-------------------------------------------------------------------------------

\begin{problem}[Posted by \href{https://artofproblemsolving.com/community/user/1991}{orl}]
	Find all the functions $f: \mathbb{R} \to\mathbb{R}$ such that
\[f(x-f(y))=f(f(y))+xf(y)+f(x)-1\]
for all $x,y \in \mathbb{R} $.
	\flushright \href{https://artofproblemsolving.com/community/c6h19782}{(Link to AoPS)}
\end{problem}



\begin{solution}[by \href{https://artofproblemsolving.com/community/user/1991}{orl}]
	Please post your solutions. This is just a solution template to write up your solutions in a nice way and formatted in LaTeX. But maybe your solution is so well written that this is not required finally. For more information and instructions regarding the ISL\/ILL problems please look here: [url=http://www.mathlinks.ro/Forum/viewtopic.php?t=15623]introduction for the IMO ShortList\/LongList project[\/url] and regarding[url=http://www.mathlinks.ro/Forum/viewtopic.php?t=15623]solutions[\/url]  :)
\end{solution}



\begin{solution}[by \href{https://artofproblemsolving.com/community/user/26}{grobber}]
	Making $x=f(y)$, we get $f(f(y))=\frac{f(0)+1}2-\frac{f(y)^2}2$, so if we denote $\frac{f(0)+1}2$ by $\alpha$, we have $f(t)=\alpha-\frac{t^2}2\ (*)$ for all $t\in f(\mathbb R)$. Take some $x=f(z)+f(y)\in f(\mathbb R)+f(\mathbb R)$. Then $f(x-f(y))=\alpha-\frac{(x-f(y))^2}2$ and $f(f(y))=\alpha-\frac{f(y)^2}2$, so if we plug these two into the initial equation we get $f(x)=1-\frac{x^2}2$, which must thus hold for all $x\in f(\mathbb R)+f(\mathbb R)$. 

Now fix any $x\in f(\mathbb R)+f(\mathbb R)$ in the initial equation. We have $f(x-f(y))=\alpha-\frac{(x-f(y))^2}2$, so $(*)$ holds for all $t\in f(\mathbb R)+f(\mathbb R)-f(\mathbb R)$. Since $-xf(y)+1=f(f(y))+f(x)-f(x-f(y))\in f(\mathbb R)+f(\mathbb R)-f(\mathbb R),\ (*)$ holds for all $t$ of the form $-xf(y)+1$, and by choosing a $y$ with $f(y)\ne 0$ ($f$ is clearly not identically zero) and making $x$ vary over the reals, we find that every real has that form, so $(*)$ actually holds for all $t\in\mathbb R$, and since $f(x)$ takes the value $1-\frac{x^2}2$ when $x\in f(\mathbb R)+f(\mathbb R)$, we find $\alpha=1$, and $f(t)=1-\frac{t^2}2,\ \forall t\in\mathbb R$. 

Conversely, it's easy to check that this is actually a solution, i.e. it satisfies the functional equation.
\end{solution}



\begin{solution}[by \href{https://artofproblemsolving.com/community/user/10277}{Philip_Leszczynski}]
	When you say

$-xf(y)+1 = f(f(y)) + f(x) - f(x-f(y)) \in f(\mathbb R) + f(\mathbb R) - f(\mathbb R)$

where did you get that range?

Also, how did you conclude that (*) holds for all $t$ in the form $-xf(y)+1$?
\end{solution}



\begin{solution}[by \href{https://artofproblemsolving.com/community/user/16465}{abdurashidjon}]
	Let $f(z)=0$ for some real number $z$. Then set $y=z$ it will become
$f(x-f(z))=f(f(z))+xf(z)+f(x)-1$ (since $f(z)=0$) then we will get $f(0)=1$   $(i)$
Set $x=0$, then 
$f(0-f(y)=f(f(y)+0\times f(y)+f(0)-1$ using $(i)$ we will get that $f(f(-y))=f(f(y))$    $(ii)$ so our function is even function
(this only satisfy our solution  ). Now here is:
$f(x+f(y))=f(x-(-f(y)))=f(-f(y))-xf(y)+f(x)-1$   $(iii)$
$f(x)=f(x+f(y)-f(y))=f(f(y))+(x+f(y))f(y)+f(x+f(y))-1$ then $f(x)=f(f(y))+(x+f(y))f(y)+f(x+f(y))-1$ now using $(iii)$ we will get that $f(f(y))=1-\frac{f^2(y)}{2}$
As before we have taken $f(z)=0$. If it really exist in our function then ita true. Set  $f(y)=z$ then we will have that 
$z=-\sqrt 2$ or $\sqrt 2$. this two numbers is real numbers and our function is even and bot of them satisfy. 
Set $f(y)=x$ then $f(x)=1-\frac{x^2}{2}$.
so answer will be $f(x)=1-\frac{x^2}{2}$ for all real $x$.
\end{solution}



\begin{solution}[by \href{https://artofproblemsolving.com/community/user/19718}{lordWings}]
	\begin{tcolorbox}$f(f(-y))=f(f(y))$    $(ii)$ so our function is even function\end{tcolorbox}
I'm not sure if we can say that.

\begin{tcolorbox}Set $f(y)=x$ then $f(x)=1-\frac{x^2}{2}$.
so answer will be $f(x)=1-\frac{x^2}{2}$ for all real $x$.\end{tcolorbox}
I think that's only valid if there's a $y$ with $x=f(y)$, and we still didn't know anything about the range of $f$.

If you wanted to avoid the assumption that there exists a $z$ with $f(z)=0$, you could have followed the next path:
$x=2f(y)$, $f(\frac{x}{2})=f(\frac{x}{2})+\frac{x^2}{2}+f(x)-1$, $f(x)=1-\frac{x^2}{2}$.
But of course, this is only valid if there's a $y$ with $x=2f(y)$. So, we still should work a little harder before assuming anything about the range of $f$.
\end{solution}



\begin{solution}[by \href{https://artofproblemsolving.com/community/user/16465}{abdurashidjon}]
	\begin{tcolorbox}[quote="abdurashidjon"]$f(f(-y))=f(f(y))$    $(ii)$ so our function is even function\end{tcolorbox}
I'm not sure if we can say that.
.\end{tcolorbox}
sorry for mistake it is like $f(-f(y))=f(f(y))$ this if you check you will get
the next i will post later 
Abdurashid
\end{solution}



\begin{solution}[by \href{https://artofproblemsolving.com/community/user/18832}{BogG}]
	\begin{tcolorbox}Let $f(z)=0$ for some real number $z$.\end{tcolorbox}

How can you admite that ??
\end{solution}



\begin{solution}[by \href{https://artofproblemsolving.com/community/user/20652}{Nekruzjon_eko}]
	Truely we can take $f(z)=0$ for any $z$.
Because there can be such that $z$. And then we will look for it.
\end{solution}



\begin{solution}[by \href{https://artofproblemsolving.com/community/user/18420}{aviateurpilot}]
	we take $ S=\{f(a)-f(b)|\ (a,b)\in R^{2}\},E=\{f(a)|\ a\in R\}$.
$ x=f(y),f(0)=c$ gives $ c=2fof(y)+(f(y))^{2}-1$
$ changing\ x\ by\ f(x)$ gives $ {f(f(x)-f(y))=fof(y)+f(x)f(y)+fof(x)-1=\frac{2(c+1)-(f(x)-f(y))^{2}}{2}-1}$
so $ \forall x\in S: \ f(x)=c-\frac{x^{2}}{2}$
we find easly that $ f$ is not a constante so, $ \exists m\in R: \ f(m)\neq 0$
and $ \forall x\in R: \ xf(m)+fof(m)-1=f(x-f(y))-f(x)\in S$
$ g: R\to R,g(x)=xf(m)+fof(m)-1$ is bijiectiv so, $ R=g(R)\subseteq S$
then $ S=R$
and $ \forall x\in S=R: \ f(x)=c-\frac{x^{2}}{2}$
$ x=y=0$ gives $ f(-c)=f(c)+c-1$ then $ c=1+f(-c)-f(c)=1$

$ \forall x\in S=R: \ f(x)=1-\frac{x^{2}}{2}$
\end{solution}



\begin{solution}[by \href{https://artofproblemsolving.com/community/user/23840}{cckek}]
	\begin{tcolorbox}Determine all functions f : R → R such that $ f (x-f (y)) = f (f (y))+x f (y)+f (x)-1$
for all x, y ε R\end{tcolorbox}

First we solve the functional equation: $ f(x-t)=f(t)+xt+f(x)-1$ 

If $ x=t=0\implies f(0)=1$, if $ x=t\implies f(0)=2f(x)+x^{2}-1\implies f(x)=1-\frac{x^{2}}{2}$

If $ x=f(y)\in Im f$ then $ c=f(0)=f(x)+x^{2}+f(x)-1\implies f(x)=-\frac{x^{2}}{2}+\frac{c+1}{2}$, where $ c=f(0)$.
So $ t=f(y)\in Im f$ therefore $ f(x-t)=-\frac{t^{2}}{2}+\frac{c-1}{2}+xt+f(x)$
If 

$ x-t=y\in Imf\implies-\frac{y^{2}}{2}+\frac{c+1}{2}=f(y)=-\frac{t^{2}}{2}+\frac{c-1}{2}+yt+t^{2}+f(y+t)\implies f(y+t)=-\frac{(y+t)^{2}}{2}+1$

 so 

$ f(x)=-\frac{x^{2}}{2}+1, x\in Imf+Imf\implies Imf+ImF\subset(-\infty,1]\implies \forall x\in \mathbb{R}\; \exists y,t\in Im f, x=y+t\implies f(x)=-\frac{x^{2}}{2}+1, \forall x\in \mathbb{R}$
\end{solution}



\begin{solution}[by \href{https://artofproblemsolving.com/community/user/74510}{filipbitola}]
	I found a solution to this problem in 2 pages. It requires a really interesting trick. I am hoping that anyone can come up with a simpler solution. For those who are interested in my solution you can find it at:

http://www.4shared.com\/document\/krpZ5Oeg\/IMO_1999_-_6_Solution.html

Link to the original problem:
http://www.artofproblemsolving.com/Forum\/resources.php?c=1&cid=16&year=1999&sid=5777a41e27950f68ccb7f28b8760fa1b
\end{solution}



\begin{solution}[by \href{https://artofproblemsolving.com/community/user/65976}{mudok}]
	Is following solution true?
Let $f(o)=c$, $f(y)=a$
put $x=a$, then \[f(a)=\frac{c+1-a^2}{2}=f(-a)\]
put $x=0$,then \[f(-a)=f(a)+c-1\]
then $c=1$ and \[f(x)=1-\frac{x^2}{2}\]
\end{solution}



\begin{solution}[by \href{https://artofproblemsolving.com/community/user/64716}{mavropnevma}]
	NO. Not to mention that the fact that $f(-a) = \frac {c+1-a^2} {2}$ is not justified, the final conclusion would in fact be $f(a) = 1 - \dfrac {a^2} {2}$, i.e.  $f(f(y)) = 1 - \dfrac {f(y)^2} {2}$, or, if you want to change variables, $f(f(x)) = 1 - \dfrac {f(x)^2} {2}$.

This means the quadratic form you propose as unique solution holds over $\textrm{Im} f = f(\mathbb{R})$. How do you propose to show it holds over the whole $\mathbb{R}$ ? If you just looked at the first proof provided (post #2), you would have noted that this is the crucial issue ... so your attempt is just a simple\begin{bolded} first step \end{bolded}towards a much more sofisticated full solution.
\end{solution}



\begin{solution}[by \href{https://artofproblemsolving.com/community/user/65976}{mudok}]
	Thanks Mavropnevma. I understand that it was true only for x-es that are in the set of values of the given function(is there a name of this set?)
  
PS: it was not my solution. I can't find even incomplete solution like this. 

PS: What is the difference between range of  a function and set of values of a function? Or they are equal sets?  Sorry, i am beginner.
\end{solution}



\begin{solution}[by \href{https://artofproblemsolving.com/community/user/208529}{vlad1m1r}]
	I don't quite think that you can admit that there exists a real $z$ such that $f(z)=0$ (eg: $f(x)=x^2+1$ doesn't have zeroes on reals, the same way you didn't know if the function in the problem had zeroes), but this problem can be solved preety easilly knowing a quite simple and usefull trick.

First of all simple check can show that the function can't be zero on the whole domain so take $t$ such that $f(t)=a$ is not $0$. Putting $y=t$ we get:
$f(x-a)-f(x)=ax+f(a)-1$, so we have that every real number $x$ can be written as $x=f(u)-f(v)$, for some real numbers $u,v$. Putting $y=f(y)$ we get:
$f(f(x)-f(y))=f(f(x))+f(x)f(y)+f(f(y))-1$. Putting $x=y$ in this relation we get:
$f(f(x))=\cfrac{1}{2}\cdot (1+f(1)-f(x)^2)$. Introducing this relation in the previous one we get that:
$f(f(x)-f(y))=f(0)-\cfrac{1}{2}\cdot (f(x)-f(y))^2$. So by putting $x=u,y=v$ we get that:
$f(x)=f(0)-\cfrac{x^2}{2}$. A simple verification leads us to the single solution:
\[f(x)=1-\frac{x^2}{2}\]
\end{solution}



\begin{solution}[by \href{https://artofproblemsolving.com/community/user/183149}{JuanOrtiz}]
	Let $\Omega = \text{Img}(f)$ and $g(x)=f(x)-(1-\frac{x^2}{2})$. By letting $x \in \Omega$ we verify that $f(0)=1$ and by letting $x=f(y)$ we verify $g(\omega)=0 \forall \omega \in \Omega$. Notice that $g(x)=g(x+\omega) \forall x\in \mathbb{R},\omega \in \Omega$, therefore $g$ is periodic with period $\omega_1-\omega_2$ for all $\omega_1,\omega_2 \in \Omega$. Finally notice that if $\omega_1 = f(x_1),\omega_2=f(x_2)$ with $x_2=x_1+\omega$ for any $\omega \in \Omega,\omega \neq 0$ (for example, $\omega=1$) then

$\omega_1 - \omega_2 = x \omega - (\omega^2\/2)$, which can take any real value.

This implies $g$ is periodic modulo anything, so it is constant. Since it takes the value $0$, $g \equiv 0$ and so $f \equiv 1-(x^2\/2)$
\end{solution}



\begin{solution}[by \href{https://artofproblemsolving.com/community/user/222968}{rkm0959}]
	Denote $F=\{f(y)| y \in \mathbb{R}\}$ and $A=\{a-b|a,b\in F\}$.

First, for $b \in F$, plugging $x=a$ and $f(y)=b$ to the equation gives $f(a-b)=f(b)+ab+f(a)-1$ for all $a \in \mathbb{R}$ and $b \in F$.
Setting $a=b$ gives $f(0)=2f(a)+a^2-1$, so $f(a)=-\frac{1}{2}a^2+\frac{f(0)+1}{2}$ for all $a \in F$.
For all $c$ such that $f(c) \not= 0$, we have $f(x-f(c))-f(x)=f(f(c))+xf(c)-1$. 
Since the R.H.S can take all real values, we have $A \equiv \mathbb{R}$.

Now for $a, b \in F$, we have $f(a-b)=ab+f(a)+f(b)-1=ab-\frac{1}{2}a^2-\frac{1}{2}b^2+f(0)+1-1=-\frac{1}{2}(a-b)^2+f(0)$. 
Therefore, $f(x)=-\frac{1}{2}x^2+f(0)$ for all $x \in \mathbb{R}$. Note that $\frac{f(0)+1}{2}=f(0)$, so $f(0)=1$.

This gives $f(x)=-\frac{1}{2}x^2+1$ as our only answer.
\end{solution}



\begin{solution}[by \href{https://artofproblemsolving.com/community/user/165984}{sturdyoak2012}]
	Let $x = f(y).$  Then

\[f(0) = f(f(y)) + f(y)^2 + f(f(y)) - 1 = 2f(f(y)) + f(y)^2 - 1.\]

Let $t = f(y).$  Then

\[f(0) = 2f(t) + t^2 - 1.\]

Thus,

\[f(t) = \dfrac{f(0) + 1}{2} - \dfrac{t^2}{2}\]

Replace $t$ with $x - f(y).$  Then

\[f(x - f(y)) = \dfrac{f(0) + 1}{2} - \dfrac{x^2 - 2xf(y) + f(y)^2}{2} = f(f(y)) + xf(y) + f(x) - 1.\]

Thus,

\[\dfrac{f(0) + 1}{2} = \dfrac{x^2}{2} - xf(y) + \dfrac{f(y)^2}{2} + f(f(y)) + xf(y) + f(x) - 1.\]

This simplifies down to

\[\dfrac{f(0) + 1}{2} = \dfrac{x^2}{2} + \dfrac{f(y)^2}{2} + f(f(y)) + f(x) - 1.\quad\quad(1)\]

Replace $t$ with $f(y).$  Then

\[f(f(y)) = \dfrac{f(0) + 1}{2} - \dfrac{f(y)^2}{2}.\]

Substituting this into $(1),$ we get

\[\dfrac{f(0) + 1}{2} = \dfrac{x^2}{2} + \dfrac{f(y)^2}{2} + \dfrac{f(0)+1}{2} - \dfrac{f(y)^2}{2} + f(x) - 1.\]

There are many terms that cancel, leading us to the equation

\[\dfrac{x^2}{2} + f(x) - 1 = 0.\]

This is

\[f(x) = 1 - \dfrac{x^2}{2}.\]

Replacing this back into the original equation, we get

\begin{align*}
1 - \dfrac{x^2 - 2xf(y) + f(y)^2}{2} &= 1 - \dfrac{1 - y^2 + \frac{y^4}{4}}{2} + x - \dfrac{xy^2}{2} + 1 - \dfrac{x^2}{2} - 1\\
1 - \dfrac{x^2}{2} + x - \dfrac{xy^2}{2} - \dfrac{1-y^2+\frac{y^2}{4}}{2} &= - \dfrac{1-y^2+\frac{y^2}{4}}{2} + x - \dfrac{xy^2}{2} + 1 - \dfrac{x^2}{2}.
\end{align*}

Everything cancels out, so $\boxed{f(x) = 1 - \dfrac{x^2}{2}}$ for all $x\in \mathbb{R}.$
\end{solution}



\begin{solution}[by \href{https://artofproblemsolving.com/community/user/146669}{trumpeter}]
	\begin{tcolorbox}
Replace $t$ with $x - f(y).$\end{tcolorbox}

I don't quite think that you can do this, as you defined $t$ to be a member of the image of $f$. For example, if $f\equiv0$, then $t$ can only be $0$, so then when we make this substitution, $x$ must be $0$ (then $x$ cannot range all real numbers).
\end{solution}



\begin{solution}[by \href{https://artofproblemsolving.com/community/user/193803}{suli}]
	Guided Solve:
[hide=Hint 1]
Notice that $f(y)$ can only take on values in the range of $f$. If we can prove $f$ is surjective that would be great; however, there exists at least one non-surjective solution. What else is surjective?
[\/hide]
[hide=Hint 2]
Rewrite the condition as
$$f(x - c) = f(c) + xc + f(x) - 1,$$
where $c$ is in the range of $f$. Now plugging in $c$ is obvious: it leads
$$f(c) = \frac{1 + f(0) - c^2}{2}.$$
Thus
$$f(x - c) = \frac{-1 + f(0) - c^2}{2} + xc + f(x).$$
Unfortunately, we still have a quadratic term. Is there anything we can do to make this look nicer?
[\/hide]
[hide=Hint 3]
If you're smart you already guessed $f(x) = 1 - \frac{x^2}{2}$ is a solution. Thus make the substitution
$$f(x) = g(x) + 1 - \frac{x^2}{2}.$$
Then $f(0) = g(0) + 1$, so substitute $g$ into $f$ and cancel to yield
$$g(x - c) = \frac{g(0)}{2} + g(x).$$
This holds for any $c = f(y).$ This looks like a line were $c$ fixed. But $c$ is not fixed....
[\/hide]

[hide=Solution]
Notice that $g(x-d) = \frac{g(0)}{2} + g(x) = g(x-e)$ for any $d, e$ in range of $f$. Thus
$$g(x) = g(x + d - e).$$
Now clearly the zero function is not a valid solution, so assume $f$ is not completely zero. Thus there must exist $c \neq 0$ in its range. Let
$$d = f(y), e = f(y - c)$$
and see that
$$f(y - c) - f(y) = \frac{-1 + f(0) - c^2}{2} + yc.$$
Thus $d - e$ can take on any value, so $g(x)$ is constant. But $g(x - c) = \frac{g(0)}{2} + g(x)$, so $g(0) = 0$. Hence $f(x) = 1 - \frac{x^2}{2}$ for any $x$. This is indeed a valid solution.
[\/hide]
\end{solution}



\begin{solution}[by \href{https://artofproblemsolving.com/community/user/85576}{larkl}]
	$f(x-f(y)) = f(f(y)) + xf(y) + f(x) - 1$                                                        (equation  1)

$f(x) = 1 - \dfrac {x^2} 2$ satisfies the relation, so that's one possible answer.  

If $f(y)=0$ for some $y$, then from equation (1), $f(0)=1$.  So $f(0)\neq 0$.  

Then setting $x=\dfrac 1 {f(0)}, y = 0$ in equation 1, 

$f\left(\dfrac 1 {f(0)} - f(0)\right) 
= f(f(0)) + f\left(\dfrac 1 {f(0)}\right)$                                                            (equation 2)

For $r\in \text{range(f)}$, setting $x=r, f(y)=r$ in equation 1, we get 

$f(r) = \dfrac {1+f(0) - r^2} 2$  

For $r_1,r_2 \in \text{range}(f)$, setting $x=r_1+r_2, y=r_2$ in equation 1, we get

$f(r_1+r_2) = 1 - \dfrac {(r_1+r_2)^2} 2$.  

But the left side of equation 2 is of the form $f(x)$, and the right side is of the form $f(x_1)+f(x_2)$.  

So applying $f$ to both sides of equation 2, we get that $f(0)=1$.  

Let $S$ be the set of $x$ such that $f(x) = 1-\dfrac {x^2} 2$.  

Then for $s\in S$ and $r\in \text{range}(f), s-r\in S$ and $s+r\in S$, as can be verified from equation 1.

$f(x-1) = f(x-f(0)) = f(1) + x + f(x) - 1$, so

$x = f(x-1) - f(1) - f(x) + f(0)$.  So $x\in S$.  

So $f(x) = 1 - \dfrac {x^2} 2$.
\end{solution}



\begin{solution}[by \href{https://artofproblemsolving.com/community/user/368111}{Anis2017}]
	we can solve it directly by putting $x=2f(y)$
is this right??
\end{solution}
*******************************************************************************
-------------------------------------------------------------------------------

\begin{problem}[Posted by \href{https://artofproblemsolving.com/community/user/1991}{orl}]
	Let $ {\mathbb Q}^ +$ be the set of positive rational numbers. Construct a function $ f : {\mathbb Q}^ + \rightarrow {\mathbb Q}^ +$ such that
\[ f(xf(y)) = \frac {f(x)}{y}
\]
for all $ x$, $ y$ in $ {\mathbb Q}^ +$.
	\flushright \href{https://artofproblemsolving.com/community/c6h60742}{(Link to AoPS)}
\end{problem}



\begin{solution}[by \href{https://artofproblemsolving.com/community/user/26}{grobber}]
	It suffices to construct such a function satisfying $f(ab)=f(a)f(b),\ \forall a,b\in\mathbb Q^+\ (*)$ (this implies $f(1)=1$) and $f(f(x))=\frac 1x,\ \forall x\in\mathbb Q^+\ (**)$.

All we need to do is define $f(p_i)$ s.t. $(*)$ whenever $x=p_i$ for some $i\ge 1$, where $(p_n)_{n\ge 1}$ is the sequence of primes, and then extend it to the rest of $\mathbb Q^+$ so that $(**)$ holds. Then it's clear that $(*)$ will automatically hold.
\end{solution}



\begin{solution}[by \href{https://artofproblemsolving.com/community/user/4527}{Amir.S}]
	as Grobber said(he didn't prove it) we have $ f(ab) = f(a)f(b)$ , this  implies $ f(\prod_{i = 1}^np_i^{\alpha_i}) = \prod_{i = 1}^nf(p_i)^{\alpha_i}$ , hence we must deifne the function on all primes, let $ p_i$ denote the $ i - th$ prime number we define $ f$ as:
$ f(p_{2k - 1}) = p_{2k}\ ,\ f(p_{2k}) = \frac {1}{p_{2k - 1}}$
this function satisfies the problem , clearly.
\end{solution}



\begin{solution}[by \href{https://artofproblemsolving.com/community/user/28136}{Rofler}]
	So how do you extend to Q?
\end{solution}



\begin{solution}[by \href{https://artofproblemsolving.com/community/user/38170}{aznlord1337}]
	$ f(f(y)) = f(1)\/y$. This implies that $ f$ is injective. $ f(f(1)) = f(1) \longrightarrow f(1) = 1$

Therefore $ f(f(y)) = 1\/y$. Let $ y=f(y)$, so $ f(x\/y) = f(x)\/f(y)$. Then $ f(1\/f(y)) = f(1)\/f(f(y)) = y$

From the original equation, letting $ y=1\/f(y)$ implies $ f(xy) = f(x)f(y)$. A function on primes like Amir's works.
\end{solution}



\begin{solution}[by \href{https://artofproblemsolving.com/community/user/41655}{triplebig}]
	I don't understand how you can conclude that $ f$ is injective, can anyone please share some light?
\end{solution}



\begin{solution}[by \href{https://artofproblemsolving.com/community/user/29428}{pco}]
	\begin{tcolorbox}I don't understand how you can conclude that $ f$ is injective, can anyone please share some light?\end{tcolorbox}
$ f(y_1) = f(y_2)$ $ \implies$ $ f(xf(y_1)) = f(xf(y_2))$ $ \implies$ $ \frac {f(x)}{y_1} = \frac {f(x)}{y_2}$ $ \implies$ $ y_1 = y_2$ (since $ f(x)\neq 0$)
\end{solution}



\begin{solution}[by \href{https://artofproblemsolving.com/community/user/41655}{triplebig}]
	Got it, thank you for the help
\end{solution}



\begin{solution}[by \href{https://artofproblemsolving.com/community/user/72819}{Dijkschneier}]
	\begin{tcolorbox}So how do you extend to Q?\end{tcolorbox}
Can sameone answer to this, please ?
\end{solution}



\begin{solution}[by \href{https://artofproblemsolving.com/community/user/29428}{pco}]
	\begin{tcolorbox}[quote="Rofler"]So how do you extend to Q?\end{tcolorbox}
Can sameone answer to this, please ?\end{tcolorbox}

There is no need for extension : the problem is just for Q+ and we know that any positive rational may be written in a unique manner as the product of prime numbers raised to integer powers.

What do you want more ?
\end{solution}



\begin{solution}[by \href{https://artofproblemsolving.com/community/user/72819}{Dijkschneier}]
	Thank you.
\end{solution}



\begin{solution}[by \href{https://artofproblemsolving.com/community/user/95245}{CPT_J_H_Miller}]
	Sorry to revive this topic, but can someone please explain why this doesn't work? :

Note that from the above we've already established that $f(xy)=f(x)f(y)$ and $f$ is injective.
Also, from $ f(f(y)) = \frac{1}{y} $, we know that $ f $ is surjective, therefore $ f^{-1}(x) $ exists for all positive rationals $ x $.

So set $ f^{-1}(y)$ as $ y $ in $ f(f(y)) = \frac{1}{y} \Rightarrow f(y)f^{-1}(y) = 1 $

Thus set $ f^{-1}(x) $ as $ x $ and $ y $ as $ y $ into the original equation and we obtain:
$ f(f^{-1}(x)f(y)) = \frac{x}{y} $
$ \Rightarrow f^{-1}(x)f(y) = \frac{x}{y} $
$ \Rightarrow \frac{f(y)}{f(x)} = \frac{x}{y} $
$ \Rightarrow f(x) = \frac{1}{x} $ $\forall$  $ x \in \mathbb{Q}^{+} $
which is obviously not a solution to the equation.

Can someone please explain what went wrong? Thanks.
\end{solution}



\begin{solution}[by \href{https://artofproblemsolving.com/community/user/64716}{mavropnevma}]
	\begin{tcolorbox}Thus set $ f^{-1}(x) $ as $ x $ and $ y $ as $ y $ into the original equation and we obtain:
$ f(f^{-1}(x)f(y)) = \frac{x}{y} $
$ \Rightarrow f^{-1}(x)f(y) = \frac{x}{y} $.\end{tcolorbox}
The implication is abusive; from $f(A) = B$ you infer $A=B$.
\end{solution}



\begin{solution}[by \href{https://artofproblemsolving.com/community/user/95245}{CPT_J_H_Miller}]
	Argh... silly mistake... yes it should be $ f(f^{-1}(x)f(y)) = xf(f(y)) = \frac{x}{y} $.
Thanks!
\end{solution}



\begin{solution}[by \href{https://artofproblemsolving.com/community/user/74734}{flare}]
	I was able to determine the conditions for the function, but not able to construct it. 
Out of curiosity, how many points would I get for this (on the actual thing I would probably spend time finding it since the conditions take a very small time to find, but...)?
\end{solution}



\begin{solution}[by \href{https://artofproblemsolving.com/community/user/141397}{subham1729}]
	:mad: Plugging in x = 1 we get f(f(y)) = f(1)\/y and hence f(y1) = f(y2)
implies y1 = y2 i.e. that the function is bijective. Plugging in y = 1 gives
us f(xf(1)) = f(x) ⇒ xf(1) = x ⇒ f(1) = 1. Hence f(f(y)) = 1\/y.
Plugging in y = f(z) implies 1\/f(z) = f(1\/z). Finally setting y = f(1\/t)
into the original equation gives us f(xt) = f(x)\/f(1\/t) = f(x)f(t).
Conversely, any functional equation on Q+ satisfying (i) f(xt) = f(x)f(t)
and (ii) f(f(x)) = 1
x for all x, t ∈ Q+ also satisfies the original functional
equation: f(xf(y)) = f(x)f(f(y)) = f(x)
y . Hence it suffices to find
a function satisfying (i) and (ii).
We note that all elements q ∈ Q+ are of the form q = $n
i=1 pai
i where
pi are prime and ai ∈ Z. The criterion (a) implies f(q) = f($n
i=1 pai
i ) = $n
i=1 f(pi)ai . Thus it is sufficient to define the function on all primes. For
the function to satisfy (b) it is necessary and sufficient for it to satisfy
f(f(p)) = 1
p for all primes p. Let qi denote the i-th smallest prime. We
define our function f as follows:
f(q2k−1) = q2k, f(q2k) =
1
q2k−1
, k ∈ N .
Such a function clearly satisfies (b) and along with the additional condition
f(xt) = f(x)f(t) it is well defined for all elements of Q+ and it satisfies
the original functional equation. :P  :mad:  :mad:
\end{solution}



\begin{solution}[by \href{https://artofproblemsolving.com/community/user/223099}{MathPanda1}]
	\begin{tcolorbox}as Grobber said(he didn't prove it) we have $ f(ab) = f(a)f(b)$ , this implies $ f(\prod_{i = 1}^np_i^{\alpha_i}) = \prod_{i = 1}^nf(p_i)^{\alpha_i}$ , hence we must deifne the function on all primes, let $ p_i$ denote the $ i - th$ prime number we define $ f$ as:
$ f(p_{2k - 1}) = p_{2k}\ ,\ f(p_{2k}) = \frac {1}{p_{2k - 1}}$
 this function satisfies the problem , clearly.\end{tcolorbox}

What is the motivation for constructing such a function? Thank you very much!
\end{solution}



\begin{solution}[by \href{https://artofproblemsolving.com/community/user/130234}{vsathiam}]
	\begin{tcolorbox}[quote=Amir.S]as Grobber said(he didn't prove it) we have $ f(ab) = f(a)f(b)$ , this implies $ f(\prod_{i = 1}^np_i^{\alpha_i}) = \prod_{i = 1}^nf(p_i)^{\alpha_i}$ , hence we must deifne the function on all primes, let $ p_i$ denote the $ i - th$ prime number we define $ f$ as:
$ f(p_{2k - 1}) = p_{2k}\ ,\ f(p_{2k}) = \frac {1}{p_{2k - 1}}$
 this function satisfies the problem , clearly.\end{tcolorbox}

What is the motivation for constructing such a function? Thank you very much!\end{tcolorbox}

First you try out some algebraic methods: none of them are fruitful. Then you note that the problem said construction, which implies a numbertheoretic approach. This immediately applies looking for multiplicity and a way to define f(1), which just happen to be related to each other. (Shows that you are on the right track). Then you obtain the relation:

$f(f(p)) = \frac{1}{p}$ for all primes.

So it is clear that you cannot manipulate the power of prime p to get from an exponent of 1 to -1 in two steps over $\mathbb{Q^{+}}$. So you have to manipulate the primes in some other method, with a group of elements acting as a medium. (In other words, f(f(p)) maps A $\rightarrow$ B $\rightarrow$ C, where you know that {p} = A, B is unknown, and {$\frac{1}{p}$} = C.

This suggests bipartitioning the set of primes, which suggests considering the sets {$p_{2k}$}, {$p_{2k-1}$}, {$\frac{1}{p_{2k}}$} and {$\frac{1}{p_{2k-1}}$}. Playing around with directed arrows that map elements between the sets gives you the function.
\end{solution}
*******************************************************************************
-------------------------------------------------------------------------------

\begin{problem}[Posted by \href{https://artofproblemsolving.com/community/user/11927}{dondigo}]
	Find all functions $f : \mathbb{N}\rightarrow \mathbb{N}$ satisfying following condition:
\[f(n+1)>f(f(n)), \quad \forall n \in \mathbb{N}.\]
	\flushright \href{https://artofproblemsolving.com/community/c6h75980}{(Link to AoPS)}
\end{problem}



\begin{solution}[by \href{https://artofproblemsolving.com/community/user/16261}{Rust}]
	There are more solution. For example 
$1) \ f(n)=n,$
$2) \ f(n)=n-(-1)^n,$
and combinations 1), 2).
\end{solution}



\begin{solution}[by \href{https://artofproblemsolving.com/community/user/13}{enescu}]
	[url=http://www.kalva.demon.co.uk\/imo\/isoln\/isoln776.html]www.kalva.demon.co.uk\/imo\/isoln\/isoln776.html[\/url]
\end{solution}



\begin{solution}[by \href{https://artofproblemsolving.com/community/user/1838}{paladin8}]
	\begin{tcolorbox}There are more solution. For example 
$1) \ f(n)=n,$
$2) \ f(n)=n-(-1)^n,$
and combinations 1), 2).\end{tcolorbox}

The second one doesn't seem to work because the inequality is strict. For instance, take any $n$ odd:

$f(n+1) = (n+1)-1 = n$
$f(f(n)) = f(n+1) = n$

so strict inequality does not hold.
\end{solution}



\begin{solution}[by \href{https://artofproblemsolving.com/community/user/10088}{silouan}]
	I have a solution please tell me if I am wrong .We suppose that 
$f$ is decreasing in an interval $G$ . Then for $k\in G$ we have from ypothesis
$f(k+1)>f(f(k))$ so $f(k)>k+1$  so $f(p)>p+1$ for any $p$ in $G$ . So $f(p)=p+m$ where $m$ is a positive integer $m\geq 2$

By hypothesis  we find that $p+m+1>p+2m$ or $1>m$ contradiction .
So $f$ is an increasing function .So by hypothesis again $f(n+1)>f(f(n))$ implies that $n+1>f(n)$ . But from the fact that $f$ is an increasing function we find that $f(n)\geq n$ so $f(n)=n$
\end{solution}



\begin{solution}[by \href{https://artofproblemsolving.com/community/user/16261}{Rust}]
	\begin{tcolorbox}$f$ is decreasing in an interval $G$ . Then for $k\in G$ we have from ypothesis
$f(k+1)>f(f(k))$ so $f(k)>k+1$  so $f(p)>p+1$ for any $p$ in $G$ . \end{tcolorbox}
Exactly for any $p$ in $G$, suth that $f(p)\in G$ and $p+1\in G$.
\end{solution}



\begin{solution}[by \href{https://artofproblemsolving.com/community/user/8288}{gopherhole112}]
	A fairly simple solution is as follows. It assumes that the mapping $ f$ is bijective, and increasing. For the sake of brevity (I'm lazy) those lemmata have been left out.

$ f(n+1)>f(f(n))$
We apply the inverse of $ f$ to both sides
$ n+1>f(n)$

Similarly, we can say that

$ f(n)>f(f(n-1))$
$ \therefore n>f(n-1)$

Putting these together, we get that
$ f(n+1)>n>f(n-1)$

This means that $ n+1>f^{-1}(n)>n-1$, but since $ f$ is bijective, $ f^{-1}$ must be an integer between $ n-1$ and $ n+1$ exclusive.

Thus $ f^{-1}(n)=n$
$ f(n)=n$
QED
\end{solution}



\begin{solution}[by \href{https://artofproblemsolving.com/community/user/153087}{Weirdos34}]
	I thought this was a new problem and so I posted it elsewhere to share it with others(It was locked and I was referred to this thread).Anyway,I shall post my solution now:

Let $r=\text{min}\{f(n):n\in \mathbb{N}\}$ and $m\in \mathbb{N}$ so that $r=f(m)$.For the sake of contradiction, assume that $m>1$.Then $r=f(m)>f(f(m-1))$, which contradicts the minimality of $r$.So, $m=1$ and $f(1)<f(2)$.In the set $\{f(n):n\ge 2\}$,$f(2)$ is the minimum(using the same argument).If $f(1)>1$ then $f(1)\ge 2$.Clearly,$f(f(1))\ge 2$, a contradiction to the initial condition.So, $f(1)=1$.Consider $g(n)=f(n+1)-1$. So, $g(g(n))=g(f(n+1)-1)<f(n+2)-1=g(n+1)$.$g$ satisfies the same condition as $f$.Therefore,$g(1)=1$ and hence $f(2)=2$. By induction on $n$, $f(n)=n$
\end{solution}



\begin{solution}[by \href{https://artofproblemsolving.com/community/user/86983}{GaryTam}]
	\begin{bolded}Lemma.\end{bolded} If $n\geq m$, then $f(n)\geq m$.
\begin{italicized}Proof.\end{italicized} We proceed by induction on $m$. For $m=1$ this is trivial. Suppose $f(n)\geq m$ for all $n\geq m$. If $n\geq m+1$, using the induction hypothesis twice,
$$ n-1\geq m\hspace{4mm}\Rightarrow\hspace{4mm}f(n-1)\geq m\hspace{4mm}\Rightarrow\hspace{4mm}f\left(f(n-1)\right)\geq m, $$
and it follows that $f(n) > f\left(f(n-1)\right)\geq m$. $\square$

Now, going back to the original question, the lemma implies that $f\left(f(n)\right)\geq f(n)$, so $f(n+1) > f(n)$, i.e., $f$ is strictly increasing. Since $f$ is strictly increasing, $f(n+1) > f\left(f(n)\right)$ implies $n+1 > f(n)$, i.e., $f(n)\leq n$. Combining this with the lemma, which asserts $f(n)\geq n$, we get $f(n) = n$.
\end{solution}
*******************************************************************************
-------------------------------------------------------------------------------

\begin{problem}[Posted by \href{https://artofproblemsolving.com/community/user/4618}{andre.l}]
	Find all functions $ f: \mathbb{R}\to\mathbb{R}$ such that $ f(x+y)+f(x)f(y)=f(xy)+2xy+1$ for all real numbers $ x$ and $ y$.

\begin{italicized}Proposed by B.J. Venkatachala, India\end{italicized}
	\flushright \href{https://artofproblemsolving.com/community/c6h78909}{(Link to AoPS)}
\end{problem}



\begin{solution}[by \href{https://artofproblemsolving.com/community/user/15232}{vietnamesegauss89}]
	Given $x=y=0 \Rightarrow (f(0))^2=1$
$*)f(0)=1$
Given$y=0\Rightarrow f(x)\equiv 1 \forall x$
$)f(0)=-1$
Given $x=-y$we have:
            $f(x)f(-x)-1=f(-x^2)-2x^2+1$(*)
Given $x=1$we have
$f(-1)(1-f(1))=0\Rightarrow f(-1)=0$or$f(1)=1$
$+)f(1)=1$
Given y=1,we have:
              $f(x+1)=2x+1$
              $\Rightarrow f(x)=2x-1 \forall x$
$+)f(-1)=0$
Given $y=-1$,we have:
$f(-x)=f(x-1)+2x-1$
Given$y=-1-x$we have:
$f(x)(f(x)+2x+1)=f(x^2+x-1)$
Given$x=1$we have:
$f(1)=f(1)(f(1)+3)$
$\Rightarrow f(1)=0$or$f(1)=-2$
.)$f(1)=0$
Given$y=1$we have
$f(x+1)=f(x)+2x+1$
$\Rightarrow f(2)=3$
Given$y=2$we have:
$f(x+2)+3f(x)=f(2x)+4x+1$
$f(x+2)=f(x)+4x+1 \Rightarrow f(2x)=4f(x)+3$
Given $x=x+1,y=x-1$we have:
$f(2x)+f(x-1)f(x+1)=f(x^2-1)+2x^2-1$
But $f(x^2-1)=f(x^2)\2x^2+1)$
$\Rightarrow 4f(x)+3+(f(x)-2x+1)(f(x)+2x+1)=f(x^2)$
$\Rightarrow (f(x))^2+6f(x)+4-4x^2=f(x^2)$(1)
Given $x=y$we have:
$f(2x)+(f(x))^2=f(x^2)+2x^2+1$
$\Rightarrow f(x^2)=(f(x))^2+4f(x)+2-2x^2$(2)
(1)(2)$\Rightarrow 2f(x)+2-2x^2=0$
$\Rightarrow f(x)=x^2-1\forall x$
.)Similar with $f(1)=-2$ we have:$f(x)=-x-1$
The function satisfly:
$f(x)=2x-1\forall x$
$f(x)=x^2-1\forall x$
$f(x)=-x-1\forall x$
\end{solution}



\begin{solution}[by \href{https://artofproblemsolving.com/community/user/18111}{mps}]
	\begin{tcolorbox}
$\Rightarrow 4f(x)+3+(f(x)-2x+1)(f(x)+2x+1)=f(x^2)$
\end{tcolorbox}
How you get that ? 
I get  $4f(x)+3+(f(x)-2x+1)(f(x)+2x+1)=f(-x^2)$

Anyone help me please!
\end{solution}



\begin{solution}[by \href{https://artofproblemsolving.com/community/user/16383}{M4RI0}]
	Is there a shorter solution?
\end{solution}



\begin{solution}[by \href{https://artofproblemsolving.com/community/user/15581}{Davron}]
	can anyone explain how he could get :

Given $y=-1-x$ we have , $f(x)(f(x)+2x+1)=f(x^2+x-1)$

thanks
\end{solution}



\begin{solution}[by \href{https://artofproblemsolving.com/community/user/16465}{abdurashidjon}]
	Hi Here you have found as follows
\begin{tcolorbox}Given$y=-1-x$we have:
$f(x)(f(x)+2x+1)=f(x^2+x-1)$
Given$x=1$we have:
$f(1)=f(1)(f(1)+3)$
$\Rightarrow f(1)=0$or$f(1)=-2$\end{tcolorbox}
$\Rightarrow f(1)=0$or$f(1)=-2$. you used "or"  but it could be the same or I am wrong. explain it how to discover that they are different 
Abdurashid
\end{solution}



\begin{solution}[by \href{https://artofproblemsolving.com/community/user/35129}{Zhero}]
	They are allowed to be the same; the or is not exclusive in that scenario. 

Here's my solution: 

Let $ P(x,y)$ be the assertion that $ f(x + y) + f(x)f(y) = f(xy) + 2xy + 1$. 

$ P(0,0)$ yields $ f(0) = \pm 1$. If $ f(0) = 1$, then $ P(x,0)$ yields $ f(x) = 1$ for all $ x$, which $ P(3,3)$ shows is impossible, so $ f(0) = - 1$. 

$ P(x, - x)$ now gives $ f(x)f( - x) = f( - x^2) - 2x^2 + 2$. Setting $ x = 1$ yields $ f(1)f( - 1) = f( - 1)$, so $ f(1) = 1$ or $ f( - 1) = 0$. 

If $ f(1) = 1$, then $ P(x,1)$ yields $ f(x + 1) + f(x) = f(x) + 2x + 1$, so $ f(x + 1) = 2x + 1$, so $ f(x) = 2x - 1$ for all real $ x$. It can easily be seen that $ f(x) = 2x - 1$ indeed satisfies this functional equation. 

Suppose now that $ f(1) \neq 1$; then $ f( - 1) = 0$. $ P(1, - 2)$ yields $ f( - 1) + f(1)f( - 2) = f( - 2) - 3$, that is, $ f(1)f( - 2) = f( - 2) - 3$. $ P( - 1,2)$ gives $ f(1) + f( - 1)f(2) = f( - 2) - 3$, that is, $ f(1) = f( - 2) - 3$. Therefore, $ f(1) = f(1)f( - 2)$, so $ f(1) = 0$ or $ f( - 2) = 1$. 

$ f(1) = 0$ will give $ f(x) = x^2 - 1$. $ P(x, - 1)$ and $ P( - x,1)$ yield $ f(x - 1) = f(1 - x) = f( - x) - 2x + 1$, so $ f$ is even. $ P(x, - x)$, combined with the fact that $ f$ is even, yields $ f(x)^2 = f(x^2) - 2x^2 + 2$, while $ P(x,x)$ gives $ f(2x) + f(x)^2 = f(x^2) + 2x^2 + 1$. Since $ f(x)^2 = f(x^2) - 2x^2 + 2$, we see that $ f(2x) = 4x^2 - 1$, so $ f(x) = x^2 - 1$ for all $ x$; it can easily be verified that $ f(x) = x^2 - 1$ satisfies the functional equation. 

If $ f( - 2) = 1$, then $ f(1) = f( - 2) - 3 = - 2$. $ P(1, - x)$ yields $ f(1 - x) = 3f( - x) - 2x + 1$, while $ P( - 1,x)$ yields $ f(x - 1) = f( - x) - 2x + 1$. Subtracting these two equations gives $ 2f( - x) = f(1 - x) - f(x - 1)$. Substituting $ x = 1 - y$ yields $ 2f( - 1 + y) = f(y) - f( - y)$. Replacing $ y$ with $ - y$ yields $ 2f( - 1 - y) = f( - y) - f(y) = - 2f( - 1 + y)$, so $ f( - 1 + y) = - f( - 1 - y)$ for all real $ y$. 

$ P(x, - 1)$ yields $ f(x - 1) = f( - x) - 2x + 1$, and $ P( - x, - 1)$ yields $ f( - x - 1) = f(x) + 2x + 1$. Since $ f( - x - 1) = - f(x - 1)$, we have that $ f(x - 1) = - f(x) - 2x - 1 = f( - x) - 2x + 1$, so $ f(x) + f( - x) = - 2$ for all real $ x$. 

$ P(0,0)$ yields $ f(x)f( - x) = f( - x^2) - 2x^2 + 2$, so $ f(x)( - 2 - f(x)) = - 2 - f(x^2) - 2x^2 + 2$. Thus, $ 2f(x) + f(x)^2 = f(x^2) + 2x^2$. On the other hand, $ P(x,x)$ yields $ f(2x) + f(x)^2 = f(x^2) + 2x^2 + 1$. Subtracting $ 2f(x) + f(x)^2 = f(x)^2 + 2x^2$ gives $ f(2x) - 2f(x) = 1$. 

$ f(2) = - 2 - f( - 2) = - 3$, so $ P(x,2)$ yields $ f(x + 2) - 3f(x) = f(2x) + 4x + 1$. But $ f(2x) = 2f(x) + 1$, so $ f(x + 2) = 5f(x) + 4x + 2$. On the other hand, $ P(x,1)$ gives $ f(x + 1) = 3f(x) + 2x + 1$; substituting $ x = y + 1$ here yields $ f(y + 2) = 3f(y + 1) + 2y + 3$. But $ f(y + 1) = 3f(y) + 2y + 1$, so $ f(y + 2) = 9f(y) + 8y + 6$. However, $ f(y + 2) = 5f(y) + 4y + 2$ as well, so $ 4f(y) = - 4y - 4$, yielding $ f(y) = - y - 1$ for all real $ y$. It can easily be verified that $ f(x) = - x - 1$ satisfies the functional equation. 

Hence, our three solutions to this functional equation are $ \boxed{f(x) = 2x - 1, x^2 - 1, - x - 1}$.
\end{solution}



\begin{solution}[by \href{https://artofproblemsolving.com/community/user/37259}{math154}]
	This is my solution from WOOT some time ago...

Let $P(x,y)\implies f(x+y)+f(x)f(y)=f(xy)+2xy+1$. First, note that $P(1,-1)\implies f(-1)[f(1)-1]=0$.

\begin{bolded}Case 1:\end{bolded} $f(1)=1$. Then
\[P(x,1)\implies f(x+1)=2x+1\implies f(x)=2x-1\]for all $x$.

\begin{bolded}Case 2:\end{bolded} $f(1)\ne1\implies f(-1)=0$. Note that $P(x,0)\implies f(0)=-1$ (otherwise, $f(x)$ is constant, which is clearly impossible since $2xy$ is surjective). Now let $c=1-f(1)$ (by assumption, $c\ne0$). Then
\[P(x,1)\implies f(x+1)=cf(x)+2x+1.\]The second order difference gives us
\[f(x+3)-(c+2)f(x+2)+(2c+1)f(x+1)-cf(x)=0\]for all $x$.

\begin{bolded}Subcase 2.1:\end{bolded} If $c=1$ (i.e. $f(1)=0$), so $f(x+1)=f(x)+2x+1$ for all $x$ and we find by simple induction that $f(n)=n^2-1$ and $f(x+n)=f(x)+(x+n)^2-x^2$ for all integers $n$, whence
\[P(x,\pm n)\implies f(-xn)=f(xn)+1=n^2[f(x)+1].\]Thus $f$ is even, and
\[P(x,x)\implies f(x^2)=[f(x)]^2+4f(x)-2x^2+2\]while
\[P(x,-x)\implies f(x^2)=f(-x^2)=[f(x)]^2+2x^2-2.\]Equating, we arrive at
\[f(x)=x^2-1,\]which indeed satisfies the original equation.

\begin{bolded}Subcase 2.2:\end{bolded} If $c\ne1$ (i.e. $f(1)\ne0$), then the characteristic polynomial of the sequence $f(n)$ has roots $1,1,c$ and
\[f(n)=\alpha+\beta n+\gamma c^n\]for all integers $n$ (for some real constants $\alpha,\beta,\gamma$). Considering $f(-1),f(0),f(1)$, we find that
\[0=\alpha-\beta+\frac{\gamma}{c},\quad -1=\alpha+\gamma,\quad 1-c=\alpha+\beta+\gamma c.\]Solving this system,
\[f(n)=-\frac{c+1}{(c-1)^2}+\frac{2n}{1-c}-\frac{c(c-3)}{(c-1)^2}c^n.\]Now,
\[P(2,2)\implies f(2)=\pm3.\]If $f(2)=3$, then
\[3=f(2)=cf(1)+2(1)+1=cf(1)+3\implies cf(1)=0.\]But we have $c\ne0$ and $f(1)\ne0$. So $f(2)=-3$, and
\[-3=f(2)=cf(1)+2(1)+1\implies [f(1)-3][f(1)+2]=0.\]
\begin{bolded}Subcase 2.2.1:\end{bolded} $f(1)=3\implies c=-2$. Then
\[f(n)=\frac{1}{9}+\frac{2n}{3}-\frac{10}{9}(-2)^n\]for all integers $n$, so $f(2)=-3$, $f(-2)=-3\/2$, and $f(-4)=-21\/8$. This contradicts $P(2,-2)\implies f(2)f(-2)=f(-2^2)$.

\begin{bolded}Subcase 2.2.2:\end{bolded} $f(1)=-2\implies c=3$. Then
\[f(n)=-1-n\]for all integers $n$. By induction, we find $f(x+n)=3^nf(x)+(3^n-1)x+(3^n-n-1)$. Thus
\[P(x,n)\implies f(xn)+xn+1=(3^n-n-1)[f(x)+x+1].\]So
\begin{align*}
(3^n-n-1)^2[f(x)+x+1]&=(3^n-n-1)[f(xn)+xn+1]\\
&=f(xn^2)+xn^2+1=(3^{n^2}-n^2-1)[f(x)+x+1]
\end{align*}for all real $x$ and integers $n$. Take $n=-1$. We find
\[f(x)=-x-1.\]
Finally, our solutions are $f(x)=2x-1\forall x$, $f(x)=x^2-1\forall x$, and $f(x)=-x-1\forall x$ (all three work).
\end{solution}



\begin{solution}[by \href{https://artofproblemsolving.com/community/user/29428}{pco}]
	\begin{tcolorbox}Find all functions $ f: \mathbb{R}\to\mathbb{R}$ such that $ f\left(x+y\right)+f\left(x\right)f\left(y\right)=f\left(xy\right)+2xy+1$ for all real numbers $ x$ and $ y$.\end{tcolorbox}
Let $P(x,y)$ be the assertion $f(x+y)+f(x)f(y)=f(xy)+2xy+1$
Let $f(0)=a$
Let $f(1)=b$
Let $f(-1)=c$

(e1) : $P(x,-1)$ $\implies$ $f(x-1)+cf(x)=f(-x)-2x+1$
(e2) : $P(x-1,1)$ $\implies$ $f(x)+(b-1)f(x-1)=2x-1$
(1-b)e1 + e2 $\implies$ $(1+c-bc)f(x)+(b-1)f(-x)=-2bx-b$
changing $x\to -x$, we get $(1+c-bc)f(-x)+(b-1)f(x)=2bx-b$

And so the system :
$(1+c-bc)f(x)+(b-1)f(-x)=-2bx-b$
$(b-1)f(x)+(1+c-bc)f(-x)=2bx-b$

If the determinant of the system is non zero, this gives $f(x)=ux+v$ for some real $u,v$ and plugging this in original equation, 
we get two solutions $\boxed{f(x)=-x-1}$ and $\boxed{f(x)=2x-1}$

If the determinant of the system is zero, this means :
either $1+c-bc=b-1$ and so $-2bx-b=2bx-b$ and so $b=0$
either $1+c-bc=1-b$ and so $-2bx-b=-(2bx-b)$ and so $b=0$

So $b=0$ and :
$P(0,1)$ $\implies$ $b+ab=a+1$ and so $a=-1$
$P(-1,1)$ $\implies$ $a+bc=c-1$ and so $c=0$
The system above becomes then $f(x)=f(-x)$ and the function is even.

$P(\frac x2,\frac x2)$ $\implies$ $f(x)+f(\frac x2)^2=f(\frac {x^2}4)+\frac{x^2}2+1$

$P(\frac x2,-\frac x2)$ $\implies$ $-1+f(\frac x2)^2=f(\frac {x^2}4)-\frac{x^2}2+1$

Subtracting these two lines implies $\boxed{f(x)=x^2-1}$ which indeed is a solution
\end{solution}



\begin{solution}[by \href{https://artofproblemsolving.com/community/user/105386}{Bertus}]
	Let $P(x,y)$ the assertion : $f(x+y)+f(x)f(y)=f(xy)+2xy+1$
$P(x,0) : (f(0)+1)(f(x)-1)=0$, we can see easily that the constant function $f \equiv 1$ is not a solution then $f(0)=-1$
$P(1,-1): f(-1)(f(1)-1)=0$ 
- If $f(1)=1$ then :
$P(x,1): f(x+1)=2x+1$ which give us $\forall x \in \mathbb{R} : f(x)=2x-1$ which is indeed a solution.
- If $f(-1)=0$ : 
$P(2,-1) : f(-2)=f(1)+3$
$P(-2,1): f(-2)(1-f(1))=3$ hence either $f(1)=0$ either $f(1)=2$
-- If $f(1)=0$
$P(x,1): f(x+1)=f(x)+2x+1$
$P(x,-1):f(x-1)=f(-x)-2x+1$
$P(-x,1):f(-x+1)=f(-x)-2x+1$ hence we get $f(-x+1)=f(x-1)$ which mean that f is even.
$P(x,-x): (f(x))^{2}=f(x^{2})+2-2x^{2}$
$P(x,x): f(2x)+(f(x))^{2}=f(x^{2})+2x^{2}+1$ hence $ \forall x \in \mathbb{R} : f(2x)=4x^{2}-1=(2x)^{2}-1$ hence we get $\forall x \in \mathbb{R}: f(x)=x^{2}-1$
-- If $f(1)=-2$:
$P(x,1):f(x+1)=3f(x)+2x+1$
Hence :$f(x+2)=9f(x)+8x+6 \forall x$ (*)
Otherwise : 
$P(x,-1): f(x-1)=f(-x)-2x+1$
$P(-x,1): f(-x+1)-2f(-x)=f(-x)-2x+1 \Rightarrow f(-x)=f(x-1)+2x-1=\frac{f(-x+1)+2x-1}{3}$
Hence : $f(-x+1)=3f(x-1)+4(x-1)+2$ which mean $\forall x \in \mathbb{R} f(-x)=3f(x)+4x+2$
$P(x,-x): -1+f(x)(3f(x)+4x+2)=3f(x^{2})+4x^{2}+2$ which is equivalent to : $f(x^{2}) = \frac{-3-4x^{2}+f(x)(3f(x)+4x+2)}{3}$ (1)

From (*) we get $f(2)=-9+6=-3$ 
$P(x,2) f(x+2)-3f(x)=f(2x)+4x+1 \Rightarrow f(2x)=6f(x)+4x+5$ (2)
$P(x,x) : f(2x)+(f(x))^{2}=f(x^{2})+4x^{2}+1$ (3)
Using (1) +(2) +(3) we get :
$ \forall x \in \mathbb {R} : f(x)=-x-1$
Finally the F.E have 3 solutions which are :
$\forall x \in \mathbb{R} : f(x)=-x-1 , f(x)=x^{2}-1, f(x)=2x-1$
\end{solution}



\begin{solution}[by \href{https://artofproblemsolving.com/community/user/187896}{Ashutoshmaths}]
	Let $P(x,y)$ be the assertion.
First see that no constant functions exist.
$P(x,0)\implies f(x)+f(x)f(0)=f(0)+1\implies (f(0)+1)(f(x)-1)=0$
As $f$ is non constant $\exists x$ such that $f(x)\neq 1\implies f(0)=-1$
$P(1,-1)\implies f(0)+f(1)f(-1)=f(-1)-1\implies f(-1)(f(1)-1)=0$
If $f(1)=1$
$P(x,1)\implies f(x+1)=2x+1\implies \boxed{f(x)=2x-1\text{ for all } x\in\mathbb{R}}$
If $f(-1)=0$.
$P(-1,-1)\implies f(-2)+f(-1)^2=f(1)+3\implies f(-2)=f(1)+3$.
$P(1,-2)\implies f(-1)+f(1)f(-2)=f(-2)-3\implies f(-2)(f(1)-1)=-3$
$\implies (f(1)+3)(f(1)-1)=-3$
$\implies f(1)^2+2f(1)=0\implies f(1)=0\text{ or } f(1)=-2$
$\text{ If }f(1)=0$
$P(x,1)\implies f(x+1)=f(x)+2x+1\implies f(x-1+1)=f(x-1)+2x-2+1$
$\implies f(x)=f(x-1)+2x-1$
$P(x,1)\implies f(x-1)=f(-x)-2x+1$
but we have just proved $f(x)=f(x-1)+2x-1$.
Comparing these two equations, we get $f(x)=f(-x)\text{ for all } x\in\mathbb{R}$
$P(x,-y)\implies f(x-y)+f(x)f(-y)=f(-x)-2xy+1\text{ but as }f(x)=f(-x)$
$f(x-y)=f(xy)-f(x)f(y)-2xy+1$
$P(x,y)\implies f(x+y)+f(x)f(y)=f(xy)+2xy+1$
Comparing these two equations, we get 
$f(x+y)-f(x-y)=4xy$
Take $x=y$ to get $f(2x)=4x^2-1$, 
so, $x\rightarrow \frac{x}{2}\implies \boxed{f(x)=x^2-1\text{ for all } x\in\mathbb{R}}\text{ is another solution}$
If $f(1)=-2$
$P(x,1)\implies f(x+1)=3f(x)+2x+1\implies f((x-1)+1)=3f(x-1)+2x-1\implies f(x)=3f(x-1)+2x-1$
$P(x,-1)\implies f(x-1)=f(-x)-2x+1$
$\implies 3(f(x-1))=3f(-x)-6x+3=f(x)-2x+1$
$\implies f(x)+4x=3f(-x)+2$
Now $x\rightarrow -x\implies f(-x)-4x=3f(x)+2$
Adding these two, we get, $f(x)+f(-x)=-2$
$P(x,-y)\implies f(x-y)+f(x)f(-y)=f(-xy)-2xy+1$
$\implies f(x-y)+f(x)\left[-2-f(y)\right]=-2-f(xy)-2xy+1$
$\implies f(x-y)-2f(x)+2xy+1=f(x)f(y)-f(xy)$
But $P(x,y)\implies f(x)f(y)-f(xy)=2xy+1-f(x+y)$
Therefore, $2xy+1-f(x+y)=f(x-y)-2f(x)+2xy+1$
$\implies f(x-y)+f(x+y)=2f(x)$
$x=y\implies -1+f(2x)=2f(x)\implies f(2x)-2f(x)=1$
We have already proved $f(x)=3f(x-1)+2x-1\stackrel{x=-1}{\implies} f(-2)=1$
As $f(x)+f(-x)=-2\implies f(-2)=3$
$P(x,2)\implies f(x+2)+f(x).f(2)=f(2x)+4x+1\implies f(2x)=f(x+2)-3f(x)-4x-1$
We have already proved 
$f(x+1)=3f(x)+2x+1$
$\implies f((x+1)+1)=3f(x+1)+2x+3=9f(x)+6x+3+2x+3$
$\implies f(x+2)=9f(x)+8x+6$
Therefore $9f(x)+8x+6-3f(x)-4x-1=f(2x)$
But $f(2x)=1+2f(x)\implies 6f(x)+4x+5-2f(x)=1\implies 4f(x)=-4x-4$
$\implies \boxed{f(x)=-x-1\text{ for all }x\in\mathbb{R}}$
\end{solution}



\begin{solution}[by \href{https://artofproblemsolving.com/community/user/148207}{Particle}]
	[hide="Solution"]Easy to get $f(0)=-1$ and $(f(1)-1)f(-1)=0$. So if $f(1)=1$, then $P(x-1,1)\implies f(x)=2x-1$ which is definitely a solution. So assume $f(-1)=0$. Let $f(1)=c$.

$P(-x+1,-1)\implies f(-x)-f(x-1)=2x-1$
$P(x-1,1)\implies f(x)+cf(x-1)-f(x-1)=2x-1$
Combining these we get $f(-x)=f(x)+cf(x-1)\quad (1)$
So $f(x)=f(-x)+cf(-x-1)$
Adding them gives $c[f(x-1)+f(-x-1)]=0$ So at least one of them is zero. Suppose $c=0$. Then (1) yields $f(x)=f(-x)$. Now subtracting $P(x,-x)$ from $P(x,x)$ gives $f(2x)=4x^2-1\implies f(x)=x^2-1$ which is another valid solution.

Now assume $f(x-1)+f(-x-1)=0\implies f(x)+f(-x-2)=0$. So $f(-2)=-f(0)=1$. And now $P(-1,-1)$ implies $f(1)=-2$.
$P(x,-1)\implies f(x-1)=f(-x)-2x+1$
$P(x,1)\implies f(x+1)=3f(x)+2x+1$
So $-f(x)=f(-x-2)=f(x+1)+2x+3=3f(x)+4x+2$ $\implies f(x)=-(x+1)$

So all the functions are $f(x)=2x-1,x^2-1,-x-1$[\/hide]
\end{solution}



\begin{solution}[by \href{https://artofproblemsolving.com/community/user/175572}{Goblik}]
	Why $f(x)=1$ for all $x$ don't satisfy in this problem?
\end{solution}



\begin{solution}[by \href{https://artofproblemsolving.com/community/user/29428}{pco}]
	\begin{tcolorbox}Why $f(x)=1$ for all $x$ don't satisfy in this problem?\end{tcolorbox}
Because with $f(x)=1$ $\forall x$ :

$LHS$ of functional equation is $2$
$RHS$ of functional equation is $2+2xy$

And obviously we dont have $LHS=RHS$ $\forall x,y$
\end{solution}



\begin{solution}[by \href{https://artofproblemsolving.com/community/user/160913}{fireclaw105}]
	I hate casework.
[hide=Solution]
The functions are $2x-1, x^2-1,$ and $-x-1$.

Let $P(x, y)$ be the assertion that $f(x+y) + f(x)f(y) = f(xy) + 2xy + 1$. 
$P(0, 0)$ yields that $f(0)^2 = 1$, so $f(0) = 1$ or $f(0) = -1$. However, if $f(0) = 1$, then $P(x, 0)$ tells us that $f(x)$ is a constant function, which is clearly false. Thus, $f(0) = -1$. $P(1, -1)$ gives us $f(1)f(-1) = f(-1)$, so $f(1) = 1$ or $f(-1) = 0$. 

\begin{bolded}Case 1: $f(1) = 1$\end{bolded}
$P(x - 1, 1)$ shows that $f(x) = 2x - 1$.

\begin{bolded}Case 2:$f(-1) = 0$\end{bolded}
$P(-2, 1)$ shows that $f(-2)f(1) = f(-2) - 3$ and $P(-1, -1)$ implies $f(-2) = f(1) + 3$. Putting together these two observations, we see that $f(1) = f(1)f(-2)$, so $f(1) = 0$ or $f(-2) = 1$.

\begin{bolded}Subcase 1:$f(1) = 0$\end{bolded}
$P(x - 1, 1)$ and $P(-x + 1, -1)$ tell us that $f(x) = f(-x)$. $P(-x, x)$ implies $-1 + f(-x)f(x) = f(-x^2) - 2x^2 + 1$, and substituting $f(x)$ for $f(-x)$ and $f(x^2)$ for $f(-x^2)$ gives us $f(x)^2 = f(x^2) - 2x^2 + 2$. $P(x, x)$ shows that $f(2x) + f(x)^2 = f(x^2) + 2x + 1$. Subtracting this equation from the previous one, we get $f(2x) = 4x^2 - 1$, or $f(x) = x^2 - 1$.

\begin{bolded}Subcase 2: $f(-2) = 1$\end{bolded}
We can easily show that $f(-2) = 1$ implies that $f(1) = -2$. 
$P(-x, -1)$ gives us 
$f(-x-1) = f(x) + 2x + 1$. (1)
 $P(x, 1)$ gives us 
$f(x+1) = 3f(x) + 2x + 1$.      (2)
Subtracting, we get
$f(x+1) - f(-x-1) = 2f(x)$, or equivalently $f(x) - f(-x) = 2f(x-1)$. Substituting $x$ for $-x$, we can easily see that $f(x-1) = -f(-x-1)$. (3)
Substituting $x-1$ in (2), multiply (1) by 3, and subtracting those two equations, we get $f(-x-1) = 3f(x-1) + 4x$. Using (3), we immediately get $f(-x-1) = x$, or $f(x) = -x - 1$.

All those functions work, so QED.
[\/hide]
\end{solution}



\begin{solution}[by \href{https://artofproblemsolving.com/community/user/298310}{Blackpanther}]
	very beatiful function.
\end{solution}



\begin{solution}[by \href{https://artofproblemsolving.com/community/user/194761}{raxu}]
	This problem took a really long while for me... Well, here's my solution. Feedback would be greatly appreciated - it is loooong (4 o's to indicate it's pretty long, but not insane).
[hide="Spoilers!"]
We claim that the only possible functions $f$ are $f(x)=2x-1, f(x)=-x-1, f(x)=x^2-1$. It is easy to see that these values satisfy the problem condition.

Using $x=0, y\in\mathbb{R}$ is any real number, we have $f(y)(f(0)+1)=f(0)+1$.

If $f(0)\neq -1, f(y)=1$ for all real number $y$. Choosing $x=y=10$ gives us $1+1=1+200+1$, which is clearly not valid.
Therefore, $f(0)=-1$.

Using $x=1, y=-1$, we have $f(0)+f(1)f(-1)=f(-1)-2+1, f(-1)(f(1)-1)=0$. Therefore, $f(1)=1$ or $f(-1)=0$.

If $f(1)=1$, using $y=1$ gives us $f(x+1)+f(x)f(1)=f(x)+2x+1, f(x+1)=2x+1$ for all real $x$. Therefore, $f(x)=2x-1$ for all real $x$.

If $f(1)\neq 1$, $f(-1)=0$ and denote $c=f(1)$.
Using $y=1$, we have $f(x+1)+f(x)c=f(x)+2x+1$, or $f(x+1)=(1-c)f(x)+2x+1(\clubsuit)$ for all real $x$.
Using $y=-1$, we have  $f(x-1)+f(x)f(-1)=f(-x)-2x+1, f(x-1)=f(-x)-2x+1$ for all real $x$.
Therefore, $f(-x-1)=f(x)+2x+1(\star)$ for all real $x$.

Multiplying $(\star)$ by $1-c$ and subtract $(\clubsuit)$gives us $(1-c)f(-x-1)-f(x+1)=(2x+1)(-c)$ for any real $x$. Therefore, we also have $(1-c)f(x)-f(-x)=(2x-1)(-c)$, $(1-c)f(-x)-f(x)=(-2x-1)(-c)$ for all real $x$.

If $c\neq 0$, Solving for $f(x)$, we get that $f(x)=\frac{-2xc^2}{c^2-2c}-1$ for all real $x$. Using $x=1$, we get $c(c+2)(c-1)=0$. Since $c\neq 0,1$, we get $f(x)=-x-1$.

If $c=0$, then our equation gives us $f(x)=f(-x)$. Using $y=x$, we have $f(2x)+f(x)^2=f(x^2)+2x^2+1$. Furthermore, using $y=-x$, we have $f(0)+f(x)f(-x)=f(-x^2)-2x^2+1$.
Using $f(x)=f(-x)$ and subtracting, we have $f(2x)=4x^2-1$ for all real $x$. Therefore, $f(x)=x^2-1$.

Having exhausted all possible value of $f(1)=c$, the only possible functions $f$ are $f(x)=x^2-1, 2x-1, -x-1$, as desired.
[\/hide]
\end{solution}



\begin{solution}[by \href{https://artofproblemsolving.com/community/user/243741}{anantmudgal09}]
	\begin{tcolorbox}Find all functions $ f: \mathbb{R}\to\mathbb{R}$ such that $ f(x+y)+f(x)f(y)=f(xy)+2xy+1$ for all real numbers $ x$ and $ y$.

\begin{italicized}Proposed by B.J. Venkatachala, India\end{italicized}\end{tcolorbox}

[hide=solution] Answer: The only functions which work are $f(x)=2x-1$ for all $x$, $f(x)=x^2-1$ for all $x$ and $f(x)=-(x+1)$ for all $x$. It is clear that they indeed satisfy the desired conditions.

Let $P(x,y)$ denote the given assertion and $f$ be a function which works. Let $h(x)=f(x)-x^2+1$ and $g(x)=f(x)+x+1$. I shall use them later on.

Note that $P(0,0) \Longrightarrow f(0)=\pm 1$ so if $P(0)=1$ then $P(x, 0) \Longrightarrow f(x)=1$ for all $x$ which is not a valid function. So $f(0)=-1$. Observe $$P(x, 1) \Longrightarrow f(x+1)+f(x)\cdot \left(f(1)-1\right)=2x+1$$ and $P(2, 2) \Longrightarrow f(2)=\pm 3$. Henceforth, $P(x, 1)$ is called the $(+)$. Plugging $x=1$ in $(+)$ yields $$f(1)(f(1)-1)=3-f(2) \in \{0, 6\} \Longrightarrow f(1) \in \{0, 1, -2, 3\}.$$ Consider the four cases for different values of $f(1)$ as follow. 

Firstly, the easier ones.

\begin{bolded}Case 1.\end{bolded} $f(1)=1$

Our equation $(+)$ becomes $f(x+1)=2x+1 \Longrightarrow f(x)=2x-1$ for all reals $x$ which is one of the claimed solutions.

\begin{bolded}Case 2.\end{bolded} $f(1)=3$

Put $x=-1$ in $(+)$ to get $f(-1)\cdot \left(f(1)-1\right)=0 \Longrightarrow f(-1)=0$. Observe $P(-1, -1) \Longrightarrow f(-2)=f(1)+3 \Longrightarrow f(-2)=6$ but $x=-2$ in $(+)$ gives $f(-2)\cdot \left(f(1)-1\right)=-3 \Longrightarrow f(-2)=-1.5$ so we have a contradiction! 

Harder cases incoming...

\begin{bolded}Case 3.\end{bolded} $f(1)=0$

Notice that $P(x, y)$ is equivalent to the assertion $$Q(x, y) \overset{\text{def}}{:=}h(xy)=h(x+y)+h(x)h(y)+(x^2-1)h(y)+(y^2-1)h(x). $$ Equation $(+)$ boils down to $f(1+x)-f(x)=2x+1 \Longrightarrow h(1+x)=h(x)$ for all reals $x$. Notice that $$Q(1+x, y)-Q(x, y) \Longrightarrow h(xy+y)-h(xy)=(2x+1)h(y).$$ Plugging $x=\frac{1}{y}$ for non zero $y$ in the last equation gives $\frac{h(y)}{y}=0$ so $h \equiv 0$ and we get $f(x)=x^2-1$ for all reals $x$, which is again one of the stated solutions.

\begin{bolded}Case 4.\end{bolded} $f(1)=-2$

Equation $(+)$ is equivalent to $g(1+x)=3g(x)$ and the assertion $P(x, y)$ is equivalent to $$R(x, y) \overset{\text{def}}{:=} g(x+y)+g(x)g(y)=g(xy)+(1+x)g(y)+(1+y)g(x).$$ Evidently, $$R(1+x, y)-3R(x, y) \Longrightarrow g(xy+y)-3g(xy)=(2x+1)g(y).$$ Plugging $x=\frac{1}{y}$ we conclude $2\left(\frac{1}{y}-1\right)g(y)=0$ so $g(y)=0$ for all $y \ne 1$, which combined with $g(1)=0$ yield $g \equiv 0$ or $f(x)=-(x+1)$ for all reals $x$, another one of the mentioned solutions.

By our rather long analysis, it is clear that these are the only valid functions. 

End of Story.

[\/hide]
\end{solution}



\begin{solution}[by \href{https://artofproblemsolving.com/community/user/303386}{AlgebraFC}]
	[hide=Solution]
Answer: $f(x)=2x-1, f(x)=x^2-1,$ and $f(x)=-1-x$ are the three solutions.

Let $P(x, y)$ denote the assertion that \[f(x+y)+f(x)f(y)=f(xy)+2xy+1\] for all real numbers $x, y$. Then $P(0, 0)\implies f(0)^2=1\implies f(0)=1 \ \text{or} \ f(0)=-1$.

\begin{bolded}Case 1\end{bolded}: $f(0)=1$. Then $P(x, 0)\implies 2f(x)=2\implies f(x)=1$, which clearly doesn't satisfy the original FE.

\begin{bolded}Case 2\end{bolded}: $f(0)=-1$. Then \begin{align*}P(1, -1)&\implies -1+f(1)f(-1)=f(-1)-1 \\ &\implies f(-1)=0 \ \text{or} \ f(1)=1.\end{align*}
[list]
[*]\begin{bolded}Subcase 2.1\end{bolded}: $f(1)=1$. Then \begin{align*}P(x, 1)&\implies f(x+1)+f(x)=f(x)+2x+1) \\ &\implies f(x+1)=2x+1 \\ &\implies f(x)=2x-1.\end{align*} We can easily check that $\boxed{f(x)=2x-1}$ indeed satisfies the original equation, as \[f(x+y)+f(x)f(y)=2x+2y-1+4xy-2x-2y+1=f(xy)+2xy+1.\]
[*]\begin{bolded}Subcase 2.2\end{bolded}: $f(-1)=0$. Let $g(x)=f(x)+1$, and we have that $P(x, y)$ is equivalent to the assertion \[Q(x, y):=g(x+y)+g(x)g(y)-g(x)-g(y)=g(xy)+2xy\] for all reals $x, y$. Note that $g(0)=0$ and $g(-1)=1$. 

$Q(-1, -1)$ gives $g(-2)=g(1)+3$, and $Q(1, -2)$ gives \[1+g(1)g(-2)-g(1)-g(-2)=g(-2)-4 \implies g(1)=1 \ \text{or} \ -1\] after substituting $g(-2)=g(1)+3$.
[list]
[*]\begin{bolded}Subcase 2.2.1\end{bolded}: $g(1)=1$. Then $Q(x, 1)\implies g(x+1)-1=g(x)+2x$. We also have that $Q(-x, -1)\implies g(-x-1)-1=g(x)+2x$, so $g(x+1)=g(-(x+1))$ for all $x$, and $g$ is even. But \begin{align*}Q(x, -x)&\implies g(x^2)-2g(x)=g(x^2)-2x^2 \\ &\implies g(x)=x^2.\end{align*} Because $f(x)=g(x)-1$, the $g(x)=x^2$ corresponds to $\boxed{f(x)=x^2-1}$ for all real $x$. This can indeed be checked to satisfy the original FE, as \[f(x+y)+f(x)f(y)=x^2+2xy+y^2+(xy)^2-x^2-y^2+1=f(xy)+2xy+1.\]
[*]\begin{bolded}Subcase 2.2.2\end{bolded}: $g(1)=-1$. Then $Q(x, 1)\implies g(x+1)=3g(x)+2x-1 \ \ \ (*)$. Plugging $x=-\tfrac{1}{2}$ into $(*)$ gives \[g\left(\tfrac{1}{2}\right)=3g\left(-\tfrac{1}{2}\right)-2. \ \ \ (1)\] Replacing $x$ with $2x-\tfrac{1}{2}$ in $(*)$ gives \[g\left(2x+\tfrac{1}{2}\right)=3g\left(2x-\tfrac{1}{2}\right)+4x-2. \ \ \ (2)\] Write \begin{eqnarray*}Q\left(2x, -\tfrac{1}{2}\right)\implies g\left(2x-\tfrac{1}{2}\right)+g(2x)g\left(-\tfrac{1}{2}\right)-g(2x)-g\left(-\tfrac{1}{2}\right)=g(-x)-2x\end{eqnarray*} and then use $(1), (2)$ to get \begin{align*}Q\left(2x, \tfrac{1}{2}\right)&\implies g\left(2x+\tfrac{1}{2}\right)+g(2x)g\left(\tfrac{1}{2}\right)-g(2x)-g\left(\tfrac{1}{2}\right)=g(x)+2x \\ &\implies 3\left[g\left(2x-\tfrac{1}{2}\right)+g(2x)g\left(-\tfrac{1}{2}\right)-g(2x)-g\left(-\tfrac{1}{2}\right)\right]=g(x)-2x \\ &\implies 3\left[g(-x)-2x\right] = g(x)-2x\ \\ &\implies 3g(-x) = g(x) + 4x. \ \ \ (3)\end{align*} Replacing $x$ with $-x$ in $3g(-x)=g(x)+4x$ yields \[3g(x)=g(-x)-4x, \ \ \ (4)\] and solving for $g(x)$ in $(3)$ and $(4)$ returns $g(x)=-x$. The corresponding function in $f$ for $g(x)=-x$ is $\boxed{f(x)=-1-x}$, and it is easy to check that it works: \[f(x+y)+f(x)f(y)=-1-x-y+xy+x+y+1=f(xy)+2xy+1.\] 

[\/list]
[\/list]
---------
We are done, having exhausted all cases.
[\/hide]
\end{solution}



\begin{solution}[by \href{https://artofproblemsolving.com/community/user/274926}{rmtf1111}]
	Let $P(x,y)$ denote the assertion that $ f(x+y)+f(x)f(y)=f(xy)+2xy+1$.
$$P(x,0) \implies (1+f(0))(f(x))=1+f(0) \implies f(0)=-1$$
If $f(1)=1$, then $P(x-1,1) \implies f(x)=2x-1$. Now suppose that $f(1) \neq 1$ and let $c=f(1)$.
$$P(1,-1) \implies f(1)f(-1)=f(-1) \implies f(-1)=0$$
$$P(x,-1) \implies f(x-1)=f(-x)-2x+1$$
$$P(x-1,1) \implies f(x)+cf(x-1)=f(x-1)+2x-1 \implies f(x)+c(f(-x)-2x+1)=f(-x) \implies (c-1)f(-x)=2cx-c-f(x) \ \ (Q(x))$$
$$Q(-1) \implies c(c-1)=-3c \implies c=0 \  \text{or} \ c=-2$$
Let's check the case $c=-2$.
$$(c-1)Q(-x)-Q(x) \implies (c-1)f(-x)=-(c-1)(2cx+c+(c-1)f(x))=2cx-c-f(x) \implies 3(-4x-2-3f(x))=-4x+2-f(x) \implies f(x)=-x-1$$
Now let's check the case when $c=0$. Note that $Q(x) \implies f(x)=f(-x)$
$$P(x,x)-P(x,-x) \implies f(2x)+1=4x^2 \implies f(x)=x^2-1$$
To conclude, the only solutions are $f(x)=x^2-1$ , $f(x)=-x-1$ and $f(x)=2x-1$.
\end{solution}



\begin{solution}[by \href{https://artofproblemsolving.com/community/user/403767}{mkhayech}]
	1. y=1
2. y=y+1 (expand)
3. use symmetry
Done. That easy (try it)
\end{solution}
*******************************************************************************
-------------------------------------------------------------------------------

\begin{problem}[Posted by \href{https://artofproblemsolving.com/community/user/18812}{pohoatza}]
	Prove that there is no bijective function $f : \left\{1,2,3,\ldots \right\}\rightarrow \left\{0,1,2,3,\ldots \right\}$ such that $f(mn)=f(m)+f(n)+3f(m)f(n)$.
	\flushright \href{https://artofproblemsolving.com/community/c6h145400}{(Link to AoPS)}
\end{problem}



\begin{solution}[by \href{https://artofproblemsolving.com/community/user/16261}{Rust}]
	Let $g(n)=3f(n)+1$, then $g(mn)=g(m)g(n)$ (g is strongly multilikative).
If f is bijective, then $g: N\to 3N+1$ too bijective.
Let $q_{1},q_{2},q_{3},q_{4}$ primes form $q_{i}=2(mod \ 3).$
Then exist distinct primes $p_{1},p_{2},p_{3},p_{4}$, suth that 
$g(p_{1})=q_{1}q_{2},g(p_{2})=q_{3}q_{4},g(p_{3})=q_{1}q_{3},g(p_{4})=q_{2}q_{4}.$
It give contradition with bijective, because $g(p_{1}p_{2})=g(p_{3})g(p_{4}).$
\end{solution}
*******************************************************************************
-------------------------------------------------------------------------------

\begin{problem}[Posted by \href{https://artofproblemsolving.com/community/user/3272}{freemind}]
	Prove that every bijective function $ f: \mathbb{Z}\rightarrow\mathbb{Z}$ can be written in the way $ f=u+v$ where $ u,v: \mathbb{Z}\rightarrow\mathbb{Z}$ are bijective functions.
	\flushright \href{https://artofproblemsolving.com/community/c6h187664}{(Link to AoPS)}
\end{problem}



\begin{solution}[by \href{https://artofproblemsolving.com/community/user/15035}{Ilthigore}]
	I vaguely remember having seen this problem somewhere else, but not vividly enough to prevent me spending a while finding the solution.

We can wlog let $ f(n)=n$ (by permuting the arguments of u and v accordingly).

I claim the following algorithm generates bijections u and v summing to identity, where I make both odd functions (u(-x)=-u(x) etc) and u(0)=v(0)=0.

For k, starting at k=0 and continuing upwards, let $ u(2k+1)=-m, v(2k+2)=-l$ where $ m$ is the least positive integer such that $ u(x)$ does not already contain $ m$ in its image between $ -2k$ and $ 2k$, and similar for $ v,l$, and correspondingly let $ v(2k+1)=2k+1+m, u(2k+2)=2k+2+l$

Suppose this algorithm fails to generate a bijection. It clearly can exclude no integer from its image, because it specifically demands inclusion of all integers not yet in its image, so it will only break if injectivity fails. Suppose k is the least integer for which we obtain u(x)=u(y) or v(x)=v(y) for |x|,|y|<=2(k+1).

If u(x)=u(y) then 2k+2+l = u(z) for |z|<2k+2. z must be odd, for obvious reasons (if it's even, l hasn't been chosen correctly). This would imply that the image of u(x) as x varies on -(|z|-1),|z|-1 must contain all the positive integers between 1 and 2k+2+l. This is clearly absurd, since |z|-1<2k+1<2k+2+l.

We get similar contradictions in the v(x)=v(y) case, and thus conclude that u,v are both bijections, as required.
\end{solution}



\begin{solution}[by \href{https://artofproblemsolving.com/community/user/9336}{cadge_nottosh}]
	I think we can do even less work.  
Draw a table containing three rows.  The first row contains the integers (in, say, the order 0,1,-1,2,-2,...), while the second and third rows shall contain the values of u(n) and v(n) respectively (n being the value in the first row).
We repeat the following process to fill in all the columns:

i)look at the first column which we have not yet filled, with first row n.  Choose u(n) and v(n) to have absolute value larger than anything we have yet chosen so that they sum to f(n).

ii)look at the first number, k, which we have not yet chosen as a value u(i).  Choose n such that f(n)-k has not yet been chosen as a value for v, and fill in the n-column accordingly.

iii)repeat ii with u and v reversed.

Cycling through i, ii and iii generates bijections u, v with u+v=f

Jack
\end{solution}



\begin{solution}[by \href{https://artofproblemsolving.com/community/user/335829}{vjdjmathaddict}]
	[size=100]is this trivial :maybe:  simply define v=f(f)+f and u=-f(f).[\/size]
\end{solution}



\begin{solution}[by \href{https://artofproblemsolving.com/community/user/209049}{math90}]
	They are injective, but not surjective.
\end{solution}
*******************************************************************************
-------------------------------------------------------------------------------

\begin{problem}[Posted by \href{https://artofproblemsolving.com/community/user/1991}{orl}]
	Let $ a \in \mathbb{R}, 0 < a < 1,$ and $ f$ a continuous function on $ [0, 1]$ satisfying $ f(0) = 0, f(1) = 1,$ and

\[ f \left( \frac{x+y}{2} \right) = (1-a) f(x) + a f(y) \quad \forall x,y \in [0,1] \text{ with } x \leq y.\]

Determine $ f \left( \frac{1}{7} \right).$
	\flushright \href{https://artofproblemsolving.com/community/c6h226724}{(Link to AoPS)}
\end{problem}



\begin{solution}[by \href{https://artofproblemsolving.com/community/user/16261}{Rust}]
	If $ a\not =\frac 12$ function is not continiosly. If $ a=\frac 12$, then for continiosly function $ f(x)=x$.
\end{solution}



\begin{solution}[by \href{https://artofproblemsolving.com/community/user/40426}{reza1370}]
	let $ f(\frac {1}{7})=k$.
$ x=0,y=\frac {2}{7}$ then $ f(\frac {1}{7})=(1-a)f(0)+af(\frac {2}{7})$ so $ f(\frac {2}{7})=\frac {k}{a} (1)$.
$ x=\frac {1}{7},y=1$ then $ f(\frac {4}{7})=(1-a)k+a (2)$.
$ x=\frac {2}{7},y=\frac {4}{7}$ then with $ (1),(2)$:
$ f(\frac {3}{7})=\frac {1-a}{a}.k+((1-a)k+a).a (3)$.
$ x=0,y=1$ then $ f(\frac {1}{2})=a (4)$.
$ x=\frac {3}{7},y=\frac {4}{7}$ then with $ (2),(3),(4)$:
$ a=(1-a).(\frac {1-a}{a}.k+((1-a)k+a).a)+a.((1-a)k+a)$
so
$ k=\frac {a-a^3}{a^3+2a^2-a+1}$.
\end{solution}



\begin{solution}[by \href{https://artofproblemsolving.com/community/user/40922}{mehdi cherif}]
	since the left part of the equation is symmetric , so the right part must be symmetric too : 

then :

$ (1-a)f(x)+af(y)=af(x)+(1-a)f(y)$

$ \longrightarrow$   $ a=1\/2$  or $ f(x)=f(y)$

then $ f(x)=x$

:)
\end{solution}



\begin{solution}[by \href{https://artofproblemsolving.com/community/user/29126}{MellowMelon}]
	You dropped the assumption $ x \leq y$ by switching the variables.
\end{solution}



\begin{solution}[by \href{https://artofproblemsolving.com/community/user/29145}{Aneo.}]
	My solution and answer are different than those previously mentioned, so hopefully if I have an inaccuracy someone can tell me what I did wrong.
[hide]
We first write the given equation as:
\[ f(x + y) = (1 - a) f(2x) + a f(2y)\]
with $ x,y\in [0,\frac {1}{2}]$. With $ x = 0$, we have $ f(x) = af(2x)$. From here, we can easily show inductively that $ f(\frac {1}{2^k}) = a^k$. We can again rewrite the given equation again as:
\[ f(x + y) = \frac {(1 - a)}{a} f(x) + f(y)\]
with the same previous constraints. Now, consider the sequence $ s_1 = 1\/8$, $ s_{k + 1} = s_{k} + \frac {1}{8^{k + 1}}$. We make two observations: Firstly, since $ {s_n}$ is a geometric series, that $ \lim_{k\rightarrow \infty}s_k = \frac {1}{7}$. Also, since $ s_i > 0$ for all $ i > 0$, we have:
\[ s_k = s_{k - 1} + \frac {1}{8^k} > \frac {1}{8^k} > \frac {1}{8^{k + 1}}\]
We now have:
\[ f(s_{k + 1}) = f(s_k + \frac {1}{8^{k + 1}}) = (\frac {1 - a}{a})f(\frac {1}{8^{k + 1}}) + f(s_k)\]
Since $ f(\frac {1}{2^k}) = a^k,$ we have $ f(\frac {1}{8^{k + 1}}) = a^{3k + 3}$, and consequentially:
\[ f(s_{k + 1}) = a^{3k + 2}\cdot (1 - a) + f(s_k)\]
Now, we have:
\[ f(s_{k + 1}) - f(s_1) = \sum_{i = 1}^k f(s_{i + 1}) - f(s_i) = \sum_{i = 1}^k a^{3i + 2}\cdot (1 - a)\]

\[ = (1 - a)a^5 \cdot (\frac {1 - a^{3k}}{1 - a^3}) = (1 - a^{3k})\cdot(\frac {a^5}{a^2 + a + 1})\]
Hence,
\[ f(s_{k + 1}) = (1 - a^{3k})\cdot(\frac {a^5}{a^2 + a + 1}) + f(1\/8) = (1 - a^{3k})\cdot(\frac {a^5}{a^2 + a + 1}) + a^3\]
Finally, we have:
\[ f(\frac {1}{7}) = f(\lim_{k\rightarrow \infty} s_k) = \lim_{k\rightarrow \infty} f(s_k) = \lim_{k\rightarrow \infty} (1 - a^{3k - 3})\cdot(\frac {a^5}{a^2 + a + 1}) + a^3\]

\[ = \frac {2a^5 + a^4 + a^3}{a^2 + a + 1}\]
since $ f$ is continuous. 


[\/hide]
\end{solution}



\begin{solution}[by \href{https://artofproblemsolving.com/community/user/16261}{Rust}]
	\begin{tcolorbox}If $ a\not = \frac 12$ function is not continiosly. If $ a = \frac 12$, then for continiosly function $ f(x) = x$.\end{tcolorbox}
I repeat my answer.
\end{solution}



\begin{solution}[by \href{https://artofproblemsolving.com/community/user/29145}{Aneo.}]
	Why is $ f$ not continuous if $ a\neq\frac{1}{2}$?
\end{solution}



\begin{solution}[by \href{https://artofproblemsolving.com/community/user/16261}{Rust}]
	I give prove in this forum for function
$ f(bx+(1-b)y)=af(x)+(1-a)f(y)$ if f is not const and $ a\not =b$, then f can not be continiosly.
\end{solution}



\begin{solution}[by \href{https://artofproblemsolving.com/community/user/66698}{Vincent Gilbert}]
	\begin{tcolorbox}Let $ a \in \mathbb{R}, 0 < a < 1,$ and $ f$ a continuous function on $ [0, 1]$ satisfying $ f(0) = 0, f(1) = 1,$ and
\[ f \left( \frac {x + y}{2} \right) = (1 - a) f(x) + a f(y) \quad \forall x,y \in [0,1] \text{ with } x \leq y.\]
Determine $ f \left( \frac {1}{7} \right).$\end{tcolorbox}
$ P(0,x)\Rightarrow  f( \frac{x}{2} )=af(x)$ 
$ P(2x,4x) \Rightarrow  af(3x)=(1-a)af(2x)+a^2 f(4x)=(2-a) f(x)$
$ P(x,3x) \Rightarrow  f(x)=af(2x)=(1-a)af(x)+ a^2 f(3x)$
$ =(1-a)af(x)+a(2-a) f(x)=a(3-2a)f(x)$
$ \Leftrightarrow f(x)(2a-1)(a-1)=0$
It gives that if $ a \ne \frac{1}{2}$ then  $ f(x)=0 \forall  0 \le x \le \frac{1}{4}$ which is wrong because :$ f(\frac{1}{4})=a^2 f(1)=a^2 \ne 0$
$ \Rightarrow a=\frac{1}{2}$
$ \Rightarrow f(\frac{1}{7})=\frac{1}{7}$
Have I gotten mistake somewhere ?

To rust: Does your problem have some condition as orl's ?
\end{solution}



\begin{solution}[by \href{https://artofproblemsolving.com/community/user/16261}{Rust}]
	\begin{tcolorbox}[quote="orl"]
Have I gotten mistake somewhere ?

To rust: Does your problem have some condition as orl's ?\end{tcolorbox}\end{tcolorbox}
No, have not mistake. $ a=1\to f=const$.
Posted before similar problem and I prove if $ f(bx+(1-b)y)=af(x)+(1-a)f(y), 0<a,b<1$ and $ f\not =const$, then $ f$ can not be continiosly, when $ a\not =b$.
\end{solution}



\begin{solution}[by \href{https://artofproblemsolving.com/community/user/66846}{anirudh215}]
	\begin{tcolorbox}I give prove in this forum for function
$ f(bx + (1 - b)y) = af(x) + (1 - a)f(y)$ if f is not const and $ a\not = b$, then f can not be continiosly.\end{tcolorbox}

I don't get this. Can you elaborate a little more?
\end{solution}



\begin{solution}[by \href{https://artofproblemsolving.com/community/user/109704}{dien9c}]
	\begin{tcolorbox}Let $ a \in \mathbb{R}, 0 < a < 1,$ and $ f$ a continuous function on $ [0, 1]$ satisfying $ f(0) = 0, f(1) = 1,$ and

\[ f ( \frac{x+y}{2} ) = (1-a) f(x) + a f(y) \quad \forall x,y \in [0,1] \text{ with } x \leq y.\]

Determine $ f ( \frac{1}{7} ).$\end{tcolorbox}
\begin{bolded}The general problem \end{bolded}
Let $f:[ 0;1 ]\to \mathbb{R}$ such that $f$ continouous on $[ 0,1 ]$
i) $f(0)=0,\,\,f(1)=1$
ii) $f( \frac{mx+ny}{m+n} )=\alpha f( x )+( 1-\alpha  )f( y )$ when $x\ge y,\,\,x,y\in [ 0,1 ]$ and $\alpha \in ( 0;1 )$.
Find value of $f( \frac{{{m}^{3}}}{{{( m+n )}^{3}}-n{{m}^{2}}} )$

\begin{bolded}Solution.\end{bolded}
(Note: We don’t need the continuous condition)
We have 
$ f( \frac{{{m}^{3}}}{{{( m+n )}^{3}}-n{{m}^{2}}} )
=f( \frac{m.\frac{{{m}^{2}}( m+n )}{{{( m+n )}^{3}}-n{{m}^{2}}}+n.0}{m+n} ) $
$=\alpha .f( \frac{{{m}^{2}}( m+n )}{{{( m+n )}^{3}}-n{{m}^{2}}} )+( 1-\alpha  )f( 0 )  $
$=\alpha .f( \frac{{{m}^{2}}( m+n )}{{{( m+n )}^{3}}-n{{m}^{2}}} ) 
$
Similary
$f( \frac{{{m}^{2}}( m+n )}{{{( m+n )}^{3}}-n{{m}^{2}}} )$
$=f( \frac{m.\frac{m{{( m+n )}^{2}}}{{{( m+n )}^{3}}-n{{m}^{2}}}+n.0}{m+n} )$
$=\alpha .f( \frac{m{{( m+n )}^{2}}}{{{( m+n )}^{3}}-n{{m}^{2}}} )$
Finally 
$ f( \frac{m{{( m+n )}^{2}}}{{{( m+n )}^{3}}-n{{m}^{2}}}  $
$=f( \frac{m.1+n.\frac{{{m}^{3}}}{{{( m+n )}^{3}}-n{{m}^{2}}}}{m+n} )$
$=\alpha f( 1 )+( 1-\alpha  )f( \frac{{{m}^{3}}}{{{( m+n )}^{3}}-n{{m}^{2}}} ) $
 $=\alpha +( 1-\alpha  )f( \frac{{{m}^{3}}}{{{( m+n )}^{3}}-n{{m}^{2}}} ) 
$
So we conclude that
	\[f( \frac{{{m}^{3}}}{{{( m+n )}^{3}}-n{{m}^{2}}} )=\frac{{{\alpha }^{3}}}{1-{{\alpha }^{2}}+{{\alpha }^{3}}}\]
\end{solution}



\begin{solution}[by \href{https://artofproblemsolving.com/community/user/67223}{Amir Hossein}]
	Also posted [url=https:\/\/artofproblemsolving.com\/community\/c6h338030p1808772]here[\/url].
\end{solution}
*******************************************************************************
-------------------------------------------------------------------------------

\begin{problem}[Posted by \href{https://artofproblemsolving.com/community/user/43536}{nguyenvuthanhha}]
	Let $ f  :  [ 0 ; 1] \to\mathbb{R}$ be a strictly increasing function. We know that $ f(0) = 0$ and $f(1) = 1$. Moreover,
\[ \frac{1}{2} \le \frac{ f(x+y) - f(x)}{ f(x) - f(x-y)} \le 2, \] for all $x$ and $y$ such that $ 0 < y \le x < x+y \le 1$. Prove that \[ f \left( \frac{1}{3} \right) \le \frac{76}{135}.\]
	\flushright \href{https://artofproblemsolving.com/community/c6h321886}{(Link to AoPS)}
\end{problem}



\begin{solution}[by \href{https://artofproblemsolving.com/community/user/43536}{nguyenvuthanhha}]
	\begin{italicized}My lovely teacher , sir . Tran Nam Dung gave me a very good solution as follows  

  Solution :      
      From the hypothesis ,     notice that :  $ \frac {1}{3} f(x + y) + \frac {2}{3}f(x - y) \le f(x) \le \frac {1}{3} f(x - y) + \frac {2}{3}f(x + y)$

 for all $ 0 \le y \le x \le x + y \le 1 \ \ (*)$

   From that property , we can have some relative inequalities :

  $ a\/ \ \ \frac {3}{2} f \left( \frac { 2 }{3 } \right) - \frac {1}{2} f \left( \frac {7 }{12 } \right) \le f \left( \frac { 3 }{ 4} \right) \le \frac {2}{3} + \frac {1}{3} f \left( \frac { 1}{ 2 } \right)$ ( in $ (*)$ put $ x \ = \ \frac {2}{3} ; y \ = \ \frac {1}{6}$ and $ x = \frac {3}{4} \ ; \ y \ = \ \frac {1}{4}$ )

   $ b\/ \ \ f \left( \frac { 7 }{ 12 } \right) \le 3f \left( \frac {1 }{ 2 } \right) \ - \ 2 f \left( \frac { 5 }{12 } \right)$ ( in $ (*)$ put $ x \ = \ \frac {1}{2} ; y \ = \ \frac {1}{12}$ )    
   $ c\/ \ \ f \left( \frac { 5 }{12 } \right) \ge \frac {3}{2} f \left( \frac { 1 }{ 3 } \right) \ - \ \frac {1}{2} f \left( \frac { 1 }{ 4 } \right)$( in $ (*)$ put $ x \ = \ \frac {1}{3} ; y \ = \ \frac {1}{12}$ ) 

     $ d\/ \ \ f \left( \frac { 1 }{ 4 } \right) \le \frac {2}{3} f \left( \frac {1 }{ 2 } \right)$

     $ e\/ \ \ f \left( \frac { 1 }{ 3 } \right) \le \frac {2}{3} f \left( \frac {2 }{ 3 } \right) \rightarrow \frac {3}{2}f \left( \frac { 1 }{ 3 } \right) \le f \left( \frac {2 }{ 3 } \right)$

    $ f\/ \ \ f \left( \frac {1 }{ 2 } \right) \le \frac {2}{3}$
   Now , It's time to eat a cake  :blush: 
From $ a\/$ , we have : $ \frac {3}{2} f \left( \frac { 2 }{3 } \right) - \frac {1}{2} f \left( \frac {7 }{12 } \right) \le \frac {2}{3} + \frac {1}{3} f \left( \frac { 1}{ 2 } \right)$

   Thus , $ \frac {3}{2} f \left( \frac { 2 }{3 } \right) \le \frac {1}{2} f \left( \frac {7 }{12 } \right) + \frac {2}{3} + \frac {1}{3} f \left( \frac { 1}{ 2 } \right) \le \frac {2}{3} + \frac {1}{3} f \left( \frac { 1}{ 2 } \right) + \frac {3}{2} f \left( \frac {1 }{ 2 } \right) \ - \ f \left( \frac { 5 }{12 } \right)$ , due to $ b\/$ :)

   Now , using $ c\/$ , we have : 

   Hence $ \frac {3}{2} f \left( \frac { 2 }{3 } \right) \le \frac {2}{3} + \frac {1}{3} f \left( \frac { 1}{ 2 } \right) + \frac {3}{2} f \left( \frac {1 }{ 2 } \right) - \frac {3}{2} f \left( \frac { 1 }{ 3 } \right) \ + \ \frac {1}{2} f \left( \frac { 1 }{ 4 } \right)$

  So that :  
$ \frac {3}{2} f \left( \frac { 2 }{3 } \right) + \frac {3}{2} f \left( \frac { 1 }{ 3 } \right) \le \frac {2}{3} + \frac {11}{6} f \left( \frac { 1}{ 2 } \right) \ + \ \frac {1}{2} f \left( \frac { 1 }{ 4 } \right)$

$ \le \frac {2}{3} + \frac {11}{6} f \left( \frac { 1}{ 2 } \right) \ + \ \frac {1}{3} f \left( \frac { 1 }{ 4 } \right) = \frac {2}{3} + \frac {13}{6} f \left( \frac { 1}{ 2 } \right) \le \frac {2}{3} + \frac {13}{6} \cdot \frac {2}{3} \ = \ \frac {19}{9}$

  $ \rightarrow \frac {19}{9} \ge \frac {3}{2} f \left( \frac { 2 }{3 } \right) + \frac {3}{2} f \left( \frac { 1 }{ 3 } \right) \ge \frac {9}{4} f \left( \frac { 2 }{3 } \right) + \frac {3}{2} f \left( \frac { 1 }{ 3 } \right)$ , due to $ e\/$

   Thus , $ f\left( \frac { 1 }{ 3 } \right) \le \frac { 76}{135}$ , Done

   So nice problem :love: \end{italicized}
\end{solution}



\begin{solution}[by \href{https://artofproblemsolving.com/community/user/29428}{pco}]
	I got only $ f(\frac 13)\le\frac 47$ ...  :blush:
\end{solution}



\begin{solution}[by \href{https://artofproblemsolving.com/community/user/29428}{pco}]
	Here is my proof for my weaker result :

Setting $ x = y = \frac 13$ in the inequation, we get $ \frac 12\le\frac {f(\frac 23) - f(\frac 13)}{f(\frac 13)}\le 2$

And since $ f(x)$ is strictly increasing and $ f(0) = 0$, we get $ f(\frac 13) > 0$ and the above inequality becomes $ \frac 32f(\frac 13)\le f(\frac 23)\le 3f(\frac 13)$

Setting now $ x = \frac 23$ and $ y = \frac 13$, we get $ \frac 12\le\frac {1 - f(\frac 23)}{f(\frac 23) - f(\frac 13)}$ and so : $ \frac 32f(\frac 23)\le 1 + \frac 12f(\frac 13)$

And since we got in the beginning $ \frac 32f(\frac 13)\le f(\frac 23)$, we can conclude :

$ \frac 94f(\frac 13)\le \frac 32f(\frac 23)\le 1 + \frac 12f(\frac 13)$ and so $ \frac 94f(\frac 13)\le 1 + \frac 12f(\frac 13)$ and so : $ \boxed{f(\frac 13)\le\frac 47}$ 

and $ \frac 47$ is near of $ \frac {76}{135}$ but this is not enough for showing your request. ;)
\end{solution}
*******************************************************************************
-------------------------------------------------------------------------------

\begin{problem}[Posted by \href{https://artofproblemsolving.com/community/user/46840}{behdad.math.math}]
	Find all functions $f: \mathbb R^{+} \to \mathbb R^{+}$ such that
\[ f\left(\frac{2xy}{x+y}\right) = \frac{2f(x)f(y)}{f(x)+f(y)}\]
for all positive reals $x$ and $y$.
	\flushright \href{https://artofproblemsolving.com/community/c6h321964}{(Link to AoPS)}
\end{problem}



\begin{solution}[by \href{https://artofproblemsolving.com/community/user/29428}{pco}]
	\begin{tcolorbox}Find all functions $ f: R^ + \longrightarrow R^ +$ such that:

$ f(\frac {2xy}{x + y}) = \frac {2f(x)f(y)}{f(x) + f(y)}$\end{tcolorbox}

I suppose $ 0\notin\mathbb R^+$

Let $ g(x)=\frac 1{f(\frac 1x)}$. The equation becomes $ g(\frac{\frac 1x+\frac 1y}2)=\frac{g(\frac 1x)+g(\frac 1y)}2$ and so $ g(\frac {x+y}2)=\frac{g(x)+g(y)}2$ with $ g(x): \mathbb R^+\to\mathbb R^+$

This is a very simple and well known equation whose solution (remember the constraint $ \mathbb R^+\to\mathbb R^+$) is $ g(x)=ax+b$ with $ a,b\ge 0$ and $ a+b>0$

Hence the solutions of original equation : $ \boxed{f(x)=\frac{x}{bx+a}}$ where $ a,b\ge 0$ and $ a+b>0$
\end{solution}
*******************************************************************************
-------------------------------------------------------------------------------

\begin{problem}[Posted by \href{https://artofproblemsolving.com/community/user/46840}{behdad.math.math}]
	Find all continuous functions $ f: \mathbb R \to \mathbb R$ such that for all $ x\in \mathbb R$,
\[f(1 - x) = 1 - f(f(x)).\]
	\flushright \href{https://artofproblemsolving.com/community/c6h321967}{(Link to AoPS)}
\end{problem}



\begin{solution}[by \href{https://artofproblemsolving.com/community/user/29428}{pco}]
	\begin{tcolorbox}Find all continuouse functions $ f: R \longrightarrow R$ such that for all $ x\in R$:

$ f(1 - x) = 1 - f(f(x))$\end{tcolorbox}

1) replacing the equation by a simpler one : $ g(g(x))=-g(-x)$
=============================================
Let $ g(x)=\frac 12-f(\frac 12-x)$ and so $ f(x)=\frac 12-g(\frac 12-x)$ and the equation becomes $ -g(x-\frac 12)=g(g(\frac 12-x))$ and so :

$ g(x)$ is a continuous function such that $ g(g(x))=-g(-x)$
Q.E.D.

2) $ g(x)=x$ $ \forall x\in g(\mathbb R)$
=======================

Let $ A=g(\mathbb R)$
From 1) above, we get that $ a\in A\implies -a\in A$ and so, since $ g(x)$ is continuous :
Either $ A=\mathbb R$, either $ A=[-u,u]$ for some $ u\ge 0$.

$ g(g(x))=-g(-x)$ $ \implies$ $ g(g(g(g(x))))=-g(-g(g(x)))=-g(g(-x))=g(x)$ and so $ g(g(g(x)))=x$ $ \forall x\in A$

Let then $ g_r(x)$ : $ A\to A$ the restriction of $ g(x)$ to $ A$ :
$ g(g(g(x)))=x$ $ \forall x\in A$ $ \implies$ $ g_r(g_r(g_r(x)))=x$ $ \forall x\in A$ and so $ g_r(x)$ is a bijection, so is monotonous
$ g(g(x))=-g(-x)$ $ \implies$ $ g_r(g_r(x))=-g_r(-x)$ implies then $ -g_r(-x)$ is increasing and so $ g_r(x)$ is increasing.

So, $ g_r(a)> a$ $ \implies$ $ g_r(g_r(a))>g_r(a)>a$ and so $ g_r(g_r(g_r(a)))>g_r(a)>a$ which is impossible and so $ g_r(a)\le a$
Same, $ g_r(a)< a$ $ \implies$ $ g_r(g_r(a))<g_r(a)<a$ and so $ g_r(g_r(g_r(a)))<g_r(a)<a$ which is impossible and so $ g_r(a)\ge a$

So $ g_r(x)=x$
Q.E.D.

3) general solution of $ g(g(x))=-g(-x)$ with $ g(x)$ continuous
===========================================================

Then the original equation $ g(g(x))=-g(-x)$ becomes $ g(x)=-g(-x)$ (since $ g(g(x))=g_r(g(x))=g(x)$) and so :

The general continuous solutions of $ g(g(x))=-g(-x)$ are any odd continuous function $ g(x)$ such that $ g(x)=x$ $ \forall x\in g(\mathbb R)$
A simple way to build them is :
3.1) If $ g(\mathbb R)=\mathbb R$ : $ g(x)=x$ $ \forall x$

3.2) If $ g(\mathbb R)=[-u,u]$ for some $ u\ge 0$ :
Let any continuous fonction $ h(x)$ from $ [u,+\infty)\to[-u,+u]$ such that $ h(u)=u$ and define $ g(x)$ as :
$ \forall x\in(-\infty,-u)$ : $ g(x)=-h(-x)$
$ \forall x\in[-u,+u]$ : $ g(x)=x$
$ \forall x\in(u,+\infty)$ : $ g(x)=h(x)$

Notice that $ u=0$ gives the solution $ g(x)=0$

4) general solution of $ f(1-x)=1-f(f(x))$ with $ f(x)$ continuous
=============================================================
The general solution of original equation is $ f(x)=\frac 12-g(\frac 12-x))$ where $ g(x)$ is any function described in 3) above

5) examples of solutions
======================
$ g(x)=0$ gives the solution $ f(x)=\frac 12$

$ g(x)=x$ gives the solution $ f(x)=x$

$ u=1$ and $ h(x)=\sin(\frac{\pi}2x)$ gives the solution : 
If $ |x-\frac 12|<1$ : $ f(x)=x$
If $ |x-\frac 12|\ge 1$ : $ f(x)=\frac 12+\sin(\frac{\pi}2x-\frac{\pi}4)$

$ u=2$ and $ h(x)=4|2\{\frac{x-2}8\}-1|-2$ gives the solution : $ f(x)=\frac 52-4|2\{\frac{-2x-3}{16}\}-1|$ :)

Notice that the only differentiable solutions are $ f(x)=\frac 12$ and $ f(x)=x$
\end{solution}



\begin{solution}[by \href{https://artofproblemsolving.com/community/user/66201}{basketball9}]
	Is f bijective?
\end{solution}



\begin{solution}[by \href{https://artofproblemsolving.com/community/user/139716}{asjeykg}]
	are there any other more easy solutions? ) Please help. Thanks in advance!
\end{solution}



\begin{solution}[by \href{https://artofproblemsolving.com/community/user/29428}{pco}]
	\begin{tcolorbox}are there any other more easy solutions? ) Please help. Thanks in advance!\end{tcolorbox}
Are you looking for a simpler method to get the ugly set of solution or for any method to get a simpler set of solutions ?

In case of the first choice, what is the part of the current proof that you consider as "not easy" in order we can try to improve ?
\end{solution}



\begin{solution}[by \href{https://artofproblemsolving.com/community/user/177353}{Parnpaniti}]
	if problem can't have  continuouse functions
\end{solution}



\begin{solution}[by \href{https://artofproblemsolving.com/community/user/292092}{Erkhes}]
	PCO how did this kind of substitution come to your head ?

\end{solution}



\begin{solution}[by \href{https://artofproblemsolving.com/community/user/330078}{Delray}]
	\begin{tcolorbox}[quote="behdad.math.math"]Find all continuouse functions $ f: R \longrightarrow R$ such that for all $ x\in R$:

$ f(1 - x) = 1 - f(f(x))$\end{tcolorbox}

1) replacing the equation by a simpler one : $ g(g(x))=-g(-x)$
=============================================
Let $ g(x)=\frac 12-f(\frac 12-x)$ 
\end{tcolorbox}
What was your thought process in arriving at this substitution?

\end{solution}



\begin{solution}[by \href{https://artofproblemsolving.com/community/user/29428}{pco}]
	As often : tests and trials
When $f(f(x))$ occurs somewhere, it may be clever to test $g(x)=u(f(u^{-1}(x)))$ where $u(x)$ is any bijection since then $g(g(x))=u(f(f(u^{-1}(x)))$

\end{solution}
*******************************************************************************
-------------------------------------------------------------------------------

\begin{problem}[Posted by \href{https://artofproblemsolving.com/community/user/48364}{cnyd}]
	Prove that there is no function $ f: \mathbb{R^{ + }}\mapsto \mathbb{R^{ + }}$ such that for all $x,y\in\mathbb{R^{ + }}$,
\[f(x)^{2}\geq f(x + y)[f(x) + y].\]
	\flushright \href{https://artofproblemsolving.com/community/c6h321998}{(Link to AoPS)}
\end{problem}



\begin{solution}[by \href{https://artofproblemsolving.com/community/user/29428}{pco}]
	\begin{tcolorbox}Prove that there is no function, $ f: \mathbb{R^{ + }}\mapsto \mathbb{R^{ + }}$,

$ \forall x,y\in\mathbb{R^{ + }},f(x)^{2}\geq f(x + y)[f(x) + y]$\end{tcolorbox}

Are brackets in $ [f(x) + y]$ just parenthesis "$ (f(x) + y)$" or the marks of floor function "$ \lfloor f(x) + y\rfloor$" ?
\end{solution}



\begin{solution}[by \href{https://artofproblemsolving.com/community/user/48364}{cnyd}]
	they are just parenthesis
\end{solution}



\begin{solution}[by \href{https://artofproblemsolving.com/community/user/68920}{prester}]
	\begin{tcolorbox}Prove that there is no function, $ f: \mathbb{R^{ + }}\mapsto \mathbb{R^{ + }}$,

$ \forall x,y\in\mathbb{R^{ + }},f(x)^{2}\geq f(x + y)[f(x) + y]$\end{tcolorbox}

Let suppose that there exist a function $ f(x): \mathbb{R^{ + }}\mapsto \mathbb{R^{ + }}$ that satisfy the given condition. 

Let $ x_0 > 0.$ $ f(x_0) > 0$ and $ \forall y\in \mathbb{R^ + }$ we should have
\[ y \le \frac {f(x_0)}{f(x_0 + y)} \left [f(x_0) - f(x_0 + y)\right] \;\;\;(*)\]

where $ f(x_0) - f(x_0 + y) > 0$ since that it must be $ f(x_0)f(x_0 + y) < f(x_0)^2.$

The inequality $ (*)$ is a contradiction. So there is no function $ f(x): \mathbb{R^{ + }}\mapsto \mathbb{R^{ + }}$ that verifies the given property.
\end{solution}



\begin{solution}[by \href{https://artofproblemsolving.com/community/user/48552}{ocha}]
	\begin{tcolorbox}
where $ f(x_0) - f(x_0 + y) > 0$ since that it must be $ f(x_0)f(x_0 + y) < f(x_0)^2.$

This is a contradiction. \end{tcolorbox}

sorry, I don't see the contradiction, can you please explain. thanks
\end{solution}



\begin{solution}[by \href{https://artofproblemsolving.com/community/user/68920}{prester}]
	\begin{tcolorbox}[quote="prester"]
where $ f(x_0) - f(x_0 + y) > 0$ since that it must be $ f(x_0)f(x_0 + y) < f(x_0)^2.$

This is a contradiction. \end{tcolorbox}

sorry, I don't see the contradiction, can you please explain. thanks\end{tcolorbox}

The contradiction is the inequality above $ y \le ...$
\end{solution}



\begin{solution}[by \href{https://artofproblemsolving.com/community/user/68920}{prester}]
	\begin{tcolorbox}[quote="ocha"]\begin{tcolorbox}
where $ f(x_0) - f(x_0 + y) > 0$ since that it must be $ f(x_0)f(x_0 + y) < f(x_0)^2.$

This is a contradiction. \end{tcolorbox}

sorry, I don't see the contradiction, can you please explain. thanks\end{tcolorbox}

The contradiction is the inequality above $ y \le ...$\end{tcolorbox}

I cannot delete my previous post.  However, \begin{italicized}ocha\end{italicized} is right. There is no contradiction since the right hand side of the inequality is a function of $ y$ and it is increasing. So the inequality is $ y \le g(y)$ $ \forall y \in \mathbb{R^+}$ and there is no contradiction... :(  :blush:
\end{solution}



\begin{solution}[by \href{https://artofproblemsolving.com/community/user/48552}{ocha}]
	[hide="my solution"]

From the condition it is obvious that $ f$ is strictly decreasing, futhermore if we fix $ x$ and take the limit as $ y \to \infty$ then $ f(x) \to 0$ 


We have $ f(x)^{2}\geq f(x + y)(f(x) + y) \Longrightarrow f(x)\ge f(x + y) + \frac {1}{\frac {1}{f(x)} + \frac {1}{y}} \qquad(1)$

Also by AM-GM: $ f(x + y) + \frac {f(x) + y}{4} \ge \sqrt {f(x + y)(f(x) + y)} = f(x) \qquad(2)$


So $ f(x + y) + \frac {f(x) + y}{4} \ge f(x) \ge f(x + y) + \frac {1}{\frac {1}{f(x)} + \frac {1}{y}}$

If we set $ y = f(x)$ then we get $ f(x + f(x)) = \frac {1}{2}f(x) \qquad(*)$

If now set $ x\to x + f(x)$ in $ (*)$ we get $ f(x + \frac {3}{2}f(x)) = \frac {1}{4}f(x)$

By the same substitution we get by induction $ f(x + 2f(x) - \frac {1}{2^n}f(x)) = \frac {1}{2^n}f(x) \qquad (**)$

If we send $ n\to \infty$ in $ (**)$ then we get $ f(x + 2f(x)) = 0$ 

This is an obvious contradiction so we cannot have any functions that satisfy the condition  :)[\/hide]
\end{solution}



\begin{solution}[by \href{https://artofproblemsolving.com/community/user/29428}{pco}]
	\begin{tcolorbox} ... $ f(x + 2f(x) - \frac {1}{2^n}f(x)) = \frac {1}{2^n}f(x) \qquad (**)$

If we send $ n\to \infty$ in $ (**)$ then we get $ f(x + 2f(x)) = 0$ \end{tcolorbox}

You need continuity to get this conclusion.
And we dont have ...  :(
\end{solution}



\begin{solution}[by \href{https://artofproblemsolving.com/community/user/43536}{nguyenvuthanhha}]
	\begin{italicized}Here is the solution I read in somewhere that I can't remember  :blush: :

  We have     $ \forall x,y\in\mathbb{R^{ + }},f(x)^{2}\geq f(x + y)[f(x) + y] (1)$

   From the hypothesis , we can deduce that $ f$ is a strictly decreasing function  

   In $ (1) \rightarrow f(x + y) \leq \frac { f(x)^{2}}{f(x) + y}$ 

 Fix $ x$ , and let $ y$ come to infty , we can see easily that : $ \lim_{x \to + \infty } f(x) \ = \ 0$

    On the other hand , change $ y$ to $ f(x)$ , we have : $ f(x + f(x)) \le \frac {f(x)}{2}$

  For each positive real number $ x_0$ , construct the sequence $ (x_n)$ as follows :

   $ x_{n + 1} = x_n + f(x_n)$

  So , $ f(x_{n + 1}) \le \frac {f(x_n)}{2} \rightarrow f(x_{n}) \le \frac {f(x_0)}{2^n} \ \ \forall n \ \in \ \mathbb{N}$

   $ x_{n + 1} \ = \ x_0 + f(x_0) + f(x_1) + .... + f(x_n) \le x_0 + f(x_0) \left( 1 + \frac {1}{2} + ... + \frac {1}{2^n} \right) < x_0 + 2f(x_0)$

    Thus , the  sequence $ (x_n)$  is up  bounded by the constent $ m \ = \ x_0 + 2f(x_0)$

   So,    $ 0 < f(m) \le f(x_n) \le \frac {f(x_0)}{2^n} \ \ \forall n \ \in \ \mathbb{N}$

  But , $ \lim_{n \to + \infty} \frac {f(x_0)}{2^n} = 0$ , contradiction  

       We conclude that there in no function $ f$ satisfies the condition \end{italicized}
\end{solution}



\begin{solution}[by \href{https://artofproblemsolving.com/community/user/48552}{ocha}]
	\begin{tcolorbox}[quote="ocha"] ... $ f(x + 2f(x) - \frac {1}{2^n}f(x)) = \frac {1}{2^n}f(x) \qquad (**)$

If we send $ n\to \infty$ in $ (**)$ then we get $ f(x + 2f(x)) = 0$ \end{tcolorbox}

You need continuity to get this conclusion.
And we dont have ...  :(\end{tcolorbox}

\begin{bolded}EDIT: \end{bolded} \begin{italicized} on second thought wouldn't monotonicity be enough the get the conclusion?\end{italicized}


Thanks PCO. Can it still be fixed though? Say we take $ x_0$ as a constant, then we still get

$ f(x_0 + 2f(x_0) - \frac {1}{2^n}f(x_0)) = \frac {1}{2^n}f(x_0)$

So we can send $ n\to \infty$ and get $ f(x_0 + 2f(x_0)) = 0$

If we let $ x = x_0 + 2f(x_0)$ in our original inequality we get

$ 0 = f(x_0 + 2f(x_0))^2 \ge yf(x_0 + 2f(x_0) + y)$

Since $ y\not = 0$ we have $ f(x_0 + 2f(x_0) + y) = 0 \forall y \in \mathbb{R}$ which is absurd
\end{solution}
*******************************************************************************
-------------------------------------------------------------------------------

\begin{problem}[Posted by \href{https://artofproblemsolving.com/community/user/68920}{prester}]
	Let $ f : \mathbb{R} \rightarrow \mathbb{R}$ be a continuous function that satisfies the following conditions: 
a) $ f(1000)=999$, and
b) $ f(x) \cdot f(f(x)) = 1$ for all $x \in \mathbb{R}$.

Find $ f(500)$.
	\flushright \href{https://artofproblemsolving.com/community/c6h323159}{(Link to AoPS)}
\end{problem}



\begin{solution}[by \href{https://artofproblemsolving.com/community/user/29428}{pco}]
	\begin{tcolorbox}Let $ f : \mathbb{R} \rightarrow \mathbb{R}$ a \begin{italicized}continuous function\end{italicized} that satisfies the following conditions: 

$ a)$ $ f(1000) = 999$

$ b)$ $ f(x) \cdot f(f(x)) = 1,$   $ \forall x \in \mathbb{R}$

Find $ f(500).$\end{tcolorbox}

$ f(x)f(f(x))=1$ $ \implies$ $ f(x)\ne 0$ and so $ f(f(x))=\frac 1{f(x)}$ $ \forall x\in\mathbb R$ and so $ f(x)=\frac 1x$ $ \forall x\in f(\mathbb R)$

$ f(1000)=999$ $ \implies$ $ 999\in f(\mathbb R)$ $ \implies$ $ f(999)=\frac 1{999}$ $ \implies$ (using continuity of $ f(x)$) $ [\frac 1{999},999]\subseteq f(\mathbb R)$ $ \implies$ $ 500\in f(\mathbb R)$ $ \implies$ $ \boxed{f(500)=\frac 1{500}}$
\end{solution}



\begin{solution}[by \href{https://artofproblemsolving.com/community/user/44887}{Mathias_DK}]
	\begin{tcolorbox}$ f(1000) = 999$ $ \implies$ $ 999\in f(\mathbb R)$ $ \implies$ $ f(999) = \frac 1{999}$ $ \implies$ (using continuity of $ f(x)$) $ [\frac 1{999},999]\subseteq f(\mathbb R)$ $ \implies$ $ 500\in f(\mathbb R)$ $ \implies$ $ \boxed{f(500) = \frac 1{500}}$\end{tcolorbox}
How can one prove $ f(a_1) = b_1,f(a_2) = b_2 \Rightarrow [b_1;b_2] \in f(\mathbb{R})$?
\end{solution}



\begin{solution}[by \href{https://artofproblemsolving.com/community/user/29428}{pco}]
	\begin{tcolorbox}[quote="pco"]$ f(1000) = 999$ $ \implies$ $ 999\in f(\mathbb R)$ $ \implies$ $ f(999) = \frac 1{999}$ $ \implies$ (using continuity of $ f(x)$) $ [\frac 1{999},999]\subseteq f(\mathbb R)$ $ \implies$ $ 500\in f(\mathbb R)$ $ \implies$ $ \boxed{f(500) = \frac 1{500}}$\end{tcolorbox}
How can one prove $ f(a_1) = b_1,f(a_2) = b_2 \Rightarrow [b_1;b_2] \in f(\mathbb{R})$?\end{tcolorbox}

This is just the intermediate value theorem for continuous functions. see http://en.wikipedia.org\/wiki\/Intermediate_value_theorem
\end{solution}



\begin{solution}[by \href{https://artofproblemsolving.com/community/user/44887}{Mathias_DK}]
	Thanks a lot :)
\end{solution}



\begin{solution}[by \href{https://artofproblemsolving.com/community/user/68920}{prester}]
	It was a problem from a context in Italy (Cesenatico). As said above by pco,  the solution of this problem use a theorem called "Intermediate value theorem" that say that for continuous functions in an interval $ [a,b]$ where $ f(a) \ne f(b)$, the function takes all the values between $ f(a)$ and $ f(b)$. 
As I know, this theorem was attributed to $ Bolzano$. See wikipedia for more informations.

Thanks a lot for your work.
\end{solution}
*******************************************************************************
-------------------------------------------------------------------------------

\begin{problem}[Posted by \href{https://artofproblemsolving.com/community/user/29428}{pco}]
	Find all functions $f: \mathbb R^+\to\mathbb R^+$ such that \[ f(xf(y))=f(x+y)\] for all positive reals $x$ and $y$.
	\flushright \href{https://artofproblemsolving.com/community/c6h323169}{(Link to AoPS)}
\end{problem}



\begin{solution}[by \href{https://artofproblemsolving.com/community/user/19427}{TRAN THAI HUNG}]
	\begin{tcolorbox}@angelstt : since you seem interested in this functional equation (4 different problems about it), here is a fifth one :

Find all functions from $ \mathbb R^ + \to\mathbb R^ +$ (the set of all positive real numbers) such that : $ f(xf(y)) = f(x + y)$ $ \forall x,y > 0$\end{tcolorbox}
Is my solution right?
$ f(x+y)=f(xf(y))=f(yf(x))$
\begin{bolded}Case1\end{bolded} if there aren't  $ y_1  \ne y_2$ such that $ f(y_1)=f(y_2)$ then xf(y)=yf(x)
Then, $ f(x)=ax$ . Then check it which is not suitable.
\begin{bolded}Case2 \end{bolded}if there are $ y_1  \ne y_2$ such that $ f(y_1)=f(y_2)$ 
      then $ f(x+y_1)=f(xf(y_1))=f(xf(y_2))=f(x+y_2)$ .
Then, $ f(x)=f(x+c)$ .($ c\geq 0$, c is the smallest frequency)
      \begin{bolded} If\end{bolded} $ c=0$then $ f(x)=const$.Check it suitable.
     \begin{bolded}  If \end{bolded}$ c>0$then, there exits $ a$ that $ f(a) \ne 1$ 
         +$ f(a)>1$ 
           $ f(xf(a))=f(x+a)$ so $ xf(a)=x+a+nc$( $ n \in N^*$ take x large enough)
so $ x(f(a)-1)=a+nc$
Then , take $ x_0=x+\frac{c}{2(f(a)-1)}$we got $ x_0.(f(a)-1)= a+nc+c\/2=a+mc$ ($ n,m \in N^*$) Which can not true.
          Do the same for $ f(a)<1$
Then we obtain $ f(x) =const$
\end{solution}



\begin{solution}[by \href{https://artofproblemsolving.com/community/user/68920}{prester}]
	\begin{tcolorbox}
      ...
      \begin{bolded} If\end{bolded} $ c = 0$then $ f(x) = const$.Check it suitable.
      ...
\end{tcolorbox}

sorry, i don't see that. If $ c=0$ we have $ f(x)=f(x)$. Can you explain? Thanks
\end{solution}



\begin{solution}[by \href{https://artofproblemsolving.com/community/user/29428}{pco}]
	\begin{tcolorbox}[quote="pco"]@angelstt : since you seem interested in this functional equation (4 different problems about it), here is a fifth one :

Find all functions from $ \mathbb R^ + \to\mathbb R^ +$ (the set of all positive real numbers) such that : $ f(xf(y)) = f(x + y)$ $ \forall x,y > 0$\end{tcolorbox}
Is my solution right?
$ f(x + y) = f(xf(y)) = f(yf(x))$
\begin{bolded}Case1\end{bolded} if there aren't  $ y_1 \ne y_2$ such that $ f(y_1) = f(y_2)$ then xf(y)=yf(x)
Then, $ f(x) = ax$ . Then check it which is not suitable.
\begin{bolded}Case2 \end{bolded}if there are $ y_1 \ne y_2$ such that $ f(y_1) = f(y_2)$ 
      then $ f(x + y_1) = f(xf(y_1)) = f(xf(y_2)) = f(x + y_2)$ .
Then, $ f(x) = f(x + c)$ .($ c\geq 0$, c is the smallest frequency)
      \begin{bolded} If\end{bolded} $ c = 0$then $ f(x) = const$.Check it suitable.
     \begin{bolded}  If \end{bolded}$ c > 0$then, there exits $ a$ that $ f(a) \ne 1$ 
         +$ f(a) > 1$ 
           $ f(xf(a)) = f(x + a)$ so $ xf(a) = x + a + nc$( $ n \in N^*$ take x large enough)
so $ x(f(a) - 1) = a + nc$
Then , take $ x_0 = x + \frac {c}{2(f(a) - 1)}$we got $ x_0.(f(a) - 1) = a + nc + c\/2 = a + mc$ ($ n,m \in N^*$) Which can not true.
          Do the same for $ f(a) < 1$
Then we obtain $ f(x) = const$\end{tcolorbox}

1) As prester said, $ \inf\{c$ such that $ f(x+c)=f(x) \forall x\}=0$ does not imply $ f(x)$ is constant (it would be true if $ f(x)$ was continuous, but this is not a constraint of the problem.

2) If $ f(x)=f(x+c)$ $ \forall x$, it does not allow you to conclude $ f(a)=f(b)$ $ \implies$ $ a=b+nc$ as you wrote ("$ f(xf(a)) = f(x + a)$ so $ xf(a) = x + a + nc$")
\end{solution}



\begin{solution}[by \href{https://artofproblemsolving.com/community/user/19427}{TRAN THAI HUNG}]
	1) I agree I forgot Dirichlete function also have that condition.(@to prester:when I write c=0, I mean there is no minimum of c). :) 
2) I agree If  $ f(x) = f(x + c)$,  it does not allow me to conclude   $ f(a) = f(b) \implies a = b + nc$ but if we also have $ f(y_1) = f(y_2)\implies\ y_1 - y_2$ is a frequency then we can conclude that since c is the smallest frequency.
Am I right? :) Thanks.
However, I still fell in solve this problem. Can you show your proof, please?
\end{solution}



\begin{solution}[by \href{https://artofproblemsolving.com/community/user/29428}{pco}]
	\begin{tcolorbox}1) I agree.
2) I agree If  $ f(x) = f(x + c)$,  it does not allow me to conclude   $ f(a) = f(b) \implies a = b + nc$ but if we also have $ f(y_1) = f(y_2)\implies\ y_1 - y_2$ is a frequency then we can conclude that since c is the smallest frequency.
Am I right? :) Thanks.\end{tcolorbox}

 :blush:  You're right. $ f(a)=f(b)$ $ \implies$ $ |b-a|$ is a period and so $ c$ divides $ |b-a|$ if $ c\ne 0$
Sorry.
\end{solution}



\begin{solution}[by \href{https://artofproblemsolving.com/community/user/73892}{Nixmtp}]
	Pco , can u please post your solution ?
\end{solution}



\begin{solution}[by \href{https://artofproblemsolving.com/community/user/26129}{The QuattoMaster 6000}]
	[hide="Solution"]
We claim that only constant functions work; it is clear that they do work. 


\begin{bolded}Observation:\end{bolded} If $ f(a) = f(b)$ with $ b - a = s > 0$, then $ f(y + a) = f(yf(a)) = f(yf(b)) = f(y + b)$ for positive $ y$.  Set $ y = x - a$ to get that $ f(x) = f(x + s)$ for all $ x > a$.  


If $ f$ is injective, then $ xf(y) = x + y$, so $ f(1) = 1 + \frac {1}{x}$.  However, this value is nonconstant, which is a contradiction, so $ f$ cannot be injective.  By the observation, there are reals $ t$ and $ u$ so that for all $ x > u$, $ f(x) = f(x + t)$.  Assume that $ f$ is nonconstant so that there is a $ z$ so that $ f(z) \ne 1$. 

Let $ r$ be an arbitrary real greater than $ z$.  If $ f(z) > 1$, then set $ x = \frac {z + r}{f(z) - 1}$.  If $ f(z) < 1$, then set $ x = \frac {r - z}{1 - f(z)}$.  In the first case, we get $ xf(z) = x + z + r$, meaning that $ f(x + z) = f(x + z + r)$, and in the second case, we get that $ xf(z) = x + z - r$, so $ f(x + z - r) = f(x + z)$.  Either way, we have found a $ k$ and an $ l$ so that $ |k - l| = r$ and $ f(k) = f(l)$.  By the observation, there is a function $ g$ that maps reals to reals so that whenever $ x > g(r)$, $ f(x) = f(x + r)$.

Set $ h(r) = g(r) - t\lfloor {\frac {g(r)}{t} }\rfloor$ so that $ h(r) < t$ for all $ r$.  Let $ \max \{t, u \} = v$; if $ x > v$, then $ f(x) = f(x + g(r) - h(r)) = f(x + g(r) - h(r) + r) = f(x + r)$ for all $ r$.  Then, there is a constant $ c$ so that $ f(x) = c$ is constant whenever $ x > v$.  For any $ w$, put in $ x = \frac {w}{c}$ and set $ y > v$.  This gives that $ f(w) = f(xf(y)) = f(x + y) = c$, so $ f$ is constant everywhere. 

Hence, only constant functions work.

[\/hide]
\end{solution}



\begin{solution}[by \href{https://artofproblemsolving.com/community/user/29428}{pco}]
	\begin{bolded}@QuattroMaster \end{bolded}\end{underlined}:
I do agree with your solution with a little correction :
\begin{tcolorbox} ... then $ f(x) = f(x + g(r) - h(r)) = f(x + g(r) - h(r) + r) = f(x + r)$ for all $ r$.  \end{tcolorbox}
In fact, this is not $ \forall r$ : it is $ \forall r > z$ but this does not destroy the proof.

Congrats.

\begin{bolded}@Nixmtp\end{bolded}\end{underlined} :
Here is my solution, using the same idea than QuattroMaster, but with a little bit different end :

Let $ P(x,y)$ be the equation $ f(xf(y)) = f(x + y)$

$ f(x) = 1$ $ \forall x$ is a solution. So we'll from now consider $ \exists u$ such that $ f(u)\ne 1$

1) If $ \exists a,T > 0$ such that $ f(a) = f(a + T)$ then $ f(x) = f(x + T)$ $ \forall x\ge a$
=================================================================================
Let $ x > a$ : 
$ P(x - a,a)$ $ \implies$ $ f((x - a)f(a)) = f(x)$
$ P(x - a,a + T)$ $ \implies$ $ f((x - a)f(a + T)) = f(x + T)$ and so $ f(x) = f(x + T)$ $ \forall x > a$
And, since $ f(a) = f(a + T)$ : $ f(x) = f(x + T)$ $ \forall x\ge a$
Q.E.D.

2) $ \exists a,b > 0$ such that $ f(x) = f(x + T)$ $ \forall x > a,\forall T > b$
====================================================================
Let $ u$ such that $ f(u)\ne 1$
2.1) if $ f(u) > 1$
---------------
Let $ T > 0$ : $ P(\frac {u + T}{f(u) - 1},u)$ $ \implies$ $ f(\frac {uf(u) + Tf(u)}{f(u) - 1}) = f(\frac {u + T}{f(u) - 1} + u)$ $ = f(\frac {uf(u) + Tf(u)}{f(u) - 1} - T)$
And so, using the point 1) above : $ f(x) = f(x + T)$ $ \forall x\ge \frac {uf(u) + Tf(u)}{f(u) - 1} - T$
Q.E.D.

2.2) if $ f(u) < 1$
----------------
Let $ T > u$ : $ P(\frac {u - T}{f(u) - 1},u)$ $ \implies$ $ f(\frac {uf(u) - Tf(u)}{f(u) - 1}) = f(\frac {u - T}{f(u) - 1} + u)$ $ = f(\frac {uf(u) - Tf(u)}{f(u) - 1} + T)$
And so, using the point 1) above : $ f(x) = f(x + T)$ $ \forall x\ge \frac {uf(u) - Tf(u)}{f(u) - 1}$
Q.E.D.

3) The only solutions are constant functions
============================================
Let $ a,b > 0$ such that $ f(x) = f(x + T)$ $ \forall x > a,\forall T > b$ (according to point 2) above)
Let $ x > y > a$ 

Let $ T_1 = b + 1 > b$. Using point 2, we get $ f(x) = f(x + b + 1)$
Let $ T_2 = b + 1 + x - y > b$. Using point 2, we get $ f(y) = f(y + b + 1 + x - y) = f(x + b + 1)$
And so $ f(x) = f(y) = c$ $ \forall x,y > a$ for some constant $ c$

Let then $ u > 0$ $ P(\frac u{f(a + 1)},a + 1)$ $ \implies$ $ f(u) = f(a + 1 + u{f(a + 1)}$ and, since $ a + 1 + u{f(a + 1)} > a$, we get $ f(u) = c$ $ \forall u > 0$
Q.E.D


And so (since the case $ f(x) = 1$ $ \forall x$ is also a constant solution and since obviously $ f(x) = c$ fits the requested equation :

The only solutions to this equation are $ \boxed{f(x) = c}$ $ \forall x > 0$ where $ c$ is a positive constant
\end{solution}



\begin{solution}[by \href{https://artofproblemsolving.com/community/user/73892}{Nixmtp}]
	Thank you very much , i really enjoyed your beautiful solutions.
\end{solution}



\begin{solution}[by \href{https://artofproblemsolving.com/community/user/125018}{horizon}]
	my solution:
$x=1$,we have $f(f(y))=f(y=1)$,so $f(x(f(y+1))=f(x+f(y))=f(x+y+1)$
so $f(y)-y-1=T$ is period for any positive reals $y$
$f(xf(y))-xf(y)-1=T$,$f(x+y)-xf(y)-1=T$,$f(x+y)-(x+y)-1=T$,
so $T=x(1-f(y))+y$,if $f(x)$ is not constant,then we have $T$ can be all reals on interval $T>k$ or $T<k'$
so $f$ must be constant
\end{solution}



\begin{solution}[by \href{https://artofproblemsolving.com/community/user/330078}{Delray}]
	\begin{tcolorbox}my solution:
$x=1$,we have $f(f(y))=f(y=1)$,so $f(x(f(y+1))=f(x+f(y))=f(x+y+1)$
so $f(y)-y-1=T$ is period for any positive reals $y$
$f(xf(y))-xf(y)-1=T$,$f(x+y)-xf(y)-1=T$,$f(x+y)-(x+y)-1=T$,
so $T=x(1-f(y))+y$,if $f(x)$ is not constant,then we have $T$ can be all reals on interval $T>k$ or $T<k'$
so $f$ must be constant\end{tcolorbox}
Where does $T=x(1-f(y))+y$ come from?
Shouldn't it be $0=x(1-f(y))+y$
\end{solution}
*******************************************************************************
-------------------------------------------------------------------------------

\begin{problem}[Posted by \href{https://artofproblemsolving.com/community/user/25017}{Leonhard Euler}]
	Find all functions $f: {\mathbb{R^+}}\to{\mathbb{R^+}}$ such that
\[ f(1+xf(y))=yf(x+y)\]
for all $x,y\in\mathbb{R^+}$.
	\flushright \href{https://artofproblemsolving.com/community/c6h323174}{(Link to AoPS)}
\end{problem}



\begin{solution}[by \href{https://artofproblemsolving.com/community/user/29428}{pco}]
	\begin{tcolorbox}Find all functions $ \,f: {\mathbb{R^ + }}\rightarrow{\mathbb{R^ + }}\,$ such that
$ f(1 + xf(y)) = yf(x + y)$ $ \text{for all}\,x,y\in\mathbb{R^ + }.$\end{tcolorbox}
Let $ P(x,y)$ be the assertion $ f(1 + xf(y)) = yf(x + y)$

1) $ f(x)$ is a surjective function
=====================
$ P(\frac 1{f(\frac {f(2)}x)},\frac {f(2)}x)$ $ \implies$ $ f(2) = \frac {f(2)}xf(\frac 1{f(\frac {f(2)}x)} + \frac {f(2)}x)$

And so $ x = f(\text{something})$
Q.E.D.



2) $ f(x)$ is an injective function
=====================
Let $ a > b > 0$ such that $ f(a) = f(b)$
Let $ T = b - a > 0$
Comparing $ P(x,a)$ and $ P(x,b)$, we get $ af(x + a) = bf(x + b)$ and so $ f(x) = \frac baf(x + T)$ $ \forall x > a$
And so $ f(x) = \left(\frac ba\right)^nf(x + nT)$ $ \forall x > a,n\in\mathbb N$

Let then $ y$ such that $ f(y) > 1$ (such $ y$ exists since $ f(x)$ is a surjection, according to 1) above)
Let $ n$ great enough to have $ y + nT - 1 > 0$

$ P(\frac {y + nT - 1}{f(y) - 1},y)$ $ \implies$ $ f(1 + \frac {yf(y) + (nT - 1)f(y)}{f(y) - 1}) = yf(\frac {y + nT - 1}{f(y) - 1} + y)$ which may be written :

$ f(\frac {yf(y) + nT - 1}{f(y) - 1} + nT) = yf(\frac {yf(y) + nT - 1}{f(y) - 1})$

and since $ f(\frac {yf(y) + nT - 1}{f(y) - 1} + nT) = \left(\frac ab\right)^nf(\frac {yf(y) + nT - 1}{f(y) - 1})$, we get $ y = \left(\frac ab\right)^n$ $ \forall n$, which is impossible
Q.E.D.

3) $ f(1) = 1$
===========
$ P(1,1)$ $ \implies$ $ f(1 + f(1)) = f(2)$ and so, since $ f(x)$ is injective, $ f(1) = 1$
Q.E.D.


4) The only solution is $ f(x) = \frac 1x$
==========================
$ P(1,x)$ $ \implies$ $ f(1 + f(x)) = xf(1 + x)$ and so $ f(1 + x) = \frac 1xf(1 + f(x))$

$ P(\frac x{f(\frac 1x)},\frac 1x)$ $ \implies$ $ f(1 + x) = \frac 1xf(\frac x{f(\frac 1x)} + \frac 1x)$

And so (comparing these two lines) : $ f(1 + f(x)) = f(\frac x{f(\frac 1x)} + \frac 1x)$

And so (using injectivity) : $ 1 + f(x) = \frac x{f(\frac 1x)} + \frac 1x$ and so $ f(\frac 1x) = \frac x{f(x) + 1 - \frac 1x}$

This implies (changing $ x\to\frac 1x$) : $ f(x) = \frac {\frac 1x}{f(\frac 1x) + 1 - x}$

And so $ f(x) = \frac {\frac 1x}{\frac x{f(x) + 1 - \frac 1x} + 1 - x}$

Which gives $ x^2f(x)^2 - 2xf(x) + 1 = 0$

And so $ \boxed{f(x) = \frac 1x}$, which indeed is a solution
\end{solution}



\begin{solution}[by \href{https://artofproblemsolving.com/community/user/101044}{erfan_Ashorion}]
	we khnow that :
1)f(X) is surjective!!by the up solution
2)we proof that f(X) is injective:
suppose that f(x)=f(y) we have:
p(0,x) =>f(x)=f(1)\/x
f(x)=f(y)=>f(1)\/x=f(1)\/y =>x=y ;)
3)we khnow that f(1)=1
4)the proof is end f(x)=1\/x!!!
\end{solution}



\begin{solution}[by \href{https://artofproblemsolving.com/community/user/29428}{pco}]
	\begin{tcolorbox} p(0,x) =>f(x)=f(1)\/x\end{tcolorbox}
$0\notin\mathbb R^+$
\end{solution}



\begin{solution}[by \href{https://artofproblemsolving.com/community/user/103150}{Djurre}]
	\begin{tcolorbox}
$ P(\frac 1{f(\frac {f(2)}x)},\frac {f(2)}x)$ 
\end{tcolorbox}

How do you come up with this? Is there some way to figure this fast out?
\end{solution}



\begin{solution}[by \href{https://artofproblemsolving.com/community/user/29428}{pco}]
	\begin{tcolorbox}[quote="pco"]
$ P(\frac 1{f(\frac {f(2)}x)},\frac {f(2)}x)$ 
\end{tcolorbox}

How do you come up with this? Is there some way to figure this fast out?\end{tcolorbox}
Just tests and trials in order to get $f(something)=x$ :

Setting $x=\frac 1{f(y)}$ we get $f(2)=yf(y+\frac 1{f(y)})$

Setting then $y=\frac{f(2)}z$ we get $f(2)=\frac{f(2)}zf(\frac{f(2)}z+\frac 1{f(\frac{f(2)}z)})$ and so $z=f(\frac{f(2)}z+\frac 1{f(\frac{f(2)}z)})$

It remains to merge these two steps in a unique one.
\end{solution}



\begin{solution}[by \href{https://artofproblemsolving.com/community/user/125018}{horizon}]
	a new way to prove that $f$ is non-increasing
if we have $a<b$,$f(a)<f(b)$,then we find $x_{1},x_{2}$ such that $1+x_{1}a=1+x_{2}f(b)$
$x_{1}+a=x_{2}+b$,let $(a,b)$ be $(x_{1},a)$ and $(x_{2},b)$. a contradiction!
\end{solution}



\begin{solution}[by \href{https://artofproblemsolving.com/community/user/139716}{asjeykg}]
	\begin{tcolorbox}we khnow that :
1)f(X) is surjective!!by the up solution
2)we proof that f(X) is injective:
suppose that f(x)=f(y) we have:
p(0,x) =>f(x)=f(1)\/x
f(x)=f(y)=>f(1)\/x=f(1)\/y =>x=y ;)
3)we khnow that f(1)=1
4)the proof is end f(x)=1\/x!!!\end{tcolorbox}

but we cannot plug in zero for x since the equation holds for positive reals
\end{solution}



\begin{solution}[by \href{https://artofproblemsolving.com/community/user/184873}{amatysten}]
	I just wanted to post my own solution. It's a bit different than pco's.
\begin{tcolorbox}f(1 + xf(y)) = yf(x + y)\end{tcolorbox}
1) f is injective:
$f(m) = f(n) \Rightarrow mf(m + y) = nf(n + y),$

$R_{LHS}(mx', m + y') = R_{LHS}(nx', n + y') \Rightarrow$
$(m + y')f(m + y' + mx') = (n + y')f(n + y' + nx'),$ (1)

$R_{LHS}(2m + 2, m + 1 + 2m) = R_{LHS}(2n + 2, n + 1 + 2n) \Rightarrow$
$(3m + 1)f(5m + 3) = (3n + 1)f(5n + 3),$ (2)

now plug in (1) $(x', y') = (4, 3) \Rightarrow (m + 3)f(5m + 3) = (n + 3)f(5n + 3),$ (3)
dividing (2) \/ (3) we get $\frac{3m + 1}{m + 3} = \frac{3n + 1}{n + 3} \Rightarrow m = n.$
2) $f(1) = 1$ is simple.
3) Ending:
$f(z) = f(1 + \frac{z - 1}{f(p)}f(p)) = pf(p + \frac{z - 1}{f(p)}),$ if z > 1
$f(1 + xf(y)) = yf(y + x), f(y + x) = \frac{1}{y}f(\frac{1}{y} + \frac{y + x - 1}{f(\frac{1}{y})}),$ if $y + x > 1 \Rightarrow$
$1 + xf(y) = \frac{1}{y} + \frac{y + x - 1}{f(\frac{1}{y})}.$ denote this as S(x, y),
$S(2, y) - 2S(1, y): 1 = \frac{1}{y} + \frac{y - 1}{f(\frac{1}{y})} \Rightarrow f(\frac{1}{y}) = y,$ if $y \neq 1.$
\end{solution}



\begin{solution}[by \href{https://artofproblemsolving.com/community/user/200539}{samariddin}]
	почему F (1) = 1?
\end{solution}



\begin{solution}[by \href{https://artofproblemsolving.com/community/user/200539}{samariddin}]
	where F (1) = 1?
\end{solution}



\begin{solution}[by \href{https://artofproblemsolving.com/community/user/184873}{amatysten}]
	The injectivity was proved. Now plugging $x=1,y=1$ we get $f(1+f(1))=f(2)\Rightarrow f(1)=1$.
\end{solution}



\begin{solution}[by \href{https://artofproblemsolving.com/community/user/184652}{CanVQ}]
	\begin{tcolorbox}Find all functions $ \,f: {\mathbb{R^+}}\rightarrow{\mathbb{R^+}}\,$ such that
$ f(1+xf(y))=yf(x+y)\quad (1)$ $ \text{for all}\,x,y\in\mathbb{R^+}.$\end{tcolorbox}
This is my solution:

Replacing $x$ by $\frac{x}{f(y)}$ in $(1),$ we have
\[f(1+x)=y\cdot f\left(\frac{x}{f(y)}+y\right),\quad \forall x ,\,y \in \mathbb R^+. \quad (2)\]
Next, we replace $y$ by $\frac{f(1+x)}{y}$ in $(2)$ to get
\[y=f\left(\frac{x}{f\left(\frac{f(1+x)}{y}\right)}+\frac{f(1+x)}{y}\right),\quad \forall x,\,y \in \mathbb R^+. \quad (3)\]
From $(3),$ it follows that $f$ is surjective. Now, we will prove that $f$ is decreasing. Replacing $x$ by $x+z$ in $(1),$ we get
\[f\big(1+(x+z)\cdot f(y)\big)=y\cdot f(x+y+z),\quad \forall x,\,y,\,z \in \mathbb R^+. \quad (4)\]
Replacing $y$ by $y+z$ in $(1),$ we also have
\[f\big(1+x\cdot f(y+z)\big)=(y+z)\cdot f(x+y+z),\quad \forall x,\,y,\,z \in \mathbb R^+. \quad  (5)\]
Dividing $(4)$ for $(5),$ side by side, we obtain
\[\frac{f\big(1+(x+z)\cdot f(y)\big)}{f\big(1+x\cdot f(y+z)\big)}=\frac{y}{y+z},\quad \forall x,\,y,\,z \in \mathbb R^+. \quad (6)\]
Now, assume that there exists a pair $(y,\,z)$ such that $f(y+z)>f(y).$ In this case, by choosing $x=\frac{z\cdot f(y)}{f(y+z)-f(y)}$ vào $(6),$ we obtain $y=y+z,$ which is a contradiction. So we must have
\[f(y+z) \le f(y),\quad \forall y,\,z \in \mathbb R^+. \quad (7)\]
Now, we will prove that $f$ is injective. Assume that there are two numbers $a,\,b$ such that $f(a)=f(b).$ Replacing $y=a$ and $y=b$ in $(1)$ respectively in $(1),$ we get
\[a\cdot f(x+a)=b\cdot f(x+b),\quad \forall x\in \mathbb R^+. \quad (8)\]
From this, it follows that
\[1+a(y-1)\cdot f(x+a)=1+b(y-1)\cdot f(x+b),\quad \forall x, \, y \in \mathbb R^+,\, y>1. \quad  (9)\]
Plugging this into $f$ and using $(1),$ we get
\[(x+a)\cdot f(x+ay)=(x+b)\cdot f(x+by),\quad \forall x,\,y \in \mathbb R^+ ,\, y>1. \quad (10)\]
From $(10),$ it follows that, for any $x,\,y,\,z \in \mathbb R^+,\, y>1,$
\[1+(xz+az)\cdot f(x+ay)=1+(xz+bz)\cdot f(x+by). \quad (11)\]
Again, we plug this into $f$ and using $(1).$ It follows that
\[(x+ay)\cdot f(x+ay+az+xz)=(x+by)\cdot f(x+by+bz+xz)\quad  (12)\]
for any $x,\,y,\,z \in \mathbb R^+$ and $ y>1.$ On the other hand, according to $(10),$ we also have
\[\big[(x+xz)+a\big]\cdot f\big( (x+xz)+a(y+z)\big)=\big[(x+xz)+b\big] \cdot f\big((x+xz)+b(y+z)\big),\]
or
\[(x+xz+a)\cdot f(x+ay+az+xz)=(x+xz+b)\cdot f(x+ay+az+xz). \quad (13)\]
Dividing $(12)$ for $(13),$ side by side, we obtain
\[\frac{x+ay}{x+xz+a}=\frac{x+by}{x+xz+b},\quad \forall x,\,y,\,z \in \mathbb R^+,\, y>1. \quad (14)\]
It is easy to deduce that $a=b$ here, so $f $ is injective. Now, replacing $x=y=1$ in $(1)$ with notice that $f $ is injective, we  have $f(1)=1.$ Since $f$ is strictly decreasing ($f$ is decreasing and injective), we have
\[f(x)<1,\quad  \forall x>1.\quad (15)\]
Now, let us consider the case $y>x.$ Replacing $y$ by $y-x$ in $(1)$ and using the above remark, we get
\[f(y)=\frac{f\big(1+x\cdot f(y-x)\big)}{y-x}<\frac{1}{y-x},\quad \forall x,\,y \in \mathbb R^+,\, x<y. \quad  (16)\]
In $(16),$ we let $x\to 0^+$ and obtain
\[f(y) \le \frac{1}{y},\quad \forall y >0. \quad (17)\] 
Next, replacing $x$ by $x-1$ and $y$ by $f(y)$ in $(3),$ we get
\[y=\frac{x-1}{f\left(\frac{f(x)}{f(y)}\right)}+\frac{f(x)}{f(y)},\quad \forall x ,\, y \in \mathbb R^+,\, x >1. \quad (18)\]
Since $f\left(\frac{f(x)}{f(y)}\right) \le \frac{f(y)}{f(x)},$ from $(18),$ we deduce that
\[y\ge \frac{(x-1)\cdot f(x)}{f(y)}+\frac{f(x)}{f(y)},\]
or
\[y\cdot f(y) \ge x \cdot f(x),\quad \forall x,\,y \in \mathbb R^+,\, x>1. \quad (19)\]
Changin the position of $x$ and $y$ in $(19),$ we also have
\[x\cdot f(x) \ge y\cdot f(y),\quad \forall x,\,y \in \mathbb R^+,\, y>1. \quad (20)\]
From the inequalities $(19)$ and $(20),$ we can easily deduce that
\[f(x)=\frac{k}{x},\quad \forall x>1. \quad (21)\]
Now, taking $x=1$ in $(1)$ and using $(21),$ we have
\[\frac{1}{1+f(y)}=\frac{y}{1+y},\]
or
\[f(y)=\frac{1}{y},\quad \forall y\in \mathbb R^+. \quad (22)\]
Clearly, the function $f(x)=\frac{1}{x}$ satisfies our equation.
\end{solution}



\begin{solution}[by \href{https://artofproblemsolving.com/community/user/231508}{john10}]
	\begin{tcolorbox}[quote="Leonhard Euler"]Find all functions $ \,f: {\mathbb{R^ + }}\rightarrow{\mathbb{R^ + }}\,$ such that
$ f(1 + xf(y)) = yf(x + y)$ $ \text{for all}\,x,y\in\mathbb{R^ + }.$\end{tcolorbox}
Let $ P(x,y)$ be the assertion $ f(1 + xf(y)) = yf(x + y)$

1) $ f(x)$ is a surjective function
=====================
$ P(\frac 1{f(\frac {f(2)}x)},\frac {f(2)}x)$ $ \implies$ $ f(2) = \frac {f(2)}xf(\frac 1{f(\frac {f(2)}x)} + \frac {f(2)}x)$

And so $ x = f(\text{something})$
Q.E.D.



2) $ f(x)$ is an injective function
=====================
Let $ a > b > 0$ such that $ f(a) = f(b)$
Let $ T = b - a > 0$
Comparing $ P(x,a)$ and $ P(x,b)$, we get $ af(x + a) = bf(x + b)$ and so $ f(x) = \frac baf(x + T)$ $ \forall x > a$
And so $ f(x) = \left(\frac ba\right)^nf(x + nT)$ $ \forall x > a,n\in\mathbb N$

Let then $ y$ such that $ f(y) > 1$ (such $ y$ exists since $ f(x)$ is a surjection, according to 1) above)
Let $ n$ great enough to have $ y + nT - 1 > 0$

$ P(\frac {y + nT - 1}{f(y) - 1},y)$ $ \implies$ $ f(1 + \frac {yf(y) + (nT - 1)f(y)}{f(y) - 1}) = yf(\frac {y + nT - 1}{f(y) - 1} + y)$ which may be written :

$ f(\frac {yf(y) + nT - 1}{f(y) - 1} + nT) = yf(\frac {yf(y) + nT - 1}{f(y) - 1})$

and since $ f(\frac {yf(y) + nT - 1}{f(y) - 1} + nT) = \left(\frac ab\right)^nf(\frac {yf(y) + nT - 1}{f(y) - 1})$, we get $ y = \left(\frac ab\right)^n$ $ \forall n$, which is impossible
Q.E.D.

3) $ f(1) = 1$
===========
$ P(1,1)$ $ \implies$ $ f(1 + f(1)) = f(2)$ and so, since $ f(x)$ is injective, $ f(1) = 1$
Q.E.D.


4) The only solution is $ f(x) = \frac 1x$
==========================
$ P(1,x)$ $ \implies$ $ f(1 + f(x)) = xf(1 + x)$ and so $ f(1 + x) = \frac 1xf(1 + f(x))$

$ P(\frac x{f(\frac 1x)},\frac 1x)$ $ \implies$ $ f(1 + x) = \frac 1xf(\frac x{f(\frac 1x)} + \frac 1x)$

And so (comparing these two lines) : $ f(1 + f(x)) = f(\frac x{f(\frac 1x)} + \frac 1x)$

And so (using injectivity) : $ 1 + f(x) = \frac x{f(\frac 1x)} + \frac 1x$ and so $ f(\frac 1x) = \frac x{f(x) + 1 - \frac 1x}$

This implies (changing $ x\to\frac 1x$) : $ f(x) = \frac {\frac 1x}{f(\frac 1x) + 1 - x}$

And so $ f(x) = \frac {\frac 1x}{\frac x{f(x) + 1 - \frac 1x} + 1 - x}$

Which gives $ x^2f(x)^2 - 2xf(x) + 1 = 0$

And so $ \boxed{f(x) = \frac 1x}$, which indeed is a solution\end{tcolorbox}
T=b-a<0 , and this will change the condition for n such that it be great enough to have y+nT-1>0 . Can you correct please ?  :blush: 


\end{solution}



\begin{solution}[by \href{https://artofproblemsolving.com/community/user/334227}{reveryu}]
	How can we get the contradiction from  \begin{tcolorbox}$ y = \left(\frac ab\right)^n$   \end{tcolorbox}   could you please explain this a bit more?
\end{solution}



\begin{solution}[by \href{https://artofproblemsolving.com/community/user/29428}{pco}]
	The contradiction is not obtained from $y=\left(\frac ab\right)^n$

It is obtained from $y=\left(\frac ab\right)^n$ $\boxed{\forall n}$
Since $a\ne b$, such equality can not be true $\forall n$. Is it really necessary to explain more ?

\end{solution}



\begin{solution}[by \href{https://artofproblemsolving.com/community/user/334227}{reveryu}]
	thanks . :-D
\end{solution}



\begin{solution}[by \href{https://artofproblemsolving.com/community/user/135388}{giorgigona}]
	\begin{tcolorbox}[quote="Leonhard Euler"]Find all functions $ \,f: {\mathbb{R^ + }}\rightarrow{\mathbb{R^ + }}\,$ such that
$ f(1 + xf(y)) = yf(x + y)$ $ \text{for all}\,x,y\in\mathbb{R^ + }.$\end{tcolorbox}
Let $ P(x,y)$ be the assertion $ f(1 + xf(y)) = yf(x + y)$

1) $ f(x)$ is a surjective function
=====================
$ P(\frac 1{f(\frac {f(2)}x)},\frac {f(2)}x)$ $ \implies$ $ f(2) = \frac {f(2)}xf(\frac 1{f(\frac {f(2)}x)} + \frac {f(2)}x)$

And so $ x = f(\text{something})$
Q.E.D.



2) $ f(x)$ is an injective function
=====================
Let $ a > b > 0$ such that $ f(a) = f(b)$
Let $ T = b - a > 0$
Comparing $ P(x,a)$ and $ P(x,b)$, we get $ af(x + a) = bf(x + b)$ and so $ f(x) = \frac baf(x + T)$ $ \forall x > a$
And so $ f(x) = \left(\frac ba\right)^nf(x + nT)$ $ \forall x > a,n\in\mathbb N$

Let then $ y$ such that $ f(y) > 1$ (such $ y$ exists since $ f(x)$ is a surjection, according to 1) above)
Let $ n$ great enough to have $ y + nT - 1 > 0$

$ P(\frac {y + nT - 1}{f(y) - 1},y)$ $ \implies$ $ f(1 + \frac {yf(y) + (nT - 1)f(y)}{f(y) - 1}) = yf(\frac {y + nT - 1}{f(y) - 1} + y)$ which may be written :

$ f(\frac {yf(y) + nT - 1}{f(y) - 1} + nT) = yf(\frac {yf(y) + nT - 1}{f(y) - 1})$

and since $ f(\frac {yf(y) + nT - 1}{f(y) - 1} + nT) = \left(\frac ab\right)^nf(\frac {yf(y) + nT - 1}{f(y) - 1})$, we get $ y = \left(\frac ab\right)^n$ $ \forall n$, which is impossible
Q.E.D.

3) $ f(1) = 1$
===========
$ P(1,1)$ $ \implies$ $ f(1 + f(1)) = f(2)$ and so, since $ f(x)$ is injective, $ f(1) = 1$
Q.E.D.


4) The only solution is $ f(x) = \frac 1x$
==========================
$ P(1,x)$ $ \implies$ $ f(1 + f(x)) = xf(1 + x)$ and so $ f(1 + x) = \frac 1xf(1 + f(x))$

$ P(\frac x{f(\frac 1x)},\frac 1x)$ $ \implies$ $ f(1 + x) = \frac 1xf(\frac x{f(\frac 1x)} + \frac 1x)$

And so (comparing these two lines) : $ f(1 + f(x)) = f(\frac x{f(\frac 1x)} + \frac 1x)$

And so (using injectivity) : $ 1 + f(x) = \frac x{f(\frac 1x)} + \frac 1x$ and so $ f(\frac 1x) = \frac x{f(x) + 1 - \frac 1x}$

This implies (changing $ x\to\frac 1x$) : $ f(x) = \frac {\frac 1x}{f(\frac 1x) + 1 - x}$

And so $ f(x) = \frac {\frac 1x}{\frac x{f(x) + 1 - \frac 1x} + 1 - x}$

Which gives $ x^2f(x)^2 - 2xf(x) + 1 = 0$

And so $ \boxed{f(x) = \frac 1x}$, which indeed is a solution\end{tcolorbox}
pco you have mistake in second part. at first T < 0. if you take f(y) > 1 than y<1 definitely. so you can not take such y that y+nT-1>0.
\end{solution}



\begin{solution}[by \href{https://artofproblemsolving.com/community/user/220735}{Leicich}]
	A needlessly long and complicated solution, but a solution nonetheless. As always, let $P(x,y)$ be the assertion $f(1+xf(y))=yf(x+y)$
	
Part 1: If $f(y) \neq 1$ and $y \neq 1$, then $y < 1 \iff f(y) > 1$
	[hide]
Assume otherwise. Then $\frac{y-1}{f(y)-1}$ is a positive real and
	
$P(\frac{y-1}{f(y)-1},y) \implies f(1+\frac{y-1}{f(y)-1} \cdot y) = yf(\frac{y-1}{f(y)-1} + y) \implies y = 1$
	
Therefore, if $f(y) \neq 1$, then $y > 1 \iff f(y) < 1$ and $y < 1 \iff f(y) > 1$. One important conclusion is that, if we don't move around the functions, then the $LHS$ in $P(x,y)$ is $\leq 1$, and so is the $RHS$.[\/hide]
	
Part 2: $xf(x) \leq 1$
	
[hide]Assume $xf(x) = 1 + \epsilon$, where $\epsilon > 0$. Then
	
$P(x-\frac{x}{1+0.5\epsilon},\frac{x}{1+0.5\epsilon}) \implies f(1+(x-\frac{x}{1+0.5\epsilon})f(\frac{x}{1+0.5\epsilon})) = \frac{x}{1+0.5\epsilon}f(x-\frac{x}{1+0.5\epsilon}+\frac{x}{1+0.5\epsilon}) \implies something\ at\ most\ 1 = \frac{1}{1+0.5\epsilon} \cdot (1+\epsilon) = something\ greater\ than\ 1$, contradiction

Therefore, $xf(x) \leq 1$.[\/hide]
	
Part 3: $f^{-1}(1)$ exists
	
[hide]$P(\frac{1}{f(f(2))},f(2)) \implies f(1+\frac{1}{f(f(2))} \cdot f(2)) = f(2)(\frac{1}{f(f(2))}+f(2)) \implies f(\frac{1}{f(f(2))} + f(2)) = 1 $[\/hide]
	
Part 4: $f(1) = 1$
	
[hide]By part 2, we know that $f(1) \leq 1$. Assume $f(1) < 1$. Then
	
$P(\frac{1}{f(1)},1) \implies f(2) = f(\frac{1}{f(1)}+1)$
	
$P(\frac{1}{(f(1))^n},1) \implies f(1+\frac{1}{(f(1))^{n-1}})=f(\frac{1}{(f(1))^n}+1)$
	
Then by induction $f(2) = f(\frac{1}{(f(1))^n}+1)$ for all positive integers $n$.
	
Let $f(2) = c$. By part 2, we know that $c = \frac{1}{2} - \epsilon$ for some non-negative $\epsilon$ less than $\frac{1}{2}$. We also know by part 2 that $f(\frac{1}{(f(1))^n}+1) \leq \frac{(f(1))^n}{(f(1))^n+1} = 1-\frac{1}{(f(1))^n+1}$. 
	
But $0 < f(1) < 1$ implies that $\lim_{n \to +\infty}(1-\frac{1}{(f(1))^n+1}) = 0$. Then we can pick $n$ sufficiently large such that $f(\frac{1}{f(1)^n}+1) \leq 1-\frac{1}{(f(1))^n+1} < \frac{1}{2} - \epsilon$, which contradicts the induction result. 
	
Therefore, $f(1) = 1$[\/hide]
	
Part 5: $f^{-1}(1) = \{1\}$
	
[hide]Let $f(\lambda)=1$. By part 2, we know that $\lambda \leq 1$. Assume there is a $\lambda$ value less than one such that $f(\lambda)=1$.
	
$P(1-\lambda,\lambda) \implies f(1+(1-\lambda)f(\lambda))=\lambda f(1-\lambda+\lambda) \implies f(2-\lambda) = \lambda$
	
$P(\frac{1-\lambda}{\lambda},2-\lambda) \implies f(1+\frac{1-\lambda}{\lambda}f(2-\lambda)) = (2-\lambda)f(\frac{1-\lambda}{\lambda}+2-\lambda) \implies \frac{\lambda}{2-\lambda}=f(1+\frac{1}{\lambda}-\lambda)$
	
$P(\frac{1}{\lambda}-\lambda,\lambda) \implies f(1+(\frac{1}{\lambda}-\lambda)f(\lambda))=\lambda f(\frac{1}{\lambda}-\lambda+\lambda) \implies \frac{\lambda}{2-\lambda} = \lambda f(\frac{1}{\lambda}) \implies f(\frac{1}{\lambda}) = \frac{1}{2-\lambda}$
	
But by part 2 we know that $f(\frac{1}{\lambda}) \leq \lambda$. Therefore $\lambda (2-\lambda) \geq 1$. But by part 2 as well we know that $(2-\lambda)f(2-\lambda)=(2-\lambda) \lambda \leq 1$. Therefore $\lambda (2-\lambda)=1$, which implies $\lambda=1$, contradiction.
	
Therefore, $f^{-1} = \{1\}$, and this implies that $f(x)=1 \iff x=1$. Furthermore, this strengthens part 1, because we may now say $y > 1 \iff f(y) < 1$ and $y < 1 \iff f(y) > 1$.[\/hide]
	
Part 6: $f$ is strictly decreasing from $x > 1$
	
[hide]Assume $x > 1$. Then
	
$P(\frac{x-1}{f(f(x))},f(x)) \implies f(1+\frac{x-1}{f(f(x))}f(f(x))) = f(x)f(\frac{x-1}{f(f(x))}+f(x)) \implies f(\frac{x-1}{f(f(x))}+f(x)) = 1$
	
By part 5, we get that $\frac{x-1}{f(f(x))} = 1-f(x)$
	
Let $\epsilon > 0$. Then,
	
$P(\frac{x-1+\epsilon}{f(f(x))},f(x)) \implies f(1+\frac{x-1+\epsilon}{f(f(x))}f(f(x)))=f(x)f(1-f(x)+\frac{\epsilon}{f(f(x))}+f(x)) \implies f(x+\epsilon)=f(x)f(1+\frac{\epsilon}{f(f(x))})$
	
By the corollary of part 5, we get that $f(1+\frac{\epsilon}{f(f(x))}) < 1$. Therefore, $f(x+\epsilon)<f(x) \ \forall \epsilon > 0$. 
	
Now assume there exists $x,y > 1$ where $x < y$ and $f(x) \leq f(y)$. Then, take $\epsilon=y-x$. Therefore, $f(x+\epsilon)=f(y)<f(x) \leq f(y)$, contradiction. Therefore, no such $x,y$ exists and $f$ is strictly decreasing from $x > 1$.[\/hide]
	
Part 7: $xf(2x) \geq f(2)$
	
[hide]By part 2, $1+xf(x) \leq 2$. Then, by part 6, $f(1+xf(x)) \geq f(2)$. But
	
$P(x,x) \implies f(1+xf(x))=xf(2x)$
	
Therefore, $xf(2x) \geq f(2)$[\/hide]
	
Part 8: $f(2) = \frac{1}{2}$
	
[hide]By the proof in part 2, $f(\frac{1}{f(f(2))} + f(2)) = 1$. By part 5, $\frac{1}{f(f(2))} + f(2) = 1$. But using part 2 we get $f(2) \leq \frac{1}{2}$. Putting this in the previous equation we get that $f(f(2)) \leq 2$.
	
However, using part 7 with $x = \frac{f(2)}{2}$ we get that $\frac{f(2)}{2} f(2\frac{f(2)}{2}) \geq f(2) \implies f(f(2)) \geq 2$. Therefore, $f(f(2))=2$. Plugging this in the previous equation yields $f(2)=\frac{1}{2}$.
	
A corollary is that we may strengthen part 7, because we now have that $xf(2x) \geq \frac{1}{2}$[\/hide]
	
Part 9: $f(x) = \frac{1}{x}$
	
[hide]Using the corollary from part 8 and multiplying it by $2$ we get that $2xf(2x) \geq 1$. But, using part 2 we have that $2xf(2x) \leq 1$. Therefore, $2xf(2x) = 1$. We may reassign $2x \to z$ and we get $f(z) = \frac{1}{z}$, which clearly works, because
	
$f(1+xf(y))=f(1+\frac{x}{y})=\frac{y}{x+y}=yf(x+y)$[\/hide]
\end{solution}



\begin{solution}[by \href{https://artofproblemsolving.com/community/user/391072}{akmathworld}]
	The domain and co-domain of the function is only the set of positive real numbers .But you found that f(x)=1\/x and the domain and  co-domain of this function is the set of real numbers which contradicts the condition.Will you please explain it? :blush:  :love:  :spam: :wallbash_red: 
\end{solution}



\begin{solution}[by \href{https://artofproblemsolving.com/community/user/220735}{Leicich}]
	Proving that $f(x)$ is of a certain form (and seeing that it verifies the given equation) does not imply anything about a well-defined domain and the codomain. If anything, the function has to be defined in the entire domain in order for it to be a function. While $f(x)=1\/x$ works for positive and negative reals, proving $f(x)=1\/x$ does not mean the domain has to include negative values.
\end{solution}
*******************************************************************************
-------------------------------------------------------------------------------

\begin{problem}[Posted by \href{https://artofproblemsolving.com/community/user/67223}{Amir Hossein}]
	1. Find all functions $f: \mathbb R \to \mathbb R$ such that for all reals $x$ and $y$,
\[ f(xy) = yf(x) + xf(y).\]
2. Find all functions $f: \mathbb R \to \mathbb R$ satisfying the following functional equation for every real number $x$:
\[ f(x) + \left(x + \frac {1}{2}\right)f(1 - x) = 1.\]
3. Find all functions$f: \mathbb R \to \mathbb R$ such that for all real numbers $x$ we have
\[ f(2x)+f(1-x) = 5x+9.\]
	\flushright \href{https://artofproblemsolving.com/community/c6h323393}{(Link to AoPS)}
\end{problem}



\begin{solution}[by \href{https://artofproblemsolving.com/community/user/29428}{pco}]
	\begin{tcolorbox}$ 1$.Find all functions $ f: R\rightarrow R$ such that for every $ x,y$ from real numbers we have:
$ f(xy) = yf(x) + xf(y)$\end{tcolorbox}

Let $ P(x,y)$ be the assertion $ f(xy)=yf(x)+xf(y)$

$ P(2,0)$ $ \implies$ $ f(0)=0$
$ P(x,x)$ $ \implies$ $ f(x^2)=2xf(x)$
$ P(-x,-x)$ $ \implies$ $ f(x^2)=-2xf(-x)$ 
So $ f(-x)=-f(x)$ $ \forall x\ne 0$ and so $ f(-x)=-f(x)$ $ \forall x$

Let then $ h(x)=e^{-x}f(e^x)$

So, $ \forall x>0$ : $ h(\ln(x))=\frac{f(x)}{x}$

So, $ \forall x\ne 0$ : $ h(\ln(|x|))=\frac{f(x)}{x}$ and $ f(x)=xh(\ln(|x|)$

Then the equation becomes $ xyh(\ln(|xy|)=xyh(\ln(|x|)+xyh(\ln(|y|)$ and so $ h(u+v)=h(u)+h(v)$ $ \forall u,v$ and so $ h(x)$ is any solution of Cauchy's equation.

And it is easy to check back that this indeed is a solution.

Hence the result :
Let $ h(x)$ be any solution of Cauchy's equation. Then :
$ f(0)=0$
$ f(x)=xh(\ln(|x|)$ $ \forall x\ne 0$

If you require continuity, then $ h(x)=ax$ and the only continuous solutions are $ ax\ln(|x|)$ (with continuous extension value $ 0$ at $ x=0$)
\end{solution}



\begin{solution}[by \href{https://artofproblemsolving.com/community/user/29428}{pco}]
	\begin{tcolorbox}problem $ 2$.find all functions $ f: R\rightarrow R$ satisfying the following functional equation:
$ f(x) + (x + \frac {1}{2})f(1 - x) = 1$   (for every real number $ x$)\end{tcolorbox}

Let $ P(x)$ be the assertion $ f(x)+(x+\frac 12)f(1-x)=1$

$ P(1-x)$ $ \implies$ $ f(1-x)+(\frac 32-x)f(x)=1$

Multiplying this equality by $ (x+\frac 12)$, we get $ (x+\frac 12)f(1-x)+(x+\frac 12)(\frac 32-x)f(x)=x+\frac 12$

Subtracting $ P(x)$ from this equality implies $ (x+\frac 12)(\frac 32-x)f(x)-f(x)=x-\frac 12$

And so ${ \boxed{f(x)=\frac 2{1-2x}\forall x\ne\frac 12}}$ and $ P(\frac 12)$ $ \implies$ $ \boxed{f(\frac 12)=\frac 12}$

and this indeed is a solution (quick easy check)

\begin{bolded}About problem 1\end{bolded}\end{underlined}, just point the parts you dont understand.
\end{solution}



\begin{solution}[by \href{https://artofproblemsolving.com/community/user/29428}{pco}]
	\begin{tcolorbox}problem $ 3$.find all functions $ f: R\rightarrow R$ such that for every real number $ x$ we have:
$ f(2x) + f(1 - x) = 5x + 9$\end{tcolorbox}

First let us find a solution in the form $ ax+b$. A quick identification gives $ f(x)=5x+2$

Let then $ g(x)=f(x)-5x-2$ and we get $ g(2x)+g(1-x)=0$

Let then $ h(x)=g(x+\frac 23)$ and we get $ h(2(x-\frac 13))+h(\frac 13-x)=0$ and so $ h(2x)=-h(-x)$

As a consequence : $ h(4x)=h(x)$ and we get a general solution for $ h(x)$ :

Let $ u(x)$ any function defined over $ [0,1)$ :
$ \forall x>0$ : $ h(x)=u(\{\log_4(x)\})$
$ h(0)=0$
$ \forall x<0$ : $ h(x)=-h(-2x)=-u(\{\log_4(-2x)\})$

And the general solution of the requested equation is $ f(x)=5x+2+h(x-\frac 23)$ for any $ h(x)$ defined as above.

Once again, if you add continuity constraint, $ h(4x)=h(x)$ implies $ h(x)=c$ and then $ h(x)=0$ and the only continuous solution is $ 5x+2$
\end{solution}
*******************************************************************************
-------------------------------------------------------------------------------

\begin{problem}[Posted by \href{https://artofproblemsolving.com/community/user/37845}{gks921217}]
	1. Find all $ f: (1, \inf ) \to \mathbb R$ which satisfy \[f(xy)(x-y)=f(y)-f(x)\]  for all $x,y>1$.

2. Find all $ f: \mathbb R^+ \to \mathbb R^+$ such that \[f\left( \frac{x-y}{f(y)} \right)=\frac{f(y)}{f(x)}\] for all $x>y>0$.
	\flushright \href{https://artofproblemsolving.com/community/c6h323638}{(Link to AoPS)}
\end{problem}



\begin{solution}[by \href{https://artofproblemsolving.com/community/user/29428}{pco}]
	\begin{tcolorbox}1. $ f: (1, \inf ) \rightarrow R, f(xy)(x - y) = f(y) - f(x) \forall x,y > 1$. Find all $ f$.\end{tcolorbox}

Let $ P(x,y)$ be the assertion $ f(xy)(x - y) = f(y) - f(x)$

1) If $ \exists u > v > 1$ such that $ f(u) = f(v) = 0$, then $ f(x) = 0$ $ \forall x > 1$
================================================
Adding $ P(x,y)$, $ P(y,z)$ and $ P(z,x)$, we get a new assertion $ Q(x,y,z)$ : $ f(xy)(x - y) + f(yz)(y - z) + f(zx)(z - x) = 0$

Suppose now $ \exists u > v > 1$ such that $ f(u) = f(v) = 0$. Let then $ w\in(\frac uv,uv)$ : $ Q(\sqrt {\frac {vw}u},\sqrt {\frac {uw}v},\sqrt {\frac {uv}w})$ $ \implies$ $ f(w) = 0$ $ \forall w\in(\frac uv,uv)$
And since $ P(u,v)$ $ \implies$ $ f(uv) = 0$ and $ P(\frac uv,v)$ $ \implies$ $ f(\frac uv) = 0$, we get : $ f(x) = 0$ $ \forall x\in[\frac uv,uv]$

Taking then $ \epsilon$ such that $ u - \frac uv > \epsilon > 0$, we have $ uv > u > u - \epsilon > \frac uv$ and so $ f(u) = f(u - \epsilon) = 0$ and so, using the same method : $ f(x) = 0$ $ \forall x\in[\frac {u}{u - \epsilon},u(u - \epsilon)]$
And so, taking $ \epsilon$ as little as we want : $ f(x) = 0$ $ \forall x\in(1,u^2)$
And so (just replay the mechanism with $ u^2,v$, then $ u^4,v$, ... : $ f(x) = 0$ $ \forall x > 1$
Q.E.D.

2) The only solutions are $ f(x) = \frac ax$
==========================
Clearly these functions are solutions.
Let then $ g(x)$ any solution 
The function $ f(x) = g(x) - \frac {3g(3)}{x}$ is clearly a solution and is such that $ f(3) = 0$
Then $ P(3,2)$ $ \implies$ $ f(6) = f(2)$
Then $ P(6,2)$ $ \implies$ $ f(12) = 0$

And so $ f(3) = f(12) = 0$ and so $ f(x) = 0$ $ \forall x > 0$, according to 1) above.

And so $ g(x) = \frac {3g(3)}{x}$
Q.E.D.

Hence the unique family of solutions $ \boxed{f(x) = \frac ax}$
\end{solution}



\begin{solution}[by \href{https://artofproblemsolving.com/community/user/29428}{pco}]
	\begin{tcolorbox}2. $ f: R^ + \rightarrow R^ + , f( \frac {x - y}{f(y)} ) = \frac {f(y)}{f(x)} \forall x > y > 0$. Find all $ f$.\end{tcolorbox}

Let $ P(x,y)$ be the assertion $ f(\frac{x-y}{f(y)})=\frac{f(y)}{f(x)}$

Let $ 1+f(1)=a$

$ P(1+f(1),1)$ $ \implies$ $ f(1)=\frac{f(1)}{f(1+f(1))}$ and so $ f(1+f(1))=1$ and so $ f(a)=1$

$ P(2a,a)$ $ \implies$ $ 1=\frac 1{f(2a)}$ and so $ f(2a)=1$

$ P(x,a)$ $ \implies$ $ f(x-a)=\frac 1{f(x)}$ $ \forall x>a$ and so $ f(x-2a)=\frac 1{f(x-a)}=f(x)$ $ \forall x>2a$

$ P(x,2a)$ $ \implies$ $ f(x-2a)=\frac 1{f(x)}$ $ \forall x>2a$

Comparing these two lines, we get $ f(x)=1$ $ \forall x>2a$

So $ f(2a+x)=1$ $ \forall x>0$ and so $ P(2a+x,2a)$ $ \implies$ $ f(x)=1$ $ \forall x>0$

And this indeed is a solution. Hence the result : $ \boxed{f(x)=1\forall x}$
\end{solution}
*******************************************************************************
-------------------------------------------------------------------------------

\begin{problem}[Posted by \href{https://artofproblemsolving.com/community/user/64868}{mahanmath}]
	Find all functions $f: \mathbb N \to \mathbb N$ such that
\[f(n+f(m))=f(n)+m\]
holds for all $m,n \in \mathbb N$.
	\flushright \href{https://artofproblemsolving.com/community/c6h323817}{(Link to AoPS)}
\end{problem}



\begin{solution}[by \href{https://artofproblemsolving.com/community/user/44083}{jgnr}]
	Substitute $ n: =f(n)$, $ f(f(n)+f(m))=f(f(n))+m$. By symmetry of LHS, $ f(f(n))+m=f(f(m))+n$, so $ f(f(n))=n+k$, where $ k$ is a constant. It is well-known (and has been posted many times) that functions on natural numbers which satisfies $ f(f(n))=n+k$ has a unique solution $ f(n)=n+\frac{k}2$, where $ k$ is even. Let $ \frac{k}2=a$. Substitute this to the given equation, we get $ a=0$, so $ f(n)=n$ for all $ n$.
\end{solution}



\begin{solution}[by \href{https://artofproblemsolving.com/community/user/50028}{hophinhan}]
	\begin{tcolorbox}Find all function $ f: N\longrightarrow N$ such that :

$ f(n + f(m)) = f(n) + m$\end{tcolorbox}

Let $ P(x,y)$ be the assertion $ f(n + f(m)) = f(n) + m$

$ \blacksquare \ f(m_1) = f(m_2)$ . $ P(n,m_1)$ and $ P(n,m_2) \Longrightarrow m_1 = m_2$ . It's mean $ f(x)$ is an injective funtion.

$ P(0,0) \Longrightarrow f(f(0)) = f(0) \Longrightarrow f(0) = 0$

$ \blacksquare P(0,m) \Longrightarrow f(f(m)) = m \ \ \forall n\in\mathbb{N}$

$ P(n,f(m))\ \Longrightarrow f(n + f(f(m))) = f(n) + f(m)$
$ \Longrightarrow\ f(n + m) = f(n) + f(m) \ \ \forall m,n \in \mathbb{N}$

We get : $ f(n) = n \ \forall n\in \mathbb{N}$
\end{solution}



\begin{solution}[by \href{https://artofproblemsolving.com/community/user/44083}{jgnr}]
	\begin{tcolorbox}[quote="mahanmath"]Find all function $ f: N\longrightarrow N$ such that :

$ f(n + f(m)) = f(n) + m$\end{tcolorbox}

Let $ P(x,y)$ be the assertion $ f(n + f(m)) = f(n) + m$

$ \blacksquare \ f(m_1) = f(m_2)$ . $ P(n,m_1)$ and $ P(n,m_2) \Longrightarrow m_1 = m_2$ . It's mean $ f(x)$ is an injective funtion.

$ P(0,0) \Longrightarrow f(f(0)) = f(0) \Longrightarrow f(0) = 0$

$ \blacksquare P(0,m) \Longrightarrow f(f(m)) = m \ \ \forall n\in\mathbb{N}$

$ P(n,f(m))\ \Longrightarrow f(n + f(f(m))) = f(n) + f(m)$
$ \Longrightarrow\ f(n + m) = f(n) + f(m) \ \ \forall m,n \in \mathbb{N}$

We get : $ f(n) = n \ \forall n\in \mathbb{N}$\end{tcolorbox}N does not contain 0.

anyway, i haven't found the topic in which the function $ f(f(n))=n+k$ is solved, so my proof is not finished yet...
\end{solution}



\begin{solution}[by \href{https://artofproblemsolving.com/community/user/50028}{hophinhan}]
	\begin{tcolorbox}[quote="hophinhan"][quote="mahanmath"]Find all function $ f: N\longrightarrow N$ such that :

$ f(n + f(m)) = f(n) + m$\end{tcolorbox}

Let $ P(x,y)$ be the assertion $ f(n + f(m)) = f(n) + m$

$ \blacksquare \ f(m_1) = f(m_2)$ . $ P(n,m_1)$ and $ P(n,m_2) \Longrightarrow m_1 = m_2$ . It's mean $ f(x)$ is an injective funtion.

$ P(0,0) \Longrightarrow f(f(0)) = f(0) \Longrightarrow f(0) = 0$

$ \blacksquare P(0,m) \Longrightarrow f(f(m)) = m \ \ \forall n\in\mathbb{N}$

$ P(n,f(m))\ \Longrightarrow f(n + f(f(m))) = f(n) + f(m)$
$ \Longrightarrow\ f(n + m) = f(n) + f(m) \ \ \forall m,n \in \mathbb{N}$

We get : $ f(n) = n \ \forall n\in \mathbb{N}$\end{tcolorbox}N does not contain 0.

anyway, i haven't found the topic in which the function $ f(f(n)) = n + k$ is solved, so my proof is not finished yet...\end{tcolorbox}
 :rotfl:    "N does not contain 0"...
\end{solution}



\begin{solution}[by \href{https://artofproblemsolving.com/community/user/44083}{jgnr}]
	http://en.wikipedia.org\/wiki\/Natural_number
\end{solution}



\begin{solution}[by \href{https://artofproblemsolving.com/community/user/64868}{mahanmath}]
	\begin{bolded}Johan Gunardi \end{bolded}, You are right.$ 0 \notin {N}$ (At least in this problem  :P  !!)
\end{solution}



\begin{solution}[by \href{https://artofproblemsolving.com/community/user/62475}{hqthao}]
	dear Gunadi, 
when we have : $ f(f(n))=n+c$ (c:const and c must be nonnegative)
we have: change $ n$ by $ f(n)$ :  $ f(f(n)+f(m))=n+m+c$;
also, change $ m$ by $ f(m)$ : $ f(n+m+c)=f(n)+f(m) => f(f(n)+f(m))=f(f(n+m+c))=n+m+2c$
$ =>c=0
=> f(f(n))=n;
=> f( f(n)+f(m))=n+m$. in this, we change $ n$ by $ f(n)$ and m by $ f(m) => f(n+m)=f(n)+f(m)
=>f(n)=kn$ (k:const and k is positive)
$ =>f(f(n))=f(kn)=>n=k*k*n=>k=1
=>f(n)=n;$
dear Hophinhan, I think you right. 0 belongs to $ N$ (and the link that Gunardi give to us say that too :D) so I don't know why he say that 0 not belongs to $ N$. 0 not belongs to $ N*$. but of course, the problem we had solved more beutiful :D
\end{solution}



\begin{solution}[by \href{https://artofproblemsolving.com/community/user/29428}{pco}]
	Once again, once again, once again. For all members who did not walk thru these forum before posting their comments :

$ 0\in\mathbb N$ in some countries (mine, for example)
$ 0\notin\mathbb N$ in some other countries (more, I think).

The current situation in mathlinks forums is to consider $ 0\notin\mathbb N$.

The best thing would be that posters give the precision when they use $ \mathbb N$ in their problems.
\end{solution}



\begin{solution}[by \href{https://artofproblemsolving.com/community/user/71459}{x164}]
	Or just stop using $ \mathbb{N}$ and use $ \mathbb{Z}_{ > 0}$ and $ \mathbb{Z}_{\geq 0}$ :D
\end{solution}
*******************************************************************************
-------------------------------------------------------------------------------

\begin{problem}[Posted by \href{https://artofproblemsolving.com/community/user/57591}{KMK00009}]
	find all functions $f,g,h: \mathbb R \to \mathbb R$ such that
\[f(x+y)+g(x-y)=2h(x)+2h(y), \quad \forall x,y \in \mathbb R.\]
	\flushright \href{https://artofproblemsolving.com/community/c6h324002}{(Link to AoPS)}
\end{problem}



\begin{solution}[by \href{https://artofproblemsolving.com/community/user/29428}{pco}]
	\begin{tcolorbox}find all real functions f,g,h such that
f(x+y)+g(x-y)=2h(x)+2h(y)\end{tcolorbox}

Let $ P(x,y)$ be the assertion $ f(x+y)+g(x-y)=2h(x)+2h(y)$

$ P(x,0)$ $ \implies$ $ f(x)+g(x)=2h(x)+2h(0)$ and so $ g(x)=2h(x)-f(x)+2h(0)$

$ P(x,x)$ $ \implies$ $ f(2x)+g(0)=4h(x)$ and so $ f(x)=4h(\frac x2)-g(0)=4h(\frac x2)-4h(0)+f(0)$

So we have a mandatory expression for $ f(x)$ and $ g(x)$ depending from $ h(x)$ :

$ f(x)=4h(\frac x2)-4h(0)+a$
$ g(x)=2h(x)-4h(\frac x2)+6h(0)-a$

Plugging this in the original equation, we get a functional equation on $ h(x)$ :

$ 4h(\frac {x+y}2)-4h(0)+a+2h(x-y)-4h(\frac {x-y}2)+6h(0)-a=2h(x)+2h(y)$

$ \iff$ $ 2(h(\frac {x+y}2)-h(0))+$ $ h(x-y)-h(0)-2(h(\frac {x-y}2)-h(0))$ $ =(h(x)-h(0)+(h(y)-h(0)$

And so, using $ u(x)=h(x)-h(0)$ : 
New assertion $ Q(x,y)$ : $ 2u(\frac {x+y}2)+u(x-y)-2u(\frac {x-y}2)=u(x)+u(y)$ with $ u(0)=0$

adding $ Q(x,y)$ and $ Q(x,-y)$, we get new assertion $ R(x,y)$ : $ u(x-y)+u(x+y)=2u(x)+u(y)+u(-y)$

subtracting $ R(y,x)$ from $ R(x,y)$, we get $ u(x-y)-u(y-x)=u(x)-u(-x)-u(y)+u(-y)$

And so $ v(x-y)=v(x)-v(y)$ where $ v(x)=u(x)-u(-x)$ and so $ v(x)$ is a solution of Cauchy's equation.

So $ u(x)=p(x)+c(x)$ where $ p(x)$ is an even function and $ c(x)=\frac{v(x)}2$ is a solution of Cauchy's equation and $ p(0)=c(0)=0$

Plugging this is $ R(x,y)$, we get $ p(x-y)+p(x+y)=2p(x)+2p(y)$ which implies $ p(2x)=2p(x)$ and so $ u(2x)=2u(x)$

Using this property, $ Q(x,y)$ may be written : $ u(x+y)=u(x)+u(y)$ and $ u(x)$ is also a solution of Cauchy's equation.

And it's immediate to check back that $ Q(x,y)$ and $ u(0)=0$ $ \iff$ $ u(x)$ is a solution of Cauchy's equation.

And this gives the general solution of the requested equation :

Let $ u(x)$ any solution of Cauchy's equation :

$ f(x)=2u(x)+a$
$ g(x)=4b-a$
$ h(x)=u(x)+b$
\end{solution}



\begin{solution}[by \href{https://artofproblemsolving.com/community/user/29428}{pco}]
	\begin{tcolorbox}[quote="KMK00009"]find all real functions f,g,h such that
f(x+y)+g(x-y)=2h(x)+2h(y)\end{tcolorbox}

... Plugging this is $ R(x,y)$, we get $ p(x - y) + p(x + y) = 2p(x) + 2p(y)$ which implies $ p(2x) = 2p(x)$ ...\end{tcolorbox}

Sorry, but I made a mistake here : This implies $ p(2x) = 4p(x)$ and not $ p(2x) = 2p(x)$ :(

So we have to solve $ p(x - y) + p(x + y) = 2p(x) + 2p(y)$ with $ p(0) = 0$

It's very easy to show that $ p(ax) = a^2p(x)$ $ \forall a\in\mathbb Q$, $ \forall x$

Which gives infinitely many solutions :
Using an hamel basis $ {b_i}$ and $ u_i\in\mathbb R$, you get solution $ f(\sum x_ib_i) = \sum u_ix_i^2b_i$

And so at least the family of solutions :

Let $ c(x)$ any solution of Cauchy's equation
Let $ p(x)$ any function in the form above


$ f(x) = p(x) + 2c(x) + a$
$ g(x) = p(x) + 4b - a$
$ h(x) = p(x) + c(x) + b$

\begin{bolded}And maybe some others \end{bolded}\end{underlined}... (I dont know) .

\begin{bolded}If continuity constraint is added\end{bolded}\end{underlined}, we have then $ p(x) = cx^2$ and $ c(x) = dx$ and the solutions :
$ f(x) = cx^2 + 2dx + a$
$ g(x) = cx^2 + 4b - a$
$ h(x) = cx^2 + dx + b$
\end{solution}



\begin{solution}[by \href{https://artofproblemsolving.com/community/user/177353}{Parnpaniti}]
	\begin{tcolorbox}\begin{tcolorbox}[quote="KMK00009"]find all real functions f,g,h such that
f(x+y)+g(x-y)=2h(x)+2h(y)\end{tcolorbox}

... Plugging this is $ R(x,y)$, we get $ p(x - y) + p(x + y) = 2p(x) + 2p(y)$ which implies $ p(2x) = 2p(x)$ ...\end{tcolorbox}

Sorry, but I made a mistake here : This implies $ p(2x) = 4p(x)$ and not $ p(2x) = 2p(x)$ :(

So we have to solve $ p(x - y) + p(x + y) = 2p(x) + 2p(y)$ with $ p(0) = 0$

It's very easy to show that $ p(ax) = a^2p(x)$ $ \forall a\in\mathbb Q$, $ \forall x$

Which gives infinitely many solutions :
Using an hamel basis $ {b_i}$ and $ u_i\in\mathbb R$, you get solution $ f(\sum x_ib_i) = \sum u_ix_i^2b_i$

And so at least the family of solutions :

Let $ c(x)$ any solution of Cauchy's equation
Let $ p(x)$ any function in the form above


$ f(x) = p(x) + 2c(x) + a$
$ g(x) = p(x) + 4b - a$
$ h(x) = p(x) + c(x) + b$

\begin{bolded}And maybe some others \end{bolded}\end{underlined}... (I dont know) .

\begin{bolded}If continuity constraint is added\end{bolded}\end{underlined}, we have then $ p(x) = cx^2$ and $ c(x) = dx$ and the solutions :
$ f(x) = cx^2 + 2dx + a$
$ g(x) = cx^2 + 4b - a$
$ h(x) = cx^2 + dx + b$\end{tcolorbox}
c(x) be solution of Cauchy's equation but c is function from R to R so c is accord by 3 condition,monotone or continuous or bound function
\end{solution}
*******************************************************************************
-------------------------------------------------------------------------------

\begin{problem}[Posted by \href{https://artofproblemsolving.com/community/user/43536}{nguyenvuthanhha}]
	The function $ f :  \mathbb{R} \to \mathbb{R}$ satisfies the equation
\[ f(x+f(y))+f(f(x)+y)=2x+2y \quad \forall x,y  \in \mathbb{R}.\]
Prove that $ f(2x) = 2f(x)$ for all real $x$. Open problem:  find all functions $f$ which satisfy the above functional equation.
	\flushright \href{https://artofproblemsolving.com/community/c6h324143}{(Link to AoPS)}
\end{problem}



\begin{solution}[by \href{https://artofproblemsolving.com/community/user/68920}{prester}]
	\begin{tcolorbox}\begin{italicized}The function $ f \ : \ \mathbb{R} \to \mathbb{R}$ satisfies the equation :

   $ f(x + f(y)) + f(f(x) + y) = 2x + 2y \ \forall \ \ x;y \ \in \ \mathbb{R} \ \ (*)$

   Prove that : $ f(2x) \ = \ 2f(x)$

  Open problem :  Find all function satisfy  $ (*)$\end{italicized}\end{tcolorbox}

$ f(x)=x$, $ \forall x\in \mathbb{R}$ is a solution of $ (*)$. 

So the $ (*)$ may be written as $ 2f(x+y)=2x+2y$

$ x=y$ $ \implies$ $ 2f(2x)=4x$ $ \implies$ $ f(2x)=2x=2f(x)$, $ \forall x\in \mathbb{R}$

Q.E.D.
\end{solution}



\begin{solution}[by \href{https://artofproblemsolving.com/community/user/29428}{pco}]
	@prester : you proved the result only for the solution $ f(x)=x$ (in which case it is obvious) and not for all the solutions.
\begin{tcolorbox}\begin{italicized}The function $ f \ : \ \mathbb{R} \to \mathbb{R}$ satisfies the equation :

   $ f(x + f(y)) + f(f(x) + y) = 2x + 2y \ \forall \ \ x;y \ \in \ \mathbb{R} \ \ (*)$

   Prove that : $ f(2x) \ = \ 2f(x)$
\end{tcolorbox}

Let $ P(x,y)$ be the assertion $ f(x + f(y)) + f(f(x) + y) = 2x + 2y$

$ P(0,0)$ $ \implies$ $ f(f(0))=0$.

$ P(f(0),f(0))$ $ \implies$ $ 0=f(0)$

$ P(x,x)$ $ \implies$ $ f(x+f(x))=2x$ and so $ f(f(x+f(x))=f(2x)$

$ P(x+f(x),0)$ $ \implies$ $ f(x+f(x))+f(f(x+f(x)))=2x+2f(x)$ and so $ f(f(x+f(x))=2f(x)$

Comparing these two lines, we get $ f(2x)=2f(x)$
Q.E.D.
\end{solution}



\begin{solution}[by \href{https://artofproblemsolving.com/community/user/68920}{prester}]
	\begin{tcolorbox}@prester : you proved the result only for the solution $ f(x) = x$ (in which case it is obvious) and not for all the solutions.
\end{tcolorbox}

Yes, but it should be sufficient to prove that for just one solution...Otherwise it should not valid for any solution. I mean that you proved that indipendenlty on the solution of (*) and this is obviously perfect, but I suppose that if you have a solution of (*) and you prove it, it should be ok too... 
...Maybe I am wrong...but sincerely I cannot see why... :oops:
\end{solution}



\begin{solution}[by \href{https://artofproblemsolving.com/community/user/29428}{pco}]
	\begin{tcolorbox}[quote="pco"]@prester : you proved the result only for the solution $ f(x) = x$ (in which case it is obvious) and not for all the solutions.
\end{tcolorbox}

Yes, but it should be sufficient to prove that for just one solution...Otherwise it should not valid for any solution. I mean that you proved that indipendenlty on the solution of (*) and this is obviously perfect, but I suppose that if you have a solution of (*) and you prove it, it should be ok too... 
...Maybe I am wrong...but sincerely I cannot see why... :oops:\end{tcolorbox}

I dont understand.

The question is to prove for all solutions, not for just one.

For example, the property $ f(1)=1$ is true for the solution $ f(x)=x$ but is not true for all solutions (it's false for solution $ f(x)=-2x$, for example).
\end{solution}



\begin{solution}[by \href{https://artofproblemsolving.com/community/user/68920}{prester}]
	Dear pco,  you're rigth as usual. I've made another stupid mistake...Sorry
\end{solution}



\begin{solution}[by \href{https://artofproblemsolving.com/community/user/67223}{Amir Hossein}]
	Thank you for your nice solution pco.
\end{solution}
*******************************************************************************
-------------------------------------------------------------------------------

\begin{problem}[Posted by \href{https://artofproblemsolving.com/community/user/67223}{Amir Hossein}]
	Find all functions $f: \mathbb R \to \mathbb R$ such that
\[f(f(x)+f(y)+xy)=x+y+f(xy)\]
for all reals $x$ and $y$.
	\flushright \href{https://artofproblemsolving.com/community/c6h324153}{(Link to AoPS)}
\end{problem}



\begin{solution}[by \href{https://artofproblemsolving.com/community/user/67223}{Amir Hossein}]
	when I solve this function, I get to this: $ f(f(x))=x$. and I dont know how to solve it.
or the problem changes to this:
find all functions $ f: R \rightarrow R$ which $ f(f(x))=x$.




amparvardi
\end{solution}



\begin{solution}[by \href{https://artofproblemsolving.com/community/user/29428}{pco}]
	\begin{tcolorbox}when I solve this function, I get to this: $ f(f(x)) = x$. and I dont know how to solve it.
or the problem changes to this:
find all functions $ f: R \rightarrow R$ which $ f(f(x)) = x$.
\end{tcolorbox}

1) about involutive functions :
Your equation implies $ f(f(x))=x$ but is not equivalent to.
$ f(f(x))=x$ has as solution any involutive function (it's their definition) and you have infinitely many involutive functions. One general form for them could be :

Let $ A,B,C$ any split of $ \mathbb R$ such that $ A$ and $ B$ are equinumerous and let $ h(x)$ any bijecton from $ A\to B$ :

$ \forall x\in A$ : $ f(x)=h(x)$
$ \forall x\in B$ : $ f(x)=h^{-1}(x)$
$ \forall x\in C$ : $ f(x)=x$

2) about your problem :
We have infinitely many solutions.
For example at least all involutive solutions of Cauchy's equation.
But I'm unable up to now  to prove if these are all the solutons or if some other exist.
\end{solution}



\begin{solution}[by \href{https://artofproblemsolving.com/community/user/47539}{snain}]
	I am sorry for my bad english

y = 0 => f(f(x)+f(0)) = x+f(0) => f is injective
y = -x => f(f(x)+f(-x)-x^2) = f(-x^2) => f(x)+f(-x)-x^2 = -x^2 => f(-x) = -f(x)

take g(x) = f(x) - x
if g is null function f(x) = x wich is solution.
otherwise f(x) injective so g(x) is too.
g(-x) = f(-x) + x = -f(x)+x = -g(x).

Replacing f(x) by g(x) in the equation we take :

g(g(x)+x+g(y)+y+xy) + g(x)+x+g(y)+y+xy = x+y +g(xy)+xy

g(g(x)+g(y)+x+y+xy) = g(xy)-g(x)-g(y) (1)

take y = 1

g(g(x)+g(1)+x+1+x)=-g(1)=g(-1)

g is injective => g(x)+2x+1+g(1) = -1 => f(x) = -x-2-g(1) (2)

take x = 1 and y = 1

g(g(1)+g(1)+3)=g(1)-g(1)-g(1)=g(-1)

2g(1)+3 = -1 => g(1) = -2 => f(x) = -x

the only solution are f(x) = x and f(x) = -x
\end{solution}



\begin{solution}[by \href{https://artofproblemsolving.com/community/user/29428}{pco}]
	\begin{tcolorbox}y = 0 => f(f(x)+f(0)) = x+f(0) => f is injective
y = -x => f(f(x)+f(-x)-x^2) = f(-x^2) => f(x)+f(-x)-x^2 = -x^2 => f(-x) = -f(x)

take g(x) = f(x) - x
if g is null function f(x) = x wich is solution.
otherwise f(x) injective so g(x) is too.\end{tcolorbox}
Surely not. $ f(x)$ injective does not imply $ f(x)-x$ injective !


\begin{tcolorbox}the only solution are f(x) = x and f(x) = -x\end{tcolorbox}
Surely not. As I said in my previous post, any involutive solution of Cauchy's equation is solution. And so infinitely many solutions (with axiom of choice).
\end{solution}



\begin{solution}[by \href{https://artofproblemsolving.com/community/user/67223}{Amir Hossein}]
	\begin{tcolorbox}...
the only solution are f(x) = x and f(x) = -x\end{tcolorbox}


It has infinitely many solutions.
For Example, let $ x=y=0$,so we have: $ f(2f(0))=f(0) \Longrightarrow f(0)=0$.
Then let $ y=0$, It becomes to $ f(f(x))=x$. And a soloution for this function is this: $ f(x)=k-x$.(check it)
\end{solution}



\begin{solution}[by \href{https://artofproblemsolving.com/community/user/29428}{pco}]
	\begin{tcolorbox}
It has infinitely many solutions.
For Example, let $ x = y = 0$,so we have: $ f(2f(0)) = f(0) \Longrightarrow f(0) = 0$.
Then let $ y = 0$, It becomes to $ f(f(x)) = x$. And a soloution for this function is this: $ f(x) = k - x$.(check it)\end{tcolorbox}
Once again, same error. Your equation implies $ f(x)$ is involutive bus \begin{bolded}is not equivalent to \end{bolded}\end{underlined}.

All involutive functions are not solutions of your equation.
For example $ f(x)=k-x$ is involutive but is solution only if $ k=0$, so is not a good answer to snain.
\end{solution}



\begin{solution}[by \href{https://artofproblemsolving.com/community/user/67223}{Amir Hossein}]
	\begin{tcolorbox}
Once again, same error. Your equation implies $ f(x)$ is involutive bus \begin{bolded}is not equivalent to \end{bolded}\end{underlined}.

All involutive functions are not solutions of your equation.
For example $ f(x) = k - x$ is involutive but is solution only if $ k = 0$, so is not a good answer to snain.\end{tcolorbox}


yes, you are right ,but [size=150]snain[\/size] wrote that there are only two functions  $ f(x) = x$ and $ f(x) = -x$ satisfying the problem.
it was an example I wrote,not all soloutions of the question.
((my question:why the soloutions of the $ f(f(x))=x$ aren't exactly soloutions of the original function?))



amparvardi
\end{solution}



\begin{solution}[by \href{https://artofproblemsolving.com/community/user/29428}{pco}]
	\begin{tcolorbox}
yes, you are right ,but [size=150]snain[\/size] wrote that there are only two functions  $ f(x) = x$ and $ f(x) = - x$ satisfying the problem.
it was an example I wrote,not all soloutions of the question.
((my question:why the soloutions of the $ f(f(x)) = x$ aren't exactly soloutions of the original function?))



amparvardi\end{tcolorbox}
But you gave him an example to check : $ f(x) = k - x$ and you did not yourself check it : this does not fit if $ k\ne 0$ and so your example is wrong!

And about your question : 
You got $ f(f(x)) = x$ by using $ \implies$, not $ \iff$ (I said this to you already twice)
For example : $ f(x) = 1 - x$ is such that $ f(f(x)) = x$ but is not a solution of your equation.
\end{solution}



\begin{solution}[by \href{https://artofproblemsolving.com/community/user/67223}{Amir Hossein}]
	Got it!
you're right,and I get the mistake.
thank you for your guidence!
\end{solution}
*******************************************************************************
-------------------------------------------------------------------------------

\begin{problem}[Posted by \href{https://artofproblemsolving.com/community/user/67223}{Amir Hossein}]
	Find all functions $ f: \mathbb Z \rightarrow \mathbb Z$ satisfying:
(i) $ f(0)=1$,
(ii) $ f(f(n))=n$, and
(iii) $ f(f(n+2)+2)=n$ for all integers $\mathbb Z$.
	\flushright \href{https://artofproblemsolving.com/community/c6h324155}{(Link to AoPS)}
\end{problem}



\begin{solution}[by \href{https://artofproblemsolving.com/community/user/29428}{pco}]
	\begin{tcolorbox}find all functions $ f: Z \rightarrow Z$ satisfying:
$ ( \imath)$ $ f(0) = 1$
$ ( \imath \imath)$ $ f(f(n)) = n$
$ (\imath \imath \imath)$ $ f(f(n + 2) + 2) = n$   .  $ \forall n \in Z$\end{tcolorbox}

$ f(f(n))=n$ shows that $ f(n)$ is injective.

Then $ f(f(n+2)+2)=n=f(f(n))$ $ \implies$ $ f(n+2)=f(n)-2$

Then $ f(0)=1$ and $ f(n+2)=f(n)-2$ $ \implies$ $ f(2n)=1-2n$

Then $ f(0)=1$ $ \implies$ $ f(1)=f(f(0))=0$ and $ f(n+2)=f(n)-2$ $ \implies$ $ f(2n+1)=-2n=1-(2n+1)$

And it is easy to check back that this indeed is a solution.

Hence the solution: $ \boxed{f(x)=1-x}$
\end{solution}



\begin{solution}[by \href{https://artofproblemsolving.com/community/user/67223}{Amir Hossein}]
	\begin{tcolorbox}

Then $ f(0) = 1$ and $ f(n + 2) = f(n) - 2$ $ \implies$ $ f(2n) = 1 - 2n$

\end{tcolorbox}

How do you can say since ''$ f(0) = 1$ and $ f(n + 2) = f(n) - 2$ $ \implies$ $ f(2n) = 1 - 2n$''?
\end{solution}



\begin{solution}[by \href{https://artofproblemsolving.com/community/user/29428}{pco}]
	\begin{tcolorbox}[quote="pco"]

Then $ f(0) = 1$ and $ f(n + 2) = f(n) - 2$ $ \implies$ $ f(2n) = 1 - 2n$

\end{tcolorbox}

How do you can say since ''$ f(0) = 1$ and $ f(n + 2) = f(n) - 2$ $ \implies$ $ f(2n) = 1 - 2n$''?\end{tcolorbox}

In order to see it :
$ f(0)=1$ and then $ f(2)=f(0+2)=f(0)-2=-1$ and then $ f(4)=f(2+2)=f(2)-2=-3...$
$ f(0)=1$ and then $ f(-2)=f(0-2)=f(0)+2=3$ and then $ f(-4)=f(-2-2)=f(-2)+2=5...$

In order to prove it : just use induction
\end{solution}
*******************************************************************************
-------------------------------------------------------------------------------

\begin{problem}[Posted by \href{https://artofproblemsolving.com/community/user/75045}{KapioPulsar}]
	Find all functions $f: \mathbb R \to \mathbb R$ such that $f(0)=0$ and
\[f(x \cos B)-f(x(\cos B)^2)=x-x^2 \cos B.\]
For each real number $x$ and each $B  \in \mathbb R \setminus \left\{B : B=k\pi \text{ or } B=k \pi+ \frac{\pi}{2}, k \in \mathbb Z\right\}.$
	\flushright \href{https://artofproblemsolving.com/community/c6h324225}{(Link to AoPS)}
\end{problem}



\begin{solution}[by \href{https://artofproblemsolving.com/community/user/75045}{KapioPulsar}]
	its not so hard....!!!!
\end{solution}



\begin{solution}[by \href{https://artofproblemsolving.com/community/user/44887}{Mathias_DK}]
	\begin{tcolorbox}find all fruction R->R ,$ f(0) = 0$, $ f(xcosB) - f(x(cosB)^2) = x - x^2cosB$,  x->R and for every B belongs R\begin{italicized}\begin{bolded}-{B:B=kπ or B=kπ+π\/2, k belongs to Z}\end{bolded}\end{italicized}\end{underlined}\end{tcolorbox}
If $ B = 0$, we have: $ f(0)-f(0)=x - x^2 \cdot 0 \iff x =0$, which cannot hold for all $ x$, so there is nu such function... Maybe you missed something?
\end{solution}



\begin{solution}[by \href{https://artofproblemsolving.com/community/user/68920}{prester}]
	\begin{tcolorbox}[quote="KapioPulsar"]find all fruction R->R ,$ f(0) = 0$, $ f(xcosB) - f(x(cosB)^2) = x - x^2cosB$,  x->R and for every B belongs R\begin{italicized}\begin{bolded}-{B:B=kπ or B=kπ+π\/2, k belongs to Z}\end{bolded}\end{italicized}\end{underlined}\end{tcolorbox}
If $ B = 0$, we have: $ f(0) - f(0) = x - x^2 \cdot 0 \iff x = 0$, which cannot hold for all $ x$, so there is nu such function... Maybe you missed something?\end{tcolorbox}

You should say $ \cos B=0$ but if I understood well, $ \cos B = 0$ and  $ \cos B=\pm 1$ are not included in the problem's domain (see the statement).
\end{solution}



\begin{solution}[by \href{https://artofproblemsolving.com/community/user/44887}{Mathias_DK}]
	\begin{tcolorbox}[quote="Mathias_DK"][quote="KapioPulsar"]find all fruction R->R ,$ f(0) = 0$, $ f(xcosB) - f(x(cosB)^2) = x - x^2cosB$,  x->R and for every B belongs R\begin{italicized}\begin{bolded}-{B:B=kπ or B=kπ+π\/2, k belongs to Z}\end{bolded}\end{italicized}\end{underlined}\end{tcolorbox}
If $ B = 0$, we have: $ f(0) - f(0) = x - x^2 \cdot 0 \iff x = 0$, which cannot hold for all $ x$, so there is nu such function... Maybe you missed something?\end{tcolorbox}

You should say $ \cos B = 0$ but if I understood well, $ \cos B = 0$ and  $ \cos B = \pm 1$ are not included in the problem's domain (see the statement).\end{tcolorbox}
$ B=0 \Rightarrow \cos B = 0$ ;) I didn't really understand what he have written, but the result is the same when $ \cos B \notin \{-1,0,1\}$..
Then we have:
$ x = \frac{1}{\cos B}$ gives
$ f(\cos B) = f(1)$
Plugging in $ x=1$ gives $ f(\cos B) - f( (\cos B)^2) = 1-\cos B$. But $ f(\cos B) = f(1) = f( (\cos B)^2 )$, so $ \cos B = 1$, which cannot hold for all $ B$. So there is no solution.
I still think there may be some mistake in the problem statement..
\end{solution}



\begin{solution}[by \href{https://artofproblemsolving.com/community/user/68920}{prester}]
	\begin{tcolorbox}
$ B = 0 \Rightarrow \cos B = 0$....
\end{tcolorbox}
Sorry but I believed that $ B = 0 \Rightarrow \cos B = 1$...

\begin{tcolorbox}
$ ...f(1) = f( (\cos B)^2 )...$
\end{tcolorbox}
Sorry but I don't see that. 

Instead I derived the property $ f(x) - f(-x)=4x^2$ putting $ \cos B = \pm \frac12$ in the original equation and summating the results...and this result is the best I have done until now...
\end{solution}



\begin{solution}[by \href{https://artofproblemsolving.com/community/user/75045}{KapioPulsar}]
	\begin{tcolorbox}[quote="KapioPulsar"]find all fruction R->R ,$ f(0) = 0$, $ f(xcosB) - f(x(cosB)^2) = x - x^2cosB$,  x->R and for every B belongs R\begin{italicized}\begin{bolded}-{B:B=kπ or B=kπ+π\/2, k belongs to Z}\end{bolded}\end{italicized}\end{underlined}\end{tcolorbox}
If $ B = 0$, we have: $ f(0) - f(0) = x - x^2 \cdot 0 \iff x = 0$, which cannot hold for all $ x$, so there is nu such function... Maybe you missed something?\end{tcolorbox} \begin{bolded}$ B \in \mathbb{R} - \{ B: B=k \pi ,,or,, B=k \pi +\frac{ \pi}{2} \ , k \in \mathbb{Z}\}$\end{underlined}\end{bolded}
\end{solution}



\begin{solution}[by \href{https://artofproblemsolving.com/community/user/75045}{KapioPulsar}]
	\begin{italicized}\begin{bolded}Now? \end{bolded}\end{italicized}\end{underlined}
\end{solution}



\begin{solution}[by \href{https://artofproblemsolving.com/community/user/44887}{Mathias_DK}]
	\begin{tcolorbox}
\begin{tcolorbox}
$ ...f(1) = f( (\cos B)^2 )...$
\end{tcolorbox}
Sorry but I don't see that.\end{tcolorbox}
$ \cos B$ can be any number in the interval $ (-1;1) \backslash \{0\}$, so $ f(x) = f(1) \forall x \in (-1;1) \backslash \{0\}$ and therefore $ f ( \cos^2 B ) = f(1)$
\end{solution}



\begin{solution}[by \href{https://artofproblemsolving.com/community/user/75045}{KapioPulsar}]
	\begin{tcolorbox}find all fruction R->R ,$ f(0) = 0$, $ f(xcosB) - f(x(cosB)^2) = x - x^2cosB$,  x->R and for every B belongs R\begin{italicized}\begin{bolded}-{B:B=kπ or B=kπ+π\/2, k belongs to Z}\end{bolded}\end{italicized}\end{underlined}\end{tcolorbox} Sorry i forgot to tell that f is continious  :blush:  :oops:
\end{solution}



\begin{solution}[by \href{https://artofproblemsolving.com/community/user/44887}{Mathias_DK}]
	\begin{tcolorbox}\begin{tcolorbox}find all fruction R->R ,$ f(0) = 0$, $ f(xcosB) - f(x(cosB)^2) = x - x^2cosB$,  x->R and for every B belongs R\begin{italicized}\begin{bolded}-{B:B=kπ or B=kπ+π\/2, k belongs to Z}\end{bolded}\end{italicized}\end{underlined}\end{tcolorbox} Sorry i forgot to tell that f is continious  :blush:  :oops:\end{tcolorbox}
I have already proved that no such $ f$ exists, even without the additional constraint, haven't I?
\end{solution}



\begin{solution}[by \href{https://artofproblemsolving.com/community/user/75045}{KapioPulsar}]
	\begin{tcolorbox}[quote="KapioPulsar"][quote="KapioPulsar"]find all fruction R->R ,$ f(0) = 0$, $ f(xcosB) - f(x(cosB)^2) = x - x^2cosB$,  x->R and for every B belongs R\begin{italicized}\begin{bolded}-{B:B=kπ or B=kπ+π\/2, k belongs to Z}\end{bolded}\end{italicized}\end{underlined}\end{tcolorbox} Sorry i forgot to tell that f is continious  :blush:  :oops:\end{tcolorbox}
I have already proved that no such $ f$ exists, even without the additional constraint, haven't I?\end{tcolorbox} f exist i solve it but litle complicated
\end{solution}



\begin{solution}[by \href{https://artofproblemsolving.com/community/user/68920}{prester}]
	\begin{tcolorbox}[quote="prester"]
\begin{tcolorbox}
$ ...f(1) = f( (\cos B)^2 )...$
\end{tcolorbox}
Sorry but I don't see that.\end{tcolorbox}
$ \cos B$ can be any number in the interval $ ( - 1;1) \backslash \{0\}$, so $ f(x) = f(1) \forall x \in ( - 1;1) \backslash \{0\}$ and therefore $ f ( \cos^2 B ) = f(1)$\end{tcolorbox}

Yes I see it now. Thanks...You're right, no solutions
\end{solution}



\begin{solution}[by \href{https://artofproblemsolving.com/community/user/75045}{KapioPulsar}]
	\begin{tcolorbox}\end{tcolorbox}[quote="Mathias_DK"]\begin{tcolorbox}
\begin{tcolorbox}
$ ...f(1) = f( (\cos B)^2 )...$
\end{tcolorbox}
Sorry but I don't see that.\end{tcolorbox}
$ \cos B$ can be any number in the interval $ ( - 1;1) \backslash \{0\}$, so $ f(x) = f(1) \forall x \in ( - 1;1) \backslash \{0\}$ and therefore $ f ( \cos^2 B ) = f(1)$\end{tcolorbox}\begin{tcolorbox}

Yes I see it now. Thanks...You're right, no solutions\end{tcolorbox} Solutions exist !! I solve it !try think it again!
\end{solution}



\begin{solution}[by \href{https://artofproblemsolving.com/community/user/44887}{Mathias_DK}]
	\begin{tcolorbox} Solutions exist !! I solve it !try think it again!\end{tcolorbox}
The way Prester and I interpreted the problem makes it obvious that there are no solution. Can you try to write it down again? Is $ B$ perhaps a given constant, such that the equation only holds for one value of $ B$?
\end{solution}



\begin{solution}[by \href{https://artofproblemsolving.com/community/user/75045}{KapioPulsar}]
	\begin{tcolorbox}[quote="KapioPulsar"] Solutions exist !! I solve it !try think it again!\end{tcolorbox}
The way Prester and I interpreted the problem makes it obvious that there are no solution. Can you try to write it down again? Is $ B$ perhaps a given constant, such that the equation only holds for one value of $ B$?\end{tcolorbox} i say $ B \in \mathbb{R} - \{ B: B = k \pi ,,or,, B = k \pi + \frac {\pi}{2} \ , k \in \mathbb{Z}\}$ this problem is from math olumpiad in greece 1997-1998 and its as exactly as i write it
\end{solution}



\begin{solution}[by \href{https://artofproblemsolving.com/community/user/68920}{prester}]
	\begin{tcolorbox}[quote="Mathias_DK"]\begin{tcolorbox} Solutions exist !! I solve it !try think it again!\end{tcolorbox}
The way Prester and I interpreted the problem makes it obvious that there are no solution. Can you try to write it down again? Is $ B$ perhaps a given constant, such that the equation only holds for one value of $ B$?\end{tcolorbox} i say $ B \in \mathbb{R} - \{ B: B = k \pi ,,or,, B = k \pi + \frac {\pi}{2} \ , k \in \mathbb{Z}\}$ this problem is from math olumpiad in greece 1997-1998 and its as exactly as i write it\end{tcolorbox}

@mathias_dk: I think that we must apologize to KapioPulsar, because $ f (\cos B) \ne f( \cos^2B)$...I did not understand the statement well since $ B$ and so $ \cos B$ is a parameter and not a variable... :oops:  :?:
\end{solution}



\begin{solution}[by \href{https://artofproblemsolving.com/community/user/29428}{pco}]
	\begin{tcolorbox}find all fruction R->R ,$ f(0) = 0$, $ f(xcosB) - f(x(cosB)^2) = x - x^2cosB$,  x->R and for every B belongs R\begin{italicized}\begin{bolded}-{B:B=kπ or B=kπ+π\/2, k belongs to Z}\end{bolded}\end{italicized}\end{underlined}\end{tcolorbox}

I dont like at all this $ \cos(B)$ thing and I suggest the following transformation (equivalent, IMHO) :

Find all functions from $ \mathbb R\to\mathbb R$ such that $ f(0)=0$ and $ f(xy)-f(xy^2)=x-x^2y$ $ \forall x\in\mathbb R$, $ \forall y\in(-1,0)\cup(0,1)$

Let $ P(x,y)$ be the assertion $ f(xy)-f(xy^2)=x-x^2y$

Let $ x\in(0,\frac 14)$ :

$ x\in(0,\frac 14)$ $ \implies$ $ 2x\in(0,1)$ and so $ P(\frac 1{4x},2x)$ $ \implies$ $ f(x)=f(\frac 12)-\frac 1{8x}$

$ x\in(0,\frac 14)$ $ \implies$ $ x\in(0,1)$ and so $ P(\frac 1x,x)$ $ \implies$ $ f(x)=f(1)$

And so $ f(\frac 12)-\frac 1{8x}=f(1)$ $ \forall x\in(0,\frac 14)$ : contradiction.

So no solution.
Sorry  :roll: .
\end{solution}



\begin{solution}[by \href{https://artofproblemsolving.com/community/user/75530}{kyros}]
	\begin{tcolorbox}[quote="KapioPulsar"]find all fruction R->R ,$ f(0) = 0$, $ f(xcosB) - f(x(cosB)^2) = x - x^2cosB$,  x->R and for every B belongs R\begin{italicized}\begin{bolded}-{B:B=kπ or B=kπ+π\/2, k belongs to Z}\end{bolded}\end{italicized}\end{underlined}\end{tcolorbox}\end{tcolorbox}
If $ B = 0$, we have: $ f(0) - f(0) = x - x^2 \cdot 0 \iff x = 0$, which cannot hold for all $ x$, so there is nu such function... Maybe you missed something?[\/q
b is never 0 because b belongs to R-{b=kπ or b=kπ+π\/2, κεΖ} so for k =0 we have b=0 that's not acceptable...we cannot pace b=0
\end{solution}



\begin{solution}[by \href{https://artofproblemsolving.com/community/user/44887}{Mathias_DK}]
	\begin{tcolorbox}
b is never 0 because b belongs to R-{b=kπ or b=kπ+π\/2, κεΖ} so for k =0 we have b=0 that's not acceptable...we cannot pace b=0\end{tcolorbox}
Don't be such a fool! Read the rest of the posts ;)
\end{solution}
*******************************************************************************
-------------------------------------------------------------------------------

\begin{problem}[Posted by \href{https://artofproblemsolving.com/community/user/46787}{moldovan}]
	Determine all the functions $ f: \mathbb{N} \rightarrow \mathbb{N}\setminus \{1\}$ so that:
\[ f(n+1)+f(n+3)=f(n+5) \cdot f(n+7) - 2001, \quad \forall n \ge 0.\]
	\flushright \href{https://artofproblemsolving.com/community/c6h324618}{(Link to AoPS)}
\end{problem}



\begin{solution}[by \href{https://artofproblemsolving.com/community/user/29428}{pco}]
	\begin{tcolorbox}Determine all the functions $ f: \mathbb{N}^* \rightarrow \mathbb{N}^* - \{1\}$ so that:
\[ f(n + 1) + f(n + 3) = f(n + 5) \cdot f(n + 7) - 2001, \forall n \ge 0\]
\end{tcolorbox}

This kind of problem has been posted thousands of times ....
You should read the forum sometimes and learn from it. :(

$ f(n + 1) + f(n + 3) = f(n + 5) f(n + 7) - 2001$
$ f(n + 3) + f(n + 5) = f(n + 7) f(n + 9) - 2001$

Subtracting :
$ f(n+5)-f(n+1)=f(n+7)(f(n+9)-f(n+5))$ $ \forall n\ge 0$

And so $ f(n+5)-f(n+1)$ $ =f(n+7)f(n+11)f(n+15)$ $ ....f(n+7+4p)(f(n+4p+9)-$ $ f(n+4p+5))$

And, since $ f(n)\ne 1$ $ \forall n$ : $ f(n+5)=f(n+1)$ $ \forall n$ and the sequence $ f(n)$ is $ a,b,c,d,a,b,c,d,...$ with :
$ a+c=ac-2001$ $ \iff$ $ (a-1)(c-1)=2002=2\cdot 7\cdot 11\cdot 13$ and so $ 16$ possibilities for $ (a,c)$ :
$ (2,2003)$
$ (3,1002),(8,287),(12,183),(14,155)$
$ (15,144),(23,92),(27,78),(78,27),(92,23),144,15)$
$ (155,14),(183,12),(287,8),(1002,3)$
$ (2003,2)$

$ b+d=bd-2001$ $ \iff$ $ (b-1)(d-1)=2002=2\cdot 7\cdot 11\cdot 13$ and so the same $ 16$ possibilities for $ (b,d)$ 

And so $ 256$ solutions (what an interesting problem). For example :
$ 23,3,92,1002,23,3,92,1002,23,3,92,1002,...$
$ 2,2003,2003,2,2,2003,2003,2,2,2003,2003,2,...$
...
\end{solution}
*******************************************************************************
-------------------------------------------------------------------------------

\begin{problem}[Posted by \href{https://artofproblemsolving.com/community/user/46787}{moldovan}]
	Determine the functions $ f: \mathbb{R} \rightarrow \mathbb{R}$ which satisfy: 

\[ f(x+y)=f( \ln (x \cdot y ))\]

for any $ x,y \in (0,\infty).$
	\flushright \href{https://artofproblemsolving.com/community/c6h324621}{(Link to AoPS)}
\end{problem}



\begin{solution}[by \href{https://artofproblemsolving.com/community/user/29428}{pco}]
	\begin{tcolorbox}Determine the functions $ f: \mathbb{R} \rightarrow \mathbb{R}$ which satisfy:
\[ f(x + y) = f( \ln (x \cdot y ))\]
for any $ x,y \in (0,\infty).$\end{tcolorbox}

Let $ u,v\in\mathbb R$ such that $ u\ge 2e^{\frac v2}$ and let $ a,b$ the two positive roots of $ x^2-ux+e^v=0$. 

Then $ f(a+b)=f(\ln(ab))$ and so $ f(u)=f(v)$

And so $ f(u)=f(v)$ $ \forall u\ge 2e^{\frac v2}$

And so $ f(u)=f(v)$ and $ f(u)=f(w)$ $ \forall u\ge\max( 2e^{\frac v2}, 2e^{\frac w2})$

And so $ f(v)=f(w)$ $ \forall v,w$ and it is easy to see that constant functions indeed are solutions.

Hence the solutions : $ \boxed{f(x)=c}$
\end{solution}



\begin{solution}[by \href{https://artofproblemsolving.com/community/user/52090}{Dumel}]
	let's put $ y= \frac{1}{x}$. then $ \forall_{x>2} \ f(x)=f(0)=a$
for x=2 we have:
$ a=f(2+y)=f( \ln(2y))$
so $ f(x)=a$
\end{solution}
*******************************************************************************
-------------------------------------------------------------------------------

\begin{problem}[Posted by \href{https://artofproblemsolving.com/community/user/68719}{MJ GEO}]
	Find all functions $ f: [0,1] \to  [0,1]$ and \[ f(2x - f(x)) = x\] for all $x \in [0,1]$.
	\flushright \href{https://artofproblemsolving.com/community/c6h324846}{(Link to AoPS)}
\end{problem}



\begin{solution}[by \href{https://artofproblemsolving.com/community/user/29428}{pco}]
	\begin{tcolorbox}$ f$ form $ [0,1]$ to $ [0,1]$ and $ f(2x - f(x)) = x$\end{tcolorbox}

This immediately implies $ f((n+1)x-nf(x))=nx-(n-1)f(x)$ and so $ 1\ge nx-(n-1)f(x)\ge 0$ and so $ \frac n{n-1}x\ge f(x)\ge \frac{nx-1}{n-1}$ and so $ \boxed{f(x)=x}$ which, indeed, is a solution.
\end{solution}



\begin{solution}[by \href{https://artofproblemsolving.com/community/user/68719}{MJ GEO}]
	i cant understand your solution.coud you write more about it? its mine
$ g(x)=2x-f(x)$ so $ f(g(x))=x$. $ g(g(x))=2g(x)-f(g(x))=(2g(x)-2x)+x$ and similiary $ g^n(x)=(ng(x)-nx)+x$ so $ g(x)=x$ so $ f(x)=x$
\end{solution}
*******************************************************************************
-------------------------------------------------------------------------------

\begin{problem}[Posted by \href{https://artofproblemsolving.com/community/user/68555}{trbst}]
	Find all functions $ f:\mathbb{N}\rightarrow \mathbb{N}$ so that
\[ f(f(n+1))=f(f(n)+1)=n+2009,\quad \forall n \in \mathbb{N}.\]
	\flushright \href{https://artofproblemsolving.com/community/c6h325261}{(Link to AoPS)}
\end{problem}



\begin{solution}[by \href{https://artofproblemsolving.com/community/user/29428}{pco}]
	\begin{tcolorbox}Find the function $ f : \mathbb{N}\rightarrow \mathbb{N}$ so that : 

$ f(f(n + 1)) = f(f(n) + 1) = n + 2009\ ,\ \forall n\ \in \mathbb{N}$ .\end{tcolorbox}

$ f(f(n+1))=n+2009$ $ \implies$ "$ f(a)=f(b)$ implies $ a=b$ $ \forall a,b>1$"

So $ f(f(n+1))=f(f(n)+1)$ implies either $ f(n+1)=1$, either $ f(n+1)=f(n)+1$

And since $ f(n+1)=1$ implies $ f(1)=n+2009$ and so can occur for at most one $ n$, ge get :

Either $ f(n)=n+a$ $ \forall n$ and so $ a=1004$ 
Either $ f(n)=n+a$ $ \forall n\in[1,p]$ and $ f(n)=n-p$ $ \forall n>p$ and this gives no solution (we get $ a=p+2008$ and the a contradiction with $ f(2p+2)$)

Hence the unique solution $ \boxed{f(n)=n+1004}$ $ \forall n$
\end{solution}



\begin{solution}[by \href{https://artofproblemsolving.com/community/user/44887}{Mathias_DK}]
	\begin{tcolorbox}
$ f(f(n + 1)) = n + 2009$ $ \implies$ "$ f(a) = f(b)$ implies $ a = b$ $ \forall a,b > 1$"\end{tcolorbox}
Actually $ f(a) = f(b) \Rightarrow f(f(a)+1) = f(f(b)+1) \iff a=b$, which makes the solution slightly shorter.
\end{solution}



\begin{solution}[by \href{https://artofproblemsolving.com/community/user/29428}{pco}]
	\begin{tcolorbox}[quote="pco"]
$ f(f(n + 1)) = n + 2009$ $ \implies$ "$ f(a) = f(b)$ implies $ a = b$ $ \forall a,b > 1$"\end{tcolorbox}
Actually $ f(a) = f(b) \Rightarrow f(f(a) + 1) = f(f(b) + 1) \iff a = b$, which makes the solution slightly shorter.\end{tcolorbox}

:) you're right. I just looked at the other equality and so created a useless complexity.
Thanks for your remark.
\end{solution}
*******************************************************************************
-------------------------------------------------------------------------------

\begin{problem}[Posted by \href{https://artofproblemsolving.com/community/user/67223}{Amir Hossein}]
	Find all pairs of functions $f, g: \mathbb R \to \mathbb R$ such that
(a) If $ x < y$, then $ f(x) < f(y)$, and
(b) For all $ x,y \in \mathbb R$, \[ f(xy) = g(y)f(x) + f(y).\]
	\flushright \href{https://artofproblemsolving.com/community/c6h325439}{(Link to AoPS)}
\end{problem}



\begin{solution}[by \href{https://artofproblemsolving.com/community/user/29428}{pco}]
	\begin{tcolorbox}Find all pairs of functions $ f,g : R\rightarrow R$ such that

(a) if $ x < y$, then $ f(x) < f(y)$;

(b) for all $ x,y \in R$, $ f(xy) = g(y)f(x) + f(y)$.\end{tcolorbox}
Let $ P(x,y)$ be the assertion $ f(xy) = g(y)f(x) + f(y)$

$ P(x,1)$ $ \implies$ $ f(x)(1 - g(1)) = f(1)$ and so $ g(1) = 1$ (else $ f(x)$ is constant and condition (a) is no longer true) and so $ f(1) = 0$ and so $ f(0) < 0$ (using (a))

Since $ f(x)$ solution implies $ cf(x)$ is also solution (with $ c > 0$), wlog say $ f(0) = - 1$

Then $ P(0,x)$ $ \implies$ $ g(x) = f(x) + 1$ and $ P(x,y)$ becomes $ f(xy) = f(x)f(y) + f(x) + f(y)$ and so $ f(xy) + 1 = (f(x) + 1)(f(y) + 1)$

and so $ g(xy) = g(x)g(y)$ with $ g(x)$ strictly increasing and so $ g(x) = x^a$ with $ a > 0$ and such that $ x^a$ is an increasing function defined even for $ x < 0$ 

And it is easy to check back that this indeed is a solution.

Hence the result :

$ f(x) = \alpha(x^a - 1)$
$ g(x) = x^a$

For any $ \alpha > 0$ and any $ a > 0$ such that $ x^a$ is an increasing function defined even for $ x < 0$ (example : $ 2p + 1, \frac 1{2q + 1}, ...$)

(be careful that acceptance of $ x^a$ for $ x < 0$ when $ a\notin\mathbb N$ may vary in different countries. But $ f(x) = \sqrt [3] x - 1$ is indeed a solution, for example).
\end{solution}



\begin{solution}[by \href{https://artofproblemsolving.com/community/user/29428}{pco}]
	\begin{tcolorbox}It was supposed to be strictly increasing, i just mistranslated the task. I have a question regarding your solution. 
\begin{tcolorbox}
$ g(xy) = g(x)g(y)$ with $ g(x)$ strictly increasing and so $ g(x) = x^a$
\end{tcolorbox}

How do we know that we can assume any strictly increasing function here? I mean what shows us that all strictly increasing functions g are the solution?\end{tcolorbox}

I dont understand your question.

$ f(x)$ is increasing, so $ g(x)=f(x)+1$ is increasing too.
\end{solution}



\begin{solution}[by \href{https://artofproblemsolving.com/community/user/58355}{NikolayKaz}]
	I understand this but my question is how do we know that it holds for ANY increasing function.
\end{solution}



\begin{solution}[by \href{https://artofproblemsolving.com/community/user/29428}{pco}]
	\begin{tcolorbox}I understand this but my question is how do we know that it holds for ANY increasing function.\end{tcolorbox}

Problem \end{underlined}: $ g(xy) = g(x)g(y)$ and $ g(x)$ increasing.
Let $ P(x,y)$ be the assertion $ g(xy) = g(x)g(y)$

If $ g(u) = 0$ for some $ u\ne 0^$, $ P(u,\frac xu)$ $ \implies$ $ g(x) = 0$ $ \forall x$. Impossible since $ g(x)$ is increasing. So $ g(x)\ne 0$ $ \forall x\ne 0$
$ P(x,1)$ $ \implies$ $ g(x)(g(1) - 1) = 0$ and so $ g(1) = 1$ else $ g(x) = 0$ $ \forall x$, in contradiction with $ g(x)$ increasing.
$ P(x,0)$ $ \implies$ $ g(0)(g(x) - 1) = 0$ and so $ g(0) = 0$ else $ g(x) = 1$ $ \forall x$, in contradiction with $ g(x)$ increasing.
$ P( - 1, - 1)$ $ \implies$ $ 1 = g( - 1)^2$ and so $ g( - 1) = - 1$ since $ g(x)$ is increasing and $ g(0) = 0$
$ P(x, - 1)$ $ \implies$ $ g( - x) = - g(x)$

Let then $ h(x)$ : $ \mathbb R\to\mathbb R$ such that $ h(x) = \ln(g(e^x))$ : $ h(x)$ is increasing and $ h(x + y) = h(x) + h(y)$
This is Cauchy's equation and the only monotonic solutions are $ h(x) = cx$ for any $ c > 0$ 

So $ g(x) = x^c$ $ \forall x > 0$ and $ g(x) = - ( - x)^c$ $ \forall x < 0$

And so \begin{bolded}I was wrong : there are some other solutions \end{bolded}\end{underlined}!

The general solution for $ g(x)$ is $ g(x) = sign(x)|x|^c$ for any $ c > 0$

So for example $ g(x) = sign(x)x^2$ is a solution (that I did not give in my previous posts).

Hence the result for the original problem :

$ f(x) = \alpha(sign(x)|x|^a - 1)$
$ g(x) = sign(x)|x|^a$

For any $ \alpha > 0$ and any $ a > 0$

And thanks, NikolayKaz, for your question.  :)
\end{solution}
*******************************************************************************
-------------------------------------------------------------------------------

\begin{problem}[Posted by \href{https://artofproblemsolving.com/community/user/68025}{Pirkuliyev Rovsen}]
	Find all functions $f: \mathbb R \to \mathbb R$ such that \[ f((x+z)(y+z))=(f(x)+f(z))(f(y)+f(z)), \quad x,y,z \in \mathbb R.\]
	\flushright \href{https://artofproblemsolving.com/community/c6h325471}{(Link to AoPS)}
\end{problem}



\begin{solution}[by \href{https://artofproblemsolving.com/community/user/29428}{pco}]
	\begin{tcolorbox}Find all functions $ f: R\rightarrow R$ such that $ f((x + z)(y + z)) = (f(x) + f(z))(f(y) + f(z))$,$ x,y,z \in R$
\end{tcolorbox}
Let $ P(x,y,z)$ be the assertion $ f((x+z)(y+z))=(f(x)+f(z))(f(y)+f(z))$

1) Suppose $ \exists u$ such that $ f(u)+f(-u)\ne 0$. Then $ P(x,u,-u)$ $ \implies$ $ f(0)=(f(x)+f(-u))(f(u)+f(-u))$ and so $ f(x)=c$ with $ c=4c^2$ hence two solutions : $ f(x)=0$ and $ f(x)=\frac 14$

2) consider now $ f(x)+f(-x)=0$ $ \forall x$
This implies $ f(0)=0$ 
$ P(x,y,0)$ $ \implies$ $ f(xy)=f(x)f(y)$ and $ P(x,y,z)$ becomes $ f(x+z)f(y+z)=(f(x)+f(z))(f(y)+f(z))$

Using then $ x=y$ in this equality, we get $ f(x+z)^2=(f(x)+f(z))^2$ and so $ |f(x+z)|=|f(x)+f(z)|$

But $ f(xy)=f(x)f(y)$ implies $ f(x)\ge 0$ $ \forall x\ge 0$ and so $ f(x+y)=f(x)+f(y)$ $ \forall x,y\ge 0$

This is a classical Cauchy's equation with bounds, and so $ f(x)=ax$ and so $ f(x)=x$ (using $ f(xy)=f(x)f(y)$) and this, indeed, is a solution.

Hence the three solutions :
$ f(x)=x$

$ f(x)=0$

$ f(x)=\frac 14$
\end{solution}



\begin{solution}[by \href{https://artofproblemsolving.com/community/user/334227}{reveryu}]
	\begin{tcolorbox}and so $ f(x)=c$ with $ c=4c^2$ \end{tcolorbox}
sorry, I don't understand how you get this.
\end{solution}



\begin{solution}[by \href{https://artofproblemsolving.com/community/user/29428}{pco}]
	$f(0)=(f(x)+f(-u))(f(u)+f(-u))$ and $f(u)+f(-u)\ne 0$ implies

$f(x)=\frac{f(0)}{f(u)+f(-u)}-f(-u)=c$ constant $\forall x$

Plugging this back in original equation, we get $c=(c+c)(c+c)=4c^2$

\end{solution}
*******************************************************************************
-------------------------------------------------------------------------------

\begin{problem}[Posted by \href{https://artofproblemsolving.com/community/user/68920}{prester}]
	Let $ \mathbb{F(Z)}$ the set of all functions $ f: \mathbb{Z} \to \mathbb{Z}$ such that $ f(x)=f(x^2+x+1)$ for all integers $x \in \mathbb{Z}$.

a) Find all $ f \in \mathbb{F(Z)}$ such that $ f(x)=f(-x)$,for any $x \in \mathbb{Z}$.

b) Find all $ f \in \mathbb{F(Z)}$ such that $ f(x)=-f(-x)$, for all $x \in \mathbb{Z}$.
	\flushright \href{https://artofproblemsolving.com/community/c6h325520}{(Link to AoPS)}
\end{problem}



\begin{solution}[by \href{https://artofproblemsolving.com/community/user/29428}{pco}]
	\begin{tcolorbox}Let $ \mathbb{F(Z)}$ the set of functions $ f: \mathbb{Z} \to \mathbb{Z}$ such that $ f(x) = f(x^2 + x + 1)$, $ \forall x \in \mathbb{Z}$

$ a)$ find all $ f \in \mathbb{F(Z)}$ such that $ f(x) = f( - x)$, $ \forall x \in \mathbb{Z}$

$ b)$ find all $ f \in \mathbb{F(Z)}$ such that $ f(x) = - f( - x)$, $ \forall x \in \mathbb{Z}$\end{tcolorbox}

a) $ f(x) = f(x^2 + x + 1) = f((x+1)^2-(x+1)+1) = f(-(x+1)) = f(x+1)$ and so $ \boxed{f(x)=c}$ $ \forall x$

b) $ f(x) = f(x^2 + x + 1) = f((x+1)^2-(x+1)+1) = f(-(x+1)) = -f(x+1)$ and so $ f(2p)=a$ and $ f(2p+1)=-a$
But then $ f(0)=f(1)$ and so $ a=-a$ and so $ \boxed{f(x)=0}$ $ \forall x$
\end{solution}
*******************************************************************************
-------------------------------------------------------------------------------

\begin{problem}[Posted by \href{https://artofproblemsolving.com/community/user/63660}{Victory.US}]
	Determine all functions $f: \mathbb R \to \mathbb R$ such that
\[ f(x + y) + f(xy) = f(x) + f(y) + f(x)f(y),\quad \forall x,y \in \mathbb R.\]
	\flushright \href{https://artofproblemsolving.com/community/c6h325937}{(Link to AoPS)}
\end{problem}



\begin{solution}[by \href{https://artofproblemsolving.com/community/user/29428}{pco}]
	\begin{tcolorbox}Determine all function $ f: R \to R$ such that:
$ f(x + y) + f(xy) = f(x) + f(y) + f(x)f(y),$ $ \forall x,y \in R$\end{tcolorbox}

Let $ P(x,y)$ be the assertion $ f(x+y)+f(xy)=f(x)+f(y)+f(x)f(y)$

If $ f(1)=0$, then $ P(x-1,1)$ $ \implies$ $ f(x)=0$ which, indeed is a solution. So we'll from now consider $ f(1)\ne 0$

$ P(0,0)$ $ \implies$ $ f(0)=0$
$ P(-1,1)$ $ \implies$ $ f(1)(1+f(-1))=0$ and so $ f(-1)=-1$

$ P(-1,-1)$ $ \implies$ $ f(-2)=-1-f(1)$
$ P(-2,1)$ $ \implies$ $ f(-2)f(1)=-1-f(1)$
Comparing these two lines, we get $ f(1)^2=1$

If $ f(1)=-1$ :
$ P(x,1)$ $ \implies$ $ f(x+1)=-1-f(x)$ and so $ f(x+2)=f(x)$
Then subtracting $ P(x,y)$ from $ P(x+2,y)$ gives $ f(xy+2y)-f(xy)=0$ and so, setting $ x=\frac 2{u-1}$ and $ y=\frac{u-1}2$ implies $ f(u)=-1$, which, clearly, is not a solution

So $ f(1)=1$
$ P(x,1)$ $ \implies$ $ f(x+1)=f(x)+1$
Then subtracting $ P(x,y)$ from $ P(x+1,y)$ gives $ f(xy+y)-f(xy)=f(y)$ and so $ f(u+v)=f(u)+f(v)$
Then $ P(x,y)$ becomes $ f(xy)=f(x)f(y)$
And the system $ f(x+y)=f(x)+f(y)$ and $ f(xy)=f(x)f(y)$ is a very classical equation whose non allzero solution is $ f(x)=x$ 

Hence the solutions (after easy check that these indeed are solutions) :
$ f(x)=0$
$ f(x)=x$
\end{solution}



\begin{solution}[by \href{https://artofproblemsolving.com/community/user/63660}{Victory.US}]
	Computing $ f(1)$ is all of the solution :)
\end{solution}



\begin{solution}[by \href{https://artofproblemsolving.com/community/user/44887}{Mathias_DK}]
	\begin{tcolorbox}Determine all function $ f: R \to R$ such that:
$ f(x + y) + f(xy) = f(x) + f(y) + f(x)f(y),$ $ \forall x,y \in R$\end{tcolorbox}
It is the same as http://www.mathlinks.ro/viewtopic.php?t=325061, since after substituting $ g(x) = f(x) + 1$ we get: 
$ g(x+y) = g(x)g(y) - g(xy) + 1$
\end{solution}
*******************************************************************************
-------------------------------------------------------------------------------

\begin{problem}[Posted by \href{https://artofproblemsolving.com/community/user/68025}{Pirkuliyev Rovsen}]
	Find all a function $f: \mathbb R^{\geq 0} \to \mathbb R^{\geq 0}$ satisfying the following conditions:
i) For any $x \geq 0$, \[ f(f(x))=2009\cdot 2010x-f(x),\]
ii) For any $x>0$, we have $f(x)>0$.
	\flushright \href{https://artofproblemsolving.com/community/c6h325950}{(Link to AoPS)}
\end{problem}



\begin{solution}[by \href{https://artofproblemsolving.com/community/user/29428}{pco}]
	\begin{tcolorbox}Find all a function  $ f: [0; + \infty)\rightarrow [0; + \infty)$ satisfying the following conditions 
i)$ f(f(x)) = 2009\cdot 2010x - f(x)$ ,
ii)$ f(x) > 0$ $ x > 0$
$ x \in [0; + \infty)$\end{tcolorbox}
For a given $ x>0$, Consider the sequence :
$ u_0=x$
$ u_1=f(x)$
$ u_{n+2}=f(u_{n+1})=-u_{n+1}+2009\cdot 2010 u_n$

It's easy to solve this sequence : $ u_n=\frac{2010x+f(x)}{4019}2009^n+\frac{2009x-f(x)}{4019}(-2010)^n$

Which may be written : $ u_n=\frac{2009^n}{4019}(2010x+f(x)+(2009x-f(x))(-\frac{2010}{2009})^n)$

We know that $ u_n>0$ and so $ f(x)=2009x$, else $ u_n$ would be negative for any odd or even (depending on sign of $ f(x)-2009x$) n great enough.

And it is easy to check back that this indeed is a solution.

Hence the unique solution : $ \boxed{f(x)=2009x}$ $ \forall x\ge 0$
\end{solution}



\begin{solution}[by \href{https://artofproblemsolving.com/community/user/68025}{Pirkuliyev Rovsen}]
	Bravo Patrick !!! 
\end{solution}
*******************************************************************************
-------------------------------------------------------------------------------

\begin{problem}[Posted by \href{https://artofproblemsolving.com/community/user/51029}{mathVNpro}]
	Find all continuous functions $f: \mathbb R^{+} \to \mathbb R^+$
\[ f(f(x)) = x \quad \text{and} \quad \ f(x + 1) = \frac {f(x)}{f(x) + 1}\]
for all $x>0$.
	\flushright \href{https://artofproblemsolving.com/community/c6h326222}{(Link to AoPS)}
\end{problem}



\begin{solution}[by \href{https://artofproblemsolving.com/community/user/29428}{pco}]
	\begin{tcolorbox}Find all function $ f$ which is continuous and defined on all positive real numbers such that:
\[ f(f(x)) = x,\ f(x + 1) = \frac {f(x)}{f(x) + 1}\ \forall x > 0\]
\end{tcolorbox}

Since $ f(x)$ is defined only for $ x > 0$, then we know $ f(x) > 0$ $ \forall x > 0$ else $ f(f(x))$ would not be defined.

Let $ P(x)$ be the assertion $ f(x + 1) = \frac {f(x)}{f(x) + 1}$

Let $ Q(x)$ be the assertion $ x + 1 = f(\frac {f(x)}{f(x) + 1})$ obtained by taking $ f(LHS) = f(RHS)$ in $ P(x)$

$ P(x)$ $ \implies$ $ f(x + 1) < 1$ $ \forall x > 0$ and so $ x > 1$ $ \implies$ $ f(x) < 1$

For $ 0 < x < 1$ : $ Q(f(\frac x{1 - x}))$ $ \implies$ $ f(\frac x{1 - x}) + 1 = f(x)$ and so $ x < 1$ $ \implies$ $ f(x) > 1$

From this, we get $ f(1) = 1$ and an immediate induction gives $ f(n) = \frac 1n$

Then :

$ \frac 1{f(x + 1)} = \frac 1{f(x)} + 1$ and so $ \frac 1{f(x + n)} = \frac 1{f(x)} + n$ and so $ f(x + n) = \frac 1{n + \frac 1{f(x)}}$

Then $ x + n = f(f(x + n)) = f(\frac 1{n + \frac 1{f(x)}})$ and so (using $ f(\frac 1x)$ instead of $ x$) : $ f(\frac 1{n + x}) = n + f(\frac 1x)$

Considering $ x\in\mathbb Q^ +$ and using these two equalities and an induction on the length of continuous fraction of $ x$, we easily get $ f(x) = \frac 1x$ $ \forall x\in\mathbb Q^ +$

And so $ \boxed{f(x) = \frac 1x}$ $ \forall x\in\mathbb R^ +$ (using continuity)
\end{solution}



\begin{solution}[by \href{https://artofproblemsolving.com/community/user/68920}{prester}]
	\begin{tcolorbox}
Considering $ x\in\mathbb Q^ +$ and using these two equalities and an induction on the length of continuous fraction of $ x$, we easily get $ f(x) = \frac 1x$ $ \forall x\in\mathbb Q^ +$
....
\end{tcolorbox}

Please pco, can you explain better the above statement? Thanks
\end{solution}



\begin{solution}[by \href{https://artofproblemsolving.com/community/user/29428}{pco}]
	\begin{tcolorbox}[quote="pco"]
Considering $ x\in\mathbb Q^ +$ and using these two equalities and an induction on the length of continuous fraction of $ x$, we easily get $ f(x) = \frac 1x$ $ \forall x\in\mathbb Q^ +$
....
\end{tcolorbox}

Please pco, can you explain better the above statement? Thanks\end{tcolorbox}

Sure, I know this was not a very clear phrase. :)

Consider $ u\in\mathbb Q^+$ and its (finite, since rational) continuous fraction $ [a_1,a_2,...,a_n]$.

\begin{bolded}Start of induction \end{underlined}\end{bolded}:
If $ n=1$, $ x\in\mathbb N$ and we already established that $ f(x)=\frac 1x$

\begin{bolded}Induction step \end{underlined}\end{bolded}:
If $ n>1$, induction hypothesis says that $ f([a_2,...,a_n])=\frac 1{[a_2,...,a_n]}$ and so (using $ f(f(x))=x$) : $ f(\frac 1{[a_2,...,a_n]})=[a_2,...,a_n]$. Then two cases :

1) If $ a_1\ge 1$, $ u=[a_1,...,a_n]=a_1+\frac 1{[a_2,...,a_n]}$

Using then $ f(x+n)=\frac 1{n+\frac 1{f(x)}}$, we get $ f(u)=\frac 1{a_1+\frac 1{f(\frac 1{[a_2,...,a_n]})}}$ $ =\frac 1{a_1+\frac 1{[a_2,...,a_n]}}$ $ =\frac 1u$

2) If $ a_1=0$, then $ f(u)=f([0,a_2,...,a_n])=f(\frac 1{[a_2,...,a_n]})$ $ =[a_2,...,a_n]=\frac 1u$
\end{solution}



\begin{solution}[by \href{https://artofproblemsolving.com/community/user/68920}{prester}]
	Finally, after reading several times your previous post, I understood.   
Very nice. 

I have last question: \begin{italicized}Where do you use the assert below to get the final result? \end{italicized}

\begin{tcolorbox}
$ f(\frac {1}{n+x}) = n+f(\frac{1}{x})$
\end{tcolorbox}

Thank you very much, pco, for your time  :oops:
\end{solution}



\begin{solution}[by \href{https://artofproblemsolving.com/community/user/29428}{pco}]
	\begin{tcolorbox}Finally, after reading several times your previous post, I understood.   
Very nice. 

I have last question: \begin{italicized}Where do you use the assert below to get the final result? \end{italicized}

\begin{tcolorbox}
$ f(\frac {1}{n + x}) = n + f(\frac {1}{x})$
\end{tcolorbox}

\end{tcolorbox}
never  :blush: 
I used it in a more complex proof.
\end{solution}
*******************************************************************************
-------------------------------------------------------------------------------

\begin{problem}[Posted by \href{https://artofproblemsolving.com/community/user/68025}{Pirkuliyev Rovsen}]
	Determine all functions $ f: \mathbb{Q}\to\mathbb{R}$ such that
\[ f(x+y)+f(x-y)=f(x)+f(y)+f(f(x)-f(y))\]
for all $x,y \in \mathbb Q$.
	\flushright \href{https://artofproblemsolving.com/community/c6h326470}{(Link to AoPS)}
\end{problem}



\begin{solution}[by \href{https://artofproblemsolving.com/community/user/68920}{prester}]
	\begin{tcolorbox}Determine all functions $ f: \mathbb{Q}\to\mathbb{R}$ such that
$ f(x + y) + f(x - y) = f(x) + f(y) + f(f(x) - f(y))$
$ \forall x,y \in Q$
 





_____________________________
Azerbaijan Land of Fire  :ninja:\end{tcolorbox}

Let $ P(x,y)$ the assertion $ f(x + y) + f(x - y) = f(x) + f(y) + f(f(x) - f(y))$

$ P(0,0) \iff 2f(0)=3f(0) \implies f(0)=0$

$ P(x,0) \iff f(x)=f(f(x))\ \forall x \in \mathbb{Q}$

So $ f(\mathbb{Q}) \subseteq \mathbb{Q}$ and $ \forall x \in f(\mathbb{Q})\ \implies f(x)=x$

Let $ x,y \in f(\mathbb{Q})$ then $ P(x,y)$ $ \implies f(x+y)=f(x)+f(y)$ that is cauchy functional equation

The solutions are $ f(x)=0$ and $ f(x)=cx\ c,x \in \mathbb{Q}$ and for both we can consider that $ f(\mathbb{Q})=\mathbb{Q}$

Furthermore if $ f(x)=cx$ then $ f(f(x))=f(x) \implies c^2x=cx\ \implies c=1$ since $ c=0$ is already included in the $ f(x)=0$ solution 

So $ \forall x \in \mathbb{Q}$ we have $ f(x)=0$ and $ f(x)=x$ that are indeed solution of the equation
\end{solution}



\begin{solution}[by \href{https://artofproblemsolving.com/community/user/29428}{pco}]
	\begin{tcolorbox} Let $ P(x,y)$ the assertion $ f(x + y) + f(x - y) = f(x) + f(y) + f(f(x) - f(y))$

$ P(0,0) \iff 2f(0) = 3f(0) \implies f(0) = 0$

$ P(x,0) \iff f(x) = f(f(x))\ \forall x \in \mathbb{Q}$

So $ f(\mathbb{Q}) \subseteq \mathbb{Q}$ and $ \forall x \in f(\mathbb{Q})\ \implies f(x) = x$

Let $ x,y \in f(\mathbb{Q})$ then $ P(x,y)$ $ \implies f(x + y) = f(x) + f(y)$ that is cauchy functional equation

The solutions are $ f(x) = 0$ and $ f(x) = cx\ c,x \in \mathbb{Q}$ \end{tcolorbox}

You cant use this conclusion since $ f(x+y)=f(x)+f(y)$ is currently only true for $ x,y \in f(\mathbb{Q})$, not for $ x,y \in \mathbb{Q}$. So you cant use the normal conclusion.

For example, the function $ f(x)=2\left[\frac x2\right]$ is such that $ f(x+y)=f(x)+f(y)$ $ \forall x,y \in f(\mathbb{Q})$  and is neither $ 0$, neither $ cx$
\end{solution}



\begin{solution}[by \href{https://artofproblemsolving.com/community/user/68920}{prester}]
	\begin{tcolorbox}[quote="prester"] Let $ P(x,y)$ the assertion $ f(x + y) + f(x - y) = f(x) + f(y) + f(f(x) - f(y))$

$ P(0,0) \iff 2f(0) = 3f(0) \implies f(0) = 0$

$ P(x,0) \iff f(x) = f(f(x))\ \forall x \in \mathbb{Q}$

So $ f(\mathbb{Q}) \subseteq \mathbb{Q}$ and $ \forall x \in f(\mathbb{Q})\ \implies f(x) = x$

Let $ x,y \in f(\mathbb{Q})$ then $ P(x,y)$ $ \implies f(x + y) = f(x) + f(y)$ that is cauchy functional equation

The solutions are $ f(x) = 0$ and $ f(x) = cx\ c,x \in \mathbb{Q}$ \end{tcolorbox}

You cant use this conclusion since $ f(x + y) = f(x) + f(y)$ is currently only true for $ x,y \in f(\mathbb{Q})$, not for $ x,y \in \mathbb{Q}$. So you cant use the normal conclusion.

For example, the function $ f(x) = 2\left[\frac x2\right]$ is such that $ f(x + y) = f(x) + f(y)$ $ \forall x,y \in f(\mathbb{Q})$  and is neither $ 0$, neither $ cx$\end{tcolorbox}

I supposed to be wrong... :blush:
\end{solution}



\begin{solution}[by \href{https://artofproblemsolving.com/community/user/29428}{pco}]
	\begin{tcolorbox}Determine all functions $ f: \mathbb{Q}\to\mathbb{R}$ such that
$ f(x + y) + f(x - y) = f(x) + f(y) + f(f(x) - f(y))$
$ \forall x,y \in Q$\end{tcolorbox}

Let $ P(x,y)$ be the assertion $ f(x + y) + f(x - y) = f(x) + f(y) + f(f(x) - f(y))$

$ P(0,0)$ $ \implies$ $ f(0) = 0$
$ P(x,0)$ $ \implies$ $ f(f(x)) = f(x)$

Then  :
$ P(x,x)$ $ \implies$ $ f(2x) = 2f(x)$
$ P(2x,x)$ $ \implies$ $ f(3x) = 3f(x)$
And an easy induction (using $ P(nx,x)$) shows that $ f(nx) = nf(x)$ $ \forall n\in \mathbb N$ and so $ f(ux) = uf(x)$ $ \forall u\in\mathbb Q^ +$

So $ f(x) = ax$ $ \forall x\in\mathbb Q^ +$ and $ f(x) = - bx$ $ \forall x\in\mathbb Q^ -$ (where $ a = f(1)$ and $ b = f( - 1)$. Then :

$ P(1, - 1)$ $ \implies$ $ 2a = a + b + f(a - b)$ and so $ f(a - b) = a - b$
$ P( - 1,1)$ $ \implies$ $ 2b = b + a + f(b - a)$ and so $ f(b - a) = b - a$
If $ a > b$ : the first line implies $ a(a - b) = a - b$ and so $ a = 1$ and the second implies $ - b(b - a) = b - a$ and so $ b = - 1$, and we get the solution $ f(x) = x$, which indeed is a solution.
If $ a < b$ : the first line implies $ - b(a - b) = a - b$ and so $ b = - 1$ and the second implies $ a(b - a) = b - a$ and so $ a = 1$, impossible since $ a < b$
If $ a = b$, we get $ f(x) = a|x|$ and $ P(1,0)$ $ \implies$ $ a = a|a|$ and so $ a\in\{ - 1,0,1\}$ which, indeed, are solutions.

Hence $ 4$ solutions to this equation:
$ f(x) = 0$
$ f(x) = x$
$ f(x) = |x|$
$ f(x) = - |x|$
\end{solution}
*******************************************************************************
-------------------------------------------------------------------------------

\begin{problem}[Posted by \href{https://artofproblemsolving.com/community/user/68025}{Pirkuliyev Rovsen}]
	Given rational numbers $a$ and $b$, find all functions $ f: \mathbb{Q}\to\mathbb{Q}$ such that 
\[ f(x+a+f(y))=f(x+b)+y\]
for all $x,y \in \mathbb Q$.
	\flushright \href{https://artofproblemsolving.com/community/c6h326471}{(Link to AoPS)}
\end{problem}



\begin{solution}[by \href{https://artofproblemsolving.com/community/user/29428}{pco}]
	\begin{tcolorbox}Given $ a,b \in Q$,$ f: \mathbb{Q}\to\mathbb{Q}$  such that  $ f(x + a + f(y)) = f(x + b) + y$  $ \forall x,y \in Q$\end{tcolorbox}

Let $ g(x)=f(x)+a-b$. The equation becomes $ g(x + b + g(y)) = g(x + b) + y$ and so $ g(x  + g(y)) = g(x) + y$

Let $ P(x,y)$ be the assertion $ g(x  + g(y)) = g(x) + y$

Let $ g(y_1)=g(y_2)$. Comparing $ P(x,y_1)$ and $ P(x,y_2)$, we immediately get $ y_1=y_2$ and so $ g(x)$ is injective.

$ P(0,0)$ $ \implies$ $ g(g(0))=g((0)$ and so $ g(0)=0$ since $ g(x)$ is injective.
$ P(0,x)$ $ \implies$ $ g(g(x))=x$
$ P(x,g(y))$ $ \implies$ $ g(x+y)=g(x)+g(y)$ $ \forall x,y\in\mathbb Q$

So $ g(x)=cx$ and $ g(g(x))$ $ \implies$ $ c=\pm 1$

Hence two solutions (easy to check back that these two necessary forms are sufficient) :

$ f(x)=x+b-a$

$ f(x)=-x+b-a$
\end{solution}
*******************************************************************************
-------------------------------------------------------------------------------

\begin{problem}[Posted by \href{https://artofproblemsolving.com/community/user/68025}{Pirkuliyev Rovsen}]
	Determine all functions $ f: \mathbb{N}\to\mathbb{R}$ such that for all integers $n \geq 1$, we have
\[ \log_{n+1}f(n)=\log_{f(n+2)}(n+3).\]
	\flushright \href{https://artofproblemsolving.com/community/c6h326472}{(Link to AoPS)}
\end{problem}



\begin{solution}[by \href{https://artofproblemsolving.com/community/user/29428}{pco}]
	\begin{tcolorbox}Determine the functions $ f: \mathbb{N}\to\mathbb{R}$   $ n \neq 1$ such that $ \log_{n + 1}f(n) = \log_{f(n + 2)}(n + 3)$\end{tcolorbox}

I suppose that $ n\ne 1$ in the problem statement means that $ f$ : $ \mathbb N\backslash\{1\}\to\mathbb R$
In order $ \log_{n + 1}f(n)$ be defined, we may also say that $ f$ : $ \mathbb N\backslash\{1\}\to\mathbb R^ +$

The equation may be written : $ \frac {\ln(f(n))}{\ln(n + 1)} = \frac {\ln(n + 3)}{\ln(f(n + 2))}$

Let then $ g(n) = \frac {\ln(f(n))}{\ln(n + 1)}$. The equation becomes $ g(n + 2) = \frac 1{g(n)}$

Hence the solutions :

Let $ a,b\in\mathbb R^*$ :
$ n = 0\pmod 4$ $ \implies$ $ f(n) = (n + 1)^a$
$ n = 1\pmod 4$ $ \implies$ $ f(n) = (n + 1)^b$
$ n = 2\pmod 4$ $ \implies$ $ f(n) = (n + 1)^{\frac 1a}$
$ n = 3\pmod 4$ $ \implies$ $ f(n) = (n + 1)^{\frac 1b}$
\end{solution}
*******************************************************************************
-------------------------------------------------------------------------------

\begin{problem}[Posted by \href{https://artofproblemsolving.com/community/user/25405}{AndrewTom}]
	A function $ f$ is defined over the set of all positive integers and satisfies $ f(1)=1996$ and 
\[f(1)+f(2)+\cdots +f(n) =n^{2}f(n)\]
for all $ n>1$. Calculate the value of $ f(1996)$.
	\flushright \href{https://artofproblemsolving.com/community/c6h326736}{(Link to AoPS)}
\end{problem}



\begin{solution}[by \href{https://artofproblemsolving.com/community/user/29428}{pco}]
	\begin{tcolorbox}A function $ f$ is defined over the set of all positive integers and satisfies

$ f(1) = 1996$

and 

$ f(1) + f(2) + ... + f(n) = n^{2}f(n)$ for all $ n > 1$.

Calculate the exact value of $ f(1996)$.\end{tcolorbox}

Subtracting $ f(1) + f(2) + ... + f(n-1) = (n-1)^{2}f(n-1)$ from $ f(1) + f(2) + ... + f(n) = n^{2}f(n)$, we get $ f(n)=n^2f(n)-(n-1)^2f(n-1)$ and so $ f(n)=\frac{n-1}{n+1}f(n-1)$

So $ f(n)=\frac{2f(1)}{n(n+1)}$ and so $ \boxed{f(1996)=\frac{2}{1997}}$
\end{solution}



\begin{solution}[by \href{https://artofproblemsolving.com/community/user/64868}{mahanmath}]
	Hi !
$ f(1) + f(2) + ... + f(n) = n^{2}f(n)$
 
$ f(1) + f(2) + ... + f(n) + f(n + 1) = (n + 1)^{2}f(n + 1)$

 :arrow:  So we have : 

$ \frac {f(n + 1)}{f(n)} = \frac {n}{n + 2}$

$ \prod_{1}^{1995} \frac {f(n + 1)}{f(n)} = \frac {f(1996)}{f(1)}$

But we can calculate this product from another way :

$ \prod_{1}^{1995} \frac {f(n + 1)}{f(n)} = \prod_{1}^{1995} \frac {n}{n + 2} = \frac {2}{1997*1996}$ 

thus $ \frac {2}{1997*1996} = \frac {f(1996)}{f(1)} = \frac {f(1996)}{1996}$

So $ f(1996) = \frac {2}{1997}$

EDIT : sorry , for double posting .I didn`t see your post  :blush: .
\end{solution}
*******************************************************************************
-------------------------------------------------------------------------------

\begin{problem}[Posted by \href{https://artofproblemsolving.com/community/user/29381}{james digol}]
	Find all functions $ f: \mathbb{R} \rightarrow  \mathbb{R}$ that satisfy \[ f(m+nf(m))=f(m)+mf(n)\] for all $ m$ and $ n$.
	\flushright \href{https://artofproblemsolving.com/community/c6h326890}{(Link to AoPS)}
\end{problem}



\begin{solution}[by \href{https://artofproblemsolving.com/community/user/68920}{prester}]
	\begin{tcolorbox}Find all functions $ f: \mathbb{R} \rightarrow \mathbb{R}$ that satisfy
\[ f(m + nf(m)) = f(m) + mf(n)\]
for all $ m$ and $ n$.\end{tcolorbox}

$ m,n \in  \mathbb{R} ?$ Usually we assume that $ m,n \in N$...
\end{solution}



\begin{solution}[by \href{https://artofproblemsolving.com/community/user/29428}{pco}]
	\begin{tcolorbox}Find all functions $ f: \mathbb{R} \rightarrow \mathbb{R}$ that satisfy
\[ f(m + nf(m)) = f(m) + mf(n)\]
for all $ m$ and $ n$.\end{tcolorbox}

\begin{italicized}(already posted, I think)\end{italicized}

Let $ P(x,y)$ be the assertion $ f(x + yf(x)) = f(x) + xf(y)$

$ f(x) = 0$ $ \forall x$ is a solution and let us consider from now that $ \exists a$ such that $ f(a)\ne 0$

1) $ f(x) = 0$ $ \iff$ $ x = 0$
=======================
$ P(1,0)$ $ \implies$ $ f(0) = 0$
Let $ u$ such that $ f(u) = 0$ : $ P(u,a)$ $ \implies$ $ uf(a) = 0$ and so $ u = 0$
Q.E.D.

2) $ f( - 1) = - 1$ and $ f(1) = 1$
==========================
$ P( - 1, - 1)$ $ \implies$ $ f( - 1 - f( - 1)) = 0$ and so, according to 1) above : $ - 1 - f( - 1) = 0$ and so $ f( - 1) = - 1$

Let then $ u = 1 - f(1)$ : $ P(1, - 1)$ $ \implies$ $ f(u) = - u$
Then $ P(u,1)$ $ \implies$ $ f(u + f(u)) = f(u) + uf(1)$ $ \implies$ $ 0 = u(f(1) - 1) = - u^2$ and so $ u = 0$ and $ f(1) = 1$
Q.E.D.

3) $ f(px) = pf(x)$ $ \forall p\in\mathbb Q$, $ \forall x$ 
=====================================================

$ P(1,x)$ $ \implies$ $ f(x + 1) = f(x) + 1$ and so $ f(x + n) = f(x) + n$ and $ f(n) = n$ $ \forall n\in\mathbb Z$, $ \forall x$
$ P(n,x)$ $ \implies$ $ n + f(nx) = n + nf(x)$ and so $ f(nx) = nf(x)$ $ \forall n\in\mathbb Z$, $ \forall x$
Then, using $ f(\frac pqx) = f(p\frac xq) = pf(\frac xq)$ and $ f(x) = f(q\frac xq) = qf(\frac xq)$, we get the result.
Q.E.D.

4) $ f(x) = x$
===========
Let $ x\ne 0$ : 
$ P(x,\frac {y - x}{f(x)})$ $ \implies$ $ f(y) = f(x) + xf(\frac {y - x}{f(x)})$
$ P(2x,\frac {y - x}{f(2x)})$ $ \implies$ $ f(x + y) = f(2x) + 2xf(\frac {y - x}{f(2x)})$ $ = 2f(x) + xf(\frac {y - x}{f(x)})$

And so, subtracting these lines : $ f(x + y) = f(x) + f(y)$, obviously still true for $ x = 0$

Then $ P(x,y)$ may be written $ f(yf(x)) = xf(y)$ and so $ f(f(x)) = x$ and so $ f(xy) = f(x)f(y)$

And we got the very classical system :
$ f(x + y) = f(x) + f(y)$
$ f(xy) = f(x)f(y)$

Whose unique non constant solution is $ f(x) = x$ which, indeed, is a solution of original equation.

5) Synthesis of solutions :
=========================
$ f(x) = 0$ $ \forall x$
$ f(x) = x$ $ \forall x$
\end{solution}



\begin{solution}[by \href{https://artofproblemsolving.com/community/user/29381}{james digol}]
	Job well done pco...  
The solution that you have is way too simple and beautiful than mine..so I don't have to post it coz it's really long..i'm slow with latex..but i'll try posting it if anybody cares..Anyway, the problem was originally proposed by Wu Wei Chao (the lecturer) and I'm not sure if it's already posted somewhere here in ML..but sure there are a lot of similar problems like this in which the solutions are the zero and identity functions...
\end{solution}



\begin{solution}[by \href{https://artofproblemsolving.com/community/user/100418}{DVDthe1st}]
	Very nice problem! I liked the difficulties presented in this problem.

I find that part (4) of pco's solution is quite magical... maybe the following can be motivated more easily?

[hide="Solution"]Let $(x,y)$ denote the assertion that $f(x+yf(x))=f(x)+xf(y)$

1) $f(a)=0 \Leftrightarrow a=0$   (apart from when $f(x)=0 \forall x$)

$(x,0): xf(0)=0$
Thus $f(0)=0$.

If $f(a)=0$ for some $a$,
$(x,a): f(x+af(x))=f(x)$
Thus either $f$ is constant (in which case we see that $f(x)=0 \forall x$ is indeed a solution, or $a=0$.

2) $f$ is injective.

Suppose $f(x)=f(x+k)$ (for some $x\neq 0$, since $x=0 \Rightarrow k=0$.
$(x,\frac{k}{f(x)}): f(x+k) = f(x) + \frac{k}{f(x)}x$
$\Rightarrow \frac{kx}{f(x)} = 0$
Since $x,f(x)\neq 0$, thus $k=0$ and hence $f$ is injective.

3) $f(1)=1$

$(1,y): f(1+yf(1))=f(1)+f(y)$
If $f(1)\neq 1$, then set $y=\frac{1}{1-f(1)}$ which is equivalent to $1+yf(1)=y$.
$\Rightarrow f(1)=0$, which is clearly false due to injectivity.

4) $f(x+1)=f(x)+1, f(2x)=2f(x)\forall x$

$(1,y): f(y+1)=f(y)+1$
$(1,1): f(2)=2$
$(2,y): f(2y+2) = 2+2f(y)$
$\Rightarrow 2+2f(y) = f(2y+2) = f(2y+1) + 1 = f(2y) + 2$
$\Rightarrow f(2y)=2f(y)$

5) $f(x)=x\forall x$

$(x,\frac{x}{f(x)}): f(2x)=f(x)+xf(\frac{x}{f(x)}$
$\Rightarrow f(\frac{x}{f(x)})=\frac{f(x)}{x}$

$(y,y): f(y+yf(y))= y+yf(y)$
If $yf(y)=1$, then $f(y)=y$.

Combining the two, we see that $f(x)^2 = x^2$.

If for some $b\neq 0$, $f(b)=-b$, then
$(b,b): b^2-b = 0$
$\Rightarrow b=0 \text{ or } 1$, either of which will contradict with $f(b)=-b$. 

Thus either $f(x)=x\forall x$ hhor $f(x)=0\forall x$[\/hide]
\end{solution}



\begin{solution}[by \href{https://artofproblemsolving.com/community/user/221848}{yassino}]
	4) $ f(x) = x$
===========
Let $ x\ne 0$ : 
$ P(x,\frac {y - x}{f(x)})$ $ \implies$ $ f(y) = f(x) + xf(\frac {y - x}{f(x)})$
$ P(2x,\frac {y - x}{f(2x)})$ $ \implies$ $ f(x + y) = f(2x) + 2xf(\frac {y - x}{f(2x)})$ $ = 2f(x) + xf(\frac {y - x}{f(x)})$
@Pco , how did you get up this magical substituion ?
\end{solution}



\begin{solution}[by \href{https://artofproblemsolving.com/community/user/223099}{MathPanda1}]
	Is this comparable to this year's IMO #5 in beauty and difficulty?
\end{solution}



\begin{solution}[by \href{https://artofproblemsolving.com/community/user/180203}{codyj}]
	IMO #5 was neither beautiful nor difficult
\end{solution}



\begin{solution}[by \href{https://artofproblemsolving.com/community/user/223099}{MathPanda1}]
	Yet it only had 30 perfects  :P ! Would this problem (not the IMO one) be harder then?
\end{solution}



\begin{solution}[by \href{https://artofproblemsolving.com/community/user/180203}{codyj}]
	much easier
\end{solution}



\begin{solution}[by \href{https://artofproblemsolving.com/community/user/223099}{MathPanda1}]
	Which one is much easier?
\end{solution}



\begin{solution}[by \href{https://artofproblemsolving.com/community/user/180203}{codyj}]
	this one is much easier
\end{solution}
*******************************************************************************
-------------------------------------------------------------------------------

\begin{problem}[Posted by \href{https://artofproblemsolving.com/community/user/34524}{Algadin}]
	Find all functions $f: \mathbb R \to \mathbb R$ which satisfy
\[ f(y-f(x))=f(x^{2010}-y)-2009yf(x)\]
for all reals $x$ and $y$.
	\flushright \href{https://artofproblemsolving.com/community/c6h326924}{(Link to AoPS)}
\end{problem}



\begin{solution}[by \href{https://artofproblemsolving.com/community/user/29428}{pco}]
	\begin{tcolorbox}Find all $ f: R\to R$ which satisfy:
$ f(y - f(x)) = f(x^{2010} - y) - 2009yf(x)$\end{tcolorbox}

Let $ P(x,y)$ be the assertion $ f(y - f(x)) = f(x^{2010} - y) - 2009yf(x)$

$ P(x,\frac{f(x)+x^{2010}}2)$ $ \implies$ $ f(x)(f(x)+x^{2010})=0$ and so $ \forall x$ : either $ f(x)=0$, either $ f(x)=-x^{2010}$

As a consequence, $ f(0)=0$. 
Suppose now $ \exists u\ne 0$ such that $ f(u)=-u^{2010}$ : $ P(u,u^{2010})$ $ \implies$ $ f(2u^{2010}) =  2009u^{4020}$ and, since $ RHS\ne 0$, we get $ f(2u^{2010}) =-(2u^{2010}) ^{2010}$ and so :

$ 2009u^{4020}=-(2u^{2010}) ^{2010}$ which is impossible since $ LHS>0$ while $ RHS <0$

Hence the unique solution : $ \boxed{f(x)=0}$ $ \forall x$ (which, indeed, is a solution).
\end{solution}
*******************************************************************************
-------------------------------------------------------------------------------

\begin{problem}[Posted by \href{https://artofproblemsolving.com/community/user/68920}{prester}]
	Find all functions $f: \mathbb R \to \mathbb R$ such that for all real $x$ and $y$,
\[xf(y)+yf(x)=(x+y)f(x)f(y).\]
	\flushright \href{https://artofproblemsolving.com/community/c6h326946}{(Link to AoPS)}
\end{problem}



\begin{solution}[by \href{https://artofproblemsolving.com/community/user/29428}{pco}]
	\begin{tcolorbox}Find all function $ f$ which are defined for all $ x \in \mathbb{R}$ and, for any $ x,y$ satisfy
$ xf(y) + yf(x) = (x + y)f(x)f(y)$\end{tcolorbox}
Let $ P(x,y)$ be the assertion $ xf(y)+yf(x)=(x+y)f(x)f(y)$

If $ \exists u\ne 0$ such that $ f(u)=0$, then $ P(x,u)$ $ \implies$ $ f(x)=0$ $ \forall x$. So from now, we'll consider $ f(x)\ne 0$ $ \forall x\ne 0$

The equation may then be written $ x(f(x)-1)f(y)+yf(x)(f(y)-1)=0$ and, considering $ x,y\ne 0$ : $ \frac{x(f(x)-1)}{f(x)}+\frac{y(f(y)-1)}{f(y)}=0$

And so $ g(x)+g(y)=0$ $ \forall x,y\ne 0$ where $ g(x)=\frac{x(f(x)-1)}{f(x)}$ and so $ g(x)=0$ and so $ f(x)=1$ $ \forall x\ne 0$ and we got a second family of solutions.

Hence the solutions :
=============
$ f(x)=0$ $ \forall x$

$ f(x)=1$ $ \forall x\ne 0$ and $ f(0)=a$ for any real $ a$
\end{solution}



\begin{solution}[by \href{https://artofproblemsolving.com/community/user/68920}{prester}]
	\begin{tcolorbox}So from now, we'll consider $ f(x)\ne 0$ $ \forall x\ne 0$

\end{tcolorbox}

Dear pco, why do you exclude function that are not zero only for some $ x \in \mathbb{R}$ and not for any? I mean what about discontinous functions?
\end{solution}



\begin{solution}[by \href{https://artofproblemsolving.com/community/user/29428}{pco}]
	\begin{tcolorbox}[quote="pco"]So from now, we'll consider $ f(x)\ne 0$ $ \forall x\ne 0$

\end{tcolorbox}

Dear pco, why do you exclude function that are not zero only for some $ x \in \mathbb{R}$ and not for any? I mean what about discontinous functions?\end{tcolorbox}

Because I have shown that if $ \exists u\ne 0$ such that $ f(u)=0$, then $ f(x)=0$ $ \forall x$

So, either $ f(x)=0$ $ \forall x$, either $ f(x)\ne 0$ $ \forall x\ne 0$
\end{solution}



\begin{solution}[by \href{https://artofproblemsolving.com/community/user/44887}{Mathias_DK}]
	\begin{tcolorbox}Find all function $ f$ which are defined for all $ x \in \mathbb{R}$ and, for any $ x,y$ satisfy
$ xf(y) + yf(x) = (x + y)f(x)f(y)$\end{tcolorbox}
Let $ P(x,y) \iff xf(y) + yf(x) = (x + y)f(x)f(y)$
$ P(x,0) \iff xf(0) = xf(x)f(0)$
$ x \neq 0, P(x,x) \iff 2xf(x) = 2xf(x)^2 \iff f(x) = 0$ or $ f(x) = 1$, for all $ x \neq 0$
Assume that $ a,b \neq 0, f(a) = 0, f(b) = 1$. Then:
$ P(a,b) \iff a = 0$ contradiction. So either $ f(x) = 0 \forall x \neq 0$ or $ f(x) = 1 \forall x \neq 0$.
If $ f(x) = 0 \forall x \neq 0$, then $ P(0,1) \iff f(0) = 0$ and hence $ f(x) = 0 \forall x \in \mathbb{R}$, which is seen to be a solution.
If $ f(x) = 1 \forall x \neq 0$, then $ f(x) = 1, \forall x \neq 0, f(0) = c$ is a solution for all $ c \in \mathbb{R}$
\end{solution}



\begin{solution}[by \href{https://artofproblemsolving.com/community/user/68920}{prester}]
	Yes, ok. Thank you guys for your solutions. My solution is the same as mathias

As pco and mathias have proved, there are no discontinuous solutions of type $ f(x)=0,\ \forall x \in \mathbb{A}$ and $ f(x)=1\ \forall x \in \mathbb{R}\setminus A$ where $ \mathbb{A}$ is any subset of $ \mathbb{R}$ as indicated in the book from which I got this problem. Obvioulsy the second solution $ f(x)=1\ \forall x \ne 0$ can be discontinuous at $ x=0$ when $ f(0) \ne 1$
\end{solution}
*******************************************************************************
-------------------------------------------------------------------------------

\begin{problem}[Posted by \href{https://artofproblemsolving.com/community/user/34524}{Algadin}]
	Find all functions $g: \mathbb R \to \mathbb R$ such that for all real $x$ and $y$,
 \[g(x+y)+g(xy)=g(x)g(y)+g(x)+g(y).\]
	\flushright \href{https://artofproblemsolving.com/community/c6h327089}{(Link to AoPS)}
\end{problem}



\begin{solution}[by \href{https://artofproblemsolving.com/community/user/29428}{pco}]
	\begin{tcolorbox}Find all functions $ g: R\to R$ such that for arbitrary real number $ x$ and $ y$:
$ g(x + y) + g(xy) = g(x)g(y) + g(x) + g(y)$\end{tcolorbox}
\begin{italicized}Already posted, I think\end{italicized}

Let $ P(x,y)$ be the assertion $ f(x+y)+f(xy)=f(x)f(y)+f(x)+f(y)$

$ P(0,0)$ $ \implies$ $ f(0)=0$
If $ f(1)=0$, then $ P(x-1,1)$ $ \implies$ $ f(x)=0$ $ \forall x$ which indeed is a solution. So we'll consider from now that $ f(1)\ne 0$

$ P(-1,1)$ $ \implies$ $ f(-1)=-1$

$ P(-1,-1)$ $ \implies$ $ f(-2)=-1-f(1)$
$ P(-2,1)$ $ \implies$ $ f(-2)f(1)=-1-f(1)$
Comparing these two lines, we get $ f(1)^2=1$

If $ f(1)=-1$, then $ P(x,1)$ $ \implies$ $ f(x+1)=-f(x)-1$ and so $ f(x+2)=f(x)$ and so $ f(2)=f(0)=0$, but then $ P(\frac 12,2)$ $ \implies$ $ f(1)=0$, contradiction

So $ f(1)=1$ and $ P(x,1)$ $ \implies$ $ f(x+1)=f(x)+1$ 

Let then $ x\ne 0$ : $ P(x,\frac yx+1)$ $ \implies$ $ f(x+\frac yx)+f(x+y)=f(x)f(\frac yx)+2f(x)+f(\frac yx)$
Subtracting $ P(x,\frac yx)$ from this equality, we get : $ f(x+y)=f(x)+f(y)$, still true for $ x=0$

Then $ P(x,y)$ becomes $ f(xy)=f(x)f(y)$ and we have the very classical system :

$ f(x+y)=f(x)+f(y)$
$ f(xy)=f(x)f(y)$

Whose unique non constant solution is $ f(x)=x$ which, indeed, is a solution of your equation.

Hence the two solutions :
$ f(x)=0$ $ \forall x$
$ f(x)=x$ $ \forall x$
\end{solution}
*******************************************************************************
-------------------------------------------------------------------------------

\begin{problem}[Posted by \href{https://artofproblemsolving.com/community/user/29381}{james digol}]
	Let a function $ f: \mathbb{R} \rightarrow \mathbb{R}$ satisfy \[ f(x^n-y^n)=(x-y)\left[ f(x)^{n-1}+f(x)^{n-2}f(y)+\cdots+f(x)f(y)^{n-2}+f(y)^{n-1} \right].\] Prove that $ f(rx)=rf(x)$ for all rational $ r$ and all real $ x$.
	\flushright \href{https://artofproblemsolving.com/community/c6h327154}{(Link to AoPS)}
\end{problem}



\begin{solution}[by \href{https://artofproblemsolving.com/community/user/29428}{pco}]
	\begin{tcolorbox}Let a function $ f: \mathbb{R} \rightarrow \mathbb{R}$ satisfy
\[ f(x^n - y^n) = (x - y)\left[ f(x)^{n - 1} + f(x)^{n - 2}f(y) + \cdots + f(x)f(y)^{n - 2} + f(y)^{n - 1} \right].\]
Prove that $ f(rx) = rf(x)$ for all rational $ r$ and all real $ x$.\end{tcolorbox}

I suppose $ n$ is a parameter $ \in\mathbb N$

Let $ P(x,y)$ be the assertion $ f(x^n-y^n)=(x-y)\sum_{k=0}^{n-1}f(y)^kf(x)^{n-1-k}$
Let $ Q(z)$ be the assertion $ f(zx)=zf(x)$ $ \forall x\in\mathbb R$

$ P(0,0)$ $ \implies$ $ f(0)=0$
$ P(x,0)$ $ \implies$ $ f(x^n)=xf(x)^{n-1}$
$ P(0,x)$ $ \implies$ $ f(-x^n)=-xf(x)^{n-1}$

So (comparing the two last lines) : $ f(-x)=-f(x)$

1) $ n$ even
===========
Multiplying both sides of $ P(x,y)$ by $ f(x)-f(y)$, we get $ R(x,y)$ : $ (f(x)-f(y))f(x^n-y^n)=(x-y)(f(x)^n-f(y)^n)$
Consider $ x,y\ne 0$ such that $ |x|\ne|y|$ and $ |f(x)|\ne |f(y)|$
Then, dividing $ R(x,y)$ by $ R(x,-y)$, we get $ \frac {f(x)-f(y)}{f(x)+f(y)}=\frac{x-y}{x+y}$ and so $ \frac{f(x)}x=\frac{f(y)}y$
And it is easy to get rid of the restrictions $ |f(x)|\ne |f(y)|$ and to conclude $ f(x)=xf(1)$ $ \forall x$
And so $ Q(z)$ true $ \forall z$
Q.E.D.


2) $ n$ odd
========
$ f(-x)=-f(x)$ $ \implies$ $ Q(z)\implies Q(-z)$

Suppose now $ Q(z)$ : 
$ P(x,zx)$ $ \implies$ $ f((1-z^n)x^n)=(x-zx)\sum_{k=0}^{n-1}z^kf(x)^{n-1}$ $ =xf(x)^{n-1}(1-k^n)$ $ =(1-k^n)f(x^n)$
So $ f((1-z^n)x)=(1-z^n)f(x)$ $ \forall x\ge 0$ and, since $ f(-x)=-f(x)$ : $ f((1-z^n)x)=(1-z^n)f(x)$ $ \forall x$
So $ Q(z)$ $ \implies$ $ Q(1-z^n)$

Suppose now $ Q(z)$ :
$ P(|z|^{\frac 1n}x,0)$ $ \implies$ $ f(|z|x^n)=|z|^{\frac 1n}xf(|z|^{\frac 1n}x)^{n-1}$
Since $ Q(z)$ $ \implies$ $ Q(|z|)$, this last equlity becomes $ |z|f(x^n)=|z|^{\frac 1n}xf(|z|^{\frac 1n}x)^{n-1}$
And so $ |z|xf(x)^{n-1}=|z|^{\frac 1n}xf(|z|^{\frac 1n}x)^{n-1}$
And so $ f(|z|^{\frac 1n}x)^{n-1}=(|z|^{\frac 1n}f(x))^{n-1}$
Since $ n$ odd, $ f(x^n)=xf(x)^{n-1}$ shows that $ f(x)$ and $ x$ has same signs and so $ f(|z|^{\frac 1n}x)=|z|^{\frac 1n}f(x))$ and so $ Q(|z|^{\frac 1n})$


So we know :
T1 : $ Q(z)\implies Q(-z)$
T2 : $ Q(z)\implies Q(|z|^{\frac 1n})$
T3 : $ Q(z)\implies Q(1-z^n)$

From this, it is easy to conclude :
For any $ z>0$ : $ Q(z)$ $ \implies$ $ Q(z^{\frac 1n})$ $ \implies$ $ Q(-z^{\frac 1n})$ $ \implies$ $ Q(1-(-z^{\frac 1n})^n)$ and so $ Q(z+1)$
And, since $ Q(1)$ is true, $ Q(n)$ is true $ \forall n\in\mathbb N$ and so $ Q(n)$ is true $ \forall n\in\mathbb Z$

And it's immediate to conclude $ Q(z)$ true $ \forall z\in\mathbb Q$
Q.E.D.
\end{solution}
*******************************************************************************
-------------------------------------------------------------------------------

\begin{problem}[Posted by \href{https://artofproblemsolving.com/community/user/67223}{Amir Hossein}]
	Find all functions $f: \mathbb Q \to \mathbb Q$ such that for all rational $x$ and $y$,
\[ f(x+y)+f(x-y)=2f(x)+2f(y).\]
	\flushright \href{https://artofproblemsolving.com/community/c6h327357}{(Link to AoPS)}
\end{problem}



\begin{solution}[by \href{https://artofproblemsolving.com/community/user/29428}{pco}]
	\begin{tcolorbox}find all functions $ f: Q \rightarrow Q$ such that :

$ f(x + y) + f(x - y) = 2f(x) + 2f(y)$\end{tcolorbox}

Let $ P(x,y)$ be the assertion $ f(x + y) + f(x - y) = 2f(x) + 2f(y)$

$ P(0,0)$ $ \implies$ $ f(0)=0$
$ P(x,x)$ $ \implies$ $ f(2x)=4f(x)$
$ P(2x,x)$ $ \implies$ $ f(3x)=9f(x)$
And an immediate induction using $ P(nx,x)$ shows that $ f(nx)=n^2f(x)$ $ \forall n\in\mathbb N$
$ P(0,x)$ $ \implies$ $ f(-x)=f(x)$ and so $ f(nx)=n^2f(x)$ $ \forall n\in\mathbb Z$

And so $ f(nx)=n^2f(x)$ $ \forall n\in\mathbb Q$

Hence the result : $ \boxed{f(x)=f(1)x^2}$ which, indeed, is a solution.
\end{solution}
*******************************************************************************
-------------------------------------------------------------------------------

\begin{problem}[Posted by \href{https://artofproblemsolving.com/community/user/68920}{prester}]
	Find all continuous functions $f,g: \mathbb R \to \mathbb R$ such that for all real $x$ and $y$,
\[ f(x-y)=f(x)f(y) + g(x)g(y).\]
	\flushright \href{https://artofproblemsolving.com/community/c6h327556}{(Link to AoPS)}
\end{problem}



\begin{solution}[by \href{https://artofproblemsolving.com/community/user/29428}{pco}]
	\begin{tcolorbox}Find all continuous solutions of $ f(x - y) = f(x)f(y) + g(x)g(y),\ x,y \in \mathbb{R}$\end{tcolorbox}
Let $ P(x,y)$ be the assertion $ f(x-y)=f(x)f(y)+g(x)g(y)$

1) let us look first for solutions where $ f(x)=c$ is contant.
=====================================
$ P(x,x)$ $ \implies$ $ g(x)^2=c-c^2$ and so $ g(x)^2$ is constant and so $ g(x)$ is constant (since continuous).
Then $ f(x)=c$ and $ g(x)=b$ $ \implies$ $ c=c^2+b^2$ and we got a set of solutions.

2) We'll consider from now that $ f(x)$ is not a constant function.
=========================================
2.1) $ g(0)=0$ and $ f(0)=1$ and $ f(x)^2+g(x)^2=1$
-------------------------------------------------------------
$ P(x,0)$ $ \implies$ $ f(x)(1-f(0))=g(x)g(0)$

If $ g(0)\ne 0$, then $ g(x)=af(x)$ where $ a=\frac{1-f(0)}{g(0)}$ and $ P(x,x)$ becomes $ f(0)=(a^2+1)f(x)^2$ and so $ f(x)^2$ is constant and so $ f(x)$ is constant (since continuous). 
So $ g(0)=0$
So $ f(0)=1$, else $ P(x,0)$ $ \implies$ $ f(x)(1-f(0))=g(x)g(0)$ $ \implies$ $ f(x)=0$, constant.
And then $ P(x,x)$ $ \implies$ $ f(x)^2+g(x)^2=1$
Q.E.D.

2.2) $ f(x)$ is an even function and $ g(x)$ is an odd function which may be $ 0$ only on isolated points.
----------------------------------------------------------------------------------------------------------------------
Comparing $ P(x,0)$ and $ P(0,x)$ immediately implies $ f(-x)=-f(x)$
From $ f(x)^2+g(x)^2=1$, we then get that $ g(-x)^2=g(x)^2$
Suppose then that there is a non empty interval $ (a,b)$ ($ b>a$) over which $ g(x)=g(-x)$. Continuity implies $ g(x)=g(-x)$ $ \forall x\in[a,b]$
Then $ P(x,-x)$, for $ x\in[a,b]$  $ \implies$ $ f(2x)=f(x)^2+g(x)^2=1$ and so $ f(x)=1$ and so $ g(x)=0$ $ \forall x\in[2a,2b]$
Then $ P(x,y)$, for $ x,y\in[2a,2b]$ $ \implies$ $ f(x)=1$ (and so $ g(x)=0$) $ \forall x\in[2a-2b,2b-2a]$

So $ g(x)=g(-x)$ $ \forall x\in[a,b]$ implies $ g(x)=g(-x)$ $ \forall x\in[2a-2b,2b-2a]$ and $ f(x)=1$ and $ g(x)=0$ $ \forall x\in[2a-2b,2b-2a]$
Repeating this process with $ [2a-2b,2b-2a]$ instead of $ [a,b]$ implies $ g(x)=g(-x)$ $ \forall x\in[8a-8b,8b-8a]$ and $ f(x)=1$ and $ g(x)=0$ $ \forall x\in[8a-8b,8b-8a]$
And an immediate induction gives $ f(x)=1$ $ \forall x$ and $ g(x)=0$ $ \forall x$

So $ g(x)=g(-x)$ only on isolated points and $ g(x)=-g(-x)$ everywhere else.
So $ g(-x)=-g(x)$ $ \forall x$ (using continuity)
Q.E.D.

2.3) $ \exists x_0>0$ such that $ f(x_0)=0$
------------------------------------------------
Since $ f(x)=-f(x)$, we just have to show that $ f(x)=0$ for some $ x\ne 0$
Since $ f(0)=1>0$ and $ f(x)$ continuous, we just have to show that $ f(x)\le 0$ for some $ x\ne 0$

Since $ f(x)$ is not constant and $ |f(x)|\le 1$, $ \exists u$ such that $ f(u)\in(-1,+1)$
If $ f(u)\le 0$, we got the result.
If $ f(u)>0$, then $ P(x,-x)$ $ \implies$ $ f(2x)=f(x)^2-g(x)^2$ and so $ f(2x)=2f(x)^2-1$ and it's easy to see that the sequence defined as $ a_1=f(u)>0$ and $ a_{n+1}=2a_n^2-1$ contains some negative numbers. Hence the result.
Q.E.D.

2.4) $ \exists a>0$ such that $ f(a)=0$ and $ f(x)\in(0,1)$ $ \forall x\in(0,a)$
---------------------------------------------------------------------------------------
Let $ A=\{x>0$ such that $ f(x)=0\}$. $ A$ is non empty (according to 2.3 above)
Let $ a=\inf(A)$. Since $ f(x)$ is continuous, $ f(a)=0$ and $ a>0$ ($ a\ne 0$ since $ f(0)=1$)
And we get $ f(x)>0$ $ \forall x\in[0,a)$

If $ \exists b\in(0,a)$ such that $ f(b)=1$, then $ g(b)=0$ and :
$ P(\frac b2,-\frac b2)$ $ \implies$ $ 1=f(b)=2f(\frac b2)^2_1$ and so $ f(\frac b2)^2=1$ and, since $ f(x)>0$ $ \forall x\in[0,a)$ : $ f(\frac b2)=1$

So $ f(\frac b{2^n})=1$ and $ f(\frac b{2^n})=0$ but then $ P(x,\frac b{2^n})$ $ \implies$ $ f(x-\frac b{2^n})=f(x)$ and so $ f(x)$ is periodic with a period as little as we want, and so is constant, impossible in this paragraph.

So $ f(x)\in(0,1)$ $ \forall x\in(0,a)$
Q.E.D

2.5) $ f(x)=\cos(\frac{\pi x}{2a})$ and either $ g(x)=\sin(\frac{\pi x}{2a})$, either $ g(x)=-\sin(\frac{\pi x}{2a})$
----------------------------------------------------------------------------------------------------------------------------------
Since $ f(x)\in(0,1)$ $ \forall x\in(0,a)$, $ g(x)\ne 0$ $ \forall x\in(0,a)$ and so $ g(x)$ has a constant sign on this interval.
Since $ g(x)$ solution implies $ -g(x)$ solution, wlog say $ g(x)>0$ $ \forall x\in(0,a)$ and so $ g(x)=\sqrt{1-f(x)^2}$
From $ f(2x)=2f(x)^2-1$ and $ f(x)\ge 0$, we get $ f(\frac x2)=\sqrt{\frac{f(x)+1}2}$

It's then easy to establish with induction that $ f(\frac a{2^n})=\cos(\frac{\pi}{2^{n+1}})$ and $ g(\frac a{2^n})=\sin(\frac{\pi}{2^{n+1}})$ $ \forall n\in\mathbb N_0$


Using then $ P(x,y)$, it's easy to establish withe induction that $ f(\frac {ka}{2^n})=\cos(\frac{k\pi}{2^{n+1}})$ and $ g(\frac {ka}{2^n})=\sin(\frac{k\pi}{2^{n+1}})$ $ \forall n\in\mathbb N_0$, $ \forall$ integer $ k\in[0,2^n]$

And so, with continuity, $ f(x)=\cos(\frac{\pi x}{2a})$ and $ g(x)=\sin(\frac{\pi x}{2a})$ $ \forall x\in[0,a]$

Then, using $ f(2x)=2f(x)^2-1$, we get $ f(x)=\cos(\frac{\pi x}{2a})$ $ \forall x\in[0,2a]$
And immediate induction using $ f(2x)=2f(x)^2-1$ and the fact that $ f(x)$ is even gives $ f(x)=\cos(\frac{\pi x}{2a})$ $ \forall x$

Then $ P(x,a)$ $ \implies$ $ \cos(\frac{\pi (x-a)}{2a})=g(x)g(a)$ where $ g(a)=\pm 1$
And so either $ g(x)=\sin(\frac{\pi x}{2a})$, either $ g(x)=-\sin(\frac{\pi x}{2a})$

And it is easy to check back that this indeed is a solution.

3) Synthesis of solutions 
================
We got :
$ f(x)=c$ and $ g(x)=b$ with  $ (2c-1)^2+(2b)^2=1$

$ f(x)=\cos(ux)$ and ${ g(x)=\sin(ux})$

$ f(x)=\cos(ux)$ and ${ g(x)=-\sin(ux})$
\end{solution}



\begin{solution}[by \href{https://artofproblemsolving.com/community/user/68920}{prester}]
	I am so impressioned by your fantastic work. Many thanks pco. However not all steps are clear to me and may be there is some typo. I will come back to you with some question when I will finish to read all your proof.
\end{solution}
*******************************************************************************
-------------------------------------------------------------------------------

\begin{problem}[Posted by \href{https://artofproblemsolving.com/community/user/49205}{beginner01}]
	Does there exist a monotonic bijective function from the set of real numbers to itself that is discontinuous at some point in $ \mathbb{R}$?
	\flushright \href{https://artofproblemsolving.com/community/c6h327574}{(Link to AoPS)}
\end{problem}



\begin{solution}[by \href{https://artofproblemsolving.com/community/user/29428}{pco}]
	\begin{tcolorbox}Does there exist a monotonic bijective function from the set of real numbers to itself that is discontinuous at some point in $ \mathbb{R}$?\end{tcolorbox}

No : Wlog say $ f(x)$ is increasing and suppose $ f(x)$ is not continuous at $ x=a$

Choose any increasing sequence $ x_n$ whose limit is $ a$ such that the limit of $ f(x_n)$ is not $ f(a)$. $ f(x_n)$ is increasing and $ f(x_n)<f(a)$. And so $ f(x_n)$ has a limit $ b<f(a)$

Consider now $ d\in(b,f(a))$ and $ c$ such that $ f(c)=d$ (which exists since $ f(x)$ is bijective) :
Since $ f(c)<f(a)$ : $ c<a$
Since $ f(c)>b$, $ c>x_n$ $ \forall n$ and so $ c\ge a$
Hence contradiction.

So $ f(x)$ is continuous.
\end{solution}
*******************************************************************************
-------------------------------------------------------------------------------

\begin{problem}[Posted by \href{https://artofproblemsolving.com/community/user/76369}{peter117}]
	Find all functions $f: \mathbb R \to \mathbb R$ which satisfy $f(0) = 0$ and
\[ f(f(x)+y)=2x+f(f(y)-x)\]
for all real $x$ and $y$.
	\flushright \href{https://artofproblemsolving.com/community/c6h327587}{(Link to AoPS)}
\end{problem}



\begin{solution}[by \href{https://artofproblemsolving.com/community/user/51029}{mathVNpro}]
	\begin{tcolorbox}Find all $ f: R\to R$ which satisfy
\[ f(0) = 0;\ f(f(x) + y) = 2x + f(f(y) - x)\ (1)\]
\end{tcolorbox}
Take $ y: = - f(x)$ into $ (1)$, we obtain $ - 2x = f(f( - f(x)) - x)$, $ \forall x\in \mathbb {R}$, implying $ f$ is surjective function.
Denote $ g(x)$ by the function defined on $ \mathbb {R}$ such that $ g(x) = f(x) - x$. Then from $ (1)$, we get:
$ g(g(x) + x + y) + g(x) = g(g(y) + y - x) + g(y)$ $ (2)$, $ \forall x,y\in \mathbb {R}$.
Take $ x: = 0$ into $ (2)$, we have $ g(g(y) + y) = 0$, $ \forall y\in \mathbb {R}$ or $ g(g(x) + x) = 0$, $ \forall x\in \mathbb {R}$.
But since $ f$ is surjective, then $ g(x) = 0$, $ \forall x\in \mathbb {R}$, yielding that $ f(x) = x$, $ \forall x\in \mathbb {R}$.

\begin{italicized}Conclusion.\end{italicized} $ f(x) = x$ is only the solution to our problem.
\end{solution}



\begin{solution}[by \href{https://artofproblemsolving.com/community/user/29428}{pco}]
	\begin{tcolorbox}[quote]Find all $ f: R\to R$ which satisfy
\[ f(0) = 0;\ f(f(x) + y) = 2x + f(f(y) - x)\ (1)\]
\end{tcolorbox}
Take $ y: = - f(x)$ into $ (1)$, we obtain $ - 2x = f(f( - f(x)) - x)$, $ \forall x\in \mathbb {R}$, implying $ f$ is injective function.\end{tcolorbox}

Why should $ - 2x = f(f( - f(x)) - x)$ imply $ f$ injective ?  :huh:
\end{solution}



\begin{solution}[by \href{https://artofproblemsolving.com/community/user/51029}{mathVNpro}]
	\begin{tcolorbox}[quote="mathVNpro"][quote]Find all $ f: R\to R$ which satisfy
\[ f(0) = 0;\ f(f(x) + y) = 2x + f(f(y) - x)\ (1)\]
\end{tcolorbox}
Take $ y: = - f(x)$ into $ (1)$, we obtain $ - 2x = f(f( - f(x)) - x)$, $ \forall x\in \mathbb {R}$, implying $ f$ is injective function.\end{tcolorbox}

Why should $ - 2x = f(f( - f(x)) - x)$ imply $ f$ injective ?  :huh:\end{tcolorbox}
Sorry, I mean "surjective"  :blush: . I have edited it!
\end{solution}



\begin{solution}[by \href{https://artofproblemsolving.com/community/user/29428}{pco}]
	\begin{tcolorbox}Find all $ f: R\to R$ which satisfy
$ f(0) = 0$
and $ f(f(x) + y) = 2x + f(f(y) - x)$\end{tcolorbox}

Setting $ x=0$, we get $ f(f(y))=f(y)$ and so $ f(x)=x$ $ \forall x\in f(\mathbb R)$

And since mathVNpro proved that $ f(x)$ is surjective, $ f(\mathbb R)=\mathbb R$ and $ f(x)=x$ $ \forall x$
\end{solution}
*******************************************************************************
-------------------------------------------------------------------------------

\begin{problem}[Posted by \href{https://artofproblemsolving.com/community/user/76369}{peter117}]
	Find an integer $m>0$ such that there exists a function $f: \mathbb R^{+} \to \mathbb R^{+}$ which satisfies
\[ f(xf(y)) = x^2y^m, \quad \forall x,y>0.\]
	\flushright \href{https://artofproblemsolving.com/community/c6h327589}{(Link to AoPS)}
\end{problem}



\begin{solution}[by \href{https://artofproblemsolving.com/community/user/29428}{pco}]
	\begin{tcolorbox}Find $ m > 0$ such that
$ \exists f: R^ + \to R^ +$ which satisfy $ f(xf(y)) = x^2y^m$ $ \forall x,y > 0$\end{tcolorbox}

Let $ P(x,y)$ be the assertion $ f(xf(y)) = x^2y^m$

$ P(1,1)$ $ \implies$ $ f(f(1))=1$
$ P(x,f(1))$ $ \implies$ $ f(x)=x^2f(1)^m$

Plugging $ f(x)=ax^2$ in the original equation, we get $ a^3x^2y^4=x^2y^m$ and so $ a=1$ and $ \boxed{m=4}$ and the solution $ f(x)=x^2$
\end{solution}



\begin{solution}[by \href{https://artofproblemsolving.com/community/user/76369}{peter117}]
	Oh! This problem easy. Thanks pco
\end{solution}
*******************************************************************************
-------------------------------------------------------------------------------

\begin{problem}[Posted by \href{https://artofproblemsolving.com/community/user/68719}{MJ GEO}]
	Find all functions $f: \mathbb N \to \mathbb N$ such that for all positive integers $n$,
\[ f(f(n) - n) = 2n.\]
	\flushright \href{https://artofproblemsolving.com/community/c6h327597}{(Link to AoPS)}
\end{problem}



\begin{solution}[by \href{https://artofproblemsolving.com/community/user/29428}{pco}]
	\begin{tcolorbox}find all function $ f$ from $ N$ to $ N$ and $ f(f(n) - n) = 2n$\end{tcolorbox}

Since $ f(n): \mathbb N\to\mathbb N$, and in order $ f(f(n)-n)$ be defined, we need $ f(n)>n$

then $ f(f(n)-n)>f(n)-n$ and so $ 2n>f(n)-n$ and $ f(n)<3n$ and so $ 3n>f(n)>n$

Consider now that we got $ an>f(n)>bn$. This implies $ a(f(n)-n)>2n>b(f(n)-n)$ and so $ \frac{b+2}bn>f(n)>\frac{a+2}an$

So we can define the two sequences :
$ a_1=3$ and $ b_1=1$ and $ a_{n+1}=\frac{b_n+2}{b_n}$ and $ b_{n+1}=\frac{a_n+2}{a_n}$

And we get $ a_{2p+1}n>f(n)>b_{2p+1}n$ and $ b_{2p}n>f(n)>a_{2p}n$

And it is easy to show that these two sequences have the same limit $ 2$ and so $ \boxed{f(n)=2n}$ which, indeed, is a solution.
\end{solution}



\begin{solution}[by \href{https://artofproblemsolving.com/community/user/44887}{Mathias_DK}]
	\begin{tcolorbox}find all function $ f$ from $ N$ to $ N$ and $ f(f(n) - n) = 2n$\end{tcolorbox}
$ f: \mathbb{N} \to \mathbb{N}$
Let $ g(n) = f(n) - n$, then $ g: \mathbb{N} \to \mathbb{N}$. And $ f(g(n)) = 2n \iff g(n) + g(g(n)) = 2n$.
If $ g(a) = g(b)$ then $ g(g(a)) + g(a) = g(g(b)) + g(b) \iff a=b$, so $ g$ is injective. I will prove by induction that $ g(n) = n$.
$ g(1) + g(g(1)) = 2$ gives $ g(1)=1$ since $ g(1),g(g(1)) \ge 1$.
Assume that $ g(1)=1,g(2)=2, \cdots, g(n) = n$. Then $ g(n+1) \ge n+1$ and $ g(g(n+1)) \ge n+1$, since $ g$ is injective. And $ g(n+1)+g(g(n+1)) = 2(n+1)$ gives $ g(n+1)=n+1$. So $ g(n) = n$ and hence $ f(n)=2n$, which is easily seen to be a solution.
\end{solution}



\begin{solution}[by \href{https://artofproblemsolving.com/community/user/29428}{pco}]
	\begin{tcolorbox}[quote="MJ GEO"]find all function $ f$ from $ N$ to $ N$ and $ f(f(n) - n) = 2n$\end{tcolorbox}
$ f: \mathbb{N} \to \mathbb{N}$
Let $ g(n) = f(n) - n$, then $ g: \mathbb{N} \to \mathbb{N}$. And $ f(g(n)) = 2n \iff g(n) + g(g(n)) = 2n$.
If $ g(a) = g(b)$ then $ g(g(a)) + g(a) = g(g(b)) + g(b) \iff a = b$, so $ g$ is injective. I will prove by induction that $ g(n) = n$.
$ g(1) + g(g(1)) = 2$ gives $ g(1) = 1$ since $ g(1),g(g(1)) \ge 1$.
Assume that $ g(1) = 1,g(2) = 2, \cdots, g(n) = n$. Then $ g(n + 1) \ge n + 1$ and $ g(g(n + 1)) \ge n + 1$, since $ g$ is injective. And $ g(n + 1) + g(g(n + 1)) = 2(n + 1)$ gives $ g(n + 1) = n + 1$. So $ g(n) = n$ and hence $ f(n) = 2n$, which is easily seen to be a solution.\end{tcolorbox}

Nice, and prettier than mine  :blush:
\end{solution}
*******************************************************************************
-------------------------------------------------------------------------------

\begin{problem}[Posted by \href{https://artofproblemsolving.com/community/user/51029}{mathVNpro}]
	Find all functions $f: \mathbb R^+ \to \mathbb R^+$ such that for all positive reals $x$ and $y$,
\[ f(x + y) =f(x^2 + y^2).\]
	\flushright \href{https://artofproblemsolving.com/community/c6h328046}{(Link to AoPS)}
\end{problem}



\begin{solution}[by \href{https://artofproblemsolving.com/community/user/29428}{pco}]
	\begin{tcolorbox}Find all $ f: \mathbb {{R}^{*}}_{ + }\longrightarrow \mathbb {{R}^{*}}_{ + }$ such that:
\[ f(x + y) = f(x^2 + y^2);\ \forall x,y\in \mathbb {{R}^{*}}_{ + }\]
where $ \mathbb {{R}^{*}}_{ + }$ is the set of positive real numbers.\end{tcolorbox}

Let $ u,v>0$ such that $ u^2>4v>0$. The equation $ X^2-uX+v$ has two positive roots $ r_1$ and $ r_2$. Plugging $ x=r_1$ and $ y=r_2$ in the functional equation, we get :

$ f(u)=f(u^2-2v)$ $ \forall u,v\in\mathbb R^+$ such that $ u^2>4v>0$

And so (choosing any $ v\in(0,\frac{u^2}4)$),  $ f(x)$ is constant over $ (\frac{u^2}2,u^2)$ $ \forall u$ and so $ \boxed{f(x)=c}$ $ \forall x$
\end{solution}
*******************************************************************************
-------------------------------------------------------------------------------

\begin{problem}[Posted by \href{https://artofproblemsolving.com/community/user/76369}{peter117}]
	Let $f: \mathbb R \to \mathbb R$ be a function which satisfies the following conditions:
1) For all $x \in \mathbb R$, \[ f^2(2x)\leq 8x^2f(x).\]
2) $ f(x)$ is bounded on $ [-1;1]$.

Prove that for all reals $x$, \[f(x)\leq\frac{x^2}{2}.\]
	\flushright \href{https://artofproblemsolving.com/community/c6h328098}{(Link to AoPS)}
\end{problem}



\begin{solution}[by \href{https://artofproblemsolving.com/community/user/29428}{pco}]
	\begin{tcolorbox}Let $ f: R\to R$ which satisfy:

1) $ \forall x\in R$  $ f^2(2x)\leq 8x^2f(x)$
2) $ f(x)$ have bounded on $ [ - 1;1]$
           Prove that $ f(x)\leq\frac {x^2}{2}$\end{tcolorbox}
$ f(x)\ge 0$ $ \forall x$
For $ x\ne 0$, let $ g(x) = \frac {2f(x)}{x^2}$. We clearly get $ g(x)\ge 0$ $ \forall x$

2) may be written $ (\frac {2f(2x)}{4x^2})^2\le \frac {2f(x)}{x^2}$ $ ^\forall x\ne 0$ or again $ g^2(2x)\le g(x)$ 

$ \implies$ $ g^{2^n}(2^nx)\le g(x)$ which may also be written $ g(x)^{2^n}\le g(x2^{ - n})$ $ \forall x\ne0,\forall n\in\mathbb Z$

If $ \exists u\ne 0$ such that $ g(u) = a > 1$, then $ g(u2^{ - n})\ge a^{2^n}$ $ \iff$ $ 2f(u2^{ - n})\ge a^{2^n}u^22^{ - 2n}$

For $ n$ great enough, $ u2^{ - n}\in[ - 1,1]$ while $ a^{2^n}u^22^{ - 2n}$ is as great as we want. Hence $ f(x)$ has no bounds on $ [ - 1,1]$, in contradiction with 2).

So $ g(u)\le 1$ $ \forall u\ne 0$ and so $ f(u)\le \frac {u^2}2$ $ \forall u\ne 0$.
Setting $ x = 0$ in 1), we get $ f(0) = 0$ and so  $ f(u)\le \frac {u^2}2$ for $ u = 0$

And so $ f(x)\le \frac {x^2}2$ $ \forall x$.
\end{solution}
*******************************************************************************
-------------------------------------------------------------------------------

\begin{problem}[Posted by \href{https://artofproblemsolving.com/community/user/67223}{Amir Hossein}]
	Find all functions $f: \mathbb R \to \mathbb R$ such that for all reals $x$ and $y$,
\[f(xf(x) + f(y)) = (f(x))^2 + y.\]
	\flushright \href{https://artofproblemsolving.com/community/c6h328119}{(Link to AoPS)}
\end{problem}



\begin{solution}[by \href{https://artofproblemsolving.com/community/user/29428}{pco}]
	\begin{tcolorbox}find all functions $ f: R \rightarrow R$ such that $ f(xf(x) + f(y)) = (f(x))^2 + y$.\end{tcolorbox}

Let $ (x,y)$ be the assertion  $ f(xf(x) + f(y)) = (f(x))^2 + y$

$ P(0,x)$ $ \implies$ $ f(f(x))=x+f(0)^2$ and so $ f(x)$ is bijective and $ \exists u$ such that $ (u)=0$

$ P(u,x)$ $ \implies$ $ f(f(x))=x$ and, since we already got $ f(f(x))=x+f(0)^2$ : $ f(0)=0$

$ P(x,0)$ $ \implies$ $ f(xf(x))=f(x)^2$
$ P(f(x),0)$ $ \implies$ $ f(xf(x))=x^2$

And so $ f(x)^2=x^2$ and so $ \forall x$ : either $ f(x)=x$, either $ f(x)=-x$

Suppose now $ \exists a,b$ such that $ f(a)=a$ and $ f(b)=-b$ :
$ P(a,b)$ $ \implies$ $ f(a^2-b)=a^2+b$ and so :
either $ f(a^2-b)=a^2-b$ and so $ a^2-b=a^2+b$ $ \implies$ $ b=0$
either $ f(a^2-b)=-(a^2-b)$ and so $ -a^2+b=a^2+b$ $ \implies$ $ a=0$

And so either $ f(x)=x$ $ \forall x$, either $ f(x)=-x$ $ \forall x$, which both are indeed solutions
\end{solution}



\begin{solution}[by \href{https://artofproblemsolving.com/community/user/67223}{Amir Hossein}]
	\begin{tcolorbox}[quote="amparvardi"]find all functions $ f: R \rightarrow R$ such that $ f(xf(x) + f(y)) = (f(x))^2 + y$.\end{tcolorbox}

$ P(0,x)$ $ \implies$ $ f(f(x)) = x + f(0)^2$ and so $ f(x)$ is bijective and $ \exists u$ such that $ (u) = 0$

$ P(u,x)$ $ \implies$ $ f(f(x)) = x$ and, since we already got $ f(f(x)) = x + f(0)^2$ : $ f(0) = 0$\end{tcolorbox}

Thank you for your nice solution but I don't understand why f(0)=0.please explain it.
why we can say" $ P(0,x)$ $ \implies$ $ f(f(x)) = x + f(0)^2$ and so $ f(x)$ is bijective and $ \exists u$ such that $ (u) = 0$"??
\end{solution}



\begin{solution}[by \href{https://artofproblemsolving.com/community/user/68920}{prester}]
	I think this problem was already posted

[url]http://www.mathlinks.ro/viewtopic.php?t=56415[\/url]
\end{solution}



\begin{solution}[by \href{https://artofproblemsolving.com/community/user/67223}{Amir Hossein}]
	\begin{tcolorbox}I think this problem was already posted

[url]http://www.mathlinks.ro/viewtopic.php?t=56415[\/url]\end{tcolorbox}


Thanks!!
\end{solution}



\begin{solution}[by \href{https://artofproblemsolving.com/community/user/29428}{pco}]
	\begin{tcolorbox}[quote="pco"]\begin{tcolorbox}find all functions $ f: R \rightarrow R$ such that $ f(xf(x) + f(y)) = (f(x))^2 + y$.\end{tcolorbox}

$ P(0,x)$ $ \implies$ $ f(f(x)) = x + f(0)^2$ and so $ f(x)$ is bijective and $ \exists u$ such that $ (u) = 0$

$ P(u,x)$ $ \implies$ $ f(f(x)) = x$ and, since we already got $ f(f(x)) = x + f(0)^2$ : $ f(0) = 0$\end{tcolorbox}

Thank you for your nice solution but I don't understand why f(0)=0.please explain it.
why we can say" $ P(0,x)$ $ \implies$ $ f(f(x)) = x + f(0)^2$ and so $ f(x)$ is bijective and $ \exists u$ such that $ (u) = 0$"??\end{tcolorbox}

$ P(0,x)$ $ \implies$ $ f(f(x))=x+f(0)^2$ : just replace arguments in the assertion
And this equality shows that $ f(a)=f(b)$ $ \implies$ $ f(f(a))=f(f(b))$ $ \implies$ $ a+f(0)^2=b+f(0)^2$ $ \implies$ $ a=b$ and $ f(x)$ injective
In the same equality, set $ x=t-f(0)^2$ and you get $ f(f(t-f(0)^2))=t$ and so any real is image of some other real and $ f(x)$ is surjective.
Since surjective, there is some $ u$ such that $ f(u)=0$ (in fact $ u=-f(0)^2$)

Then $ P(u,x)$ $ \implies$ $ f(f(x))=x$ (just replace the arguments in $ P(x,y)$
And  since we already got $ f(f(x))=x+f(0)^2$, we now have $ x+f(0)^2=x$ and so $ f(0)=0$
\end{solution}
*******************************************************************************
-------------------------------------------------------------------------------

\begin{problem}[Posted by \href{https://artofproblemsolving.com/community/user/67223}{Amir Hossein}]
	Find all functions $ f: \mathbb R \to \mathbb R$ such that 
\[f(f(x)+y)=f(x^{2}-y)+4yf(x) \qquad \forall x,y \in \mathbb R.\]
	\flushright \href{https://artofproblemsolving.com/community/c6h328125}{(Link to AoPS)}
\end{problem}



\begin{solution}[by \href{https://artofproblemsolving.com/community/user/29428}{pco}]
	\begin{tcolorbox}find all functions $ f: R \rightarrow R$ such that $ f(f(x) + y) = f(x^2 - y) + 4yf(x)$.

\begin{italicized}[size=100]P.S. It's my $ 100^{th}$ post!![\/size]\end{italicized}\end{tcolorbox}

Set $ y=\frac {x^2-f(x)}2$ in the equation and you get $ f(x)(x^2-f(x))=0$ and so : $ \forall x$, either $ f(x)=0$, either $ f(x)=x^2$

Suppose now $ \exists a,b\ne 0$ such that $ f(a)=a^2$ and $ f(b)=0$ :

$ P(a,a^2-b)$ $ \implies$ $ f(2a^2-b)=4(a^2-b)a^2$ and so :
either $ f(2a^2-b)=(2a^2-b)^2$ and so $ (2a^2-b)^2=4(a^2-b)a^2$ and so $ b=0$, impossible
either $ f(2a^2-b)=0$ and so $ b=a^2$

But $ P(b,b)$ $ \implies$ $ f(b^2-b)=0$ and so either $ b^2-b=0$, either $ b^2-b=b=a^2$ and so $ b\in\{0,1,2\}$ and $ a\in\{0,1,\sqrt 2\}$, impossible.

So either $ f(x)=0$ $ \forall x$, either $ f(x)=x^2$ $ \forall x$ and it is easy to check back that these two functions indeed are solutions.
\end{solution}



\begin{solution}[by \href{https://artofproblemsolving.com/community/user/139716}{asjeykg}]
	\begin{tcolorbox}[quote="amparvardi"]find all functions $ f: R \rightarrow R$ such that $ f(f(x) + y) = f(x^2 - y) + 4yf(x)$.

\begin{italicized}[size=100]P.S. It's my $ 100^{th}$ post!![\/size]\end{italicized}\end{tcolorbox}


Set $ y=\frac {x^2-f(x)}2$ in the equation.\end{tcolorbox}
 

sorry, can you explain when we can set $ y=\frac {x^2-f(x)}2$, so that y contains f(x). I thought that if we dont know about the surjectiveness of f, we cant do it. Please can u explain. Thanks in advance
\end{solution}



\begin{solution}[by \href{https://artofproblemsolving.com/community/user/29428}{pco}]
	I did not say that for any $y$, there exists a $x$ such that ... . I just use $\frac{x^2-f(x)}2$ as a value for the second parameter of the equation, which is always possible simply because $f(x)$ is defined $\forall x$
\end{solution}



\begin{solution}[by \href{https://artofproblemsolving.com/community/user/149744}{ryuzaki}]
	another solution : 
  
 y=x² : f(f(x)+x²)=f(0)+4x²f(x) 
 
 y=-f(x) : f(0)=f(x²+f(x))-4f(x)² 
   
 f(x)=x² or f(x)=0
\end{solution}



\begin{solution}[by \href{https://artofproblemsolving.com/community/user/64716}{mavropnevma}]
	Notice the similarity with this BMO problem [url]http://www.artofproblemsolving.com/Forum/viewtopic.php?p=827178&sid=f4a997cd943660217101bfb826de86a1#p827178[\/url].
\end{solution}



\begin{solution}[by \href{https://artofproblemsolving.com/community/user/231508}{john10}]
	
Sorry to revive . how de we get either $ b^2-b=0$ either $ b^2-b=b $ . :blush: 
But $ P(b,b)$ $ \implies$ $ f(b^2-b)=0$ and so either $ b^2-b=0$, either $ b^2-b=b=a^2$ and so $ b\in\{0,1,2\}$ and $ a\in\{0,1,\sqrt 2\}$, impossible.

\end{solution}



\begin{solution}[by \href{https://artofproblemsolving.com/community/user/29428}{pco}]
	\begin{tcolorbox}Sorry to revive . how de we get either $ b^2-b=0$ either $ b^2-b=b $ . :blush: 
But $ P(b,b)$ $ \implies$ $ f(b^2-b)=0$ and so either $ b^2-b=0$, either $ b^2-b=b=a^2$ and so $ b\in\{0,1,2\}$ and $ a\in\{0,1,\sqrt 2\}$, impossible.\end{tcolorbox}

Because we proved that if $f(a)=a^2$ and $f(b)=0$ for some $a,b\ne 0$, then $b=a^2$ 

So, if $c=b^2-b\ne 0$ , and since $f(b^2-b)=0$ we have : 
$f(a)=a^2$ and $f(c)=0$ for some $a,c\ne 0$, then $c=a^2$  and so $b^2-b=b=a^2$

And so either $b^2-b=0$, either $b^2-b=b=a^2$

\end{solution}



\begin{solution}[by \href{https://artofproblemsolving.com/community/user/334227}{reveryu}]
	Why we suppose that $ \exists a,b\ne 0$. Please could you explain why a,b can't be 0.  :maybe: 
\end{solution}



\begin{solution}[by \href{https://artofproblemsolving.com/community/user/29428}{pco}]
	\begin{tcolorbox}Why we suppose that $ \exists a,b\ne 0$. Please could you explain why a,b can't be 0.  :maybe:\end{tcolorbox}
Because we are looking for existence of functions which are not the trivial $f(x)=0$ $\forall x$ or $f(x)=x^2$ $\forall x$

If such a non trivial function exists then :
1) $\exists a$ such that $f(a)\ne 0$ (and so $f(a)=a^2$ and $a\ne 0$)
(because $f(a)=a^2$ and $a=0$ would also imply $f(a)=0$)

2) $\exists b$ such that $f(b)\ne b^2$ (and so $f(b)=0$ and $b\ne 0$)
(because $f(b)=0$ and $b=0$ would also imply $f(b)=b^2$)


\end{solution}



\begin{solution}[by \href{https://artofproblemsolving.com/community/user/330078}{Delray}]
	Let $P(x,y)$, be the assertion $f(f(x)+y)=f(x^2-y)+4yf(x)$.
$P(x,x^2)$ yields $f(f(x)+x^2)=f(0)+4x^2f(x)$.
$P(x,-f(x))$ yields $f(0)=f(f(x)+x^2)-4f(x)^2$.
Combining, we have $4f(x)^2=4x^2f(x)$, implying either $f(x)=x^2$ or $f(x)=0$, both of which satisfy the original expression.$\square$
\end{solution}



\begin{solution}[by \href{https://artofproblemsolving.com/community/user/29428}{pco}]
	\begin{tcolorbox}Let $P(x,y)$, be the assertion $f(f(x)+y)=f(x^2-y)+4yf(x)$.
$P(x,x^2)$ yields $f(f(x)+x^2)=f(0)+4x^2f(x)$.
$P(x,-f(x))$ yields $f(0)=f(f(x)+x^2)-4f(x)^2$.
Combining, we have $4f(x)^2=4x^2f(x)$, implying either $f(x)=x^2$ or $f(x)=0$, both of which satisfy the original expression.$\square$\end{tcolorbox}
No.

$4f(x)^2=4x^2f(x)$ implies :
$\forall x$, either $f(x)=0$, either $f(x)=x^2$

Which is quite different from :
Either $f(x)=0$ $\forall x$, either $f(x)=x^2$ $\forall x$

For example the function $f(x)=\frac{x(x+|x|)}2$ matches $4f(x)^2=4x^2f(x)$ and must be discarded as a solution in some way (for example following my method in post #2 above)



\end{solution}



\begin{solution}[by \href{https://artofproblemsolving.com/community/user/330078}{Delray}]
	\begin{tcolorbox}[quote=Delray]Let $P(x,y)$, be the assertion $f(f(x)+y)=f(x^2-y)+4yf(x)$.
$P(x,x^2)$ yields $f(f(x)+x^2)=f(0)+4x^2f(x)$.
$P(x,-f(x))$ yields $f(0)=f(f(x)+x^2)-4f(x)^2$.
Combining, we have $4f(x)^2=4x^2f(x)$, implying either $f(x)=x^2$ or $f(x)=0$, both of which satisfy the original expression.$\square$\end{tcolorbox}
No.

$4f(x)^2=4x^2f(x)$ implies :
$\forall x$, either $f(x)=0$, either $f(x)=x^2$

Which is quite different from :
Either $f(x)=0$ $\forall x$, either $f(x)=x^2$ $\forall x$

For example the function $f(x)=\frac{x(x+|x|)}2$ matches $4f(x)^2=4x^2f(x)$ and must be discarded as a solution in some way (for example following my method in post #2 above)\end{tcolorbox}

How would I proceed from that step then?
\end{solution}



\begin{solution}[by \href{https://artofproblemsolving.com/community/user/335559}{Duarti}]
	\begin{tcolorbox}
But $ P(b,b)$ $ \implies$ $ f(b^2-b)=0$ and so either $ b^2-b=0$, either $ b^2-b=b=a^2$ and so $ b\in\{0,1,2\}$ and $ a\in\{0,1,\sqrt 2\}$, impossible.\end{tcolorbox}
I do not understand why this is impossible.
\end{solution}



\begin{solution}[by \href{https://artofproblemsolving.com/community/user/29428}{pco}]
	\begin{tcolorbox}[quote=pco]... $ b\in\{0,1,2\}$ and $ a\in\{0,1,\sqrt 2\}$, impossible.\end{tcolorbox}
I do not understand why this is impossible.\end{tcolorbox}
Because this would mean for example that :
$f(3)=0$ since $3\notin \{0,1,\sqrt 2\}$
$f(3)=3^2$ since $3\notin\{0,1,2\}$
While $0\ne 9$



\end{solution}
*******************************************************************************
-------------------------------------------------------------------------------

\begin{problem}[Posted by \href{https://artofproblemsolving.com/community/user/76594}{mhmhm}]
	Find all functions $f: \mathbb R \to \mathbb R$ such that for all reals $x$ and $y$,
\[f(x+f(y))=f(x)+x+f(x-y).\]
	\flushright \href{https://artofproblemsolving.com/community/c6h328465}{(Link to AoPS)}
\end{problem}



\begin{solution}[by \href{https://artofproblemsolving.com/community/user/68920}{prester}]
	\begin{tcolorbox}hi, find all f:R--->R that 
f(x+f(y))=f(x)+x+f(x-y)
 :)\end{tcolorbox}

Let $ P(x,y)\ \ \ f(x + f(y)) = f(x) + x + f(x - y),\ x,y \in \mathbb{R}$

Let $ f(0) = a$

$ P(0,0)$ $ \implies f(a) = 2a$

$ P( - a,a)$ $ \implies f(a) = f( - a)$

$ P( - a,0)$ $ \implies f(0) = a = f( - a) - a + f( - a)\ \implies f( - a) = a\ \implies 2a = a\ \implies a = 0$

Hence $ f(0) = 0$

$ P(x,0)$ $ \implies f(x) = f(x) + x + f(x) \ \implies f(x) = - x$ and this is indeed a solution of the original equation

Hence the result $ \boxed{f(x) = - x,\ x \in \mathbb{R}}$
\end{solution}



\begin{solution}[by \href{https://artofproblemsolving.com/community/user/29428}{pco}]
	\begin{tcolorbox}hi, find all f:R--->R that 
f(x+f(y))=f(x)+x+f(x-y)
 :)\end{tcolorbox}Let $ P(x,y)$ be the assertion $ f(x + f(y)) = f(x) + x + f(x - y)$

1) $ f(x)$ is injective
======================
If $ f(a) = f(b)$, comparing $ P(x + b,a)$ and $ P(x + b,b)$ implies $ f(x + b - a) = f(x)$
Then subtracting $ P(x,y)$ from $ P(x + b - a,y)$, we get $ b - a = 0$
Q.E.D.

2) $ \exists a$ such that $ f(a) = 0$
================================
$ P( - f(0), - f(0))$ $ \implies$ $ f( - f(0) + f( - f(0))) = f( - f(0))$
And, since $ f(x)$ is injective : $ - f(0) + f( - f(0)) = - f(0)$ and so $ f( - f(0)) = 0$
Q.E.D.

3) $ f(x) = - x$ $ \forall x$
========================
Let $ a$ such that $ f(a) = 0$
$ P(x + a,a)$ $ \implies$ $ f(x + a) = f(x + a) + x + a + f(x)$ and so $ f(x) = - (x + a)$
Setting $ x = a$ in this equation, we get $ 0 = - 2a$ and so $ a = 0$
Hence the result : $ \boxed{f(x) = - x}$ $ \forall x$ which, indeed, is a solution

\begin{bolded}edited \end{bolded}\end{underlined}: too late :) , but :
\begin{tcolorbox} $ P( - a,a)$ $ \implies f(a) = f( - a)$\end{tcolorbox}

I dont undestand this line  :blush:
\end{solution}



\begin{solution}[by \href{https://artofproblemsolving.com/community/user/68920}{prester}]
	\begin{tcolorbox}
\begin{bolded}edited \end{bolded}\end{underlined}: too late :) , but :
\begin{tcolorbox} $ P( - a,a)$ $ \implies f(a) = f( - a)$\end{tcolorbox}

I dont undestand this line  :blush:\end{tcolorbox}

Yes, you do not understand because it is wrong... :mad: 

Correct line should be $ P(-a,a)=f(0)=a=f(-a)-a+f(-2a)$ and this is not very useful

I tried so quick and I did not check with sufficient care my steps...I always try to be faster than you but I can't...without errors.... :blush: 

Next time may be....
 
\end{solution}
*******************************************************************************
-------------------------------------------------------------------------------

\begin{problem}[Posted by \href{https://artofproblemsolving.com/community/user/76594}{mhmhm}]
	Find all functions $f: (1,\infty) \to \mathbb R$ such that for all $x, y \in (1,\infty)$,
\[f(x)-f(y)=(y-x)f(xy).\]
	\flushright \href{https://artofproblemsolving.com/community/c6h328695}{(Link to AoPS)}
\end{problem}



\begin{solution}[by \href{https://artofproblemsolving.com/community/user/45762}{FelixD}]
	\begin{tcolorbox}Find all $ f: (1$,$ +\infty)\to \mathbb{R}$ such that 
\[ f(x)-f(y)=(y-x)f(xy)\]\end{tcolorbox}
Let $ P(x$,$ y)$ be the assertion $ f(x)-f(y)=(y-x)f(xy)$. [hide="@pco"] The introduction of the assertions is useful :D[\/hide] 
$ P(x$,$ 1)$ gives $ f(x)-f(1)=(1-x)f(x)$ or equivalently $ f(x)=f(1)\/x$ for all $ x\ge 1$. Substituting back into the original equation shows that this is indeed a solution.
\end{solution}



\begin{solution}[by \href{https://artofproblemsolving.com/community/user/71459}{x164}]
	But $ 1$ is not in the domain of $ f$...  :o
\end{solution}



\begin{solution}[by \href{https://artofproblemsolving.com/community/user/45762}{FelixD}]
	sry, I'm never sure if (1,...) includes one or not...^^
\end{solution}



\begin{solution}[by \href{https://artofproblemsolving.com/community/user/29428}{pco}]
	\begin{tcolorbox}find all $ f$ : $ (1,+\infty)\to\mathbb R$ such that  $ f(x)-f(y)=(y-x)f(xy)$ $ \forall x,y>1$\end{tcolorbox}
I think there is something simpler, but here is a solution :
Let $ P(x,y)$ be the assertion $ f(x)-f(y)=(y-x)f(xy)$

Let $ a>1$ and $ x\in(a,a^3)$ : then $ \sqrt{\frac xa}>1$ and $ \sqrt{ax}>1$ and $ a\sqrt{\frac ax}>1$ and so :

Adding $ P(\sqrt{\frac xa},\sqrt{ax})$, $ P(\sqrt{ax},a\sqrt{\frac ax})$ and $ P(a\sqrt{\frac ax},\sqrt{\frac xa})$, we get : $ (\sqrt{ax}-\sqrt{\frac xa})f(x)+(a\sqrt{\frac ax}-\sqrt{ax})f(a^2)+(\sqrt{\frac xa}-a\sqrt{\frac ax})f(a)=0$, 

$ \iff$ $ (\sqrt{a}-\sqrt{\frac 1a})f(x)+(\frac{a\sqrt a}x-\sqrt{a})f(a^2)+(\sqrt{\frac 1a}-\frac{a\sqrt a}x)f(a)=0$

And so $ f(x)=c(a)+\frac {d(a)}x$. Choosing then $ x,y\in(a,a\sqrt a)$ (so that $ xy\in(a,a^3)$) and plugging in $ P(x,y)$, we get $ c(a)=0$

So $ \forall a>1$, $ \exists d(a)$ such that $ f(x)=\frac{d(a)}x$ $ \forall x\in(a,a\sqrt a)$

Choosing then different values of $ a$ such that intervals $ (a,a\sqrt a)$ have common part and cover $ (1,+\infty)$, we get $ d(a)=$constant and $ f(x)=\frac dx$, which indeed is a solution.
\end{solution}



\begin{solution}[by \href{https://artofproblemsolving.com/community/user/76594}{mhmhm}]
	I don't understand. Why?
\end{solution}



\begin{solution}[by \href{https://artofproblemsolving.com/community/user/29428}{pco}]
	\begin{tcolorbox}I don't understand. Why?\end{tcolorbox}

Why what ?
\end{solution}



\begin{solution}[by \href{https://artofproblemsolving.com/community/user/76594}{mhmhm}]
	why c(a)=0 & d(a)=constant ?
\end{solution}



\begin{solution}[by \href{https://artofproblemsolving.com/community/user/29428}{pco}]
	\begin{tcolorbox}why c(a)=0 & d(a)=constant ?\end{tcolorbox}

1) $ c(a) = 0$
========
As I suggested, choose $ x,y\in(a,a\sqrt a)$ (so that $ xy\in(a,a^3)$) and plug $ f(x) = c(a) + \frac {d(a)}x$ in $ P(x,y)$ :

$ \frac {(y - x)d(a)}{xy} = (y - x)c(a) + \frac {(y - x)d(a)}{xy}$ $ \forall x,y\in(a,a\sqrt a)$ and so $ c(a) = 0$

2) $ d(a) = c$
=========
2.1) How to see it
---------------------
Choose $ a = 9$, we got $ f(x) = \frac {d(9)}x$ $ \forall x\in(9,27)$

Choose $ a = 16$, we got $ f(x) = \frac {d(16)}x$ $ \forall x\in(16,64)$ and so $ \frac {d(9)}x = \frac {d(16)}x$ $ \forall x\in(16,27)$ and so $ d(9) = d(16) = d$ and $ f(x) = \frac dx$ $ \forall x\in(9,64)$

Choose $ a = 25$, we got $ f(x) = \frac {d(25)}x$ $ \forall x\in(25,125)$ and so $ \frac {d}x = \frac {d(25)}x$ $ \forall x\in(25,64)$ and so $ d(25) = d$ and $ f(x) = \frac dx$ $ \forall x\in(9,125)$
...

2.2) how to prove it
-----------------------
Choose a sequence $ a_1 = 2$ and $ a_{n + 1} = \frac {a_n + a_n\sqrt {a_n}}2$ and use the previous method to show with induction that $ f(x) = \frac dx$ $ \forall x\in(2, + \infty)$
(using the fact that the next interval $ (a_{n+1},a_{n+1}\sqrt{a_{n+1}})$ starts in the middle of the interval $ (a_n,a_n\sqrt{a_n})$)

Choose then a sequence $ b_1 = 2$ and $ b_{n + 1} = \left(\frac {b_n + b_n\sqrt {b_n}}2\right)^{\frac 23}$ and use the previous method to show with induction that $ f(x) = \frac {d'}x$ $ \forall x\in(1,2\sqrt 2)$
(using the fact that the next interval $ (b_{n+1},b_{n+1}\sqrt{b_{n+1}})$ ends in the middle of the interval $ (b_n,b_n\sqrt{b_n})$)


Hence the result.
\end{solution}



\begin{solution}[by \href{https://artofproblemsolving.com/community/user/76594}{mhmhm}]
	thank you :)
\end{solution}
*******************************************************************************
-------------------------------------------------------------------------------

\begin{problem}[Posted by \href{https://artofproblemsolving.com/community/user/37447}{mr.danh}]
	Find all functions $ f: \mathbb R\to \mathbb R$ such that $ f(x)f(y)=f(x+y)+xy$ for all $ x,y\in \mathbb R$.
	\flushright \href{https://artofproblemsolving.com/community/c6h329072}{(Link to AoPS)}
\end{problem}



\begin{solution}[by \href{https://artofproblemsolving.com/community/user/45762}{FelixD}]
	\begin{tcolorbox}Find all functions $ f: \mathbb R\to \mathbb R$ such that $ f(x)f(y) = f(x + y) + xy$ for all $ x,y\in \mathbb R$\end{tcolorbox}
Let $ P(x,y)$ be the assertion $ f(x)f(y)=f(x+y)+xy$. $ P(x,0)$ gives $ f(x)f(0)=f(x)$. If $ f(x) \equiv 0$ for all $ x$, we obtain a contradiction. Hence there exists an $ x$ such that $ f(x) \ne 0$, which yields $ f(0)=1$. 
$ P(x,-x)$ gives $ f(x)f(-x)=1-x^2$. Plugging in $ x=1$, we get $ f(1)f(-1)=0$, so at least one of $ f(1)$, $ f(-1)$ is equal to $ 0$.
Suppose we have $ f(1)=0$. $ P(x,1)$ gives $ 0=f(x+1)+x$ for all $ x$ and hence $ f(x)=1-x$, which is indeed a solution.
Suppose we have $ f(-1)=0$. $ P(x,-1)$ gives $ 0=f(x-1)-x$ and hence $ f(x)=x+1$, which is a solution too.
\end{solution}



\begin{solution}[by \href{https://artofproblemsolving.com/community/user/29428}{pco}]
	\begin{tcolorbox}Find all functions $ f: \mathbb R\to \mathbb R$ such that $ f(x)f(y) = f(x + y) + xy$ for all $ x,y\in \mathbb R$\end{tcolorbox}

Let $ P(x,y)$ be the assertion $ f(x)f(y) = f(x + y) + xy$

$ P(x,0)$ $ \implies$ $ f(x)f(0)=f(x)$ and so :

If $ f(0)\ne 1$, we got $ f(x)=0$ $ \forall x$ which is not a solution.
So $ f(0)=1$ and then $ P(1,-1)$ $ \implies$ $ f(1)f(-1)=0$ and so :

1) either $ f(1)=0$ and $ P(x-1,1)$ $ \implies$ $ f(x)=1-x$ which, indeed, is a solution.
2) either $ f(-1)=0$ and $ P(x+1,-1)$ $ \implies$ $ f(x)=x+1$ which, indeed, is a solution.

Hence the two solutions :
$ f(x)=1-x$ $ \forall x$
$ f(x)=x+1$ $ \forall x$

\begin{bolded}edited \end{bolded}\end{underlined}: too late. Congrats, FelixD  
\end{solution}



\begin{solution}[by \href{https://artofproblemsolving.com/community/user/45762}{FelixD}]
	Thanks pco :) ^^
Actually both solutions are identical :D
\end{solution}



\begin{solution}[by \href{https://artofproblemsolving.com/community/user/44887}{Mathias_DK}]
	\begin{tcolorbox}Find all functions $ f: \mathbb R\to \mathbb R$ such that $ f(x)f(y) = f(x + y) + xy$ for all $ x,y\in \mathbb R$\end{tcolorbox}
Let $ g(x) = f(x) + 1$.
Then $ Q(x,y) \iff g(x)g(y) + g(x) + g(y) = g(x + y) + xy$ from the original statement.

$ Q(x,0) \iff g(0)(g(x) + 1) = 0$. So either $ g(0) = 0$ or $ g(x) = - 1 \forall x$. $ g(x) = - 1 \forall x$ is not a solution, so $ g(0) = 0$.

Let $ g(1) = a$
$ Q(1,1) \iff g(2) = a^2 + 2a - 1$
$ Q(2,1) \iff g(3) = a^3 + 3a^2 + 2a - 3$
$ Q(2,2) \iff g(4) = a^4 + 4a^3 + 4a^2 - 5$
$ Q(3,1) \iff g(4) = a^4 + 4a^3 + 5a^2 - 6$
Hence $ g(4) = g(4) \iff a^4 + 4a^3 + 4a^2 - 5 = a^4 + 4a^3 + 5a^2 - 6 \iff a^2 = 1$. So $ a = 1$ or $ a = - 1$.
That is $ g(1) = 1$ or $ g(1) = - 1$

Case $ g(1) = 1$: 
$ Q(x,1) \iff g(x + 1) = 2g(x) + 1 - x$, and hence $ g(1 - x) = 2g( - x) + x + 1$
$ Q(x,1 - x) \iff g(x)g(1 - x) + g(x) + g(1 - x) - x(1 - x) = g(1) = 1 \iff$
$ 2g(x)g( - x) + (x + 2)g(x) + 2g( - x) + x^2 = 0$
$ Q(x, - x) \iff g(x)g( - x) + g(x) + g( - x) + x^2 = 0$. Using that in $ Q(x,1 - x)$ gives:
$ xg(x) = x^2 \iff g(x) = x$.
So $ g(x) = x$ which gives $ f(x) = x + 1$. It is seen to be a a solution.

Case $ g(1) = - 1$:
$ Q(x,1) \iff g(x + 1) = - (x + 1)$, which gives $ g(x) = - x \forall x$ and hence $ f(x) = 1 - x \forall x$, which is also seen to be a solution.

So $ f(x) = x + 1 \forall x$ or $ f(x) = - x + 1 \forall x$
\end{solution}
*******************************************************************************
-------------------------------------------------------------------------------

\begin{problem}[Posted by \href{https://artofproblemsolving.com/community/user/68025}{Pirkuliyev Rovsen}]
	Find all functions $ f: \mathbb{R}\to\mathbb{R}$ such that 
\[ f(x^{5}-y^{5})=x^{2}f(x^{3})-y^{2}f(y^{3})\]
 for all $ x,y \in \mathbb{R}$.
	\flushright \href{https://artofproblemsolving.com/community/c6h329261}{(Link to AoPS)}
\end{problem}



\begin{solution}[by \href{https://artofproblemsolving.com/community/user/29428}{pco}]
	\begin{tcolorbox}Find all function $ f: \mathbb{R}\to\mathbb{R}$ such that $ f(x^{5} - y^{5}) = x^{2}f(x^{3}) - y^{2}f(y^{3})$ for all $ x,y \in \mathbb{R}$.\end{tcolorbox}

Let $ P(x,y)$ be the assertion $ f(x^5-y^5)=x^2f(x^3)-y^2f(y^3)$

$ P(x,0)$ $ \implies$ $ f(x^5)=x^2f(x^3)$ and so $ f(x^5-y^5)=f(x^5)-f(y^5)$ and so $ f(x-y)=f(x)-f(y)$ and so $ f(x+y)=f(x)+f(y)$

So $ f(x)=ax$ $ \forall x\in\mathbb Q$ and $ f(px)=pf(x)$ $ \forall x,\forall p\in\mathbb Q$

Let then $ y\in\mathbb Q$ :  $ P(x+y,0)$ $ \implies$ $ f(x^5+5x^4y+10x^3y^2+10x^2y^3+5xy^4+y^5)$ $ =(x^2+2xy+y^2)f(x^3+3x^2y+3xy^2+y^3)$

$ \implies$ $ f(x^5)+5yf(x^4)+10y^2f(x^3)+10y^3f(x^2)+5y^4f(x)+ay^5$ $ =(x^2+2xy+y^2)(f(x^3)+3yf(x^2)+3y^2f(x)+ay^3)$

$ \implies$ $ 2y^4(f(x)-ax)+y^3(7f(x^2)-6xf(x)-ax^2)$ $ +y^2(9f(x^3)-6xf(x^2)-3x^2f(x))+$ $ y(5f(x^4)-2xf(x^3)-3x^2f(x^2))+f(x^5)-x^2f(x^3)=0$ $ \forall x\in\mathbb R$, $ \forall y\in\mathbb Q$

So, when $ x$ is choosen, we have a $ 4^{th}$ degree polynomial in $ y$ which have infinitely many roots (each rational number) and so which must be the null polynomial. This implies all coefficients are $ 0$ and $ f(x)=ax$ $ \forall x\in\mathbb R$

And this, indeed, is a solution.

Hence the unique family of solutions : $ \boxed{f(x)=ax}$ $ \forall x$
\end{solution}
*******************************************************************************
-------------------------------------------------------------------------------

\begin{problem}[Posted by \href{https://artofproblemsolving.com/community/user/68920}{prester}]
	Find all continuous functions $ f: \mathbb{R} \to \mathbb{R}$ such that \[ f(x+y)f(x-y)=(f(x)f(y))^2\] for all $x,y \in \mathbb{R}$.
	\flushright \href{https://artofproblemsolving.com/community/c6h329548}{(Link to AoPS)}
\end{problem}



\begin{solution}[by \href{https://artofproblemsolving.com/community/user/29428}{pco}]
	\begin{tcolorbox}Find all continuous functions $ f: \mathbb{R} \to \mathbb{R}$ such that $ f(x + y)f(x - y) = (f(x)f(y))^2\ x,y \in \mathbb{R}$\end{tcolorbox}

Let $ P(x,y)$ be the assertion $ f(x+y)f(x-y)=f(x)^2f(y)^2$

If $ f(0)^2\ne 1$, $ P(x,0)$ $ \implies$ $ f(x)=0$ $ \forall x$ which indeed is a solution
So we'll from now consider $ f(0)^2=1$
Since $ f(x)$ solution implies $ -f(x)$ solution, WLOG consider $ f(0)=+1$

Suppose $ \exists a$ such that $ f(a)=0$. Then $ P(\frac a2,\frac a2)$ $ \implies$ $ f(a)=f(\frac a2)^4$ and so $ f(\frac a2)=0$
So $ f(\frac a{2^n})=0$ and so, setting $ n\to +\infty$ and using continuity, $ f(0)=0$, contradiction. So $ f(x)\ne 0$ $ \forall x$

$ P(\frac x2,\frac x2)$ $ \implies$ $ f(x)=f(\frac x2)^4$ and so $ f(x)\ge 0$ $ \forall x$. So $ f(x)>0$ $ \forall x$

Let then $ g(x)=\ln(f(x))$ : $ g(x)$ is continuous and $ g(0)=0$ and $ P(x,y)$ implies $ Q(x,y)$ : $ g(x+y)+g(x-y)=2g(x)+2g(y)$

$ Q(x,x)$ $ \implies$ $ g(2x)=4g(x)$
$ Q(2x,x)$ $ \implies$ $ g(3x)=9g(x)$
An immediate induction gives $ g(nx)=n^2g(x)$ and then $ g(px)=p^2g(x)$ $ \forall p\in\mathbb Q^+$
And since $ P(0,x)$ $ \implies$ $ g(-x)=g(x)$, we get $ g(x)=x^2g(1)$ $ \forall x\in\mathbb Q$ 
Continuity gives then $ g(x)=ax^2$ $ \forall x$ which indeed is a solution.

Hence the solutions of initial equation :

$ f(x)=0$ $ \forall x$

$ f(x)=e^{ax^2}$ $ \forall x$

$ f(x)=-e^{ax^2}$ $ \forall x$
\end{solution}



\begin{solution}[by \href{https://artofproblemsolving.com/community/user/46171}{tuandokim}]
	I can't solve the same problem can you help me pco?
The problem doesn't have this condition :f is a continuous
\end{solution}



\begin{solution}[by \href{https://artofproblemsolving.com/community/user/29428}{pco}]
	\begin{tcolorbox}I can't solve the same problem can you help me pco?
The problem doesn't have this condition :f is a continuous\end{tcolorbox}

Are you sure that you encountered this problem in an olympiad contest \/ training without the continuity constraint ?

If you are sure, I'll try searching. But it seems we have a lot of strange solutions and I'm not sure we can get a general solution. Some examples :

1) $ f(x)=e^{ah(x)^2}$ with $ h(x)$ being any solution of Cauchy's equation

2) $ f(x)=\chi_{\mathbb Q}(x)$, the indicator function of $ \mathbb Q$ in $ \mathbb R$

and a lot of other functions.

So, tuandokim, please confirm that you know there is a general solution to this problem, thanks.
\end{solution}



\begin{solution}[by \href{https://artofproblemsolving.com/community/user/46171}{tuandokim}]
	\begin{tcolorbox}[quote="tuandokim"]I can't solve the same problem can you help me pco?
The problem doesn't have this condition :f is a continuous\end{tcolorbox}

Are you sure that you encountered this problem in an olympiad contest \/ training without the continuity constraint ?

If you are sure, I'll try searching. But it seems we have a lot of strange solutions and I'm not sure we can get a general solution. Some examples :

1) $ f(x) = e^{ah(x)^2}$ with $ h(x)$ being any solution of Cauchy's equation

2) $ f(x) = \chi_{\mathbb Q}(x)$, the indicator function of $ \mathbb Q$ in $ \mathbb R$

and a lot of other functions.

So, tuandokim, please confirm that you know there is a general solution to this problem, thanks.\end{tcolorbox}
i'm sorry for my big mistake. 
my friend has just said that his problem was wrong.
 :blush: 
thks
\end{solution}
*******************************************************************************
-------------------------------------------------------------------------------

\begin{problem}[Posted by \href{https://artofproblemsolving.com/community/user/37447}{mr.danh}]
	Suppose a function  $ f: \mathbb R\to \mathbb R$ satisfies $ f(f(x)) = - x$ for all $ x\in \mathbb R$. Prove that $ f$ has infinitely many points of discontinuity.
	\flushright \href{https://artofproblemsolving.com/community/c6h329705}{(Link to AoPS)}
\end{problem}



\begin{solution}[by \href{https://artofproblemsolving.com/community/user/29428}{pco}]
	\begin{tcolorbox}Suppose a function  $ f: \mathbb R\to \mathbb R$ satisfies $ f(f(x)) = - x$ for all $ x\in \mathbb R$. Prove that $ f$ has infinitely many points of discontinuity.\end{tcolorbox}

See http://www.mathlinks.ro/Forum/viewtopic.php?t=263561 (found using the subject description)

or directly http://www.mathlinks.ro/Forum/viewtopic.php?t=113408 (found using keywords)
\end{solution}



\begin{solution}[by \href{https://artofproblemsolving.com/community/user/68641}{R.Maths}]
	But , there are no function satisfying this condition ,
\end{solution}



\begin{solution}[by \href{https://artofproblemsolving.com/community/user/29428}{pco}]
	\begin{tcolorbox}But , there are no function satisfying this condition ,\end{tcolorbox}

There are infinitely many such functions. For example :

Let $ A,B$ two equinumerous subsets of $ \mathbb R^+$ such that $ A\cup B=\mathbb R^+$ and $ A\cap B=\emptyset$ and let $ h(x)$ any bijection from $ A\to B$ (I think you agree there are infinitely many such  $ A,B,h(x)$). 
Define $ f(x)$ as :

$ f(0)=0$
$ \forall x\in A$ : $ f(x)=h(x)$
$ \forall x\in B$ : $ f(x)=-h^{-1}(x)$
$ \forall x$ such that $ -x\in A$ : $ f(x)=-h(-x)$
$ \forall x$ such that $ -x\in B$ : $ f(x)=h^{-1}(-x)$
\end{solution}



\begin{solution}[by \href{https://artofproblemsolving.com/community/user/13}{enescu}]
	see also this [url]http://www.mathlinks.ro/Forum/viewtopic.php?t=49191[\/url]
\end{solution}



\begin{solution}[by \href{https://artofproblemsolving.com/community/user/77688}{noman}]
	can anyone please give me some sugestion about where can i learn the basic  theory of these functional equations??? any links please??? :(
\end{solution}



\begin{solution}[by \href{https://artofproblemsolving.com/community/user/13}{enescu}]
	\begin{tcolorbox}can anyone please give me some sugestion about where can i learn the basic  theory of these functional equations??? any links please??? :(\end{tcolorbox}
If you can read French, see this: [url]http://www.animath.fr\/old\/cours\/eqfonc.pdf[\/url]
\end{solution}



\begin{solution}[by \href{https://artofproblemsolving.com/community/user/77688}{noman}]
	sir thank you very much for appreasiating for really a good note on this subject.
but sir i am sorry i really cant read french.
any other link or name of book such as introduction to functional equation and process to solve them,or any link....
thank sir again.
 :(
\end{solution}



\begin{solution}[by \href{https://artofproblemsolving.com/community/user/13}{enescu}]
	Try this:[url]http://www.imomath.com\/tekstkut\/funeqn_mr.pdf[\/url]
\end{solution}



\begin{solution}[by \href{https://artofproblemsolving.com/community/user/143628}{MANMAID}]
	Let $f$ is discontinuous at $r$.
Now take two sequences $R^{+}=\{r_1,r_2,...\}$ and $R^{-}=\{r'_1,r'_2,...\}$, both converges to $r$ from left and right sides respectively. 
So $f(r)^{+}\neq{f(r)\neq{f(r)^{-}}}$. Let $f(r_n)=\epsilon_1$ & $f(r'_m)=\epsilon_2$, for large value of $n,m$.$f(f(r_n))=-r_n=f(\epsilon_1)$ and $f(f(r'_m))=-r'_m=f(\epsilon_2)$. Now we have $R^{+}\bigcup{\{r\}}\bigcup{R^{-}}$, so $f(f(r)^{-})=f(f(r))=f(f(r)^{+})$ ,but $f(r)^{-}\neq{f(r)\neq{f(r)^{+}}}$, which is a contradiction.This forces in two cases:

\begin{bolded}case 1:\end{bolded}$f$ is continuous in every point. Then $f(c)=-x$ for every points in the interval $c\in{R}$.It implies $c=\pm{x}$. Now take $c=x$,then $f(x)=-x$, then from $f(f(x))=-x$ , we can get $f(x)=x$ for all $x$ , which is a contradiction.The other case is similar. Then $f$ is not continuous. 

\begin{bolded}case 2:\end{bolded}Note that $f$ is odd. Since $f$ is not continuous then $f$ satisfies two different equation and every point satisfies exactly one of these equations. By the equation $f(f(x))=-x$ , we can say that if $f$ satisfies one of these eq. then $f^2$ is another. That gives  $f$ has infinitely many points of discontinuity.
\end{solution}
*******************************************************************************
-------------------------------------------------------------------------------

\begin{problem}[Posted by \href{https://artofproblemsolving.com/community/user/77007}{armon}]
	Find all functions $f: \mathbb R \to \mathbb R$ such that for all reals $x$ and $y$,
\[(x+y)(f(x)-f(y))=(x-y)f(x+y).\]
	\flushright \href{https://artofproblemsolving.com/community/c6h329724}{(Link to AoPS)}
\end{problem}



\begin{solution}[by \href{https://artofproblemsolving.com/community/user/29428}{pco}]
	\begin{tcolorbox}Find all functions f:R--->R such that:
(x+y)(f(x)-f(y))=(x-y)f(x+y)\end{tcolorbox}

Let $ P(x,y)$ be the assertion $ (x+y)(f(x)-f(y)=(x-y)f(x+y)$

$ P(\frac 12,-\frac 12)$ $ \implies$ $ f(0)=0$

If $ g(x)$ is a solution, so is $ f(x)=g(x)-g(1)x$ and so we can look for solutions such that $ f(0)=f(1)=0$

Then $ P(x,1-x)$ $ \implies$ $ f(x)=f(1-x)$.

$ P(\frac{x+y}2,\frac{x-y}2)$ $ \implies$ $ x(f(\frac{x+y}2)-f(\frac{x-y}2))=yf(x)$

$ P(1-\frac{x+y}2,\frac{x-y}2)$ $ \implies$ $ (1-y)(f(\frac{x+y}2)-f(\frac{x-y}2))=(1-x)f(y)$

And so $ y(1-y)f(x)=x(1-x)f(y)$ and so $ f(x)=ax(x-1)$ an so $ g(x)=ax^2+bx$ which indeed is a solution

Hence the answer : $ \boxed{f(x)=ax^2+bx}$ $ \forall x$
\end{solution}



\begin{solution}[by \href{https://artofproblemsolving.com/community/user/243907}{IstekOlympiadTeam}]
	\begin{tcolorbox}If $ g(x)$ is a solution, so is $ f(x)=g(x)-g(1)x$ and so we can look for solutions such that $ f(0)=f(1)=0$\end{tcolorbox} Why?


\end{solution}



\begin{solution}[by \href{https://artofproblemsolving.com/community/user/168801}{joyce_tan}]
	You can just substitutethat equivalence into the original equation and check for consistency, assuming g is a solution. The linear factors cancel out nicely.
\end{solution}



\begin{solution}[by \href{https://artofproblemsolving.com/community/user/343113}{gmail.com}]
	Please see the attachment at https:\/\/artofproblemsolving.com\/community\/c6t31139f6h411461s1_100_functional_equations_problems_with_solutions
\end{solution}
*******************************************************************************
-------------------------------------------------------------------------------

\begin{problem}[Posted by \href{https://artofproblemsolving.com/community/user/77007}{armon}]
	Find all functions $f: \mathbb R \to \mathbb R$ such that for all $x \in \mathbb R$,
\[ f(x)= \max_{y \in \mathbb{R}}(2xy-f(y)).\]
	\flushright \href{https://artofproblemsolving.com/community/c6h329733}{(Link to AoPS)}
\end{problem}



\begin{solution}[by \href{https://artofproblemsolving.com/community/user/29428}{pco}]
	\begin{tcolorbox}functions f:R--->R such that: 
f(x)=max(2xy-f(y))\end{tcolorbox}

Generally, the function $ \max$ has at least two arguments $ \max(u,v)$ What is the meaning here of $ \max(2xy - f(y))$ ? Is it just $ 2xy - f(y)$ ?

You are new on this forum, may I suggest you some hints ? :
- Read the link ampavardi gave you (for example about titles )
- Check carefully your problems before posting (at least two errors in your first 6 problems)
- Avoid double posting, overall with just 10 minutes interval :)
- show to people who gave you some help or solutions that you read their post and are glad of (posting a solution and seeing as answer that the problem is posted a second time seems that you said "shut up, I undestood nothing, maybe doubleposting will bring me some other solutions".
- dont hesitate to use $ \text{\LaTeX}$ (enclose formulas with dollars and look at Latex tutorial (top of page))
- dont hesitate to apologize when you make an error.

Politeness is not inconsistent with the request for assistance :).
\end{solution}



\begin{solution}[by \href{https://artofproblemsolving.com/community/user/77007}{armon}]
	maximum 2xy-f(y)
\end{solution}



\begin{solution}[by \href{https://artofproblemsolving.com/community/user/44887}{Mathias_DK}]
	\begin{tcolorbox}[quote="armon"]functions f:R--->R such that: 
f(x)=max(2xy-f(y))\end{tcolorbox}

Generally, the function $ \max$ has at least two arguments $ \max(u,v)$ What is the meaning here of $ \max(2xy - f(y))$ ? Is it just $ 2xy - f(y)$ ?\end{tcolorbox}
I think what he means is that $ \max{2xy-f(y)}$ is maximal element of $ S_x = \{ 2xy-f(y) \mid y \in \mathbb{R} \}$ but I'm not sure.
\end{solution}



\begin{solution}[by \href{https://artofproblemsolving.com/community/user/29428}{pco}]
	\begin{tcolorbox}[quote="pco"][quote="armon"]functions f:R--->R such that: 
f(x)=max(2xy-f(y))\end{tcolorbox}

Generally, the function $ \max$ has at least two arguments $ \max(u,v)$ What is the meaning here of $ \max(2xy - f(y))$ ? Is it just $ 2xy - f(y)$ ?\end{tcolorbox}
I think what he means is that $ \max{2xy - f(y)}$ is maximal element of $ S_x = \{ 2xy - f(y) \mid y \in \mathbb{R} \}$ but I'm not sure.\end{tcolorbox}
Thanks, Mathias_DK.

So I'll consider that the equation is $ \boxed{f(x)=\max_{y\in\mathbb R}(2xy-f(y))}$

From $ f(x)+f(y)\ge 2xy$ $ \forall x,y$, we get, using $ y=x$ : $ f(x)\ge x^2$

Let $ u\in\mathbb R$ and let $ y_n$ a sequence of reals such that $ \lim_{n\to+\infty}(2uy_n-f(y_n))=f(u)$ 

$ f(u)+f(y_n)-2uy_n\ge u^2+y_n^2-2uy_n=(u-y_n)^2$
Setting $ n\to+\infty$ in $ f(u)+f(y_n)-2uy_n\ge (u-y_n)^2\ge 0$, we get that $ \lim_{n\to+\infty}y_n= u$ (since $ LHS\to 0$)

So $ \lim_{n\to+\infty}(f(u)+f(y_n)-2uy_n)=0$ $ \implies$ $ \lim_{n\to+\infty} f(y_n)=2u^2-f(u)$

So $ \lim_{n\to+\infty} (f(y_n)-y_n^2)=u^2-f(u)$

But $ f(y_n)-y_n^2\ge 0$ and so $ u^2-f(u)\ge 0$

And, since we already got $ f(u)\ge u^2$, we can deduce $ f(u)=u^2$ which, indeed, is a solution.

Hence the answer : $ \boxed{f(x)=x^2}$ $ \forall x$

Thanks again, Mathias_DK
\end{solution}



\begin{solution}[by \href{https://artofproblemsolving.com/community/user/44887}{Mathias_DK}]
	Can't we just say: $ f(x) \ge x^2$. Let $ f(x) = g(x) + x^2$, $ g(x) \ge 0 \forall x$
$ g(x) + x^2 = \max_{y \in \mathbb{R}} \{2xy - y^2 - g(y) \} \iff$
$ \max_{y \in \mathbb{R}} \{ - (x - y)^2 - g(x) - g(y) \} = 0$
This must be true for any $ x$. If $ g(x) > 0$ it is not possible, so $ g(x) = 0$, and therefore $ f(x) = x^2$.
\end{solution}



\begin{solution}[by \href{https://artofproblemsolving.com/community/user/29428}{pco}]
	\begin{tcolorbox}Can't we just say: $ f(x) \ge x^2$. Let $ f(x) = g(x) + x^2$, $ g(x) \ge 0 \forall x$
$ g(x) + x^2 = \max_{y \in \mathbb{R}} \{2xy - y^2 - g(y) \} \iff$
$ \max_{y \in \mathbb{R}} \{ - (x - y)^2 - g(x) - g(y) \} = 0$
This must be true for any $ x$. If $ g(x) > 0$ it is not possible, so $ g(x) = 0$, and therefore $ f(x) = x^2$.\end{tcolorbox}

Yes we can !     

Much more pretty and simple than mine ! Congrats :)
\end{solution}
*******************************************************************************
-------------------------------------------------------------------------------

\begin{problem}[Posted by \href{https://artofproblemsolving.com/community/user/77007}{armon}]
	Find all monotone functions $f: \mathbb R^+ \to \mathbb R^+$ such that for all positive reals $x$ and $y$,
\[f(xy)f\left(\frac{f(y)}{x}\right)=1.\]
	\flushright \href{https://artofproblemsolving.com/community/c6h329735}{(Link to AoPS)}
\end{problem}



\begin{solution}[by \href{https://artofproblemsolving.com/community/user/67223}{Amir Hossein}]
	@ armon:I think reading [url]http://www.mathlinks.ro/viewtopic.php?t=135914[\/url] would be good for you.
\end{solution}



\begin{solution}[by \href{https://artofproblemsolving.com/community/user/29428}{pco}]
	\begin{tcolorbox}functions f:R+--->R+ such that: 
f(xy)f(f(y)\/x)=1\end{tcolorbox}

Strange problem. Where is it coming from ?

There are infinitely many solutions and I dont know how to describe all of them. Here is just a family of solutions :

Let $ g(x)$ from $ \mathbb R\to\mathbb Q$ any non continuous solution of Cauchy's equation with values in $ \mathbb Q$ such that $ g(x)=-x$ $ \forall x\in\mathbb Q$

Then $ f(x)=e^{g(\ln(x))}$ is a solution to your equation.
\end{solution}



\begin{solution}[by \href{https://artofproblemsolving.com/community/user/77007}{armon}]
	Mathematical Olympiad in Sweden (1990)
\end{solution}



\begin{solution}[by \href{https://artofproblemsolving.com/community/user/29428}{pco}]
	\begin{tcolorbox}Mathematical Olympiad in Sweden (1990)\end{tcolorbox}

I did not find it : olympiad 1990 was in China and Olympiad in Sweden was in 1991 and I found this problem in none.

If you have an original text, could you verify if the problem statement is OK ? Maybe you missed some constraint (for example continuity) ?

Thanks for checking.
\end{solution}



\begin{solution}[by \href{https://artofproblemsolving.com/community/user/77007}{armon}]
	sorry!
f: monotonic function :)
\end{solution}



\begin{solution}[by \href{https://artofproblemsolving.com/community/user/29428}{pco}]
	\begin{tcolorbox}sorry!
f: monotonic function :)\end{tcolorbox}

Ahhhhh ! With this constraint, this problem becomes quite easier :)

Let $ P(x,y)$ be the assertion $ f(xy)f(\frac {f(y)}x) = 1$

1) Suppose first $ \exists a > 0$ such that $ af(a)\ne f(1)$
==================================

Then, comparing $ P(x,1)$ and $ P(\frac xa,a)$, we get $ f(\frac {f(1)}x) = f(\frac {af(a)}{x})$ $ \forall x$

And, since $ af(a)\ne f(1)$ and $ f(x)$ is monotonic, we get that $ f(x)$ is constant over $ [\frac {f(1)}x,\frac {af(a)}x]$ (reverse bounds if necessary)

Then, it is easy to show (choosing successive appropriate values for $ x$) that $ f(x)$ is constant over $ (0, + \infty)$
and so, plugging $ f(x) = c$ in the original equation, we get the solution $ \boxed{f(x) = 1}$

2) suppose then $ xf(x) = f(1)$ $ \forall x$
=========================
Then $ f(x) = \frac {f(1)}x$ and, plugging this in original equation, we get $ f(1) = 1$ and the solution $ \boxed{f(x) = \frac 1x}$
\end{solution}
*******************************************************************************
-------------------------------------------------------------------------------

\begin{problem}[Posted by \href{https://artofproblemsolving.com/community/user/67007}{Axes AM}]
	Find all functions $f: \mathbb R \to \mathbb R$ such that for all reals $x$ and $y$,
\[f(f(x)+y)=f(x^2-y)+4f(x)y.\]
	\flushright \href{https://artofproblemsolving.com/community/c6h329761}{(Link to AoPS)}
\end{problem}



\begin{solution}[by \href{https://artofproblemsolving.com/community/user/29428}{pco}]
	\begin{tcolorbox}Hii !
  
Find all The Functions $ f$ : $ \mathbb{R}$  $ \longrightarrow$  $ \mathbb{R}$ such that :
 
$ f((f(x) + y) = f(x^2 - y) + 4f(x)y$\end{tcolorbox}

Let $ P(x,y)$ be the assertion $ f(f(x) + y) = f(x^2 - y) + 4f(x)y$

$ P(x,\frac{x^2-f(x)}2)$ $ \implies$ $ f(x)(f(x)-x^2)=0$ and so : $ \forall x$ : either $ f(x)=0$, either $ f(x)=x^2$

$ f(x)=0$ is obviously a solution. So let us now consider $ f(x)$ is not all zero. Suppose $ \exists a\ne 0$ such that $ f(a)=0$. we know that $ \exists b\ne 0$ such that $ f(b)=b^2$ :

$ P(a,b)$ $ \implies$ $ b^2=f(a^2-b)$ and since $ b \ne 0$, $ f(a^2-b)=(a^2-b)^2$ and so $ a^2=2b$ So there is at most one such $ b$ and at most two such $ a$, which is impossible.

Hence : either $ f(x)=0$ $ \forall x$, either $ f(x)=x^2$ $ \forall x$ which indeed are solutions.

Hence the answer : Two solutions :
$ f(x)=0$ $ \forall x$
$ f(x)=x^2$ $ \forall x$
\end{solution}
*******************************************************************************
-------------------------------------------------------------------------------

\begin{problem}[Posted by \href{https://artofproblemsolving.com/community/user/67007}{Axes AM}]
	Find all functions $f: \mathbb R \to \mathbb R$ such that for all $x \in \mathbb{R}\setminus (1,-1)$,
\[f\left(\frac{x-3}{x+1}\right) + f\left(\frac{x+3}{1-x} \right) = x.\]
	\flushright \href{https://artofproblemsolving.com/community/c6h329994}{(Link to AoPS)}
\end{problem}



\begin{solution}[by \href{https://artofproblemsolving.com/community/user/29428}{pco}]
	\begin{tcolorbox}Hi !

Find all the functions $ f$ : $ \mathbb{R} \longrightarrow \mathbb{R}$ , Such That , $ \forall x \in \mathbb{R} - (1, - 1)$ :

$ f(\frac {x - 3}{x + 1} ) + f(\frac {x + 3}{1 - x} ) = x$\end{tcolorbox}

I suppose you mean $ \forall x \in \mathbb{R} - \{1, - 1\}$

Let $ g(x) = \frac {x - 3}{x + 1}$. We get $ g(g(x)) = \frac {x + 3}{1 - x}$ and $ g(g(g(x))) = x$

Let $ P(x)$ be the original assertion $ f(g(x)) + f(g(g(x))) = x$

$ P(x)$ $ \implies$ $ f(g(x)) + f(g(g(x)) = x$ $ \forall x\notin\{ - 1, + 1\}$
$ P(g(x))$ $ \implies$ $ f(g(g(x))) + f(x) = g(x)$ $ \forall x$ such that $ g(x)\notin\{ - 1, + 1\}$ so $ x\notin\{1\}$
$ P(g(g(x)))$ $ \implies$ $ f(x) + f(g(x)) = g(g(x))$ $ \forall x$ such that $ g(g(x))\notin\{ - 1, + 1\}$ so $ x\notin\{ - 1\}$

Adding second and third line and then subtracting the first, we get :

$ 2f(x) = g(x) + g(g(x)) - x$ $ \forall x\notin\{ - 1, + 1\}$ and so $ f(x) = \frac {x(x^2 + 7)}{2(1 - x^2)}$ $ \forall x\notin\{ - 1, + 1\}$

It is (rather) easy to check that this indeed is a solution.
It is also easy to check that $ f(1)$ and $ f( - 1)$ can take any values.

Hence the solution :
$ f(1) = a$, $ f( - 1) = b$ and $ \boxed{f(x) = \frac {x(x^2 + 7)}{2(1 - x^2)}}$ $ \forall x\notin\{ - 1, + 1\}$
\end{solution}
*******************************************************************************
-------------------------------------------------------------------------------

\begin{problem}[Posted by \href{https://artofproblemsolving.com/community/user/74705}{shortlist}]
	Find all functions $f: \mathbb R \to \mathbb R$ which satisfy for all reals $x$ and $y$,
\[ f(y^4 + f(x) - x))= (f(y))^4.\]
	\flushright \href{https://artofproblemsolving.com/community/c6h330377}{(Link to AoPS)}
\end{problem}



\begin{solution}[by \href{https://artofproblemsolving.com/community/user/68920}{prester}]
	\begin{tcolorbox}Find all $ f: R - > R$ satisfy
$ f(y^4 + f(x) - x)) = (f(y))^4$\end{tcolorbox}

Setting  $ y=0$ we have $ f(f(x)-x)=f^4(0)\ \forall x \in \mathbb{R}$

So we have either $ (a)\ f(x)-x=const$ or $ (b)\ f(x)=const$ 

$ (a) f(x)=c+x$

By substitution of $ (a)$ in the original equation we have $ c=0$ and $ f(x)=x\ \forall x \in \mathbb{R}$ that is indeed a solution of original equation

$ (b) f(x)=c$

By substitution of $ (b)$ in the original equation we have either $ f(x)=0\ \forall x \in \mathbb{R}$ or $ f(x)=1\ \forall x \in \mathbb{R}$ that are indeed solutions of original equation

Hence the list of solutions is the following:

$ \boxed{f(x)=x\ \forall x \in \mathbb{R}}$

$ \boxed{f(x)=0\ \forall x \in \mathbb{R}}$ 

$ \boxed{f(x)=1\ \forall x \in \mathbb{R}}$
\end{solution}



\begin{solution}[by \href{https://artofproblemsolving.com/community/user/74705}{shortlist}]
	\begin{tcolorbox}[quote="shortlist"]Find all $ f: R - > R$ satisfy
$ f(y^4 + f(x) - x)) = (f(y))^4$\end{tcolorbox}

Setting  $ y = 0$ we have $ f(f(x) - x) = f^4(0)\ \forall x \in \mathbb{R}$

So we have either $ (a)\ f(x) - x = const$ or $ (b)\ f(x) = const$ 

$ (a) f(x) = c + x$


\end{tcolorbox}
I don't understand ? Can you say more?
\end{solution}



\begin{solution}[by \href{https://artofproblemsolving.com/community/user/29428}{pco}]
	\begin{tcolorbox}[quote="shortlist"]Find all $ f: R - > R$ satisfy
$ f(y^4 + f(x) - x)) = (f(y))^4$\end{tcolorbox}

Setting  $ y = 0$ we have $ f(f(x) - x) = f^4(0)\ \forall x \in \mathbb{R}$

So we have either $ (a)\ f(x) - x = const$ or $ (b)\ f(x) = const$ \end{tcolorbox}
You cant take such a conclusion just from  $ f(f(x) - x) = f^4(0)\ \forall x \in \mathbb{R}$. Counter example : $ f(x)=x$ $ \forall x\ge 1$ and $ f(x)=x+1$ $ \forall x<1$
\end{solution}



\begin{solution}[by \href{https://artofproblemsolving.com/community/user/68920}{prester}]
	\begin{tcolorbox}[quote="prester"][quote="shortlist"]Find all $ f: R - > R$ satisfy
$ f(y^4 + f(x) - x)) = (f(y))^4$\end{tcolorbox}

Setting  $ y = 0$ we have $ f(f(x) - x) = f^4(0)\ \forall x \in \mathbb{R}$

So we have either $ (a)\ f(x) - x = const$ or $ (b)\ f(x) = const$ \end{tcolorbox}
You cant take such a conclusion just from  $ f(f(x) - x) = f^4(0)\ \forall x \in \mathbb{R}$. Counter example : $ f(x) = x$ $ \forall x\ge 1$ and $ f(x) = x + 1$ $ \forall x < 1$\end{tcolorbox}

yes, I understand my mistake. We need of continuity for that? But here we don't have...
\end{solution}



\begin{solution}[by \href{https://artofproblemsolving.com/community/user/29428}{pco}]
	\begin{tcolorbox}Find all $ f: R - > R$ satisfy
$ f(y^4 + f(x) - x)) = (f(y))^4$\end{tcolorbox}Let $ P(x,y)$ be the assertion $ f(y^4+f(x)-x)=f(y)^4$

1) If $ f(x)-x$ is not constant, then the only solutions are constant functions $ f(x)=0$ and $ f(x)=1$
===================================================================

1.1) $ \exists T>0$ and $ a$ such that $ f(x)=f(x+T)$ $ \forall x\ge a$
-------------------------------------------------------------------
Let $ u_1$ and $ u_2$ such that $ f(u_1)-u_1>f(u_2)-u_2$ and let $ T=(f(u_1)-u_1)-(f(u_2)-u_2)>0$
Comparing $ P(u_1,x)$ and $ P(u_2,x)$, we get $ f(x^4+f(u_1)-u_1)=f(x^4+f(u_2)-u_2)$
$ \implies$ $ f(x)=f(x+T)$ $ \forall x\ge f(u_2)-u_2$
Q.E.D.

1.2) $ f(x)\ge 0$ $ \forall x$
------------------------------
Let $ T>0$ and $ a$ such that $ f(x)=f(x+T)$ $ \forall x\ge a$
$ f(x)=f(x+kT)$ $ \forall x\ge a$, $ \forall k\in\mathbb N$

$ P(x+kT,y)$ $ \implies$ $ f(y^4+f(x)-x-kT)=f(y)^4\ge 0$ $ \forall x\ge a$, $ \forall k\in\mathbb N$

$ \implies$ $ f(y)\ge 0$ $ \forall y\ge f(x)-x-kT$, $ \forall x\ge a$, $ \forall k\in\mathbb N$

Setting $ k\to +\infty$, we get $ f(x)\ge 0$ $ \forall x$
Q.E.D.

1.3) $ f(-x)=f(x)$ $ \forall x$
-------------------------------
Comparing $ P(0,x)$ and $ P(0,-x)$, we get $ f(x)=\pm f(-x)$
And since $ f(x)\ge 0$, we get immediately $ f(x)=f(-x)$
Q.E.D.

1.4) $ f(x)$ is the constant function
-------------------------------------
Let $ T>0$ and $ a$ such that $ f(x)=f(x+T)$ $ \forall x\ge a$

$ P(x,y)$ $ \implies$ $ f(y^4+f(x)-x)=f(y)^4$
$ P(-x,y)$ $ \implies$ $ f(y^4+f(x)+x)=f(y)^4$

So $ f(y^4+f(x)-x)=f(y^4+f(x)+x)$ and so $ f(z)=f(z+2x)$ $ \forall x$, $ \forall z\ge f(x)-x$

Let $ x\ge 0$
Let $ u\ge a$ such that $ f(u)=f(u+nT)$. 
Let $ k$ such that $ u+kT\ge f(x)-x$
We get $ f(u+kT)=f(u+kT+2x)$ and so $ f(u)=f(u+2x)$ $ \forall u\ge a$, $ \forall x\ge 0$
So $ f(x)=c$ $ \forall x\ge a$

Let then $ x\in\mathbb R$ :
Let $ y$ such that $ y^4\ge 2a-c$ and $ y^4\ge a+x-c$ :

$ P(a,y)$ $ \implies$ $ f(y^4+c-a)=f(y)^4$ and so, since $ y^4+c-a\ge a$ : $ c=f(y)^4$
$ P(y^4+c-x,y)$ $ \implies$ $ f(f(y^4+c-x)+x-c)=f(y)^4=c$ and, since $ y^4+c-x\ge a$, we get $ f(y^4+c-x)=c$ and so $ f(x)=c$
Q.E.D

1.5) The only solutions are $ f(x)=0$ and $ f(x)=1$
---------------------------------------------------
Plugging $ f(x)=c$ in the original equation, we get $ c=c^4$, hence the result


2) If $ f(x)-x$ is constant, then the only solution is $ f(x)=x$
========================================
Just plug $ f(x)=x+a$ in the original equation, and you get : $ y^4+2a=(y+a)^4$ $ \forall y$ and so $ a=0$
Q.E.D.

3) Synthesis of solutions.
================
We got three solutions :
$ f(x)=0$ $ \forall x$
$ f(x)=1$ $ \forall x$
$ f(x)=x$ $ \forall x$
\end{solution}
*******************************************************************************
-------------------------------------------------------------------------------

\begin{problem}[Posted by \href{https://artofproblemsolving.com/community/user/36998}{sandu2508}]
	Let $ p$ be a positive integer. Define the function $ f: \mathbb{N}\to\mathbb{N}$ by $ f(n)=a_1^p+a_2^p+\cdots+a_m^p$, where $ a_1, a_2,\ldots, a_m$ are the decimal digits of $ n$ ($ n=\overline{a_1a_2\ldots a_m}$). Prove that every sequence $ (b_k)^\infty_{k=0}$ of positive integer that satisfy $ b_{k+1}=f(b_k)$ for all $ k\in\mathbb{N}$, has a finite number of distinct terms. $ \mathbb{N}=\{1,2,3\ldots\}$
	\flushright \href{https://artofproblemsolving.com/community/c6h330413}{(Link to AoPS)}
\end{problem}



\begin{solution}[by \href{https://artofproblemsolving.com/community/user/29428}{pco}]
	\begin{tcolorbox}Let $ p$ be a positive integer. Define the function $ f: \mathbb{N}\to\mathbb{N}$ by $ f(n) = a_1^p + a_2^p + \cdots + a_m^p$, where $ a_1, a_2,\ldots, a_m$ are the decimal digits of $ n$ ($ n = \overline{a_1a_2\ldots a_m}$). Prove that every sequence $ (b_k)^\infty_{k = 0}$ of positive integer that satisfy $ b_{k + 1} = f(b_k)$ for all $ k\in\mathbb{N}$, has a finite number of distinct terms. $ \mathbb{N} = \{1,2,3\ldots\}$\end{tcolorbox}

It's easy to see that the equation $ 9^p(1 + \log_{10}(x)) = x$ has two positive roots $ x_1 < x_2$ and that $ 9^p(1 + \log_{10}(x))\le x$ $ \forall x\ge x_2$
Let $ M = \max(b_0,x_2)$

It's easy to show with induction that $ b_n\le M$ $ \forall n\in\mathbb N_0$ :

It's obviously true for $ n = 0$ : $ b_0\le M$ (definition of $ M$).

If $ b_k\le M$, then :
if $ x_2\le b_k\le b_0$, then $ b_{k + 1} = f(b_k)\le 9^p(1 + [\log_{10}(b_k)])\le b_k$ (definition of $ x_2$) and so $ b_{k + 1}\le M$
if $ b_k < x_2$ then $ b_{k + 1} = f(b_k)\le 9^p(1 + [\log_{10}(b_k)])\le 9^p(1 + \log_{10}(x_2)) = x_2 \le M$

Then, the sequence of positive integers has an upper bound and so has a finite number of distinct terms.
Q.E.D.
\end{solution}



\begin{solution}[by \href{https://artofproblemsolving.com/community/user/36998}{sandu2508}]
	Beautiful solution. Thanks you.
\end{solution}
*******************************************************************************
-------------------------------------------------------------------------------

\begin{problem}[Posted by \href{https://artofproblemsolving.com/community/user/61082}{Pain rinnegan}]
	Find all functions $ f: \mathbb{Z}\to\mathbb{R}$ such that
\[ f(m+n-mn)=f(m)+f(n)-f(mn), \quad \forall m,n\in \mathbb{Z}.\]
	\flushright \href{https://artofproblemsolving.com/community/c6h331073}{(Link to AoPS)}
\end{problem}



\begin{solution}[by \href{https://artofproblemsolving.com/community/user/29428}{pco}]
	\begin{tcolorbox}Find all the functions $ f: \mathbb{Z}\rightarrow \mathbb{R}$ such that:
\[ f(m + n - mn) = f(m) + f(n) - f(mn)\ ,\ (\forall)m,n\in \mathbb{Z}\]
\end{tcolorbox}

Here is a rather ugly proof and I hope somebody else will find a prettier one (I'm interested in your own solution Pain rinnegan) : 

Let $ P(x,y)$ be the assertion $ f(x+y-xy)=f(x)+f(y)-f(xy)$

The set $ \mathbb S$ of solutions is clearly a $ \mathbb R$-vector space ($ f(x)$ and $ g(x)$ solutions imply $ af(x)$ and $ f(x)+g(x)$ solutions too).
We'll show that this vector space has a dimension less than or equal to $ 4$
We'll then show four independant solutions, establishing that the dimension is $ 4$ and producing a basis, and so producing all the solutions.

1) knowledge of $ f(x)$ for $ x\in E=\{-1,0,1,2,3,5\}$ allows full knowledge of $ f(x)$ and so dim$ (\mathbb S)\le 6$
=========================================================================

(a) : $ P(-x-1,3)$ $ \implies$ $ f(2x+5)=f(-x-1)+f(3)-f(-3x-3)$
(b) : $ P(x+1,-1)$ $ \implies$ $ f(2x+1)=f(x+1)+f(-1)-f(-x-1)$
(c) : $ P(x+1,-3)$ $ \implies$ $ f(4x+1)=f(x+1)+f(-3)-f(-3x-3)$
(d) : $ P(3,-1)$ $ \implies$ $ f(5)=f(3)+f(-1)-f(-3)$
(e) : $ P(x,-1)$ $ \implies$ $ f(2x-1)=f(x)+f(-1)-f(-x)$
(f) : $ P(-x,-1)$ $ \implies$ $ f(-2x-1)=f(-x)+f(-1)-f(x)$
(g) : $ P(2x+1,-1)$ $ \implies$ $ f(4x+1)=f(2x+1)+f(-1)-f(-2x-1)$

(a)+(b)-(c)-(d)-(e)-(f)+(g) : $ f(2x+5)=f(2x-1)+f(5)-f(-1)$

(h) : $ P(x+2,2)$ $ \implies$ $ f(-x)=f(x+2)+f(2)-f(2x+4)$
(i) : $ P(x,-1)$ $ \implies$ $ f(2x-1)=f(x)+f(-1)-f(-x)$

(h)-(i) : $ f(2x+4)=f(2x-1)+f(x+2)-f(x)+f(2)-f(-1)$

And so we got :
$ f(2x+4)=f(2x-1)+f(x+2)-f(x)+f(2)-f(-1)$ which allows computation of $ f(x)$ for all even $ x\ge 4$ from knowlegge of $ f(E)$
$ f(2x+5)=f(2x-1)+f(5)-f(-1)$ which allows computation of $ f(x)$ for all odd $ x\ge 7$ from knowlegge of $ f(E)$

And since $ P(x,-1)$ $ \implies$ $ f(-x)=-f(2x-1)+f(x)+f(-1)$ and so full knowledge of $ f(x)$ for any $ x\ge -1$ implies full knowledge of $ f(x)$ over $ \mathbb Z$
Q.E.D

2) $ f(-1),f(0),f(1),f(2),f(3)$ and $ f(5)$ are not independant and  dim$ (\mathbb S)\le 4$
========================================================
Let $ Q(x)$ be the assertion $ f(2x+5)=f(2x-1)+f(5)-f(-1)$
Let $ R(x)$ be the assertion $ f(2x+4)=f(2x-1)+f(x+2)-f(x)+f(2)-f(-1)$
both established in the previous paragraph

(a) : $ P(2,-2)$ $ \implies$ $ f(-4)=f(2)+f(-2)-f(4)$
(b) : $ P(2,-1)$ $ \implies$ $ f(3)=f(2)+f(-1)-f(-2)$
(c) : $ P(4,-2)$ $ \implies$ $ f(7)=f(4)+f(-1)-f(-4)$
(d) : $ Q(1)$ $ \implies$ $ f(7)=f(1)+f(5)-f(-1)$
(e) : $ R(0)$ $ \implies$ $ f(4)=2f(2)-f(0)$

(a)+(b)-(c)+(d)-2(e) : $ f(3)=-f(-1)+f(1)+f(5)-2f(2)+2f(0)$ and so $ f(5)=f(3)+f(-1)-f(1)+2f(2)-2f(0)$

(a) : $ P(3,-2)$ $ \implies$ $ f(7)=f(3)+f(-2)-f(-6)$
(b) : $ P(2,-1)$ $ \implies$ $ f(3)=f(2)+f(-1)-f(-2)$
(c) : $ P(6,-1)$ $ \implies$ $ f(11)=f(6)+f(-1)-f(-6)$
(d) : $ Q(3)$ $ \implies$ $ f(11)=2f(5)-f(-1)$
(e) : $ Q(1)$ $ \implies$ $ f(7)=f(1)+f(5)-f(-1)$
(f) : $ R(1)$ $ \implies$ $ f(6)=f(3)+f(2)-f(-1)$

(a)+(b)-(c)+(d)-(e)-(f) : $ f(3)=f(5)-f(1)+f(-1)$ and so $ f(5)=f(3)+f(1)-f(-1)$

Comparing these two expression for $ f(5)$, we get $ f(2)=f(1)-f(-1)+f(0)$

So we got :
$ f(2)=f(1)-f(-1)+f(0)$
$ f(5)=f(3)+f(1)-f(-1)$

And so knowledge of $ f(x)$ for $ x\in\{-1,0,1,3\}$ allows full knowledge of $ f(x)$ and so dim$ (\mathbb S)\le 4$
Q.E.D

3) we can show 4 independant solutions and so dim$ (\mathbb S)= 4$
=============================================

3.1) $ f_1(x)=x$ is a solution
---------------------------
That's trivial

3.2) $ f_2(x)=1$ is a solution
---------------------------
trivial too

3.3) $ f_3(x)=0$ if $ x$ is even and $ f_3(x)=1$ if $ x$ is odd is a solution
---------------------------------------------------------------------
Just test the four possibilities for $ x,y\pmod 2$ (in fact, due to symetry, just 3 tests to do)

3.4) $ f_4(x)=0$ if $ x=0\pmod 3$ and $ f_4(x)=2$ if $ x=1\pmod 3$ and $ f_4(x)=1$ if $ x=2\pmod 3$ is a solution
-----------------------------------------------------------------------------------------------------
Just test the nine possibilities for $ x,y\pmod 3$ (in fact, due to symetry, just 6 tests to do)

3.5) these 4 solutions are independant and dim$ (\mathbb S)= 4$
--------------------------------------------------------------
If $ g(x)=a\cdot f_1(x)+b\cdot f_2(x)+c\cdot f_3(x)+d\cdot f_4(x)=0$ $ \forall x$ :

$ a=0$ else $ g(x)$ is unbounded
$ g(0)=0$ $ \implies$ $ b=0$
$ g(1)=0$ $ \implies$ $ b+c+2d=0$
$ g(2)=0$ $ \implies$ $ b+d=0$
So $ a=b=c=d=0$ and we got four independant solutions in a vector space whose dimension is at most $ 4$
Hence the result

4) synthesis of solutions
=========================
So All solutions are any $ g(x)=ax+b+c\cdot f_3(x)+d\cdot f_4(x)$ where :

$ f_3(x)=0$ if $ x$ is even and $ f_3(x)=1$ if $ x$ is odd
$ f_4(x)=0$ if $ x=0\pmod 3$ and $ f_4(x)=2$ if $ x=1\pmod 3$ and $ f_4(x)=1$ if $ x=2\pmod 3$
\end{solution}



\begin{solution}[by \href{https://artofproblemsolving.com/community/user/29428}{pco}]
	\begin{tcolorbox} 4) synthesis of solutions
=========================
So All solutions are any $ g(x) = ax + b + c\cdot f_3(x) + d\cdot f_4(x)$ where :

$ f_3(x) = 0$ if $ x$ is even and $ f_3(x) = 1$ if $ x$ is odd
$ f_4(x) = 0$ if $ x = 0\pmod 3$ and $ f_4(x) = 2$ if $ x = 1\pmod 3$ and $ f_4(x) = 1$ if $ x = 2\pmod 3$\end{tcolorbox}

Other form of solution :

$ (f_1,f_2,f_3,f_4)$ basis of $ \mathbb S$ $ \implies$ $ (f_1,f_2,f_1+f_3,f_1+f_4)$ is also a basis

And since $ f_1(x)+f_3(x)=2\left\lceil\frac x2\right\rceil$ and $ f_1(x)+f_4(x)=3\left\lceil\frac x3\right\rceil$, we get a nicer general form of solution :

$ \boxed{f(x)=ax+b+c\left\lceil\frac x2\right\rceil+d\left\lceil\frac x3\right\rceil}$
\end{solution}



\begin{solution}[by \href{https://artofproblemsolving.com/community/user/61082}{Pain rinnegan}]
	Since pco said it's interested in it, here is the solution (i hope someone will read it to the end):

Observe that the equation is the same if we change $ f(q)$ with $ f(q) - f(0)$. So we can consider $ f(0) = 0$.
\[ m = - 1\Rightarrow f(2n - 1) = f( - 1) + f(n) - f( - n)\]
Then the function $ g(n) = f(2n - 1) - f( - 1) = f(n) - f( - n)$ is odd.

Now setting $ m = 2m - 1$ and $ n = 2n - 1$, then:
\[ f(4m + 4n - 4mn - 3) = f(2m - 1) + f(2n - 1) - f(4mn - 4m - 4n + 1)\Leftrightarrow\]

\[ \Leftrightarrow g(2m + 2n - 2mn - 1) = g(m) + g(n) - g(2mn - 2m - 2n + 1)\Leftrightarrow\]

\[ \Leftrightarrow g(2mn - 2m - 2n + 1) = g( - 2m - 2n + 2mn + 1) + g(m) + g(n)\ ,\ (\forall)m,n\in \mathbb{Z}\ (*)\]
Put in the above $ m = 0$. Then:
\[ g(2n - 1) = g(n - 1) + g(n)\ ,\ (\forall)n\in \mathbb{Z}\ (**)\]
Now if $ m = - 1$ and using $ (**)$ we can get :
\[ g(3n - 2) = - g(n - 1) - g(2n - 2) - g(1)\]
which after switching $ n$ with $ n + 1$ becomes:
\[ g(2n) = g(3n + 1) - g(n) - g(1)\ ,\ (\forall)n\in \mathbb{Z}\]
Now if we put $ m = 2$ ans switching $ n$ with $ - n$ in $ (*)$ we get:
\[ - g(3n + 1) = - g(2n + 3) + g(2) - g(n)\ ,\ (\forall)n\in \mathbb{Z}\]
which combined with the previous one gives:
\[ g(2n) = g(n + 1) + g(n + 2) - g(1) - g(2)\ ,\ (\forall)n\in \mathbb{Z}\]
Using this results convenable we have:

$ g(3) = g(1) + g(2)$

$ g(4) = 2g(1) + g(2)$

$ g(5) = g(1) + 2g(2)$

$ g(6) = 2g(1) + 2g(2)$

$ g(7) = 3g(1) + 2g(2)$

$ g(8) = 2g(1) + 3g(2)$

$ g(9) = 3g(1) + 3g(2)$

$ g(10) = 4g(1) + 3g(2)$

$ g(11) = 3g(1) + 4g(2)$

$ g(12) = 4g(1) + 4g(2)$

Using the above results we can constuct the following statement:
\[ g(q) = \left(q - 2\left\lfloor \frac {q + 1}{3}\right\rfloor \right)g(1) + \left\lfloor \frac {q + 1}{3}\right\rfloor g(2)\ ,\ (\forall)q\in \mathbb{N}\]
Let's assume that the above statement is true for all $ q < p$ and $ p\ge 12$ is fixed and let's prove it for $ p$. In the case $ p = 2k + 1$ we have:
\[ g(p) = g(2k + 1) = g(k) + g(k + 1) =\]

\[ = \left(2k + 1 - 2\left\lfloor \frac {2k + 2}{3}\right\rfloor \right) g(1) + \left\lfloor \frac {2k + 2}{3}\right\rfloor g(2) =\]

\[ = \left(p - 2\left\lfloor \frac {p + 1}{3}\right\rfloor \right)g(1) + \left\lfloor \frac {p + 1}{3}\right\rfloor g(2)\]
where i used the identity
\[ \left\lfloor \frac {k + 1}{3}\right\rfloor + \left\lfloor \frac {k + 2}{3}\right\rfloor = \left\lfloor \frac {2k + 2}{3}\right\rfloor\]
which can be solved by checking the cases $ k\in \{3a,3a + 1,3a + 2\}\ ,\ a\in \mathbb{N}$.

If $ p = 2k$, then:
\[ g(p) = g(2k) = g(k + 1) + g(k + 2) - g(1) - g(2) =\]

\[ = \left(2k - 2\left\lfloor \frac {2k + 1}{3}\right\rfloor \right)g(1) + \left\lfloor \frac {2k + 1}{3}\right\rfloor g(2)\]
where i used the identity
\[ \left\lfloor \frac {k + 2}{3}\right\rfloor + \left\lfloor \frac {k + 3}{3}\right\rfloor - 1 = \left\lfloor \frac {2k + 1}{3}\right\rfloor\]
which can be proved in the same way as the above one.

Then we have:
\[ g( - n) = - g(n) = - \left(n - 2\left\lfloor \frac {n + 1}{3}\right\rfloor \right)g(1) - \left\lfloor \frac {n + 1}{3}\right\rfloor g(2) =\]

\[ = \left( - n - 2\left\lfloor \frac { - n + 1}{3}\right\rfloor \right)g(1) + \left\lfloor \frac { - n + 1}{3}\right\rfloor g(2)\]
due to the identity:
\[ - \left\lfloor \frac {n + 1}{3}\right\rfloor = \left\lfloor \frac { - n + 1}{3}\right\rfloor\]
So we can say now that:
\[ g(q) = \left(q - 2\left\lfloor \frac {q + 1}{3}\right\rfloor \right)g(1) + \left\lfloor \frac {q + 1}{3}\right\rfloor g(2)\ ,\ (\forall)q\in \mathbb{Z}\]
Then
\[ f(2n - 1) = g(n) + f( - 1) = \frac {(f(1) - f( - 1))(2k - 1)}{2} +\]

\[ + (f(2) - f( - 2) - 2f(1) + 2f( - 1))\left\lfloor \frac {2k - 1 + 3}{6}\right\rfloor + \frac {f(1) + f( - 1)}{2}\]
So
\[ f(q) = aq + b\left\lfloor \frac {q + 3}{6}\right\rfloor + c_1\ ,\ (***)\]
where $ a = \frac {f(1) - f( - 1)}{2},\ b = f(2) - f( - 2) - 2f(1) + 2f( - 1),\ c_1 = \frac {f(1) + f( - 1)}{2}$

Getting back to the initial equation putting $ m = 2$ gives:
\[ f(2n) - f(n) = f(2) - f( - n + 2)\ ,\ (\forall)n\in \mathbb{Z}(***\ *)\]
So using $ (***)$ and $ (***\ *)$, for $ x\ge 3\ ,\ x\in \mathbb{N}$ we have:
\[ f(2^x) - f(2^{x - 1}) = f(2) - f( - 2^{x - 1} + 2) = f(2^{x - 2} + 1) - f(1 - 2^{x - 2}) =\]

\[ = a(2^{x - 2} + 1) + b\left\lfloor \frac {2^{x - 2} + 4}{6}\right\rfloor - a(1 - 2^{x - 2}) - b\left\lfloor\frac {4^x - 2^{x - 2}}{6}\right\rfloor =\]

\[ a2^{x - 1} + bH(x)\]
where
\[ H(x) = \left\lfloor \frac {2^{x - 3} + 2}{3}\right\rfloor - \left\lfloor \frac {2 - 2^{x - 3}}{2}\right\rfloor\]
As $ \frac {2^{2y} - 1}{3}\ ,\ \frac {2^{2y + 1} - 2}{3}\in \mathbb{Z}\ ,\ (\forall)y\in \mathbb{N}$, for even $ x$ we have:
\[ H(x) = \frac {2^{x - 3} + 1}{3} - \frac {2 - 2^{x - 3}}{3} = \frac {2^{x - 2} - 1}{3}\]
and for odd $ x$:
\[ H(x) = \frac {2^{x - 3} + 2}{3} - \frac {1 - 2^{x - 3}}{3} = \frac {2^{x - 2} + 1}{3}\]
Summing the equalities:
\[ f(2^x) - f(2^{x - 1}) = a2^{x - 1} + bH(x)\]

\[ f(2^{x - 1}) - f(2^{x - 2}) = a2^{x - 2} + bH(x - 1)\]

\[ \ldots \ldots \ldots \ldots \ldots \ldots \ldots\]

\[ f(2^3) - f(2^2) = a2^2 + bH(3)\]
we get:
\[ f(2^n) = a2^x + bK(x) + f(4) - 4a\]
where:
\[ K(x) = H(3) + H(4) + ... + H(x)\]
For $ x = 2p,\ p\in \mathbb{N}\ ,\ p\ge 3$ we have:
\[ K(x) = H(3) + H(4) + ... + H(2p) = \frac {2^{2p} - 4}{3} = \frac {2^x - 4}{6} = \left\lfloor \frac {2^x}{6}\right\rfloor\]
Now if we put $ n = 2$ in $ (***\ *)$  we get $ f(4) = 2f(2)$ so:
\[ f(2^x) = a2^x + b\left\lfloor \frac {2^x}{6}\right\rfloor + c_2\ ,\ (\forall)x\ge 2\ (*****)\]
where $ c_2 = 2f(2) - 2f(1) + 2f( - 1)$. Now put in the initial equation $ m = q\in 2\mathbb{Z} + 1\ ,\ n = 2^x$, and using $ (***)$ and $ (*****)$ we can say:
\[ f(2^xq) = f(2^x) + f(q) - f(q + 2^x - 2^xq) = aq + b\left\lfloor \frac {q + 3}{6}\right\rfloor + a2^x + b\left\lfloor \frac {2^x}{6}\right\rfloor - a(q - (q - 1)2^x) - b\left\lfloor \frac {q - (q - 1)2^x + 3}{6}\right\rfloor + c_2 = a2^xq - bL(q,x) + c_2\]
where
\[ L(q,x) = \left\lfloor \frac {q + 3}{6}\right\rfloor + \left\lfloor \frac {2^{x - 1}}{6}\right\rfloor + \left\lfloor \frac {q - (q - 1)2^x + 3}{6}\right\rfloor\]
\begin{bolded}First case:\end{bolded} $ q = 6k + 1\ ,\ k\in \mathbb{Z}$. Then for even $ x$ we have:
\[ L(q,x) = \left\lfloor \frac {2^{x - 1} - 2}{3}\right\rfloor + k2^x = k2^x + \frac {2^{x - 1} - 2}{3} = \frac {2^xq - 4}{6} = \left\lfloor \frac {2^xq}{6}\right\rfloor\]
and for odd $ x$:
\[ L(q,x) = \left\lfloor \frac {2^{x - 1} - 1}{3}\right\rfloor + k2^x = \frac {2^xq - 2}{6} = \left\lfloor \frac {2^xq}{6}\right\rfloor\]
\begin{bolded}Second case:\end{bolded} $ q = 6k + 3\ ,\ k\in \mathbb{Z}$. Then for even $ x$ we have:
\[ L(q,x) = \left\lfloor \frac {2^{x - 1}}{3}\right\rfloor + k2^x - \left\lfloor \frac { - 2^x}{3}\right\rfloor = k2^x + \frac {2^{x - 1} - 2}{3} + \frac {2^x + 2}{3} = \frac {2^xq}{6} = \left\lfloor \frac {2^xq}{6}\right\rfloor\]
and for odd $ x$:
\[ L(q,x) = k2^x + \frac {2^{x - 1} - 2}{3} + \frac {2^x + 2}{3} = \frac {2^xq}{6} = \left\lfloor \frac {2^xq}{6}\right\rfloor\]
\begin{bolded}Third case:\end{bolded} $ q = 6k + 5\ ,\ k\in \mathbb{Z}$. Then for even $ x$:
\[ L(q,x) = \left\lfloor \frac {2^{x - 1}}{3}\right\rfloor + k2^x - \left\lfloor \frac {1 - 2^{x + 1}}{3}\right\rfloor =\]

\[ k2^x + \frac {2^{x - 1} - 2}{3} + \frac {2^{x + 1} + 1}{3} = \frac {2^xq - 2}{6} = \left\lfloor \frac {2^xq}{6}\right\rfloor\]
and for odd $ x$:
\[ L(q,x) = k2^x + \frac {2^{x - 1} - 1}{3} + \frac {2^{x + 1} - 1}{3} = \frac {2^xq - 4}{6} = \left\lfloor \frac {2^xq}{6}\right\rfloor\]
So:
\[ f(q) = aq + b\left\lfloor \frac {q}{6}\right\rfloor + c_2\ ,\ (\forall)k\in 4\mathbb{Z}\ (******)\]
Getting back to the initial equation for $ m = - n$, we get $ f(n^2) + f( - n^2) = f(n) + f( - n)$ and using $ (******)$ :
\[ f(2) + f( - 2) = f(4) + f( - 4) = f(16) + f( - 16) =\]

\[ = a\cdot 16 + b\left\lfloor\frac {16}{6}\right\rfloor + c_2 - 16a + b\left\lfloor\frac {16}{6}\right\rfloor = 2c_2 - b =\]

\[ = - f(2) + f( - 2) + 2f(1) - 2f( - 1) + 4f(2) - 4f(1) + 4f( - 1)\]

\[ \Rightarrow f(2) = f( - 1) - f(1)\Rightarrow c_2 = 0\]
Now we put $ m = 2$ and $ n = q\ ,\ q\in 2\mathbb{Z} + 1$ in the initial equation and using $ (***)$ we have:
\[ f(2q) = f(q) + f(2) - f(2 - q) =\]

\[ = f(2) + aq + b\left\lfloor \frac {q + 3}{6}\right\rfloor + c_1 - a(2 - q) - b\left\lfloor \frac {5 - q}{6}\right\rfloor - c_1 = a\cdot 2q + b\left\lfloor \frac {2q}{6}\right\rfloor\]
I used
\[ \left\lfloor \frac {q + 3}{6}\right\rfloor - \left\lfloor \frac {5 - q}{6}\right\rfloor = \left\lfloor \frac {2q}{6}\right\rfloor\]
which is easy to check for $ q\in \{6a + 1,6a + 3,6a + 5\}\ ,\ a\in \mathbb{Z}$ and also the fact that $ f(2) - 2c_1 = c_2 = 0$. So we have:
\[ f(q) = \left\{\begin{aligned} aq + b\left\lfloor \frac {q + 3}{6}\right\rfloor + c_1\ ,\ \text{if}\ q\in 2\mathbb{Z} + 1 \\
aq + b\left\lfloor \frac {q}{6}\right\rfloor\ ,\ \text{if}\ q\in 2\mathbb{Z}\end{aligned} \right.\]
We have:
\[ q + 3\left(q - 2\left\lfloor \frac {q}{2}\right\rfloor \right) = 4 - 6\left\lfloor \frac {q}{2}\right\rfloor = \left\{\begin{aligned} q + 3\ ,\ \text{if}\ q\in2\mathbb{Z} + 1 \\
q\ ,\ \text{if}\ q\in 2\mathbb{Z}\end{aligned} \right.\]
and then:
\[ \left\lfloor \frac {2q}{3}\right\rfloor - \left\lfloor \frac {q}{2}\right\rfloor = \left\{\begin{aligned} \left\lfloor \frac {q + 3}{6}\right\rfloor,\ \text{if}\ q\in 2\mathbb{Z} + 1 \\
\left\lfloor \frac {q}{6}\right\rfloor\ ,\ \text{if}\ q\in 2\mathbb{Z}\end{aligned} \right.\]
Now if we put $ c = - \frac {c_1}{2}$ and $ c^{\prime} = - c$ we have
\[ ( - 1)^qc + c^{\prime} = \left\{\begin{aligned} c_1\ ,\ \text{if}\ q\in 2\mathbb{Z} + 1 \\
0\ ,\ \text{if}\ q\in 2\mathbb{Z}\end{aligned} \right.\]
Finally putting $ d = c^{\prime} + f(0)$, we obtain the following form for $ f$:
\[ f(q) = aq + b\left(\left\lfloor \frac {2q}{3}\right\rfloor - \left\lfloor \frac {q}{2}\right\rfloor \right) + c( - 1)^q + d\ ,\ \text{where}\ a,b,c,d\in \mathbb{R}\]
The proof ends here. It remains only to verify, which is not hard. I'm sorry for all the eventual typos.
\end{solution}



\begin{solution}[by \href{https://artofproblemsolving.com/community/user/29428}{pco}]
	Wow   

Thanks for posting it (a lot of users dont, even when asked)

i'll read it. Just give me some time :)
\end{solution}



\begin{solution}[by \href{https://artofproblemsolving.com/community/user/29428}{pco}]
	\begin{tcolorbox} Finally putting $ d = c^{\prime} + f(0)$, we obtain the following form for $ f$:
\[ f(q) = aq + b\left(\left\lfloor \frac {2q}{3}\right\rfloor - \left\lfloor \frac {q}{2}\right\rfloor \right) + c( - 1)^q + d\ ,\ \text{where}\ a,b,c,d\in \mathbb{R}\]
\end{tcolorbox}

My first interrogation is to check that we got the same solution. 

You used a basis : $ (g_1,g_2,g_3,g_4)$ where :

$ g_1(q)=q$

$ g_2(q)=\left\lfloor \frac {2q}{3}\right\rfloor - \left\lfloor \frac {q}{2}\right\rfloor$

$ g_3(q)=(-1)^q$

$ g_4(q)=1$

Considering my basis $ (f_1,f_2,f_3,f_4)$ as defined in my post, we get  : 

$ g_1(q)=f_1(q)$

$ g_2(q)=\frac 16f_1(q)+\frac 12f_3(q)-\frac 13f_4(q)$

$ g_3(q)=f_2(q)-2f_3(q)$

$ g_4(q)=f_2(q)$

And so these are two basis of the same vector space.

So we got the same solution.
Good thing  :)
\end{solution}



\begin{solution}[by \href{https://artofproblemsolving.com/community/user/29428}{pco}]
	\begin{tcolorbox}Since pco said it's interested in it, here is the solution (i hope someone will read it to the end):\end{tcolorbox}

I did, but not to the end ... for the moment

\begin{tcolorbox}
Now setting $ m = 2m - 1$ and $ n = 2n - 1$, then:
$ f(4m + 4n - 4mn - 3) = f(2m - 1) + f(2n - 1) - f(4mn - 4m - 4n + 1)$
$ \iff$ $ g(2m + 2n - 2mn - 1) = g(m) + g(n) - g(2mn - 2m - 2n + 1)$
$ \iff$ $ g(2mn - 2m - 2n + 1) = g( - 2m - 2n + 2mn + 1) + g(m) + g(n)$ $ \forall m,n\in \mathbb{Z}$(*)
\end{tcolorbox}
Typo :
$ f(4m + 4n - 4mn - 3) = f(2m - 1) + f(2n - 1) - f(4mn - 2m - 2n + 1)$
$ \iff$ $ g(2m + 2n - 2mn - 1) = g(m) + g(n) - g(2mn - m - n + 1)$
$ \iff$ $ g(2mn - m - n + 1) = g( - 2m - 2n + 2mn + 1) + g(m) + g(n)$ $ \forall m,n\in \mathbb{Z}$(*)

\begin{tcolorbox}
$ g(2mn - 2m - 2n + 1) = g( - 2m - 2n + 2mn + 1) + g(m) + g(n)\ ,\ (\forall)m,n\in \mathbb{Z}$(*)

Put in the above $ m = 0$. Then:
$ g(2n - 1) = g(n - 1) + g(n)\ ,\ (\forall)n\in \mathbb{Z}$(**)
\end{tcolorbox}
Second error : setting $ m=0$ in this equation would lead to $ g(- 2n + 1) = g(- 2n + 1)+ g(n)$
But this second error compensate the first one, since putting $ m=0$ in the good equation $ g(2mn - m - n + 1) = g( - 2m - 2n + 2mn + 1) + g(m) + g(n)$ gives :

$ g(- n + 1) = g( - 2n + 1)  + g(n)$
and so, setting $ n\to -n$ : $ g(2n - 1) = g(n - 1) + g(n)\ ,\ (\forall)n\in \mathbb{Z}$(**)

\begin{tcolorbox}
Now if $ m = - 1$ and using $ (**)$ we can get :

$ g(3n - 2) = - g(n - 1) - g(2n - 2) - g(1)$\end{tcolorbox}

I found, after applying $ m=-1$ :
$ g(3n - 2) = g(n - 1) + g(2n - 2) + g(1)$


\begin{tcolorbox}
which after switching $ n$ with $ n + 1$ becomes:
$ g(2n) = g(3n + 1) - g(n) - g(1)$ $ \forall n\in \mathbb{Z}$\end{tcolorbox}

No, setting $ n\to n+1$ in $ g(3n - 2) = - g(n - 1) - g(2n - 2) - g(1)$, we get $ g(2n)=-g(3n +1) - g(n) - g(1)$

But this second error compensate the first, since setting $ n\to n+1$ in the correct equation $ g(3n - 2) = g(n - 1) + g(2n - 2) + g(1)$ gives :

$ g(2n) = g(3n + 1) - g(n) - g(1)$ :)

\begin{tcolorbox}

Using this results convenable we have:

$ g(3) = g(1) + g(2)$

$ g(4) = 2g(1) + g(2)$

$ g(5) = g(1) + 2g(2)$

$ g(6) = 2g(1) + 2g(2)$

$ g(7) = 3g(1) + 2g(2)$

$ g(8) = 2g(1) + 3g(2)$

$ g(9) = 3g(1) + 3g(2)$

$ g(10) = 4g(1) + 3g(2)$

$ g(11) = 3g(1) + 4g(2)$

$ g(12) = 4g(1) + 4g(2)$\end{tcolorbox}

I do agree but had some difficulties to $ g(4)$ :
We got up to now :
$ P(n)$ : $ g(2n)=g(n+1)+g(n+2)-g(1)-g(2)$
$ Q(n)$ : $ g(2n+1)=g(n)+g(n+1)$
$ R(n)$ : $ g(2n) = g(3n + 1) - g(n) - g(1)$

$ Q(1)$ $ \implies$ $ g(3)=g(1)+g(2)$
$ R(1)$ $ \implies$ $ g(4)=2g(1)+g(2)$
$ Q(2)$ $ \implies$ $ g(5)=g(2)+g(3)=g(1)+2g(2)$

Applying then $ P(n)$ and $ Q(n)$ for $ n=3,4,5,6$, we get the results.


\begin{tcolorbox}
So we can say now that:
$ g(q) = \left(q - 2\left\lfloor \frac {q + 1}{3}\right\rfloor \right)g(1) + \left\lfloor \frac {q + 1}{3}\right\rfloor g
(2)$ $ \forall q\in \mathbb{Z}$

Then

$ f(2n - 1) = g(n) + f( - 1) = \frac {(f(1) - f( - 1))(2k - 1)}{2}$ $ + (f(2) - f( - 2) - 2f(1) +$ $ 2f( - 1))\left\lfloor \frac {2k - 1 + 3}{6}\right\rfloor + \frac {f(1) + f( - 1)}{2}$
\end{tcolorbox}

Here, i'm stuck  :oops: . 

For me, we got $ f(2n - 1) = g(n) + f( - 1) =(n - 2\lfloor \frac {n + 1}{3}\rfloor)g(1) + \lfloor \frac {n + 1}{3}\rfloor g(2) + f(-1)$

And so $ f(2n - 1) =n(f(1)-f(-1)) - 2\lfloor \frac {n + 1}{3}\rfloor(f(1)-f(-1))$ $ + \lfloor \frac {n + 1}{3}\rfloor (f(3)-f(-1)) + f(-1)$

And I dont understand at all what is $ k$ in your equation and where are the $ f(2)$ coming from ....

I stop here ..., sorry.
\end{solution}
*******************************************************************************
-------------------------------------------------------------------------------

\begin{problem}[Posted by \href{https://artofproblemsolving.com/community/user/63660}{Victory.US}]
	Determine all continuous functions $f: \mathbb R \to \mathbb R$ such that for all reals $x$ and $y$,
\[f(x+f(y+f(z)))=f(x)+f(f(y))+f(f(f(z))).\]
	\flushright \href{https://artofproblemsolving.com/community/c6h331303}{(Link to AoPS)}
\end{problem}



\begin{solution}[by \href{https://artofproblemsolving.com/community/user/29428}{pco}]
	\begin{tcolorbox}determine all continuous function $ f: R\to R$ such that :

$ f(x + f(y + f(z))) = f(x) + f(f(y)) + f(f(f(z)))$\end{tcolorbox}

I'm quite sure this problem has already been posted. 
Here is a solution :
Let $ P(x,y,z)$ be the assertion $ f(x + f(y + f(z))) = f(x) + f(f(y)) + f(f(f(z)))$
Let $ E=f(\mathbb R)$

Subtracting $ P(0,y,z)$ from $ P(x,y,z)$, we get $ f(x+f(y+f(z)))=f(x)+f(f(y+f(z)))-f(0)$ $ \forall x,y,z$

Setting $ y=t-f(z)$ in this equality, we get $ f(x+f(t))=f(x)+f(f(t))-f(0)$ $ \forall x,t$

And so $ f(x+y)=f(x)+f(y)-f(0)$ $ \forall x$, $ \forall y\in E$

From there, it's easy to show that :
$ f(x+y+z)=f(x)+f(y+z)-f(0)$ $ \forall x$, $ \forall y,z\in E$
[hide="How?"]
$ f(x+y+z)=f(x+y)+f(z)-f(0)$ since $ z\in E$
$ f(x+y+z)=f(x)+f(y)+f(z)-2f(0)$ since $ y\in E$
$ f(y+z)=f(y)+f(z)-f(0)$ since $ z\in E$
Subtracting these two last lines, we get $ f(x+y+z)=f(x)+f(y+z)-f(0)$ [\/hide]
$ f(x+y-z)=f(x)+f(y-z)-f(0)$ $ \forall x$, $ \forall y,z\in E$
[hide="How?"]
$ f(x-y+y)=f(x-y)+f(y)-f(0)$ since $ y\in E$ and so $ f(x-y)=f(x)-f(y)+f(0)$ $ \forall x$, $ \forall y\in E$
$ f(x+y-z)=f(x+y)-f(z)+f(0)$ since $ z\in E$
$ f(x+y-z)=f(x)+f(y)-f(z)$ since $ y\in E$
$ f(y-z)=f(y)-f(z)+f(0)$ since $ z\in E$
Subtracting these two last lines, we get $ f(x+y-z)=f(x)+f(y-z)-f(0)$ [\/hide]

And, by induction, we get $ f(x+y)=f(x)+f(y)-f(0)$ $ \forall x\in\mathbb R$, $ \forall y$ finite sum or difference of elements of $ E$. Then two cases :

If $ E$ contains a unique element, then plugging $ f(x)=c$ in the original equation, we get $ f(x)=0$ $ \forall x$

If $ E$ contains at least two elements $ a,b$, then it contains $ [a,b]$ (since $ f(x)$ is continuous) and it is easy to show that any real may be obtained as finite sum and difference of elements of $ [a,b]$

So $ f(x+y)=f(x)+f(y)-f(0)$ $ \forall x,y$ and so $ f(x)=ax+b$ (since continuous).

Plugging this in the original equation, we get $ b=0$ or $ a=-2$

Hence the two families of solutions :
$ f(x)=ax$ $ \forall x$ (and $ a=0$ gives the solution $ f(x)=0$
$ f(x)=a-2x$ $ \forall x$
\end{solution}



\begin{solution}[by \href{https://artofproblemsolving.com/community/user/63660}{Victory.US}]
	thank you for your very nice solution,mr Pco  :)
\end{solution}



\begin{solution}[by \href{https://artofproblemsolving.com/community/user/66674}{thuyanh158}]
	\begin{tcolorbox}it is easy to show that any real may be obtained as finite sum and difference of elements of $ [a,b]$\end{tcolorbox}

i don't understand it :blush:  could you please explain clearly  :?:
\end{solution}



\begin{solution}[by \href{https://artofproblemsolving.com/community/user/29428}{pco}]
	\begin{tcolorbox}[quote="pco"]it is easy to show that any real may be obtained as finite sum and difference of elements of $ [a,b]$\end{tcolorbox}

i don't understand it :blush:  could you please explain clearly  :?:\end{tcolorbox}

Let $ x\in\mathbb R$ and $ n=\left\lfloor\frac{x-a}{b-a}\right\rfloor$ so that $ x-a=n(b-a)+r$ with $ r\in[0,b-a)$

Then $ x=nb-na+(r+a)$ with $ r+a\in[a,b)$ and so $ x$ is a finite sum and difference of elements (a,b,r+a) of $ [a,b]$
\end{solution}
*******************************************************************************
-------------------------------------------------------------------------------

\begin{problem}[Posted by \href{https://artofproblemsolving.com/community/user/36998}{sandu2508}]
	Find all function $ f: \mathbb{N}\to\mathbb{N}$ that satisfy the following relation for all $ x, y\in\mathbb{N}$:
\[xf(x) + yf(y) = (x + y)f(x^2 + y^2).\]
	\flushright \href{https://artofproblemsolving.com/community/c6h331331}{(Link to AoPS)}
\end{problem}



\begin{solution}[by \href{https://artofproblemsolving.com/community/user/29428}{pco}]
	\begin{tcolorbox}Find all function $ f: \mathbb{N}\to\mathbb{N}$ that satisfy the following relation for all $ x, y\in\mathbb{N}$
$ xf(x) + yf(y) = (x + y)f(x^2 + y^2)$


$ \mathbb{N} = \{1,2,3\ldots\}$\end{tcolorbox}

Suppose $ \exists x\ne y$ such that $ f(x)\ne f(y)$, WLOG say $ f(x) < f(y)$

$ xf(x) + yf(y) = (x + y)f(x + y)$ $ \implies$ $ f(x) < f(x^2 + y^2) < f(y)$ and so $ 0 < f(x^2 + y^2) - f(x) < f(y) - f(x)$

And so, repeating this process with $ (x,x^2 + y^2)$ instead of $ (x,y)$, this allows us to create a strictly decreasing sequence of positive integers, which is impossible.

So such $ x,y$ do not exist and $ f(x) = f(y)$ $ \forall x,y$ and $ f(x)$ must be constant, which is indeed a solution.

Hence the answer : $ \boxed{f(x) = c}$ $ \forall x$
\end{solution}
*******************************************************************************
-------------------------------------------------------------------------------

\begin{problem}[Posted by \href{https://artofproblemsolving.com/community/user/61082}{Pain rinnegan}]
	Let $a$ and $b$ be given non-zero real numbers and $ f: \mathbb{R}\to\mathbb{R}$ be a function such that 
\[ f(x)=
\begin{cases} ax,& \mbox{if }x\in \mathbb{Q} \\
bx, & \mbox{if } x\in \mathbb{R}\setminus \mathbb{Q}\end{cases}.
\]
Prove that $ f$ is bijective if and only if $a\/b $ is a rational number.
	\flushright \href{https://artofproblemsolving.com/community/c6h331414}{(Link to AoPS)}
\end{problem}



\begin{solution}[by \href{https://artofproblemsolving.com/community/user/29428}{pco}]
	\begin{tcolorbox}Let $ a,b\in \mathbb{R}^*$ and $ f: \mathbb{R}\rightarrow \mathbb{R}$ such that $ f(x) = \{\begin{aligned} ax,\ \text{if}\ x\in \mathbb{Q} \\
bx\ ,\ \text{if}\ x\in \mathbb{R}\backslash \mathbb{Q}\end{aligned} $. Prove that $ f$ is bijective $ \Leftrightarrow \frac {a}{b}\in \mathbb{Q}$.\end{tcolorbox}

For easier proof, let $ p=\frac ab$ and $ g(x)=\frac{f(x)}b$ such that $ g(x)=px$ $ \forall x\in\mathbb Q$ and $ g(x)=x$ $ \forall x\notin\mathbb Q$

Obviously we have $ f(x)$ bijective $ \iff$ $ g(x)$ bijective


$ p\notin \mathbb Q$ $ \implies$ $ 1\ne p$ and $ g(1)=g(p)=p$ and so $ g(x)$ not injective.
So $ g(x)$ injective $ \implies$ $ p\in\mathbb Q$ $ \implies$ $ p\in\mathbb Q^*$ (since $ a\ne 0$)
So $ g(x)$ bijective $ \implies$ $ p\in\mathbb Q^*$

If $ p\in\mathbb Q^*$ and $ g(x)=g(y)$ : 
If $ x\in\mathbb Q$ and $ y\in\mathbb Q$ : $ g(x)=px$ and $ g(y)=py$ and so $ px=py$ and $ x=y$
If $ x\in\mathbb Q$ and $ y\notin\mathbb Q$ : $ g(x)=px$ and $ g(y)=y$ and so $ px=y$, impossible since $ LHS\in\mathbb Q$ while $ RHS\notin\mathbb Q$ 
If $ x\notin\mathbb Q$ and $ y\in\mathbb Q$ : $ g(x)=x$ and $ g(y)=py$ and so $ x=py$, impossible since $ LHS\notin\mathbb Q$ while $ RHS\in\mathbb Q$ 
If $ x\notin\mathbb Q$ and $ y\notin\mathbb Q$ : $ g(x)=x$ and $ g(y)=y$ and so $ x=y$
So $ p\in\mathbb Q^*$ $ \implies$ $ g(x)$ is injective

If $ p\in\mathbb Q^*$ : $ \forall x\notin Q$ : $ g(x)=x$ and $ \forall x\in\mathbb Q$ : $ \frac xp\in\mathbb Q$ and so $ g(\frac xp)=x$
So $ p\in\mathbb Q^*$ $ \implies$ $ g(x)$ is surjective

So $ p\in\mathbb Q^*$ $ \implies$ $ g(x)$ is bijective

And so we proved $ \frac ab\in\mathbb Q^*$ $ \iff$ $ g(x)$ is bijective $ \iff$ $ f(x)$ is bijective.
Q.E.D.
\end{solution}



\begin{solution}[by \href{https://artofproblemsolving.com/community/user/13}{enescu}]
	The solution given by pco can be rephrased as follows:
Assume that $ f$ is a bijection and suppose, by way of contradiction, that $ \frac {a}{b}$ is not rational. But then $ f(1) = a = f\left(\frac {a}{b}\right)$, hence $ 1 = \frac {a}{b}$, contradiction.
Assume that $ \frac {a}{b}$ is rational . We prove that $ f$ is injective. Suppose that $ f(x) = f(y)$. If $ x,y$ are both rational or both irrational, $ x = y$ follows easily. If $ x$ is rational and $ y$ is not, we obtain $ ax = by$, which is impossible, since it implies $ \frac {x}{y} = \frac {b}{a}\in Q$.
We prove that $ f$ is surjective. Pick some real number $ y$. Now, $ \frac {y}{a}$ and $ \frac {y}{b}$ are both rational or both irrational (since their ratio is a rational number). In the first case $ f\left(\frac {y}{a}\right) = y$, while in the second one $ f\left(\frac {y}{b}\right) = y$.
\end{solution}
*******************************************************************************
-------------------------------------------------------------------------------

\begin{problem}[Posted by \href{https://artofproblemsolving.com/community/user/76369}{peter117}]
	Find all functions $f: \mathbb R \to \mathbb R$ which satisfy the following conditions:
1) $ f(x^2)=f^2(x)$ for all $x\in \mathbb R$, and
2) $f(x+1)=f(x)+1$ for all  $x\in \mathbb R$.
	\flushright \href{https://artofproblemsolving.com/community/c6h331537}{(Link to AoPS)}
\end{problem}



\begin{solution}[by \href{https://artofproblemsolving.com/community/user/29428}{pco}]
	\begin{tcolorbox}Find all function $ f: R\to R$ such that
1)$ f(x^2) = f^2(x)$ $ \forall x\in R$
2)$ f(x + 1) = f(x) + 1$ $ \forall x\in R$\end{tcolorbox}

A) $ f(x+n)=f(x)+n$ $ \forall x$, $ \forall n\in\mathbb Z$
====================================
Immediate consequence of 2)

B) $ f(x)=x$ $ \forall x\in\mathbb Q$
=======================
Let $ p\in\mathbb Z$ and $ q\in\mathbb Z^*$
$ f((\frac pq+q)^2)=f(\frac{p^2}{q^2}+2p+q^2)=f(\frac{p^2}{q^2})+2p+q^2$

$ (f(\frac pq+q))^2=(f(\frac pq)+q)^2=(f(\frac pq))^2+2qf(\frac pq)+q^2$

So $ f(\frac{p^2}{q^2})+2p+q^2=(f(\frac pq))^2+2qf(\frac pq)+q^2$

And, since $ f(\frac{p^2}{q^2})=(f(\frac pq))^2$, we get $ 2p=2qf(\frac pq)$ and so $ f(\frac pq)=\frac pq$
Q.E.D.

C) $ f(-x)=-f(x)$ $ \forall x$
============
From 1), we get either $ f(-x)=f(x)$, either $ f(-x)=-f(x)$
Suppose now that $ \exists a$ such that $ f(-a)=f(a)$
Then $ f(-a-1)=f(a-1)$ and, since either $ f(-a-1)=f(a+1)$, either $ f(-a-1)=-f(a+1)$, we get :
If $ f(-a-1)=f(a+1)$ : $ f(a+1)=f(a-1)$ and so $ f(a)-1=f(a)+1$, impossible
If $ f(-a-1)=-f(a+1)$ : $ f(a-1)=-f(a+1)$ and so $ f(a)-1=-f(a)-1$ and so $ f(a)=0$ and so $ f(-a)=f(a)=-f(a)$
Q.E.D.

D) $ [x]\le f(x) \le [x]+1$ $ \forall x$
==================
From 1), we get $ f(x)\ge 0$ $ \forall x\ge 0$
So $ f(x-[x])\ge 0$ and so $ f(x)\ge [x]$
So $ f(-x)\ge [-x]$ $ \implies$ $ -f(x)\ge [-x]\ge -[x]-1$ $ \implies$ $ f(x)\le [x]+1$
Q.E.D.

E) $ f(x)$ is non decreasing
=================
Suppose $ x<y$ and $ f(x)>f(y)$ 
Adding an appropriate integer to both $ x$ and $ y$, we get $ 1<x_1<y_1$ and $ f(x_1)>f(y_1)>1$
Squaring then $ x_1$ and $ y_1$ as much as needed, we get $ 1<x_2<y_2$ and $ f(x_2)>f(y_2)+2$
But then $ [y_2]+1\ge [x_2]+1\ge f(x_2) > f(y_2)+2 \ge [y_2] +2$, which is impossible
Q.E.D

F) $ f(x)=x$ $ \forall x$
=======================
From B) we got $ f(x)=x$ $ \forall x\in\mathbb Q$
From E) we got $ f(x)$ is non decreasing
Hence the result.
\end{solution}



\begin{solution}[by \href{https://artofproblemsolving.com/community/user/62475}{hqthao}]
	this solutiuon of mine may be right  :P 
by induction, we have: $ f(x + n) = f(x) + n,\forall n\in N$
we also have: $ f(x) > 0,\forall x > 0$
we have: $ f{^2}(1 - x) = f{^2}(x - 1) = > f( - x) = - f(x)$
$ = > f(0) = 0 = > f(n) = n, \forall n \in N$
Now, we begin: $ \forall x \in [n;n + 1]:$
$ 0 < = f(x - n) = f(x) - n = > f(x) > = n$
$ 0 > = f(x - (n + 1)) = f(x) - (n + 1) = > f(x) < = n + 1$
$ = |f(x) - x)| < = 1$
change $ x$ by $ x^k = > 1 > = |f((x^k) - x^k| = |f^{k}(x) - x^k| > = \|f(x) - x|*x^{k - 1}$
so we have: $ |f(x) - x| < = \frac {1}{x^{k - 1}}$
put $ k$ to very big,$ = > f(x) - x --> 0$
$ = = > f(x) = x$ 
\end{solution}
*******************************************************************************
-------------------------------------------------------------------------------

\begin{problem}[Posted by \href{https://artofproblemsolving.com/community/user/77636}{Igor1234}]
	Find all functions $f: \mathbb R \to \mathbb R$ such that for all reals $x$ and $y$,
\[ f(x(f(y)) + f(x)) = 2f(x) + xy.\]
	\flushright \href{https://artofproblemsolving.com/community/c6h331555}{(Link to AoPS)}
\end{problem}



\begin{solution}[by \href{https://artofproblemsolving.com/community/user/29428}{pco}]
	\begin{tcolorbox}\begin{bolded}Find all the funtions of\end{bolded} $ F: \mathbb{R}\to\mathbb{R}$ 
\begin{bolded}belonging to the real numbers x,y.\end{bolded}

$ f(x(f(y)) + f(x)) = 2f(x) + xy$\end{tcolorbox}

Let $ P(x,y)$ be the assertion $ f(xf(y)+f(x))=2f(x)+xy$

1) $ f(x)$ is bijective
=============
$ P(1,x-2f(1))$ $ \implies$ $ f(\text{something})=x$ and so $ f(x)$ is surjective
If $ f(y_1)=f(y_2)$, Subtracting $ P(1,y_1)$ from $ P(1,y_2)$ implies $ y_1=y_2$ and so $ f(x)$ is injective.
Q.E.D.

2) $ f(-1)=0$ and $ f(0)=1$ and $ f(1)=2$
==========================
Since $ f(x)$ is bijective, let $ a$ the unique value such that $ f(a)=0$
$ P(a,0)$ $ \implies$ $ f(af(0))=0=f(a)$ and so $ af(0)=a$ (since $ f(x)$ is injective). So either $ a=0$, either $ f(0)=1$ :

If $ a=0$, $ P(x,0)$ $ \implies$ $ f(f(x))=2f(x)$ and so, since $ f(x)$ is surjective, $ f(x)=2x$, which is not a solution.
So $ a\ne 0$ and $ f(0)=1$
Then $ P(-1,-1)$ $ \implies$ $ f(0)=2f(-1)+1$ and so $ f(-1)=0$

Then $ P(0,0)$ $ \implies$ $ f(1)=2$
Q.E.D.

3) $ f(x+4)=f(x)+4$ $ \forall x$
=================
$ P(-1,x)$ $ \implies$ $ f(-f(x))=-x$
$ P(1,-f(x))$ $ \implies$ $ f(2-x)=4-f(x)$
$ P(-f(x-2),-1)$ $ \implies$ $ f(2-x)=4-2x+f(x-2)$

subtracting these two lines, we get $ f(x)+f(x-2)=2x$
And so $ f(x-2)+f(x-4)=2x-4$
Subtracting these two lines, we get the result $ f(x)=f(x-4)+4$
Q.E.D.

4) $ f(x)=x+1$
==========
Let $ x\ne 0$
Since $ f(x)$ is surjective, $ \exists y$ such that $ f(y)=-4\frac{f(\frac x4)}x$

Then $ P(\frac x4,y)$ $ \implies$ $ 1=2f(\frac x4)+\frac {xy}4$

Then $ P(\frac x4,y+4)$ $ \implies$ $ f(x)=2f(\frac x4)+\frac {xy}4+x$ $ =1+x$ $ \forall x\ne 0$ and we know that this is still true for $ x=0$

And it is easy to check back that this indeed is a solution.

Hence the answer : $ \boxed{f(x)=x+1}$ $ \forall x$
\end{solution}



\begin{solution}[by \href{https://artofproblemsolving.com/community/user/46171}{tuandokim}]
	Your solution is nice :D
and I have an other solution,can you check for me pco?

Let x=1 then $ y+2f(1)=f(f(y)+f(1))$

so f is surjective and there is only one $ \alpha\in R$ satisfies:$ f(\alpha)=0$

Let $ x=\alpha$ then $ f(\alpha*f(y))=\alpha*y$              (1)

Let $ y=\alpha$ then $ f(f(x))=2f(x)+x\alpha$

so $ f(y\alpha)=f(f(\alpha*f(y)))=2y\alpha+(\alpha)^2f(y)$                (2)

let y=0 (2) then $ f(0)=f(0)*\alpha^2$

********
Case 1: f(0)=0 then f(f(x))=2f(x) then f(x)=2x (because f(x) is surjective)

which is not true
********
Case 2: $ \alpha^2=1$

let $ y=\alpha$ (1) then $ f(0)=\alpha^2$ then f(0)=1

let $ y=\alpha$ (2) then $ f(\alpha^2)=2\alpha^2$ so f(1)=2

so we get $ f(f(x))=2f(x)-x$ 

then by induction we get $ f(n)=n+1$ for every $ n\in N$
we have

$ f(2f(x)+xy)=f(f(xf(y)+f(x)))=2f(xf(y)+f(x))-xf(y)-f(x)=3f(x)+2xy-xf(y)$          (3)

let $ g(x)=f(x)-x-1$ then from (3) we get 

$ g(2f(x)+xy)=g(x)-xg(y)$ (4)


let x=1 (4) then $ g(y+4)=g(1)-g(y)$ so $ g(y+4)=-g(y)$

let x=2  (4) then $ g(2y+6)=-2g(y)$ so $ g(2y)=-2g(y-3)=2g(y+1)$

let x=3  (4) then $ g(3y+8)=-3g(y)$   (5)    and $ g(3y+5)=-2g(y-1)$

let y=2y  (5) then $ g(6y+8)=-3g(2y)=-6g(y+1)$ but $ g(6y+8)=2g(3y+5)=-6g(y-1)$

so $ g(y-1)=g(y+1)$

so $ g(x)=g(x+4)$

so $ g(x)=0$

and finally we get 
$ \boxed(f(x)=x+1)$
\end{solution}



\begin{solution}[by \href{https://artofproblemsolving.com/community/user/29428}{pco}]
	\begin{tcolorbox}Your solution is nice :D
and I have an other solution,can you check for me pco?\end{tcolorbox}

I do agree with your proof. Here are some little typos (without consequences on the result) :

\begin{tcolorbox}
Let x=1 then $ y + 2f(1) = f(f(y) + f(1))$

so f is surjective and there is only one $ \alpha\in R$ satisfies:$ f(\alpha) = 0$\end{tcolorbox}

Not exactly : you cant say "there is \begin{bolded}only \end{bolded}\end{underlined}one" since you did not show injectivity. You just can say "there is at least one $ \alpha$ ..." and this is enough for the remainder of your proof.

\begin{tcolorbox}Let $ x = \alpha$ then $ f(\alpha*f(y)) = \alpha*y$              (1)

Let $ y = \alpha$ then $ f(f(x)) = 2f(x) + x\alpha$

so $ f(y\alpha) = f(f(\alpha*f(y))) = 2y\alpha + (\alpha)^2f(y)$                (2)

let y=0 (2) then $ f(0) = f(0)*\alpha^2$

********
Case 1: f(0)=0 then f(f(x))=2f(x) then f(x)=2x (because f(x) is surjective)

which is not true
********
Case 2: $ \alpha^2 = 1$

let $ y = \alpha$ (1) then $ f(0) = \alpha^2$ then f(0)=1

let $ y = \alpha$ (2) then $ f(\alpha^2) = 2\alpha^2$ so f(1)=2

so we get $ f(f(x)) = 2f(x) - x$ \end{tcolorbox}
You're right but it would be fine to explain why : $ \alpha^2=1$ $ \implies$ $ \alpha=1$ or $ \alpha=-1$. But since $ f(1)=2$; $ \alpha\ne 1$, hence your equality :)

\begin{tcolorbox}then by induction we get $ f(n) = n + 1$ for every $ n\in N$
we have

$ f(2f(x) + xy) = f(f(xf(y) + f(x))) = 2f(xf(y) + f(x)) - xf(y) - f(x) = 3f(x) + 2xy - xf(y)$          (3)

let $ g(x) = f(x) - x - 1$ then from (3) we get 

$ g(2f(x) + xy) = g(x) - xg(y)$ (4)


let x=1 (4) then $ g(y + 4) = g(1) - g(y)$ so $ g(y + 4) = - g(y)$

let x=2  (4) then $ g(2y + 6) = - 2g(y)$ so $ g(2y) = - 2g(y - 3) = 2g(y + 1)$

let x=3  (4) then $ g(3y + 8) = - 3g(y)$   (5)    and $ g(3y + 5) = - 2g(y - 1)$\end{tcolorbox}

No, the second equality is $ g(3y + 5) = - 3g(y - 1)$. But this error is of poor importance since you only use the left part (5)

\begin{tcolorbox}let y=2y  (5) then $ g(6y + 8) = - 3g(2y) = - 6g(y + 1)$ but $ g(6y + 8) = 2g(3y + 5) = - 6g(y - 1)$

so $ g(y - 1) = g(y + 1)$

so $ g(x) = g(x + 4)$

so $ g(x) = 0$

and finally we get 
$ \boxed{f(x) = x + 1}$\end{tcolorbox}

Seems quite fine for me :)
Congrats.
\end{solution}
*******************************************************************************
-------------------------------------------------------------------------------

\begin{problem}[Posted by \href{https://artofproblemsolving.com/community/user/77226}{wya}]
	Find all functions $ f: (0,1)\to (0,1)$ such that $ f\left(\frac{1}{2}\right)=\frac{1}{2}$, and
\[ \left(f(ab)\right)^{2}=\left(af(b)+f(a)\right)\left(bf(a)+f(b)\right)\]
for all $ a,b\in(0,1)$.
	\flushright \href{https://artofproblemsolving.com/community/c6h331738}{(Link to AoPS)}
\end{problem}



\begin{solution}[by \href{https://artofproblemsolving.com/community/user/29428}{pco}]
	\begin{tcolorbox}Find all functions $ f: (0,1)\rightarrow(0,1)$ such that $ f(\frac {1}{2}) = \frac {1}{2}$, and
$ \left(f(ab)\right)^{2} = \left(af(b) + f(a)\right)\left(bf(a) + f(b)\right)$       for all $ a,b\in(0,1).$

Thanks in advance.\end{tcolorbox}
Here is a strange proof that $ f(x)=1-x$. I'm sure there is simpler but I dont get it.

Let $ P(x,y)$ be the assertion $ f(xy)^2=(xf(y)+f(x))(yf(x)+f(y))$

1) $ f(x)\le 1-x$
=============
$ P(x,x)$ $ \implies$ $ f(x^2)=(x+1)f(x)$ and so $ \frac{f(x^2)}{1-x^2}=\frac{f(x)}{1-x}$ and so $ \frac{f(x^{2^n})}{1-x^{2^n}}=\frac{f(x)}{1-x}$

So, since $ f(x^{2^n})<1$ : $ f(x)<\frac{1-x}{1-x^{2^n}}$ and setting $ n\to +\infty$ gives the result.


2) $ f(x)$ is left continuous
=======================
From 1) above, we get that $ \lim_{x\to 1^-}f(x)=0$

Let then $ 0<y<x<1$ : $ P(x,\frac yx)$ is $ f(y)^2=(xf(\frac yx)+f(x))(\frac yxf(x)+f(\frac yx))$

Setting $ y\to x^-$ in this equalty and using the fact that $ \lim_{x\to 1^-}f(x)=0$, we get the result.

3) $ f(x)=1-x$
=============
Let $ u\in(0,1)$ and $ a\le 1$ such that $ f(u)=a(1-u)$

Let $ E_a=\{x\in(0,1)$ such that $ f(x)=a(1-x)\}$

$ E_a$ is not empty since $ u\in E_a$ and it is easy to establish :
(using $ P(x,y)$) : $ x,y\in E_a$ $ \implies$ $ xy\in E_a$
(using $ P(\sqrt x,\sqrt x)$) : $ x\in E_a$ $ \implies$ $ \sqrt x\in E_a$

And it's easy to conclude from this that $ E_a$ is dense in $ (0,1)$

Then, just build an increasing sequence $ \{x_n\}$ of elements of $ E_a$ whose limit is $ \frac 12$ :
$ \lim_{n\to+\infty}f(x_n)=\lim_{n\to+\infty}a(1-x_n)=\frac a2$

But, since $ f(x)$ is left-continuous, this limit must be $ f(\frac 12)=\frac 12$ and so $ a=1$
So $ f(u)=1-u$ $ \forall u$
Q.E.D

And, since $ f(x)=1-x$ is indeed a solution, we got the answer : $ \boxed{f(x)=1-x}$ $ \forall x\in(0,1)$
\end{solution}



\begin{solution}[by \href{https://artofproblemsolving.com/community/user/77226}{wya}]
	Nice proof, pco. This is my solution.
\begin{bolded}1)\end{bolded}proof: $ f(a)\le1-a$, my proof is the same.
\begin{bolded}2)\end{bolded}proof: $ \forall \frac{1}{2}<a<1 ,f(a)=1-a$
   Consider for each $ a>\frac{1}{2}$, let $ b=\frac{1}{2a}<1\Rightarrow f(ab)=\frac{1}{2}=1-ab;$
$ P(a,b)\Rightarrow(1-ab)^{2}=\left(f(ab)\right)^{2}=\left(af(b)+f(a)\right)\left(bf(a)+f(b)\right)\le\left(a(1-b)+f(a)\right)\left(b(1-a)+(1-b)\right)=(1-ab)\left(f(a)+a-ab\right)$
$ \Rightarrow1-ab\le f(a)+a-ab\Rightarrow 1-a\le f(a)\Rightarrow f(a)=1-a$
\begin{bolded}3)\end{bolded}Consider for each $ \alpha$ such that $ f(\alpha)=1-\alpha\Rightarrow f(\alpha^{2})=(\alpha+1)f(\alpha)=(\alpha+1)(1-\alpha)=1-\alpha^{2}$ (by induction) $ \Rightarrow f(\alpha^{2^{n}})=1-\alpha^{2^{n}}$ for all natural number  $ n.$
\begin{bolded}4)\end{bolded}Consider for each $ 0<a\le\frac{1}{2}$, there exist a natural number $ n$ such that $ a^{\frac{1}{2^{n}}}>\frac{1}{2}\Rightarrow f\left(a^{\frac{1}{2^{n}}}\right)=1-a^{\frac{1}{2^{n}}}\Rightarrow f(a)=f\left(\left(a^{\frac{1}{2^{n}}}\right)^{2^{n}}\right)=1-\left(a^{\frac{1}{2^{n}}}\right)^{2^{n}}=1-a.$
\end{solution}



\begin{solution}[by \href{https://artofproblemsolving.com/community/user/29428}{pco}]
	\begin{tcolorbox}Nice proof, pco. This is my solution.
\begin{bolded}1)\end{bolded}proof: $ f(a)\le1 - a$, my proof is the same.
\begin{bolded}2)\end{bolded}proof: $ \forall \frac {1}{2} < a < 1 ,f(a) = 1 - a$
   Consider for each $ a > \frac {1}{2}$, let $ b = \frac {1}{2a} < 1\Rightarrow f(ab) = \frac {1}{2} = 1 - ab;$
$ P(a,b)\Rightarrow(1 - ab)^{2} = \left(f(ab)\right)^{2} = \left(af(b) + f(a)\right)\left(bf(a) + f(b)\right)\le\left(a(1 - b) + f(a)\right)\left(b(1 - a) + (1 - b)\right) = (1 - ab)\left(f(a) + a - ab\right)$
$ \Rightarrow1 - ab\le f(a) + a - ab\Rightarrow 1 - a\le f(a)\Rightarrow f(a) = 1 - a$
\begin{bolded}3)\end{bolded}Consider for each $ \alpha$ such that $ f(\alpha) = 1 - \alpha\Rightarrow f(\alpha^{2}) = (\alpha + 1)f(\alpha) = (\alpha + 1)(1 - \alpha) = 1 - \alpha^{2}$ (by induction) $ \Rightarrow f(\alpha^{2^{n}}) = 1 - \alpha^{2^{n}}$ for all natural number  $ n.$
\begin{bolded}4)\end{bolded}Consider for each $ 0 < a\le\frac {1}{2}$, there exist a natural number $ n$ such that $ a^{\frac {1}{2^{n}}} > \frac {1}{2}\Rightarrow f\left(a^{\frac {1}{2^{n}}}\right) = 1 - a^{\frac {1}{2^{n}}}\Rightarrow f(a) = f\left(\left(a^{\frac {1}{2^{n}}}\right)^{2^{n}}\right) = 1 - \left(a^{\frac {1}{2^{n}}}\right)^{2^{n}} = 1 - a.$\end{tcolorbox}

That's quite ok for me. Congrats.
I looked for a proof like that but did not succeed finding "$ f(x)\ge 1-x$"
\end{solution}
*******************************************************************************
-------------------------------------------------------------------------------

\begin{problem}[Posted by \href{https://artofproblemsolving.com/community/user/77636}{Igor1234}]
	Find all the functions $f: \mathbb Z \to\mathbb Z$ that satisfy for all $n \in \mathbb Z$
i) $f(f(n))=f(n+1)$, and 
ii) $f(2009n+ 2008)=2009f(n)$.
	\flushright \href{https://artofproblemsolving.com/community/c6h331851}{(Link to AoPS)}
\end{problem}



\begin{solution}[by \href{https://artofproblemsolving.com/community/user/48552}{ocha}]
	The trivial solution is $ f(n) = 0$, so assume now that $ f$ is not constant

\begin{bolded}Poof that $ f$ is injective:\end{bolded}

Assume there exists distinct $ a,b\in \mathbb{Z}$ such that $ f(a) = f(b)$.

$ f(a) = f(b) \Rightarrow f(f(a)) = f(f(b)) \Rightarrow f(a + 1) = f(b + 1)$. Therefore $ f$ is periodic, with period $ p = |a - b|$

Since $ f$ is periodic and is $ \mathbb{Z} \rightarrow \mathbb{Z}$, $ f$ must be bounded. Therefore it attains either a positive maximum or negative minimum. Since both cases are treated the same, assume wlog that $ f$ attain a positive maximum.

Let $ M\in \mathbb{Z}$ be an integer such that $ f(M)$ is maximum. Therefore $ f(2009M + 2008) = 2009f(M) > f(M)$ contradiction. Therefore $ f$ is injective.

Now from ouw conditions $ f(2009f(n)) = f(f(2009n + 2008)) = f(2009(n + 1))$

Hence $ 2009f(n) = 2009(n + 1) \Longrightarrow \boxed{f(n) = n + 1}$\end{bolded}
\end{solution}



\begin{solution}[by \href{https://artofproblemsolving.com/community/user/29428}{pco}]
	\begin{tcolorbox} $ f(a) = f(b) \Rightarrow f(f(a)) = f(f(b)) \Rightarrow f(a + 1) = f(b + 1)$. Therefore $ f$ is periodic, with period $ p = |a - b|$

Since $ f$ is periodic and is $ \mathbb{Z} \rightarrow \mathbb{Z}$, $ f$ must be bounded. \end{tcolorbox}

Wlog say $ a < b$ : you only proved that $ f(x) = f(x + (b - a))$ $ \forall x\ge a$, so $ f(x)$ is periodic for $ x\ge a$ and so bounded for $ x\ge a$

But since $ f$ is defined over $ \mathbb Z$, you cant directly conclude that $ f(x)$ is bounded over $ ( - \infty,a)$

Maybe this may be easily fixed (I did not try at all).

\begin{bolded}edited \end{bolded}\end{underlined}: I think it cant be fixed since here is a solution which is neither constant, neither $ n+1$ $ \forall n$ : $ f(n)=\min(n+1,0)$
\end{solution}



\begin{solution}[by \href{https://artofproblemsolving.com/community/user/29428}{pco}]
	\begin{tcolorbox}\begin{bolded}Find all the functions $ F: \mathbb^{Z}\to\mathbb^{Z}$ that satisfy:

i)f(f(n))=f(n+1) for all n $ \epsilon Z$
ii)f(2009n+ 2008)=2009f(n) for all n $ \epsilon Z$\end{bolded}\end{tcolorbox}

$ f(x)=x+1$ is a solution.
$ f(x)=0$ is a solution

Suppose now that $ \exists a$ such that $ f(a)\ne a+1$ and $ \exists b$ such that $ f(b)\ne 0$

Then, using i) we get that $ f(x)=f(x+T)$, where $ T=|f(a)-a-1|$ $ \forall x>a$
So $ f([a,+\infty))$ is bounded and so $ f(x)=0$ $ \forall x\ge 0$, else, applying $ x\to 2009x+2008$ as much as needed, we'll be out of bounds.
So, using $ f(x)=f(x+T)$ as much as needed, $ f(x)=0$ $ \forall x>a$

This implies $ f(x)=x+1$ $ \forall x<b$ and so $ \exists c$ greatest integer such that $ f(x)=x+1$ $ \forall x<c$
The function is then $ f(x)=x+1$ $ \forall x<c$ and $ f(x)=0$ $ \forall x>c$. Then :

Since $ f(x)=0$ $ \forall x\ge 0$ we get that $ c\le 0$
But, if $ c<0$ and $ f(c)\ne c+1$, we get $ 2009c+2008<c$ and $ f(2009c+2008)=2009f(c)\ne 2009c+2009$, so contradiction

So $ c=0$ and we got the only three solutions (which indeed are solutions) :

$ f(x)=0$ $ \forall x$
$ f(x)=x+1$ $ \forall x$
$ f(x)=\min(x+1,0)$ $ \forall x$
\end{solution}
*******************************************************************************
-------------------------------------------------------------------------------

\begin{problem}[Posted by \href{https://artofproblemsolving.com/community/user/77226}{wya}]
	Find all natural numbers $n$ such that there exists a non-constant function $ f: \mathbb R\to\mathbb R$ which satisfies
\[f\left(yf(x)\right)=f\left(f(x)\right)+f\left(x^{n}(y-1)\right)\]
for all $ x,y\in \mathbb R$.
	\flushright \href{https://artofproblemsolving.com/community/c6h332286}{(Link to AoPS)}
\end{problem}



\begin{solution}[by \href{https://artofproblemsolving.com/community/user/57817}{mathboy2710}]
	Very interesting !!! 
I think we just have to point out some or at least one function for each case of $ n$. So, how about $ f(x)=0$ ????
I guess for all natural numbers $ n$, that function sastifies
\end{solution}



\begin{solution}[by \href{https://artofproblemsolving.com/community/user/29428}{pco}]
	\begin{tcolorbox}Find all natural numbers n such that there exist a function $ f: R\rightarrow R$ such that

for all $ x,y\in R,$       $ f\left(yf(x)\right) = f\left(f(x)\right) + f\left(x^{n}(y - 1)\right).$\end{tcolorbox}

Answer : $ \boxed{\text{any }n\in\mathbb N}$ since, $ \forall n$ : $ f(x) = 0$ always fits.

If you add the constraint "$ f(x)$ non constant", then :

Let $ P(x,y)$ be the assertion $ f(yf(x)) = f(f(x)) + f(x^n(y - 1))$
$ P(x,1)$ $ \implies$ $ f(0) = 0$
let $ a$ such that $ f(a) = 0$ : $ P(a,y)$ $ \implies$ $ f(a^n(y - 1)) = 0$ $ \implies$ $ a = 0$ (else $ f(x) = 0$ $ \forall x$)

Suppose now $ \exists b$ such that $ f(b)\ne b^n$ : $ P(b,\frac {b^n}{b^n - f(b)})$ $ \implies$ $ f(f(b)) = 0$ and so $ f(b) = 0$ and so $ b = 0$, hence contradiction since $ f(0) = 0^n$

So $ f(x) = x^n$ and, plugging this in the equation, we get the answer : $ \boxed{n = 1}$
\end{solution}



\begin{solution}[by \href{https://artofproblemsolving.com/community/user/77226}{wya}]
	Sorry for my mistake. :blush: 
I'm forgot said that "$ f$ is non constant".
My solution is the same to pco.
\end{solution}
*******************************************************************************
-------------------------------------------------------------------------------

\begin{problem}[Posted by \href{https://artofproblemsolving.com/community/user/48364}{cnyd}]
	Let $f: \mathbb R \to \mathbb R$ be a function such that
\[ f(x^{3}+y^{3})=(x+y)(f(x)^{2}-f(x)f(y)+f(y)^{2})\]
holds for all real numbers $x$ and $y$. Prove that $ f(1996x)=1996f(x)$ for all real $x$.
	\flushright \href{https://artofproblemsolving.com/community/c6h332803}{(Link to AoPS)}
\end{problem}



\begin{solution}[by \href{https://artofproblemsolving.com/community/user/29428}{pco}]
	\begin{tcolorbox}$ f: \mathbb{R}\mapsto\mathbb{R}$

$ f(x^{3} + y^{3}) = (x + y)(f(x)^{2} - f(x)f(y) + f(y)^{2})$

$ \implies$  $ f(1996x) = 1996f(x)$\end{tcolorbox}

Let $ P(x,y)$ be the assertion $ f(x^3+y^3)=(x+y)(f(x)^2-f(x)f(y)+f(y)^2)$

$ P(x,0)$ $ \implies$ $ f(x^3)=xf(x)^2$ and from this, we get that $ f(x)$ and $ x$ have same signs

Let $ A=\{x$ such that $ f(xy)=xf(y)$ $ \forall y\}$. 
$ P(0,0)$ $ \implies$ $ f(0)=0$ and so $ 0\in A$

Let $ a\in A$ : 
$ P(x,0)$ $ \implies$ $ f(x^3)=xf(x)^2$
$ P(x,ax)$ $ \implies$ $ f((1+a^3)x^3)=x(1+a)f(x)^2(1-a+a^2)=(1+a^3)xf(x)^2$
And so $ f((1+a^3)x^3)=(1+a^3)f(x^3)$
And so : $ (P1)$ : $ a\in A$ $ \implies$ $ 1+a^3\in A$

Let $ a\ne 0\in A$
$ P(x,0)$ $ \implies$ $ f(x^3)=xf(x)^2$
$ P(\sqrt[3]ax,0)$ $ \implies$ $ f(ax^3)=\sqrt[3]axf(\sqrt[3]ax)^2$ and so $ af(x^3)=\sqrt[3]axf(\sqrt[3]ax)^2$
And so $ axf(x)^2=\sqrt[3]axf(\sqrt[3]ax)^2$
And so $ \sqrt[3]af(x)=f(\sqrt[3]ax)$ $ \forall x\ne 0$ (remember that $ f(x)$ and $ x$ have same signs)
And so $ (P2)$ : $ a\in A$ $ \implies$ $ \sqrt[3]a\in A$

So $ a\in A$ $ \implies$ $ \sqrt[3]a\in A$ $ (P1)$ $ \implies$ $ (1+(\sqrt[3]a)^3)\in A$ $ (P2)$

so $ a\in A$ $ \implies$ $ 1+a\in A$

And, since $ 0\in A$ : $ 1996\in A$
Q.E.D.
\end{solution}
*******************************************************************************
-------------------------------------------------------------------------------

\begin{problem}[Posted by \href{https://artofproblemsolving.com/community/user/78358}{kewen}]
	Given a function $f: \mathbb R \to \mathbb R$ such that $ f(1) > 0$ and \[ f^2(x + y) \ge f^2(x) + 2f(xy) + f^2(y)\] for all $x,y \in \mathbb R$, prove that $ f$ is unbounded.
	\flushright \href{https://artofproblemsolving.com/community/c6h333395}{(Link to AoPS)}
\end{problem}



\begin{solution}[by \href{https://artofproblemsolving.com/community/user/29428}{pco}]
	\begin{tcolorbox}Given function $ f: R \to R$ such that :$ f(1) > 0$ and $ f^2(x + y) \ge f^2(x) + 2f(xy) + f^2(y)$
Prove that $ f$ is unbounded\end{tcolorbox}

Using $ y=\frac 1x$ in the above inequality, we get $ f^2(x+\frac 1x)\ge f^2(x)+2f(1)+f^2(\frac 1x)\ge f^2(x)+2f(1)$

Let then the sequence $ x_1=1$ and $ x_{n+1}=x_n+\frac 1{x_n}$. We got $ f^2(x_{n+1})\ge f^2(x_n)+2f(1)$ and so $ f^2(x_{n+1})\ge 2nf(1)+f^2(x_1)$

Hence the result.
\end{solution}



\begin{solution}[by \href{https://artofproblemsolving.com/community/user/68025}{Pirkuliyev Rovsen}]
	Let us assume  $ x_{1}\neq 0$ and $ y_{1}=\frac{1}{x_{1}}$ take .Then  $ f^{2}(x_{1}+y_{1})\ge f^{2}(x_{1})+2f^{2}(x_{1})+2f(1)+f^{2}(y_{1})\ge f^{2}(x_{1})+a$ 
here $ a=2f(1)>0$ $ x_{n}=x_{n-1}+y_{n-1}$ , $ y_{n}=\frac{1}{x_{n}}$,  $ n\ge 2$ the must take.Then  $ f^{2}(x_{n}+y_{n})\ge f^{2}(x_{n})+a =f^{2}(x_{n-1}+y_{n-1})+a\ge f^{2}(x_{n-1})+2a\ge ...\ge f^{2}(x_{1})+na$
It is clear that $ f(x_{1})$ ,$ f(x_{2})... f (x_{n})$ unbounded  :!:
\end{solution}



\begin{solution}[by \href{https://artofproblemsolving.com/community/user/29428}{pco}]
	\begin{tcolorbox}Let us assume  $ x_{1}\neq 0$ and $ y_{1} = \frac {1}{x_{1}}$ take .Then  $ f^{2}(x_{1} + y_{1})\ge f^{2}(x_{1}) + 2f^{2}(x_{1}) + 2f(1) + f^{2}(y_{1})\ge f^{2}(x_{1}) + a$ 
here $ a = 2f(1) > 0$ $ x_{n} = x_{n - 1} + y_{n - 1}$ , $ y_{n} = \frac {1}{x_{n}}$,  $ n\ge 2$ the must take.Then  $ f^{2}(x_{n} + y_{n})\ge f^{2}(x_{n}) + a = f^{2}(x_{n - 1} + y_{n - 1}) + a\ge f^{2}(x_{n - 1}) + 2a\ge ...\ge f^{2}(x_{1}) + na$
It is clear that $ f(x_{1})$ ,$ f(x_{2})... f (x_{n})$ unbounded  :!:\end{tcolorbox}

nice proof, indeed.
Congrats.
\end{solution}



\begin{solution}[by \href{https://artofproblemsolving.com/community/user/68025}{Pirkuliyev Rovsen}]
	Thanks you :)
\end{solution}
*******************************************************************************
-------------------------------------------------------------------------------

\begin{problem}[Posted by \href{https://artofproblemsolving.com/community/user/60032}{Stephen}]
	Find all functions $ f: \mathbb{N}\to\mathbb{N}$ that satisfy \[ \frac{1}{f(1)f(2)}+\frac{1}{f(2)f(3)}+\cdots+\frac{1}{f(n)f(n+1)}=\frac{f(f(n)}{f(n+1)}\] for all positive intergers $ n$.
	\flushright \href{https://artofproblemsolving.com/community/c6h333996}{(Link to AoPS)}
\end{problem}



\begin{solution}[by \href{https://artofproblemsolving.com/community/user/62475}{hqthao}]
	Noone solve for this nice function equation. I just see that $ f(1)=1$, if we can find that $ f(2)=2$ then by induction, $ f(n)=n$. but to now, I still cannot find $ f(2)$, any idea, the owner  
\end{solution}



\begin{solution}[by \href{https://artofproblemsolving.com/community/user/29428}{pco}]
	I would be interested too, Stephen, in your solution.  I searched and did not find anything serious.
\end{solution}



\begin{solution}[by \href{https://artofproblemsolving.com/community/user/60032}{Stephen}]
	Well, we can find out $ f(n+1)>f(f(n)$ by induction. It is well thown that if $ f(n+1)>f(f(n)$, then $ f(n)=n$.
\end{solution}



\begin{solution}[by \href{https://artofproblemsolving.com/community/user/29428}{pco}]
	\begin{tcolorbox}Well, we can find out $ f(n + 1) > f(f(n)$ by induction. \end{tcolorbox}

Could you show us how ?
\end{solution}



\begin{solution}[by \href{https://artofproblemsolving.com/community/user/60032}{Stephen}]
	Of course. :) 

Since $ \frac {1}{f(1)f(2)} + \frac {1}{f(2)f(3)} + ... + \frac {1}{f(n-1)f(n)} = \frac {f(f(n - 1))}{f(n)}$, 

$ \frac {f(f(n - 1))}{f(n)} + \frac {1}{f(n)f(n + 1)} = \frac {f(f(n))}{f(n + 1)}$.

So $ f(f(n - 1))f(n + 1) + 1 = f(f(n))f(n)$.

Now let's use induction.

If n=1, we can easily show that $ f(f(1)) = 1$, and $ f(2)\ge 2$.

If $ f(k) > f(f(k - 1))$, then $ f(k)\ge f(f(k - 1)) + 1$.

So $ f(f(k - 1))f(k + 1) + 1 = f(f(k))f(k) \ge f(f(k))(f(f(k - 1)) + 1) > f(f(k))f(f(k - 1)) + 1$.

So $ f(k + 1) > f(f(k))$.

The proof is complete. :)
\end{solution}



\begin{solution}[by \href{https://artofproblemsolving.com/community/user/29428}{pco}]
	\begin{tcolorbox}So $ f(f(k - 1))f(k + 1) + 1 =$ $ f(f(k))f(k) \ge f(f(k))(f(f(k - 1)) + 1)$ $ > f(f(k))f(f(k - 1)) + 1$.

\end{tcolorbox}

How do you show, for the last $ >$, that $ f(f(k)) > 1$ ?
\end{solution}



\begin{solution}[by \href{https://artofproblemsolving.com/community/user/60032}{Stephen}]
	We can show that $ f(n)=1$ if and only if $ n=1$.
\end{solution}



\begin{solution}[by \href{https://artofproblemsolving.com/community/user/29428}{pco}]
	\begin{tcolorbox}We can show that $ f(n) = 1$ if and only if $ n = 1$.\end{tcolorbox}

You're right.  :blush: 

Sorry for my stupid questions.

Congrats for the solution. I really looked for it a rather long time without any success.
And thanks for your patient replies.
\end{solution}



\begin{solution}[by \href{https://artofproblemsolving.com/community/user/62475}{hqthao}]
	Can you post a detail solution for me, please  :oops: I easy to prove that $ f(n+1)>f(k+1)$ by parallel induction. but to that, I still cannot have $ f(n)=n$.
\end{solution}



\begin{solution}[by \href{https://artofproblemsolving.com/community/user/62475}{hqthao}]
	Noone answered my question  :maybe: but, now I can solve it by myself. I don't know this problem posted before?. that:
$ f: N\rightarrow N$
$ f(n + 1) > f(f(n))$ 
from this, we have: $ f(n) = n$.
really nice, huh  
\end{solution}



\begin{solution}[by \href{https://artofproblemsolving.com/community/user/55721}{Thjch Ph4 Trjnh}]
	IMO 1977.
http://www.mathlinks.ro/Forum/viewtopic.php?t=75980
\end{solution}
*******************************************************************************
-------------------------------------------------------------------------------

\begin{problem}[Posted by \href{https://artofproblemsolving.com/community/user/35882}{marsupilami}]
	Let $ a>\frac{3}{4}$ be a real number. Find all functions $ f$ defined on the set of real numbers such that
\[ f(f(x))+a=x^2\]
holds true for all $ x \in \mathbb R$.
	\flushright \href{https://artofproblemsolving.com/community/c6h334093}{(Link to AoPS)}
\end{problem}



\begin{solution}[by \href{https://artofproblemsolving.com/community/user/44753}{gilcu3}]
	Try with the fixed points of $ f(f(x))$ and $ f(f(f(f(x))))$, and you will obtein that there is no function.
\end{solution}



\begin{solution}[by \href{https://artofproblemsolving.com/community/user/29428}{pco}]
	\begin{tcolorbox}Try with the fixed points of $ f(f(x))$ and $ f(f(f(f(x))))$, and you will obtein that there is no function.\end{tcolorbox}

Sorry, but I dont understand your solution   :blush: 

Could you explain a bit more, please ?
Thanks.
\end{solution}



\begin{solution}[by \href{https://artofproblemsolving.com/community/user/44753}{gilcu3}]
	Let $ g(x)=f(f(x))$ Then $ g(x)=x^2-a$ and $ g(g(x))=x^2-2ax^2+a^2-a$
$ g(x)$ has only two fixed  points ($ x_1$ and $ x_2$) and $ g(g(x))$ has exactly four fixed points ($ x_1$, $ x_2$, $ x_3$, $ x_4$).It is easy to see that all are distinct.
$ x_1=\frac{1+\sqrt{1+4a}}{2}$, $ x_2=\frac{1-\sqrt{1+4a}}{2}$
$ x_3=\frac{-1+\sqrt{4a-3}}{2}$, $ x_2=\frac{-1-\sqrt{4a-3}}{2}$
All are real because $ a>\frac{3}{4}$
Let $ f(x_1)=y_1$, $ f(x_2)=y_2$, $ f(x_3)=y_3$, $ f(x_4)=y_4$
Its obvious that $ {x_1,x_2}= {y_1,y_2}$ and $ {x_1,x_2,x_3,x_4}={y_1,y_2,y_3,y_4}$

1-  If $ y_3=x_1$ we have $ x_3=g(g(x_3))=f(x_1)={x_1,x_2}$ which is impossible.
2-  If $ y_3=x_2$ we see the same thing.
3-  If $ y_3=x_3$ we have $ g(x_3)=x_3$ which is false.

4-  If $ y_3=x_4$  If $ y_4={x_1,x_2,x_4}$ we have the same that the preceding three cases.
                        Then $ y_4=x_3$ and so we have $ g(x_3)=x_3$ which is false.
Then there is no function.
\end{solution}



\begin{solution}[by \href{https://artofproblemsolving.com/community/user/29428}{pco}]
	Quite clear and simple !
Thanks and congrats :)
\end{solution}



\begin{solution}[by \href{https://artofproblemsolving.com/community/user/70365}{Maharjun}]
	hello, im a little new to a few of the topics you have used. could you please explain what a fixed point refers to? i cant understand.

thnks in advance
\end{solution}



\begin{solution}[by \href{https://artofproblemsolving.com/community/user/29428}{pco}]
	\begin{tcolorbox}hello, im a little new to a few of the topics you have used. could you please explain what a fixed point refers to? i cant understand.

thnks in advance\end{tcolorbox}

Fixed point of $ f(x)$ is any solution of $ f(x)=x$
\end{solution}
*******************************************************************************
-------------------------------------------------------------------------------

\begin{problem}[Posted by \href{https://artofproblemsolving.com/community/user/74705}{shortlist}]
	1. Find all surjective functions $f: \mathbb R \to \mathbb R$ such that for all $ x,y \in \mathbb R$,
\[f(f(x - y)) = f(x) - f(y).\]

2. Find all functions $f: \mathbb R \to \mathbb R$ satisfying the following relation for all $ x,y \in \mathbb R$:
\[ f(x - f(y)) = 4f(x) - (y) - 3x.\]

3. Find all functions $ f: [0,1] \to [0,1]$ such that for all $x, y \in [0,1]$, we have
\[ f(xy) = xf(x) + yf(y).\]

4. Find all functions $ f: (1, \infty) \to \mathbb R$ such that for all $x>1$ and $y>1$,
\[ f(x) - f(y) = (y - x)f(xy).\]
	\flushright \href{https://artofproblemsolving.com/community/c6h334392}{(Link to AoPS)}
\end{problem}



\begin{solution}[by \href{https://artofproblemsolving.com/community/user/29428}{pco}]
	\begin{tcolorbox}1 find all fruction satisfy:$ f: R - - > R$
i) $ f(f(x - y)) = f(x) - f(y)$ for all $ x,y \in R$
ii) All $ x \in R$ there exist $ y$ such that $ x = f(y)$ \end{tcolorbox}
Let $ P(x,y)$ be the assertion $ f(f(x - y)) = f(x) - f(y)$

Let $ x,y\in\mathbb R$. Let $ u_x$ a real such that $ f(u_x) = x$

$ P(x,x - u_x)$ $ \implies$ $ f(x) = f(x) - f(x - u_x)$ and so $ f(x - u_x) = 0$
$ P(x,u_x)$ $ \implies$ $ f(0) = f(x) - x$ and so $ f(x) = x + a$
Plugging this in the original equation, we get $ a = 0$ and the unique solution $ \boxed{f(x) = x}$

\begin{tcolorbox}2 Find all fruction satisfy:f: R-->R
$ f(x - f(y)) = 4f(x) - (y) - 3x$ for all $ x,y \in R$\end{tcolorbox}
Let $ P(x,y)$ be the assertion $ f(x - f(y)) = 4f(x) - y - 3x$
Let $ a = 4f(0)$
$ P(0,a)$ $ \implies$ $ f( - f(a)) = 0$
$ P(x, - f(a))$ $ \implies$ $ f(x) = 4f(x) + f(a) - 3x$ and so $ f(x) = x + u$
Plugging this in the original equation, we get $ u = 0$ and the unique solution $ \boxed{f(x) = x}$

\begin{tcolorbox}3)Find all fruction $ f: [0,1] - - > [0,1]$
$ f(xy) = xf(x) + yf(y)$ for all $ x,y \in R$\end{tcolorbox}
I suppose you mean $ f(xy) = xf(x) + yf(y)$ $ \forall x,y\in[0,1]$ (and not $ \forall x,y\in\mathbb R$)
Let $ P(x,y)$ be the assertion $ f(xy) = xf(x) + yf(y)$
$ P(0,0)$ $ \implies$ $ f(0) = 0$
$ P(x,0)$ $ \implies$ $ 0 = xf(x)$ and so the unique solution $ \boxed{f(x) = 0}$

\begin{tcolorbox}4) Find all fruction $ f: (1, + \propto ) - - > R$ satisfy:
$ f(x) - f(y) = (y - x)f(xy)$ for all $ x,y > 1$\end{tcolorbox}

Let $ P(x,y)$ be the assertion $ f(x) - f(y) = (y - x)f(xy)$
Adding $ P(\sqrt {2x},\sqrt {\frac x2})$ and $ P(\sqrt {\frac x2},\sqrt {\frac 2x})$ and $ P(\sqrt {\frac 2x},\sqrt {2x})$, we get :

$ (\sqrt {\frac x2} - \sqrt {2x})f(x) +$ $ (\sqrt {\frac 2x} - \sqrt {\frac x2})f(1)$ $ + (\sqrt {2x} - \sqrt {\frac 2x})f(2) = 0$

And so $ f(x) = a + \frac bx$. 
Plugging this in the original equation, we get $ a = 0$ and the unique family of solutions $ \boxed{f(x) = \frac bx}$
\end{solution}



\begin{solution}[by \href{https://artofproblemsolving.com/community/user/29428}{pco}]
	\begin{tcolorbox} [quote]4) Find all fruction $ f: (1, + \propto ) - - > R$ satisfy:
$ f(x) - f(y) = (y - x)f(xy)$ for all $ x,y > 1$\end{tcolorbox}

Let $ P(x,y)$ be the assertion $ f(x) - f(y) = (y - x)f(xy)$
Adding $ P(\sqrt {2x},\sqrt {\frac x2})$ and $ P(\sqrt {\frac x2},\sqrt {\frac 2x})$ and $ P(\sqrt {\frac 2x},\sqrt {2x})$, we get :

$ (\sqrt {\frac x2} - \sqrt {2x})f(x) +$ $ (\sqrt {\frac 2x} - \sqrt {\frac x2})f(1)$ $ + (\sqrt {2x} - \sqrt {\frac 2x})f(2) = 0$

And so $ f(x) = a + \frac bx$. 
Plugging this in the original equation, we get $ a = 0$ and the unique family of solutions $ \boxed{f(x) = \frac bx}$\end{tcolorbox}
Sorry, but, as all of you have certainly seen, this proof is wrong because the equation is true only for $ x>1$ and not $ x>0$
So some modification must be done :

Let $ a>b>1$
Let $ ab>x>\frac ab > 1$

$ \sqrt{\frac {xb}a} >1$ and $ \sqrt{\frac {xa}b}>1$ and so $ P(\sqrt{\frac {xb}a},\sqrt{\frac {xa}b})$ $ \implies$ $ f(\sqrt{\frac {xb}a}) - f(\sqrt{\frac {xa}b}) = (\sqrt{\frac {xa}b} - \sqrt{\frac {xb}a})f(x)$

$ \sqrt{\frac {xa}b} >1$ and $ \sqrt{\frac{ab}x} >1$ and so $ P(\sqrt{\frac {xa}b} ,\sqrt{\frac{ab}x})$ $ \implies$ $ f(\sqrt{\frac {xa}b} )-f(\sqrt{\frac{ab}x})=(\sqrt{\frac{ab}x}-\sqrt{\frac {xa}b})f(a)$

$ \sqrt{\frac{ab}x} >1$ and $ \sqrt{\frac {xb}a} >1$ and so $ P(\sqrt{\frac{ab}x}, \sqrt{\frac {xb}a})$ $ \implies$ $ f(\sqrt{\frac{ab}x})-f(\sqrt{\frac {xb}a} )=(\sqrt{\frac {xb}a} - \sqrt{\frac{ab}x})f(b)$

And then, adding these three lines, we get $ (\sqrt{\frac {xa}b} - \sqrt{\frac {xb}a})f(x)$ $ +(\sqrt{\frac{ab}x}-\sqrt{\frac {xa}b})f(a)$ $ +(\sqrt{\frac {xb}a} - \sqrt{\frac{ab}x})f(b)=0$

And so $ f(x)=u+\frac vx$ and so $ f(x)=\frac vx$ $ \forall x\in(\frac ab,ab)$

And then, we can extend this equality to $ f(x)=\frac vx$ $ \forall x\in(1,+\infty)$ (choosing appropriate $ a,b$)
\end{solution}
*******************************************************************************
-------------------------------------------------------------------------------

\begin{problem}[Posted by \href{https://artofproblemsolving.com/community/user/67949}{aktyw19}]
	Find all functions $f: \mathbb R \to \mathbb R$ such that for all non-zero reals $x$ and $y$,
\[ f(x+y)=x^2f\left(\frac{1}{x}\right)+y^2f\left(\frac{1}{y}\right).\]
	\flushright \href{https://artofproblemsolving.com/community/c6h334624}{(Link to AoPS)}
\end{problem}



\begin{solution}[by \href{https://artofproblemsolving.com/community/user/29428}{pco}]
	\begin{tcolorbox}Find all functions $ f: R - > R$ such that
$ f(x + y) = x^2f(\frac {1}{x}) + y^2f(\frac {1}{y})$ for all x,y\end{tcolorbox}

No solution with your statement since the equation cannot be true for $ x=0$ or $ y=0$

If we replace in your problem statement "for all x,y" by $ \forall x\ne0,y\ne 0$, then here is a solution :

Let $ P(x,y)$ be the assertion $ f(x+y)=x^2f(\frac 1x)+y^2f(\frac 1y)$

1) $ f(0)=0$ and $ f(x)=-f(-x)$
=============================
(a) : $ P(1,1)$ $ \implies$ $ f(2)=2f(1)$
(b) : $ P(-1,-1)$ $ \implies$ $ f(-2)=2f(-1)$
(c) : $ P(1,-1)$ $ \implies$ $ f(0)=f(1)+f(-1)$
(d) : $ P(\frac 12,-\frac 12)$ $ \implies$ $ f(0)=\frac 14(f(2)+f(-2))$

(a)+(b)-2(c)+4(d) : $ f(2)+f(-2)-2f(0)+4f(0)=2f(1)+2f(-1)-2f(1)-2f(-1)+f(2)+f(-2)$ and so $ f(0)=0$

Then $ P(\frac 1x,-\frac 1x)$ $ \implies$ $ f(x)+f(-x)=0$ $ \forall x\ne 0$
And, since $ f(0)=0$, we got $ f(x)+f(-x)=0$ $ \forall x$
Q.E.D.

2) $ f(x+y)=f(x)+f(y)$ $ \forall x$ 
=================================

$ P(x,1)$ $ \implies$ $ f(x+1)=x^2f(\frac 1x)+f(1)$ $ \forall x\ne 0$
$ P(y,-1)$ $ \implies$ $ f(y-1)=f(-1)+y^2f(\frac 1y)$ $ \forall y\ne 0$

Adding these two lines, we get $ f(x+1)+f(y-1)=f(x+y)+f(1)+f(-1)$ and so :

$ f(x+y)=f(x)+f(y)$ $ \forall x\ne 1,y\ne -1$

For symetry reason, we can extend this equality to $ f(x+y)=f(x)+f(y)$ $ \forall (x,y)\in\mathbb R^2\backslash\{(-1,-1),(1,1)\}$

But $ P(1,1)$ shows that $ f(x+y)=f(x)+f(y)$ for $ x=y=1$
And $ P(-1,-1)$ shows that $ f(x+y)=f(x)+f(y)$ for $ x=y=-1$
Q.E.D.

As a consequence, we get that $ f(px)=pf(x)$ $ \forall x\in\mathbb R$, $ \forall p\in\mathbb Q$


3) $ f(x)=x^2f(\frac 1x)$ $ \forall x\ne 0$
==========================================
Using point 2 above, we get $ f(2x)=2f(x)$
And $ P(x,x)$ $ \implies$ $ f(2x)=2x^2f(\frac 1x)$ $ \forall x\ne 0$
Q.E.D.

From there, I suggest to look at inio's solution in http://www.mathlinks.ro/viewtopic.php?p=1787412#1787412


And the result is $ \boxed{f(x)=ax}$ $ \forall x$
\end{solution}



\begin{solution}[by \href{https://artofproblemsolving.com/community/user/67949}{aktyw19}]
	thanks  
\end{solution}
*******************************************************************************
-------------------------------------------------------------------------------

\begin{problem}[Posted by \href{https://artofproblemsolving.com/community/user/67949}{aktyw19}]
	Find all injective functions $f: \mathbb R \to \mathbb R$ such that for all reals $x$ and $y$ with $x \neq y$, we have 
\[ f\left(\frac{x+y}{x-y}\right) = \frac{f(x)+f(y)}{f(x)-f(y)}.\]
	\flushright \href{https://artofproblemsolving.com/community/c6h334677}{(Link to AoPS)}
\end{problem}



\begin{solution}[by \href{https://artofproblemsolving.com/community/user/29428}{pco}]
	\begin{tcolorbox}Find all injective $ f: R\rightarrow R$  such that for all reals x not equal to y, we have $ f\left(\frac {x + y}{x - y}\right) = \frac {f(x) + f(y)}{f(x) - f(y)}$.\end{tcolorbox}

Let $ P(x,y)$ be the assertion $ f\left(\frac{x+y}{x-y}\right)=\frac{f(x)+f(y)}{f(x)-f(y)}$

1) $ f(0)=0$ and $ f(1)=1$ and $ f(-x)=-f(x)$
==================================
$ P(x,0)$ $ \implies$ $ f(1)=\frac{f(x)+f(0)}{f(x)-f(0)}$ and so $ f(0)=0$ else we would have $ f(x)=c$ not injective.
So $ f(1)=1$
And $ P(x,-x)$ implies $ f(x)+f(-x)=0$
Q.E.D.

2) $ f(xy)=f(x)f(y)$ $ \forall x,y$
====================
Let $ x\ne 1$ and $ x,y\ne 0$ :
Comparing $ P(x,1)$ and $ P(xy,y)$, we get $ \frac{f(x)+1}{f(x)-1}$ $ =\frac{f(xy)+f(y)}{f(xy)-f(y)}$ and so $ f(xy)=f(x)f(y)$

And this equality is obviously true if $ x=0$ or $ x=1$ or $ y=0$ 
Q.E.D.

3) $ f(x+y)=f(x)+f(y)$ $ \forall x,y$
=========================
$ f\left(\left(\frac{x+y}{x-y}\right)^2\right)$ $ =\left(f\left(\frac{x+y}{x-y}\right)\right)^2$ $ =\frac{f(x)^2+2f(x)f(y)+f(y)^2}{f(x)^2-2f(x)f(y)+f(y)^2}$

But $ f\left(\left(\frac{x+y}{x-y}\right)^2\right)$ $ =f\left(\frac {(x^2+y^2)+(2xy)}{(x^2+y^2)-(2xy)}\right)$ $ =\frac{f(x^2+y^2)+f(2xy)}{f(x^2+y^2)-f(2xy)}$ $ =\frac{f(x^2+y^2)+f(2)f(x)f(y)}{f(x^2+y^2)-f(2)f(x)f(y)}$

So $ \frac{f(x)^2+2f(x)f(y)+f(y)^2}{f(x)^2-2f(x)f(y)+f(y)^2}$ $ =\frac{f(x^2+y^2)+f(2)f(x)f(y)}{f(x^2+y^2)-f(2)f(x)f(y)}$

And $ f(x^2+y^2)=\frac{f(2)}2(f(x^2)+f(y^2))$ $ \forall x\ne y$ and $ x,y\ne 0$ (using the fact that $ f(x)^2=f(x^2)$)

Using $ x=\sqrt 2$ and $ y=1$ in this equation and comparing with $ P(2,1)$, we get $ f(2)=2$ (remember $ f(-1)=-1$ and so $ f(2)\ne -1$ since $ f(x)$ is injective)

So $ f(x^2+y^2)=f(x^2)+f(y^2)$ $ \forall x\ne y$ and $ x,y\ne 0$ and then it's clear that the equality is true $ \forall x,y$

So $ f(x+y)=f(x)+f(y)$ $ \forall x,y\ge 0$ and, since $ f(-x)=-f(x)$ : $ f(x+y)=f(x)+f(y)$ $ \forall x,y$
Q.E.D

4) $ f(x)=x$ $ \forall x$
=======================
$ f(x+y)=f(x)+f(y)$ and $ f(xy)=f(x)f(y)$ are a classical equation whose solutions are well known :
$ f(x)=0$ $ \forall x$ or $ f(x)=x$ $ \forall x$
And , since $ f(x)$ is injective, we get the required result.

And since obviously $ f(x)=x$ matches the initial requirements, we get the answer : $ \boxed{f(x)=x}$ $ \forall x$
\end{solution}



\begin{solution}[by \href{https://artofproblemsolving.com/community/user/67949}{aktyw19}]
	nice, thanks
\end{solution}
*******************************************************************************
-------------------------------------------------------------------------------

\begin{problem}[Posted by \href{https://artofproblemsolving.com/community/user/28717}{litongyang}]
	Find all injective functions $ f : \mathbb{N}\to\mathbb{N}$  which satisfy
\[ f(f(n))\le\frac{n+f(n)}{2}\] for each $ n\in\mathbb{N}$.
	\flushright \href{https://artofproblemsolving.com/community/c6h334842}{(Link to AoPS)}
\end{problem}



\begin{solution}[by \href{https://artofproblemsolving.com/community/user/29428}{pco}]
	\begin{tcolorbox}Find all injective functions $ f : \mathbb{N}\to\mathbb{N}$  which satisfy
\[ f(f(n))\le\frac {n + f(n)}{2}\]
for each $ n\in\mathbb{N}$\end{tcolorbox}

Suppose $ \exists m$ such that $ f(m)<m$. Then $ f(f(m))\le\frac {m + f(m)}{2}<m$ and a simple induction implies $ f^{[k]}(m)<m$ $ \forall k>0$
So (pigeonhole) $ \exists k_1> k_2$ such that $ f^{[k_1]}(m)=f^{[k_2]}(m)$ and injectivity implies $ f^{[k_1-k_2]}(m)=m$, in contradiction with $ f^{[k]}(m)<m$ $ \forall k>0$.

So $ f(n)\ge n$ $ \forall n$

So $ f(f(n))\ge f(n)$ but $ f(f(n))\le\frac {n + f(n)}{2}\le f(n)$ and so $ f(f(n))=f(n)$ $ \forall n$ and so $ f(n)\le \frac {n + f(n)}{2}$ which implies $ f(n)\le n$

So $ \boxed{f(n)=n}$, which, indeed, is a solution.
\end{solution}



\begin{solution}[by \href{https://artofproblemsolving.com/community/user/72819}{Dijkschneier}]
	Here is an other one (same idea as pco's, but expressed in an other way), though I prefer pco's solution which is more direct.
$f(f(n)) \leq \frac{(n+f(n))}{2}$.
By straight induction, $f^m(n) \leq \frac{(f(n)(2^{m-1}-1)+n)}{2^{m-1}}$
Fix n. Since $\lim _{m \to \infty} \frac{(f(n)(2^{m-1}-1)+n)}{2^{m-1}} = f(n)$, and $f^m(n)$ is an integer, then for all m>N, $f^m(n) \in \{0,1,2,\cdots,f(n)\}$, and so there exists a,b>N such that $f^a(n)=f^{a+b}(n)$, which implies (injectivity) $n=f^b(n)$, and so for all t>0, $f^t(n)=f^{t+b}(n) \in \{0,1,2,\cdots,f(n)\}$. So, for all t>=m, $f^t(n) \in \{0,1,2,\cdots,f^m(n)\}$, and so in particular, for all m>0,$f^{m+1}(n) \leq f^m(n)$(*). Suppose $f(n) \neq n$. (*) and injectivity implies that the sequence of positive integers $(f^m(n))_{m>0}$ is strictly decreasing, which is impossible. Hence f(n)=n as desired.
\end{solution}
*******************************************************************************
-------------------------------------------------------------------------------

\begin{problem}[Posted by \href{https://artofproblemsolving.com/community/user/60032}{Stephen}]
	Find all functions $ f: \mathbb{R}\to\mathbb{R}$ that satisfies \[ f(f(x)+y)=2x+f(f(f(y))-x)\] for all real numbers $ x$ and $y$.
	\flushright \href{https://artofproblemsolving.com/community/c6h335232}{(Link to AoPS)}
\end{problem}



\begin{solution}[by \href{https://artofproblemsolving.com/community/user/62475}{hqthao}]
	first, we have $ f(x)$ bijective.
so put $ x = 0$, we have: $ f(f(0) + y) = f_{3}(y) = > f(0) + x = f(f(x)) (1)$
apply (1) to the origin equation, we have: $ f(f(x) + y) = 2x + f(f(0) + y - x).(2)$
put $ y = 0$ to (2), we have: $ f(f(x)) = 2x + f(f(0) - x) = f(0) + x = > f(0) = x + f(f(0) - x).$
put $ x = f(0)$ to this equation, we have: $ f(0) = 0$. so:$ f(f(x)) = x$
apply this equation to the origin equation, we have: $ f(f(x)+y)=2x+f(y-x)(3)$
change $ x$ by $ f(x)$ to (3), we have $ f(x+y)=2f(x)+f(y-f(x))(4)$
change $ x$ by $ y$ and $ y$ by $ x$ to (4) we have: $ f(x+y)=2f(y)+f(x-f(y))$
compare two line, we have: $ 2f(x)+f(y-f(x))=2f(y)+f(x-f(y))$
change $ x$ by $ f(y)$, we have: $ 2f(x)=2x$

From now, we have: $ f(x)=x$  
\end{solution}



\begin{solution}[by \href{https://artofproblemsolving.com/community/user/29428}{pco}]
	\begin{tcolorbox}first, we have $ f(x)$ bijective.\end{tcolorbox}

Why ?
Neither surjectivity, neither injectivity seem obvious to me   :blush:
\end{solution}



\begin{solution}[by \href{https://artofproblemsolving.com/community/user/62475}{hqthao}]
	So funny, pco, may be you play a trick on me, I learn that $ f$ is injective from you.  
first, change $ y$ by $ - f(x)$ we have that:$ f(0) - 2x = f(something of x)$. so $ f$ is surbjective.
if exist $ y_1$ and $ y_2$ that $ f(y_1) = f(y_2)$ so $ f(f(x) + y_1) = f(f(x) + y_2)$, change $ f(x)$ by $ t$($ t$ belong to R, because $ f$ surbjective  ). we have: $ f(t+y_1)=f(t+y_2)$ from that, we have that $ f$ period with $ T$. ($ T=y_1-y_2$) change $ x$ by $ x + T$ to origin equation, we have that $ x + T = x$, so that $ T = 0$. so $ f$ injective.
 @pco: sorry, your post to prove $ f$ injective I forget the link  :oops:
\end{solution}



\begin{solution}[by \href{https://artofproblemsolving.com/community/user/29428}{pco}]
	\begin{tcolorbox}So funny, pco, may be you play a trick on me, I learn that $ f$ is injective from you.  
first, change $ y$ by $ - f(x)$ we have that:$ f(0) - 2x = f(something of x)$. so $ f$ is surbjective.
if exist $ y_1$ and $ y_2$ that $ f(y_1) = f(y_2)$ so $ f(f(x) + y_1) = f(f(x) + y_2)$, change $ f(x)$ by $ t$($ t$ belong to R, because $ f$ surbjective  ). we have: $ f(t + y_1) = f(t + y_2)$ from that, we have that $ f$ period with $ T$. ($ T = y_1 - y_2$) change $ x$ by $ x + T$ to origin equation, we have that $ x + T = x$, so that $ T = 0$. so $ f$ injective.
 @pco: sorry, your post to prove $ f$ injective I forget the link  :oops:\end{tcolorbox}

:)

I did not say $ f(x)$ was not bijective. 
I said it was not obvious and needed to be proved in some lines.

And I do agree with your lines
\end{solution}



\begin{solution}[by \href{https://artofproblemsolving.com/community/user/30342}{nicetry007}]
	Suppose $ f$ is a bijection.
$ x = y = 0 \Rightarrow f(f(0)) = f(f(f(0))) \Rightarrow 0 = f(0)$.
$ x = 0 \Rightarrow f(y) = f(f(f(y))) \Rightarrow y = f(f(y)) - - - - - - (*)$
$ y = 0 \Rightarrow f(f(x)) = 2x + f( - x) \Rightarrow - x = f( - x)\;\;\;\;(\because f(f(x)) = x )$.
Therefore $ f(x) = x$.
\end{solution}



\begin{solution}[by \href{https://artofproblemsolving.com/community/user/30342}{nicetry007}]
	Easy to see $ f$ is surjective.
Suppose $ f(x) = f(z)$. Pick a $ y$ which satisfies $ f(f(y)) = x + z$. Note that such a $ y$ exists. (Reason : By surjectivity, there exists an $ \alpha$ such that $ f(\alpha) = x + z$ and a $ y$ such that $ f(y) = \alpha$. Combining the two, we get $ f(f(y)) = x + z$.)
$ f(f(x) + y )  = f(f(z) + y ) \Rightarrow 2x + f(f(f(y)) -x) = 2z + f(f(f(y)) - z) \Rightarrow 2x + f(x+ z -x) = 2z + f(x+z-z) \Rightarrow 2x + f(z) = 2z + f(x) \Rightarrow x= z.$
\end{solution}
*******************************************************************************
-------------------------------------------------------------------------------

\begin{problem}[Posted by \href{https://artofproblemsolving.com/community/user/68555}{trbst}]
	Find all functions $ f: \mathbb R\to\mathbb R$ so that $ f(x+y)+f(x-y)=2f(x)\cos y$ for all $x,y \in \mathbb R$.
	\flushright \href{https://artofproblemsolving.com/community/c6h335486}{(Link to AoPS)}
\end{problem}



\begin{solution}[by \href{https://artofproblemsolving.com/community/user/29428}{pco}]
	\begin{tcolorbox}Find all functions $ f: \mathbb R\to\mathbb R$ so that $ f(x + y) + f(x - y) = 2f(x)\cos y$ .\end{tcolorbox}
Let $ P(x,y)$ be the assertion $ f(x + y) + f(x - y) = 2f(x)\cos y$

(a) : $ P(\frac x2,\frac x2)$ $ \implies$ $ f(x) + f(0) = 2f(\frac x2)\cos(\frac x2)$

(b) : $ P(\frac x2,-\frac x2)$ $ \implies$ $ f(0) + f(-x) = 2f(\frac x2)\cos(\frac x2)$

(c) : $ P(0,x)$ $ \implies$ $ f(x)+f(-x)=2f(0)\cos(x)$

(a) - (b) + (c) : $ 2f(x)=2f(0)\cos(x)$

And it is easy to check that this indeed is a solution

Hence the answer : $ \boxed{f(x)=a\cdot \cos(x)}$ $ \forall x$
\end{solution}



\begin{solution}[by \href{https://artofproblemsolving.com/community/user/68920}{prester}]
	\begin{tcolorbox}[quote="trbst"]Find all functions $ f: \mathbb R\to\mathbb R$ so that $ f(x + y) + f(x - y) = 2f(x)\cos y$ .\end{tcolorbox}
Let $ P(x,y)$ be the assertion $ f(x + y) + f(x - y) = 2f(x)\cos y$

(a) : $ P(\frac x2,\frac x2)$ $ \implies$ $ f(x) + f(0) = 2f(\frac x2)\cos(\frac x2)$

(b) : $ P(\frac x2, - \frac x2)$ $ \implies$ $ f(0) + f( - x) = 2f(\frac x2)\cos(\frac x2)$

(c) : $ P(0,x)$ $ \implies$ $ f(x) + f( - x) = 2f(0)\cos(x)$

(a) - (b) + (c) : $ 2f(x) = 2f(0)\cos(x)$

And it is easy to check that this indeed is a solution

Hence the answer : $ \boxed{f(x) = a\cdot \cos(x)}$ $ \forall x$\end{tcolorbox}

I think that (b) is wrong because $ f(\frac x2-\frac x2) + f( \frac x2 - (-\frac x2)) = 2f(\frac x2)\cos(\frac x2) \implies$ the same of (a): $ f(x) + f(0) = 2f(\frac x2)\cos(\frac x2)$
\end{solution}



\begin{solution}[by \href{https://artofproblemsolving.com/community/user/29428}{pco}]
	\begin{tcolorbox}I think that (b) is wrong because $ f(\frac x2 - \frac x2) + f( \frac x2 - ( - \frac x2)) = 2f(\frac x2)\cos(\frac x2) \implies$ the same of (a): $ f(x) + f(0) = 2f(\frac x2)\cos(\frac x2)$\end{tcolorbox}

 :blush:  I think you're right .... 

Sorry.
\end{solution}



\begin{solution}[by \href{https://artofproblemsolving.com/community/user/29428}{pco}]
	\begin{tcolorbox}Find all functions $ f: \mathbb R\to\mathbb R$ so that $ f(x + y) + f(x - y) = 2f(x)\cos y$ .\end{tcolorbox}
Second trial (more complex :( )
Let $ P(x,y)$ be the assertion $ f(x + y) + f(x - y) = 2f(x)\cos(y)$

1) the only even solutions are $ f(x) = a\cos(x)$
=================================
$ P(0,x)$ $ \implies$ $ f(x) + f( - x) = 2f(0)\cos(x)$ and so $ f(x) = f(0)\cos(x)$
Q.E.D.

2) the only odd solutions are $ f(x) = a\sin(x)$
==================================
$ P(x + \frac {\pi}2,y)$ shows that $ g(x) = f(x + \frac {\pi}2)$ is a solution.

$ P( - x,\frac {\pi}2)$ $ \implies$ $ f( - x + \frac {\pi}2) + f( - x - \frac {\pi}2) = 0$ and so $ f( - x + \frac {\pi}2) = - f( - x - \frac {\pi}2)$

$ g( - x) = f( - x + \frac {\pi}2) = - f( - x - \frac {\pi}2)$ $ = f(x + \frac {\pi}2) = g(x)$ (using the previous line plus the fact that $ f(x)$ is odd)

So $ g(x)$ is an even solution and, according to 1), must be $ a\cos(x)$ and so $ f(x) = g(x - \frac {\pi}2) = a\sin(x)$

3) The general solution is $ f(x) = a\cos(x) + b\sin(x)$
========================================
If $ f(x)$ is solution, it's easy to see that $ f(x) + f( - x)$ is an even solution and so is $ a\cos(x)$

If $ f(x)$ is solution, it's easy to see that $ f(x) - f( - x)$ is an odd solution and so is $ b\sin(x)$

Hence the result.

And it's easy to check that this necessary form is sufficient.

Notice that this may also be written $ a\cos(x + b)$
\end{solution}



\begin{solution}[by \href{https://artofproblemsolving.com/community/user/70887}{rogueknight}]
	Can you tell me more, first, why $ f(x)$ is a solution so $ f(x) + f( - x)$ is a solution, too. second, why if $ f(x) + f( - x)$ is an even solution so that all the solution we need  :roll:
\end{solution}



\begin{solution}[by \href{https://artofproblemsolving.com/community/user/29428}{pco}]
	\begin{tcolorbox}Can you tell me more, first, why $ f(x)$ is a solution so $ f(x) + f( - x)$ is a solution too,\end{tcolorbox}


$ P(x,y)$ $ \implies$ $ f(x+y)+f(x-y)=2f(x)\cos(y)$
$ P(-x,-y)$ $ \implies$ $ f(-x-y)+f(-x+y)=2f(-x)\cos(y)$

Adding these two lines : $ (f(x+y)+f(-x-y))+(f(x-y)+f(-x+y))=2(f(x)+f(-x))\cos(y)$

And so $ g(x)=f(x)+f(-x)$ is such that $ g(x+y)+g(x-y)=2g(x)\cos(y)$ and so is a solution too.

\begin{tcolorbox} second, why if $ f(x) + f( - x)$ is an even solution so that all the solution we need  :roll:\end{tcolorbox}

I dont understand your question.

$ f(x)+f(-x)$ is a solution
$ f(x)+f(-x)$ is even
So $ f(x)+f(-x)$ is an even solution, and so $ f(x)+f(-x)=a\cos(x)$, according to 1)


$ f(x)-f(-x)$ is a solution
$ f(x)-f(-x)$ is odd
So $ f(x)-f(-x)$ is an odd solution, and so $ f(x)-f(-x)=b\sin(x)$, according to 2)

And so $ f(x)=\frac{f(x)+f(-x)}2+\frac{f(x)-f(-x)}2=\frac a2\cos(x)+\frac b2\sin(x)$

Hence my result (just change the coefficients names)
\end{solution}



\begin{solution}[by \href{https://artofproblemsolving.com/community/user/70887}{rogueknight}]
	oh, I understand   thanks so much. your solution is so strange because I had never see that solution before. So amazing.  :)
\end{solution}
*******************************************************************************
-------------------------------------------------------------------------------

\begin{problem}[Posted by \href{https://artofproblemsolving.com/community/user/77155}{mno}]
	Find all functions $ f:\mathbb{R}\to\mathbb{R}$ such that 
\[ f(x-f(y))=f(x)+x \cdot f(y)+f(f(y))\]
holds for all $x,y\in\mathbb{R} $.
	\flushright \href{https://artofproblemsolving.com/community/c6h335522}{(Link to AoPS)}
\end{problem}



\begin{solution}[by \href{https://artofproblemsolving.com/community/user/62475}{hqthao}]
	I remember that this problem had post somewhere in the forum   (because I use a method to solve your problem, and this method I remember that I used somewhere in this forum, too  :oops: )
In someline, your will do:
1) $ f(x) - f(y)$ surbjective.
2)$ f(f(y)) = \frac {f(0) - f^{2}(y)}{2}$ (by change $ x$ by $ f(y)$)
3)change $ x$ by $ f(x)$
\end{solution}



\begin{solution}[by \href{https://artofproblemsolving.com/community/user/29428}{pco}]
	\begin{tcolorbox}find all functions $ f:\mathbb{R}\to\mathbb{R}$ such that 
$ \forall x,y\in\mathbb{R}$  $ f(x - f(y)) = f(x) + x.f(y) + f(f(y))$\end{tcolorbox}
To be more precise, hqthao's solution is :

Let $ P(x,y)$ be the assertion $ f(x-f(y))=f(x)+xf(y)+f(f(y))$

$ f(x)=0$ $ \forall x$ is a solution.
let us now look for non all-zero solutions.

1) For non all-zero solution, any real $ x$ may be written as $ f(a)-f(b)$ for some $ a,b$
========================================================
Since $ f(x)$ is non all-zero. Let $ u$ such that $ f(u)\ne 0$
$ P(\frac{x-f(f(u))}{f(u)},u)$ $ \implies$ $ f(\frac{x-f(f(u))}{f(u)}-f(u))$ $ -f(\frac{x-f(f(u))}{f(u)})=x$
Q.E.D.

2) Non all-zero solutions are $ f(x)=-\frac{x^2}2$
================================

$ P(f(x),x)$ $ \implies$ $ f(f(x))=\frac{f(0)-f(x)^2}2$

$ P(f(x),y)$ $ \implies$ $ f(f(x)-f(y))=f(f(x))+f(x)f(y)+f(f(y))$ and, using the previous line :

$ f(f(x)-f(y))=\frac{f(0)-f(x)^2}2+f(x)f(y)+\frac{f(0)-f(x)^2}2$ $ =f(0)-\frac{(f(x)-f(y))^2}2$

So $ f(u)=a-\frac {u^2}2$ for any real $ u$ which may be written as $ f(x)-f(y)$ for some $ x,y$. So for any real $ u$, according to 1) above.

Now, you just have to plug $ f(x)=a-\frac {x^2}2$ in the original equation to get $ a=0$ and so the two solutions to this problem :

$ \boxed{f(x)=0}$ and $ \boxed{f(x)=-\frac {x^2}2}$
\end{solution}
*******************************************************************************
-------------------------------------------------------------------------------

\begin{problem}[Posted by \href{https://artofproblemsolving.com/community/user/78044}{eivos}]
	Let $ a>1$ be a real number. Prove that there exists a function $ f: (0,+\infty) \to (0,+\infty)$ satisfying 
\[ f\left(f(x)+\frac{1}{f(x)}\right)=x+a\]
for all $x>0$.
	\flushright \href{https://artofproblemsolving.com/community/c6h335581}{(Link to AoPS)}
\end{problem}



\begin{solution}[by \href{https://artofproblemsolving.com/community/user/29428}{pco}]
	\begin{tcolorbox}Let $ a > 1$ be a real number.Prove that there exists a function $ f: (0, + \infty) \to (0, + \infty)$ satisfying $ f(f(x) + \frac {1}{f(x)}) = x + a$ $ \forall x > 0$.\end{tcolorbox}

Here is a rather complex solution (Dont hesitate to look for simpler ones) :

Let $ g(x)=f(x)+\frac 1{f(x)}$. 

The equation implies $ g(g(x))=h(x)$ with $ h(x)=x+a+\frac 1{x+a}$

1) $ \exists$ increasing solutions $ g(x)$ with $ g(0)>2$ to the equation $ g(g(x))=h(x)$ $ \forall x\ge 0$
===============================================================
Since $ a>1$, $ h(x)$ is an increasing function over $ [0,+\infty)$ and $ h(0)=a+\frac 1a>2$ and $ h(x)>x+a>x$ $ \forall x\ge 0$
It's then easy to build a solution $ g(x)$ with the classical "piece per piece" method 
[hide="How ?"]
Let $ b\in(2,a+\frac 1a)$ and the sequence : $ x_0=0$, $ x_1=b$ and $ x_{n+2}=h(x_n)$
$ x_n$ is a strictly increasing sequence whose limit is $ +\infty$ and we can build a function $ g(x)$ in the following manner :

Let $ u_n(x)$ a sequence of functions from $ [x_n,x_{n+1}]\to[x_{n+1},x_{n+2}]$ defined as :

Let $ u_0(x)$ any continuous increasing function defined on $ [0,b]$ and such that $ u_0(0)=b$ and $ u_0(b)=h(0)$
$ u_0(x)$ is a continuous bijection from $ [x_0,x_1]\to[x_1,x_2]$ such that $ u_0(x_0)=x_1$ and $ u_0(x_1)=x_2$

Considering thru induction that $ u_n(x)$ is a continuous increasing bijection from $ [x_n,x_{n+1}]\to[x_{n+1},x_{n+2}]$ such that $ u_n(x_n)=x_{n+1}$ and $ u_n(x_{n+1})=x_{n+2}$, define $ u_{n+1}(x)=h(u_n^{[-1]}(x))$

It's easy to see that $ u_{n+1}(x)$ is a continuous increasing bijection from $ [x_{n+1},x_{n+2}]\to[x_{n+2},x_{n+3}]$ such that $ u_{n+1}(x_{n+1})=x_{n+2}$ and $ u_{n+1}(x_{n+2})=x_{n+3}$

And we clearly have $ u_{n+1}(u_n(x))=h(x)$

We just have now to define $ g(x)$ as $ u_n(x)$ over $ [x_n,x_{n+1}]$
And $ g(x)$ is a continuous increasing bijection from $ [0,+\infty)\to [b,+\infty)$ such that $ g(g(x))=h(x)$ $ \forall x\ge 0$ [\/hide]


2) $ \exists$ $ f(x)$ matching the requirements
===============================
Let $ g(x)$ an increasing continuous bijection from $ [0,+\infty)\to [b,+\infty)$ with $ a+\frac 1a>b=g(0)>2$ solution of the equation $ g(g(x))=h(x)$ $ \forall x\ge 0$

The equation $ X+\frac 1X=g(x)$ always has solutions since $ g(x)>2$ and let $ f(x)$ be the greatest solution of this equation.

So we have $ g(x)=f(x)+\frac 1{f(x)}$ and $ f(x)>1$

And since $ g(g(x))=h(x)$, we get $ f(f(x)+\frac 1{f(x)})+\frac 1{f(f(x)+\frac 1{f(x)})}=x+a+\frac 1{x+a}$

And so, either $ f(f(x)+\frac 1{f(x)})=x+a$, either $ f(f(x)+\frac 1{f(x)})=\frac 1{x+a}$ and, since $ f(x)>1$ : 

$ f(f(x)+\frac 1{f(x)})=x+a$

Notice that the $ f(x)$ we built is continuous and increasing since we have $ f(x)=\frac{g(x)+\sqrt{g(x)^2-4}}2$ and $ g(x)$ is an increasing continuous bijection from $ [0,+\infty)\to [b,+\infty)$

Q.E.D.
\end{solution}
*******************************************************************************
-------------------------------------------------------------------------------

\begin{problem}[Posted by \href{https://artofproblemsolving.com/community/user/67949}{aktyw19}]
	Given a positive integer $ n\geq 2$, find all functions $ f: \mathbb R\to \mathbb R$ such that for all $x,y \in  \mathbb R$, we have
\[ f(x^{n}+2f(y))=(f(x))^{n}+y+f(y).\]
	\flushright \href{https://artofproblemsolving.com/community/c6h336057}{(Link to AoPS)}
\end{problem}



\begin{solution}[by \href{https://artofproblemsolving.com/community/user/29428}{pco}]
	\begin{tcolorbox}Given positive integer $ n\geq 2$ .Find all $ f: R\rightarrow R$ such that: $ \forall x,y \in R$ we have:
$ f(x^{n} + 2f(y)) = (f(x))^{n} + y + f(y)$.\end{tcolorbox}
Here is a rather complex solution.
Could you, aktyw19, tell us where is this problem coming from ?

Let $ P(x,y)$ be the assertion $ f(x^n+2f(y))=f^n(x)+y+f(y)$

1) $ f(x)$ is injective
=============
If $ f(a)=f(b)$, just compare $ P(0,a)$ and $ P(0,b)$ and we get $ a=b$

2) Some identities
=============
$ P(x,-f^n(x))$ $ \implies$ $ f(x^n+2f(-f^n(x)))=f(-f^n(x))$ and so, since injective, $ x^n+2f(-f^n(x))=-f^n(x)$ (*)
$ P(0,x)$ $ \implies$ $ f(2f(x))=x+f(x)+f^n(0)$ (**)

3) If $ n$ is even, the only solution is $ f(x)=x$
==============================
3.1) $ f(x)$ is an odd function
------------------------------
Comparing $ P(x,0)$ and $ P(-x,0)$, we get $ f^n(x)=f^n(-x)$ and so $ f(x)=-f(x)$ $ \forall x\ne 0$ (using injectivity)
Since $ f(x)$ is injective, $ \exists a\ne 0$ such that $ f(a)\ne 0$ and then :
$ P(0,a)$ $ \implies$ $ f(2f(a))=f^n(0)+a+f(a)$
$ P(0,-a)$ $ \implies$ $ -f(2f(a))=f^n(0)-a-f(a)$ (we needed here $ a\ne 0$ and $ f(a)\ne 0$
Adding these two lines implies $ f(0)=0$ and so $ f(x)=-f(-x)$ $ \forall x$
Q.E.D.

3.2) $ f(x)=x$
--------------
$ P(x,0)$ $ \implies$ $ f(x^n)=f^n(x)$
So  $ x^n+2f(-f^n(x))=-f^n(x)$ (*) implies  $ x^n-2f(f(x^n))=-f(x^n)$  and so $ x-2f(f(x))=-f(x)$ $ \forall x\ge 0$
And, since $ f(x)$ is an odd function, this is still true for $ x<0$ and we got $ f(f(x))=\frac{x+f(x)}2$ $ \forall x$

So $ f(f(x^n))=\frac{x^n+f(x^n)}2$

But $ f(f(x^n))=(f(f(x)))^n=\left(\frac{x+f(x)}2\right)^n$ and so $ \frac{x^n+f(x^n)}2=\left(\frac{x+f(x)}2\right)^n$

And since $ g(x)=x^n$ is a convex function $ \forall n>1$, this can only be true if $ f(x)=x$ which, indeed, is a solution.
Q.E.D

4) If $ n$ is odd, the only solution is $ f(x)=x$
=============================
4.1) $ f(x)$ is surjective
-------------------------
$ P((-x^n-2f(-x^n))^{\frac 1n},-x^n)$ $ \implies$ $ f(-x^n)=f^n((-x^n-2f(-x^n))^{\frac 1n})-x^n+f(-x^n)$ and so $ f(\text{something})=x$
Q.E.D.

4.2) $ f(x+y)=f(x)+f(y)-f(0)$ and $ f(x^n)=f^n(x)+c$
--------------------------------------------------
Let $ c$ such that $ f(x)=0$
$ P(x,c)$ $ \implies$ $ f(x^n)=f^n(x)+c$
$ P(0,y)$ $ \implies$ $ f(2f(y))=f(0)^n+y+f(y)$

So $ f(x^n+2f(y))=f(x^n)-c+f(2f(y))-f(0)^n$
And, since both $ x^n$ and $ 2f(x)$ are surjective :

$ f(x+y)=f(x)+f(y)-c-f(0)^n$
And setting $ y=0$ in this equality gives $ f(0)=c+f(0)^n$
Q.E.D.

4.3) $ f(x+y)=f(x)+f(y)$ and $ f(x^n)=f^n(x)$
-------------------------------------------
From 4.2, we get that $ f(2x)=2f(x)-f(0)$ and so $ f(2^nx)=2^nf(x)-(2^n-1)f(0)$
Setting $ x\to x^n$, we get $ f(2^nx^n)=2^nf(x^n)-(2^n-1)f(0)$ and so $ f(2^nx^n)=2^nf^n(x)+2^nc-(2^n-1)f(0)$

But $ f(2^nx^n)=f((2x)^n)=f^n(2x)+c$ $ =(2f(x)-f(0))^n+c$

And so $ (2f(x)-f(0))^n+c=2^nf^n(x)+2^nc-(2^n-1)f(0)$ and since $ f(x)$ is surjective, we have the identity between polynomials :

$ (2X-f(0))^n+c=2^nX^n+2^nc-(2^n-1)f(0)$
which implies, since $ n>1$, $ f(0)=0$ and so $ c=0$
Q.E.D.

4.4) $ f(x)=x$ $ \forall x$
-------------------------
$ f(2f(x))=x+f(x)+f^n(0)$ (**) becomes $ 2f(f(x))=x+f(x)$ and so $ f(f(x))=\frac{x+f(x)}2$ $ \forall x$

So $ f(f(x^n))=\frac{x^n+f(x^n)}2$

But $ f(f(x^n))=(f(f(x)))^n=\left(\frac{x+f(x)}2\right)^n$ and so $ \frac{x^n+f(x^n)}2=\left(\frac{x+f(x)}2\right)^n$

And since $ g(x)=x^n$ is a convex function $ \forall n>1$, this can only be true if $ f(x)=x$ which, indeed, is a solution.
Q.E.D



Hence the unique solution : $ \boxed{f(x)=x}$ $ \forall x$
\end{solution}



\begin{solution}[by \href{https://artofproblemsolving.com/community/user/62475}{hqthao}]
	oh, so nice. I thought this problem for many hours  :blush: maybe I failed because I always compare this problem with a other problem I had solved. (they really have something same). And here:
$ f: R \rightarrow R$
$ f(x^n + f(y)) = y + f^{n}(x)$
if anyone like that problem, please solve. I sure that's nice to  :)
\end{solution}



\begin{solution}[by \href{https://artofproblemsolving.com/community/user/29428}{pco}]
	\begin{tcolorbox}oh, so nice. I thought this problem for many hours  :blush: maybe I failed because I always compare this problem with a other problem I had solved. (they really have something same). And here:
$ f: R \rightarrow R$
$ f(x^n + f(y)) = y + f^{n}(x)$
if anyone like that problem, please solve. I sure that's nice to  :)\end{tcolorbox}

See http://www.mathlinks.ro/Forum/viewtopic.php?t=307854
\end{solution}



\begin{solution}[by \href{https://artofproblemsolving.com/community/user/29428}{pco}]
	\begin{tcolorbox} 
4.4) $ f(x) = x$ $ \forall x$
-------------------------
$ f(2f(x)) = x + f(x) + f^n(0)$ (**) becomes $ 2f(f(x)) = x + f(x)$ and so $ f(f(x)) = \frac {x + f(x)}2$ $ \forall x$

So $ f(f(x^n)) = \frac {x^n + f(x^n)}2$

But $ f(f(x^n)) = (f(f(x)))^n = \left(\frac {x + f(x)}2\right)^n$ and so $ \frac {x^n + f(x^n)}2 = \left(\frac {x + f(x)}2\right)^n$

And since $ g(x) = x^n$ is a convex function $ \forall n > 1$, this can only be true if $ f(x) = x$ which, indeed, is a solution.
\end{tcolorbox}

There is a mistake here.I used the same proof as in 3.2 but, with $ n$ odd, $ x^n$ is no longer convex.

In fact $ \frac {x^n + f(x)^n}2 = \left(\frac {x + f(x)}2\right)^n$ implies $ f(x)=x$ or $ f(x)=-x$

And we need to eliminate the case $ fx)=-x$ for some $ x$ (easy using $ f(2f(x))=x+f(x)$ and injectivity).
\end{solution}
*******************************************************************************
-------------------------------------------------------------------------------

\begin{problem}[Posted by \href{https://artofproblemsolving.com/community/user/55239}{bvarici}]
	Find all continuous functions $f: \mathbb R \to \mathbb R$ such that for all reals $x$ and $y$,
\[ f (xy) = x f (y)+y f (x).\]
	\flushright \href{https://artofproblemsolving.com/community/c6h336136}{(Link to AoPS)}
\end{problem}



\begin{solution}[by \href{https://artofproblemsolving.com/community/user/29428}{pco}]
	\begin{tcolorbox}Find all continuous functions f : R→R such that f (xy) = x f (y)+y f (x).\end{tcolorbox}

Let $ g(x)$ from $ \mathbb R^*=\mathbb R\backslash\{0\}\to\mathbb R$ defined as $ g(x)=\frac{f(x)}x$
$ g(x)$ is a continuous function over $ \mathbb R^*$ and is such that $ g(xy)=f(x)+g(y)$

Let $ P(x,y)$ be the assertion $ g(xy)=g(x)+g(y)$

$ P(x,1)$ $ \implies$ $ g(1)=1$
$ P(-1,-1)$ $ \implies$ $ g(-1)=0$
$ P(x,-1)$ $ \implies$ $ g(-x)=g(x)$

Let then $ h(x)$ from $ \mathbb R\to\mathbb R$ defined as $ h(x)=g(e^x)$
$ h(x)$ is continuous and $ P(e^x,e^y)$ $ \implies$ $ h(x+y)=h(x)+h(y)$

So $ h(x)=ax$
So $ g(x)=a\ln(x)$ $ \forall x>0$ and, since $ g(-x)=g(x)$, we get $ g(x)=a\ln(|x|)$

So $ f(x)=ax\ln(|x|)$ $ \forall x\ne 0$ and then continuity implies $ f(0)=0$

It's then easy to check back that this indeed is a solution.

Hence the answer :
$ f(x)=ax\ln(|x|)$ $ \forall x\ne 0$
$ f(0)=0$
\end{solution}



\begin{solution}[by \href{https://artofproblemsolving.com/community/user/55239}{bvarici}]
	you said:
g(x) is a continuous function over \mathbb R^* and is such that g(xy)=f(x)+g(y)

I think anyway g(xy)= f(xy) \/ xy = xf(y) + yf(x) \/  xy = f(y)\/y + f(x)\/x= g(x) + g(y)
\end{solution}



\begin{solution}[by \href{https://artofproblemsolving.com/community/user/29428}{pco}]
	\begin{tcolorbox}you said:
g(x) is a continuous function over \mathbb R^* and is such that g(xy)=f(x)+g(y)

I think anyway g(xy)= f(xy) \/ xy = xf(y) + yf(x) \/  xy = f(y)\/y + f(x)\/x= g(x) + g(y)\end{tcolorbox}

Just a typo :)
\end{solution}



\begin{solution}[by \href{https://artofproblemsolving.com/community/user/67223}{Amir Hossein}]
	\begin{tcolorbox}you said:
g(x) is a continuous function over \mathbb R^* and is such that g(xy)=f(x)+g(y)

I think anyway g(xy)= f(xy) \/ xy = xf(y) + yf(x) \/  xy = f(y)\/y + f(x)\/x= g(x) + g(y)\end{tcolorbox}

Please write with $ \text{\LaTeX}$.I don't understand what you write.
Read the attachment.
For your problem.
Let $ g(x) = \frac {f(x)}{x}$ then the problem become this:$ g(xy) = g(x) + g(y)$ which is one of famous functions.

@pco: Thanks for your quick reply!
\end{solution}



\begin{solution}[by \href{https://artofproblemsolving.com/community/user/55239}{bvarici}]
	\begin{tcolorbox}[quote="bvarici"]you said:
g(x) is a continuous function over \mathbb R^* and is such that g(xy)=f(x)+g(y)

I think anyway g(xy)= f(xy) \/ xy = xf(y) + yf(x) \/  xy = f(y)\/y + f(x)\/x= g(x) + g(y)\end{tcolorbox}

Please write with $ \text{\LaTeX}$.I don't understand what you write.
Read the attachment.
For your problem.
Let $ g(x) = \frac {f(x)}{x}$ then the problem become this:$ g(xy) = g(x) + g(y)$ which is one of famous functions.

@pco: Thanks for your quick reply!\end{tcolorbox}

thanks for suggestion I know this is'nt appear nice 
I'm sorry but I don't know use LaTeX
\end{solution}
*******************************************************************************
-------------------------------------------------------------------------------

\begin{problem}[Posted by \href{https://artofproblemsolving.com/community/user/60032}{Stephen}]
	$ a, b, c$ are fixed positive integers. For all positive integers $ n$, function $ f: \mathbb{N}\to\mathbb{N}$ satisfies $ f(f(...f(f(n))...)) = bn^c$.

($ f$ is composited $ a$ times. For example, if $ a = 1$, $ f(n)=bn^c$. And if $ a = 3$, $ f(f(f(n)))=bn^c$.)

Does $ f$ exist?
	\flushright \href{https://artofproblemsolving.com/community/c6h336300}{(Link to AoPS)}
\end{problem}



\begin{solution}[by \href{https://artofproblemsolving.com/community/user/29428}{pco}]
	\begin{tcolorbox}$ a, b, c$ are fixed positive integers. For all positive integers $ n$, function $ f: \mathbb{N}\to\mathbb{N}$ satisfies $ f(f(...f(f(n))...)) = bn^c$.

($ f$ is composited $ a$ times. For example, if $ a = 1$, $ f(n) = bn^c$. And if $ a = 3$, $ f(f(f(n))) = bn^c$.)

Does $ f$ exist?\end{tcolorbox}

Yes :

If $ b = c = 1$, choose $ f(x) = x$
If $ bc\ne 1$ :

Let $ A = \{bn^c \forall n\in\mathbb N\}$
Let $ B = \mathbb N\backslash A$

$ B$ has infinitely many elements and so $ \exists B_1,B_2,...,B_a$ such that :
(a) $ B_i\cap B_j = \emptyset$ $ \forall i\ne j$
(b) $ \cup B_i = B$
(c) all $ B_i$ are equinumerous.

Let then $ \{f_i\}_{i = 1}^{a - 1}$ a set of bijections $ f_i(x)$ from $ B_i\to B_{i + 1}$

And choose $ f(x)$ defined as :

$ \forall x\in B_1$ : $ f(x) = f_1(x)$
$ \forall x\in B_2$ : $ f(x) = f_2(x)$
...
$ \forall x\in B_{a - 1}$ : $ f(x) = f_{a - 1}(x)$

$ \forall x\in B_a$ : $ f(x) = bn\left(f_1^{ - 1}(f_2{ - 1}(...f_{a - 1}^{ - 1}(x))...)\right)^c$

$ \forall x\in A$ : $ f(x) = bf((\frac xb)^{\frac 1c})$ \begin{bolded}*** edited ***\end{bolded}\end{underlined} : since $ f(bn^c)=bf(n)^c$, we should have here : $ f(x) = bf((\frac xb)^{\frac 1c})^c$
\end{solution}



\begin{solution}[by \href{https://artofproblemsolving.com/community/user/60032}{Stephen}]
	Well, how can we now that $ b\left(f_{1}^{ - 1}(f_{2}^{ - 1}(...f_{a - 1}^{ - 1}(x))...)\right)^{c}$ is a element from set $ B_1$?
\end{solution}



\begin{solution}[by \href{https://artofproblemsolving.com/community/user/29428}{pco}]
	\begin{tcolorbox}Well, how can we now that $ b\left(f_{1}^{ - 1}(f_{2}^{ - 1}(...f_{a - 1}^{ - 1}(x))...)\right)^{c}$ is a element from set $ B_1$?\end{tcolorbox}

It's not.

What is true is that $ f_{1}^{ - 1}(f_{2}^{ - 1}(...f_{a - 1}^{ - 1}(x))\in B_1$ :

$ x\in B_a$
so $ f_{a-1}^{-1}(x)\in B_{a-1}$
so $ f_{a-2}^{-1}(f_{a-1}^{-1}(x))\in B_{a-2}$
...
So $ f_{1}^{ - 1}(f_{2}^{ - 1}(...f_{a - 1}^{ - 1}(x))\in B_1$
\end{solution}



\begin{solution}[by \href{https://artofproblemsolving.com/community/user/60032}{Stephen}]
	If it's not, it doesn't work.

To explain more, then how can we define $ f(b\left(f_{1}^{-1}(f_{2}^{-1}(...f_{a-1}^{-1}(x))...)\right)^{c})$?
\end{solution}



\begin{solution}[by \href{https://artofproblemsolving.com/community/user/29428}{pco}]
	\begin{tcolorbox}If it's not, it doesn't work.

To explain more, then how can we define $ f(b\left(f_{1}^{ - 1}(f_{2}^{ - 1}(...f_{a - 1}^{ - 1}(x))...)\right)^{c})$?\end{tcolorbox}

Obviously, $ b\left(f_{1}^{ - 1}(f_{2}^{ - 1}(...f_{a - 1}^{ - 1}(x))...)\right)^{c}\in A$ 

And I wrote that $ \forall x\in A$ : $ f(x)=bf((\frac xb)^{\frac 1c})$ 
There was obviously a typo here (since we know that $ f(bn^c)=bf(n)^c$ and we should have $ f(x)=bf((\frac xb)^{\frac 1c})^c$ 

But this changes nothing to the fact that $ f()$ is defined over A as well as over B.

We get $ f(b\left(f_{1}^{ - 1}(f_{2}^{ - 1}(...f_{a - 1}^{ - 1}(x))...)\right)^{c})$ $ =b(f(f_{1}^{ - 1}(f_{2}^{ - 1}(...f_{a - 1}^{ - 1}(x))...))^c$
\end{solution}



\begin{solution}[by \href{https://artofproblemsolving.com/community/user/29428}{pco}]
	In fact, there is a general solution to the equation $ f^{[n]}(n)=g(n)$ where :
$ f(n)$ is a function from $ \mathbb N\to\mathbb N$
$ g(n)$ is an injective function from $ %Error. "mathBB" is a bad command.
N\to\mathbb N$ such that :
a) $ B=\mathbb N\backslash g(\mathbb N)$ may be split in $ n$ equinumerous non empty sets.
b) $ \forall x\in \mathbb N$, $ \exists r(x)\in\mathbb N\backslash g(\mathbb N)$ and $ n(x)\in\mathbb N\cup\{0\}$ such that $ x=g^{[n(x)]}(r(x))$

Notice that $ n(x)$ and $ r(x)$, since they exist, are obviously unique for a given $ x$.

Let $ A=g(\mathbb N)$
Let $ B=\mathbb N\backslash A$
Let $ \{B_k\}_{k=1}^n$ a partition of $ B$ in $ n$ equinumerous sets.
Let $ \{f_k\}_{k=1}^{n-1}$ a family of bijections from $ B_k\to B_{k+1}$

Then define $ f(x)$ as :
$ \forall x\in B_k$ with $ k\in[1,n-1]$ : $ f(x)=f_k(x)$
$ \forall x\in B_{n}$ : $ f(x)=g(f_1^{[-1]}(f_2^{[-1]}(...f_{n-1}^{[-1]}(x)...)))$
$ \forall x\in A$ : $ f(x)=g(f(g^{-1}(n)))$

Notice that this last line is a recursive definition which may be written $ f(x)=g^{[n(x)]}(f(r(x))$ and, since $ r(x)\in B$, this complete a full definition of $ f(x)$


1) Proof that the function such defined is a solution.
=================================
1.1) If $ x\in B$ $ f^{[n]}(x)=g(x)$
---------------------------------------
Let $ k$ such that $ x\in B_k$
Let $ u=f_1^{[-1](f_2^{[-1]}(...f_{k-1}^{[-1]}(x)...))}$

We get $ u \in B_1$ and $ f^{[k-1]}(u)=x$

So $ f^{[n-1]}(u)=f_{n-1}(f_{n-2}(...f_1(u)...))\in B_n$
So $ f^{[n]}(u)=g(f_1^{[-1]}(f_2^{[-1]}(...f_{n-1}^{[-1]}(f_{n-1}(f_{n-2}(...f_1(u)...)))...))))$ $ =g(u)$

So $ f^{[n+1]}(u)=f(g(u))=g(f(g^{[-1]}(g(u))=g(f(u))$ (remember $ g(u)\in A$)
So $ f^{[n+2]}(u)=f(g(f(u)))=g(f(g^{[-1]}(g(f(u)))=g(f^{[2]}(u))$

And so $ f^{[n+k-1]}(u)=g(f^{[k-1]}(u))$ which means $ f^{[n]}(x)=g(x)$
Q.E.D

1.2) If $ x\in A$ $ f^{[n]}(x)=g(x)$
---------------------------------------
$ f(x)=g^{[n(x)]}(f(r(x))$ with $ r(x)\in B$ and $ n(x)>0$

The definition of $ f(x)$ implies $ f(g(x))=g(f(x))$ and so $ f^{[p]}(g^{[q]}(x))=g^{[q]}(f^{[p]}(x)$

So $ f^{[n]}(x)=f^{[n-1]}(g^{[n(x)]}(f(r(x)))$ $ =g^{[n(x)]}(f^{[n]}(r(x)))$

And, since $ r(x)\in B$, we got from 1.1) that $ f^{n]}(r(x))=g(r(x))$ and so $ f^{[n]}(x)=g^{[n(x)]}(g(r(x)))$

And so $ f^{[n]}(x)=g(g^{[n(x)]}(r(x)))$ $ =g(x)$
Q.E.D

2) Proof that any solution can be written is such a way and so this is a general solution
=======================================================
Let $ f(x)$ be a solution.
We obviously have $ f(g(x))=g(f(x))$ and so $ f(x)=g(f(g^{[-1]}(x)))$ $ \forall x\in A$

We know that $ f(x)\in A$ $ \implies$ $ f(x)=g(y)$ $ \implies$ $ f(f(x))=f(g(y))=g(f(y))\in A$
We also now that $ f^{[n]}(x)\in A$

So, $ \forall x\in B$, $ \exists k(x)\in[1,n]$ such that $ f^{\begin{italicized}}(x)\in B$ $ \forall 0\le i<k$ and $ f^{\begin{italicized}}(x)\in A$ $ \forall i\ge k$

And so we can define $ B_k=\{x\in B$ such that $ k(x)=n+1-k\}$

And it is easy to check that $ f(x)$ is a bijection from $ B_k\to B_{k+1}$ $ \forall k\in[1,n-1]$ so that the set of $ B_k$ is a partition of $ B$ in $ n$ equinumerous subsets. and that, considering $ f_i(x)=f(x)$, we get the required form.
Q.E.D

3) application to the problem
===================
$ g(n)=bn^c$ is injective.
If $ bc\ne 1$, $ B$ is infinite and so $ B_k$ may be choosen so condition a) is OK
And obviously condition b) is OK too.

4) some complementary remarks
======================
If condition b) is not respected, solutions may exist too. But not in all cases and the proof is more complex.
\end{solution}



\begin{solution}[by \href{https://artofproblemsolving.com/community/user/60032}{Stephen}]
	Thank you for the good repiles, pco. 

But I still don't think it'll work.

I think that you should explain more about $ x\in A$.

Sorry if I had offend you so much :maybe:
\end{solution}



\begin{solution}[by \href{https://artofproblemsolving.com/community/user/29428}{pco}]
	\begin{tcolorbox}Thank you for the good repiles, pco. 

But I still don't think it'll work.

I think that you should explain more about $ x\in A$.

Sorry if I had offend you so much :maybe:\end{tcolorbox}

What is not clear in the 5 lines explanation of my last post (paragraph 1.2)?
Just tell me what is the first line you dont understand.
\end{solution}



\begin{solution}[by \href{https://artofproblemsolving.com/community/user/60032}{Stephen}]
	\begin{tcolorbox}[quote="Stephen"]$ a, b, c$ are fixed positive integers. For all positive integers $ n$, function $ f: \mathbb{N}\to\mathbb{N}$ satisfies $ f(f(...f(f(n))...)) = bn^c$.

($ f$ is composited $ a$ times. For example, if $ a = 1$, $ f(n) = bn^c$. And if $ a = 3$, $ f(f(f(n))) = bn^c$.)

Does $ f$ exist?\end{tcolorbox}
$ \forall x\in B_1$ : $ f(x) = f_1(x)$
$ \forall x\in B_2$ : $ f(x) = f_2(x)$
...
$ \forall x\in B_{a - 1}$ : $ f(x) = f_{a - 1}(x)$

$ \forall x\in B_a$ : $ f(x) = b\left(f_1^{ - 1}(f_2{ - 1}(...f_{a - 1}^{ - 1}(x))...)\right)^c$

$ \forall x\in A$ : $ f(x) = bf((\frac xb)^{\frac 1c})$ \begin{bolded}*** edited ***\end{bolded}\end{underlined} : since $ f(bn^c) = bf(n)^c$, we should have here : $ f(x) = bf((\frac xb)^{\frac 1c})^c$\end{tcolorbox}

I understand the definition you have stated.

What I mean is that if $ x \in A$, then why $ f(f(...f(f(x))...)=bx^c$?

I agreed when $ x$ is not an element of $ A$.

So if you explain more kindly in that case($ x \in A$), I'll ask you no more questions in this topic. :what?:
\end{solution}



\begin{solution}[by \href{https://artofproblemsolving.com/community/user/29428}{pco}]
	\begin{tcolorbox} What I mean is that if $ x \in A$, then why $ f(f(...f(f(x))...) = bx^c$?

I agreed when $ x$ is not an element of $ A$.

So if you explain more kindly in that case($ x \in A$), I'll ask you no more questions in this topic. :what?:\end{tcolorbox}

1.2) If $ x\in A$ $ f^{[n]}(x) = g(x)$
---------------------------------------
$ f(x) = g^{[n(x)]}(f(r(x))$ with $ r(x)\in B$ and $ n(x) > 0$

The definition of $ f(x)$ implies $ f(g(x)) = g(f(x))$ and so $ f^{[p]}(g^{[q]}(x)) = g^{[q]}(f^{[p]}(x)$

So $ f^{[n]}(x) = f^{[n - 1]}(g^{[n(x)]}(f(r(x)))$ $ = g^{[n(x)]}(f^{[n]}(r(x)))$

And, since $ r(x)\in B$, we got from 1.1) that $ f^{n]}(r(x)) = g(r(x))$ and so $ f^{[n]}(x) = g^{[n(x)]}(g(r(x)))$

And so $ f^{[n]}(x) = g(g^{[n(x)]}(r(x)))$ $ = g(x)$
Q.E.D
\end{solution}
*******************************************************************************
-------------------------------------------------------------------------------

\begin{problem}[Posted by \href{https://artofproblemsolving.com/community/user/67223}{Amir Hossein}]
	Define $ f$ on the positive integers by $ f(n) = k^2 + k + 1$, where $ n = 2^k(2l + 1)$ for some non-negative integers $k$ and $l$. 
Find the smallest $ n$ such that \[ f(1) + f(2) + \cdots + f(n) \geq 123456.\]
	\flushright \href{https://artofproblemsolving.com/community/c6h336594}{(Link to AoPS)}
\end{problem}



\begin{solution}[by \href{https://artofproblemsolving.com/community/user/29428}{pco}]
	\begin{tcolorbox}Define $ f$ on the positive integers by $ f(n) = k^2 + k + 1$.where $ n = 2^k(2l + 1)$ for some $ k, l$ nonnegative integers. 
Find the smallest $ n$ such that $ f(1) + f(2) + ... + f(n) \geq 123456$.\end{tcolorbox}

Here is an ugly solution  :blush: :

So $ f(n)=v_2(n)^2+v_2(n)+1$

The numbers of integers in $ [1,n]$ such that $ v_2(n)=k$ is exactly $ \left\lfloor\frac n{2^k}\right\rfloor$ $ -\left\lfloor\frac n{2^{k+1}}\right\rfloor$

So the required sum is $ S(n)=\sum_{k=0}^{+\infty}(k^2+k+1)($ $ \left\lfloor\frac n{2^k}\right\rfloor$ $ -\left\lfloor\frac n{2^{k+1}}\right\rfloor)$

So $ s(n)=n+\sum_{k=1}^{+\infty}\left\lfloor\frac n{2^k}\right\rfloor(k^2+k+1-(k-1)^2-(k-1)-1)$

So $ s(n)=n+2\sum_{k=1}^{+\infty}k\left\lfloor\frac n{2^k}\right\rfloor$

From there, I just find a rather brute computation :

We have $ S(2^p)=5\cdot 2^p-2p-4$. So, looking for the greatest power of $ 2$ around $ \frac {123456}5$, we get :
$ S(2^{14})=S(16384)=5\cdot 16384-32=81888$
$ S(2^{15})=S(32768)=5\cdot 32768-34=163806$

Then, dichotomy research :( :

$ S(24000)=24000+2($ $ \left\lfloor\frac{24000}{16384}\right\rfloor$ $ +2\left\lfloor\frac{24000}{16384}\right\rfloor$ $ +3\left\lfloor\frac{24000}{12288}\right\rfloor$ $ +4\left\lfloor\frac{24000}{8192}\right\rfloor$ $ +5\left\lfloor\frac{24000}{5120}\right\rfloor$ $ +6\left\lfloor\frac{24000}{3072}\right\rfloor$ $ +7\left\lfloor\frac{24000}{1792}\right\rfloor$ $ +8\left\lfloor\frac{24000}{1024}\right\rfloor$ $ +9\left\lfloor\frac{24000}{576}\right\rfloor$ $ +10\left\lfloor\frac{24000}{320}\right\rfloor$ $ +11\left\lfloor\frac{24000}{176}\right\rfloor$ $ +12\left\lfloor\frac{24000}{96}\right\rfloor$ $ +13\left\lfloor\frac{24000}{52}\right\rfloor$ $ +14\left\lfloor\frac{24000}{28}\right\rfloor$  $ )=119836$
$ S(28000)=28000+2($ $ \left\lfloor\frac{28000}{12000}\right\rfloor$ $ +2\left\lfloor\frac{28000}{12000}\right\rfloor$ $ +3\left\lfloor\frac{28000}{9000}\right\rfloor$ $ +4\left\lfloor\frac{28000}{6000}\right\rfloor$ $ +5\left\lfloor\frac{28000}{3750}\right\rfloor$ $ +6\left\lfloor\frac{28000}{2250}\right\rfloor$ $ +7\left\lfloor\frac{28000}{1309}\right\rfloor$ $ +8\left\lfloor\frac{28000}{744}\right\rfloor$ $ +9\left\lfloor\frac{28000}{414}\right\rfloor$ $ +10\left\lfloor\frac{28000}{230}\right\rfloor$ $ +11\left\lfloor\frac{28000}{121}\right\rfloor$ $ +12\left\lfloor\frac{28000}{60}\right\rfloor$ $ +13\left\lfloor\frac{28000}{26}\right\rfloor$ $ +14\left\lfloor\frac{28000}{14}\right\rfloor$  $ )=139838$
$ S(26000)=26000+2($ $ \left\lfloor\frac{26000}{14000}\right\rfloor$ $ +2\left\lfloor\frac{26000}{14000}\right\rfloor$ $ +3\left\lfloor\frac{26000}{10500}\right\rfloor$ $ +4\left\lfloor\frac{26000}{7000}\right\rfloor$ $ +5\left\lfloor\frac{26000}{4375}\right\rfloor$ $ +6\left\lfloor\frac{26000}{2622}\right\rfloor$ $ +7\left\lfloor\frac{26000}{1526}\right\rfloor$ $ +8\left\lfloor\frac{26000}{872}\right\rfloor$ $ +9\left\lfloor\frac{26000}{486}\right\rfloor$ $ +10\left\lfloor\frac{26000}{270}\right\rfloor$ $ +11\left\lfloor\frac{26000}{143}\right\rfloor$ $ +12\left\lfloor\frac{26000}{72}\right\rfloor$ $ +13\left\lfloor\frac{26000}{39}\right\rfloor$ $ +14\left\lfloor\frac{26000}{14}\right\rfloor$  $ )=129864$
$ S(25000)=25000+2($ $ \left\lfloor\frac{25000}{13000}\right\rfloor$ $ +2\left\lfloor\frac{25000}{13000}\right\rfloor$ $ +3\left\lfloor\frac{25000}{9750}\right\rfloor$ $ +4\left\lfloor\frac{25000}{6500}\right\rfloor$ $ +5\left\lfloor\frac{25000}{4060}\right\rfloor$ $ +6\left\lfloor\frac{25000}{2436}\right\rfloor$ $ +7\left\lfloor\frac{25000}{1421}\right\rfloor$ $ +8\left\lfloor\frac{25000}{808}\right\rfloor$ $ +9\left\lfloor\frac{25000}{450}\right\rfloor$ $ +10\left\lfloor\frac{25000}{250}\right\rfloor$ $ +11\left\lfloor\frac{25000}{132}\right\rfloor$ $ +12\left\lfloor\frac{25000}{72}\right\rfloor$ $ +13\left\lfloor\frac{25000}{39}\right\rfloor$ $ +14\left\lfloor\frac{25000}{14}\right\rfloor$  $ )=124876$
$ S(24500)=24500+2($ $ \left\lfloor\frac{24500}{12500}\right\rfloor$ $ +2\left\lfloor\frac{24500}{12500}\right\rfloor$ $ +3\left\lfloor\frac{24500}{9375}\right\rfloor$ $ +4\left\lfloor\frac{24500}{6248}\right\rfloor$ $ +5\left\lfloor\frac{24500}{3905}\right\rfloor$ $ +6\left\lfloor\frac{24500}{2340}\right\rfloor$ $ +7\left\lfloor\frac{24500}{1365}\right\rfloor$ $ +8\left\lfloor\frac{24500}{776}\right\rfloor$ $ +9\left\lfloor\frac{24500}{432}\right\rfloor$ $ +10\left\lfloor\frac{24500}{240}\right\rfloor$ $ +11\left\lfloor\frac{24500}{132}\right\rfloor$ $ +12\left\lfloor\frac{24500}{72}\right\rfloor$ $ +13\left\lfloor\frac{24500}{39}\right\rfloor$ $ +14\left\lfloor\frac{24500}{14}\right\rfloor$  $ )=122296$
$ S(24750)=24750+2($ $ \left\lfloor\frac{24750}{12250}\right\rfloor$ $ +2\left\lfloor\frac{24750}{12250}\right\rfloor$ $ +3\left\lfloor\frac{24750}{9186}\right\rfloor$ $ +4\left\lfloor\frac{24750}{6124}\right\rfloor$ $ +5\left\lfloor\frac{24750}{3825}\right\rfloor$ $ +6\left\lfloor\frac{24750}{2292}\right\rfloor$ $ +7\left\lfloor\frac{24750}{1337}\right\rfloor$ $ +8\left\lfloor\frac{24750}{760}\right\rfloor$ $ +9\left\lfloor\frac{24750}{423}\right\rfloor$ $ +10\left\lfloor\frac{24750}{230}\right\rfloor$ $ +11\left\lfloor\frac{24750}{121}\right\rfloor$ $ +12\left\lfloor\frac{24750}{60}\right\rfloor$ $ +13\left\lfloor\frac{24750}{26}\right\rfloor$ $ +14\left\lfloor\frac{24750}{14}\right\rfloor$  $ )=123632$
$ S(24625)=24625+2($ $ \left\lfloor\frac{24625}{12375}\right\rfloor$ $ +2\left\lfloor\frac{24625}{12374}\right\rfloor$ $ +3\left\lfloor\frac{24625}{9279}\right\rfloor$ $ +4\left\lfloor\frac{24625}{6184}\right\rfloor$ $ +5\left\lfloor\frac{24625}{3865}\right\rfloor$ $ +6\left\lfloor\frac{24625}{2316}\right\rfloor$ $ +7\left\lfloor\frac{24625}{1351}\right\rfloor$ $ +8\left\lfloor\frac{24625}{768}\right\rfloor$ $ +9\left\lfloor\frac{24625}{432}\right\rfloor$ $ +10\left\lfloor\frac{24625}{240}\right\rfloor$ $ +11\left\lfloor\frac{24625}{132}\right\rfloor$ $ +12\left\lfloor\frac{24625}{72}\right\rfloor$ $ +13\left\lfloor\frac{24625}{39}\right\rfloor$ $ +14\left\lfloor\frac{24625}{14}\right\rfloor$  $ )=123033$
$ S(24700)=24700+2($ $ \left\lfloor\frac{24700}{12312}\right\rfloor$ $ +2\left\lfloor\frac{24700}{12312}\right\rfloor$ $ +3\left\lfloor\frac{24700}{9234}\right\rfloor$ $ +4\left\lfloor\frac{24700}{6156}\right\rfloor$ $ +5\left\lfloor\frac{24700}{3845}\right\rfloor$ $ +6\left\lfloor\frac{24700}{2304}\right\rfloor$ $ +7\left\lfloor\frac{24700}{1344}\right\rfloor$ $ +8\left\lfloor\frac{24700}{768}\right\rfloor$ $ +9\left\lfloor\frac{24700}{432}\right\rfloor$ $ +10\left\lfloor\frac{24700}{240}\right\rfloor$ $ +11\left\lfloor\frac{24700}{132}\right\rfloor$ $ +12\left\lfloor\frac{24700}{72}\right\rfloor$ $ +13\left\lfloor\frac{24700}{39}\right\rfloor$ $ +14\left\lfloor\frac{24700}{14}\right\rfloor$  $ )=123378$
$ S(24725)=24725+2($ $ \left\lfloor\frac{24725}{12350}\right\rfloor$ $ +2\left\lfloor\frac{24725}{12350}\right\rfloor$ $ +3\left\lfloor\frac{24725}{9261}\right\rfloor$ $ +4\left\lfloor\frac{24725}{6172}\right\rfloor$ $ +5\left\lfloor\frac{24725}{3855}\right\rfloor$ $ +6\left\lfloor\frac{24725}{2310}\right\rfloor$ $ +7\left\lfloor\frac{24725}{1344}\right\rfloor$ $ +8\left\lfloor\frac{24725}{768}\right\rfloor$ $ +9\left\lfloor\frac{24725}{432}\right\rfloor$ $ +10\left\lfloor\frac{24725}{240}\right\rfloor$ $ +11\left\lfloor\frac{24725}{132}\right\rfloor$ $ +12\left\lfloor\frac{24725}{72}\right\rfloor$ $ +13\left\lfloor\frac{24725}{39}\right\rfloor$ $ +14\left\lfloor\frac{24725}{14}\right\rfloor$  $ )=123521$
$ S(24712)=24712+2($ $ \left\lfloor\frac{24712}{12362}\right\rfloor$ $ +2\left\lfloor\frac{24712}{12362}\right\rfloor$ $ +3\left\lfloor\frac{24712}{9270}\right\rfloor$ $ +4\left\lfloor\frac{24712}{6180}\right\rfloor$ $ +5\left\lfloor\frac{24712}{3860}\right\rfloor$ $ +6\left\lfloor\frac{24712}{2316}\right\rfloor$ $ +7\left\lfloor\frac{24712}{1351}\right\rfloor$ $ +8\left\lfloor\frac{24712}{768}\right\rfloor$ $ +9\left\lfloor\frac{24712}{432}\right\rfloor$ $ +10\left\lfloor\frac{24712}{240}\right\rfloor$ $ +11\left\lfloor\frac{24712}{132}\right\rfloor$ $ +12\left\lfloor\frac{24712}{72}\right\rfloor$ $ +13\left\lfloor\frac{24712}{39}\right\rfloor$ $ +14\left\lfloor\frac{24712}{14}\right\rfloor$  $ )=123470$
$ S(24711)=24711+2($ $ \left\lfloor\frac{24711}{12356}\right\rfloor$ $ +2\left\lfloor\frac{24711}{12356}\right\rfloor$ $ +3\left\lfloor\frac{24711}{9267}\right\rfloor$ $ +4\left\lfloor\frac{24711}{6176}\right\rfloor$ $ +5\left\lfloor\frac{24711}{3860}\right\rfloor$ $ +6\left\lfloor\frac{24711}{2316}\right\rfloor$ $ +7\left\lfloor\frac{24711}{1351}\right\rfloor$ $ +8\left\lfloor\frac{24711}{768}\right\rfloor$ $ +9\left\lfloor\frac{24711}{432}\right\rfloor$ $ +10\left\lfloor\frac{24711}{240}\right\rfloor$ $ +11\left\lfloor\frac{24711}{132}\right\rfloor$ $ +12\left\lfloor\frac{24711}{72}\right\rfloor$ $ +13\left\lfloor\frac{24711}{39}\right\rfloor$ $ +14\left\lfloor\frac{24711}{14}\right\rfloor$  $ )=123457$
$ S(24710)=24710+2($ $ \left\lfloor\frac{24710}{12355}\right\rfloor$ $ +2\left\lfloor\frac{24710}{12354}\right\rfloor$ $ +3\left\lfloor\frac{24710}{9264}\right\rfloor$ $ +4\left\lfloor\frac{24710}{6176}\right\rfloor$ $ +5\left\lfloor\frac{24710}{3860}\right\rfloor$ $ +6\left\lfloor\frac{24710}{2316}\right\rfloor$ $ +7\left\lfloor\frac{24710}{1351}\right\rfloor$ $ +8\left\lfloor\frac{24710}{768}\right\rfloor$ $ +9\left\lfloor\frac{24710}{432}\right\rfloor$ $ +10\left\lfloor\frac{24710}{240}\right\rfloor$ $ +11\left\lfloor\frac{24710}{132}\right\rfloor$ $ +12\left\lfloor\frac{24710}{72}\right\rfloor$ $ +13\left\lfloor\frac{24710}{39}\right\rfloor$ $ +14\left\lfloor\frac{24710}{14}\right\rfloor$  $ )=123456$


Hence the answer : $ \boxed{n=24710}$
\end{solution}



\begin{solution}[by \href{https://artofproblemsolving.com/community/user/29428}{pco}]
	\begin{tcolorbox}Define $ f$ on the positive integers by $ f(n) = k^2 + k + 1$.where $ n = 2^k(2l + 1)$ for some $ k, l$ nonnegative integers. 
Find the smallest $ n$ such that $ f(1) + f(2) + ... + f(n) \geq 123456$.\end{tcolorbox}
There is a simpler way :

Let $ S(n)=f(1)+f(2)+...+f(n)$ (strictly increasing sequence).

$ f(n+2^k)=f(n)$ $ \forall n<2^k$ and so $ S(n+2^k)=S(2^k)+S(n)$ $ \forall n<2^k$

So $ S(\sum_{n_i\ne n_j}2^{n_i})=\sum S(2^{n_i})$

The number of elements in $ [1,2^n]$ such that $ v_2(x)=k$ is $ 2^{n-1-k}$ $ \forall k\in [0,n-1]$

So $ S(2^n)=$ $ 2^{n-1}(0^2+0+1)+2^{n-2}(1^2+1+1)+$ $ 2^{n-3}(2^2+2+1) + ...$ $ +1((n-1)^2+(n-1)+1) + n^2+n+1$ $ =5\cdot 2^n-2n-4$ (use induction)

So :
$ S(1)=1$
$ S(2^{1})=5\cdot 2^{1}-2\cdot 1-4=4$
$ S(2^{2})=5\cdot 2^{2}-2\cdot 2-4=12$
$ S(2^{3})=5\cdot 2^{3}-2\cdot 3-4=30$
$ S(2^{4})=5\cdot 2^{4}-2\cdot 4-4=68$
$ S(2^{5})=5\cdot 2^{5}-2\cdot 5-4=146$
$ S(2^{6})=5\cdot 2^{6}-2\cdot 6-4=304$
$ S(2^{7})=5\cdot 2^{7}-2\cdot 7-4=622$
$ S(2^{8})=5\cdot 2^{8}-2\cdot 8-4=1260$
$ S(2^{9})=5\cdot 2^{9}-2\cdot 9-4=2538$
$ S(2^{10})=5\cdot 2^{10}-2\cdot 10-4=5096$
$ S(2^{11})=5\cdot 2^{11}-2\cdot 11-4=10214$
$ S(2^{12})=5\cdot 2^{12}-2\cdot 12-4=20452$
$ S(2^{13})=5\cdot 2^{13}-2\cdot 13-4=40930$
$ S(2^{14})=5\cdot 2^{14}-2\cdot 14-4=81888$
$ S(2^{15})=5\cdot 2^{15}-2\cdot 15-4=163806$

So :

The littlest number $ x$ such that $ S(x)\ge 123456$ is $ x=2^{14}+x_1$ where $ x_1$ is the littlest number such that $ S(x_1)\ge 123456-S(2^{14})=123456-81888=41568$
The littlest number $ x_1$ such that $ S(x_1)\ge 41568$ is $ x_1=2^{13}+x_2$ where $ x_2$ is the littlest number such that $ S(x_2)\ge 41568-S(2^{13})=41568-40930=638$
The littlest number $ x_2$ such that $ S(x_2)\ge 638$ is $ x_2=2^{7}+x_3$ where $ x_3$ is the littlest number such that $ S(x_3)\ge 638-S(2^{7})=638-622=16$
The littlest number $ x_3$ such that $ S(x_3)\ge 16$ is $ x_3=2^{2}+x_4$ where $ x_4$ is the littlest number such that $ S(x_4)\ge 16-S(2^{2})=16-12=4$
and so $ x_4=2$

And the required number is $ \boxed{2^{14}+2^{13}+2^{7}+2^2+2=24710}$ and we have $ S(24710)=123456$
\end{solution}
*******************************************************************************
-------------------------------------------------------------------------------

\begin{problem}[Posted by \href{https://artofproblemsolving.com/community/user/64682}{KDS}]
	Find all continuous functions $f: \mathbb R \to \mathbb R$ that satisfy \[ f(x + y) + f(xy) + 1 = f(x) + f(y) + f(xy + 1), \quad \forall x,y \in \mathbb R.\]
	\flushright \href{https://artofproblemsolving.com/community/c6h336764}{(Link to AoPS)}
\end{problem}



\begin{solution}[by \href{https://artofproblemsolving.com/community/user/29428}{pco}]
	\begin{tcolorbox}Find all continuous functions $ f: R \to R$ that satisfy $ f(x + y) + f(xy) + 1 = f(x) + f(y) + f(xy + 1)$ $ \forall x,y \in R$.\end{tcolorbox}

Let $ P(x,y)$ be the assertion $ f(x + y) +f(xy) + 1 = f(x)+f(y) + f(xy + 1)$

1) Let us solve the easier equation $ (E1)$ :
===========================
"Find all functions $ g(x)$ from $ \mathbb N\to\mathbb R$ such that : $ g(2x + y) - g(2x) - g(y) = g(2y + x) - g(2y) - g(x)$ $ \forall x,y\in\mathbb N$"

The set $ \mathbb S$ of solutions is a $ \mathbb R$-vector space.
Setting $ y = 1$, we get $ g(2x + 1) = g(2x) + g(1) + g(x + 2) - g(2) - g(x)$
Setting $ y = 2$, we get $ g(2x + 2) = g(2x) + g(2) + g(x + 4) - g(4) - g(x)$
From these two equations, we see that knowledge of $ g(1),g(2),g(3),g(4)$ and $ g(6)$ gives knowledge of $ g(x)$ $ \forall x\in\mathbb N$ and so dimension of $ \mathbb S$ is at most $ 5$.
But the $ 5$ functions below are independant solutions :
$ g_1(x) = 1$
$ g_2(x) = x$
$ g_3(x) = x^2$
$ g_4(x) = 1$ if $ x = 0\pmod 2$ and $ g_4(x) = 0$ if $ x\neq 0\pmod 2$
$ g_5(x) = 1$ if $ x = 0\pmod 3$ and $ g_5(x) = 0$ if $ x\neq 0\pmod 3$
And the general solution of $ (E1)$ is $ g(x) = a\cdot x^2 + b\cdot x + c + d\cdot g_4(x) + e\cdot g_5(x)$

2) Solutions of the original equation :
========================
$ P(x,0)$ $ \implies$ $ f(1)=1$
Comparing $ P(xy,z)$ and $ P(xz,y)$, we get $ Q(x,y,z)$ : $ f(xy + z) - f(xy) - f(z) = f(xz + y) - f(xz) - f(y)$

2.1) $ f(x)=ax^2+bx+c$ $ \forall x> 0$
--------------------------------------------
Let $ p$ a positive integer.
$ Q(2,\frac mp,\frac np)$ $ \implies$ $ f(\frac {2m + n}{p}) - f(\frac {2m}{p}) - f(\frac np) = f(\frac {2n + m}{p}) - f(\frac {2n}{p}) - f(\frac mp)$

So $ f(\frac xp)$ is a solution of $ (E1)$ and so $ f(\frac xp) = a_p\cdot x^2 + b_p\cdot x + c_p + d_p\cdot g_4(x) + e_p\cdot g_5(x)$ $ \forall x\in\mathbb N$
Choosing $ x = kp$, it's easy to see that $ a_p = \frac {a}{p^2}$, then that $ b_p = \frac bp$
Choosing $ x = 2kp$, $ x = 3kp$ and $ x = 6kp$, it's easy to see that $ c_p = c$ and $ d_p = e_p = 0$

And so $ f(\frac xp) = a(\frac xp)^2 + b(\frac xp) + c$ $ \forall x,p\in\mathbb N$
And so $ f(x) = ax^2 + bx + c$ $ \forall x\in\mathbb Q^{ + *}$

Now, $ f(x)$ continuous implies $ f(x) = ax^2 + bx + c$ $ \forall x\in\mathbb R^+$
Q.E.D.

2.2) $ f(x)=a'x^2+b'x+c'$ $ \forall x< 0$
----------------------------------------------
$ Q(2,-\frac mp,-\frac np)$ $ \implies$ $ f(-\frac {2m + n}{p}) - f(-\frac {2m}{p}) - f(-\frac np) = f(-\frac {2n + m}{p}) - f(-\frac {2n}{p}) - f(-\frac mp)$
So $ f(-\frac xp)$ is a solution of $ (E1)$ and the same method as in 2.1 above gives the result.

2.3) $ f(x)=ax^2+bx+1-a-b$ $ \forall x$
---------------------------------------------
We got $ f(x)=ax^2+bx+c$ $ \forall x>0$
and $ f(x)=a'x^2+b'x+c'$ $ \forall x<0$

Continuity at $ 0$ implies $ c=c'$ and $ f(1)=1$ implies $ c=1-a-b$
$ P(-1,-1)$ $ \implies$ $ a'=a$
$ P(-2,3)$ $ \implies$ $ b'=b$
Q.E.D

It is then easy to check back that this necessary form is indeed a solution and we got the result :

$ \boxed{f(x)=ax^2+bx+1-a-b}$ $ \forall x$
\end{solution}



\begin{solution}[by \href{https://artofproblemsolving.com/community/user/55721}{Thjch Ph4 Trjnh}]
	$ f(x+y)-f(x)-f(y)=f(xy+1)-f(xy)-1$.$ (1)$
Let $ x=-x, y=-y$:
$ f(-x-y)-f(-x)-f(-y)=f(xy+1)-f(xy)-1$.
Denote $ g(x)=f(x)-f(-x)$ then $ g(x+y)=g(x)+g(y)$.
$ g(x)$ is continuous so $ g(x)=ax$.
$ \Rightarrow f(x)-f(-x)=ax$.
Let $ y=-y$, we obtain:
$ f(x-y)-f(x)-f(y)+ay=f(xy-1)-a(xy-1)-f(xy)+axy-1$.
$ f(x-y)-f(x)-f(y)=f(xy-1)-f(xy)+a-ay-1$.$ (2)$
$ (1)$ and $ (2)$, we have:
$ f(x+y)-f(x-y)=f(xy+1)-f(xy-1)-a+ay$.
Denote $ h(x)=f(x)-\frac{a}{2}x$, then:
$ h(x+y)-h(x-y)=h(xy+1)-h(xy-1)$.
So $ h(x+y)-h(x-y)=h(2\sqrt{xy})-h(0)$, $ x,y\in R^+$.
$ u(x)=h(\sqrt{x})-h(0)$, $ x\in R^+$, then:
$ u((x+y)^2)-u((x-y)^2)=u(4xy)$, $ x,y\in R^+$.
$ u(x+y)=u(x)+u(y)$, hence $ u(x)=bx$, $ x\in R^+$.
Then $ h(x)=u(x^2)+h(0)=bx^2+c$, $ x\in R^+$.
But $ h(-x)=f(-x)-\frac{a}{2}(-x)=f(x)-\frac{a}{2}x=h(-x)$, so $ h(x)=bx^2+c, x\in R$.
$ \Rightarrow f(x)=h(x)+\frac{a}{2}x=bx^2+\frac{a}{2}x+c$.
\end{solution}



\begin{solution}[by \href{https://artofproblemsolving.com/community/user/62475}{hqthao}]
	can you tell me more, how can you think the function:$ g: N\rightarrow R$: $ g(2x+y)-g(2x)-g(y)=g(2y+x)-g(2y)-g(x)$, please.  :)  And, maybe something I still don't know in the way you solve $ g(x,y)$  :oops:
\end{solution}
*******************************************************************************
-------------------------------------------------------------------------------

\begin{problem}[Posted by \href{https://artofproblemsolving.com/community/user/67949}{aktyw19}]
	Find all functions $ f: \mathbb{R} \rightarrow \mathbb{R}$ such that \[  f(x)f(yf(x)-1)=x^2f(y)-f(x)\]
for all real numbers $x$ and $y$.
	\flushright \href{https://artofproblemsolving.com/community/c6h336781}{(Link to AoPS)}
\end{problem}



\begin{solution}[by \href{https://artofproblemsolving.com/community/user/29428}{pco}]
	\begin{tcolorbox}Find all functions $ f: \mathbb{R} \rightarrow \mathbb{R}$ such that $ f(x)f(yf(x) - 1) = x^2f(y) - f(x)$
for all real numbers x and y\end{tcolorbox}
Let $ P(x,y)$ be the assertion $ f(x)f(yf(x)-1)=x^2f(y)-f(x)$

$ P(x,0)$ $ \implies$ $ f(x)(1+f(-1))=x^2f(0)$

If $ f(-1)\ne -1$, this implies $ f(x)=ax^2$ and plugging this in original equation, this implies $ a=0$ and the solution $ f(x)=0$ $ \forall x$

So let us consider from now that $ f(x)$ is not all zero and so we get $ f(-1)=-1$ and $ f(0)=0$

1) Non all-zero solutions are surjective
==========================
If $ f(x)=0$, $ P(x,y)$ $ \implies$ $ x^2f(y)=0$ and so $ x=0$ else we would have the all-zero function.

$ P(x,x)$ $ \implies$ $ f(x)f(xf(x)-1)=x^2f(x)-f(x)$
$ P(-1,-xf(x))$ $ \implies$ $ f(xf(x)-1)=-f(-xf(x))-1$ and so $ f(x)f(xf(x)-1)=-f(x)f(-xf(x))-f(x)$ 

and so : $ f(x)f(-xf(x))=-x^2f(x)$ and so $ f(-xf(x))=-x^2$ $ \forall x$ (it's true for $ x=0$ and, if $ x\ne 0$, we saw that $ f(x)\ne 0$ and we can divide by $ f(x)$)

From there, we get that any nonpositive real belongs to $ f(\mathbb R)$
Let then $ x>0$ and $ u$ such that $ f(u)=-(x+1)$ :
$ P(-1,-u-1)$ $ \implies$ $ -f(u)-1=f(-u-1)$ and so $ x=f(u+1)$ and so any positive real belongs to $ f(\mathbb R)$
And $ f(x)$ is surjective.
Q.E.D.

2) For all non all-zero solutions, $ f(x+y)=f(x)+f(y)$ $ \forall x,y$
=========================================

$ P(-1,-yf(x))$ $ \implies$ $ f(yf(x)-1)=-f(-yf(x))-1$ and so $ f(x)f(yf(x)-1)=-f(x)f(-yf(x))-f(x)$
Comparing with $ P(x,y)$ we get $ x^2f(y)=-f(x)f(-yf(x))$
$ P(x,\frac 1{f(x)}-y)$ $ \implies$ $ f(x)f(-yf(x))=x^2f(\frac 1{f(x)}-y)-f(x)$

And so (using previous line) : $ -x^2f(y)=x^2f(\frac 1{f(x)}-y)-f(x)$
Setting $ y=0$ in this line  : $ 0=x^2f(\frac 1{f(x)})-f(x)$
Subtracting the two lines : $ -x^2f(y)=x^2f(\frac 1{f(x)}-y)-x^2f(\frac 1{f(x)})$

And so : $ -f(y)=f(\frac 1{f(x)}-y)-f(\frac 1{f(x)})$ $ \forall x\ne 0$
And, since $ f(x)$ is surjective, $ \frac 1{f(x)}$ can take any non zero value and we got :

$ f(z-y)=f(z)-f(y)$ $ \forall y,\forall z\ne 0$
Swapping $ y$ and $ z$, we show that $ f(-x)=-f(x)$ and so $ f(z-y)=f(z)-f(y)$ $ \forall z,y$
Q.E.D.

3) For all non all-zero solutions, $ f(x)f(\frac 1x)=1$ $ \forall x\ne 0$
============================================
Let $ x\ne 0$

$ P(-1,f(x))$ $ \implies$ $ f(-f(x)-1)=-f(f(x))-1$ and so $ f(x)f(-f(x)-1)=-f(x)f(f(x))-f(x)$
Comparing with $ P(x,-1)$ we get $ f(x)f(f(x))=x^2$ and so $ f(f(x))=\frac {x^2}{f(x)}$

Let then $ x\ne 0$ : $ f(x)\ne 0$ and so $ P(x,\frac 1{f(x)})$ $ \implies$ $ x^2f(\frac 1{f(x)})=f(x)$ and so $ f(\frac 1{f(x)})=\frac {f(x)}{x^2}$

An so $ f(f(x))f(\frac 1{f(x)})=1$ $ \forall x\ne 0$

And since $ f(x)$ is surjective : $ f(x)f(\frac 1x)=1$ $ \forall x\ne 0$
Q.E.D

4) The only non all-zero solution is $ f(x)=x$
=============================================
From $ f(0)=0$ and $ f(-1)=-1$, we get $ f(1)=1$

Let $ x\notin\{0,1\}$ : 

$ f(\frac 1x+\frac 1{1-x})=f(\frac 1x)+f(\frac 1{1-x})$ $ =\frac 1{f(x)}+\frac 1{f(1-x)}$ $ =\frac 1{f(x)}+\frac 1{1-f(x)}$ $ =\frac 1{f(x)-f(x)^2}$

But $ f(\frac 1x+\frac 1{1-x})=f(\frac 1{x-x^2})=\frac 1{f(x)-f(x^2)}$

And so $ f(x^2)=f(x)^2$ $ \forall x\notin\{0,1\}$ and this is still true for $ x=0$ or $ x=1$

So $ f(x)>0$ $ \forall x>0$  and $ f(x+y)=f(x)+f(y)$ implies then $ f(x)$ is an increasing function.

So we have an increasing solution of Cauchy equation and so $ f(x)=f(1)x=x$ which, indeed, is a solution.

5) Synthesis of solutions 
=========================
We got two solutions :

$ f(x)=0$ $ \forall x$
$ f(x)=x$ $ \forall x$
\end{solution}



\begin{solution}[by \href{https://artofproblemsolving.com/community/user/44753}{gilcu3}]
	If $ y=0$

$ (1+f(-1))f(x)=x^2f(0)$
If $ f(0)\neq 0$ then $ f(-1)\neq -1$ which is a contradiction when $ x=0$.

So $ f(0)=0$, then $ f(x)=0$ or $ f(-1)=-1$
If $ f(x)\neq 0$ then $ f(-1)=-1$
If $ x=y$ and $ f(x)\neq0$

Then $ f(xf(x)-1)=x^2-1$ (1)

We will prove that $ f$ is injective:

If $ f(x)=f(z)$
Then $ x^2f(y)=z^2f(y)$, so $ x^2=z^2$

Then in (1) $ f(xf(x)-1)=f(zf(z)-1)$, and $ (xf(x)-1)^2=(zf(z)-1)^2$, so $ x=z$ as claimed.
So the only zero of the function is $ f(0)$

Then in (1) changing $ x$ by $ -x$ we get:

$ f(xf(x)-1)= f(-xf(-x)-1)$
So $ xf(x)-1=-xf(-x)-1$, and $ -f(x)=f(-x)$ (4)



In the original equation change $ y$ by $ f(y)$

We get: $ f(x)f(f(y)f(x)-1)=x^{2}f(f(y))-f(x)$
Then $ f(f(y)f(x)-1)+1=\frac{x^{2}f(f(y))}{f(x)}$

Then $ \frac{x^{2}f(f(y))}{f(x)}=\frac{y^{2}f(f(x))}{f(y)}$

$ \frac{f(y)f(f(y))}{y^{2}}=\frac{f(x)f(f(x))}{x^{2}}=\frac{f(-1)f(f(-1))}{(-1)^{2}}=1$

(2) $ \frac{f(y)f(f(y))}{y^{2}}=1$

Then changing $ x$ by $ f(x)$ and $ y$ by $ f(x)$ and using (2):
We get $ f(x^2-1)=x^2-1$, which is equivalent to $ f(x)=x$ if we use (3).

The solutions are $ f(x)=x$ or $ f(x)=0$
\end{solution}



\begin{solution}[by \href{https://artofproblemsolving.com/community/user/29428}{pco}]
	\begin{tcolorbox} ...
(2) $ \frac {f(y)f(f(y))}{y^{2}} = 1$

Then changing $ x$ by $ f(x)$ and $ y$ by $ f(x)$ and using (2):
We get $ f(x^2 - 1) = x^2 - 1$\end{tcolorbox}

I get $ f(x^2-1)=f(x)^2-1$ and not $ f(x^2-1)=x^2-1$  :blush:
\end{solution}



\begin{solution}[by \href{https://artofproblemsolving.com/community/user/44753}{gilcu3}]
	Yes, you are right [color=blue]pco\end{underlined}[\/color]

We obtain (4) $ f(x^{2}-1)=f(x)^{2}-1$
In the original if we change $ x$ by $ -1$
we get: $ f(y+1)=y+1$ and using that in (4), we get

$ f(x)^{2}=x^2$ and we have $ f(x^{2}-1)=x^{2}-1$ again.
\end{solution}



\begin{solution}[by \href{https://artofproblemsolving.com/community/user/29428}{pco}]
	\begin{tcolorbox}Yes, you are right [color=blue]pco\end{underlined}[\/color]

We obtain (4) $ f(x^{2} - 1) = f(x)^{2} - 1$
In the original if we change $ x$ by $ - 1$
we get: $ f(y + 1) = y + 1$ and using that in (4), we get

$ f(x)^{2} = x^2$ and we have $ f(x^{2} - 1) = x^{2} - 1$ again.\end{tcolorbox}

That's OK :)

And much more simple than mine !  :blush: 
Congrats !
\end{solution}



\begin{solution}[by \href{https://artofproblemsolving.com/community/user/9049}{nsato}]
	\begin{tcolorbox}In the original if we change $ x$ by $ -1$
we get: $ f(y+1)=y+1$\end{tcolorbox}

I don't think so; I think you get $f(y + 1) = f(y) + 1$.  (If you get $f(y +1) = y + 1$, then you are already done.)
\end{solution}
*******************************************************************************
-------------------------------------------------------------------------------

\begin{problem}[Posted by \href{https://artofproblemsolving.com/community/user/76267}{tmath}]
	Find all functions $f: \mathbb R \to \mathbb R$ such that for all reals $x$ and $y$,
\[ f(x^3+y^3)=xf(x^2)+yf(y^2).\]
	\flushright \href{https://artofproblemsolving.com/community/c6h336995}{(Link to AoPS)}
\end{problem}



\begin{solution}[by \href{https://artofproblemsolving.com/community/user/57888}{baleanu}]
	It is problem 3 proposed at the National Mathematical Olympiad in Romania at 10 th grade in 2009.
\end{solution}



\begin{solution}[by \href{https://artofproblemsolving.com/community/user/29428}{pco}]
	\begin{tcolorbox}Find all functions $ f: R \rightarrow R$  satisfying for any real $ x,y$
$ f(x^3 + y^3) = xf(x^2) + yf(y^2)$\end{tcolorbox}
Setting $ y=0$ in the equation, we get $ f(x^3)=xf(x^2)$ and so $ f(x^3+y^3)=f(x^3)+f(y^3)$ and so the problem is equivalent to :

$ f(x+y)=f(x)+f(y)$ and $ f(x^3)=xf(x^2)$
Let then $ f(1)=a$

$ f((x+1)^3)=f(x^3+3x^2+3x+1)=f(x^3)+3f(x^2)+3f(x)+a$ $ =xf(x^2)+3f(x^2)+3f(x)+a$
$ f((x+1)^3)=(x+1)f((x+1)^2)=(x+1)(f(x^2)+2f(x)+a)$

Comparing these two lines gives $ f(x^2)=xf(x)+\frac{ax-f(x)}2$

Then, using this last equality : $ f(x^6)=x^3f(x^3)+\frac{ax^3-f(x^3)}2$ $ =x^4f(x^2)+\frac{ax^3-xf(x^2)}2$ 

And, using $ f(x^3)=xf(x^2)$ : $ f(x^6)=x^2f(x^4)$ $ =x^2(x^2f(x^2)+\frac{ax^2-f(x^2)}2)$ $ =x^4f(x^2)+\frac{ax^4-x^2f(x^2)}2$

And so $ ax^3-xf(x^2)=ax^4-x^2f(x^2)$ and so $ f(x^2)=ax^2$ $ \forall x\notin\{0,1\}$ and so $ f(x)=ax$ $ \forall x\ne 1>0$

This is still true for $ x=1$ (definition of $ a$)
And since $ f(x+y)=f(x)+f(y)$ implies $ f(0)=0$ and $ f(-x)=-f(x)$, we get $ \boxed{f(x)=ax \forall x}$ which, indeed, is a solution
\end{solution}



\begin{solution}[by \href{https://artofproblemsolving.com/community/user/62475}{hqthao}]
	maybe same with pco ( and I think my solution I read somewhere). just count $ f((x-1)^3+(x+1)^3)$ by two way.
\end{solution}



\begin{solution}[by \href{https://artofproblemsolving.com/community/user/30342}{nicetry007}]
	PCO's solution could be shortened a bit.

Suppose we have shown the following.
$ f(x+y)=f(x)+f(y)$, $ f(x^3)=xf(x^2)$ and 
$ f(x^2)=xf(x)+\frac{ax-f(x)}2 -----------(*)$
We note that $ f(-x^3) = -xf(x^2) = -f(x^3) \Rightarrow f(-x) = -f(x)$.
Setting $ x = -x$ in $ (*)$, we get 
$ f(x^2)=xf(x)-\frac{ax-f(x)}2 -----------(**)$.
Comparing $ (*)$ and $ (**)$, we get
$ f(x) = ax$.
\end{solution}



\begin{solution}[by \href{https://artofproblemsolving.com/community/user/29428}{pco}]
	\begin{tcolorbox}PCO's solution could be shortened a bit.

Suppose we have shown the following.
$ f(x + y) = f(x) + f(y)$, $ f(x^3) = xf(x^2)$ and 
$ f(x^2) = xf(x) + \frac {ax - f(x)}2 - - - - - - - - - - - (*)$
We note that $ f( - x^3) = - xf(x^2) = - f(x^3) \Rightarrow f( - x) = - f(x)$.
Setting $ x = - x$ in $ (*)$, we get 
$ f(x^2) = xf(x) - \frac {ax - f(x)}2 - - - - - - - - - - - (**)$.
Comparing $ (*)$ and $ (**)$, we get
$ f(x) = ax$.\end{tcolorbox}

Quite true! Thanks   :)
\end{solution}



\begin{solution}[by \href{https://artofproblemsolving.com/community/user/93393}{r31415}]
	Sorry for the revival, but I believe that I have a solution:
Set y=0 and $f(x^3)=xf(x^2)$. Therefore $\frac {f(x^3)} {f(x^2)} =x$, $\frac {f(x^2)} {f(x^{4\/3})} = x^{2\/3}$, etc. Multiplying these, we eventually get $\frac {f(x^3)} {f(x^0)}=x^{1+2\/3+4\/9+ \cdots}$, or $f(x^3)=x^3f(1)$, and letting $f(x)=c$, $f(x)=cx$ for real c.
\end{solution}



\begin{solution}[by \href{https://artofproblemsolving.com/community/user/45928}{auj}]
	\begin{tcolorbox}Sorry for the revival, but I believe that I have a solution:
Set y=0 and $f(x^3)=xf(x^2)$. Therefore $\frac {f(x^3)} {f(x^2)} =x$, $\frac {f(x^2)} {f(x^{4\/3})} = x^{2\/3}$, etc. Multiplying these, we eventually get $\frac {f(x^3)} {f(x^0)}=x^{1+2\/3+4\/9+ \cdots}$, or $f(x^3)=x^3f(1)$, and letting $f(x)=c$, $f(x)=cx$ for real c.\end{tcolorbox}

Dear "r31415":
Sorry for my disbelieving comment and question.
But, please justify your \begin{italicized}"we eventually get ..."\end{italicized}.
Thanks,
auj
\end{solution}



\begin{solution}[by \href{https://artofproblemsolving.com/community/user/29428}{pco}]
	You need continuity for this conclusion, and you dont have.
\end{solution}



\begin{solution}[by \href{https://artofproblemsolving.com/community/user/64716}{mavropnevma}]
	Another simplification of \begin{bolded}pco\end{bolded}'s solution (in fact the official solution) is, once we get the equation $ f(x^2)=xf(x)+\frac{ax-f(x)}2$, to also write it again for $x\mapsto x+1$. Solving for $f(x)$ the system made by these two equations yields the result.

As for the comment above - and once we assume continuity the problem would become trivial, since $f$ was shown to be a solution to Cauchy's equation $f(x+y) = f(x) + f(y)$ ...
\end{solution}



\begin{solution}[by \href{https://artofproblemsolving.com/community/user/93393}{r31415}]
	\begin{tcolorbox}[quote="r31415"]Sorry for the revival, but I believe that I have a solution:
Set y=0 and $f(x^3)=xf(x^2)$. Therefore $\frac {f(x^3)} {f(x^2)} =x$, $\frac {f(x^2)} {f(x^{4\/3})} = x^{2\/3}$, etc. Multiplying these, we eventually get $\frac {f(x^3)} {f(x^0)}=x^{1+2\/3+4\/9+ \cdots}$, or $f(x^3)=x^3f(1)$, and letting $f(x)=c$, $f(x)=cx$ for real c.\end{tcolorbox}

Dear "r31415":
Sorry for my disbelieving comment and question.
But, please justify your \begin{italicized}"we eventually get ..."\end{italicized}.
Thanks,
auj\end{tcolorbox}
It's a telescoping series, with the denominator $f(x^n)$, and n decreasing towards 0 every time. Eventually we get to $f(x^0)=f(1)$.
\end{solution}



\begin{solution}[by \href{https://artofproblemsolving.com/community/user/64716}{mavropnevma}]
	Precisely. Having $x_n = x^{1\/n}$, with $\lim_{n\to \infty} x_n = \lim_{n\to \infty} x^{1\/n} = x^0 = 1$ does not warrant $\lim_{n\to \infty} f(x_n) = f(\lim_{n\to \infty} x_n) = f(1)$, unless $f$ is known to be continuous at $1$.

And $(1-1) + (1-1) + \cdots + (1-1) + \cdots $ also seems to be a telescoping series, and still its value \begin{bolded}is not \end{bolded}$1$. One needs convergence before blindly telescoping infinite sums or products.
\end{solution}



\begin{solution}[by \href{https://artofproblemsolving.com/community/user/45928}{auj}]
	\begin{tcolorbox}Precisely. Having $x_n = x^{1\/n}$, with $\lim_{n\to \infty} x_n = \lim_{n\to \infty} x^{1\/n} = x^0 = 1$ does not warrant $\lim_{n\to \infty} f(x_n) = f(\lim_{n\to \infty} x_n) = f(1)$, unless $f$ is known to be continuous at $1$.

And $(1-1) + (1-1) + \cdots + (1-1) + \cdots $ also seems to be a telescoping series, and still its value \begin{bolded}is not \end{bolded}$1$. One needs convergence before blindly telescoping infinite sums or products.\end{tcolorbox}

Good morning & thanks, "mavropnevma", for getting it to the point!
Regards, auj
\end{solution}



\begin{solution}[by \href{https://artofproblemsolving.com/community/user/187896}{Ashutoshmaths}]
	\begin{tcolorbox}Find all functions $ f: R \rightarrow R$  satisfying for any real $ x,y$

$ f(x^3+y^3)=xf(x^2)+yf(y^2)$\end{tcolorbox}
Please point out any flaw in my solution:
Let $P(x,y)$ be the assertion.
$P(0,0)\implies f(0)=0$.
$P(x,0)\implies f(x^3)=xf(x^2)\cdots\cdots\ast\ast$
Hence $f(x^3+y^3)=f(x^3)+f(y^3)\implies f(x+y)=f(x)+f(y)$
Let $n\in\mathbb{N}$,$f(nx)=f(x+x+\cdots \text{ n times })=nf(x)\cdots\cdots \ast$. 
$f((x+y)^3)=f(x^3+y^3+3x^2y+3xy^2)=f(x^3)+f(y^3)+ f(3x^2y)+(3xy^2)$
$=f(x^3)+f(y^3)+ 3f(x^2y)+3(xy^2)$(from $\ast$)
$=xf(x^2)+yf(y^2)+3f(x^2y)+3(xy^2)$
But 
$f((x+y)^3))$
$=(x+y)f((x+y)^2)$
$=(x+y)f(x^2+2xy+y^2)$
$=(x+y)(f(x^2)+2f(xy)+f(y^2))$
$=xf(x^2)+yf(y^2)+yf(x^2)+xf(y^2)+f(2xy)(x+y)$
$=xf(x^2)+yf(y^2)+yf(x^2)+xf(y^2)+2f(xy)(x+y)$(from $\ast$)
Hence $=xf(x^2)+yf(y^2)+3f(x^2y)+3(xy^2)$
$=xf(x^2)+yf(y^2)+yf(x^2)+xf(y^2)+2f(xy)(x+y)$
$\implies 3f(x^2y)+3f(xy^2)=yf(x^2)+xf(y^2)+2f(xy)(x+y)$
From $\ast\ast$,
we get,
$3f(x^2y)+3f(xy^2)=y\left(\frac{f(x^3)}{x}\right)+x\left(\frac{f(y^3)}{y}\right)+2f(xy)(x+y)$
$\implies 3xyf(x^2y)+3xyf(xy^2)=y^2f(x^3)+x^2f(y^3)+2xyf(xy)(x+y)$
$P(x,1)\implies 3xf(x^2)+3xf(x)=f(x^3)+x^2f(1)+2xf(x)(x+1)$
$\implies 2xf(x^2)+3xf(x)=x^2f(1)+2xf(x)(x+1)$
$\implies 2xf(x^2)+3xf(x)=x^2f(1)+2x^2f(x)+2xf(x)$
$\implies 2xf(x^2)+xf(x)=x^2f(1)+2x^2f(x)$
$\implies xf(x)=x^2f(1)\implies x(f(x)-f(1)x)=0$
Hence $f(x)=cx$,$c=f(1)$(for $x\neq 0)$
Plugging this in the original equation,we get $f(x)=cx\forall x\in\mathbb{R},c=f(1)$ or $f(x)=0\forall x\in\mathbb{R}$
It is easy to check that $x=0$ also satisfies.$\blacksquare$
\end{solution}



\begin{solution}[by \href{https://artofproblemsolving.com/community/user/294255}{Ryoma.Echizen}]
	Can we say if $f(x+y)=f(x)+f(y)\forall x,y$ and $f(x^n)=xf(x^{n-1})$ then $f(x)=ax$
\end{solution}



\begin{solution}[by \href{https://artofproblemsolving.com/community/user/29428}{pco}]
	\begin{tcolorbox}Can we say if $f(x+y)=f(x)+f(y)\forall x,y$ and $f(x^n)=xf(x^{n-1})$ then $f(x)=ax$\end{tcolorbox}
Yes.
Consider $k\in\mathbb Q$ and write $f((x+k)^n)=(x+k)f((x+k)^{n-1})$
This is a polynomial in $k$ with infinitely many roots (all rational numbers) and so all coefficients are zero.
Write this constraint for coefficient of $k^{n-1}$ and you get $f(x)=xf(1)$
\end{solution}



\begin{solution}[by \href{https://artofproblemsolving.com/community/user/335559}{Duarti}]
	\begin{tcolorbox}[quote=Ryoma.Echizen]Can we say if $f(x+y)=f(x)+f(y)\forall x,y$ and $f(x^n)=xf(x^{n-1})$ then $f(x)=ax$\end{tcolorbox}
Yes.
Consider $k\in\mathbb Q$ and write $f((x+k)^n)=(x+k)f((x+k)^{n-1})$
This is a polynomial in $k$ with infinitely many roots (all rational numbers) and so all coefficients are zero.
Write this constraint for coefficient of $k^{n-1}$ and you get $f(x)=xf(1)$\end{tcolorbox}
I do not understand this, can you explain better and slowly please?


\end{solution}



\begin{solution}[by \href{https://artofproblemsolving.com/community/user/29428}{pco}]
	\begin{tcolorbox}I do not understand this, can you explain better and slowly please?\end{tcolorbox}
What is the first line you dont understand ?


\end{solution}
*******************************************************************************
-------------------------------------------------------------------------------

\begin{problem}[Posted by \href{https://artofproblemsolving.com/community/user/60032}{Stephen}]
	Let $ f, g: \mathbb R \to \mathbb R$ be continuous bijective functions which satisfy \[ f(g^{ - 1}(x)) + g(f^{ - 1}(x)) = 2x\] for all $ x \in \mathbb R$. If there exists a real number $ x_0$ such that $ f(x_0) = g(x_0)$, prove that $ f = g$.
	\flushright \href{https://artofproblemsolving.com/community/c6h337001}{(Link to AoPS)}
\end{problem}



\begin{solution}[by \href{https://artofproblemsolving.com/community/user/29428}{pco}]
	\begin{tcolorbox}$ f, g: R\rightarrow R$ is a bijective functions that satisfies $ f(g^{ - 1}(x)) + g(f^{ - 1}(x)) = 2x$ for all $ x \in R$.

If there exists a real number $ x_0$ that satisfies $ f(x_0) = g(x_0)$, prove that $ f = g$.\end{tcolorbox}

Wrong.

Counter-example :

$ f(x)=x$ $ \forall x\in\mathbb Z$ and $ f(x)=x+1$ $ \forall x\notin\mathbb Z$
$ g(x)=x$

So $ f^{[-1]}(x)=x$ $ \forall x\in\mathbb Z$ and $ f^{[-1]}(x)=x-1$ $ \forall x\notin\mathbb Z$

So $ f(g^{[-1]}(x))+g(f^{[-1]}(x))$ $ =f(x)+f^{[-1]}(x)=2x$ $ \forall x$ and $ f(0)=g(0)$ but these two bijections are not identical.
\end{solution}



\begin{solution}[by \href{https://artofproblemsolving.com/community/user/60032}{Stephen}]
	Oh, sorry. $ f, g$ are continuous functions. :blush:
\end{solution}



\begin{solution}[by \href{https://artofproblemsolving.com/community/user/29428}{pco}]
	\begin{tcolorbox}Oh, sorry. $ f, g$ are continuous functions. :blush:\end{tcolorbox}

Ohhh :(

The problem is then equivalent (taking $ h(x)=f(g^{-1}(x)))$ to show that the only continuous bijection $ h(x)$ such that $ h(x)+h^{-1}(x)=2x$ $ \forall x$ and $ h(x_1)=x_1$ for some real $ x_1$ is $ h(x)=x$

The equation is equivalent to $ h(h(x))+x=2h(x)$ and so $ h(x)$ and $ h^{-1}(x)$ both are increasing.

If $ h\ne Id$, then let $ u$ such that $ h(u)\ne u$ and WLOG say $ h(u)>u$ (else swap $ h(x)$ and $ h^{-1}(x)$)

Let $ a=h(u)-u$ with $ a>0$

$ h(h(u))=2h(u)-u$ implies $ h(u+a)=(u+a)+a$ and : $ \forall x\in(u,u+a)$ : $ h(x)>h(u)=u+a>x$ and so $ h(x)>x$ $ \forall x\ge u$

$ h^{-1}(u)=2u-h(u)=u-a$ and so $ h(u-a)=(u-a)+a$ and : $ \forall x\in(u-a,u)$ : $ h(x)>h(u-a)=u>x$ and so $ h(x)>x$ $ \forall x\le u$

And so $ h\ne Id$ $ \implies$ $ h(x)\ne x$ $ \forall x$
Q.E.D.
\end{solution}



\begin{solution}[by \href{https://artofproblemsolving.com/community/user/30342}{nicetry007}]
	I do not see how $ h(x)$ and $ h^{-1}(x)$ are both increasing. I would agree that $ h(x) + h^{-1}(x)$ is increasing. Could you please elaborate.

thanks
\end{solution}



\begin{solution}[by \href{https://artofproblemsolving.com/community/user/29428}{pco}]
	\begin{tcolorbox}I do not see how $ h(x)$ and $ h^{ - 1}(x)$ are both increasing. I would agree that $ h(x) + h^{ - 1}(x)$ is increasing. Could you please elaborate.

thanks\end{tcolorbox}

$ h(x)$ is monotonic (since bijective and continuous)
So $ h(h(x))$ is increasing
So $ 2h(x)=h(h(x))+x$ is increasing

And $ h(x)$ increasing implies $ h^{-1}(x)$ increasing
\end{solution}



\begin{solution}[by \href{https://artofproblemsolving.com/community/user/30342}{nicetry007}]
	Thanks for clarifying. In hindsight that was a stupid question. Totally forgot the fact that $ h(x)$ was a bijection.
\end{solution}
*******************************************************************************
-------------------------------------------------------------------------------

\begin{problem}[Posted by \href{https://artofproblemsolving.com/community/user/56873}{duythuc_lqd}]
	Find all functions $ f: \mathbb Z \times \mathbb Z\to \mathbb R$ which satisfy the following conditions:
1. For all integers $x,y,$ and $z$,
\[ f\left ( x, y \right) \cdot f\left ( y, z \right) \cdot f \left (z, x \right) = 1.\]
2. For all integers $x$, 
\[ f\left ( x+1,x \right )=2.\]
	\flushright \href{https://artofproblemsolving.com/community/c6h337198}{(Link to AoPS)}
\end{problem}



\begin{solution}[by \href{https://artofproblemsolving.com/community/user/29428}{pco}]
	\begin{tcolorbox}Find all functions $ f: Z\texttt{x}Z\rightarrow R$ satisfy two conditions:
1. $ f\left ( x;y \right ).f\left ( y;z \right ).f\left ( z;x \right ) = 1, \forall x,y,z\in Z$
2. $ f\left ( x + 1;x \right ) = 2,\forall x\in Z$\end{tcolorbox}

Using $ x=y=z$ in 1. implies $ f(x,x)=1$

Using $ z=x$ in 1. implies $ f(y,x)=\frac 1{f(x,y)}$

And so 1. implies $ f(x,y)f(yz)=f(x,z)$

Using then 2. and a simple induction gives $ f(x+n,x)=2^n$ $ \forall n\in\mathbb Z$

And so $ \boxed{f(x,y)=2^{x-y}}$ $ \forall x,y$ which, indeed, is a solution.
\end{solution}
*******************************************************************************
-------------------------------------------------------------------------------

\begin{problem}[Posted by \href{https://artofproblemsolving.com/community/user/50550}{malcolm}]
	Find all functions $f$ from the set $\mathbb{R}$ of real numbers into $\mathbb{R}$ which satisfy for all $x, y, z \in \mathbb{R}$ the identity \[f(f(x)+f(y)+f(z))=f(f(x)-f(y))+f(2xy+f(z))+2f(xz-yz).\]
	\flushright \href{https://artofproblemsolving.com/community/c6h337225}{(Link to AoPS)}
\end{problem}



\begin{solution}[by \href{https://artofproblemsolving.com/community/user/29428}{pco}]
	\begin{tcolorbox}Find all functions $ f$ from the set $ R$ of real numbers into $ R$ which satisfy $ \forall x,y\in R$ the identity
$ f(f(x) + f(y) + f(z)) = f(f(x) - f(y)) + f(2xy + f(z)) + 2f(xz - yz)$.\end{tcolorbox}

Let $ P(x,y,z)$ be the assertion $ f(f(x) + f(y) + f(z)) = f(f(x) - f(y)) + f(2xy + f(z)) + 2f(xz - yz)$

The only constant solution is $ f(x)=0$ $ \forall x$
So we'll from now consider non constant solutions.

1) $ f(0)=0$
===========
I'm sure there is a shorter method for this part.

1.1) If $ f(z_1)=f(z_2)$, then $ f(xz_1)=f(xz_2)$ $ \forall x$
-----------------------------------------------------------
Let $ a=f(z_1)=f(z_2)$
$ P(x,0,z_1)$ $ \implies$ $ f(f(x) + f(0) + a) = f(f(x) - f(0)) + f(a) + 2f(xz_1)$
$ P(x,0,z_2)$ $ \implies$ $ f(f(x) + f(0) + a) = f(f(x) - f(0)) + f(a) + 2f(xz_2)$
Subtracting gives the required result.
Q.E.D.

1.2) $ f(-x)=f(x)$ $ \forall x$
-----------------------------
Since $ f(x)$ is not constant, let $ x\ne y$ such that $ f(x)-f(y)=b\ne 0$
$ P(x,y,0)$ $ \implies$ $ f(f(x) + f(y) + f(0)) = f(b) + f(2xy + f(0)) + 2f(0)$
$ P(y,x,0)$ $ \implies$ $ f(f(x) + f(y) + f(0)) = f(-b) + f(2xy + f(0)) + 2f(0)$
Subtracting implies $ f(b)=f(-b)$ and then 1.1) gives $ f(bx)=f(-bx)$ $ \forall x$ and since $ b\ne 0$, we get the required result.
Q.E.D.

1.3) $ f(z_1)=f(z_2)$ $ \implies$ $ z_1=\pm z_2$
--------------------------------------------
If $ f(z_1)=f(z_2)$ :
$ P(xz_2,yz_1,1)$ $ \implies$ $ f(f(xz_2) + f(yz_1) + f(1)) = f(f(xz_2) - f(yz_1)) + f(2xyz_1z_2 + f(1)) + 2f(xz_2 - yz_1)$
$ P(xz_1,yz_2,1)$ $ \implies$ $ f(f(xz_1) + f(yz_2) + f(1)) = f(f(xz_1) - f(yz_2)) + f(2xyz_1z_2 + f(1)) + 2f(xz_1 - yz_2)$
Subtracting (and using 1.1) gives $ f(xz_1 - yz_2)=f(xz_2 - yz_1)$ $ \forall x,y$

If $ z_1^2\ne z_2^2$, the system $ xz_1 - yz_2=u$ and $ xz_2 - yz_1=v$ always has solution and $ f(u)=f(v)$ $ \forall u,v$
And so, since we are considering non constant solutions, $ z_1^2=z_2^2$
Q.E.D.

1.4) $ f(0)=0$
-----------------
We know (using 1.2) that $ f(1)=f(-1)$. Then 
$ P(1,1,0)$ $ \implies$ $ f(2f(1) + f(0)) = f(2 + f(0)) + 3f(0)$
$ P(1,-1,0)$ $ \implies$ $ f(2f(1) + f(0)) = f(-2 + f(0)) + 3f(0)$
So $ f(2+f(0))=f(-2+f(0))$ and so (using 1.3)  : either $ 2+f(0)=-2+f(0)$, either $ 2+f(0)=2-f(0)$ and so $ f(0)=0$
Q.E.D.

2) $ f(x)=x^2$ $ \forall x$
==============
$ P(x,x,0)$ $ \implies$ $ f(2f(x))=f(2x^2)$ and so, using 1.3) : $ \forall x$ : either $ f(x)=x^2$, either $ f(x)=-x^2$

$ P(x,x,x)$ $ \implies$ $ f(3f(x))=f(2x^2+f(x))$ and so, using 1.3) : $ \forall x$ : either $ f(x)=x^2$, either $ f(x)=-\frac{x^2}2$
Hence the result.


3) synthesis of solutions
================
It's easy to check back that the required form $ f(x)=x^2$ for non constant solutions is indeed a solution and so we got two solutions :

$ f(x)=0$ $ \forall x$
$ f(x)=x$ $ \forall x$
\end{solution}



\begin{solution}[by \href{https://artofproblemsolving.com/community/user/35129}{Zhero}]
	[hide]We claim that the only solutions are $ f(x) = 0$ and $ f(x) = x^2$. Let $ P(x,y,z)$ be the assertion that $ f(f(x) + f(y) + f(z)) = f(f(x) - f(y)) + f(2xy + f(z)) + 2f(xz - yz)$.

\begin{bolded}Lemma 1\end{bolded}: For all real $ x$, $ f(x) = f( - x)$.
\begin{italicized}Proof:\end{italicized} $ P(x,y,z)$ and $ P(y,x,z)$ give $ f(f(x) + f(y) + f(z)) - f(2xy + f(z)) = f(f(x) - f(y)) + 2f(xz - yz) = f(f(y) - f(x)) + 2f(yz - xz)$, that is, $ f(f(x) - f(y)) + 2f(xz - yz) = f(f(y) - f(x)) + 2f(yz - xz)$ for all real $ x$, $ y$, and $ z$. Setting $ x = 2$ and $ y = 1$ here reveals that $ f(z) - f( - z) = (f(f(1) - f(2)) - f(f(2) - f(1)))\/2 = m$ for all real $ z$. Hence, $ 0 = [f(f(x) - f(y)) - f( - (f(y) - f(x)))] + 2[f(xz - yz) - f( - (xz - yz)] = 3m$, so $ m = 0$. Therefore, $ f(z) = f( - z)$ for all real $ z$.

\begin{bolded}Lemma 2\end{bolded}: $ f(0) = 0$.
\begin{italicized}Proof:\end{italicized} Let $ c = f(0)$. $ P(x,x,0)$ reveals that $ f(2f(x) + c) = c + f(2x^2 + c) + 2c = 3c + f(2x^2 + c)$, while $ P(x, - x,0)$ reveals that $ f(2f(x) + c) = f(f(x) + f( - x) + c) = c + f(c - 2x^2) + 2c = 3c + f(c - 2x^2)$, that is, $ f(2x^2 + c) = f(c - 2x^2)$ for all real $ x$. Setting $ x = \sqrt {|c|}$ reveals that $ f(3c) = f( - c) = f(c)$.

Since $ f(2f(x) + c) = 3c + f(2x^2 + c)$, setting $ x = 0$ reveals that $ f(3c) = 3c + f(c)$. But since $ f(3c) = f(c)$, we have that $ 3c = 0$, that is, $ c = 0$, which completes the proof of our lemma.

If $ f$ is constant, since $ f(0) = 0$, we must have that $ f(x) = 0$ for all real $ x$. Hence, we may now assume that $ f$ is nonconstant.

\begin{bolded}Lemma 3\end{bolded}: If $ f(a) = f(b)$, then for any real $ k$, $ f(ka) = f(kb)$.
\begin{italicized}Proof:\end{italicized} Let $ f(a) = f(b) = n$. The assertions $ P(k + 1,1,a)$ and $ P(k + 1,1,b)$ reveal that $ f(f(k + 1) + f(1) + n) - f(f(k + 1) - f(1)) - f(2(k + 1)(1) + n) = 2f(ka) = 2f(kb)$, which clearly completes our proof.

\begin{bolded}Corollary 1\end{bolded}: If $ f(x) = 0$, then $ x = 0$.
\begin{italicized}Proof:\end{italicized} If $ f(x) = 0$, then $ f(kx) = f(x) = 0$ for all real $ k$. If $ x$ is nonzero, then the set $ \{kx\}$ spans all reals, so $ f(y) = 0$ for all real $ y$. Therefore, $ f$ is constant, which is a contradiction.

\begin{bolded}Lemma 4\end{bolded}: $ f(f(x)) = f(x^2)$ for all real $ x$.
\begin{italicized}Proof:\end{italicized} $ P(x,x,0)$ reveals that $ f(2f(x)) = 0 + f(2x^2) + 0 = f(2x^2)$ for all real $ x$. By lemma 3, we have that $ f(x^2) = f(f(x))$.

\begin{bolded}Lemma 5\end{bolded}: $ f(f(1)x) = f(x)$ for all real $ x$.
\begin{italicized}Proof:\end{italicized} $ P(x^2,1,0)$ and $ P(f(x),f(1),0)$, by lemma 4, yield $ f(f(x^2) + f(1)) - f(f(x^2) - f(1)) = f(2x^2) = f(2f(x)f(1))$. Hence, by lemma 3, $ f(x^2) = f(f(x) f(1))$. By lemmas 3 and 4, $ f(f(x)) = f(x^2)$ implies that $ f(f(1)f(x)) = f(f(1)x^2)$, that is, $ f(x^2) = f(f(1)x^2)$. Since $ f$ is even, $ f( - x^2) = f(f(1)( - x^2))$ as well. Since the set of all $ x^2$ and the set of all $ - x^2$ together span the reals, we may conclude that $ f(f(1)x) = f(x)$ for all real $ x$.

\begin{bolded}Lemma 6\end{bolded}: $ f(1) = 1$.
\begin{italicized}Proof:\end{italicized} Since $ f$ is nonconstant, we can find some $ z$ such that $ f(z) \neq 0$. If we let $ y = \frac { - f(z)}{2}$, then $ P(1,y, z)$ and $ P(f(1),f(1)y, z)$, combined with lemmas 3, 4, and 5 yield $ f(f(1) + f(y) + f(z)) - f(f(1) - f(y)) = f(2y + f(z)) + f(z - yz) = f(2yf(1)^2 + f(z)) + 2f(f(1)z(1 - y)) = f(2yf(1)^2 + f(z)) + 2f(z(1 - y))$. Hence, $ f(2yf(1)^2 + f(z)) = f(2y + f(z))$. But since $ y = \frac { - f(z)}{2}$, we have that $ 2y + f(z) = 0$. Hence, $ f(2y + f(z)) = 0$. But since $ f(2yf(1)^2 + f(z)) = 0$, by corollary 1, we must have that $ 2yf(1)^2 + f(z) = 0$. Substitution reveals that $ f(z)(f(1)^2 - 1) = 0$. Since we chose $ z$ such that $ f(z) \neq 0$, $ f(1)^2 = 1$. Hence, either $ f(1) = 1$ or $ f(1) = - 1$.

If $ f(1) = - 1$, then $ f( - 1) = - 1$ by lemma 1. $ P(1,0,1)$ yields $ f( - 2) = - 4$ (and hence, by lemma 1, that $ f(2) = - 4$.) $ P(\sqrt {2}, \sqrt {2}, 0)$ and lemma 4 gives $ f(2 \sqrt {2}) = f(2) + f(2) + 2f(2)$, while $ P(\sqrt {2}, 0, \sqrt {2})$ gives $ f(2 \sqrt {2}) = f(4)$. Hence, $ - 16 = 4f(2) = f(4)$. $ P(2,0,1)$ therefore gives $ f(3) = f(4) + f(1) + 2f(2) = - 9$. On the other hand, $ P(1,1,1)$ gives $ f(3) = - 1$, which is a contradiction. It follows that $ f(1) = 1$, which completes the proof of this lemma.

\begin{bolded}Corollary 2\end{bolded}: $ f(2f(x)^2 + f(x)) = 2f(x^2 f(x) + f(x))$.
\begin{italicized}Proof:\end{italicized} $ P(x,x,1)$ yields $ f(2f(x) + 1) = f(2x^2 + 1)$. By lemma 3, $ f(2f(x)^2 + f(x)) = 2f(x^2 f(x) + f(x))$.

------------------------------------------------------------------------------------

$ P(x^2,f(x),x)$ yields $ f(2f(x^2) + f(x)) = f(2x^2 f(x) + f(x)) + 2f(x(x^2 - f(x))$. $ P(f(x), f(x), x)$ yields $ f(2f(x^2) + f(x)) = f(2f(x)^2 + f(x))$. Therefore, $ f(2f(x)^2 + f(x)) = f(2x^2 f(x) + f(x)) + 2f(x(x^2 - f(x)))$. By corollary 2, $ f(2f(x)^2 + f(x)) = f(2x^2 f(x) + f(x))$, so $ 2f(x(x^2 - f(x)) = 0$. If $ x \neq 0$, by corollary 1, we must have that $ f(x) = x^2$. If $ x = 0$, then $ f(0) = 0 = 0^2$. Therefore, $ f(x) = x^2$ for all real $ x$, as desired.[\/hide]

pco, I must say that I found your step 1.3 amazing. It completely blew away my solution (which is probably flawed somewhere)  :D
\end{solution}



\begin{solution}[by \href{https://artofproblemsolving.com/community/user/29428}{pco}]
	Thanks.

I carefully read your proof and found one or two typos and one error (for which a correction surely exists) but I do agree with you proof.

Here are my remarks :

\begin{tcolorbox}\begin{bolded}Lemma 6\end{bolded}: $ f(1) = 1$.
\begin{italicized}Proof:\end{italicized} Since $ f$ is nonconstant, we can find some $ z$ such that $ f(z) \neq 0$. If we let $ y = \frac { - f(z)}{2}$, then $ P(1,y, z)$ and $ P(f(1),f(1)y, z)$, combined with lemmas 3, 4, and 5 yield $ f(f(1) + f(y) + f(z)) - f(f(1) - f(y)) = f(2y + f(z)) + f(z - yz) = f(2yf(1)^2 + f(z)) + 2f(f(1)z(1 - y)) = f(2yf(1)^2 + f(z)) + 2f(z(1 - y))$. \end{tcolorbox}

Typo : $ f(f(1) + f(y) + f(z)) - f(f(1) - f(y)) = f(2y + f(z)) + 2f(z - yz)$

\begin{tcolorbox} \begin{bolded}Lemma 6\end{bolded}: ...
If $ f(1) = - 1$, then $ f( - 1) = - 1$ by lemma 1. $ P(1,0,1)$ yields $ f( - 2) = - 4$ (and hence, by lemma 1, that $ f(2) = - 4$.) $ P(\sqrt {2}, \sqrt {2}, 0)$ and lemma 4 gives $ f(2 \sqrt {2}) = f(2) + f(2) + 2f(2)$, while $ P(\sqrt {2}, 0, \sqrt {2})$ gives $ f(2 \sqrt {2}) = f(4)$. Hence, $ - 16 = 4f(2) = f(4)$. $ P(2,0,1)$ therefore gives $ f(3) = f(4) + f(1) + 2f(2) = - 9$. On the other hand, $ P(1,1,1)$ gives $ f(3) = - 1$, which is a contradiction. It follows that $ f(1) = 1$, which completes the proof of this lemma.\end{tcolorbox}
Typo : 
$ P(\sqrt {2}, \sqrt {2}, 0)$  gives $ f(2 \sqrt {2}) = f(4)$ 
$ P(\sqrt {2}, 0, \sqrt {2})$ gives $ f(2 \sqrt {2}) = f(2) + f(2) + 2f(2)$

Error :
$ P(2,0,1)$  gives $ f(5) = f(4) + f(1) + 2f(2) = - 25$
So this does not imply immediately the contradiction

\begin{tcolorbox}\begin{bolded}Corollary 2\end{bolded}: $ f(2f(x)^2 + f(x)) = 2f(x^2 f(x) + f(x))$.
\begin{italicized}Proof:\end{italicized} $ P(x,x,1)$ yields $ f(2f(x) + 1) = f(2x^2 + 1)$. By lemma 3, $ f(2f(x)^2 + f(x)) = 2f(x^2 f(x) + f(x))$.\end{tcolorbox}

Typo : I think we have $ f(2f(x)^2 + f(x)) = f(2x^2f(x)+f(x))$ and indeed you use this form in the end of your proof.
\end{solution}



\begin{solution}[by \href{https://artofproblemsolving.com/community/user/35129}{Zhero}]
	[hide="Revised version"]We claim that the only solutions are $ f(x) = 0$ and $ f(x) = x^2$. Let $ P(x,y,z)$ be the assertion that $ f(f(x) + f(y) + f(z)) = f(f(x) - f(y)) + f(2xy + f(z)) + 2f(xz - yz)$.

\begin{bolded}Lemma 1\end{bolded}: For all real $ x$, $ f(x) = f( - x)$.
\begin{italicized}Proof:\end{italicized} $ P(x,y,z)$ and $ P(y,x,z)$ give $ f(f(x) + f(y) + f(z)) - f(2xy + f(z)) = f(f(x) - f(y)) + 2f(xz - yz) = f(f(y) - f(x)) + 2f(yz - xz)$, that is, $ f(f(x) - f(y)) + 2f(xz - yz) = f(f(y) - f(x)) + 2f(yz - xz)$ for all real $ x$, $ y$, and $ z$. Setting $ x = 2$ and $ y = 1$ here reveals that $ f(z) - f( - z) = (f(f(1) - f(2)) - f(f(2) - f(1)))\/2 = m$ for all real $ z$. Hence, $ 0 = [f(f(x) - f(y)) - f( - (f(y) - f(x)))] + 2[f(xz - yz) - f( - (xz - yz)] = 3m$, so $ m = 0$. Therefore, $ f(z) = f( - z)$ for all real $ z$.

\begin{bolded}Lemma 2\end{bolded}: $ f(0) = 0$.
\begin{italicized}Proof:\end{italicized} Let $ c = f(0)$. $ P(x,x,0)$ reveals that $ f(2f(x) + c) = c + f(2x^2 + c) + 2c = 3c + f(2x^2 + c)$, while $ P(x, - x,0)$ reveals that $ f(2f(x) + c) = f(f(x) + f( - x) + c) = c + f(c - 2x^2) + 2c = 3c + f(c - 2x^2)$, that is, $ f(2x^2 + c) = f(c - 2x^2)$ for all real $ x$. Setting $ x = \sqrt {|c|}$ reveals that $ f(3c) = f( - c) = f(c)$.

Since $ f(2f(x) + c) = 3c + f(2x^2 + c)$, setting $ x = 0$ reveals that $ f(3c) = 3c + f(c)$. But since $ f(3c) = f(c)$, we have that $ 3c = 0$, that is, $ c = 0$, which completes the proof of our lemma.

If $ f$ is constant, since $ f(0) = 0$, we must have that $ f(x) = 0$ for all real $ x$. Hence, we may now assume that $ f$ is nonconstant.

\begin{bolded}Lemma 3\end{bolded}: If $ f(a) = f(b)$, then for any real $ k$, $ f(ka) = f(kb)$.
\begin{italicized}Proof:\end{italicized} Let $ f(a) = f(b) = n$. The assertions $ P(k + 1,1,a)$ and $ P(k + 1,1,b)$ reveal that $ f(f(k + 1) + f(1) + n) - f(f(k + 1) - f(1)) - f(2(k + 1)(1) + n) = 2f(ka) = 2f(kb)$, which clearly completes our proof.

\begin{bolded}Corollary 1\end{bolded}: If $ f(x) = 0$, then $ x = 0$.
\begin{italicized}Proof:\end{italicized} If $ f(x) = 0$, then $ f(kx) = f(x) = 0$ for all real $ k$. If $ x$ is nonzero, then the set $ \{kx\}$ spans all reals, so $ f(y) = 0$ for all real $ y$. Therefore, $ f$ is constant, which is a contradiction.

\begin{bolded}Lemma 4\end{bolded}: $ f(f(x)) = f(x^2)$ for all real $ x$.
\begin{italicized}Proof:\end{italicized} $ P(x,x,0)$ reveals that $ f(2f(x)) = 0 + f(2x^2) + 0 = f(2x^2)$ for all real $ x$. By lemma 3, we have that $ f(x^2) = f(f(x))$.

\begin{bolded}Lemma 5\end{bolded}: $ f(f(1)x) = f(x)$ for all real $ x$.
\begin{italicized}Proof:\end{italicized} $ P(x^2,1,0)$ and $ P(f(x),f(1),0)$, by lemma 4, yield $ f(f(x^2) + f(1)) - f(f(x^2) - f(1)) = f(2x^2) = f(2f(x)f(1))$. Hence, by lemma 3, $ f(x^2) = f(f(x) f(1))$. By lemmas 3 and 4, $ f(f(x)) = f(x^2)$ implies that $ f(f(1)f(x)) = f(f(1)x^2)$, that is, $ f(x^2) = f(f(1)x^2)$. Since $ f$ is even, $ f( - x^2) = f(f(1)( - x^2))$ as well. Since the set of all $ x^2$ and the set of all $ - x^2$ together span the reals, we may conclude that $ f(f(1)x) = f(x)$ for all real $ x$.

\begin{bolded}Lemma 6\end{bolded}: $ f(1) = 1$.
\begin{italicized}Proof:\end{italicized} Since $ f$ is nonconstant, we can find some $ z$ such that $ f(z) \neq 0$. If we let $ y = \frac { - f(z)}{2}$, then $ P(1,y, z)$ and $ P(f(1),f(1)y, z)$, combined with lemmas 3, 4, and 5 yield $ f(f(1) + f(y) + f(z)) - f(f(1) - f(y)) = f(2y + f(z)) + 2 f(z - yz) = f(2yf(1)^2 + f(z)) + 2f(f(1)z(1 - y)) = f(2yf(1)^2 + f(z)) + 2f(z(1 - y))$. Hence, $ f(2yf(1)^2 + f(z)) = f(2y + f(z))$. But since $ y = \frac { - f(z)}{2}$, we have that $ 2y + f(z) = 0$. Hence, $ f(2y + f(z)) = 0$. But since $ f(2yf(1)^2 + f(z)) = 0$, by corollary 1, we must have that $ 2yf(1)^2 + f(z) = 0$. Substitution reveals that $ f(z)(f(1)^2 - 1) = 0$. Since we chose $ z$ such that $ f(z) \neq 0$, $ f(1)^2 = 1$. Hence, either $ f(1) = 1$ or $ f(1) = - 1$.

Suppose for the sake of contradiction that $ f(1) = - 1$. $ P(1,0,1)$ yields $ f( - 2) = - 4$. $ P(1,1,1)$ gives $ f( - 3) = - 1$. $ P(2,1,2)$ gives $ f( - 9) = f( - 3) + f(4 + f(2)) + 2f(2) = - 9$. $ P(3,0,1)$ gives us $ f( - 2) = f(f( - 3)) + f(f( - 1)) + 2f(3) = f( - 9) - 1 - 2 = - 12$, a contradiction, since $ f( - 2) = 4$. 

Therefore, we may conclude that $ f(1) = 1$. 

\begin{bolded}Corollary 2\end{bolded}: $ f(2f(x)^2 + f(x)) = 2f(x^2 f(x) + f(x))$.
\begin{italicized}Proof:\end{italicized} $ P(x,x,1)$ yields $ f(2f(x) + 1) = f(2x^2 + 1)$. By lemma 3, $ f(2f(x)^2 + f(x)) = f(2x^2 f(x) + f(x))$.

------------------------------------------------------------------------------------

$ P(x^2,f(x),x)$ yields $ f(2f(x^2) + f(x)) = f(2x^2 f(x) + f(x)) + 2f(x(x^2 - f(x))$. $ P(f(x), f(x), x)$ yields $ f(2f(x^2) + f(x)) = f(2f(x)^2 + f(x))$. Therefore, $ f(2f(x)^2 + f(x)) = f(2x^2 f(x) + f(x)) + 2f(x(x^2 - f(x)))$. By corollary 2, $ f(2f(x)^2 + f(x)) = f(2x^2 f(x) + f(x))$, so $ 2f(x(x^2 - f(x)) = 0$. If $ x \neq 0$, by corollary 1, we must have that $ f(x) = x^2$. If $ x = 0$, then $ f(0) = 0 = 0^2$. Therefore, $ f(x) = x^2$ for all real $ x$, as desired.[\/hide]
Thanks for your remarks and correction, pco. Hopefully it's all good now.  
\end{solution}



\begin{solution}[by \href{https://artofproblemsolving.com/community/user/112643}{proglote}]
	Solution by me and applepi2000 :

[hide="Solution"]

Let $P(x,y,z)$ denote the statement $f(f(x) + f(y) + f(z)) = f(f(x) - f(y)) + f(2xy + f(z)) + 2f(xz - yz).$

1) $f(-x) = f(x) \ \ \forall x \in \mathbb{R}.$ 

First notice that, for any reals $a,b$, comparing $P(a,b,0)$ and $P(b,a,0)$ we get $f(f(a) - f(b)) = f(f(b) - f(a)).$

If $f$ is constant, then it's clear that $f(x) = 0$ for all reals $x$, which is indeed a solution. Otherwise, there exist two reals $a$ and $b$ such that $f(a) \neq f(b)$. Let $x = f(a) - f(b)$ and $y = -x.$ Comparing $P(x, y, z)$ with $P(y, x, z)$ and using the above relation, we get $f(xz - yz) = f(yz - xz)$, and setting $z = k\/(x-y)$ we get $f(k) = f(-k) \ \ \forall k \in \mathbb{R}.$

2) $f(0) = 0.$

Let $x \in \mathbb{R}.$ Comparing $P(x, x, 0)$ and $P(x, -x, 0)$ we get $f(2x^2 + f(0)) = f(-2x^2 + f(0)) = f(2x^2 - f(0))$.

Therefore $f(k + f(0)) = f(k- f(0))$ for any $k \in \mathbb{R}^{+}$, and since $f$ is even, this holds for any real $k.$ Equivalently, $f(k) = f(k + 2f(0)).$

Suppose that $f(0) \neq 0.$ Notice that $f$ is periodic with period $2f(0)$, so $f(k + 2nf(0)) = f(k + 2mf(0))$ for any integers $m,n$, and choose them such that the numbers $k+2nf(0)$ and $k+2mf(0)$ are nonzero and not opposite. Comparing $P(x,y, k+2nf(0))$ and $P(x,y,k +2mf(0))$ for some reals $x \neq y$, we get $f((k+2nf(0))(x-y)) = f((k+2mf(0)) (x-y))$. Take $x - y = d\/(2n+f(0))$ for some real $d$, so we have $f(d) = f(d \cdot (k+2mf(0))\/(k+2nf(0)))$.

Since the function $g(k) = \frac{k+2mf(0)}{k+2nf(0)}$ is surjective on the reals (except 1), the above relation implies that $f$ is constant, so $f(x) = 0$ for all reals, a contradiction. Therefore, $f(0) = 0.$

3) $f(a) = f(b) \implies f(na) = f(nb) \ \ \forall a, b, n \in \mathbb{R}.$

Indeed, comparing $f(n+y, y, a)$ and $f(n+y, y, b)$ gives the desired result.

4) If $f$ is not constant, $f(x) = 0 \iff x = 0.$

Suppose $f(x) = 0.$ Then $P(x, 0, z) \implies 2f(xz) = 0$, and so if $f$ is not constant, we get $x =0.$

5) If $f$ is not constant, $f(x) = \pm x^{2}.$

From 3), setting $n = 1\/a$ and $n = 1\/b$, we get $f(a) = f(b) \implies f(a\/b) = f(b\/a) = f(1).$

Taking $P(a\/b, b\/a, z)$ for $z \neq 0$ gives $f(2f(1) + f(z)) = f(2 + f(z)) + 2f(z \cdot(a\/b - b\/a)).$
Taking $P(1, 1, z)$ gives $f(2f(1) + f(z)) = f(2 + f(z))$, hence $f(z \cdot(a\/b - b\/a)) = 0 \implies a\/b = b\/a \implies |a| = |b|.$
Finally, taking $P(x, x, 0)$ gives $f(2f(x)) = f(2x^2) \implies |f(x)| = |x^2|$, and so $f(x) = \pm x^2.$ 

6) If $f$ is not constant, then $f(x) = x^2$ for all reals $x.$

Suppose $f(x) = -x^2$ for some $x.$ Then $P(x,x,x)$ gives $f(-3x^2) = f(x^2) \implies \pm3x^2 = x^2$, so $x = 0.$ So $f(x) = x^2$ for all reals $x.$

Hence the solutions : $f(x) = 0$ and $f(x) = x^2.$

[\/hide]
\end{solution}



\begin{solution}[by \href{https://artofproblemsolving.com/community/user/264001}{tarzanjunior}]
	\begin{tcolorbox}
If $ z_1^2\ne z_2^2$, the system $ xz_1 - yz_2=u$ and $ xz_2 - yz_1=v$ always has solution and $ f(u)=f(v)$ $ \forall u,v$
And so, since we are considering non constant solutions, $ z_1^2=z_2^2$\end{tcolorbox}
How is $f(u)=f(v)$? Can someone explain me this step? 
\end{solution}



\begin{solution}[by \href{https://artofproblemsolving.com/community/user/29428}{pco}]
	\begin{tcolorbox}How is $f(u)=f(v)$? Can someone explain me this step?\end{tcolorbox}

We established that $f(z_1)=f(z_2)$ $\implies$ $f(xz_1-yz_2)=f(xz_2-yz_1)$ $\forall x,y$
We also established that if $z_1^2\ne z_2^2$, then system $xz_1-yz_2=u$ and $xz_2-yz_1=v$ always a solution $(x,y)$ whatever are $(u,v)$

And so if $f(z_1)=f(z_2)$ and $z_1^2\ne z_2^2$, just choose in $f(xz_1-yz_2)=f(xz_2-yz_1)$ the values $(x,y)$ solutions of the system $xz_1-yz_2=u$ and $xz_2-yz_1=v$ an you get $f(u)=f(v)$ whatever are $u,v)$

\end{solution}



\begin{solution}[by \href{https://artofproblemsolving.com/community/user/243741}{anantmudgal09}]
	\begin{tcolorbox}Find all functions $f$ from the set $\mathbb{R}$ of real numbers into $\mathbb{R}$ which satisfy for all $x, y, z \in \mathbb{R}$ the identity $$f(f(x)+f(y)+f(z))=f(f(x)-f(y))+f(2xy+f(z))+2f(xz-yz).$$\end{tcolorbox}

[hide=Solution]

[hide=Answer] We claim that the only functions $f$ for which $P$ is satisfied are $f \equiv 0$ and $f(x)=x^2$ for all $x \in \mathbb{R}$. Clearly, both of these satisfy the condition. It remains to show that these are the only such functions. [\/hide]

[hide=Notation] Let $P(x,y,z)$ denote the given assertion, namely, $$f(f(x)+f(y)+f(z))=f(f(x)-f(y))+f(2xy+f(z))+2f(xz-yz),$$ for all reals $x,y,z$. Assume that $f$ is non-constant (if it is the constant function, then $f \equiv 0$). We will show that $f(x)=x^2$ for all $x \in \mathbb{R}$. [\/hide]

[hide=Prelim 1] \begin{bolded}Lemma 1:\end{bolded} $f(x)=f(-x)$ for all $x \in \mathbb{R}$.

\begin{italicized}Proof:\end{italicized} Fix $x,y \in \mathbb{R}$ with $x \not= y$. From $P(x,y,z)$ and $P(y,x,z)$ we get that $$2c=f(f(x)-f(y))-f(f(y)-f(x))=2\left( f((x-y)z)-f((y-x)z) \right).$$ As $z$ varies, $(x-y)z$ spans all real numbers and so $f(t)-f(-t)=c$ for all $t$. Thus, $2c=c \Longrightarrow c=0 \Longrightarrow f(t)=f(-t)$ as desired. $\square$ [\/hide]

[hide=Prelim 2] \begin{bolded}Lemma 2:\end{bolded} For reals $a,b$, $f(a)=f(b) \Longrightarrow f(xa)=f(xb)$ for all $x \in \mathbb{R}$.

\begin{italicized}Proof:\end{italicized} From $P(x,y,a)$ and $P(x,y,b)$, it follows that $f((x-y)a)=f((x-y)b),$ and as $(x-y)$ spans all real numbers we are done. $\square$ [\/hide]

[hide=Prelude] Define the set $S$ of real numbers as $$\mathcal{S} := \left\{\frac{b}{a} : f(a)=f(b) \right \}$$ and note that \begin{bolded}Lemma 2\end{bolded} implies that for all real numbers $x$, $s \in \mathcal{S} \Longrightarrow f(x)=f(xs).$ [\/hide]

[hide=Main Step] \begin{bolded}Lemma 3:\end{bolded} For reals $a,b$, $f(a)=f(b) \Longrightarrow a=\pm \, b$.

\begin{italicized}Proof:\end{italicized} Fix $s \in \mathcal{S}$ and for the sake of contradiction, assume that $|s| \ne 1$. Fix a real $z$ such that $f(z) \ne 0$. From $P(x,y,z), P(xs,ys,z)$ we get $$f(2xys^2+f(z))=f(2xy+f(z)) \Longrightarrow \frac{2xys^2+f(z)}{2xy+f(z)} \in \mathcal{S}$$ for all $x,y$. Put $t=2xy$ and note that $$\frac{2xys^2+f(z)}{2xy+f(z)}=\frac{ts^2+f(z)}{t+f(z)}=s^2+\frac{f(z)\cdot (1-s^2)}{t+f(z)} \in \mathcal{S}$$ and as $t$ varies over the reals, we conclude that $\mathcal{S}=\mathbb{R}$. It follows that $f$ is a constant function, a contradiction to our initial assumption. $\square$ [\/hide]

[hide=Endgame] \begin{bolded}Conclusion:\end{bolded} Now, we observe that $P(x,y,0)$ and $P(x,-y,0)$ together imply that $$f(2xy+f(0))=f(-2xy+f(0))$$ and with the aid of \begin{bolded}Lemma 3\end{bolded} we get $f(0)=0$. Thus, $P(x,x,z)$ implies $$f(2f(x)+f(z))=f(2x^2+f(z))$$ for all reals $x,z$. Fix $x \in \mathbb{R}$ such that $f(x) \ne x^2$. Thus, $f(z)=-(x^2+f(x))$ for all reals $z$ implying that $f$ is a constant function; false! Therefore, $f(x)=x^2$ for all $x \in \mathbb{R}$. [\/hide]

[\/hide]
\end{solution}
*******************************************************************************
-------------------------------------------------------------------------------

\begin{problem}[Posted by \href{https://artofproblemsolving.com/community/user/55239}{bvarici}]
	Let $n$ be a positive integer. Find all monotone functions $f: \mathbb R \to \mathbb R$ such that  
\[f(x+f(y))=f(x)+y^n\]
holds for all reals $x$ and $y$.
	\flushright \href{https://artofproblemsolving.com/community/c6h337794}{(Link to AoPS)}
\end{problem}



\begin{solution}[by \href{https://artofproblemsolving.com/community/user/29428}{pco}]
	\begin{tcolorbox}Let n ∈ N. Find all monotone functions f : R→R such that
f (x+ f (y)) = f (x)+y^n\end{tcolorbox}

Let $ P(x,y)$ be the assertion $ f(x+f(y))=f(x)+y^n$

$ P(x,0)$ $ \implies$ $ f(x+f(0))=f(x)$.
If $ f(0)\ne 0$, this means that $ f(x)$ is a periodic monotonic function, so is constant, which is impossible.

So $ f(0)=0$ 
$ P(0,x)$ $ \implies$ $ f(f(x))=x^n$ 
$ f(x)$ monotonic implies $ f(f(x))$ increasing and so $ n$ must be odd.

$ P(0,f(x))$ $ \implies$ $ f(x^n)=f(x)^n$
$ P(x,f(y))$ $ \implies$ $ f(x+y^n)=f(x)+f(y)^n$ $ =f(x)+f(y^n)$ and so, since $ n$ is odd, $ f(x+y)=f(x)+f(y)$  and so, since monotonic, $ f(x)=ax$

Plugging back in original equation, we get $ n=1$ and $ a=\pm 1$

Hence the answer : solutions only exist if $ n=1$ and are then $ f(x)=x$ and $ f(x)=-x$
\end{solution}



\begin{solution}[by \href{https://artofproblemsolving.com/community/user/30342}{nicetry007}]
	$ f(x +f(y)) = f(x) + y^n ----------(*)$.
$ x = 0 \Rightarrow f(f(y)) = f(0) +y^n \Rightarrow f(f(0)) = f(0)$
Set $ y = f(y)$ in $ (*)$.
$ f(x+y^n + f(0)) = f(x) + f(y)^n ---------(**)$
Set $ x = -y^n$ in $ (**)$.
$ f(f(0)) = f(-y^n) +f(y)^n \Rightarrow f(0) = f(-y^n) + f(y)^n ----------(***)$
Set $ y = 0$ in $ (***)$.
$ f(0) = f(0) + f(0)^n \Rightarrow f(0)^n = 0 \Rightarrow f(0) = 0$.
This implies 
(i)$ f(f(y)) = y^n--------(+)$ ,
(ii)$ f(x+y^n) = f(x) + f(y)^n \Rightarrow f(y^n) = f(y)^n ----------(++)$
Applying $ f$ to both sides of $ (*)$, we get
$ f(f(x + f(y))) = f(f(x) + y^n) \Rightarrow (x + f(y))^n = f(y^n) + x^n$ 
$ \Rightarrow (x + f(y))^n = f(y)^n + x^n-----(****)$
Suppose $ n \geq 2$.
From $ (****)$, we have
$ x^n + nx^{n-1}f(y) + ...+f(y)^n = x^n + f(y)^n$ 
$ \Rightarrow nx^{n-1}f(y) + ...+nxf(y)^{n-1} = 0----(+++)$.
Since $ f(f(y)) = y^n$ , $ f(y) \neq 0 \;\;\forall y \neq 0$. We note that $ (+++)$ holds for all $ x$ and $ y$.  
For a given value of $ y\neq 0$, the equation in $ (+)$ is of degree $ n-1$ and has infinitely many values of $ x$ satisfying it. 
Hence, the co-efficient of $ x^i$ is zero for $ 1\leq i \leq n-1$. This implies $ nf(y) = 0 \Rightarrow n = 0$ which is not possible.
Hence, $ n=1$.
Relations in $ (+)$ and $ (++)$ would become
(i) $ f(f(y)) = y \Rightarrow f$ is a bijection.
(ii)$ f(x +y) = f(x)+ f(y) \Rightarrow f(q)= f(1)q \;\;\forall q \in \mathbb{Q}$
$ f(f(q)) = qf(1)^2 = q \Rightarrow f(1) = 1$ or $ -1$.
Suppose $ f$ is increasing. Then for $ y > 0$, $ f(x+y) = f(x) + f(y) \geq f(x) \Rightarrow f(y) \geq 0 \Rightarrow f(1) = 1$.
$ \Rightarrow f(q) = q \;\;\forall q \in \mathbb{Q} \Rightarrow f(x) = x \;\;\forall x \in \mathbb{R}(\because f \text{ is increasing })$.
Suppose $ f$ is decreasing. Then for $ y > 0$, $ f(x+y) = f(x) + f(y) \leq f(x) \Rightarrow f(y) \leq 0 \Rightarrow f(1) = -1$.
$ \Rightarrow f(q) = -q \;\;\forall q \in \mathbb{Q} \Rightarrow f(x) = -x \;\;\forall x \in \mathbb{R}(\because f \text{ is decreasing })$.
\end{solution}
*******************************************************************************
-------------------------------------------------------------------------------

\begin{problem}[Posted by \href{https://artofproblemsolving.com/community/user/49928}{alex2008}]
	Find all the functions $ f: \mathbb{N}\rightarrow \mathbb{N}$ such that
\[ 3f(f(f(n))) + 2f(f(n)) + f(n) = 6n, \quad \forall n\in \mathbb{N}.\]
	\flushright \href{https://artofproblemsolving.com/community/c6h337982}{(Link to AoPS)}
\end{problem}



\begin{solution}[by \href{https://artofproblemsolving.com/community/user/29428}{pco}]
	\begin{tcolorbox}Find all the functions $ f: \mathbb{N}\rightarrow \mathbb{N}$ such that:
\[ 3f(f(f(n))) + 2f(f(n)) + f(n) = 6n\ , \ (\forall)n\in \mathbb{N}\]
\end{tcolorbox}

$ f(n)$ is injective and an immediate induction gives $ \boxed{f(n)=n}$ $ \forall n$, which indeed is a solution.

[hide="Induction"]
If $ f(1)>1$, $ LHS>6$ and so $ f(1)=1$

If $ f(k)=k$ $ \forall k\le n$, then :
$ f(n+1)\ge n+1$ (else $ f(n+1)=k\le n$ would imply $ f(n+1)=f(k)$ and so $ n+1=k\le n$)
$ f(f(n+1))\ge n+1$ (else $ f(f(n+1))=k\le n$ would imply $ f(f(n+1))=f(k)$ and so $ f(n+1)=k=f(k)$ and so $ n+1=k\le n$)
$ f(f(f(n+1)))\ge n+1$ (else $ f(f(f(n+1)))=k\le n$ would imply $ f(f(f(n+1)))=f(k)$ and so $ f(f(n+1))=k=f(k)$ $ f(n+1)=k=f(k)$ and so $ n+1=k\le n$)

And since $ 3f(f(f(n+1))) + 2f(f(n+1)) + f(n+1) = 6(n+1)$, we need to have $ f(n+1)=n+1$
[\/hide]
\end{solution}



\begin{solution}[by \href{https://artofproblemsolving.com/community/user/78332}{AforMath}]
	Let $ a \in N$ be arbitrary. Set $ x_0 = a$, and $ x_n = f(x_{n - 1})$ for $ n > 0$. Then form equation we have this recurrence relation:
\[ 3x_{n + 3} + 2x_{n + 2} + x_{n + 1} - 6x_n = 0\]
Characteristic equation is  $ 3y^3 + 2y^2 + y - 6 = 0$ with roots $ 1$ , $ \frac 16( - 5 + i\sqrt {47})$,$ \frac 16( - 5 - i\sqrt {47})$. Let $ r = - 5 + i\sqrt {47}, q = - 5 - i\sqrt {47}$
    So the general solution of recurrence relation is
\[ x_n = A + B(\frac 16)^nr^n + C(\frac { - 1}{6})^nq^n\]
We must have that  $ B = C = 0$, because for all $ n > 0$ we have $ x_n \in N$ . So the only possible solution is $ x_n = A$. 
  And easily we have $ x_n = a$ for all $ n \in N$. Substituting  $ n = 1$  we obtain  $ f(a) = a$. 
  So  for all $ n \in N$ we have $ f(n) = n$, which indeed is a solution
\end{solution}



\begin{solution}[by \href{https://artofproblemsolving.com/community/user/70887}{rogueknight}]
	Aformath, I think your solution isnot right, because we doesn't have $ A \in N$, so we cannot have that $ B=c=0$. :)
\end{solution}



\begin{solution}[by \href{https://artofproblemsolving.com/community/user/78332}{AforMath}]
	\begin{tcolorbox}Aformath, I think your solution isnot right, because we doesn't have $ A \in N$, so we cannot have that $ B = c = 0$. :)\end{tcolorbox}

To get $ B = C = 0$, we don't need to have $ A \in N$. 
 Just look at  $ B(\frac {1}{6})^{n}r^{n} + C(\frac { - 1}{6})^{n}q^{n}$
We have, that for all $ n$ , $ x_n \in N$, so  if $ B$ and $ C$ is not $ 0$ then for sufficient large $ n$ we get that $ x_n \notin N$, since $ A$ does not change. 
If you want, you can easily get (with $ n=0$) that, $ A=a-B-C$, so $ x_{n}= a+ B((\frac{1}{6})^{n}r^{n}-1)+C((\frac{-1}{6})^{n}q^{n}-1)$
(More if $ B \neq C$ and not equal $ 0$, then $ x_n$ will be comlex number )

I hope you understand.
\end{solution}
*******************************************************************************
-------------------------------------------------------------------------------

\begin{problem}[Posted by \href{https://artofproblemsolving.com/community/user/49928}{alex2008}]
	Determine all the functions $ f: \mathbb{N}\rightarrow \mathbb{N}$ such that
\[ f(n)+f(n+1)+f(f(n))=3n+1, \quad \forall n\in \mathbb{N}.\]
	\flushright \href{https://artofproblemsolving.com/community/c6h338042}{(Link to AoPS)}
\end{problem}



\begin{solution}[by \href{https://artofproblemsolving.com/community/user/29428}{pco}]
	\begin{tcolorbox}Determine all the functions $ f: \mathbb{N}^*\rightarrow \mathbb{N}^*$ such that:
\[ f(n) + f(n + 1) + f(f(n)) = 3n + 1\ ,\ (\forall)n\in \mathbb{N}^*\]
\end{tcolorbox}

$ f(1)+f(2)+f(f(1))=4$ and so each of these term is $ 1$ or $ 2$ :

1) If $ f(1)=1$ : we get $ f(2)=2$ and an immediate induction gives $ \boxed{f(n)=n}$

2) If $ f(1)=2$ we get $ f(2)=1$ and an immediate induction gives $ \boxed{f(n)=n+(-1)^{n+1}}$
\end{solution}
*******************************************************************************
-------------------------------------------------------------------------------

\begin{problem}[Posted by \href{https://artofproblemsolving.com/community/user/33535}{wangsacl}]
	1. Find all monotone functions $ f: \mathbb{R}\to \mathbb{R}$ which satisfy for any $ x,y\in\mathbb{R}$,
\[ f(x+f(y))=f(x)+y.\]
2. Find all functions $ f: \mathbb{R}\to \mathbb{R}$ such that for any $ x,y\in\mathbb{R}$,
\[ f(x+f(y))=f(x)+y.\]
3. Find all functions $ f: \mathbb{R}^ + \rightarrow\mathbb{R}^ +$ such that for any $ x,y\in\mathbb{R}^ +$,
\[ f(x + f(y)) = f(x) + y.\]
	\flushright \href{https://artofproblemsolving.com/community/c6h338430}{(Link to AoPS)}
\end{problem}



\begin{solution}[by \href{https://artofproblemsolving.com/community/user/29428}{pco}]
	\begin{tcolorbox}1. Find any monotone function $ f: \mathbb{R}\rightarrow\mathbb{R}$ which satisfies for any $ x,y\in\mathbb{R}$,
\[ f(x + f(y)) = f(x) + y\]
2. Find any function $ f: \mathbb{R}\rightarrow\mathbb{R}$ which satisfies for any $ x,y\in\mathbb{R}$,
\[ f(x + f(y)) = f(x) + y\]
\end{tcolorbox}

Let $ P(x,y)$ be the assertion $ f(x+f(y))=f(x)+y$

$ P(0,x)$ $ \implies$ $ f(f(x))=x+f(0)$ and so $ f(x)$ is bijective
$ P(x,0)$ $ \implies$ $ f(x+f(0))=f(x)$ and so, since injective, $ f(0)=0$ and, as a consequence : $ f(f(x))=x$
$ P(x,f(y))$ $ \implies$ $ f(x+y)=f(x)+f(y)$ and $ f(x)$ is a solution of Cauchy equation.

Then :
Problem 1 : $ f(x)$ is a monotonic solution of Cauchy equation and so is $ ax$. Plugging in original equation, we get two solutions : $ f(x)=x$ and $ f(x)=-x$

Problem 2 : $ f(x)$ is any involutary solution of Cauchy equation (infinitely many such solutions)
\end{solution}



\begin{solution}[by \href{https://artofproblemsolving.com/community/user/33535}{wangsacl}]
	3. Find all function $ f: \mathbb{R}^ + \rightarrow\mathbb{R}^ +$ such that for any $ x,y\in\mathbb{R}^ +$,
\[ f(x + f(y)) = f(x) + y\]
whereas $ \mathbb{R}^ +$ denote the set of positive reals.
\end{solution}



\begin{solution}[by \href{https://artofproblemsolving.com/community/user/29428}{pco}]
	\begin{tcolorbox}3. Find all function $ f: \mathbb{R}^ + \rightarrow\mathbb{R}^ +$ such that for any $ x,y\in\mathbb{R}^ +$,
\[ f(x + f(y)) = f(x) + y\]
whereas $ \mathbb{R}^ +$ denote the set of positive reals.\end{tcolorbox}

You're welcome. Glad to have helped you for your two first problems.


problem 3 :
Let $ P(x,y)$ be the assertion $ f(x + f(y)) = f(x) + y$

Comparing $ P(x,a)$ and $ P(x,b)$, we get that "$ f(a) = f(b)\implies a = b$" and so $ f(x)$ is injective.

If $ f(a) > f(b)$, then $ P(b,f(a) - f(b))$ $ \implies$ $ f(b + f(f(a) - f(b))) = f(a)$ and so, since injective, $ f(f(a) - f(b)) = a - b$ and so $ a > b$ 
An immediate consequence is that $ a < b$ $ \implies$ $ f(a) < f(b)$ and so $ f(x)$ is increasing.

$ x > \frac x2$ and so $ f(x) > f(\frac x2)$ : $ P(f(x) - f(\frac x2),\frac x2)$ $ \implies$ $ f(f(x)) = f(f(x) - f(\frac x2)) + \frac x2 = x$ (using the fact that $ f(f(a) - f(b)) = a - b$ $ \forall a > b$)

Then $ P(x,f(y))$ $ \implies$ $ f(x + y) = f(x) + f(y)$ and so $ f(x)$ is an increasing solution of Cauchy equation, so is $ ax$.

Plugging this back in original equation, we get the unique solution $ f(x) = x$
\end{solution}



\begin{solution}[by \href{https://artofproblemsolving.com/community/user/33535}{wangsacl}]
	Nice solution!

Here's my solution:
\[ f(x + f(y)) = f(x) + y\]

\[ x + f(x + f(y)) = x + y + f(x)\]

\[ f(x + f(x + f(y)) = f(x + y + f(x))\]

\[ f(x) + x + f(y) = f(x + y) + x\]

\[ f(x) + f(y) = f(x + y)\]
If $ a > b$, then
\[ f(a) = f(a - b + b) = f(a - b) + f(b) > f(b)\]
showing $ f$ is a monotone increasing function. Because $ f$ is a Cauchy equation, $ f$ have some solutions $ f(x) = cx$ for a certain constant $ c$
Plugging $ f(x) = cx$ to the original fq, we get $ c = 1$, and $ f(x) = x$
\end{solution}
*******************************************************************************
-------------------------------------------------------------------------------

\begin{problem}[Posted by \href{https://artofproblemsolving.com/community/user/75308}{mr.fahlan}]
	Let $f: \mathbb Z \to \mathbb Z$ be a functions such that
\[ f(x+y)+f(xy-1)=f(x)f(y)-2\]
for all integers $x$ and $y$.

a) Prove that $f$ is not constant.
b) Find $f(0)$, $f(-1)$, and $f(1)$.
c) Find all such functions $f$.
	\flushright \href{https://artofproblemsolving.com/community/c6h338433}{(Link to AoPS)}
\end{problem}



\begin{solution}[by \href{https://artofproblemsolving.com/community/user/33535}{wangsacl}]
	1. assume the contrary, let $ f(x)=c$ for an integer $ c$ and for all $ x\in\mathbb{Z}$. Thus, 
\[ 2c=c^2-2\]
\[ c^2-2c-2=0\]
which obviously has no solution for $ c\in\mathbb{Z}$. Contradiction.

2. Let $ P(x,y)$ be the assertion $ f(x+y)+f(xy-1)=f(x)f(y)-2$
$ P(0,0): f(0)+f(-1)=(f(0))^2-2 \cdots (1)$
$ P(0,-1): 2f(-1)=f(0)f(-1)-2\cdots (2)$
Let $ f(0)=k$,
$ 2f(-1)=kf(-1)-2$
$ f(-1)=\frac{2}{k-2}$
Plugging this information to (1), 
$ k+\frac{2}{k-2}=k^2-2$
$ k^2-2k+2=k^3-2k^2-2k+4$
$ k^3-3k^2+2=0$
$ (k-1)(k^2-2k-2)=0$
$ k$ is an integer, thus $ k=1$. Consequently, $ f(-1)=\frac{2}{k-2}=\frac{2}{1-2}=-2$
And additionally, 
$ P(-1,-1): f(-2)+f(0)=(f(-1))^2-2$
$ f(-2)=(-2)^2-2-1=1$.
$ f(0)=1, f(-1)=-2, f(-2)=1$

Really weird... Maybe I got it wrong? CMIIW.
\end{solution}



\begin{solution}[by \href{https://artofproblemsolving.com/community/user/52090}{Dumel}]
	\begin{tcolorbox}Really weird... Maybe I got it wrong? CMIIW\end{tcolorbox}it's doubtful. I got the same results.  :)
\end{solution}



\begin{solution}[by \href{https://artofproblemsolving.com/community/user/75308}{mr.fahlan}]
	\begin{tcolorbox}1. assume the contrary, let $ f(x) = c$ for an integer $ c$ and for all $ x\in\mathbb{Z}$. Thus,
\[ 2c = c^2 - 2\]

\[ c^2 - 2c - 2 = 0\]
which obviously has no solution for $ c\in\mathbb{Z}$. Contradiction.

2. Let $ P(x,y)$ be the assertion $ f(x + y) + f(xy - 1) = f(x)f(y) - 2$
$ P(0,0): f(0) + f( - 1) = (f(0))^2 - 2 \cdots (1)$
$ P(0, - 1): 2f( - 1) = f(0)f( - 1) - 2\cdots (2)$
Let $ f(0) = k$,
$ 2f( - 1) = kf( - 1) - 2$
$ f( - 1) = \frac {2}{k - 2}$
Plugging this information to (1), 
$ k + \frac {2}{k - 2} = k^2 - 2$
$ k^2 - 2k + 2 = k^3 - 2k^2 - 2k + 4$
$ k^3 - 3k^2 + 2 = 0$
$ (k - 1)(k^2 - 2k - 2) = 0$
$ k$ is an integer, thus $ k = 1$. Consequently, $ f( - 1) = \frac {2}{k - 2} = \frac {2}{1 - 2} = - 2$
And additionally, 
$ P( - 1, - 1): f( - 2) + f(0) = (f( - 1))^2 - 2$
$ f( - 2) = ( - 2)^2 - 2 - 1 = 1$.
$ f(0) = 1, f( - 1) = - 2, f( - 2) = 1$

Really weird... Maybe I got it wrong? CMIIW.\end{tcolorbox}

Thank you, you are amazing :)
\end{solution}



\begin{solution}[by \href{https://artofproblemsolving.com/community/user/29428}{pco}]
	Let $ P(x,y)$ be the assertion $ f(x+y)+f(xy-1)=f(x)f(y)-2$

$ P(0,-1)$ $ \implies$ $ (f(0)-2)f(-1)=2$ and so $ f(-1)\ne 0$

$ P(x,-1)$ $ \implies$ $ f(x-1)+f(-x-1)=f(x)f(-1)-2$
$ P(-x,-1)$ $ \implies$ $ f(-x-1)+f(x-1)=f(-x)f(-1)-2$

Since $ f(-1)\ne 0$, this implies $ f(x)=f(-x)$ and $ P(x,1)$ becomes $ f(x-1)+f(x+1)=f(x)f(-1)-2$

Let then $ f(-1)=f(1)=a$ and $ f(0)=2+\frac 2a$. $ P(0,0)$ $ \implies$ $ 2+\frac 2a+a=(2+\frac 2a)^2-2$ and so $ (a+2)(a^2-2a-2)=0$ and so $ a=-2$

And so $ f(x)$ is fully defined with $ f(0)=1$ and $ f(1)=-2$ and $ f(x+1)=-2f(x)-f(x-1)-2$

which gives : $ \boxed{f(n)=\frac{3(-1)^n-1}2} \forall n\in\mathbb Z$ which, indeed is a solution.
\end{solution}
*******************************************************************************
-------------------------------------------------------------------------------

\begin{problem}[Posted by \href{https://artofproblemsolving.com/community/user/67223}{Amir Hossein}]
	If $a_i  \ (i = 1, 2, \ldots, n)$ are distinct non-zero real numbers, prove that the equation
\[\frac{a_1}{a_1-x} + \frac{a_2}{a_2-x}+\cdots+\frac{a_n}{a_n-x} = n\]
has at least $n - 1$ real roots.
	\flushright \href{https://artofproblemsolving.com/community/c6h368347}{(Link to AoPS)}
\end{problem}



\begin{solution}[by \href{https://artofproblemsolving.com/community/user/89144}{sumanguha}]
	let wlog order the a's as wlog
$ \ a_{1}<a_{2}<...a_{n}$
let among them $ \ a_{1}<a_{2}<...a_{m}<0<a_{m+1}..<a_{n}$
notice 0 is a root
then see if we approach a_{1} from right it blows to +infinity  but as we approach a_{2} from left it blows to -infinity.
but in  $ \ (a_{1},a_{2})$ the fn is continuous so atleast at one point in the interval it take value n.
apply same logic upto a_{m}.
between a_{m} and a_{m+1} automatically a root $ \ 0 $.
then apply again from a_{m+1}to thye next a etc....
so we get atleast $ \ n-1  $ real roots
done
\end{solution}



\begin{solution}[by \href{https://artofproblemsolving.com/community/user/29428}{pco}]
	\begin{tcolorbox}If $a_i  \ (i = 1, 2, \ldots, n)$ are distinct non-zero real numbers, prove that the equation
\[\frac{a_1}{a_1-x} + \frac{a_2}{a_2-x}+\cdots+\frac{a_n}{a_n-x} = n\]
has at least $n - 1$ real roots.\end{tcolorbox}

Wlog say $a_1<a_2<...<a_n$

$f(x)=\frac{a_1}{a_1-x} + \frac{a_2}{a_2-x}+\cdots+\frac{a_n}{a_n-x}$ is continuous over $(a_k,a_{k+1})$ $\forall k\in[1,n-1]$

If $a_k<0$ and $a_{k+1}<0$ : $\lim_{x\to a_k^+}f(x)=+\infty$  and $\lim_{x\to a_{k+1}^-}f(x)=-\infty$ and at least one root in $(a_k,a_{k+1)}$

If $a_k<0$ and $a_{k+1}>0$ : $f(0)=n$ and at least one root in $(a_k,a_{k+1)}$

If $a_k>0$ and $a_{k+1}>0$ : $\lim_{x\to a_k^+}f(x)=-\infty$  and $\lim_{x\to a_{k+1}^-}f(x)=+\infty$ and at least one root in $(a_k,a_{k+1)}$

So at least one root of $f(x)=n$ in each interval $(a_k,a_{k+1})$ for any positive integer $k\in[1,n-1]$
Hence the result.
\end{solution}



\begin{solution}[by \href{https://artofproblemsolving.com/community/user/191127}{sayantanchakraborty}]
	If all $a_j$'s had been of same sign the problem would have been very easy.
\end{solution}



\begin{solution}[by \href{https://artofproblemsolving.com/community/user/64716}{mavropnevma}]
	Write the equation
\[\frac{a_1}{a_1-x} + \frac{a_2}{a_2-x}+\cdots+\frac{a_n}{a_n-x} = n\]
as 
\[\frac{x}{a_1-x} + \frac{x}{a_2-x}+\cdots+\frac{x}{a_n-x} = 0.\]
Consider the polynomial $p(x) = (x-a_1)(x-a_2)\cdots (x-a_n)$; then 
\[\frac{x}{a_1-x} + \frac{x}{a_2-x}+\cdots+\frac{x}{a_n-x} = -\dfrac {xp'(x)}{p(x)},\]
so the problem comes to showing $xp'(x) = 0$ has at least $n-1$ real roots. But $p'$ indeed \begin{bolded}does\end{bolded} have $n-1$ distinct real roots, by the definition of $p$. Moreover, if all $a_i$ are not-null, we also have $x=0$ as (maybe multiple) root, so there will be $n$ total real roots.
\end{solution}
*******************************************************************************
-------------------------------------------------------------------------------

\begin{problem}[Posted by \href{https://artofproblemsolving.com/community/user/72235}{Goutham}]
	Find all functions $f$ defined for all $x$ that satisfy the condition $xf(y) + yf(x) = (x + y)f(x)f(y),$ for all $x$ and $y.$ Prove that exactly two of them are continuous.
	\flushright \href{https://artofproblemsolving.com/community/c6h369291}{(Link to AoPS)}
\end{problem}



\begin{solution}[by \href{https://artofproblemsolving.com/community/user/29428}{pco}]
	\begin{tcolorbox}$(BUL 2)$ Find all functions $f$ defined for all $x$ that satisfy the condition $xf(y) + yf(x) = (x + y)f(x)f(y),$ for all $x$ and $y.$ Prove that exactly two of them are continuous.\end{tcolorbox}
Let $P(x,y)$ be the assertion $xf(y)+yf(x)=(x+y)f(x)f(y)$

If $f(0)\ne 0$ $P(x,0)$ $\implies$ $xf(x)=x$ and so $f(x)=1$ $\forall x\ne 0$ and this is indeed a solution, whatever is $f(0)$

If $f(0)=0$ and $\exists u\ne 0$ such that $f(u)=0$, then $P(x,u)$ $\implies$ $f(x)=0$ $\forall x$

If $f(0)=0$ and $f(x)\ne 0$ $\forall x\ne 0$, then $P(x,x)$ $\implies$ $f(x)=1$ $\forall x\ne 0$

And so the solutions :

$f(x)=0$ $\forall x$
$f(x)=1$ $\forall x\ne 0$ and $f(0)=a$

And obviously only two of these solutions are continuous :
$f(x)=0$ $\forall x$
$f(x)=1$ $\forall x$
\end{solution}



\begin{solution}[by \href{https://artofproblemsolving.com/community/user/86695}{Thrax}]
	Excuse me there's another solution that's not mentioned
$ f(x) = 0$  $\forall x \ne 0 $ and  $f(0) = a $

It follows from the assertion $ P(x,x) $ 
Another solution follows from the assertion $ P(x,x) $ which is $ f(x) = 1 $ $\forall x \in S $ and $f(x) = 0 $  $\forall x \in S`-\{0\} $ and $f(0) = k $ for a real constant $ k $, where S is a subset of the set of real numbers that doesn't contain the element 0. but this solution is falsified if $ x \in S $ and $y \in S`-\{0\} $

so another solution would be the one mentioned at the beginning.
\end{solution}



\begin{solution}[by \href{https://artofproblemsolving.com/community/user/29428}{pco}]
	\begin{tcolorbox}Excuse me there's another solution that's not mentioned
$ f(x) = 0$  $\forall x \ne 0 $ and  $f(0) = a $\end{tcolorbox}
I'm sorry but if $a\ne 0$ this is not a solution :

Choose $x=0$ and $y=1$ so that $f(x)=a$ and $f(y)=0$

$xf(y)+yf(x)=a$
$(x+y)f(x)f(y)=0$

And so we dont have equality.
\end{solution}



\begin{solution}[by \href{https://artofproblemsolving.com/community/user/86695}{Thrax}]
	I'm sorry , you are right, this solution implies that k = 0;

Thanks :)
\end{solution}



\begin{solution}[by \href{https://artofproblemsolving.com/community/user/216909}{PatrikP}]
	wrong...
\end{solution}



\begin{solution}[by \href{https://artofproblemsolving.com/community/user/89198}{chaotic_iak}]
	Wrong:
1. $f(x) = f(x)^2$ only when $x \neq 0$, since you divided by $x$.
2. It's possible that $f(x) = 1$ for some $x$ and $f(x) = 0$ for the rest.
\end{solution}
*******************************************************************************
-------------------------------------------------------------------------------

\begin{problem}[Posted by \href{https://artofproblemsolving.com/community/user/85501}{taylorcute56}]
	Find all functions $g,h : \mathbb R\to mathbb R$ such that \[g(x)-g(y)=\cos(x+y)h(x-y)\] for all $x,y \in \mathbb R$.
	\flushright \href{https://artofproblemsolving.com/community/c6h369561}{(Link to AoPS)}
\end{problem}



\begin{solution}[by \href{https://artofproblemsolving.com/community/user/29428}{pco}]
	\begin{tcolorbox}Find all funtion g,h : $R\rightarrow R$ such that $g(x)-g(y)=cos(x+y)h(x-y)$\end{tcolorbox}
Setting $y=0$ in the equation, we get $g(x)=g(0)+\cos(x)h(x)$ and plugging this in the equation we get :

Assertion $P(x,y)$ : $\cos(x)h(x)-\cos(y)h(y)=\cos(x+y)h(x-y)$


Adding $P(x,0)$, $P(0,-\frac {\pi}2)$ and $P(-\frac{\pi}2,x)$, we get :
(1) : $\cos(x)h(x)+\sin(x)h(-x-\frac{\pi}2)=0$

Adding $P(\frac {\pi}4+x,\frac {\pi}4)$, $P(\frac {\pi}4,-\frac {\pi}4)$ and $P(-\frac {\pi}4,\frac {\pi}4+x)$, we get :
(2) : $-\sin(x)h(x)+h(\frac{\pi}2)+\cos(x)h(-x-\frac{\pi}2)=0$

Multiplying (1) by $\cos(x)$ and (2) by $-\sin(x)$ and adding the two results, we get :

$h(x)=\sin(x)h(\frac{\pi}2)$ $\forall x$

This gives the mandatory form :
$h(x)=2a\cdot\sin(x)$
$g(x)=a\cdot\sin(2x)+b$

which indeed is a solution.
\end{solution}
*******************************************************************************
-------------------------------------------------------------------------------

\begin{problem}[Posted by \href{https://artofproblemsolving.com/community/user/89954}{Sutuxam}]
	Does there exist a function $f: \mathbb R \to \mathbb R$ satisfying \[f(x+y) \geq f(x)+yf((x))\] for all $x,y \in \mathbb R$ and $f(0)>0$?
	\flushright \href{https://artofproblemsolving.com/community/c6h369909}{(Link to AoPS)}
\end{problem}



\begin{solution}[by \href{https://artofproblemsolving.com/community/user/29428}{pco}]
	\begin{tcolorbox}are there exits function f:R-R satisfying  f(x+y)>= f(x)+yf((x)) with all x,y in R and f(0)>0.\end{tcolorbox}

If $f((x))$ means $f(x)$, the answer is yes : choose for example $f(x)=e^x$ :

$f(0)=1>0$
$f(x+y)=e^{x+y}$ and $f(x)+yf(x)=(y+1)e^x$ and the inequality $e^xe^y\ge e^x(y+1)$ is true.
\end{solution}



\begin{solution}[by \href{https://artofproblemsolving.com/community/user/89954}{Sutuxam}]
	\begin{tcolorbox}[quote="Sutuxam"]are there exits function f:R-R satisfying  f(x+y)>= f(x)+yf((x)) with all x,y in R and f(0)>0.\end{tcolorbox}

If $f((x))$ means $f(x)$, the answer is yes : choose for example $f(x)=e^x$ :

$f(0)=1>0$
$f(x+y)=e^{x+y}$ and $f(x)+yf(x)=(y+1)e^x$ and the inequality $e^xe^y\ge e^x(y+1)$ is true.\end{tcolorbox}
but f(f(x))?
\end{solution}



\begin{solution}[by \href{https://artofproblemsolving.com/community/user/29428}{pco}]
	\begin{tcolorbox}[quote="pco"]\begin{tcolorbox}are there exits function f:R-R satisfying  f(x+y)>= f(x)+yf((x)) with all x,y in R and f(0)>0.\end{tcolorbox}

If $f((x))$ means $f(x)$, the answer is yes : choose for example $f(x)=e^x$ :

$f(0)=1>0$
$f(x+y)=e^{x+y}$ and $f(x)+yf(x)=(y+1)e^x$ and the inequality $e^xe^y\ge e^x(y+1)$ is true.\end{tcolorbox}
but f(f(x))?\end{tcolorbox}

You are welcome, glad to have helped you.

But what do you mean ?

Did you make a mistake in the problem ? Is the real problem $f(x+y)\ge f(x)+yf(f(x)) $ ?
\end{solution}



\begin{solution}[by \href{https://artofproblemsolving.com/community/user/89954}{Sutuxam}]
	sorry for my bad english.but the problem is f(f(x)),isnot f(x).can you help me? this problem from china traning
\end{solution}



\begin{solution}[by \href{https://artofproblemsolving.com/community/user/29428}{pco}]
	\begin{tcolorbox}are there exits function f:R-R satisfying  f(x+y)>= f(x)+yf((x)) with all x,y in R and f(0)>0.\end{tcolorbox}
If the inequation is $P(x,y)$ : $f(x+y)\ge f(x)+yf(f(x))$, then here is a rather complex proof that no such function exists (I'm not very proud of it and hope something simpler exists) :

1) $f(x)$ is a convex $C_1$ function such that $f(f(x))=f'(x)$
[hide="15 lines proof"]
================= begin of hidden proof ===================
$P(x+y,-y)$ $\implies$ $f(x)\ge f(x+y)-yf(f(x+y))$ and so :

$f(x)+yf(f(x+y))\ge f(x+y) \ge f(x)+yf(f(x))$

From this, we get :

a) $f(f(x))$ is non decreasing
$yf(f(x+y))\ge yf(f(x))$ and so $f(f(x))$ is non decreasing

b) $f(x)$ is continuous
If $y\in(0,1)$ : $f(f(x+y))\le f(f(x+1))$ and the inequality becomes 
$f(x)+yf(f(x+1))\ge f(x+y) \ge f(x)+yf(f(x))$ an setting $y\to 0^+$ in this inequality, we get $f(x)$ right continuous

If $y\in (-1,0)$ : $f(f(x+y))\ge f(f(x-1))$ and the inequality becomes
$f(x)+yf(f(x-1))\ge f(x+y) \ge f(x)+yf(f(x))$ an setting $y\to 0^+$ in this inequality, we get $f(x)$ left continuous

c) $f(x)$ is $C_1$ and $f(f(x))=f'(x)$
$f(x)+yf(f(x+y))\ge f(x+y) \ge f(x)+yf(f(x))$ implies :

$\max(f(f(x+y)),f(f(x)))\ge \frac{f(x+y)-f(x)}y \ge \min(f(f(x+y)),f(f(x))$
Setting $y\to 0$ in this inequality and using continuity of $f(x)$, we get that $\lim_{y\to 0}\frac{f(x+y)-f(x)}y =f(f(x))$

And since $f(f(x))=f'(x)$ is non decreasing, we get thet $f(x)$ is convex.
=================  end of hidden proof  ===================
[\/hide]

2) $f'(0)\ge 0$
[hide="2 lines proof"]
================= begin of hidden proof ===================
If $f'(0)<0$, we get $f(f(0))<0$ and so $\exists u\in(0,f(0))$ such that $f(u)=u$ and $f'(u)<0$ (remember $f(x)$ is continuous and convex)
But $f(u)=u$ $\implies$ $f'(u)=f(f(u))=u>0$ and so contradiction.
=================  end of hidden proof  ===================
[\/hide]

3) $f(x)>x$ $\forall x>a$ for some real $a$
[hide=" 6 lines proof"]
================= begin of hidden proof ===================
$f'(0)\ge 0$ implies that $f'(x)>0$ for some $x>0$ else $f(x)$ would be constant over $\mathbb R^+$, and $f'(x)=0$ and so $f(x)=0$, which is wrong since $f(0)>0$
$f'(x)>0$ above some point and $f(x)$ convex implies $\lim_{x\to +\infty}f(x)=+\infty$

And so $\lim_{x\to +\infty}f(f(x))=+\infty$
And so $\lim_{x\to +\infty}f'(x)=+\infty$
And so $f'(x)>1$ above some point
And so $f(x)>x$ above some point
Q.E.D
=================  end of hidden proof  ===================
[\/hide]

4) $f(x)<x$ $\forall x>b$ for some real $b$
[hide="12 lines proof"]
================= begin of hidden proof ===================
Suppose $f(x)>x$ for some $x>\max(4,f(0)+2)$
Let $a_0=x$
Since $a_0>f(0)$, $\exists $a_1<a_0$ such that $f(a_1)=a_0$
Since $f(x)$ convex, $\frac{f(a_0)-f(a_1)}{a_0-a_1}\ge f'(a_1)$
So $\frac{f(a_0)-a_0}{a_0-a_1}\ge f(a_0)$ and so $a_0-a_1\le 1-\frac{a_0}{f(a_0)}<1$
So $a_1>\max(4,f(0)+2)$ and we can build now $a_2<a_1$ such that $f(a_2)=a_1$ and we get 
$a_1-a_2\le 1-\frac{a_1}{f(a_1)}=1-\frac{a_1}{a_0}=\frac{a_0-a_1}{a_0}<\frac{a_0-a_1}2\le \frac 12$
And we can build a decreasing sequence $a_n$ such that $a_n\in[a_0-2,a_0)$ and $f(a_{n+1})=a_n$
So $a_n has a limit $a\in(a_0-2,a_0)$ such that $f(a)=a$ (continuity of $f(x)$)

So, for any $x>\max(4,f(0)+2)$ such that $f(x)>x$, we get a $x'\in(x-2,x)$ such that $f(x')=x'$
And since $f(x)$ is convex, we know that the equation $f(x)=x$ has at most two solutions (except if $f(x)=x$ on some interval, which is easy to exclude), we get the required result.
=================  end of hidden proof  ===================
[\/hide]

And, since obviously 3) and 4) before are a contradiction, we proved that no such function exist.
\end{solution}



\begin{solution}[by \href{https://artofproblemsolving.com/community/user/64716}{mavropnevma}]
	Two very closely related problems

1. From Italian MO of recent years. Show that for any function $f : (0,\infty) \to (0,\infty)$ there exist real numbers $x,y > 0$ such that $f(x+y) < f(x) + yf(f(x))$.

2. Strengthened by me and sat at 2009 IMAR contest (Romania). Show that for any function $f : (0,\infty) \to (0,\infty)$ there exist real numbers $x,y > 0$ such that $f(x+y) < yf(f(x))$. (Indeed, even $f(x+y) < yf^n(x)$, where $f^n$ is the iterate of order $n \geq 2$ of $f$; not true for $n=1$.)
\end{solution}
*******************************************************************************
-------------------------------------------------------------------------------

\begin{problem}[Posted by \href{https://artofproblemsolving.com/community/user/67223}{Amir Hossein}]
	Let $f$ and $g$ be functions from the set $A$ to the same set $A$. We define $f$ to be a functional $n$-th root of $g$ ($n$ is a positive integer) if $f^n(x) = g(x)$, where $f^n(x) = f^{n-1}(f(x)).$

(a) Prove that the function $g : \mathbb R  \to \mathbb R, g(x) = 1\/x$ has an infinite number of $n$-th functional roots for each positive integer $n.$

(b) Prove that there is a bijection from $\mathbb R$ onto $\mathbb R$ that has no nth functional root for each positive integer $n.$
	\flushright \href{https://artofproblemsolving.com/community/c6h370142}{(Link to AoPS)}
\end{problem}



\begin{solution}[by \href{https://artofproblemsolving.com/community/user/29428}{pco}]
	\begin{tcolorbox}Let $f$ and $g$ be functions from the set $A$ to the same set $A$. We define $f$ to be a functional $n$-th root of $g$ ($n$ is a positive integer) if $f^n(x) = g(x)$, where $f^n(x) = f^{n-1}(f(x)).$

\begin{bolded}(a)\end{bolded} Prove that the function $g : \mathbb R  \to \mathbb R, g(x) = 1\/x$ has an infinite number of $n$-th functional roots for each positive integer $n.$\end{tcolorbox}

Little remark : $g(x)=\frac 1x$ is not a function from $\mathbb R\to\mathbb R$ since it is not defined for $x=0$

So I supposed you defined $g(x)$ as :
$g(0)=0$ and $g(x)=\frac 1x$ $\forall x\ne 0$

Let $A_1,A_2, ... A_n$ any split of $(0,1)$ in $n$ equinumerous sets (intersections are empty and union is $(0,1)$

Let $\{b_k\}$ for $k\in[1,n-1]$ bijections from $A_{k}\to A_{k+1}$ and let $b_n=b_1^{[-1]}(b_2^{[-1]}(....(b_{n-1}^{[-1]}(x)))))$ a bijection from $A_n\to A_1$

The then $f(x)$ defined as :

$f(0)=0$
$f(1)=1$
$\forall x\in(0,1)$ :
If $x\in A_k$ for some $k<n$ : $f(x)=b_k(x)$ (and so $f(x)\in A_{k+1}$)
If $x\in A_n$ : $f(x)=\frac 1{b_n(x)}$

$\forall x\in(1,+\infty)$ : $f(x)=\frac 1{f(\frac 1x)}$

$\forall x<0$ : $f(x)=-f(-x)$

It's easy to see that $f^n(x)=g(x)$ $\forall x$

And we obviously have infinitely many possibilities (varying $A_i$ or $b_i$)
\end{solution}



\begin{solution}[by \href{https://artofproblemsolving.com/community/user/29428}{pco}]
	\begin{tcolorbox}Let $f$ and $g$ be functions from the set $A$ to the same set $A$. We define $f$ to be a functional $n$-th root of $g$ ($n$ is a positive integer) if $f^n(x) = g(x)$, where $f^n(x) = f^{n-1}(f(x)).$
\begin{bolded}(b)\end{bolded} Prove that there is a bijection from $\mathbb R$ onto $\mathbb R$ that has no nth functional root for each positive integer $n.$\end{tcolorbox}
Let $a$ a fixed point of $g(g(x))$

Let $f(x)$ a n-th functional root of $g(x)$

$f^{2n}(a)=g(g(a))=a$ and so $f^{2n+1}(a)=f(a)$ and so $g(g(f(a)))=f(a)$

So the image by $f(x)$ of any fixed point of $g(g(x))$ is a fixed point of $g(g(x))$

Choose then $g(x)$ as any bijection without fixed point such that $g(g(x))$ has only two fixed point $a$ and $b$.

Either $f(a)=a$ but then $g(a)=f^{n}(a)=a$, impossible
Either $f(a)=b$ but then $f(b)=a$ and, if $n$ is even, $g(a)=a$ again, impossible.

So all bijection $g(x)$ without fixed point such that $g(g(x))$ has only two fixed point has no n-th functional root for $n$ even.
\end{solution}



\begin{solution}[by \href{https://artofproblemsolving.com/community/user/114585}{anonymouslonely}]
	for (b) ...
for your proof... means that that kind of functions doesn't have nth functional root for n even... but what happens when n is odd?
can you explain,please? thank you.
\end{solution}



\begin{solution}[by \href{https://artofproblemsolving.com/community/user/96532}{dgrozev}]
	Yes, for b), Patrick's argument only proves that there exists function $g$, so that if $g$ has $n$-th functional root, $n$ must be odd.
And btw, I came across it through IMO Longlist, 1983, p.20. 

b) First some analysis to help us in construction of a function $g$ without functional roots. 
If $g: \mathbb{R}\to \mathbb{R}$, let $O_x = \{g^i(x)\,|\,i=0,1,\cdots\}$. If $O_x$ is finite, $|O_x|$ be the number of elements of $O_x$ and if $O_x$ is infinite let $|O_x|=\infty$. 
Suppose now that $\forall x\in \mathbb{R},\, g(x)=f^n(x) $. Then $f(g(x))=g(f(x)) $. This means $|O_x|=|O_{f(x)}| $. So if we take care, that $g$ does not have two different finite orbits with equal number of elements, there will remain not so many possibilities.

\begin{bolded}An example of\end{bolded}\end{underlined} $g$ \begin{bolded}without functional roots\end{bolded}\end{underlined}.

Let $g$ is a bijection that satisfies:
(i) $g$ does not have a fixed point, i.e. $\forall x \in \mathbb{R}\, g(x)\neq x$.
(ii) If $|O_x|=|O_y| < \infty \, \Rightarrow \, O_x = O_y$.
(iii) $\forall p \in \mathbb{P}, \, \exists x\in \mathbb{R},\, |O_x|=p$.

Let postpone the explicit construction of $g$ until the end and first show why it is imposible $g$ to have a functional root. 
Assume that $\forall x\in\mathbb{R},\, g(x)=f^{n_0}(x)$. Let $p_0\in \mathbb{P},\, p_0 | n_0$. According to (iii) ${\exists x_0,\, |O_{x_0}|=p_0}$.
Because $f\circ g = g \circ f \,\Rightarrow \, |O_{x_0}|=|O_{f(x_0)}|$. According to (ii), $O_{x_0}= O_{f(x_0)}$. So, $f(x_0)=g^i(x_0),\, 0<i<p_0$.(It is impossible $f(x_0)=x_0$, because it will contradict to (i) ).
We have:

(1) $f(x_0)=f^{in_0}(x_0)   $.
(2) $f^{n_0p_0}(x_0)=x_0$.

Let $y_0=f(x_0)$. (1)and (2) yield:

(3) $f^{in_0-1}(y_0)=y_0$.
(4) $f^{n_0p_0}(y_0)=y_0$.

It is clear that $\gcd(n_0p_0, in_0-1)=1\,\Rightarrow \, \exists k,l\in \mathbb{N},\, kn_0p_0=l(in_0-1)+1\, \Rightarrow$
(5) $y_0=f^{kn_0p_0}(y_0)=f^{l(in_0-1)+1}(y_0)=f(y_0)$
Now (5) contradics (i) which proves that it is not possible $g$ to have any functional root.

Finally, to construct $g$ satisfying (i),(ii),(iii). 
Let difine $g$ by: 
(6) $0\to 1\to 0;\,-1\to -2 \to -3 \to -1;\, 2 \to 3\to 4 \to 5 \to 6 \to 2; \cdots $ 
(7) $(0,1)\to (1,2)\to (2,3) \to \cdots \,; (0,1) \leftarrow (-1,0)\leftarrow (-2,-1)\leftarrow (-3,-2)\leftarrow \cdots $

So the only finite orbits $O_x$ of $g$ are when $x\in \mathbb{Z}$. (6) guarantees that for every $p\in \mathbb{P},\, \exists x\in \mathbb{Z},\, |O_x|=p$.
In (7), $(n,n+1)\to (n+1,n+2)$ is an arbitrary bijection.
\end{solution}
*******************************************************************************
-------------------------------------------------------------------------------

\begin{problem}[Posted by \href{https://artofproblemsolving.com/community/user/72235}{Goutham}]
	Determine all continuous functions $f: \mathbb R \to \mathbb R$ such that
\[f(x + y)f(x - y) = (f(x)f(y))^2, \quad \forall(x, y) \in\mathbb{R}^2.\]
	\flushright \href{https://artofproblemsolving.com/community/c6h371123}{(Link to AoPS)}
\end{problem}



\begin{solution}[by \href{https://artofproblemsolving.com/community/user/89670}{man111}]
	[list]$f(x+y)f(x-y) = (f(x)f(y))^2$, taking $log$ on both side, we get..

$log_{e}f(x+y) + log_{e}f(x-y) = 2(log_{e}f(x) + log_{e}f(y))$

and Let $log_{e}f(x) = g(x)$, we get $g(x+y) + g(x-y) = 2(g(x) + g(y))$

Now put $y = 1$,We get $g(x+1) + g(x-1) = 2(g(x) + g(1))$

$(g(x+1)  - g(x)) -  (g(x) - g(x-1)) = 2(g(1)) = constant.$

means $g(x) = ax^2 +bx +c$, now put into this equation , and get value of $a , b, c.$



 [\/list]
\end{solution}



\begin{solution}[by \href{https://artofproblemsolving.com/community/user/29428}{pco}]
	\begin{tcolorbox}Determine all continuous functions $f$ such that
\[f(x + y)f(x - y) = (f(x)f(y))^2\:\:\:\: \forall(x, y) \in\mathbb{R}^2\]\end{tcolorbox}

Let $P(x,y)$ be the assertion $f(x+y)f(x-y)=f(x)^2f(y)^2$

If $f(0)=0$ : $P(x,x)$ implies $f(x)=0$ $\forall x$ which indeed is a solution
If $f(0)\ne 0$ : $P(0,0)$ $\implies$ $f(0)=\pm 1$. But $f(x)$ solution implies $-f(x)$ solution.

WLOG say then from now : $f(0)=1$

If $f(a)=0$ for some $a$, then $P(\frac a2,\frac a2)$ $\implies$ $f(\frac a2)=0$ and so $f(\frac a{2^n})=0$ and so, using continuity, $f(0)=0$, which is wrong. 

So $f(x)>0$ $\forall x$

$P(x,x)$ $\implies$ $f(2x)=f(x)^4$
$P(2x,x)$ $\implies$ $f(3x)=f(x)^9$
And a simple induction gives then $f(nx)=f(x)^{n^2}$ $\forall n\in\mathbb N$

$P(0,x)$ $\implies$ $f(-x)=f(x)$ and so $f(nx)=f(x)^{n^2}$ $\forall n\in\mathbb Z$

$f(n\frac xn)=f(\frac xn)^{n^2}$ and $f(\frac xn)=f(x)^{\frac 1{n^2}}$

And so $f(x)=f(1)^{x^2}$ $\forall x\in\mathbb Q$

And so $f(x)=a^{x^2}$ $\forall x$ (using continuity) where $a>0$ which indeed is a solution.

Hence the solutions :
$f(x)=0$ $\forall x$
$f(x)=a^{x^2}$ $\forall x$ with any $a>0$
$f(x)=-a^{x^2}$ $\forall x$ with any $a>0$
\end{solution}



\begin{solution}[by \href{https://artofproblemsolving.com/community/user/29428}{pco}]
	\begin{tcolorbox}[list]$f(x+y)f(x-y) = (f(x)f(y))^2$, taking $log$ on both side, we get..

$log_{e}f(x+y) + log_{e}f(x-y) = 2(log_{e}f(x) + log_{e}f(y))$

and Let $log_{e}f(x) = g(x)$, we get $g(x+y) + g(x-y) = 2(g(x) + g(y))$

Now put $y = 1$,We get $g(x+1) + g(x-1) = 2(g(x) + g(1))$

$(g(x+1)  - g(x)) -  (g(x) - g(x-1)) = 2(g(1)) = constant.$

means $g(x) = ax^2 +bx +c$, now put into this equation , and get value of $a , b, c.$

 [\/list]\end{tcolorbox}
Your idea is interesting but you miss a lot of rigor. For example :

1) Before taking logs, you need to study the cases where $f(x)\le 0$ for some $x$ for example (which would forbid taking logs)

2) there are a lot of continuous functions such that $g(x+1) + g(x-1) = 2(g(x) + g(1))$ and quadratic are just a [very] specific solution. So your conclusion $g(x) = ax^2 +bx +c$ is wrong without further explanations.
\end{solution}



\begin{solution}[by \href{https://artofproblemsolving.com/community/user/73492}{sea rover}]
	\begin{tcolorbox}[quote="gouthamphilomath"]Determine all continuous functions $f$ such that
\[f(x + y)f(x - y) = (f(x)f(y))^2\:\:\:\: \forall(x, y) \in\mathbb{R}^2\]\end{tcolorbox}

Let $P(x,y)$ be the assertion $f(x+y)f(x-y)=f(x)^2f(y)^2$

If $f(0)=0$ : $P(x,x)$ implies $f(x)=0$ $\forall x$ which indeed is a solution
If $f(0)\ne 0$ : $P(0,0)$ $\implies$ $f(0)=\pm 1$. But $f(x)$ solution implies $-f(x)$ solution.

WLOG say then from now : $f(0)=1$

If $f(a)=0$ for some $a$, then $P(\frac a2,\frac a2)$ $\implies$ $f(\frac a2)=0$ and so $f(\frac a{2^n})=0$ and so, using continuity, $f(0)=0$, which is wrong. 

So $f(x)>0$ $\forall x$

$P(x,x)$ $\implies$ $f(2x)=f(x)^4$
$P(2x,x)$ $\implies$ $f(3x)=f(x)^9$
And a simple induction gives then $f(nx)=f(x)^{n^2}$ $\forall n\in\mathbb N$

$P(0,x)$ $\implies$ $f(-x)=f(x)$ and so $f(nx)=f(x)^{n^2}$ $\forall n\in\mathbb Z$

$f(n\frac xn)=f(\frac xn)^{n^2}$ and $f(\frac xn)=f(x)^{\frac 1{n^2}}$

And so $f(x)=f(1)^{x^2}$ $\forall x\in\mathbb Q$

And so $f(x)=a^{x^2}$ $\forall x$ (using continuity) where $a>0$ which indeed is a solution.

Hence the solutions :
$f(x)=0$ $\forall x$
$f(x)=a^{x^2}$ $\forall x$ with any $a>0$
$f(x)=-a^{x^2}$ $\forall x$ with any $a>0$\end{tcolorbox}

nice prove! :)
\end{solution}



\begin{solution}[by \href{https://artofproblemsolving.com/community/user/89670}{man111}]
	Yes Patric you are correct.....I have missed lot of things (domain of function....)

 and another nice solution.........
\end{solution}
*******************************************************************************
-------------------------------------------------------------------------------

\begin{problem}[Posted by \href{https://artofproblemsolving.com/community/user/3535}{hungvuong}]
	Find all continuous functions $f: \mathbb{R}\to \mathbb{R}$ such that 
\[f(10x +10y+2010) = 10 f(x) +10 f(y) +1010\]
for all $x,y \in \mathbb R$.
	\flushright \href{https://artofproblemsolving.com/community/c6h371159}{(Link to AoPS)}
\end{problem}



\begin{solution}[by \href{https://artofproblemsolving.com/community/user/29428}{pco}]
	\begin{tcolorbox}Find all f: \mathbb{R}\to \mathbb{R} continuous such that 
              f(10x +10y+2010) = 10 f(x) +10 f(y) +1010  with all x;y\end{tcolorbox}

Let $g(x)=f(x-\frac{2010}{19})+\frac{1010}{19}$

So $f(x)=g(x+\frac{2010}{19})-\frac{1010}{19}$ and the equation becomes :

$g(10x+10y+2010+\frac{2010}{19})-\frac{1010}{19}$ $=10g(x+\frac{2010}{19})+10g(y+\frac{2010}{19})-20\frac{1010}{19}+1010$ 

$\iff$ $g(10(x+\frac{2010}{19})+10(y+\frac{2010}{19}))$ $=10g(x+\frac{2010}{19})+10g(y+\frac{2010}{19})$ 

$\iff$ $g(10x+10y)=10g(x)+10g(y)$

Setting $x=y=0$, we get then $g(0)=0$
Setting $y=0$, we get $g(10x)=10g(x)$ and so $g(10y)=10g(y)$ and so $g(10x+10y)=g(10x)+g(10y)$ and so $g(x+y)=g(x)+g(y)$

This is the very very classical Cauchy equatgion whose continuous solutions are $g(x)=ax$

And so $\boxed{f(x)=a(x+\frac{2010}{19})-\frac{1010}{19}}$
\end{solution}
*******************************************************************************
-------------------------------------------------------------------------------

\begin{problem}[Posted by \href{https://artofproblemsolving.com/community/user/74657}{ArefS}]
	Find all functions $f:\mathbb R \rightarrow \mathbb R$ satisfying
\[f(x)f(y)\le f(xy) \quad \text{and} \quad f(x)+f(y)\le f(x+y)\]
for all reals $x$ and $y$.
	\flushright \href{https://artofproblemsolving.com/community/c6h371349}{(Link to AoPS)}
\end{problem}



\begin{solution}[by \href{https://artofproblemsolving.com/community/user/29428}{pco}]
	\begin{tcolorbox}Find All functions $f:\mathbb R \rightarrow \mathbb R$ satisfying:

$f(x)f(y)\le f(xy)$

and

$f(x)+f(y)\le f(x+y)$.\end{tcolorbox}
$f(x)=0$ $\forall x$ is a solution.
So let us from now look for non all-zero solutions

Let $a$ such that $f(a)\ne 0$
Let $P(x,y)$ be the assertion $f(x)f(y)\le f(xy)$
Let $Q(x,y)$ be the assertion $f(x)+f(y)\le f(x+y)$

$P(0,0)$ $\implies$ $f(0)\ge f(0)^2\ge 0$
$Q(0,0)$ $\implies$ $f(0)\le 0$
So $f(0)=0$

$P(a,a)$ $\implies$ $f(a^2)>0$
$Q(a^2,-a^2)$ $\implies$ $f(-a^2)<0$
$P(a^2,1)$ $\implies$ $f(a^2)f(1)\le f(a^2)$ and so $f(1)\le 1$
$P(-a^2,1)$ $\implies$ $f(-a^2)f(1)\le f(-a^2)$ and so $f(1)\ge 1$
So $f(1)=1$

$Q(-1,1)$ $\implies$ $f(-1)\le -1$
$P(-1,-1)$ $\implies$ $f(-1)^2\le 1$
So $f(-1)=-1$

$P(x,-1)$ $\implies$ $-f(x)\le f(-x)$
$Q(x,-x)$ $\implies$ $f(-x)+f(x)\le 0$
So $f(-x)=-f(x)$

$P(x,-y)$ $\implies$ $-f(x)f(y)\le -f(xy)$
$P(x,y)$ $\implies$ $f(x)f(y)\le f(xy)$
So $f(xy)=f(x)f(y)$

$Q(-x,-y)$ $\implies$ $-f(x)-f(y)\le -f(x+y)$
$Q(x,y)$ $\implies$ $f(x)+f(y)\le f(x+y)$
So $f(x+y)=f(x)+f(y)$

And so this is a very very classical equation (Cauchy + $f(xy)=f(x)f(y)$) whose non all zero solution is $f(x)=x$

Hence the two solutions :
$f(x)=0$ $\forall x$
$f(x)=x$ $\forall x$
\end{solution}



\begin{solution}[by \href{https://artofproblemsolving.com/community/user/74657}{ArefS}]
	\begin{tcolorbox}
And so this is a very very classical equation (Cauchy +$ f(xy)=f(x)f(y)$) whose non all zero solution is $f(x)=x$\end{tcolorbox}

Dear pco,

I have got some problems with the statement you mentioned, Could you please solve this classical equation?
\end{solution}



\begin{solution}[by \href{https://artofproblemsolving.com/community/user/29428}{pco}]
	\begin{tcolorbox}[quote="pco"]
And so this is a very very classical equation (Cauchy +$ f(xy)=f(x)f(y)$) whose non all zero solution is $f(x)=x$\end{tcolorbox}

Dear pco,

I have got some problems with the statement you mentioned, Could you please solve this classical equation?\end{tcolorbox}

Let $P(x,y)$ be the assertion $f(x+y)=f(x)+f(y)$

$P(0,0)$ $\implies$ $f(0)=0$
$P(x,-x)$ $\implies$ $f(-x)=-f(x)$

A quick induction (starting with $f(0)=0$) implies $f(nx)=nf(x)$ $\forall n\in\mathbb N\cup\{0\}$

So $f(q\frac xq)=qf(\frac xq)$ and so $f(\frac xq)=\frac 1qf(x)$ $\forall q\in\mathbb N$

So $f(\frac pqx)=\frac pqf(x)$ $\forall p\in\mathbb N\cup\{0\}$ $\forall q\in\mathbb N$

And since $f(-x)=-f(x)$, we get $f(x)=xf(1)$ $\forall x\in\mathbb Q$

$f(x)f(y)=f(xy)$ $\implies$ $f(x^2)=f(x)^2\ge 0$ and so $f(x)\ge 0$ $\forall x\ge 0$
So, $y>x$ $\implies$ $f(y)=f(x+(y-x))=f(x)+f(y-x)\ge f(x)$ and $f(x)$ is non decreasing

Let then any $x\notin \mathbb Q$
Let $a_n$ any increasing sequence of rational numbers whose limit is $x$
Let $b_n$ any decreasing sequence of rational numbers whose limit is $x$

$a_n<x<b_n$ $\implies$ $f(a_n)\le f(x)\le f(b_n)$ since $f(x)$ is non decreasing

So $a_nf(1)\le f(x)\le b_nf(1)$

Setting $n\to+\infty$ in this inequality, we get $f(x)=xf(1)$

So $f(x)=ax$ $\forall x$
Then $f(xy)=f(x)f(y)$ $\implies$ $axy=a^2xy$ $\forall x,y$ and so $a=a^2$ and so $a=1$ (we were looking only for non allzero solutions)

And so the only non all zero solution is $f(x)=x$
\end{solution}
*******************************************************************************
-------------------------------------------------------------------------------

\begin{problem}[Posted by \href{https://artofproblemsolving.com/community/user/31067}{ridgers}]
	Find all functions $f: \{1,2,\ldots,10\} \to \{1,2,\ldots,100 \}$ such that $f$ is strictly increasing and $xf(x)+yf(y)$ is divisible by $x+y$ for all $x,y \in \{1,2,\ldots,10\}$.
	\flushright \href{https://artofproblemsolving.com/community/c6h371566}{(Link to AoPS)}
\end{problem}



\begin{solution}[by \href{https://artofproblemsolving.com/community/user/29428}{pco}]
	\begin{tcolorbox}Define all the functions $f: \{1,2,\ldots,10\} \to \{1,2,\ldots,100 \}$ such as :They are  monoton increasing functions and $xf(x)+yf(y)$ is divisible by $x+y$ for all $x,y \in \{1,2,\ldots,10\}.$\end{tcolorbox}
I suppose that "monoton increasing" means "$x>y\implies f(x)>f(y)$"

Let $y=x+1$

$x+x+1|xf(x)+(x+1)f(x+1)$ $\implies$ $2x+1|xf(x)+(x+1)f(x+1)-(2x+1)f(x)=(x+1)(f(x+1)-f(x))$ $\implies$ $2x+1|f(x+1)-f(x)$

So $f(x+1)\ge f(x)+2x+1$

So (summing) $f(x)\ge x^2+f(1)-1$

So $f(10)\ge f(1)+99$ and $f(1)=1$ and $f(10)=100$

So $f(9)\ge 81$ and $f(10)\ge f(9)+19$ and so $f(9)=81$

And it's immediate to conclude $\boxed{f(x)=x^2}$ which indeed is a solution
\end{solution}



\begin{solution}[by \href{https://artofproblemsolving.com/community/user/31067}{ridgers}]
	sorry PCO but I couldn't understand this part: $ 2x+1|xf(x)+(x+1)f(x+1)-(2x+1)f(x)=(x+1)(f(x+1)-f(x)) $
\end{solution}



\begin{solution}[by \href{https://artofproblemsolving.com/community/user/29428}{pco}]
	\begin{tcolorbox}sorry PCO but I couldn't understand this part: $ 2x+1|xf(x)+(x+1)f(x+1)-(2x+1)f(x)=(x+1)(f(x+1)-f(x)) $\end{tcolorbox}

$x+(x+1)$ divides $xf(x)+(x+1)f(x+1)$

$2x+1$ divides $xf(x)+(x+1)f(x+1)$

$2x+1$ divides $xf(x)+(x+1)f(x+1)-(2x+1)\times\text{ anything}$

$2x+1$ divides $xf(x)+(x+1)f(x+1)-(2x+1)f(x)$

$2x+1$ divides $(x+1)f(x+1)-(x+1)f(x)$

$2x+1$ divides $(x+1)(f(x+1)-f(x))$

And since $\gcd(2x+1,x+1)=1$, we get $2x+1$ divides $f(x+1)-f(x)$
\end{solution}
*******************************************************************************
-------------------------------------------------------------------------------

\begin{problem}[Posted by \href{https://artofproblemsolving.com/community/user/91306}{ndk09}]
	Find all one-to-one continuous functions $f: \mathbb R \to \mathbb R$ such that:
i) For all real $x$, $f(2x - f(x)) = x$, and
ii) There exists $x_0 \in \mathbb R $ such that $ f(x_0) =x_0$.
	\flushright \href{https://artofproblemsolving.com/community/c6h371746}{(Link to AoPS)}
\end{problem}



\begin{solution}[by \href{https://artofproblemsolving.com/community/user/61513}{Obel1x}]
	\begin{tcolorbox}Find all one-to-one countinous function $f : R \longrightarrow R $ such that:
$ i) f(2x - f(x)) = x \forall x \in R$
$ ii) \exists x_0 \in R $ such that $ f(x_0) =x_0$\end{tcolorbox}
It's obviously that $f$ is a linear function. Let $f(x)=ax+b$, where $a,b \in \mathbb{R}$. From $i)$ we have:
\begin{eqnarray*}f\{2x-f(x)\}&=&x \\ f(2x-ax-b)&=&x\\ a\{x(2-a)-b\}+b&=&x \\ x(2a-a^2)-ab+b&=&x\end{eqnarray*}
This holds $2a-a^2=1$ and $-ab+b=0$. This implies $a=1$ and $\forall \ b \in \mathbb{R}$. Hence the function that satisfies the condition $i)$ is $ f(x)=x+b, \ \ \forall \ b \in \mathbb{R}$. This function satisfies condition $ii)$ when $b=0$. Hence the solution is $\boxed{f(x)=x}$.
\end{solution}



\begin{solution}[by \href{https://artofproblemsolving.com/community/user/20335}{powerchess}]
	why is it linear?
\end{solution}



\begin{solution}[by \href{https://artofproblemsolving.com/community/user/61513}{Obel1x}]
	\begin{tcolorbox}why is it linear?\end{tcolorbox}

Because for higher degree functions, we have more parameters, hence we must have more conditions in order to solve it.
\end{solution}



\begin{solution}[by \href{https://artofproblemsolving.com/community/user/20335}{powerchess}]
	I'm sorry, I do not really understand your reasoning. I mean how do you rule out exponential or other such functions?
\end{solution}



\begin{solution}[by \href{https://artofproblemsolving.com/community/user/91306}{ndk09}]
	\begin{tcolorbox}[quote="ndk09"]Find all one-to-one countinous function $f : R \longrightarrow R $ such that:
$ i) f(2x - f(x)) = x \forall x \in R$
$ ii) \exists x_0 \in R $ such that $ f(x_0) =x_0$\end{tcolorbox}
It's obviously that $f$ is a linear function. Let $f(x)=ax+b$, where $a,b \in \mathbb{R}$. From $i)$ we have:
\begin{eqnarray*}f\{2x-f(x)\}&=&x \\ f(2x-ax-b)&=&x\\ a\{x(2-a)-b\}+b&=&x \\ x(2a-a^2)-ab+b&=&x\end{eqnarray*}
This holds $2a-a^2=1$ and $-ab+b=0$. This implies $a=1$ and $\forall \ b \in \mathbb{R}$. Hence the function that satisfies the condition $i)$ is $ f(x)=x+b, \ \ \forall \ b \in \mathbb{R}$. This function satisfies condition $ii)$ when $b=0$. Hence the solution is $\boxed{f(x)=x}$.\end{tcolorbox}


I think it isn't true.
\end{solution}



\begin{solution}[by \href{https://artofproblemsolving.com/community/user/29428}{pco}]
	\begin{tcolorbox}Find all one-to-one countinous function $f : R \longrightarrow R $ such that:
$ i) f(2x - f(x)) = x \forall x \in R$
$ ii) \exists x_0 \in R $ such that $ f(x_0) =x_0$\end{tcolorbox}
Let $P(x)$ be the assertion $f(2x-f(x))=x$

$f(x)$ is monotonous (since injective and continuous) and so $f^{[2n]}(x)$ is increasing (composition of function $f(x)$ $2n$ times)

$P(f(x))$ $\implies$ $f(2f(x)-f(f(x))=f(x)$ and so $f(f(x))=2f(x)-x$ since $f(x)$ is injective.
Simple induction shows then $f^{[n]}(x)=n(f(x)-x)+x$

Let then $x_1>x_2$ : $f^{[2n]}(x_1)-f^{[2n]}(x_2)=2n(f(x_1)-f(x_2))-(2n-1)(x_1-x_2)>0$ and so $f(x_1)-f(x_2)>(1-\frac 1{2n})(x_1-x_2)$
Setting $n\to +\infty$, we get $f(x_1)-f(x_2)\ge x_1-x_2$

Let then $g(x)=f(x)-x$ 

We have just seen that $x_1>x_2$ $\implies$ $g(x_1)\ge g(x_2)$ and so $g(x)$ is non decreasing.
So, if $a<b$ and $g(a)=g(b)$, we get that $g(x)=g(a)=g(b)$ $\forall x\in[a,b]$

$f(2x-f(x))=x$ $\implies$ $g(x-g(x))=g(x)$ and so, with easy induction, $g(x-ng(x))=g(x)$ $\forall n\in\mathbb N$
$f(f(x))=2f(x)-x$ $\implies$ $g(x+g(x)=g(x)$ and so, with easy induction, $g(x+ng(x))=g(x)$ $\forall n\in\mathbb N$

And so $g(x)=g(y)$ $\forall y\in[x-n|g(x)|,x+n|g(x)|]$ and so $g(x)=c$ is a constant function

So $f(x)=x+c$ and the constraint ii) implies that $\boxed{f(x)=x}$ $\forall x$
\end{solution}



\begin{solution}[by \href{https://artofproblemsolving.com/community/user/20335}{powerchess}]
	why is $f^{2n}(x)$ increasing? I agree it must be monotone, but am not sure why it is monotone increasing.
\end{solution}



\begin{solution}[by \href{https://artofproblemsolving.com/community/user/29428}{pco}]
	\begin{tcolorbox}why is $f^{2n}(x)$ increasing? I agree it must be monotone, but am not sure why it is monotone increasing.\end{tcolorbox}

If $h(x)$ is increasing, then $h(h(x))$ is increasing
If $h(x)$ is decreasing, then $h(h(x))$ is increasing

So $h(x)$ monotonous always implies $h(h(x))$ monotonous increasing

So $f(x)$ monotonous implies $f^{[n]}(x)$ monotonous so $f^{[2n]}(x)=f^{[n]}(f^{[n]}(x))$ increasing
\end{solution}
*******************************************************************************
-------------------------------------------------------------------------------

\begin{problem}[Posted by \href{https://artofproblemsolving.com/community/user/86021}{Headhunter}]
	Find all functions $F: \mathbb R \to \mathbb R$ such that for all reals $x$ and $y$, \[F(x+F(y))=y+F(x+1).\]
	\flushright \href{https://artofproblemsolving.com/community/c6h371903}{(Link to AoPS)}
\end{problem}



\begin{solution}[by \href{https://artofproblemsolving.com/community/user/29428}{pco}]
	\begin{tcolorbox}Hello.

Find all continuous functions $F:R\rightarrow R$ such that for all real numbers $x$ and $y$, $F(x+F(y))=y+F(x+1)$\end{tcolorbox}

let $f(x)=g(x)+1$ : the equation becomes $g(x+1+g(y))=y+g(x+1)$ and so assertion $P(x,y)$ : $g(x+g(y))=g(x)+y$

$P(0,x)$ $\implies$ $g(g(x))=x+g(0)$ and so $g(x)$ is bijective
$P(x,0)$ $\implies$ $g(x+g(0))=g(x)$ and so $x+g(0)=x$ (since bijective) and so $g(0)=0$ and $g(g(x))=x$
$P(x,g(y))$ $\implies$ $g(x+y)=g(x)+g(y)$ and so, since continuous solution of Cauchy's equation, $g(x)=ax$

So $f(x)=ax+1$ and, plugging back in the original equation, $a=\pm 1$ and so two solutions :
$f(x)=x+1$
$f(x)=1-x$
\end{solution}
*******************************************************************************
-------------------------------------------------------------------------------

\begin{problem}[Posted by \href{https://artofproblemsolving.com/community/user/31067}{ridgers}]
	Suppose that $n$ is a fixed natural number. We denote by $P(n)$ the number of all functions $ f: \mathbb R\to \mathbb R $ of the form $f(x)=ax^2+bx+c$ where $a,b,c \in \{1,2,\ldots,n\}$ and such that the roots of $f(x)=0$ are integral. Prove that for every integer $n\geq 4$, the following inequality holds: \[n<P(n)<n^2.\]
	\flushright \href{https://artofproblemsolving.com/community/c6h372007}{(Link to AoPS)}
\end{problem}



\begin{solution}[by \href{https://artofproblemsolving.com/community/user/29428}{pco}]
	\begin{tcolorbox}Given $n$ a fixed natural number. We denote by $P(n)$ the number of all functions $ F:R\rightarrow R $ of the form: $f(x)=ax^2+bx+c$ where $a,b,c$ $\in$ $\{$ $1,2,...,n$ $\}$ and such that the roots of $f(x)=0$ are real. Prove that for every $n\geq 4$ , the following inequality holds: $n<P(n)<n^2$.\end{tcolorbox}
Wrong :

Choose $n=6$ and you get at least $43$ solutions and $n^2=36<43\le P(6)$ :
$1$ : $(a,b,c)=(1,2,1)$ $\implies$ equation $x^2+2x+1$ whose discrimant is $0\ge 0$ and so has all its roots real
$2$ : $(a,b,c)=(1,3,1)$ $\implies$ equation $x^2+3x+1$ whose discrimant is $5\ge 0$ and so has all its roots real
$3$ : $(a,b,c)=(1,3,2)$ $\implies$ equation $x^2+3x+2$ whose discrimant is $1\ge 0$ and so has all its roots real
$4$ : $(a,b,c)=(1,4,1)$ $\implies$ equation $x^2+4x+1$ whose discrimant is $12\ge 0$ and so has all its roots real
$5$ : $(a,b,c)=(1,4,2)$ $\implies$ equation $x^2+4x+2$ whose discrimant is $8\ge 0$ and so has all its roots real
$6$ : $(a,b,c)=(1,4,3)$ $\implies$ equation $x^2+4x+3$ whose discrimant is $4\ge 0$ and so has all its roots real
$7$ : $(a,b,c)=(1,4,4)$ $\implies$ equation $x^2+4x+4$ whose discrimant is $0\ge 0$ and so has all its roots real
$8$ : $(a,b,c)=(1,5,1)$ $\implies$ equation $x^2+5x+1$ whose discrimant is $21\ge 0$ and so has all its roots real
$9$ : $(a,b,c)=(1,5,2)$ $\implies$ equation $x^2+5x+2$ whose discrimant is $17\ge 0$ and so has all its roots real
$10$ : $(a,b,c)=(1,5,3)$ $\implies$ equation $x^2+5x+3$ whose discrimant is $13\ge 0$ and so has all its roots real
$11$ : $(a,b,c)=(1,5,4)$ $\implies$ equation $x^2+5x+4$ whose discrimant is $9\ge 0$ and so has all its roots real
$12$ : $(a,b,c)=(1,5,5)$ $\implies$ equation $x^2+5x+5$ whose discrimant is $5\ge 0$ and so has all its roots real
$13$ : $(a,b,c)=(1,5,6)$ $\implies$ equation $x^2+5x+6$ whose discrimant is $1\ge 0$ and so has all its roots real
$14$ : $(a,b,c)=(1,6,1)$ $\implies$ equation $x^2+6x+1$ whose discrimant is $32\ge 0$ and so has all its roots real
$15$ : $(a,b,c)=(1,6,2)$ $\implies$ equation $x^2+6x+2$ whose discrimant is $28\ge 0$ and so has all its roots real
$16$ : $(a,b,c)=(1,6,3)$ $\implies$ equation $x^2+6x+3$ whose discrimant is $24\ge 0$ and so has all its roots real
$17$ : $(a,b,c)=(1,6,4)$ $\implies$ equation $x^2+6x+4$ whose discrimant is $20\ge 0$ and so has all its roots real
$18$ : $(a,b,c)=(1,6,5)$ $\implies$ equation $x^2+6x+5$ whose discrimant is $16\ge 0$ and so has all its roots real
$19$ : $(a,b,c)=(1,6,6)$ $\implies$ equation $x^2+6x+6$ whose discrimant is $12\ge 0$ and so has all its roots real
$20$ : $(a,b,c)=(2,3,1)$ $\implies$ equation $2x^2+3x+1$ whose discrimant is $1\ge 0$ and so has all its roots real
$21$ : $(a,b,c)=(2,4,1)$ $\implies$ equation $2x^2+4x+1$ whose discrimant is $8\ge 0$ and so has all its roots real
$22$ : $(a,b,c)=(2,4,2)$ $\implies$ equation $2x^2+4x+2$ whose discrimant is $0\ge 0$ and so has all its roots real
$23$ : $(a,b,c)=(2,5,1)$ $\implies$ equation $2x^2+5x+1$ whose discrimant is $17\ge 0$ and so has all its roots real
$24$ : $(a,b,c)=(2,5,2)$ $\implies$ equation $2x^2+5x+2$ whose discrimant is $9\ge 0$ and so has all its roots real
$25$ : $(a,b,c)=(2,5,3)$ $\implies$ equation $2x^2+5x+3$ whose discrimant is $1\ge 0$ and so has all its roots real
$26$ : $(a,b,c)=(2,6,1)$ $\implies$ equation $2x^2+6x+1$ whose discrimant is $28\ge 0$ and so has all its roots real
$27$ : $(a,b,c)=(2,6,2)$ $\implies$ equation $2x^2+6x+2$ whose discrimant is $20\ge 0$ and so has all its roots real
$28$ : $(a,b,c)=(2,6,3)$ $\implies$ equation $2x^2+6x+3$ whose discrimant is $12\ge 0$ and so has all its roots real
$29$ : $(a,b,c)=(2,6,4)$ $\implies$ equation $2x^2+6x+4$ whose discrimant is $4\ge 0$ and so has all its roots real
$30$ : $(a,b,c)=(3,4,1)$ $\implies$ equation $3x^2+4x+1$ whose discrimant is $4\ge 0$ and so has all its roots real
$31$ : $(a,b,c)=(3,5,1)$ $\implies$ equation $3x^2+5x+1$ whose discrimant is $13\ge 0$ and so has all its roots real
$32$ : $(a,b,c)=(3,5,2)$ $\implies$ equation $3x^2+5x+2$ whose discrimant is $1\ge 0$ and so has all its roots real
$33$ : $(a,b,c)=(3,6,1)$ $\implies$ equation $3x^2+6x+1$ whose discrimant is $24\ge 0$ and so has all its roots real
$34$ : $(a,b,c)=(3,6,2)$ $\implies$ equation $3x^2+6x+2$ whose discrimant is $12\ge 0$ and so has all its roots real
$35$ : $(a,b,c)=(3,6,3)$ $\implies$ equation $3x^2+6x+3$ whose discrimant is $0\ge 0$ and so has all its roots real
$36$ : $(a,b,c)=(4,4,1)$ $\implies$ equation $4x^2+4x+1$ whose discrimant is $0\ge 0$ and so has all its roots real
$37$ : $(a,b,c)=(4,5,1)$ $\implies$ equation $4x^2+5x+1$ whose discrimant is $9\ge 0$ and so has all its roots real
$38$ : $(a,b,c)=(4,6,1)$ $\implies$ equation $4x^2+6x+1$ whose discrimant is $20\ge 0$ and so has all its roots real
$39$ : $(a,b,c)=(4,6,2)$ $\implies$ equation $4x^2+6x+2$ whose discrimant is $4\ge 0$ and so has all its roots real
$40$ : $(a,b,c)=(5,5,1)$ $\implies$ equation $5x^2+5x+1$ whose discrimant is $5\ge 0$ and so has all its roots real
$41$ : $(a,b,c)=(5,6,1)$ $\implies$ equation $5x^2+6x+1$ whose discrimant is $16\ge 0$ and so has all its roots real
$42$ : $(a,b,c)=(6,5,1)$ $\implies$ equation $6x^2+5x+1$ whose discrimant is $1\ge 0$ and so has all its roots real
$43$ : $(a,b,c)=(6,6,1)$ $\implies$ equation $6x^2+6x+1$ whose discrimant is $12\ge 0$ and so has all its roots real
\end{solution}



\begin{solution}[by \href{https://artofproblemsolving.com/community/user/31067}{ridgers}]
	Sorry PCO, I saw the question again and instead of real it should be integral. Sorry again.
\end{solution}



\begin{solution}[by \href{https://artofproblemsolving.com/community/user/48552}{ocha}]
	This turned out to be much harder than i first thought, do you have a nicer solution?

\begin{bolded}Lower Bound:\end{bolded} $f(x)=x^2+(k+1)x+k = (x+1)(x+k)$ gives $n-1$ functions, then because $n\ge 4$ we can include $x^2+4x+3$ and $2x^2+4x+2$ giving $P(n)>n$

\begin{bolded}Upper Bound\end{bolded}: for $n=1$ the claim is trivial. Assume $P(k)<k^2$ then consider $n=k+1$, we examine the cases where at least one of $a,b,c=k+1$. We cannot have $a= k+1$. Write $(a,b,c)=(a,a\cdot (r_1+r_2),a\cdot r_1\cdot r_2)$ where $r_1,r_2$ are the integer roots of $f$, clearly $r_1,r_2$ must be positive integers.

\begin{bolded}1) \end{bolded} First we count the number of solutions to $b=a\cdot (r_1+r_2)=k+1$. 

If $\max\{r_1,r_2\}\ge 2$ we get $c =a\cdot r_1\cdot r_2 \ge a(r_1+r_2)=k+1$ with equality iff $r_1=r_2=2$. Hence the only soutions are $(r_1, r_2, a) = (1,r_2,\textstyle\frac{k+1}{r_2})$ (at most k solutions) or $(2,2, \frac{k+1}{4})$ (1 solution). So there are at most $k+1$ such functions when $b=k+1$

\begin{bolded}2) \end{bolded}
\begin{bolded}Lemma:\end{bolded} For some fixed $n\in \mathbb{N}$, the number of solutions to to $a\cdot b \cdot c = n$ for $(a,b,c)\in \mathbb{N}^3$, with $b\ge c$  is at most $n$. 

[hide="proof "]
============================Proof===================================
Let $n = p_1^{e_1}\cdot p_2^{e_2} \cdot \cdot \cdot p_k^{e_k}$ for distinct primes $p_1,...,p_k$ then the number of triples $(a,b,c)$ where we ignore the fact that $b\ge c$ is equal to $\prod_{j=1}^k \binom{e_j+2}{2}$ (*). This is because $\textstyle\binom{e_j+2}{2}$ is the number of ways to distibute $e_j$ copies of prime $p_j$ among $a,b,c$). 

The number of triples $(a,b,c)$ where $b=c$ is given by $\lfloor\frac{e_1}{2}+1\rfloor\cdot\cdot\cdot \lfloor\frac{e_k}{2}+1 \rfloor$. Because we either give no primes, $p_i$, to $b,c$ or give both $b$ and $c$ the sume number of primes, $p_i$. Now it follows that the number of triples $(a,b,c)$ with $b\ge c$ is

\[ \frac{1}{2}(\prod_{j=1}^k \binom{e_j+2}{2} + \lfloor \frac{e_1}{2}+1 \rfloor \cdot \cdot \cdot \lfloor\frac{e_k}{2}+1\rfloor) \qquad (**)\] 

Now we will show that (**) is less that $n$. 

First if all $e_i \le 1$ then ${\lfloor \frac{e_1}{2}+1\rfloor \cdot \cdot \cdot \lfloor\frac{e_k}{2}+1\rfloor})=1$
Now using the fact that $\binom{e_j+2}{2} > 2^{e_j+1}$ and $\binom{e_j+2}{2} \ge p^{e_j}$ where $p$ is a prime greater than $2$. And the fact that $n \ge 2^{e_1}\cdot 3^{e_2} \cdot 5^{e_3}\cdot\cdot\cdot$ If follows from (**) that we must have at most $n$ solutions $(a,b,c)$. Now if any value $e_i$ is increased by one, then $n=p_1^{e_1}\cdot\cdot p_k^{e_k}$ will at least double, while (**) will at most double. Hence $(**)$ is at most $n$
===========================End of Proof========================================
[\/hide]

From the lemma we know that if $c=a\cdot r_1\cdot r_2 = k+1$. then there is at most $k+1$ triples $(a,r_1,r_2)$. But one of these triples is $(k+1,1,1)$ which gives the function $f(x)=(k+1)x^2 + 2(k+1)x + k+1$ but this is not allowed since $2(k+1) > k+1$ hence there are only $k$ such tiples

From (1) and (2) we have at most $2k+1$ functions with at least one of $a,b,c=k+1$. so $P(k+1) \le P(k) + 2k+1 < (k+1)^2$ so we are done. Phhhew!
\end{solution}
*******************************************************************************
-------------------------------------------------------------------------------

\begin{problem}[Posted by \href{https://artofproblemsolving.com/community/user/70363}{hoangvn.fix}]
	Find all functions $f: \mathbb R \to \mathbb R$ such that for all reals $x $ and $y$,
\[(f(x))^2+2yf(x)+f(y)=f(y+f(x)).\]
	\flushright \href{https://artofproblemsolving.com/community/c6h372154}{(Link to AoPS)}
\end{problem}



\begin{solution}[by \href{https://artofproblemsolving.com/community/user/29428}{pco}]
	\begin{tcolorbox}Find all the function $ f:R \rightarrow R $ such that
$(f(x))^2+2yf(x)+f(y)=f(y+f(x))$ for all the real numbers $x,y$\end{tcolorbox}
$f(x)=0$ $\forall x$ is obviously a solution.
Let us from now look for non all-zero solutions.

Let then $P(x,y)$ be the assertion $f(x)^2+2yf(x)+f(y)=f(y+f(x))$
Let $a=f(0)$
Let $b$ such that $f(b)\ne 0$

$P(b,\frac x{2f(b)}-\frac{f(b)}2)$ $\implies$ $x=f(\frac x{2f(b)}+\frac{f(b)}2)$ $-f(\frac x{2f(b)}-\frac{f(b)}2)$ $=f(u)-f(v)$ where :

$u=\frac x{2f(b)}+\frac{f(b)}2$
$v=\frac x{2f(b)}-\frac{f(b)}2$
Then :

$P(v,-f(v))$ $\implies$ $f(v)^2+a=f(-f(v))$
$P(u,-f(v))$ $\implies$ $f(u)^2-2f(u)f(v)+f(-f(v))=f(f(u)-f(v))=f(x)$
Adding these two lines, we get $f(x)=f(u)^2-2f(u)f(v)+f(v)^2+a=(f(u)-f(v))^2+a=x^2+a$

And it is easy to check back that this function indeed is a solution.

Hence the two solutions :
$f(x)=0$ $\forall x$
$f(x)=x^2+a$ $\forall x$
\end{solution}
*******************************************************************************
-------------------------------------------------------------------------------

\begin{problem}[Posted by \href{https://artofproblemsolving.com/community/user/85501}{taylorcute56}]
	1. Find all functions $f: \mathbb R \to \mathbb R$ such that for all reals $x$ and $y$,
\[ f(x+y-xy) = f(x) + f(y) - f(xy) .\]
2. Find all functions $f: \mathbb N \to \mathbb N$ such that $f(f(n)) = 2n$ for all positive integers $n$.
	\flushright \href{https://artofproblemsolving.com/community/c6h372186}{(Link to AoPS)}
\end{problem}



\begin{solution}[by \href{https://artofproblemsolving.com/community/user/29428}{pco}]
	\begin{tcolorbox}1. Find all funtion continious f : R -> R such that : f(x+y-xy) = f(x) + f(y) - f(xy) 
\end{tcolorbox}
Let $P(x,y)$ be the assertion $f(x+y-xy)=f(x)+f(y)-f(xy)$
If $f(x)$ is a solution, so is $f(x)+c$. So Wlog say $f(0)=0$

Let $h(x)=f(x)+f(-x)$
$P(x,-x)$ $\implies$ $h(x^2)=h(x)$ and so, since continuous, $h(x)=c$ and so $h(x)=0$ and $f(-x)=-f(x)$
[hide="why?"]
======================= begin of hidden part =========================
Consider $x\in(0,1)$ : $h(x)=h(x^2)$ implies $h(x)=h(x^{2^n})$ and taking $n\to+\infty$ and using continuity : $h(x)=h(0)$
Consider $x\in(1,+\infty)$ : $h(x)=h(x^{\frac 12})$ implies $h(x)=h(x^{2^{-n}})$ and taking $n\to+\infty$ and using continuity : $h(x)=h(1)$
Continuity at $1$ implies $h(1)=h(0)$ and $h(x)=h(0)$ $\forall x\ge 0$
$h(x^2)=h(x)$ $\implies$ $h(-x)=h(x)$ and so $h(x)=h(0)$ $\forall x$
=======================  end of hidden part  =========================[\/hide]

$P(x,y)$ $\implies$ $f(x+y-xy)=f(x)+f(y)-f(xy)$
$P(-x,-y)$ $\implies$ $f(x+y+xy)=f(x)f(y)+f(xy)$
Subtracting, we get $f(x+y+xy)=f(x+y-xy)+2f(xy)$

Let then $u,v$ with $v\le 0$ : $\exists x,y$ such that $x+y=u$ and $xy=v$ and so $f(u+v)=f(u-v)+2f(v)$
Let then $u,v$ with $v>0$ : $\exists x,y$ such that $x+y=u$ and $xy=-v$ and so $f(u-v)=f(u+v)+2f(-v)$

And so $f(u+v)=f(u-v)+2f(v)$ $\forall u,v$
So $f(u+2v)=f(u)+2f(v)$ and so $f(2v)=2f(v)$ and $f(x+y)=f(x)+f(y)$ $\forall x,y$

This is the classical Cauchy equation whose solution, since continuous, is $f(x)=ax$

Hence the general solution : $\boxed{f(x)=ax+b}$
\end{solution}



\begin{solution}[by \href{https://artofproblemsolving.com/community/user/29428}{pco}]
	\begin{tcolorbox}2. Find all funtion f : N* -> N* such that f(f(n)) = 2n\end{tcolorbox}

1) please, try to post one problem per thread
2) This has been posted many many many times.

The general solution is :
(I dont know the meaning of $\mathbb N^*$. If this means $N\cup\{0\}$, then it's easy to show that $f(0)=0$)

Let $A$ be the set of all odd positive integers.
Let $B,C$ a partition of $A$ ($B\cap C=\emptyset$ and $B\cup C=A$) in two equinumerous subsets.
Let $g(x)$ a bijection from $B\to C$

For any $x\in \mathbb N$, let $x=2^{n(x)}a(x)$ where $a(x)\in A$ is odd. Then :

If $a(x)\in B$ : $f(x)=2^{n(x)}g(a(x))$
If $a(x)\in C$ : $f(x)=2^{n(x)+1}g^{[-1]}(a(x))$
\end{solution}
*******************************************************************************
-------------------------------------------------------------------------------

\begin{problem}[Posted by \href{https://artofproblemsolving.com/community/user/84155}{truongtansang89}]
	Find all functions $f: \mathbb R \to \mathbb R$ such that for all reals $x$ and $y$,
\[f(|x|+y+f(y+f(y))) = 3y + |f(x)|.\]
	\flushright \href{https://artofproblemsolving.com/community/c6h372591}{(Link to AoPS)}
\end{problem}



\begin{solution}[by \href{https://artofproblemsolving.com/community/user/29428}{pco}]
	\begin{tcolorbox}Find all $f : R \rightarrow R$ satisfying :

   $f(|x|+y+f(y+f(y))) = 3y + |f(x)|$
\end{tcolorbox}
Let $P(x,y)$ be the assertion $f(|x|+y+f(y+f(y)))=3y+|f(x)|$

1) $f(x)=0$ $\iff$ $x=0$
=================
$f(x)$ is obviously a surjection and so let $u$ such that $f(u)=0$.
Comparing $P(u,0)$ and $P(-u,0)$, we get $f(-u)=0$ and so wlog say $u\ge 0$

$P(u,-u)$ $\implies$ $f(0)=-3u\le 0$ and we know that $u=-\frac{f(0)}3$ is unique. (and so at most two values $u,-u$ such that $f(x)=0$).

$P(u,0)$ $\implies$ $f(u+f(f(0)))=0$ and so either $u+f(f(0))=u$, either $u+f(f(0))=-u$ and so $f(f(0))\le 0$

$P(f(f(0)),0)$ $\implies$ $f(|f(f(0))|+f(f(0)))=|f(f(f(0)))|\ge 0$ and so $f(0)\ge 0$ 

And, since $f(0)=-3u\le 0$, we get $f(0)=0$ and so $u=0$
Q.E.D.

2) $f(x_1)=f(x_2)$ with $x_1,x_2\ge 0$ $\implies$ $x_1=x_2$
==========================================
$P(x,0)$ $\implies$ $f(|x|)=|f(x)|$ and so $f(x)>0$ $\forall x>0$

Let then $x>0$
$P(x,-\frac{f(x)}3)$ $\implies$ $f(x-\frac{f(x)}3+f(-\frac{f(x)}3+f(-\frac{f(x)}3)))=0$ and so  :

$x-\frac{f(x)}3+f(-\frac{f(x)}3+f(-\frac{f(x)}3))=0$ 
and so $f(x_1)=f(x_2)$ with $x_1,x_2\ge 0$ implies $x_1=x_2$
Q.E.D.

3) $f(x+y)=f(x)+f(y)$ $\forall x,y\ge 0$
========================
Let $x\ge 0$ $P(0,\frac{f(x)}3)$ $\implies$ $f(\frac{f(x)}3+f(\frac{f(x)}3+f(\frac{f(x)}3)))=f(x)$

And so $\frac{f(x)}3+f(\frac{f(x)}3+f(\frac{f(x)}3))=x$ (since $LHS\ge 0$ and using 2) above)

And so $g(x)=x+f(x+f(x))$ is a surjection from $\mathbb R^+\cup\{0\}\to\mathbb R^+\cup\{0\}$

Let $x,y\ge 0$. We got $f(x+g(y))=3y+f(x)$, and so $f(x+g(y))=f(x)+f(g(y))$ and, since $g(y)$ is a surjection :
$f(x+y)=f(x)+f(y)$ $\forall x,y\ge 0$

4) $f(x)=x$ $\forall x$
==============
$f(x+y)=f(x)+f(y)$ $\forall x,y\ge 0$ and $f(x)> 0$ $\forall x> 0$ (so $f(x)$ increasing) immediately implies $f(x)=ax$ $\forall x\ge 0$ with some $a\ge 0$

Plugging this in $P(1,1)$, we get $a^3+a^2+a-3=0$, so $(a-1)(a^2+2a+3)=0$ and so $a=1$

So $f(x)=x$ $\forall x\ge 0$

$P(x,0)$ $\implies$ $f(|x|)=|f(x)|$ and so for any $x<0$, either $f(x)=x$, either $f(x)=-x$

Let then $v\le 0$ such that $f(v)=-v$
$P(-v,v)$ $\implies$ $0=2v$ and so $v=0$

So $f(x)=x$ $\forall x<0$

So $\boxed{f(x)=x \forall x}$ which indeed is a solution
Q.E.D.
\end{solution}
*******************************************************************************
-------------------------------------------------------------------------------

\begin{problem}[Posted by \href{https://artofproblemsolving.com/community/user/84905}{BaronShadeNight}]
	Find all continuous functions $f: \mathbb R \to \mathbb R$ such that for all reals $x$ and $y$,
\[f(x^2-y^2)=(x-y)(f(x)-f(y)).\]
	\flushright \href{https://artofproblemsolving.com/community/c6h372818}{(Link to AoPS)}
\end{problem}



\begin{solution}[by \href{https://artofproblemsolving.com/community/user/29428}{pco}]
	\begin{tcolorbox}Find all continuous functions $f:R\rightarrow R$ such that for all real numbers $x$ and $y$:

$f(x^2-y^2)=(x-y)[f(x)-f(y)]$\end{tcolorbox}

is $[f(x)-f(y)]$ integer part of $f(x)-f(y)$ or just $(f(x)-f(y))$ ?
\end{solution}



\begin{solution}[by \href{https://artofproblemsolving.com/community/user/84905}{BaronShadeNight}]
	I edited it.
\end{solution}



\begin{solution}[by \href{https://artofproblemsolving.com/community/user/29428}{pco}]
	\begin{tcolorbox}Find all continuous functions $f:R\rightarrow R$ such that for all real numbers $x$ and $y$:

$f(x^2-y^2)=(x-y)(f(x)-f(y))$\end{tcolorbox}
Let $P(x,y)$ be the assertion $f(x^2-y^2)=(x-y)(f(x)-f(y))$
Let $g(x)=\frac{f(x)}x$ continuous over $(0,+\infty)$

$P(1,1)$ $\implies$ $f(0)=0$
$P(x,0)$ $\implies$ $f(x^2)=xf(x)$ and so $g(x^2)=g(x)$ $\forall x\ne 0$
$P(0,x)$ $\implies$ $f(-x^2)=xf(x)=f(x^2)$ and $f(x)$ is an even function.

Let $x>0$ : $g(x)=g(x^{\frac 12})=g(x^{\frac 14})=g(x^{2^{-n}})$ Setting $n\to +\infty$ and using continuity, we get $g(x)=g(1)=f(1)$

And so $f(x)=f(1)x$ $\forall x>0$
Plugging this in the original equation, we get $f(1)=0$ and so $f(x)=0$ $\forall x\ge 0$

And so $\boxed{f(x)=0}$ $\forall x$ which indeed is a solution.
\end{solution}



\begin{solution}[by \href{https://artofproblemsolving.com/community/user/84905}{BaronShadeNight}]
	\begin{tcolorbox}[quote="BaronShadeNight"]Find all continuous functions $f:R\rightarrow R$ such that for all real numbers $x$ and $y$:

$f(x^2-y^2)=(x-y)(f(x)-f(y))$\end{tcolorbox}
Let $P(x,y)$ be the assertion $f(x^2-y^2)=(x-y)(f(x)-f(y))$
Let $g(x)=\frac{f(x)}x$ continuous over $(0,+\infty)$

$P(1,1)$ $\implies$ $f(0)=0$
$P(x,0)$ $\implies$ $f(x^2)=xf(x)$ and so $g(x^2)=g(x)$ $\forall x\ne 0$
$P(0,x)$ $\implies$ $f(-x^2)=xf(x)=f(x^2)$ and $f(x)$ is an even function.

Let $x>0$ : $g(x)=g(x^{\frac 12})=g(x^{\frac 14})=g(x^{2^{-n}})$ Setting $n\to +\infty$ and using continuity, we get $g(x)=g(1)=f(1)$

And so $f(x)=f(1)x$ $\forall x>0$
Plugging this in the original equation, we get $f(1)=0$ and so $f(x)=0$ $\forall x\ge 0$

And so $\boxed{f(x)=0}$ $\forall x$ which indeed is a solution.\end{tcolorbox}
thanks you very much!
\end{solution}



\begin{solution}[by \href{https://artofproblemsolving.com/community/user/90621}{Love_Math1994}]
	I have an another approach:because f(x) is a solution then  $f(x)+c$ is an solution so we can assume f(0)=0.subtitue  y=0 we have $f(x^2)=xf(x)$ the remain is not difficult. 
\end{solution}



\begin{solution}[by \href{https://artofproblemsolving.com/community/user/90621}{Love_Math1994}]
	it is an old problem
\end{solution}



\begin{solution}[by \href{https://artofproblemsolving.com/community/user/29428}{pco}]
	\begin{tcolorbox} because f(x) is a solution then  $f(x)+c$ is an solution \end{tcolorbox}


Surely not.
\end{solution}



\begin{solution}[by \href{https://artofproblemsolving.com/community/user/90621}{Love_Math1994}]
	:oops:  :oops: so sorry.i am very negligence...it is not true.but,in my memory,this problem difficult at how to compute f(0) and in official solution they assume f(0)=1 (after a reasoning like we can sub f(0) by f(0)+c :oops: )...i will find it in my book. :!: sorry
\end{solution}



\begin{solution}[by \href{https://artofproblemsolving.com/community/user/29428}{pco}]
	\begin{tcolorbox}:oops:  :oops: so sorry.i am very negligence...it is not true.but,in my memory,this problem difficult at how to compute f(0) and in official solution they assume f(0)=1 (after a reasoning like we can sub f(0) by f(0)+c :oops: )...i will find it in my book. :!: sorry\end{tcolorbox}

As I said in my solution, just set $x=y=1$ in the equation and you get $f(0)=0$ without any difficulty.
\end{solution}



\begin{solution}[by \href{https://artofproblemsolving.com/community/user/64716}{mavropnevma}]
	Why that insistence on value $1$, pco? Is it not enough for you that taking $x=y$ yields $f(0)=0$?  :)
\end{solution}



\begin{solution}[by \href{https://artofproblemsolving.com/community/user/90621}{Love_Math1994}]
	My stupid...embarassed :oops:  :oops:
\end{solution}



\begin{solution}[by \href{https://artofproblemsolving.com/community/user/29428}{pco}]
	\begin{tcolorbox}Why that insistence on value $1$, pco? Is it not enough for you that taking $x=y$ yields $f(0)=0$?  :)\end{tcolorbox}
Indeed :)
\end{solution}



\begin{solution}[by \href{https://artofproblemsolving.com/community/user/84905}{BaronShadeNight}]
	Sorry. I made a mistake.
\end{solution}



\begin{solution}[by \href{https://artofproblemsolving.com/community/user/29428}{pco}]
	\begin{tcolorbox} hello pco, you made a mistake. But the idea is still true ..!
thanks you!\end{tcolorbox}

Ohhh, I'm really sorry about my mistake!
Could you just show me where it is ?
I dont see it, very sorry  :oops:
\end{solution}



\begin{solution}[by \href{https://artofproblemsolving.com/community/user/84905}{BaronShadeNight}]
	\begin{tcolorbox}[quote="BaronShadeNight"] hello pco, you made a mistake. But the idea is still true ..!
thanks you!\end{tcolorbox}

Ohhh, I'm really sorry about my mistake!
Could you just show me where it is ?
I dont see it, very sorry  :oops:\end{tcolorbox}

It was bold.
\end{solution}



\begin{solution}[by \href{https://artofproblemsolving.com/community/user/29428}{pco}]
	\begin{tcolorbox}[quote="pco"]\begin{tcolorbox} hello pco, you made a mistake. But the idea is still true ..!
thanks you!\end{tcolorbox}

Ohhh, I'm really sorry about my mistake!
Could you just show me where it is ?
I dont see it, very sorry  :oops:\end{tcolorbox}

It was bold.\end{tcolorbox}

It seems quite right to me : plugging $f(x)=ax$ in original equation, we get :

$a(x^2-y^2)=a(x-y)^2$ $\iff$ $ay(y-x)=0$ $\forall x\ge y \ge 0$ and so $a=0$. 

Where is the mistake ?
\end{solution}



\begin{solution}[by \href{https://artofproblemsolving.com/community/user/84905}{BaronShadeNight}]
	\begin{tcolorbox}
It seems quite right to me : plugging $f(x)=ax$ in original equation, we get :

$a(x^2-y^2)=a(x-y)^2$ $\iff$ $ay(y-x)=0$ $\forall x\ge y \ge 0$ and so $a=0$. 

Where is the mistake ?\end{tcolorbox}

 Sorry pco, you did right.
\end{solution}
*******************************************************************************
-------------------------------------------------------------------------------

\begin{problem}[Posted by \href{https://artofproblemsolving.com/community/user/67223}{Amir Hossein}]
	Find all functions $f : \mathbb Q \to \mathbb R^+$ such that

(i) $f(x) \geq 0 \quad \forall x \in \mathbb Q$, and $f(x)=0 \iff x=0,$

(ii) $f(xy)= f(x)\cdot f(y),$ for all $x,y \in \mathbb Q$, and

(iii) $f(x+y) \leq \max \{ f(x), f(y) \}$ for all $x,y \in \mathbb Q$.
	\flushright \href{https://artofproblemsolving.com/community/c6h372842}{(Link to AoPS)}
\end{problem}



\begin{solution}[by \href{https://artofproblemsolving.com/community/user/29428}{pco}]
	\begin{tcolorbox}Find all functions $f : \mathbb Q \to \mathbb R^+$ such that

\begin{bolded}(i)\end{bolded} $f(x) \geq 0 \quad \forall x \in \mathbb Q, \qquad f(x)=0 \iff x=0,$

\begin{bolded}(ii)\end{bolded} $f(xy)= f(x)\cdot f(y),$

\begin{bolded}(iii)\end{bolded} $f(x+y) \leq \max \{ f(x), f(y) \}$\end{tcolorbox}
Nice problem :), thanks.

Using (ii) with $x=y=1$, we get $f(1)^2=f(1)$ and so $f(1)=1$ (using (i) right part)
Using (ii) with $x=y=-1$, we get $f(-1)^2=1$ and so $f(-1)=1$ (using (i) left part)
So $f(-x)=f(x)$

From (iii), it's easy to get $f(2x)\le f(x)$ and then $f(3x)\le f(x)$ ... and then $f(nx)\le f(x)$ and so $f(n)\le 1$ $\forall n\in\mathbb N$

If $f(n)=1$ $\forall n\in\mathbb N$, we get the solution $f(0)=0$ and $f(x)=1$ $\forall x\in\mathbb Q\setminus\{0\}$
If $\exists n>1$ such that $f(n)\ne 1$, let $p$ the littlest positive integer such that $f(p)=a\ne 1$. 
Obviously, $p$ is prime ($p=uv$ would imply $f(p)=f(u)f(v)=1$)

Let $n>0$ such that $p\not|n$ and $r\in\{1,2,...,p-1\}$ such that $n\equiv r\pmod p$
Let $k>0$ such that $m=(p+r)^{p^k}>n$ and $m\equiv r\pmod p$

$1=f(p+r-p)\le \max(f(p+r),f(p))$ and so $f(p+r)=1$ and so $f(m)=1$

$1=f(p+(m-p))\le \max(f(m-p),f(p))$ and so $f(m-p)=1$ and so $f(m-qp)=1$ and so $f(n)=1$

So $f(n)=1$ $\forall n\not\equiv 0\pmod p$ and $f(n)=a^{v_p(n)}$ $\forall n\in\mathbb N$

Then $f(\frac mn)=a^{v_p(|m|)-v_p(|n|)}$ which indeed is a solution

\begin{bolded}Hence the solutions \end{bolded}\end{underlined}:

$f(0)=0$ and $f(x)=1$ $\forall x\in\mathbb Q\setminus\{0\}$

$f(0)=0$ and $f(\frac mn)=a^{v_p(|m|)-v_p(|n|)}$ $\forall m,n\in\mathbb Z\setminus\{0\}$ for some prime $p$ and some real $a\in(0,1)$
\end{solution}



\begin{solution}[by \href{https://artofproblemsolving.com/community/user/67223}{Amir Hossein}]
	And thanks for your nice solution  :)
\end{solution}



\begin{solution}[by \href{https://artofproblemsolving.com/community/user/72819}{Dijkschneier}]
	Using (ii) with x=y=1, we get $f(1)^2=f(1)$ and so $f(1)=1$ (using (i) right part)
Using (ii) with x=y=-1, we get $f(-1)^2=1$ and so $f(-1)=1$ (using (i) left part)
So $f(-x)=f(x)$, and since f is multiplicative, it is enough to study it on $\mathbb{N}$.
Using (iii) with x=y=1, and then x=2, y=1, and then x=3,y=1... we get (by induction) : $\forall n \in \mathbb{N} f(n) \leq 1$.
Let $A=\{n\in \mathbb{N} \/ f(n)<1\}$. If A is empty, then f is determined with f(0)=0 and f(x)=1 if $x\neq 0$ and conversely it verifies the functional equation. If not, let $p = min(A)$. 
We claim that p is prime and that A is exactly the set of multiples of p. 
p is prime since if p=uv, then $f(p)=f(u)f(v)=1$ which is a contradiction. Let $a\in A$ and $d=gcd(a,p)$. If d=1, then we can write ua+vp=1, which gives : $1=f(1)=f(ua+vp) \leq max(f(ua),f(vp)) < 1$, which is a contradiction. Hence d=p and we get $p | a$. Conversely, all the multiples of p obviously belong to A since : $f(kp)=f(k)f(p)<1$. 
So $f(n)=1 \forall n\not\equiv 0\pmod p$ and $f(n)=a^{v_p(n)} \forall n\equiv 0\pmod p$, and we can summarize : $f(n)=a^{v_p(n)}\forall n\in\mathbb N$.
Then $f(\frac mn)=a^{v_p(|m|)-v_p(|n|)}$ which indeed is a solution.

Hence the solutions :\end{underlined}
$f(0)=0$ and $f(x)=1 \forall x\in\mathbb Q\setminus\{0\}$
$f(0)=0$ and$ f(\frac mn)=a^{v_p(|m|)-v_p(|n|)} \forall m,n\in\mathbb Z\setminus\{0\}$ for some prime p and some real $a\in(0,1)$
\end{solution}



\begin{solution}[by \href{https://artofproblemsolving.com/community/user/31915}{Batominovski}]
	This problem is a special case of Ostrowski's Valuation Theorem.
\end{solution}



\begin{solution}[by \href{https://artofproblemsolving.com/community/user/72819}{Dijkschneier}]
	\begin{tcolorbox}This problem is a special case of Ostrowski's Valuation Theorem.\end{tcolorbox}
I searched on Google after solving that problem and found that it is related indeed to Ostrowski's Theorem.
Can you please show how it is related to Ostrowski's Theorem since its statement is a little hard for me ? (how to deduce this problem from Ostrowski).
Thanks.
\end{solution}



\begin{solution}[by \href{https://artofproblemsolving.com/community/user/334227}{reveryu}]
	why when we get $ f(x) \geq f(nx) $ we can't do like this? ; $ f(x) \geq f(nx) , f(x\/n) \geq f(x)$  ,setting n infinity imply $0 \geq f(x) $so $f(x)=0$

and how did you get this $f(n)=a^{v_p(n)} \forall n\equiv 0\pmod p$

thank you
\end{solution}
*******************************************************************************
-------------------------------------------------------------------------------

\begin{problem}[Posted by \href{https://artofproblemsolving.com/community/user/89198}{chaotic_iak}]
	Let a function $f : \mathbb{R} \to \mathbb{R}$ has the property
\[(x-y)f(x+y) - (x+y)f(x-y) = 4xy(x^2 - y^2)\]
for all reals $x$ and $y$. Find all possible functions $f$.
	\flushright \href{https://artofproblemsolving.com/community/c6h373196}{(Link to AoPS)}
\end{problem}



\begin{solution}[by \href{https://artofproblemsolving.com/community/user/79866}{ateet74}]
	hint use like

$ \frac{f(x+y)}{x+y}-\frac{f(x-y)}{x-y} = (x+y)^2-(x-y)^2  $
\end{solution}



\begin{solution}[by \href{https://artofproblemsolving.com/community/user/37878}{hsbhatt}]
	[hide="Solution"]With the substitution $f(x) =x^3+g(x)$ the equation reduces to $(x+y)g(x-y) = (x-y) g(x+y)$

Its obvious that $g(0) = 0$. 

Now, if $x \ne y$ for $x,y \in \mathbb b{R}-\{0\}$, setting $x \rightarrow \frac{x+y}{2}$ and $y \rightarrow \frac{x-y}{2}$ we obtain $yg(x) = xg(y)$ or $\frac{g(x)}{x}$ is a constant so that $g(x)=cx \ \forall \ x \in \mathbb{R}$

Hence all solutions are of the form $f(x) = x^3+cx$


[\/hide]
\end{solution}



\begin{solution}[by \href{https://artofproblemsolving.com/community/user/29428}{pco}]
	\begin{tcolorbox}Let a function $f : \mathbb{R} \to \mathbb{R}$ has the property:
$(x-y)f(x+y) - (x+y)f(x-y) = 4xy(x^2 - y^2)$
Find all possible functions $f$.\end{tcolorbox}
Let $g(x)=f(x)-x^3$. The equation becomes $(x-y)g(x+y)=(x+y)g(x-y)$

Setting $x=\frac{u+1}2$ and $y=\frac{u-1}2$ in this equality, we get $g(u)=g(1)u$ and so $g(x)=ax$ which indeed is a solution.

Hence the answer : $\boxed{f(x)=x^3+ax}$ for any $a\in\mathbb R$

\begin{bolded}edited \end{bolded}\end{underlined}: too late again :) Congrats hsbhatt
\end{solution}
*******************************************************************************
-------------------------------------------------------------------------------

\begin{problem}[Posted by \href{https://artofproblemsolving.com/community/user/89157}{duck1606}]
	Find all function $f: \mathbb R^{+} \to \mathbb R^{+}$ such that \[f(x) \cdot f(yf(x))=f(x+y)\] for all $x,y>0$.
	\flushright \href{https://artofproblemsolving.com/community/c6h373453}{(Link to AoPS)}
\end{problem}



\begin{solution}[by \href{https://artofproblemsolving.com/community/user/29428}{pco}]
	\begin{tcolorbox}Find all function $f:R^{+}\rightarrow R^{+}$ such that: $f(x).f(yf(x))=f(x+y)$ with $x,y>0$\end{tcolorbox}

Let $P(x,y)$ be the assertion $f(x)f(yf(x))=f(x+y)$
$f(x)=1$ $\forall x$ is obviously a solution. So let us from now look for non all-one solutions.
Let then $u>0$ such that $f(u)\ne 1$


1) $f(x)\le 1$ $\forall x>0$
=================
If $\exists a>0$ such that $f(a)>1$, then :

$P(a,\frac{a}{f(a)-1})$ $\implies$ $f(a)f(\frac{af(a)}{f(a)-1})$ $=f(\frac{af(a)}{f(a)-1})$ and so $f(a)=1$ and contradiction

Q.E.D

2) $f(x)$ is injective
==============
As a consequence of 1) above : $f(u)<1$
If $\exists a>0$ and $\Delta>0$ such that $f(a)=f(a+\Delta)$ :
Comparing $P(a,x)$ and $P(a+\Delta,x)$, we get $f(x+a)=f(x+a+\Delta)$ and so $f(x)=f(x+\Delta)$ $\forall x>a$
So $f(x)=f(x+n\Delta)$ $\forall x>a$, $\forall n\in\mathbb N\cup\{0\}$

Let then $n$ great enough such that $\frac{(n\Delta-u)f(u)}{1-f(u)}>a$
Let $y=\frac{n\Delta-u}{1-f(u)}$ such that $yf(u)>a$ and $y+u=yf(u)+n\Delta$

$P(u,y)$ $\implies$ $f(u)f(yf(u))=f(y+u)=f(yf(u)+n\Delta)=f(yf(u))$ and so $f(u)=1$, contradiction
Q.E.D.

3) $f(x)<1$ $\forall x$
===============
If $\exists a>0$ such that $f(a)=1$, then $P(a,x)$ $\implies$ $f(x)=f(x+a)$, impossible, since $f(x)$ is injective
Q.E.D.

4) $f(x)=\frac 1{ax+1}$
================
$P(x,\frac 1{f(x)})$ $\implies$ $f(x)f(1)=f(x+\frac 1{f(x)})$

$\frac 1{f(x)}>1$. So : $P(1,x+\frac 1{f(x)}-1)$ $\implies$ $f(1)f((x+\frac 1{f(x)}-1)f(1))=f(x+\frac 1{f(x)})$

And so $f(x)f(1)=f(1)f((x+\frac 1{f(x)}-1)f(1))$ and so, since injective :

$x=(x+\frac 1{f(x)}-1)f(1)$ and so $f(x)=\frac 1{x(\frac 1{f(1)}-1)+1}$ which may be written $f(x)=\frac 1{ax+1}$ with $a>0$
which indeed is a solution.

5) Synthesis of solutions
=================
We got :
$f(x)=1$ $\forall x>0$
$f(x)=\frac 1{ax+1}$ with $a>0$

And so the family of solutions $\boxed{f(x)=\frac 1{ax+1}}$ $\forall x>0$ with $a\ge 0$
\end{solution}



\begin{solution}[by \href{https://artofproblemsolving.com/community/user/104682}{momo1729}]
	A somewhat shorter solution can be found in the last page of this document (in French) : http://www.animath.fr\/IMG\/pdf\/OFM_2011-2012-envoi2-corrige.pdf
\end{solution}
*******************************************************************************
-------------------------------------------------------------------------------

\begin{problem}[Posted by \href{https://artofproblemsolving.com/community/user/63660}{Victory.US}]
	Determine all functions $f:[0,\infty )  \to [0,\infty ) $ such that
\[f(f(x)-x)=2x, \quad \forall x\in [0,\infty ) .\]
	\flushright \href{https://artofproblemsolving.com/community/c6h374009}{(Link to AoPS)}
\end{problem}



\begin{solution}[by \href{https://artofproblemsolving.com/community/user/29428}{pco}]
	\begin{tcolorbox}determine all function $f:[0;+\infty )  \to [O;\infty ) $ such that

$ f(f(x)-x)=2x$ ,$\forall x\in [O;\infty ) $\end{tcolorbox}
In order to have the equation defined, we need $f(x)-x\ge 0$ and so $f(x)\ge x$

Then $f(f(x)-x)\ge f(x)-x$ and so $2x\ge f(x)-x$ and so $f(x)\le 3x$

Suppose now we have $a_nx\ge f(x)\ge b_nx$ $\forall x\ge 0$ with $a_n\ge b_n> 0$

We get $a_n(f(x)-x)\ge f(f(x)-x) \ge b_n (f(x)-x)$

And so $a_n(f(x)-x)\ge 2x\ge b_n (f(x)-x)$

And so $\frac{b_n+2}{b_n}x\ge f(x)\ge \frac{a_n+2}{a_n}x$

And so we have two sequences $a_n,b_n$ defined as :
$a_0=3$
$b_0=1$

$a_{n+1}=\frac{b_n+2}{b_n}$

$b_{n+1}=\frac{a_n+2}{a_n}$

And it's easy to show that these two sequences are convergent with same limit $2$

And so a unique solution $\boxed{f(x)=2x}$ which indeed is a solution.
\end{solution}



\begin{solution}[by \href{https://artofproblemsolving.com/community/user/66674}{thuyanh158}]
	\begin{tcolorbox}

And so $a_n(f(x)-x)\ge 2x\ge b_n (f(x)-x)$

And so $\frac{b_n+2}{b_n}x\ge f(x)\ge \frac{a_n+2}{a_n}x$
.\end{tcolorbox}

i wonder if u show me why ? i don't really understand this line. thanks
\end{solution}



\begin{solution}[by \href{https://artofproblemsolving.com/community/user/29428}{pco}]
	\begin{tcolorbox}[quote="pco"]

And so $a_n(f(x)-x)\ge 2x\ge b_n (f(x)-x)$

And so $\frac{b_n+2}{b_n}x\ge f(x)\ge \frac{a_n+2}{a_n}x$
.\end{tcolorbox}

i wonder if u show me why ? i don't really understand this line. thanks\end{tcolorbox}
We suppose that $a_nx\ge f(x)\ge b_nx$ $\forall x>0$ (it's true at the beginning since we proved $3x\ge f(x)\ge x$ $\forall x>0$

Since this is true for any $x>0$, it is true for $f(x)-x$ and so :
 $a_n(f(x)-x)\ge f(f(x)-x))\ge b_n(f(x)-x)$

But we know that $f(f(x)-x)=2x$

So $a_n(f(x)-x)\ge 2x\ge b_n(f(x)-x)$

$\iff$ $a_nf(x)-a_nx\ge 2x\ge b_nf(x)-b_nx$

So (left part) :  $a_nf(x)-a_nx\ge 2x$
$\iff$  $a_nf(x)\ge (a_n+2)x$
$\iff$  $f(x)\ge \frac{a_n+2}{a_n}x$ since $a_n>0$

Same (right part) : $2x\ge b_nf(x)-b_nx$
$\iff$ $(b_n+2)x\ge b_nf(x)$
$\iff$ $\frac{b_n+2}{b_n}\ge f(x)$ since $b_n>0$

And so we got $\frac{b_n+2}{b_n}\ge f(x)\ge \frac{a_n+2}{a_n}x$

Is it more understandable now ?
Dont hesitate to ask if there is still some difficulty.
\end{solution}
*******************************************************************************
-------------------------------------------------------------------------------

\begin{problem}[Posted by \href{https://artofproblemsolving.com/community/user/63660}{Victory.US}]
	Determine all function $f: \mathbb R \to \mathbb R$ such that
\[f(f(x)+y)=2x+f(f(y)-x), \quad \forall x,y \in \mathbb R.\]
	\flushright \href{https://artofproblemsolving.com/community/c6h374010}{(Link to AoPS)}
\end{problem}



\begin{solution}[by \href{https://artofproblemsolving.com/community/user/29428}{pco}]
	\begin{tcolorbox}Determine all fuction $f: R \to R$ such that :

$f(f(x)+y)=2x+f(f(y)-x), \forall x,y \in R$\end{tcolorbox}
Let $P(x,y)$ be the assertion $f(f(x)+y)=2x+f(f(y)-x)$

$P(x,-f(x))$ $\implies$ $f(0)-2x=f(f(-f(x))-x)$ and so $f(x)$ is surjective

Let then $u$ such that $f(u)=0$

$P(u,x)$ $\implies$ $f(f(x)-u)=(f(x)-u)-u$ and, since $f(x)$ is surjective : $f(x)=x-u$ which indeed is a solution

Hence the answer : $\boxed{f(x)=x+a}$
\end{solution}



\begin{solution}[by \href{https://artofproblemsolving.com/community/user/72731}{goodar2006}]
	[url=http://www.artofproblemsolving.com/Forum\/resources.php?c=1&cid=17&year=2002]IMO Shortlist 2002[\/url]
\end{solution}



\begin{solution}[by \href{https://artofproblemsolving.com/community/user/63660}{Victory.US}]
	yes,i found this nice problem in IMO Shortlist :)
\end{solution}



\begin{solution}[by \href{https://artofproblemsolving.com/community/user/92964}{dyta}]
	\begin{tcolorbox}yes,i found this nice problem in IMO Shortlist :)\end{tcolorbox}

It is a problem in Italy TST 2003.
[url]http://www.artofproblemsolving.com/Forum/viewtopic.php?f=38&t=303332[\/url]
\end{solution}
*******************************************************************************
-------------------------------------------------------------------------------

\begin{problem}[Posted by \href{https://artofproblemsolving.com/community/user/19427}{TRAN THAI HUNG}]
	Does there exist a function $ f: \mathbb R^+\to \mathbb R^+$ such that \[ f(x+y)\geq yf(f(x))\] for all $x,y >0$?
	\flushright \href{https://artofproblemsolving.com/community/c6h374022}{(Link to AoPS)}
\end{problem}



\begin{solution}[by \href{https://artofproblemsolving.com/community/user/29428}{pco}]
	\begin{tcolorbox}Is that exist $ f:R^+\rightarrow R^+$ such that: $ f(x+y)\geq yf(f(x))$ for all $x,y >O$\end{tcolorbox}
Let $P(x,y)$ be the assertion $f(x+y)\ge yf(f(x))$

1) $f(x)\le x+1$ $\forall x>0$
===================
If $f(x)\le x$, we get obviously $f(x)\le x+1$
If $f(x)>x$ for some $x$, then $P(x,f(x)-x)$ $\implies$ $f(f(x))\ge (f(x)-x)f(f(x))$ $\implies$ $f(x)\le x+1$
Q.E.D.

2) $f(f(x))\le 1$ $\forall x>0$
====================
If $f(f(x))>1$ for some $x$, then Let $y>\frac{x+1}{f(f(x))-1}$ so that $yf(f(x))>x+y+1$ : 
$P(x,y)$ $\implies$ $f(x+y)\ge yf(f(x))>x+y+1$ and so contradiction with 1) above.
Q.E.D.

3) $\exists M>0$ such that $f(x)<M$ $\forall x>0$
==================================
Suppose that $\forall a>0,\exists u_a>0$ such that $f(u_a)>a$
Let $x>0$ and $a>x$ and $u_a>0$ such that $f(u_a)>a>x$
$P(x,f(u_a)-x)$ $\implies$ $f(f(u_a))\ge (f(u_a)-x)f(f(x))>(a-x)f(f(x))$ $\implies$ (using 2) above) $1>(a-x)f(f(x))$

So $0<f(f(x))<\frac 1{a-x}$ $\forall a>x$ which is impossible
Q.E.D.

4) No such function exists
==================
$P(x,y)$ $\implies$ $M>yf(f(x))$ and so $0<f(f(x))<\frac My$ $\forall y$ which is impossible
Q.E.D.
\end{solution}
*******************************************************************************
-------------------------------------------------------------------------------

\begin{problem}[Posted by \href{https://artofproblemsolving.com/community/user/72235}{Goutham}]
	Given a nonconstant function $f : \mathbb{R}^+ \longrightarrow\mathbb{R}$ such that $f(xy) = f(x)f(y)$ for any $x, y > 0$, find functions $c, s : \mathbb{R}^+ \longrightarrow \mathbb{R}$ that satisfy $c\left(\frac{x}{y}\right) = c(x)c(y)-s(x)s(y)$ for all $x, y > 0$ and $c(x)+s(x) = f(x)$ for all $x > 0$.
	\flushright \href{https://artofproblemsolving.com/community/c6h374217}{(Link to AoPS)}
\end{problem}



\begin{solution}[by \href{https://artofproblemsolving.com/community/user/29428}{pco}]
	\begin{tcolorbox}Given a nonconstant function $f : \mathbb{R}^+ \longrightarrow\mathbb{R}$ such that $f(xy) = f(x)f(y)$ for any $x, y > 0$, find functions $c, s : \mathbb{R}^+ \longrightarrow \mathbb{R}$ that satisfy $c\left(\frac{x}{y}\right) = c(x)c(y)-s(x)s(y)$ for all $x, y > 0$ and $c(x)+s(x) = f(x)$ for all $x > 0$.\end{tcolorbox}
We get :
$c(x)+s(x)=f(x)$
$c(x)^2-s(x)^2=c(1)$ and so $(c(x)-s(x))f(x)=c(1)$

If $f(x)=0$ for some $x$, then $f(x)=0$ $\forall x$, which is impossible since we know that $f(x)$ is non constant.
So $f(x)\ne 0$ $\forall x>0$ and $f(1)=1$

Then we get :
$c(x)+s(x)=f(x)$
$c(x)-s(x)=\frac{c(1)}{f(x)}$

And so :
$c(x)=\frac 12({f(x)+\frac{c(1)}{f(x)})}$ and $s(x)=\frac 12{f(x)-\frac{c(1)}{f(x)})}$

It's then immediate to get $c(1)=1$ and the result :

$c(x)=\frac 12\left(f(x)+\frac 1{f(x)}\right)$ and $s(x)=\frac 12\left(f(x)-\frac 1{f(x)}\right)$

and it's easy to check back that this mandatory form is indeed a solution
\end{solution}
*******************************************************************************
-------------------------------------------------------------------------------

\begin{problem}[Posted by \href{https://artofproblemsolving.com/community/user/89818}{gauman}]
	Given additive functions $f,g : \mathbb R \to \mathbb R$ such that
\[f(x) \cdot g(x)=2x^2\]
for all $x \in \mathbb R$, prove that $f(x)=ax$ for some real $a$.
	\flushright \href{https://artofproblemsolving.com/community/c6h374847}{(Link to AoPS)}
\end{problem}



\begin{solution}[by \href{https://artofproblemsolving.com/community/user/91689}{novae}]
	$f(x)=2$ and $g(x)=x^2$
\end{solution}



\begin{solution}[by \href{https://artofproblemsolving.com/community/user/89818}{gauman}]
	$f(x)=2$ not additive . Are there some body?
\end{solution}



\begin{solution}[by \href{https://artofproblemsolving.com/community/user/29428}{pco}]
	\begin{tcolorbox}Given $f(x),g(x) :R\to R$ additive:
$f(x).g(x)=2x^2$ with $x \in R$
Prove that $f(x)=ax$\end{tcolorbox}
$f(x+1)g(x+1)=2(x+1)^2$ $\implies$ $((f(x)+f(1))((g(x)+g(1))=2x^2+4x+2$ 

$\implies$ $f(1)g(x)+f(x)g(1)=4x$ $\implies$ $g(x)=\frac{4}{f(1)}x-\frac{g(1)}{f(1)}f(x)$ (remember $f(1)g(1)=2$ and so $f(1)\ne 0$)

Plugging this value of $g(x)$ in $f(x)g(x)=2x^2$, we get $\frac{4}{f(1)}xf(x)-\frac{g(1)}{f(1)}f(x)^2=2x^2$

Which is a quadratic $g(1)f(x)^2-4xf(x)+2x^2f(1)=0$ which may be written, since $f(1)g(1)=2$ :

$(g(1)f(x)-2x)^2=0$  and so $f(x)=\frac 2{g(1)}x$ which indeed is a solution.
Q.E.D.
\end{solution}



\begin{solution}[by \href{https://artofproblemsolving.com/community/user/64716}{mavropnevma}]
	Not so very hard. For $x=1$ we have $f(1)g(1) = 2$, so both $f(1)$ and $g(1)$ are not null.
Now, $f(x+1)g(x+1) = 2(x+1)^2 = 2x^2 + 4x + 2$, but also $f(x+1)g(x+1) = (f(x) + f(1))(g(x)+g(1)) =$ $f(x)g(x) + f(x)g(1) + f(1)g(x) + f(1)g(1) = 2x^2 + f(x)g(1) + f(1)g(x) + 2$, hence $f(x)g(1) + f(1)g(x) = 4x$, or $\dfrac {f(x)} {f(1)} + \dfrac {g(x)} {g(1)} = 2x$.

From that, and $\dfrac {f(x)} {f(1)} \cdot \dfrac {g(x)} {g(1)} = x^2$, it follows $\dfrac {f(x)} {f(1)}$ and $\dfrac {g(x)} {g(1)}$ are the roots of the equation $\lambda^2 - 2x\lambda + x^2 = 0$, thus both are equal to $x$, hence $f(x) = f(1)x$ and $g(x) = g(1)x$, with the only condition $f(1)g(1)=2$.

Edit. It turns out pco subterranously worked out the same idea, while I was painstakingly writing mine.
\end{solution}



\begin{solution}[by \href{https://artofproblemsolving.com/community/user/82334}{bappa1971}]
	\begin{tcolorbox}Given $f(x),g(x) :R\to R$ additive:
$f(x).g(x)=2x^2$ with $x \in R$
Prove that $f(x)=ax$\end{tcolorbox}

$f(0)=g(0)=0$
$f(x)=\frac{2x^2}{g(x)}$ for $x\neq 0$
$f(x+y)g(x+y)=(f(x)+f(y))(g(x)+g(y))$
$\Longrightarrow 4x y=f(x)g(y)+g(x)f(y)$
$\Longrightarrow 4x y=f(x)\frac{2y^2}{f(y)}+f(y)\frac{2x^2}{f(x)}$
$\Longrightarrow 2x y f(x)f(y)=y^2 f(x)^2 +x^2 f(y)^2$
$\Longrightarrow (yf(x)-xf(y))^2=0$
$\Longrightarrow f(x)=\frac{f(y)}{y} x$

$x=1$ yeilds
\[f(x)=ax\] where $a=f(1)$


\begin{bolded}Edit\end{bolded}::  
:oops_sign:  i'm late!!
didn't see \begin{bolded}mavropnevma\end{bolded} and\begin{bolded} pco\end{bolded}'s solution!
\end{solution}



\begin{solution}[by \href{https://artofproblemsolving.com/community/user/89818}{gauman}]
	if you want a real hard problem please do it:
Given $a \in R$ and $f_1,f_2,...,f_n: R \to R$ additive such that:
$f_1(x).f_2(x)...f_n(x) =ax^n$ with every $x \in R$
Prove that exist $b \in R$ and $i \in${$1,2,...,n$} such that: $f_i(x)=bx$ with every $x \in R$
\end{solution}
*******************************************************************************
-------------------------------------------------------------------------------

\begin{problem}[Posted by \href{https://artofproblemsolving.com/community/user/67223}{Amir Hossein}]
	Find all functions $f : \mathbb R \to \mathbb R$ such that for all $x, y,$
\[f(f(x) + y) = f(x^2 - y) + 4f(x)y.\]
	\flushright \href{https://artofproblemsolving.com/community/c6h375050}{(Link to AoPS)}
\end{problem}



\begin{solution}[by \href{https://artofproblemsolving.com/community/user/29428}{pco}]
	\begin{tcolorbox}Find all functions $f : \mathbb R \to \mathbb R$ such that for all $x, y,$
\[f(f(x) + y) = f(x^2 - y) + 4f(x)y.\]\end{tcolorbox}
Let $P(x,y)$ be the assertion $f(f(x)+y)=f(x^2-y)+4f(x)y$

$P(x,\frac{x^2-f(x)}2)$ $\implies$ $f(x)(f(x)-x^2)=0$ and so : $\forall x$, either $f(x)=0$, either $f(x)=x^2$

Suppose now that $\exists a,b\ne 0$ such that $f(a)=0$ and $f(b)=b^2$

$P(a,b)$ $\implies$ $b^2=f(a^2-b)$ and so either $b^2=0$, either $b^2=(a^2-b)^2$ $\iff$ $a^2(a^2-2b)=0$ $\implies$ $b=\frac{a^2}2$

Choose then $c\in\mathbb R\setminus\{0,a,-a,\frac{a^2}2\}$

If $f(c)=0$, then $P(c,b)$ $\implies$ $b=\frac {c^2}2$ and so $c^2=a^2$, impossible
If $f(c)=c^2$, then $P(a,c)$ $\implies$ $c=\frac {a^2}2$, impossible

So either $f(x)=0$ $\forall x\ne 0$, either $f(x)=x^2$ $\forall x\ne 0$

So either $f(x)=0$ $\forall x$, either $f(x)=x^2$ $\forall x$ and it's easy to check that both are solutions.

Hence the answer :
$f(x)=0$ $\forall x$
$f(x)=x^2$ $\forall x$
\end{solution}
*******************************************************************************
-------------------------------------------------------------------------------

\begin{problem}[Posted by \href{https://artofproblemsolving.com/community/user/48364}{cnyd}]
	Find all functions $f: \mathbb R \to \mathbb R$ such that for all reals $x$ and $y$,
\[f(f(x))=x,f(xy)+f(x)+f(y)=f(x+y)+f(x)f(y).\]
Prove that $f$ is an increasing function.
	\flushright \href{https://artofproblemsolving.com/community/c6h375103}{(Link to AoPS)}
\end{problem}



\begin{solution}[by \href{https://artofproblemsolving.com/community/user/29428}{pco}]
	\begin{tcolorbox}$f$ from the reals to the reals such that 

$f(f(x))=x,f(xy)+f(x)+f(y)=f(x+y)+f(x)f(y)$  

Prove that $f$ is increasing function.\end{tcolorbox}
$f(f(x))=x$ implies that $f(x)$ is a bijection.
Let $P(x,y)$ be the assertion $f(xy)+f(x)+f(y)=f(x+y)+f(x)f(y)$
Let $a=f(1)$

$P(a,0)$ $\implies$ $f(0)=0$

If $a\ne 1$ : Let $u$ such that $f(u)=\frac a{a-1}$

$P(u,1)$ $\implies$ $f(u+1)=\frac a{a-1}=f(u)$, which is impossible since $f(x)$ is an injection. So $a=1$.

$P(x,1)$ $\implies$ $f(x+1)=f(x)+1$ 
$P(x,y+1)$ $\implies$ $f(xy+x)+f(y)=f(x+y)+f(x)f(y)$
Subtracting $P(x,y)$ from the previous line, we get $f(xy+x)=f(xy)+f(x)$

So $f(x)$ is additive and $P(x,y)$ implies $f(xy)=f(x)f(y)$ and so $f(x^2)=f(x)^2$ and so $f(x)>0$ $\forall x>0$

And since $f(x+y)=f(x)+f(y)$, we get $f(x+y)>f(x)$ $\forall x,\forall y>0$
Q.E.D

And, btw, we immediately get the unique solution $f(x)=x$
\end{solution}
*******************************************************************************
-------------------------------------------------------------------------------

\begin{problem}[Posted by \href{https://artofproblemsolving.com/community/user/3182}{Kunihiko_Chikaya}]
	Find all positive integers $n$ such that there exists the polynomial with degree $n$ satisfying $f(x^2+1)=f(x)^2+1$.
	\flushright \href{https://artofproblemsolving.com/community/c6h375387}{(Link to AoPS)}
\end{problem}



\begin{solution}[by \href{https://artofproblemsolving.com/community/user/29428}{pco}]
	\begin{tcolorbox}Find all positive integers $n$ such that there exists the polynomial with degree $n$ satisfying $f(x^2+1)=f(x)^2+1$.

\begin{italicized}2010 Tokyo Institute of Technology Admission Office entrance exam, Problem Ⅱ-1\/Science\end{italicized}\end{tcolorbox}
Let $P(x)$ be the assertion $f(x^2+1)=f(x)^2+1$

1) The only odd solution is $f(x)=x$
========================
Suppose that $f(x)$ is odd. We can then write $f(x)=xg(x^2)$ for some $g(x)$ and we get :
$(x^2+1)g((x^2+1)^2)=x^2g(x^2)^2+1$ and so $(x+1)g((x+1)^2)=xg(x)^2+1$

From there, we see that $g(x)=1$ for $x\ne -1$ $\implies$ $g((x+1)^2)=1$
Setting $x=0$, we get $g(1)=1$
So $g((1+1)^2)=g(4)=1$
So $g((4+1)^2)=g(25)=1$
And so the equation $g(x)=1$ has infinitely many roots and so $g(x)=1$ and $f(x)=x$
Q.E.D.

2) The only possible degrees are $2^p$
============================
Comparing $P(x)$ and $P(-x)$, we get $f(x)^2=f(-x)^2$ and so either $f(x)$ is odd, either $f(x)$ is even.
If $f(x)$ is odd, then $f(x)=x$ (see 1) above)
If $f(x)$ is even, then $f(x)=g(x^2)$ and the equation becomes :

$g((x^2+1)^2)=g(x^2)^2+1$ and so $g((x+1)^2)=g(x)^2+1$
Let then $g(x)=h(x+1)$ and the equation becomes $h((x+1)^2+1)=h(x+1)^2+1$ and so $h(x^2+1)=h(x)^2+1$

And so either $f(x)$ has degree 1, either $f(x)$ has degree $2k$ and it exists a function $h(x)$ of degree $k$ which is solution too.
Q.E.D

3) It exists solution for any degree $n=2^p$ with $p\in\mathbb N\cup\{0\}$
======================================================
Let $g(x)=x^2+1$
The equation is then $f(g(x))=g(f(x))$ and so at least any $f(x)=g^{[p]}(x)$ with $p\ge 0$ is solution (composition of function)

And since degree of $g^{[p]}(x)=2^p$, we got the answer.
Q.E.D

And so the answer is $\boxed{n\in\{2^k,\forall k\in\mathbb N\cup\{0\}\}}$
\end{solution}



\begin{solution}[by \href{https://artofproblemsolving.com/community/user/3182}{Kunihiko_Chikaya}]
	Splendid! How do you think the problem is difficult, perhaps it is easy for you?

Generally speaking, Japanese High School Students don't study Functional Equation in high school curriculum.
\end{solution}



\begin{solution}[by \href{https://artofproblemsolving.com/community/user/29428}{pco}]
	\begin{tcolorbox}Splendid! How do you think the problem is difficult, perhaps it is easy for you?

Generally speaking, Japanese High School Students don't study Functional Equation in high school curriculum.\end{tcolorbox}

I quickly found the even case but got some difficulties with the odd case

It's not an easy problem (generally, this kind of problem implies root usage and it does not work here).
\end{solution}



\begin{solution}[by \href{https://artofproblemsolving.com/community/user/3182}{Kunihiko_Chikaya}]
	I agree with you.
\end{solution}



\begin{solution}[by \href{https://artofproblemsolving.com/community/user/92964}{dyta}]
	\begin{tcolorbox}Find all positive integers $n$ such that there exists the polynomial with degree $n$ satisfying $f(x^2+1)=f(x)^2+1$.\end{tcolorbox}

Thanks pco.
But I don't understand. If we let $f(n)=n$ then $f(x^2+1)=f(x)^2+1$ with every $n \in N$
\end{solution}



\begin{solution}[by \href{https://artofproblemsolving.com/community/user/29428}{pco}]
	\begin{tcolorbox}[quote="kunny"]Find all positive integers $n$ such that there exists the polynomial with degree $n$ satisfying $f(x^2+1)=f(x)^2+1$.\end{tcolorbox}

Thanks pco.
But I don't understand. If we let $f(n)=n$ then $f(x^2+1)=f(x)^2+1$ with every $n \in N$\end{tcolorbox}

In the problem, $n$ is the degree of the polynomial. If you write $f(n)=n$, you mean $f(x)=x$ which indeed is the solution for $n=1$ (degree 1).
\end{solution}



\begin{solution}[by \href{https://artofproblemsolving.com/community/user/92964}{dyta}]
	\begin{tcolorbox}[quote="dyta"][quote="kunny"]Find all positive integers $n$ such that there exists the polynomial with degree $n$ satisfying $f(x^2+1)=f(x)^2+1$.\end{tcolorbox}

Thanks pco.
But I don't understand. If we let $f(n)=n$ then $f(x^2+1)=f(x)^2+1$ with every $n \in N$\end{tcolorbox}

In the problem, $n$ is the degree of the polynomial. If you write $f(n)=n$, you mean $f(x)=x$ which indeed is the solution for $n=1$ (degree 1).\end{tcolorbox}

Sorry, it is a stupid question.

Please help me a problem:
Find $P(x) \in \mathbb{Z}$[x] such that: $P(x^2-2)=P^2 (x)-2$
\end{solution}



\begin{solution}[by \href{https://artofproblemsolving.com/community/user/36147}{Edward_Tur}]
	\begin{tcolorbox}Find all positive integers $n$ such that there exists 

Find $P(x) \in \mathbb{Z}$[x] such that: $P(x^2-2)=P^2 (x)-2$\end{tcolorbox}
$P(x)=2T_n(x\/2)$, see [url]http://en.wikipedia.org\/wiki\/Chebyshev_polynomials[\/url].
Also see:
[url]http://www.math.lsa.umich.edu\/~zieve\/papers\/peter.pdf[\/url]
[url]http://kvant.mirror1.mccme.ru\/1979\/04\/kommutiruyushchie_mnogochleny.htm[\/url].
\end{solution}



\begin{solution}[by \href{https://artofproblemsolving.com/community/user/93909}{magical}]
	\begin{tcolorbox}Find all positive integers $n$ such that there exists the polynomial with degree $n$ satisfying $f(x^2+1)=f(x)^2+1$.

\begin{italicized}2010 Tokyo Institute of Technology Admission Office entrance exam, Problem Ⅱ-1\/Science\end{italicized}\end{tcolorbox}
\begin{bolded}Expand\end{underlined}\end{bolded}
Find all positive integers $n$ such that there exists the polynomial with degree $n$ satisfying $f(x^3+1)=f(x)^3+1$.
\end{solution}



\begin{solution}[by \href{https://artofproblemsolving.com/community/user/36147}{Edward_Tur}]
	\begin{tcolorbox}
\begin{bolded}Expand\end{underlined}\end{bolded}
Find all positive integers $n$ such that there exists the polynomial with degree $n$ satisfying $f(x^3+1)=f(x)^3+1$.\end{tcolorbox}
[url]http://books.google.com.ua\/books?id=qIJPxdwSqlcC&pg=PA102&lpg=PA102&dq=commuting+polynomials&source=bl&ots=6rv-8oCwha&sig=VSJSWb7sKzSmm2cb5X3HuJZC5KU&hl=uk&ei=B30XTdyMM8ms8QOO-NmDBw&sa=X&oi=book_result&ct=result&resnum=6&ved=0CD0Q6AEwBTgK[\/url]
\end{solution}



\begin{solution}[by \href{https://artofproblemsolving.com/community/user/87195}{SCP}]
	\begin{tcolorbox}[quote="kunny"]Find all positive integers $n$ such that there exists the polynomial with degree $n$ satisfying $f(x^2+1)=f(x)^2+1$.

\begin{italicized}2010 Tokyo Institute of Technology Admission Office entrance exam, Problem Ⅱ-1\/Science\end{italicized}\end{tcolorbox}
\begin{bolded}Expand\end{underlined}\end{bolded}
Find all positive integers $n$ such that there exists the polynomial with degree $n$ satisfying $f(x^3+1)=f(x)^3+1$.\end{tcolorbox}

Maybe we can search it in generally: $f(x^s+1)=f(x)^s+1$ with $s \in N$
\end{solution}



\begin{solution}[by \href{https://artofproblemsolving.com/community/user/92964}{dyta}]
	\begin{tcolorbox}[quote="dyta"]Find all positive integers $n$ such that there exists 

Find $P(x) \in \mathbb{Z}$[x] such that: $P(x^2-2)=P^2 (x)-2$\end{tcolorbox}
$P(x)=2T_n(x\/2)$, see [url]http://en.wikipedia.org\/wiki\/Chebyshev_polynomials[\/url].
Also see:
[url]http://www.math.lsa.umich.edu\/~zieve\/papers\/peter.pdf[\/url]
[url]http://kvant.mirror1.mccme.ru\/1979\/04\/kommutiruyushchie_mnogochleny.htm[\/url].\end{tcolorbox}

Sorry, I can't understand them. Can you give me the main idea for my problem?
\end{solution}
*******************************************************************************
-------------------------------------------------------------------------------

\begin{problem}[Posted by \href{https://artofproblemsolving.com/community/user/68025}{Pirkuliyev Rovsen}]
	Let $k \ge 3$ be an integer and $f: \mathbb{R}\to\mathbb{R}$ be a function such that for all real $x$,
\[f(x-1)+f(x+1)=2\left(\cos \frac{2\pi}{k}\right)f(x).\]
Prove that $f$ is periodic.
	\flushright \href{https://artofproblemsolving.com/community/c6h375413}{(Link to AoPS)}
\end{problem}



\begin{solution}[by \href{https://artofproblemsolving.com/community/user/29428}{pco}]
	\begin{tcolorbox}Let $k \ge 3$ ,$k\in N$ and $f: \mathbb{R}\to\mathbb{R}$ be a function such that
$f(x-1)+f(x+1)=2(\cos \frac{2\pi}{k})f(x)$.Prove that function f periodical\end{tcolorbox}
$f(x+2)=2\cos\frac{2\pi}kf(x+1)-f(x)$

The classical solution of linear recurences gives then $f(x+n)=ze^{\frac{2i\pi}kn}+\overline ze^{-\frac{2i\pi}kn}$ where $z$ depends on $f(x)$ and $f(x+1)$

And so $f(x+k)=f(x)$ $\forall x$
Hence the result.
\end{solution}



\begin{solution}[by \href{https://artofproblemsolving.com/community/user/68025}{Pirkuliyev Rovsen}]
	Patrick,Please write in detail
\end{solution}



\begin{solution}[by \href{https://artofproblemsolving.com/community/user/29428}{pco}]
	\begin{tcolorbox}Patrick,Please write in detail\end{tcolorbox}
$f(x+2)=2\cos\frac{2\pi}kf(x+1)-f(x)$

Let then any $x$ and the sequence $a_n$ defined as :
$a_0=f(x)$
$a_1=f(x+1)$
$a_{n+2}=2\cos\frac{2\pi}ka_{n+1}-a_n$

The characteristic polynomial associated to this sequence is $X^2-2\cos\frac{2\pi}k X+1$ whose roots are both complex and are  $e^{\frac{2i\pi}k}$ and $e^{\frac{2i\pi}k}$ 

And so $a_n=ze^{\frac{2i\pi}kn}+\overline ze^{-\frac{2i\pi}kn}$ for some $z$ which may be calculated (but it's of no interest) using :
$a_0=f(x)=z+\overline z$
$a_1=f(x+1)=ze^{\frac{2i\pi}k}+\overline ze^{-\frac{2i\pi}k}$

The interesting thing is that setting $n=k$, we get $a_k=ze^{2i\pi}+\overline ze^{-2i\pi}=a_0$

And since obviously $a_n=f(x+n)$, we got $a_k=a_0$ $\implies$ $f(x+k)=f(x)$ $\forall x$
\end{solution}



\begin{solution}[by \href{https://artofproblemsolving.com/community/user/82091}{Solving}]
	BEAUTIFUL Patrick! King!!
\end{solution}



\begin{solution}[by \href{https://artofproblemsolving.com/community/user/68025}{Pirkuliyev Rovsen}]
	Thanks you Patrick  :!:
\end{solution}
*******************************************************************************
-------------------------------------------------------------------------------

\begin{problem}[Posted by \href{https://artofproblemsolving.com/community/user/92175}{TigerBen}]
	Find all $f:\mathbb R\to \mathbb R$ such that  $f(x+y)+f(x)f(y)=f(xy)+2xy+1$ for all $x,y\in\mathbb R$.
	\flushright \href{https://artofproblemsolving.com/community/c6h375416}{(Link to AoPS)}
\end{problem}



\begin{solution}[by \href{https://artofproblemsolving.com/community/user/29428}{pco}]
	\begin{tcolorbox}Find all $f:\mathbb R\to \mathbb R$ such that  $f(x+y)+f(x)f(y)=f(xy)+2xy+1$ for all $x,y\in\mathbb R$.\end{tcolorbox}
Let $P(x,y)$ be the assertion $f(x+y)+f(x)f(y)=f(xy)+2xy+1$
Let $f(1)=a$

$P(x,0)$ $\implies$ $(f(x)-1)(f(0)+1)=0$ and since $f(x)=1\forall x$ is not a solution, we get $f(0)=-1$

$P(x,1)$ $\implies$ $f(x+1)=(1-a)f(x)+2x+1$

$P(x,y+1)$ $\implies$ $f(x+y+1)+f(x)f(y+1)=f(xy+x)+2xy+2x+1$

So $(1-a)f(x+y)+2x+2y+1+(1-a)f(x)f(y)+(2y+1)f(x)$ $=f(xy+x)+2xy+2x+1$
So $(1-a)(f(x+y)+f(x)f(y))=f(xy+x)+2xy-2y-(2y+1)f(x)$
So $(1-a)(f(xy)+2xy+1)=f(xy+x)+2xy-2y-(2y+1)f(x)$
So $f(xy+x)=(1-a)f(xy)+(2y+1)f(x)-2axy+2y+a-1$

So $f(x+y)=(1-a)f(y)+(2\frac yx+1)f(x)-2ay+2\frac yx+a-1$ $\forall x\ne 0$
Swapping $x,y$, we get :
$f(x+y)=(1-a)f(x)+(2\frac xy+1)f(y)-2ax+2\frac xy+a-1$ $\forall y\ne 0$

Subtracting the two lines :

$(f(x)+1)(2\frac yx+a)-(f(y)+1)(2\frac xy+a)=-2ax+2ay$ $\forall x,y\ne 0$

Setting $y=1$ : $f(x)=\frac{2x^2+a(a+2)x-2}{ax+2}$ $\forall x\ne 0$ and $\ne -\frac 2a$

Plugging this in the equation $P(1,1)$ : $f(2)=-a^2+a+3$, we get $a\in\{0,1,-2\}$

$a=0$ $\implies$ $f(x)=x^2-1$ $\forall x\ne 0$ and so $f(x)=x^2-1$ $\forall x$ which indeed is a solution.

$a=1$ : $P(x-1,1)$ $\implies$ $f(x)=2x-1$ which indeed is a solution.

$a=-2$ $\implies$ $f(x)=-x-1$ $\forall x\ne 0,1$ and so $f(x)=-x-1$ $\forall x$ which indeed is a solution.

\begin{bolded}Hence the solutions \end{bolded}\end{underlined}:
$f(x)=-x-1$
$f(x)=2x-1$
$f(x)=x^2-1$
\end{solution}
*******************************************************************************
-------------------------------------------------------------------------------

\begin{problem}[Posted by \href{https://artofproblemsolving.com/community/user/91306}{ndk09}]
	Let $n$ be a even positive integer. Find all function $f: \mathbb R \to \mathbb R$ such that
\[f(x-f(y)=f(x+y^n)+f(f(y)+y^n)+1\]
holds for all $x,y \in \mathbb R$.
	\flushright \href{https://artofproblemsolving.com/community/c6h375438}{(Link to AoPS)}
\end{problem}



\begin{solution}[by \href{https://artofproblemsolving.com/community/user/91306}{ndk09}]
	please help me!
\end{solution}



\begin{solution}[by \href{https://artofproblemsolving.com/community/user/29428}{pco}]
	\begin{tcolorbox}Let $n$ is a even positive interger. Find all function $f : R \longrightarrow R$ such that:
\[f(x-f(y)=f(x+y^n)+f(f(y)+y^n)+1\]
$\forall x,y \in R$\end{tcolorbox}
Missing parenthesis in LHS : must we understand $f(x-f(y))$ or $f(x)-f(y)$ ?
\end{solution}



\begin{solution}[by \href{https://artofproblemsolving.com/community/user/91306}{ndk09}]
	\begin{tcolorbox}[quote="ndk09"]Let $n$ is a even positive interger. Find all function $f : R \longrightarrow R$ such that:
\[f(x-f(y)=f(x+y^n)+f(f(y)+y^n)+1\]
$\forall x,y \in R$\end{tcolorbox}
Missing parenthesis in LHS : must we understand $f(x-f(y))$ or $f(x)-f(y)$ ?\end{tcolorbox}
$f(x-f(y))$
\end{solution}



\begin{solution}[by \href{https://artofproblemsolving.com/community/user/46171}{tuandokim}]
	\begin{tcolorbox}Let $n$ is a even positive interger. Find all function $f : R \longrightarrow R$ such that:
\[f(x-f(y)=f(x+y^n)+f(f(y)+y^n)+1\] (*)
$\forall x,y \in R$\end{tcolorbox}
Let x=f(y) so $f(f(y)+y^n)=\frac{f(0)-1}{2}$
so we can rewrite the problem 
$f(x-f(y))=f(x+y^n)+\frac{f(0)+1}{2}$
because n is even then $f(x-f(y))=f(x-f(-y))$
we have 2 cases:
*****
case 1: there exists a which $(f(a)-f(-a))^2>0$ 
so there existed b which $f(x)=f(x+b)$ for every x in R
so we have $f(x+y^n)=f(x+(y+b)^n)$, let $x=-y^n $
so $f(0)=f(y^{n-1}*n+....+b^n)$
remember that n is even so (n-1) is odd so for every $x\in R$,there exist y which satisfied
$x=y^{n-1}*n+....+b^n$
so $f(x)=const=-1$ which is wrong 
*****
case 2: f(a)=f(-a) for every $a\in R$
Let x=y=0 in (*) we get f(0)=-1 and f(1)=f(-1)=-1
and $f(x-f(y))=f(x+y^n)$  (**)
let x=-f(x) in (**)
we get $f(f(x)+f(y))=f(y^n-f(x))=f(x^n+y^n)$
let y=1,y=0 so $f(x^n)=f(x^n+1)$
so $f(x)=f(x+1)$ for every $x\ge 0$
do the same with case 1 so f(x)=const=-1
*********
 finally,$f(x)=-1$
\end{solution}



\begin{solution}[by \href{https://artofproblemsolving.com/community/user/29428}{pco}]
	Quite nice, congrats !

I specially appreciated the idea $f(x)=f(x+b)$ $\implies$ $f(0)=f(bny^{n-1}+...)$  :)

I looked for a solution of this problem during a long time without any success !
\end{solution}
*******************************************************************************
-------------------------------------------------------------------------------

\begin{problem}[Posted by \href{https://artofproblemsolving.com/community/user/91306}{ndk09}]
	Find all continuous functions $f : [0,1] \to [0,1]$ such that
\[f(x)\geq 2xf(x^2), \quad 0\leq x\leq 1.\]
	\flushright \href{https://artofproblemsolving.com/community/c6h375440}{(Link to AoPS)}
\end{problem}



\begin{solution}[by \href{https://artofproblemsolving.com/community/user/22804}{nayel}]
	Note that 
\[\int_0^1f(x)\,dx\ge\int_0^12xf(x^2)\,dx=\int_0^1f(u)\,du.\]
Hence we must have equality in the original equaion, i.e.
\begin{align}f(x)=2xf(x^2)\quad\forall x\in [0,1].\end{align}
Therefore
\[\int_a^bf(x)\,dx=\int_a^b2xf(x^2)\,dx=\int_{a^2}^{b^2}f(u)\,du.\]
Let $F(x)=\int_0^xf(x)\,dx$. Then from the above, $F(b)-F(a)=F(b^2)-F(a^2)$ for all $a,b\in [0,1]$. Since, from $(1)$ we have $f(0)=0=f(1)$, $F(0)=0$ and so $F(x)=F(x^2)$ for all $x\in [0,1]$. Thus
\[F(x)=F(x^{2^n})\quad\forall n\in\mathbb N.\]
But since $0\le x\le 1$, $x^{2^n}\to 0$ for very large $n$, and $F$ is continuous. Therefore we must have $F(x)=F(0)=0$ for all $x$. Thus $f(x)\equiv 0$ is the only solution.
\end{solution}



\begin{solution}[by \href{https://artofproblemsolving.com/community/user/29428}{pco}]
	Nice !

We can also write $x^2f(x^2)\le \frac 12 xf(x)$ and so $xf(x)\le \frac 12x^{\frac 12}f(x^{\frac 12})$

And so $xf(x)\le \frac 1{2^n}x^{2^{-n}}f(x^{2^{-n}})$

And setting $n\to +\infty$ : $xf(x)\le 0$ $\forall x$ and so $f(x)=0$

(and we need only continuity at $0$ and $1$)
\end{solution}



\begin{solution}[by \href{https://artofproblemsolving.com/community/user/91306}{ndk09}]
	Sorry, $f: [0,1] \longrightarrow R$
\end{solution}



\begin{solution}[by \href{https://artofproblemsolving.com/community/user/52090}{Dumel}]
	it doesn't change anything in the solutions above.
\end{solution}



\begin{solution}[by \href{https://artofproblemsolving.com/community/user/29428}{pco}]
	\begin{tcolorbox}it doesn't change anything in the solutions above.\end{tcolorbox}
It does, at least in mine.

My solution is based upon the fact that $f(x)\ge 0$ and so the conclusion $f(x)=0$

If $f(x)$ may be negative, I can no longer conclude  $f(x)=0$
\end{solution}
*******************************************************************************
-------------------------------------------------------------------------------

\begin{problem}[Posted by \href{https://artofproblemsolving.com/community/user/88228}{625gs}]
	1. Find all functions $ f: \mathbb{Q}\to\mathbb{Q} $ such that
\[f(x^2+y+f(xy))=(x-2)f(x)+f(x)f(y)+3\]
for all rationals $x$ and $y$.

2. For $a,b>0,n\in \mathbb{N}$ prove that
\[\frac {1}{a+b}+\frac {1}{a+2b}+\cdots+\frac {1}{a+nb}<\frac{n}{\sqrt{a(a+nb)}}.\]
	\flushright \href{https://artofproblemsolving.com/community/c6h375644}{(Link to AoPS)}
\end{problem}



\begin{solution}[by \href{https://artofproblemsolving.com/community/user/88228}{625gs}]
	please help me!
\end{solution}



\begin{solution}[by \href{https://artofproblemsolving.com/community/user/29428}{pco}]
	\begin{tcolorbox}1.Find all function $ f: \mathbb{Q}\to\mathbb{Q} $ such that
$f(x^2+y+f(xy))=(x-2)f(x)+f(x)f(y)+3$\end{tcolorbox}
Let $P(x,y)$ be the assertion $f(x^2+y+f(xy))=(x-2)f(x)+f(x)f(y)+3$
Let $f(0)=a$


1) Ugly proof that $a=f(0)=1$
===========================
$(e1)$ : $P(0,0)$ $\implies$ $f(a)=a^2-2a+3$
$(e2)$ : $P(a,0)$ $\implies$ $f(a^2+a)=2(a-1)f(a)+3$
$(e3)$ : $P(-a,0)$ $\implies$ $f(a^2+a)=-2f(-a)+3$
$(e4)$ : $P(0,-a)$ $\implies$ $3a-3=af(-a)$

$-2a(a-1)\times(e1) -a\times(e2) +a\times(e3) +2\times(e4)$ $\implies$  $(a-1)(a^3-2a^2+3a+3)=0$

But $a^3-2a^2+3a+3=0$ has no rational root and, since $a\in\mathbb Q$, we get $a=1$
Q.E.D.

2) $f(x)=x+1$ $\forall x\in\mathbb Q$
==================================
$P(0,x)$ $\implies$ $f(x+1)=f(x)+1$ and so $f(x)=x+1$ $\forall x\in\mathbb Z$ and $f(x+n)=f(x)+n$

$P(q,\frac pq)$ $\implies$ $f(q^2+\frac pq+f(p))=(q-2)f(q)+f(q)f(\frac pq)+3$

$\implies$ $f(\frac pq)+q^2+p+1=(q-2)(q+1)+(q+1)f(\frac pq)+3$

$\implies$ $f(\frac pq)=\frac pq+1$
Q.E.D

And so $\boxed{f(x)=x+1}$ which indeed is a solution.
\end{solution}



\begin{solution}[by \href{https://artofproblemsolving.com/community/user/64716}{mavropnevma}]
	2. We have $\dfrac {1} {(a+kb)(a+(k+1)b)} = \dfrac {1} {b} \left (\dfrac {1} {a+kb} - \dfrac {1} {a+(k+1)b} \right )$ for all $0\leq k \leq n-1$. This telescopes to  $\sum_{k=0}^{n-1} \dfrac {1} {(a+kb)(a+(k+1)b)} = \dfrac {1} {b} \left (\dfrac {1} {a} - \dfrac {1} {a+nb} \right ) = \dfrac {n} {a(a+nb)}$.

Now, $\dfrac {a+kb} {a+(k+1)b} < 1$  for all $0\leq k \leq n-1$, so $\sum_{k=0}^{n-1} \dfrac {a+kb} {a+(k+1)b} < n$.
Finally, by Cauchy-Schwarz, $\dfrac {n^2} {a(a+nb)} = n\cdot \dfrac {n} {a(a+nb)} >$ $\left (\sum_{k=0}^{n-1} \dfrac {a+kb} {a+(k+1)b}\right)\cdot \left(\sum_{k=0}^{n-1} \dfrac {1} {(a+kb)(a+(k+1)b)}\right ) \geq$ $\left (\sum_{k=0}^{n-1} \dfrac {1} {a+(k+1)b}\right)^2$, hence $ \dfrac {n} {\sqrt{a(a+nb)}} > \sum_{k=1}^{n} \dfrac {1} {a+kb}$.
\end{solution}
*******************************************************************************
-------------------------------------------------------------------------------

\begin{problem}[Posted by \href{https://artofproblemsolving.com/community/user/90621}{Love_Math1994}]
	Find all functions $f: \mathbb R \to \mathbb R$ such that for all reals $x$ and $y$, we have
\[f(x+y+f(y))=f(x).\]
	\flushright \href{https://artofproblemsolving.com/community/c6h375658}{(Link to AoPS)}
\end{problem}



\begin{solution}[by \href{https://artofproblemsolving.com/community/user/29428}{pco}]
	\begin{tcolorbox}Find all  $f:R-->R$ satify the equation 
 $ f(x+y+f(y))=f(x)$\end{tcolorbox}
Looks like http://www.artofproblemsolving.com/Forum/viewtopic.php?f=38&t=271589 (a very little bit different) :

Here is a general solution :

1) the solution :
===========
Let $\mathbb A$ any additive subgroup of $\mathbb R$
Let $\sim$ the equivalence relation $x\sim y$ $\iff$ $x-y\in \mathbb A$ and $c(x)$ any choice function which associates at each real number a representant of its equivalence class (unique per class)
Let $h(x)$ any function from $\mathbb R\to \mathbb A$

Then $f(x)=h(c(x))-c(x)$

2) proof that any function in the form given in (1) is a solution :
===========================================
$\forall x\in\mathbb R$ :
$x\sim c(x)$ and so $x-c(x)\in\mathbb A$
$h(c(x))\in\mathbb A$
$\mathbb A$ is an additive subgroup and so $f(x)+x=h(c(x))+x-c(x)\in \mathbb A$

So $x+y+f(y)\sim x$ and so $c(x+y+f(y))=c(x)$ and so $f(x+y+f(y))=f(x)$
Q.E.D.

3) proof that any solution may be written in the form given in (1), so that it is a "general" solution
=====================================================================
Let $f(x)$ such that $f(x+y+f(y))=f(x)$ $\forall x,y\in\mathbb R$
Let $\mathbb A=\{x\in\mathbb R : f(x+y)=f(y)$ $\forall y\in\mathbb R\}$

3.1) $\mathbb A$ is an additive subgroup
----------------------------------------
$0\in \mathbb A$
If $x\in \mathbb A$, we have $f(x+y)=f(y)$ $\forall y\in\mathbb R$. Let then $y_1\in\mathbb R$ and $y=y_1-x$ :
$f(x+y_1-x)=f(y_1-x)$ and so $f(y_1-x)=f(y_1)$ $\forall y_1$ and so $-x\in \mathbb A$

Let then $x_1,x_2\in\mathbb A$ and $y\in\mathbb R$ : $f(x_1+x_2+y)=f(x_1+(x_2+y))=f(x_2+y)=f(y)$ and so $x_1+x_2\in\mathbb A$

3.2) $\forall x\in\mathbb R$ : $f(x)+x\in \mathbb A$
---------------------------------------------------
This is an immediate result from $f(y+x+f(x))=f(y)$ $\forall x,y$

3.3) Final result :
------------------
Since $\mathbb A$ is a group, the relation $\sim$ : $x\sim y$ $\iff$ $x-y\in\mathbb A$ is an equivalence relation
Let $r(x)$ any choice function which associates at each real number a representant of its equivalence class (unique per class)
Let $h(x)=f(x)+x$
Since $f(x)+x\in\mathbb A$, $h(x)$ is a function from $\mathbb R\to\mathbb A$
Then :
$x\sim c(x)$ $\implies$ $x-c(x)\in\mathbb A$ $\implies$ $f(c(x)+(x-c(x)))=f(c(x))$ and so $f(x)=f(c(x))=f(c(x))+c(x)-c(x)$
$\implies$ $f(x)=h(c(x))-c(x)$
Q.E.D


4) Application : some examples
======================

4.1) $\mathbb A=\mathbb R$, $c(x)=0$ and $h(x)=a$ gives the solution $\boxed{f(x)=a}$

4.2) $\mathbb A=\{0\}$, $c(x)=x$ and $h(x)=0$ gives the solution $\boxed{f(x)=-x}$

4.3) $\mathbb A=\mathbb Z$, $c(x)=x-[x]$ and $h(x)=n$ gives the solution $\boxed{f(x)=n+[x]-x}$

4.4) $\mathbb A=\mathbb Z$, $c(x)=x-[x]$ and $h(x)=\left\lfloor 5\sin(\pi x)\right\rfloor$ gives the solution $\boxed{f(x)=\left\lfloor 5\sin(\pi (x-[x]))\right\rfloor+[x]-x}$

and infinitely many other solutions
\begin{tcolorbox}I think it is an open question :blush:\end{tcolorbox}
No longer consider it as an open question :)
\end{solution}



\begin{solution}[by \href{https://artofproblemsolving.com/community/user/90621}{Love_Math1994}]
	wow.my brother said that he think it is open question when he talk with him.
So give a huge thanks to you.very kind :D
\end{solution}
*******************************************************************************
-------------------------------------------------------------------------------

\begin{problem}[Posted by \href{https://artofproblemsolving.com/community/user/86097}{hurricane}]
	Find all functions $f:\mathbb{R}^+ \rightarrow \mathbb{R}^+$ such that $2f(x+yf(x))=f(x)f(y)$ for all $x, y>0$.
	\flushright \href{https://artofproblemsolving.com/community/c6h376062}{(Link to AoPS)}
\end{problem}



\begin{solution}[by \href{https://artofproblemsolving.com/community/user/29428}{pco}]
	\begin{tcolorbox}A difficult problem
Find all $f:\mathbb{R}^+ \rightarrow \mathbb{R}^+$ function such that $2f(x+yf(x))=f(x)f(y)$.\end{tcolorbox}
Let $P(x,y)$ be the assertion $2f(x+yf(x))=f(x)f(y)$

1) $f(x)\ge 1$ $\forall x$
=============
Suppose $\exists u$ such that $0<f(u)<1$

Then $P(u,\frac{u}{1-f(u)})$ $\implies$ $2f(\frac{u}{1-f(u)})=f(u)f(\frac{u}{1-f(u)})$ and so $f(u)=2$, contradiction
Q.E.D

2) $f(x)\ge 2$ $\forall x$
=============
Suppose $f(x)\ge a$ $\forall x$ for some $a\in[1,2)$ : $P(x,x)$ $\implies$ $f(x)^2=2f(x+xf(x))\ge 2a$ and so $f(x)\ge\sqrt {2a}$

And since $\sqrt{2a}\in(a,2)$ we can build a sequence $a_0=1$ and $a_{n+1}=\sqrt{2a_n}$ such that $f(x)>a_n$ $\forall n$

But $a_n$ is a convergent increasing sequence whose limit is $2$ and so $f(x)\ge 2$ $\forall x$
Q.E.D

3) $f(x)$ is non decreasing function
=======================
$P(x,\frac y{f(x)})$ $\implies$ $2f(x+y)=f(x)f(\frac y{f(x)})$ $\ge 2f(x)$ and so $f(x+y)\ge f(x)$
Q.E.D.

4) $f(x)$ is not injective
================
$P(x,1)$ $\implies$ $2f(x+f(x))=f(x)f(1)$
$P(1,x)$ $\implies$ $2f(1+xf(1))=f(x)f(1)$

So $f(x+f(x))=f(1+xf(1))$
So, if $f(x)$ is injective : $x+f(x)=1+xf(1)$
So $f(x)=(f(1)-1)x+1$ $=ax+1$ for some $a\ge 1$
and, plugging this in original equation, we get that this is not a solution.
Q.E.D.

5) $f(x)=2$ $\forall x$
============
Since $f(x)$ is not injective, $\exists u<v$ such that $f(u)=f(v)$

$P(u,\frac{v-u}{f(u)})$ $\implies$ $2f(v)=f(u)f(\frac{v-u}{f(u)})$ $\implies$ $f(\frac{v-u}{f(u)})=2$

Let $a=\frac{v-u}{f(u)}$ : $P(a,a)$ $\implies$ $2f(a+af(a))=f(a)^2=4$ and so $f(3a)=2$
And so $f(3^na)=2$ and so $f(x)=2$ $\forall x$ (remember $f(x)$ is non decreasing)
and this indeed is a solution
Q.E.D

Hence the unique solution : $\boxed{f(x)=2}$ $\forall x$
\end{solution}
*******************************************************************************
-------------------------------------------------------------------------------

\begin{problem}[Posted by \href{https://artofproblemsolving.com/community/user/89818}{gauman}]
	Given $f: \mathbb R^+\to \mathbb R$ with $f(xy) \le \frac{f(x)}{y}+f(y)$ for all $x,y>0$, prove that there exists a real number $s$ such that $|f(x)| \leq s$ for all $x \geq 1$.
	\flushright \href{https://artofproblemsolving.com/community/c6h376109}{(Link to AoPS)}
\end{problem}



\begin{solution}[by \href{https://artofproblemsolving.com/community/user/29428}{pco}]
	\begin{tcolorbox}Given $f: R+\to R : f(xy) \le \frac{f(x)}{y}+f(y)$ with $x,y>0$. Prove that exist $s$: $|f(x)| \le s$ with $x \ge1$\end{tcolorbox}
Let $P(x,y)$ be the assertion $f(xy)\le \frac{f(x)}y+f(y)$

1) $f(x)$ is upper bounded for $x\ge 1$
======================
$P(\frac 12,2x)$ $\implies$ $f(x)\le \frac{f(\frac 12)}{2x}+f(2x)$

$P(x,2)$ $\implies$ $f(2x)\le \frac{f(x)}2+f(2)$

Adding these two lines, we get $f(x)\le \frac{f(\frac 12)}{2x}+\frac{f(x)}2+f(2)$

And so $f(x)\le \frac{f(\frac 12)}{x}+2f(2)$ and so $f(x)\le |f(\frac 12)|+2|f(2)|$ $\forall x\ge 1$
Q.E.D.

2) $f(x)$ is lower bounded for $x\ge 1$
========================
$P(2,\frac x2)$ $\implies$ $f(x)\le \frac{2f(2)}x+f(\frac x2)$

$P(x,\frac 12)$ $\implies$ $f(\frac x2)\le 2f(x)+f(\frac 12)$

Adding these two lines, we get $f(x)\le \frac{2f(2)}x+2f(x)+f(\frac 12)$

And so $ -\frac{2f(2)}x-f(\frac 12)\le f(x)$ and so $-|2f(2)|-|f(\frac 12)|\le f(x)$ $\forall x\ge 1$
Q.E.D.

Hence the result : $\boxed{|f(x)|\le |f(\frac 12)|+2|f(2)|\text{   }\forall x\ge 1}$
\end{solution}
*******************************************************************************
-------------------------------------------------------------------------------

\begin{problem}[Posted by \href{https://artofproblemsolving.com/community/user/92753}{WakeUp}]
	Find all functions $f:\mathbb{Q}^{+}\rightarrow \mathbb{Q}^{+}$ which for all $x \in \mathbb{Q}^{+}$ fulfil
\[f\left(\frac{1}{x}\right)=f(x) \ \ \text{and} \ \ \left(1+\frac{1}{x}\right)f(x)=f(x+1). \]
	\flushright \href{https://artofproblemsolving.com/community/c6h376120}{(Link to AoPS)}
\end{problem}



\begin{solution}[by \href{https://artofproblemsolving.com/community/user/29428}{pco}]
	\begin{tcolorbox}Find all functions $f:\mathbb{Q}^{+}\rightarrow \mathbb{Q}^{+}$ which for all $x \in \mathbb{Q}^{+}$ fulfil
\[f\left(\frac{1}{x}\right)=f(x) \ \ \text{and} \ \ \left(1+\frac{1}{x}\right)f(x)=f(x+1). \]\end{tcolorbox}
Let $p,q\in\mathbb N$ such that $\gcd(p,q)=1$. Let then $h(\frac pq)=\frac{f(\frac pq)}{pq}$ 

The first equation implies $h(x)=h(\frac 1x)$ and the second equation implies $h(x+1)=h(x)$

And so, $h([a_1;a_2,a_3,...,a_n])$ $=h([0;a_2,a_3,...,a_n])$ $=h([a_2;a_3,...,a_n])$ $=...=h(a_n)=h(1)$

Hence the solution : $\boxed{f(\frac pq)=a\frac{pq}{\gcd(p,q)^2}}$ $\forall p,q\in\mathbb N$ and for any $a\in\mathbb Q^+$
\end{solution}



\begin{solution}[by \href{https://artofproblemsolving.com/community/user/92964}{dyta}]
	\begin{tcolorbox}And so, $h([a_1;a_2,a_3,...,a_n])$ $=h([0;a_2,a_3,...,a_n])$ $=h([a_2;a_3,...,a_n])$ $=...=h(a_n)=h(1)$\end{tcolorbox}

What is $a_1,a_2,...$ and [a_1;a_2,a_3,...,a_n],please? Are they integer numbers?
\end{solution}



\begin{solution}[by \href{https://artofproblemsolving.com/community/user/29428}{pco}]
	\begin{tcolorbox}[quote="pco"]And so, $h([a_1;a_2,a_3,...,a_n])$ $=h([0;a_2,a_3,...,a_n])$ $=h([a_2;a_3,...,a_n])$ $=...=h(a_n)=h(1)$\end{tcolorbox}

What is $a_1,a_2,...$ and [a_1;a_2,a_3,...,a_n],please? Are they integer numbers?\end{tcolorbox}


$a_i$ all are positive integers and $[a_1;a_2,...,a_n]$ is one classical representation of the continued fraction of a rational.
see http://en.wikipedia.org\/wiki\/Continued_fraction
\end{solution}
*******************************************************************************
-------------------------------------------------------------------------------

\begin{problem}[Posted by \href{https://artofproblemsolving.com/community/user/59935}{hana1122}]
	Let \[f(x)=\prod_{k=0}^{n-1} \left(x-\cos\frac{2k\pi}n\right).\] Prove that 
\[f'(x)=n\prod_{k=1}^{n-1} \left(x-\cos\frac{k\pi}n\right).\]
	\flushright \href{https://artofproblemsolving.com/community/c6h376400}{(Link to AoPS)}
\end{problem}



\begin{solution}[by \href{https://artofproblemsolving.com/community/user/29428}{pco}]
	\begin{tcolorbox}Let $f(x)=\prod_{k=0}^n \left(x-\cos\frac{2k\pi}n\right)$. prove that 
\[f'(x)=n\prod_{k=1}^n \left(x-\cos\frac{k\pi}n\right)\]\end{tcolorbox}


Wrong :( :(

Choose $n=1$ (very difficult check and I understand you did not try it) and then :

$f(x)=\prod_{k=0}^1 \left(x-\cos 2k\pi\right)$ $=(x-1)^2$

$f'(x)=2(x-1)\ne 1\prod_{k=1}^1 \left(x-\cos k\pi\right)$ $=x+1$
\end{solution}



\begin{solution}[by \href{https://artofproblemsolving.com/community/user/64716}{mavropnevma}]
	It is enough to notice the leading coefficients do not match; $\dfrac {d} {dx} x^{n+1} = (n+1)x^n$.
Have you tried with slightly changed limits of summation? Probably a snafu between $n$ and $n-1$ ...
\end{solution}



\begin{solution}[by \href{https://artofproblemsolving.com/community/user/29428}{pco}]
	Oh Yes, Oh yes! Let us play to the great game "Here is a wrong problem! please modify it in order to find a good problem and solve it" !!

$\left(\prod_{k=0}^n \left(x-\cos\frac{2k\pi}n\right)\right)'$ $=n\prod_{k=1}^n \left(x-\cos\frac{k\pi}n\right)$
Wrong (test $n=1$ or see that leading coefficients dont match)

$\left(\prod_{k=0}^n \left(x-\cos\frac{2k\pi}n\right)\right)'$ $=(n+1)\prod_{k=1}^n \left(x-\cos\frac{k\pi}n\right)$
Wrong (test $n=1$)

$\left(\prod_{k=1}^{n} \left(x-\cos\frac{2k\pi}n\right)\right)'$ $=n\prod_{k=2}^{n} \left(x-\cos\frac{k\pi}n\right)$
wrong (test $n=2$)

$\left(\prod_{k=0}^n \left(x-\cos^2\frac{k\pi}n\right)\right)'$ $=(n+1)\prod_{k=1}^n \left(x-\cos\frac{k\pi}n\right)$
Wrong (test $n=1$)

$\left(\prod_{k=0}^n \left(x-0\times\cos\frac{2k\pi}n\right)\right)'$ $=n\prod_{k=1}^n \left(x-0\times\cos\frac{k\pi}n\right)$
It's true ! $LHS=x^{n+1}$ (I dont give the proof but it's not very difficult) and $RHS=nx^n$ (I dont give the proof).

I won ! I won ! What an interesting game !
\end{solution}



\begin{solution}[by \href{https://artofproblemsolving.com/community/user/59935}{hana1122}]
	\begin{tcolorbox}[quote="hana1122"]Let $f(x)=\prod_{k=0}^n \left(x-\cos\frac{2k\pi}n\right)$. prove that 
\[f'(x)=n\prod_{k=1}^n \left(x-\cos\frac{k\pi}n\right)\]\end{tcolorbox}


Wrong :( :(

Choose $n=1$ (very difficult check and I understand you did not try it) and then :

$f(x)=\prod_{k=0}^1 \left(x-\cos 2k\pi\right)$ $=(x-1)^2$

$f'(x)=2(x-1)\ne 1\prod_{k=1}^1 \left(x-\cos k\pi\right)$ $=x+1$\end{tcolorbox}
Ok! I edited it. Now it seems really a nice problem. try it!
\end{solution}



\begin{solution}[by \href{https://artofproblemsolving.com/community/user/64716}{mavropnevma}]
	What did I tell you, pco?  :)
\end{solution}



\begin{solution}[by \href{https://artofproblemsolving.com/community/user/29428}{pco}]
	\begin{tcolorbox}What did I tell you, pco?  :)\end{tcolorbox}

You were right, as often  :)
\end{solution}



\begin{solution}[by \href{https://artofproblemsolving.com/community/user/16261}{Rust}]
	consider numbers $x_m=\cos\frac{m\pi}{n},m=1,...,n-1$. If $m=2k$ - even, then $f(x)$ had double root $x_m$, because $\cos \frac{2k\pi}{n}=\cos\frac{2(n-k)}{n}$. Because $x_m$ is double root of $f(x)$, it is root of $f'(x)$. 
\[f(-x)=(-1)^n\prod_{k=0}^{n-1}(x+\cos\frac{2k\pi}{n})=f(x), \ if \ n \ is \ even.\]
If $n$ is odd $f(x)=(x-1)g(x), g(-x)=g(x)$.
If $m,n$ are odd, then $x_m=-x_{n-m}$ is root of $f'(x)$ too, because $n-m$ is even.
If $m$ is odd, $n$ is even, by symmetric $f'(x_m)=0$.
\end{solution}



\begin{solution}[by \href{https://artofproblemsolving.com/community/user/59935}{hana1122}]
	\begin{tcolorbox}consider numbers $x_m=\cos\frac{m\pi}{n},m=1,...,n-1$. If $m=2k$ - even, then $f(x)$ had double root $x_m$, because $\cos \frac{2k\pi}{n}=\cos\frac{2(n-k)}{n}$. Because $x_m$ is double root of $f(x)$, it is root of $f'(x)$. 
\[f(-x)=(-1)^n\prod_{k=0}^{n-1}(x+\cos\frac{2k\pi}{n})=f(x), \ if \ n \ is \ even.\]
If $n$ is odd $f(x)=(x-1)g(x), g(-x)=g(x)$.
If $m,n$ are odd, then $x_m=-x_{n-m}$ is root of $f'(x)$ too, because $n-m$ is even.
If $m$ is odd, $n$ is even, by symmetric $f'(x_m)=0$.\end{tcolorbox}
I don't understand your solution in the case $n$ is even. I accept that for any even $m$, $x_m$ is a double root of $f$ and so is a root of $f'$. but in this case, $x_m=x_{2n-m}=-x_{n-m}=-x_{n+m}$ and so the relation $f(x)=f(-x)$ doesn't imply anything.
\end{solution}
*******************************************************************************
-------------------------------------------------------------------------------

\begin{problem}[Posted by \href{https://artofproblemsolving.com/community/user/83439}{Zeus93}]
	Find all functions $f: \mathbb R \to \mathbb R$ such that for all reals $x$ and $y$,
\[f(x+y) +f(x)f(y) = f(xy) +f(x)+f(y).\]
	\flushright \href{https://artofproblemsolving.com/community/c6h376663}{(Link to AoPS)}
\end{problem}



\begin{solution}[by \href{https://artofproblemsolving.com/community/user/29428}{pco}]
	\begin{tcolorbox}find all $f: \mathbf{R} \rightarrow \mathbf{R} $which satisfy:
$f(x+y) +f(x)f(y) = f(xy) +f(x)+f(y) \forall x,y\in \mathbf{R}$\end{tcolorbox}
Let $P(x,y)$ be the assertion $f(x+y)+f(x)f(y)=f(xy)+f(x)+f(y)$
Let $f(0)=a$
Let $f(1)=b$

$P(x,0)$ $\implies$ $a(f(x)-2)=0$
If $f(x)=2$ $\forall x$, we got a first solution
If $f(t)\ne 2$ for some $t$, then $a=0$

$P(-1,1)$ $\implies$ $(b-2)(f(-1)-1)=2$ and so $b\ne 2$
$P(x,1)$ $\implies$ $f(x+1)=(2-b)f(x)+b$ with $b\ne 2$ 

1) $b=1$
======
Then $f(x+1)=f(x)+1$ and so $f(x+n)=f(x)+n$ and $f(n)=n$

$P(x,y+1)$ $\implies$ $f(x+y)+f(x)f(y)+f(x)=f(xy+x)+f(x)+f(y)$
Subtracting $P(x,y)$ from this equality, we get : $f(xy+x)=f(xy)+f(x)$

And so $f(x+y)=f(x)+f(y)$ $\forall x,y$ and so (using $P(x,y)$) : $f(xy)=f(x)f(y)$

And the system "$f(1)=1$ and $f(x+y)=f(x)+f(y)$ and $f(xy)=f(x)f(y)$" is a very classical equation whose solution is $f(x)=x$ which indeed is a solution.

2) $b\ne 1$
=======
Let $u=\frac b{1-b}$ and the equation $f(x+1)=(2-b)f(x)+b$ implies $f(x+1)+u=(2-b)(f(x)+u)$ and so $f(x+n)=(2-b)^n(f(x)+u)-u$

So $f(n)=u((2-b)^n-1)$

$P(m,n)$ $\implies$ $u((2-b)^{m+n}-1)+u^2(2-b)^n-1)((2-b)^m-1)$ $=u((2-b)^{mn}-1)+u((2-b)^m-1)+u((2-b)^n-1)$

2.1) $u=0$ and so $b=0$
---------------------------
$P(x,1)$ $\implies$ $f(x+1)=2f(x)$
$P(x,y+1)$ $\implies$ $2f(x+y)+2f(x)f(y)=f(xy+x)+f(x)+2f(y)$
Subtracting $2P(x,y)$ from the above equality, we get $f(xy+x)=2f(xy)+f(x)$ and so $f(x+y)=f(x)+2f(y)$ $\forall y,\forall x\ne 0$
And so $f(x)+2f(y)=f(y)+2f(x)$ $\forall x,y\ne 0$ and so $f(x)=f(1)=0$ $\forall x\ne 0$

and so $f(x)=0$ $\forall x$ which indeed is a solution.

2.2) $b\ne 0$ and so $u\ne 0$
----------------------------------
Setting $c=2-b\notin\{0,1,2\}$, the equation $P(m,n)$ becomes : $(u+1)(c^m-1)(c^n-1)=c^{mn}-1$
And so $|c|=1$ (consider the two cases $|c|>1$ and $|c|<1$ and set $m,n\to +\infty$) and $c=-1$ and $b=3$

$P(x,1)$ $\implies$ $f(x+1)=3-f(x)$ and so $f(x+2)=f(x)$ and $f(2)=0$

$P(x,2)$ $\implies$ $f(x+2)=f(2x)+f(x)$ and so $f(2x)=0$, and so contradiction with $f(1)=3$


\begin{bolded}Hence the solutions :\end{bolded}\end{underlined}
$f(x)=0$
$f(x)=2$
$f(x)=x$
\end{solution}
*******************************************************************************
-------------------------------------------------------------------------------

\begin{problem}[Posted by \href{https://artofproblemsolving.com/community/user/85498}{nicolasteo}]
	Given positive integers $k$ and $n$, find all functions $f:\mathbb{R}\to \mathbb{R}$ which satisfy
\[f\left ( x-f\left ( y \right ) \right )=f\left ( x+y^{n} \right )+f\left ( f\left ( y \right )+y^n \right )+k\] 
for all $x,y\in \mathbb{R}$.
	\flushright \href{https://artofproblemsolving.com/community/c6h376670}{(Link to AoPS)}
\end{problem}



\begin{solution}[by \href{https://artofproblemsolving.com/community/user/29428}{pco}]
	\begin{tcolorbox}Given $k,n$ are positive integer numbers. Find all functions $f:\mathbb{R}\rightarrow \mathbb{R}$ satisfy
\[f\left ( x-f\left ( y \right ) \right )=f\left ( x+y^{n} \right )+f\left ( f\left ( y \right )+y^n \right )+k\] $\forall x,y\in \mathbb{R}$\end{tcolorbox}
I wonder if you are serious. Is it a really olympiad exercise you got somewhere ?. Please, give us an honest response.

It's according to me a very difficult problem far above the level of olympiad contests (IMHO).

Just look at the case $n=1$, for example (I think we can try to prove that $f(x)$ is constant if $n>1$ since $k>0$ (else $k=0$ and $f(x)=-x^n$ is also a solution)) :

The equation is $f(x-f(y))=f(x+y)+f(y+f(y))+k$ and so $f(x-f(y))=f(x+y)+\frac{a+k}2$ where $a=f(0)$

So $f(x)=f(x+y+f(y))+\frac{a+k}2$ which is equivalent to the system :

$f(x+a)=f(x+y+f(y))$
$f(x+a)=f(x)-\frac{a+k}2$

Let then $g(x)=f(x)-a$ and the first equation becomes $g(x)=g(x+y+g(y))$ with $g(0)=0$

This is a hard problem whose general solution without constraint $g(0)=0$ (see http://www.artofproblemsolving.com/Forum/viewtopic.php?f=36&t=375658) is :
Let $\mathbb A$ any additive subgroup of $\mathbb R$
Let $\sim$ the equivalence relation $x\sim y$ $\iff$ $x-y\in \mathbb A$ and $c(x)$ any choice function which associates at each real number a representant of its equivalence class (unique per class)
Let $h(x)$ any function from $\mathbb R\to \mathbb A$

Then $g(x)=h(c(x))-c(x)$

The constraint $g(0)=0$ just implies $h(c(0))=c(0)$ which is always possible since $c(0)\in \mathbb A$

So $f(x)=h(c(x))-c(x)+a$ and it remains to fullfill the second equation $f(x+a)=f(x)-\frac{a+k}2$

And so $h(c(x+a))-c(x+a)=h(c(x))-c(x)-\frac{a+k}2$

I did not succeed up to now finding a general solution for this complementary equation, but it's easy to find infinitely many solutions :

Just choose $a=-k$ and $-k\in\mathbb A$

And so a set of solutions when n=1 (but not all the solutions)\end{underlined} :
Let $\mathbb A$ any additive subgroup of $\mathbb R$ such that $k\in\mathbb A$
Let $\sim$ the equivalence relation $x\sim y$ $\iff$ $x-y\in \mathbb A$ and $c(x)$ any choice function which associates at each real number a representant of its equivalence class (unique per class)
Let $h(x)$ any function from $\mathbb R\to \mathbb A$ such that $h(c(0))=c(0)$

Then $f(x)=h(c(x))-c(x)-k$ 

\begin{bolded}Examples of solutions :\end{bolded}\end{underlined}

1) $\mathbb A=\mathbb R$ and $c(x)=0$ and so $\boxed{f(x)=-k}$

2) $\mathbb A=\mathbb Z$ and $c(x)=x-[x]$ and $h(x)=0$ and so $\boxed{f(x)=[x]-x-k}$

3) $\mathbb A=\mathbb Z$ and $c(x)=x-[x]$ and $h(x)=[xt(x)]$ and so $\boxed{f(x)=[(x-[x])t(x-[x])]+[x]-x-k}$ for any function $t(x)$

4) $\mathbb A=k\mathbb Z$ ... and so on

So, just for $n=1$, a big family of solutions (and likely not all the solutions) after complex considerations.

I cant believe you really got this problem in olympiad training :( :(
\end{solution}



\begin{solution}[by \href{https://artofproblemsolving.com/community/user/43269}{JoeBlow}]
	I think if we make a further (not too strong) assumption that there exists a real number $\xi $ such that $f$ is bounded on the interval $[\xi,\xi+k)$, it is not hard to show that $f$ is periodic with period $k$, i.e. that $f(0)=-k$ (or $a=-k$ in pco's notation, simplifying some of his analysis).
\end{solution}



\begin{solution}[by \href{https://artofproblemsolving.com/community/user/85498}{nicolasteo}]
	Thank you, pco.

\begin{tcolorbox}I can't believe you really got this problem in olympiad training :( :(\end{tcolorbox}

This problem was appeared in my books. My books have a hint and not a solution.
\begin{tcolorbox}Given $n$ are positive integer numbers. Find all functions $f:\mathbb{R}\rightarrow \mathbb{R}$ satisfy
\[f\left ( x-f\left ( y \right ) \right )=f\left ( x+y^{n} \right )+f\left ( f\left ( y \right )+y^n \right )+2004\] $\forall x,y\in \mathbb{R}$\end{tcolorbox} 
Hint is $f(x)=-2004$

In another my book
\begin{tcolorbox}Given $n$ are positive integer numbers. Find all functions $f:\mathbb{R}\rightarrow \mathbb{R}$ satisfy
\[f\left ( x-f\left ( y \right ) \right )=f\left ( x+y^{n} \right )+f\left ( f\left ( y \right )+y^n \right )+2009\] $\forall x,y\in \mathbb{R}$\end{tcolorbox}
Hint is $f(x)=-2009$

And my teacher gave me a problem
\begin{tcolorbox}Find all functions $f:\mathbb{R}\rightarrow \mathbb{R}$ satisfy
\[f\left ( x-f\left ( y \right ) \right )=f\left ( x+y^{2009} \right )+f\left ( f\left ( y \right )+y^{2009} \right )+1\] $\forall x,y\in \mathbb{R}$\end{tcolorbox}
Can you help me solve the problem of my teacher, please ?
Thank you again, pco.
\end{solution}



\begin{solution}[by \href{https://artofproblemsolving.com/community/user/29428}{pco}]
	\begin{tcolorbox}Thank you, pco.

\begin{tcolorbox}I cant believe you really got this problem in olympiad training :( :(\end{tcolorbox}

This problem was appeared in my book. My book have a hint and not a solution.
\begin{tcolorbox}Given $n$ are positive integer numbers. Find all functions $f:\mathbb{R}\rightarrow \mathbb{R}$ satisfy
\[f\left ( x-f\left ( y \right ) \right )=f\left ( x+y^{n} \right )+f\left ( f\left ( y \right )+y^n \right )+2004\] $\forall x,y\in \mathbb{R}$\end{tcolorbox} 
Hint is $f(x)=-2004$

In another my book
\begin{tcolorbox}Given $n$ are positive integer numbers. Find all functions $f:\mathbb{R}\rightarrow \mathbb{R}$ satisfy
\[f\left ( x-f\left ( y \right ) \right )=f\left ( x+y^{n} \right )+f\left ( f\left ( y \right )+y^n \right )+2009\] $\forall x,y\in \mathbb{R}$\end{tcolorbox}
Hint is $f(x)=-2009$

And my teacher gave me a problem
\begin{tcolorbox}Find all functions $f:\mathbb{R}\rightarrow \mathbb{R}$ satisfy
\[f\left ( x-f\left ( y \right ) \right )=f\left ( x+y^{2009} \right )+f\left ( f\left ( y \right )+y^{2009} \right )+1\] $\forall x,y\in \mathbb{R}$\end{tcolorbox}
Can you help me solve the problem of my teacher, please ?
Thank you again, pco.\end{tcolorbox}


Obviously $f(x)=-k$ is a solution $\forall n$

Assuming that $k>0$, it's not very difficult to show that it is the unique solution if $n>1$

But obviously these hints are wrong if $n=1$ : just look at the infinitely many solutions I gave you in this case.
\end{solution}



\begin{solution}[by \href{https://artofproblemsolving.com/community/user/29428}{pco}]
	\begin{tcolorbox} Assuming that $k>0$, it's not very difficult to show that it is the unique solution if $n>1$.\end{tcolorbox}
Let us consider $n>1$

Let $P(x,y)$ be the assertion $f(x-f(y))=f(x+y^n)+f(f(y)+y^n)+k$
Let $f(0)=a$

$P(f(x),x)$ $\implies$ $f(f(x)+x^n)=\frac {a-k}2$ and so $P(x,y)$ becomes :

New assertion $Q(x,y)$ : $f(x-f(y))=f(x+y^n)+\frac{a+k}2$

Then two situations :

1) $|f(x)+x^n|=c$ constant $\forall x$
=========================
Then $f(x)=e(x)c-x^n$ where $e(x)=\pm 1$

$Q(0,y)$ $\implies$ $e(-f(y))c-(y^n-e(y)c)^n=e(y^n)c-y^{n^2}+\frac{a+k}2$

and so $y^{n^2}-(y^n-e(y)c)^n=e(y^n)c-e(-f(y))c+\frac{a+k}2$

And, since $n>1$, and setting $y\to +\infty$, we get $LHS\equiv ny^{n^2-n}e(y)c$ while RHS is bounded.

So $c=0$ and then $f(x)=-x^n$ and so $a=0$ and $Q(x,y)$ becomes :

$0=\frac{a+k}2$ and so $k=0$, which is impossible since $k>0$
So no solution in this case.

2) $|f(x)+x^n|$ is not a contant $\forall x$
============================
Then $\exists x,y,u,v$ with $|u|\ne |v|$ such that $f(x)+x^n=u$ and $f(y)+y^n=v$ and so $f(u)=f(v)$

Then, comparing $Q(x,u)$ and $Q(x,v)$, we get $f(x+u^n)=f(x+v^n)$ $\forall x$

Then, comparing $Q(y,x+u^n)$ and $Q(y,x+v^n)$, we get $f(y+(x+u^n)^n)=f(y+(x+v^n)^n)$ $\forall x,y$

Consider now the system of equation :
$y+(x+u^n)^n=a$
$y+(x+v^n)^n=b$

equivalent to :

$y=a-(c+u^n)^n$
$(x+u^n)^n-(x+v^n)^n=a-b$

Since $n>1$ and $|u|\ne|v|$, this last equation has always solution if $n$ is even but have also solution for $n$ odd as soon as $a-b>A$ if $u^n>v^n$ or $a-b<A$ is $u^n<v^n$ for some real $A$

So $f(a)=f(b)$ $\forall a,b$ such that $a>b+A$ (the case $b>a-A$ is exactly the same)

So $f(b+x)=f(b)$ $\forall x>A$
So $f(x)=constant$ $\forall x>b+A$, $\forall b$
So $f(x)$ is constant

And plugging back in original equation, we get $f(x)=-k$
\end{solution}
*******************************************************************************
-------------------------------------------------------------------------------

\begin{problem}[Posted by \href{https://artofproblemsolving.com/community/user/72731}{goodar2006}]
	Find all functions $f: \mathbb R \to \mathbb R$ such that for all reals $x$ and $y$,
\[f(y^4+f(x)-x)=(f(y))^4.\]
	\flushright \href{https://artofproblemsolving.com/community/c6h376683}{(Link to AoPS)}
\end{problem}



\begin{solution}[by \href{https://artofproblemsolving.com/community/user/72731}{goodar2006}]
	any idea??
\end{solution}



\begin{solution}[by \href{https://artofproblemsolving.com/community/user/64868}{mahanmath}]
	I`m really eager to see a solution , does anybody have an idea to start ? :maybe:
\end{solution}



\begin{solution}[by \href{https://artofproblemsolving.com/community/user/92334}{vanstraelen}]
	The function f with condition:
$ f(y^{4}+f(x)-x)=(f(y))^{4} $

If $f(x)=x$, then $f(y)=y$
If $f(x)=x$, then also $f(y^{4})=y^{4}=(f(y))^{4}$               (1)
The condition:
$ f(y^{4}+0)=(f(y))^{4} $
$ f(y^{4})=(f(y))^{4} $, see (1)
\end{solution}



\begin{solution}[by \href{https://artofproblemsolving.com/community/user/29428}{pco}]
	\begin{tcolorbox}$f:\mathbb R \longrightarrow \mathbb R$
$f(y^4+f(x)-x)=(f(y))^4$\end{tcolorbox}
Sorry for the long delay. This equation seemed to me not so easy :)

Let $P(x,y)$ be the assertion $f(y^4+f(x)-x)=f(y)^4$
Let $A=\{f(x)-x$ $\forall x\in\mathbb R\}$
Let $T=\{f(x)^4-x^4$ $\forall x\in\mathbb R\}$

Let $a\in A$ : $f(x^4+a)-(x^4+a)=f(x)^4-x^4-a$ and so $u-a\in A$ $\forall u\in T,a\in A$
So $u-(v-a)=a+(u-v)\in A$ and so $a+n(u-v)\in A$ $\forall u,v\in T$, $\forall a\in A$, $\forall n\in \mathbb N$

1) if $T=\{c\}$ has a unique element
==========================
Then $f(x)^4=x^4+c$ $\forall x$
$\implies$ $f(x^4+a)=f(x)^4=x^4+c$ and so $f(x+a)=x+c$ $\forall x\ge 0$, $\forall a\in A$
$\implies$ $f(x)=x+c-a$ $\forall x\ge a$, $\forall a\in A$ and so $A=\{a\}$ and $f(x)=x+a$ $\forall x$
And so, plugging this in original equation, $a=0$ and $f(x)=x$ which indeed is a solution.

2) if $\exists u>v\in T$
===============
2.1) preliminary results : $A$ is unbounded and $f(-x)=f(x)\ge 0$ $\forall x$
----------------------------------------------------------------------------
$a+n(u-v)\in A$ and $a+n(v-u)\in A$ implies that $A$ is neither upper bounded, neither lower bounded.

$\implies$ $\forall x$, $\exists a\in A$ such that $a<x$ and so $f(x)=f((\sqrt[4]{x-a})^4+a)=f(\sqrt[4]{x-a})^4$ and so $f(x)\ge 0$ $\forall x$
Comparing $P(0,x)$ and $P(0,-x)$, we get thet $f(x)^4=f(-x)^4$ and so $f(-x)=f(x)$ $\forall x$ since $f(x)\ge 0$ $\forall x$

2.2) $f(x)$ is constant
-----------------------
Let $x>y$
Let $a=f(\frac{x-y}2)-\frac{x-y}2\in A$

Let $b=f(\frac{y-x}2)-\frac{y-x}2\in A$. Notice that $b-a=x-y$ since $f(x)$ is an even function (see 2.1)

Let $u\ne v\in T$ and $n\in\mathbb N$ such that $y\ge a+n(u-v)$ and let $z=\sqrt[4]{y-a-n(u-v)}$ Then :

$a+n(u-v)\in A$ $\implies$ $f(z^4+a+n(u-v))=f(z)^4$ and so $f(y)=f(z)^4$
$b+n(u-v)\in A$ $\implies$ $f(z^4+b+n(u-v))=f(z)^4$ and so $f(y+b-a)=f(z)^4$ and so $f(x)=f(z)^4$

And so $f(x)=f(y)$
Q.E.D.

3) Solutions
========
The only constant solutions are $f(x)=0$ and $f(x)=1$
Hence the result :
$f(x)=x$
$f(x)=0$
$f(x)=1$
\end{solution}



\begin{solution}[by \href{https://artofproblemsolving.com/community/user/72731}{goodar2006}]
	thanks pco. interesting solution! :D
\end{solution}
*******************************************************************************
-------------------------------------------------------------------------------

\begin{problem}[Posted by \href{https://artofproblemsolving.com/community/user/92753}{WakeUp}]
	For a positive integer $k$, let $f_1(k)$ be the square of the sum of the digits of $k$. Define $f_{n+1}$ = $f_1 \circ f_n$ . Evaluate $f_{2007}(2^{2006} )$.
	\flushright \href{https://artofproblemsolving.com/community/c6h376708}{(Link to AoPS)}
\end{problem}



\begin{solution}[by \href{https://artofproblemsolving.com/community/user/29428}{pco}]
	\begin{tcolorbox}For a positive integer $k$, let $f_1(k)$ be the square of the sum of the digits of $k$. Define $f_{n+1}$ = $f_1 \circ f_n$ . Evaluate $f_{2007}(2^{2006} )$.\end{tcolorbox}
Let $n(x)$ the number of digits of $x$ and $s(x)$ the sum of digits of $x$.

$n(x)\le 1+\log_{10}(x)$ and so $s(x)\le 9n(x)=9(1+\log_{10}x)$ and $f(x)\le 81(1+\log_{10}x)^2$

So $f(2^{2006})\le 81(1+2006\log_{10} 2)^2$ $\le 81(1+\frac {2006}3)^2$ $<81\cdot 700^2 <10^8$

$f(f(2^{2006}))\le 81(1+8)^2 < 10^4$
$f(f(f(2^{2006}))) \le 81(1+4)^2=2025$
$f^{[4]}(2^{2006}) \le (1+9+9+9)^2= 784$
$f^{[5]}(2^{2006}) \le (6+9+9)^2= 576$
$f^{[6]}(2^{2006}) \le (4+9+9)^2= 484$
$f^{[7]}(2^{2006}) \le (3+9+9)^2= 441$ and this process enters then a loop on $441$

So we know that, from this point, $s(f^{[n]}(2^{2006}))\le 21$

But $2^{2006}\equiv 4\pmod 9$ and so it's clear that $s(f^{[n]}(2^{2006}))\equiv 4,7\pmod 9$ and so :

$s(f^{[n]}(2^{2006}))\in\{4,7,13,16\}$ $\forall n\ge 7$

And since :
$s(x)=4$ $\implies$ $f(x)=16$ whose sum of digits is $7$
$s(x)=7$ $\implies$ $f(x)=49$ whose sum of digits is $13$
$s(x)=13$ $\implies$ $f(x)=169$ whose sum of digits is $16$
$s(x)=16$ $\implies$ $f(x)=256$ whose sum of digits is $13$

we get that $f^{[n]}(2^{2006})\in\{169,256\}$ $\forall n\ge 10$

It remains to look at the values modulus $9$ to get that $f^{[2p]}(2^{2006})\equiv 4\pmod 9$ while $f^{[2p+1]}(2^{2006})\equiv 7\pmod 9$

And so $\boxed{f^{[2007]}(2^{2006})=169}$  since $2007$ is odd
\end{solution}
*******************************************************************************
-------------------------------------------------------------------------------

\begin{problem}[Posted by \href{https://artofproblemsolving.com/community/user/92753}{WakeUp}]
	Find all functions $f:\mathbb{R}^+\rightarrow\mathbb{R}^+$ which satisfy the following conditions:
$(\text{i})$ $f(x+f(y))=f(x)f(y)$ for all $x,y>0;$
$(\text{ii})$ there are at most finitely many $x$ with $f(x)=1$.
	\flushright \href{https://artofproblemsolving.com/community/c6h376745}{(Link to AoPS)}
\end{problem}



\begin{solution}[by \href{https://artofproblemsolving.com/community/user/29428}{pco}]
	\begin{tcolorbox}Find all functions $f:\mathbb{R}^+\rightarrow\mathbb{R}^+$ which satisfy the following conditions:
$(\text{i})$ $f(x+f(y))=f(x)f(y)$ for all $x,y>0;$
$(\text{ii})$ there are at most finitely many $x$ with $f(x)=1$.\end{tcolorbox}
Let $P(x,y)$ be the assertion $f(x+f(y))=f(x)f(y)$

1) preliminary results : $f(x)>1$ and $f(x)\ge x$ $\forall x$
=====================================
If $f(u)=1$, then $P(x,u)$ $\implies$ $f(x+1)=f(x)$ and so $f(x+n)=f(x)$ and so $f(u+n)=1$ $\forall n$ and so contradiction with $(ii)$.
So $f(x)\ne 1$ $\forall x$

If $f(u)<u$, then $P(u-f(u),u)$ $\implies$ $f(u-f(u))=1$ which is in contradiction with the above result.
So $f(x)\ge x$ $\forall x$

$P(x,x)$ $\implies$ $f(x+f(x))=f(x)^2$
$P(x+f(x),x)$ $\implies$ $f(x+2f(x))=f(x)^3$
And so $f(x+(n-1)f(x))=f(x)^n$ and so $f(x)^n\ge x+(n-1)f(x)>x$ and so $f(x)>x^{\frac 1n}$
And, setting $n\to +\infty$, we get $f(x)\ge 1$ $\forall n$ and so $f(x)>1$ $\forall x$
Q.E.D.

2) no such function exists
=================
Let $A=f(\mathbb R)$

$P(f(x),1)$ $\implies$ $f(f(x)+f(1))=f(f(x))f(1)$
$P(f(1),x)$ $\implies$ $f(f(x)+f(1))=f(f(1))f(x)$

And so $f(f(x))=cf(x)$ for some constant $c=\frac{f(f(1))}{f(1)}>1$

So $f(x)=cx$ $\forall x\in A$
$P(x,y)$ shows that $x,y\in A$ $\implies$ $xy\in A$ and so $x\in A$ $\implies$ $x^2\in A$ and so $f(x^2)=cx^2$

$f(x)\in A$ $\implies$ $f(x)>1$ and so $f(x)^2>f(x)$
$P(f(x)^2-f(x),x)$ $\implies$ $f(f(x)^2)=f(f(x)^2-f(x))f(x)$
$f(x)\in A$ $\implies$ $f(x)^2\in A$ $\implies$ $f(f(x)^2)=cf(x)^2$

And so $cf(x)^2=f(f(x)^2-f(x))f(x)$ and so $f(f(x)^2-f(x))=cf(x)>f(x)^2-f(x)$ and so $f(x)<c+1$

But this is in contradiction with the fact that $f(x)>1$ and that $f(x)\in A$ $\implies$ $f(x)^2\in A$
So no such function.
\end{solution}
*******************************************************************************
-------------------------------------------------------------------------------

\begin{problem}[Posted by \href{https://artofproblemsolving.com/community/user/92753}{WakeUp}]
	Determine all functions $f:\mathbb{R}\rightarrow\mathbb{R}$ such that for all $x, y, z,$
\[f(x+y^2+z)=f(f(x))+yf(y)+f(z). \]
	\flushright \href{https://artofproblemsolving.com/community/c6h376891}{(Link to AoPS)}
\end{problem}



\begin{solution}[by \href{https://artofproblemsolving.com/community/user/29428}{pco}]
	\begin{tcolorbox}Determine all functions $f:\mathbb{R}\rightarrow\mathbb{R}$ such that for all $x, y, z,$
\[f(x+y^2+z)=f(f(x))+yf(y)+f(z) \]\end{tcolorbox}
Let $P(x,y,z)$ be the assertion $f(x+y^2+z)=f(f(x))+yf(y)+f(z)$
Let $f(0)=a$

$P(x,0,1)$ $\implies$ $f(x+1)=f(f(x))+f(1)$
$P(1,0,x)$ $\implies$ $f(1+x)=f(f(1))+f(x)$

And so $f(f(x))=f(x)+b$ whith $b=f(f(1))-f(1)$

Then $P(x,0,y)$ becomes $f(x+y)=f(x)+f(y)+b$ and $P(0,0,0)$ gives then $b=-a$

$f(x+y)=f(x)+f(y)-a$ $\implies$ $f(x+y^2+z)=f(x)+f(y^2)+f(z)-2a$ and so $P(x,y,z)$ becomes $f(x)+f(y^2)+f(z)-2a=f(x)-a+yf(y)+f(z)$

And so $f(y^2)=yf(y)+a$ which implies (setting $y=1$) : $a=0$

So $f(x+y)=f(x)+f(y)$ and $f(f(x))=f(x)$ and $f(x^2)=xf(x)$

Then :
$f((x+1)^2)=f(x^2+x+x+1)=f(x^2)+2f(x)+f(1)$ $=xf(x)+2f(x)+f(1)$
$f((x+1)^2)=(x+1)f(x+1)$ $=xf(x)+f(x)+xf(1)+f(1)$

And so (subtracting these two lines) : $f(x)=f(1)x$ and so, plugging this in original equation : $f(x)=x$ or $f(x)=0$
 which indeed are solutions.

Hence the two answers : $\boxed{f(x)=x}$ and $\boxed{f(x)=0}$
\end{solution}



\begin{solution}[by \href{https://artofproblemsolving.com/community/user/109774}{littletush}]
	let $x=z=0,y=1$
then $f(y^2)=yf(y)+f(0)$
so $f(0)=0$.
hence $f(x^2)=xf(x)$
so $f(y^2+z)=f(y^2)+f(z)$
we get that when at least one of $x,y>0$,
$f(x+y)=f(x)+f(y)$.
since f is odd,hence $f(x+y)=f(x)+f(y)$
for all reals x,y
this's Cauchy's function,and what follows is tricial.
\end{solution}



\begin{solution}[by \href{https://artofproblemsolving.com/community/user/29428}{pco}]
	\begin{tcolorbox}let $x=z=0,y=1$
then $f(y^2)=yf(y)+f(0)$
so $f(0)=0$.\end{tcolorbox}
According to me $x=z=0$ implies $f(y^2)=yf(y)+f(f(0))+f(0)$ and not $f(y^2)=yf(y)+f(0)$
So you only got $f(f(0))+f(0)=0$


\begin{tcolorbox}...hence $f(x+y)=f(x)+f(y)$
for all reals x,y
this's Cauchy's function,and what follows is tricial.\end{tcolorbox}
I think that you might precise this trivial part. One should think that you claim that all solutions of Cauchy are trivial, which is obviously wrong.
\end{solution}



\begin{solution}[by \href{https://artofproblemsolving.com/community/user/151349}{andrejilievski}]
	Setting $ x=y=z=0, f(f(0))=f(0), x=z=0, f(y^2)=yf(y)+f(0), y=1, f(0)=0 $ Now for $ y=z=0, f(x)=f(f(x)), z=0, f(x+y^2)=f(x)+f(y^2) $ So $ f(x+y)=f(x)+f(y) $ and $ f(x^2)=f(x)x $ so $ f((x+1)^2)=(x+1)f(x+1)=(x+1)(f(x)+f(1))=xf(x)+xf(1)+f(x)+f(1) $ and $ f((x+1)^2) = f(x^2)+2f(x)+f(1) $ so $ f(x)=xf(1)=kx $ and now it is easy to see that $ k=1 $ so $ f(x)=0 $ and $ f(x)=x $
\end{solution}



\begin{solution}[by \href{https://artofproblemsolving.com/community/user/243907}{IstekOlympiadTeam}]
	Let $P(x,y,z)$ be the assertion $f(x+y^2+z)=f(f(x))+yf(y)+f(z)$
From $P(x,0,z) \to f(x+z)=f(f(x))+f(z)$ $\blacksquare$

From $P(x,0,0) \to f(0)=f(x)-f(f(x))$ $(1)$

From $P(x,0,-x) \to f(0)=f(f(x))+f(-x)$ $(2)$

From $(1)$ and $(2)$ $f(f(x))=f(x)$ using this in $\blacksquare$  $f(x+z)=f(x)+f(z)$   From $P(0,0,0) \to f(f(0))=0$ From $P(0,y,z) \to f(y^2+z)=yf(y)+f(z)=f(y^2)+f(z)$  which proves   $f(y^2)=yf(y)$. From $f(y^2)=yf(y)$  and  $f(x+y)=f(f(x))+f(y)$  it is easy find $f(x)=cx$   and plugging back we get  $c=1$
\end{solution}
*******************************************************************************
-------------------------------------------------------------------------------

\begin{problem}[Posted by \href{https://artofproblemsolving.com/community/user/92753}{WakeUp}]
	Does there exist a function $f:\mathbb{N}\rightarrow\mathbb{N}$ such that
\[f(f(n-1)=f(n+1)-f(n)\quad\text{for all}\ n\ge 2\text{?} \]
	\flushright \href{https://artofproblemsolving.com/community/c6h376900}{(Link to AoPS)}
\end{problem}



\begin{solution}[by \href{https://artofproblemsolving.com/community/user/29428}{pco}]
	\begin{tcolorbox}Does there exist a function $f:\mathbb{N}\rightarrow\mathbb{N}$ such that
\[f(f(n-1)=f(n+1)-f(n)\quad\text{for all}\ n\ge 2\text{?} \]\end{tcolorbox}
So $f(n+1)=f(2)+\sum_{k=1}^{n-1}f(f(k))$ $\forall n\ge 2$

As an immediate consequence : $f(n+1)\ge n$ $\forall n\ge 2$ and $f(n)\ge n-1$ $\forall n\ge 3$

So $f(k)\ge k-1\ge 3$ $\forall k\ge 4$ and so $f(f(k))\ge f(k)-1\ge k-2$ $\forall k\ge 4$

So $f(n+1)\ge  4+\sum_{k=4}^{n-1}(k-2)$   $\forall n\ge 5$ and so $f(7)\ge 9$ 

Consider then $n=f(7)-1\ge 8\ge 2$

$f(n+1)=f(f(7))=f(2)+\sum_{k=1}^{f(7)-2}f(f(k))$

But $f(7)-2\ge 7$ and so one of the elements in the sum $\sum_{k=1}^{f(7)-2}f(f(k))$ is $f(f(7))$ and we get :

$0=f(2)+\sum_{k=1,k\ne 7}^{f(7)-2}f(f(k))>0$ which is impossible.

So no such function

[color=#f00][mod edit: posted [url=https:\/\/artofproblemsolving.com\/community\/c6h381661]here[\/url] as well.][\/color]
\end{solution}
*******************************************************************************
-------------------------------------------------------------------------------

\begin{problem}[Posted by \href{https://artofproblemsolving.com/community/user/87034}{Mahan17}]
	Find all functions $f: \mathbb R \to \mathbb R$ such that for all reals $x$ and $y$,
\[ f(x+xy+f(y) )=\left(f(x)+\frac 12 \right)\left(f(y)+\frac 12 \right).\]
	\flushright \href{https://artofproblemsolving.com/community/c6h377568}{(Link to AoPS)}
\end{problem}



\begin{solution}[by \href{https://artofproblemsolving.com/community/user/5962}{yair}]
	at first take x=0 and y=inv f(0) to get f(0)=1\/2
then take x=0 and y= inv f(s) and find that f(s) = s +1\/2.
\end{solution}



\begin{solution}[by \href{https://artofproblemsolving.com/community/user/87034}{Mahan17}]
	does anyone know a correct formula for the answer
?
\end{solution}



\begin{solution}[by \href{https://artofproblemsolving.com/community/user/29428}{pco}]
	\begin{tcolorbox}at first take x=0 and y=inv f(0) to get f(0)=1\/2
then take x=0 and y= inv f(s) and find that f(s) = s +1\/2.\end{tcolorbox}
You cant use inverses without first showing that $f(x)$ is a surjection.
\end{solution}



\begin{solution}[by \href{https://artofproblemsolving.com/community/user/29428}{pco}]
	\begin{tcolorbox}find al functions like f:R->R such that for real numbers x,y: f(x+xy+f(y) )=(f(x)+1\/2)(f(y)+1\/2)\end{tcolorbox}
Let $P(x,y)$ be the assertion $f(x+xy+f(y))=(f(x)+\frac 12)(f(y)+\frac 12)$

It's first easy to see that the equation $x=(x+\frac 12)^2$ has no real solution and so there is no constant solution to the given functional equation.

$P(x,-1)$ $\implies$ $f(f(-1))=(f(x)+\frac 12)(f(-1)+\frac 12)$ and so $f(-1)=-\frac 12$ else $f(x)$ would be constant

In the inverse path : if $\exists u$ such that $f(u)=-\frac 12$ : $P(u,x)$ $\implies$ $f((u+1)x-\frac 12)=0$ and so $u=-1$, else $f(x)$ would be constant.

Let then $y_1\ne y_2$ :

$P(\frac{f(y_1)-f(y_2)}{y_2-y_1},y_1)$ $\implies$ $f(\frac{f(y_1)(y_2+1)-f(y_2)(y_1+1)}{y_2-y_1})$ $=(f(\frac{f(y_1)-f(y_2)}{y_2-y_1})+\frac 12)$ $(f(y_1)+\frac 12)$

$P(\frac{f(y_1)-f(y_2)}{y_2-y_1},y_2)$ $\implies$ $f(\frac{f(y_1)(y_2+1)-f(y_2)(y_1+1)}{y_2-y_1})$ $=(f(\frac{f(y_1)-f(y_2)}{y_2-y_1})+\frac 12)$ $(f(y_2)+\frac 12)$

And so $(f(\frac{f(y_1)-f(y_2)}{y_2-y_1})+\frac 12)$ $(f(y_1)+\frac 12)$ $=(f(\frac{f(y_1)-f(y_2)}{y_2-y_1})+\frac 12)$ $(f(y_2)+\frac 12)$ and so :

either $f(\frac{f(y_1)-f(y_2)}{y_2-y_1})=-\frac 12$ and so $\frac{f(y_1)-f(y_2)}{y_2-y_1}=-1$

either $f(y_1)=f(y_2)$

This may also be written : $\forall x\ne y$ :
either $f(x)-x=f(y)-y$
either $f(x)=f(y)$

Setting $y=-1$ in the above conclusion, we get 
$\forall x\ne -1$ :
either $f(x)-x=\frac 12$
either $f(x)=-\frac 12$ but this would imply $x=-1$

So : $\forall x\ne -1$ : $f(x)=x+\frac 12$
and since $f(-1)=-1+\frac 12$ too :

we get $\boxed{f(x)=x+\frac 12}$ $\forall x$ which indeed is a solution.
\end{solution}
*******************************************************************************
-------------------------------------------------------------------------------

\begin{problem}[Posted by \href{https://artofproblemsolving.com/community/user/3535}{hungvuong}]
	Find all functions $f: \mathbb{R} \rightarrow \mathbb{R}$ that satisfiy these conditions:
1) $f$ is continuous;
2) $f(Ax + By +C) = a f(x) + bf(y) + c$ for all real numbers $x$ and $y$,  and $abAB$ not zero.
	\flushright \href{https://artofproblemsolving.com/community/c6h377695}{(Link to AoPS)}
\end{problem}



\begin{solution}[by \href{https://artofproblemsolving.com/community/user/29428}{pco}]
	\begin{tcolorbox}Find all function f: \mathbb R \rightarrow \mathbb R that satisfies these conditions:
1) continuous
2) f (Ax + By +C) = a f(x) + bf(y) + c for all real numbers x and y and abAB not zero.\end{tcolorbox}
Let $P(x,y)$ be the assertion $f(Ax+By+C)=af(x)+bf(y)+c$

$P(x,0)$ $\implies$ $f(Ax+C)=af(x)+bf(0)+c$ and so $af(x)=f(Ax+C)-bf(0)-c$
$P(0,y)$ $\implies$ $f(By+C)=af(0)+bf(y)+c$ and so $bf(y)=f(By+C)-af(0)-c$

And so $P(x,y)$ becomes $f(Ax+By+C)=f(Ax+C)+f(By+C)-(a+b)f(0)-c$

Let then $g(x)=f(x+C)-(a+b)f(0)-c$. This equation becomes $g(Ax+By)=g(Ax)+g(By)$ and so ($A,B\ne 0$) $g(x+y)=g(x)+g(y)$

So, since continuous, $g(x)=ux$ and $f(x)=g(x-C)+(a+b)f(0)+c$ $=u(x-C)+(a+b)f(0)+c$ and so $f(x)=dx+e$ for some real $d,e$

Plugging this in original equation, we get :

$d(Ax+By+C)+e=adx+ae+bdy+be+c$ and so :
$dA=da$
$dB=db$
$dC+e=ae+be+c$

And so :
either $d=0$ and $e(a+b-1)=-c$
either $A=a$ and $B=b$ and $e(a+b-1)=dC-c$

And so the different cases :

1) If $A\ne a$ or $B\ne b$ and $a+b\ne 1$, the only solution is $f(x)=-\frac c{a+b-1}$ $\forall x$

2) If $A\ne a$ or $B\ne b$ and $a+b=1$ and $c=0$, we get the solution $f(x)=e$ for any real $e$

3) If $A\ne a$ or $B\ne b$ and $a+b=1$ and $c\ne 0$ : there are no solutions

4) If $A=a$ and $B=b$ and $a+b\ne 1$, then solutions : $f(x)=dx+\frac{dC-c}{a+b-1}$ for any real $d$

5) If $A=a$ and $B=b$ and $a+b=1$ and $C=0$ and $c=0$ , then solutions : $f(x)=dx+e$, for any real $d,e$

6) If $A=a$ and $B=b$ and $a+b=1$ and $C=0$ and $c\ne 0$ , then no solution

7) If $A=a$ and $B=b$ and $a+b=1$ and $C\ne 0$, then solutions $f(x)=\frac cCx+e$, for any real $e$
\end{solution}
*******************************************************************************
-------------------------------------------------------------------------------

\begin{problem}[Posted by \href{https://artofproblemsolving.com/community/user/67223}{Amir Hossein}]
	Find all functions $f: [0, +\infty) \to [0, +\infty)$ satisfying the equation
\[(y+1)f(x+y) = f\left(xf(y)\right)\]
For all non-negative real numbers $x$ and $y.$
	\flushright \href{https://artofproblemsolving.com/community/c6h377951}{(Link to AoPS)}
\end{problem}



\begin{solution}[by \href{https://artofproblemsolving.com/community/user/29428}{pco}]
	\begin{tcolorbox}Find all functions $f: [0, +\infty) \to [0, +\infty)$ satisfying the equation
\[(y+1)f(x+y) = f\left(xf(y)\right)\]
For all non-negative real numbers $x$ and $y.$\end{tcolorbox}
Setting $x=0$ in this equality, we get $(y+1)f(y)=f(0)$ and so $f(x)=\frac a{x+1}$

Plugging this in original equation, we get $a=0$ or $a=1$

Hence the answer :
$f(x)=0$ $\forall x\ge 0$

$f(x)=\frac 1{x+1}$ $\forall x\ge 0$
\end{solution}



\begin{solution}[by \href{https://artofproblemsolving.com/community/user/191127}{sayantanchakraborty}]
	I think this Slovenia paper was too easy.....isn't it?
\begin{bolded}I want to pose a problem\end{bolded}\end{underlined}

Find all functions $f:\mathbb{R}\rightarrow \mathbb{R}$ such that

$f(x+y)=f(x)f(y)f(xy)$   for all $x,y$ in $\mathbb{R}$





Maths is the doctor of science....


Sayantan...
\end{solution}



\begin{solution}[by \href{https://artofproblemsolving.com/community/user/29428}{pco}]
	\begin{tcolorbox}...
\begin{bolded}I want to pose a problem\end{bolded}\end{underlined}
...\end{tcolorbox}
Post as a new thread, not as a second new problem in an old thread.
\end{solution}



\begin{solution}[by \href{https://artofproblemsolving.com/community/user/187896}{Ashutoshmaths}]
	Why are you posting all the Indian MO problems which are already discussed?
http://www.artofproblemsolving.com/Forum/viewtopic.php?p=344412&sid=76a84a7b75c6d8d3bfe2d332309154f3#p344412
\end{solution}
*******************************************************************************
-------------------------------------------------------------------------------

\begin{problem}[Posted by \href{https://artofproblemsolving.com/community/user/70108}{EhsanNamdari}]
	Find all functions $f: \mathbb R \to \mathbb R$ such that $f(f(0))=f(0)$ and for all reals $x$, $y$, and $z$,
\[f\left(\frac{x+f(x)}{2}+y+f(2z)\right)=2x-f(x)+f(f(f(y)))+2f(f(z)).\]
	\flushright \href{https://artofproblemsolving.com/community/c6h378164}{(Link to AoPS)}
\end{problem}



\begin{solution}[by \href{https://artofproblemsolving.com/community/user/29428}{pco}]
	\begin{tcolorbox}Find all functions $f:R\to R$ such that:
$1)f(\frac{x+f(x)}{2}+y+f(2z))=2x-f(x)+f(f(f(y)))+2f(f(z))$

$2)f(f(0))=f(0)$\end{tcolorbox}
Let $P(x,y,z)$ be the assertion $f(\frac{x+f(x)}2+y+f(2z))=2x-f(x)+f(f(f(y)))+2f(f(z))$

1) $f(0)=0$
===========
[hide="rather simple proof"]
=============================== begin ==============================
$P(0,y,0)$ $\implies$ $f(y+\frac{3f(0)}2)=f(f(f(y)))+f(0)$ (using $f(f(0))=f(0)$)

$P(f(0),y,0)$ $\implies$ $f(y+2f(0))=f(f(f(y)))+3f(0)$

And, comparing these two lines and setting $x=y+\frac{3f(0)}2$ : $f(x+\frac{f(0)}2)=f(x)+2f(0)$ $\forall x$
And so $f(x+n\frac{f(0)}2)=f(x)+2nf(0)$ $\forall x,\forall n\in\mathbb Z$

From this, we get $f(2f(0))=f(0+4\frac{f(0)}2)=f(0)+8f(0)=9f(0)$

But $P(f(0),0,0)$ $\implies$ $f(2f(0))=4f(0)$ and so $9f(0)=4f(0)$ and $f(0)=0$
Q.E.D. 
=============================== end ==============================
[\/hide]

2) $f(f(x))=f(x)$ $\forall x$
===========================
$P(f(f(x)),0,0)$ $\implies$ $f(\frac{f(f(x))+f(f(f(x)))}2)=2f(f(x))-f(f(f(x)))$

$P(0,x,0)$ $\implies$ $f(x)=f(f(f(x)))$ and the previous line becomes : $f(\frac{f(f(x))+f(x)}2)=2f(f(x))-f(x)$

But $P(f(x),0,0)$ $\implies$ $f(\frac{f(x)+f(f(x))}2)=2f(x)-f(f(x))$

And so $2f(f(x))-f(x)=2f(x)-f(f(x))$ and so $f(f(x))=f(x)$ $\forall x$
Q.E.D.

3) $f(x)=x$ $\forall x$
======================
$P(0,0,x)$ $\implies$ $f(f(2x))=2f(f(x))$ and so $f(2x)=2f(x)$

Then $\frac{x+f(x)}2=\frac x2+f(\frac x2)$

And $P(x,0,0)$ $\implies$ $f(\frac x2+f(\frac x2))=2x-f(x)$

But $P(f(\frac x2),\frac x2,0)$ $\implies$ $f(f(\frac x2)+\frac x2)=2f(\frac x2)=f(x)$

And so $2x-f(x)=f(x)$ and $f(x)=x$, which indeed is a solution.

Hence the unique solution : $\boxed{f(x)=x}$
\end{solution}



\begin{solution}[by \href{https://artofproblemsolving.com/community/user/70108}{EhsanNamdari}]
	\begin{tcolorbox}Find all functions $f:R\to R$ such that:
$1)f(\frac{x+f(x)}{2}+y+f(2z))=2x-f(x)+f(f(f(y)))+2f(f(z))$

$2)f(f(0))=f(0)$\end{tcolorbox}
I forgot to say that this problem was proposed by Mohammad Jafari!
\end{solution}



\begin{solution}[by \href{https://artofproblemsolving.com/community/user/334227}{reveryu}]
	another way to prove @pco's third step.

$P(0,0,z) \implies f(f(2z))=2f(f(z)) $
plugging in $P(x,y,z)$ and changes $z$ to $\frac{z}{2}$ and use $f(f(x))=f(x)$ give us new assertion,  $Q(x,y,z) :  f(\frac{x+f(x)}{2}+y+f(z))=2x-f(x)+f(y)+f(z)$

3.1) $f(x)+f(y)$ is onto
proof: $Q(\frac{-f(y)-f(z)+x}{2},y,z)$ and then the result follows.
  
3.2) $f(x)=x \forall x\in R$
proof: $Q(0,f(x),y) \implies f(f(x)+f(y))=f(x)+f(y)$
from $f(x)+f(y)$ is onto then we're done.

\end{solution}
*******************************************************************************
-------------------------------------------------------------------------------

\begin{problem}[Posted by \href{https://artofproblemsolving.com/community/user/70108}{EhsanNamdari}]
	Find all functions $f: \mathbb Q \to \mathbb Q$ such that for all rationals $x$ and $y$,
\[f(x+f(x)+2y)=2x+2f(f(y)).\]
	\flushright \href{https://artofproblemsolving.com/community/c6h378365}{(Link to AoPS)}
\end{problem}



\begin{solution}[by \href{https://artofproblemsolving.com/community/user/29428}{pco}]
	\begin{tcolorbox}Find all functions $f:Q \to Q$ such that:
$f(x+f(x)+2y)=2x+2f(f(y))$\end{tcolorbox}
Let $P(x,y)$ be the assertion $f(x+f(x)+2y)=2x+2f(f(y))$

$P(0,0)$ $\implies$ $f(f(0))=0$
If $f(u)=0$ for some $u$, then $P(u,0)$ $\implies$ $u=0$ ans so, since $f(f(0))=0$, we get $f(0)=0$

$P(0,x)$ $\implies$ $f(2x)=2f(f(x))$ ans so $P(x,y)$ may be written $f(x+f(x)+2y)=2x+f(2y)$
and so new assertion $Q(x,y)$ : $f(x+y+f(x))=2x+f(y)$

$Q(x,0)$ $\implies$ $f(x+f(x))=2x$
$Q(x,y+f(y))$ $\implies$ $f(x+f(x)+y+f(y))=2x+2y$
Setting $y=-x$ in the above equality, xe get $f(f(x)+f(-x))=0$ and so $f(-x)=-f(x)$

$Q(x,-x)$ $\implies$ $f(f(x))=-f(x)+2x$ and so $f(f(x))+f(x)=2x$ and the function $f(x)+x$ is surjective.

But $Q(x,y)$ may also be written $f((x+f(x))+y)=f(x+f(x))+f(y)$ and so, since $f(x)+x$ is surjective : $f(x+y)=f(x)+f(y)$

And so, since $f(x)$ is from $\mathbb Q\to\mathbb Q$ : $f(x)=xf(1)$ and, plugging this in original equation : $f(1)=1$

Hence the unique solution : $\boxed{f(x)=x}$
\end{solution}



\begin{solution}[by \href{https://artofproblemsolving.com/community/user/70108}{EhsanNamdari}]
	\begin{tcolorbox}Find all functions $f:Q \to Q$ such that:
$f(x+f(x)+2y)=2x+2f(f(y))$\end{tcolorbox}
I forgot to say that this problem was proposed by Mohammad Jafari!
\end{solution}



\begin{solution}[by \href{https://artofproblemsolving.com/community/user/121558}{Bigwood}]
	It is obvious that $f$ is surjective, and $P(0,0)$ implies $f(f(0))=0$, $f(x+f(x))=2x$ from $P(x,0)$.
$P(x+f(x),0)\Rightarrow f(3x+f(x))=2x+2f(x)$
$P(x,x)\Rightarrow f(3x+f(x))=2x+2f(f(x))$ then $f(f(x))=f(x)$. Since $f$ is surjective, $\boxed{f(x)=x\ (\forall x)}$

We can solve this in $\mathbb{R}$, I wanna see the author's solution :D
\end{solution}
*******************************************************************************
-------------------------------------------------------------------------------

\begin{problem}[Posted by \href{https://artofproblemsolving.com/community/user/92753}{WakeUp}]
	Let $\mathbb{R}$ denote the set of real numbers. Find all functions $f:\mathbb{R}\rightarrow\mathbb{R}$ such that
\[f(x^2)+f(xy)=f(x)f(y)+yf(x)+xf(x+y)\]
for all $x,y\in\mathbb{R}$.
	\flushright \href{https://artofproblemsolving.com/community/c6h378553}{(Link to AoPS)}
\end{problem}



\begin{solution}[by \href{https://artofproblemsolving.com/community/user/29428}{pco}]
	\begin{tcolorbox}Let $R$ denote the set of real numbers. Find all functions $f:\mathbb{R}\rightarrow\mathbb{R}$ such that
\[f(x^2)+f(xy)=f(x)f(y)+yf(x)+xf(x+y)\]
for all $x,y\in\mathbb{R}$.\end{tcolorbox}
Let $P(x,y)$ be the assertion $f(x^2)+f(xy)=f(x)f(y)+yf(x)+xf(x+y)$

$P(0,x)$ $\implies$ $f(0)(f(x)+x-2)$

If $f(0)\ne 0$, this implies $f(x)=2-x$ which indeed is a solution.

Let us from know consider that $f(0)=0$

$P(x,0)$ $\implies$ $f(x^2)=xf(x)$
Then : $P(x,y)$ $\implies$ $xf(x)+f(xy)=f(x)f(y)+yf(x)+xf(x+y)$
Same : $P(y,x)$ $\implies$ $yf(y)+f(xy)=f(x)f((y)+xf(y)+yf(x+y)$
Subtracting implies $(x-y)f(x+y)-f(x)-f(y))=0$

and so $f(x+y)=f(x)+f(y)$ $\forall x\ne y$

Plugging this in $xf(x)+f(xy)=f(x)f(y)+yf(x)+xf(x+y)$, we get $f(xy)=f(x)f(y)+yf(x)+xf(y)$ $\forall x\ne y$

$\iff$ $f(xy)+xy=(f(x)+x)(f(y)+y)$

Let then $g(x)=f(x)+x$. We got :
$g(0)=0$
$g(x+y)=g(x)+g(y)$ $\forall x\ne y$
$g(xy)=g(x)g(y)$ $\forall x\ne y$

From the first, we get $g(-x)=-g(x)$ and so $g(2x+(-x))=g(2x)+g(-x)$ and so $g(2x)=2g(x)$ and so $g(x+y)=g(x)+g(y)$ $\forall x,y$
From the second, we get $g(x(x+1))=g(x)g(x+1)=g(x)^2+g(x)$
But also $g(x(x+1))=g(x^2+x)=g(x^2)+g(x)$
And so $g(x^2)=g(x)^2$ and so $g(xy)=g(x)g(y)$ $\forall x,y$

So :
$g(x+y)=g(x)+g(y)$ $\forall x,y$
$g(xy)=g(x)g(y)$ $\forall x,y$
And so, very classical, $g(x)=x$ and $f(x)=0$

Hence the two solutions :
$f(x)=2-x$
$f(x)=0$
\end{solution}



\begin{solution}[by \href{https://artofproblemsolving.com/community/user/92753}{WakeUp}]
	\begin{tcolorbox}
From the second, we get $g(x(x+1))=g(x)g(x+1)=g(x)^2+g(x)$
But also $g(x(x+1))=g(x^2+x)=g(x^2)+g(x)$
And so $g(x^2)=g(x)^2$ and so $g(xy)=g(x)g(y)$ $\forall x,y$
\end{tcolorbox}

Hi pco, could you please explain this part of the solution? Note also $f(x)=-x$ is a solution.
\end{solution}



\begin{solution}[by \href{https://artofproblemsolving.com/community/user/82091}{Solving}]
	SO im right?

x=y
2f(x^2)=f(x)^2+x(f(x)+f(2x))
2f(0)=f(0)^2
f(0)=0
or
f(0)=1\/2
let x=x, y=0
f(x^2)+f(0)=f(x)f(0)+xf(x)
f(0)=1\/2
f(x^2)+1\/2=f(x)(1\/2+x)
f(x)=ax+b
ax^2+b+1\/2=(ax\/2+ax^2+b\/2+bx)
b\/2=b+1\/2
b=-1
a=2
f(x)=2x-1 is the solution
for
f(0)=0
f(x)=x
\end{solution}



\begin{solution}[by \href{https://artofproblemsolving.com/community/user/29428}{pco}]
	\begin{tcolorbox}[quote="pco"]
From the second, we get $g(x(x+1))=g(x)g(x+1)=g(x)^2+g(x)$
But also $g(x(x+1))=g(x^2+x)=g(x^2)+g(x)$
And so $g(x^2)=g(x)^2$ and so $g(xy)=g(x)g(y)$ $\forall x,y$
\end{tcolorbox}

Hi pco, could you please explain this part of the solution? \end{tcolorbox}
Yes, :oops:, I wrote too quickly !
First we can see that $g(x)=0$ is a solution (and so $f(x)=-x$ is indeed !
If $g(x)$ is not the all zero function, let then $u$ such that $g(u)\ne 0$. If $u=1$, choose instead $u=-1$.
Then the second equation gives us $g(u)(g(1)-1)=0$ and so $g(1)=1$ 

Then $g(x(x+1))=g(x)g(x+1)$ (using second equation since $x\ne x+1$) $=g(x)(g(x)+g(1))=g(x)^2+g(x)$
But $g(x(x+1))=g(x^2+x)=g(x^2)+g(x)$
And so, $g(x^2)=g(x)^2$ and so the second equation $g(xy)=g(x)g(y)$ is also true if $x=y$
...


\begin{tcolorbox}Note also $f(x)=-x$ is a solution.\end{tcolorbox}
Yes, :oops: $g(x)=0$ is also a solution (I forgot it) 

And so :
$f(x)=0$
$f(x)=-x$
$f(x)=2-x$
\end{solution}



\begin{solution}[by \href{https://artofproblemsolving.com/community/user/152203}{borntobeweild}]
	This is just about as interesting as a FE can get while still dying to the standard strategies of plugging stuff in, taking cases, and testing. Nevertheless, it was a fun problem.

[hide="Reading this won't teach you anything except for what tricks you should have tried"]Let $P$ be the given assertion. Then we get:
$P(0,0): 2f(0)=f(0)^2\implies f(0)=0\,\, \text{or}\, f(0)=2$.

We then take cases:

Case 1 (easier): $f(0)=2$:

We have:
$P(0,x): 4=2f(x)+2x\implies f(x)=2-x$

Case 2 (trickier): $f(0)=0$:

We find that:
$P(x, 0): f(x^2)=xf(x)$

and furthermore:
$P(x, -x): f(x^2)+f(-x^2)=f(x)f(-x)-xf(x)$
$P(-x, x): f(x^2)+f(-x^2)=f(x)f(-x)+xf(-x)$

Using the above two equations, we get that $f(-x)=-f(x)$ (as $f(0)=0$ is known).

Then the first of those two equations becomes:
$f(x)^2=-xf(x)$

Therefore $\forall x$, either $f(x)=-x$ or $f(x)=0$.
Let $f(a)=0, f(b)=-b$. Then as $f(x^2)=xf(x)$:
$P(a, b): f(ab)=af(a+b)$
$P(b, a): -b^2+f(ab)=-ab+bf(a+b)$

Subtracting these gives:
$b^2=ab+af(a+b)-bf(a+b)$
$-b(a-b)=(a-b)f(a+b)$

If $a=b$, then $-b=f(b)=f(a)=0$, so $a=b=0$.
Else, $f(a+b)=-b$. Then if $f(a+b)=0$, we get $b=0$, while if $f(a+b)=-a-b$, then $a=0$.

Therefore, $a$ and $b$ cannot both be nonzero, so the function must be $f(x)=0 \,\, \forall x$, or $f(x)=-x \,\,\forall x$

Summarizing over both the cases, we get the solution set:

$f(x)=2-x$
$f(x)=-x$
$f(x)=0$[\/hide]
\end{solution}



\begin{solution}[by \href{https://artofproblemsolving.com/community/user/204311}{Onlygodcanjudgeme}]
	put $ (x,y) = (0,0) $  then we take that 1) $ f(0) =0 $  2)$ f(0) =2 $
1) put  $ (x,y) = (x,0) $ then we take that $ f(x^2) = x \cdot f(x) $ and this is odd function .
put  $ (x,y) = (x,-x) $ then we take that  f(-x) = x , f(x) =0
2)put  $ (x,y) = (0,x) $ then we take that f(x) = 2-x 
so answer is f(x)=0 , f(x) = -x ,f(x) = 2-x
\end{solution}
*******************************************************************************
-------------------------------------------------------------------------------

\begin{problem}[Posted by \href{https://artofproblemsolving.com/community/user/70108}{EhsanNamdari}]
	Find all functions $f: \mathbb R \to \mathbb R$ such that for all reals $x$ and $y$,
\[f(f(x)-y^{2})=f(x)^{2}-2f(x)y^{2}+f(f(y)).\]
	\flushright \href{https://artofproblemsolving.com/community/c6h378717}{(Link to AoPS)}
\end{problem}



\begin{solution}[by \href{https://artofproblemsolving.com/community/user/29428}{pco}]
	\begin{tcolorbox}Find all functions $f:R \to R$ such that:
$f(f(x)-y^{2})=f(x)^{2}-2f(x)y^{2}+f(f(y))$\end{tcolorbox}
$f(x)=0$ $\forall x$ is a solution and let us from now look for non all zero solutions.

Let $P(x,y)$ be the assertion $f(f(x)-y^2)=f(x)^2-2f(x)y^2+f(f(y))$
Let $A=f(\mathbb R)$

$P(0,0)$ $\implies$ $f(f(0))=f(0)^2+f(f(0))$ and so $f(0)=0$

$P(y,0)$ $\implies$ $f(f(y))=f(y)^2$ and so $f(x)=x^2$ $\forall x\in A$

So $P(x,y)$ may be written $f(f(x)-y^2)=f(x)^2-2f(x)y^2+f(y)^2$

Let $y=f(z)\in A$ : $f(y)=y^2$ and so the above equality may be written : $f(f(x)-f(y))=f(x)^2-2f(x)f(y)+f(y)^2$

And so $f(f(x)-f(y))=(f(x)-f(y))^2$ $\forall x\in \mathbb R,y\in A$

Let then $u$ such that $f(u)\ne 0$ : $f(f(u))=f(u)^2$ and so $f(v)>0$ where $v=f(u)$

Let then $t\le 0$ and $s$ such that $s^2=\frac{f(v)^2-t}{2f(v)}$ 

$P(v,s)$ $\implies$ $f(f(v)-s^2)-f(f(s))=t$ and so any non positive real $t$ may be written as $t=f(x)-f(y)$ with $x\in\mathbb R$ and $y\in A$

And since we previously got that $f(f(x)-f(y))=(f(x)-f(y))^2$ $\forall x\in \mathbb R,y\in A$, we now have :

$f(x)=x^2$ $\forall x\le 0$

But then $x\in A$ $\forall x\ge 0$ since $x=f(-\sqrt x)$ and since we know that $f(x)=x^2$ $\forall x\in A$, we get $f(x)=x^2$ $\forall x\ge 0$

And so $f(x)=x^2$ $\forall x$ which indeed is a solution.

Hence the solutions : $\boxed{f(x)=0}$ and $\boxed{f(x)=x^2}$
\end{solution}



\begin{solution}[by \href{https://artofproblemsolving.com/community/user/70108}{EhsanNamdari}]
	\begin{tcolorbox}Find all functions $f:R \to R$ such that:
$f(f(x)-y^{2})=f(x)^{2}-2f(x)y^{2}+f(f(y))$\end{tcolorbox}
I forgot to say that this problem was proposed by Mohammad Jafari!
\end{solution}
*******************************************************************************
-------------------------------------------------------------------------------

\begin{problem}[Posted by \href{https://artofproblemsolving.com/community/user/70108}{EhsanNamdari}]
	Find all functions $f: \mathbb R \to \mathbb R$ such that for all reals $x$ and $y$,
\[f(x-f(y))=f(f(y))-2xf(y)+f(x).\]
	\flushright \href{https://artofproblemsolving.com/community/c6h378718}{(Link to AoPS)}
\end{problem}



\begin{solution}[by \href{https://artofproblemsolving.com/community/user/29428}{pco}]
	\begin{tcolorbox}Find all functions $f:R\to R$ such that:
$f(x-f(y))=f(f(y))-2xf(y)+f(x)$\end{tcolorbox}
Already posted .
Dont hesitate to :
- search before posting
- use intelligent title

$f(x)=0$ $\forall x$ is a solution. So let us from now look for non all-zero solutions.

Let $P(x,y)$ be the assertion $f(x-f(y))=f(f(y))-2xf(y)+f(x)$
Let $f(0)=2a$
Let $u$ such that $f(u)\ne 0$

$P(f(x),x)$ $\implies$ $f(f(x))=f^2(x)+a$

$P(f(x),y)$ $\implies$ $f(f(x)-f(y))=(f(x)-f(y))^2+2a$

$P(\frac{f(f(u))-x}{2f(u)},u)$ $\implies$ $f(\frac{f(f(u))-x}{2f(u)}-f(u))-f(\frac{f(f(u))-x}{2f(u)})=x$ and so any real $x$ may be written as $f(t)-f(t')$ for some real $t,t'$

And since we already got $f(f(t)-f(t'))=(f(t)-f(t'))^2+2a$ $\forall t,t'$, we can conclude $f(x)=x^2+2a$ $\forall x$

Plugging this in original equation, we get $a=0$ and so the solutions :

$\boxed{f(x)=x^2}$ and $\boxed{f(x)=0}$
\end{solution}



\begin{solution}[by \href{https://artofproblemsolving.com/community/user/70108}{EhsanNamdari}]
	\begin{tcolorbox}Find all functions $f:R\to R$ such that:
$f(x-f(y))=f(f(y))-2xf(y)+f(x)$\end{tcolorbox}
I forgot to say that this problem was proposed by Mohammad Jafari!
\end{solution}
*******************************************************************************
-------------------------------------------------------------------------------

\begin{problem}[Posted by \href{https://artofproblemsolving.com/community/user/74529}{bigbang195}]
	Find all functions $f : \mathbb R \to \mathbb R$ such that 
\[f(x+y)+f(x)f(y)=f(xy)+f(x)+f(y), \quad \forall x,y \in \mathbb R.\]
	\flushright \href{https://artofproblemsolving.com/community/c6h379512}{(Link to AoPS)}
\end{problem}



\begin{solution}[by \href{https://artofproblemsolving.com/community/user/94087}{siddhubhai007}]
	e ka ans. f(x)=x hai babua
\end{solution}



\begin{solution}[by \href{https://artofproblemsolving.com/community/user/89495}{Olympic}]
	I found two functions. Making $ x=y=0 $ : $ f(0) + [f(0)]^2 = f(0.0) + f(0) + f(0) $ so
$ [f(0)]^2 - 2f(0) = 0 $ __ $ f(0) = 0  or  f(0) = 2 $. 
If $ f(0) = 0 $: $ f(x) = x $ satisfy the problem. (Sorry! :(  I cannot prove that, but see - this is truth!)
For the case $ f(0) = 2 $, making $ y=0 $: $ f(x) + f(x)f(0) = f(0) + f(x) + f(0) $ __ $ 2f(x) = 4 $ __ $ f(x) = 2 (const.) (OK!) $
\end{solution}



\begin{solution}[by \href{https://artofproblemsolving.com/community/user/29428}{pco}]
	\begin{tcolorbox}Find  f : $R \to R $such that 

\[f(x+y)+f(x)f(y)=f(xy)+f(x)+f(y), \forall x,y \in R\]\end{tcolorbox}
Let $P(x,y)$ be the assertion $f(x+y)+f(x)f(y)=f(xy)+f(x)+f(y)$
Let $f(1)=b$

$P(x,1)$ $\implies$ $f(x+1)=(2-b)f(x)+b$
$P(x,y+1)$ $\implies$ $(2-b)f(x+y)+(2-b)f(x)f(y)+bf(x)=f(xy+x)+f(x)+(2-b)f(y)$

Subtracting $(2-b)P(x,y)$ from this equation, we get :
$f(xy+x)=(2-b)f(xy)+f(x)$ and so $f(x+y)=(2-b)f(x)+f(y)$

If $b=1$, this implies $f(x+y)=f(x)+f(y)$
Plugging this in original equation, we get $f(xy)=f(x)f(y)$

And the system $f(x+y)=f(x)+f(y)$ and $f(x)f(y)$ and $f(1)=1$ is very classical and has the unique solution $f(x)=x$ which indeed is a solution of our equation.

If $b\ne 1$, this implies $(2-b)f(x)+f(y)=(2-b)f(y)+f(x)$ and so $(1-b)f(x)=(1-b)f(y)$ and so $f(x)=f(y)$ and so $f(x)$ is constant and it's immediate to see that the only constant solutions are $f(x)=0$ and $f(x)=2$

Hence the three solutions :
$f(x)=0$ $\forall x$
$f(x)=2$ $\forall x$
$f(x)=x$ $\forall x$
\end{solution}



\begin{solution}[by \href{https://artofproblemsolving.com/community/user/74529}{bigbang195}]
	\begin{bolded}Thank you very much,pco                                                                                                                                                        
\end{bolded}
\end{solution}
*******************************************************************************
-------------------------------------------------------------------------------

\begin{problem}[Posted by \href{https://artofproblemsolving.com/community/user/91306}{ndk09}]
	Find all continuous functions $f: \mathbb R \to \mathbb R$ such that for all reals $x$ and $y$,
\[f(f(xy))=f(x)f(y).\]
	\flushright \href{https://artofproblemsolving.com/community/c6h379603}{(Link to AoPS)}
\end{problem}



\begin{solution}[by \href{https://artofproblemsolving.com/community/user/29428}{pco}]
	\begin{tcolorbox}Find all continous function $f:R \longrightarrow R$ such that:
\[ f(f(xy))=f(x)f(y)\]\end{tcolorbox}
Let $P(x,y)$ be the assertion $f(f(xy))=f(x)f(y)$
Let $f(0)=a$
Let $f(1)=b$

$P(x,0)$ $\implies$ $f(a)=af(x)$ and so either $a=0$, either $f(x)$ is constant and so is all zero or all one.
Let us from now consider that $f(x)$ is not constant and $a=0$

Comparing $P(x,y)$ and $P(xy,1)$, we get $bf(xy)=f(x)f(y)$ and so $b\ne 0$ else we would have $f(x)=0$ constant.

So we get $f(xy)=\frac 1bf(x)f(y)$ and so $g(xy)=g(x)g(y)$ where $g(x)=\frac 1bf(x)$ with $g(1)=1$

And the general solution of $g(xy)=g(x)g(y)$ with $g(x)$ continuous and $g(1)=1$ is :
either $g(x)=1$ $\forall x$ which gives $f(x)=b$ constant already studied

either $g(x)=|x|^c$ $\forall x$ and for some $c>0$ which gives $f(x)=b|x|^c$
and, plugging in original equation, the only non constant solution is $f(x)=b|x|$ for any $b>0$

either $g(x)=\text{sign}(x)|x|^c$ $\forall x$ and for some $c>0$
and, plugging in original equation, the only non constant solution is $f(x)=b\text{sign}(x)|x|=bx$ for any $b$

\begin{bolded}Hence the four solutions \end{bolded}\end{underlined}:
$f(x)=0$ $\forall x$
$f(x)=1$ $\forall x$
$f(x)=a|x|$ $\forall x$ for any $a>0$
$f(x)=ax$ $\forall x$ for any $a$
\end{solution}
*******************************************************************************
-------------------------------------------------------------------------------

\begin{problem}[Posted by \href{https://artofproblemsolving.com/community/user/91306}{ndk09}]
	Find all functions $f: \mathbb R \to \mathbb R$ such that for all reals $x$ and $y$,
\[f(x+y)(f(f(x))-y)=xf(x)-yf(y).\]
	\flushright \href{https://artofproblemsolving.com/community/c6h379993}{(Link to AoPS)}
\end{problem}



\begin{solution}[by \href{https://artofproblemsolving.com/community/user/29428}{pco}]
	\begin{tcolorbox}  Sorry, I have a mistake  :)
Can find all function $f:R\longrightarrow R$ such that :
\[f(x+y)(f(f(x))-y)=xf(x)-yf(y)\] ?\end{tcolorbox}
Let $P(x,y)$ be the assertion $f(x+y)(f(f(x))-y)=xf(x)-yf(y)$
Let $U=\{x$ such that $f(x)=0\}$

1) Preliminary results
============
1.1) $\forall u\in U$ $\forall v\in U\setminus\{f(0)\}$ : $u+v\in U$
-------------------------------------------------------
Let $u\in U$ and $v\in U\setminus\{f(0)\}$ : $P(u,v)$ $\implies$ $f(u+v)(f(0)-v)=0$ and so $u+v\in U$
Q.E.D.

1.2) $\forall u,v\in U$ such that $u\ne v$ : $u-v\in U$ and $v-u\in U$
---------------------------------------------------------------------------
$P(u,v-u)$ $\implies$ $f(v)(f(f(u))-v+u)=uf(u)-(v-u)f(v-u)$ and so $(v-u)f(v-u)=0$ and so $f(v-u)=0$
Q.E.D.

1.3) $f(0)\in U$
---------------
$P(0,0)$ $\implies$ $f(0)f(f(0))=0$ and so :
either $f(0)=0$ and so $f(f(0))=f(0)=0$ and so $f(0)\in U$
either $f(0)\ne 0$ and so $f(f(0))=0$ and so $f(0)\in U$
Q.E.D.

1.4) $\forall x\notin U$ $f(f(x))=x$
---------------------------------------
Let $x\notin U$ : $f(x)\ne 0$
$P(x,0)$ $\implies$ $f(x)f(f(x))=xf(x)$ and so $f(f(x))=x$
Q.E.D.

1.5) $\forall x\notin U$, $\forall y\in U\setminus\{f(0)\}$ : $x-y\in U$
-----------------------------------------------------------
Let $x\notin U$
Let $y\in U\setminus\{f(0)\}$
$P(y,x-y)$ $\implies$ $f(x)(f(f(y))-x+y)=yf(y)-(x-y)f(x-y)$
$P(x-y,y)$ $\implies$ $f(x)(f(f(x-y))-y)=(x-y)f(x-y)-yf(y)$
Adding these two lines : $f(x)(f(f(y))-x+f(f(x-y)))=0$
And so, since $f(x)\ne 0$ and $f(y)=0$ : $f(f(x-y))=x-f(0)$
So $f(f(x-y))\ne x-y$ and so, using 1.4, $x-y\in U$
Q.E.D

2) If $U\ne\{f(0)\}$ then $f(x)=0$ $\forall x$
============================
2.1) $f(0)=0$
------------
Let $u,v\in U$ such that $u\ne v$
Using 1.2) we know that $u-v\in U$ and $v-u\in U$
Since $u\ne v$, we get that $u-v\ne v-u$ and so at least one of these two numbers is different from $f(0)$
WLOG say $u-v\ne f(0)$
Using then 1.1 with $v-u$ and $u-v$, we get $(v-u)+(u-v)\in U$ and so $0\in U$
Q.E.D.

2.2) $\forall u,v\in U$ : $u+v\in U$ and $u-v\in U$
-----------------------------------------------
Just use 1.1 and 1.2 considering $f(0)=0$
Q.E.D.

2.3) $U=\mathbb R$ and so $f(x)=0$ $\forall x$
-----------------------------------------
Suppose $\exists a\notin U$
Let $u\ne 0\in U$ (such $u$ exists since, in this paragraph 2, we supposed ${U\ne\{f(0)}$ and we know that $f(0)=0$
Using 1.5 with $x=a$ and $y=u$, we get $a-u\in U$
But, $u\in U$ and $a-u\in U$ implies, using 2.2 : $a\in U$,a nd so contradiction
So such $a$ does not exist
so $U=\mathbb R$
so $f(x)=0$ $\forall x$
Q.E.D.

3) If $U=\{f(0)\}$ then $f(x)=x$ $\forall x$ or $f(x)=f(0)-x$ $\forall x$
============================================
3.1) $f(f(x))=x$ $\forall x$
---------------------------
Using 1.4, we get $f(f(x))=x$ $\forall x\ne f(0)$
We also have $f(f(0))=0$ and so $f(f(f(0)))=f(0)$ and so $f(f(x))=x$ for $x=f(0)$
Q.E.D.

3.2) $\forall x$ : either $f(x)=x$, either $f(x)=f(0)-x$
-------------------------------------------------
$P(x,f(x))$ $\implies$ $f(x+f(x))(f(f(x))-f(x))=xf(x)-f(x)f(f(x))$ and so $f(x+f(x))(x-f(x))=0$ and so :
either $f(x)=x$
either $f(x+f(x))=0$ and so $x+f(x)\in U$ and so $x+f(x)=f(0)$ and so $f(x)=f(0)-x$
Q.E.D.

3.3) either $f(x)=x$ $\forall x$, either $f(x)=f(0)-x$ $\forall x$
-------------------------------------------------------------
Suppose :
$\exists x$ such that $f(x)=x$ and $f(x)\ne f(0)-x$
$\exists y$ such that $f(y)=f(0)-y$ and $f(y)\ne y$

$P(x,y)$ $\implies$ $f(x+y)(x-y)=x^2+y^2-yf(0)$

If $f(x+y)=f(0)-x-y$, then this becomes $(f(0)-x-y)(x-y)=x^2+y^2-yf(0)$ and so $2x^2=xf(0)$ and so :
- either $x=0$ but this is impossible since then we would have $f(x)=x$ and $f(x)=f(0)-x$
- either $x=\frac{f(0)}2$ but this is impossible since then we would have $f(x)=f(0)-x$ (since $f(x)=x)$
If $f(x+y)=x+y$, then this becomes $(x+y)(x-y)=x^2+y^2-yf(0)$ and so $2y^2=yf(0)$ and so :
- either $y=\frac{f(0)}2$ but this is impossible since then we would have $f(y)=f(0)-\frac{f(0)}2=y$
- either $y=0$

And so the only possibility would be $y=0$ and $f(0)\ne 0$
But then, let $z=f(0)$ : we have $f(z)=0$ (using 1.3) $f(z)=f(0)-z$ and $f(z)\ne z$
And, repeating the same process than above with $x,z$ instead of $x,y$, we get $z=0$ and so contradictoin.

So no such $x,y$ exist
So either $f(x)=x$ $\forall x$, either $f(x)=f(0)-x$ $\forall x$

4) synthesis of solutions.
=================
We got three possibilities (and it's easy to check back that each indeed is a solution) :

$f(x)=0$ $\forall x$
$f(x)=x$ $\forall x$
$f(x)=a-x$ $\forall x$
\end{solution}



\begin{solution}[by \href{https://artofproblemsolving.com/community/user/91306}{ndk09}]
	pco, thankyou very much, It's a great solution! :blush:
\end{solution}



\begin{solution}[by \href{https://artofproblemsolving.com/community/user/92334}{vanstraelen}]
	Possibility 4
$ \forall x $: $f(x)=-x$
is not included in: $f(x)=a-x= f(0)-x$, because $f(0) \neq 0$
\end{solution}



\begin{solution}[by \href{https://artofproblemsolving.com/community/user/29428}{pco}]
	\begin{tcolorbox}Possibility 4
$ \forall x $: $f(x)=-x$
is not included in: $f(x)=a-x= f(0)-x$, because $f(0) \neq 0$\end{tcolorbox}
Nobody wrote $f(0)\ne 0$

And so $f(x)=-x$ is perfectly included in the form $f(0)-x$
\end{solution}
*******************************************************************************
-------------------------------------------------------------------------------

\begin{problem}[Posted by \href{https://artofproblemsolving.com/community/user/94534}{mcrasher}]
	find all functions $f$ from the positive real numbers to itself such that
$f(f(x))=6x-f(x)$, for all positive real numbers $x$.
	\flushright \href{https://artofproblemsolving.com/community/c6h380149}{(Link to AoPS)}
\end{problem}



\begin{solution}[by \href{https://artofproblemsolving.com/community/user/16261}{Rust}]
	Obviosly $f(x)=2x$ and $f(x)=-3x$ are solutions. If $f(x)$ is continiosly there are not another functions. Else
we can represent $R$ as union $U$ and $V$ with unique intersection $0$, satysfyed condition: for any $x\in U$ $2x,-3x\in U$ for any $x\in V$ $2x,-3x\in V$ and define $f(x)=2x$ if $x\in U$ and $f(x)=-3x$ if $x\in V$.
\end{solution}



\begin{solution}[by \href{https://artofproblemsolving.com/community/user/29428}{pco}]
	\begin{tcolorbox}find all functions $f$ from the positive real numbers to itself such that
$f(f(x))=6x-f(x)$, for all positive real numbers $x$.\end{tcolorbox}
It's immediate to establish that $f^{[n]}(x)=\frac{3x+f(x)}52^n+\frac{2x-f(x)}5(-3)^n$

And so, if, for some $x>0$, we get $f(x)\ne 2x$, then for $n$ great enough (odd if $f(x)<2x$ and even if $f(x)>2x$), we get $f^{[n]}(x)<0$, which is impossible.

So $\boxed{f(x)=2x}$ and it's easy to check back that this indeed is a solution.
\end{solution}



\begin{solution}[by \href{https://artofproblemsolving.com/community/user/94534}{mcrasher}]
	Thanks to pco and rust, for nice solution. 
\end{solution}
*******************************************************************************
-------------------------------------------------------------------------------

\begin{problem}[Posted by \href{https://artofproblemsolving.com/community/user/92964}{dyta}]
	Find all functions $f: (0,1) \to \mathbb{R}$ such that \[f(xyz)=xf(x)+yf(y)+zf(z)\] for all real numbers $x,y,z \in (0,1)$.
	\flushright \href{https://artofproblemsolving.com/community/c6h380683}{(Link to AoPS)}
\end{problem}



\begin{solution}[by \href{https://artofproblemsolving.com/community/user/29428}{pco}]
	\begin{tcolorbox}Find all functions $f: (0;1) \rightarrow \mathbb{R} : f(xyz)=xf(x)+yf(y)+zf(z)$ for all real numbers $x,y,z \in (0;1)$\end{tcolorbox}
Let $P(x,y,z)$ be the assertion $f(xyz)=xf(x)+yf(y)+zf(z)$

$P(\frac 12,\frac 12,\frac 12)$ $\implies$ $f(\frac 18)=\frac 32f(\frac 12)$

$P(x,\frac 12,\frac 18)$ $\implies$ $f(\frac {x}{16})=xf(x)+\frac 12f(\frac 12)+\frac 18f(\frac 18)$ and so : $f(\frac {x}{16})=xf(x)+\frac {11}{16}f(\frac 12)$


$P(x,\frac 12,\frac 12)$ $\implies$ $f(\frac x4)=xf(x)+f(\frac 12)$

$P(\frac x4,\frac 12,\frac 12)$ $\implies$ $f(\frac {x}{16})=\frac x4f(\frac x4)+f(\frac 12)$ and so : $f(\frac {x}{16})=\frac{x^2}4f(x)+\frac x4f(\frac 12)+f(\frac 12)$

And so $xf(x)+\frac {11}{16}f(\frac 12)=\frac{x^2}4f(x)+\frac x4f(\frac 12)+f(\frac 12)$

$(x-\frac{x^2}4)f(x)=\frac x4f(\frac 12)+\frac 5{16}f(\frac 12)$

And so $f(x)=\frac 14f(\frac 12)\frac {4x+5}{4x-x^2}$ 

Plugging this in original equation, we get $f(\frac 12)=0$

Hence the unique solution : $\boxed{f(x)=0}$ $\forall x$
\end{solution}



\begin{solution}[by \href{https://artofproblemsolving.com/community/user/35201}{peregrinefalcon88}]
	An alternative solution would be to plug in 1\/3, 1\/3, 1\/3 to get f(1\/3) = f(1\/27) and repeat the process to get f(1\/27) = 3^2f(3^-9), f(3^-9) = 3^7f(3^-27) => f(3^-27) = 3^9f(1\/3) and now plug in 1\/3, 1\/3, 1\/3^25 to get 

f(3^-27) = 2\/3f(1\/3)+1\/3^25f(1\/3^25) > 2\/3f(1\/3) = 2*3^8f(1\/3^27) which is a contradiction so no such functions exists. Unlike PCO I am assuming the notation means that 0 and 1 are excluded from the domain and range.
\end{solution}
*******************************************************************************
-------------------------------------------------------------------------------

\begin{problem}[Posted by \href{https://artofproblemsolving.com/community/user/67223}{Amir Hossein}]
	Let $f : \mathbb N \to \mathbb N$ be a function satisfying
\[f(f(m)+f(n))=m+n, \quad \forall m,n \in \mathbb N.\]
Prove that $f(x)=x$ for all $x \in \mathbb N$.
	\flushright \href{https://artofproblemsolving.com/community/c6h381298}{(Link to AoPS)}
\end{problem}



\begin{solution}[by \href{https://artofproblemsolving.com/community/user/29428}{pco}]
	\begin{tcolorbox}Let $f : \mathbb N \to \mathbb N$ be a function satisfying
\[f(f(m)+f(n))=m+n \qquad \forall m,n \in \mathbb N.\]
Prove that $f(x)=x \quad \forall x \in \mathbb N.$

\begin{italicized}Note. $\mathbb N$ is the set of all positive integers.\end{italicized}\end{tcolorbox}
Obviously, $f(x)$ is injective and, since $f(f(m+1)+f(1))=m+2=f(f(m)+f(2))$, we get $f(m+1)=f(m)+f(2)-f(1)$

So $f(n)=(f(2)-f(1))(n-1)+f(1)$

Plugging then $f(n)=an+b$ in the original equation, we get $a=1$ and $b=0$ and so the unique solution $\boxed{f(x)=x}$
\end{solution}



\begin{solution}[by \href{https://artofproblemsolving.com/community/user/85314}{mathmdmb}]
	If $f(n)<n,f(1)<1$ but $f(1)\in \mathbb N$.
Now if $f(x)>x,f(f(m)+f(n))>m+n$,contradiction.So $f(x)=x$
\end{solution}



\begin{solution}[by \href{https://artofproblemsolving.com/community/user/29428}{pco}]
	\begin{tcolorbox}If $f(n)<n,f(1)<1$ but $f(1)\in \mathbb N$.
Now if $f(x)>x,f(f(m)+f(n))>m+n$,contradiction.So $f(x)=x$\end{tcolorbox}

Your proof is false.
What you proved is that :
1) $f(x)<x$ $\forall x$ is impossible
2) $f(x)>x$ $\forall x$ is impossible too

But you cant conclude that $f(x)=x$ $\forall x$
Maybe $f(x)>x$ for some $x$ (for example $x=1$) and $f(x)<x$ for some other.
\end{solution}



\begin{solution}[by \href{https://artofproblemsolving.com/community/user/308400}{Dilshodbek}]
	\begin{tcolorbox}[quote="amparvardi"]Let $f : \mathbb N \to \mathbb N$ be a function satisfying
\[f(f(m)+f(n))=m+n \qquad \forall m,n \in \mathbb N.\]
Prove that $f(x)=x \quad \forall x \in \mathbb N.$

\begin{italicized}Note. $\mathbb N$ is the set of all positive integers.\end{italicized}\end{tcolorbox}
Obviously, $f(x)$ is injective and, since $f(f(m+1)+f(1))=m+2=f(f(m)+f(2))$, we get $f(m+1)=f(m)+f(2)-f(1)$

So $f(n)=(f(2)-f(1))(n-1)+f(1)$

Plugging then $f(n)=an+b$ in the original equation, we get $a=1$ and $b=0$ and so the unique solution $\boxed{f(x)=x}$\end{tcolorbox}

very nice PCO thank you very much
\end{solution}
*******************************************************************************
-------------------------------------------------------------------------------

\begin{problem}[Posted by \href{https://artofproblemsolving.com/community/user/70108}{EhsanNamdari}]
	Find all functions $f,g: \mathbb R \to \mathbb R$ such that for all reals $x$ and $y$,
\[f(x)-f(y)=(x-y)g(x+y).\]
	\flushright \href{https://artofproblemsolving.com/community/c6h381877}{(Link to AoPS)}
\end{problem}



\begin{solution}[by \href{https://artofproblemsolving.com/community/user/29428}{pco}]
	\begin{tcolorbox}Find all functions $f,g:R\to R$ such that:
$f(x)-f(y)=(x-y)g(x+y)$.\end{tcolorbox}
Let $P(x,y)$ be the assertion $f(x)-f(y)=(x-y)g(x+y)$

$P(\frac{x+1}2,\frac{x-1}2)$ $\implies$ $f(\frac{x+1}2)-f(\frac{x-1}2)=g(x)$

$P(\frac{x-1}2,\frac{1-x}2)$ $\implies$ $f(\frac{x-1}2)-f(\frac{1-x}2)=(x-1)g(0)$

$P(\frac{1-x}2,\frac{x+1}2)$ $\implies$ $f(\frac{1-x}2)-f(\frac{x+1}2)=-xg(1)$

Adding these three lines, we get $g(x)=x(g(1)-g(0))+g(0)$ and so $g(x)=ax+b$

then $P(x,0)$ $\implies$ $f(x)=ax^2+bx+f(0)$ and so $f(x)=ax^2+bx+c$

Plugging these two values in original equation, we find that this indeed is a solution whatever are values of $a,b,c$

\begin{bolded}Hence the solution \end{bolded}\end{underlined}:
$f(x)=ax^2+bx+c$
$g(x)=ax+b$
\end{solution}
*******************************************************************************
-------------------------------------------------------------------------------

\begin{problem}[Posted by \href{https://artofproblemsolving.com/community/user/74529}{bigbang195}]
	Find all functions $f,g: \mathbb R \to \mathbb R$ such that for all reals $x$ and $y$,
\[f(x+f(y))+f(xf(y))=y+f(x)+yf(x)\]
and
\[g(x-f(y))=4g(x)-g(y)-3x.\]
	\flushright \href{https://artofproblemsolving.com/community/c6h382161}{(Link to AoPS)}
\end{problem}



\begin{solution}[by \href{https://artofproblemsolving.com/community/user/29428}{pco}]
	\begin{tcolorbox}Find  $f,g: R \to R$ such that 

$a\/f(x+f(y))+f(xf(y))=y+f(x)+yf(x)$


$b\/g(x-f(y))=4g(x)-g(y)-3x $\end{tcolorbox}
Let $P(x,y)$ be the assertion $f(x+f(y))+f(xf(y))=y+f(x)+yf(x)$
Let $Q(x,y)$ be the assertion $g(x-f(y))=4g(x)-g(y)-3x$
Let $f(0)=a$

$P(0,x)$ $\implies$ $f(f(x))=(a+1)x$

If $a=-1$, this implies $f(f(x))=0$ $\forall x$ but then $-1=f(0)=f(f(f(x)))=0$ and so contradiction.

So $a\ne -1$ and $f(f(x))=(a+1)x$ implies that $f(x)$ is a bijection.

Let then $u$ such that $f(u)=0$
$P(x,u)$ $\implies$ $uf(x)=a-u$ and so $a=u=0$, else $f(x)$ is constant and no longer a bijection.
So $f(0)=0$

Then $Q(0,0)$ $\implies$ $g(0)=0$ and $Q(x,0)$ $\implies$ $g(x)=x$ and $Q(x,y)$ becomes $x-f(y)=4x-y-3x$ and so $f(x)=x$

And this indeed is a solution. Hence the unique solution $\boxed{f(x)=g(x)=x\forall x}$
\end{solution}
*******************************************************************************
-------------------------------------------------------------------------------

\begin{problem}[Posted by \href{https://artofproblemsolving.com/community/user/93909}{magical}]
	Find all functions $f: \mathbb N \to \mathbb N$ such that
\[f(mf(n))=n^2f(m), \quad \forall m,n \in \mathbb N.\]
	\flushright \href{https://artofproblemsolving.com/community/c6h382208}{(Link to AoPS)}
\end{problem}



\begin{solution}[by \href{https://artofproblemsolving.com/community/user/79494}{oneplusone}]
	I am not sure if this is correct.

Suppose there exists $a,b\in\mathbb{N}$ such that $f(a)=f(b)$. Then sub $n=a,b$ we get $a^2f(m)=f(mf(a))=f(mf(b))=b^2f(m)$, so $a=b$. So we have proven $f$ is injective. Sub $n=1$ we have $f(mf(1))=f(m)$ so $f(1)=1$. Now let $m=1$, then we have $f(f(n))=n^2$. Sub $m=f(m)$ into the original equation, $f(f(m)f(n))=m^2n^2=f(f(mn))$, so $f(m)f(n)=f(mn)$. Thus for all composite $c$, $f(c)$ is dependent only on $f$ of primes. Suppose $f(p)=q$ then $f(q)=p^2$, where $p$ is prime. If $q$ is composite, let $q=ar$, then $f(q)=f(a)f(r)=p^2$, so both must be $p$, contradiction. Thus $q$ is prime. So in the entire universe of primes, we can pair all of them up like $(p,q)$ such that $f(p)=q$ and $f(q)=p^2$. Now consider such a function with primes paired up like above and composites defined using $f(mn)=f(m)f(n)$, then $f$ is defined on $\mathbb{N}$ ($f(1)=1$ of course). Then it is easy to check $f(mn)=f(m)f(n)$ for all naturals $m,n$. Also $f(f(n))=n^2$ is true for all primes $n$. Assume $n=pq$ where $p$ is prime. Then $f(f(pq))=f(f(p)f(q))=f(f(p))f(f(q))=p^2f(f(q))$. Then repeat the same for $q$, and we have shown for all naturals. Thus $f(mn)=f(m)f(n)$ and $f(f(n))=n^2$ for all naturals $m,n$. Sub $n=f(n)$ for the first equation, we have $f(mf(n))=n^2f(m)$, the original equation.
\end{solution}



\begin{solution}[by \href{https://artofproblemsolving.com/community/user/74529}{bigbang195}]
	\begin{tcolorbox}I am not sure if this is correct.

Sub $m=f(m)$ into the original equation, $f(f(m)f(n))=m^2n^2=f(f(mn))$,\end{tcolorbox}

I don't understand , because $f(m)=m$ then we have done.

and if it's right then $f(f(m)f(n))=mn^2  \not= (mn)^2$
\end{solution}



\begin{solution}[by \href{https://artofproblemsolving.com/community/user/29428}{pco}]
	\begin{tcolorbox}Find $f:N^+ \rightarrow N^+ $ such that
$f(mf(n))=n^2f(m)$\end{tcolorbox}
I suppose that $N^+=\mathbb N$ is the set of positive integers.
Let $P(x,y)$ be the assertion $f(xf(y))=y^2f(x)$

If $f(y_1)=f(y_2)$, comparaison of $P(1,y_1)$ and $P(1,y_2)$ implies $f(1)(y_1^2-y_2^2)=0$ and so $y_1=y_2$ and $f(x)$ is injective.

$P(1,1)$ $\implies$ $f(f(1))=f(1)$ and so $f(1)=1$

$P(1,x)$ $\implies$ $f(f(x))=x^2$

$P(f(x),y)$ $\implies$ $f(f(x)f(y))=y^2f(f(x))=x^2y^2=f(f(xy))$ $\implies$ $f(xy)=f(x)f(y)$

Let then $p$ prime and $f(p)=\prod q_i^{a_i}$ with $a_i>0$ and $q_i$ primes (notice that $f(p)=1$ is impossible since $f(1)=1$ and $f(x)$ is injective).

We get $f(f(p))=p^2=\prod f(q_i)^{a_i}$ and so $f(q_i)=p^{b_i}$ with $b_i>0$ and $\sum a_ib_i=2$ and so :
- either $f(p)=qr$ and $f(q)=f(r)=p$, impossible (injectivity)
- either $f(p)=q^2$ and $f(q)=p$
- either $f(p)=q$ and $f(q)=p^2$

And so the general solution :
Let $A,B$ a split of the set $\mathbb P$ of all primes in two equinumerous subsets ($A\cap B=\emptyset$ and $A\cup B=\mathbb P$) and $g(x)$ a bijection from $A\to B$

$\forall p\in A$ : $f(p)=g(p)$
$\forall p\in B$ : $f(p)=\left(g^{[-1]}(p)\right)^2$
$\forall x\in\mathbb N\setminus\mathbb P$ : $f(\prod p_i^{n_i})=\prod f(p_i)^{n_i}$ (where $p_i\in\mathbb P$)
\end{solution}



\begin{solution}[by \href{https://artofproblemsolving.com/community/user/79494}{oneplusone}]
	\begin{tcolorbox}[quote="oneplusone"]I am not sure if this is correct.

Sub $m=f(m)$ into the original equation, $f(f(m)f(n))=m^2n^2=f(f(mn))$,\end{tcolorbox}

I don't understand , because $f(m)=m$ then we have done.

and if it's right then $f(f(m)f(n))=mn^2  \not= (mn)^2$\end{tcolorbox}
What I mean is sub $m=f(a)$ into $f(mf(n))=n^2f(m)$ and use the fact $f(f(m))=m^2$, then $f(f(a)f(n))=n^2f(f(a))=a^2n^2=f(f(an))$ for all $a,n$.
\end{solution}



\begin{solution}[by \href{https://artofproblemsolving.com/community/user/74529}{bigbang195}]
	\begin{tcolorbox}[quote="bigbang195"]\begin{tcolorbox}I am not sure if this is correct.

Sub $m=f(m)$ into the original equation, $f(f(m)f(n))=m^2n^2=f(f(mn))$,\end{tcolorbox}

I don't understand , because $f(m)=m$ then we have done.

and if it's right then $f(f(m)f(n))=mn^2  \not= (mn)^2$\end{tcolorbox}
What I mean is sub $m=f(a)$ into $f(mf(n))=n^2f(m)$ and use the fact $f(f(m))=m^2$, then $f(f(a)f(n))=n^2f(f(a))=a^2n^2=f(f(an))$ for all $a,n$.\end{tcolorbox}
\begin{bolded}
thanks you :D                                                                                                                                              .\end{bolded}
\end{solution}



\begin{solution}[by \href{https://artofproblemsolving.com/community/user/93909}{magical}]
	Thanks all.
\end{solution}
*******************************************************************************
-------------------------------------------------------------------------------

\begin{problem}[Posted by \href{https://artofproblemsolving.com/community/user/74871}{chuyentoan}]
	1. Find all functions $f: \mathbb{R} \to\mathbb{R}$ such that $f(0)=0$ and for all $x,y \in \mathbb R$,
\[f(x^2+y+2f(y))=y+2f(x)^2.\]
2. Find all functions $f: \mathbb{R} \to\mathbb{R}$ such that
\[f(|x|+f(y+f(y)))=3y+|f(x)|\]
for all $x,y \in \mathbb R$.
	\flushright \href{https://artofproblemsolving.com/community/c6h382907}{(Link to AoPS)}
\end{problem}



\begin{solution}[by \href{https://artofproblemsolving.com/community/user/29428}{pco}]
	\begin{tcolorbox}\begin{bolded}Problem 1:\end{bolded} Find all $f: \mathbb{R} \rightarrow \mathbb{R}$ satisfied:
$i)          f(0)=0$
$ii)         f(x^2+y+2f(y))=y+2f(x)^2$\end{tcolorbox}
Let $P(x,y)$ be the assertion $f(x^2+y+2f(y))=y+2f(x)^2$

$P(x,0)$ $\implies$ $f(x^2)=2f(x)^2$
As a consequence, $f(x)\ge 0$ $\forall x\ge 0$ (*)

Let $u$ such that $f(u)=0$. Then $P(0,u)$ $\implies$ $u=0$ and so $f(u)=0$ $\iff$ $u=0$

$P(x,-2f(x)^2)$ $\implies$ $f(x^2-2f(x)^2+2f(-2f(x)^2))=0$ $\implies$ $x^2-2f(x)^2+2f(-2f(x)^2)=0$
This last equality implies that $f(u)=f(v)$ $\implies$ $u^2=v^2$ (**)

$P(0,x)$ $\implies$ $f(x+2f(x))=x$ (***)

(*)+(**)+(***) implies that $f(x)$ is a bijection from $\mathbb R^+\cup\{0\}\to\mathbb R^+\cup\{0\}$ (I'm speaking obviously about the restriction of $f(x)$ to the set of non negative reals)

Using $f(x^2)=2f(x)^2$ and $f(x+2f(x))=x$, $P(x,y)$ becomes $f(x^2+y+2f(y))=f(x^2)+f(y+2f(y))$
And so $f(x+y)=f(x)+f(y)$ $\forall x,y\ge 0$
And so (since $f(x)\ge 0$ $\forall x\ge 0$) $f(x)=ax$ $\forall x\ge 0$
Plugging this in $f(x+2f(x))=x$, we get $f(x)=\frac x2$ $\forall x\ge 0$

Using then $x^2-2f(x)^2+2f(-2f(x)^2)=0$ with $x\ge 0$, we get $\frac{x^2}2+2f(-\frac{x^2}2)=0$ and so $f(-\frac{x^2}2)=-\frac{x^2}4$

Hence the result : $\boxed{f(x)=\frac x2}$ $\forall x$
\end{solution}



\begin{solution}[by \href{https://artofproblemsolving.com/community/user/63660}{Victory.US}]
	\begin{tcolorbox}\begin{bolded}
\begin{bolded}Problem 2:\end{bolded} Find all $f: \mathbb{R} \rightarrow \mathbb{R}$ satisfied:
$f(|x|+f(y+f(y)))=3y+|f(x)|$\end{tcolorbox}

you can see it was used in the test section team of PTNK, HCM city
\end{solution}



\begin{solution}[by \href{https://artofproblemsolving.com/community/user/29428}{pco}]
	\begin{tcolorbox}\begin{bolded}Problem 2:\end{bolded} Find all $f: \mathbb{R} \rightarrow \mathbb{R}$ satisfied:
$f(|x|+f(y+f(y)))=3y+|f(x)|$\end{tcolorbox}
Let $P(x,y)$ be the assertion $f(|x|+f(y+f(y)))=3y+|f(x)|$

Comparing $P(x,y)$ and $P(-x,y)$, we get $|f(x)|=|f(-x)|$ $\forall x$ 

$f(x)$ is obviously surjective. Let then $u$ such that $f(u)=0$ : $P(x,u)$ $\implies$ $f(|x|)=3u+|f(x)|$
But we previously got $|f(x)|=|f(|x|)|$ and so $f(|x|)=3u+|f(|x|)|$ and so : 
Then, if $u\ne 0$ :  $f(x)=\frac 32u$ $\forall x\ge 0$ which is not a solution
So $u=0$ and $f(x)\ge 0$ $\forall x\ge 0$

$P(x,-\frac{|f(x)|}3)$ $\implies$ $f(|x|+f(-\frac{|f(x)|}3+f(-\frac{|f(x)|}3)))=0$ and so $|x|+f(-\frac{|f(x)|}3+f(-\frac{|f(x)|}3))=0$

From this last equality, we get that $f(x)=f(y)$ $\implies$ $|x|=|y|$
$P(0,x)$ $\implies$ $f(f(x+f(x)))=3x$ 

And so the restriction of $f$ to $\mathbb R^+\cup\{0\}$ is a bijection from $\mathbb R^+\cup\{0\}\to\mathbb R^+\cup\{0\}$

Using then $f(|x|)=|f(x)|$ and $f(f(y+f(y)))=3y$, we can rewrite $P(x,y)$ as $f(|x|+f(y+f(y)))=f(|x|)+f(f(y+f(y)))$
And so $f(x+y)=f(x)+f(y)$ $\forall x,y\ge 0$
And since $f(x)\ge 0$ $\forall x\ge 0$, we get $f(x)=ax$ $\forall x\ge 0$ and plugging in $f(f(x+f(x)))=3x$, we get :

$a=\frac{-1+\sqrt[3]{\frac{79-9\sqrt{77}}2}+\sqrt[3]{\frac{79+9\sqrt{77}}2}}3$ $\sim 1.1746. . .$

Then, for $x<0$, we have either $f(x)=ax$, either $f(x)=-ax$.
Suppose that, for some $x<0$, we get $f(x)=-ax$. Then $P(0,x)$ $\implies$ $f(f((1-a)x))=3x$  and since $(1-a)x>0$, this implies $a^2(1-a)=3$ which is wrong since $a$ is root of $a^2(1+a)=3$

Hence the result (after having checked that this indeed is a solution) : 

$\boxed{f(x)=ax}$ $\forall x$ with $a=\frac{-1+\sqrt[3]{\frac{79-9\sqrt{77}}2}+\sqrt[3]{\frac{79+9\sqrt{77}}2}}3$ $\sim 1.1746. . .$
\end{solution}
*******************************************************************************
-------------------------------------------------------------------------------

\begin{problem}[Posted by \href{https://artofproblemsolving.com/community/user/82678}{hEatLove}]
	Find all real numbers $a$ such that there exists a function $f:\mathbb R\to \mathbb{R}$ satisfying the inequality 
\[x+af(y)\le{y+f(f(x))}\] for all $x,y \in \mathbb R$.
	\flushright \href{https://artofproblemsolving.com/community/c6h383375}{(Link to AoPS)}
\end{problem}



\begin{solution}[by \href{https://artofproblemsolving.com/community/user/29428}{pco}]
	\begin{tcolorbox}Find all real a such that therer exists a function $f:R\to{R}$ satisfying the inequality 
$x+af(y)\le{y+f(f(x))}$ for all $x,y$ real numbsers.\end{tcolorbox}
Answer is $\boxed{a\in(-\infty,0)\cup[1,1]}$

1) $a<0$ fits
===========
Choose $f(x)=b|x|$ with $b=\max(1,-\frac 1a)$ : it would be enough to show :
$x\le f(f(x))$ $\forall x$
$af(x)\le x$ $\forall x$

1.1) $x\le f(f(x))$ $\forall x$
------------------------
$f(f(x))=b^2|x|$ and so :
If $x\le 0$, we obviously have $x\le 0\le b^2|x|$
If $x>0$, we have $1\le b^2$ and so $x\le b^2x=b^2|x|$
Q.E.D.

1.2) $af(x)\le x$ $\forall x$
---------------------------
If $x\ge 0$ : $af(x)=ab|x|\le 0\le x$ (remember $a<0$ and $b\ge 1>0$)
If $x<0$ : $b\ge -\frac 1a$ and so $-ab\ge 1$ and so $af(x)=ab|x|=-abx\le x$
Q.E.D.

2) $a=0$ does not fit
===================
Obviously $x\le y+f(f(x))$ cant be true for all $x,y$ : just set $y\to -\infty$
Q.E.D.

3) If $a>0$ : only $a=1$ fits
============================
Let $a>0$
The inequality is $f(f(x))-x\ge af(y)-y$ and so $\exists u$ such that $f(f(x))-x\ge u\ge af(x)-x$ $\forall x$

But $af(x)\le x+u$ $\implies$ $f(x)\le \frac xa+\frac ua$ $\implies$ $f(f(x))\le  \frac {f(x)}a+\frac ua$ $\le \frac {x}{a^2}+\frac u{a^2}+\frac ua$

And so $x+u\le \frac {x}{a^2}+\frac u{a^2}+\frac ua$ $\forall x$ and so $a^2=1$ and $a=1$

So, only $a=1$ may fit.
And using $f(x)=x$, it's immediate to check that $a=1$ fits.
Q.E.D.
\end{solution}
*******************************************************************************
-------------------------------------------------------------------------------

\begin{problem}[Posted by \href{https://artofproblemsolving.com/community/user/92753}{WakeUp}]
	For all positive real numbers $x$ and $y$ let
\[f(x,y)=\min\left( x,\frac{y}{x^2+y^2}\right) \]
Show that there exist $x_0$ and $y_0$ such that $f(x, y)\le f(x_0, y_0)$ for all positive $x$ and $y$, and find $f(x_0,y_0)$.
	\flushright \href{https://artofproblemsolving.com/community/c6h383417}{(Link to AoPS)}
\end{problem}



\begin{solution}[by \href{https://artofproblemsolving.com/community/user/29428}{pco}]
	\begin{tcolorbox}For all positive real numbers $x$ and $y$ let
\[f(x,y)=\min\left( x,\frac{y}{x^2+y^2}\right) \]
Show that there exist $x_0$ and $y_0$ such that $f(x, y)\le f(x_0, y_0)$ for all positive $x$ and $y$, and find $f(x_0,y_0)$.\end{tcolorbox}
We'll show that $\boxed{f(x,y)\le f\left(\frac 1{\sqrt 2},\frac 1{\sqrt 2}\right)=\frac 1{\sqrt 2}}$ $\forall x,y>0$

If $x\le \frac 1{\sqrt 2}$, then $f(x,y)\le x\le \frac 1{\sqrt 2}$

If $x\ge\frac 1{\sqrt 2}$, then $y^2-\sqrt 2 y+x^2\ge 0$ (consider it as a quadratic in $y$) and so $\frac y{x^2+y^2}\le \frac 1{\sqrt 2}$ and so $f(x,y)\le\frac y{x^2+y^2}\le \frac 1{\sqrt 2}$

Q.E.D.
\end{solution}



\begin{solution}[by \href{https://artofproblemsolving.com/community/user/15392}{TZF}]
	\begin{bolded}Hint:\end{bolded} In polar, we have $f(r, \theta) = \text{min} \left( r \cos \theta, \; \frac{1}{r} \sin \theta \right)$
\end{solution}
*******************************************************************************
-------------------------------------------------------------------------------

\begin{problem}[Posted by \href{https://artofproblemsolving.com/community/user/67223}{Amir Hossein}]
	Find all continuous functions $f: \mathbb R \to \mathbb R$ such that
\[f(x^2-y^2)=f(x)^2 + f(y)^2, \quad \forall x,y \in \mathbb R.\]
	\flushright \href{https://artofproblemsolving.com/community/c6h383466}{(Link to AoPS)}
\end{problem}



\begin{solution}[by \href{https://artofproblemsolving.com/community/user/29428}{pco}]
	\begin{tcolorbox}Find all continuous functions $f: \mathbb R \to \mathbb R$ such that
\[f(x^2-y^2)=f(x)^2 + f(y)^2 \qquad \forall x,y \in \mathbb R.\]\end{tcolorbox}
Setting $y=x$, we get $f(0)=2f(x)^2$ and so $f(x)=\pm c$ and so, since continuous at $0$, $f(x)=a$, constant.

Plugging this in original equation, we find two solutions : $\boxed{f(x)=0}$ $\forall x$ and $\boxed{f(x)=\frac 12}$ $\forall x$
\end{solution}



\begin{solution}[by \href{https://artofproblemsolving.com/community/user/36770}{diego627}]
	\begin{tcolorbox}Find all continuous functions $f: \mathbb R \to \mathbb R$ such that
\[f(x^2-y^2)=f(x)^2 + f(y)^2 \qquad \forall x,y \in \mathbb R.\]\end{tcolorbox}
Let $f(0)=c$. Set $x=y=0$, then $c=2c^2$.
Case 1. $c=0$.
Set $x=y$, then $2f(x)^2=f(0)=0$ ,then $f(x)=0$ for all $x$.
Case 2. $c=1\/2$
Set $x=y$ then $2f(x)^2=f(0)=1\/2$ therefore $f(x)^2=1\/4$ for all x.
If $a \geq 0$ set $x=\sqrt{a},y=0$ then $f(a)=f(x)^2+f(y)^2=1\/4+1\/4=1\/2$.
If $a < 0$ set $x=\sqrt{-a}, y=0$ then $f(a)=f(x)^2+f(y)^2=1\/4+1\/4=1\/2$.
So $f(x)=1\/2$ for all x.
Testing both functions, they both work. Note that we didnt need that $f$ is continous.
\end{solution}
*******************************************************************************
-------------------------------------------------------------------------------

\begin{problem}[Posted by \href{https://artofproblemsolving.com/community/user/91306}{ndk09}]
	Find all continuous functions $f:\left[ -1,1\right] \to \left[-1,1\right]$ such that for all $x\in \left[ -1,1\right]$,
\[f(2x^2-1)=2xf(x).\]
	\flushright \href{https://artofproblemsolving.com/community/c6h383866}{(Link to AoPS)}
\end{problem}



\begin{solution}[by \href{https://artofproblemsolving.com/community/user/29428}{pco}]
	\begin{tcolorbox}Find all continous function $f:\left[ -1,1\right] \longrightarrow \left[-1,1\right]$ such that :
\[f(2x^2-1)=2xf(x)\] $\forall x\in \left[ -1,1\right]$\end{tcolorbox}
First notice that $xf(x)=-xf(-x)$ and so $f(-x)=-f(x)$ $\forall x\ne 0$ and so $f(-x)=-f(x)$ $\forall x$ (continuity)
As a consequence, $f(0)=0$

Then $f(2\cos^2x-1)=2\cos xf(\cos x)$ and $f(2\sin^2x-1)=2\sin xf(\sin x)$ and so $\cos xf(\cos x)=-\sin xf(\sin x)$

Let then $\theta\in(0,\frac {\pi}2)$ : $f(2\cos^2\frac{\theta}2-1)=2\cos\frac{\theta}2f(\cos\frac{\theta}2)$

And so $\frac{f(\cos \theta)}{\sin \theta}=$ $\frac{f(\cos \frac{\theta}2)}{\sin \frac{\theta}2}$

And so $\frac{f(\cos \theta)}{\sin \theta}=$ $\frac{f(\cos \frac{\theta}{2^n})}{\sin \frac{\theta}{2^n}}$

And so $\frac{f(\cos \theta)}{\sin \theta}=$ $-\frac{f(\sin \frac{\theta}{2^n})}{\cos \frac{\theta}{2^n}}$ (see third line of this post)

Setting $n\to +\infty$ and using continuity, we get then $\frac{f(\cos \theta)}{\sin \theta}=-f(0)=0$ $\forall \theta\in(0,\frac{\pi}2)$

Hence the unique solution :$\boxed{f(x)=0}$ $\forall x$
\end{solution}
*******************************************************************************
-------------------------------------------------------------------------------

\begin{problem}[Posted by \href{https://artofproblemsolving.com/community/user/67223}{Amir Hossein}]
	Let $f: \mathbb N \to \mathbb N$ be a function such that $f(1)=1$ and
\[f(n)=n - f(f(n-1)), \quad \forall n \geq 2.\]
Prove that $f(n+f(n))=n $ for each positive integer $n.$
	\flushright \href{https://artofproblemsolving.com/community/c6h383979}{(Link to AoPS)}
\end{problem}



\begin{solution}[by \href{https://artofproblemsolving.com/community/user/29428}{pco}]
	\begin{tcolorbox}Let $f: \mathbb N \to \mathbb N$ be a function such that $f(1)=1$ and
\[f(n)=n - f(f(n-1)) \qquad n \geq 2.\]
Prove that $f(n+f(n))=n $ for each positive integer $n.$

\begin{italicized}Note. $\mathbb N$ denotes the set of positive integers.\end{italicized}\end{tcolorbox}

I'm sure it's not the requested solution, but there is a simple way : 

It's not very difficult to see that $f(x)$ is unique.
let then $f(n)=\left\lfloor\frac{2n+2}{1+\sqrt 5}\right\rfloor$
It's rather easy (with some careful casework) to prove that this $f(x)$ matches the two equations.
Hence the result.
\end{solution}



\begin{solution}[by \href{https://artofproblemsolving.com/community/user/74529}{bigbang195}]
	\begin{tcolorbox}[quote="amparvardi"]Let $f: \mathbb N \to \mathbb N$ be a function such that $f(1)=1$ and
\[f(n)=n - f(f(n-1)) \qquad n \geq 2.\]
Prove that $f(n+f(n))=n $ for each positive integer $n.$

\begin{italicized}Note. $\mathbb N$ denotes the set of positive integers.\end{italicized}\end{tcolorbox}

I'm sure it's not the requested solution, but there is a simple way : 

It's not very difficult to see that $f(x)$ is unique.
let then $f(n)=\left\lfloor\frac{2n+2}{1+\sqrt 5}\right\rfloor$
It's rather easy (with some careful casework) to prove that this $f(x)$ matches the two equations.
Hence the result.\end{tcolorbox}

i see $f(2)=1$ so
\end{solution}



\begin{solution}[by \href{https://artofproblemsolving.com/community/user/29428}{pco}]
	\begin{tcolorbox}i see $f(2)=1$ so\end{tcolorbox}
So what ?
\end{solution}



\begin{solution}[by \href{https://artofproblemsolving.com/community/user/74529}{bigbang195}]
	$f(2)=\left\lfloor\frac{2.2+2}{1+\sqrt 5}\right\rfloor=2 \not=1$
\end{solution}



\begin{solution}[by \href{https://artofproblemsolving.com/community/user/29428}{pco}]
	\begin{tcolorbox}$f(2)=\left\lfloor\frac{2.2+2}{1+\sqrt 5}\right\rfloor=2 \not=1$\end{tcolorbox}
In my country, $\frac{2.2+2}{1+\sqrt 5}=1.854101966...$ and so $f(2)=\left\lfloor\frac{2.2+2}{1+\sqrt 5}\right\rfloor=1$
\end{solution}



\begin{solution}[by \href{https://artofproblemsolving.com/community/user/74529}{bigbang195}]
	\begin{bolded}i'm sorry,  it's stupid mistake :-s

you can  resolved this problem carefully

i just understand :( \end{bolded}
\end{solution}



\begin{solution}[by \href{https://artofproblemsolving.com/community/user/87195}{SCP}]
	\begin{tcolorbox}[quote="amparvardi"]Let $f: \mathbb N \to \mathbb N$ be a function such that $f(1)=1$ and
\[f(n)=n - f(f(n-1)) \qquad n \geq 2.\]
Prove that $f(n+f(n))=n $ for each positive integer $n.$

\begin{italicized}Note. $\mathbb N$ denotes the set of positive integers.\end{italicized}\end{tcolorbox}

I'm sure it's not the requested solution, but there is a simple way : 

It's not very difficult to see that $f(x)$ is unique.
let then $f(n)=\left\lfloor\frac{2n+2}{1+\sqrt 5}\right\rfloor$
It's rather easy (with some careful casework) to prove that this $f(x)$ matches the two equations.
Hence the result.\end{tcolorbox}

I understand there can only be one solution (induction from 1, we see always only 1 f(n) if we know f(n-1),
but how can we guess the only solution;$f(n)=\left\lfloor\frac{2n+2}{1+\sqrt 5}\right\rfloor$ ?
\end{solution}



\begin{solution}[by \href{https://artofproblemsolving.com/community/user/29428}{pco}]
	\begin{tcolorbox}[quote="pco"][quote="amparvardi"]Let $f: \mathbb N \to \mathbb N$ be a function such that $f(1)=1$ and
\[f(n)=n - f(f(n-1)) \qquad n \geq 2.\]
Prove that $f(n+f(n))=n $ for each positive integer $n.$

\begin{italicized}Note. $\mathbb N$ denotes the set of positive integers.\end{italicized}\end{tcolorbox}

I'm sure it's not the requested solution, but there is a simple way : 

It's not very difficult to see that $f(x)$ is unique.
let then $f(n)=\left\lfloor\frac{2n+2}{1+\sqrt 5}\right\rfloor$
It's rather easy (with some careful casework) to prove that this $f(x)$ matches the two equations.
Hence the result.\end{tcolorbox}
I understand there can only be one solution (induction from 1, we see always only 1 f(n) if we know f(n-1),
but how can we guess the only solution;$f(n)=\left\lfloor\frac{2n+2}{1+\sqrt 5}\right\rfloor$ ?\end{tcolorbox}
According to me, there is no way to guess.
I remembered some old exercices about $f(n+f(n))=n$ whose solution was something similar to $\lfloor\frac n{\varphi}\rfloor$ and I made some trials in order to find the good value.
Then , there is just some work to prove it.

Since it's impoissible to guess, that's the reason why I wrote "I'm sure it's not the requested solution" : I think there is surely a simple direct way.
\end{solution}



\begin{solution}[by \href{https://artofproblemsolving.com/community/user/40375}{benimath}]
	I think I got an idea. It is clear that we can determine all the values of the function, and therefore the function is unique. By checking the first values by hand we see that our function is increasing and it increases with at most $1$ at a time. 

It is clear that $f(n)\leq n,\ \forall n \in \Bbb{N}$. Now suppose that $f(n) \in \{f(n-1),f(n-1)+1\},\  (\star)$. This can be seen to be true for small $n$. Suppose now that $(\star)$ is true for all $n\leq N$. Then $f(N+1)-f(N)=1-(f(f(N))-f(f(N-1)))$. 
Let's analyze $h=f(f(N))-f(f(N-1))$. If $f(N)=f(N-1)$ then $h=0$. If $f(N)=f(N-1)+1$ then, because of the inductive hypothesis we need to have $h \in \{0,1\}$. Anyway, $h$ can only take the values $0,1$. This proves by induction that $(\star)$ is true for all positive integers.

Now let's try to prove that $f(n+f(n))=n$ by induction. Small cases are true. 
Suppose that $f(n+f(n))=n,\ \forall n \leq N$. We know that $f(N+1)=N+1-f(f(N))$, and we calculate $f(N+1+f(N+1))=N+1+f(N+1)-f(f(N+f(N+1))\ (0)$.
We have two cases:
1. $f(N+1)=f(N)$. Then $f(f(N+f(N+1)))=f(f(N+f(N)))=f(N)=f(N+1)$ which implies that $(0) = N+1$.

2. $f(N+1)=f(N)+1$. Then $f(f(N+f(N+1)))=f(f(N+1+f(N)))=$$f(N+1+f(N)-f(f(N+f(N))))=f(N+1+f(N)-f(N))=f(N+1)$ which implies that $(0)=N+1$.

Thus in both cases we have $f(N+1+f(N+1))=N+1$, which proves by induction that the identity is true for all positive integers $n\geq 1$.
\end{solution}



\begin{solution}[by \href{https://artofproblemsolving.com/community/user/85501}{taylorcute56}]
	\begin{tcolorbox}[quote="amparvardi"]Let $f: \mathbb N \to \mathbb N$ be a function such that $f(1)=1$ and
\[f(n)=n - f(f(n-1)) \qquad n \geq 2.\]
Prove that $f(n+f(n))=n $ for each positive integer $n.$

\begin{italicized}Note. $\mathbb N$ denotes the set of positive integers.\end{italicized}\end{tcolorbox}

I'm sure it's not the requested solution, but there is a simple way : 

It's not very difficult to see that $f(x)$ is unique.
let then $f(n)=\left\lfloor\frac{2n+2}{1+\sqrt 5}\right\rfloor$
It's rather easy (with some careful casework) to prove that this $f(x)$ matches the two equations.
Hence the result.\end{tcolorbox}
Do you can prove f(x) is unique?
\end{solution}



\begin{solution}[by \href{https://artofproblemsolving.com/community/user/29428}{pco}]
	\begin{tcolorbox} Do you can prove f(x) is unique?\end{tcolorbox}
$f(n)\le n$ $\forall n$ and so, if $f(x)$ is known over $[1,n]$, we also know $f(f(n))$, and so $f(n+1)=n+1-f(f(n))$ is uniquely known.

And since $f(n)$ is unique over $[1,1]$, we get that $f(n)$ is unique over $\mathbb N$
\end{solution}



\begin{solution}[by \href{https://artofproblemsolving.com/community/user/29428}{pco}]
	\begin{tcolorbox}I think I got an idea. It is clear that we can determine all the values of the function, and therefore the function is unique. By checking the first values by hand we see that our function is increasing and it increases with at most $1$ at a time. 

It is clear that $f(n)\leq n,\ \forall n \in \Bbb{N}$. Now suppose that $f(n) \in \{f(n-1),f(n-1)+1\},\  (\star)$. This can be seen to be true for small $n$. Suppose now that $(\star)$ is true for all $n\leq N$. Then $f(N+1)-f(N)=1-(f(f(N))-f(f(N-1)))$. 
Let's analyze $h=f(f(N))-f(f(N-1))$. If $f(N)=f(N-1)$ then $h=0$. If $f(N)=f(N-1)+1$ then, because of the inductive hypothesis we need to have $h \in \{0,1\}$. Anyway, $h$ can only take the values $0,1$. This proves by induction that $(\star)$ is true for all positive integers.

Now let's try to prove that $f(n+f(n))=n$ by induction. Small cases are true. 
Suppose that $f(n+f(n))=n,\ \forall n \leq N$. We know that $f(N+1)=N+1-f(f(N))$, and we calculate $f(N+1+f(N+1))=N+1+f(N+1)-f(f(N+f(N+1))\ (0)$.
We have two cases:
1. $f(N+1)=f(N)$. Then $f(f(N+f(N+1)))=f(f(N+f(N)))=f(N)=f(N+1)$ which implies that $(0) = N+1$.

2. $f(N+1)=f(N)+1$. Then $f(f(N+f(N+1)))=f(f(N+1+f(N)))=$$f(N+1+f(N)-f(f(N+f(N))))=f(N+1+f(N)-f(N))=f(N+1)$ which implies that $(0)=N+1$.

Thus in both cases we have $f(N+1+f(N+1))=N+1$, which proves by induction that the identity is true for all positive integers $n\geq 1$.\end{tcolorbox}
I do agree.
Quite nice direct solution which does not need to compute a closed form for $f(n)$.
Congrats !
\end{solution}



\begin{solution}[by \href{https://artofproblemsolving.com/community/user/93464}{taiiyama}]
	\begin{tcolorbox}I think I got an idea. It is clear that we can determine all the values of the function, and therefore the function is unique. By checking the first values by hand we see that our function is increasing and it increases with at most $1$ at a time. 

It is clear that $f(n)\leq n,\ \forall n \in \Bbb{N}$. Now suppose that $f(n) \in \{f(n-1),f(n-1)+1\},\  (\star)$. This can be seen to be true for small $n$. Suppose now that $(\star)$ is true for all $n\leq N$. Then $f(N+1)-f(N)=1-(f(f(N))-f(f(N-1)))$. 
Let's analyze $h=f(f(N))-f(f(N-1))$. If $f(N)=f(N-1)$ then $h=0$. If $f(N)=f(N-1)+1$ then, because of the inductive hypothesis we need to have $h \in \{0,1\}$. Anyway, $h$ can only take the values $0,1$. This proves by induction that $(\star)$ is true for all positive integers.

Now let's try to prove that $f(n+f(n))=n$ by induction. Small cases are true. 
Suppose that $f(n+f(n))=n,\ \forall n \leq N$. We know that $f(N+1)=N+1-f(f(N))$, and we calculate $f(N+1+f(N+1))=N+1+f(N+1)-f(f(N+f(N+1))\ (0)$.
We have two cases:
1. $f(N+1)=f(N)$. Then $f(f(N+f(N+1)))=f(f(N+f(N)))=f(N)=f(N+1)$ which implies that $(0) = N+1$.

2. $f(N+1)=f(N)+1$. Then $f(f(N+f(N+1)))=f(f(N+1+f(N)))=$$f(N+1+f(N)-f(f(N+f(N))))=f(N+1+f(N)-f(N))=f(N+1)$ which implies that $(0)=N+1$.

Thus in both cases we have $f(N+1+f(N+1))=N+1$, which proves by induction that the identity is true for all positive integers $n\geq 1$.\end{tcolorbox}
i had the same thing like your process but i used contradiction while induction by plugging in the number between f(n+f(n)) and f(n+1+f(n+1))
depending on whether f(n) or f(n)+1=f(n+1) the rest is the same.
i really want to know how pco generated the function.well thanks though(the difference of 1 you see can be proved when you see that f(n+1)>=f(n)for all n.you can prove that if a>b
f(a)-f(b)<=(a-b)  )
\end{solution}



\begin{solution}[by \href{https://artofproblemsolving.com/community/user/29428}{pco}]
	\begin{tcolorbox} i really want to know how pco generated the function.\end{tcolorbox}

As I previously said, I remembered an old problem rather similar whose solution was $\left\lfloor\frac x{\varphi}\right\rfloor$ and so I made some trials and found this value which is not very difficult to prove (when you know it :) )
\end{solution}



\begin{solution}[by \href{https://artofproblemsolving.com/community/user/95505}{hxthanh}]
	\begin{tcolorbox}[quote="amparvardi"]Let $f: \mathbb N \to \mathbb N$ be a function such that $f(1)=1$ and
\[f(n)=n - f(f(n-1)) \qquad n \geq 2.\]
Prove that $f(n+f(n))=n $ for each positive integer $n.$

\begin{italicized}Note. $\mathbb N$ denotes the set of positive integers.\end{italicized}\end{tcolorbox}

I'm sure it's not the requested solution, but there is a simple way : 

It's not very difficult to see that $f(x)$ is unique.
let then $f(n)=\left\lfloor\frac{2n+2}{1+\sqrt 5}\right\rfloor$
It's rather easy (with some careful casework) to prove that this $f(x)$ matches the two equations.
Hence the result.\end{tcolorbox}
I think so...
$f(n)=n-\left(\left\lfloor\frac{n+1}{21}\right\rfloor+\left\lfloor\frac{n+3}{21}\right\rfloor+\left\lfloor\frac{n+6}{21}\right\rfloor+\left\lfloor\frac{n+8}{21}\right\rfloor+\left\lfloor\frac{n+11}{21}\right\rfloor+\left\lfloor\frac{n+14}{21}\right\rfloor+\left\lfloor\frac{n+16}{21}\right\rfloor+\left\lfloor\frac{n+19}{21}\right\rfloor\right)$
\end{solution}



\begin{solution}[by \href{https://artofproblemsolving.com/community/user/29428}{pco}]
	\begin{tcolorbox} I think so...
$f(n)=n-\left(\left\lfloor\frac{n+1}{21}\right\rfloor+\left\lfloor\frac{n+3}{21}\right\rfloor+\left\lfloor\frac{n+6}{21}\right\rfloor+\left\lfloor\frac{n+8}{21}\right\rfloor+\left\lfloor\frac{n+11}{21}\right\rfloor+\left\lfloor\frac{n+14}{21}\right\rfloor+\left\lfloor\frac{n+16}{21}\right\rfloor+\left\lfloor\frac{n+19}{21}\right\rfloor\right)$\end{tcolorbox}

I dont understand what you mean ?
Do you mean your function is a solution ?
If so, you are wrong ... .

Choose for example $n=54$. With your function :
$f(n)=f(54)=34$
$f(n-1)=f(53)=33$
$f(f(n-1))=f(33)=21$
$n-f(f(n-1))=54-21=33\ne f(n)$
\end{solution}



\begin{solution}[by \href{https://artofproblemsolving.com/community/user/95505}{hxthanh}]
	Thanks for your checked!
I'm wrong...
[url=http://www.wolframalpha.com\/input\/?i=plot+n-%28floor%5B%28n%2B1%29%2F21%5D%2Bfloor%5B%28n%2B3%29%2F21%5D%2Bfloor%5B%28n%2B6%29%2F21%5D%2Bfloor%5B%28n%2B8%29%2F21%5D%2Bfloor%5B%28n%2B11%29%2F21%5D%2Bfloor%5B%28n%2B14%29%2F21%5D%2Bfloor%5B%28n%2B16%29%2F21%5D%2Bfloor%5B%28n%2B19%29%2F21%5D%29%2C+floor%5B%282n%2B2%29%2F%281%2Bsqrt%5B5%5D%29%5D%2C+n%3D1+to+50]\begin{bolded}ref\end{bolded}[\/url]!
\end{solution}



\begin{solution}[by \href{https://artofproblemsolving.com/community/user/87322}{paul1703}]
	Please correct me... to conjencture that value for the function you need to use the linear aproximation tehnique so 
you nead that $f(n)\sim{an}$ so $f(n)=n-f(f(n))$ $<=>$ $an\sim{n-a^2n}$ by dividing with n we get that $a^2+a-1=0$ so a =$\frac{1+\sqrt{5}}{2}$ instead of $\frac{2}{1+\sqrt{5}}$ where am i wrong?????
\end{solution}



\begin{solution}[by \href{https://artofproblemsolving.com/community/user/29428}{pco}]
	\begin{tcolorbox}Please correct me... to conjencture that value for the function you need to use the linear aproximation tehnique so 
you nead that $f(n)\sim{an}$ so $f(n)=n-f(f(n))$ $<=>$ $an\sim{n-a^2n}$ by dividing with n we get that $a^2+a-1=0$ so a =$\frac{1+\sqrt{5}}{2}$ instead of $\frac{2}{1+\sqrt{5}}$ where am i wrong?????\end{tcolorbox}

In fact, $\frac{1+\sqrt{5}}{2}$ is not root of $a^2+a-1=0$. The two roots are $a_1=\frac{-1-\sqrt 5}2<0$ and $a_2=\frac{-1+\sqrt 5}2=\frac 2{\sqrt 5+1}$
\end{solution}



\begin{solution}[by \href{https://artofproblemsolving.com/community/user/87322}{paul1703}]
	sorry :blush: this is the way you conjenctured the solution right?
\end{solution}



\begin{solution}[by \href{https://artofproblemsolving.com/community/user/201437}{thangtoancvp}]
	\begin{tcolorbox}Let $f: \mathbb N \to \mathbb N$ be a function such that $f(1)=1$ and
\[f(n)=n - f(f(n-1)) \qquad n \geq 2.\]
Prove that $f(n+f(n))=n $ for each positive integer $n.$

\begin{italicized}Note. $\mathbb N$ denotes the set of positive integers.\end{italicized}\end{tcolorbox}
We prove by induction $f\left( n \right) = k,f\left( {n + 21} \right) = k + 13$
\end{solution}



\begin{solution}[by \href{https://artofproblemsolving.com/community/user/29428}{pco}]
	\begin{tcolorbox}We prove by induction $f\left( n \right) = k,f\left( {n + 21} \right) = k + 13$\end{tcolorbox}
Dont hesitate to post the proof of your interesting [but unfortunately wrong] claim.

[hide]Just compare $f(33)$ and $f(54)$ and check that their difference is not $13$[\/hide]
\end{solution}



\begin{solution}[by \href{https://artofproblemsolving.com/community/user/236427}{AmirAlison}]
	\begin{tcolorbox}Iwe see that our function is increasing and it increases with at most $1$ at a time.  \end{tcolorbox}
As a matter of a fact, that's what we need to prove, as it's the most difficult part of the problem. If you can prove it, then write the proof please :-P


\end{solution}



\begin{solution}[by \href{https://artofproblemsolving.com/community/user/250510}{galav}]
	\begin{tcolorbox}[quote="bigbang195"]$f(2)=\left\lfloor\frac{2.2+2}{1+\sqrt 5}\right\rfloor=2 \not=1$\end{tcolorbox}
In my country, $\frac{2.2+2}{1+\sqrt 5}=1.854101966...$ and so $f(2)=\left\lfloor\frac{2.2+2}{1+\sqrt 5}\right\rfloor=1$\end{tcolorbox}

That is called real sarcasm! Oh After so many days!


\end{solution}
*******************************************************************************
-------------------------------------------------------------------------------

\begin{problem}[Posted by \href{https://artofproblemsolving.com/community/user/67223}{Amir Hossein}]
	Let $f: \mathbb R \to \mathbb R$ and $g: \mathbb R \to \mathbb R$ be two functions satisfying
\[\forall x,y \in \mathbb R: \begin{cases} f(x+y)=f(x)f(y),\\ f(x)= x g(x)+1\end{cases} \quad \text{and} \quad \lim_{x \to 0} g(x)=1.\]
Find the derivative of $f$ in an arbitrary point $x.$
	\flushright \href{https://artofproblemsolving.com/community/c6h383984}{(Link to AoPS)}
\end{problem}



\begin{solution}[by \href{https://artofproblemsolving.com/community/user/94534}{mcrasher}]
	\begin{tcolorbox}Let $f: \mathbb R \to \mathbb R$ and $g: \mathbb R \to \mathbb R$ be two functions satisfying
\[\forall x,y \in \mathbb R: \{\begin{array}{cc}f(x+y)=f(x)f(y) \\ \text{ } \\ f(x)= x g(x)+1\end{array}, \quad \lim_{x \to 0} g(x)=1.\]
Find the derivative of $f$ in an arbitrary point $x.$\end{tcolorbox}
we know 
$f'(x)=\lim_{h\to0}\frac{f(x+h)-f(x)}{h}$

from $f(x+y)=f(x)f(y)$

$f'(x)=\lim_{h\to0}\frac{f(x)f(h)-f(x)}{h}$

$f'(x)=f(x)\lim_{h\to0}\frac{f(h)-1}{h}$

from $f(x)= x g(x)+1$
$f(h)-1=g(h)$

$f'(x)=f(x)\lim_{h\to0}\frac{hg(h)}{h}=f(x)\lim_{h\to0}{g(h)}=f(x).1$

hence $f'(x)=f(x)$,   $\frac{f'(x)}{f(x)}=1$
on integrating 
$ln(f(x))=x+c$, $\implies{f(x)=ke^x}$

from $f(x+y)=f(x)f(y)$ and $f(x)= x g(x)+1$

$f(0)=1$, so $k=1$ ,   ${f(x)=e^x}$ and ${f'(x)=e^x}$
\end{solution}



\begin{solution}[by \href{https://artofproblemsolving.com/community/user/29428}{pco}]
	\begin{tcolorbox}Let $f: \mathbb R \to \mathbb R$ and $g: \mathbb R \to \mathbb R$ be two functions satisfying
\[\forall x,y \in \mathbb R: \{\begin{array}{cc}f(x+y)=f(x)f(y) \\ \text{ } \\ f(x)= x g(x)+1\end{array}, \quad \lim_{x \to 0} g(x)=1.\]
Find the derivative of $f$ in an arbitrary point $x.$\end{tcolorbox}
If $f(u)=0$ for some $u$, then $f(x)=f(x-u)f(u)=0$ $\forall x$ and $g(x)=-\frac 1x$ $\forall x\ne 0$ and so $\lim_{x\to 0}g(x)\ne 1$

So $f(x)\ne 0$ $\forall x$ and $f(x)=f(\frac x2)^2$ and so $f(x)>0$ $\forall x$

Let then $f(x)=e^{h(x)}$ and the first equation becomes $h(x+y)=h(x)+h(y)$ $\forall x$ and we also have $\lim_{x\to 0}\frac{e^{h(x)}-1}x=1$. So $h(x)$ is bounded around $0$ and so (Cauchy equation property) $h(x)=ax$

$\lim_{x\to 0}\frac{e^{ax}-1}x=1$ implies $a=1$ and so $f(x)=e^x$ and $\boxed{f'(x)=e^x}$
\end{solution}
*******************************************************************************
-------------------------------------------------------------------------------

\begin{problem}[Posted by \href{https://artofproblemsolving.com/community/user/25546}{Yuriy Solovyov}]
	Function $f: \mathbb R \to \mathbb R$ is such that $ |f(a)-f(b)|<|a-b|$ for all reals $ a\ne b$. Prove that if $ f(f(f(0)))=0$ then $ f(0)=0$.
	\flushright \href{https://artofproblemsolving.com/community/c7h199604}{(Link to AoPS)}
\end{problem}



\begin{solution}[by \href{https://artofproblemsolving.com/community/user/24822}{Svejk}]
	First I would consider the following cases : 
If $ f \circ f(0) = f(0)$ $ \Rightarrow$ $ 0 = f \circ f\circ f(0) = f \circ f(0) = f(0)$
If $ f \circ f \circ f(0) = f \circ f(0)$ $ \Rightarrow$ $ 0 = f \circ f \circ f(0) = f\circ f(0)$ $ \Rightarrow$ $ 0 = f \circ f \circ f(0) = f(0)$.
So in both situations we get that $ f(0) = 0$ 

Now, in order to prove that $ f(0) = 0$ I will use the contradiction method.Let's suppose that $ f(0) \neq 0$. So from the relations from the hypothesis we get : 
$ |f \circ f(0) - f(0)| < |f(0) - 0| = |f(0)|$.Furthermore we can suppose that $ f \circ f \neq f(0)$ and $ f \circ f \circ f(0) \neq f \circ f(0)$.So from the relation from the hypothesis we have : 
$ |f \circ f \circ f \circ f(0) - f \circ f \circ f(0)| < |f \circ f \circ f(0) - f \circ f(0)| < |f \circ f(0) - f(0)| < |f(0)|$ $ \Rightarrow$ $ |f(0)| < |f(0)|$ contradiction.
\end{solution}



\begin{solution}[by \href{https://artofproblemsolving.com/community/user/27886}{sos440}]
	Proposition)  Define $ f^n = \underbrace{f \circ \cdots \circ f}_{n \text{ tuples}}$. If $ |f(x) - f(y)| < |x - y|$ for all $ x \neq y$ and $ f^n(0) = 0$ for some positive integer $ n$, then $ f(0) = 0$.


proof)  By hypothesis, $ | f^n(x) - f^n(y)| < |f^{n-1}(x) - f^{n-1}(y)| < \cdots < |x - y|$ for $ x \neq y$. In particular, $ |f^n(x)| = |f^n(x) - f^n(0)| < |x-0| = |x|$ for $ x \neq 0$.

Now assume $ f^n$ has a fixed point other than $ 0$, namely $ x_0$. Then $ |x_0| = |f^n(x_0)| < |x_0|$, a contradiction. Thus $ f^n$ has a unique fixed point.

Then $ f^n(f(0)) = f^{n+1}(0) = f(f^n(0)) = f(0)$, hence $ f(0) = 0$.
\end{solution}



\begin{solution}[by \href{https://artofproblemsolving.com/community/user/24822}{Svejk}]
	sos440  , I believe you are wrong when saying that 
"$ |f^{k + 1}(x) - f^{k + 1}(y)| < |f^k(x) - f^k(y)|$ for $ x\neq y$ "since if $ x\neq y$ it is not necessary that $ f^k(x) \neq f^k(y)$ so you cannot apply the property from our hypothesis.
\end{solution}



\begin{solution}[by \href{https://artofproblemsolving.com/community/user/27886}{sos440}]
	\begin{tcolorbox}sos440  , I believe you are wrong when saying that 
"$ |f^{k + 1}(x) - f^{k + 1}(y)| < |f^k(x) - f^k(y)|$ for $ x\neq y$ "since if $ x\neq y$ it is not necessary that $ f^k(x) \neq f^k(y)$ so you cannot apply the property from our hypothesis.\end{tcolorbox}

Ah, you're right. Hmm... maybe this would work:

... Exploiting hypothesis, $ |f(x) - f(y)| \leq |x - y|$ for all $ x, y$. Then for $ x \neq y$, $ |f^{n}(x) - f^{n}(y)| \leq |f^{n - 1}(x) - f^{n - 1}(y)| \leq \cdots \leq |f(x) - f(y)| < |x - y|$.
\end{solution}



\begin{solution}[by \href{https://artofproblemsolving.com/community/user/25546}{Yuriy Solovyov}]
	Let $ f(0)=x,\; f(x)=y$. Then $ f(y)=0$. Now using our condition:
$ |x-0|\geq|f(x)-f(0)|=|y-x|\geq|f(y)-f(x)|=|0-y|\geq|f(0)-f(y)|=|x-0|$. 
And this means $ x=y=0$, or $ f(0)=0$.
\end{solution}
*******************************************************************************
-------------------------------------------------------------------------------

\begin{problem}[Posted by \href{https://artofproblemsolving.com/community/user/79212}{jeag}]
	Let $f:\mathbb{R} \rightarrow \mathbb{R}$ be a continuous function such that for every real $x$,
\[f(x+f(x))=f(x).\]
Show that $f$ is constant.
	\flushright \href{https://artofproblemsolving.com/community/c7h347779}{(Link to AoPS)}
\end{problem}



\begin{solution}[by \href{https://artofproblemsolving.com/community/user/6542}{fedja}]
	Assume that it is not and find $x$ and $y$ such that $f(x)\ne f(y)$ but $\frac{x-y}{f(y)-f(x)}$ is a positive integer.
\end{solution}



\begin{solution}[by \href{https://artofproblemsolving.com/community/user/81769}{arshakus}]
	$ f:\mathbb{R}\rightarrow\mathbb{R} $
$ f(x+f(x))=f(x) $
$x=-f(x)=>f(0)=f(-f(x))=> \forall x f(-f(x)) $ is const!!!!!!!!
\end{solution}



\begin{solution}[by \href{https://artofproblemsolving.com/community/user/6542}{fedja}]
	Arshakus, can you understand what you wrote yourself? I should confess that I cannot understand a single word in your proof. :(
\end{solution}



\begin{solution}[by \href{https://artofproblemsolving.com/community/user/82533}{edooo}]
	sorry for interupting, but what means $confess$??
\end{solution}



\begin{solution}[by \href{https://artofproblemsolving.com/community/user/6542}{fedja}]
	Nowdays you can easily look unknown words in some [url=http://www.merriam-webster.com\/dictionary\/confess]online dictionary[\/url]. ;)
\end{solution}



\begin{solution}[by \href{https://artofproblemsolving.com/community/user/81769}{arshakus}]
	yes I understand what I wrote, and if you don't understand just ask for explanation, ok?
\end{solution}



\begin{solution}[by \href{https://artofproblemsolving.com/community/user/6542}{fedja}]
	OK, please, explain the very first step $x=-f(x)\implies f(0)=f(-f(x))$. What is $x$ here?
\end{solution}



\begin{solution}[by \href{https://artofproblemsolving.com/community/user/29428}{pco}]
	\begin{tcolorbox}yes I understand what I wrote, and if you don't understand just ask for explanation, ok?\end{tcolorbox}

I'm sorry, but what you wrote is completely wrong.

In $f(y+f(x))$, for example, you are allowed to choose $y$ such that $y=-f(x)$
But in $f(x+f(x))$, you have no proof that it exists a "$x$" such that $x=-f(x)$

And even if $x=-f(x)$ for some real values, then $f(0)=f(-f(x))$ would be true only for those values, not for all x

And, btw, even if $f(-f(x))=f(0)$ was true for any $x$, you could not conclude $f(x)=$constant (look, for example, at $f(x)=x+|x|$)
\end{solution}



\begin{solution}[by \href{https://artofproblemsolving.com/community/user/81769}{arshakus}]
	instead for $x$ I use $-f(x)$, if you don't understand it=> you can't solve such kind of problems!
\end{solution}



\begin{solution}[by \href{https://artofproblemsolving.com/community/user/81769}{arshakus}]
	explain me why I can't use $-f(x)$ instead of $x$??
\end{solution}



\begin{solution}[by \href{https://artofproblemsolving.com/community/user/29428}{pco}]
	\begin{tcolorbox}instead for $x$ I use $-f(x)$, \end{tcolorbox}

So the equation becomes  $f(-f(x)+f(-f(x)))=f(-f(x))$ and not $f(0)=f(-f(x))$

\begin{tcolorbox}if you don't understand it=> you can't solve such kind of problems!\end{tcolorbox}

You are surely right. I'm well known on this forum to be unable to solve functional equations  :rotfl:
\end{solution}



\begin{solution}[by \href{https://artofproblemsolving.com/community/user/6542}{fedja}]
	That is exactly what I was afraid of: Pco's critics completely missed the point because he criticised his own (very refined) interpretation of the text and what arshakus really had in mind was nowhere close.

@arshakus: If you use the identity $f(x+f(x))=f(x)$ for $-f(x)$ instead of $x$, you'll get $f(-f(x)+f(-f(x)))=f(-f(x))$ but not $f(0)=f(-f(x))$.
\end{solution}



\begin{solution}[by \href{https://artofproblemsolving.com/community/user/81769}{arshakus}]
	yes yes pco is right now I realized how I was wrong thank you boys for explanation, and good luck!
\end{solution}



\begin{solution}[by \href{https://artofproblemsolving.com/community/user/79212}{jeag}]
	It is easy to show inductively that for any integer $k \geq 0$.
$f(x+kf(x))=f(x)$. $(*)$
Now if we can find a positive integer $l$, and $x,y$, $f(x) \neq f(y)$ s.t.
$\frac{x-y}{f(y)-f(x)}=l$
then
$x+lf(x)=y+lf(y)$
and by $(*)$ we get
$f(x)=f(y)$, a contradiction.
but I don't see how to prove that such pair $x,y$ should exist if $f$ is not constant...
\end{solution}



\begin{solution}[by \href{https://artofproblemsolving.com/community/user/6542}{fedja}]
	Here is where continuity comes into play (and it has to be used somewhere because without it the statement is just false). 
[hide="Hint"]
Try to look at this difference as at a continuous function of $x$ and $y$. What can you say if it misses all positive integers?
[hide="Solution"]
Let's assume that the function takes 2 different positive values $a>b>0$. Then each of those values is attained on an arithmetic half-progression going to the right. In particular, we can choose $x>y$ with a huge difference $x-y$ such that $f(x)=b$ and $f(y)=a$. Now consider the ratio $\frac{x-y}{f(y)-f(x)}$ and start moving $x$ towards the closest point at which the value is $a$ (since we have a half-progression of such points starting at $y$, there is such a point not too far away). The numerator will stay large all the way while the denominator will approach $0$ from above, so the ratio will tend to $+\infty$. But it is rather hard to continuously escape to $+\infty$ and miss all positive integers on the way. A similar argument works for 2 distinct negative values but now the half-progressions go to the left. The rest should be clear.
[\/hide][\/hide]
\end{solution}



\begin{solution}[by \href{https://artofproblemsolving.com/community/user/29428}{pco}]
	\begin{tcolorbox}That is exactly what I was afraid of: Pco's critics completely missed the point because he criticised his own (very refined) interpretation of the text and what arshakus really had in mind was nowhere close.\end{tcolorbox}

I'm very sorry.
I just wanted to help but I'm not a professor as you are, so obviously my help was useless.

Since, in two variables functional equations, it's frequent to choose a value of one of the two variables in order to get an interesting equation, I really thought arshakus tried to make the argument of $f(x+f(x))$ be zero.

I'm sorry to have been the realization of your fears. 

 :blush:
\end{solution}



\begin{solution}[by \href{https://artofproblemsolving.com/community/user/6542}{fedja}]
	\begin{tcolorbox}
I'm sorry to have been the realization of your fears. 
\end{tcolorbox}
It wasn't your post that was "realization of my fears" but arshakus' reaction to it. What I learnt from dealing with students is that it is useless to try to interpret what they wrote in some reasonable way if what is written is completely unclear. More often than not, what was really meant is not what your best guess would be. :). It was clear from the beginning that the proof was wrong (for instance, because whatever it was, it was pure algebra and the problem statement is false without the continuity assumption) but it was completely unclear \begin{italicized}what exactly\end{italicized} it was and it is next to impossible to correctly pinpoint an error in an incomprehensible argument. You discussed three \begin{italicized}possible\end{italicized} errors (and did that perfectly well) but, as it turned out, the \begin{italicized}actual\end{italicized} error was different from all of them and was made at the level far below the one your interpretation would suggest.

\begin{tcolorbox}
I just wanted to help but I'm not a professor as you are, so obviously my help was useless.
\end{tcolorbox}
First, it wasn't "useless" at all and second, my being a professor has absolutely nothing to do with it. There have been many cases even on this forum where I tried to explain something to somebody without much success and then some high schooler interfered and gave an explanation that finally went through. So, you haven't done anything wrong.
\end{solution}
*******************************************************************************
-------------------------------------------------------------------------------

\begin{problem}[Posted by \href{https://artofproblemsolving.com/community/user/80321}{ahaanomegas}]
	Let $f(x)$ be a continuous function such that $f(2x^2-1)=2xf(x)$ for all $x$. Show that $f(x)=0$ for $-1\le x \le 1$.
	\flushright \href{https://artofproblemsolving.com/community/c7h429344}{(Link to AoPS)}
\end{problem}



\begin{solution}[by \href{https://artofproblemsolving.com/community/user/2948}{Kent Merryfield}]
	The first observation is this: for $x \ne 0, f(-x) = -f(x)$ since $f(x) =  \frac{f(2x^2 -1)}{2x}  .$ Since $f$ is continuous, it has a limit as $x \to 0$ but that limit is the negative of itself, hence $0.$ Thus $f(0) = 0$ and $f$ is an odd function. Now consider the mapping $\phi: [-1, 1] \to [-1, 1]$ given by $\phi(x) = 2x^2 - 1.$ Let $\phi^{(k)}$ be the $k$-th iterate of $\phi.$ That is, let $\phi^{(1)}(x) = \phi(x)$ and for $k \ge 1,$ let
$\phi^{(k+1)}(x) = \phi(\phi^{(k)}(x)).$ Suppose $x \ne 0$ is a point for which for some $k,$ $\phi^{(k)}(x) = 0.$ Assume WLOG that k is the smallest integer for which this is true for this particular x. Then
$0 = f(0) = f(\phi^{(k)}(x)) = 2\phi^{(k-1)}(x)f(\phi^{(k-1)}(x)) = \cdots =$ $2^k \phi^{(k-1)}(x) \phi^{(k-2)}(x) \cdots \phi(x)xf(x).$ Since none of the factors multiplying $f(x)$ are zero, we must have $f(x) = 0.$ If we can only prove that such $x$’s for which some iterate of $\phi$ of $x$ is zero are dense in $[-1, 1],$ then we would have that $f$ is identically zero, because a continuous function that is zero on a dense set is zero.

To this end, let $t$ be any number for which $\cos(t) = x.$ Then $\phi(x) = 2x^2 - 1= 2 \cos^2t - 1 = \cos(2t).$ By induction, $\phi^{(k)}(x) = \cos(2^kt).$ Consider the set $A$ of all numbers t of the form $t =  \frac{(m + \frac12 )\pi}{2^k}$   where $k$ is any positive integer and $m$ is any integer. For every member of $A,$ $\cos(2^kt) = 0$ for the particular $k$ involved in the construction. $A$ is dense in the line. (For each $k$ they are spaced $2^{-k}\pi$ apart, and $k$ may be arbitrarily large.) Let $B = \{x = \cos(t)\, |\, t \in A\}.$ Since cosine is a continuous function, $B$ must be dense in the range of cosine, namely $[-1, 1].$ But every point in $B$ is a point at which $\phi^{(k)}(x) = 0$ for some $k,$ hence a point at which $f(x) = 0.$
\end{solution}
*******************************************************************************
-------------------------------------------------------------------------------

\begin{problem}[Posted by \href{https://artofproblemsolving.com/community/user/64868}{mahanmath}]
	Find all functions $f: \mathbb N \to \mathbb N$ such that
\[f(n+f(m))=f(n)+m\]
holds for all $m,n \in \mathbb N$.
	\flushright \href{https://artofproblemsolving.com/community/q1h323817}{(Link to AoPS)}
\end{problem}



\begin{solution}[by \href{https://artofproblemsolving.com/community/user/44083}{jgnr}]
	Substitute $ n: =f(n)$, $ f(f(n)+f(m))=f(f(n))+m$. By symmetry of LHS, $ f(f(n))+m=f(f(m))+n$, so $ f(f(n))=n+k$, where $ k$ is a constant. It is well-known (and has been posted many times) that functions on natural numbers which satisfies $ f(f(n))=n+k$ has a unique solution $ f(n)=n+\frac{k}2$, where $ k$ is even. Let $ \frac{k}2=a$. Substitute this to the given equation, we get $ a=0$, so $ f(n)=n$ for all $ n$.
\end{solution}



\begin{solution}[by \href{https://artofproblemsolving.com/community/user/50028}{hophinhan}]
	\begin{tcolorbox}Find all function $ f: N\longrightarrow N$ such that :

$ f(n + f(m)) = f(n) + m$\end{tcolorbox}

Let $ P(x,y)$ be the assertion $ f(n + f(m)) = f(n) + m$

$ \blacksquare \ f(m_1) = f(m_2)$ . $ P(n,m_1)$ and $ P(n,m_2) \Longrightarrow m_1 = m_2$ . It's mean $ f(x)$ is an injective funtion.

$ P(0,0) \Longrightarrow f(f(0)) = f(0) \Longrightarrow f(0) = 0$

$ \blacksquare P(0,m) \Longrightarrow f(f(m)) = m \ \ \forall n\in\mathbb{N}$

$ P(n,f(m))\ \Longrightarrow f(n + f(f(m))) = f(n) + f(m)$
$ \Longrightarrow\ f(n + m) = f(n) + f(m) \ \ \forall m,n \in \mathbb{N}$

We get : $ f(n) = n \ \forall n\in \mathbb{N}$
\end{solution}



\begin{solution}[by \href{https://artofproblemsolving.com/community/user/44083}{jgnr}]
	\begin{tcolorbox}[quote="mahanmath"]Find all function $ f: N\longrightarrow N$ such that :

$ f(n + f(m)) = f(n) + m$\end{tcolorbox}

Let $ P(x,y)$ be the assertion $ f(n + f(m)) = f(n) + m$

$ \blacksquare \ f(m_1) = f(m_2)$ . $ P(n,m_1)$ and $ P(n,m_2) \Longrightarrow m_1 = m_2$ . It's mean $ f(x)$ is an injective funtion.

$ P(0,0) \Longrightarrow f(f(0)) = f(0) \Longrightarrow f(0) = 0$

$ \blacksquare P(0,m) \Longrightarrow f(f(m)) = m \ \ \forall n\in\mathbb{N}$

$ P(n,f(m))\ \Longrightarrow f(n + f(f(m))) = f(n) + f(m)$
$ \Longrightarrow\ f(n + m) = f(n) + f(m) \ \ \forall m,n \in \mathbb{N}$

We get : $ f(n) = n \ \forall n\in \mathbb{N}$\end{tcolorbox}N does not contain 0.

anyway, i haven't found the topic in which the function $ f(f(n))=n+k$ is solved, so my proof is not finished yet...
\end{solution}



\begin{solution}[by \href{https://artofproblemsolving.com/community/user/50028}{hophinhan}]
	\begin{tcolorbox}[quote="hophinhan"][quote="mahanmath"]Find all function $ f: N\longrightarrow N$ such that :

$ f(n + f(m)) = f(n) + m$\end{tcolorbox}

Let $ P(x,y)$ be the assertion $ f(n + f(m)) = f(n) + m$

$ \blacksquare \ f(m_1) = f(m_2)$ . $ P(n,m_1)$ and $ P(n,m_2) \Longrightarrow m_1 = m_2$ . It's mean $ f(x)$ is an injective funtion.

$ P(0,0) \Longrightarrow f(f(0)) = f(0) \Longrightarrow f(0) = 0$

$ \blacksquare P(0,m) \Longrightarrow f(f(m)) = m \ \ \forall n\in\mathbb{N}$

$ P(n,f(m))\ \Longrightarrow f(n + f(f(m))) = f(n) + f(m)$
$ \Longrightarrow\ f(n + m) = f(n) + f(m) \ \ \forall m,n \in \mathbb{N}$

We get : $ f(n) = n \ \forall n\in \mathbb{N}$\end{tcolorbox}N does not contain 0.

anyway, i haven't found the topic in which the function $ f(f(n)) = n + k$ is solved, so my proof is not finished yet...\end{tcolorbox}
 :rotfl:    "N does not contain 0"...
\end{solution}



\begin{solution}[by \href{https://artofproblemsolving.com/community/user/44083}{jgnr}]
	http://en.wikipedia.org\/wiki\/Natural_number
\end{solution}



\begin{solution}[by \href{https://artofproblemsolving.com/community/user/64868}{mahanmath}]
	\begin{bolded}Johan Gunardi \end{bolded}, You are right.$ 0 \notin {N}$ (At least in this problem  :P  !!)
\end{solution}



\begin{solution}[by \href{https://artofproblemsolving.com/community/user/62475}{hqthao}]
	dear Gunadi, 
when we have : $ f(f(n))=n+c$ (c:const and c must be nonnegative)
we have: change $ n$ by $ f(n)$ :  $ f(f(n)+f(m))=n+m+c$;
also, change $ m$ by $ f(m)$ : $ f(n+m+c)=f(n)+f(m) => f(f(n)+f(m))=f(f(n+m+c))=n+m+2c$
$ =>c=0
=> f(f(n))=n;
=> f( f(n)+f(m))=n+m$. in this, we change $ n$ by $ f(n)$ and m by $ f(m) => f(n+m)=f(n)+f(m)
=>f(n)=kn$ (k:const and k is positive)
$ =>f(f(n))=f(kn)=>n=k*k*n=>k=1
=>f(n)=n;$
dear Hophinhan, I think you right. 0 belongs to $ N$ (and the link that Gunardi give to us say that too :D) so I don't know why he say that 0 not belongs to $ N$. 0 not belongs to $ N*$. but of course, the problem we had solved more beutiful :D
\end{solution}



\begin{solution}[by \href{https://artofproblemsolving.com/community/user/29428}{pco}]
	Once again, once again, once again. For all members who did not walk thru these forum before posting their comments :

$ 0\in\mathbb N$ in some countries (mine, for example)
$ 0\notin\mathbb N$ in some other countries (more, I think).

The current situation in mathlinks forums is to consider $ 0\notin\mathbb N$.

The best thing would be that posters give the precision when they use $ \mathbb N$ in their problems.
\end{solution}



\begin{solution}[by \href{https://artofproblemsolving.com/community/user/71459}{x164}]
	Or just stop using $ \mathbb{N}$ and use $ \mathbb{Z}_{ > 0}$ and $ \mathbb{Z}_{\geq 0}$ :D
\end{solution}
*******************************************************************************
-------------------------------------------------------------------------------

\begin{problem}[Posted by \href{https://artofproblemsolving.com/community/user/76369}{peter117}]
	Find all functions $ f : \mathbb R\to \mathbb R$ such that
\[ x^2y^2 \left( f(x+y)-f(x)-f(y) \right)=3(x+y)f(x)f(y)\]
holds for all reals $x$ and $y$.
	\flushright \href{https://artofproblemsolving.com/community/q1h337211}{(Link to AoPS)}
\end{problem}



\begin{solution}[by \href{https://artofproblemsolving.com/community/user/29428}{pco}]
	\begin{tcolorbox}Find all $ f : R\to R$ such that
$ x^2y^2[f(x + y) - f(x) - f(y)] = 3(x + y)f(x)f(y)$\end{tcolorbox}
I'm not in a good period and I find only ugly proofs. Sorry.
I would be interested in yours, peter117, thanks :)

Let $ P(x,y)$ be the assertion $ x^2y^2(f(x + y) - f(x) - f(y)) = 3(x + y)f(x)f(y)$

$ P(x,0)$ $ \implies$ $ xf(x)f(0) = 0$ $ \forall x$

If $ f(0)\ne 0$, this implies $ f(x) = 0\forall x\ne 0$ and then $ P(1, - 1)$ implies $ f(0) = 0$ and we got $ f(x) = 0$ $ \forall x$, which indeed is a solution.
So we'll from now consider $ f(0) = 0$ and $ f(x)$ not all zero.

1) $ f(x)$ is a quotient of polynomials in $ x$ for all $ x$ but a finite set
============================================
The original equation implies $ f(x + y) = f(x) + f(y) + 3\frac {x + y}{x^2y^2}f(x)f(y)$ $ \forall x,y\ne 0$

this allow computation of $ f(x + y + z)$ in two ways :
Considering $ x,y,z,y + z\ne 0$ :

$ f(x + (y + z)) = f(x) + f(y + z) + 3\frac {x + y + z}{x^2(y + z)^2}f(x)f(y + z)$ 
$ f(x + (y + z)) = f(x) + f(y) + f(z)$ $ + 3\frac {y + z}{y^2z^2}f(y)f(z)$ $ + 3\frac {x + y + z}{x^2(y + z)^2}f(x)f(y)$ $ + 3\frac {x + y + z}{x^2(y + z)^2}f(x)f(z)$ $ + 9\frac {x + y + z}{x^2y^2z^2(y + z)}f(x)f(y)f(z)$

Same, considering $ x,y,z,x + y\ne 0$ :

$ f(y + (x + z)) = f(y) + f(x + z) + 3\frac {y + x + z}{y^2(x + z)^2}f(y)f(x + z)$ 
$ f(y + (x + z)) = f(y) + f(x) + f(z)$ $ + 3\frac {x + z}{x^2z^2}f(x)f(z)$ $ + 3\frac {y + x + z}{y^2(x + z)^2}f(y)f(x)$ $ + 3\frac {y + x + z}{y^2(x + z)^2}f(y)f(z)$ $ + 9\frac {y + x + z}{y^2x^2z^2(x + z)}f(y)f(x)f(z)$

Comparing these two expressions, we get $ \forall x,y,z,x + y,y + z\ne 0$:

$ h_1(x,y,z)f(x)f(y) +$ $ h_2(x,y,z)f(x)f(z) +$ $ h_3(x,y,z)f(y)f(z) +$ $ h_4(x,y,z)f(x)f(y)f(z) = 0$
Where :

$ h_1(x,y,z) = 3\frac {x + y + z}{x^2(y + z)^2} - 3\frac {y + x + z}{y^2(x + z)^2}$

$ h_2(x,y,z) = 3\frac {x + y + z}{x^2(y + z)^2} - 3\frac {x + z}{x^2z^2}$

$ h_3(x,y,z) = 3\frac {y + z}{y^2z^2} - 3\frac {y + x + z}{y^2(x + z)^2}$

$ h_4(x,y,z) = 9\frac {x + y + z}{x^2y^2z^2(y + z)} - 9\frac {y + x + z}{y^2x^2z^2(x + z)}$

This result may be written $ f(x)((h_1 + h_4f(z))f(y) + h_2f(z)) = - h_3f(y)f(z)$

$ h_4$ is not zero for all $ z$ so :
either $ h_1 + h_4f(z) = 0$ $ \forall z$ but a finite set and so $ f(z) = - \frac {h_1}{h_4}$ and the result is achieved
either $ h_1 + h_4f(z_0)\ne 0$ for some $ z_0$ and then :

either $ (h_1 + h_4f(z_0))f(y) + h_2f(z_0) = 0$ $ \forall y$ but a finite set and so $ f(y) = - \frac {h_2f(z_0)}{h_1 + h_4f(z_0)}$ and the result is achieved
either $ (h_1 + h_4f(z_0))f(y_0) + h_2f(z_0)\ne 0$ for some $ y_0$ and then $ f(x) = - \frac {h_3f(y)f(z)}{(h_1 + h_4f(z_0))f(y_0) + h_2f(z_0)}$ and the result is achieved

2) $ f(x) = x^3$ $ \forall x$
================
From 1;, we get $ f(x) = \frac {P(x)}{Q(x)}$ $ \forall x$ but a finite set, with $ P(x)$ and $ Q(x)$ having no common root.

Let $ y$ fixed (not in the finite set). The original equation becomes :
$ x^2y^2(P(x + y)Q(x)Q(y)$ $ - P(x)Q(y)Q(x + y) - P(y)Q(x)Q(x + y))$ $ = 3(x + y)P(x)P(y)Q(x + y)$ $ \forall x$ but a finite set.

And since $ P,Q$ are polynomial, this equality is true $ \forall x$ And so $ \forall x,y$

Let then $ x = z_i$ any complex root of $ P(x)$. This equation becomes :
$ z_i^2y^2(P(z_i + y)Q(z_i)Q(y) - P(y)Q(z_i)Q(z_i + y)) = 0$ $ \forall y$ and so $ z_i = 0$ (else $ \frac {P(x)}{Q(x)}$ would be periodic with period $ z_i$).

So $ P(x) = ax^n$

Let then $ y = w_i - x$ where $ w_i$ is any complex root of $ Q(x)$. This equation becomes : $ x^2(w_i - x)^2Q(x)Q(w_i - x) = 0$ which is impossible.

So $ Q(x) = b$

And $ f(x) = cx^n$ with $ c\ne 0$

Plugging this in the original equation and comparing the terms in $ x^{n + 1}$ both sides, we get $ n = 3$ and it is easy to check that this indeed is a solution.

3) Synthesis of solutions :
=================
$ f(x) = 0$ $ \forall x$
$ f(x) = x^3$ $ \forall x$
\end{solution}



\begin{solution}[by \href{https://artofproblemsolving.com/community/user/29428}{pco}]
	\begin{tcolorbox}Find all $ f : R\to R$ such that
$ x^2y^2[f(x + y) - f(x) - f(y)] = 3(x + y)f(x)f(y)$\end{tcolorbox}

Second solution, prettier :)

Let $ P(x,y)$ be the assertion $ x^2y^2(f(x + y) - f(x) - f(y)) = 3(x + y)f(x)f(y)$

$ P(x,0)$ $ \implies$ $ xf(x)f(0) = 0$ $ \forall x$

If $ f(0)\ne 0$, this implies $ f(x) = 0\forall x\ne 0$ and then $ P(1, - 1)$ implies $ f(0) = 0$ and we got $ f(x) = 0$ $ \forall x$, which indeed is a solution.
So we'll from now consider $ f(0) = 0$ and $ f(x)$ not all zero.

$ P(x, - x)$ $ \implies$ $ f( - x) = - f(x)$ $ \forall x\ne 0$ and so $ f(x) = f( - x)$ $ \forall x$
Let $ f(1) = a$

$ P(1,1)$ $ \implies$ $ f(2) = 6a^2 + 2a$
$ P(2, - 1)$ $ \implies$ $ 4(2a - f(2)) = - 3af(2)$ and so $ f(2)(3a - 4) = - 8a$

So $ (6a^2 + 2a)(3a - 4) = - 8a$ $ \iff$ $ a^2(a - 1) = 0$

If $ a = 0$, then :
$ P(x,1)$ $ \implies$ $ x^2(f(x + 1) - f(x)) = 0$ and so $ f(x + 1) = f(x)$ $ \forall x\ne 0$ and so $ f(x + 1) = f(x)$ $ \forall x$
$ P(x + 1,y)$ $ \implies$ $ (x + 1)^2y^2(f(x + y) - f(x) - f(y)) = 3(x + y + 1)f(x)f(y)$
Subtracting $ P(x,y)$ : $ (2x + 1)y^2(f(x + y) - f(x) - f(y)) = 3f(x)f(y)$

And so $ (x + y)(2x + 1)f(x)f(y) = x^2f(x)f(y)$ $ \forall x,y$ from which it is easy to deduce $ f(x) = 0$ $ \forall x$

So $ f(1) = 1$ and $ f(2) = 8$

$ P(x,1)$ $ \implies$ $ x^2(f(x + 1) - f(x) - 1) = 3(x + 1)f(x)$ and so $ f(x + 1) = \frac {x^2 + 3x + 3}{x^2}f(x) + 1$ $ \forall x\ne 0$

$ P(x + 1,1)$ $ \implies$ $ f(x + 2) = \frac {x^2 + 5x + 7}{(x + 1)^2}f(x + 1) + 1$ $ = \frac {x^2 + 5x + 7}{(x + 1)^2}$ $ \frac {x^2 + 3x + 3}{x^2}f(x) + \frac {x^2 + 5x + 7}{(x + 1)^2} + 1$ $ \forall x\notin\{ - 1,0\}$

$ P(x,2)$ $ \implies$ $ f(x + 2) = \frac {x^2 + 6x + 12}{x^2}f(x) + 8$ $ \forall x\ne 0$

And so $ \frac {x^2 + 5x + 7}{(x + 1)^2}$ $ \frac {x^2 + 3x + 3}{x^2}f(x) + \frac {x^2 + 5x + 7}{(x + 1)^2} + 1 = \frac {x^2 + 6x + 12}{x^2}f(x) + 8$

So $ (x^2 + 5x + 7)(x^2 + 3x + 3)f(x)$ $ + (x^2 + 5x + 7)x^2 + x^2(x + 1)^2$ $ = (x^2 + 6x + 12)(x + 1)^2f(x)$ $ + 8x^2(x + 1)^2$

And simplifying this, we get $ (6x + 9)f(x) = x^3(6x + 9)$ and so $ f(x) = x^3$ $ \forall x\notin\{ - \frac 32, - 1, 0\}$

We already know that $ f(0) = 0$ and $ f(1) = 1$ and $ P( - 1, - \frac 12)$ gives $ f( - \frac 32) = - \frac {27}8$

And so $ \boxed{f(x) = x^3}$ $ \forall x$ which indeed is a solution.
\end{solution}



\begin{solution}[by \href{https://artofproblemsolving.com/community/user/30342}{nicetry007}]
	$ x^2y^2[f(x + y) - f(x) - f(y)] = 3(x + y)f(x)f(y) \;\;\forall x,y \in \mathbb{R}$
We start by investigating constant solutions. Let $ f \equiv c$.
Setting $ x = - y \neq 0$, we get $ - cx^4 = 0 \Rightarrow c = 0$.
Hence, the only constant solution is $ f \equiv 0$.
Suppose $ f \not\equiv 0$ and $ \exists y_0$ such that $ f(y_0) \neq 0$.
Setting $ x = 0$, we get $ 3yf(y)f(0) = 0 \;\;\forall y \in \mathbb{R}$.
 We either have (i) $ f(0) \neq 0$ and $ f(y) = 0 \;\;\forall y\neq 0$ or (ii) $ f(0) = 0$ and $ y_0 \neq 0$.
Suppose (i) holds. Setting $ x = - y \neq 0$, we get
              $ x^4f(0) = 0 \Rightarrow f(0) = 0$. 
This implies $ f \equiv 0$ which contradicts $ f \not\equiv 0$. Hence, $ f(0) = 0$ and $ y_0 \neq 0$.
Setting $ x = y = y_0$, we get
           $ y_0^4[f(2y_0) - 2f(y_0)] = 6y_0f(y_0)^2 \neq 0 \Rightarrow f(2y_0) - 2f(y_0) \neq 0$.
Suppose $ \exists x_0 \neq 0$ such that $ f(x_0) = 0$. Setting $ x = x_0$, we get
           $ x_0^2y^2[f(x_0 + y) - f(y)] = 0 \Rightarrow f(y + x_0) = f(y) \;\;\forall y \neq 0$
Since $ f(x_0) = f(0) = 0$, we have $ f(y + x_0) = f(y) \;\;\forall y \in \mathbb{R} \Rightarrow f$ is either constant or periodic with period $ T > 0$. 
Since $ f$ is not a contant function, $ f$ has to be periodic.
Setting $ x = y_0 - kT, y = y_0 + kT$, we get
$ (y_0 - kT)^2(y_0 + kT)^2 = \frac {6y_0f(y_0)^2}{f(2y_0) - 2f(y_0)} = \alpha \neq 0 \;\;\forall k \in \mathbb{Z}$
$ \Rightarrow (y_0^2 - k^2T^2)^2 = \alpha \neq 0 \;\;\forall k \in \mathbb{Z} \Rightarrow T = 0$
This contradicts $ T > 0$. Hence $ f(x) \neq 0 \;\;\forall x \neq 0$.

Now comes the nice part.

Setting $ y = y + z$, we get
$ x^2(y + z)^2[f(x + y + z) - f(x) - f(y + z)] = 3(x + y + z)f(x)f(y + z)$
Setting $ x + y + z = 0$, we get  $ x^2(y + z)^2[ - f(x) - f(y + z)] = 0$.
We let $ x,y,z \neq 0$. This implies $ x + y,y + z,z + x \neq 0$.
Therefore, $ f(x) + f(y + z) = 0 \;\;\forall x,y,z \in \mathbb{R}\setminus {0}$ and $ x + y + z = 0$.
Expanding $ f(y + z)$, we get $ f(x) + f(y) + f(z) = - \frac {3(y + z)f(y)f(z)}{y^2z^2} = - \frac {3(x + z)f(x)f(z)}{x^2z^2}$
Replacing $ y + z$ by $ - x$ and $ x + z$ by $ - y$ and cancelling $ z^2$ and $ f(z) (\neq 0)$ from the last two expressions, we get
$ \frac {3xf(y)}{y^2} = \frac {3yf(x)}{x^2}$.
Setting $ y = 1$, we get $ f(x) = cx^3 \;\;\forall x \neq - 1,0 \Rightarrow f(2) = 8c$.
Setting $ y = 2$, we get $ f( - 1) = - c$. Since $ f(0) = 0 = 0\cdot c$, we have $ f(x) = cx^3 \;\;\forall x$.
Plugging $ f(x) = cx^3$ and $ x = y = 1$, we get $ c = 0$ or $ 1$.  Since $ f \not\equiv 0$, $ c = 1$ and $ f(x) = x^3$.
\end{solution}



\begin{solution}[by \href{https://artofproblemsolving.com/community/user/184652}{CanVQ}]
	\begin{tcolorbox}Find all functions $ f : \mathbb R\to \mathbb R$ such that
\[ x^2y^2 \left( f(x+y)-f(x)-f(y) \right)=3(x+y)f(x)f(y)\quad (1)\]\end{tcolorbox}
It is clear that $f(x) \equiv 0$ is a solution. Let us consider the case $f(x) \not \equiv 0,$ then there exists $x_0$ such that $f(x_0) \ne 0.$ Replacing $x=x_0$ and $y=0$ in $(1),$ we get $f(0)=0.$ Now, substituting $y=-x \ne 0$ in $(1)$ and combining with $f(0)=0,$ we get \[f({-x})=-f(x),\quad \forall x \in \mathbb R. \quad (2)\] Replacing $y=x \ne 0$ in $(1),$ we have \[f(2x)=\frac{6\cdot f(x)}{x^3}+2\cdot f(x),\quad \forall x \ne 0. \quad (3)\] On the other hand, replacing $x$ by $2x \ne 0$ and $y=-x$ in $(1),$ we also have \[f(x)-f(2x)-f({-x})=\frac{3\cdot f(2x)\cdot f(-x)}{4x^3},\] or \[2\cdot f(x)+\left[ \frac{3\cdot f(x)}{4x^3}-1\right] \cdot f(2x)=0,\quad \forall x \ne 0. \quad (4)\] Plugging $(3)$ into $(4),$ we get \[2\cdot f(x)+ \left[\frac{3\cdot f(x)}{4x^3}-1\right] \left[\frac{6\cdot f(x)}{x^3}+2\cdot f(x)\right] =0,\quad \forall x \ne 0.\] From this, we can easily deduce that \[f(x)=0 \vee f(x)=x^3,\quad \forall x \ne 0.\] Assume that there are $a,\,b \ne 0$ such that $f(a)=0$ and $f(b)=b^3.$ Replacing $x=a$ and $y=b$ in $(1),$ we get \[f(a+b)=f(a)+f(b)=b^3 \ne 0,\] so $f(a+b)=(a+b)^3$ and \[(a+b)^3=b^3.\] It follows that $a=0,$ a contradiction. So we must have $f(x)=0,\, \forall x \in \mathbb R$ or $f(x)=x^3,\, \forall x \in \mathbb R.$ Since $f \not \equiv 0,$ we have $f(x)=x^3,\, \forall x \in \mathbb R.$ This function satisfies our condition.

In conclusion, there are two solutions: $f(x)=0$ and $f(x)=x^3.$
\end{solution}
*******************************************************************************
-------------------------------------------------------------------------------

\begin{problem}[Posted by \href{https://artofproblemsolving.com/community/user/56439}{srinath.r}]
	Let $ f(x)=x^{4} + ax^{3} + bx^{2} + cx + d$, where $a,b,c$, and $d$ are real numbers. Find the conditions on $ a,b,c,d$ in order $ f(x)$ may be written as $ f(x)=(x^{2} + px + q )^{2}$ for some reals $p$ and $q$.
	\flushright \href{https://artofproblemsolving.com/community/q1h344373}{(Link to AoPS)}
\end{problem}



\begin{solution}[by \href{https://artofproblemsolving.com/community/user/29428}{pco}]
	So, I dont know what was exactly the problem OP found in his book or exam sheet, but : \begin{tcolorbox}Find the condition for the expression $ f(x) = ax^{4} + bx^{3} + cx + d$ to be a perfect square -(by perfect square , I mean $ f(x) = (g(x))^{2}$ for some $ g(x)$  ) \end{tcolorbox}

\begin{tcolorbox}Oh , sorry then , but the original expression should be $ x^{4} + ax^{3} + bx^{2} + cx + d$ , we can have it for real $ x$ , something like $ x^{4} + ax^{3} + bx^{2} + cx + d = (x^{2} + px + q )^{2}$ , then finding conditions . All the coefficients are real\end{tcolorbox}

So the problem is "Let $ f(x)=x^{4} + ax^{3} + bx^{2} + cx + d$. Find the conditions on $ a,b,c,d$ in order $ f(x)$ may be written $ f(x)=(x^{2} + px + q )^{2}$"

So $ x^{4} + ax^{3} + bx^{2} + cx + d$ $ =x^4+2px^3+(p^2+2q)x^2+2pqx+q^2$

From $ a=2p$ and $ b=p^2+2q$, we get $ p=\frac a2$ and $ q=\frac{4b-a^2}8$

Then $ c=2pq$ and $ d=q^2$ imply the equations $ c=\frac{4ab-a^3}8$ and $ d=\frac{(4b-a^2)^2}{64}$

Hence the required result : $ \boxed{4ab-a^3-8c=0\text{  and  }(4b-a^2)^2-64d=0}$
\end{solution}
*******************************************************************************
-------------------------------------------------------------------------------

\begin{problem}[Posted by \href{https://artofproblemsolving.com/community/user/85501}{taylorcute56}]
	Let $f(x)$ be a continuous function from $\mathbb R$ to $\mathbb R$ such that $f(4)=3$ and \[f(x) \cdot f(f(f(f(x))))=1\] 
 for all $x\in\mathbb R$. Find $f(2)$.
	\flushright \href{https://artofproblemsolving.com/community/q1h358421}{(Link to AoPS)}
\end{problem}



\begin{solution}[by \href{https://artofproblemsolving.com/community/user/29428}{pco}]
	So the real original problem is :

Let $f(x)$ be a continuous function from $\mathbb R\to\mathbb R$ such that $f(4)=3$ and $f(x).f(f(f(f(x))))=1$ $\forall x\in\mathbb R$
Find $f(2)$

From $f(x).f(f(f(f(x))))=1$, we can conclude that $0\notin f(\mathbb R)$ and $u\in f(\mathbb R)\implies \frac 1u\in f(\mathbb R)$

Since $f(4)=3$, we get $3\in f(\mathbb R)$ and so $\frac 13\in f(\mathbb R)$ and so $[\frac 13,3]\subseteq f(\mathbb R)\subseteq(0,+\infty)$ (using continuity)

From $f(x).f(f(f(f(x))))=1$, we get $f(f(f(x)))=\frac 1x$ $\forall x\in f(\mathbb R)$

So $f_r(x)$, restriction of $f(x)$ from $f(\mathbb R)\to f(\mathbb R)$ is a continuous bijection, and so is monotonous decreasing.

Let us from now consider only $f_r(x)$ :
We got $f_r(f_r(f_r(x)))=\frac 1x$ and $f_r(x).f_r(\frac 1x)=1$

From this, we get $f_r(1)=1$
Let then $A=(-\infty,1]\cap f(\mathbb R)$ and $B=f(\mathbb R)\cap[1,+\infty)$

We can define two continuous decreasing functions:
$g(x)$ from $A\to B$ as $g(x)=f_r(x)$
$h(x)$ from $B\to A$ as $h(x)=f_r(x)$

From $f_r(x).f_r(\frac 1x)=1$, we get $h(x)=\frac 1{g(\frac 1x)}$ $\forall x\in B$

From $f_r(f_r(f_r(x)))=\frac 1x$, we get now $g(\frac 1{g(\frac 1{g(\frac 1x)})}=x$ $\forall x\in B$

Let then $k(x)=g(\frac 1x)$ continuous bijection from $B\to B$. We got $k(k(k(x)))=x$ and so $k(x)=x$

Hence $f_r(x)=\frac 1x$ and so $f(2)=\frac 12$ since $2\in[\frac 13,3]\subseteq f(\mathbb R)$
\end{solution}
*******************************************************************************
-------------------------------------------------------------------------------

\begin{problem}[Posted by \href{https://artofproblemsolving.com/community/user/92753}{WakeUp}]
	Find all functions $f:\mathbb{Q}^{+}\rightarrow \mathbb{Q}^{+}$ which for all $x \in \mathbb{Q}^{+}$ fulfil
\[f\left(\frac{1}{x}\right)=f(x) \ \ \text{and} \ \ \left(1+\frac{1}{x}\right)f(x)=f(x+1). \]
	\flushright \href{https://artofproblemsolving.com/community/q1h376120}{(Link to AoPS)}
\end{problem}



\begin{solution}[by \href{https://artofproblemsolving.com/community/user/29428}{pco}]
	\begin{tcolorbox}Find all functions $f:\mathbb{Q}^{+}\rightarrow \mathbb{Q}^{+}$ which for all $x \in \mathbb{Q}^{+}$ fulfil
\[f\left(\frac{1}{x}\right)=f(x) \ \ \text{and} \ \ \left(1+\frac{1}{x}\right)f(x)=f(x+1). \]\end{tcolorbox}
Let $p,q\in\mathbb N$ such that $\gcd(p,q)=1$. Let then $h(\frac pq)=\frac{f(\frac pq)}{pq}$ 

The first equation implies $h(x)=h(\frac 1x)$ and the second equation implies $h(x+1)=h(x)$

And so, $h([a_1;a_2,a_3,...,a_n])$ $=h([0;a_2,a_3,...,a_n])$ $=h([a_2;a_3,...,a_n])$ $=...=h(a_n)=h(1)$

Hence the solution : $\boxed{f(\frac pq)=a\frac{pq}{\gcd(p,q)^2}}$ $\forall p,q\in\mathbb N$ and for any $a\in\mathbb Q^+$
\end{solution}



\begin{solution}[by \href{https://artofproblemsolving.com/community/user/92964}{dyta}]
	\begin{tcolorbox}And so, $h([a_1;a_2,a_3,...,a_n])$ $=h([0;a_2,a_3,...,a_n])$ $=h([a_2;a_3,...,a_n])$ $=...=h(a_n)=h(1)$\end{tcolorbox}

What is $a_1,a_2,...$ and [a_1;a_2,a_3,...,a_n],please? Are they integer numbers?
\end{solution}



\begin{solution}[by \href{https://artofproblemsolving.com/community/user/29428}{pco}]
	\begin{tcolorbox}[quote="pco"]And so, $h([a_1;a_2,a_3,...,a_n])$ $=h([0;a_2,a_3,...,a_n])$ $=h([a_2;a_3,...,a_n])$ $=...=h(a_n)=h(1)$\end{tcolorbox}

What is $a_1,a_2,...$ and [a_1;a_2,a_3,...,a_n],please? Are they integer numbers?\end{tcolorbox}


$a_i$ all are positive integers and $[a_1;a_2,...,a_n]$ is one classical representation of the continued fraction of a rational.
see http://en.wikipedia.org\/wiki\/Continued_fraction
\end{solution}
*******************************************************************************
-------------------------------------------------------------------------------

\begin{problem}[Posted by \href{https://artofproblemsolving.com/community/user/92753}{WakeUp}]
	Determine all functions $f:\mathbb{R}\rightarrow\mathbb{R}$ such that
\[f(x+yf(x))=f(x)+xf(y) \quad \text{for all}\ x,y \in\mathbb{R}\]
	\flushright \href{https://artofproblemsolving.com/community/q1h376448}{(Link to AoPS)}
\end{problem}



\begin{solution}[by \href{https://artofproblemsolving.com/community/user/29428}{pco}]
	\begin{tcolorbox}Determine all functions $f:\mathbb{R}\rightarrow\mathbb{R}$ such that
\[f(x+yf(x))=f(x)+xf(y) \quad \text{for all}\ x,y \in\mathbb{R}\]\end{tcolorbox}
http://www.artofproblemsolving.com/Forum/viewtopic.php?f=36&t=301766
\end{solution}
*******************************************************************************
-------------------------------------------------------------------------------

\begin{problem}[Posted by \href{https://artofproblemsolving.com/community/user/49928}{alex2008}]
	Prove that any continuos function $ f: \mathbb{R}\rightarrow \mathbb{R}$ with 

\[ f(x)=\left\{ \begin{aligned} a_1x+b_1\ ,\ \text{for } x\le 1  \\
a_2x+b_2\ ,\ \text{for } x>1 \end{aligned} \right.\]

where $ a_1,a_2,b_1,b_2\in \mathbb{R}$, can be written as:

\[ f(x)=m_1x+n_1+\epsilon|m_2x+n_2|\ ,\ \text{for } x\in \mathbb{R}\]

where $ m_1,m_2,n_1,n_2\in \mathbb{R}$ and $ \epsilon\in \{-1,+1\}$.
	\flushright \href{https://artofproblemsolving.com/community/c6h338993}{(Link to AoPS)}
\end{problem}



\begin{solution}[by \href{https://artofproblemsolving.com/community/user/29428}{pco}]
	\begin{tcolorbox}Prove that any continuos function $ f: \mathbb{R}\rightarrow \mathbb{R}$ with
\[ f(x) = \left\{\begin{aligned} a_1x + b_1\ ,\ \text{for } x\le 1 \\
a_2x + b_2\ ,\ \text{for } x > 1 \end{aligned} \right.\]
where $ a_1,a_2,b_1,b_2\in \mathbb{R}$, can be written as:
\[ f(x) = m_1x + n_1 + \epsilon|m_2x + n_2|\ ,\ \text{for } x\in \mathbb{R}\]
where $ m_1,m_2,n_1,n_2\in \mathbb{R}$ and $ \epsilon\in \{ - 1, + 1\}$.\end{tcolorbox}

Notice that $ a_1+b_1=a_2+b_2$ (continuity at $ x=1$). Then :

If $ a_1=a_2=a$, $ f(x)=ax+a+|0x+0|$

If $ a_1>a_2$, $ f(x)=\frac{a_1+a_2}2x+\frac{b_1+b_2}2-|\frac{a_2-a_1}2x+\frac{a_1-a_2}2|$

If $ a_1<a_2$, $ f(x)=\frac{a_1+a_2}2x+\frac{b_1+b_2}2+|\frac{a_2-a_1}2x+\frac{a_1-a_2}2|$
\end{solution}



\begin{solution}[by \href{https://artofproblemsolving.com/community/user/8638}{me@home}]
	[hide="Solution"]
\begin{align*}f(x)&=\frac{a_1+a_2}{2}x+\frac{b_1+b_2}{2}+\left\{\begin{aligned}\frac{a_1-a_2}{2}x+\frac{b_1-b_2}{2}&\qquad x\leq 1\\\frac{a_2-a_1}{2}x+\frac{b_2-b_1}{2}&\qquad x\geq1\end{aligned}\right.\\
&=\frac{a_1+a_2}{2}x+\frac{b_1+b_2}{2}\pm\frac{a_2-a_1}{2}|x-1|\qquad(\text{by continuity})\\
\end{align*}
The result is trivially in the desired form.
[\/hide]
\end{solution}
*******************************************************************************
-------------------------------------------------------------------------------

\begin{problem}[Posted by \href{https://artofproblemsolving.com/community/user/57963}{saeedghodsi}]
	Find all functions $ f: \mathbb{R}^{+}\rightarrow\mathbb{R}^{+}$ such that for each $ x,y\in\mathbb{R}^{+}$,
\[f\left(\frac{2xy}{x+y}\right)= \frac{2f(x)f(y)}{f(x)+f(y)}.\]
	\flushright \href{https://artofproblemsolving.com/community/c6h339196}{(Link to AoPS)}
\end{problem}



\begin{solution}[by \href{https://artofproblemsolving.com/community/user/29428}{pco}]
	\begin{tcolorbox}Find all functions $ f: \mathbb{R}^{ + }\rightarrow\mathbb{R}^{ + }$ such that for each $ x,y\in\mathbb{R}^{ + }$:
$ f(\frac {2xy}{x + y}) = \frac {2f(x)f(y)}{f(x) + f(y)}$\end{tcolorbox}

Let $ h(x)=\frac 1{f(\frac 1x)}$ and the equation becomes $ h(\frac {x+y}2)=\frac{h(x)+h(y)}2$ with $ h(x)$ from $ \mathbb R^+\to\mathbb R^+$

For $ \Delta>0$, and setting $ y=x+2\Delta$ in this equality, we get $ h(x+2\Delta)-h(x+\Delta)=h(x+\Delta)-h(x)$ and so $ h(x+\Delta)\ge h(x)$ and $ h(x)$ is an increasing function, else, for some $ p$, $ h(x+p\Delta)<0$

And $ h(\frac {x+y}2)=\frac{h(x)+h(y)}2$ PLUS $ h(x)$ increasing PLUS $ h(x)$ from $ \mathbb R^+\to\mathbb R^+$implies  $ h(x)=bx+a$ with $ a,b\ge 0$ and $ a+b>0$

And so $ \boxed{f(x)=\frac{x}{ax+b}}$ for some $ a,b\ge 0$ and $ a+b>0$ and it is easy to check back that this is indeed a solution.
\end{solution}
*******************************************************************************
-------------------------------------------------------------------------------

\begin{problem}[Posted by \href{https://artofproblemsolving.com/community/user/73589}{mathmen}]
	Find all continuous functions $f: \mathbb R \to \mathbb R$ such that for all reals $x$ and $y$,
\[ f({x}^{2}+{y}^{2})=f({x}^{2}-{y}^{2})+f(2xy).\]
	\flushright \href{https://artofproblemsolving.com/community/c6h339434}{(Link to AoPS)}
\end{problem}



\begin{solution}[by \href{https://artofproblemsolving.com/community/user/29428}{pco}]
	\begin{tcolorbox}find all function such that :
1)$ lim_{x \to x_0}f(x) = f(x_0)$($ x_0 \in \mathbb{R}$)
2)$ f({x}^{2} + {y}^{2}) = f({x}^{2} - {y}^{2}) + f(2xy)$
for all $ x,y \in \mathbb{R}$\end{tcolorbox}

\begin{italicized}I always wonder how could a poster post in "unsolved" category (which means he did not succeed in solving the problem and doest not know tthe solution) and, at the same time, claim that the problem is "very easy", or "very nice" or any similar thing \end{italicized} :huh: 

Let $ P(x,y)$ be the assertion $ f(x^2 + y^2) = f(x^2 - y^2) + f(2xy)$

$ P(0,0)$ $ \implies$ $ f(0) = 0$
$ P(0,x)$ $ \implies$ $ f(x^2) = f( - x^2)$ and so $ f(x) = f( - x)$ $ \forall x$

Let $ x\ge y\ge 0$ : $ P(\sqrt {\frac {x + y}2},\sqrt {\frac {x - y}2})$ $ \implies$ $ f(x) = f(y) + f(\sqrt {x^2 - y^2})$

let $ g(x) = f(\sqrt {x})$ $ \forall x\ge 0$ and $ g(x) = -f(\sqrt {-x})$ $ \forall x\le 0$ such that $ g(x)$ is an odd function and $ f(x) = g(x^2)$ $ \forall x$. 

The last equation becomes $ g(x^2) = g(y^2) + g(x^2 - y^2)$ $ \forall x\ge y\ge 0$

And so, since $ g(x)$ is an odd function :  $ g(u + v) = g(u) + g(v)$ $ \forall u,v$
Since $ f(x)$ is continuous at $ x_0$, $ g(x)$ is continuous at $ \sqrt {|x_0|}$ and so continuous on $ \mathbb R$

So $ g(x) = ax$ and $ \boxed{f(x) = ax^2}$ which indeed is a solution.
\end{solution}
*******************************************************************************
-------------------------------------------------------------------------------

\begin{problem}[Posted by \href{https://artofproblemsolving.com/community/user/68719}{MJ GEO}]
	Let $f: \mathbb N \to \mathbb R$ be a function such that for all integers $a$ and $b$ larger than $1$,
\[ f(ab) = f(d)\left(f\left(\frac {a}{d}\right) + f\left(\frac {b}{d}\right)\right),\]
where $d$ is the greatest common divisor of $a$ and $b$. Find all possible values of $ f(2001)$.
	\flushright \href{https://artofproblemsolving.com/community/c6h339655}{(Link to AoPS)}
\end{problem}



\begin{solution}[by \href{https://artofproblemsolving.com/community/user/68719}{MJ GEO}]
	this is my solution,but i think it is wrong,please check it for me:
$ f(a^2)=2f(a)f(1)$,(a,b)=1 so $ f(ab)=f(1)(f(a)+f(b))$
$ f(a^4)=2f(a^2)f(1)=f(a.a^3)=f(a)f(1)+f(a)f(a^2)$,so  if f(1)=0 not occur we have $ 1+2f(a)=4f(1)$
\end{solution}



\begin{solution}[by \href{https://artofproblemsolving.com/community/user/29428}{pco}]
	\begin{tcolorbox}$ f$ is from $ N$ to $ R$,such that for $ a,b > 1$ $ f(ab) = f(d)(f(\frac {a}{d}) + f(\frac {b}{d}))$ and $ d = (a,b)$ find all possible solutions for $ f(2001)$\end{tcolorbox}

1) If $ f(1)=0$ :
=========
Then $ f(2001)=f(3\times 667)=f(1)(f(3)+f(667))=0$
And such $ f(x)$ exists (choose for example $ f(x)=0$ $ \forall x$

2) If $ f(1)\ne 0$
===========
Let $ p$ prime
$ f(p^2)=f(p\times p)=f(p)(f(1)+f(1))=2f(1)f(p)$
$ f(p^3)=f(p\times p^2)=f(p)(f(1)+f(p))=f(p)^2+f(1)f(p)$
$ f(p^4)=f(p^2\times p^2)=f(p^2)(f(1)+f(1))=2f(1)f(p^2)=4f(1)^2f(p)$
$ f(p^4)=f(p\times p^3)=f(p)(f(1)+f(p^2))=2f(1)f(p)^2+f(1)f(p)$
$ f(p^5)=f(p\times p^4)=f(p)(f(1)+f(p^3))=f(p)^3+f(1)f(p)^2+f(1)f(p)$
$ f(p^5)=f(p^2\times p^3)=f(p^2)(f(1)+f(p))=2f(1)f(p)^2+2f(1)^2f(p)$

The two expressions of $ f(p^4)$ imply (since $ f(1)\ne 0$) $ f(p)=0$ or $ f(p)=2f(1)-\frac 12$

If $ f(p)\ne 0$, then $ f(p)=2f(1)-\frac 12$ and the two expressions of $ f(p^5)$ imply $ f(1)=\frac 12$

2.1) $ f(1)\ne \frac 12$
-------------------------
Then $ f(p)=0$ $ \forall$ prime $ p$
Then $ f(2001)=f(3\times 667)=f(1)(f(3)+f(667))=0$

2.2) $ f(1)=\frac 12$
-----------------------
Let $ p,q,r$ three prime numbers.

We get $ f(pqr)=f((pq)r)=\frac 12(f(pq)+f(r))$ $ =\frac 14f(p)+\frac 14f(q)+\frac 12f(r)$
And also $ f(pqr)=f(p(qr))=f(1)(f(p)+f(qr))$ $ =\frac 12f(p)+\frac 14f(q)+\frac 14f(r)$

and so $ f(p)=f(r)$ $ \forall p,r$
And since we saw that $ f(p)=0$ or $ f(p)=2f(1)-\frac 12=\frac 12$ we get :

2.2.1) either $ f(p)=0$ $ \forall$ prime $ p$
------------------------------------------------
And then $ f(2001)=f(3\times 667)=f(1)(f(3)+f(667))=0$

2.2.2) either $ f(p)=\frac 12$ $ \forall$ prime $ p$
------------------------------------------------------
And then $ f(2001)=f(3\times 667)=f(1)(f(3)+f(667))=\frac 12$
And such $ f(x)$ exists (choose for example $ f(x)=\frac 12$ $ \forall x$)

Hence the answer : $ \boxed{f(2001)\in\{0,\frac 12\}}$
\end{solution}



\begin{solution}[by \href{https://artofproblemsolving.com/community/user/55721}{Thjch Ph4 Trjnh}]
	$ f(d^2ab) = f(d)(f(a) + f(b)), (a,b) = 1$.
If $ (d^2,ab) = 1$ then 
$ f(d^2ab) = f(1).(f(d^2) + f(ab)) = f(1).(2f(1)f(d) + f(1)(f(a) + f(b)))$
$ = (f(1))^2(2f(d) + f(a) + f(b))$.
$ \Rightarrow f(d)(f(a) + f(b)) = (f(1))^2(2f(d) + f(a) + f(b))$.
$ \Rightarrow f(d)(f(a) + f(b) - 2(f(1))^2) = (f(1))^2(f(a) + f(b))$.
*) If $ f(1)\ne 0$.
If $ f(a) + f(b) - 2(f(1))^2 = 0$, then $ f(a) + f(b) = 0$ hence $ f(1) = 0$, contradict.
$ \Rightarrow f(d) = \frac {(f(1))^2(f(a) + f(b))}{f(a) + f(b) - 2(f(1))^2}$, for $ (d,a) = (d,b) = (a,b) = 1$.
For $ a,b\in N, a,b > 1$, there exits $ p > q$ are primes $ > a,b$.
We have $ (a,p) = (a,q) = (b,p) = (b,q) = (p,q) = 1$, then:
$ f(a) = f(b) = \frac {(f(1))^2(f(p) + f(q))}{f(p) + f(q) - 2(f(1))^2}$.
$ \Rightarrow f(a) = f(b), a,b > 1$.
$ f(n) = c, n > 1$.
We obtain: $ c = 2c^2$ so $ c = 0,\frac {1}{2}$.
_$ c = 0$, $ f(2001) = 0$.
_$ c = \frac {1}{2}, f(2001) = \frac {1}{2}$.
*) If $ f(1) = 0$.
$ f(2001) = 0$.
\end{solution}



\begin{solution}[by \href{https://artofproblemsolving.com/community/user/67223}{Amir Hossein}]
	This is Baltic Way 2001: https:\/\/artofproblemsolving.com\/community\/c6h378184
\end{solution}
*******************************************************************************
-------------------------------------------------------------------------------

\begin{problem}[Posted by \href{https://artofproblemsolving.com/community/user/68719}{MJ GEO}]
	Find all functions $f: \mathbb R^+ \to \mathbb R$ such that for all reals $x$ and $y$,
\[ f(x+y)=f(x^2+y^2).\]
	\flushright \href{https://artofproblemsolving.com/community/c6h339865}{(Link to AoPS)}
\end{problem}



\begin{solution}[by \href{https://artofproblemsolving.com/community/user/29428}{pco}]
	\begin{tcolorbox}$ f$ from $ R^ +$ to $ R$ and $ f(x + y) = f(x^2 + y^2)$.find all functions.\end{tcolorbox}

I suppose that equation is true $ \forall x,y>0$. Let then $ P(x,y)$ be the assertion $ f(x+y)=f(x^2+y^2)$

Let $ u,v$ such that $ v\sqrt 2>u\ge v>0$ and let then $ w\in[\frac{u^2}2,v^2]$

$ u^2> w\ge \frac {u^2}2$ and so the system "$ x+y=u$ and $ xy=\frac{u^2-w}2$" has at least two real positive roots $ x_1,y_1$ and $ P(x_1,y_1)$ $ \implies$ $ f(u)=f(w)$

$ v^2> w\ge \frac {v^2}2$ and so the system "$ x+y=v$ and $ xy=\frac{v^2-w}2$" has at least two real positive roots $ x_2,y_2$ and $ P(x_2,y_2)$ $ \implies$ $ f(v)=f(w)$

So $ f(u)=f(v)$ $ \forall u,v$ such that  $ v\sqrt 2>u\ge v>0$ and it is easy to conclude from this that $ f(x)$ is constant.

Hence the answer : $ \boxed{f(x)=c}$ $ \forall x>0$ which, indeed, is a solution
\end{solution}
*******************************************************************************
-------------------------------------------------------------------------------

\begin{problem}[Posted by \href{https://artofproblemsolving.com/community/user/68719}{MJ GEO}]
	Find all functions $f: \mathbb R \to \mathbb R$ which satisfy the following conditions:
1) For all reals $x$ and $y$,
\[ f(x + f(y)) = y + f(x).\]
2) The set 
\[ A = \left\{ \frac {f(x)}{x} \mid x \in \mathbb R, x \neq 0\right\}\]
is finite.
	\flushright \href{https://artofproblemsolving.com/community/c6h339999}{(Link to AoPS)}
\end{problem}



\begin{solution}[by \href{https://artofproblemsolving.com/community/user/31915}{Batominovski}]
	First, I'll show that $ f(0) = 0$.  Suppose contrary that $ f(0) = c \neq 0$.  Thus,
\[ f(x + c) = f\left(x + f(0)\right) = 0 + f(x) = f(x)\,.\]
Therefore, $ f(c) = c$, and by induction, $ f(nc) = c$ for every $ n \in \mathbb{N}$.  This means $ \frac {1}{n} \in A$ for all positive integers $ n$, contradicting the second requirement.  

Now, $ f(0) = 0$.  We have
\[ f(f(x)) = f\left(0 + f(x)\right) = x + f(0) = x\,.\]
That is,
\[ f(x + y) = f\big(x + f\left(f(y)\right)\big) = f(y) + f(x)\,,\]
which is Cauchy's functional equation.  We conclude that there exists $ k$ s.t. $ f(r) = kr$ for all rationals $ r$.  If there exists $ a \notin \mathbb{Q}$ for which $ f(a) \neq ka$, we get
\[ f(a + r) = f(a) + f(r) = f(a) + kr\,.\]
Thus, $ \frac {f(a + r)}{a + r} = \frac {f(a) + kr}{a + r} = k + \frac {f(a) - ka}{a + r}$.  Note that, for all rational numbers $ r$, $ a + r \neq 0$ and the expression $ \frac {f(a) - ka}{a + r}$ takes distinct values.  This contradicts the second requirement again.  Hence, $ f(x) = kx$ for all $ x$.  Now, it's easy to see that $ k = \pm 1$.
\end{solution}



\begin{solution}[by \href{https://artofproblemsolving.com/community/user/29428}{pco}]
	\begin{tcolorbox}find all functions from $ R$ to $ R$ such that 
1)$ f(x + f(y)) = y + f(x)$ 
2)the set $ A = {\frac {f(x)}{x}}$ be finite.\end{tcolorbox}

Notice first that condition 2) implies that $ f(x)$ is bounded over $ [0,1]$. Let then $ P(x,y)$ be the assertion $ f(x+f(y))=y+f(x)$

$ P(0,x-f(0))$ $ \implies$ $ f(f(x-f(0)))=x$
$ P(x-f(0),f(y-f(0)))$ $ \implies$ $ f(x+y-f(0))=f(x-f(0))+f(y-f(0))$

Let then $ g(x)=f(x-f(0))$. We got $ g(x+y)=g(x)+g(y)$ and so $ g(x)$ is a solution of Cauchy's equation bounded on $ [f(0),f(0)+1]$, and so is $ ax$ and so $ f(x)=ax+b$

Plugging in 1), we get $ a^2=1$ and $ b=0$ and both solutions match 2). hence the answers : $ f(x)=x$ and $ f(x)=-x$
\end{solution}
*******************************************************************************
-------------------------------------------------------------------------------

\begin{problem}[Posted by \href{https://artofproblemsolving.com/community/user/74122}{vaibhav2903}]
	If a function $f: \mathbb R \to \mathbb R$ satisfies the following conditions, determine f($ \sqrt{1996}$).
(i) $ f(1)=1$,
(ii) $ f(x+y)= f(x)+f(y)$ for all $x,y \in \mathbb R$, and
(iii) For all non-zero real $x$,
\[ f\left(\frac 1x \right)= \frac{f(x)}{x^{2}}.\]
	\flushright \href{https://artofproblemsolving.com/community/c6h340091}{(Link to AoPS)}
\end{problem}



\begin{solution}[by \href{https://artofproblemsolving.com/community/user/29428}{pco}]
	\begin{tcolorbox}If a function f satisfies the conditions (i)–(iii),
determine f($ \sqrt {1996}$) .
(i) $ f(1) = 1$;
(ii) $ f(x + y) = f(x) + f(y)$for all x,y ? R;
2
(iii) $ f(1\/x)$= $ \frac {f(x)}{x^{2}}$ for all x ? R, x=0\end{tcolorbox}

Obviously $ f(1996)=1996$

$ f(\frac 1{\sqrt{1996}(1-\sqrt{1996})})=\frac{f(\sqrt{1996}-1996)}{1996(1-\sqrt{1996})^2}$ $ =\frac{f(\sqrt{1996})-1996}{1996(1-\sqrt{1996})^2}$

$ f(\frac 1{\sqrt{1996}(1-\sqrt{1996})})=f(\frac 1{\sqrt{1996}}+\frac 1{1-\sqrt{1996}})$ $ =f(\frac 1{\sqrt{1996}})+f(\frac 1{1-\sqrt{1996}})$ $ =\frac{f(\sqrt{1996})}{1996}+\frac{f(1-\sqrt{1996})}{(1-\sqrt{1996})^2}$ $ =\frac{f(\sqrt{1996})}{1996}+\frac{1-f(\sqrt{1996})}{(1-\sqrt{1996})^2}$

And so $ \frac{f(\sqrt{1996})-1996}{1996(1-\sqrt{1996})^2}$ $ =\frac{f(\sqrt{1996})}{1996}+\frac{1-f(\sqrt{1996})}{(1-\sqrt{1996})^2}$

And $ f(\sqrt{1996})=\sqrt{1996}$
\end{solution}



\begin{solution}[by \href{https://artofproblemsolving.com/community/user/74122}{vaibhav2903}]
	cant we just define the function that $ f(x)=x$.

  :| is that wrong?
\end{solution}



\begin{solution}[by \href{https://artofproblemsolving.com/community/user/29428}{pco}]
	\begin{tcolorbox}cant we just define the function that $ f(x) = x$.

  :| is that wrong?\end{tcolorbox}

It's longer to prove that $ f(x)=x$ than to prove that $ f(\sqrt{1996})=\sqrt{1996}$
\end{solution}
*******************************************************************************
-------------------------------------------------------------------------------

\begin{problem}[Posted by \href{https://artofproblemsolving.com/community/user/29381}{james digol}]
	Let $ \mathbb{R}^*$ be the set of all real numbers, except $ 1$. Find all functions $ f: \mathbb{R}^*\rightarrow \mathbb{R}$ that satisfy the functional equation \[ x+f(x)+2f\left ( \frac{x+2009}{x-1} \right ) =2010.\]
	\flushright \href{https://artofproblemsolving.com/community/c6h340538}{(Link to AoPS)}
\end{problem}



\begin{solution}[by \href{https://artofproblemsolving.com/community/user/29428}{pco}]
	\begin{tcolorbox}Let $ \mathbb{R}^*$ be the set of all real numbers, except $ 1$. Find all functions $ f: \mathbb{R}^*\rightarrow \mathbb{R}$ that satisfy the functional equation
\[ x + f(x) + 2f\left ( \frac {x + 2009}{x - 1} \right ) = 2010.\]
\end{tcolorbox}

[hide="My solution"]
Let $ P(x)$ be the assertion $ x + f(x) + 2f\left ( \frac {x + 2009}{x - 1} \right ) = 2010$

(a) : $ P(x)$ $ \implies$ $ x + f(x) + 2f\left ( \frac {x + 2009}{x - 1} \right ) = 2010$

(b) : $ P(\frac {x + 2009}{x - 1})$ $ \implies$ $ \frac {x + 2009}{x - 1}+ f(\frac {x + 2009}{x - 1}) + 2f ( x) = 2010$

(a)-2(b) : $ x+f(x)-2\frac {x + 2009}{x - 1}-4f(x)=-2010$

And so $ \boxed{f(x)=\frac{x^2+2007x-6028}{3(x-1)}}$ and it is easy to check back that this indeed is a solution.[\/hide]
\end{solution}



\begin{solution}[by \href{https://artofproblemsolving.com/community/user/29381}{james digol}]
	Very good pco!
\end{solution}
*******************************************************************************
-------------------------------------------------------------------------------

\begin{problem}[Posted by \href{https://artofproblemsolving.com/community/user/60102}{Luba1123}]
	Find all functions $f: \mathbb R \to \mathbb R$ such that for all reals $x$ and $y$,
\[ f(x+y)+f(xy+1)=f(x)+f(y)+f(xy).\]
	\flushright \href{https://artofproblemsolving.com/community/c6h340545}{(Link to AoPS)}
\end{problem}



\begin{solution}[by \href{https://artofproblemsolving.com/community/user/29428}{pco}]
	\begin{tcolorbox}$ f: R\rightarrow R$ 
if   $ f(x + y) + f(xy + 1) = f(x) + f(y) + f(xy)$ then 
find all $ f$ function.\end{tcolorbox}

Let $ P(x,y)$ be the assertion $ f(x+y)+f(xy+1)=f(x)+f(y)+f(xy)$

$ P(xy,1)$ $ \implies$ $ f(xy+1)=f(xy)+\frac{f(1)}2$

Subtracting this from $ P(x,y)$, we get $ f(x+y)=f(x)+f(y)-\frac{f(1)}2$ and so $ g(x)=f(x)-\frac{f(1)}2$ is such that $ g(x+y)=g(x)+g(y)$


And so $ \boxed{f(x)=g(x)+g(1)}$ where $ g(x)$ is any solution of Cauchy's equation and it is easy to check back that this indeed is a solution.

For example, the only continuous solutions are $ a(x+1)$
\end{solution}
*******************************************************************************
-------------------------------------------------------------------------------

\begin{problem}[Posted by \href{https://artofproblemsolving.com/community/user/72731}{goodar2006}]
	Find all functions $f:\mathbb R \to\mathbb R$ such that
\[f(f(x)+f(y))+f(f(x))=2f(x)+y, \quad \forall x,y \in \mathbb R.\]
	\flushright \href{https://artofproblemsolving.com/community/c6h340552}{(Link to AoPS)}
\end{problem}



\begin{solution}[by \href{https://artofproblemsolving.com/community/user/29428}{pco}]
	\begin{tcolorbox}find all functions f in real numbers such that :
f(f(x)+f(y))+f(f(x))=2f(x)+y\end{tcolorbox}

Let $ P(x,y)$ be the assertion $ f(f(x)+f(y))+f(f(x))=2f(x)+y$

Subtracting $ P(0,x)$ from $ P(x,0)$ implies $ f(f(x))-f(f(0))=2f(x)-2f(0)-x$ and so $ f(f(x))=2f(x)-x+c$ (where $ c=f(f(0))-2f(0)$)

Plugging this in original equation, we get $ f(f(x)+f(y))=x+y-c$
So $ f(x)$ is bijective and $ \exists u$ such that $ f(u)=0$

Setting $ y=u$ in $ f(f(x)+f(y))=x+y-c$ implies $ f(f(x))=x+u-c$
And since  $ f(f(x))=2f(x)-x+c$, we get $ 2f(x)-x+c=x+u-c$ and so $ f(x)=x+a$ for some $ a$.

Plugging this in original equation, we get $ a=0$

Hence the unique solution : $ \boxed{f(x)=x}$ $ \forall x$
\end{solution}



\begin{solution}[by \href{https://artofproblemsolving.com/community/user/125018}{horizon}]
	my solution
let $y=0$ ,we have $f(f(x))+f(f(x)+f(0))=2f(x)$
let $x=0$, we have $f(f(y)+f(0))+f(f(0))=2f(0)+y$,so
$f(f(y))-2f(y)+y=f(f(0))-2f(0)$,then we have
$f(f(x)+f(y))+f(f(0))-2f(0)=x+y$,then we have that $f$ is surjective,
so there exists a $x$ so that $f(x)=0$,$f(f(y))+f(0)=y$
change $f(x)$ by $x$,we have $f(x+f(y))+f(x)=2x+y$,change $y$ by $f(y)$
$f(x+y-f(0))+f(x)=2x+f(y)$ let $y=f(0)$ we have $f(x)=x+c$ ,$c$ is constant
plugging this in original equation,$c=0$,so $f(x)=x$
\end{solution}
*******************************************************************************
-------------------------------------------------------------------------------

\begin{problem}[Posted by \href{https://artofproblemsolving.com/community/user/78977}{lexmarkx125}]
	Find all functions $f: \mathbb R \to \mathbb R$  satisfying the following:
(1) If $ x > 0$, then $ f(x) > 0$, and
(2) For all real values of $y$ and $z$, 
\[ f(2f(y) + f(z) ) = 2y+z.\]
	\flushright \href{https://artofproblemsolving.com/community/c6h340930}{(Link to AoPS)}
\end{problem}



\begin{solution}[by \href{https://artofproblemsolving.com/community/user/29428}{pco}]
	\begin{tcolorbox}Find all function such that $ f: \Re \rightarrow \Re$ satisfy following:
(1) if $ x > 0$, then $ f(x) > 0$
(2) for all real values y and z , $ f(2f(y) + f(z) ) = 2y + z$\end{tcolorbox}

Let $ P(x,y)$ be the assertion $ f(2f(x) + f(y) ) = 2x + y$

Let $ u=3f(0)$ : $ P(0,0)$ $ \implies$ $ f(u)=0$ and so $ u\le 0$ (using 1)
If $ u<0$, $ P(u,-u)$ $ \implies$ $ f(f(-u))=u$ But $ u<0$ $ \implies$ $ -u>0$ and so $ f(-u)>0$ and so $ f(f(-u))>0$ and so $ u>0$, impossible.

So $ u=0$ and $ f(0)=0$ and then $ P(0,x)$ $ \implies$ $ f(f(x))=x$

$ P(f(x),0)$ $ \implies$ $ f(2x)=2f(x)$
$ P(f(x),f(y))$ $ \implies$ $ f(2x+y)=2f(x)+f(y)=f(2x)+f(y)$ and so $ f(x+y)=f(x)+f(y)$ $ \forall x,y$

So $ f(x)$ is an increasing (since $ y>0$ $ \implies$ $ f(y)>0$) solution of Cauchy equation, so is $ ax$ and $ f(x)$ increasing plus $ f(f(x))=x$ imply $ a=1$

Hence the unique solution $ \boxed{f(x)=x}$ $ \forall x$ which indeed is a solution.
\end{solution}
*******************************************************************************
-------------------------------------------------------------------------------

\begin{problem}[Posted by \href{https://artofproblemsolving.com/community/user/50375}{Phun_TZ}]
	Prove that there does not exists a function $ f : \mathbb{R}^{+}\to\mathbb{R}^{+}$ which satisfies  
\[ f(x)^2\geq f(x+y)(f(x)+y)\]
for all positive reals $x$ and $y$.
	\flushright \href{https://artofproblemsolving.com/community/c6h340983}{(Link to AoPS)}
\end{problem}



\begin{solution}[by \href{https://artofproblemsolving.com/community/user/29428}{pco}]
	\begin{tcolorbox}Prove that there does not exists a function $ f : \mathbb{R}^{ + }\rightarrow\mathbb{R}^{ + }$
satisfies
\[ f(x)^2\geq f(x + y)(f(x) + y)\]
\end{tcolorbox}
1) $ f(x)$ is a strictly decreasing function 
==========================
$ f(x+y)f(x)<f(x+y)(f(x)+y)\le f(x)^2$ and so $ f(x+y)<f(x)$ 
Q.E.D.

2) Assertion $ Q(x)$ : $ f(x+f(x))\le \frac{f(x)}2$ is true $ \forall x>0$
=============================================
Just set $ y=f(x)$ in original inequation

3) Let $ f(1)=a$ and let $ u_n=1+2a(1-2^{-n})$. Then $ f(u_n)\le a2^{-n}$
===============================================
This can easily be shown with induction :
Start of induction \end{underlined}: $ f(u_0)=f(1)=a\le a2^{-0}$
Induction step \end{underlined}: If $ f(u_n)\le a2^{-n}$, then $ Q(u_n)$ implies $ f(u_n+f(u_n))\le \frac {f(u_n}2$
But $ f(u_n)\le a2^{-n}$ and $ f(x)$ decreasing implies then :

$ f(u_{n+1})=f(u_n+a2^{-n})\le f(u_n+f(u_n))\le \frac{f(u_n)}2\le a2^{-n-1}$
Q.E.D.

3) no such function exists
================
Since $ u_n<2a+1$, we get then $ f(2a+1)\le a2^{-n}$ $ \forall n$ which is impossible since $ f(2a+1)>0$
Q.E.D.
\end{solution}



\begin{solution}[by \href{https://artofproblemsolving.com/community/user/50375}{Phun_TZ}]
	That's cool!pco....Thank for your solution :)
\end{solution}
*******************************************************************************
-------------------------------------------------------------------------------

\begin{problem}[Posted by \href{https://artofproblemsolving.com/community/user/57963}{saeedghodsi}]
	Suppose that $ f: \mathbb{R}\to\mathbb{R}$ is a bounded function such that for all $x \in \mathbb{R}$,
\[ f \left( x + \frac {1}{6} \right) + f \left( x + \frac {1}{7} \right) = f (x) + f \left( x + \frac {13}{42} \right).\]
Prove that there exists a positive real number $d$ such that $ f ( x ) = f ( x + d )$ for all $x \in \mathbb{R}$.
	\flushright \href{https://artofproblemsolving.com/community/c6h341011}{(Link to AoPS)}
\end{problem}



\begin{solution}[by \href{https://artofproblemsolving.com/community/user/29428}{pco}]
	\begin{tcolorbox}$ f: \mathbb{R}\to\mathbb{R}$ is a bounded function such that:
$ f ( x + \frac {1}{6} ) + f ( x + \frac {1}{7} ) = f (x) + f ( x + \frac {13}{42} )$
prove that there exists $ d\in R$ such that $ f ( x ) = f ( x + d )$\end{tcolorbox}

$ d=0$
\end{solution}



\begin{solution}[by \href{https://artofproblemsolving.com/community/user/29428}{pco}]
	\begin{tcolorbox}$ f: \mathbb{R}\to\mathbb{R}$ is a bounded function such that:
$ f ( x + \frac {1}{6} ) + f ( x + \frac {1}{7} ) = f (x) + f ( x + \frac {13}{42} )$
prove that there exists $ d\in R^{ + }$ such that $ f ( x ) = f ( x + d )$\end{tcolorbox}

With the modification made by OP ($ d\in\mathbb R\to d\in\mathbb R^+$) :

For easier writing, Let $ g(x)=f(\frac x{42})$ such that the equation becomes $ g(x+7)+g(x+6)=g(x)+g(x+13)$

Let then $ h(x)=g(x+7)-g(x)$ such that $ h(x+6)=h(x)$

$ h(x+35)+h(x+28)+h(x+21)+h(x+14)+h(x+7)+h(x)=g(x+42)-g(x)$ and so :

$ g(x+42)=g(x)+k(x)$ where $ k(x)=h(x+35)+h(x+28)+h(x+21)+h(x+14)+h(x+7)+h(x)$

Since $ h(x+6)=h(x)$, we get that $ k(x+6)=k(x)$ and so $ k(x+42)=k(x)$ and so $ g(x+42n)=g(x)+nk(x)$

So, since $ g(x)$ is bounded, $ k(x)=0$ $ \forall x$ and so $ g(x+42)=g(x)$ $ \forall x$ and so $ \boxed{f(x+1)=f(x)}$ $ \forall x$
\end{solution}
*******************************************************************************
-------------------------------------------------------------------------------

\begin{problem}[Posted by \href{https://artofproblemsolving.com/community/user/61896}{Mateescu Constantin}]
	Find all functions $ f : \mathbb{Q}\to\mathbb{Q}$ so that \[ f(xy+f(x))=f(x)f(y)+x, \quad \forall\ x,y\in\mathbb{Q}\] and $ f(1)=1$.
	\flushright \href{https://artofproblemsolving.com/community/c6h341272}{(Link to AoPS)}
\end{problem}



\begin{solution}[by \href{https://artofproblemsolving.com/community/user/29428}{pco}]
	\begin{tcolorbox}Find all functions $ f : \mathbb{Q}\to\mathbb{Q}$ so that $ f(xy + f(x)) = f(x)f(y) + x\ ,\ \forall\ x,y\in\mathbb{Q}$ and $ f(1) = 1$ .\end{tcolorbox}

Let $ P(x,y)$ be the assertion $ f(xy + f(x)) = f(x)f(y) + x$

$ P(1,x)$ $ \implies$ $ f(x+1)=f(x)+1$ and so $ f(x)=x$ $ \forall x\in\mathbb Z$

$ P(n,x-1)$ $ \implies$ $ f(nx)=nf(x)$ and so $ \boxed{f(x)=x\text{   }\forall x\in\mathbb Q}$ which indeed is a solution.
\end{solution}
*******************************************************************************
-------------------------------------------------------------------------------

\begin{problem}[Posted by \href{https://artofproblemsolving.com/community/user/68025}{Pirkuliyev Rovsen}]
	If $f:\mathbb{N}\to\mathbb{Z}$ is a function so that \[ (f(n+1)-f(n))(f(n+1)+f(n)+4)\leq 0\] for all $n \in \mathbb{N}$. Prove that $f$ is not injective.
	\flushright \href{https://artofproblemsolving.com/community/c6h341370}{(Link to AoPS)}
\end{problem}



\begin{solution}[by \href{https://artofproblemsolving.com/community/user/29428}{pco}]
	\begin{tcolorbox}If f: \mathbb{N}\to\mathbb{Z}  so that $ (f(n + 1) - f(n))(f(n + 1) + f(n) + 4)\leq 0$ ,$ \forall n \in \mathbb{N}$ .Prove that f is not infectiva.\end{tcolorbox}

$ (f(n + 1) - f(n))(f(n + 1) + f(n) + 4)\leq 0$ $ \iff$ $ (f(n+1)+2)^2\le (f(n)+2)^2$ 

$ \implies$ $ (f(n)+2)^2$ is a positive non increasing function 

$ \implies$  $ (f(n)+2)^2$ has a limit $ a^2$ when $ n\to+\infty$ 

$ \implies$ $ f(n)=\pm a-2$ $ \forall n>n_0$ 

Hence the result
\end{solution}



\begin{solution}[by \href{https://artofproblemsolving.com/community/user/68025}{Pirkuliyev Rovsen}]
	Thanks you :!:
\end{solution}
*******************************************************************************
-------------------------------------------------------------------------------

\begin{problem}[Posted by \href{https://artofproblemsolving.com/community/user/48947}{pablo221092}]
	Let $ a,b,$ and $c$ be real numbers. Consider the functions
\[ f(x) = ax^2 + bx + c \quad \text{and} \quad g(x) = cx^2 + bx + a.\]
Knowing that
\[\left| f( - 1)\right| \leq 1, \quad \left |f(0)\right |\leq1, \quad \text{and} \quad \left |f(1)\right |\leq1,\]
prove that if $ \qquad - 1\leq x \leq1$, then
\[ \left |f(x)\right |\leq \frac {5}{4} \quad \text{and} \quad \left |g(x)\right |\leq2.\]
	\flushright \href{https://artofproblemsolving.com/community/c6h341669}{(Link to AoPS)}
\end{problem}



\begin{solution}[by \href{https://artofproblemsolving.com/community/user/78843}{lethanhdatvn}]
	appling lagrande formula's 
$ f(x)$= $ f(1)$.$ \frac{x(x+1)}{(1-0)(1+1)}$ +$ f(0)$.$ \frac{(x-1)(x+1)}{(0-1)(1+1)}$ +$ f(-1)$.$ \frac{x(x-1)}{(-1-0)(1-1)}$ 
$ \Leftrightarrow$ $ f(x)$= $ \frac{1}{2}$ $ f(1)$.$ x(x+1)$ +$ \frac{1}{2}$ $ f(-1)$.$ x(x-1)$ - $ f(0)$.$ (x-1)(x+1)$
$ \Rightarrow$  $ \mid$  $ f(x)$ $ \mid$ $ \leq$ $ \mid$ $ \frac{1}{2}$ $ f(1)$.$ x(x+1)$ $ \mid$ + $ \mid$  $ \frac{1}{2}$ $ f(-1)$.$ x(x-1)$ $ \mid$ +$ \mid$ $ f(0)$.$ (x-1)(x+1)$ $ \mid$
$ \Rightarrow$   $ \mid$  $ f(x)$ $ \mid$ $ \leq$ $ \frac{1}{2}$ $ \mid$ $ x(x+1)$ $ \mid$ + $ \frac{1}{2}$ $ \mid$ $ x(x-1)$ $ \mid$ + $ \mid$ $ (x-1)(x+1)$ $ \mid$
If 1 $ \geq$ $ x$ $ \geq$ 0 then $ f(x)$ $ \leq$ $ \frac{5}{4}$ -$ {(x-\frac{1}{2})}^{2}$ $ \leq$ $ \frac{5}{4}$
If 0 $ \geq$ $ x$ $ >$ -1 then $ f(x)$ $ \leq$ $ \frac{5}{4}$ -$ {(x+\frac{1}{2})}^{2}$ $ \leq$ $ \frac{5}{4}$
Rewrite $ g(x)$= $ \frac{1}{{x}^{2}}$.$ f( \frac{1}{x} )$, then we have the same soluiton like $ f(x)$
 




ps: sorry because my bad english
\end{solution}



\begin{solution}[by \href{https://artofproblemsolving.com/community/user/70687}{math12061992}]
	\begin{tcolorbox}appling lagrande formula's 
$ f(x)$= $ f(1)$.$ \frac {x(x + 1)}{(1 - 0)(1 + 1)}$ +$ f(0)$.$ \frac {(x - 1)(x + 1)}{(0 - 1)(1 + 1)}$ +$ f( - 1)$.$ \frac {x(x - 1)}{( - 1 - 0)(1 - 1)}$ 
$ \Leftrightarrow$ $ f(x)$= $ \frac {1}{2}$ $ f(1)$.$ x(x + 1)$ +$ \frac {1}{2}$ $ f( - 1)$.$ x(x - 1)$ - $ f(0)$.$ (x - 1)(x + 1)$
$ \Rightarrow$  $ \mid$  $ f(x)$ $ \mid$ $ \leq$ $ \mid$ $ \frac {1}{2}$ $ f(1)$.$ x(x + 1)$ $ \mid$ + $ \mid$  $ \frac {1}{2}$ $ f( - 1)$.$ x(x - 1)$ $ \mid$ +$ \mid$ $ f(0)$.$ (x - 1)(x + 1)$ $ \mid$
$ \Rightarrow$   $ \mid$  $ f(x)$ $ \mid$ $ \leq$ $ \frac {1}{2}$ $ \mid$ $ x(x + 1)$ $ \mid$ + $ \frac {1}{2}$ $ \mid$ $ x(x - 1)$ $ \mid$ + $ \mid$ $ (x - 1)(x + 1)$ $ \mid$
If 1 $ \geq$ $ x$ $ \geq$ 0 then $ f(x)$ $ \leq$ $ \frac {5}{4}$ -$ {(x - \frac {1}{2})}^{2}$ $ \leq$ $ \frac {5}{4}$
If 0 $ \geq$ $ x$ $ >$ -1 then $ f(x)$ $ \leq$ $ \frac {5}{4}$ -$ {(x + \frac {1}{2})}^{2}$ $ \leq$ $ \frac {5}{4}$
Rewrite $ g(x)$= $ \frac {1}{{x}^{2}}$.$ f( \frac {1}{x} )$, then we have the same soluiton like $ f(x)$





ps: sorry because my bad english\end{tcolorbox}Will anyone tell me please what is \begin{bolded} lagrande formula\end{bolded} ? :blush:
\end{solution}



\begin{solution}[by \href{https://artofproblemsolving.com/community/user/29428}{pco}]
	\begin{tcolorbox} Will anyone tell me please what is \begin{bolded} lagrande formula\end{bolded} ? :blush:\end{tcolorbox}

It's "Lagrange formula". See http://en.wikipedia.org\/wiki\/Lagrange_interpolation_formula
\end{solution}



\begin{solution}[by \href{https://artofproblemsolving.com/community/user/68920}{prester}]
	\begin{tcolorbox}
Rewrite $ g(x)$= $ \frac {1}{{x}^{2}}$.$ f( \frac {1}{x} )$, then we have the same soluiton like $ f(x)$
\end{tcolorbox} 

It should be $ g(x)$= $ {x}^{2}f( \frac {1}{x} )$
\end{solution}
*******************************************************************************
-------------------------------------------------------------------------------

\begin{problem}[Posted by \href{https://artofproblemsolving.com/community/user/80281}{Flex}]
	Find all functions $f: \mathbb R \to \mathbb R$ such that for all reals $x$, $y$, and $z$,
\[  f(x^2(z^2 + 1) + f(y)(z + 1)) = 1 - f(z)(x^2 + f(y)) - z((1 + z)x^2 + 2f(y)).\]
	\flushright \href{https://artofproblemsolving.com/community/c6h341674}{(Link to AoPS)}
\end{problem}



\begin{solution}[by \href{https://artofproblemsolving.com/community/user/29428}{pco}]
	\begin{tcolorbox}$ f: R - - > R$ and $ f(x^2(z^2 + 1) + f(y)(z + 1)) = 1 - f(z)(x^2 + f(y)) - z((1 + z)x^2 + 2f(y))$ for all real $ x$ and $ y$ 
Find all functions.\end{tcolorbox}

Let $ P(x,y,z)$ the assertion $ f(x^2(z^2 + 1) + f(y)(z + 1)) = 1 - f(z)(x^2 + f(y)) - z((1 + z)x^2 + 2f(y))$

A first quick check shows that no constant function may be solution.

$ P(x,y,-1)$ $ \implies$ $ f(2x^2) = 1 - f(-1)x^2 +(2-f(-1)) f(y)$

Since $ f(y)$ cant be a constant, we need to have $ f(-1)=2$ and so $ f(2x^2)=1-2x^2$ and so $ f(x)=1-x$ $ \forall x\ge 0$ (*)

let then $ x\ge 0$ : $ P(0,x+1,0)$ $ \implies$ $ f(f(x+1)) = 1 - f(0)f(x+1)$
We know from (*) that $ f(x+1)=1-(x+1)=-x$ and $ f(0)=1$  and so :

$ f(-x) = 1 +x$ (**) $ \forall x\ge 0$

(*)+(**) imply $ \boxed{f(x)=1-x}$ $ \forall x$ which indeed is a solution
\end{solution}



\begin{solution}[by \href{https://artofproblemsolving.com/community/user/80281}{Flex}]
	\begin{tcolorbox}
A first quick check shows that no constant function may be solution.

\end{tcolorbox}
How?

\begin{tcolorbox}
Since $ f(y)$ cant be a constant, we need to have $ f( - 1) = 2$  

\end{tcolorbox}
Why?
\end{solution}



\begin{solution}[by \href{https://artofproblemsolving.com/community/user/29428}{pco}]
	\begin{tcolorbox}[quote="pco"]
A first quick check shows that no constant function may be solution.

\end{tcolorbox}
How?\end{tcolorbox}
Plugging $ f(x) = c$ in the original equation gives $ c = 1 - c(x^2 + c) - z((1 + z)x^2 + 2c)$ and so $ c^2 + c(1 + x^2 + 2z) + z(1 + z)x^2 - 1 = 0$ $ \forall x,z$
Setting $ x = 0$ and $ z = 0$ there gives $ c^2 + c - 1 = 0$
Setting $ x = 1$ and $ z = 0$ there gives $ c^2 + 2c - 1 = 0$
And no real $ c$ matches both equations.

\begin{tcolorbox}[quote="pco"]
Since $ f(y)$ cant be a constant, we need to have $ f( - 1) = 2$  

\end{tcolorbox}
Why?\end{tcolorbox}
Since $ f(y)$ is not a constant, let $ a$ and $ b$ such $ f(a)\ne f(b)$ :

$ P(x,a, - 1)$ $ \implies$ $ f(2x^2) = 1 - f( - 1)x^2 + (2 - f( - 1)) f(a)$
$ P(x,b, - 1)$ $ \implies$ $ f(2x^2) = 1 - f( - 1)x^2 + (2 - f( - 1)) f(b)$

Subtracting gives $ 0 = (2 - f( - 1))(f(a) - f(b))$ and so $ f( - 1) = 2$
\end{solution}



\begin{solution}[by \href{https://artofproblemsolving.com/community/user/195015}{Jul}]
	\begin{tcolorbox}$ f: R-->R$ and $ f(x^2(z^2+1)+f(y)(z+1))=1-f(z)(x^2+f(y))-z((1+z)x^2+2f(y))\;\;\;(1)$ for all real $ x$ and $ y$ 
Find all functions.\end{tcolorbox}

My solution :

Replacing $z=-1$ in $(1)$ :
\[f(2x^2)=1-f(-1)(x^2+f(y))+2f(y)\;\; \forall x,y\in \mathbb{R}\;\;(2)\]
If $f(-1) \neq 2$, from $(2)$, we deduce $f$ is a constant. A contracdiction !
Thus, $f(-1)=2$. In $(1)$ replace $x=0$ :
\[f(f(y).(z+1))=1-f(z)f(y)-2zf(y)\;\;\forall y,z\in \mathbb{R}\;\;\;(3)\]
Replacing $y=-1$ in $(3)$ :
\[f(2z+2)=1-2f(z)-4z\;\;\forall z\in \mathbb{R}\;\;\;(4)\]
Replacing $z=-1$ in $(4)$, we get $f(0)=1$. In $(3)$ replace $y=0$ :
\[f(z+1)=1-f(z)-2z\;\;\forall z\in \mathbb{R}\;\;\;(5)\]
From $(4)(5)$, we have :
\[f(2z+2)=-1+2f(z+1)\;\;\forall z\in \mathbb{R}\]
Or 
\[f(2z)=-1+2f(z)\;\;\forall z\in \mathbb{R}\]
We calculate $f(2z+2)$ in two ways.
\[f(2z+2)=1-2f(z)-4z\;\;\forall z\in \mathbb{R}\;\;\;\]
\[f(2z+2)=f((2z+1)+1)=1-f(2z+1)-2(2z+1)=-1-4z-f(2z+1)\]
\[=-1-4z-(1-f(2z)-4z)=-2+f(2z)=-3+2f(z)\;\forall z\in \mathbb{R}\]
Implies :
\[1-2f(z)-4z=-3+2f(z),\;\forall z\in \mathbb{R}\].
Or 
\[f(z)=1-z,\;\;\forall z\in \mathbb{R}\]
The functional equation sastifies the demand of this problem be :
\[f(x)=1-x,\;\;\forall z\in \mathbb{R}\]
\end{solution}
*******************************************************************************
-------------------------------------------------------------------------------

\begin{problem}[Posted by \href{https://artofproblemsolving.com/community/user/49420}{minhtuan_lequydon_dn}]
	1) Find all functions $f: \mathbb Q \to \mathbb Q$ which satisfy \[(f(x) + x + y) = x + f(x) + f(y)\] for all rationals $x$ and $y$.
2) Find all functions $f: \mathbb Q \to \mathbb Q$ which satisfy \[f(x) \cdot f(y) = f(x+f(y))\] for all rationals $x$ and $y$.
	\flushright \href{https://artofproblemsolving.com/community/c6h341758}{(Link to AoPS)}
\end{problem}



\begin{solution}[by \href{https://artofproblemsolving.com/community/user/29428}{pco}]
	\begin{tcolorbox}1) Find all function : Q -> Q sasfisty :f(f(x) + x + y) = x + f(x) + f(y) \end{tcolorbox}

See http://www.mathlinks.ro/viewtopic.php?t=271589

And I dont understand how such a problem could be asked in an olympiad training (it seemed to me very difficult when I had to solve it the first time)

Could you, minhtuan_lequydon_dn, please :

1) give us in what contest \/ book did you find this problem
2) give us your own solution  (you have one since you posted in "proposed and own" category.
\end{solution}



\begin{solution}[by \href{https://artofproblemsolving.com/community/user/49420}{minhtuan_lequydon_dn}]
	\begin{tcolorbox}1) Find all function : Q -> Q sasfisty :f(f(x) + x + y) = x + f(x) + f(y) 
2)  Find all function : Q -> Q sasfisty : f(x).f(y) = f(x+f(y))\end{tcolorbox}


Problem 2 in BMO 1979
\end{solution}



\begin{solution}[by \href{https://artofproblemsolving.com/community/user/29428}{pco}]
	\begin{tcolorbox}\begin{tcolorbox}1) Find all function : Q -> Q sasfisty :f(f(x) + x + y) = x + f(x) + f(y) 
2)  Find all function : Q -> Q sasfisty : f(x).f(y) = f(x+f(y))\end{tcolorbox}


Problem 2 in BMO 1979\end{tcolorbox}

Thanks for your reply, but  my question was about problem 1 

Btw, about problem 2 :
Better to indicate the source in the problem when you have it
Better to give the full statement. The BMO 1979 problem 4 contained at the end "prove that $ f(x)$ is constant"
\end{solution}



\begin{solution}[by \href{https://artofproblemsolving.com/community/user/49420}{minhtuan_lequydon_dn}]
	\begin{tcolorbox}[quote="minhtuan_lequydon_dn"][quote="minhtuan_lequydon_dn"]1) Find all function : Q -> Q sasfisty :f(f(x) + x + y) = x + f(x) + f(y) 
2)  Find all function : Q -> Q sasfisty : f(x).f(y) = f(x+f(y))\end{tcolorbox}


Problem 2 in BMO 1979\end{tcolorbox}

Thanks for your reply, but  my question was about problem 1 

Btw, about problem 2 :
Better to indicate the source in the problem when you have it
Better to give the full statement. The BMO 1979 problem 4 contained at the end "prove that $ f(x)$ is constant"\end{tcolorbox}
problem 2: Can you prove f(0) = 1 or 0 ??
\end{solution}



\begin{solution}[by \href{https://artofproblemsolving.com/community/user/29428}{pco}]
	\begin{tcolorbox}2)  Find all function : Q -> Q sasfisty : f(x).f(y) = f(x+f(y))\end{tcolorbox}

Let $ P(x,y)$ be the assertion $ f(x)f(y)=f(x+f(y))$
Let $ A=f(\mathbb Q)$
Let $ a=f(0)$

If $ a=0$, then $ P(x,0)$ $ \implies$ $ f(x)=0$ $ \forall x$ which indeed is a solution
Let us consider from now that $ a\ne 0$

$ P(0,0)$ $ \implies$ $ f(a)=a^2$
Let $ b=a-a^2$ : $ P(a-a^2,a)$ $ \implies$ $ f(b)=1$ and so $ 1\in A$

$ P(x,b)$ $ \implies$ $ f(x)=f(x+1)$ $ \forall x$ and so $ f(x)=f(x+n)$ 
A specific consequence is that $ f(n)=f(0)=a$ $ \forall n\in\mathbb Z$

From original equation, it's immediate to establish thru induction $ f(x+nf(y))=f(x)f(y)^n$ $ \forall x,y$ $ \forall n\in\mathbb N$

Let then $ f(y)=\frac pq$. This last equation implies $ f(x+p)=f(x)f(y)^q$

And since we previously got $ f(x+p)=f(x)$, we get $ f(x)f(y)^q=f(x)$
Using $ x=0$ in this equation we obtain $ f(y)\in\{-1,+1\}$ $ \forall y$
So $ f(x+f(y))=f(x)$ (since $ f(x+1)=f(x-1)=f(x)$) and so $ f(y)=1$ $ \forall y$ which indeed is a solution

So we got only two solutions :
$ f(x)=0$ $ \forall x$
$ f(x)=1$ $ \forall x$
\end{solution}
*******************************************************************************
-------------------------------------------------------------------------------

\begin{problem}[Posted by \href{https://artofproblemsolving.com/community/user/53051}{vinhhop}]
	Find all continous functions $ f: \mathbb{R}\to \mathbb{R}$ such that \[f(kx)=k\cdot f(x), \quad \forall x\in \mathbb{R},\] where the constant $ k$ is given.
	\flushright \href{https://artofproblemsolving.com/community/c6h341840}{(Link to AoPS)}
\end{problem}



\begin{solution}[by \href{https://artofproblemsolving.com/community/user/29428}{pco}]
	\begin{tcolorbox}Find all continous function $ f: \mathbb{R}\longrightarrow \mathbb{R}$ such that $ f(kx) = k\cdot f(x)\forall x\in \mathbb{R},$ where constant $ k$ is given.\end{tcolorbox}

It seems to me this is a rather classical equation but rigorous solution is rather long and I'm surprised you got this in an olympiad [training] session.

1) get rid of specific cases
=================
1.1) $ k = 0$
Solution if $ f(x) = h(x) - h(0)$ where $ h(x)$ is any continuous function

1.2) $ k = 1$
Solution is $ f(x) = h(x)$ where $ h(x)$ is any continuous function

1.3) $ k = - 1$
Soltion is $ f(x) = h(x) - h( - x)$ where $ h(x)$ is any continuous function (in fact $ f(x)$ is any continuous odd function)

2) $ k > 0$ and $ k\ne 1$
================
$ f(kx) = kf(x)$ $ \iff$ $ f(\frac xk) = \frac 1kf(x)$ and so WLOG say $ k > 1$
We have then $ f(k^nx) = k^nf(x)$ $ \forall x,\forall n\in\mathbb Z$
For $ x > 0$, we can then define :
$ f(x) = h_1(x)$ $ \forall x\in[1,k]$ where $ h_1(x)$ is any continuous function defined over $ [1,k]$ and such that $ h_1(k) = kh_1(1)$
$ f(x) = k^nh_1(xk^{ - n})$ $ \forall x\in [k^n,k^{n + 1}]$ and $ \forall n\in\mathbb Z$
From there, continuity at $ 0$ implies $ f(0) = 0$

Same method for defining $ f(x)$ for $ x < 0$ and so the solution :

Wlog say $ k > 1$ (else, swap $ k$ and $ \frac 1k$)
Let $ h_1(x)$ any continuous function defined over $ [1,k]$ and such that $ h_1(k) = kh_1(1)$
Let $ h_2(x)$ any continuous function defined over $ [ - k, - 1]$ and such that $ h_2( - k) = kh_2( - 1)$
$ f(0) = 0$
$ \forall x > 0$, $ f(x) = k^{\left\lfloor\frac {\ln(x)}{\ln(k)}\right\rfloor}$ $ h_1(xk^{ - \left\lfloor\frac {\ln(x)}{\ln(k)}\right\rfloor})$

$ \forall x < 0$, $ f(x) = k^{\left\lfloor\frac {\ln( - x)}{\ln(k)}\right\rfloor}$ $ h_1(xk^{ - \left\lfloor\frac {\ln( - x)}{\ln(k)}\right\rfloor})$

3) $ k < 0$ and $ k\ne - 1$
=================
We use the same method as in $ 2)$ above with base functions $ h_1$ and $ h_2$ but the base function used for a given value of $ x$ depend on $ \left\lfloor\frac {\ln(|x|)}{\ln(|k|)}\right\rfloor$ odd or even
\end{solution}



\begin{solution}[by \href{https://artofproblemsolving.com/community/user/53051}{vinhhop}]
	Thankyou very much, pco.
\end{solution}
*******************************************************************************
-------------------------------------------------------------------------------

\begin{problem}[Posted by \href{https://artofproblemsolving.com/community/user/35641}{DreamTeam}]
	Find all continous functions $ f: \mathbb R\rightarrow \mathbb R$, such that:

                         $ f(x)f(y)=f(\sqrt[3]{x^3+y^3})$
	\flushright \href{https://artofproblemsolving.com/community/c6h342100}{(Link to AoPS)}
\end{problem}



\begin{solution}[by \href{https://artofproblemsolving.com/community/user/29428}{pco}]
	\begin{tcolorbox}Find all continous functions $ f: \mathbb R\rightarrow \mathbb R$, such that:

                         $ f(x)f(y) = f(\sqrt [3]{x^3 + y^3})$\end{tcolorbox}

Let $ g(x)=f(\sqrt[3]x)$. We get : $ g(x^3)g(y^3)=g(x^3+y^3)$ and so $ g(x)$ is a continuous solution of $ g(x+y)=g(x)g(y)$ and so is $ 0$ or $ e^{ax}$

Hence the solutions :
$ f(x)=0$ $ \forall x$
$ f(x)=e^{ax^3}$ $ \forall x$
\end{solution}
*******************************************************************************
-------------------------------------------------------------------------------

\begin{problem}[Posted by \href{https://artofproblemsolving.com/community/user/54175}{Finchy}]
	Find the minimal value of the function $ f: [5,+\infty)\to \mathbb{R}$

\[ f(m)=m+\sqrt{m^2+40-2\sqrt{m^2-25}}\]
	\flushright \href{https://artofproblemsolving.com/community/c6h342140}{(Link to AoPS)}
\end{problem}



\begin{solution}[by \href{https://artofproblemsolving.com/community/user/61204}{BarbieRocks}]
	it looks like an increasing function to me, because $ m^2+40$ dominates $ -2\sqrt{m^2-25}$...so I would guess it's when x=5, or $ f(5)=5+\sqrt{65}$
\end{solution}



\begin{solution}[by \href{https://artofproblemsolving.com/community/user/29428}{pco}]
	\begin{tcolorbox}it looks like an increasing function to me, because $ m^2 + 40$ dominates $ - 2\sqrt {m^2 - 25}$...so I would guess it's when x=5, or $ f(5) = 5 + \sqrt {65}$\end{tcolorbox}

If the minimum is searched for $ x$ real (the usage of $ m$ could mean that OP is looking for integer $ m\ge 5$), then I dont know the result, but it's surely not $ f(5)$ : it's easy to see that $ \lim_{x\to 5^+}f'(x)=-\infty$ and so $ f(x)$ is decreasing on $ [5,x_0]$ for some $ x_0\in(5,6)$

So the minimum is less than $ f(5)$
\end{solution}



\begin{solution}[by \href{https://artofproblemsolving.com/community/user/29428}{pco}]
	\begin{tcolorbox}Find the minimal value of the function $ f: [5, + \infty)\to \mathbb{R}$
\[ f(m) = m + \sqrt {m^2 + 40 - 2\sqrt {m^2 - 25}}\]
\end{tcolorbox}

Brute solution :

$ f(x) = x + \sqrt {x^2 + 40 - 2\sqrt {x^2 - 25}}$

$ f'(x) = 1 + \frac {2x - 2\frac {2x}{2\sqrt {x^2 - 25}}}{2\sqrt {x^2 + 40 - 2\sqrt {x^2 - 25}}}$

$ f'(x) = 1 + \frac {x\sqrt {x^2 - 25} - x}{\sqrt {x^2 - 25}\sqrt {x^2 + 40 - 2\sqrt {x^2 - 25}}}$

And so sign of $ f'(x)$ is the same as sign of $ a(x) = \sqrt {x^2 - 25}\sqrt {x^2 + 40 - 2\sqrt {x^2 - 25}}$ $ + x\sqrt {x^2 - 25} - x$

So we are interested in roots above $ 5$ of $ a(x) = 0$, so 
$ 5 < x$ and $ \sqrt {x^2 - 25}\sqrt {x^2 + 40 - 2\sqrt {x^2 - 25}}$ $ = x - x\sqrt {x^2 - 25}$

$ \iff$ $ 5 < x < \sqrt {26}$ and $ (x^2 - 25)(x^2 + 40 - 2\sqrt {x^2 - 25})$ $ = x^2 + x^2(x^2 - 25) - 2x^2\sqrt {x^2 - 25}$

$ \iff$ $ 5 < x < \sqrt {26}$ and $ 50\sqrt {x^2 - 25}) = 1000 - 39x^2$

$ \iff$ $ 5 < x < \sqrt {\frac {1000}{39}}$ and $ 2500(x^2 - 25) = (1000 - 39x^2)^2$

$ \iff$ $ 5 < x < \sqrt {\frac {1000}{39}}$ and $ 39^2x^4 - 80500x^2 + 1062500 = 0$

Discriminant is $ 80500^2 - 4\cdot 39^2\cdot 1062500 = 4000^2$ and so we got $ x^2 = \frac {40250\pm 2000}{39^2}$

It's easy to check that the highest root is $ x^2 = \frac {42250}{39^2} > \frac {1000}{39}$

while the lowest is $ x^2 = \frac {4250}{169}\in(25,\frac {1000}{39})$ and so $ a(x)$ has a unique root over $ 5$ : $ x_0 = \sqrt {\frac {4250}{169}} = \frac {5\sqrt {170}}{13}$

From there, and since $ a(x)$ is continuous, we get that $ f(x)$ is decreasing over $ [5,\frac {5\sqrt {170}}{13}]$ and increasing after this value.

And the required minimum is obtained when $ x = \frac {5\sqrt {170}}{13}$

And its value is $ \boxed{f(\frac {5\sqrt {170}}{13}) = \sqrt {170}}$

IMHO, there is likely a geometrical solution ...
\end{solution}



\begin{solution}[by \href{https://artofproblemsolving.com/community/user/29428}{pco}]
	Yes, there is a geometrical one :

Take in plane points $ A(0,5)$, $ B(1,-8)$ and $ M(u,0)$ such that $ AM=x$

Then the required quantity is $ AM+MB$ and the minimum is $ AB=\sqrt{1+(5+8)^2}=\sqrt{170}$  
\end{solution}



\begin{solution}[by \href{https://artofproblemsolving.com/community/user/61204}{BarbieRocks}]
	ya i should stop browsing this forum when 
1. I can't even get top 16 at mc State
2. I got 6 on AIME.
\end{solution}



\begin{solution}[by \href{https://artofproblemsolving.com/community/user/36230}{varunrocks}]
	An AIME score does not reflect on your proof writing skills.
\end{solution}
*******************************************************************************
-------------------------------------------------------------------------------

\begin{problem}[Posted by \href{https://artofproblemsolving.com/community/user/77226}{wya}]
	1. Find all functions $ f: \mathbb{R}\to\mathbb{R}$ such that 
for all $ x,y\in\mathbb{R},$
\[ f\left(x+xf(y)\right)=x+yf(x).\]

2. Find all functions $ f: \mathbb{R}\to\mathbb{R}$ such that 
for all $ x,y\in\mathbb{R},$
\[ f\left(x+f(x)f(y)\right)=x+yf(x).\]

3. Find all functions $ f: \mathbb{R}\to\mathbb{R}$ such that 
for all $ x,y\in\mathbb{R},$
\[ f\left(x+xf\left(f(y)\right)\right)=x+yf(x).\]

4. Find all functions $ f: \mathbb{R}\to\mathbb{R}$ such that 
for all $ x,y\in\mathbb{R},$
\[ f\left(x+xf(y)\right)=x+f(y)f(x).\]

5. Find all functions $ f: \mathbb{R}\to\mathbb{R}$ such that 
for all $ x,y\in\mathbb{R},$
\[ f\left(x+xf(y)\right)=x+yf\left(f(x)\right).\]
	\flushright \href{https://artofproblemsolving.com/community/c6h343045}{(Link to AoPS)}
\end{problem}



\begin{solution}[by \href{https://artofproblemsolving.com/community/user/29428}{pco}]
	\begin{tcolorbox}\begin{bolded}1.\end{bolded} Find all functions $ f: \mathbb{R}\to\mathbb{R}$ such that 
for all $ x,y\in\mathbb{R},$
$ f\left(x + xf(y)\right) = x + yf(x).$\end{tcolorbox}
Let $ P(x,y)$ be the assertion $ f(x + xf(y)) = x + yf(x)$
$ P(0,0)$ $ \implies$ $ f(0) = 0$
$ P(x,0)$ $ \implies$ $ f(x) = x$ which indeed is a solution
Hence the answer : $ \boxed{f(x) = x}$

\begin{tcolorbox}\begin{bolded}2.\end{bolded} Find all functions $ f: \mathbb{R}\to\mathbb{R}$ such that 
for all $ x,y\in\mathbb{R},$
$ f\left(x + f(x)f(y)\right) = x + yf(x).$\end{tcolorbox}
Let $ P(x,y)$ be the assertion $ f(x + f(x)f(y)) = x + yf(x)$
$ P(0,0)$ $ \implies$ $ f(u) = 0$ with $ u = f(0)^2$
$ P(u,1)$ $ \implies$ $ u = 0$
$ P(x,0)$ $ \implies$ $ f(x) = x$ which indeed is a solution
Hence the answer : $ \boxed{f(x) = x}$

\begin{tcolorbox}\begin{bolded}3.\end{bolded} Find all functions $ f: \mathbb{R}\to\mathbb{R}$ such that 
for all $ x,y\in\mathbb{R},$
$ f\left(x + xf\left(f(y)\right)\right) = x + yf(x).$\end{tcolorbox}
Let $ P(x,y)$ be the assertion $ f(x + xf(f(y))) = x + yf(x)$
$ P(0,0)$ $ \implies$ $ f(0) = 0$
$ P(x,0)$ $ \implies$ $ f(x) = x$ which indeed is a solution
Hence the answer : $ \boxed{f(x) = x}$

\begin{tcolorbox}\begin{bolded}4.\end{bolded} Find all functions $ f: \mathbb{R}\to\mathbb{R}$ such that 
for all $ x,y\in\mathbb{R},$
$ f\left(x + xf(y)\right) = x + f(y)f(x).$\end{tcolorbox}
Let $ P(x,y)$ be the assertion $ f(x + xf(y)) = x + f(y)f(x)$
$ P(2,2)$ $ \implies$ $ f(u) = 2 + f(2)^2\ne 1$. where $ u = 2 + 2f(2)$
$ P(0,u)$ $ \implies$ $ f(0) = f(0)f(u)$ and so, since $ f(u)\ne 1$ : $ f(0) = 0$
$ P(x,0)$ $ \implies$ $ f(x) = x$ which indeed is a solution
Hence the answer : $ \boxed{f(x) = x}$

\begin{tcolorbox}\begin{bolded}5.\end{bolded} Find all functions $ f: \mathbb{R}\to\mathbb{R}$ such that 
for all $ x,y\in\mathbb{R},$
$ f\left(x + xf(y)\right) = x + yf\left(f(x)\right).$\end{tcolorbox}
Let $ P(x,y)$ be the assertion $ f(x + xf(y)) = x + yf(f(x))$
$ P(0,0)$ $ \implies$ $ f(0) = 0$
$ P(x,0)$ $ \implies$ $ f(x) = x$ which indeed is a solution
Hence the answer : $ \boxed{f(x) = x}$
\end{solution}



\begin{solution}[by \href{https://artofproblemsolving.com/community/user/77226}{wya}]
	Nice solution, pco. Thank you for your interest.  
\end{solution}
*******************************************************************************
-------------------------------------------------------------------------------

\begin{problem}[Posted by \href{https://artofproblemsolving.com/community/user/77226}{wya}]
	Find all functions $ f: \mathbb{R}\to\mathbb{R}$ such that for all $ x,y\in\mathbb{R},$
\[ f\left(x+xf(y)\right)=f(x)+yf(x).\]
	\flushright \href{https://artofproblemsolving.com/community/c6h343047}{(Link to AoPS)}
\end{problem}



\begin{solution}[by \href{https://artofproblemsolving.com/community/user/29428}{pco}]
	\begin{tcolorbox}Find all functions $ f: \mathbb{R}\to\mathbb{R}$ such that 
for all $ x,y\in\mathbb{R},$

$ f\left(x + xf(y)\right) = f(x) + yf(x).$\end{tcolorbox}

Let $ P(x,y)$ be the assertion $ f(x+xf(y))=f(x)+yf(x)$

$ f(x)=0$ $ \forall x$ is a solution and we'll from now look for non all-zero solutions.
Let then $ a$ such that $ f(a)\ne 0$

1) $ f(x)$ is a bijection, $ f(0)=0$ and $ f(-1)=-1$
====================================
If $ f(y_1)=f(y_2)$, comparaison between $ P(a,y_1)$ and $ P(a,y_2)$ implies $ y_1=y_2$ and so $ f(x)$ is injective
$ P(a,\frac{u}{f(a)}-1)$ $ \implies$ $ f(\text{something})=u$ and so $ f(x)$ is surjective, and so bijective.

Let then $ v$ such that $ f(v)=0$ : $ P(a,v)$ $ \implies$ $ vf(a)=0$ and so $ v=0$ and we got $ f(x)=0$ $ \iff$ $ x=0$

Then $ P(1,-1)$ $ \implies$ $ f(1+f(-1))=0$ and so $ f(-1)=-1$
Q.E.D.

2) $ f(xy)=f(x)f(y)$
===============
Let $ a=f(1)\ne 0$
Let $ g(x)=f(x)+1$. The equation becomes :
New assertion $ Q(x,y)$ : $ g(xg(y))=g(x)+yg(x)-y$
$ Q(1,x)$ $ \implies$ $ g(g(x))=ax+a+1$ and so $ g(ax+a+1)=ag(x)+a+1$ which may be written $ g(\frac xa-1-\frac 1a)=\frac{g(x)}a-1-\frac 1a$

$ Q(x,g(\frac ya-1-\frac 1a))$ $ \implies$ $ g(xy)=g(x)+g(\frac ya-1-\frac 1a)g(x)-g(\frac ya-1-\frac 1a)$ $ g(x)+(\frac {g(y)}a-1-\frac 1a)g(x)-\frac{g(y)}a+1+\frac 1a$ and so 

$ g(xy)-1=\frac 1a(g(x)-1)(g(y)-1)$

And so $ f(xy)=\frac{f(x)f(y)}{f(1)}$
Using $ x=y=-1$ in the above equation, we get $ f(1)^2=1$ and since $ f(1)\ne -1$ (since $ f(-1)=-1$ and $ f(x)$ injective) : $ f(1)=1$

And so $ f(xy)=f(x)f(y)$
Q.E.D.

3) $ f(x)=x$
==========
From $ f(xy)=f(x)f(y)$, we get $ f(x^2)=f(x)^2$ and so $ f(x)>0$ $ \forall x>0$ 
Then $ f(x+xf(y))=f(x)+yf(x)$ and $ f(x)$ surjective and $ f(x)$ and $ x$ same signs implies $ f(x)$ increasing.

$ f(xy)=f(x)f(y)$ and $ f(x)$ increasing implies then $ f(x)=x$
Q.E.D

4) synthesis of solutions
================
We got :
$ f(x)=0$
$ f(x)=x$
and it's easy to check back that these two functions indeed are solutions.
\end{solution}



\begin{solution}[by \href{https://artofproblemsolving.com/community/user/77226}{wya}]
	Very nice, pco.
Here is my solution:

First, we known that $ f(x)=0$ $ \forall x\in\mathbb{R}$ is a solution, and consider for non all-zero solutions. 
And prove that $ f$ is a bijection, $ f(0)=0, f(-1)=-1$ (same to yours proof)

\begin{bolded}2)\end{bolded} $ f(1)=1, f(-x)+f(x)=0$ $ \forall x\in\mathbb{R}$:
Let $ \alpha$ be a real number such that $ f(\alpha)=1$ (from $ f$ surjective $ \Rightarrow$ there exist $ \alpha$)
$ P(-1,y)\Rightarrow f\left(-1-f(y)\right)=-1-y$
$ P(\alpha,y)\Rightarrow f\left(\alpha\left(1+f(y)\right)\right)=1+y$
$ \Rightarrow f\left(-1-f(y)\right)+f\left(\alpha\left(1+f(y)\right)\right)=0,$
because $ f$ is a surjective, we have
$ f(-x)+f(\alpha x)=0$ $ \forall x\in\mathbb{R}$
$ \Rightarrow f\left(-\alpha^{2}\right)=-f(\alpha)=-1=f(-1)$
$ \Rightarrow-\alpha^{2}=-1\Rightarrow\alpha=1$ ($ f(-1)=-1$)
Therefore $ f(1)=1,$
and so $ f(-x)+f(x)=0$ $ \forall x\in\mathbb{R}$

\begin{bolded}3)\end{bolded} $ f(x+y)=f(x)+f(y)$
$ P(x,-y)\Rightarrow f\left(x-xf(y)\right)=f(x)-yf(x)$
$ P(x,y)\Rightarrow f\left(x+xf(y)\right)=f(x)+yf(x)$
$ \Rightarrow f\left(x-xf(y)\right)+f\left(x+xf(y)\right)=2f(x)$
because $ f$ is a surjective, we have
$ f(x-y)+f(x+y)=2f(x)$ for all $ x\in\mathbb{R}-\{0\}$ and $ y\in\mathbb{R},$
and because for $ x=0$ the equality become $ f(-y)+f(y)=0,$ which is also hold,
therefore $ f(x-y)+f(x+y)=2f(x)$ for all $ x,y\in\mathbb{R}$
$ \Rightarrow f(x-y)+f(x+y)=2f(x)=f(x-x)+f(x+x)=f(2x)$ 
$ \Rightarrow f(x+y)=f(x)+f(y)$

\begin{bolded}4)\end{bolded} $ f(x)=x$
From 3) and $ P(x,y)$ we have $ f(x)+f\left(xf(y)\right)=f\left(x+xf(y)\right)=f(x)+yf(x)$
$ \Rightarrow f\left(xf(y)\right)=yf(x)$
$ \Rightarrow f\left(f(y)\right)=yf(1)=y$
And so $ f\left(x^{2}\right)=f\left(xf\left(f(x)\right)\right)=f(x)f(x)=\left(f(x)\right)^{2}$ and so $ f(x)>0$ $ \forall x>0$
Consider $ f\left(f(x)-x\right)=f\left(f(x)\right)-f(x)=-\left(f(x)-x\right),$
because $ f(y)$ and $ y$ same signs, therefore
$ f(x)-x=0,$
or $ f(x)=x.$
\end{solution}



\begin{solution}[by \href{https://artofproblemsolving.com/community/user/77226}{wya}]
	By the way, we can solve the general:

Given $ \alpha\ne 0$ be a real number. Find all functions $ f: \mathbb{R}\to\mathbb{R}$ such that 
for all $ x,y\in\mathbb{R},$
$ f\left(x+xf(y)\right)=f(x)+\alpha yf(x).$

And other interesting problems (easy):

\begin{bolded}1.\end{bolded}Find all values for a real parameter $ \alpha$ for which there exists a non constant function $ f: \mathbb{R}\to\mathbb{R}$ such that
for all $ x,y\in\mathbb{R},$
$ f\left(x+xf(y)\right)=\alpha f(x)+yf(x).$

\begin{bolded}2.\end{bolded}Find all values for a real parameter $ \alpha$ for which there exists a non constant function $ f: \mathbb{R}\to\mathbb{R}$ such that
for all $ x,y\in\mathbb{R},$
$ \alpha f\left(x+xf(y)\right)=f(x)+yf(x).$
\end{solution}
*******************************************************************************
-------------------------------------------------------------------------------

\begin{problem}[Posted by \href{https://artofproblemsolving.com/community/user/78444}{Babai}]
	If $f: \mathbb R \to \mathbb R$ is a function such that $|f(x)| \leq 1$ and 
\[ f\left(x +\frac{15}{56}\right) + f(x) = f\left(x +\frac 18 \right) + f\left(x + \frac 17\right)\]
for all reals $x$, then prove that $f$ is periodic.
	\flushright \href{https://artofproblemsolving.com/community/c6h343855}{(Link to AoPS)}
\end{problem}



\begin{solution}[by \href{https://artofproblemsolving.com/community/user/29428}{pco}]
	\begin{tcolorbox}If f is a function from real to real such that |f(x)| <= 1 and 
 
 f(x +15\/56) + f(x) = f(x +1\/8) + f(x + 1\/7) then prove that f is periodic\end{tcolorbox}

Just for easier writing, let $ g(x)=f(\frac x{56})$ and the equation is $ g(x+15)+g(x)=g(x+7)+g(x+8)$

Let then $ h(x)=g(x+7)-g(x)$ so that the equation becomes $ h(x+8)=h(x)$

From $ h(x)=g(x+7)-g(x)$, we get :
$ g(x+56)=g(x)+(h(x)+$ $ h(x+7)+h(x+14)+$ $ h(x+21)+h(x+28)+$ $ h(x+35)+h(x+42)+$ $ h(x+49))$ $ =g(x)+k(x)$

And since $ h(x+8)=h(x)$, we get $ k(x+8)=k(x)$ and $ k(x+56)=k(x)$ and so $ g(x+56n)=g(x)+nk(x)$

Then, if $ k(x)\ne 0$, we can find $ n$ great enough to have $ |g(x+56n)|>1$, which is impossible.

So $ k(x)=0$ and so $ g(x+56)=g(x)$ and so $ \boxed{f(x+1)=f(x)}$
Q.E.D.
\end{solution}



\begin{solution}[by \href{https://artofproblemsolving.com/community/user/44887}{Mathias_DK}]
	\begin{tcolorbox}If f is a function from real to real such that |f(x)| <= 1 and 
 
 f(x +15\/56) + f(x) = f(x +1\/8) + f(x + 1\/7) then prove that f is periodic\end{tcolorbox}
Let $ a = \frac{1}{7}, b = \frac{1}{8}$ then $ a+b = \frac{15}{56}$, so we have:
$ f(x+a+b) + f(x) = f(x+a) + f(x+b) \iff$
$ f( (x+a)+b) - f(x+a) = f(x+b) - f(x)$
Let $ g(x) = f(x+b) - f(x)$
The fact that $ g(x)$ is periodic with period $ a$ implies that $ h(x) = g(x+7b) + g(x+6b) + \cdots + g(x)$ is periodic with period $ a$.
And $ h(x) = f(x+8b) - f(x) = f(x+1) - f(x)$.
Now $ h(x+1) = h(x+7a) = h(x+6a) = \cdots = h(x)$, so $ h$ is periodic with period $ 1$.
So assume that for some $ x_0$ we have $ f(x_0+1) \neq f(x_0) \iff h(x_0) = f(x_0+1) - f(x_0) = c \neq 0$.
Then $ f(x_0+n) - f(x_0) = h(x_0) + h(x_0+1) + \cdots + h(x_0+(n-1)) = nc$. But $ n > \frac{1}{|c|} \Rightarrow |nc| > 1$, and therefore we obtain a contradiction. So $ f(x+1) \neq f(x)$ is not possible and $ f(x) = f(x+1) \forall x \in \mathbb{R}$, and $ f$ is periodic.
\end{solution}
*******************************************************************************
-------------------------------------------------------------------------------

\begin{problem}[Posted by \href{https://artofproblemsolving.com/community/user/78770}{thuanspdn}]
	Find all functions $ f,g: \mathbb{R}\to\mathbb{R}$ such that
\[2f(x)-g(x) = f(y)-y \quad \text{and} \quad f(x) \cdot g(x) \ge x +1 \]
for all $x,y \in \mathbb {R}$.
	\flushright \href{https://artofproblemsolving.com/community/c6h344357}{(Link to AoPS)}
\end{problem}



\begin{solution}[by \href{https://artofproblemsolving.com/community/user/29428}{pco}]
	\begin{tcolorbox}Find all function $ f,g: \mathbb{R}\longrightarrow \mathbb{R}$ ,such that
$ \{\begin{array}{c}\ \ 2f(x) - g(x) = f(y) - y \
f(x).g(x) \ge x + 1 \end{array} $              , $ \forall x,y \in \mathbb {R}$\end{tcolorbox}

Setting $ x = 0$ in the first equation, we get $ f(y) = y + a$ for some $ a$
Plugging this in the first equation, we get $ g(x) = 2x + a$

Plugging in second line, we get $ (2x + a)(x + a)\ge x + 1$ $ \forall x$

And so $ 2x^2 + (3a - 1)x + a^2 - 1\ge 0$ $ \forall x$

And so $ (3a - 1)^2 - 8(a^2 - 1)\le 0$ 

And so $ a^2 - 6a + 9\le 0$

And so $ a = 3$

And the unique solution (after having easily checked this indeed was a solution) :
$ f(x) = x + 3$
$ g(x) = 2x + 3$
\end{solution}



\begin{solution}[by \href{https://artofproblemsolving.com/community/user/54772}{enndb0x}]
	\begin{tcolorbox}

Setting $ x = 0$ in the first equation, we get $ f(y) = y + a$ for some $ a$
Plugging this in the first equation, we get $ g(x) = 2x + a$
\end{tcolorbox}
Can you explain how we get $ g(x)=2x +a$ ? I think plugging $ f(y)=y+a$ into first equation we have: $ 2f(x) -g(x)=(y+a) -y \implies g(x)=2f(x) -a$
\end{solution}



\begin{solution}[by \href{https://artofproblemsolving.com/community/user/61082}{Pain rinnegan}]
	\begin{tcolorbox}[quote="pco"]

Setting $ x = 0$ in the first equation, we get $ f(y) = y + a$ for some $ a$
Plugging this in the first equation, we get $ g(x) = 2x + a$
\end{tcolorbox}
Can you explain how we get $ g(x) = 2x + a$ ? I think plugging $ f(y) = y + a$ into first equation we have: $ 2f(x) - g(x) = (y + a) - y \implies g(x) = 2f(x) - a$\end{tcolorbox}

He got $ f(y)=y+a\ ,\ (\forall)y\in \mathbb{R}$. So for $ y=x$, he finds $ f(x)=x+a$. Now you only need to substitute in your last relation.
\end{solution}



\begin{solution}[by \href{https://artofproblemsolving.com/community/user/78770}{thuanspdn}]
	@pco:Nice your proof :P ,Thank you very much
\end{solution}
*******************************************************************************
-------------------------------------------------------------------------------

\begin{problem}[Posted by \href{https://artofproblemsolving.com/community/user/78770}{thuanspdn}]
	Find all continuous functions$f, g: \mathbb R \to \mathbb R$ such that
\[ f(x^2) - f(y^2) = g(x-y).g(x + y)\]
for all $ x,y \in \mathbb R$.
	\flushright \href{https://artofproblemsolving.com/community/c6h344946}{(Link to AoPS)}
\end{problem}



\begin{solution}[by \href{https://artofproblemsolving.com/community/user/29428}{pco}]
	\begin{tcolorbox}Find all continuous functions: $ f(x), g(x)$  on $ R$ such that

 $ f(x^2) - f(y^2) = g(x - y).g(x + y)$    , for  all  $ x,y \in R$\end{tcolorbox}

This is, according to me, a rather ugly exercise for an olympiad test.
here is my proof.
There is maybe a simpler method by trying to show that $ g(x)$ is differentiable and using then some differential equation (I did nit look in this direction).

Let $ P(x,y)$ be the assertion $ f(x^2) - f(y^2) = g(x + y)g(x - y)$

$ P(0,0)$ $ \implies$ $ g(0) = 0$
$ P(x,0)$ $ \implies$ $ f(x^2) - f(0) = g(x)^2$

So $ f(x^2) = f(0) + g(x)^2$ and $ g(x)$ verifies the new assertion $ Q(x,y)$ : $ g(x)^2 - g(y)^2 = g(x + y)g(x - y)$


$ g(x) = 0$ is a solution of $ Q(x,y)$. Let us from now consider non all-zero solutions.

1) If $ \exists u\ne 0$ such that $ g(u) = 0$, then $ g(x) = c\sin(ax)$ for some reals $ a,c$
======================================================
[hide="Long proof"]
Notice that if $ g(t) = 0$ then $ Q(x + t,x)$ $ \implies$ $ g(x + t)^2 = g(x)^2$ and so $ g(x)^2$ is a continuous periodic function and 

$ t$ is one of its periods.

$ Q(u, - u)$ $ \implies$ $ g( - u) = 0$
Let then $ A = \{x > 0$ such that $ g(x) = 0\}$. This set is non empty since $ |u|\in A$
Let $ a = \inf(A)$ : continuity implies $ g(a) = 0$

If $ a = 0$, we can find $ t$ as little as we want in A and so $ g(x)^2$ is perodic with periods as little as we want. So $ g (x)^2 = c$ (continuity) and so $ g(x) = 0$ $ \forall x$, which is wrong.
So $ a > 0$
Since $ g(x)$ solution implies $ bg(cx)$ solution, WLOG say $ a = \pi$ and $ g(x) > 0$ $ \forall x\in(0,a)$

Let then $ q > 2\in\mathbb N$ and the sequence $ a_n = g(n\frac {\pi}q)$ such that $ a_q = 0$ and $ a_n > 0$ $ \forall n < q$
$ Q((n + 2)\frac {\pi}q,\frac {\pi}q)$ $ \implies$ $ a_{n + 2}^2 - a_1^2 = a_{n + 3}a_{n + 1}$
$ Q((n + 1)\frac {\pi}q,\frac {\pi}q)$ $ \implies$ $ a_{n + 1}^2 - a_1^2 = a_{n + 2}a_n$
Subtracting : $ a_{n + 2}^2 - a_{n + 1}^2 = a_{n + 3}a_{n + 1} - a_{n + 2}a_n$
$ \iff$ $ a_{n + 2}(a_{n + 2} + a_n) = a_{n + 1}(a_{n + 1} + a_{n + 3})$
And so $ \frac {a_{n + 2} + a_n}{a_{n + 1}} = \frac {a_{n + 1} + a_{n + 3}}{a_{n + 2}}$ $ \forall n < q - 2$

which implies $ \frac {a_{n + 2} + a_n}{a_{n + 1}} = \frac {a_1 + a_3}{a_2}$ $ \forall n\le q - 1$
$ Q(2\frac {\pi}q,\frac {\pi}q)$ $ \implies$ $ a_{2}^2 - a_1^2 = a_{3}a_1$ and so $ a_3 = \frac {a_2^2}{a_1} - a_1$

So $ \frac {a_{n + 2} + a_n}{a_{n + 1}} = \frac {a_2}{a_1}$ which implies $ a_{n + 2} = \frac {a_2}{a_1}a_{n + 1} - a_n$ $ \forall n < q$

From this, we get that $ \frac {a_2}{a_1} < 2$, else $ a_{n}$ would be non decreasing and $ a_q$ could not be $ 0$

It's then easy to establish $ a_n = a_1\frac {\sin(n\theta)}{\sin(\theta)}$ where $ \frac {a_2}{a_1} = 2\cos(\theta)$ $ \forall n\le q + 1$ with $ \theta\in[0,\frac {\pi}2)$

Then $ a_q = 0$ implies $ \theta = \frac {k\pi}q$ with $ k < \frac q2$ and so $ k = 1$ (else some $ a_n$ would be nefative for $ n < q$

And so ${ g(n\frac {\pi}q) = g(\frac\{pi}q)\frac {\sin(n\frac {\pi}q)}{\sin(\frac {\pi}q)}$ $ \forall q > 2$, $ \forall n\in[1,q]$

So $ \frac {g(n\frac {\pi}q)}{\sin(n\frac {\pi}q)} = \frac {g(m\frac {\pi}q)}{\sin(m\frac {\pi}q)}$ $ \forall q > 2$, $ \forall m,n\in [1,q)$

And so, with continuity, $ g(x) = c\sin(x)$ $ \forall x\in[0,\pi]$
Then $ Q(x + \pi,x)$ implies  $ g(x + \pi) = \pm g(x)$
But $ g(\pi,x)$ $ \implies$ $ g(\pi + x)g(\pi - x)\le 0$ 

And so $ g(x) = c\sin(x)$ $ \forall x$
And the general solution in this case is $ g(x) = c\sin(bx)$
Q.E.D.
[\/hide]

2) If $ g(x)\ne 0$ $ \forall x\ne 0$, the only solutions for $ g(x)$ are $ g(x) = cx$ and $ g(x) = c\sinh(ax)$
================================================================
[hide="Another long proof"]
Wlog say $ g(x) > 0$ $ \forall x > 0$ (remember $ g(x)$ is continuous)
Let then $ x > 0$ and the sequence $ a_n = g(nx)$
$ Q((n + 2)x,x)$ $ \implies$ $ a_{n + 2}^2 - a_1^2 = a_{n + 3}a_{n + 1}$
$ Q((n + 1)x,x)$ $ \implies$ $ a_{n + 1}^2 - a_1^2 = a_{n + 2}a_n$
Subtracting : $ a_{n + 2}^2 - a_{n + 1}^2 = a_{n + 3}a_{n + 1} - a_{n + 2}a_n$
$ \iff$ $ a_{n + 2}(a_{n + 2} + a_n) = a_{n + 1}(a_{n + 1} + a_{n + 3})$

And so $ \frac {a_{n + 2} + a_n}{a_{n + 1}} = \frac {a_{n + 1} + a_{n + 3}}{a_{n + 2}}$

which implies $ \frac {a_{n + 2} + a_n}{a_{n + 1}} = \frac {a_1 + a_3}{a_2}$
$ Q(2x,x)$ $ \implies$ $ a_{2}^2 - a_1^2 = a_{3}a_1$ and so $ a_3 = \frac {a_2^2}{a_1} - a_1$

So $ \frac {a_{n + 2} + a_n}{a_{n + 1}} = \frac {a_2}{a_1}$ which implies $ a_{n + 2} = \frac {a_2}{a_1}a_{n + 1} - a_n$

It's easy to see that if $ \frac {a_2}{a_1} < 2$, then $ a_n = a_1\frac {\sin(n\theta)}{\sin(\theta)}$ where $ \frac {a_2}{a_1} = 2 \cos(\theta)$ $ \forall n$ with $ \theta\in[0,\frac {\pi}2)$
And this quantity is negative for infinitely many $ n$, which is impossible.

So $ \frac {a_2}{a_1}\ge 2$

2.1) If $ \frac {a_2}{a_1} = 2$, the only solutions are $ g(x) = cx$
-------------------------------------------------------------
[hide="rather short"]
If $ \frac {a_2}{a_1} = 2$, the equation for the sequence becomes $ a_{n + 2} = 2a_{n + 1} - a_n$ and so $ a_n = na_1$

Then $ g(nx) = ng(x)$ $ \forall x,\forall n\in\mathbb N$ and so $ g(x) = g(1)x$ $ \forall x\in\mathbb Q +$ and so $ g(x) = cx$ $ \forall x\in\mathbb R +$
Comparing then $ Q(x,0)$ and $ Q(0,x)$, we get get $ g(x) = g( - x)$ and so $ g(x) = cx$ $ \forall x\in\mathbb R$
Q.E.D.
[\/hide]


2.2) If $ \frac {a_2}{a_1} > 2$, the only solutions are $ g(x) = c\sinh(ax)$
-------------------------------------------------------------
[hide="mean length"]
If $ \frac {a_2}{a_1} > 2$, the characteristic equation $ x^2 - \frac {a_2}{a_1}x + 1 = 0$ has two real roots $ r_1$ and $ r_2$ and :
$ a_n = a_1\frac {\sinh(nu)}{\sinh(u)}$ where $ u > 0$ is such that $ \frac {a_2}{a_1} = 2\cosh(u)$

Let then $ u(x) = %Error. "argcosh" is a bad command.
(\frac {g(2x)}{2g(x)})$ : we get $ g(nx) = g(x)\frac {\sinh(nu(x))}{\sinh(u(x))}$

But $ \frac {g(2nx)}{2g(nx)} = \frac {a_{2n}}{2a_n}$ $ = \frac {\sinh(2nu)}{2\sinh(nu)}$ $ = \cosh(nu)$
And so $ u(nx) = nu(x)$ and so $ u(x) = cx$ since $ u(x)$ is continuous.

So $ g(nx) = g(x)\frac {\sinh(cnx)}{\sinh(cx)}$ and so $ \frac {g(nx)}{\sinh(cnx)} = \frac {g(x)}{\sinh(cx)}$

And so $ \frac {g(x)}{\sinh(cx)} = d$ $ \forall x\in\mathbb Q^ +$ 

And so $ \frac {g(x)}{\sinh(cx)} = d$ $ \forall x\in\mathbb R^ +$ 

And so $ g(x) = d\sinh(cx)$  $ \forall x\in\mathbb R^ +$ 

And since $ g( - x) = - g(x)$ : $ g(x) = d\sinh(cx)$  $ \forall x\in\mathbb R$ 
Q.E.D
[\/hide] :)
[\/hide]


3) Solutions of the original problem.
=======================
We found three families of solution for $ g(x)$ (I consider the family $ g(x) = 0$ is included thru $ c = 0$) :
$ g(x) = c\sin(ax)$
$ g(x) = cx$
$ g(x) = c\sinh(ax)$

We had $ f(x^2) = f(0) + g(x)^2$ and so $ f(x)$ is fully defined for any $ x\ge 0$
And clearly $ f(x)$ may be any continuous function $ x\le 0$ (with continuity at $ 0$)

Hence :
3.1) $ \sin$ based solutions 
---------------------------
$ f(x) = b + c^2\sin(a\sqrt x)^2$ $ \forall x\ge 0$ and $ f(x)$ may be any continuous function with $ \lim_{x\to 0^ - }f(x) = b$ when $ x < 0$
$ g(x) = c\sin(ax)$

3.2) Id based solutions
----------------------
$ f(x) = b + c^2|x|$ $ \forall x\ge 0$ and $ f(x)$ may be any continuous function with $ \lim_{x\to 0^ - }f(x) = b$ when $ x < 0$
$ g(x) = cx$

3.3 $ \sinh$ based solutions
---------------------------
$ f(x) = b + c^2\sinh(a\sqrt x)^2$ $ \forall x\ge 0$ and $ f(x)$ may be any continuous function with $ \lim_{x\to 0^ - }f(x) = b$ when $ x < 0$
$ g(x) = c\sinh(ax)$

And it is easy to check back that these indeed are solutions.
\end{solution}



\begin{solution}[by \href{https://artofproblemsolving.com/community/user/81418}{nogeom}]
	Thanks, pco.  :) 
I appreciate your effort.
\end{solution}
*******************************************************************************
-------------------------------------------------------------------------------

\begin{problem}[Posted by \href{https://artofproblemsolving.com/community/user/52090}{Dumel}]
	Functions $ f,g: \mathbb{R} \to \mathbb{R}$ satisfy
\[ f(x + y) + f(x - y) = 2f(x)g(y)\]
for all $x,y \in \mathbb{R}$. If $ f \not\equiv 0$, prove that $ g(x) \ge - 1$.
	\flushright \href{https://artofproblemsolving.com/community/c6h345315}{(Link to AoPS)}
\end{problem}



\begin{solution}[by \href{https://artofproblemsolving.com/community/user/29428}{pco}]
	\begin{tcolorbox}functions $ f,g: \mathbb{R} \to \mathbb{R}$ satisfy
$ f(x + y) + f(x - y) = 2f(x)g(x)$ for all $ x \in \mathbb{R}$
and
$ f \not\equiv 0$
prove that $ g(x) \ge - 1$\end{tcolorbox}

Wrong :

Let $ f(x)=x$ and $ g(x)=1$ $ \forall x\ne 0$ and $ g(0)=-2$
\end{solution}



\begin{solution}[by \href{https://artofproblemsolving.com/community/user/52090}{Dumel}]
	I edited 1st post
\end{solution}



\begin{solution}[by \href{https://artofproblemsolving.com/community/user/29428}{pco}]
	\begin{tcolorbox}functions $ f,g: \mathbb{R} \to \mathbb{R}$ satisfy
$ f(x + y) + f(x - y) = 2f(x)g(y)$ for all $ x,y \in \mathbb{R}$
and
$ f \not\equiv 0$
prove that $ g(x) \ge - 1$\end{tcolorbox}
Let $ P(x,y)$ be the assertion $ f(x+y)+f(x-y)=2f(x)g(y)$


1) If $ f(0)\ne 0$, then $ g(x)\ge -1$ 
=======================
Let $ u$ such that $ f(u)\ne 0$ (since we know $ f(x)$ is not the all-zero function)

(a) : $ P(u,x)$ $ \implies$ $ f(u+x)+f(u-x)=2f(u)g(x)$
(b) : $ P(u,-x)$ $ \implies$ $ f(u-x)+f(u+x)=2f(u)g(-x)$

(a) - (b) gives $ g(-x)=g(x)$

(a) : $ P(0,x)$ $ \implies$ $ f(x)+f(-x)=2f(0)g(x)$
(b) : $ P(0,2x)$ $ \implies$ $ f(2x)+f(-2x)=2f(0)g(2x)$
(c) : $ P(x,-x)$ $ \implies$ $ f(0)+f(2x)=2f(x)g(x)$
(d) : $ P(-x,x)$ $ \implies$ $ f(0)+f(-2x)=2f(-x)g(x)$
2g(x)*(a)-(b)+(c)+(d) gives $ 2f(0)=4f(0)g(x)^2-2f(0)g(2x)$ and so $ g(2x)=-1+2g(x)^2\ge -1$
Q.E.D.
(notice such couple $ (f(x),g(x))$ with $ f(0)\ne 0$ exist. Choose for example $ (f,g)=(x+1,1)$ 

2) If $ f(0)=0$, then $ g(x)\ge -1$
=====================

2.1) $ \forall x$ such that $ f(x)\ne 0$ : $ g(2x)\ge -1$
------------------------------------------------------
(a) : $ P(x,x)$ $ \implies$ $ f(2x)=2f(x)g(x)$
(b) : $ P(0,x)$ $ \implies$ $ f(x)+f(-x)=0$
(c) : $ P(2x,x)$ $ \implies$ $ f(3x)+f(x)=2f(2x)g(x)$
(d) : $ P(x,2x)$ $ \implies$ $ f(3x)+f(-x)=2f(x)g(2x)$

2g(x)*(a)+(b)+(c)-(d) : $ 2f(x)=-2f(x)g(2x)+4f(x)g(x)^2$ and so $ f(x)(g(2x)+1-g(x)^2)=0$

So, if $ f(x)\ne 0$, $ g(2x)=-1+g(x)^2\ge -1$
Q.E.D.

2.2) $ f(x)=0$ $ \implies$ $ g(2x)\ge -1$
-------------------------------------
Let $ x$ such that $ f(x)=0$
$ P(x,x)$ $ \implies$ $ f(2x)=0$

(a) : $ P(y,2x)$ $ \implies$ $ f(y+2x)+f(y-2x)=2f(y)g(2x)$
(b) : $ P(2x,y)$ $ \implies$ $ f(2x+y)+f(2x-y)=0$
(c) : $ P(0,2x-y)$ $ \implies$ $ f(2x-y)+f(y-2x)=0$

(a)+(b)-(c) : $ f(y+2x)=f(y)g(2x)$
So $ f(y+4x)=f(y+2x)g(2x)=f(y)g(2x)^2$ (*)

$ P(y+2x,2x)$ $ \implies$ $ f(y+4x)+f(y)=2f(y+2x)g(2x)=2f(y)g(2x)^2$ and so $ f(y+4x)=f(y)(2g(2x)^2-1)$ (**)

Comparing (*) and (**), we get $ f(y)(g(2x)^2-1)=0$
And since $ f(x)$ is not the all-zero function, choosing $ y$ such that $ f(y)\ne 0$, we get $ g(2x)^2=1$ and so $ g(2x)\ge -1$
Q.E.D.

Hence the required result
\end{solution}
*******************************************************************************
-------------------------------------------------------------------------------

\begin{problem}[Posted by \href{https://artofproblemsolving.com/community/user/67223}{Amir Hossein}]
	Let $ f: \mathbb{R} \rightarrow \mathbb{R}$ be a function such that \[ f(x)= \displaystyle \frac{\text{arctan} \displaystyle \frac{x}{2}+\arctan \displaystyle \frac{x}{3}}{\text{arccot} \displaystyle \frac{x}{2}+\text{arccot}\displaystyle \frac{x}{3}}, \quad \forall x\in\mathbb{R}.\] Find (with proof) the value of $ f(1)$.
	\flushright \href{https://artofproblemsolving.com/community/c6h345546}{(Link to AoPS)}
\end{problem}



\begin{solution}[by \href{https://artofproblemsolving.com/community/user/29428}{pco}]
	\begin{tcolorbox}Let $ f: \mathbb{R} \rightarrow \mathbb{R}$ be a function such that $ f(x) = \displaystyle \frac {\text{arctan} \displaystyle \frac {x}{2} + \arctan \displaystyle \frac {x}{3}}{\text{arccot} \displaystyle \frac {x}{2} + \text{arccot}\displaystyle \frac {x}{3}}$ ,$ \forall x\in\mathbb{R}$.Find (with proof) the value of $ f(1)$.\end{tcolorbox}

$ \tan(\arctan(\frac 12)+\arctan(\frac 13))=\frac{\tan(\arctan(\frac 12))+\tan(\arctan(\frac 13))}{1-\tan(\arctan(\frac 12))\tan(\arctan(\frac 13))}$ $ =\frac{\frac 12+\frac 13}{1-\frac 12\frac 13}$ $ =1$

So $ \arctan(\frac 12)+\arctan(\frac 13)=\frac {\pi}4$ (after having checked that this sum is indeed in $ (-\frac {\pi}2,\frac{\pi}2))$

So $ \text{arccot}(\frac 12)+\text{arccot}(\frac 13)=\pi-\frac {\pi}4=\frac{3\pi}4$

Hence the result : $ \boxed{f(1)=\frac 13}$
\end{solution}
*******************************************************************************
-------------------------------------------------------------------------------

\begin{problem}[Posted by \href{https://artofproblemsolving.com/community/user/67949}{aktyw19}]
	Does there exist a differentiable function $ f: \mathbb{R} \to\mathbb{R}$ such that $ f(f(x)) = \sin x$ for all real $x$?
	\flushright \href{https://artofproblemsolving.com/community/c6h345705}{(Link to AoPS)}
\end{problem}



\begin{solution}[by \href{https://artofproblemsolving.com/community/user/29428}{pco}]
	\begin{tcolorbox}whether there is a differentiable function  $ f: \mathbb{R} \longrightarrow \mathbb{R}$,such that $ (f \circ f)(x) = \sin x$?\end{tcolorbox}

If "differentiable" means that it exists a derivative at each point, then yes it does.
But it's not sure we can obtain that the derivative is continuous at points $ k\pi$ (but you did not ask for)
\end{solution}



\begin{solution}[by \href{https://artofproblemsolving.com/community/user/29428}{pco}]
	Here is my construction.
Since it is rather heavy, i will be very interested in your own solution, aktyw19.
Thanks for giving it to us.

====== Building a differentiable solution to $ f(f(x)) = \sin(x)$ ======
\begin{italicized}(i'm too lazy to check all the post, so forgive me for unchecked typos and dont hesitate to ask for any explanation)\end{italicized}


We'll build in part I a function defined over $ [0,\frac {\pi}2]$ and such that :
(a) $ \forall x\in [0,\frac {\pi}2]$ : $ \sin(x)\le f(x)\le x$
(b) $ f(x)$ is differentiable over $ [0,\frac {\pi}2]$
(c) $ f'(0^ + ) = + 1$
(d) $ f'(\frac {\pi}2^ - ) = 0$
(e) $ \forall x\in [0,\frac {\pi}2]$ : $ f(f(x)) = \sin(x)$

We'll then show in part II that this function may be extended over $ \mathbb R$ in a solution to the required problem

1) Part I
=========
Let $ a\in(1,\frac {\pi}2)$
Let the sequence $ a_n$ as :
$ a_0 = \frac {\pi}2$
$ a_1 = a$
$ a_{n + 2} = \sin(a_n)$
(It's easy to see thet $ a_n$ is a stricly decreasing sequence whose limit is $ 0$)


Let $ h(x)$ a $ C_2$ function defined over $ [a,\frac {\pi}2]$ such that :
(c1) : $ h(a) = 1$, $ h'(a) = 1$, $ h(\frac {\pi}2) = a$, $ h'(\frac {\pi}2) = 0$ and $ h"(\frac {\pi}2) = - 1$
(c2) : $ h'(x) > 0$ $ \forall x\in[a,\frac {\pi}2)$

1.1) construction (induction) of a sequence of continuous increasing bijections from $ [a_{n + 1},a_n]\to[a_{n + 2},a_{n + 1}]$
--------------------------------------------------------------------------------------------------
Let us define $ h_n(x)$ as an increasing bijection from $ [a_{n + 1},a_n]\to[a_{n + 2},a_{n + 1}]$ as :
$ h_0(x) = h(x)$
$ h_{n + 1}(x) = \sin(h_n^{ - 1}(x))$

Start of induction : 
$ h_0(x) = h(x)$ is a continuous strictly increasing function from $ [a,\frac {\pi}2]\to[1,a]$, so from $ [a_1,a_0]\to[a_2,a_1]$

Induction step :
If $ h_n(x)$ is a continuous increasing bijection from $ [a_{n + 1},a_n]\to[a_{n + 2},a_{n + 1}]$, then :
$ h_{n}^{ - 1}(x)$ exists and is a continuous increasing bijection from $ [a_{n + 2},a_{n + 1}]\to[a_{n + 1},a_{n}]$
$ \sin(h_{n}^{ - 1}(x))$ exists and is a continuous increasing bijection from $ [a_{n + 2},a_{n + 1}]\to[\sin(a_{n + 1}),\sin(a_ {n})]$

So $ \sin(h_{n}^{ - 1}(x))$ exists and is a continuous increasing bijection from $ [a_{n + 2},a_{n + 1}]\to[a_{n + 3},a_{n + 2}]$

So $ h_{n + 1}(x))$ exists and is a continuous increasing bijection from $ [a_{n + 2},a_{n + 1}]\to[a_{n + 3},a_{n + 2}]$

And so we really can build such a sequence.

1.2) $ h_{n + 1}(a_{n + 1}) = h_n(a_{n + 1})$
-----------------------------------
This is a direct consequence of the construction : $ h_{n + 1}(a_{n + 1}) = h_n(a_{n + 1}) = a_{n + 2}$
This will allow us later to claim that the catenation of $ h_k$ is continuous.

1.3) $ h_n(x)$ is differentiable and $ h'_{n + 1}(a_{n + 1}) = h'_n(a_{n + 1})$ $ \forall n\ge 0$
--------------------------------------------------------------------------------------
This is a difficult part which will allow us later to claim that the catenation of $ h_k$ is differentiable.

1.3.1) the difficult situation of $ h_1(x)$
-------------------------------------------
$ h_0(x)$ is differentiable over $ [a_1,a_0]$ and $ h'(x) > 0$ $ \forall x\in [a_1,a_0)$. So $ h_0^{ - 1}(x)$ is differentiable over $ [a_2,a_1)$ (not in $ a_1$ since $ h'(a_0) = 0$)
So $ h_1(x) = \sin(h^{ - 1}(x))$ is differentiable over $ [a_2,a_1)$

For the case $ a_1$ : 
$ \lim_{x\to a_1^ - }\frac {h_1(a_1) - h_1(x)}{a_1 - x}$ $ = \lim_{x\to a_1^ - }\frac {1 - \sin(h^{ - 1}(x))}{a_1 - x}$ $ = \lim_{x\to a_0^ - } \frac {1 - \sin(x)}{1 - h(x)}$

And $ \lim_{x\to a_0^ - }\frac {1 - \sin(x)}{1 - h(x)} = 1$ since $ h(a_0) = 1$, $ h'(a_0) = 0$ and $ h"(a_0) = - 1$
So $ h_1(x)$ is differentiable over $ [a_2,a_1]$

And $ h_1'(x) > 0$ $ \forall x\in[a_2,a_1]$
Last, we got : $ h_1'(a_1) = 1$ and we know that $ h_0'(a_1) = h'(a) = 1$

So $ h_n(x)$ is differentiable and $ h'_{n + 1}(a_{n + 1}) = h'_n(a_{n + 1})$ for $ n = 0$

1.3.2) general case.
-------------------
$ h_{n + 1}(x) = \sin(h_n^{ - 1}(x))$ allows to conclude that $ h_{n + 1}(x)$ is differentiable since $ h_n(x)$ is differentiable and 

such that $ h_n'(x) > 0$ (the case $ h_0'(a_0) = 0$ was a peculiar case we dealt with in 1.3.1. above).

$ h_{n + 1}'(a_{n + 1}) = \cos(h_n^{ - 1}(a_{n + 1}))h_n^{ - 1}'(a_{n + 1})$ $ = \frac {\cos(a_n)}{h_n'(h_n^{ - 1}(a_{n + 1}))}$ $ = \frac {\cos (a_n)}{h_n'(a_n)}$

$ h_n'(a_{n + 1}) = \cos(h_{n - 1}^{ - 1}(a_{n + 1}))h_{n - 1}^{ - 1}'(a_{n + 1})$ $ = \frac {\cos(a_n)}{h_{n - 1}'(h_{n - 1}^{ - 1}(a_{n + 1}))}$ $ = \frac {\cos(a_n)}{h_{n - 1}'(a_n)}$

And since induction step tells us that $ h_n'(a_n) = h_{n - 1}'(a_n)$, we get $ h_{n + 1}'(a_{n + 1}) = h_n'(a_{n + 1})$

1.4) Construction of $ f(x)$ over $ [0,\frac {\pi}2]$
--------------------------------------------------
We can then define $ f(x)$ as :

$ f(0) = 0$
$ \forall x\in(a_{n + 1},a_n]$ : $ f(x) = h_n(x)$

1.5) $ \sin(x) \le f(x)\le x$ $ \forall x\in[0,\frac {\pi}2]$
----------------------------------------------------------

For $ x = 0$, this is obviously true (with equality)

For  $ x\in (a_{n + 1},a_n]$ and since $ h_n(x)$ is increasing : 
$ h_n(x) > h_n(a_{n + 1}) = a_{n + 2} = \sin(a_{n})\ge\sin(x)$
$ h_n(x)\le h_n(a_n) = a_{n + 1} < x$
Q.E.D


1.6) $ f(x)$ is continuous over $ [0,\frac {\pi}2]$
-----------------------------------------------
Continuity over $ (a_{n + 1},a_n)$ is obtained thru continuity of $ h_n(x)$ (see paragraph 1.1)
Continuity at each $ a_n$ is obtained thru $ h_{n + 1}(a_{n + 1}) = h_n(a_{n + 1})$ (see paragraph 1.2)
Continuity at $ 0$ may be obtained thru 1.5 above : $ \sin(x) \le f(x)\le x$ $ \implies$ $ \lim_{x\to 0 + }f(x) = 0 = f(0)$

1.7) $ f(x)$ is differentiable over $ [0,\frac {\pi}2]$
-------------------------------------------------------
Differentiability over $ (a_{n + 1},a_n)$ is obtained thru differentiability of $ h_n(x)$ (see paragraph 1.3)
Differentiability at each $ a_n$ is obtained thru $ h'_{n + 1}(a_{n + 1}) = h'_n(a_{n + 1})$ (see paragraph 1.3)
Differentiability at $ 0$ may be obtained thru 1.5 above : 
$ \sin(x) \le f(x)\le x$ $ \implies$ $ \frac {\sin(x)}x \le \frac {f(x) - f(0)}{x - 0}\le 1$ and so $ f'(0) = 1$

Notice that nothing allow us to think that $ f'(x)$ is continuous at $ 0$.

1.8) $ f(f(x)) = \sin(x)$ $ \forall x\in[0,\frac {\pi}2]$
-----------------------------------------------------

For $ x = 0$, $ f(f(x)) = f(0) = 0 = \sin(0)$

$ \forall x\in(a_{n + 1},a_n]$, $ f(x) = h_n(x)\in(a_{n + 2},a_n]$ and so $ f(f(x)) = h_{n + 1}(h_n(x))$

And since $ h_{n + 1}(x) = \sin(h_n^{ - 1}(x))$, we get $ f(f(x)) = \sin(h_n^{ - 1}(h_n(x))) = \sin(x)$

1.9) Conclusion of part I
-------------------------
So we built a function $ f(x)$ such that :
(a) $ \forall x\in [0,\frac {\pi}2]$ : $ \sin(x)\le f(x)\le x$ (see 1.5)
(b) $ f(x)$ is differentiable over $ [0,\frac {\pi}2]$ (see 1.7)
(c) $ f'(0^ + ) = + 1$ (see 1.7)
(d) $ f'(\frac {\pi}2^ - ) = 0$ (see definition of $ h(x)$)
(e) $ \forall x\in [0,\frac {\pi}2]$ : $ f(f(x)) = \sin(x)$ (see 1.8)

2) Part II
===========
2.1) extension over $ [0,\pi]$
----------------------------
Extend $ f(x)$ over $ (\frac {\pi}2,\pi]$ with $ f(x) = f(\pi - x)$
Since $ f'(\frac {\pi}2^ - ) = 0$, the extension is differentiable over $ [0,\pi]$

$ \forall x\in(\frac {\pi}2,\pi]$ : $ f(f(x)) = f(f(\pi - x)) = \sin(\pi - x) = \sin(x)$

2.2) extension over $ [ - \pi,\pi]$
--------------------------------
Extend $ f(x)$ over $ [ - \pi,0)$ with $ f(x) = - f( - x)$
Since $ f'(0 + ) = 1$, the extension is differentiable at $ 0$ and so over $ [ - \pi,\pi]$

$ \forall x\in[ - \pi,0)$ : $ f(f(x)) = f( - f( - x)) = - f( - ( - f( - x))) = - f(f( - x)) = - \sin( - x) = \sin(x)$

2.3) extension over $ \mathbb R$
------------------------------
Using $ f(x + 2\pi) = f(x)$ and $ f(x - 2\pi) = f(x)$, it's clear that we can extend over $ \mathbb R$ keeping differentiability.
\end{solution}



\begin{solution}[by \href{https://artofproblemsolving.com/community/user/67949}{aktyw19}]
	very nice, you are the best \begin{bolded}pco\end{bolded}  :first:
\end{solution}



\begin{solution}[by \href{https://artofproblemsolving.com/community/user/29428}{pco}]
	\begin{tcolorbox}very nice, you are the best \begin{bolded}pco\end{bolded}  :first:\end{tcolorbox}

Thanks. Could you give us your own solution, as required, please.
\end{solution}



\begin{solution}[by \href{https://artofproblemsolving.com/community/user/67949}{aktyw19}]
	Unfortunately, I do not have a solution to this problem :oops:
\end{solution}



\begin{solution}[by \href{https://artofproblemsolving.com/community/user/29428}{pco}]
	\begin{tcolorbox}Unfortunately, I do not have a solution to this problem :oops:\end{tcolorbox}
Please, read the forum definitions and respect the usage. It's just a matter of politeness :

\begin{italicized}Unsolved category\end{underlined}\end{italicized} : "Problems you couldn't solve and to which you know that there is a solution (i.e. a problem from a contest, etc.) but you don't know it."

\begin{italicized}Proposed and own category\end{underlined}\end{italicized} : "Problems that \begin{bolded}you have already solved \end{bolded}\end{underlined}and you are interested in second opinions or solutions."

\begin{italicized}Open category\end{underlined}\end{italicized} : "An open question is a question that has no known solution up to this moment, and it is not known wheter the problem has one or not."

Some users are interested in only some categories (personnaly, I generally spend more time on "proposed and own" than on "Open" since I'm sure there is a clever answer (olympiad level) and I know, when I fail findind it, that original poster will post the answer when asked.
\end{solution}



\begin{solution}[by \href{https://artofproblemsolving.com/community/user/67949}{aktyw19}]
	you do not need to respond to my posts
\end{solution}



\begin{solution}[by \href{https://artofproblemsolving.com/community/user/29428}{pco}]
	\begin{tcolorbox}you do not need to respond to my posts\end{tcolorbox}

Sure, but why dont you respect the rules ?
If you have the solution : post in this category
If you dont and know there is one : post in "unsolved"
If you dont know there is one : post in "open"

That is simple and honest.
Posting in wrong forum is just unfair for other users.
And it's not an error from you : you were already told about this problem.
\end{solution}
*******************************************************************************
-------------------------------------------------------------------------------

\begin{problem}[Posted by \href{https://artofproblemsolving.com/community/user/78843}{lethanhdatvn}]
	Find all functions $f: \mathbb R \to \mathbb R$ such that for all reals $x$ and $y$,
\[ f(f(x)+y)= f(x+y) + xf(y) -xy - x + 1.\]
	\flushright \href{https://artofproblemsolving.com/community/c6h345834}{(Link to AoPS)}
\end{problem}



\begin{solution}[by \href{https://artofproblemsolving.com/community/user/29428}{pco}]
	\begin{tcolorbox}Find all $ f : R \rightarrow R$ such that $ :$
$ f(f(x) + y) = f(x + y) + xf(y) - xy - x + 1$\end{tcolorbox}
Let $ P(x,y)$ be the assertion $ f(f(x)+y)=f(x+y)+xf(y)-xy-x+1$

$ P(0,x)$ $ \implies$ $ f(x+f(0))=f(x)+1$

$ P(x,f(0))$ $ \implies$ $ f(f(x)+f(0))=f(x+f(0))+xf(f(0))-xf(0)-x+1$ and so $ f(f(x))+1=f(x)+1+x(f(0)+1)-xf(0)-x+1$ and so :

$ P(x,f(0))$ $ \implies$ $ f(f(x))=f(x)+1$
$ P(x,0)$ $ \implies$ $ f(f(x))=f(x)+xf(0)-x+1$

Subtracting and setting $ x=1$ gives $ f(0)=1$ and so $ f(x+1)=f(x)+1$

Then $ P(x+1,y)$ $ \implies$ $ f(f(x)+y)+1=f(x+y)+1+(x+1)f(y)-(x+1)y-(x+1)+1$ and so $ f(f(x)+y)=f(x+y)+(x+1)f(y)-(x+1)y-x$
Subtracting $ P(x,y)$ from the previous line gives $ 0=f(y)-y-1$

And so $ \boxed{f(x)=x+1}$ $ \forall x$ which indeed is a solution.
\end{solution}



\begin{solution}[by \href{https://artofproblemsolving.com/community/user/44887}{Mathias_DK}]
	\begin{tcolorbox}Find all $ f : R \rightarrow R$ such that $ :$
$ f(f(x) + y) = f(x + y) + xf(y) - xy - x + 1$\end{tcolorbox}
Define $ g(x) = f(x)-x-1 \iff f(x) = g(x)+x+1$. Then insert:
$ g(g(x)+x+y+1)+g(x) = g(x+y) + xg(y) \iff Q(x,y)$
$ Q(0,y) \iff g(g(0)+1+y) + g(0) = g(y)$. Let $ c = g(0)+1$:
$ g(x+c) = g(y) + (1-c)$
$ Q(x,y+c) \iff g(g(x)+x+y+c+1)+g(x) = g(x+y+c) + xg(y+c) \iff$
$ g(g(x)+x+y+1)+g(x) + 1-c = g(x+y)+1-c + xg(y) + x(1-c) \Rightarrow$
$ x(1-c) = 0 \Rightarrow 1=c$.
So: $ g(x+1) = g(x)$.
$ Q(x+1,y) \iff g(g(x+1)+x+1+y+1) + g(x+1) = g(x+1+y) + (x+1)g(y) \iff$
$ g(g(x)+x+y+1)+g(x) = g(x+y) + (x+1)g(y) \Rightarrow$
$ xg(y) = (x+1)g(y) \iff g(y) = 0$.
So $ g(x) = 0 \forall x \in \mathbb{R}$, and hence:

$ f(x) = x+1 \forall x \in \mathbb{R}$. This is easily seen to be a solution.
\end{solution}
*******************************************************************************
-------------------------------------------------------------------------------

\begin{problem}[Posted by \href{https://artofproblemsolving.com/community/user/70363}{hoangvn.fix}]
	Suppose that the function $f: \mathbb R^{+} \to \mathbb R^{+}$ satisfies
\[ f(3x)\ge f\left(\frac{1}{2}f(2x)\right)+2x\]
for all the $x>0$. Prove that $ f(x) \ge x$ for all $x>0$.
	\flushright \href{https://artofproblemsolving.com/community/c6h345882}{(Link to AoPS)}
\end{problem}



\begin{solution}[by \href{https://artofproblemsolving.com/community/user/29428}{pco}]
	\begin{tcolorbox}Let the function $ f: R_{ + }\longrightarrow R_{ + }$ satisfy 
$ f(3x)\ge f(\frac {1}{2}f(2x)) + 2x$ for all the positive x
Prove that:$ f(x) \ge x$ for all the positive x\end{tcolorbox}
Suppose we proved that $ f(x) > ax$ $ \forall x > 0$ and for some $ a > 0$

Then $ f(\frac 12f(2x)) > \frac a2f(2x) > a^2x$ and $ f(3x) > (a^2 + 2)x$ and so $ f(x) > \frac {a^2 + 2}3x$

And since we obviously have $ f(3x) > 2x$ and so $ f(x) > \frac 23x$, we get $ f(x) > a_nx$ for any element $ a_n$ of the sequence :
$ a_1 = \frac 23$
$ a_{n + 1} = \frac {a_n^2 + 2}3$

And it's quite easy to show that $ a_n$ is an increasing sequence whose limit is $ 1$ and so $ f(x)\ge x$ $ \forall x$
Q.E.D.
\end{solution}
*******************************************************************************
-------------------------------------------------------------------------------

\begin{problem}[Posted by \href{https://artofproblemsolving.com/community/user/75406}{seifi-seifi}]
	Find all functions $f: \mathbb R \to \mathbb R$ such that for all reals $x$ and $y$,
\[f(xf(y))+f(f(x)+f(y))=yf(x)+f(x+f(y)).\]
	\flushright \href{https://artofproblemsolving.com/community/c6h346289}{(Link to AoPS)}
\end{problem}



\begin{solution}[by \href{https://artofproblemsolving.com/community/user/29428}{pco}]
	\begin{tcolorbox}find all function $f : R\rightarrow R$ such that : 

$f(xf(y))+f(f(x)+f(y))=yf(x)+f(x+f(y))$\end{tcolorbox}

Let $P(x,y)$ be the assertion $f(xf(y))+f(f(x)+f(y))=yf(x)+f(x+f(y))$
Let $a=f(0)$

$f(x)=0$ $\forall x$ is a solution and we'll from now consider non all-zero solutions.
Let then $u$ such that $f(u)\ne 0$

If $f(x_1)=f(x_2)$, comparing $P(u,x_1)$ and $P(u,x_2)$ implies $x_1=x_2$ and so $f(x)$ is injective.

$P(0,1)$ $\implies$  $f(a+f(1))=f(f(1))$ and so, since injective, $a+f(1)=f(1)$ and so $f(0)=a=0$

$P(x,0)$ $\implies$ $f(f(x))=f(x)$ and so, since injective, $f(x)=x$ which indeed is a solution.

Hence the two solutions :
$f(x)=0$ $\forall x$
$f(x)=x$ $\forall x$
\end{solution}
*******************************************************************************
-------------------------------------------------------------------------------

\begin{problem}[Posted by \href{https://artofproblemsolving.com/community/user/73965}{Django}]
	Find all functions $f: \mathbb R \to \mathbb R$ such that for $x,y \in \mathbb R$,
\[f(x-f(y)) = f(f(y)) + xf(y) + f(x) - 1.\]
	\flushright \href{https://artofproblemsolving.com/community/c6h346403}{(Link to AoPS)}
\end{problem}



\begin{solution}[by \href{https://artofproblemsolving.com/community/user/29428}{pco}]
	\begin{tcolorbox}Find all functions $f:R \rightarrow R$ such that for $x,y \in R$:

$f(x-f(y)) = f(f(y)) + xf(y) + f(x) - 1$\end{tcolorbox}

Let $P(x,y)$ be the assertion $f(x-f(y))=f(f(y))+xf(y)+f(x)-1$
$f(x)=0$ $\forall x$ is not a solution. So let $u$ such that $f(u)\ne 0$.

$P(\frac{x+1-f(f(u))}{f(u)}  ,u)$ $\implies$ $x=f\left(\frac{x+1-f(f(u))}{f(u)} -f(u)\right) - f\left(\frac{x+1-f(f(u))}{f(u)}\right ) $  and so any real may be written as $x=f(a)-f(b)$ for some $a,b$

Let then $x\in\mathbb R$ and $a,b$ such that $x=f(a)-f(b)$ : 

$P(f(a),a)$ $\implies$ $f(f(a))=\frac{f(0)+1}2-\frac 12f(a)^2$

$P(f(b),b)$ $\implies$ $f(f(b))=\frac{f(0)+1}2-\frac 12f(b)^2$

$P(f(a),b)$ $\implies$ $f(f(a)-f(b))=f(f(b))+f(a)f(b)+f(f(a))-1$
Adding these three lines, we get :

$f(f(a)-f(b))=f(0)-\frac 12(f(a)-f(b))^2$ and so $f(x)=f(0)-\frac{x^2}2$ $\forall x$

Pluging back in original equation, we get $a=1$

Hence the unique solution : $\boxed{f(x)=1-\frac{x^2}2}$
\end{solution}
*******************************************************************************
-------------------------------------------------------------------------------

\begin{problem}[Posted by \href{https://artofproblemsolving.com/community/user/68983}{jestrada}]
	Find all functions from $\mathbb R$ to $\mathbb R$ satisfying the following condition: for all reals $x$ and $y$, 
\[f(xf(x+y))=f(yf(x))+x^2.\]
	\flushright \href{https://artofproblemsolving.com/community/c6h346514}{(Link to AoPS)}
\end{problem}



\begin{solution}[by \href{https://artofproblemsolving.com/community/user/29428}{pco}]
	\begin{tcolorbox}Find all functions from $R$ to $R$ satisfying the following condition:
For all real $x, y$ $f(xf(x+y))=f(yf(x))+x^2$\end{tcolorbox}
Let $P(x,y)$ be the assertion $f(xf(x+y))=f(yf(x))+x^2$

1) $f(x)=0$ $\iff$ $x=0$
If $f(0)\ne 0$ : $P(0,\frac{x}{f(0)})$ $\implies^$ $f(0)=f(x)$ and so $f(x)$ is constant, which is wrong. So $f(0)=0$
If $f(x)=0$ : $P(x,0)$ $\implies$ $0=x^2$
Q.E.D.

2) $f(xf(x))=x^2$
This is an immediate application of $P(x,0)$ and $f(0)=0$

3) $f(x)$ is injective
$P(a,b-a)$ $\implies$ $f(af(b))=f((b-a)f(a)+a^2=f((b-a)f(a)+f(af(a))$ and so $f((b-a)f(a))=f(af(b))-f(af(a))$
If $f(a)=f(b)$, the previous line implies $f((b-a)f(a))=0$ and so, using 1) $(b-a)f(a)=0$ and :
either $f(a)=f(b)=0$ and so $a=b=0$
either $b-a=0$ and so $a=b$
Q.E.D.

4) $f(-x)=-f(x)$
$f(xf(x))=x^2=(-x)^2=f(-x(f-x))$ and so, since $f(x)$ is injective, $xf(x)=-xf(-x)$ and so $f(-x)=-f(x)$ $\forall x$ (inclding the special case $x=0$)
Q.E.D.

5) $f(x)$ is bijective
Since $f(xf(x))=x^2$, we get that $\mathbb R^+\cup\{0\}\subseteq f(\mathbb R)$
Since $f(-x)=-f(x)$, we get that $f(-xf(x))=-x^2$ and so $\mathbb R^-\subseteq f(\mathbb R)$
Q.E.D.

6) The only two solutions are $f(x)=x$ and $f(x)=-x$
Since $f(x)$ is surjective, let $u$ such that $f(u)=1$
$f(uf(u))=u^2$ becomes $u^2=1$ and so $u=\pm 1$
6.1) If $u=1$
Then $f(1)=1$ and :
$P(1,x-1)$ $\implies$ $f(f(x))=f(x-1)+1$
$P(1,-x-1)$ $\implies$ $f(f(-x))=f(-x-1)+1$ and so $-f(f(x))=-f(x+1)+1$
So $f(x+1)=f(x-1)+2$ and so $f(x+2)=f(x)+2$ and $f(x+4)=f(x)+4$

$P(2,x)$ $\implies$ $f(2f(x+2))=f(xf(2))+4=f(2x+4)$ and so, since $f(x)$ is injective, $2f(x+2)=2x+4$ and $f(x)=x$ which indeed is a solution.

6.2) If $u=-1$
Then $f(-1)=1$ and :
$P(-1,x+1)$ $\implies$ $f(f(x))=f(x+1)+1$
$P(-1,-x+1)$ $\implies$ $f(f(-x))=f(-x+1)+1$ and so $-f(f(x))=-f(x-1)+1$
So $f(x+1)=f(x-1)-2$ and so $f(x+2)=f(x)-2$ and $f(x-4)=f(x)+4$

$P(2,x)$ $\implies$ $f(2f(x+2))=f(xf(2))+4=f(-2x-4)$ and so, since $f(x)$ is injective, $2f(x+2)=-2x-4$ and $f(x)=-x$ which indeed is a solution.
Q.E.D.

Hence the only two solutions :
$f(x)=x$ $\forall x$
$f(x)=-x$ $\forall x$
\end{solution}



\begin{solution}[by \href{https://artofproblemsolving.com/community/user/68983}{jestrada}]
	Thanks, Patrick!
I had proved from 1 to 5, but was stuck there...
\end{solution}



\begin{solution}[by \href{https://artofproblemsolving.com/community/user/213024}{Adrienmath}]
	Here is my solution.

$f(xf(x+y))=f(yf(x))+x^2$ $(1)$

Setting $x=0$ in $(1)$ we get $f(0)=f(yf(0))$ so $f(0)=0$ because otherwise $f$ is constant, impossible.

Setting $y=0$ in $(1)$ we get $f(xf(x))=x^2$ and setting $y=-x$ in $(1)$ we have $f(-xf(x))=-x^2$ and so $f$ is surjective.

Let $a$ such that $f(a)=0$. Setting $x=a$ and $y=0$ in $(1)$ we have $a=0$.

From that we deduce not so hardly that $f$ is invective (but it's a bit long...) so $f$ is bijective.

We have $f(-xf(x))=-x^2=f(xf(-x))$ so $f(-x)=-f(x)$ using injectivity and $f(0)=0$.

Setting $y=-2x$ in $(1)$ we have $f(2a)=2f(a)$ for all $a$ of the form $xf(x)$ and so we have the same relation for all $a$ of the form $-xf(x)$. But it's clear that every number can be written either as $xf(x)$ either as $-xf(x)$ for some real number $x$. So we have $f(2x)=2f(x)$.

Now we have $f(2xf(2x+y))=f(yf(2x))+4x^2$ so $f(xf(x+z))=x^2+f(xf(z))$ (simplify and take $z=x+y$).

Using $(1)$ we deduce $f(zf(x))=f(xf(z))$ so we have that $ f(x)\/x = f(z)\/z $ if $x$ and $z$ are nonzero. $f(0)=0$ so $f$ is linear.

It is now obvious that $f(x)=x$ and $f(x)=-x$ are the only solutions of the problem, QED.
\end{solution}
*******************************************************************************
-------------------------------------------------------------------------------

\begin{problem}[Posted by \href{https://artofproblemsolving.com/community/user/57963}{saeedghodsi}]
	Find all continuous functions $ f:\mathbb{R}\to\mathbb{R} $ such that for any $x \in \mathbb R$,
\[ f(x)=f\left(x^{2}+\frac{x}{3}+\frac{1}{9}\right) .\]
	\flushright \href{https://artofproblemsolving.com/community/c6h346648}{(Link to AoPS)}
\end{problem}



\begin{solution}[by \href{https://artofproblemsolving.com/community/user/29428}{pco}]
	\begin{tcolorbox}find all continuous functions $ f:\mathbb{R}\to\mathbb{R} $ such that :

$ \forall x \in \mathbb R $ $ f(x)=f\left(x^{2}+\frac{x}{3}+\frac{1}{9}\right) $\end{tcolorbox}

If $x\in[-\frac 16,\frac 13)$ :
The sequence $a_0=x$ and $a_{n+1}=a_n^2+\frac{a_n}3+\frac 19$ is an increasing sequence whose limit is $\frac 13$
Since $f(a_{n+1})=f(a_n)$ and $f(x)$ is continuous, we get $f(x)=f(\frac 13)$ $\forall x\in[-\frac 16,\frac 13]$

If $x\in(\frac 13,+\infty)$ :
The sequence $a_0=x$ and $a_{n+1}=\frac{\sqrt{36a_n-3}-1}6$ is a decreasing sequence whose limit is $\frac 13$
Since $f(a_{n+1})=f(a_n)$ and $f(x)$ is continuous, we get $f(x)=f(\frac 13)$ $\forall x\in[\frac 13,+\infty)$

So $f(x)=f(\frac 13)$ $\forall x\in[-\frac 16,+\infty)$

And since $f(x)=f(x^2+\frac x3+\frac 19)$ $=f((-x-\frac 13)^2+\frac{-x-\frac 13}3+\frac 19)=f(-x-\frac 13)$ :

$f(x)=f(\frac 13)$ $\forall x$

Hence the answer : $f(x)=c$ constant.
\end{solution}



\begin{solution}[by \href{https://artofproblemsolving.com/community/user/38228}{Bijection}]
	\begin{tcolorbox}[quote="saeedghodsi"]find all continuous functions $ f:\mathbb{R}\to\mathbb{R} $ such that :

$ \forall x \in \mathbb R $ $ f(x)=f\left(x^{2}+\frac{x}{3}+\frac{1}{9}\right) $\end{tcolorbox}

If $x\in[-\frac 16,\frac 13)$ :
The sequence $a_0=x$ and $a_{n+1}=a_n^2+\frac{a_n}3+\frac 19$ is an increasing sequence whose limit is $\frac 13$
Since $f(a_{n+1})=f(a_n)$ and $f(x)$ is continuous, we get $f(x)=f(\frac 13)$ $\forall x\in[-\frac 16,\frac 13]$

If $x\in(\frac 13,+\infty)$ :
The sequence $a_0=x$ and $a_{n+1}=\frac{\sqrt{36a_n-3}-1}6$ is a decreasing sequence whose limit is $\frac 13$
Since $f(a_{n+1})=f(a_n)$ and $f(x)$ is continuous, we get $f(x)=f(\frac 13)$ $\forall x\in[\frac 13,+\infty)$

So $f(x)=f(\frac 13)$ $\forall x\in[-\frac 16,+\infty)$

And since $f(x)=f(x^2+\frac x3+\frac 19)$ $=f((-x-\frac 13)^2+\frac{-x-\frac 13}3+\frac 19)=f(-x-\frac 13)$ :

$f(x)=f(\frac 13)$ $\forall x$

Hence the answer : $f(x)=c$ constant.\end{tcolorbox}
Great solution, where did you see this technique? I have a question, how did you get the numbers $\frac{-1}{6}$ and $\frac{1}{3}$ as the boundary cases?
\end{solution}



\begin{solution}[by \href{https://artofproblemsolving.com/community/user/29428}{pco}]
	\begin{tcolorbox} Great solution, where did you see this technique? I have a question, how did you get the numbers $\frac{-1}{6}$ and $\frac{1}{3}$ as the boundary cases?\end{tcolorbox}

Thanks.
Just study $g(x)=x^2+\frac x3 +\frac 19$

$x=-\frac 16$ is the minimum and $g(x)$ is monotonous over any interval which does not contain $-\frac 16$

$x=\frac 13$ is the unique solution (double) of $g(x)=x$
\end{solution}
*******************************************************************************
-------------------------------------------------------------------------------

\begin{problem}[Posted by \href{https://artofproblemsolving.com/community/user/57963}{saeedghodsi}]
	Find all functions $ f: \mathbb R^{+} \to \mathbb R^{+} $ such that $ f(x) > (x-y) f(y)^{2} $ for all $x,y \in \mathbb R^{+}$.
	\flushright \href{https://artofproblemsolving.com/community/c6h346649}{(Link to AoPS)}
\end{problem}



\begin{solution}[by \href{https://artofproblemsolving.com/community/user/29428}{pco}]
	\begin{tcolorbox}find all functions $ f: \mathbb R^{+} \to \mathbb R^{+} $ such that :
$ \forall x,y \in \mathbb R^{+} : $ $ f(x) > (x-y) f(y)^{2} $\end{tcolorbox}
Let $P(x,y)$ be the assertion $f(x)>(x-y)f(y)^2$

Let $a=1+\frac 1{f(1)^2}$ : $P(a,1)$ $\implies$ $f(a)>1$

Let then the increasing sequence $a_n$ defined as $a_0=a$ and $a_{n+1}=a_n+\frac 2{f(a_n)}$

$P(a_{n+1},a_n)$ $\implies$ $f(a_{n+1})>2f(a_n)$ and so $f(a_n)>2^n$ and induction shows then that $a_n<a+4$

$P(a+5,a_n)$ $\implies$ $f(a+5)>(a+5-a_n)f(a_n)^2>2^{2n}$ $\forall n$ which is impossible.

So no solution to this functional inequation.
\end{solution}
*******************************************************************************
-------------------------------------------------------------------------------

\begin{problem}[Posted by \href{https://artofproblemsolving.com/community/user/78770}{thuanspdn}]
	Find all continuous functions $f, g: \mathbb R \to \mathbb R$ such that
\[ f(x^2-y^2) = (x+y)  g(x -  y)\]
for  all  $ x,y  \in \mathbb R$.
	\flushright \href{https://artofproblemsolving.com/community/c6h347121}{(Link to AoPS)}
\end{problem}



\begin{solution}[by \href{https://artofproblemsolving.com/community/user/29428}{pco}]
	\begin{tcolorbox}Find all continuous functions: $ f(x), g(x)  $  on $ R$ ,such that

 $ f(x^2-y^2) = (x+y)  g(x -  y)                                  $,  for  all  $ x,y  \in R$\end{tcolorbox}
Let $P(x,y)$ the assertion $f(x^2-y^2)=(x+y)g(x-y)$

$P(0,0)$ $\implies$ $f(0)=0$
$P(1,1)$ $\implies$ $g(0)=0$
$P(x,0)$ $\implies$ $f(x^2)=xg(x)$ and so $g(x)=\frac{f(x^2)}x$ $\forall x\ne 0$

And so the original equation may be written $f(x^2-y^2)=\frac{x+y}{x-y}f((x-y)^2))$ $\forall x\ne y$

So $\frac{f(x^2-y^2)}{x^2-y^2}=\frac{f((x-y)^2)}{(x-y)^2}$ $\forall x\ne \pm y$

For any $x\ne 0$ : let $h(x)=\frac{f(x)}x$
$h(x^2-y^2)=h((x-y)^2)$ $\forall x\ne \pm y$

Let $u>0$ and $v\ne -u$. Set $x=\sqrt u+\frac v{2\sqrt u}$ and $y=\frac v{2\sqrt u}$ in this equation and you get $h(u+v)=h(u)$ and so $h(x)=c$ $\forall x\ne 0$

So $f(x)=cx$ $\forall x\ne 0$ and so $\forall x$ and so $cx^2=xg(x)$ and $g(x)=cx$ $\forall x\ne 0$ and so $\forall x$

Hence the solution : $\boxed{f(x)=g(x)=cx \forall x}$ which indeed is a solution
And we dont need continuity.
\end{solution}
*******************************************************************************
-------------------------------------------------------------------------------

\begin{problem}[Posted by \href{https://artofproblemsolving.com/community/user/78770}{thuanspdn}]
	Find all continuous functions $f, g: \mathbb R \to \mathbb R$ such that
\[f(x)-f(y) = (  x^2 -y^2) g(x-y)\]
for all $ x,y  \in \mathbb R$.
	\flushright \href{https://artofproblemsolving.com/community/c6h347123}{(Link to AoPS)}
\end{problem}



\begin{solution}[by \href{https://artofproblemsolving.com/community/user/29428}{pco}]
	\begin{tcolorbox}Find all continuous functions: $ f(x), g(x) $  on $ R$ ,such that

 $ f(x)-f(y) = (  x^2 -y^2)                g(x-y)$,  for  all  $ x,y  \in R$\end{tcolorbox}
Let $P(x,y)$ be the assertion $f(x)-f(y)=(x^2-y^2)g(x-y)$

(a) : $P(x,0)$ $\implies$ $f(x)-f(0)=x^2g(x)$
(b) : $P(2x,0)$ $\implies$ $f(2x)-f(0)=4x^2g(2x)$
(c) : $P(2x,x)$ $\implies$ $f(2x)-f(x)=3x^2g(x)$
(a)-(b)+(c) : $x^2g(x)=x^2g(2x)$

And so $g(2x)=g(x)$ $\forall x\ne 0$ 
And so $g(2x)=g(x)$ $\forall x$ 
And so $g(x)=g(2^{-n}x)$ $\forall x$ 
Setting $n\to +\infty$ and using continuity, we get $g(x)=g(0)=c$

$P(x,0)$ $\implies$ $f(x)=f(0)+cx^2$

And so $\boxed{f(x)=a+cx^2\text{ and }g(x)=c}$ which indeed are solutions.
\end{solution}



\begin{solution}[by \href{https://artofproblemsolving.com/community/user/13}{enescu}]
	Alternatively, we can write
\[\frac{f(x)-f(y)}{x-y}=(x+y)g(x-y),\]
for $x\ne y.$
Now, if $y\rightarrow x$ we obtain that $f$ is differentiable at $x$ and that $f'(x)=2xg(0).$
Clearly, this implies that $f(x)=g(0)x^2+a,$ for some constant $a$.
Replacing in the original condition, we deduce that $g$ is constant.
\end{solution}



\begin{solution}[by \href{https://artofproblemsolving.com/community/user/29428}{pco}]
	\begin{tcolorbox}Alternatively, we can write
\[\frac{f(x)-f(y)}{x-y}=(x+y)g(x-y),\]
for $x\ne y.$
Now, if $y\rightarrow x$ we obtain that $f$ is differentiable at $x$ and that $f'(x)=2xg(0).$
Clearly, this implies that $f(x)=g(0)x^2+a,$ for some constant $a$.
Replacing in the original condition, we deduce that $g$ is constant.\end{tcolorbox}

Oh yes! Quite nice and simpler than mine :)
Congrats!
\end{solution}
*******************************************************************************
-------------------------------------------------------------------------------

\begin{problem}[Posted by \href{https://artofproblemsolving.com/community/user/78770}{thuanspdn}]
	Find all continuous functions $ f , g: (1,+\infty) \to \infty $ such that
\[ f(xy)=xg(y) + yg(x), \quad \forall   x,y \in (1,+\infty).\]
	\flushright \href{https://artofproblemsolving.com/community/c6h347216}{(Link to AoPS)}
\end{problem}



\begin{solution}[by \href{https://artofproblemsolving.com/community/user/29428}{pco}]
	\begin{tcolorbox}Find all continuous functions: $ f(x), g(x) $  on $ (1,+\infty)            $ ,such that

 $ f(xy)=xg(y) + yg(x)$     ,$\forall   x,y \in (1,+\infty)$\end{tcolorbox}
\begin{italicized}I suppose that $f,g$ are from $(1,+\infty)\to\mathbb R$ (problem is not very clear about codomain)\end{italicized}

Let $P(x,y)$ be the assertion $f(xy)=xg(y)+yg(x)$

$P(x,x)$ $\implies$ $f(x^2)=2xg(x)$ and the equation becomes $2\sqrt{xy}g(\sqrt{xy})=xg(y)+yg(x)$

Let then $h(x)=e^{-x}g(e^x)$ defined and continuous over $(0,+\infty)$

Replacing $g(x)$ by $xh(\ln(x))$ in $2\sqrt{xy}g(\sqrt{xy})=xg(y)+yg(x)$, we get : $h(\frac{x+y}2)=\frac{h(x)+h(y)}2$ $\forall x,y>0$

This is a very classical equation whose continuous solutions are $h(x)=ax+b$ and so $g(x)=ax\ln(x)+bx$ and $f(x)=ax\ln(x)+2bx$
... which indeed are solutions.

Hence the answer : $\boxed{(f,g)=(ax\ln(x)+2bx,ax\ln(x)+bx)}$
\end{solution}
*******************************************************************************
-------------------------------------------------------------------------------

\begin{problem}[Posted by \href{https://artofproblemsolving.com/community/user/78770}{thuanspdn}]
	Find all continuous functions $f, g, h: \mathbb R \to \mathbb R$ such that
\[f(x+y)=g(x) \cdot h(y), \quad  \forall x,y\in \mathbb R.\]
	\flushright \href{https://artofproblemsolving.com/community/c6h347219}{(Link to AoPS)}
\end{problem}



\begin{solution}[by \href{https://artofproblemsolving.com/community/user/29428}{pco}]
	\begin{tcolorbox}Find all continuous functions: $ f(x), g(x) , h(x):    \mathbb{R}\to\mathbb{R}$ ,such that
 $  f(x+y)=g(x).h(y)$ ,  $ \forall x,y\in R$\end{tcolorbox}
Let $P(x,y)$ be the assertion $f(x+y)=g(x)h(y)$

If $h(x)=0$ $\forall x$, then $f(x)=0$ $\forall x$ and we got the solution $(f,g,h)=(0,g,0)$
If $g(x)=0$ $\forall x$, then $f(x)=0$ $\forall x$ and we got the solution $(f,g,h)=(0,0,h)$
So we'll from now consider $g,h$ are not the all-zero functions

Let $a$ such that $g(a)\ne 0$. Comparing $P(x,a)$ and $P(a,x)$, we get $g(x)h(a)=g(a)h(x)$ and so $h(x)=cg(x)$ with $c\ne 0$

Let then $Q(x,y)$ be the new assertion $f(x+y)=cg(x)g(y)$
$Q(x,0)$ $\implies$ $f(x)=cg(x)g(0)$ and we get the equation $g(0)g(x+y)=g(x)g(y)$

This a very classical equation whose general solution is $g(x)=ae^{bx}$ and so $h(x)=ce^{bx}$ and $f(x)=ace^{bx}$ which indeed are solutions.

\begin{bolded}Hence the answer :\end{bolded}\end{underlined}
1) $(f,g,h)=(abe^{\lambda x},ae^{\lambda x},be^{\lambda x})$ where $a,b,\lambda$ are any real numbers
2) $(f,g,h)=(0,u(x),0)$ where $u(x)$ is any continuous function
3) $(f,g,h)=(0,0,u(x))$ where $u(x)$ is any continuous function
\end{solution}
*******************************************************************************
-------------------------------------------------------------------------------

\begin{problem}[Posted by \href{https://artofproblemsolving.com/community/user/44659}{uglysolutions}]
	Find all functions $f: \mathbb R \rightarrow \mathbb R$ such that
\[f(x+xy+f(y)) = \left(f(x)+\frac{1}{2}\right) \left(f(y)+\frac{1}{2}\right)\]
holds for all real numbers $x,y$.
	\flushright \href{https://artofproblemsolving.com/community/c6h347727}{(Link to AoPS)}
\end{problem}



\begin{solution}[by \href{https://artofproblemsolving.com/community/user/34083}{Mithril}]
	Notice that if $f(x)=c$ then $c= (c+\frac 12)^2$, thus $c^2 + \frac 14 = 0$, absurd. So we may assume $f$ is not constant.

If $y = -1$ we get $f(f(-1)) = \left ( f(-1)+\frac 12 \right ) \left ( f(y)+\frac 12 \right ) $. If $f(-1) + \frac 12 \neq 0$, that gives $f$ constant, which we know cannot happen. Then $f(-1) = -\frac 12$.

Suppose $f(a) = -\frac 12$. For $y = a$ we get $f(x(a+1) +f(a)) = 0$. If $a \neq -1$, $x(a+1)+f(a)$ can take any value, so $f$ would be constant $0$, absurd. Then $f(a) = -\frac 12 \iff a=-1$.

Now, if $x = -1$ we get $f(-1 -y +f(y)) = 0$. If $y = -1 -a + f(a)$, we get $f(-1 +1 +a - f(a) +f(-1 -a +f(a)))=0$, thus $f(a-f(a))=0$ for all $a$.

Now, for $y \neq -1$, if $x = \frac{y-2f(y)}{1+y}$, we get $f(y-f(y)) = \left(f(x)+\frac{1}{2}\right)\left(f(y)+\frac{1}{2}\right)$. As $f(y-f(y)) = 0$, we have $f \left( \frac{y-2f(y)}{1+y}\right) = -\frac 12$ or $f(y) = -\frac 12$ for all $y \neq -1$. But as $f(-1) = -\frac 12$, the second case would imply $f$ constant. Then $f\left( \frac{y-2f(y)}{1+y}\right) = -\frac 12$.

But that implies $\frac{y-2f(y)}{1+y} = -1$ for $y \neq -1$. It follows that $f(y) = y + \frac 12$ for all $y$, and it is easy to check that it satisfies the conditions.
\end{solution}



\begin{solution}[by \href{https://artofproblemsolving.com/community/user/48552}{ocha}]
	[hide="solution"]
First note that $f$ is not constant, otherwise $f(x)\equiv k \Longrightarrow k = (k+\frac{1}{2})^2$ which has no solutions. contradiction.

Now we show that $f$ is injective
[hide="proof"]

Suppose that $f(a-1)=f(b-1)$ for some distinct $a,b \in \mathbb{R}$. then from the condition we have: 

$ f(ax + f(a-1)) = f(bx + f(b-1))$

denote $g_a(x)=ax + f(a-1)$ and $g_b(x)=bx+f(b-1)$
Now if one of $a,b=0$ then $f$ would be constant so assume $ab\not = 0$, and w.l.o.g $a >b$

Now pick some arbitrary $x_0 \in \mathbb{R}$, and define a sequence $ax_n + f(a-1) = bx_{n-1} + f(b-1)$

We choose this sequence so that $f(g_a(x_0)) = f(g_b(x_0)) = f(g_a(x_1)) =...=f\left(g_a\left(\lim_{n\to \infty} x_n\right)\right) $

But from the sequence we have

$\left( x_n - \frac{f(b-1)-f(a-1)}{a-b}\right) = \frac{b}{a}\left( x_{n-1} - \frac{f(b-1)-f(a-1)}{a-b}\right)$

since $\frac{b}{a} < 1$, $\lim_{n\to \infty} x_n \to \frac{f(b-1)-f(a-1)}{a-b} = L$

therefore $f(g_a(x_0)) = f( g_a (L) )$ implies $f$ is constant. contradiction. 

therefore $f(a-1)=f(b-1) \Longrightarrow a-1 = b-1$, and $f$ is injective.
[\/hide]

Since $f$ is injective
$f(x+xy + f(y)) = \left(f(x)+\frac{1}{2}\right)\left(f(y)+\frac{1}{2}\right) = f(y+yx + f(x))$

$\therefore x + xy + f(y) = y + yx + f(x) \Longrightarrow f(x)-x$ is constant.

Subbing $f(x)=x+c$ into the original equation shows $c=\frac{1}{2}$

Hence $\boxed{ f(x) = x+\frac{1}{2} }$
[\/hide]

Edit: fixed a typo
\end{solution}



\begin{solution}[by \href{https://artofproblemsolving.com/community/user/29428}{pco}]
	\begin{tcolorbox} [We choose this sequence so that $=...=f\left(g_a\left(\lim_{n\to \infty} x_n\right)\right) $\end{tcolorbox}

I'm afraid you need continuity to write this.
And we dont have ... :(
\end{solution}



\begin{solution}[by \href{https://artofproblemsolving.com/community/user/48552}{ocha}]
	\begin{tcolorbox}[quote="ocha"] [We choose this sequence so that $=...=f\left(g_a\left(\lim_{n\to \infty} x_n\right)\right) $\end{tcolorbox}

I'm afraid you need continuity to write this.
And we dont have ... :(\end{tcolorbox}

i think we only need continuity of $g_a(x)$ and $g_b(x)$, which we do have
\end{solution}



\begin{solution}[by \href{https://artofproblemsolving.com/community/user/29428}{pco}]
	\begin{tcolorbox}[quote="pco"]\begin{tcolorbox} [We choose this sequence so that $=...=f\left(g_a\left(\lim_{n\to \infty} x_n\right)\right) $\end{tcolorbox}

I'm afraid you need continuity to write this.
And we dont have ... :(\end{tcolorbox}

i think we only need continuity of $g_a(x)$ and $g_b(x)$, which we do have\end{tcolorbox}

Let $u_n=g_a(x_n)$ and let $c=f(g_a(x_0))$

Continuity of $g_a$ (that we indeed have) allows you to write $\lim g_a(x_n) = g_a(\lim x_n))$ (when such limits exist)

So $\lim u_n=g_a(\lim x_n)$ and this is OK for me.

But from $f(u_n)=c$ $\forall n$, you conclude then $f(g_a(\lim x_n))=f(\lim u_n)=c[=\lim(f(u_n))]$ 

And writing $f(\lim u_n)=\lim(f(u_n))$ demands continuity of $f$.
\end{solution}



\begin{solution}[by \href{https://artofproblemsolving.com/community/user/48552}{ocha}]
	you are right, sorry  :blush:
\end{solution}



\begin{solution}[by \href{https://artofproblemsolving.com/community/user/99754}{manifestdestiny}]
	Very nice solution mithril, just wondering what prompted you to use the substitution $ x =\frac{y-2f(y)}{1+y} $?
\end{solution}



\begin{solution}[by \href{https://artofproblemsolving.com/community/user/139716}{asjeykg}]
	\begin{tcolorbox}Very nice solution mithril, just wondering what prompted you to use the substitution $ x =\frac{y-2f(y)}{1+y} $?\end{tcolorbox} 


just try to solve the equation x+xy+f(y)=y-f(y) with respect to x. Since we need to find f(y-f(y)).
\end{solution}



\begin{solution}[by \href{https://artofproblemsolving.com/community/user/201437}{thangtoancvp}]
	\begin{tcolorbox}Find all functions $f: \mathbb R \rightarrow \mathbb R$ such that
$f(x+xy+f(y)) = \left(f(x)+\frac{1}{2}\right) \left(f(y)+\frac{1}{2}\right)$
holds for all real numbers $x,y$.\end{tcolorbox}

$x = 0 \Rightarrow f\left( {f\left( y \right)} \right) = \left( {f\left( 0 \right) + \frac{1}{2}} \right)\left( {f\left( y \right) + \frac{1}{2}} \right),\forall y \in R$
$y =  - 1 \Rightarrow f\left( {f\left( { - 1} \right)} \right) = \left( {f\left( x \right) + \frac{1}{2}} \right)\left( {f\left( { - 1} \right) + \frac{1}{2}} \right) = \left( {f\left( 0 \right) + \frac{1}{2}} \right)\left( {f\left( { - 1} \right) + \frac{1}{2}} \right)$
We consider the following two cases
a) $f\left( { - 1} \right) \ne  - \frac{1}{2} \Rightarrow f\left( x \right) = f\left( 0 \right),\forall x \in R$, Try again we see conflict
b) $f\left( { - 1} \right) =  - \frac{1}{2}$
If there exists $a$ such that $f\left( a \right) =  - \frac{1}{2},a \ne  - 1$.
$y = a \Rightarrow f\left( {x + ax + f\left( a \right)} \right) = \left( {f\left( x \right) + \frac{1}{2}} \right)\left( {f\left( a \right) + \frac{1}{2}} \right) = 0 \Rightarrow f\left( x \right) = 0,\forall x \in R$ Try not satisfied $ \Rightarrow f\left( a \right) =  - \frac{1}{2} \Leftrightarrow a =  - 1$
If there exists $a$ such that $f\left( a \right) = 0,a \ne  - \frac{1}{2}$
We ha ve $f\left( {f\left( { - 1} \right)} \right) = \left( {f\left( x \right) + \frac{1}{2}} \right)\left( {f\left( { - 1} \right) + \frac{1}{2}} \right) = 0 \Rightarrow f\left( { - \frac{1}{2}} \right) = 0$
$y =  - \frac{1}{2} \Rightarrow f\left( {\frac{x}{2}} \right) = \frac{1}{2}\left( {f\left( x \right) + \frac{1}{2}} \right),\forall x \in R$
$y = a \Rightarrow f\left( {x + ax} \right) = \frac{1}{2}\left( {f\left( x \right) + \frac{1}{2}} \right),\forall x \in R$
Hence $f\left( {x + ax} \right) = f\left( {\frac{x}{2}} \right),\forall x \in R\,\,\,\left( 1 \right).$
$x = \frac{{ - 1}}{{a + 1}} \Rightarrow f\left( { - 1} \right) = f\left( { - \frac{1}{{2\left( {a + 1} \right)}}} \right) =  - \frac{1}{2} \Leftrightarrow  - \frac{1}{{2\left( {a + 1} \right)}} =  - 1 \Leftrightarrow a =  - \frac{1}{2}$ conflict.
Hence $f\left( a \right) = 0 \Leftrightarrow a =  - \frac{1}{2}$
$x =  - 1 \Rightarrow f\left( { - 1 - y + f\left( y \right)} \right) = \left( {f\left( { - 1} \right) + \frac{1}{2}} \right)\left( {f\left( y \right) + \frac{1}{2}} \right) = 0 \Rightarrow f\left( { - 1 - y + f\left( y \right)} \right) = 0$
$ \Rightarrow  - 1 - y + f\left( y \right) =  - \frac{1}{2},\forall y \in R \Rightarrow f\left( y \right) = y + \frac{1}{2},\forall y \in R$ Try to be satisfied.
So $f\left( y \right) = y + \frac{1}{2},\forall y \in R$
\end{solution}



\begin{solution}[by \href{https://artofproblemsolving.com/community/user/173116}{Sardor}]
	See also here:
http://www.artofproblemsolving.com/Forum/viewtopic.php?p=2154206&sid=fe7e79ac1f19199836adc92f1e310dfb#p2154206
\end{solution}



\begin{solution}[by \href{https://artofproblemsolving.com/community/user/293525}{uraharakisuke_hsgs}]
	$f(f(-1)) = (f(x)+\frac{1}{2})(f(-1)+\frac{1}{2}) \implies f(-1) = \frac{-1}{2} , f(\frac{-1}{2}) = 0$
Suppose that $f(a) = \frac{-1}{2}$ for some $a$ , plug in $y = a$ then $a = -1 $
Suppose that $f(a) = 0$ for some $a$ 
$P(x,a):f(x(a+1)) = \frac{1}{2}(f(x)+\frac{1}{2})$
$P(-1,a) : f(-a-1) = 0$
$P(x,-a-1) : f(-ax) = \frac{1}{2}f(x)+\frac{1}{2}) = f(x(a+1))$ , then , $f(-ax) = f(x(a+1)) $. Plug in $x = \frac{-1}{a+1}$ we have $\frac{a}{a+1} = -1$ so $a = \frac{1}{2}$ . Then , $f(x) = 0$ if and only if $a = \frac{-1}{2}$ 
$P(-1,y) : f(f(y)-y-1) = 0 \implies f(y) = y+\frac{1}{2}$
\end{solution}
*******************************************************************************
-------------------------------------------------------------------------------

\begin{problem}[Posted by \href{https://artofproblemsolving.com/community/user/78444}{Babai}]
	Find all functions $f: \mathbb R \to \mathbb R$ such that
\[f(x+y)+f(xy)=f(x)f(y)+1\]
for all reals $x$ and $y$.
	\flushright \href{https://artofproblemsolving.com/community/c6h347778}{(Link to AoPS)}
\end{problem}



\begin{solution}[by \href{https://artofproblemsolving.com/community/user/29428}{pco}]
	\begin{tcolorbox}Help me.I am not able to solve this.\end{tcolorbox}
Let $P(x,y)$ be the assertion $f(x+y)+f(xy)=f(x)f(y)+1$
Let $a=f(1)-1$

$P(0,0)$ $\implies$ $2f(0)=f(0)^2+1$ $\implies$ $(f(0)-1)^2=0$ $\implies$ $f(0)=1$

$P(x,1)$ $\implies$ $f(x+1)=af(x)+1$

Let $x\ne 0$
$P(x,\frac yx)$ $\implies$ $f(x+\frac yx)+f(y)=f(x)f(\frac yx)+1$ $\implies$ $af(x+\frac yx)+af(y)=af(x)f(\frac yx)+a$

$P(x,\frac yx +1)$ $\implies$ $f(x+\frac yx+1)+f(x+y)=f(x)f(\frac yx+1)+1$ $\implies$ $af(x+\frac yx)+f(x+y)=af(x)f(\frac yx)+f(x)$

Subtracting, we get : $f(x+y)=f(x)+af(y)-a$ $\forall x\ne 0$

Let then $g(x)=f(x)-1$. The above line becomes $g(x+y)=g(x)+g(1)g(y)$ $\forall x\ne 0$ with $g(0)=0$

Swapping $x,y$ and subtracting, we get $(g(x)-g(y))(g(1)-1)=0$ $\forall x,y\ne 0$

If $g(1)\ne 1$, this implies $g(x)=c$ $\forall x,y\ne 0$ and so $c=c+c^2$ and so $c=0$ and so $f(x)=1$ $\forall x$ which indeed is a solution.

If $g(1)=1$, we get $g(x+y)=g(x)+g(y)$ $\forall x\ne 0$ with $g(0)=0$ and $g(1)=1$
And so  $g(x+y)=g(x)+g(y)$ with $g(0)=0$ and $g(1)=1$

$g(x)$ is a solution of Cauchy's equation.
The original equation may then be written $g(x+y)+1+g(xy)+1=(g(x)+1)(g(y)+1)+1$ 

$\implies$ $g(xy)=g(x)g(y)$ an so $g(x)\ge 0$ $\forall x\ge 0$
So $g(x)$ is non decreasing and so $g(x)=g(1)x=x$ and $f(x)=x+1$ which indeed is a solution.

\begin{bolded}Hence the two solutions \end{bolded}\end{underlined}:
$f(x)=1$ $\forall x$
$f(x)=x+1$ $\forall x$
\end{solution}



\begin{solution}[by \href{https://artofproblemsolving.com/community/user/81769}{arshakus}]
	helooo,
$f:Q->Q$
$f(xy)=f(x)f(y)-f(x+y)+1$
$x=y=0=> f(0)=1$ and by induction we can proof that $f(x)=1$
$y=1=>f(x+1)=f(x)+1=> \forall n =>f(x+n)=f(x)+n =>$ by induction we can proof that $f(x)=x+1;$
\end{solution}



\begin{solution}[by \href{https://artofproblemsolving.com/community/user/82533}{edooo}]
	I think it is wrong solution!!!
\end{solution}



\begin{solution}[by \href{https://artofproblemsolving.com/community/user/29428}{pco}]
	\begin{tcolorbox}helooo,
$f:Q->Q$
$f(xy)=f(x)f(y)-f(x+y)+1$
$x=y=0=> f(0)=1$ and by induction we can proof that $f(x)=1$
$y=1=>f(x+1)=f(x)+1=> \forall n =>f(x+n)=f(x)+n =>$ by induction we can proof that $f(x)=x+1;$\end{tcolorbox}

1) the problem is for $f$ from $\mathbb R\to\mathbb R$, not $f:Q->Q$

2) I dont understand with what kind of induction you can prove $f(x)=1$ from $f(0)=1$  :maybe: 

3) your third part need you prove first that $f(1)=2$
You need then a first induction for $f(n)=n+1$
You need then a second induction for rational values
And then it remains to deal with $\mathbb R\setminus\mathbb Q$
\end{solution}
*******************************************************************************
-------------------------------------------------------------------------------

\begin{problem}[Posted by \href{https://artofproblemsolving.com/community/user/46787}{moldovan}]
	Determine $f:\mathbb{R} \rightarrow (0,1]$ so that $f(f(x)+y)=f(x)+f(y)$ for all reals $x$ and $y$.
	\flushright \href{https://artofproblemsolving.com/community/c6h347789}{(Link to AoPS)}
\end{problem}



\begin{solution}[by \href{https://artofproblemsolving.com/community/user/29428}{pco}]
	\begin{tcolorbox}Determine $f:\mathbb{R} \rightarrow (0,1]$ so that: $f(f(x)+y)=f(x)+f(y)$.\end{tcolorbox}
Let $x,y\in\mathbb R$ and let then the sequence $a_n$ be :
$a_0=x$
$a_{n+1}=f(a_n)+y$

We get $f(a_{n+1})=f(f(a_n)+y)=f(a_n)+f(y)$

And so $f(a_n)=f(x)+nf(y)$

And, since $f(y)\in(0,1]$, this implies obviously $f(a_n)>1$ for some $n$, which is impossible.

So no solution for this equation.
\end{solution}
*******************************************************************************
-------------------------------------------------------------------------------

\begin{problem}[Posted by \href{https://artofproblemsolving.com/community/user/31915}{Batominovski}]
	Find all $f:\mathbb{R}\rightarrow\mathbb{R}$ such that
\[f\left(x^2+f(y)\right) = \big(f(x)\big)^2+\big(f(y)\big)^2+2x^2y^2\]
for all reals $x$ and $y$.
	\flushright \href{https://artofproblemsolving.com/community/c6h347921}{(Link to AoPS)}
\end{problem}



\begin{solution}[by \href{https://artofproblemsolving.com/community/user/29428}{pco}]
	\begin{tcolorbox}Find all $f:\mathbb{R}\rightarrow\mathbb{R}$ such that
\[f\left(x^2+f(y)\right) = \big(f(x)\big)^2+\big(f(y)\big)^2+2x^2y^2\,.\]\end{tcolorbox}
Let $P(x,y)$ be the assertion $f(x^2+f(y))=f(x)^2+f(y)^2+2x^2y^2$
Let $f(0)=u$

If $f(a)=f(b)$, then, subtracting $P(1,a)$ from $P(1,b)$, we get $a^2=b^2$

Comparing $P(x,0)$ and $P(0,x)$, we get $f(x^2+u)=f(f(x))$ and so : $\forall x$, either $f(x)=x^2+u$, either $f(x)=-x^2-u$

Then $P(x,0)$ $\implies$ $f(x^2+u)=(x^2+u)^2+u^2$
But we know that either $f(x^2+u)=(x^2+u)^2+u$, either $f(x^2+u)=-(x^2+u)^2-u$
Since it is impossible to have $-(x^2+u)^2-u=(x^2+u)^2+u^2$ $\forall x$, then $f(x^2+u)=(x^2+u)^2+u$ for some $x$ and so $u^2=u$

1) If $u=0$
$\forall x$, either $f(x)=x^2$, either $f(x)=-x^2$
Suppose $\exists a\ne 0$ such that $f(a)=a^2$ and $b\ne 0$ such that $f(b)=-b^2$
$P(a,b)$ $\implies$ $f(a^2-b^2)=(a^2+b^2)^2$ and so :
either $f(a^2-b^2)=(a^2-b^2)^2$ and so $(a^2-b^2)^2=(a^2+b^2)^2$ and so $ab=0$, impossible
either $f(a^2-b^2)=-(a^2-b^2)^2$ and so $-(a^2-b^2)^2=(a^2+b^2)^2$ and so $a=b=0$, impossible
So two possibilities :
either $f(x)=x^2$ $\forall x$, which indeed is a solution
either $f(x)=-x^2$ $\forall x$, which is not a solution.

2) If $u=1$
$\forall x$, either $f(x)=x^2$+1, either $f(x)=-x^2-1$
Suppose now $\exists a$ such that $f(a)=a^2+1$ and $\exists b$ such that $f(b)=-b^2-1$
$P(a,b)$ $\implies$ $f(a^2-b^2-1)=a^4+2a^2+1+b^4+2b^2+1+2a^2b^2$ and so :
either $f(a^2-b^2-1)=(a^2-b^2-1)^2+1=a^4-2a^2+1+b^4+2b^2+1-2a^2b^2$ and so $a^2(b^2+1)=0$ and so $a=0$
either $f(a^2-b^2-1)=-(a^2-b^2-1)^2-1=-a^4+2a^2-1-b^4-2b^2-1+2a^2b^2$ and so $a^4+b^4+2b^2+2=0$, impossible

So :
either $f(x)=x^2+1$ $\forall x$, which indeed is a solution
either $f(x)=-x^2-1$ $\forall x$, which is not a solution (look at $f(0)$)
either $f(x)=-x^2-1$ $\forall x\ne 0$ and $f(0)=1$, which is not a solution (look at $P(0,1)$)

Hence the solutions :
$f(x)=x^2$ $\forall x$
$f(x)=x^2+1$ $\forall x$
\end{solution}
*******************************************************************************
-------------------------------------------------------------------------------

\begin{problem}[Posted by \href{https://artofproblemsolving.com/community/user/82748}{smotas}]
	Prove that  for all functions $f: \mathbb R \rightarrow \mathbb R$, there exist $x,y\in\mathbb  R$ such that \[f(x-f(y)) > yf(x)+x.\]
	\flushright \href{https://artofproblemsolving.com/community/c6h348176}{(Link to AoPS)}
\end{problem}



\begin{solution}[by \href{https://artofproblemsolving.com/community/user/29428}{pco}]
	\begin{tcolorbox}$f: \mathbb R \rightarrow \mathbb R$ Prove that there exist such $x,y$ that $f(x-f(y)) > yf(x)+x$\end{tcolorbox}

What are you asking for ?

1) Prove that  $\forall f: \mathbb R \rightarrow \mathbb R$ , $\exists x,y\in\mathbb  R$ such that $f(x-f(y)) > yf(x)+x$

2) Prove that $\exists f: \mathbb R \rightarrow \mathbb R$ , such that $\exists x,y\in\mathbb  R$ such that $f(x-f(y)) > yf(x)+x$

3) Prove that $\exists f: \mathbb R \rightarrow \mathbb R$ , such that $\forall x,y\in\mathbb  R$ : $f(x-f(y)) > yf(x)+x$

4) Prove that $\not\exists f: \mathbb R \rightarrow \mathbb R$ , such that $\forall x,y\in\mathbb  R$ : $f(x-f(y)) > yf(x)+x$

5) Find all $f: \mathbb R \rightarrow \mathbb R$ , such that $\forall x,y\in\mathbb  R$ : $f(x-f(y)) > yf(x)+x$

Sorry for asking but your question was not very clear for me.  :blush:
\end{solution}



\begin{solution}[by \href{https://artofproblemsolving.com/community/user/82748}{smotas}]
	Thanks, pco, it's the first variant of yours. Now I've corrected it.
\end{solution}



\begin{solution}[by \href{https://artofproblemsolving.com/community/user/29428}{pco}]
	\begin{tcolorbox}Prove that  $\forall f: \mathbb R \rightarrow \mathbb R$ , $\exists x,y\in\mathbb  R$ such that $f(x-f(y)) > yf(x)+x$\end{tcolorbox}

So the problem is equivalent to prove that $\not\exists f:\mathbb R\to\mathbb R$ such that $f(x-f(y))\le yf(x)+x$ $\forall x,y$

Let us assume such a $f(x)$ exists.
Let $P(x,y)$ be the assertion $f(x-f(y))\le yf(x)+x$

1) $f(x)\le x+f(0)$ $\forall x$
=================
This is an immediate consequence of $P(x+f(0),0)$

2) $f(x)\le 0$ $\forall x$
==============
If $\exists a$ such that $f(a)>0$, let $b<\min(\frac{-1-f(0)-a}{f(a)},a-f(0))$ and let $c=a-f(b)$ :
$b<a-f(0)$ $\implies$ $f(b)<a$ $\implies$ $c>0$
$b<\frac{-1-f(0)-a}{f(a)}$ $\implies$ $bf(a)+a<-1-f(0)$
Then $P(a,b)$ $\implies$ $f(c)<-1-f(0)$ and so $f(f(c))<-1$

$P(f(c),c)$ $\implies$ $f(0)\le cf(f(c))+f(c)\le cf(f(c))+c+f(0)$ and so (remember $c>0$) $f(f(c))\ge -1$, and contradiction.
So no such $a$ exists
Q.E.D.

3) $\exists x>0$ such that $f(x)<0$
======================
Suppose the contrary : $f(x)=0$ $\forall x>0$
Let then $a<0$. We know that $f(a)\le a+f(0)<0$
Let $x>0$ (and so $f(x)=0$) : $P(a,x)$ $\implies$ $f(a)\le xf(a)+a$, which is impossible since $f(a)<0$ and so we can choose positive $x$ great enough to make RHS as negative as we want.
Q.E.D

4) No such function exists
==================
Let $a>0$ such that $f(a)<0$ ($a$ exists, according to 3. above)
Let $b>\frac{f(0)-a}{f(a)}$
Let $c=a-f(b)>0$

$P(a,b)$ $\implies$ $f(c)\le bf(a)+a<f(0)$

$P(f(c),c)$ $\implies$ $f(0)\le cf(f(c))+f(c)$ $\implies$ $f(c)\ge f(0)-cf(f(c))\ge f(0)$ (since $c>0$ and $f(x)\le 0$ $\forall x$)

Hence contradicton.
Q.E.D.
\end{solution}



\begin{solution}[by \href{https://artofproblemsolving.com/community/user/67111}{Abdek}]
	Very nice proof pco !!
\end{solution}
*******************************************************************************
-------------------------------------------------------------------------------

\begin{problem}[Posted by \href{https://artofproblemsolving.com/community/user/77076}{jaydoubleuel}]
	Find all functions $f: \mathbb R \to \mathbb R$ such that for all reals $x$ and $y$,
\[f(x+f(x)+y)=f(y)+2x.\]
	\flushright \href{https://artofproblemsolving.com/community/c6h349063}{(Link to AoPS)}
\end{problem}



\begin{solution}[by \href{https://artofproblemsolving.com/community/user/29428}{pco}]
	\begin{tcolorbox}Find all $f: R \longrightarrow R$ such that
$f(x+f(x)+y)=f(y)+2x$\end{tcolorbox}

Please, in your future posts, try to choose better titles.

Let $P(x,y)$ be the assertion $f(x+f(x)+y)=f(y)+2x$
Let $a=f(0)$

$P(0,x)$ $\implies$ $f(x+a)=f(x)$
$P(x+a,y)$ $\implies$ $f(x+f(x)+y)=f(y)+2x+2a$ and so $a=0$
$P(x,0)$ $\implies$ $f(x+f(x))=2x$ and $f(x)$ is surjective
$P(x,-f(x))$ $\implies$ $f(x)=f(-f(x))+2x$ and $f(x)$ is injective, so bijective.

Since $f(x)$ is bijective and since $f(x+f(x))=2x$, we get that $x+f(x)$ is bijective too

$f(x+f(x))=2x$ implies that $P(x,y)$ may be written $f(x+f(x)+y)=f(x+f(x))+f(y)$ and, since $x+f(x)$ is surjective : $f(x+y)=f(x)+f(y)$

So the problem is equivalent to :
$f(x+y)=f(x)+f(y)$
and $f(x)+f(f(x))=2x$


hence the only two continuous solutions $f(x)=x$ and $f(x)=-2x$
And infinitely many non continuous solutions (using for example Hamel basis with the two solutions above mixed together)
\end{solution}



\begin{solution}[by \href{https://artofproblemsolving.com/community/user/43015}{modularmarc101}]
	\begin{tcolorbox}[quote="jaydoubleuel"]Find all $f: R \longrightarrow R$ such that
$f(x+f(x)+y)=f(y)+2x$\end{tcolorbox}

Please, in your future posts, try to choose better titles.

Let $P(x,y)$ be the assertion $f(x+f(x)+y)=f(y)+2x$
Let $a=f(0)$

$P(0,x)$ $\implies$ $f(x+a)=f(x)$
$P(x+a,y)$ $\implies$ $f(x+f(x)+y)=f(y)+2x+2a$ and so $a=0$
$P(x,0)$ $\implies$ $f(x+f(x))=2x$ and $f(x)$ is surjective
$P(x,-f(x))$ $\implies$ $f(x)=f(-f(x))+2x$ and $f(x)$ is injective, so bijective.

Since $f(x)$ is bijective and since $f(x+f(x))=2x$, we get that $x+f(x)$ is bijective too

$f(x+f(x))=2x$ implies that $P(x,y)$ may be written $f(x+f(x)+y)=f(x+f(x))+f(y)$ and, since $x+f(x)$ is surjective : $f(x+y)=f(x)+f(y)$

So the problem is equivalent to :
$f(x+y)=f(x)+f(y)$
and $f(x)+f(f(x))=2x$


hence the only two continuous solutions $f(x)=x$ and $f(x)=-2x$
And infinitely many non continuous solutions (using for example Hamel basis with the two solutions above mixed together)\end{tcolorbox}


Doesn't $P(x+a, y) \implies f(x + a + y + f(x)) = f(y) + 2x + 2a$ ?

EDIT: Oh ok awesome :)
\end{solution}



\begin{solution}[by \href{https://artofproblemsolving.com/community/user/29428}{pco}]
	\begin{tcolorbox} Doesn't $P(x+a, y) \implies f(x + a + y + f(x)) = f(y) + 2x + 2a$ ?\end{tcolorbox}

Yes, but $f(x+a)=f(x)$ implies also $f(x + a + y + f(x))=f(x+y+f(x))$ and so  $f(x +  y + f(x)) = f(y) + 2x + 2a$
\end{solution}
*******************************************************************************
-------------------------------------------------------------------------------

\begin{problem}[Posted by \href{https://artofproblemsolving.com/community/user/61082}{Pain rinnegan}]
	Find all the functions $f:\mathbb{R}\rightarrow \mathbb{R}$ such that
\[x(f(x)+f(-x)+2)+2f(-x)=0, \quad \forall x\in \mathbb{R}.\]
	\flushright \href{https://artofproblemsolving.com/community/c6h349402}{(Link to AoPS)}
\end{problem}



\begin{solution}[by \href{https://artofproblemsolving.com/community/user/29428}{pco}]
	\begin{tcolorbox}Find all the functions $f:\mathbb{R}\rightarrow \mathbb{R}$ such that:

\[x(f(x)+f(-x)+2)+2f(-x)=0\ ,\ (\forall)x\in \mathbb{R}\]\end{tcolorbox}

Let $P(x)$ be the assertion $x(f(x)+f(-x)+2)+2f(-x)=0$

Adding $P(x)$ and $P(-x)$, we get $f(x)+f(-x)=0$ and so $P(x)$ becomes $f(-x)=-x$ and $f(x)=x$,  which indeed is a solution.

Hence the answer : $\boxed{f(x)=x}$
\end{solution}



\begin{solution}[by \href{https://artofproblemsolving.com/community/user/3640}{Dr Sonnhard Graubner}]
	hello, setting $x:=-x$ in your first equation we get
$-x(f(-x)+f(x)+2)+2f(x)=0$
from our first equation we get
$f(-x)=-\frac{2x+xf(x)}{x+2}$
inserting this in the equation above and solving for $f(x)$ we get the solution $f(x)=x$.
Sonnhard.
\end{solution}
*******************************************************************************
-------------------------------------------------------------------------------

\begin{problem}[Posted by \href{https://artofproblemsolving.com/community/user/73438}{leviethai}]
	Find all functions $f: \mathbb R \to \mathbb R$ such that for all reals $x$ and $y$,
\[{f^2}\left( {x + y} \right) = f\left( {{x^2}} \right) + 2f\left( x \right)f\left( y \right) + f\left( {{y^2}} \right).\]
	\flushright \href{https://artofproblemsolving.com/community/c6h350697}{(Link to AoPS)}
\end{problem}



\begin{solution}[by \href{https://artofproblemsolving.com/community/user/73438}{leviethai}]
	Is there anyone being interested in this equation :) ?

There are 4 functions which satisfy the equation

$f(x)=0,f(x)=-2,f(x)=x,f(x)=x-2$
\end{solution}



\begin{solution}[by \href{https://artofproblemsolving.com/community/user/29428}{pco}]
	\begin{tcolorbox}Is there anyone being interested in this equation :) ?

There are 4 functions which satisfy the equation

$f(x)=0,f(x)=-2,f(x)=x,f(x)=x-2$\end{tcolorbox}

Finding these 4 solutions is the trivial part.
The real problem is to show that these are the only solutions.

I did not succeed in this part.
And it seems nobody else found the solution.

So feel free, please, to post your own proof.
Thanks in advance.
\end{solution}



\begin{solution}[by \href{https://artofproblemsolving.com/community/user/73438}{leviethai}]
	\begin{tcolorbox}[quote="leviethai"]Is there anyone being interested in this equation :) ?

There are 4 functions which satisfy the equation

$f(x)=0,f(x)=-2,f(x)=x,f(x)=x-2$\end{tcolorbox}

Finding these 4 solutions is the trivial part.
The real problem is to show that these are the only solutions.

I did not succeed in this part.
And it seems nobody else found the solution.

So feel free, please, to post your own proof.
Thanks in advance.\end{tcolorbox}

My solution is rather long, but it's simple, elementary, and easy to understand but cool :). I use the following lemma to complete the solution

\begin{italicized}\begin{bolded}Lemma.\end{bolded}\end{italicized} Let function $f:R \to R$ satisfies the following conditions:

i) $f\left( {x + y} \right) = f\left( x \right) + f\left( y \right),\forall x,y \in R$
ii) $f\left( {{x^2}} \right) = {f^2}\left( x \right),\forall x \in R$

The function $f$ can only be $f(x)=0$ or $f(x)=x$.

Try it. :) (Sorry for my bad English)
\end{solution}



\begin{solution}[by \href{https://artofproblemsolving.com/community/user/29428}{pco}]
	\begin{tcolorbox}[quote="pco"]...So feel free, please, to post your own proof.
Thanks in advance.\end{tcolorbox}

My solution is rather long, but it's simple, elementary, and easy to understand but .... Try it. :)\end{tcolorbox}
Ok, so you dont have any solution. :(
Thanks for the information.
\end{solution}



\begin{solution}[by \href{https://artofproblemsolving.com/community/user/73438}{leviethai}]
	\begin{tcolorbox}[quote="leviethai"]\begin{tcolorbox}...So feel free, please, to post your own proof.
Thanks in advance.\end{tcolorbox}

My solution is rather long, but it's simple, elementary, and easy to understand but .... Try it. :)\end{tcolorbox}
Ok, so you dont have any solution. :(
Thanks for the information.\end{tcolorbox}

The following is the first part of the solution:
${f^2}\left( {x + y} \right) = f\left( {{x^2}} \right) + 2f\left( x \right)f\left( y \right) + f\left( {{y^2}} \right)$
For $x=y=0$, we have: 
${f^2}(0) = 2f(0) + 2{f^2}(0) \Leftrightarrow f(0) = 0 \vee f(0) =  - 2$

\begin{italicized}\begin{bolded}Case 1. \end{bolded}\end{italicized}\end{underlined}  $f(0) =  - 2$
For $y=0$, we have: ${f^2}(x)=f({x^2}) - 4f(x) - 2 \Leftrightarrow f({x^2}) = {f^2}\left( x \right) + 4f(x) + 2$
Therefore:
 $\begin{aligned}{f^2}\left( {x + y} \right) &= {f^2}\left( x \right) + 4f(x) + 2 + 2f\left( x \right)f\left( y \right) + {f^2}\left( y \right) + 4f(y) + 2 \\&= {\left( {f(x) + f(y) + 2} \right)^2}\end{aligned}$

For $y=-x$, we have: ${\left( {f(x) + f( - x) + 2} \right)^2} = {f^2}(0) = 4$

\begin{italicized}\begin{bolded}Case 1.1 \end{bolded}\end{italicized}\end{underlined}We assume that there exists a number $a$ such that: $f(a) + f( - a) + 2 = 2 \Leftrightarrow f(a) + f( - a) = 0$
We have $a \ne 0$ ( because $f(0)+f(0)=-4$), we may assume that $a>0$.

Since $f({x^2}) = {f^2}\left( x \right) + 4f(x) + 2$, we have:
${f^2}(a) = f({a^2}) - 4f(a) - 2$
${f^2}( - a) = f\left( {{{( - a)}^2}} \right) - 4f( - a) - 2 = f({a^2}) - 4f( - a) - 2$

Because ${f^2}(a) = {f^2}( - a)$, we get: $f(a) = f( - a) \Rightarrow f(a) = f( - a) = 0$

We also have: $f({a^2}) = 2$ from: ${f^2}(a) = f({a^2}) - 4f(a) - 2$, since $f(a)=0$

For $x=y=a$: ${f^2}(2a) = 2f({a^2}) + 2.0.0 = 4$

For $x=2a,y=-a$:
$0 = {f^2}(a) = {f^2}(2a - a) = f(4{a^2}) + 2f(2a)f( - a) + f({a^2}) = f(4{a^2}) + 2 $
$\Rightarrow f(4{a^2}) =  - 2$

For $x=y=2a$: 
${f^2}(4a) = 2f(4{a^2}) + 2{f^2}(2a) = 2.( - 2) + 2.4 = 4$
$ \Rightarrow f(4a) = 2 \vee f(4a) =  - 2$

For $x = 2\sqrt a $: 
 ${f^2}\left( {2\sqrt a } \right) = f(4a) - 4f\left( {2\sqrt a } \right) - 2 $
$  \Rightarrow {f^2}\left( {2\sqrt a } \right) =  - 4f\left( {2\sqrt a } \right) \vee {f^2}\left( {2\sqrt a } \right) =  - 4f\left( {2\sqrt a } \right) - 4$
$\Rightarrow f\left( {2\sqrt a } \right) = 0 \vee f\left( {2\sqrt a } \right) =  - 4 \vee f\left( {2\sqrt a } \right) =  - 2$

However, 

For $x=\sqrt a$: 
 ${f^2}\left( {\sqrt a } \right) = f(a) - 4f\left( {\sqrt a } \right) - 2 =  - 4f\left( {\sqrt a } \right) - 2 $ 
 $ \Leftrightarrow {f^2}\left( {\sqrt a } \right) + 4f\left( {\sqrt a } \right) + 2 = 0 $
  $\Leftrightarrow f\left( {\sqrt a } \right) = \sqrt 2  - 2 \vee f\left( {\sqrt a } \right) =  - \sqrt 2  - 2 $

For $x = y = \sqrt a $: 
 ${f^2}\left( {2\sqrt a } \right) = f(a) + 2{f^2}\left( {\sqrt a } \right) + f(a) = 2{f^2}\left( {\sqrt a } \right) $
  $\Rightarrow {f^2}\left( {2\sqrt a } \right) = 2{\left( {\sqrt 2  - 2} \right)^2} \vee {f^2}\left( {2\sqrt a } \right) = 2{\left( {\sqrt 2  + 2} \right)^2} $

We have the contradition !

So that, we must have $f(x)+f(-x)+2=-2$ for every $x$, which is equilvalent to $f(x)+f(-x)=-4$.

\begin{italicized}\begin{bolded}Case 1.2 \end{bolded}\end{italicized}\end{underlined} $f(x)+f(-x)=-4$ for every $x$

( I'll post it soon)
\end{solution}



\begin{solution}[by \href{https://artofproblemsolving.com/community/user/73438}{leviethai}]
	\begin{bolded}\begin{italicized}Case 1.2\end{underlined}\end{italicized}\end{bolded} $f(x)+f(-x)=-4$ for every $x$
Setting $g(x)=f(x)+2$, so $g(x)+g(-x)=0$
${f^2}\left( {x + y} \right) = f\left( {{x^2}} \right) + 2f\left( x \right)f\left( y \right) + f\left( {{y^2}} \right)$

We have :
${\left( {g(x + y) - 2} \right)^2} = g({x^2}) + 2\left( {g(x) - 2} \right)\left( {g(y) - 2} \right) + g({y^2}) - 4$

After expanding, we get:
${g^2}(x + y) - 4g(x + y) = g({x^2}) + g({y^2}) + 2g(x)g(y) - 4\left( {g(x) + g(y)} \right)$

From $f({x^2}) = {f^2}\left( x \right) + 4f(x) + 2$, we have:
$g({x^2}) = {\left( {g(x) - 2} \right)^2} + 4\left( {g(x) - 2} \right) + 2 \Leftrightarrow g({x^2}) = {g^2}(x)$

Therefore, replacing $x$ and $y$ with $-x$ and $-y$, we have:
 ${g^2}( - x - y) - 4g( - x - y) = g({x^2}) + g({y^2}) + 2g( - x)g( - y) - 4\left( {g( - x) + g( - y)} \right)$
  $\Leftrightarrow {g^2}(x + y) + 4g(x + y) = g({x^2}) + g({y^2}) + 2g(x)g(y) + 4\left( {g(x) + g(y)} \right)$

So, $g(x+y)=g(x)+g(y)$, and we also have ${g^2}(x) = g({x^2})$, from the \begin{bolded}\begin{italicized}Lemma.\end{underlined}\end{italicized}\end{bolded}, we get $g(x)=0$ or $g(x)=x$. Hence, $f(x)=-2$ or $f(x)=x-2$.
\end{solution}



\begin{solution}[by \href{https://artofproblemsolving.com/community/user/77832}{abhinavzandubalm}]
	leviethai 

we can shorten your proof to a large extent

and also get in the missing cases

Setting $ f $ as constant function we get 

$ \ f(x) = 0 \ or -2 $

put $ x = y = 0 $

then we get 

$ \ f(0) = 0 \ or \ f(0) = -2 $

\begin{bolded}Case 1 : $ \ f(0) = 0 $\end{bolded}

then putting

$ y = 0 $

we get 

$ f^{2}(x) = f(x^{2}) $

and putting 

$ y = -x $ 

We get 

$ f(-x) = -f(x) $

Then The Question Becomes

$ f^{2}(x+y) = f^{2}(x) + f^{2}(y) + 2f(x)f(y) $

$ f(x+y) = f(x) + f(y) $

therefore $ f(x) = cx $

but $ f(x^{2}) = f^{2}(x) $

therefore $ c = 1 \ or \ 0 $ 

\begin{bolded}Case 2 : $ f(0) = -2 \end{bolded} $

putting $ y = 0 $

We get 

$ {f^{2}}(x)=f({x^{2}})-4f(x)-2\Leftrightarrow f({x^{2}}) ={f^{2}}\left( x\right)+4f(x)+2 $

Therefore

$ f^{2}(x + y) = f^{2}(x) + f^{2}(y) + 4 + 4f(x) + 4f(y) + 2f(x)f(y) $

Therefore

$ f(x+y) = f(x) + f(y) + 2 $

Therefore 

$ f(x) = cx - 2 $

but due to similar conditions on $ f $ 

$ c = 1 \ or \ 0 $

therefore the only solutions are 

$ f(x) = cx $

$ f(x) = cx - 2 $

with $ c = 1 \ or \ 0 $ giving the answers
\end{solution}



\begin{solution}[by \href{https://artofproblemsolving.com/community/user/73438}{leviethai}]
	\begin{tcolorbox}leviethai 

we can shorten your proof to a large extent

and also get in the missing cases

Setting $ f $ as constant function we get 

.....

with $ c = 1 \ or \ 0 $ giving the answers\end{tcolorbox}

I'm sorry, but there seems to be some mistakes in your solution.

Firstly, only the result "$f(x+y)=f(x)+f(y)$" doesn't yield $f(x)=cx$.

Secondly, $ f^{2}(x + y) = f^{2}(x) + f^{2}(y) + 4 + 4f(x) + 4f(y) + 2f(x)f(y) $ is equilvalent to ${f^{2}(x+y}=(f(x)+f(y)+2)^{2}$, so $ f(x+y) = f(x) + f(y) + 2 $ or $f(x+y)=-f(x)-f(y)-2$ for every $x,y$; or $ f(x+y) = f(x) + f(y) + 2 $ for some $x,y$ and $f(x+y)=-f(x)-f(y)-2$ for some $x,y$.

So I think that your solution is not correct, but thanks for the solution. :)

Actually, my solution is completed, but I want someone else to do the \begin{bolded}Second Case. $f(0)=0$\end{bolded} and complete the solution :).
\end{solution}
*******************************************************************************
-------------------------------------------------------------------------------

\begin{problem}[Posted by \href{https://artofproblemsolving.com/community/user/43536}{nguyenvuthanhha}]
	Let $f$ be a function $f : \mathbb{R} \to \mathbb{R}$ such that $|f(x) - f(y) | \leq |x-y|$ for all $x, y \in \mathbb{R}$ and $ f(f(f(0))) = 0$. Prove that
\[|f(x)|  \leq |x|, \quad \forall x \in \mathbb{R}.\]
	\flushright \href{https://artofproblemsolving.com/community/c6h350905}{(Link to AoPS)}
\end{problem}



\begin{solution}[by \href{https://artofproblemsolving.com/community/user/64716}{mavropnevma}]
	One can easily compute $|f(0)| = |f(f(f(0))) - f(0)| \leq |f(f(0)) - 0| = |f(f(0))| = |f(f(f(0))) - f(f(0))| \leq |f(f(0)) - f(0)| \leq |f(0) - 0| = |f(0)|.$ Therefore $|f(0)| = |f(f(0))| = |f(f(0)) - f(0)|$, hence $f(f(0)) = f(0) = 0$. Then $|f(x)| = |f(x) - f(0)| \leq |x-0| = |x|$ for all $x$.
Notice that $f(f(0)) = 0$ is not enough, since the counterexample $f(x) = a - x$ for $a\neq 0$ denies the thesis.
\end{solution}



\begin{solution}[by \href{https://artofproblemsolving.com/community/user/43536}{nguyenvuthanhha}]
	\begin{italicized}Thanks mavropnevma , nice solution ! :D\end{italicized}
\end{solution}
*******************************************************************************
-------------------------------------------------------------------------------

\begin{problem}[Posted by \href{https://artofproblemsolving.com/community/user/31915}{Batominovski}]
	Find all continuous functions $f:\mathbb{R} \rightarrow \mathbb{R}^+$ which satisfy the functional equation
\[\frac{1}{f\left(y^2f(x)\right)} = \big(f(x)\big)^2\left(\frac{1}{f\left(x^2-y^2\right)} + \frac{2x^2}{f(y)}\right)\,\]
for all reals $x$ and $y$.  (Note: $\mathbb{R}^+$ is the set of \begin{italicized}positive\end{italicized} real numbers.)
	\flushright \href{https://artofproblemsolving.com/community/c6h351110}{(Link to AoPS)}
\end{problem}



\begin{solution}[by \href{https://artofproblemsolving.com/community/user/29428}{pco}]
	\begin{tcolorbox}Find all continuous functions $f:\mathbb{R} \rightarrow \mathbb{R}^+$ which satisfy the functional equation
\[\frac{1}{f\left(y^2f(x)\right)} = \big(f(x)\big)^2\left(\frac{1}{f\left(x^2-y^2\right)} + \frac{2x^2}{f(y)}\right)\,\]
for all reals $x$ and $y$.  (Note: $\mathbb{R}^+$ is the set of \begin{italicized}positive\end{italicized} real numbers.)\end{tcolorbox}

[hide="One solution"]One solution is $f(x)=\frac 1{x^2+1}$[\/hide]

But I'm still looking for other solutions or a proof that this is the only one.
\end{solution}



\begin{solution}[by \href{https://artofproblemsolving.com/community/user/31915}{Batominovski}]
	Hey Patrick!

Good that you replied.  Once you've solved this problem with $f$ being continuous, can you try dropping this condition and see whether you can still get something out of it?  I couldn't, but maybe you can.

Thanks!
\end{solution}
*******************************************************************************
-------------------------------------------------------------------------------

\begin{problem}[Posted by \href{https://artofproblemsolving.com/community/user/652}{Omid Hatami}]
	Find all non-decreasing functions $f:\mathbb R^+\cup\{0\}\rightarrow\mathbb R^+\cup\{0\}$ such that for each $x,y\in \mathbb R^+\cup\{0\}$
\[f\left(\frac{x+f(x)}2+y\right)=2x-f(x)+f(f(y)).\]
	\flushright \href{https://artofproblemsolving.com/community/c6h352088}{(Link to AoPS)}
\end{problem}



\begin{solution}[by \href{https://artofproblemsolving.com/community/user/31915}{Batominovski}]
	IF $f(0)=0$, then with $x:=0$, we get
\[f\big(f(y)\big)=f(y)\,.\;\;\;\text{(1)}\]
With $y:=x$, we now have
\[f\left(\frac{3x+f(x)}{2}\right)=2x+f(x)-f\big(f(x)\big)=2x\,.\]  Consequently, $f$ is onto, and therefore, (1) implies that $f(x)=x$ (which can be readily checked to be a solution).

It remains to show that $f(0)=0$ in order to justify that we have indeed found all solutions to this functional equation.  I found the proof to be rather long.  See [hide="here!"]

Let $a:=f(0)$ and $b:=f(a)$.  Suppose contrary that $a>0$.  Putting $x=0$, we have  \[f\left(\frac{a}{2}+y\right) = -a+f\big(f(y)\big) < f\big(f(y)\big)\,.\]  Thus, 
\[f(y) > y+\frac{a}{2}\,,\;\;\;\;\;\text{(2)}\] 
for all $y \geq 0$. Now, with $y:=0$ and from (2), we have  \[\frac{x+f(x)}{2}+\frac{a}{2} < f\left(\frac{x+f(x)}{2}\right)=2x-f(x)+b\,.\;\;\;\text{(3)}\]  That is,
\[f(x)<x-\frac{1}{3}a+\frac{2}{3}b\,.\]

Define the sequences $\left\{p_n\right\}$, $\left\{q_n\right\}$, $\left\{r_n\right\}$, and $\left\{s_n\right\}$ as follows:
1) $p_1=\frac{1}{2}$, $q_1=0$, $r_1=-\frac{1}{3}$, and $s_1=\frac{2}{3}$;
2) for each $n>1$, $p_n=-\frac{2}{3}r_{n-1}$, $q_n=\frac{2}{3}\left(1-s_{n-1}\right)$, $r_n=-\frac{2}{3}p_n$, and $s_n=\frac{2}{3}\left(1-q_n\right)$.  
Since the inequalities  \[x+p_na+q_nb<f(x)<x+r_na+s_nb\;\;\;\text{(4)}\]  hold when $n=1$, repeatedly applying (3), we can see that (4) holds for all $n \in \mathbb{N}$.  Note that as $n\rightarrow\infty$, $p_n\rightarrow0$, $q_n\rightarrow\frac{2}{5}$, $r_n\rightarrow0$, and $s_n\rightarrow\frac{2}{5}$.  Hence,  \[f(x)=x+\frac{2}{5}b\,.\]  This means $a=\frac{2}{5}b$ and $b=a+\frac{2}{5}b$, or $a=b=0$, which is a contradiction.
[\/hide]

[color=#f00][mod edit: also posted [url=https:\/\/artofproblemsolving.com\/community\/c6h378714]here[\/url].][\/color]
\end{solution}



\begin{solution}[by \href{https://artofproblemsolving.com/community/user/72731}{goodar2006}]
	I think we can say $f(0)=0$ in an easier way .
by putting $x=f(x)$ and $y=x$ we get $f(\frac{f(f(x))+f(x)}{2}+x)=2x$ and so $f$ is surjective .
this implies that there is $a$ that $f(a)=0$ . now put $x=a$ and $y=\frac{a}{2}$ . we get $f(f(\frac{a}{2}))=-2a$ . since $a$ and $-2a$ are both non-negative , we get $a=0$ and $f(0)=0$ .
\end{solution}



\begin{solution}[by \href{https://artofproblemsolving.com/community/user/46488}{Raja Oktovin}]
	\begin{tcolorbox}
by putting $x=f(x)$ and $y=x$ we get $f(\frac{f(f(x))+f(x)}{2}+x)=2x$ and so $f$ is surjective .\end{tcolorbox}

i think it is wrong :(
can you write it more clearly?
\end{solution}



\begin{solution}[by \href{https://artofproblemsolving.com/community/user/72731}{goodar2006}]
	yes , you are right .
we get $f(\frac{f(f(x))+f(x)}{2}+x)=2f(x)$ .
thanks for noticing me .
\end{solution}



\begin{solution}[by \href{https://artofproblemsolving.com/community/user/43727}{RaleD}]
	We can show that $f(0)=0$ in this way: because $f$ is non-decreasing we have $f(0) \le f( \frac{f(0)}{2})=f(f(0))-f(0)$ so $f(f(0)) \ge 2f(0)$. By putting $x=f(0), y=0$ we get $f( \frac{f(0)+f(f(0))}{2})=2f(0) \le f(f(0))$. That means $f(0 )\ge \frac{f(0)+f(f(0))}{2}$ or $f( \frac{f(0)+f(f(0))}{2})=2f(0)=f(f(0))$. If first then obviously $f(0)=0$. Let second be correct.  By placing $x=0, y= \frac{f(0)}{2}$ we obtain $f(f(0))=f(f( \frac{f(0)}{2}))-f(0);$ $f(f(0))=f(f(f(0))-f(0))-f(0);$
$f(f(0))=f(f(0))-f(0);$
$ f(0)=0$
\end{solution}



\begin{solution}[by \href{https://artofproblemsolving.com/community/user/37259}{math154}]
	Let $P(x,y)$ denote the assertion that $f\left(\frac{x+f(x)}{2}+y\right)=2x-f(x)+f(f(y))$.

First suppose to the contrary that some $a<b$ exist such that $\ell=f(a)=f(b)$. Then $f(x)=\ell$ for all $a\le x\le b$, so for sufficiently small $\epsilon$,
\[P(a,b),P(a+2\epsilon,b-\epsilon)\implies 2a-\ell+f(\ell)=f\left(\frac{a+\ell}{2}+b\right)=2(a+2\epsilon)-\ell+f(\ell),\]a contradiction.

Thus $f$ is injective, so $a<b\implies f(a)<f(b)$ and $f(a)\ge f(b)\implies a\ge b$, whence
\[P(0,r)\implies f(f(r))=f(0)+f\left(\frac{f(0)}{2}+r\right)\implies f(r)\ge r+\frac{f(0)}{2}\forall{r}.\]Now assume for the sake of contradiction that $c=f(0)\/2>0$. Then
\begin{align*}
P(f(r),r)\implies 2f(r)
&= f\left(\frac{f(r)+f(f(r))}{2}+r\right) \\
&\ge \frac{r+c+r+2c}{2}+r+c \\
&= 2r+\frac{5c}{2}\implies f(r)\ge r+\frac{5c}{4}
\end{align*}for all $r$, which is clearly impossible (since $(5\/4)^n$ tends to infinity).

Hence $f(0)=0$, so
\[P(0,r)\implies f(f(r))=f(r)\implies f(r)=r\forall r,\]as desired.
\end{solution}



\begin{solution}[by \href{https://artofproblemsolving.com/community/user/188896}{eulerou1997}]
	My solution is different from the upper ones.
First, like  # 6 , we can get f(0)=0
then, put x=0 , f(y)=f(f(y)), thus f( (x+f(x))  \/ 2 ) + y = 2x - f(x) +f(y)
replacing y by f(y), we have f(  x+f(x)  \/ 2 ) =  2x - f(x) 
we can have the following discusion.

(1)
if x >=  f(x)
x>=( x+f(x) )\/2 
2x= f(x)+f( (x+f(x))  \/ 2 ) <=2f(x) 
thus we have f(x)=x
(2)
if x< f(x)
like (1) we'll find out that this condition is wrong.

by the upper discusion , the answer is f(x)=x
\end{solution}



\begin{solution}[by \href{https://artofproblemsolving.com/community/user/134113}{tahanguyen98}]
	[hide]
My solution:
First, we have a statement
Given a function $f$ from $R^{+} \cup {0}$ to $R^{+} \cup {0}$ and $f$ is non-decreasing, if $f(x)>f(y)$ then $x>y$
Proof: Suppose the contrary, $x<y$ then $f(x)\le f(y)$ (because $f$ is non-decreasing) contrary to $f(x)>f(y)$, and if $x=y$ then $f(x)=f(y)$, again a contradiction


Return to our problem, let $f(0)=a$, suppose that $a>0$
$f(\dfrac{x+f(x)}{2}+y)=2x-f(x)+f(f(y))$ - $(1)$
#Subtituting $x=y=0$ in $(1)$ we have 
$f(\dfrac{a}{2})=f(a)-a$ - $(2)$
#Subtituting $x=0$ and $y=\dfrac{a}{2}$  in $(1)$ and considering and $(2)$ in concide we have 
$f(a)=f(f(\dfrac{a}{2}))-a=f(f(a)-a)-a$ thus $f(a)<f(f(a)-a$ (because $a>0$) applying the above statement we have $a\le f(a)-a \rightarrow f(a)>2a$ -  $(3)$
On the other hand if we subtituting $x=f(y)$ in $(1)$ we have 
$f(\dfrac{f(y)+f(f(y))}{2}+y)=2f(y)$ - $(4)$
#Subtituting $y=0$ in $(4)$ we have $f(\dfrac{a+f(a)}{2})=2a<f(a)$ (from $(3)$) applying the above statement again we have $\dfrac{a+f(a)}{2}<a \rightarrow f(a)<a$ - $(5)$
From $(3)(5)$ we have $2a<a$ a contradiction because $a>0$
So $a=0$
Then we easily get $f(\dfrac{x+f(x)}{2})=2x-f(x)$ $(6)$ and $f(y)=f(f(y))$ - $(7)$
From $(6)$ we have $f(\dfrac{x+f(x)}{2})+f(x)=2x$ thus $\{f(u)+f(v)\}=\{R^{+} \cup {0}\}$ - $(*)$
On the other hand subtituting $x->f(x)$ and $y->f(y)$ in $(1)$ and using $(6)(7)$ we have 
$f(f(x)+f(y))=f(x)+f(y)$
Applying $(*)$ we have $f(x)=x$ for all $x \in R^{+} \cup{0}$

[\/hide]
\end{solution}



\begin{solution}[by \href{https://artofproblemsolving.com/community/user/247657}{Ankoganit}]
	Doesn't look like we even need monotonicity :o

Let $c=f(f(0))$ and put $y=0$ in the given equation to get$$f\left(\frac{x+f(x)}2\right)=2x-f(x)+c.\qquad (\star)$$

\begin{bolded}Claim 1.\end{bolded} $f(x)\le x+c$ for all $x\ge 0$.
[hide=Proof] Define a sequence $\langle a_n\rangle_{n=0}^\infty$ as follows: $a_0=2$, and for $n>0$, $a_n=\tfrac{5a_{n-1}+4}{a_{n-1}+8}.$ It's easy to prove by induction that $a_n\ge 1,a_{n+1}\le a_{n}$ for all $n$, so by monotone convergence theorem, the sequence has a limit, say $\ell$. Then we have $\ell=\tfrac{5\ell+4}{\ell+8}\implies \ell=1.$ Now we'll show that $f(x)\le a_nx+c$ for all $x,n$ by induction on $n$. Since the range of $f$ is $\mathbb R^+\cup\{0\}$, from $(\star)$, $2x-f(x)+c\ge 0\implies f(x)\le a_0x+c$. 

Now suppose $f(x)\le a_nx+c$ for all $x$. Then $(\star)$ gives $$2x-f(x)+c=f\left(\frac{x+f(x)}2\right)\le a_n\cdot \left(\frac{x+f(x)}2\right)+c\implies f(x)\ge \frac{4-a_n}{a_n+2}x.$$
then \begin{align*}2x-f(x)+c&=f\left(\frac{x+f(x)}2\right)\ge \left(\frac{4-a_n}{a_n+2}\right)\cdot\left(\frac{x+f(x)}2\right)\\
\implies f(x)&\le \frac{5a_n+4}{a_n+8}\cdot x+\frac{2a_n+4}{a_n+8}\cdot c\implies f(x)\le a_{n+1}x+c.\end{align*}This completes the induction step. Now taking $n\to\infty$ in $f(x)\le a_nx+c$, we have $f(x)\le x+c$ for all $x\ge 0$.[\/hide]

\begin{bolded}Claim 2.\end{bolded} $f(x)=x+\tfrac25c$ for all $x\ge 0$.
[hide=Proof] Define another sequence $\langle b_n\rangle_{n=0}^\infty$ by $b_0=c$ and $b_n=\tfrac29c+\tfrac49b_{n-1}$ for $n>0$. It's easy to see that $b_n\to \tfrac25c$ as $n\to\infty$. Now we'll show that $f(x)\le x+b_n$ for all $n,x$. The base case is clear; now suppose it's true for some $n$. Then again $(\star)$ gives $$2x-f(x)+c=f\left(\frac{x+f(x)}2\right)\le \frac{x+f(x)}2+b_n\implies f(x)\ge x+\frac23(c-b_n).$$
Again, $$2x-f(x)+c=f\left(\frac{x+f(x)}2\right)\ge \frac{x+f(x)}2+\frac23(c-b_n)\implies f(x)\le x+b_{n+1}.$$Now taking $n\to\infty $ yields $f(x)\le x+\tfrac25c$. Then $2x-f(x)+c\le \tfrac{x+f(x)}2+\tfrac25c\implies f(x)\ge x+\tfrac25c$, so in fact $f(x)=x+\tfrac25c.$[\/hide]

Now using $f(x)=x+\tfrac25c$ in the given equation immediately gives $c=0$, so $\boxed{f(x)\equiv x}$ which is indeed a solution. $\blacksquare$
\end{solution}
*******************************************************************************
-------------------------------------------------------------------------------

\begin{problem}[Posted by \href{https://artofproblemsolving.com/community/user/80101}{vanthanhlhp}]
	Find all functions $f: \mathbb R \to \mathbb R$ such that for all reals $x$ and $y$,
\[f\left( {f\left( x \right) + y} \right) = 2x + f\left( {f\left( {f\left( y \right)} \right) - x} \right).\]
	\flushright \href{https://artofproblemsolving.com/community/c6h355188}{(Link to AoPS)}
\end{problem}



\begin{solution}[by \href{https://artofproblemsolving.com/community/user/29428}{pco}]
	\begin{tcolorbox}Find all function $f:R \to R$  satisfying

$f\left( {f\left( x \right) + y} \right) = 2x + f\left( {f\left( {f\left( y \right)} \right) - x} \right)$ for all $x,y \in R$\end{tcolorbox}
Let $P(x,y)$ be the assertion $f(f(x)+y)=2x+f(f(f(y))-x)$
Let $a=f(0)$

$P(\frac{a-x}2,-f(\frac{a-x}2))$ $\implies$ $x=f(f(f(-f(\frac{a-x}2)))-\frac{a-x}2)$ and $f(x)$ is a surjection.

If $f(u)=f(v)$, then, comparing $P(x,u)$ and $P(x,v)$, we get $f(u+f(x))=f(v+f(x))$ and, since $f(x)$ is a surjection, $f(x)=f(x+u-v)$ $\forall x$
Then, comparaison of $P(x,y)$ with $P(x+u-v,y)$ implies $u-v=0$ and so $f(x)$ is injective.

Then, $P(0,x)$  $\implies$ $f(x+a)=f(f(f(x)))$ and so (since injective), $f(f(x))=x+a$

$P(x,y)$ becomes then $f(f(x)+y)=2x+f(y-x+a)$
Setting $y=0$ in the line above, we get $x+a=2x+f(a-x)$ and so $f(a-x)=a-x$ and so $\boxed{f(x)=x}$, which indeed is a solution.
\end{solution}
*******************************************************************************
-------------------------------------------------------------------------------

\begin{problem}[Posted by \href{https://artofproblemsolving.com/community/user/22793}{April}]
	Find all functions $f$ from the set of real numbers into the set of real numbers which satisfy for all $x$, $y$ the identity \[ f\left(xf(x+y)\right) = f\left(yf(x)\right) +x^2\]

\begin{italicized}Proposed by Japan\end{italicized}
	\flushright \href{https://artofproblemsolving.com/community/c6h355780}{(Link to AoPS)}
\end{problem}



\begin{solution}[by \href{https://artofproblemsolving.com/community/user/42653}{Bacteria}]
	Here is my quite lengthy solution:
[hide]
Take $x=0$ to get $f(yf(0))=f(0)$ for all $y$; if $f(0)\neq0$ then $f(x)$ is constant. But take $x=1$ to easily obtain the contradiction of $f(0)=f(0)+1$. Thus $f(0)=0$
Take $y=0$ in our original equation to get 
$f(xf(x))=x^2$ for all x. (1)
Now take $y=-x$ to get
$f(-xf(x))=-x^2$ for all x. (2)
From (1) and (2) it is easily seen that $f(x)$ is onto. Say $f(w)=1$. Then take $x=a$ in (1) to get $w^2=f(wf(w))=f(w)=1$ so $w=\pm1$. 
Now say $f(a)=0$. Then from (1) we have $a^2=f(af(a))=f(0)=0$, so $a=0$. (3)
I now claim $f$ is one to one.
Now, let $f(a)=f(b)=c$ for some pair $a,b$ and $c\neq0$. 
Case 1: $c>0$
Since $f$ is onto we can find $p,q$ such that $f(p+\sqrt{c})=\dfrac{a}{\sqrt{c}}$ and $f(q+\sqrt{c})=\dfrac{b}{\sqrt{c}}$
Now, take $x=\sqrt{c}$, $y=p$ in the original equation to get $c+f(pf(\sqrt{c}))=f(\sqrt{c}f(\sqrt{c}+p))=f(a)=c$, so $f(pf(\sqrt{c}))=0$ and thus by (3) we have $pf(\sqrt{c})=0$, so $p=0$ since $\sqrt{c}\neq0$ so by (3) $f(\sqrt{c})\neq0$.
Thus we have $\sqrt{c}f(\sqrt{c})=a$.  We can perform the exact same steps with $y=q$ to obtain $q=0$ and $b=\sqrt{c}f(\sqrt{c})=a$ so $a=b$.
Case 2: c<0
Now let $p=\dfrac{a}{f(\sqrt{-c})}$ and $q=\dfrac{b}{f(\sqrt{-c})}$. Then $y=p,x=\sqrt{-c}$ in the original equation gives $f(\sqrt{-c}f(\sqrt{-c}+p))=f(pf(\sqrt{-c}))-c=f(a)-c=c-c=0$, so by (3) $\sqrt{-c}f(\sqrt{-c}+p)=0$, so $p=-\sqrt{-c}$ and thus $a=-\sqrt{-c}f(\sqrt{-c})$. Once again we can repeat the same method with $y=q$ to obtain $b=-\sqrt{-c}f(\sqrt{-c})=a$ and indeed $b=a$.

Thus we can combine (3), case 1, and case 2 to conclude that $f$ is one to one. 
Now I claim that $f(-x)=-f(x)$ for all $x$. Let $a$ be arbitrary and say $f(a)=b$
Case 1: $b\geq0$
By (1), $f(\sqrt{b}f(\sqrt{b}))=b=f(a)$, so $a=\sqrt{b}f(\sqrt{b})$, and $-a=-\sqrt{b}f(\sqrt{b})$. Thus $f(-a)=f(-\sqrt{b}f(\sqrt{b}))=-b=-f(a)$ by (2). 
Case 2: $b<0$
This follows a similar path as case 1 except we use $\sqrt{-b}$.

Now, take $x=1,-1$ in our original equation to get
$f(f(y+1))=f(yf(1))+1$
$f(-f(y-1))=f(yf(-1))+1$ so bringing out the minus signs and rearranging gives
$f(yf(1))=f(f(y-1))+1$. Replacing $y$ with $y+2$ gives
$f((y+2)f(1))=f(f(y+1))+1=f(yf(1))+2$. I now give two cases, stemming from the fact that either $f(1)$ or $f(-1)$ is 1.
Case 1: $f(1)=1$.
Then $f(y+2)=f(y)+2$, so $f(2)=f(0)+2=2$. Taking $x=2$ in the original equation gives $f(2(f(y+2))=f(yf(2))+2^2=f(2y)+4=f(2y+2)+2=f(2y+4)$, so since $f$ is one to one $2f(y+2)=2y+4$ and $f(y+2)=y+2$; $f(x)=x$ for all real x is our solution here.
Case 2: $f(-1)=1$.
Now, $f(-y-2)=f(-y)+2$, so $-f(y+2)=-f(y)+2$, and $f(y)=f(y+2)+2$. Replacing $y$ with $y-2$ gives $f(y-2)=f(y)+2$. Thus $f(-2)=f(0)+2=2$. Put $x=-2$ in our original equation to get $f(-2f(y-2))=f(yf(-2))+4=f(2y)+4=f(2y-2)+2=f(2y-4)$, so since $f$ is one to one we get $-2f(y-2)=2y-4$ and $f(y-2)=-y+2$; $f(x)=-x$ for all x.

To sum it up we have two solutions: $f(x)=x$ and $f(x)=-x$. QED
[\/hide]
\end{solution}



\begin{solution}[by \href{https://artofproblemsolving.com/community/user/31917}{daniel73}]
	I try to shorten it a little bit:

1st part, we show $f(0)=0$, $f$ is bijective and $f(x)=-f(-x)$:
[hide] Take $x=0$, or $f(0)=f(yf(0))$.  Assume that $f(0)\neq0$, then for any real $z$, $y=\frac{z}{f(0)}$ exists and $f(z)=f(0)$, or $f$ is constant and $x^2=0$ for all real $x$, absurd, hence $f(0)=0$.  Assume now that $z\neq 0$ exists such that $f(z)=0$.  Take $x=z$ and $y=-z$, or $f(zf(0))=f(0)+z^2$, and $z=0$, absurd, or $f(x)=0$ iff $x=0$.
Assume that $z\neq x\neq0$ exist such that $f(z)=f(x)$, and take $y=z-x$, or $x^2=f(xf(z))=f((z-x)f(x))$; clearly either $z-x=0$ absurd, or $x=0$ absurd, hence $f$ is injective.
Take now $y=0$, or $f(xf(x))=x^2$, and take $y=-x$, or $f(-xf(x))=-x^2$, or for any positive real $z$, $f(\sqrt{z}f(\sqrt{z}))=z$, and for any negative real $z$, $f(-\sqrt{-z}f(\sqrt{-z}))=-z$, and $f$ is surjective.
Take finally $x=-y\neq0$, then $0=f(yf(-y)+y^2=f(-yf(y))+y^2$, or since $f$ is injective, $yf(-y)=-yf(y)$, or $f(-y)=-f(y)$.
[\/hide]

2nd part, we prove that for all $z,x$, we have [hide]$f(xf(z))=xz$[\/hide]
[hide]
Take $y=z-x$, or $f(xf(z))=f((z-x)f(x))+x^2$; transform now $x$ into $x+z$, or $f((x+z)f(z))=-f(xf(x+z))+(z+x)^2$.
Take also $y=-z-x$, or $-f(xf(z))=f(xf(-z))=x^2+f(-(z+x)f(x))=x^2-f((z+x)f(x))=x^2+f(zf(x+z))-(z+x)^2$ (we have used the previous result exchanging $z$ and $x$), and finally $z^2+2zx-f(xf(z))=f(zf(x+z))=f(xf(z))+z^2$, yielding $f(xf(z))=xz$.
[\/hide]

3rd part, conclusion
[hide]
Choose $k$ such that $f(k)=1$, which exists because $f$ is surjective, and take $z=k$.  Clearly $f(x)=kx$ for all $x$, and substitution in the original equation yields $k^2x(x+y)=f(kx(x+y))=f(xf(x+y))=f(yf(x))+x^2=f(kxy)+x^2=k^2xy+x^2$, or for all $x$ we must have $k^2x^2=x^2$, or $k=\pm1$, hence exactly two solutions may exist, $f(x)=x$ for all $x$, and $f(x)=-x$, depending on whether $f(1)=1$ or $f(1)=-1$.
[\/hide]
\end{solution}



\begin{solution}[by \href{https://artofproblemsolving.com/community/user/31919}{tenniskidperson3}]
	I don't understand how $ x^{2}=f(xf(z))-f((z-x)f(x)) $ and $f(x)=f(z)$ implies $z-x=0$ or $x=0$.  Could you explain?
\end{solution}



\begin{solution}[by \href{https://artofproblemsolving.com/community/user/31917}{daniel73}]
	\begin{tcolorbox}I don't understand how $ x^{2}=f(xf(z))-f((z-x)f(x)) $ and $f(x)=f(z)$ implies $z-x=0$ or $x=0$.  Could you explain?\end{tcolorbox}

Since $f(z)=f(x)$, then $f(xf(z))=f(xf(x))=x^2$, or $f((z-x)f(x))=0$, ie by previous results $(z-x)f(x)=0$, yielding either $z=x$ or $f(x)=0$, this latter one being equivalent again to $x=0$.

Hope this clears it out...
\end{solution}



\begin{solution}[by \href{https://artofproblemsolving.com/community/user/82066}{lefao}]
	\begin{tcolorbox}[quote="tenniskidperson3"]I don't understand how $ x^{2}=f(xf(z))-f((z-x)f(x)) $ and $f(x)=f(z)$ implies $z-x=0$ or $x=0$.  Could you explain?\end{tcolorbox}

Since $f(z)=f(x)$, then $f(xf(z))=f(xf(x))=x^2$, or $f((z-x)f(x))=0$, ie by previous results $(z-x)f(x)=0$, yielding either $z=x$ or $f(x)=0$, this latter one being equivalent again to $x=0$.

Hope this clears it out...\end{tcolorbox}

What if $f((z-x)f(x)=x^2 $? It also imply $z=x$ or $x=0$?
\end{solution}



\begin{solution}[by \href{https://artofproblemsolving.com/community/user/31917}{daniel73}]
	\begin{tcolorbox}[quote="daniel73"][quote="tenniskidperson3"]I don't understand how $ x^{2}=f(xf(z))-f((z-x)f(x)) $ and $f(x)=f(z)$ implies $z-x=0$ or $x=0$.  Could you explain?\end{tcolorbox}

Since $f(z)=f(x)$, then $f(xf(z))=f(xf(x))=x^2$, or $f((z-x)f(x))=0$, ie by previous results $(z-x)f(x)=0$, yielding either $z=x$ or $f(x)=0$, this latter one being equivalent again to $x=0$.

Hope this clears it out...\end{tcolorbox}

What if $f((z-x)f(x)=x^2 $? It also imply $z=x$ or $x=0$?\end{tcolorbox}

You have a bracket missing; if you mean $f((z-x)f(x))=x^2$, then we know by previous results that $x^2=f(xf(x))$, and that $f$ is injective, or you may deduce that $f(z-x)f(x)=xf(x)$, yielding either $f(x)=0$ (which again results in $x=0$), or $f(z-x)=x$.  Since we have chosen $x\neq0$, you would obtain that $f((z-x)f(x))=x^2$ implies $f(z-x)=x$.  However, I do not know under which circumstances you would obtain $f((z-x)f(x))=x^2$, and I do not see clearly why $f(z-x)=x$ would be useful, but if it is, feel free to use it.
\end{solution}



\begin{solution}[by \href{https://artofproblemsolving.com/community/user/35129}{Zhero}]
	Set $x = 0$. $f(0) = f(yf(0))$ for all $y$. If $f(0) \neq 0$, $yf(0)$ spans the set of reals, so $f$ must be constant, which is clearly impossible. Hence, $f(0) = 0$.

Setting $y=0$ shows that $f(xf(x)) = x^2$ for all real $x$, and setting $x=-y$ shows that $f(-xf(x)) = -x^2$ for all real $x$. It follows from these that if $f(v) = 0$, then $v^2 = f(vf(v)) = f(0) = 0$, i.e., $v = 0$.

We now claim that $f$ is injective. Suppose that $f(a) = f(a+b)$. We wish to show that $b=0$. From $f(af(a)) = a^2$, we have $a^2 = f(af(a)) = f(af(a+b)) = f(bf(a)) = a^2$, whence $f(bf(a)) = 0$. Hence, $b = 0$ or $f(a) = 0$. If $f(a) = 0$, then from the above $a = b = 0$, so $f$ is injective.

From $f(xf(x)) = x^2$, we have $xf(x) f(xf(x)) = x^3 f(x)$, so $f(x^3 f(x)) = f((xf(x)) f(xf(x))) = x^2 f(x)^2$. When $x=1$, we have $1^2 = f(1 f(1)) = f(1^3 f(1)) = f(1)^2$. Similarly, $f(-1)^2 = 1$. By injectivity, either $f(1) = 1$ and $f(-1) = -1$, or $f(-1) = 1$ and $f(1) = -1$. Since $f(x)$ is a solution to the functional equation if and only if $f(-x)$ is (as we may replace each occurence of $x$ and $y$ in the original functional equation with $-x$ and $-y$), we may for now assume without loss of generality that $f(1) = 1$ and $f(-1) = -1$.

Setting $y=x$ in the original functional equation gives $f(xf(2x)) = f(xf(x)) + x^2 = 2x^2$. Since $f(\sqrt{2} x f( \sqrt{2} x)) = 2x^2$ as well, by injectivity $x f(2x) = \sqrt{2} x f( \sqrt{2} x)$. Replacing $x$ with $\frac{\sqrt{2} x}{2}$ and rearranging gives $f(\sqrt{2} x) = \sqrt{2} f(x)$. Hence, $f(2x) = f( \sqrt{2} (\sqrt{2} x)) = \sqrt{2} f(\sqrt{2} x) = 2x$.

Setting $x=1$ in the original functional equation gives $f(f(y+1)) - f(y) = 1$, and setting $x=2$ gives $f(f(y+2)) - f(y) = 2$. Subtracting these two equations gives $f(f(y+2)) - f(f(y+1)) = 1$ for all $y$, or $f(f(y+1)) = f(f(y)) + 1$. Substituting this into $f(f(y+1)) - f(y) = 1$ yields $f(f(y)) = f(y)$. By injectivity, $f(y) = y$ for all real $y$.

Recall that this is only the solution when $f(1) = 1$; if $f(1) = -1$, $f(-x)$ is the solution to this functional equation, so $f(x) = x$ and $f(x) = -x$ are the only solutions.
\end{solution}



\begin{solution}[by \href{https://artofproblemsolving.com/community/user/1991}{orl}]
	Information by Kazuaki Kobayashi, MOF (Mathematic Olympiad Foudation) of Japan about the background for this problem.
\end{solution}



\begin{solution}[by \href{https://artofproblemsolving.com/community/user/29428}{pco}]
	\begin{tcolorbox}Find all functions $f$ from the set of real numbers into the set of real numbers which satisfy for all $x$, $y$ the identity \[ f\left(xf(x+y)\right) = f\left(yf(x)\right) +x^2\]

\begin{italicized}Proposed by Japan\end{italicized}\end{tcolorbox}
Here is a long and rather ugly solution.

Let $P(x,y)$ be the assertion $f(xf(x+y))=f(yf(x))+x^2$

1) $f(x)=0$ $\iff$ $x=0$
==============
If $f(0)\ne 0$, then $P(0,\frac x{f(0)})$ $\implies$ $f(x)=f(0)$ constant, which is not a solution. So $f(0)=0$
If $f(u)=0$, then $P(u,-u)$ $\implies$ $u^2=0$
Q.E.D.

2) $f(x)$ is a bijection
=============
$P(x,0)$ $\implies$ $f(xf(x))=x^2$ and so $\mathbb R^+\cup\{0\}\subseteq f(\mathbb R)$
$P(x,-x)$ $\implies$ $f(-xf(x))=-x^2$ and so $\mathbb R^-\cup\{0\}\subseteq f(\mathbb R)$
And so $f(x)$ is a surjection

Let then $a<0$ and $u$ such that $f(u)=a$ :
$P(\sqrt{-a},\frac u{f(\sqrt{-a})})$ $\implies$ $f(\sqrt{-a}f(\sqrt{-a}+\frac u{f(\sqrt{-a})}))=0$
So (using 1. above) : $f(\sqrt{-a}+\frac u{f(\sqrt{-a})})=0$
So (using again 1) above) : $\sqrt{-a}+\frac u{f(\sqrt{-a})}=0$ and $u=-\sqrt{-a}f(\sqrt{-a})$
And so a unique $u$ for any $a<0$

Let then $a>0$ and $u\ne 0$ such that $f(u)=a$ and $v$ such that $f(v)=\frac u{\sqrt a}$ :
$P(\sqrt a,v-\sqrt a)$ $\implies$ $f((v-\sqrt a)f(\sqrt a))=0$
So (using 1. above) : $(v-\sqrt a)f(\sqrt a)=0$ and so $v=\sqrt a$ and $u=\sqrt af(\sqrt a)$
And so a unique $u$ for any $a>0$

And so $f(x)$ is an injection.
Q.E.D.

3) $f(x)$ is an odd function
=================
Let $x\ne 0$
$P(x,0)$ $\implies$ $f(xf(x))=x^2$
$P(-x,0)$ $\implies$ $f(-xf(-x))=x^2$
And so $f(xf(x))=f(-xf(-x))$ and so, since injective, $xf(x)=-xf(-x)$ and $f(-x)=-f(x)$ (true also if $x=0$)
Q.E.D.

4) $f(2x)=2f(x)$
=========
Let $x\ne 0$ :
$P(x,-2x)$ $\implies$ $-f(xf(x))=-f(2xf(x))+x^2$
$P(-x,-x)$ $\implies$ $f(xf(2x))=f(xf(x))+x^2$
Subtracting, we get : $f(xf(2x))=f(2xf(x))$ and, since injective : $xf(2x)=2xf(x)$ and so $f(2x)=2f(x)$ (true also if $x=0$)
Q.E.D.

5) $f(x+y)=f(x)+f(y)$ $\forall x,y$
=====================
$P(x,y)$ $\implies$ $f(xf(x+y))=f(yf(x))+x^2$
$P(x+y,-x)$ $\implies$ $f((x+y)f(y))=-f(xf(x+y))+(x+y)^2$
Subtracting, we get a new assertion $Q(x,y)$ : $f((x+y)f(y))+f(yf(x))=y^2+2xy$

$Q(x-y,y)$ $\implies$ $f(xf(y))+f(yf(x-y))=2xy-y^2$
$Q(x-y,2y)$ $\implies$ $f((x+y)f(y))+f(yf(x-y))=2xy$
Subtracting, we get $f(xf(y)+yf(y))-f(xf(y))=y^2$

And so $f(xf(y)+yf(y))=f(xf(y))+f(yf(y))$ (remember that $P(y,0)$ $\implies$ $f(yf(y))=y^2$)

And so $f(x+yf(y))=f(x)+f(yf(y))$ $\forall x,y$

And so $f(x+y)=f(x)+f(y)$ $\forall x\in\mathbb R,\forall y\in f^{[-1]}(\mathbb R^+\cup\{0\})$

But then $f(-x+y)=-f(x)+f(y)$ and so $f(x-y)=f(x)+f(-y)$
And so $f(x+y)=f(x)+f(y)$ $\forall x\in\mathbb R,\forall y\in f^{[-1]}(\mathbb R^-\cup\{0\})$
Q.E.D.

6) either $f(x)=x$ $\forall x$, either $f(x)=-x$ $\forall x$
================================
Using $f(x+y)=f(x)+f(y)$, $P(x,y)$ becomes $f(xf(x))+f(xf(y))=f(yf(x))+x^2$ and so $f(xf(y))=f(yf(x))$

$P(x+y,0)$ $\implies$ $f(xf(x))+f(xf(y))+f(yf(x))+f(yf(y))=(x+y)^2$ and so $f(xf(y))=xy$

Setting $y=f(1)$ and using $f(yf(y))=y^2$, we get $f(x)=xf(1)$

Plugging this in original equation, we get $f(1)^2=1$
Hence the two solutions :

either $f(1)=1$ and so $\boxed{f(x)=x}$ $\forall x$

either $f(1)=-1$ and so $\boxed{f(x)=-x}$ $\forall x$
\end{solution}



\begin{solution}[by \href{https://artofproblemsolving.com/community/user/44083}{jgnr}]
	Another proof of injectivity: $f(x)=f(y)$ implies $x^2=f(xf(x))=f(xf(y))=f((y-x)f(x))+x^2$, which gives $f((y-x)f(x))=0$. So $y=x$ or $x=0$, both of which implies $x=y$.

Another way to continue from (3):
$f(xf(y))=f((y-x)f(x))+x^2=-f((x-y)f(x))+x^2=-f(yf(x-y))-(x-y)^2+x^2=f(yf(y-x))-(x-y)^2+x^2=f(-xf(y))+y^2-(x-y)^2+x^2=-f(xf(y))+2xy$
so $f(xf(y))=xy$, and similarly we get $f(yf(x))=xy$. Hence $f(xf(y))=f(yf(x))$, which gives $xf(y)=yf(x)$. So $f(x)=cx$, and we easily get $c=\pm1$.
\end{solution}



\begin{solution}[by \href{https://artofproblemsolving.com/community/user/29428}{pco}]
	\begin{tcolorbox}Another proof of injectivity: $f(x)=f(y)$ implies $x^2=f(xf(x))=f(xf(y))=f((y-x)f(x))+x^2$, which gives $f((y-x)f(x))=0$. So $y=x$ or $x=0$, both of which implies $x=y$.

Another way to continue from (3):
$f(xf(y))=f((y-x)f(x))+x^2=-f((x-y)f(x))+x^2=-f(yf(x-y))-(x-y)^2+x^2=f(yf(y-x))-(x-y)^2+x^2=f(-xf(y))+y^2-(x-y)^2+x^2=-f(xf(y))+2xy$
so $f(xf(y))=xy$, and similarly we get $f(yf(x))=xy$. Hence $f(xf(y))=f(yf(x))$, which gives $xf(y)=yf(x)$. So $f(x)=cx$, and we easily get $c=\pm1$.\end{tcolorbox}

Yesssss, quite OK for both parts and quite nice !
Much quicker and simpler than my ugly steps. Congrats !
 
:)
\end{solution}



\begin{solution}[by \href{https://artofproblemsolving.com/community/user/72819}{Dijkschneier}]
	No need for injectivity !
$P(0,y) \implies f(0)=f(yf(0))$, and if f(0) is not zero, then f is constant and we get a contradiction. Hence f(0)=0.
$P(x,0) \implies f(xf(x))=x^2$ and $P(x,-x) \implies f(-xf(x))=-x^2$ and in particular, f is onto.
1) $f(x)=0 \iff x=0$
Let $c\in \mathbb{R}$ such that f(c)=0.
$P(c,0) \implies f(cf(c))=c^2 \implies 0=c^2 \implies c=0$
2) $f(d)=1 \implies d=\pm 1$
Let $d\in \mathbb{R}$ such that f(d)=1.
$P(d,0) \implies f(df(d))=d^2 \implies 1=d^2 \implies d=\pm 1$
3) $f(x)=y^2 \implies x=yf(y)$
Let x,y $\neq 0$ and define $s=\frac{x}{f(y)}$.
$P(s,y-s) \implies f(x)=f((y-s)f(s))+s^2$
If f(x)=s², then f((y-s)f(s))=0, hence y-s=0 or f(s)=0, y=s or s=0, y=s, f(y)=f(s), x=sf(s).
As f is onto, s take all the real values when y changes in $\mathbb{R}$, and hence the desired.
4) $f(2x)=2f(x)$
$P(x,x) \implies f(xf(2x))=f(xf(x))+x^2=2x^2=(\sqrt{2}x)^2 \\
\implies xf(2x)=\sqrt{2}xf(\sqrt{2}x) \\
\implies f(2x)=2f(x)$

Now from f(f(1))=1 and 2), we see that f(1)=1 or f(1)=-1.
1) First case : f(1)=1
$f(2x)=2f(x) \implies f(2)=2$
$P(1,y) \implies f(f(y+1))=f(y)+1$
$P(2,y) \implies f(2f(y+2))=f(2y)+4 \\
\implies f(f(y+2))=f(y)+2 \\
\implies f(y+1)+1=f(y)+2 \\
\implies f(y+1)=f(y)+1$
Now, $f(y)+1=f(f(y+1))=f(f(y)+1)=f(f(y))+1 \implies f(y)=f(f(y))$ and since f is onto we get f(x)=x for all real x.
2) Second case : f(1)=-1
$f(2x)=2f(x) \implies f(2)=-2$
$P(1,y) \implies f(f(y+1))=f(-y)+1$
$P(2,y) \implies f(2f(y+2))=f(-2y)+4 \\
\implies f(f(y+2))=f(-y)+2 \\
\implies f(-y-1)+1=f(-y)+2 \\
\implies f(-y-1)=f(-y)+1 \\
\implies f(y-1)=f(y)+1
\implies f(y+1)=f(y)-1$
Now, $f(-y)+1=f(f(y+1))=f(f(y)-1)=f(f(y))+1 \implies f(f(y))=f(-y)$.
$f(x^2)=f(f(xf(x))=f(-xf(x))=-x^2$ and$ f(-x^2)=f(f(-xf(x))=f(xf(x))=x^2$ so f(x)=-x for all real x.

Hence the two solutions : f(x)=x and f(x)=-x.
\end{solution}



\begin{solution}[by \href{https://artofproblemsolving.com/community/user/183149}{JuanOrtiz}]
	Sorry for not translating.

Sea la ecuación original $(1)$.

Con $x=0$ en $(1)$ obtenemos que $f(0)=0$ ya que $f(yf(0))=f(0)$ y $f$ no puede ser constante.

Con $y=0$ en $(1)$ obtenemos $f(xf(x))=x^2$, ecuación a la que llamaré $(2)$, y llamemos $g(x)=xf(x)$. Con $y=-x$ en $(1)$ obtenemos que $f(-xf(x))=-x^2$, ecuación a la que llamaré $(3)$. Con ésto, tenemos ya que $f$ es suryectiva.

Supongamos que existe $a \neq 0$ tal que $f(a)=0$. Entonces en $(1)$ con $a=x$ tendremos que $a^2=f(af(a+y))$ para toda $y$. Tomemos $e_0$ un número distinto de $a^2$. Por suryectividad existe $e_1$ tal que $f(e_1)=e_0$. Luego, existe $e_2$ tal que $f(e_2)=e_1\/a$. Con $y=e_2-a$ tendremos que $e_0=a^2$, contradicción. Entonces $f(a)=0$ sii $a=0$.

Ahora supongamos $f(a)=f(b)$, y sea $k=f(a)$. En $(1)$ con $x=a$, $y=b-a$ tendremos, por $(2)$, que $a^2=f(af(a))=f(af(b))=f(xf(x+y))=f(yf(x))+a^2$ y entonces $f(yf(x))=0$. Por tanto, $yf(x)=0$. Si $f(x)=0$ entonces $f(a)=f(b)=0$ entonces $a=b=0$. De lo contrario, $y=0$ y así $a=b$. Obtenemos que $f$ es inyectiva. Por tanto, $f$ es biyectiva.

Por $(2)$, $f(g(x))=x^2$ y $f(-g(x))=-x^2$. Con ésto es fácil ver que $g$ es suryectiva (dada la inyectividad de $f$). Sea $a$ un real cualquiera. Si $a=0$ tengo $f(0)=-f(-0)$. Sea $b$ tal que $g(b)=a$. Entonces vemos que $f(a)=-f(-a)$. De aquí, $f$ es impar.

Ahora mostraré que $f(nxf(-x))=nx^2$ para todo $n \ge 0$ entero, usando inducción sobre $n$. Para $n=0$ es obvio. Lo demostraré para $n$ usando que es cierto para $n-1$. Tenemos por imparidad y por $(2)$ que $f((n-1)g(x))=(n-1)f(g(x))$ para todo $x$. De aquí vemos que $f((n-1)x)=(n-1)f(x)$ dado que $g$ es suryectiva. Ahora, en $(1)$ sea $y=na$ y $x=-a$ para cualquier $a$. Tendremos que

$-a^2(n-1)=f((n-1)af(-a))=f(-af((n-1)a))=f(xf(x+y))=a^2+f(nxf(-x))$

Y de aquí, $f(nxf(-x))=-nx^2$ para toda $a$. La inducción está completa.

Entonces, usando el argumento de arriba, pues $g$ es suryectiva, encontramos que $f(na)=nf(a)$ para todo $n \ge 0$ entero. Por imparidad vemos fácilmente que $f(ra)=rf(a)$ para todo $r$ racional, para todo $a$ real. A ésta ecuación le llamo $(4)$. 

En $(2)$ con $x=1$ y con $x=f(1)$ obtenemos que $f(f(1))=1$ y luego que $f(f(1))=f(1)^2$. Luego $f(1)=1$ ó $-1$. Dado que $f$ es impar, si defino $f_1(x)=-f(x)$, seguirá cumpliendo $(1)$, y entonces puedo suponer sin pérdida de generalidad que $f(1)=1$. De $(4)$, se da $f(r)=r$ para $r$ racional.

De $(1)$ con $x=1$ observemos que $f(f(y+1))=f(y)+1$. A ésto le llamo $(5)$.

Ahora en $(1)$ sea $x \neq 0$ un racional cualquiera y usemos $(5)$ y $(4)$. Resulta que

$x(f(x+y-1)+1)=xf(f(x+y))=f(xf(x+y))=f(yf(x))+x^2$$=xf(y)+x^2$

y así $f(x+y-1)+1=f(y)+x$. Sea $r=x-1$, se da que $f(r+y)=f(y)+r$ para todo $r \neq -1$ racional y para todo $y$. A ésto le llamo $(6)$.

Utilizemos $(5)$ y $(6)$. Con $r=1$ tenemos que $f(x+1)=1+f(x)$ y entonces $f(x)=1+f(x-1)$. Resulta que $f(f(x))=f(x-1)+1=f(x)$ para todo $x$. Por inyectividad tendremos $f(x)=x$ para toda $x$. Recordemos que supusimos sin pérdida de generalidad que $f(1)=1$, entonces en realidad $-f$ también cumple.

Por ende, las únicas soluciones son:

$f(x)=x$

$f(x)=-x$
\end{solution}



\begin{solution}[by \href{https://artofproblemsolving.com/community/user/145173}{Aiscrim}]
	[hide="Solution"]
Let $P(x,y)$ the assertion that $f(xf(x+y))=f(yf(x))+x^2$.

From $P(0,x)$ we get that $f(0)=f(xf(0))$. If $f(0)\ne 0$, taking $x:=\dfrac{x}{f(0)}$, we get that $f(0)=f(x),\ \forall x\in \mathbb{R}$, so the function is a constant, contradiction. Therefore, $f(0)=0$.

From $P(x,0)$ we get that $f(xf(x))=f(0)+x^2=x^2$. From $P(x,-x)$ we get that $0=f(xf(0))=f(xf(-x))+x^2$, so $f(xf(-x))=-x^2$. We conclude that $f$ is surjective.

Let $x_0\in \mathbb{R}$ such that $f(x_0)=0$. From $P(x_0,0)$, $x_0^2=f(x_0f(x_0))=f(0)=0$, hence $x_0=0$. We conclude that $f(x)=0\Leftrightarrow x=0$.

Let $\alpha\in \mathbb{R}$ such that $f(\alpha)=x,\ x\ne 0$. From $P(f(x),\alpha-f(x))$, we get that $f(f(x)f(\alpha))=f((\alpha-f(x))f(f(x)))+x^2\Leftrightarrow f((\alpha-f(x))f(f(x)))=0\Leftrightarrow \alpha=f(x)$
 so $f(f(x))=x$.

Taking $x\rightarrow f(x)$ in $f(xf(x))=x^2$, we get that $f(f(x)x)=f(x)^2$, therefore $f(x)\in \{x,-x\},\ \forall x\in \mathbb{R}$. 

It is simple to see that we can't have $f(x)=x$ for some values of $x$ and $f(x)=-x$ for the other.

We conclude that the only functions are $\boxed{\mbox{f(x)=x,\ \forall x\in \mathbb{R}}}$ and $\boxed{\mbox{f(x)=-x,\ \forall x\in \mathbb{R}}}$

[\/hide]
\end{solution}



\begin{solution}[by \href{https://artofproblemsolving.com/community/user/202190}{Blitzkrieg97}]
	\begin{tcolorbox}[hide="Solution"]

$f(f(x)f(\alpha))=f((\alpha-f(x))f(f(x)))+x^2$

[\/hide]\end{tcolorbox}
i think there's mistake,shouldn't there be :$f(f(x)f(\alpha))=f((\alpha-f(x))f(f(x)))+f(x)^2$ ?
if i'm wrong,i apologize
\end{solution}



\begin{solution}[by \href{https://artofproblemsolving.com/community/user/145173}{Aiscrim}]
	You are right. My mistake, sorry :blush:
\end{solution}



\begin{solution}[by \href{https://artofproblemsolving.com/community/user/167924}{utkarshgupta}]
	Let $P(x,y)$ be the assertion $f(x(f(x+y))=f(yf(x))+x^2$

\begin{bolded}Lemma 1 :\end{bolded} $f(0)=0$
\begin{bolded}Proof :\end{bolded}
Let $f(0) \neq 0$
Consider $P(0,\frac{y}{f(0)})$
$$ \implies f(0)=f(y)$$
That is $f \equiv f(0)$ 
But this satisfy the conditions of the given equation.
Hence $\boxed{f(0)=0}$

\begin{bolded}Lemma 2 :\end{bolded} $f(\alpha)=0 \implies \alpha =0$
\begin{bolded}Proof :\end{bolded}
Let if possible their exist some $\alpha \neq 0$ such that $f(\alpha)=0$
Consider $P(\alpha,0)$
$$\implies f(\alpha f( \alpha ))=f(0)+\alpha ^2$$
$$\implies \alpha ^2 = 0$$
A contradiction !!!

Thus we must have only root of $f$ as $0$.


\begin{bolded}Lemma 3 :\end{bolded} $f$ is injective
\begin{bolded}Proof :\end{bolded}
Let if possible there exist some $a \neq b$ and $f(a)=f(b)$
Then consider $P(b,a-b)$
$$\implies f(bf(a))=f((a-b)f(b))+b^2$$
And now consider $P(b,0)$
$$\implies f(bf(b))=b^2$$

$$\implies f((a-b)f(b))=0$$
By Lemma 2, $\implies (a-b)f(b)=0$
Thus we have either $a=b$ or $f(b)=0=f(a)$ which again implies $a=b$(by Lemma 2)
A contradiction.
Thus we get $f$ is injective.

\begin{bolded}Lemma 4 :\end{bolded} $f(xf(x))=x^2$
\begin{bolded}Proof :\end{bolded}
Just consider $P(x,0)$

\begin{bolded}Lemma 5 :\end{bolded} $-f(s)=f(-s)$
\begin{bolded}Proof :\end{bolded} 
For $x=0$ this is obvious.
Now consider some $x \neq 0$
By Lemma 4,
$$f(xf(x))=x^2=f(-xf(-x))$$
Using injectivity,
$$xf(x)=-xf(-x)$$
$$\implies f(x)=-f(-x)$$
Hence the lemma.

\begin{bolded}Lemma 6 :\end{bolded} $f(2x)=2f(x)$
\begin{bolded}Proof :\end{bolded}
Consider $P(x,x)$
$$\implies f(xf(2x))=f(xf(x))+x^2 = 2x^2 = f(x\sqrt{2}f(x\sqrt{2}))$$
Injectivity implies
$$xf(2x)=x\sqrt{2}f(x\sqrt{2})$$
$$\implies f(2x)=\sqrt{2}f(x\sqrt{2})$$
Replacing $x$ by $\frac{x}{\sqrt{2}}$,
$$\implies f(x\sqrt{2}) = \sqrt{2}f(x)$$
Putting this back,
$$\boxed{f(2x)=2f(x)}$$

\begin{bolded}Lemma 7 :\end{bolded} $f(1)^2=1$
\begin{bolded}Proof :\end{bolded}
Let $Q(x)$ be the assertion $f(xf(x))=x^2$
$Q(1) \implies f(f(1))=1$
Consider $Q(xf(x))$
We have, $f(xf(x)f(xf(x)))=(xf(x))^2$
That is, we have $f(x^3f(x))=x^2f(x)^2$
Setting $x=1$ here, we have
$f(f(1))=f(1)^2$
Hence we have 
$$ \boxed {f(1)^2=1}$$

\begin{bolded}Lemma 8 :\end{bolded} $f(yf(1))+f((y+2)f(-1))+2=0$
\begin{bolded}Proof :\end{bolded}
Consider $P(1,y)$
$$\implies f(f(y+1))=f(yf(1))+1$$
Now consider $P(-1,y+2)$
$$\implies f(-f(y+1))=f((y+2)f(-1))+1$$

Now using $f$ is odd (Lemma 5), and adding the two equations,
$$f(f(y+1))+f(-f(y+1))=f(yf(1))+f((y+2)f(-1))+2$$
That is 
$$\boxed {f(yf(1))+f((y+2)f(-1))+2=0}$$

\begin{bolded}Main Proof :\end{bolded}

Since $f(1)^2=1$, we will form $2$ cases
Also observe that by lemma 6, $f(2)=2f(1)$,

\begin{bolded}Case 1 :\end{bolded} $f(1)=1$

Since $f$ is odd, $f(-1)=-1$
Also, $f(2)=2$
Lemma 8 yields, $f(y)+f(-y-2)+2=0$
That is, $f(y+2)=f(y)+2$

Consider $P(2,y)$ and using $f(2x)=2f(x)$,
$$f(2f(y+2))=f(yf(2))+4$$
$$\implies 2f(f(y+2)) = f(2y)+4$$
$$\implies f(f(y+2))=f(y)+2$$
$$\implies f(f(y+2))=f(y+2)$$
Usinf injectivity
$f(y+2)=y+2$
That is
$$\boxed{f(x)=x}$$
This is a solution.

\begin{bolded}Case 2 :\end{bolded} $f(1)=-1$

This yields $f(2)=-2$ and $f(-1)=1$
Lemma 8 yields $f(-y)+f(y+2)+2=0$
That is $f(y+2)=f(y)-2$

Again consider $P(2,y)$ and use $f(2x)=2f(x)$
$$f(2f(y+2))=f(yf(2))+4$$
$$\implies 2f(f(y+2))=-2f(y)+4$$
$$\implies f(f(y+2))=-f(y+2)$$
$$\implies f(f(y+2))=f(-(y+2))$$
Using injectivity,
$f(y+2)=-(y+2)$

That is
$$\boxed {f(x)=-x}$$



So the only solutions are $\boxed{f(x) \equiv x}$ and $\boxed{f(x) \equiv -x}$
\end{solution}



\begin{solution}[by \href{https://artofproblemsolving.com/community/user/243741}{anantmudgal09}]
	Here is my very algorithmic and intuitively obvious solution. Hope it is nice.

[hide=Lengths and FE D:] We claim that the only such functions are $f(x)=x, \forall x \in \mathbb{R}$ or $f(x)=-x, \forall x \in \mathbb{R}$

It is particularly easy, to check that these indeed, work well. We prove that these are the only such functions. We use a sequence of claims and solves, so bear with us please, as it is a very very long ride and we don't want you to fall asleep :D

Claim 1. If $f$ is a solution, then so is $-f$.
Proof:- Trivial!

Claim 2. $f(0)=0$
Proof:- Assume to the contrary. Put $(x,y)=(0,y)$. This gives $f(0)=f(yf(0))$ for all $y$. Replace $y$ by $\frac{t}{f(0)}$ where $t$ is any variable real number. This clearly gives $f(t)=0$ for all $t$ which is false. Thus, $f(0)=0$.

Claim 3. $f(-xf(x))=f(xf(-x))=-x^2$
Proof:- Replace $(x,y)=(x,-x)$ and see that we get $f(-xf(x))=-x^2$. Replace $(x,y)=(-x,x)$ and we get $f(xf(-x))=-x^2$

Claim 4. $f(xf(x))=x^2$
Proof:- Replace $(x,y)=(x,0)$ we get this result.

Claim 5. $f$ is surjective.
Proof:- Follows from Claim 3 and Claim 4.

Claim 6. $f$ is injective at point $0$.
Proof:- Let $x_0 \in \mathbb{R}$ such that $f(x_0)=0$. Assume that $x_0 \not=0$. Then, in claim 4, we replace $x$ by $x_0$, giving us $x_0^2=0$ and thus, $x_0=0$ establishing our said result.

Claim 7. $f$ is injective everywhere.
Proof:- The tricky thing is to let $z$ be an arbitrary real number and substitute $y=z-x$. Then, our equation becomes $f(xf(z))=f((z-x)f(x))+x^2$ for all $x,z \in \mathbb{R}$. Now, if for some $z \not=x$ with $x,z$ both non zero, we have $f(z)=f(x)$ then $x^2=_{\text claim 4}f(xf(x))=f(xf(z))=f((z-x)f(x))+x^2$ and so $f((z-x)f(x))=0$ and by claim 6, we have $(z-x)f(x)=0$ but this means that either $z=x$ or $f(x)=0$ and thus, $x=0$. Neither holds, proving, along with claim 6, injectivity at all points.

Claim 8. $f$ is an odd function.
Proof:- We know by claim 3 that $f(-xf(x))=f(xf(-x))$ and by claim 7, we have $-xf(x)=xf(-x)$ added with claim 2 gives $f(x)=-f(-x)$ yielding our result.

Claim 9. $f(1)=1$ "and even if it isnt, we can force it be"
Proof:- Again if we assume that $x_1 \in \mathbb{R}$ such that $f(x_1)=1$. Then, by claim 4, we have $x_1^2=1$. Now, comes the trick, we know that $f(1)=-f(-1)$ and so, if $f(1) \not= 1$ then it must be $-1$ and since by claim 1, we can again make it $f(1)=1$. This proves our claim.

Claim 10. $f(2x)=2f(x)$ for all real numbers $x$.
Proof:- Put $y=x$ in the original equation. Then, we get that $f(xf(2x))=f(xf(x))+x^2=2x^2=f(\sqrt{2}xf(\sqrt{2}x))$ and so, we have by injectivity of $f$ as in claim 7, $xf(2x)=\sqrt{2}xf(\sqrt{2}x)$ and now, we replace $x$ by $\frac{x}{\sqrt{2}}$ which gives that $f(\sqrt{2}x)=\sqrt{2}f(x)$ and thus, $f(2x)=2f(x)$ for all $x$.

Claim 11. $f(y+2)=f(y)+2$
Proof:- This is the final blow of insight i have to offer. We replace $(x,y)=(1,y)$ giving $f(f(1+y))=f(y)+1$ and when we replace $(x,y)$ by $(-1,y+2)$ then we get $-f(f(y+1))=-f(y+2)+1$ and so, $f(f(y+1))=f(y+2)-1$. Finally, these two yield $f(y+2)=f(y)+2$. Proves our claim.

End Game:- We now proceed to proving our initial claim. (oops, too many of them by now!)
We firstly put $(x,y)=(2,y)$. This gives $f(2f(y+2))=f(yf(2))+2^2=f(2y)+4=2f(y)+4$ and thus, we have $2f(f(y+2))=2(f(y)+2)$. Therefore, $f(f(y+2))=f(y)+2$. However, claim 11, instead says that $f(y+2)=f(y)+2$ and so, $f(f(y+2))=f(y+2)$ and now, we replace $y+2=x$, giving, $f(f(x))=f(x)$. Claim 7, now, finishes the argument by yielding $f(x)=x$ for all $x$ by injectivity. Now, claim 1 proves that out solutions are only $f(x)=x, \forall x$ and $f(x)=-x, \forall x$. These, as checked before, work.

Thus, A7, you are dead. Any questions? "No..." very well then, we are done, yay! :) [\/hide]
\end{solution}



\begin{solution}[by \href{https://artofproblemsolving.com/community/user/303386}{AlgebraFC}]
	[hide]
Let $P(x, y)$ denote the assertion that \[f(xf(x+y))=f(yf(x))+x^2.\] $P(0, y)$ gives $f(0)=f(yf(0))$. If $f(0)\neq 0$, then plugging in $y=\frac{x}{f(0)}$ implies $f(x)=f(0)$, so $f$ is constant. But clearly no constant functions satisfy the FE, so $f(0)=0$.

$P(x, 0)$ and $P(x, -x)$ imply $f(xf(x))=x^2$ and $f(-xf(x))=-x^2$, respectively. This implies $f$ is surjective, and in particular, if $f(a)=0$, then $a^2=f(af(a))=0\implies a=0$. 

For all real numbers $m$ such that $f(m)<0$, let $\sqrt{-f(m)}=b$. Then \[P\left(b, \frac{m}{f(b)}\right)\implies f\left(bf\left(b+\frac{m}{f(b)}\right)\right)=f\left(\frac{m}{f(b)}\cdot f(b)\right)-f(m)=0,\] so \[bf\left(b+\frac{m}{f(b)}\right)=0\implies b=-\frac{m}{f(b)}\implies -f(m)=f(bf(b))=f(-m),\] where we used the fact that $f(bf(b))=b^2=-f(m)$. 

Now, for all real numbers $m$ such that $f(m)>0$, let $\sqrt{f(m)}=b$ and $f(c)=\frac{m}{b}$. Then \[P(b, c-b)\implies f(m)=f(bf(c))=f((c-b)f(b))+f(m)\implies (c-b)f(b)=0\implies f(c)=f(b)\implies m=bf(b),\] so \[f(-m)=f(-bf(b))=-b^2=-f(m).\] It follows that $f$ is odd for all reals. 

Adding the equations for $P(x, y), P(y, -x-y), P(-x-y, x)$ and rearranging yields \[f(yf(x))=xy.\] From the surjectivity of $f$ there exists an $x$ such that $f(x)=1$; this implies that $f(y)=ky$ for constant $k$. Plugging this back in the original FE yields $k=\pm 1$, so the two solutions are $\boxed{f(x)=x, f(x)=-x}$.
[\/hide]
\end{solution}



\begin{solution}[by \href{https://artofproblemsolving.com/community/user/56076}{mathcool2009}]
	Proof of injectivity and main lemma:

Plug in $y = -2x$ to get \[f(xf(-x)) = f(-2xf(x)) + x^2. \] Note that $f(xf(-x)) = -(-x)^2 = -x^2$, so we have \[f(-2xf(x)) = -2x^2.\]

Suppose $f(a) = f(b)$. We plug in $x = a, y = b-a$, yielding \[f(af(b)) = f((b-a)f(a)) + a^2.\] Since $f(a) = f(b)$, we have $f(af(b)) = f(af(a)) = a^2,$ so we have $0 = f((b-a)f(a)).$ Thus \[0 = (b-a)f(a).\] If $f(a) = 0$, then $a=b=0$, and if $f(a) \not= 0$, then $b-a = 0$. In any case, we conclude that $a=b$. Thus $f$ is injective. 


Note that $f(xf(x))$ and $f(-xf(x))$ together span $\mathbb{R}$. Since $f$ is injective, we conclude that every real number can be expressed as either $xf(x)$ or $-xf(x)$. (This is because $f(u)$ must be able to be expressed as $f(xf(x))$ or $f(-xf(x))$ for some $x$.)

Plug in $y = z-x$ to get $f(xf(z)) = f((z-x)f(x)) + x^2$. Now negating $z$ yields $-f(xf(z)) = f((-z-x)f(x)) + x^2$, and adding these equations together yields \[0 = f((z-x)f(x)) + f((-z-x)f(x)) + 2x^2.\] We can express this as either \[f(-2xf(x)) = f((z-x)f(x)) + f((-z-x)f(x)) \qquad \qquad (\spadesuit)\] or \[f(2xf(x)) = f((-z+x)f(x)) + f((z+x)f(x)). \qquad \qquad (\diamondsuit)\] I claim that this implies that $f$ is Cauchy-additive. 

We wish to show that $f(a) + f(b) = f(a+b)$ for all $a,b$. If $a+b = 0$, the conclusion is obvious, so assume $a+b \not= 0$. Note that $\frac{a+b}{2}$ can be expressed as either $xf(x)$ or $-xf(x)$ for some $x$, which must be nonzero. If $\frac{a+b}{2} = xf(x)$, we can find $z$ such that \[(-z+x)f(x) = a.\] Using $(\diamondsuit)$, we conclude that $f(a+b) = f(a) + f(b)$, as desired. Similarly, if $\frac{a+b}{2} = -xf(x)$, we can find $z$ such that \[(z-x)f(x) = a.\] Using $(\spadesuit)$, we conclude that $f(a+b) = f(a) + f(b)$, as desired. Hence $f$ is Cauchy-additive.

\end{solution}



\begin{solution}[by \href{https://artofproblemsolving.com/community/user/408991}{mela_20-15}]
	Firstly do substitutions $x=0,y=0,y=-x$ so that you get the following three results:$f(0)=0 , f(xf(x))=x^2,  f(-xf(x))=-x^2$.
Now suppose that $f(k)=0$ for $k \not= 0$ ,we have $f(kf(k+y))=k^2$ , which for $y=-2k$ gives $f(kf(-k))=k^2=-k^2$ so $k=0$.
$Injectivity$:Suppose $f(a)=f(b) \not=0 , a \not=b$ we get from $x=a,y=b-a$ :$f((b-a)f(a))=0$ which is absurd.So $f$ is injective.
We have that $f(-xf(x))=f(xf(-x))=x^2$ so $f(-x)=-f(x)$
Now put $x=1$: $f(f(y+1))=f(y)+1 :(*)$ Next put $-x$ instead of $x$ and then subtract to get that $$f(xf(x+y))-f(xf(y-x))=2x^2$$
Put $x=2^{-1\/2}$ in order to use $(*)$ and injectivity to finally get $$f(\frac{f(x)}{\sqrt{2}}+1)=\frac{f(x+\sqrt{2})}{\sqrt{2}}$$
Put $x=-x-2\sqrt{2}$ and combine these two,using ijectivity, to get that $$f(x+2\sqrt{2})=f(x)+2\sqrt{2}$$
The problem is easily solved by setting $x=2^{3\/4}$ and using the last relation and injectivity to get that $2^{3\/4}f(y+2^{3\/4})=yf(2^{3\/4})+2^{3\/2}$
Hence $f$is linear .
\end{solution}
*******************************************************************************
-------------------------------------------------------------------------------

\begin{problem}[Posted by \href{https://artofproblemsolving.com/community/user/71129}{dap}]
	Find all functions $f: \mathbb R \to \mathbb R$ such that for all reals $x$ and $y$,
\[f(f(x)+y)=f(y-1) + xf(y-x-1) - f(x).\]
	\flushright \href{https://artofproblemsolving.com/community/c6h355806}{(Link to AoPS)}
\end{problem}



\begin{solution}[by \href{https://artofproblemsolving.com/community/user/29428}{pco}]
	\begin{tcolorbox}Find all the function :R to R such that:
f(f(x)+y)=f(y-1) + xf(y-x-1) - f(x)
Please solve it ! it's too hard .
I think f(x)=0 but I can't prove it\end{tcolorbox}
Let $P(x,y)$ be the assertion $f(f(x)+y)=f(y-1)+xf(y-x-1)-f(x)$

$P(x,x+1)$ $\implies$ $f(f(x)+x+1)=xf(0)$

If $f(0)\ne 0$, this implies that $f(x)$ is a surjection.
Let then $u$ such that $f(u)=-1$ : $P(u,x+u+1)$ $\implies$ $uf(x)=-1$ and so $u\ne 0$ and $f(x)=-\frac 1u$ which is not a solution (just plug back)
So $f(0)=0$

$P(0,x)$ $\implies$ $f(x)=f(x-1)$ and so $f(1)=f(0)=0$
$P(1,x)$ $\implies$ $f(x)=f(x-1)+f(x-2)$

Subtracting these two lines, we get $f(x-2)=0$ and so $\boxed{f(x)=0\text{   }\forall x}$ which indeed is a soluton.
\end{solution}



\begin{solution}[by \href{https://artofproblemsolving.com/community/user/82334}{bappa1971}]
	\begin{tcolorbox}[quote="dap"]Find all the function :R to R such that:
f(f(x)+y)=f(y-1) + xf(y-x-1) - f(x)
Please solve it ! it's too hard .
I think f(x)=0 but I can't prove it\end{tcolorbox}
Let $P(x,y)$ be the assertion $f(f(x)+y)=f(y-1)+xf(y-x-1)-f(x)$

$P(x,x+1)$ $\implies$ $f(f(x)+x+1)=xf(0)$

If $f(0)\ne 0$, this implies that $f(x)$ is a surjection.
Let then $u$ such that $f(u)=-1$ : $P(u,x+u+1)$ $\implies$ $uf(x)=-1$ and so $u\ne 0$ and $f(x)=-\frac 1u$ which is not a solution (just plug back)
So $f(0)=0$

$P(0,x)$ $\implies$ $f(x)=f(x-1)$ and so $f(1)=f(0)=0$
$P(1,x)$ $\implies$ $f(x)=f(x-1)+f(x-2)$

Subtracting these two lines, we get $f(x-2)=0$ and so $\boxed{f(x)=0\text{   }\forall x}$ which indeed is a soluton.\end{tcolorbox}

if $-1$ not belongs to the range of $f$, then?
 :?:
\end{solution}



\begin{solution}[by \href{https://artofproblemsolving.com/community/user/30342}{nicetry007}]
	Find all the function :$\mathbb{R}$ to $\mathbb{R}$ such that:
$f(f(x)+y)=f(y-1) + xf(y-x-1) - f(x)$
Set $y = f(y)$ in the functional equation.
$f(f(x)+f(y))=f(f(y)-1) + xf(f(y)-x-1) - f(x)$
$=f(-1 -1) + yf(-1-y-1) -f(y) + x[f(-x-1-1) + yf(-x-1-y-1) - f(y)] -f(x)$
$=f(-2) + yf(-y-2) + xf(-x-2) -f(y) -f(x) + xyf(-x-y-2) - xf(y)$
Since LHS is symmetric in $x$ and $y$, we get
$xf(y) = yf(x) \Rightarrow f(x) = cx$
Substituting $f(x) = cx$ in the original equation, we get $c = 0$.
Hence, $f\equiv 0$ is the only solution.
\end{solution}



\begin{solution}[by \href{https://artofproblemsolving.com/community/user/29428}{pco}]
	\begin{tcolorbox}[quote="pco"][quote="dap"]Find all the function :R to R such that:
f(f(x)+y)=f(y-1) + xf(y-x-1) - f(x)
Please solve it ! it's too hard .
I think f(x)=0 but I can't prove it\end{tcolorbox}
Let $P(x,y)$ be the assertion $f(f(x)+y)=f(y-1)+xf(y-x-1)-f(x)$

$P(x,x+1)$ $\implies$ $f(f(x)+x+1)=xf(0)$

If $f(0)\ne 0$, this implies that $f(x)$ is a surjection.
Let then $u$ such that $f(u)=-1$ : $P(u,x+u+1)$ $\implies$ $uf(x)=-1$ and so $u\ne 0$ and $f(x)=-\frac 1u$ which is not a solution (just plug back)
So $f(0)=0$

$P(0,x)$ $\implies$ $f(x)=f(x-1)$ and so $f(1)=f(0)=0$
$P(1,x)$ $\implies$ $f(x)=f(x-1)+f(x-2)$

Subtracting these two lines, we get $f(x-2)=0$ and so $\boxed{f(x)=0\text{   }\forall x}$ which indeed is a soluton.\end{tcolorbox}

if $-1$ not belongs to the range of $f$, then?
 :?:\end{tcolorbox}

I first proved that if $f(0)\ne 0$, $f(x)$ is a surjection (please read the line just above)
So in this case $-1$ is in the range of $f$ :)
\end{solution}



\begin{solution}[by \href{https://artofproblemsolving.com/community/user/66854}{Keehlzver}]
	If $f(f(x)+x+1)=xf(0)=0$ We still get $f$ is surjective?? 

What I understand is if we take $f(0)=a$ So above equation can be writen as $f(f(x)+x+1)=ax$ (Or in the form of m=f(b)=bk Which is obviously surjective) Every $x$ in its domain can map to every element in $f(f(x)+x+1)$, so If every element in its Range is equal to zero, we still get $f$ is surjective???
\end{solution}
*******************************************************************************
-------------------------------------------------------------------------------

\begin{problem}[Posted by \href{https://artofproblemsolving.com/community/user/85314}{mathmdmb}]
	Find all functions $f:\mathbb Z\to \mathbb Z$ such that $f(x+y+f(y))=f(x)+2y$ for all integers $x$ and $y$.
	\flushright \href{https://artofproblemsolving.com/community/c6h356042}{(Link to AoPS)}
\end{problem}



\begin{solution}[by \href{https://artofproblemsolving.com/community/user/29428}{pco}]
	\begin{tcolorbox}Find all functions $f:\mathbb Z\to \mathbb Z$such that $f(x+y+f(y))=f(x)+2y$\end{tcolorbox}
Let $P(x,y)$ be the assertion $f(x+y+f(y))=f(x)+2y$

$P(-f(x),x)$ $\implies$ $f(x)=f(-f(x))+2x$ and so $f(x)$ is injective.

$P(x,y+z)$ $\implies$ $f(x+y+z+f(y+z))=f(x)+2y+2z$
$P(x+y+f(y),z)$ $\implies$ $f(x+y+f(y)+z+f(z))=f(x+y+f(y))+2z=f(x)+2y+2z$

So $f(x+y+z+f(y+z))=f(x+y+f(y)+z+f(z))$ and, since injective, $x+y+z+f(y+z)=x+y+f(y)+z+f(z)$ and so $f(x+y)=f(x)+f(y)$

So $f(x)=xf(1)$ and, plugging back in the original equation, we get two solutions :
$f(x)=x$ $\forall x$
$f(x)=-2x$ $\forall x$
\end{solution}



\begin{solution}[by \href{https://artofproblemsolving.com/community/user/47085}{micf}]
	My solution is more messy but I hope it's correct.

Let's define $P(x,y)$ as \begin{bolded}pco\end{bolded} did. 

$P(x,0) \Longrightarrow f(x+f(0))=f(x)$
$P(-x,x) \Longrightarrow f(f(x))=f(-x)+2x \; \; \; (1)$
From that we deduce that
$2x+f(-x)=f(f(x))=f(f(x+f(0)))=f(-x-f(0))+2x+2f(0)=f(-x)+2x+2f(0)$ so $f(0)=0$
Let $a$ s.t. $f(a)=0$ 
$ P(0,a) \Longrightarrow a=0 $ 
therefore $\boxed{x=0 \Leftrightarrow f(x)=0}$
$P(-x+f(-x),x) \Longrightarrow f(-x)=-f(x)  \Longrightarrow (1) \Longrightarrow f(f(x))=2x-f(x)$
$P(x, f(1)) \Longrightarrow f(x+2)=f(x)+2f(1)$
Therefore $f(x)=xf(1)$
\end{solution}
*******************************************************************************
-------------------------------------------------------------------------------

\begin{problem}[Posted by \href{https://artofproblemsolving.com/community/user/86849}{abch42}]
	Find all functions $f:\mathbb{R}\rightarrow\mathbb{R}$ such that $f(1)=1$ and
\[f\left(f(x)y+\frac{x}{y}\right)=xyf\left(x^2+y^2\right)\]
for all real numbers $x$ and $y$ with $y\neq0$.
	\flushright \href{https://artofproblemsolving.com/community/c6h358223}{(Link to AoPS)}
\end{problem}



\begin{solution}[by \href{https://artofproblemsolving.com/community/user/29428}{pco}]
	\begin{tcolorbox}Find all functions $f:\mathbb{R}\rightarrow\mathbb{R}$ such that $f(1)=1$ and
$f\left(f(x)y+\frac{x}{y}\right)=xyf\left(x^2+y^2\right)$
for all real numbers $x$ and $y$ with $y\neq0$.\end{tcolorbox}

I spent a long time on this equation without solving it. And nobody proposed some solution till now.

Could you, please, confirm us that there are no extra "forgotten" constraint ?

With continuity at $1$, for example, or with restriction $f(x)$ from $\mathbb R^+\to\mathbb R^+$, it would be very easy to show that $f(x)=\frac 1x$.
\end{solution}



\begin{solution}[by \href{https://artofproblemsolving.com/community/user/86849}{abch42}]
	\begin{tcolorbox}I spent a long time on this equation without solving it. And nobody proposed some solution till now.

Could you, please, confirm us that there are no extra "forgotten" constraint ?

With continuity at $1$, for example, or with restriction $f(x)$ from $\mathbb R^+\to\mathbb R^+$, it would be very easy to show that $f(x)=\frac 1x$.\end{tcolorbox}
I also can't solving it. It is MOSP2007.
[url]http://amc.maa.org\/a-activities\/a6-mosp\/a6-1-mosparchives\/2007-ma\/mosptests.pdf[\/url]
\end{solution}



\begin{solution}[by \href{https://artofproblemsolving.com/community/user/19718}{lordWings}]
	Plugging $x=0$ in the functional equation we reach $f(f(0)y)=0$ for every $y\neq0$.
If $f(0)\neq0$, then $f(t)=0$ for every $t\neq0$. This is indeed one trivial family of solutions.
From now on, let's focus on the possible solutions with $f(0)=0$.

Plugging $y=\frac{1}{x}$ in the functional equation we reach $f\left(\frac{f(x)}{x}+x^2\right)=f\left(x^2+\frac{1}{x^2}\right)$.
The more obvious solutions are:
- Constant $f$ (and the only constant that is a solution to the original equation is $0$, another trivial solution).
- Injective $f$ (and then $\frac{f(x)}{x}+x^2=x^2+\frac{1}{x^2}$, so $f(x)=\frac{1}{x}$  for every $x\neq0$, which is indeed another solution).
But there still might be other solutions.

Let's study the couples $(x,y)$, with $x,y\neq0$, which satisfy the equation $f(x)y+\frac{x}{y}=x^2+y^2$.
Since this is equivalent to a 3rd degree polynomic equation in $y$, it has at least one real solution (no matter the values of $x$ and $f(x)$), so there's at least one valid couple $(x,y)$ for every $x$.
But, if there is such a couple, then plugging it in the functional equation we reach one of these two cases:
Case a) $f\left(f(x)y+\frac{x}{y}\right)=f(x^2+y^2)=0$.
Case b) $f\left(f(x)y+\frac{x}{y}\right)=f(x^2+y^2)\neq0$, then $xy=1$, so there's only one valid couple $(x,y)$ for every $x$, with $y=\frac{1}{x}$.  If we plug it back in the equation $f(x)y+\frac{x}{y}=x^2+y^2$, we reach $f(x)=\frac{1}{x}$.

So it remains to find out if there are solutions with $f(x)=0$ in a subset (not just for $x=0$ neither for all real $x$), and $f(x)=\frac{1}{x}$ everywhere else. I'll try to sort it out in the next days.
\end{solution}
*******************************************************************************
-------------------------------------------------------------------------------

\begin{problem}[Posted by \href{https://artofproblemsolving.com/community/user/67223}{Amir Hossein}]
	Find all functions $f : \mathbb N \to \mathbb N$ such that:

i) $f^{2000}(m)=f(m)$ for all $m \in \mathbb N$,

ii) $f(mn)=\dfrac{f(m)f(n)}{f(\gcd(m,n))}$, for all $m,n\in \mathbb N$, and

iii) $f(m)=1$ if and only if $m=1$.
	\flushright \href{https://artofproblemsolving.com/community/c6h359200}{(Link to AoPS)}
\end{problem}



\begin{solution}[by \href{https://artofproblemsolving.com/community/user/64868}{mahanmath}]
	:D 
$f(4) = \frac{f(2)f(2)}{f(2)} =f(2)$. So , $2 = f^{2000}(2) = f^{1999}(f(2)) = f^{1999}(f(4)) = f^{2000}(4)=4$
\end{solution}



\begin{solution}[by \href{https://artofproblemsolving.com/community/user/85314}{mathmdmb}]
	Well,my solution is the same but I found this at $Iran,2001$ may be
\end{solution}



\begin{solution}[by \href{https://artofproblemsolving.com/community/user/79198}{SKhan}]
	\begin{tcolorbox}:D 
$f(4) = \frac{f(2)f(2)}{f(2)} =f(2)$. So , $2 = f^{2000}(2) = f^{1999}(f(2)) = f^{1999}(f(4)) = f^{2000}(4)=4$\end{tcolorbox}
I think \begin{bolded}mahanmath\end{bolded} and \begin{bolded}mathmdmb\end{bolded} misinterpreted the first condition as being $f^{2000}(m)=m, \forall m\in\mathbb{N}$.

Here are some simple results:
$f$ is multiplicative.
$f(m^n)=f(m)$
$f\left(\prod_{i}p_{i}^{a_{i}}\right)=\prod_{i}f(p_{i})$ where $a_{i}\in\mathbb{N}$ and $p_{i}$ is a prime $i\in\{1,2,\cdots, k\}$
\end{solution}



\begin{solution}[by \href{https://artofproblemsolving.com/community/user/85314}{mathmdmb}]
	But I don't think so at all.Have you  any counter-logic?
\end{solution}



\begin{solution}[by \href{https://artofproblemsolving.com/community/user/79198}{SKhan}]
	It seems as though you and mahanmath think that there are no functions $f$.
However, I believe that there are infinitely many such functions.

Let $g$ be a function that maps the prime numbers to natural numbers.
Denote $p_{i}$ as being the $i^{th}$ prime number.
Let $h$ be a function that maps $0$ to $0$ and the natural numbers to $1$
Now let $f\left(\prod_{i\in\mathbb{N}} p_{i}^{a_{i}}\right)=\prod_{i\in\mathbb{N}}p_{i}^{g(p_i)h(a_i)}$ , where each $a_{i}$ is a non-negative integer.

It is easy to check that these functions $f$ satisfy all the three given conditions.
The difficult part is to show that these functions are the only functions satisying all the three given conditions or whether there are other functions.
\end{solution}



\begin{solution}[by \href{https://artofproblemsolving.com/community/user/85314}{mathmdmb}]
	Explain more about its satisfying these $3$ conditions.I don't think that it is easy to check this.
\end{solution}



\begin{solution}[by \href{https://artofproblemsolving.com/community/user/79198}{SKhan}]
	Actually I have found a more general function.
Let $\mathbb{A}$ and $\mathbb{B}_{i}$ be non-intersecting sets whose union is $\mathbb{N}$ and for each $i$ we have $|\mathbb{B}_{i}|=1999$
Define the function $G$ as the following:
If $x\in\mathbb{A}$ then $G(x)=x$ and $G(\mathbb{B}_{i})$ is a permutation of $\mathbb{B}_{i}$ such that there are no cycles smaller than $1999$.
Define the function $H$ as the following:
$H(0)=0$ and if $x\in\mathbb{N}$ then $H(x)=1$
Define the function $I$ as mapping the prime numbers to natural numbers.
Denote $p_{i}$ as being the $i^{th}$ prime number.
Define the function $f$ as the following:
$f\left(\prod_{i} p_{i}^{a_{i}}\right)=\prod_{i} p_{G(i)}^{H(a_{i})I(p_{i})}$
$f$ satisfies the first condition
[hide="PROOF"]
Let $m=\prod_{i} p_{i}^{a_{i}}$ where $a_i$ are non-negative integers.
From the definition of $G$ we get $G^{1999}(i)=i$
It is easy to get $H(H(a)b)=H(a)$ where $a\in\mathbb{N}\cup\{0\}$ and $b\in\mathbb{N}$
$f^k\left(\prod_{i} p_{G(i)}^{H(a_{i})I(p_{i})}\right)=f^{k-1}\left(\prod_{i} p_{G^2(i)}^{H(H(a_{i})I(p_{i}))I(p_{G(i)})}\right)$
Now $H(a_{i})\in\mathbb{N}\cup\{0\}$ and $I(p_{i})\in\mathbb{N}$ 
So $f^{k}\left(\prod_{i} p_{G(i)}^{H(a_{i})I(p_{i})}\right)=f^{k-1}\left(\prod_{i} p_{G^2(i)}^{H(a_{i})I(p_{G(i)})}\right)$
Continuing this we get $f^{k}\left(\prod_{i} p_{G(i)}^{H(a_{i})I(p_{i})}\right)=\prod_{i} p_{G^{k+1}(i)}^{H(a_{i})I(p_{G^k(i)})}$
So $f^{1999}\left(\prod_{i} p_{G(i)}^{H(a_{i})I(p_{i})}\right)=\prod_{i} p_{G^{2000}(i)}^{H(a_{i})I(p_{G^{1999}(i)})}$
So $f^{2000}\left(\prod_{i} p_{i}^{a_{i}}\right)=\prod_{i} p_{G(i)}^{H(a_{i})I(p_{i})}=f\left(\prod_{i} p_{i}^{a_{i}}\right)$
So $f^{2000}(m)=f(m)$ [\/hide]
$f$ satisfies the second condition
[hide="PROOF"]
Let $m=\prod_{i} p_{i}^{a_{i}}$ and $n=\prod_{i} p_{i}^{b_{i}}$ where $a_i$ and $b_i$ are non-negative integers.
Then $mn=\prod_{i} p_{i}^{a_{i}+b_{i}}$ and $gcd(m, n)=\prod_{i} p_{i}^{Min(a_{i}, b_{i})}$
It is easy to get $H(Max(a_{i}, b_{i}))=H(a_i+b_i)$
So $H(a_{i}+b_{i})+H(Min(a_{i}, b_{i}))=H(Max(a_{i}, b_{i}))+H(Min(a_{i}, b_{i}))=H(a_i)+H(b_i)$
So $f(mn)f(gcd(m, n))=\prod_{i} p_{G(i)}^{[H(a_i+b_i)+H(Min(a_{i}, b_{i}))]I(p_{i})}=\prod_{i} p_{G(i)}^{[H(a_i)+H(b_i)]I(p_{i})}$
So $f(mn)f(gcd(m, n))=\prod_{i} p_{G(i)}^{[H(a_i)+H(b_i)]I(p_{i})}=f(m)f(n)$ [\/hide]
$f$ satisfies the third condition
[hide="PROOF"]
Let $m=\prod_{i} p_{i}^{a_i}$ where $a_i$ are non-negative integers.
$f(m)=1\iff H(a_i)I(p_i)=0, \forall i\in\mathbb{N}$
$I(p_i)>0, \forall i\in\mathbb{N}$ so $H(a_i)I(p_i)=0, \forall i\in\mathbb{N}\iff H(a_i)=0, \forall i\in\mathbb{N}$
$H(a_i)=0, \forall i\in\mathbb{N}\iff a_i=0, \forall i\in\mathbb{N}\iff m=1$
So $f(m)=1\iff m=1$ [\/hide]

\begin{tcolorbox}Explain more about its satisfying these $3$ conditions.I don't think that it is easy to check this.\end{tcolorbox}
The functions $f$ that I mentioned in this post, also contains the functions $f$ that I mentioned in my previous post.
So those functions $f$ that I mentioned in my previous post also satisfy all the three given conditions.

The more difficult part is to show that these are the only functions that $f$ could take or whether there are other forms.

Has anyone got any ideas?
\end{solution}



\begin{solution}[by \href{https://artofproblemsolving.com/community/user/85314}{mathmdmb}]
	But I am not sure at all whether your functions can avoid this contradiction at all.All function having this property must satisfy this.
\begin{tcolorbox}:D 
$f(4) = \frac{f(2)f(2)}{f(2)} =f(2)$. So , $2 = f^{2000}(2) = f^{1999}(f(2)) = f^{1999}(f(4)) = f^{2000}(4)=4$\end{tcolorbox}
I don't understand how do you find these functions?
\end{solution}



\begin{solution}[by \href{https://artofproblemsolving.com/community/user/79198}{SKhan}]
	$f$ is multiplicative.
[hide="PROOF"]
When $m$ and $n$ are coprime $gcd(m , n)=1$ so by condition 2 and 3: $f(mn)=f(m)f(n)$ [\/hide]

$f(m^n)=f(m), \forall m,n\in\mathbb{N}$
[hide="PROOF"]
Base Case: $m=1$ (trivial)
Inductive Step: Assume that $f(m^k)=f(m)$ where $k\in\mathbb{N}$
$n=m^k$ in condition 2: $f(m^{k+1})=\frac{f(m^k)f(m)}{f(m)}=f(m^k)=f(m)$
This completes the inductive step so, by induction, $f(m^n)=f(m), \forall m,n\in\mathbb{N}$ [\/hide]

LEMMA 1: $f\left(\prod_{i=1}^{k}p_{i}^{a_i}\right)=\prod_{i=1}^{k}f(p_i)$ where $a_i\in\mathbb{N}$ and $p_i$ is a different prime number for each $i\in\{1, \cdots , k\}$
[hide="PROOF"]
$f$ is multiplicative so $f\left(\prod_{i=1}^{k}p_{i}^{a_i}\right)=\prod_{i=1}^{k}f(p_i^{a_i})$ where $a_i\in\mathbb{N}$ and $p_i$ is a different prime number for each $i\in\{1, \cdots , k\}$
But $f(m^n)=f(m), \forall m,n\in\mathbb{N}$ so $\prod_{i=1}^{k}f(p_i^{a_i})=\prod_{i=1}^{k}f(p_i)$
So $f\left(\prod_{i=1}^{k}p_{i}^{a_i}\right)=\prod_{i=1}^{k}f(p_i)$ where $a_i\in\mathbb{N}$ and $p_i$ is a different prime number for each $i\in\{1, \cdots , k\}$ [\/hide]

LEMMA 2: Suppose that $f(x)=f(p)$ where $p$ is a prime number, then $x$ is a power of $p$.
[hide="PROOF"]
If $q|x$ and $q$ is a prime number not equal to $p$.
Then by LEMMA 1: $f(q)|f(x)$ so $f(q)|f(p)$
$f(q)|f(p)\implies gcd(f(p),f(q))=f(q)$
$m=f(p)$ and $n=f(q)$ in condition two: $f(f(p)f(q))=\frac{f(f(p))f(f(q))}{f(f(q))}=f(f(p))$
$f$ is multiplicative (and after applying condition 1 twice), we get $f(p)f(q)=f(pq)=f^{2000}(pq)=f^{1999}(f(p)f(q))=f^{2000}(p)=f(p)$
$f(p)f(q)=f(p)\implies f(q)=1\implies q=1$ by condition 3, CONTRADICTION.
So if $q|x$ and $q$ is a prime then $q=p$
Hence if $f(x)=f(p)$ where $p$ is a prime number, then $x$ is a power of $p$. [\/hide]

Now $f(f^{1999}(p))=f(p)$ so by LEMMA 2, $f^{1999}(p)$ is a power of $p$.

LEMMA 3: Suppose that $f(x)$ is a power of a prime then so is $x$.
[hide="PROOF"]
Let $f(x)=p^a$ and $x=\prod_{i=1}^{k}q_{i}^{a_{i}}$ where $\forall i\in\{1, \cdots, k\}$, $q_i$ is a prime number and $a_i\in\mathbb{N}$
By LEMMA 1, we get $\prod_{i=1}^{k}f(q_i)=f(x)=p^a$
So $f(q_i)$ is a power of $p$, $\forall i\in\{1, \cdots, k\}$
Let $f(q_i)=p^{b_i}$ , where $b_i\in\mathbb{N}$ $\forall i\in\{1, \cdots, k\}$ 
If $k>1$, then $f(q_1)$ and $f(q_2)$ are power of $p$
$b_1\leq b_2$ or $b_2\leq b_1$ so $f(q_1)|f(q_2)$ or $f(q_2)|f(q_1)$.
W.L.O.G. we can assume that $f(q_1)|f(q_2)$ so $gcd(f(q_1), f(q_2))=f(q_1)$
$m=f(q_1)$ and $n=f(q_2)$ in condition two: $f(f(q_1)f(q_2))=\frac{f(f(q_1))f(f(q_2))}{f(f(q_1))}=f(f(q_2))$
$f$ is multiplicative (and after applying condition 1 twice), we get $f(q_1)f(q_2)=f(q_1q_2)=f^{2000}(q_1q_2)=f^{1999}(f(q_1)f(q_2))=f^{2000}(q_2)=f(q_2)$
$f(q_1)f(q_2)=f(q_2)\implies f(q_1)=1\implies q_1=1$ CONTRADICTION.
So $k=1\implies x$ is a power of a prime.
So if $f(x)$ is a power of a prime then so is $x$ [\/hide]

By applying LEMMA 3 on $f^{1999}(p)$ which is a power of a prime, (1998 times) we get that $f(p)$ is a power of a prime $\forall $ prime numbers $p$
Denote the $i^{th}$ prime number by $p_i$
So there must exist functions $G$ and $I$ such that $f(p_{i})=p_{G(i)}^{I(p_i)}$ because $f(p)$ is a power of a prime $\forall $ prime numbers $p$ (the prime number is dependant on $p$ which is dependant on $i$ where $p$ is the $i^{th}$ prime number, and the power of the prime is dependant on the the prime $p$)
$f(m^n)=f(m)$ where $m$ and $n$ are natural numbers but if $n=0$ then $f(m^n)=f(1)$
If we now define the function $H$ which maps $0$ to $0$ and the natural numbers to $1$
$f(m^n)=f(m^{H(n)})$
Then $f(p_{i}^{a_i})=p_{G(i)}^{H(a_i)I(p_i)}$
By LEMMA 1, we get $f\left(\prod_{i=1}p_{i}^{a_i}\right)=\prod_{i=1}p_{G(i)}^{H(a_i)I(p_i)}$
Also $f^{1999}(p_i)=p_{G^{1999}(i)}^{I(p_{G^{1999}(i)})}$ is a power of $p_i, \forall i\in\mathbb{N}$
So $G^{1999}(i)=i, \forall i\in\mathbb{N}$
$1999$  is a prime number so split $\mathbb{N}$ into sets ($\mathbb{B}_{i}$'s) containing $1999$ elements and the rest of the element be part of the set $\mathbb{A}$
Then the general solution of $G^{1999}(i)=i, \forall i\in\mathbb{N}$ is the following.
If $x\in\mathbb{A}$ then $G(i)=i$ and $G(\mathbb{B}_{i})$ is a permutation of $\mathbb{B}_{i}$ which has no cycles less than $1999$.

So the only functions satisfying all the three give conditions are those that I noted in my previous post.
\begin{tcolorbox}Actually I have found a more general function.
Let $\mathbb{A}$ and $\mathbb{B}_{i}$ be non-intersecting sets whose union is $\mathbb{N}$ and for each $i$ we have $|\mathbb{B}_{i}|=1999$
Define the function $G$ as the following:
If $x\in\mathbb{A}$ then $G(x)=x$ and $G(\mathbb{B}_{i})$ is a permutation of $\mathbb{B}_{i}$ such that there are no cycles smaller than $1999$.
Define the function $H$ as the following:
$H(0)=0$ and if $x\in\mathbb{N}$ then $H(x)=1$
Define the function $I$ as mapping the prime numbers to natural numbers.
Denote $p_{i}$ as being the $i^{th}$ prime number.
Define the function $f$ as the following:
$f\left(\prod_{i} p_{i}^{a_{i}}\right)=\prod_{i} p_{G(i)}^{H(a_{i})I(p_{i})}$
$f$ satisfies the first condition
[hide="PROOF"]
Let $m=\prod_{i} p_{i}^{a_{i}}$ where $a_i$ are non-negative integers.
From the definition of $G$ we get $G^{1999}(i)=i$
It is easy to get $H(H(a)b)=H(a)$ where $a\in\mathbb{N}\cup\{0\}$ and $b\in\mathbb{N}$
$f^k\left(\prod_{i} p_{G(i)}^{H(a_{i})I(p_{i})}\right)=f^{k-1}\left(\prod_{i} p_{G^2(i)}^{H(H(a_{i})I(p_{i}))I(p_{G(i)})}\right)$
Now $H(a_{i})\in\mathbb{N}\cup\{0\}$ and $I(p_{i})\in\mathbb{N}$ 
So $f^{k}\left(\prod_{i} p_{G(i)}^{H(a_{i})I(p_{i})}\right)=f^{k-1}\left(\prod_{i} p_{G^2(i)}^{H(a_{i})I(p_{G(i)})}\right)$
Continuing this we get $f^{k}\left(\prod_{i} p_{G(i)}^{H(a_{i})I(p_{i})}\right)=\prod_{i} p_{G^{k+1}(i)}^{H(a_{i})I(p_{G^k(i)})}$
So $f^{1999}\left(\prod_{i} p_{G(i)}^{H(a_{i})I(p_{i})}\right)=\prod_{i} p_{G^{2000}(i)}^{H(a_{i})I(p_{G^{1999}(i)})}$
So $f^{2000}\left(\prod_{i} p_{i}^{a_{i}}\right)=\prod_{i} p_{G(i)}^{H(a_{i})I(p_{i})}=f\left(\prod_{i} p_{i}^{a_{i}}\right)$
So $f^{2000}(m)=f(m)$ [\/hide]
$f$ satisfies the second condition
[hide="PROOF"]
Let $m=\prod_{i} p_{i}^{a_{i}}$ and $n=\prod_{i} p_{i}^{b_{i}}$ where $a_i$ and $b_i$ are non-negative integers.
Then $mn=\prod_{i} p_{i}^{a_{i}+b_{i}}$ and $gcd(m, n)=\prod_{i} p_{i}^{Min(a_{i}, b_{i})}$
It is easy to get $H(Max(a_{i}, b_{i}))=H(a_i+b_i)$
So $H(a_{i}+b_{i})+H(Min(a_{i}, b_{i}))=H(Max(a_{i}, b_{i}))+H(Min(a_{i}, b_{i}))=H(a_i)+H(b_i)$
So $f(mn)f(gcd(m, n))=\prod_{i} p_{G(i)}^{[H(a_i+b_i)+H(Min(a_{i}, b_{i}))]I(p_{i})}=\prod_{i} p_{G(i)}^{[H(a_i)+H(b_i)]I(p_{i})}$
So $f(mn)f(gcd(m, n))=\prod_{i} p_{G(i)}^{[H(a_i)+H(b_i)]I(p_{i})}=f(m)f(n)$ [\/hide]
$f$ satisfies the third condition
[hide="PROOF"]
Let $m=\prod_{i} p_{i}^{a_i}$ where $a_i$ are non-negative integers.
$f(m)=1\iff H(a_i)I(p_i)=0, \forall i\in\mathbb{N}$
$I(p_i)>0, \forall i\in\mathbb{N}$ so $H(a_i)I(p_i)=0, \forall i\in\mathbb{N}\iff H(a_i)=0, \forall i\in\mathbb{N}$
$H(a_i)=0, \forall i\in\mathbb{N}\iff a_i=0, \forall i\in\mathbb{N}\iff m=1$
So $f(m)=1\iff m=1$ [\/hide]

\begin{tcolorbox}Explain more about its satisfying these $3$ conditions.I don't think that it is easy to check this.\end{tcolorbox}
The functions $f$ that I mentioned in this post, also contains the functions $f$ that I mentioned in my previous post.
So those functions $f$ that I mentioned in my previous post also satisfy all the three given conditions.

The more difficult part is to show that these are the only functions that $f$ could take or whether there are other forms.

Has anyone got any ideas?\end{tcolorbox}
\end{solution}



\begin{solution}[by \href{https://artofproblemsolving.com/community/user/79198}{SKhan}]
	\begin{tcolorbox}But I am not sure at all whether your functions can avoid this contradiction at all.All function having this property must satisfy this.
\begin{tcolorbox}:D 
$f(4) = \frac{f(2)f(2)}{f(2)} =f(2)$. So , $2 = f^{2000}(2) = f^{1999}(f(2)) = f^{1999}(f(4)) = f^{2000}(4)=4$\end{tcolorbox}
I don't understand how do you find these functions?\end{tcolorbox}
1)As I had already mentioned earlier, mahanmath had misinterpreted the first condition, so the the functions $f$ do not necessary satisfy what mahanmath had said in the above quotation.
2)You said how can you find functions $f$ such that $2=4$ and since $2\neq 4$ no such functions $f$ would exist. However I had previously found functions $f$ which satisfied all the conditions that amparvardi had given, which shows that mahanmath had made a mistake.
I am not sure why you do not understand this, I hope you understand this now.
\end{solution}



\begin{solution}[by \href{https://artofproblemsolving.com/community/user/85314}{mathmdmb}]
	Well done dear Skhan.But I think it would be a bad news for you I found this is the official solution.So I can ofcourse say its true.So the problem must be in your work,whatever that is.So I and Mahanmath aren't wrong at all and didn't misinterpret anything here. :)
\end{solution}



\begin{solution}[by \href{https://artofproblemsolving.com/community/user/29428}{pco}]
	\begin{tcolorbox}Well done dear Skhan.But I think it would be a bad news for you I found this is the official solution.So I can ofcourse say its true.So the problem must be in your work,whatever that is.So I and Mahanmath aren't wrong at all and didn't misinterpret anything here. :)\end{tcolorbox}
Thanks for giving us the official solution.

\begin{tcolorbox}:D 
$f(4) = \frac{f(2)f(2)}{f(2)} =f(2)$. So , $2 = f^{2000}(2) = f^{1999}(f(2)) = f^{1999}(f(4)) = f^{2000}(4)=4$\end{tcolorbox}

Could you just explain me why " $2 = f^{2000}(2) $"
I dont understand at all.

Thanks for answering.
\end{solution}



\begin{solution}[by \href{https://artofproblemsolving.com/community/user/29428}{pco}]
	Just received this thru pm (thakns, mahanmath, for your private answer; I think it should be a public one) :
\begin{tcolorbox}[url=http://www.artofproblemsolving.com/Forum/viewtopic.php?p=1973851#p1973851]Subject: Find all N-to-N functions-Iran 3rd round-Number Theory 2007[\/url]

\begin{tcolorbox}

\begin{tcolorbox}:D 
$f(4) = \frac{f(2)f(2)}{f(2)} =f(2)$. So , $2 = f^{2000}(2) = f^{1999}(f(2)) = f^{1999}(f(4)) = f^{2000}(4)=4$\end{tcolorbox}

Could you just explain me why " $2 = f^{2000}(2) $"
I dont understand at all.

Thanks for answering.\end{tcolorbox}

I had seen a very similar problem many times ago which has a little (just a little)  :P   difference with this problem .
It said $m = f^{2000}(m)$ .

I`m so sorry for my mistake. :oops:\end{tcolorbox}

So, As Skhan already mentioned earlier, mahanmath had misinterpreted the first condition, so the the functions $f$ do not necessary satisfy what mahanmath wrote.

And so, when mathmdmb wrote  "Well done dear Skhan.But I think it would be a bad news for you I found this is the official solution.So I can ofcourse say its true.So the problem must be in your work,whatever that is.So I and Mahanmath aren't wrong at all and didn't misinterpret anything here. :)", he was laughing Skhan and he should better have checked the problem instead.

Congrats, Skhan.
\end{solution}



\begin{solution}[by \href{https://artofproblemsolving.com/community/user/85314}{mathmdmb}]
	Thank you pco but you misunderstood that I laughed at SKHAN.Not at all,rather Amparvardi,Mahanmath,SKhan & I are like freinds here.We try to reply to posts among ourselves.I also congratulated him.
\end{solution}



\begin{solution}[by \href{https://artofproblemsolving.com/community/user/79198}{SKhan}]
	\begin{tcolorbox}I found this is the official solution.\end{tcolorbox} :o  I didn't realise this was the official solution, well thanks for letting me know. Well I was just trying to find some stuff which seemed useful (I first conjectured that those lemmas were true and then tried to prove them) and it eventually led to a solution.
Also thanks mathmdmb and pco for congratulating.
\end{solution}
*******************************************************************************
-------------------------------------------------------------------------------

\begin{problem}[Posted by \href{https://artofproblemsolving.com/community/user/71129}{dap}]
	Find all functions $f: \mathbb R \to \mathbb R$ such that for all reals $x$ and $y$,
\[f(x^{2}+f(y)-y)=f(x)^{2}.\]
	\flushright \href{https://artofproblemsolving.com/community/c6h359497}{(Link to AoPS)}
\end{problem}



\begin{solution}[by \href{https://artofproblemsolving.com/community/user/29428}{pco}]
	\begin{tcolorbox}Find all function: $R \to R$ sastified :
$f(x^{2}+f(y)-y)=f(x)^{2}$\end{tcolorbox}
Let $P(x,y)$ be the assertion $f(x^2+f(y)-y)=f(x)^2$ $\forall x,y\in\mathbb R$
Let $a=f(0)$
Let $A=\{f(x)-x \forall x\in\mathbb R\}$
Let $Q(x,y)$ the equivalent assertion $f(x^2+y)=f(x)^2$ $\forall x\in\mathbb R, \forall y\in A$

$a\in A$

1) If A contains only $a$ :
=================
$f(x)-x=a$ $\forall x$ and, plugging this in the original equation, we get $a=0$ and the solution $f(x)=x$

2) If A contains at least two elements
==========================
Comparing $Q(x,u)$ and $Q(x,v)$, we get $f(x^2+u)=f(x^2+v)$ $\forall x\in\mathbb R, \forall u,v\in A$
This may be written : $f(x+u-v)=f(x)$ $\forall x\ge v, \forall u\ge v\in A$


2.1) A  contains infinitely many elements as negative as we want
---------------------------------------------------------------------------------
Let $u>v\in A$ and $\Delta=u-v>0$. We got previously $f(x+\Delta)=f(x)$ $\forall x\ge v$
So $f(x+n\Delta)=f(x)$ $\forall x\ge v$
So $f(x+n\Delta)-(x+n\Delta)=(f(x)-x)-n\Delta$
Setting $n$ as great as we want gives the required result.

2.2) $f(x)\ge 0$ $\forall x$ and $f(-x)=f(x)$
---------------------------------------------------
Let $x\in\mathbb R$
According to 2.1 above, $\exists u\in A$ such that $u<x$

Then $Q(\sqrt{x-u},u)$ $\implies$ $f(x)=f(\sqrt{x-u})^2\ge 0$ which is the first part of our result

Comparing $P(x,y)$ and $P(-x,y)$, we get $f(x)^2=f(-x)^2$ and so $f(-x)=\pm f(x)$ and, since $f(x)\ge 0$ $\forall x$ : $f(-x)=f(x)$
Q.E.D.

2.3) $f(x+u-v)=f(x)$ $\forall x\in\mathbb R$, $\forall 0\ge u\ge v\in A$
-------------------------------------------------------------------------------
Let $u\ge v\in A$. We already know that $f(x+u-v)=f(x)$ $\forall x\ge v$

Since $f(-x)=f(x)$, we get $f(-x-u+v)=f(-x)$ $\forall x\ge v$
$\iff$ $f(x-u+v)=f(x)$ $\forall x\le -v$
$\iff$ $f(x)=f(x+u-v)$ $\forall x+u-v\le -v$
$\iff$ $f(x+u-v)=f(x)$ $\forall x\le -u$

So we got :
$f(x+u-v)=f(x)$ $\forall x\ge v$ $\implies$ $f(x+u-v)=f(x)$  $\forall x\ge 0$
$f(x+u-v)=f(x)$ $\forall x\le -u$ $\implies$ $f(x+u-v)=f(x)$  $\forall x\le 0$

Hence the result.

2.4) $f(x)$ is contant and so either $f(x)=0$ $\forall x$, either $f(x)=1$ $\forall x$
------------------------------------------------------------------------------------------------
Let $c>0$
Let $-\frac c2>u>v\in A$ and $\Delta=u-v>0$
We already saw that $f(x+n\Delta)-(x+n\Delta)=(f(x)-x)-n\Delta$ $\forall x\ge v$

Using $x=\frac c2$ in this equation, we get $f(\frac c2+n\Delta)-(\frac c2+n\Delta)=(f(\frac c2)-\frac c2)-n\Delta$
Using $x=-\frac c2$ in this equation, we get $f(-\frac c2+n\Delta)-(-\frac c2+n\Delta)=(f(\frac c2)+\frac c2)-n\Delta$

Choose then $w=(f(\frac c2)+\frac c2)-n\Delta$ and $t=(f(\frac c2)-\frac c2)-n\Delta$

$w,t\in A$
Choosing $n$ great enough, we get $w,t<0$
$w-t=c>0$ and so we have $0>w>t$

Applying then 2.3 above, we get $f(x+c)=f(x)$ $\forall x\in\mathbb R$, $\forall c>0$
And so $f(x)$ is a constant.
Plugging this in the original equation, we get that the constant can only be $0$ or $1$, and both fit.

3) Synthesis of solutions
=================
We got three solutions :
$f(x)=0$ $\forall x$
$f(x)=1$ $\forall x$
$f(x)=x$ $\forall x$
\end{solution}



\begin{solution}[by \href{https://artofproblemsolving.com/community/user/74657}{ArefS}]
	I wonder how you solved this one :o
\end{solution}
*******************************************************************************
-------------------------------------------------------------------------------

\begin{problem}[Posted by \href{https://artofproblemsolving.com/community/user/43536}{nguyenvuthanhha}]
	For a positive integer $m$, express
\[\sum_{n=1}^{ \infty} \frac{n}{\gcd(n,m)} x^n\]
as a rational function of $x$.
	\flushright \href{https://artofproblemsolving.com/community/c6h360467}{(Link to AoPS)}
\end{problem}



\begin{solution}[by \href{https://artofproblemsolving.com/community/user/29428}{pco}]
	\begin{tcolorbox}For a positive integer $m$, express

 $ \sum_{n=1}^{ \infty} \frac{n}{gcd(n;m)} x^n$

as a rational function of $x$.\end{tcolorbox}
I dont have the full solution. Just some directions.

Let $f_m(x)=\sum_{n=1}^{+\infty}\frac n{\gcd(n,m)}x^n$ defined for $|x|<1$

We get $f_1(x)=\frac x{(1-x)^2}$

It's rather easy to show that if $p$ is prime and $\gcd(k,p)=1$, then $f_{kp}(x)=f_k(x)-(p-1)f_k(x^p)$

This formula may be extended to $f_{kp^q}(x)=pf_k(x)-(p-1)(f_k(x)+f_k(x^p)+f_k(x^{p^2})+...+f_k(x^{p^q}))$

This gives a method to find $f_m(x)$ for any given $m$ but this gives ugly results. For example :

$f_p(x)=\frac x{(1-x)^2}-\frac{(p-1)x^p}{(1-x^p)^2}$ for any prime $p$

$f_{p^2}(x)=\frac x{(1-x)^2}-\frac{(p-1)x^p}{(1-x^p)^2}$ $-\frac{(p-1)x^{p^2}}{(1-x^{p^2})^2}$ for any prime $p$

...

And I wonder what could be the general formula for any $m$  :(

I hope somebody will find it.
\end{solution}



\begin{solution}[by \href{https://artofproblemsolving.com/community/user/43536}{nguyenvuthanhha}]
	Oh , I was asked this problem 1 month ago

    i have just found out that this problem be the AMM 10750

  which was Proposed by Leonard Smiley, University of Alaska, Anchorage, AK

  but I can't found out the official solution 

         Can anybody help me ?
\end{solution}



\begin{solution}[by \href{https://artofproblemsolving.com/community/user/29428}{pco}]
	FYI (for those who are still searching):

Applying my previous formulas, we get  (checked with numeric verifications) :

$f_1(x)=\frac x{(1-x)^2}$

$f_2(x)=\frac x{(1-x^2)^2}(x^2+x+1)$

$f_3(x)=\frac x{(1-x^3)^2}(x^4+2x^3+x^2+2x+1)$

$f_4(x)=\frac x{(1-x^4)^2}(x^6+x^5+3x^4+x^3+3x^2+x+1)$

$f_5(x)=\frac x{(1-x^5)^2}(x^4+x^3+3x^2+x+1)(x^4+x^3-x^2+x+1)$

$f_6(x)=\frac x{(1-x^6)^2}(x^{10}+x^9+x^8+2x^7+5x^6+x^5+5x^4+2x^3+x^2+x+1)$

...

It's easy to see that we can write  $f_m(x)=\frac{x^m}{(1-x^m)^2}H_m(x+\frac 1x)$ where $H_m(x)\in\mathbb Z[X]$ and has degree $m-1$ :
$H_1(x)=1$

$H_2(x)=x+1$

$H_3(x)=x^2+2x-1$

$H_4(x)=x^3+x^2-1$

$H_5(x)=x^4+2x^3-x^2-2x-3$

...

It's rather hard to find a pattern in these results.
\end{solution}



\begin{solution}[by \href{https://artofproblemsolving.com/community/user/29428}{pco}]
	A kind of solution (using finite sum instead of infinite sum):

Let $f_m(x)=\sum_{n=1}^{+\infty}\frac n{\gcd(n,m)}x^n$

Let then $g_m(x)=(1-x^m)^2f_m(x)=(x^{2m}-2x^m+1)\sum_{n=1}^{+\infty}\frac n{\gcd(n,m)}x^n$

$g_m(x)=\sum_{n=1}^{+\infty}\frac n{\gcd(n,m)}x^{n+2m}$ $-2\sum_{n=1}^{+\infty}\frac n{\gcd(n,m)}x^{n+m}$ $+\sum_{n=1}^{+\infty}\frac n{\gcd(n,m)}x^{n}$ 

$g_m(x)=\sum_{n=2m+1}^{+\infty}\frac {n-2m}{\gcd(n-2m,m)}x^{n}$ $-2\sum_{n=m+1}^{+\infty}\frac {n-m}{\gcd(n-m,m)}x^{n}$ $+\sum_{n=1}^{+\infty}\frac n{\gcd(n,m)}x^{n}$ 

$g_m(x)=\sum_{n=2m+1}^{+\infty}(\frac {n-2m}{\gcd(n-2m,m)}-$ $2\frac {n-m}{\gcd(n-m,m)}+\frac n{\gcd(n,m)})x^{n}$ $-2\sum_{n=m+1}^{2m}\frac {n-m}{\gcd(n-m,m)}x^{n}$ $+\sum_{n=1}^{2m}\frac n{\gcd(n,m)}x^{n}$

But $\frac {n-2m}{\gcd(n-2m,m)}-2\frac {n-m}{\gcd(n-m,m)}+\frac n{\gcd(n,m)}=0$ 

And so $g_m(x)=-2\sum_{n=m+1}^{2m}\frac {n-m}{\gcd(n-m,m)}x^{n}$ $+\sum_{n=1}^{2m}\frac n{\gcd(n,m)}x^{n}$

$g_m(x)=\sum_{n=m+1}^{2m}(\frac n{\gcd(n,m)}-2\frac {n-m}{\gcd(n-m,m)})x^{n}$ $+\sum_{n=1}^{m}\frac n{\gcd(n,m)}x^{n}$

$g_m(x)=\sum_{n=1}^{m}\frac n{\gcd(n,m)}x^{n}$ $+\sum_{n=m+1}^{2m}\frac {2m-n}{\gcd(n,m)}x^{n}$

And so $\boxed{f_m(x)=\frac{\sum_{n=1}^{m}\frac n{\gcd(n,m)}x^{n}+\sum_{n=m+1}^{2m}\frac {2m-n}{\gcd(n,m)}x^{n}}{(1-x^m)^2}}$
\end{solution}



\begin{solution}[by \href{https://artofproblemsolving.com/community/user/43536}{nguyenvuthanhha}]
	Dear Pco

   Thanks you very much

  You deserve to be considered as an Algebra Genius of Mathlinks
\end{solution}
*******************************************************************************
-------------------------------------------------------------------------------

\begin{problem}[Posted by \href{https://artofproblemsolving.com/community/user/71129}{dap}]
	Find all functions $f:\mathbb N\to \mathbb R$ such that
\[f(n-m)+f(n+m)=f(3n)\] for all positive integers $m$ and $n$ with $m<n$.
	\flushright \href{https://artofproblemsolving.com/community/c6h360567}{(Link to AoPS)}
\end{problem}



\begin{solution}[by \href{https://artofproblemsolving.com/community/user/29428}{pco}]
	\begin{tcolorbox}Find all functions:$N*\to R$ such that:
$f(n-m)+f(n+m)=f(3n)$ with every $m<n\in N*$\end{tcolorbox}
$f(n-1)+f(n+1)=f(3n)$ $\forall n>1$
$f(n-2)+f(n+2)=f(3n)$ $\forall n>2$

Subtracting, we get $f(n+2)-f(n-1)=f(n+1)-f(n-2)$ $\forall n>2$ and so $f(n+3)=f(n)+d$ $\forall n>0$ 

$n=2$ and $m=1$ imply then $f(1)+f(3)=f(6)$ and so $f(1)+f(3)=f(3)+d$ and $f(1)=d$
$n=5$ and $m=1$ imply then $f(4)+f(6)=f(15)$ and so $f(1)+d+f(3)+d=f(3)+4d$ and so $f(1)=2d$

So $f(1)=0$ and $f(n+3)=f(n)$ $\forall n$

$n=3$ and $m=1$ imply then $f(2)+f(4)=f(9)$ and so $f(2)=f(3)$
$n=4$ and $m=1$ imply then $f(3)+f(5)=f(12)$ and so $f(3)+f(2)=f(3)$ and so $f(2)=0$

Hence the unique solution : $f(x)=0$ $\forall x$
\end{solution}



\begin{solution}[by \href{https://artofproblemsolving.com/community/user/43536}{nguyenvuthanhha}]
	Just for fun :

   Find all function $ f : \mathbb{N} \mapsto  \mathbb{N} $ such that :

        $ f(m-n) f(m+n) \ = \ f(m^2) \ \forall m \ ; n \ \in \mathbb{N} ; n<m$

   I think it's harder
\end{solution}



\begin{solution}[by \href{https://artofproblemsolving.com/community/user/29428}{pco}]
	\begin{tcolorbox}Just for fun :

   Find all function $ f : \mathbb{N} \mapsto  \mathbb{N} $ such that :

        $ f(m-n) f(m+n) \ = \ f(m^2) \ \forall m \ ; n \ \in \mathbb{N} ; n<m$

   I think it's harder\end{tcolorbox}
(I consider $0\notin\mathbb N$)

Same method :
$f(m-1)f(m+1)=f(m^2)$ $\forall m>1$
$f(m-2)f(m+2)=f(m^2)$ $\forall m>2$

So (dividing) $f(m+3)=af(m)$ $\forall m>0$ and for some $a\in\mathbb Q^+$

$m=2$ and $n=1$ $\implies$ $f(1)f(3)=f(4)=af(1)$ and so $f(3)=a$
$m=3$ and $n=1$ $\implies$ $f(2)f(4)=f(9)$ and so $f(2)f(1)a=a^2f(3)$ and so $f(1)f(2)=a^2$
$m=4$ and $n=1$ $\implies$ $f(3)f(5)=f(16)$ and so $a^2f(2)=a^5f(1)$ and so $f(1)=\frac 1{\sqrt a}$ and $f(2)=a^2\sqrt a$
$m=5$ and $n=1$ $\implies$ $f(4)f(6)=f(25)$ and so $\sqrt aa^2=\frac 1{\sqrt a}a^8$ and so $a=1$

Hence the unique solution : $f(n)=1$ $\forall n$
\end{solution}
*******************************************************************************
-------------------------------------------------------------------------------

\begin{problem}[Posted by \href{https://artofproblemsolving.com/community/user/65464}{sororak}]
	Find all functions $f: \mathbb R \to \mathbb R$ such that for all reals $x$ and $y$,
\[f(f(x)+y)=f(x^2-y)+4f(x)y.\]
	\flushright \href{https://artofproblemsolving.com/community/c6h361769}{(Link to AoPS)}
\end{problem}



\begin{solution}[by \href{https://artofproblemsolving.com/community/user/29428}{pco}]
	\begin{tcolorbox}Find all functions $ f:\mathbb{R}\rightarrow\mathbb{R} $ such that:
$ \forall x,y\in\mathbb{R} \ : \ f(f(x)+y)=f(x^2-y)+4f(x)y $ .\end{tcolorbox}
Let $P(x,y)$ be the assertion $f(f(x)+y)=f(x^2-y)+4f(x)y$

$P(x,\frac{x^2-f(x)}2)$ $\implies$ $f(x)(f(x)-x^2)=0$ and so : $\forall x$ either $f(x)=0$, either $f(x)=x^2$
As a consequence, $f(0)=0$

$f(x)=0$ $\forall x$ is a solution.
Let us from now consider that $\exists a$ such that $f(a)\ne 0$
So $f(a)=a^2$ and $a\ne 0$

$P(a,x)$ $\implies$ $f(a^2+x)=f(a^2-x)+4a^2x$
If $f(a^2+x)=0$, we get $f(a^2-x)=-4a^2x$ and so either $x=0$, either $(a^2-x)^2=-4a^2x$ and so $x=-a^2$
So $f(a^2+x)=(a^2+x)^2$ $\forall x\notin\{0,-a^2\}$ $\iff$ $f(x)=x^2$ $\forall x\notin\{0,a^2\}$

But then, choose any $b\notin\{0,a^2,-a,+a\}$ : we get $f(b)=b^2\ne 0$ and so $f(x)=x^2$ $\forall x\notin \{0,b^2\}$ and so $f(a^2)=a^4$

So $f(x)=x^2$ $\forall x$, which indeed is a solution

Hence the two solutions :
$f(x)=0$ $\forall x$
$f(x)=x^2$ $\forall x$
\end{solution}
*******************************************************************************
-------------------------------------------------------------------------------

\begin{problem}[Posted by \href{https://artofproblemsolving.com/community/user/86097}{hurricane}]
	Find all functions $f: \mathbb R^{+} \to \mathbb R^{+}$ that satisfy the following condition for all $x>0$ and $y>0$:
\[f(x)f(y)=f(x+yf(x)).\]
	\flushright \href{https://artofproblemsolving.com/community/c6h361817}{(Link to AoPS)}
\end{problem}



\begin{solution}[by \href{https://artofproblemsolving.com/community/user/29428}{pco}]
	\begin{tcolorbox}Let the function $f : R+ \to R+$ satisfy the following condition:
$f(x)f(y)=f(x+yf(x))$
Find all of such functions ???\end{tcolorbox}
Why is this problem posted in this forum ? IMHO, you should post the solution with the problem here. 

And, btw, could you post the solution of the previous problem you posted in this forum, please ?

Let $P(x,y)$ be the assertion $f(x)f(y)=f(x+yf(x))$

If $\exists a$ such that $f(a)<1$, then $P(a,\frac a{1-f(a)})$ $\implies$ $f(a)=1$, so contradiction and so $f(x)\ge 1$ $\forall x$

$P(x,\frac y{f(x)})$ $\implies$ $f(x+y)=f(x)f(\frac y{f(x)})\ge f(x)$ and so $f(x)$ is non decreasing.

If $\exists a$ such that $f(a)=1$, then $P(a,x)$ $\implies$ $f(x)=f(x+a)$ and so $f(x+na)=f(x)$ and, since $f(x)$ is non decreasing, $f(x)=c=1$

If $f(x)>1$ $\forall x$, then $f(x)$ is strictly increasing, so is an injection.
Then, comparing $P(x,y)$ and $P(y,x)$, we get $x+yf(x)=y+xf(y)$ and so $\frac 1y+\frac{f(x)}x=\frac 1x+\frac{f(y)}y$ and so $\frac{f(x)}x-\frac 1x=c$ and the solution $f(x)=1+cx$ for some $c>0$

Hence the general solution : $f(x)=1+cx$ for any $c\ge 0$
\end{solution}
*******************************************************************************
-------------------------------------------------------------------------------

\begin{problem}[Posted by \href{https://artofproblemsolving.com/community/user/82904}{gold46}]
	Find all functions $f: \mathbb R \to \mathbb R$ such that for all reals $x$ and $y$,
\[f(x+y)+f(xy+1)=f(x)+f(y)+f(xy).\]
	\flushright \href{https://artofproblemsolving.com/community/c6h361836}{(Link to AoPS)}
\end{problem}



\begin{solution}[by \href{https://artofproblemsolving.com/community/user/29428}{pco}]
	\begin{tcolorbox}Find all  $f$  functions such that 
$f: \mathbb{R} \rightarrow \mathbb{R}$  and 
$f(x+y)+f(xy+1)=f(x)+f(y)+f(xy)$.\end{tcolorbox}
Let $P(x,y)$ be the assertion $f(x+y)+f(xy+1)=f(x)+f(y)+f(xy)$
Let $f(0)=a$

$P(x,0)$ $\implies$ $f(1)=2a$

$P(xy,1)$ $\implies$ $f(xy+1)=f(xy)+a$

So $P(x,y)$ becomes $f(x+y)=f(x)+f(y)-a$

So the problem is equivalent to :
$f(x)=g(x)+a$
$g(x+y)=g(x)+g(y)$
$g(1)=a$

Hence the solution : $f(x)=g(x)+g(1)$ where $g(x)$ is any solution of Cauchy's equation
\end{solution}



\begin{solution}[by \href{https://artofproblemsolving.com/community/user/82904}{gold46}]
	$g(x)=x^2$ isn't solution of Cauchy equation. But $f(x)=a(x^2+1)$ is solution.
\end{solution}



\begin{solution}[by \href{https://artofproblemsolving.com/community/user/29428}{pco}]
	\begin{tcolorbox}$g(x)=x^2$ isn't solution of Cauchy equation. But $f(x)=a(x^2+1)$ is solution.\end{tcolorbox}
Surely not. Check again.
\end{solution}



\begin{solution}[by \href{https://artofproblemsolving.com/community/user/82904}{gold46}]
	Oh sorry!
it should be 
Find all  $f$  functions such that 
$f: \mathbb{R} \rightarrow \mathbb{R}$  and 
$f(x+y)+f(xy-1)=f(x)+f(y)+f(xy)$.
\end{solution}



\begin{solution}[by \href{https://artofproblemsolving.com/community/user/29428}{pco}]
	\begin{tcolorbox}Oh sorry!
it should be 
Find all  $f$  functions such that 
$f: \mathbb{R} \rightarrow \mathbb{R}$  and 
$f(x+y)+f(xy-1)=f(x)+f(y)+f(xy)$.\end{tcolorbox}
Let us consider $g(x)=f(x)-\frac{f(1)}2x^2-(\frac{f(1)}2-f(0))x-f(0)$. It's easy to check that $g(x)$ is such that :
a) $g(x+y)+g(xy-1)=g(x)+g(y)+g(xy)$
b) $g(1)=g(0)=0$

Let $P(x,y)$ be the assertion $g(x+y)+g(xy-1)=g(x)+g(y)+g(xy)$

$P(x+1,1)$ $\implies$ $g(x+2)=2g(x+1)-g(x)$ and so $g(n)=0$ $\forall n\in\mathbb Z$

And so $g(x+n)-g(x)=n(g(x+1)-g(x))$

Let $p\in\mathbb Z,q\in\mathbb Z-\{0\}$ : $P(\frac pq,q)$ $\implies$ $g(\frac pq+q)=g(\frac pq)$ and so, using previous line : $g(\frac pq+1)=g(\frac pq)$

And so $g(x+1)=g(x)$ $\forall x\in\mathbb Q$

With continuity constraint \end{underlined}(we dont have), we could conclude $g(x+1)=g(x)$ $\forall x$ and so $P(x,y)$ becomes $g(x+y)=g(x)+g(y)$ and so $g(x)=cx$ and so $c=0$ and the unique continuous family of solution is $\boxed{f(x)=ax^2+bx+a-b}$

Without continuity constraint\end{underlined}, we certainly have a lot of solutions.
For example, choosing $g(x+1)=g(x)$ $\forall x$ (which is not mandatory), we get that $g(x)$ is solution of Cauchy's equation.
And we get the family of solutions : $f(x)=g(x)+ax^2+(b-g(1))x+a-b$ where $g(x)$ is any solution of Cauchy's equation.

I think there exist some other families of non continuous solution and I'll look for it \begin{bolded}after you confirm us that the olympiad exercise you gave us really does not contain the continuity constraint\end{bolded}\end{underlined}.
\end{solution}



\begin{solution}[by \href{https://artofproblemsolving.com/community/user/29428}{pco}]
	Please, gold46, I see that you posted some other messages since I asked you my previous question but did not answer my question yet.

Could you give us an answer ? 
Could you check the original problem in your book or exam sheet and confirm us (or modify the statement) that the olympiad exercise you gave us did not contain some other constraint, like continuity ?
\end{solution}



\begin{solution}[by \href{https://artofproblemsolving.com/community/user/82904}{gold46}]
	I checked. I was in our olympiad. I forgot $f(1)=2$. Now it's clear. But it can help you?
\end{solution}



\begin{solution}[by \href{https://artofproblemsolving.com/community/user/29428}{pco}]
	Ok, Thanks for this answer.

Then :


\begin{tcolorbox}Find all  $f$  functions such that 
$f: \mathbb{R} \rightarrow \mathbb{R}$  and 
$f(x+y)+f(xy-1)=f(x)+f(y)+f(xy)$.\end{tcolorbox}
Let us consider $g(x)=f(x)-\frac{f(1)}2x^2-(\frac{f(1)}2-f(0))x-f(0)$. It's easy to check that $g(x)$ is such that :
a) $g(x+y)+g(xy-1)=g(x)+g(y)+g(xy)$
b) $g(1)=g(0)=0$

Let $P(x,y)$ be the assertion $g(x+y)+g(xy-1)=g(x)+g(y)+g(xy)$
Let $h(x)=g(x+1)-g(x)$

1) $h(x)$ is solution of Cauchy equation and $h(1)=0$
======================================

$P(x+1,1)$ $\implies$ $g(x+2)=2g(x+1)-g(x)$ and so : $h(x+1)=h(x)$ $\forall x$
An immediate consequence is that $h(n)=g(n)=0$ $\forall n\in\mathbb Z$

Another consequence is that $g(x+n)-g(x)=n(g(x+1)-g(x))$
But $P(x,n)$ $\implies$ $g(x+n)=g(x)+h(nx-1)=g(x)+h(nx)$ and so $h(nx)=nh(x)$

$P(x,y+1)$ $\implies$ $g(x+y+1)+g(xy+x-1)=g(x)+g(y+1)+g(xy+x)$
$P(x,y)$ $\implies$ $g(x+y)+g(xy-1)=g(x)+g(y)+g(xy)$
Subtracting these two lines, we get $h(x+y)=h(y)+h(xy+x-1)-h(xy-1)$ and so $h(x+y)+h(xy)=h(y)+h(xy+x)$

And so, with symetry : $h(x)+h(xy+y)=h(y)+h(xy+x)$
Replacing $x\to 2x$ in this equation implies $2h(x)+h(2xy+y)=h(y)+2h(xy+x)$

Multiplying the first of these two lines by $2$ and subtacting from the second gives :
$h(2xy+y)-2h(xy+y)=-h(y)$ and so
$h(2xy+y)+h(y)=h(2xy+2y)$

Setting then $x=\frac{u-v}{2v}$ and $y=v$, this equation becomes $h(u)+h(v)=h(u+v)$ $\forall u,v\ne 0$ and this is still true for $v=0$
Q.E.D.

2) Solution for $g(x)$
===============
The problem is equivalent to find $g,h$ such that :
a) $h(x+y)=h(x)+h(y)$
b) $h(n)=g(n)=0$ $\forall n\in\mathbb Z$
c) $g(x+y)=g(x)+g(y)+h(xy)$

Let $k(x)=g(x)-\frac{h(x^2)}2$
c) implies $k(x+y)=k(x)+k(y)$


Hence the general solution for $g(x),h(x)$ : $g(x)=k(x)+\frac{h(x^2)}2$ where :
$k(x)$ is any solution of Cauchy equation such that $k(n)=0$ $\forall n\in\mathbb Z$
$h(x)$ is any solution of Cauchy equation such that $h(n)=0$ $\forall n\in\mathbb Z$

3) General solution of the required problem
==============================

Let $u(x)$ and $v(x)$ be any two solutions of Cauchy's equation. The general solution is then :

$f(x)=u(x)-u(1)x+v(x^2)-v(1)x^2+ax^2+bx+a-b$ (and it is easy, although a little bit ugly, to verify this indeed is a solution)

If we add the constraint $f(1)=2$, then we get $a=1$ and the solution $\boxed{f(x)=u(x)-u(1)x+v(x^2)-v(1)x^2+x^2+bx+1-b}$


I am very impressed by the level of olympiad exercices in your country  :)
And dont hesitate to post the official solution when you'll get it.
\end{solution}



\begin{solution}[by \href{https://artofproblemsolving.com/community/user/82904}{gold46}]
	Thanks for your posts. 
Your solutions are very interesting and nice.
 It seems you're great.
\end{solution}
*******************************************************************************
-------------------------------------------------------------------------------

\begin{problem}[Posted by \href{https://artofproblemsolving.com/community/user/78444}{Babai}]
	Let $b$ be a positive real number. Find all functions $f:\mathbb{R}\to\mathbb{R}$ satisfying
\[f(x+y)=f(x)\cdot 3^{{b^y}+f(y)-1}+b^x(3^{{b^y}+f(y)-1}-b^y)\]
for all  $x,y\in\mathbb{R}$.
	\flushright \href{https://artofproblemsolving.com/community/c6h362621}{(Link to AoPS)}
\end{problem}



\begin{solution}[by \href{https://artofproblemsolving.com/community/user/29428}{pco}]
	\begin{tcolorbox}$\text{Let b is a positive real number. Find all functions}$  $f:\mathbb{R}\rightarrow\mathbb{R}$ $\text{satisfying}$: 
$f(x+y)=f(x).3^{{b^y}+f(y)-1}+b^x(3^{{b^y}+f(y)-1}-b^y)$ for all  $x,y\in\mathbb{R}$\end{tcolorbox}
Let $g(x)=f(x)+b^x$. The equation may be written $P(x,y)$ : $g(x+y)=g(x)3^{g(y)-1}$

$P(0,0)$ $\implies$ $g(0)=g(0)3^{g(0)-1}$ and so :

Either $g(0)=0$ and then $P(0,x)$ $\implies$ $g(x)=g(0)3^{g(x)-1}=0$ which indeed is a solution

Either $g(0)=1$ and then $P(0,x)$ $\implies$ $g(x)=3^{g(x)-1}$ 

The equation $x=3^{x-1}$ has only two real roots $1$ and $a\in(0,1)$ and so $g(x)\in\{a,1\}$

Suppose $\exists u$ such that $g(u)=a$  : $P(u,u)$ $\implies$ $g(2u)=a3^{a-1}=a^2$ and this is impossible since $a^2\notin\{a,1\}$
So $g(x)=1$ $\forall x$ which indeed is a solution.

Hence the only two solutions for the original equation :
$f(x)=-b^x$
$f(x)=1-b^x$
\end{solution}
*******************************************************************************
-------------------------------------------------------------------------------

\begin{problem}[Posted by \href{https://artofproblemsolving.com/community/user/51029}{mathVNpro}]
	Find all $f:$ $\mathbb {N}\to\mathbb {N}$ such that:
i) $f$ is a strictly increasing function, and
ii) For all positive integers $n$,
\[f(f(n)-2n)=3n.\]
	\flushright \href{https://artofproblemsolving.com/community/c6h362636}{(Link to AoPS)}
\end{problem}



\begin{solution}[by \href{https://artofproblemsolving.com/community/user/29428}{pco}]
	\begin{tcolorbox}Denote $\mathbb {N}$ by the set of $\{1,2,...,n\}$. Find all $f:$ $\mathbb {N}\longrightarrow \mathbb {N}$ such that:
$i.$ $f$ is a strictly increased function.
$ii.$ $f(f(n)-2n)=3n$, $\forall n\in \mathbb {N}$\end{tcolorbox}
Suppose we can prove $an>f(n)>bn$ $\forall n$ for some real $a>b>0$

Then $a(f(n)-2n)>f(f(n)-2n)>b(f(n)-2n)$ and so $af(n)-2an>3n>bf(n)-2bn$ and so $(2+\frac 3b)n>f(n)>(2+\frac 3a)n$

At the beginning, we know that $f(n)>2n$ else the relation ii is meaningless.
Using the above mechanism, we get as a consequence $\frac 72n>f(n)$

Let us build then the sequences $a_k$ and $b_k$ as :
$a_1=\frac 72$ and $b_1=2$

$a_{k+1}=2+\frac 3{b_k}$

$b_{k+1}=2+\frac 3{a_k}$

We get $a_kn>f(n)>b_kn$ $\forall n$, $\forall k$

And since $\lim_{k\to+\infty}a_k=\lim_{k\to+\infty}b_k=3$, we get $3n\ge f(n)\ge 3n$ and so $f(n)=3n$ which indeed is a solution.

Hence the unique solution : $\boxed{f(n)=3n}$

And, btw, the first condition (strictly increasing) is useless
\end{solution}
*******************************************************************************
-------------------------------------------------------------------------------

\begin{problem}[Posted by \href{https://artofproblemsolving.com/community/user/82904}{gold46}]
	Find all continuous functions $f:\mathbb{R} \to\mathbb{R}$ such that $f(x+f(x))=f(x)$ for all real $x$.
	\flushright \href{https://artofproblemsolving.com/community/c6h362912}{(Link to AoPS)}
\end{problem}



\begin{solution}[by \href{https://artofproblemsolving.com/community/user/29428}{pco}]
	\begin{tcolorbox}Find all $f:\mathbb{R} \rightarrow \mathbb{R}$ continuous function such that $f(x+f(x))=f(x)$.\end{tcolorbox}
Here is a not very elegant solution :

An immediate induction gives $f(x+nf(x))=f(x)$ $\forall x,\forall n\in\mathbb N$

Suppose now that $\exists a\ne b\in \mathbb R$ such that $f(a)\in\mathbb Q$ and $f(b)\notin\mathbb Q$

So $\frac{f(a)}{f(b)}\notin\mathbb Q$ and $\{n\frac{f(a)}{f(b)}\}$ is dense in $[0,1]$

So we can find positive integers $m,n$ such that $n\frac{f(a)}{f(b)}-m$ is as near of $\frac{b-a}{f(b)}$ as we want

and so $a+nf(a)$ as near of $b+mf(b)$ as we want.

Then, since $f(a)=f(a+nf(a))$ and $f(b)=f(b+mf(b))$, continuity implies then that $f(a)=f(b)$, which is impossible.

So no such $a,b$ exist and either $f(x)\in\mathbb Q$ $\forall x$, either $f(x)\notin\mathbb Q$ $\forall x$

So, continuity again implies $f(x)=c$ is a constant function, which indeed is a solution.

Hence the unique family of solutions : $f(x)=c$ $\forall x$
\end{solution}



\begin{solution}[by \href{https://artofproblemsolving.com/community/user/71459}{x164}]
	Why does $f(x) \in \mathbb{Q} \; \forall x$ imply that $f(x)$ is a constant function?
\end{solution}



\begin{solution}[by \href{https://artofproblemsolving.com/community/user/29428}{pco}]
	\begin{tcolorbox}Why does $f(x) \in \mathbb{Q} \; \forall x$ imply that $f(x)$ is a constant function?\end{tcolorbox}

Because if $f(a)<f(b)$ for some $a,b$, continuity implies that $(f(a),f(b))\subseteq f(\mathbb R)$ and since any non empty open interval contains non-rational elements, we would get a contradiction.
\end{solution}



\begin{solution}[by \href{https://artofproblemsolving.com/community/user/59935}{hana1122}]
	\begin{tcolorbox}[quote="gold46"]Find all $f:\mathbb{R} \rightarrow \mathbb{R}$ continuous function such that $f(x+f(x))=f(x)$.\end{tcolorbox}
...
and so $a+nf(a)$ as near of $b+mf(b)$ as we want.

Then, since $f(a)=f(a+nf(a))$ and $f(b)=f(b+mf(b))$, continuity implies then that $f(a)=f(b)$, which is impossible.\end{tcolorbox}
I think it is not true, because when $m$ and $n$ varies, $a+nf(a)$ and $b+mf(b)$ don't approach to a constant number, and hence we can't use continuity of function. Plz explain this part.
\end{solution}



\begin{solution}[by \href{https://artofproblemsolving.com/community/user/64716}{mavropnevma}]
	Right on! An example. Define $g(H_{2n}) = 0$, $g(H_{2n+1}) = 1$, and connect these points by segments to obtain a piecewise linear continuous function $g : [1,\infty) \to \mathbb{R}$ (here $H_k$ is the partial sum of $k$ terms of the harmonic series). Then $\lim_{n\to \infty} (H_{k+1} - H_k) = 0$, but $f(1) \neq f(3\/2)$. 

It is not exactly the case of above, where $a+nf(a)$ and $b+mf(b)$ are arithmetic progressions, but the phenomenon points to a flaw.
\end{solution}



\begin{solution}[by \href{https://artofproblemsolving.com/community/user/29428}{pco}]
	\begin{tcolorbox}Right on! An example. Define $g(H_{2n}) = 0$, $g(H_{2n+1}) = 1$, and connect these points by segments to obtain a piecewise linear continuous function $g : [1,\infty) \to \mathbb{R}$ (here $H_k$ is the partial sum of $k$ terms of the harmonic series). Then $\lim_{n\to \infty} (H_{k+1} - H_k) = 0$, but $f(1) \neq f(3\/2)$. 

It is not exactly the case of above, where $a+nf(a)$ and $b+mf(b)$ are arithmetic progressions, but the phenomenon points to a flaw.\end{tcolorbox}

Nice example !

It seems indeed that I'm wrong on this point.
Thanks both for having pointed this  :blush: 

I'll look for another proof
\end{solution}



\begin{solution}[by \href{https://artofproblemsolving.com/community/user/29428}{pco}]
	\begin{tcolorbox}Find all $f:\mathbb{R} \rightarrow \mathbb{R}$ continuous function such that $f(x+f(x))=f(x)$.\end{tcolorbox}

Next trial :
Induction gives immediately $f(x+nf(x))=f(x)$ $\forall x,\forall n\in \mathbb N\cup\{0\}$

$f(x)=0$ is a solution.
We'll from now consider non all-zero solutions.

Let $b\in f(\mathbb R)$ such that $b\ne 0$. Since $f(x)$ solution implies $-f(-x)$ solution too, wlog say $b>0$

Let then $a<c$ such that $f(a)=f(c)=b$ ( $a$ exists since $b\in f(\mathbb R)$ and choose for example $c=a+b=a+f(a)$

1) We'll show that $\exists u\in(a,c)$ such that $f(u)=b$
========================================
If $c>a+f(a)$, just choose $u=a+f(a)$
If $c\le a+f(a)$ : let then $d\in (a,c)$

1.1) If $f(d)=b$, just choose $u=d$ and we got our result

1.2) If $f(d)>b$ , then $\exists n\in\mathbb N$ such that $n>\frac{a+f(a)-d}{f(d)-b}$ and so $d+nf(d)>a+(n+1)f(a)$

So $a+nf(a)<a+(n+1)f(a)$ and $d+nf(d)>a+(n+1)f(a)$ and so continuity of $x+nf(x)$ implies $\exists u\in(a,d)$ such that $u+nf(u)=a+(n+1)f(a)$

And so $\exists u\in(a,d)$ such that $f(u+nf(u))=f(a+(n+1)f(a))$ and so $f(u)=f(a)$.

1.3) If $f(d)<b$, then $\exists n\in\mathbb N$ such that $n>\frac{d-a}{f(a)-f(d)}$ and so $d+nf(d)< a+nf(a)$

So $d+nf(d)< a+nf(a)$ and $c+nf(c)=c+nf(a)>a+nf(a)$ and so continuity of $x+nf(x)$ implies $\exists u\in(d,c)$ such that $u+nf(u)=a+nf(a)$

And so $\exists u\in (d,c)$ such that $f(u+nf(u))=f(a+nf(a))$ and so $f(u)=f(a)$.

2) the set $f^{-1}(\{f(a)\})$ is dense in $[a,+\infty)$
====================================
Suppose the contrary and let $(u,v)\subset[a,+\infty)$ such that $f(x)\ne f(a)$ $\forall x\in (u,v)$

Let then $E=\{x\le u$ such that $f(x)=f(a)\}$ and $F=\{x\ge v$ such that $f(x)=f(a)\}$
E and F are non empty (consider $f(a+nf(a))$)

So, let $w=\sup (E)$ and $z=\inf(F)$ Continuity implies $f(w)=f(z)=f(a)$

And previous paragraph implies $\exists t\in(w,z)$ such that $f(t)=f(a)$ and so contradiction

3) $f(x)=f(a)$ $\forall x$
==========================
Previous paragraph shows that $f(x)=f(a)$ $\forall x\ge a$ (continuity)
It's then immediate to conclude $f(x)=c$ is constant

And it's obvious that these functions indeed are solutions.
\end{solution}
*******************************************************************************
-------------------------------------------------------------------------------

\begin{problem}[Posted by \href{https://artofproblemsolving.com/community/user/88228}{625gs}]
	Find all functions $ f:\mathbb{R}\to\mathbb{R} $ such that \[f(f(x)-f(y))=(x-y)^2 f(x+y)\] for all reals $x$ and $y$.
	\flushright \href{https://artofproblemsolving.com/community/c6h363194}{(Link to AoPS)}
\end{problem}



\begin{solution}[by \href{https://artofproblemsolving.com/community/user/29428}{pco}]
	\begin{tcolorbox}1.Find all function $ f:\mathbb{R}\to\mathbb{R} $ such that 
$f(f(x)-f(y))=(x-y)^2 f(x+y)$\end{tcolorbox}
Let $P(x,y)$ be the assertion  $f(f(x)-f(y))=(x-y)^2f(x+y)$

$f(x)=0$ is a solution and we'll from now consider non all-zero solutions. So $\exists a$ such that $f(a)\ne 0$

$P(0,0)$ $\implies$ $f(0)=0$ (and so $a\ne 0$)
$P(x,0)$ $\implies$ $f(f(x))=x^2f(x)$
$P(0,x)$ $\implies$ $f(-f(x))=x^2f(x)$ and so $f(-f(x))=f(f(x))$

$P(x,a-x))$ $\implies$ $f(f(x)-f(a-x))=(2x-a)^2f(a)$ and so all reals with same sign as $f(a)$ may be reached
Then $f(-f(x))=f(f(x))$ implies $f(-x)=f(x)$ $\forall x$

$P(\frac{x+a}2,\frac{x-a}2)$ $\implies$ $f(f(\frac{x+a}2)-f(\frac{x-a}2))=a^2f(x)$

$P(\frac{x+a}2,\frac{a-x}2)$ $\implies$ $f(f(\frac{x+a}2)-f(\frac{x-a}2))=x^2f(a)$

And so $x^2f(a)=a^2f(x)$ and $f(x)=\frac{f(a)}{a^2}x^2$ and so $f(x)=cx^2$ and, plugging back in the original equation, we get $c=\pm 1$

Hence the three solutions :
$f(x)=0$ $\forall x$
$f(x)=x^2$ $\forall x$
$f(x)=-x^2$ $\forall x$
\end{solution}
*******************************************************************************
-------------------------------------------------------------------------------

\begin{problem}[Posted by \href{https://artofproblemsolving.com/community/user/68025}{Pirkuliyev Rovsen}]
	Let $a$ and $b$ be real numbers such that $a \neq 0$ and suppose that $f:\mathbb R\to[0;+\infty )$ be a function for which the equation \[f(x+a+b)+f(x)=f(x+a)+f(x+b)\] holds for every $ x \in \mathbb R$. Prove that if $\frac{a}{b}$ is rational, then $f$ is periodic.
	\flushright \href{https://artofproblemsolving.com/community/c6h363208}{(Link to AoPS)}
\end{problem}



\begin{solution}[by \href{https://artofproblemsolving.com/community/user/29428}{pco}]
	\begin{tcolorbox}Let $a,b \in \mathbb{R}$  and $f:R\rightarrow [0;+\infty )$

be a function such that $f(x+a+b)+f(x)=f(x+a)+f(x+b)$  for every $ x \in R$ 
Prove that if $\frac{a}{b}\in Q$ ,then function f is periodical \end{tcolorbox}

\begin{bolded}This is wrong \end{bolded}\end{underlined}: Let $f(x)=x^2$ and $a=0$ and $b=1$

Now, if we add the constraint $a\ne 0$, this becomes true :

Let $\frac ab=\frac pq$ with $q\in\mathbb N$ and $p\in\mathbb Z\setminus\{0\}$
Let $g(x)=f(\frac {bx}q)$ so that the equation becomes $g(x+p+q)+g(x)=g(x+p)+g(x+q)$

From $g(x+p+q)-g(x+q)=g(x+p)-g(x)$, it's easy to establish with induction $g(x+nq+p)-g(x+nq)=g(x+p)-g(x)$ $\forall x,\forall n\in\mathbb Z$

Adding these equalities for $x,x+p,x+2p,...x+(q-1)p$, we get $g(x+nq+pq)-g(x+nq)=g(x+pq)-g(x)$

Adding these equalities for $n=0,p,2p,...,kp$, we get $g(x+kpq)-g(x)=k(g(x+pq)-g(x))$ $\forall x,\forall k\in\mathbb Z$

If $g(x+pq)>g(x)$ and setting $k\to-\infty$, we get $g(x+kpq)\to -\infty$, impossible since $g(x)\ge 0$ $\forall x$
If $g(x+pq)<g(x)$ and setting $k\to+\infty$, we get $g(x+kpq)\to -\infty$, impossible since $g(x)\ge 0$ $\forall x$

So $g(x+pq)=g(x)$ and $f(x)=f(x+bp)=f(x+aq)$ 
Q.E.D. (since $bp=aq\ne 0$)
\end{solution}
*******************************************************************************
-------------------------------------------------------------------------------

\begin{problem}[Posted by \href{https://artofproblemsolving.com/community/user/82904}{gold46}]
	Let $f : \mathbb{R} \to\mathbb{R}$ be a continuous function such that $f(x)\cdot f(f(x))=1$ for all $x \in \mathbb R$ and $f(1000)=999$. find $f(500)$.
	\flushright \href{https://artofproblemsolving.com/community/c6h363494}{(Link to AoPS)}
\end{problem}



\begin{solution}[by \href{https://artofproblemsolving.com/community/user/44659}{uglysolutions}]
	Putting $x=1000$ yields $f(1000) \cdot f(999) = 1 \Rightarrow f(999) = \frac{1}{999}$. Hence, by the mean value theorem, there is $z \in (999,1000)$ such that $f(z) = 500$. Now, setting $x=z$ we get $f(z) \cdot f(500) = 1 \Rightarrow \boxed{f(500) = \frac{1}{500}}$.
\end{solution}



\begin{solution}[by \href{https://artofproblemsolving.com/community/user/44358}{crazyfehmy}]
	\begin{tcolorbox}Putting $x=1000$ yields $f(1000) \cdot f(999) = 1 \Rightarrow f(999) = \frac{1}{999}$. Hence, by the mean value theorem, there is $z \in (999,1000)$ such that $f(z) = 500$. Now, setting $x=z$ we get $f(z) \cdot f(500) = 1 \Rightarrow \boxed{f(500) = \frac{1}{500}}$.\end{tcolorbox}
I think the theorem you used is \begin{bolded}Intermediate Value Theorem\end{bolded}  :)
\end{solution}



\begin{solution}[by \href{https://artofproblemsolving.com/community/user/24037}{elim}]
	Very interesting. Can the function be solved from the equation and the init condition? Or, Does such a continuous function exist?
\end{solution}



\begin{solution}[by \href{https://artofproblemsolving.com/community/user/44659}{uglysolutions}]
	\begin{tcolorbox}[quote="uglysolutions"]Putting $x=1000$ yields $f(1000) \cdot f(999) = 1 \Rightarrow f(999) = \frac{1}{999}$. Hence, by the mean value theorem, there is $z \in (999,1000)$ such that $f(z) = 500$. Now, setting $x=z$ we get $f(z) \cdot f(500) = 1 \Rightarrow \boxed{f(500) = \frac{1}{500}}$.\end{tcolorbox}
I think the theorem you used is \begin{bolded}Intermediate Value Theorem\end{bolded}  :)\end{tcolorbox}

Doesn't matter how you call it, you know what I \begin{bolded}mean\end{bolded}t to say  :P
\end{solution}



\begin{solution}[by \href{https://artofproblemsolving.com/community/user/29428}{pco}]
	\begin{tcolorbox}Very interesting. Can the function be solved from the equation and the init condition? Or, Does such a continuous function exist?\end{tcolorbox}

Let $A=f(\mathbb R)$. The equation is $f(x)=\frac 1x$ $\forall x\in A$ and $f(1000)=999$

So $999\in A$ while $1000\notin A$ and so ($f(x)$ is continuous) $\exists a\in[999,1000)$ such that $A=[\frac 1a,a]$

Hence the solutions :

On $(-\infty,\frac 1a]$ : $f(x)$ takes any value you want in $[\frac 1a,a]$ such that $f(\frac 1a)=a$ and $f(x)$ is continuous.

On $[\frac 1a,a]$ : $f(x)=\frac 1x$

On $[a,+\infty)$ : $f(x)$ takes any value you want in $[\frac 1a,a]$ such that $f(a)=\frac 1a$, $f(1000)=999$ and $f(x)$ is continuous.
\end{solution}
*******************************************************************************
-------------------------------------------------------------------------------

\begin{problem}[Posted by \href{https://artofproblemsolving.com/community/user/59935}{hana1122}]
	Find all functions $f:(1,+\infty)\to\Bbb R$ such that  for any $x,y>1$,
\[f(x)-f(y)=(y-x)f(xy).\]
	\flushright \href{https://artofproblemsolving.com/community/c6h363497}{(Link to AoPS)}
\end{problem}



\begin{solution}[by \href{https://artofproblemsolving.com/community/user/29428}{pco}]
	\begin{tcolorbox}Find all functions $f:(1,+\infty)\to\Bbb R$ such that  for any $x,y>1$,
\[f(x)-f(y)=(y-x)f(xy)\]\end{tcolorbox}
Let $P(x,y)$ be the assertion $f(x)-f(y)=(y-x)f(xy)$

Let $u>v>1$ two real numbers.

Let $x\in(\frac uv,uv)$

$P(\sqrt{\frac{ux}v},\sqrt{\frac{uv}x})$ $\implies$ $f(\sqrt{\frac{ux}v})-f(\sqrt{\frac{uv}x})=$ $(\sqrt{\frac{uv}x}-\sqrt{\frac{ux}v})f(u)$

$P(\sqrt{\frac{uv}x},\sqrt{\frac{vx}u})$ $\implies$ $f(\sqrt{\frac{uv}x})-f(\sqrt{\frac{vx}u})=$ $(\sqrt{\frac{vx}u}-\sqrt{\frac{uv}x})f(v)$

$P(\sqrt{\frac{vx}u},\sqrt{\frac{ux}v})$ $\implies$ $f(\sqrt{\frac{vx}u})-f(\sqrt{\frac{ux}v})=$ $(\sqrt{\frac{ux}v}-\sqrt{\frac{vx}u})f(x)$

Adding these three lines and multiplying by $\sqrt{uvx}$ gives : $(v-x)uf(u)+(x-u)vf(v)+(u-v)xf(x)=0$

Let then $g(x)=xf(x)$. We got $g(x)=x\frac{g(u)-g(v)}{u-v}+\frac{ug(v)-vg(u)}{u-v}$ $\forall x\in (\frac uv,uv)$

So $\exists a$ such that $\frac{g(u)-g(v)}{u-v}=a$ and so $g(x)=ax+b$ and $f(x)=a+\frac bx$

Plugging this back in original equation, we get $\boxed{f(x)=\frac bx}$
\end{solution}
*******************************************************************************
-------------------------------------------------------------------------------

\begin{problem}[Posted by \href{https://artofproblemsolving.com/community/user/87348}{Sansa}]
	Find all functions $ f: \mathbb{R}\to\mathbb{R}$ such that
\[ f(f(x)-y) = f(x) + f(f(y)-f(-x)) + x \]
holds for all $x, y \in \mathbb R$.
	\flushright \href{https://artofproblemsolving.com/community/c6h364006}{(Link to AoPS)}
\end{problem}



\begin{solution}[by \href{https://artofproblemsolving.com/community/user/29428}{pco}]
	\begin{tcolorbox}Find all of the functions that $ f: \mathbb{R}\to\mathbb{R}$ and:
\[ f(f(x)-y) = f(x) + f(f(y)-f(-x)) + x \]\end{tcolorbox}
Let $P(x,y)$ be the assertion $f(f(x)-y)=f(x)+f(f(y)-f(-x))+x$

1) $f(x)$ is surjective
===============
$P(-x,f(-x)+x)$ $\implies$ $f(f(f(-x)+x)-f(x))=x$
Q.E.D.

Let then $f(0)=a$ and $u$ such that $f(u)=0$

2) $f(x)$ is injective
==============
$P(u,x)$ $\implies$ $f(-x)=u+f(f(x)-f(-u))$
Plugging this in $P(x,y)$, we get : $f(f(x)-y)=f(x)+f(f(y)-u-f(f(x)-f(-u)))+x$
And setting $f(x_1)=f(x_2)$ in this expression implies $x_1=x_2$
Q.E.D.

3) $f(0)=0$
========
$P(u,-u)$ $\implies$ $0=u+a$ $\implies$ $u=-a$ and so $f(-a)=0$
Then $P(0,u)$  $\implies$ $f(a-u)=a=f(0)$ and so (since injective) $a=u$
And so $a=u=0$
Q.E.D.

4) $f(x)=-x$
===========
$P(0,x)$ $\implies$ $f(-x)=f(f(x))$ and so, since injective : $f(x)=-x$
Q.E.D

And it is easy to check back that this indeed is a solution.
\end{solution}



\begin{solution}[by \href{https://artofproblemsolving.com/community/user/43536}{nguyenvuthanhha}]
	another nice one ( for fun )

   Find all functions $ f :\mathbb{R} \mapsto \mathbb{R}$ such that :

   $ f(x + f(y)) \ = \ f(x+y) + xf(y) - xy - x + 1  \ \ \forall x;y  \ \in \mathbb{R}$

  PS : it's my $300^{th}$ post
\end{solution}



\begin{solution}[by \href{https://artofproblemsolving.com/community/user/29428}{pco}]
	\begin{tcolorbox}another nice one ( for fun )

   Find all functions $ f :\mathbb{R} \mapsto \mathbb{R}$ such that :

   $ f(x + f(y)) \ = \ f(x+y) + xf(y) - xy - x + 1  \ \ \forall x;y  \ \in \mathbb{R}$

  PS : it's my $300^{th}$ post\end{tcolorbox}
Let $P(x,y)$ be the assertion $f(x+f(y))=f(x+y)+xf(y)-xy-x+1$

$P(f(x),y)$ $\implies$ $f(f(x)+f(y))=f(f(x)+y)+f(x)f(y)-f(x)y-f(x)+1$
$P(y,x)$ $\implies$ $f(y+f(x))=f(x+y)+yf(x)-xy-y+1$

Adding these two lines gives $f(f(x)+f(y))=f(x)f(y)-f(x)+f(x+y)-xy-y+2$
Swapping $x,y$ in this last line gives $f(f(x)+f(y))=f(x)f(y)-f(y)+f(x+y)-xy-x+2$

Subtracting these two lines gives $f(x)-x=f(y)-y$ and so $f(x)=x+a$

Plugging back in the original equation, we get $a=1$ and so the unique solution $\boxed{f(x)=x+1}$
\end{solution}



\begin{solution}[by \href{https://artofproblemsolving.com/community/user/43536}{nguyenvuthanhha}]
	Find all functions $ f :\mathbb{R} \mapsto \mathbb{R}$ such that :

   $ f(f(x) + y) \ = \ f(x+y) + xf(y) - xy - x + 1  \ \ \forall x;y  \ \in \mathbb{R}$
\end{solution}



\begin{solution}[by \href{https://artofproblemsolving.com/community/user/67223}{Amir Hossein}]
	\begin{tcolorbox}Find all functions $ f :\mathbb{R} \mapsto \mathbb{R}$ such that :

   $ f(f(x) + y) \ = \ f(x+y) + xf(y) - xy - x + 1  \ \ \forall x;y  \ \in \mathbb{R}$\end{tcolorbox}

Edit: :blush:

Find a function $f(x)$ defined for all real values of $x$ such that for all $x$,
\[f(x+ 2) -f(x) = x^2 + 2x + 4\]
And if $x \in [0, 2)$, then $ f(x) = x^2$.
\end{solution}



\begin{solution}[by \href{https://artofproblemsolving.com/community/user/29428}{pco}]
	\begin{tcolorbox}Find all functions $ f :\mathbb{R} \mapsto \mathbb{R}$ such that :

   $ f(f(x) + y) \ = \ f(x+y) + xf(y) - xy - x + 1  \ \ \forall x;y  \ \in \mathbb{R}$\end{tcolorbox}
Let $P(x,y)$ be the assertion $f(f(x)+y)=f(x+y)+xf(y)-xy-x+1$

$P(y,0)$ $\implies$ $f(f(y))=f(y)+yf(0)-y+1$ and so $xf(f(y))=xf(y)+xyf(0)-xy+x$
$P(x,f(y))$ $\implies$ $f(f(x)+f(y))=f(x+f(y))+xf(f(y))-xf(y)-x+1$
$P(y,x)$ $\implies$ $f(f(y)+x)=f(x+y)+yf(x)-xy-y+1$

Adding these three lines gives $f(f(x)+f(y))=f(x+y)+y(f(x)-1)+xy(f(0)-2)+2$
Swapping $x,y$ in this last line gives $f(f(x)+f(y))=f(x+y)+x(f(y)-1)+xy(f(0)-2)+2$

Subtracting these two lines gives $y(f(x)-1)=x(f(y)-1)$ and so (setting $y=1$) : $f(x)=x(f(1)-1)+1=ax+1$

Plugging back in the original equation, we get $a=1$ and so the unique solution $\boxed{f(x)=x+1}$
\end{solution}



\begin{solution}[by \href{https://artofproblemsolving.com/community/user/29428}{pco}]
	\begin{tcolorbox} Find a function $f(x)$ defined for all real values of $x$ such that for all $x$,
\[f(x+ 2) -f(x) = x^2 + 2x + 4\]
And if $x \in [0, 2)$, then $ f(x) = x^2$.\end{tcolorbox}

$f(x)=4\left\{\frac x2\right\}^2\left(\left\lfloor\frac x2\right\rfloor+1\right)$ $ + 4\left\{\frac x2\right\}\left\lfloor\frac x2\right\rfloor^2 $ $+\frac 43\left\lfloor\frac x2\right\rfloor\left(\left\lfloor\frac x2\right\rfloor^2+2\right)$
\end{solution}



\begin{solution}[by \href{https://artofproblemsolving.com/community/user/67223}{Amir Hossein}]
	\begin{tcolorbox}[quote="amparvardi"] Find a function $f(x)$ defined for all real values of $x$ such that for all $x$,
\[f(x+ 2) -f(x) = x^2 + 2x + 4\]
And if $x \in [0, 2)$, then $ f(x) = x^2$.\end{tcolorbox}

$f(x)=4\left\{\frac x2\right\}^2\left(\left\lfloor\frac x2\right\rfloor+1\right)$ $ + 4\left\{\frac x2\right\}\left\lfloor\frac x2\right\rfloor^2 $ $+\frac 43\left\lfloor\frac x2\right\rfloor\left(\left\lfloor\frac x2\right\rfloor^2+2\right)$\end{tcolorbox}

Wow !
How did you get it ? :maybe:  :w00t:
\end{solution}



\begin{solution}[by \href{https://artofproblemsolving.com/community/user/29428}{pco}]
	\begin{tcolorbox} How did you get it ? :maybe:  :w00t:\end{tcolorbox}
Just write $f(y+2k)-f(y+2k-2)=(y+2k-2)^2+2(y+2k-2)+4$

Sum these equalities for $k=1\to n$ and you get $f(y+2n)=f(y)+ny^2+2n^2y+\frac{4n(n^2+2)}3$

Write then $x=2\left\lfloor\frac x2\right\rfloor+2\left\{\frac x2\right\}$ and set in the previous expression :

$y= 2\left\{\frac x2\right\}$

$n= \left\lfloor\frac x2\right\rfloor$

So that $y+2n=x$ and $f(y)=y^2$ and you get :

$f(x)=4\left\{\frac x2\right\}^2\left(\left\lfloor\frac x2\right\rfloor+1\right)$ $ + 4\left\{\frac x2\right\}\left\lfloor\frac x2\right\rfloor^2 $ $+\frac 43\left\lfloor\frac x2\right\rfloor\left(\left\lfloor\frac x2\right\rfloor^2+2\right)$

Notice that combining integral and fractional parts, we can also write this as :

$f(x)=x^2+(2x-8)\left\lfloor\frac x2\right\rfloor\left\{\frac x2\right\}$ $+\frac 43\left\lfloor\frac x2\right\rfloor$ $\left(\left\lfloor\frac x2\right\rfloor-1\right)$ $\left(\left\lfloor\frac x2\right\rfloor-2\right)$

To be rigorous, a little attention need to be paid about negative values, but the result is OK.
\end{solution}



\begin{solution}[by \href{https://artofproblemsolving.com/community/user/67223}{Amir Hossein}]
	Thank you :) 

I'm posting the next:

Prove that the functional equations 
\[f(x + y) = f(x) + f(y),\]
\[\text{and } f(x + y + xy) = f(x) + f(y) + f(xy) \quad (\forall x, y \in \mathbb R)\]
are equivalent.
\end{solution}
*******************************************************************************
-------------------------------------------------------------------------------

\begin{problem}[Posted by \href{https://artofproblemsolving.com/community/user/67223}{Amir Hossein}]
	Let $f : [0, 1] \to [0, 1]$ satisfy $f(0) = 0, f(1) = 1$ and
\[f(x + y) - f(x) = f(x) - f(x - y)\]
for all $x, y \geq 0$ with $x - y, x + y \in [0, 1].$ Prove that $f(x) = x$ for all $x \in [0, 1].$
	\flushright \href{https://artofproblemsolving.com/community/c6h364198}{(Link to AoPS)}
\end{problem}



\begin{solution}[by \href{https://artofproblemsolving.com/community/user/72819}{Dijkschneier}]
	It is straightforward to see that f(nx)=nf(x) for all rational in [0,1].
Furthermore, we can prove that the function is continuous at 0 from the right.
We need the continuity on all the interval or the monotonicity to extend the result to [0,1].
\end{solution}



\begin{solution}[by \href{https://artofproblemsolving.com/community/user/29428}{pco}]
	\begin{tcolorbox}Let $f : [0, 1] \to [0, 1]$ satisfy $f(0) = 0, f(1) = 1$ and
\[f(x + y) - f(x) = f(x) - f(x - y)\]
for all $x, y \geq 0$ with $x - y, x + y \in [0, 1].$ Prove that $f(x) = x$ for all $x \in [0, 1].$\end{tcolorbox}
Let $P(x,y)$ be the assertion $f(x+y)-f(x)=f(x)-f(x-)$ $\forall x,y\ge 0$ with $x-y,x+y\in[0,1]$

$P(\frac{x+y}2,\frac {|x-y|}2)$ $\implies$ new assertion $Q(x,y)$ : $f(\frac{x+y}2)=\frac{f(x)+f(y)}2$ $\forall x,y\in[0,1]$

From there, we get with induction that :

$f(u+\frac{p}{2^q}(v-u))=f(u)+\frac p{2^q}(f(v)-f(u))$ $\forall u,v\in[0,1]$ and $p,q\in\mathbb Z$ such that $u+\frac{p}{2^q}(v-u)\in[0,1]$

Choosing then $u=0$ and $v=1$, this implies $f(x)=x$ $\forall x\in A=\{p2^{-q}$ where $q\in\mathbb N$ and $p\in\mathbb N\cap[0,2^q]\}$

Suppose now that $\exists u\in(0,1)$ such that $f(u)\ne u$, choosing $v\in A$ as near as we want of $u$ and appropriate $p,q$, we obviously get some $u+\frac{p}{2^q}(v-u)\in[0,1]$ such that $f(u)+\frac p{2^q}(f(v)-f(u))\notin[0,1]$ and so a contradiction.

Hence the result.
\end{solution}
*******************************************************************************
-------------------------------------------------------------------------------

\begin{problem}[Posted by \href{https://artofproblemsolving.com/community/user/82539}{mrbayat}]
	Find all functions $f:\mathbb{R}^{+}\rightarrow \mathbb{R}^{+}$ such that for all real numbers $x$ and $y$ with $x>y>0$, we have
\[f(x+y)-f(x-y)=4\sqrt{f(x)f(y)}.\]
Hint: first prove that $f(2x)=4f(x)$.
	\flushright \href{https://artofproblemsolving.com/community/c6h364296}{(Link to AoPS)}
\end{problem}



\begin{solution}[by \href{https://artofproblemsolving.com/community/user/86553}{YasserR}]
	\begin{tcolorbox}Find all $f:\mathbb{R}^{+}\rightarrow \mathbb{R}^{+}$ that:
                                                $\forall x>y>0\rightarrow f(x+y)-f(x-y)=4\sqrt{f(x)f(y)}$
(Suggestion: first prove $f(2x)=4f(x)$)\end{tcolorbox}
Hello,
$x>y>0$ or $x\ge y \ge0$??
\end{solution}



\begin{solution}[by \href{https://artofproblemsolving.com/community/user/82539}{mrbayat}]
	\begin{tcolorbox}[quote="mrbayat"]Find all $f:\mathbb{R}^{+}\rightarrow \mathbb{R}^{+}$ that:
                                                $\forall x>y>0\rightarrow f(x+y)-f(x-y)=4\sqrt{f(x)f(y)}$
(Suggestion: first prove $f(2x)=4f(x)$)\end{tcolorbox}
Hello,
$x>y>0$ or $x\ge y \ge0$??\end{tcolorbox}
$x>y>0$
\end{solution}



\begin{solution}[by \href{https://artofproblemsolving.com/community/user/72819}{Dijkschneier}]
	If $x\geq y \geq 0$, then we can prove that $f(x)=cx^2$ where c is an arbitrary constant.
\end{solution}



\begin{solution}[by \href{https://artofproblemsolving.com/community/user/29428}{pco}]
	\begin{tcolorbox}If $x\geq y \geq 0$, then we can prove that $f(x)=cx^2$ where c is an arbitrary constant.\end{tcolorbox}

It's easy to show that $f(x)$ is increasing and continuous, and then that it can be extended at $0$ with $f(0)=0$ and that the equility is true for $x\ge y\ge 0$
\end{solution}



\begin{solution}[by \href{https://artofproblemsolving.com/community/user/86553}{YasserR}]
	\begin{tcolorbox}
and then that it can be extended at $0$ with $f(0)=0$ \end{tcolorbox}
Do you want to extend $f$ at $0$? But $f:\mathbb{R}^+\rightarrow\mathbb{R}^+$ and $f$ is not undefined on $0$.
Please give more details about your idea...
\end{solution}



\begin{solution}[by \href{https://artofproblemsolving.com/community/user/29428}{pco}]
	\begin{tcolorbox}[quote="pco"]
and then that it can be extended at $0$ with $f(0)=0$ \end{tcolorbox}
Do you want to extend $f$ at $0$? But $f:\mathbb{R}^+\rightarrow\mathbb{R}^+$ and $f$ is not undefined on $0$.
Please give more details about your idea...\end{tcolorbox}
Let $P(x,y)$ be the assertion $f(x+y)-f(x-y)=4\sqrt{f(x)f(y)}$ $\forall x>y>0$

$P(x+y,y)$ $\implies$ $f(x+2y)>f(x)$ $\forall x,y>0$ and $f(x)$ is strictly increasing

So $\lim_{x\to 0^+}f(x)$ exists and let $a=\lim_{x\to 0+}f(x)$

$P(2x,x)$ $\implies$ $f(3x)-f(x)=4\sqrt{f(2x)f(x)}$ and, setting $x\to 0+$ in this equality, we get $0=4\sqrt{a^2}$ and so $a=0$

Let then $0<y<1$ so that $f(x+2y)-f(x)=4\sqrt{f(x+y)f(y)}<4\sqrt{f(x+1)f(y)}$. Setting $y\to 0^+$ in this expression, we get that $\lim_{y\to 0^+}f(x+2y)$ exists and is $f(x)$

Let then $0<y<x$ so that $P(x-y,y)$ $\implies$ $f(x)-f(x-2y)=4\sqrt{f(x-y)f(y)}<4\sqrt{f(x)f(y)}$. Setting $y\to 0^+$ in this expression, we get that $\lim_{y\to 0^+}f(x-2y)$ exists and is $f(x)$

So $f(x)$ is continuous.

Let then $g(x)$ from $\mathbb R^+\cup\{0\}\to\mathbb R^+\cup\{0\}$ defined as :
$g(0)=0$
$g(x)=f(x)$ $\forall x>0$

We obviously have $g(x+y)-g(x-y)=4\sqrt{g(x)g(y)}$ $\forall x>y>0$
Setting $y\to 0$ or $y\to x$, continuity  immediately implies that the expression is true $\forall x\ge y \ge 0$

I dont know if this is the quickest path towards the solution, but this is the answer to your request.
\end{solution}



\begin{solution}[by \href{https://artofproblemsolving.com/community/user/86553}{YasserR}]
	Hi PCO,
Thanks a lot for your explanation,Its very clear for me right know.
I think that it's the right way to the solution...
\end{solution}



\begin{solution}[by \href{https://artofproblemsolving.com/community/user/72819}{Dijkschneier}]
	Thank you pco.
After pco's work, we can end up like this :
Since f is continuous, setting f(0)=0, we can extend f at 0 by continuity.
So we have $f(x+y)-f(x-y)=4\sqrt{f(x)f(y)}$, ${\forall x,y \in \mathbb R^+\cup\{0\} }$ such that $x\geq y$.
For $x=y=0$, we have $f(0)=0$.
For x=y, we have $f(2x)=4f(x)$
By an immediate induction, we prove that $f(nx)=n^2f(x)$, $\forall n \in \mathbb N \cup\{0\}$.
Let $n = \frac{p}{q}$ be a rational number.
Then $f((qn)x)=(nq)^2 f(x)$ and $f(q(nx))=q^2f(nx)$, and therefore, $f(nx)=n^2f(x)$, $\forall n \in \mathbb Q^+\cup\{0\}$.
And since f is continuous, then it's also true $\forall n \in \mathbb R^+\cup\{0\}$.
Finally, putting $x=1$, we conclude that $f(n)=n^2f(1)$, which means $f(x)=cx^2$ for all nonnegatif real x and for an arbitrary constant c.
\end{solution}



\begin{solution}[by \href{https://artofproblemsolving.com/community/user/43015}{modularmarc101}]
	\begin{tcolorbox}
...
$P(x+y,y)$ $\implies$ $f(x+2y)>f(x)$ $\forall x,y>0$ and $f(x)$ is strictly increasing

So $\lim_{x\to 0^+}f(x)$ exists...\end{tcolorbox}

Why does 

$\text{f(x) is strictly increasing} \ \implies \ \lim_{x\to 0^+}f(x) \ \text{exists}$ ?
\end{solution}



\begin{solution}[by \href{https://artofproblemsolving.com/community/user/29428}{pco}]
	\begin{tcolorbox}[quote="pco"]
...
$P(x+y,y)$ $\implies$ $f(x+2y)>f(x)$ $\forall x,y>0$ and $f(x)$ is strictly increasing

So $\lim_{x\to 0^+}f(x)$ exists...\end{tcolorbox}

Why does 

$\text{f(x) is strictly increasing} \ \implies \ \lim_{x\to 0^+}f(x) \ \text{exists}$ ?\end{tcolorbox}

Any decreasing sequence $a_n$ whose limit is $0$ creates a strictly decreasing sequence of positive reals $f(a_n)$

Any strictly decreasing sequence of positive reals has a limit.

The fact that $f(x)$ is strictly increasing implies that all these limits are the same.
\end{solution}



\begin{solution}[by \href{https://artofproblemsolving.com/community/user/82539}{mrbayat}]
	\begin{tcolorbox}Thank you pco.
For $x=y=0$, we have $f(0)=0$.
For x=y, we have $f(2x)=4f(x)$
 .\end{tcolorbox}
$x>y>0$
\end{solution}



\begin{solution}[by \href{https://artofproblemsolving.com/community/user/29428}{pco}]
	\begin{tcolorbox}[quote="Dijkschneier"]Thank you pco.
For $x=y=0$, we have $f(0)=0$.
For x=y, we have $f(2x)=4f(x)$
 .\end{tcolorbox}
$x>y>0$\end{tcolorbox}

You should read the answers to your post.

I established that if a function $f(x)$ matches all your requirements (with $x>y>0$), then $\exists$ a fonction $g(x)$ from $\mathbb R^+\cup\{0\}\to \mathbb R^+\cup\{0\}$ $(g(x)$ is the extension of $f(x)$ with $g(0)=0$) which is continuous and matches all your requirements with $x\ge y\ge 0$.

And this is what Dijkschneier used.
\end{solution}
*******************************************************************************
-------------------------------------------------------------------------------

\begin{problem}[Posted by \href{https://artofproblemsolving.com/community/user/82539}{mrbayat}]
	Find all functions $f:\mathbb{R}\rightarrow \mathbb{R}$ such that
\[f(f(x)+y)=f(f(x)-y)+4f(x)y\]
holds for all reals $x$ and $y$.
	\flushright \href{https://artofproblemsolving.com/community/c6h364344}{(Link to AoPS)}
\end{problem}



\begin{solution}[by \href{https://artofproblemsolving.com/community/user/5729}{ehsan2004}]
	do the search, it's been posted before. Old Iran TST or Romania.
\end{solution}



\begin{solution}[by \href{https://artofproblemsolving.com/community/user/82539}{mrbayat}]
	\begin{tcolorbox}do the search, it's been posted before. Old Iran TST or Romania.\end{tcolorbox}
if you can please give me its link
\end{solution}



\begin{solution}[by \href{https://artofproblemsolving.com/community/user/29428}{pco}]
	\begin{tcolorbox}Find all functions $f:\mathbb{R}\rightarrow \mathbb{R}$ that
$f(f(x)+y)=f(f(x)-y)+4f(x)y$\end{tcolorbox}
It's a rather classical problem

Let $P(x,y)$ be the assertion $f(f(x)+y)=f(f(x)-y)+4f(x)y$

$f(x)=0$ $\forall x$ is a solution and we'll consider from now that $\exists u$ such that $f(u)\ne 0$

$P(u,\frac x{8f(u)})$ $\implies$ $2f(f(u)+\frac x{8f(u)})-2f(f(u)-\frac x{8f(u)})=x$ and so any real $x$ may be written $x=2f(a)-2f(b)$ for some $a,b$

$P(a,f(a)-2f(b))$ $\implies$ $f(2f(a)-2f(b))=f(2f(b))+4f(a)^2-8f(a)f(b)$
$P(b,f(b))$ $\implies$ $f(2f(b))=f(0)+4f(b)^2$

Adding these two lines, we get $f(2f(a)-2f(b))=f(0)+(2f(a)-2f(b))^2$

and so $f(x)=f(0)+x^2$ $\forall x$ since any real $x$ may be written $x=2f(a)-2f(b)$ for some $a,b$

And it is easy to check back that this indeed is a solution.

Hence the two solutions :
$f(x)=0$
$f(x)=x^2+c$
\end{solution}
*******************************************************************************
-------------------------------------------------------------------------------

\begin{problem}[Posted by \href{https://artofproblemsolving.com/community/user/43536}{nguyenvuthanhha}]
	Determine all functions $ f : \mathbb{R} \to\mathbb{R}$ such that
\[ f( x + y^2 + z^3 )   =  f(x) + (f(y))^2 + (f(z))^3, \quad \forall x,y,z  \in \mathbb{R}.\]
	\flushright \href{https://artofproblemsolving.com/community/c6h364408}{(Link to AoPS)}
\end{problem}



\begin{solution}[by \href{https://artofproblemsolving.com/community/user/29428}{pco}]
	\begin{tcolorbox}Determine all functions : $ f : \mathbb{R} \mapsto \mathbb{R}$ such that :

  $ f( x + y^2 + z^3 )  \ = \ f(x) + (f(y))^2 + (f(z))^3  \ \ \forall x;y;z  \ \ \in \mathbb{R}$\end{tcolorbox}
Let $P(x,y,z)$ be the assertion $f(x+y^2+z^3)=f(x)+f(y)^2+f(z)^3$

$P(0,0,0)$ $\implies$ $f(0)^2(f(0)+1)=0$ and so $f(0)\in\{0,-1\}$

1) If $f(0)=0$
=============
$P(0,0,z)$ $\implies$ $f(z^3)=f(z)^3$ and then $P(X,0,z)$ becomes $f(x+z^3)=f(x)+f(z^3)$ and so $f(x+y)=f(x)+f(y)$
And since $P(0,x,0)$ shows that $f(x^2)\ge 0$ $\forall x$, $f(x)$ is a non decreasing solution of Cauchy's equation.

So $f(x)=cx$ and plugging back in original equation, we get two solutions :
$f(x)=0$
$f(x)=x$

2) If $f(0)=-1$
==============
$P(0,0,z)$ $\implies$ $f(z^3)=f(z)^3$
$P(x,0,z)$ becomes then $f(x+z^3)=f(x)+f(z^3)+1$ and so $g(x+y)=g(x)+g(y)$ where $g(x)=f(x)+1$
$P(0,x,0)$ $\implies$ $f(x^2)=f(x)^2-2$ and so $g(x^2)\ge -1$ and $g(x)$ is a solution of Cauchy low bounded on a non empty interval, so is $cx$

And we got $f(x)=cx-1$
Plugging this in original equation, we get a unique solution $f(x)=-1$



Hence the three solutions :
$f(x)=0$
$f(x)=-1$
$f(x)=x$
\end{solution}
*******************************************************************************
-------------------------------------------------------------------------------

\begin{problem}[Posted by \href{https://artofproblemsolving.com/community/user/68025}{Pirkuliyev Rovsen}]
	Find all functions $f:\mathbb{R} \times \mathbb{R}\to\mathbb{R}$ such that \[f(f(x,z),f(z,y))=f(x,y)+z\] for all real numbers $x,y$, and $z$.
	\flushright \href{https://artofproblemsolving.com/community/c6h364515}{(Link to AoPS)}
\end{problem}



\begin{solution}[by \href{https://artofproblemsolving.com/community/user/29428}{pco}]
	\begin{tcolorbox}Find all function  $f: \mathbb{R \cdot R}\to\mathbb{R}$ such that $f(f(x,z),f(z,y))=f(x,y)+z$ for all real numbers x,y and z\end{tcolorbox}


Already posted here : http://www.artofproblemsolving.com/Forum/viewtopic.php?f=36&t=214287&start=0
(without any answer, unfortunately).
\end{solution}



\begin{solution}[by \href{https://artofproblemsolving.com/community/user/29428}{pco}]
	\begin{tcolorbox}Find all function  $f: \mathbb{R \cdot R}\to\mathbb{R}$ such that $f(f(x,z),f(z,y))=f(x,y)+z$ for all real numbers x,y and z\end{tcolorbox}
Let $P(x,y,z)$ be the assertion $f(f(x,z),f(z,y))=f(x,y)+z$

Let $s(x)=f(x,x)$ where "s" stands for "same"
Let $r(x)=f(0,x)$ where "r" stands for "right"
Let $l(x)=f(x,0)$ xhere "l" stands for "left"

$P(x,x,x)$ $\implies$ $s(s(x))=s(x)+x$ and so $s(x)$ is injective
$P(0,0,0)$ $\implies$ $s(s(0))=s(0)$ and so, since injective : $s(0)=0$ and so $r(0)=l(0)=0$ and $f(0,0)=0$

$P(x,0,0)$ $\implies$ $l(l(x))=l(x)$
$P(0,x,0)$ $\implies$ $r(r(x))=r(x)$

$P(x,y,0)$ $\implies$ $f(l(x),r(y))=f(x,y)$
Then, $l(l(x))=l(x)$ $\implies$ $f(x,y)=f(l(x),y)$
Same, $r(r(y))=r(y)$ $\implies$ $f(x,y)=f(x,r(y))$

$P(0,0,x)$ $\implies$ $f(r(x),l(x))=x$

Suppose $\exists u,v$ such that $l(u)=d(v)=a$. Then :
$l(l(u))=l(u)$ and so $l(a)=a$
$r(r(v))=r(v)$ and so $r(a)=a$
$a=f(r(a),l(a))$ and so $f(a,a)=a$ and so $s(a)=a$
$P(a,a,a)$ $\implies$ $s(s(a))=s(a)+a$ and so $a=2a$ and $a=0$
So $l(\mathbb R)\cap r(\mathbb R)=\{0\}$

Suppose now $\exists u,v$ such that $f(u,v)=0$
$P(u,v,u)$ $\implies$ $f(f(u,u),f(u,v))=f(u,v)+u$ $\implies$ $l(s(u))=u$ $\implies$ $u\in l(\mathbb R)$
$P(u,v,v)$ $\implies$ $f(f(u,v),f(v,v))=f(u,v)+v$ $\implies$ $r(s(v))=v$ $\implies$ $v\in r(\mathbb R)$
$l(u)=f(u,0)=f(u,f(u,v))=f(f(r(u),l(u)),f(l(u),v))=f(r(u),v)+l(u)$ $\implies$ $f(r(u),v)=0$ $\implies$ $r(u)\in l(\mathbb R)$
$r(v)=f(0,v)=f(f(u,v),v)=f(f(u,r(y)),f(r(v),l(v)))=f(u,l(v))+r(v)$ $\implies$ $f(u,l(v))=0$ $\implies$ $l(v)\in r(\mathbb R)$
So $r(u)\in l(\mathbb R)\cap r(\mathbb R)$ and so $r(u)=0$ 
and so $u=f(r(u),l(u))=r(l(u))$ and $r(u)=r(r(l(u)))=r(l(u))=u$ and so $u=0$
Same : $l(v))\in l(\mathbb R)\cap r(\mathbb R)$ and so $l(v)=0$
and so $v=f(r(v),l(v))=l(r(v))$ and so $l(v)=l(l(r(v)))=l(r(v))=v$ and so $v=0$

So $f(x,y)=0$ $\iff$ $x=y=0$

Then $P(x,y,-f(x,y))$ $\implies$ $f(f(x,-f(x,y)),f(-f(x,y),y))=0$ and so $f(x,-f(x,y))=f(-f(x,y),y)=0$ and so $x=y=0$, impossible

So no solution for this equation. :( :(
\end{solution}



\begin{solution}[by \href{https://artofproblemsolving.com/community/user/68025}{Pirkuliyev Rovsen}]
	Thanks you Patrick 
\end{solution}
*******************************************************************************
-------------------------------------------------------------------------------

\begin{problem}[Posted by \href{https://artofproblemsolving.com/community/user/77832}{abhinavzandubalm}]
	Does there exist a function $f: \mathbb N \to \mathbb N$ such that for all integers $n \geq 2$,
\[ f(f(n-1)) = f (n+1) - f(n)\, ?\]
	\flushright \href{https://artofproblemsolving.com/community/c6h364563}{(Link to AoPS)}
\end{problem}



\begin{solution}[by \href{https://artofproblemsolving.com/community/user/84155}{truongtansang89}]
	\begin{tcolorbox}Does There Exist A Function 

$ f : N \to N $ 

$ \forall n \ge 2$

$ f(f(n-1)) = f (n+1) - f(n) $\end{tcolorbox}

$ f(x) = 0$ for all $ x \in N $ is a solution.
\end{solution}



\begin{solution}[by \href{https://artofproblemsolving.com/community/user/87348}{Sansa}]
	\begin{tcolorbox}
$ f(x) = 0$ for all $ x \in N $ is a solution.\end{tcolorbox}
I think it said $ f : N \to N $ ...
\end{solution}



\begin{solution}[by \href{https://artofproblemsolving.com/community/user/29428}{pco}]
	\begin{tcolorbox}Does There Exist A Function 

$ f : N \to N $ 

$ \forall n \ge 2$

$ f(f(n-1)) = f (n+1) - f(n) $\end{tcolorbox}
$f(n+1)-f(n)\ge 1$ $\forall n\ge 2$ and so $f(n)\ge f(2)+n-2\ge n-1$ $\forall n\ge 3$

So : $\forall n\ge 5$ : $f(n-1)\ge n-2\ge 3$ and so $f(f(n-1))\ge f(n-1)-1\ge n-3$ and so $f(n+1)-f(n)\ge n-3$

Adding these lines for $n=5,6,7$, we get $f(8)-f(5)\ge 9$ and so $f(8)\ge 10$. Let then $a=f(8)\ge 10$

Adding then the lines $f(f(n-1))=f(n+1)-f(n)$ for $n=2\to a-1$, we get $f(a)-f(2)=\sum_{k=1}^{a-2}f(f(k))$

And, since $a\ge 10$, we can write $f(a)-f(2)=f(f(8))+\sum_{k=1,k\ne 8}^{a-2}f(f(k))$

and so, since $f(8)=a$, this becomes : $-f(2)=\sum_{k=1,k\ne 8}^{a-2}f(f(k))$, clearly impossible since $LHS<0$ while $RHS>0$

And so no solution.
\end{solution}



\begin{solution}[by \href{https://artofproblemsolving.com/community/user/87348}{Sansa}]
	Let $f(n_{0})$ has the minimum value of the function (because $f : {\mathbb{N}}\to{\mathbb{N}}$ we can suppose that). It easily can be considered that ${1}\leqslant{f(n_{0})}\leqslant{f(n)}$. Now we have two situation:

1)${n_{0}}\geqslant{3}$:

We have $f(f(n-1)) +f (n) = f (n+1)$ and by putting ${n}\to{n_{0}-1}$ we have:
$f(f(n_{0}-2)) + f(n_{0}-1) = f(n_{0})$ but we know that ${f(f(n_{0}-2))}\geqslant{f(n_{0})}$ and ${f(n_{0}-1)}\geqslant{f(n_{0})}$ $\Longrightarrow$ ${f(n_{0}) = f(f(n_{0}-2)) + f(n_{0}-1)}\geqslant{2f(n_{0})}$ $\Longrightarrow$ ${f(n_{0})}\leqslant{0}$ $\Longrightarrow$ there not exist such a function.

2)$f(2)$ has the minimum value:

Let $f(n-1) = a$ we have $f(a) = f(n+1) - f(n)$ and we have ${f(a)}\in{\mathbb{N}}$ so ${f(n+1) - f(n)}\geqslant{1}$ $\Longrightarrow$ ${f(n+1)}$>${f(n)}$ $\Longrightarrow$ ${f(2)}$>${f(1)}$ but we supposed $f(2)$ has the minimum value. $\Longrightarrow$ there not exist such a function.
\end{solution}



\begin{solution}[by \href{https://artofproblemsolving.com/community/user/29428}{pco}]
	\begin{tcolorbox}Let $f(n_{0})$ has the minimum value of the function (because $f : {\mathbb{N}}\to{\mathbb{N}}$ we can suppose that). It easily can be considered that ${1}\leqslant{f(n_{0})}\leqslant{f(n)}$. Now we have two situation:

1)${n_{0}}\geqslant{3}$:

We have $f(f(n-1)) +f (n) = f (n+1)$ and by putting ${n}\to{n_{0}-1}$ we have:
$f(f(n_{0}-2)) + f(n_{0}-1) = f(n_{0})$ but we know that ${f(f(n_{0}-2))}\geqslant{f(n_{0})}$ and ${f(n_{0}-1)}\geqslant{f(n_{0})}$ $\Longrightarrow$ ${f(n_{0}) = f(f(n_{0}-2)) + f(n_{0}-1)}\geqslant{2f(n_{0})}$ $\Longrightarrow$ ${f(n_{0})}\leqslant{0}$ $\Longrightarrow$ there not exist such a function.

2)$f(2)$ has the minimum value:

Let $f(n-1) = a$ we have $f(a) = f(n+1) - f(n)$ and we have ${f(a)}\in{\mathbb{N}}$ so ${f(n+1) - f(n)}\geqslant{1}$ $\Longrightarrow$ ${f(n+1)}$>${f(n)}$ $\Longrightarrow$ ${f(2)}$>${f(1)}$ but we supposed $f(2)$ has the minimum value. $\Longrightarrow$ there not exist such a function.\end{tcolorbox}

Two remarks : 

1) If $n_0<3$, we have two cases : $n_0=2$ and $n_0=1$. You looked only at $n_0=2$

b) in 2), you say $f(n+1)>f(n)$ but you forget that this is only true for $n\ge 2$, and so you cant apply $n=1$ to this inequality.
\end{solution}



\begin{solution}[by \href{https://artofproblemsolving.com/community/user/87348}{Sansa}]
	\begin{tcolorbox}
Two remarks : 

1) If $n_0<3$, we have two cases : $n_0=2$ and $n_0=1$. You looked only at $n_0=2$

b) in 2), you say $f(n+1)>f(n)$ but you forget that this is only true for $n\ge 2$, and so you cant apply $n=1$ to this inequality.\end{tcolorbox}
For the first one in the problem we have $\forall{n}\geqslant{2}$ so it doesn't need to check $n=1$.
But for the second one I have no idea. do anybody have  any idea about that???
\end{solution}
*******************************************************************************
-------------------------------------------------------------------------------

\begin{problem}[Posted by \href{https://artofproblemsolving.com/community/user/67223}{Amir Hossein}]
	Given two positive real numbers $a$ and $b$, suppose that a mapping $f : \mathbb R^+ \to \mathbb R^+$ satisfies the functional equation
\[f(f(x)) + af(x) = b(a + b)x.\]
Prove that there exists a unique solution of this equation.
	\flushright \href{https://artofproblemsolving.com/community/c6h364768}{(Link to AoPS)}
\end{problem}



\begin{solution}[by \href{https://artofproblemsolving.com/community/user/29428}{pco}]
	\begin{tcolorbox}Given two positive real numbers $a$ and $b$, suppose that a mapping $f : \mathbb R^+ \to \mathbb R^+$ satisfies the functional equation
\[f(f(x)) + af(x) = b(a + b)x.\]
Prove that there exists a unique solution of this equation.\end{tcolorbox}

$a+2b>0$ and we get thru simple induction : $f^{[n]}(x)=\frac{((a+b)x+f(x))b^n+(bx-f(x))(-a-b)^n}{a+2b}$

If, for some $x$, $f(x)-bx\ne 0$, we get that, for some $n$ great enough, $f^{[n]}(x)<0$, which is impossible. 

Hence the unique solution : $f(x)=bx$ which indeed is a solution.
\end{solution}
*******************************************************************************
-------------------------------------------------------------------------------

\begin{problem}[Posted by \href{https://artofproblemsolving.com/community/user/75241}{daigiaga1994}]
	Find all functions $f: \mathbb R \to \mathbb R$ such that for all reals $x$ and $y$,
\[f(f(y)(x+1))=y(f(x)+1).\]
	\flushright \href{https://artofproblemsolving.com/community/c6h364886}{(Link to AoPS)}
\end{problem}



\begin{solution}[by \href{https://artofproblemsolving.com/community/user/29428}{pco}]
	\begin{tcolorbox}\begin{bolded}Problem:\end{bolded} Find all function $f:R->R$ such that:
$f(f(y)(x+1))=y(f(x)+1)$  for all real numbers $x,y$\end{tcolorbox}
Let $P(x,y)$ be the assertion $f(f(y)(x+1))=y(f(x)+1)$

$P(-1,0)$ $\implies$ $f(0)=0$
$P(0,x)$ $\implies$ $f(f(x))=x$ and $f(x)$ is a bijection.
Let then $u=f(1)\ne 0$ such that $f(u)=1$

$P(x,u)$ $\implies$ $f(x+1)=u(f(x)+1)$
$P(x,f(y))$ $\implies$ $f(y(x+1))=f(y)(f(x)+1)=\frac 1uf(x+1)f(y)$ and so new assertion $Q(x,y)$ : $f(xy)=\frac 1uf(x)f(y)$

Let $y\ne 0$ : $Q(\frac xy+1,y)$ $\implies$ $f(x+y)=\frac 1uf(\frac xy+1)f(y)=$ $(f(\frac xy)+1)f(y)=f(\frac xy)f(y)+f(y)=uf(x)+f(y)$
And so new assertion $R(x,y)$ : $f(x+y)=uf(x)+f(y)$ $\forall y\ne 0$

Comparing $R(1,2)$ and $R(2,1)$ implies $u=1$ and then $R(x,y)$ is true also for $y=0$ and we get :

$f(xy)=f(x)f(y)$
$f(x+y)=f(x)+f(y)$

This classicaly gives $f(x)=0$ (which is not a solution of original equation) or $\boxed{f(x)=x}$ (which is indeed a solution)
\end{solution}



\begin{solution}[by \href{https://artofproblemsolving.com/community/user/75241}{daigiaga1994}]
	Thank your, pco!
\end{solution}
*******************************************************************************
-------------------------------------------------------------------------------

\begin{problem}[Posted by \href{https://artofproblemsolving.com/community/user/86714}{vlm}]
	Find all functions $ f:\mathbb{R}\rightarrow\mathbb{R} $ such that
\[(x+y)( f(x) - f(y) ) = (x-y)f(x+y)\]
for all $x,y \in \mathbb R$.
	\flushright \href{https://artofproblemsolving.com/community/c6h365241}{(Link to AoPS)}
\end{problem}



\begin{solution}[by \href{https://artofproblemsolving.com/community/user/29428}{pco}]
	\begin{tcolorbox}Find all functions $ f:\mathbb{R}\rightarrow\mathbb{R} $ such that
(x+y)( f(x) - f(y) ) = (x-y)f(x+y)\end{tcolorbox}
Let $P(x,y)$ be the assertion $f(x+y)=\frac{x+y}{x-y}f(x)-\frac{x+y}{x-y}f(y)$ $\forall x\ne y$

$P(x+1,2)$ $\implies$ $f(x+3)=\frac{x+3}{x-1}f(x+1)-\frac{x+3}{x-1}f(2)$ $\forall x\ne 1$

$P(x,1)$ $\implies$ $f(x+1)=\frac{x+1}{x-1}f(x)-\frac{x+1}{x-1}f(1)$ $\forall x\ne 1$

And so $f(x+3)=\frac{x+3}{x-1}\frac{x+1}{x-1}f(x)$ $-\frac{x+3}{x-1}\frac{x+1}{x-1}f(1)$ $-\frac{x+3}{x-1}f(2)$ $\forall x\ne 1$

$P(x+2,1)$ $\implies$ $f(x+3)=\frac{x+3}{x+1}f(x+2)-\frac{x+3}{x+1}f(1)$ $\forall x\ne -1$

$P(x,2)$ $\implies$ $f(x+2)=\frac{x+2}{x-2}f(x)-\frac{x+2}{x-2}f(2)$ $\forall x\ne 2$

And so $f(x+3)=\frac{x+3}{x+1}\frac{x+2}{x-2}f(x)$ $-\frac{x+3}{x+1}\frac{x+2}{x-2}f(2)$ $-\frac{x+3}{x+1}f(1)$ $\forall x\ne -1,2$

Equating these two expressions, we get $f(x)=ax^2+bx$ $\forall x\notin\{-3,-1,1,2\}$ for some $a,b\in\mathbb R$
Then $P(3,-6)$, $P(3,-4)$, $P(3,-2)$ and $P(4,-2)$ imply $f(x)=ax^2+bx$ $\forall x$

And it is immediate to check back that this indeed is a solution.

Hence the answer : $\boxed{f(x)=ax^2+bx}$ $\forall x$
\end{solution}
*******************************************************************************
-------------------------------------------------------------------------------

\begin{problem}[Posted by \href{https://artofproblemsolving.com/community/user/73172}{steve5030}]
	Let $g(x)=\arctan x $. Find the sum \[\sum_{n=0}^{\infty} g\left( \frac{2}{n^2} \right).\]
	\flushright \href{https://artofproblemsolving.com/community/c6h365407}{(Link to AoPS)}
\end{problem}



\begin{solution}[by \href{https://artofproblemsolving.com/community/user/29428}{pco}]
	\begin{tcolorbox}let f(x)=tanx 

g(x)=arc tan x 

sigma (n=0 to inf) g(2\/n^2)? (n is natural number)\end{tcolorbox}

How is defined $g(\frac 2{n^2})$ when $n=0$ ?
\end{solution}



\begin{solution}[by \href{https://artofproblemsolving.com/community/user/62473}{-Elixir-}]
	Indeed it is undefined, but I presume he was thinking of $\frac{\pi}{2},$ seeing as the singularity at $n = 0$ is removable due to the existence of a two-sided limit.
\end{solution}



\begin{solution}[by \href{https://artofproblemsolving.com/community/user/29428}{pco}]
	\begin{tcolorbox}let f(x)=tanx 

g(x)=arc tan x 

sigma (n=0 to inf) g(2\/n^2)? (n is natural number)\end{tcolorbox}

Ok, then :

$\forall n>1$ : $g(\frac 2{n^2})=g(\frac 1{n-1})-g(\frac 1{n+1})$

So $\sum_{n=2}^{+\infty}g(\frac 2{n^2})=g(1)+g(\frac 12)=\frac{\pi}4+\arctan(\frac 12)$ (a little more rigor about convergences of sums would be needed before swapping elements of the sum, but I'm a bit lazy :) )

So $\sum_{n=1}^{+\infty}g(\frac 2{n^2})=\frac{\pi}4+\arctan(2)+\arctan(1\/2)$

So $\boxed{\sum_{n=1}^{+\infty}g(\frac 2{n^2})=\frac{3\pi}4}$

And, with a kind of "convention" $g(+\infty)=\frac{\pi}2$, we could get $\sum_{n=0}^{+\infty}g(\frac 2{n^2})=\frac{5\pi}4$
\end{solution}



\begin{solution}[by \href{https://artofproblemsolving.com/community/user/73172}{steve5030}]
	\begin{tcolorbox}[quote="steve5030"]let f(x)=tanx 

g(x)=arc tan x 

sigma (n=0 to inf) g(2\/n^2)? (n is natural number)\end{tcolorbox}

Ok, then :

$\forall n>1$ : $g(\frac 2{n^2})=g(\frac 1{n-1})-g(\frac 1{n+1})$

So $\sum_{n=2}^{+\infty}g(\frac 2{n^2})=g(1)+g(\frac 12)=\frac{\pi}4+\arctan(\frac 12)$ (a little more rigor about convergences of sums would be needed before swapping elements of the sum, but I'm a bit lazy :) )

So $\sum_{n=1}^{+\infty}g(\frac 2{n^2})=\frac{\pi}4+\arctan(2)+\arctan(1\/2)$

So $\boxed{\sum_{n=1}^{+\infty}g(\frac 2{n^2})=\frac{3\pi}4}$

And, with a kind of "convention" $g(+\infty)=\frac{\pi}2$, we could get $\sum_{n=0}^{+\infty}g(\frac 2{n^2})=\frac{5\pi}4$\end{tcolorbox}

----------------------------------------------------------------------------------------------------------------------------------------------------------------------------

can you prove that arctanA+arctan1\/A=pi\/2 ?(A>0)

I tried it but when I used the tangent additive rule, the formula said it was undefined. -'-'

Anyways , Thanks for your help. I really appreciate it.
\end{solution}



\begin{solution}[by \href{https://artofproblemsolving.com/community/user/29428}{pco}]
	\begin{tcolorbox} can you prove that arctanA+arctan1\/A=pi\/2 ?(A>0)

I tried it but when I used the tangent additive rule, the formula said it was undefined. -'-'\end{tcolorbox}

Let $x\in(0,\frac{\pi}2)$ : 
$\tan(\frac{\pi}2-x)=\frac 1{\tan(x)}=u$
So, $x=\arctan(\frac 1u)$ (since $x\in(-\frac{\pi}2,\frac{\pi}2)$)
And $\frac{\pi}2-x=\arctan(u)$ (since $\frac{\pi}2-x\in(-\frac{\pi}2,\frac{\pi}2)$)

So $\arctan(u)+\arctan(\frac 1u)=\frac{\pi}2$ $\forall u>0$
\end{solution}
*******************************************************************************
-------------------------------------------------------------------------------

\begin{problem}[Posted by \href{https://artofproblemsolving.com/community/user/67223}{Amir Hossein}]
	Let $f(x)$ be a periodic function of period $T > 0$ defined over $\mathbb R$. Its first derivative is continuous on $\mathbb R$. Prove that there exist $x, y \in [0, T )$ such that $x \neq y$ and
\[f(x)f'(y)=f'(x)f(y).\]
	\flushright \href{https://artofproblemsolving.com/community/c6h365514}{(Link to AoPS)}
\end{problem}



\begin{solution}[by \href{https://artofproblemsolving.com/community/user/29428}{pco}]
	\begin{tcolorbox}Let $f(x)$ be a periodic function of period $T > 0$ defined over $\mathbb R$. Its first derivative is continuous on $\mathbb R$. Prove that there exist $x, y \in [0, T )$ such that $x \neq y$ and
\[f(x)f'(y)=f'(x)f(y).\]\end{tcolorbox}

If $f(x)$ is constant, choose $T=2$ and $x=0$ and $y=1$

If $f(x)$ is not constant, there exists in $[0,T)$ at least one point "$x=a$" where $f(x)$ is maximum  and one point "$x=b\ne a$" where $f(x)$ is minimum.

Then just choose $x=a$ and $y=b$
\end{solution}
*******************************************************************************
-------------------------------------------------------------------------------

\begin{problem}[Posted by \href{https://artofproblemsolving.com/community/user/67949}{aktyw19}]
	Let $a>1$ be a real number. Find the maximum and minimum value of the function
\[f(x)=\sqrt {a + \sin x} + \sqrt {a + \cos x}.\]
	\flushright \href{https://artofproblemsolving.com/community/c6h365566}{(Link to AoPS)}
\end{problem}



\begin{solution}[by \href{https://artofproblemsolving.com/community/user/86553}{YasserR}]
	Hello,
we see that $f$ is periodic $f(x)=f(x+2\pi)$
in the interval $[0,2\pi]$
the max of $f$ is for $x=\frac{\pi}{4}$ and the min is for $x=\pi$




____________________
$Yasser.R$
\end{solution}



\begin{solution}[by \href{https://artofproblemsolving.com/community/user/36230}{varunrocks}]
	You mean that the max occurs when x=pi\/2 or 0.
And min when x=pi, 3pi\/2.
\end{solution}



\begin{solution}[by \href{https://artofproblemsolving.com/community/user/86553}{YasserR}]
	\begin{tcolorbox}You mean that the max occurs when x=pi\/2 or 0.
And min when x=pi, 3pi\/2.\end{tcolorbox}
Hello varunrocks,

In the interval $[0,2\pi]$
the max occurs only when $x=\frac{\pi}{4}$
and the min occurs when $x=\pi$ and $x=\frac{3\pi}{2}$ as you said.
$f(\frac{\pi}{2})$ is not the max.



_________________
$Yasser.R$
\end{solution}



\begin{solution}[by \href{https://artofproblemsolving.com/community/user/29428}{pco}]
	\begin{tcolorbox}[quote="varunrocks"]You mean that the max occurs when x=pi\/2 or 0.
And min when x=pi, 3pi\/2.\end{tcolorbox}
Hello varunrocks,

In the interval $[0,2\pi]$
the max occurs only when $x=\frac{\pi}{4}$
and the min occurs when $x=\pi$ and $x=\frac{3\pi}{2}$ as you said.
$f(\frac{\pi}{2})$ is not the max.\end{tcolorbox}

$f'(x)=\frac{\cos(x)}{2\sqrt{a+\sin(x)}}-\frac{\sin(x)}{2\sqrt{a+\cos(x)}}$

And obviously $f'(\pi)\ne 0$ and $f'(\frac{3\pi}2)\ne 0$

So "min occurs when $x=\pi$ and $x=\frac{3\pi}{2}$" is obviously wrong.
\end{solution}



\begin{solution}[by \href{https://artofproblemsolving.com/community/user/73838}{Ovchinnikov Denis}]
	\begin{tcolorbox}Find max, min of
$f(x)=\sqrt {a + \sin x} + \sqrt {a + \cos x}$
a>1\end{tcolorbox}

$\sqrt{a+sinx}+\sqrt{a+cosx} \leq \sqrt{2(2a+sinx+cosx)} \leq \sqrt {4a+2\sqrt{2}}$ with equality for $x= \frac{ \pi }{4} +2 \pi n$
So maximum we found :)

I think, that minimum is $\sqrt{a}+ \sqrt{a-1}$ :)
\end{solution}



\begin{solution}[by \href{https://artofproblemsolving.com/community/user/66201}{basketball9}]
	Agreed.  Nice.

Using AM-GM...we get the result when $x=\pi\/2$
\end{solution}



\begin{solution}[by \href{https://artofproblemsolving.com/community/user/86553}{YasserR}]
	Hello everybody,
See what I found using the graph (for $a\not=1$).
And for Basketball9, please tell me how did proceed using AM-GM...
\end{solution}



\begin{solution}[by \href{https://artofproblemsolving.com/community/user/29428}{pco}]
	\begin{tcolorbox}Hello everybody,
See what I found using the graph (for $a\not=1$).
And for Basketball9, please tell me how did proceed using AM-GM...\end{tcolorbox}

Be careful when you take examples with graphs based on pecular values. Draw the graph with $a=1.0000001$, for example : the drawing is quite different. :)
\end{solution}



\begin{solution}[by \href{https://artofproblemsolving.com/community/user/73838}{Ovchinnikov Denis}]
	I draw graph and see, that for $a$ not very big (near $a \leq \sqrt{2}$) min holds for $x = \pi $ and for $a$ grater, that this constant, minimum is for $x = \frac {5 \pi}{4}$.
\end{solution}



\begin{solution}[by \href{https://artofproblemsolving.com/community/user/29428}{pco}]
	\begin{tcolorbox}I draw graph and see, that for $a$ not very big (near $a \leq \sqrt{2}$) min holds for $x = \pi $ and for $a$ grater, that this constant, minimum is for $x = \frac {5 \pi}{4}$.\end{tcolorbox}
I'm sure you have seen this.
But what you have seen is wrong.
As I already said, $f'(\pi)\ne 0$ and so $f(\pi)$ is neither a local minimum, neither a local maximum.
\end{solution}



\begin{solution}[by \href{https://artofproblemsolving.com/community/user/73838}{Ovchinnikov Denis}]
	\begin{tcolorbox}[quote="Ovchinnikov Denis"]I draw graph and see, that for $a$ not very big (near $a \leq \sqrt{2}$) min holds for $x = \pi $ and for $a$ grater, that this constant, minimum is for $x = \frac {5 \pi}{4}$.\end{tcolorbox}
I'm sure you have seen this.
But what you have seen is wrong.
As I already said, $f'(\pi)\ne 0$ and so $f(\pi)$ is neither a local minimum, neither a local maximum.\end{tcolorbox}
Yes, not $\sqrt{2}$  :blush: 
But this is for $a=1,001$ and i think, that minimum is near $ \pi$ :)
\end{solution}



\begin{solution}[by \href{https://artofproblemsolving.com/community/user/29428}{pco}]
	\begin{tcolorbox} i think, that minimum is near $ \pi$ \end{tcolorbox}

Ahhhhhh!  "minimum is near $\pi$" is quite different from "minimum is at $\pi$"  :)
\end{solution}



\begin{solution}[by \href{https://artofproblemsolving.com/community/user/73838}{Ovchinnikov Denis}]
	\begin{tcolorbox}[quote="Ovchinnikov Denis"] i think, that minimum is near $ \pi$ \end{tcolorbox}

Ahhhhhh!  "minimum is near $\pi$" is quite different from "minimum is at $\pi$"  :)\end{tcolorbox}

Yes, of course, it is different :)
But how we can find $x$ $near$ $\pi$ ?? 
So I think, that we can not solve problem for $a \approx 1 $  :(
\end{solution}



\begin{solution}[by \href{https://artofproblemsolving.com/community/user/29428}{pco}]
	$f(x)=\sqrt{a+\sin(x)}+\sqrt{a+\cos x}$ with $a>1$

$f'(x)=\frac{\cos(x)}{2\sqrt{a+\sin(x)}}-\frac{\sin(x)}{2\sqrt{a+\cos(x)}}$

$f(x+2\pi)=f(x)$ and so we can study zeroes and sign of $f'(x)$ only in $[0,2\pi)$

1) zeroes of $f'(x)$
==================
The zeroes of $f'(x)$ are the solutions of $\sin(x)\sqrt{a+\sin(x)}=\cos(x)\sqrt{a+\cos x}$

$\iff$ $\sin(x)$ and $\cos(x)$ have same signs and $\sin^2(x)(a+\sin(x))=\cos^2(x)(a+\cos x)$

$\iff$ $\sin(x)$ and $\cos(x)$ have same signs and $(\sin(x)-\cos(x))(1+\sin(x)\cos(x)+a(\sin(x)+\cos(x)))=0$

$\sin(x)=\cos(x)$ gives at least two zeroes : $x\in\{\frac{\pi}4,\frac{5\pi}4\}$

It remains to solve $\sin(x)$ and $\cos(x)$ have same signs and $1+\sin(x)\cos(x)+a(\sin(x)+\cos(x))=0$

This means $\sin(x)\le 0$ and $\cos(x)\le 0$ and $1+\sin(x)\cos(x)+a(\sin(x)+\cos(x))=0$

Let then $\sin(x)=-u$ and $\cos(x)=-\sqrt{1-u^2}$ with $u\in[0,1]$ and we are looking sor solitions of 

$u\in[0,1]$ and $1+u\sqrt{1-u^2}-a(u+\sqrt{1-u^2})=0$

$\iff$ $u\in[0,1]$ and $1-au=(a-u)\sqrt{1-u^2}$

$\iff$ $0\le u\le \frac 1a$ and $(1-au)^2=(a-u)^2(1-u^2)$

$\iff$ $0\le u\le \frac 1a$ and $u^4 -2au^3 +(2a^2-1)u^2 +1-a^2=0$

Studying $P(x)=x^4-2ax^3+(2a^2-1)x^2+1-a^2$ is rather complex, IMHO :

$P(0)=1-a^2<0$
$P(1)=(a-1)^2>0$
$P(\frac 1a)=\frac{(1-a^2)^3}{a^4}<0$
$P(\frac{\sqrt 2}2)=\frac 34-\frac a{\sqrt 2}$

So $P(x)$ has 1 or 3 roots in $(0,1)$
But, if there is only one root, it is $>\frac 1a$ and so is not a root of $f'(x)$
So it remains to study the possibility of 3 roots in $(0,1)$ :

$P'(x)=4x^3-6ax^2+(4a^2-2)x=\frac x4((4x-3a)^2 - (8-7a^2))$

So, in order to have 3 roots, we need $a^2\le \frac 87$ and then $P'(x)$ has a local maximum in $[0,1]$ for $x_M=\frac{3a-\sqrt{8-7a^2}}4$ and we are looking for situations where $P(x_M)\ge 0$

$P(x_M)=\frac{13a^4-36a^2+24+a(8-7a^2)\sqrt{8-7a^2}}{32}$

The zeroes of $P(x_M)$ are when $-13a^4+36a^2-24=a(8-7a^2)\sqrt{8-7a^2}$

$\iff$ $\frac 87 \ge a^2\ge \frac {18-2\sqrt 3}{13}$ and $(-13a^4+36a^2-24)^2=a^2(8-7a^2)^3$

$\iff$ $\frac 87 \ge a^2\ge \frac {18-2\sqrt 3}{13}$ and $(a^2-1)^3(8a^2-9)=0$ and so $a^2=\frac 98$

So :

for $a^2<\frac 98$, $P(x_M)>0$. and $P(\frac {\sqrt 2}2)>0$ and so $f'(x)$ has four roots : ${\frac{\pi}4 < \pi <x_1<\frac{5\pi}4 < x_2 < \frac 3\pi}2$

for $a^2=\frac 98$, $x_M=\frac{\sqrt 2}2$ and $P(x_M)=0$ and so $f'(x)$ has two roots : $\frac{\pi}4$ and $\frac{5\pi}4$ (triple root)

for $a^2>\frac 98$, $P(x_M)<0$ and $f'(x)$ has only two roots $\frac{\pi}4$ and $\frac{5\pi}4$.


2) conclusions on min-max
=========================

If $a\ge \frac{3\sqrt 2}4$ :
 max is reached for $x=\frac{\pi}4$ and is $2\sqrt{a+\frac{\sqrt 2}2}$
 min is reached for $x=\frac{5\pi}4$ and is $2\sqrt{a-\frac{\sqrt 2}2}$

If $a< \frac{3\sqrt 2}4$ :
The polynomial $P(x)=x^4-2ax^3+(2a^2-1)x^2+1-a^2$ has exactly four roots $u_1<0<u_2<\frac{\sqrt 2}2 < u_3 < \frac 1a < u_4 < 1$
Then $f'(x)$ has four roots in $[0,2\pi)$ :
$x_1=\frac{\pi}4$ which is a local maximum of $f(x)$

$x_2=\pi+\arcsin(u_2)\in(\pi,\frac{5\pi}4)$ which is a local minimum

$x_3=\frac{5\pi}4$ which is a local maximum

$x_4=\pi+\arcsin(u_3)\in(\pi,\frac{5\pi}4)$ which is a local minimum

It's immediate to see that $f(x_3)<f(x_1)$ and that the global maximum is $f(\frac{\pi}4)=2\sqrt{a+\frac{\sqrt 2}2}$

It's also nearly immediate to see that $u_3=\sqrt{1-u_2^2}$ and so $x_4=\frac{3\pi}2-\arcsin(u_2)$ and that $f(x_2)=f(x_4)$ is the global minimum.

This global minimum is $\sqrt{a+u_2}+\sqrt{a+u_3}$
\end{solution}



\begin{solution}[by \href{https://artofproblemsolving.com/community/user/73838}{Ovchinnikov Denis}]
	\begin{tcolorbox}[quote="pco"]$f(x)=\sqrt{a+\sin(x)}+\sqrt{a+\cos x}$ with $a>1$

$f'(x)=\frac{\cos(x)}{2\sqrt{a+\sin(x)}}-\frac{\sin(x)}{2\sqrt{a+\cos(x)}}$

$f(x+2\pi)=f(x)$ and so we can study zeroes and sign of $f'(x)$ only in $[0,2\pi)$

1) zeroes of $f'(x)$
==================
The zeroes of $f'(x)$ are the solutions of $\sin(x)\sqrt{a+\sin(x)}=\cos(x)\sqrt{a+\cos x}$

$\iff$ $\sin(x)$ and $\cos(x)$ have same signs and $\sin^2(x)(a+\sin(x))=\cos^2(x)(a+\cos x)$

$\iff$ $\sin(x)$ and $\cos(x)$ have same signs and $(\sin(x)-\cos(x))(1+\sin(x)\cos(x)+a(\sin(x)+\cos(x)))=0$

$\sin(x)=\cos(x)$ gives at least two zeroes : $x\in\{\frac{\pi}4,\frac{5\pi}4\}$

It remains to solve $\sin(x)$ and $\cos(x)$ have same signs and $1+\sin(x)\cos(x)+a(\sin(x)+\cos(x))=0$

This means $\sin(x)\le 0$ and $\cos(x)\le 0$ and $1+\sin(x)\cos(x)+a(\sin(x)+\cos(x))=0$

Let then $\sin(x)=-u$ and $\cos(x)=-\sqrt{1-u^2}$ with $u\in[0,1]$ and we are looking sor solitions of 

$u\in[0,1]$ and $1+u\sqrt{1-u^2}-a(u+\sqrt{1-u^2})=0$

$\iff$ $u\in[0,1]$ and $1-au=(a-u)\sqrt{1-u^2}$

$\iff$ $0\le u\le \frac 1a$ and $(1-au)^2=(a-u)^2(1-u^2)$

$\iff$ $0\le u\le \frac 1a$ and $u^4 -2au^3 +(2a^2-1)u^2 +1-a^2=0$

Studying $P(x)=x^4-2ax^3+(2a^2-1)x^2+1-a^2$ is rather complex, IMHO :

$P(0)=1-a^2<0$
$P(1)=(a-1)^2>0$
$P(\frac 1a)=\frac{(1-a^2)^3}{a^4}<0$
$P(\frac{\sqrt 2}2)=\frac 34-\frac a{\sqrt 2}$

So $P(x)$ has 1 or 3 roots in $(0,1)$
But, if there is only one root, it is $>\frac 1a$ and so is not a root of $f'(x)$
So it remains to study the possibility of 3 roots in $(0,1)$ :

$P'(x)=4x^3-6ax^2+(4a^2-2)x=\frac x4((4x-3a)^2 - (8-7a^2))$

So, in order to have 3 roots, we need $a^2\le \frac 87$ and then $P'(x)$ has a local maximum in $[0,1]$ for $x_M=\frac{3a-\sqrt{8-7a^2}}4$ and we are looking for situations where $P(x_M)\ge 0$

$P(x_M)=\frac{13a^4-36a^2+24+a(8-7a^2)\sqrt{8-7a^2}}{32}$

The zeroes of $P(x_M)$ are when $-13a^4+36a^2-24=a(8-7a^2)\sqrt{8-7a^2}$

$\iff$ $\frac 87 \ge a^2\ge \frac {18-2\sqrt 3}{13}$ and $(-13a^4+36a^2-24)^2=a^2(8-7a^2)^3$

$\iff$ $\frac 87 \ge a^2\ge \frac {18-2\sqrt 3}{13}$ and $(a^2-1)^3(8a^2-9)=0$ and so $a^2=\frac 98$

So :

for $a^2<\frac 98$, $P(x_M)>0$. and $P(\frac {\sqrt 2}2)>0$ and so $f'(x)$ has four roots : ${\frac{\pi}4 < \pi <x_1<\frac{5\pi}4 < x_2 < \frac 3\pi}2$

for $a^2=\frac 98$, $x_M=\frac{\sqrt 2}2$ and $P(x_M)=0$ and so $f'(x)$ has two roots : $\frac{\pi}4$ and $\frac{5\pi}4$ (triple root)

for $a^2>\frac 98$, $P(x_M)<0$ and $f'(x)$ has only two roots $\frac{\pi}4$ and $\frac{5\pi}4$.


2) conclusions on min-max
=========================

If $a\ge \frac{3\sqrt 2}4$ :
 max is reached for $x=\frac{\pi}4$ and is $2\sqrt{a+\frac{\sqrt 2}2}$
 min is reached for $x=\frac{5\pi}4$ and is $2\sqrt{a-\frac{\sqrt 2}2}$

If $a< \frac{3\sqrt 2}4$ :
The polynomial $P(x)=x^4-2ax^3+(2a^2-1)x^2+1-a^2$ has exactly four roots $u_1<0<u_2<\frac{\sqrt 2}2 < u_3 < \frac 1a < u_4 < 1$
Then $f'(x)$ has four roots in $[0,2\pi)$ :
$x_1=\frac{\pi}4$ which is a local maximum of $f(x)$

$x_2=\pi+\arcsin(u_2)\in(\pi,\frac{5\pi}4)$ which is a local minimum

$x_3=\frac{5\pi}4$ which is a local maximum

$x_4=\pi+\arcsin(u_3)\in(\pi,\frac{5\pi}4)$ which is a local minimum

It's immediate to see that $f(x_3)<f(x_1)$ and that the global maximum is $f(\frac{\pi}4)=2\sqrt{a+\frac{\sqrt 2}2}$

It's also nearly immediate to see that $u_3=\sqrt{1-u_2^2}$ and so $x_4=\frac{3\pi}2-\arcsin(u_2)$ and that $f(x_2)=f(x_4)$ is the global minimum.

This global minimum is $\sqrt{a+u_2}+\sqrt{a+u_3}$\end{tcolorbox}
[\/hide]

Very nice solution   

But $u_2$ and $u_3$ is not determined?? :(
\end{solution}



\begin{solution}[by \href{https://artofproblemsolving.com/community/user/29428}{pco}]
	\begin{tcolorbox} 
But $u_2$ and $u_3$ is not determined?? :(\end{tcolorbox}

When $1<a<\frac {3\sqrt 2}4$, the polynomial $P(x)=x^4-2ax^3+(2a^2-1)x^2+1-a^2$ has exactly four distinct roots $u_1<0<u_2<\frac{\sqrt 2}2 < u_3 < \frac 1a < u_4 < 1$

So $u_2$ and $u_3$ can always be computed using Ferrari or Descartes method to solve the quartic and then Cardan to solve the subsequent degree 3 equation.

But this is quite ugly and I'll not do it.

Obviously, there is a clever olympiad-level hint that I have not seen and I'm quite sure that OP will give us a quite simple and nice solution.
\end{solution}



\begin{solution}[by \href{https://artofproblemsolving.com/community/user/73838}{Ovchinnikov Denis}]
	\begin{tcolorbox}[quote="pco"][quote="Ovchinnikov Denis"] 
But $u_2$ and $u_3$ is not determined?? :(\end{tcolorbox}

When $1<a<\frac {3\sqrt 2}4$, the polynomial $P(x)=x^4-2ax^3+(2a^2-1)x^2+1-a^2$ has exactly four distinct roots $u_1<0<u_2<\frac{\sqrt 2}2 < u_3 < \frac 1a < u_4 < 1$

So $u_2$ and $u_3$ can always be computed using Ferrari or Descartes method to solve the quartic and then Cardan to solve the subsequent degree 3 equation.

But this is quite ugly and I'll not do it.

Obviously, there is a clever olympiad-level hint that I have not seen and I'm quite sure that OP will give us a quite simple and nice solution.\end{tcolorbox}
[\/hide]
Yes, I forgot by Ferrari method  :oops_sign: 
But I think, that full problem doesn't from school olympiad :) 
May be for any big $a$ :)

P.S. for $a=1$ $ min f(x)=f( \pi)=1$
Prove:
similarly we can found $min f(x)$ only for $x \in [ \pi; \frac{3 \pi}{2}] $
$\sqrt{1-sinx}+\sqrt{1-cosx} \geq 1 <=> sinx+cosx-1 \leq 2\sqrt{(1-sinx)(1-cosx)} <=> sin^2x+cos^2x+1-2sinz-2cosx+2sinxcosx \leq 4-4sinx-4cosx+4sinxcosx <=> 2(sinxcosx-sinx-cosx+1) \geq 0 <=> (sinx-1)(cosx-1) \geq 0$, 
which is true, so
$QED$
\end{solution}



\begin{solution}[by \href{https://artofproblemsolving.com/community/user/29428}{pco}]
	I  think I have a nicer result.

The roots of $f'(x)=0$ other than $\frac{\pi}4$ and $\frac{5\pi}4$ are roots of $1+\sin(x)\cos(x)+a(\sin(x)+\cos(x))=0$ with both $\sin(x)$ and $\cos(x)$ negative.

Then $a+\cos(x)=\cos(x)-\frac{1+\sin(x)\cos(x)}{\sin(x)+\cos(x)}$ $=\frac{\sin^2(x)}{-\sin(x)-\cos(x)}$

And $a+\sin(x)=\frac{\cos^2(x)}{-\sin(x)-\cos(x)}$

And so $\sqrt{a+\sin(x)}+\sqrt{a+\cos(x)}=$ $\frac{-\sin(x)}{\sqrt{-\sin(x)-\cos(x)}}$ $+\frac{-\cos(x)}{\sqrt{-\sin(x)-\cos(x)}}$

So $\sqrt{a+\sin(x)}+\sqrt{a+\cos(x)}=\sqrt{-\sin(x)-\cos(x)}$

Let then $t=-\sin(x)-\cos(x)\in(-\sqrt 2,-1)$ so that the equation $1+\sin(x)\cos(x)+a(\sin(x)+\cos(x))=0$ becomes $t^2-2at+1=0$
And we get $t=a+\sqrt{a^2-1}$ (the other value is not in the range).

The global minimum is $\sqrt t$

So, if $1<a<\frac {3\sqrt 2}4$, $\boxed{\text{the global minimum is }\sqrt{a+\sqrt{a^2-1}}}$
\end{solution}



\begin{solution}[by \href{https://artofproblemsolving.com/community/user/73838}{Ovchinnikov Denis}]
	\begin{tcolorbox}[quote="pco"]I  think I have a nicer result.

The roots of $f'(x)=0$ other than $\frac{\pi}4$ and $\frac{5\pi}4$ are roots of $1+\sin(x)\cos(x)+a(\sin(x)+\cos(x))=0$ with both $\sin(x)$ and $\cos(x)$ negative.

Then $a+\cos(x)=\cos(x)-\frac{1+\sin(x)\cos(x)}{\sin(x)+\cos(x)}$ $=\frac{\sin^2(x)}{-\sin(x)-\cos(x)}$

And $a+\sin(x)=\frac{\cos^2(x)}{-\sin(x)-\cos(x)}$

And so $\sqrt{a+\sin(x)}+\sqrt{a+\cos(x)}=$ $\frac{-\sin(x)}{\sqrt{-\sin(x)-\cos(x)}}$ $+\frac{-\cos(x)}{\sqrt{-\sin(x)-\cos(x)}}$

So $\sqrt{a+\sin(x)}+\sqrt{a+\cos(x)}=\sqrt{-\sin(x)-\cos(x)}$

Let then $t=-\sin(x)-\cos(x)\in(-\sqrt 2,-1)$ so that the equation $1+\sin(x)\cos(x)+a(\sin(x)+\cos(x))=0$ becomes $t^2-2at+1=0$
And we get $t=a+\sqrt{a^2-1}$ (the other value is not in the range).

The global minimum is $\sqrt t$

So, if $1<a<\frac {3\sqrt 2}4$, $\boxed{\text{the global minimum is }\sqrt{a+\sqrt{a^2-1}}}$\end{tcolorbox}
[\/hide]
This solution more prettier than first :) :coolspeak: 
But for $a=1$ we must analyze separately because then $f'(x) $ not determined for $x= \pi+2 \pi n, x=\frac{3 \pi}{2}+2 \pi n$  :)
\end{solution}



\begin{solution}[by \href{https://artofproblemsolving.com/community/user/73838}{Ovchinnikov Denis}]
	Let $n \in \mathbb{N} ; n \geq 2$ and $a_i \in \mathbb{R} $ for all $i=\{ 1,2,...,n \}$ such that $ \sum_{i=1}^n a_i^2 = 1$.
Find $max$, $min$ of $\sum_{i=1}^n \sqrt{a+a_i^2}$, where $a \geq 1$.

Let solve for $a \geq \frac{3}{2}$ and over this for small $a$.
But may be you can solve at once :)
\end{solution}
*******************************************************************************
-------------------------------------------------------------------------------

\begin{problem}[Posted by \href{https://artofproblemsolving.com/community/user/65464}{sororak}]
	Determine all functions $ f:\mathbb{R}\to\mathbb{R} $ such that for all ${x,y}\in\mathbb{R}$,
\[f(x+y)+f(x-y)=2f(x)\cdot\cos{y} .\]
	\flushright \href{https://artofproblemsolving.com/community/c6h365620}{(Link to AoPS)}
\end{problem}



\begin{solution}[by \href{https://artofproblemsolving.com/community/user/29428}{pco}]
	\begin{tcolorbox}Determine all functions $ f:\mathbb{R}\to\mathbb{R} $, such that:
$ \forall {x,y}\in\mathbb{R} \ ; \ f(x+y)+f(x-y)=2f(x)\cdot\cos{y} $\end{tcolorbox}
Let $P(x,y)$ be the assertion $f(x+y)+f(x-y)=2f(x)\cos(y)$

(a) : $P(\frac{\pi}2-x,\frac{\pi}2)$ $\implies$ $f(\pi-x)+f(-x)=0$

(b) : $P(0,x)$ $\implies$ $f(x)+f(-x)=2f(0)\cos(x)$

(c) : $P(\frac{\pi}2,\frac{\pi}2-x)$ $\implies$ $f(\pi-x)+f(x)=2f(\frac{\pi}2)\sin(x)$

===========================================================================
-(a)+(b)+(c) $\implies$ $2f(x)=2f(0)\cos(x)+2f(\frac{\pi}2)\sin(x)$

And so $\boxed{f(x)=a\cdot\cos(x)+b\cdot\sin(x)}$ which indeed is a solution (simple verification).
\end{solution}



\begin{solution}[by \href{https://artofproblemsolving.com/community/user/65464}{sororak}]
	\begin{tcolorbox}[quote="sororak"]Determine all functions $ f:\mathbb{R}\to\mathbb{R} $, such that:
$ \forall {x,y}\in\mathbb{R} \ ; \ f(x+y)+f(x-y)=2f(x)\cdot\cos{y} $\end{tcolorbox}
Let $P(x,y)$ be the assertion $f(x+y)+f(x-y)=2f(x)\cos(y)$

(a) : $P(\frac{\pi}2-x,\frac{\pi}2)$ $\implies$ $f(\pi-x)+f(-x)=0$

(b) : $P(0,x)$ $\implies$ $f(x)+f(-x)=2f(0)\cos(x)$

(c) : $P(\frac{\pi}2,\frac{\pi}2-x)$ $\implies$ $f(\pi-x)+f(x)=2f(\frac{\pi}2)\sin(x)$

===========================================================================
-(a)+(b)+(c) $\implies$ $2f(x)=2f(0)\cos(x)+2f(\frac{\pi}2)\sin(x)$

And so $\boxed{f(x)=a\cdot\cos(x)+b\cdot\sin(x)}$ which indeed is a solution (simple verification).\end{tcolorbox}
Thanks!
\end{solution}
*******************************************************************************
-------------------------------------------------------------------------------

\begin{problem}[Posted by \href{https://artofproblemsolving.com/community/user/65464}{sororak}]
	Determine all functions $ f:\mathbb{Q}\to\mathbb{Q} $ such that $f(1)=2 $ and for all ${x,y}\in\mathbb{Q}$, we have
\[f(xy)=f(x)\cdot{f(y)}-f(x+y)+1.\]
	\flushright \href{https://artofproblemsolving.com/community/c6h365622}{(Link to AoPS)}
\end{problem}



\begin{solution}[by \href{https://artofproblemsolving.com/community/user/59392}{SHP3ND1}]
	The only solution is:
$ f(x) = x+1$

You have to study these cases :
1. $ x = 0 , y = 0$
2. $ x = 0 , y = \frac{p}{q}$
3. $ x = 1 , y = \frac{p}{q}$
4. $ x = 1, y = 1$
5. $ x = \frac{p}{q} , y = \frac{q}{p}$

:D
\end{solution}



\begin{solution}[by \href{https://artofproblemsolving.com/community/user/29428}{pco}]
	\begin{tcolorbox}Determine all functions $ f:\mathbb{Q}\to\mathbb{Q} $ such that:
$ i) \ f(1)=2 $ ,
$ ii) \ \forall{x,y}\in\mathbb{Q} \ ; \ f(xy)=f(x)\cdot{f(y)}-f(x+y)+1 $\end{tcolorbox}
Let $P(x,y)$ be the assertion $f(xy)=f(x)f(y)-f(x+y)+1$

$P(x,1)$ $\implies$ $f(x+1)=f(x)+1$ $\implies$ $f(x+n)=f(x)+n$ and $f(n)=n+1$ $\forall n\in\mathbb Z$

$P(\frac pq,q)$ $\implies$ $f(p)=f(\frac pq)f(q)-f(\frac pq+q)+1$ $\implies$ $p+1=f(\frac pq)(q+1)-(f(\frac pq)+q)+1$

$\implies$ $p+q=qf(\frac pq)$ and so $f(\frac pq)=\frac pq+1$ and so $\boxed{f(x)=x+1}$ $\forall x\in\mathbb Q$

And it is easy to check back that this indeed is a solution.
\end{solution}



\begin{solution}[by \href{https://artofproblemsolving.com/community/user/65464}{sororak}]
	\begin{tcolorbox}[quote="sororak"]Determine all functions $ f:\mathbb{Q}\to\mathbb{Q} $ such that:
$ i) \ f(1)=2 $ ,
$ ii) \ \forall{x,y}\in\mathbb{Q} \ ; \ f(xy)=f(x)\cdot{f(y)}-f(x+y)+1 $\end{tcolorbox}
Let $P(x,y)$ be the assertion $f(xy)=f(x)f(y)-f(x+y)+1$

$P(x,1)$ $\implies$ $f(x+1)=f(x)+1$ $\implies$ $f(x+n)=f(x)+n$ and $f(n)=n+1$ $\forall n\in\mathbb Z$

$P(\frac pq,q)$ $\implies$ $f(p)=f(\frac pq)f(q)-f(\frac pq+q)+1$ $\implies$ $p+1=f(\frac pq)(q+1)-(f(\frac pq)+q)+1$

$\implies$ $p+q=qf(\frac pq)$ and so $f(\frac pq)=\frac pq+1$ and so $\boxed{f(x)=x+1}$ $\forall x\in\mathbb Q$

And it is easy to check back that this indeed is a solution.\end{tcolorbox}
Thanks!
\end{solution}
*******************************************************************************
-------------------------------------------------------------------------------

\begin{problem}[Posted by \href{https://artofproblemsolving.com/community/user/73838}{Ovchinnikov Denis}]
	Found all functions $f: \mathbb{R} \to \mathbb{R}$, such that for any $x,y \in \mathbb{R}$,
\[f(x^2+xy+f(y))=f^2(x)+xf(y)+y.\]
	\flushright \href{https://artofproblemsolving.com/community/c6h365859}{(Link to AoPS)}
\end{problem}



\begin{solution}[by \href{https://artofproblemsolving.com/community/user/29428}{pco}]
	\begin{tcolorbox}Found all functions $f: \mathbb{R} \to \mathbb{R}$, such that for any $x,y \in \mathbb{R}$ holds next equality:
$f(x^2+xy+f(y))=f^2(x)+xf(y)+y$.\end{tcolorbox}
Let $P(x,y)$ be the assertion $f(x^2+xy+f(y))=f^2(x)+xf(y)+y$

1) $f(0)=0$
===========
Let $f(0)=a$

(a) : $P(4a,0)$ $\implies$ $f(16a^2+a)=f^2(4a)+4a^2$
(b) : $P(-4a,0)$ $\implies$ $f(16a^2+a)=f^2(-4a)-4a^2$
(c) : $P(-4a,4a)$ $\implies$ $f(f(4a))=f^2(-4a)-4af(4a)+4a$
(d) : $P(0,4a)$ $\implies$ $f(f(4a))=4a+a^2$ 
---------------------------------------------------
(a)-(b)+(c)-(d) : $0=(f(4a)-2a)^2+3a^2$ and so $a=0$
Q.E.D.

2) $f(x)=x$
===========

(a) : $P(0,x)$ $\implies$ $f(f(x))=x$  and so $f(x)$ is a bijection
(b) : $P(-x,x)$ $\implies$ $f(f(x))=f^2(-x)-xf(x)+x$
(c) : $P(x,0)$ $\implies$ $f(x^2)=f^2(x)$
(d) : $P(-x,0)$ $\implies$ $f(x^2)=f^2(-x)$
-------------------------------------------------
-(a)+(b)+(c)-(d) : $0=f(x)(f(x)-x)$

Since $f(0)=0$ and $f(x)$ is a bijection, we have $f(x)\ne 0$ $\forall x\ne 0$ and the above equality becomes $f(x)=x$ $\forall x\ne 0$

And so $f(x)=x$ $\forall x$ which indeed is a solution.
\end{solution}
*******************************************************************************
-------------------------------------------------------------------------------

\begin{problem}[Posted by \href{https://artofproblemsolving.com/community/user/54529}{Martin N.}]
	Find all functions $f:\mathbb{R}\to\mathbb{R}$ such that for all $x, y\in\mathbb{R}$, we have
\[f(x+y)+f(x)f(y)=f(xy)+(y+1)f(x)+(x+1)f(y).\]
	\flushright \href{https://artofproblemsolving.com/community/c6h366340}{(Link to AoPS)}
\end{problem}



\begin{solution}[by \href{https://artofproblemsolving.com/community/user/61075}{binaj}]
	do you know anybody who solved this problem?
\end{solution}



\begin{solution}[by \href{https://artofproblemsolving.com/community/user/54529}{Martin N.}]
	\begin{tcolorbox}do you know anybody who solved this problem?\end{tcolorbox}
I'm sure someone solved it, but I didn't and neither did the rest of the Austrian team...
\end{solution}



\begin{solution}[by \href{https://artofproblemsolving.com/community/user/29428}{pco}]
	\begin{tcolorbox}Find all functions $f:\mathbb{R}\to\mathbb{R}$ such that for all $x\mbox{, }y\in\mathbb{R}$, we have
\[f(x+y)+f(x)f(y)=f(xy)+(y+1)f(x)+(x+1)f(y)\mbox{.}\]

\begin{italicized}(4th Middle European Mathematical Olympiad, Individual Competition, Problem 1)\end{italicized}\end{tcolorbox}
Let $P(x,y)$ be the assertion $f(x+y)+f(x)f(y)=f(xy)+(y+1)f(x)+(x+1)f(y)$

$P(0,0)$ $\implies$ $f(0)^2=2f(0)$ and so $f(0)=0$ or $f(0)=2$

If $f(0)=2$, $P(x,0)$ $\implies$ $f(x)=x+2$ which is not a solution. So $f(0)=0$

$P(1,-1)$ $\implies$ $f(1)f(-1)=3f(-1)$ and so $f(-1)=0$ or $f(1)=3$

If $f(1)=3$, $P(x-1,1)$ $\implies$ $\boxed{f(x)=3x}$ which indeed is a solution.

If $f(1)\ne 3$, then $f(-1)=0$

$P(-x,1)$ $\implies$ $f(-x+1)=(3-f(1))f(-x)+(-x+1)f(1)$ 
$P(x,-1)$ $\implies$ $f(x-1)=f(-x)$
$P(x-1,-1)$ $\implies$ $f(x-2)=f(-x+1)$

And so $f(x-2)=(3-f(1))f(x-1)+(-x+1)f(1)$
But $P(x-2,1)$ $\implies$ $f(x-1)=(3-f(1))f(x-2)+(x-1)f(1)$
Adding these two lines gives $(2-f(1))(f(x-2)+f(x-1))=0$

If $f(1)\ne 2$, this implies $f(x-2)=-f(x-1)$ and so $f(x+1)=-f(x)$ and $f(1)=0$
But then $P(x,1)$ $\implies$ $f(x+1)=3f(x)$ and so $-f(x)=3f(x)$ and $\boxed{f(x)=0}$ $\forall x$ which indeed is a solution.

If $f(1)=2$, then $P(x,1)$ $\implies$ $f(x+1)=f(x)+2x+2$ and so :

$f(x+n)=f(x)+2nx+n^2+n$
So $f(n)=n^2+n$

Then $P(\frac pq,q)$ $\implies$ $f(\frac pq)=(\frac pq)^2+\frac pq$ and we got $f(x)=x^2+x$ $\forall x\in\mathbb Q$

Let then $g(x)=f(x)-x^2-x$ such that $g(x)=0$ $\forall x\in\mathbb Q$. The original equation becomes :

New assertion $Q(x,y)$ : $g(x+y)+g(x)g(y)+g(x)(y^2-1)+g(y)(x^2-1))=g(xy)$

$Q(x,1)$ $\implies$ $g(x+1)=g(x)$

Using this property and subtracting $Q(x,y)$ from $Q(x,y+1)$, we get $g(x)(2y+1)=g(xy+x)-g(xy)$

So $g(x+y)=g(x)+g(y)(\frac{2x}y+1)$ $\forall x,y\ne 0$

So $g(x)+g(y)(\frac{2x}y+1)=g(y)+g(x)(\frac{2y}x+1)$ and so $\frac{g(y)}{y^2}=\frac{g(x)}{x^2}$ $\forall x,y\ne 0$
So $g(x)=ax^2$ and so $g(x)=0$ and $\boxed{f(x)=x^2+x}$ which indeed is a solution.

Hence the three solutions of the equation :

$f(x)=0$
$f(x)=3x$
$f(x)=x^2+x$
\end{solution}



\begin{solution}[by \href{https://artofproblemsolving.com/community/user/10156}{Matematika}]
	A bit different solution
$ f(x+y)+f(x)f(y)=f(xy)+(y+1)f(x)+(x+1)f(y) $
$P(0,0)  \implies  {f(0)}^2=2f(0) $
if $ f(0)=2  $  from $P(x,0) \implies f(x)=x+2$ which is not a solution
so $f(0)=0$
Put $P(1,-1)  \implies  f(1)f(-1)=3f(-1)$
if $ f(1)=3 $
$P(x-1,1) \implies f(x)=3x $ which is  a solution
so we have  $f(-1)=0$
$P(-1,-1) \implies f(-2)=f(1) $
$P(-2,1) \implies  f(-1)=f(-2)(3-f(1))+(-1)f(1) $
so $ 0=f(1)(2-f(1)) $
if f(1)=0 we get
$P(x,-1)  \implies f(x-1)=f(-x)$
$P(x-1,1) \implies   f(x)=3f(x-1)=3f(-x)$
so from $ f(x)=3f(-x)$ we get $ f(x)=3f(-x)=3(3f(-(-x)))=9f(x) \implies f(x)=0$ which is a solution
if f(1)=2 we get
$P(x,-1)  \implies f(x-1)=f(-x)$
$P(x-1,1) \implies   f(x)=f(x-1)+2x=f(-x)+2x$
so $f(x)=f(-x)+2x$
$P(x,-x) \implies f(0)+f(x)f(-x)=f(-x^2)+(x+1)f(-x)+(-x+1)f(x)$
$0+f(x)(f(x)-2x)=(f(x^2)-2x^2)+(x+1)(f(x)-2x)+(-x+1)f(x)   \implies  f(x)^2-2xf(x)=f(x^2)-2x^2+xf(x)-2x^2+f(x)-2x-xf(x)+f(x)$
$f(x)^2=f(x^2)-4x^2+2f(x)+2xf(x)-2x$
$P(x,x)  \implies f(2x)+ {f(x)}^2=f(x^2)+2(x+1)f(x)$
$f(2x)+(f(x^2)-4x^2+2f(x)+2xf(x)-2x)=f(x^2)+2(x+1)f(x)$
$f(2x)=4x^2+2x$ or
$f(x)=x^2+x$ which also shows to be a solution
\end{solution}



\begin{solution}[by \href{https://artofproblemsolving.com/community/user/41147}{FantasyLover}]
	[hide="Solution"]First of all, if $f(x)=c$ is a constant function, we have $c+c^2=c+(y+1)c+(x+1)c$ $\forall x,y$, and the only solution for $c$ is $c=0$. Now, assume that $f$ is not constant.

Rearranging the given functional equation as $(f(x)-x)(f(y)-y)=f(xy)-xy+f(x)-x+f(y)-y-(f(x+y)-(x+y))+2xy$, this gives motivation to let $g(x):= f(x)-x$. The assumption above implies $\exists a$ such that $g(a)\neq -a$.

Rewriting the functional equation in terms of $g$, we have $g(x)g(y)=g(xy)+g(x)+g(y)-g(x+y)+2xy$.

Let $P(x,y)$ be the assertion $g(x)g(y)=g(xy)+g(x)+g(y)-g(x+y)+2xy$.

$P(-1,1)\implies g(-1)g(1)=g(-1)+g(-1)+g(1)-2$. This is equivalent to $(g(-1)-1)(g(1)-1)=g(-1)-1$. Hence, $g(-1)=1$ or $g(1)=2$. 

Suppose $g(1)=2$. Since $P(x,1)\implies g(x+1)=g(x)+g(x)+g(1)+2x-g(x)g(1)$, this gives $g(x+1)=2x+2$. Thus, we obtain one solution $\boxed{g(x)=2x}\; \forall x$.

Now, suppose $g(-1)=1$. $P(-1,-1)\implies g(-2)=g(1)+g(-1)+g(-1)-g(-1)g(-1)+2$. As a result, we have $g(-2)=g(1)+3$.

$P(-2,1)\implies g(-2)g(1)=g(-2)+g(-2)+g(1)-g(-1)-4$. Substituting in $g(-2)=g(1)+3$, we get $g(1)^2+3g(1)=2g(1)+6+g(1)-5$. Therefore, $g(1)=1$ or $-1$.

Suppose $g(1)=-1$. Because of the assumption that $f$ is not constant, there exists at least one $a$ such that $g(a)\neq -a$. Let us choose the greatest such $a$.

$P(a,1)\implies g(a)g(1)=g(a)+g(a)+g(1)-g(a+1)+2a$. Then, we have $g(a+1)=3g(a)-1+2a\neq -a-1$. This contradicts our assumption that $a$ is the greatest real number such that $g(a)=-a$. Hence, $g(1)=-1$ leads to a contradiction, and thus $g(1)=1$.

$P(x,-1)\implies g(x)g(-1)=g(-x)+g(x)+g(-1)-g(x-1)-2x$. This gives $g(x-1)=g(-x)-2x+1\cdots \alpha$.

$P(x-1,1)\implies g(x-1)g(1)=g(x-1)+g(x-1)+g(1)-g(x)+2x-2$. This gives $g(x)=g(x-1)+2x-1\cdots \beta$.

From $\alpha+\beta$, we obtain $g(x)=g(-x)$.

$P(x,-x)\implies g(x)g(-x)=g(-x^2)+g(x)+g(-x)-g(0)-2x^2$. So $(g(x))^2=g(x^2)+2g(x)-2x^2\cdots \gamma$.

$P(x,x)\implies (g(x))^2=g(x^2)+2g(x)-g(2x)+2x^2\cdots \delta$

$\gamma-\delta$ gives $g(2x)=4x^2$, and we obtain another solution $\boxed{g(x)=x^2}\; \forall x$.

In conclusion, since $f(x)=g(x)+x$, there are three solutions $\boxed{f(x)=0, 3x, x^2+x}\;\forall x$ to the given functional equation. $\blacksquare$[\/hide]
\end{solution}



\begin{solution}[by \href{https://artofproblemsolving.com/community/user/35129}{Zhero}]
	\begin{tcolorbox}Rewriting the functional equation in terms of $g$, we have $g(x)g(y)=g(xy)+g(x)+g(y)-g(x+y)+2xy$.\end{tcolorbox}
Just a comment: if we let $h(x) = g(x) - 1$, then $(h(x) + 1)(h(y) + 1) + h(x+y) + 1 = h(xy) + 1 + h(x) + 1 + h(y) + 1 + 2xy$, or $h(x)h(y) + h(x+y) = h(xy) + 2xy + 1$, which is exactly [url=http://www.artofproblemsolving.com/Forum/viewtopic.php?f=36&t=78909&start=0&]2005 A4[\/url].
\end{solution}
*******************************************************************************
-------------------------------------------------------------------------------

\begin{problem}[Posted by \href{https://artofproblemsolving.com/community/user/85498}{nicolasteo}]
	Find all continuous functions $f:\mathbb{R}\to\mathbb{R}$ such that
\[f(x+y)=2f(x)f(y)+3f(x)+3f(y)+3\]
for all $x,y \in \mathbb R$.
	\flushright \href{https://artofproblemsolving.com/community/c6h366346}{(Link to AoPS)}
\end{problem}



\begin{solution}[by \href{https://artofproblemsolving.com/community/user/79198}{SKhan}]
	\begin{tcolorbox}Find all continuous functions $f:\mathbb{R}\to\mathbb{R}$ such that :
$f(x+y)=2f(x)f(y)+3f(x)+3f(y)+3$\end{tcolorbox}
Then $2f(x+y)+3=4f(x)f(y)+6f(x)+6f(y)+9=(2f(x)+3)(2f(y)+3)$
Let $g(x)=2f(x)+3$
$g(x)$ is a transformation of $f(x)$ (a times $2$ stretch in the y-direction, then a translation with vector $\binom{0}{3}$), so $g(x)$ is continuous iff. $f(x)$ is continuous.

Then we need to find all continuous functions $g:\mathbb{R}\to\mathbb{R}$ such that :
$g(x+y)=g(x)g(y)$

$g(0)=g(0)^2$ so $g(0)=0\text{ or }1$
If $g(0)=0$ then $g(x)=g(x)g(0)=0$ for all real numbers $x$.
Otherwise $g(0)=1$
$g(x)g(-x)=g(0)=1\implies g(x)\neq 0$ and $g(-x)=g(x)^{-1}$
$g(x)=g\left(\frac{x}{2}\right)^2\geq 0$
Induction on $n$ gives $g(nx)=g(x)^n$ for all integers $n$ and real numbers $x$
Let $r\in\mathbb{Q}$ and $p, q\in\mathbb{Z}$, s.t. $r=\frac{p}{q}$
Then $g(rx)^q=g(qrx)=g(px)=g(x)^p=g(x)^{qr}$
Then $g(rx)=g(x)^r$ for all rational  numbers $r$ and real numbers $x$
Then $g(r)=g(1)^r$
Suppose that $g(a)=g(b)$ then $g(a-b)=1$
Then $g(r(a-b))=1$ and $g(x+r(a-b))=g(x)$ for all rational  numbers $r$ and real numbers $x$
If $a\neq b$ then we can make $r(a-b)$ as close to zero as we want so $g(x)$ is the constant function, but $g(a-b)=1$ then $g(x)\equiv 1$
Otherwise $g$ is injective.
But $g$ is continuous so $g$ is a monotone function.
$g(2)=g(1)^2$ so $g$ strictly increasing if $g(1)>1$ or is strictly decreasing if $0<g(1)<1$
If $g(1)>1$
Then  $x<y$ iff. $g(x)<g(y)$ and $x<y$ iff. $g(1)^x<g(1)^y$
Suppose that $g(1)^x<g(x)$ then there exists a rational number $r$ s.t. $g(1)^x<g(1)^r=g(r)<g(x)$ then $x<r<x$, which is false.
Suppose that $g(1)^x>g(x)$ then there exists a rational number $r$ s.t. $g(1)^x>g(1)^r=g(r)>g(x)$ then $x>r>x$, which is false.
Then $g(x)=g(1)^x$ for all real numbers $x$ 
Otherwise $0<g(1)<1$
Then  $x<y$ iff. $g(x)>g(y)$ and $x<y$ iff. $g(1)^x>g(1)^y$
Suppose that $g(1)^x<g(x)$ then there exists a rational number $r$ s.t. $g(1)^x<g(1)^r=g(r)<g(x)$ then $x>r>x$, which is false.
Suppose that $g(1)^x>g(x)$ then there exists a rational number $r$ s.t. $g(1)^x>g(1)^r=g(r)>g(x)$ then $x<r<x$, which is false.
Then $g(x)=g(1)^x$ for all real numbers $x$

So we have $g(x)=g(1)^x$ where $g(1)\geq 0$
So $f(x)=\frac{g(x)-3}{2}=\frac{(2f(1)+3)^x-3}{2}$ where $2f(1)+3\geq 0$ ie. $f(1)\geq -\frac{3}{2}$
\end{solution}



\begin{solution}[by \href{https://artofproblemsolving.com/community/user/29428}{pco}]
	\begin{tcolorbox}[quote="nicolasteo"]Find all continuous functions $f:\mathbb{R}\to\mathbb{R}$ such that :
$f(x+y)=2f(x)f(y)+3f(x)+3f(y)+3$\end{tcolorbox}
Then $2f(x+y)+3=4f(x)f(y)+6f(x)+6f(y)+9=(2f(x)+3)(2f(y)+3)$
Let $g(x)=2f(x)+3$
$g(x)$ is a transformation of $f(x)$ (a times $2$ stretch in the y-direction, then a translation with vector $\binom{0}{3}$), so $g(x)$ is continuous iff. $f(x)$ is continuous.

Then we need to find all continuous functions $g:\mathbb{R}\to\mathbb{R}$ such that :
$g(x+y)=g(x)g(y)$...\end{tcolorbox}

Nice !.  But from there, you can go quicker :

Either $g(x)=0$ $\forall x$, either $g(x)>0$ $\forall x$ and then $\ln(g(x))$ is a continuous solution of Cauchy equation, so is $ax$

And so the solutions :
$f(x)=-\frac 32$ $\forall x$

$f(x)=\frac{e^{ax}-3}2$ $\forall x$ for any real $a$
\end{solution}



\begin{solution}[by \href{https://artofproblemsolving.com/community/user/79198}{SKhan}]
	\begin{tcolorbox}

Either $g(x)=0$ $\forall x$, either $g(x)>0$ $\forall x$ and then $\ln(g(x))$ is a continuous solution of Cauchy equation, so is $ax$

And so the solutions :
$f(x)=-\frac 32$ $\forall x$

$f(x)=\frac{e^{ax}-3}2$ $\forall x$ for any real $a$\end{tcolorbox}
Yes, indeed.
I  think that $g(x+y)=g(x)g(y)$ is a Cuachy-type equation, but I thought I would show how to solve it, since the problem was posted in the Unsolved Problems Section.
\end{solution}
*******************************************************************************
-------------------------------------------------------------------------------

\begin{problem}[Posted by \href{https://artofproblemsolving.com/community/user/67223}{Amir Hossein}]
	Determine all functions $f : \mathbb R \to \mathbb R$ satisfying the following two conditions:

(a) $f(x + y) + f(x - y) = 2f(x)f(y)$ for all $x, y \in \mathbb R$, and

(b) $\lim_{x\to \infty} f(x) = 0$.
	\flushright \href{https://artofproblemsolving.com/community/c6h366766}{(Link to AoPS)}
\end{problem}



\begin{solution}[by \href{https://artofproblemsolving.com/community/user/89666}{Akiyama}]
	I think the only solution is $ f(x)=0 $. Am I right?
\end{solution}



\begin{solution}[by \href{https://artofproblemsolving.com/community/user/29428}{pco}]
	\begin{tcolorbox}Determine all functions $f : \mathbb R \to \mathbb R$ satisfying the following two conditions:

\begin{italicized}(a)\end{italicized} $f(x + y) + f(x - y) = 2f(x)f(y)$ for all $x, y \in \mathbb R$,

\begin{italicized}(b)\end{italicized} $\lim_{x\to \infty} f(x) = 0.$\end{tcolorbox}
Let $P(x,y)$ be the assertion $f(x+y)+f(x-y)=2f(x)f(y)$

Suppose $\exists a$ such that $f(a)\ne 0$

$P(a,0)$ $\implies$ $2f(a)=2f(a)f(0)$ and so $f(0)=1$
$P(a,a)$ $\implies$ $f(2a)=2f(a)^2-1$

If $|f(a)|\le 1$ , $\exists u$ such that $f(a)=\cos(u)$ and we immediately get $f(2^na)=\cos(2^nu)$ $\forall n\ge 0$
If $|f(a)|>1$, $\exists u$ such that $|f(a)|=\cosh(u)$ and we immediately get $f(2^na)=\cosh(2^nu)$ $\forall n>0$

And obviously, we can have neither $\lim_{n\to+\infty}\cos(2^nu)=0$, neither $\lim_{n\to+\infty}\cosh(2^nu)=0$

And so $f(x)=0$ $\forall x$
\end{solution}
*******************************************************************************
-------------------------------------------------------------------------------

\begin{problem}[Posted by \href{https://artofproblemsolving.com/community/user/67223}{Amir Hossein}]
	For all rational $x$ satisfying $0 \leq x < 1$, the functions $f$ is defined by
\[f(x)=\begin{cases}\frac{f(2x)}{4},&\mbox{for }0 \leq x < \frac 12,\\ \frac 34+ \frac{f(2x - 1)}{4}, & \mbox{for } \frac 12 \leq x < 1.\end{cases}\]
Given that $x = 0.b_1b_2b_3 \cdots $ is the binary representation of $x$, find, with proof, $f(x)$.
	\flushright \href{https://artofproblemsolving.com/community/c6h366972}{(Link to AoPS)}
\end{problem}



\begin{solution}[by \href{https://artofproblemsolving.com/community/user/29428}{pco}]
	\begin{tcolorbox}For all rational $x$ satisfying $0 \leq x < 1$, f is defined by
\[f(x)=\{\begin{array}{cc}\frac{f(2x)}{4},&\mbox{ for }0 \leq x < \frac 12,\\ \frac 34+ \frac{f(2x - 1)}{4}, & \mbox{for } \frac 12 \leq x < 1.\end{array}\]
Given that $x = 0.b_1b_2b_3 \cdots $ is the binary representation of $x$, find $f(x).$\end{tcolorbox}

$f(0.b_1b_2b_3 \cdots )=0.b_1b_1b_2b_2b_3b_3 \cdots$
\end{solution}
*******************************************************************************
-------------------------------------------------------------------------------

\begin{problem}[Posted by \href{https://artofproblemsolving.com/community/user/67223}{Amir Hossein}]
	Find all the continuous bounded functions $f: \mathbb R \to \mathbb R$ such that 
\[(f(x))^2 -(f(y))^2  = f(x + y)f(x - y) \text{  for all } x, y \in \mathbb R.\]
	\flushright \href{https://artofproblemsolving.com/community/c6h367461}{(Link to AoPS)}
\end{problem}



\begin{solution}[by \href{https://artofproblemsolving.com/community/user/29428}{pco}]
	\begin{tcolorbox}Find all the continuous bounded functions $f: \mathbb R \to \mathbb R$ such that 
\[(f(x))^2 -(f(y))^2  = f(x + y)f(x - y) \text{  for all } x, y \in \mathbb R.\]\end{tcolorbox}
Let $P(x,y)$ be the assertion $f(x)^2-f(y)^2=f(x+y)f(x-y)$

$f(x)=0$ $\forall x$ is a solution. So let us from now look for non all-zero functions. So let $u$ such that $f(u)\ne 0$

1) $f(x)$ is an odd function and $f(0)=0$
=============================
$P(\frac{u+x}2,\frac{u-x}2)$ $\implies$ $f(\frac{u+x}2)^2-f(\frac{u-x}2)^2=f(u)f(x)$
$P(\frac{u-x}2,\frac{u+x}2)$ $\implies$ $f(\frac{u-x}2)^2-f(\frac{u+x}2)^2=f(u)f(-x)$
Adding these two lines, we get $f(-x)=-f(x)$
Q.E.D.

2) $\exists v>0$ such that $f(v)=0$
========================
If $f(x)$ is injective, it is monotonous (since continuous) and, since monotonous and bounded, limits at $\pm\infty$ exist :
Let then $\lim_{x\to+\infty}f(x)=L$ and $\lim_{x\to-\infty}f(x)=l$
$P(x+u,x)$ $\implies$ $f(x+u)^2-f(x)^2=f(2x+u)f(u)$
Setting $x\to+\infty$ in this equality, we get $Lf(u)=0$ and so $L=0$
Setting $x\to-\infty$ in this equality, we get $lf(u)=0$ and so $l=0$
And so contradiction (since $f(x)$ is not the all zero constant function).
So $f(x)$ is not injective and $\exists$ $b\ne c$ such that $f(b)=f(c)$
$P(b,c)$ $\implies$ $f(b+c)f(b-c)=0$
If $f(b-c)=0$ then since $b-c\ne 0$ and since $f(-x)=-f(x)$, just choose $v=|b-c|$. Q.E.D.
If $f(b+c)=0$ and $b+c\ne 0$, then just choose $v=|b+c|$. Q.E.D
If $f(b+c)=0$ and $b+c=0$, we get $c=-b$ and so $f(b)=f(c)=f(-b)=-f(b)$ and so $f(b)=f(c)=0$ with $b\ne c$. Then, just choose $v=\max(|b|,|c|)$. Q.E.D.

3) $\exists a>0$ such that $f(2a)=0$ and $f(x)\ne 0$ $\forall x\in(0,2a)$ and $f(x+4a)=f(x)$ $\forall x$
=========================================================================
Let $A=\{x>0$ such that $f(x)=0\}$. $A$ is non empty (from step 2 above)
Let $v\in A$, then :
(a) : $P(\frac{x+u}2+v,v)$ $\implies$ $f(\frac{x+u}2+v)^2-f(\frac{x+u}2)^2=0$
(b) : $P(\frac{x-u}2+v,v)$ $\implies$ $f(\frac{x-u}2+v)^2-f(\frac{x-u}2)^2=0$
(c) : $P(\frac{x+u}2,\frac{x-u}2)$ $\implies$ $f(\frac{x+u}2)^2-f(\frac{x-u}2)^2=f(x)f(u)$
(d) : $P(\frac{x+u}2+v,\frac{x-u}2+v)$ $\implies$ $f(\frac{x+u}2+v)^2-f(\frac{x-u}2+v)^2=f(x+2v)f(u)$
(a)-(b)+(c)-(d) : $f(x+2v)-f(x))f(u)=0$ and so $f(x+2v)=f(x)$

Let then $a=\frac 12\inf(A)$
Since $f(x)$ is continuous, $f(2a)=0$
If $a=0$, this means that it exists elements $v$ in $A$ as little as we want and so $f(x)$ is periodic with periods as little as we want. So, since $f(x)$ is continuous, this would mean $f(x)$ constant, and so $f(x)=0$ $\forall x$, impossible in this part of the proof.
So $a\ne 0$

And so we got :
$a>0$
$f(2a)=0$ and so $2a\in A$ and so $f(x+4a)=f(x)$ $\forall x$
$f(x)\ne 0$ $\forall x\in (0,2a)$ (since $2a=\inf(A)$)
Q.E.D

4) $f(x)=b\sin(\frac{\pi x}{2a})$
=======================
If $f(x)$ is known over $[0,2a]$, using $f(-x)=-f(x)$ implies knowlegde over $[-2a,+2a]$
Then $f(x+4a)=f(x)$ implis knowledge over $%Error. "mathBB" is a bad command.
R$
So we can look at $[0,2a]$ only
Since $f(x)$ solution implies $bf(x)$ solution and since $f(a)\ne 0$, wlog say $f(a)=1$
$P(x,2a)$ $\implies$ $f(x+2a)^2=f(x)^2$
If $x\in (0,2a)$, we know that $f(x)>0$. So $f(x)<0$ $\forall x\in(-2a,0)$ and so $f(x+2a)=-f(x)$ $\forall x\in(-2a,0)$ and so $f(x+2a)=-f(x)$ $\forall x$

$P(a-x,a)$ $\implies$ $f(a-x)^2-1=f(2a-x)f(-x)=f(x-2a)f(x)=-f(x)^2$ and so $f(a-x)^2+f(x)^2=1$

But $f(a-x)=-f(-a-x)=f(x+a)$ and so we got $f(x)^2=1-f(x+a)^2$

$P(x+a,x)$ $\implies$ $f(x+a)^2-f(x)^2=f(2x+a)$ and so $f(2x+a)=2f(x+a)^2-1$

Let then $g(x)=f(x+a)$. We get :
$g(0)=1$
$g(a)=0$
$g(x)>0$ $\forall x\in(0,a)$
$g(2x)=2g(x)^2-1$
$g(x)^2+g(a-x)^2=1$
$g(x)$ continuous.

Let then $h(x)=\cos(\frac{\pi x}{2a})$
From above, we get :
$g(x)=h(x)$ for $x=0$ and $x=a$
If $g(x)=h(x)$ for some $x\in[0,a]$, then $g(\frac x2)=h(\frac x2)$ and $g(a-x)=g(x)$

It's then immediate to conclude that $g(x)=h(x)$ $\forall x=\frac p{2^q}a$ with $p,q\in\mathbb N\cup\{0\}$ and $p\le 2^q$
And continuity gives then $g(x)=\cos(\frac{\pi x}{2a})$ $\forall x\in(0,a)$

So $f(x)=\sin(\frac{\pi x}{2a})$ $\forall x\in(0,a)$
Using then $f(-x)=-f(x)$ and $f(x+2a)=-f(x)$, we get $f(x)=\sin(\frac{\pi x}{2a})$ $\forall x$
And since we choosed $f(a)=1$, the general solution is $f(x)=b\sin(\frac{\pi x}{2a})$
Q.E.D.

It's easy to check back that this function indeed is a solution.

Hence the answer : $\boxed{f(x)=a\cdot\sin(bx)}$ where $a,b$ are arbitrary real numbers.
\end{solution}



\begin{solution}[by \href{https://artofproblemsolving.com/community/user/84645}{s7o0ory}]
	i really couldn't read the previous soloution ..
but here's my soloution ..
"i don't know why do i think my soloution is wrong :"
$first :
f(x)^2-f(y)^2=f(x+y)f(x-y)
 \Rightarrow [f(x)-f(y)][f(x)+f(y)]=f(x+y)f(x-y)]
 \Rightarrow \frac{{f(x + y)}}{{f(x) + f(y)}} = \frac{{f(x) - f(y)}}{{f(x - y)}}$
now : 
we assume that :
$\frac{{f(x + y)}}{{f(x) + f(y)}} = g(x + y)$
then:
$g(x+y)g(x-y)=1$
for all (x,y) :
now we plug :
$x->x-y$
$g(x)g(x + 2y) = 1$
we plug now x=n(constant)
then:
$g(2y + n) = \frac{1}{{g(n)}}
hence g is constant functin ..
so:
$f(x)=c$
but :
$f(x-y)f(x+y)=1$
so:
f(x) \equiv 1$
so:
$\frac{{f(x + y)}}{{f(x) + f(y)}}=1$
hence f is continous ..
so from Cauchy's Equation ..
$f(x)=ax$

bb :D
\end{solution}



\begin{solution}[by \href{https://artofproblemsolving.com/community/user/29428}{pco}]
	\begin{tcolorbox} $f(x)=ax$\end{tcolorbox}

Which, unfortunately, is not bounded, as requested, except if $a=0$

So you found the trivial solution $f(x)=0$ $\forall x$

Try to find the other continuous bounded solutions, now (or read my solution :) )
.
\end{solution}



\begin{solution}[by \href{https://artofproblemsolving.com/community/user/87195}{SCP}]
	\begin{tcolorbox}[quote="s7o0ory"] $f(x)=ax$\end{tcolorbox}

Which, unfortunately, is not bounded, as requested, except if $a=0$

So you found the trivial solution $f(x)=0$ $\forall x$

Try to find the other continuous bounded solutions, now (or read my solution :) )
.\end{tcolorbox}

Are you sure your solution is correct,
$f(x)=ax$ satisfies the question also:
$(f(x))^2-(f(y))^2=(ax)^2-(ay)^2=a^2(x^2-y^2)=[a(x+y)][a(x-y)]=f(x+y)*f(x-y)$,
so the correct solutions aren't all of one form, also the soluton $f(x)=a \cdot sin(bx)$ is correct.

We have to find one formal prove, who shows all the correct answers.
\end{solution}



\begin{solution}[by \href{https://artofproblemsolving.com/community/user/43015}{modularmarc101}]
	\begin{tcolorbox}[quote="pco"][quote="s7o0ory"] $f(x)=ax$\end{tcolorbox}

[color=#FF0000]Which, unfortunately, is not bounded, as requested, except if $a=0$.[\/color]

So you found the trivial solution $f(x)=0$ $\forall x$

Try to find the other continuous bounded solutions, now (or read my solution :) )
.\end{tcolorbox}

Are you sure your solution is correct,
$f(x)=ax$ satisfies the question also:
$(f(x))^2-(f(y))^2=(ax)^2-(ay)^2=a^2(x^2-y^2)=[a(x+y)][a(x-y)]=f(x+y)*f(x-y)$,
so the correct solutions aren't all of one form, also the soluton $f(x)=a \cdot sin(bx)$ is correct.

We have to find one formal prove, who shows all the correct answers.\end{tcolorbox}
\end{solution}



\begin{solution}[by \href{https://artofproblemsolving.com/community/user/87195}{SCP}]
	\begin{tcolorbox}[quote="pco"][quote="s7o0ory"] $f(x)=ax$\end{tcolorbox}

[color=#FF0000]Which, unfortunately, is not bounded, as requested, except if $a=0$.[\/color]

So you found the trivial solution $f(x)=0$ $\forall x$

Try to find the other continuous bounded solutions, now (or read my solution :) )
.\end{tcolorbox}

\end{tcolorbox}

Who can give the method which geves all the possible results?
(I misunderstood bounded, so sorry to pco, je vais voir ton résultat)
\end{solution}



\begin{solution}[by \href{https://artofproblemsolving.com/community/user/29428}{pco}]
	\begin{tcolorbox} Who can give the method which geves all the possible results?
(I misunderstood bounded, so sorry to pco, je vais voir ton résultat)\end{tcolorbox}

My post gives all the continuous bounded (which means that $f(\mathbb R)\subset[A,B]$ for some real $A,B$) solutions.
And $ax$ is not a bounded solution (look carefully at my post : I show that bounded continuous solutions are not injective, and so periodic)
\end{solution}



\begin{solution}[by \href{https://artofproblemsolving.com/community/user/29428}{pco}]
	If you are interested in non bounded continuous solutions, the steps for general solutions are :

1) $f(x)$ continuous non bounded implies $f(x)$ injective, and so monotonous 

2) limit the research at increasing solutions.

3) show that if it exists $u> 0$ such that $f(2u)=2f(u)$, then $f(x)=cx$

4) show that if it exists $u>0$ such that $f(2u)>2f(u)$, then $f(x)=c\cdot\sinh(ax)$

5) show that it does not exist any increasing solution such that $f(2x)<2f(x)$ for some $x>0$
\end{solution}
*******************************************************************************
-------------------------------------------------------------------------------

\begin{problem}[Posted by \href{https://artofproblemsolving.com/community/user/67223}{Amir Hossein}]
	Let $\mathbb Q$ be the set of all rational numbers and $\mathbb R$ be the set of real numbers. Function $f: \mathbb Q \to \mathbb R$ satisfies the following conditions:

(i) $f(0) = 0$, and for any nonzero $a \in Q, f(a) > 0.$
(ii) $f(x + y) = f(x)f(y) \qquad \forall x,y \in \mathbb Q.$
(iii) $f(x + y) \leq \max\{f(x), f(y)\} \qquad \forall x,y \in \mathbb Q , x,y \neq 0.$

Let $x$ be an integer and $f(x) \neq 1$. Prove that $f(1 + x + x^2+ \cdots + x^n) = 1$ for any positive integer $n.$
	\flushright \href{https://artofproblemsolving.com/community/c6h367483}{(Link to AoPS)}
\end{problem}



\begin{solution}[by \href{https://artofproblemsolving.com/community/user/29428}{pco}]
	\begin{tcolorbox}Let $\mathbb Q$ be the set of all rational numbers and $\mathbb R$ be the set of real numbers. Function $f: \mathbb Q \to \mathbb R$ satisfies the following conditions:

(i) $f(0) = 0$, and for any nonzero $a \in Q, f(a) > 0.$
(ii) $f(x + y) = f(x)f(y) \qquad \forall x,y \in \mathbb Q.$
(iii) $f(x + y) \leq \max\{f(x), f(y)\} \qquad \forall x,y \in \mathbb Q.$

Let $x$ be an integer and $f(x) \neq 1$. Prove that $f(1 + x + x^2+ \cdots + x^n) = 1$ for any positive integer $n.$\end{tcolorbox}

Obviously, no such function exist.
Setting $y=0$ in (ii) and using (i), we get $f(x)=0$ $\forall x\in\mathbb Q$, in contradiction with $f(a)>0$ stated in (i)
\end{solution}



\begin{solution}[by \href{https://artofproblemsolving.com/community/user/65499}{PIRISH}]
	\begin{tcolorbox}[quote="amparvardi"]Let $\mathbb Q$ be the set of all rational numbers and $\mathbb R$ be the set of real numbers. Function $f: \mathbb Q \to \mathbb R$ satisfies the following conditions:

(i) $f(0) = 0$, and for any nonzero $a \in Q, f(a) > 0.$
(ii) $f(x + y) = f(x)f(y) \qquad \forall x,y \in \mathbb Q.$
(iii) $f(x + y) \leq \max\{f(x), f(y)\} \qquad \forall x,y \in \mathbb Q.$

Let $x$ be an integer and $f(x) \neq 1$. Prove that $f(1 + x + x^2+ \cdots + x^n) = 1$ for any positive integer $n.$\end{tcolorbox}

Obviously, no such function exist.
Setting $y=0$ in (ii) and using (i), we get $f(x)=0$ $\forall x\in\mathbb Q$, in contradiction with $f(a)>0$ stated in (i)\end{tcolorbox}

No Patric you are not right
because x and y is rational numbers
\end{solution}



\begin{solution}[by \href{https://artofproblemsolving.com/community/user/29428}{pco}]
	\begin{tcolorbox}[quote="pco"][quote="amparvardi"]Let $\mathbb Q$ be the set of all rational numbers and $\mathbb R$ be the set of real numbers. Function $f: \mathbb Q \to \mathbb R$ satisfies the following conditions:

(i) $f(0) = 0$, and for any nonzero $a \in Q, f(a) > 0.$
(ii) $f(x + y) = f(x)f(y) \qquad \forall x,y \in \mathbb Q.$
(iii) $f(x + y) \leq \max\{f(x), f(y)\} \qquad \forall x,y \in \mathbb Q.$

Let $x$ be an integer and $f(x) \neq 1$. Prove that $f(1 + x + x^2+ \cdots + x^n) = 1$ for any positive integer $n.$\end{tcolorbox}

Obviously, no such function exist.
Setting $y=0$ in (ii) and using (i), we get $f(x)=0$ $\forall x\in\mathbb Q$, in contradiction with $f(a)>0$ stated in (i)\end{tcolorbox}

No Patric you are not right
because x and y is rational numbers\end{tcolorbox}

Huh ? What are you saying ?
(ii) $f(x + y) = f(x)f(y) \qquad \forall x,y \in \mathbb Q.$
I think that $0\in\mathbb Q$ and so I can set $y=0$ in this equality and I get $f(x)=f(x)f(0)$ $\forall x \in \mathbb Q.$

Using (i), I know that $f(0)=0$ and so $f(x)=0$ $\forall x \in \mathbb Q.$

And (i) says : "for any nonzero $a \in Q, f(a) > 0.$ which is in contradiction with $f(x)=0$ $\forall x \in \mathbb Q.$

isn't it ?

Could you kindly explain where I am wrong, please ?
\end{solution}



\begin{solution}[by \href{https://artofproblemsolving.com/community/user/65499}{PIRISH}]
	I think 0 is not rational
\end{solution}



\begin{solution}[by \href{https://artofproblemsolving.com/community/user/29428}{pco}]
	\begin{tcolorbox}I think 0 is not rational\end{tcolorbox}

I think it is :)
\end{solution}



\begin{solution}[by \href{https://artofproblemsolving.com/community/user/65499}{PIRISH}]
	I am not sure but i think
take ii and iii  conditions f(x)<=1 then f(x+y)<=f(x) given function is increasing and continuosly
by continuty  we take exist x>1 when f(1+x+x^{2}+\cdots+x^{n}) = 1
\end{solution}



\begin{solution}[by \href{https://artofproblemsolving.com/community/user/29428}{pco}]
	Please, amparvardi, since you have the solution, could you kindly check the problem statement ?
\end{solution}



\begin{solution}[by \href{https://artofproblemsolving.com/community/user/2616}{zhaoli}]
	The statement is translated from a Chinese book. I go through the solution in the book, and find that $x, y$ should be non-zero rationals.
Outline of the solution:

(1) $f(1)=1$.
(2) $f(-1)=1$.
(3) $f(m) \leq 1$ for all positive integers $m$.
(4) $f(m) \leq 1$ for all integers $m$.
(5) If integer $x$ satisfies $f(x) < 1$, then $f(1+x)=1$.
(6) induction on $n$.
\end{solution}
*******************************************************************************
-------------------------------------------------------------------------------

\begin{problem}[Posted by \href{https://artofproblemsolving.com/community/user/67223}{Amir Hossein}]
	Function $f(x, y): \mathbb N \times \mathbb N \to \mathbb Q$ satisfies the conditions:

(i) $f(1, 1) =1$,

(ii) $f(p + 1, q) + f(p, q + 1) = f(p, q)$ for all $p, q \in  \mathbb N$, and

(iii) $qf(p + 1, q) = pf(p, q + 1)$ for all $p, q \in  \mathbb N$.

Find $f(1990, 31).$
	\flushright \href{https://artofproblemsolving.com/community/c6h367536}{(Link to AoPS)}
\end{problem}



\begin{solution}[by \href{https://artofproblemsolving.com/community/user/29428}{pco}]
	\begin{tcolorbox}Function $f(x, y): \mathbb N \times \mathbb N \to \mathbb Q$ satisfies the conditions:

\begin{italicized}(i) \end{italicized}$f(1, 1) =1;$

\begin{italicized}(ii) \end{italicized}$f(p + 1, q) + f(p, q + 1) = f(p, q)$ for all $p, q \in  \mathbb N;$

\begin{italicized}(iii) \end{italicized}$qf(p + 1, q) = pf(p, q + 1)$ for all $p, q \in  \mathbb N$

Find $f(1990, 31).$\end{tcolorbox}
From (iii) we get $f(p+1,q)=\frac pqf(p,q+1)$ and induction gives $f(p,q)=\frac 1{\binom{p+q-2}{q-1}}f(1,p+q-1)$

Plugging this in (ii), we get $f(1,p+q)=\frac{p+q-1}{p+q}f(1,p+q-1)$ and so $f(1,n)=\frac 1n$

Setting this in previous line, we get $\boxed{f(p,q)=\frac {(p-1)!(q-1)!}{(p+q-1)!}}$ which indeed is a solution.

So $f(1990,31)=\frac{1989!30!}{2020!}$
\end{solution}



\begin{solution}[by \href{https://artofproblemsolving.com/community/user/294722}{Solumilkyu}]
	It is easy to see that for $p,q\in\mathbb{N}$,
$$f(p,q)=f(p+1,q)+f(p,q+1)=f(p+1,q)+\frac{q}{p}f(p+1,q)=\frac{p+q}{p}f(p+1,q)$$
and similarly $f(p,q)=\frac{p+q}{q}f(p,q+1)$. So we obtain the following results.
\begin{align*}
f(p+1,q)=\frac{p}{p+q}f(p,q)\quad\text{and}\quad f(p,q+1)=\frac{q}{p+q}f(p,q)
\end{align*}
Thus
\begin{align*}
f(1990,31)
&=\frac{1989}{2020}f(1989,31)=\frac{1989\cdot 1988}{2020\cdot 2019}f(1988,31)\\
&\,\,\vdots\\
&=\frac{1989!}{2020\cdot 2019\cdots 32}f(1,31)\\
&=\frac{1989!\cdot 30}{2020\cdot 2019\cdots 32\cdot 31}f(1,30)=\frac{1989!\cdot 30\cdot 29}{2020\cdot 2019\cdots 32\cdot 31\cdot 30}f(1,29)\\
&\,\,\vdots\\
&=\frac{1989!\cdot 30!}{2020!}
\end{align*}
\end{solution}
*******************************************************************************
-------------------------------------------------------------------------------

\begin{problem}[Posted by \href{https://artofproblemsolving.com/community/user/79198}{SKhan}]
	Find all functions $f:\mathbb{R}\to\mathbb{R}$ so that
\[f(x^2+y^2)=f(f(x))+f(xy)+f(f(y))\]
for all real numbers $x$ and $y$.
	\flushright \href{https://artofproblemsolving.com/community/c6h367724}{(Link to AoPS)}
\end{problem}



\begin{solution}[by \href{https://artofproblemsolving.com/community/user/29428}{pco}]
	\begin{tcolorbox}Find all functions $f$ such that:

$f:\mathbb{R}\rightarrow \mathbb{R}$
$f(x^2+y^2)=f(f(x))+f(xy)+f(f(y))$ for all real numbers $x$ and $y$\end{tcolorbox}
I think there surely exists a shorter solution, but here is the best I found up to now   :oops:  :

Let $P(x,y)$ be the assertion $f(x^2+y^2)=f(f(x))+f(xy)+f(f((y))$


$P(0,0)$ $\implies$ $f(f(0))=0$
Subtracting $P(x,0)$ and $P(0,y)$ from $P(x,y)$, we get new assertion $Q(x,y)$ : $f(x^2+y^2)=f(x^2)+f(xy)+f(y^2)-2f(0)$

Comparing $Q(x,1)$ and $Q(x,-1)$, we get $f(-x)=f(x)$

Let then $g(x)=f(x)-f(0)$. $Q(x,y)$ $\implies$ new assertion $R(x,y)$ : $g(x+y)=g(x)+g(y)+g(\sqrt{xy})$ $\forall x,y\ge 0$

Let then $x,y,z\ge 0$ :

$g(x+y+z)=g(x+(y+z))=g(x)+g(y+z)+g(\sqrt{xy+xz})$ $=g(x)+g(y)+g(z)+g(\sqrt{yz})+g(\sqrt{xy+xz})$
$g(x+y+z)=g((x+y)+z)=g(x+y)+g(z)+g(\sqrt{xz+yz})$ $=g(x)+g(y)+g(z)+g(\sqrt{xy})+g(\sqrt{xz+yz})$

And so $g(\sqrt{yz})+g(\sqrt{xy+xz})=g(\sqrt{xy})+g(\sqrt{xz+yz})$ $\forall x,y,z\ge 0$

So $g(\sqrt{x})+g(\sqrt{y+z})=g(\sqrt{x+y})+g(\sqrt{z})$ $\forall x,y,z>0$

Let $h(x)=g(\sqrt{x})$ and we got $h(x)+h(y+z)=h(x+y)+h(z)$ $\forall x,y,z>0$

From there, it's easy to get $h(x)=c(x)+b$ $\forall x>0$ where $c(x)$ is any solution of Cauchy equation.
and so $g(x)=c(x^2)+b$ $\forall x>0$

But $R(x,x)$ $\implies$ $g(2x^2)=3g(x^2)$ and so $4c(x^4)+b=3c(x^2)+3b$ and so $g(x)=$ constant $=0$ $\forall x>0$
So $g(x)=0$ $\forall x$
And $f(x)=c$
Plugging this in original equation, we get $\boxed{f(x)=0\text{    }\forall x}$
\end{solution}
*******************************************************************************
-------------------------------------------------------------------------------

\begin{problem}[Posted by \href{https://artofproblemsolving.com/community/user/81769}{arshakus}]
	Find all functions $f: \mathbb Z \to \mathbb Z$ such that $f(x)f(-x)=f(x^2)$ and $f(x+y)=f(x)+f(y)+2xy$ for all integers $x$ and $y$.
	\flushright \href{https://artofproblemsolving.com/community/c6h367902}{(Link to AoPS)}
\end{problem}



\begin{solution}[by \href{https://artofproblemsolving.com/community/user/67223}{Amir Hossein}]
	Let $g(x)$ be a function such that $f(x)=g(x)+x^2$, substitute in second condition, we get
\[ f(x+y)=f(x)+f(y)+2xy \implies g(x+y)+(x+y)^2=g(x)+x^2+g(y)+y^2+2xy\]\[\implies g(x+y)=g(x)+g(y)\]
By Cauchy famous functions, we get that $g(x)=ax \quad \forall x \in \mathbb Z$ where $a \in \mathbb Z.$ 
[PS. In general form, the only solution of the \begin{italicized}continuous\end{italicized} function $g : \mathbb R \to \mathbb R$ such that $g(x+y)=g(x)+g(y)$, is $g(x)=ax \quad  \forall x \in \mathbb R$ where $a \in \mathbb R.$
First I was writing this solution, I thought that there's a problem with continuity, but when I checked the solution of this kind of Cauchy functions, I saw that continuity is necessary just for the function $g : \mathbb R \to \mathbb R.$] 

Plugging back $g(x)=ax$ in $f(x)=g(x)+x^2$, we have $f(x)=x(a+x) \quad \forall x \in \mathbb Z$ where $a \in \mathbb Z.$
Check this result in the first condition, we get
\[f(x) \cdot f(-x)=f(x^2) \implies x(a+x) \cdot  -x(a-x) = x^2(a+x^2)\]\[\implies -a^2=a \implies a=0 \text{ or } a=-1\]
So the solutions are $f(x)=x^2$ and $f(x)=x(x-1).$

I hope I didn't miss something. Others (specially you, Mr. pco) please check my solution :) .
\end{solution}



\begin{solution}[by \href{https://artofproblemsolving.com/community/user/86553}{YasserR}]
	Hello
$f(0)=0$
$f(1)\times f(-1)=f(1)$ and $f(1)+f(-1)=2$
$f(1)=0$ or$ f(1)=1$
If $ f(1)=0 $ we get : $ f(x+1)= f(x) +2x $
$ f(x)=f(1)+2(1+2+3+........+(n-1))=n(n-1)$
If $f(1)=1$ we get : $f(x+1)=f(x)+1+2x$
$ f(x) =f(1)+(n-1)+2(1+2+3+4......+(n-1))=n^2$
we conclud :$ f(n)=n(n-1)$ Or $ g(n)=n^2$
\end{solution}



\begin{solution}[by \href{https://artofproblemsolving.com/community/user/29428}{pco}]
	\begin{tcolorbox} ...So the solutions are $f(x)=x^2$ and $f(x)=x(x-1).$

I hope I didn't miss something. Others (specially you, Mr. pco) please check my solution :) .\end{tcolorbox}

This seems quite OK for me.

And about Cauchy, you're right : $f(x+y)=f(x)+f(y)$ implies $f(x)=xf(1)$ over $\mathbb N,\mathbb Z$ or $\mathbb Q$. The only problem is the extension to $\mathbb R$ which needs any of the equivalent complementary conditions :
$f(x)$ is continuous
$f(x)$ is monotonous
$f(x)$ has a lower bound on some non empty open interval $(a,b)$ where $a<b$
$f(x)$ has an upper bound on some non empty open interval $(a,b)$ where $a<b$
\end{solution}



\begin{solution}[by \href{https://artofproblemsolving.com/community/user/81769}{arshakus}]
	ok thank u guys...
\end{solution}



\begin{solution}[by \href{https://artofproblemsolving.com/community/user/81769}{arshakus}]
	\begin{tcolorbox}Hello
$f(0)=0$
$f(1)\times f(-1)=f(1)$ and $f(1)+f(-1)=2$
$f(1)=0$ or$ f(1)=1$
If $ f(1)=0 $ we get : $ f(x+1)= f(x) +2x $
$ f(x)=f(1)+2(1+2+3+........+(n-1))=n(n-1)$
If $f(1)=1$ we get : $f(x+1)=f(x)+1+2x$
$ f(x) =f(1)+(n-1)+2(1+2+3+4......+(n-1))=n^2$
we conclud :$ f(n)=n(n-1)$ Or $ g(n)=n^2$\end{tcolorbox}
hey,
I think like u at first but then realized that it is not true, because after all this what u wrote we must prove that there was no other function for this 2 equations.....
\end{solution}
*******************************************************************************
-------------------------------------------------------------------------------

\begin{problem}[Posted by \href{https://artofproblemsolving.com/community/user/67223}{Amir Hossein}]
	Let $f(x)$ be a continuous function defined on the closed interval $0 \leq x \leq 1$. Let $G(f)$ denote the graph of $f(x): G(f) = \{(x, y) \in \mathbb R^2 | 0 \leq$$ x \leq 1, y = f(x) \}$. Let $G_a(f)$ denote the graph of the translated function $f(x - a)$ (translated over a distance $a$), defined by $G_a(f) = \{(x, y) \in \mathbb R^2 | a \leq x \leq a + 1, y = f(x - a) \}$. Is it possible to find for every $a, \  0 < a < 1$, a continuous function $f(x)$, defined on $0 \leq x \leq 1$, such that $f(0) = f(1) = 0$ and $G(f)$ and $G_a(f)$ are disjoint point sets ?
	\flushright \href{https://artofproblemsolving.com/community/c6h367916}{(Link to AoPS)}
\end{problem}



\begin{solution}[by \href{https://artofproblemsolving.com/community/user/29428}{pco}]
	\begin{tcolorbox}Let $f(x)$ be a continuous function defined on the closed interval $0 \leq x \leq 1$. Let $G(f)$ denote the graph of $f(x): G(f) = \{(x, y) \in \mathbb R^2 | 0 \leq$$ x \leq 1, y = f(x) \}$. Let $G_a(f)$ denote the graph of the translated function $f(x - a)$ (translated over a distance $a$), defined by $G_a(f) = \{(x, y) \in \mathbb R^2 | a \leq x \leq a + 1, y = f(x - a) \}$. Is it possible to find for every $a, \  0 < a < 1$, a continuous function $f(x)$, defined on $0 \leq x \leq 1$, such that $f(0) = f(1) = 0$ and $G(f)$ and $G_a(f)$ are disjoint point sets ?\end{tcolorbox}

No, it's not.

Choose for example $a=\frac 12$ so that the two curves are $f(x)$ and $f_1(x)=f(x-\frac 12)$

For $x=\frac 12$, we get $f(x)-f_1(x)=f(\frac 12)$ and for $x=1$, we get $f(x)-f_1(x)=-f(\frac 12)$

And so, since continuous, $\exists u\in[\frac 12,1]$ such that $f(u)=f_1(u)$

In fact it's possible to show that we can find such $f(x)$ for any $a\in(0,1)$ except for $a=\frac 1k$ where $k$ is a positive integer $>1$
\end{solution}
*******************************************************************************
-------------------------------------------------------------------------------

\begin{problem}[Posted by \href{https://artofproblemsolving.com/community/user/65464}{sororak}]
	Find all functions $f:\mathbb{R}^{+}\to \mathbb{R}^{+}$ such that for all positive real numbers $x$ and $y$, the following equation holds:
\[(x+y)f(f(x)y)=x^2f(f(x)+f(y)).\]
	\flushright \href{https://artofproblemsolving.com/community/c6h367952}{(Link to AoPS)}
\end{problem}



\begin{solution}[by \href{https://artofproblemsolving.com/community/user/29428}{pco}]
	\begin{tcolorbox}Let $\mathbb{R}^{+}$ be the set of positive real numbers. Find all functions $f:\mathbb{R}^{+}\to \mathbb{R}^{+}$ such that for all positive real numbers $x,y$ the equation holds:
$(x+y)f(f(x)y)=x^2f(f(x)+f(y))$\end{tcolorbox}
Let $P(x,y)$ be the assertion $(x+y)f(f(x)y)=x^2f(f(x)+f(y))$

If $f(a)=f(b)$, then comparing $P(a,x)$ and $P(b,x)$ implies $\frac{x+a}{a^2}=\frac{x+b}{b^2}$ $\forall x>0$ and so $a=b$ and $f(x)$ is injective.

$P(\frac 32,\frac 34)$ $\implies$ $f(\frac 34f(\frac 32))=f(f(\frac 32)+f(\frac 34))$ and, since injective :

$\frac 34f(\frac 32)=f(\frac 32)+f(\frac 34)$ which implies $f(\frac 34)+\frac 14f(\frac 32)=0$, which is impossible since $f(x)$ is from $\mathbb R^+\to\mathbb R^+$

So no solution to this equation.
\end{solution}



\begin{solution}[by \href{https://artofproblemsolving.com/community/user/125018}{horizon}]
	my solution
obviously,if $f(a)=f(b)$ then we have $a=b$,let $y=x^{2}-x$,here $x>1$,then we have
$f((x^{2}-x)f(x))=f(f(x)+f(x^{2}-x))$ 
hences to $(x^{2}-x-1)f(x)=f(x^{2}-x)$,let x be the larger root of $x^{2}-x-1=0$
a contradiction!
\end{solution}



\begin{solution}[by \href{https://artofproblemsolving.com/community/user/295258}{Ferid.---.}]
	Does this work?
Let $P(x,y)$ be the assertion 
$(x+y)f(f(x)y)=x^2 f(f(x)+f(y)).$
$\exists a,b\in\mathbb{R}^{+},$ such that $f(a)=f(b),$ comparing $P(a,x)$ and $P(b,x),$ we find $\frac{a^2}{a+x}=\frac{b^2}{b+x}\to (a-b)(ab+ax+bx)=0\to a=b,$ since $ab+ax+bx>0.$ Then $f$ is injective.
From $P(x,y)$ and $P(y,x)$ we find $$f(f(y)x)=f(f(x)y)\to ^{injectivity} \frac{f(y)}{y}=\frac{f(x)}{x}=a,$$ where $a\in\mathbb{R}^{+}.$
But in our functional equation we replace $f(x)=ax$ we find $x^2=xy$ but this is not true for infinitely $x,y.$
Them no solution to this fe.
\end{solution}



\begin{solution}[by \href{https://artofproblemsolving.com/community/user/342006}{m.yekta}]
	\begin{tcolorbox}.
From $P(x,y)$ and $P(y,x)$ we find $$f(f(y)x)=f(f(x)y)\to ^{injectivity} \frac{f(y)}{y}=\frac{f(x)}{x}=a,$$ where $a\in\mathbb{R}^{+}.$
\end{tcolorbox}
I guess you've made a mistake here. From $P(x,y)$ and $P(y,x)$ we get:$$f(yf(x))x^2=f(xf(y))y^2$$
\end{solution}



\begin{solution}[by \href{https://artofproblemsolving.com/community/user/295258}{Ferid.---.}]
	\begin{tcolorbox}[quote=Ferid.---.].
From $P(x,y)$ and $P(y,x)$ we find $$f(f(y)x)=f(f(x)y)\to ^{injectivity} \frac{f(y)}{y}=\frac{f(x)}{x}=a,$$ where $a\in\mathbb{R}^{+}.$
\end{tcolorbox}
I guess you've made a mistake here. From $P(x,y)$ and $P(y,x)$ we get:$$f(yf(x))x^2=f(xf(y))y^2$$\end{tcolorbox}

Thank you.
\end{solution}
*******************************************************************************
-------------------------------------------------------------------------------

\begin{problem}[Posted by \href{https://artofproblemsolving.com/community/user/90103}{Winner2010}]
	Find all functions $f:\mathbb{R}\to\mathbb{R}$ such that
\[f(x+y^2) \geq (y+1)f(x)\]
for all $x, y\in \mathbb{R}$.
	\flushright \href{https://artofproblemsolving.com/community/c6h367972}{(Link to AoPS)}
\end{problem}



\begin{solution}[by \href{https://artofproblemsolving.com/community/user/29428}{pco}]
	\begin{tcolorbox}Find all functions $f:\mathbb{R}\rightarrow\mathbb{R}$ such that

$f(x+y^2) \geq (y+1)f(x)$

for all $x, y\in \mathbb{R}$\end{tcolorbox}
Let $P(x,y)$ be the assertion $f(x+y^2)\ge (y+1)f(x)$

$P(x-1,-1)$ $\implies$ $f(x)\ge 0$
Let $y>0$ : 
$P(x+y^2,y)$ $\implies$ $f(x+2y^2)\ge(y+1)f(x+y^2)\ge (y+1)^2f(x)$
$P(x+2y^2,y)$ $\implies$ $f(x+3y^2)\ge(y+1)f(x+2y^2)\ge (y+1)^3f(x)$
and a simple induction gives $f(x+ny^2)\ge(y+1)^nf(x)$

Setting then $y=\frac 1n$, we get $f(x+\frac 1n)\ge(1+\frac 1n)^nf(x)$

So $f(x+\frac 2n)\ge(1+\frac 1n)^nf(x+\frac 1n)\ge (1+\frac 1n)^{2n}f(x)$

And a second induction gives $f(x+\frac kn)\ge (1+\frac 1n)^{kn}f(x)$

And so $f(x+1)\ge (1+\frac 1n)^{n^2}f(x)$ and $f(x+1)(1+\frac 1n)^{-n^2}\ge f(x)$

Setting $n\to +\infty$ in this inequality, we get $f(x)\le 0$

Hence the unique solution :$\boxed{f(x)=0\forall x}$
\end{solution}
*******************************************************************************
-------------------------------------------------------------------------------

\begin{problem}[Posted by \href{https://artofproblemsolving.com/community/user/90103}{Winner2010}]
	Find all functions $f: \mathbb{R}\rightarrow\mathbb{R}$ such that $f(0)\leq 0$ and $f(x+y)\leq x+f(f(x))$ for any $x,y\in \mathbb{R}$.
	\flushright \href{https://artofproblemsolving.com/community/c6h367995}{(Link to AoPS)}
\end{problem}



\begin{solution}[by \href{https://artofproblemsolving.com/community/user/29428}{pco}]
	\begin{tcolorbox}Find all functions $f:\mathbb{R}\rightarrow\mathbb{R}$ such that $f(0)\leq 0$ and $f(x+y)\leq x+f(f(x))$ for any $x,y\in \mathbb{R}$\end{tcolorbox}

Let $P(x,y)$ be the assertion $f(x+y)\le x+f(f(x))$

$P(0,f(x))$ $\implies$ $f(f(x))\le f(f(0))$ and so $P(x,y)$ implies new assertion $Q(x,y)$ : $f(x+y)\le x+f(f(0))$

$Q(y-f(f(0)),x+f(f(0))-y)$ $\implies$ $f(x)\le y$ $\forall x,y$ and so no solution.
\end{solution}
*******************************************************************************
-------------------------------------------------------------------------------

\begin{problem}[Posted by \href{https://artofproblemsolving.com/community/user/65464}{sororak}]
	Let $a,b,c,$ and $d$ be real numbers such that at least one of $c$ and $d$ is non-zero. Let $ f:\mathbb{R}\to\mathbb{R}$ be a function defined as $f(x)=\frac{ax+b}{cx+d}$. Suppose that for all $x\in\mathbb{R}$, we have $f(x) \neq x$. Prove that if there exists some real number $a$ for which $f^{1387}(a)=a$, then for all $x$ in the domain of $f^{1387}$, we have $f^{1387}(x)=x$. Notice that in this problem,
\[f^{1387}(x)=\underbrace{f(f(\cdots(f(x)))\cdots)}_{\text{1387 times}}.\]

\begin{italicized}Hint\end{italicized}. Prove that for every function $g(x)=\frac{sx+t}{ux+v}$, if the equation $g(x)=x$ has more than $2$ roots, then $g(x)=x$ for all $x\in\mathbb{R}-\left\{\frac{-v}{u}\right\}$.
	\flushright \href{https://artofproblemsolving.com/community/c6h368215}{(Link to AoPS)}
\end{problem}



\begin{solution}[by \href{https://artofproblemsolving.com/community/user/64868}{mahanmath}]
	[url=http://www.artofproblemsolving.com/Forum\/resources.php?c=25&cid=58&year=1990]Barzil 1990 - P5[\/url]
\end{solution}
*******************************************************************************

-------------------------------------------------------------------------------

\begin{problem}[Posted by \href{https://artofproblemsolving.com/community/user/67223}{Amir Hossein}]
	Prove that the functional equations
\[f(x + y) = f(x) + f(y),\]
\[ \text{and} \qquad  f(x + y + xy) = f(x) + f(y) + f(xy) \quad (x, y \in \mathbb R)\]
are equivalent.
	\flushright \href{https://artofproblemsolving.com/community/c4h366996}{(Link to AoPS)}
\end{problem}



\begin{solution}[by \href{https://artofproblemsolving.com/community/user/80482}{mathwarrior557}]
	Not true. Let f(x) be x^2. Then f(4)=16 but f(2)+f(2)=2*4=8.
\end{solution}



\begin{solution}[by \href{https://artofproblemsolving.com/community/user/67223}{Amir Hossein}]
	\begin{tcolorbox}Not true. Let f(x) be x^2. Then f(4)=16 but f(2)+f(2)=2*4=8.\end{tcolorbox}


Are you sure ?

If $f(x)=x^2$ then $f(x+y)=(x+y)^2=x^2+y^2+2xy$ and $f(x)=x^2, f(y)=y^2.$ But we had $f(x+y)=f(x)+f(y)$, this implies $f(x) = x^2$ is not true for this function.

Remark. $f(x)=cx$ is the only solution of the functional equation $f(x+y)=f(x)+f(y)$ where $c$ is a constant real number.
\end{solution}



\begin{solution}[by \href{https://artofproblemsolving.com/community/user/81458}{3333}]
	Assume f(a) + f(b) = f(a+b) but f(ab) + f(a) + f(b) =\= f(a+b+ab)

Then:
 f(a+b+ab)  =\= f(ab) + f(a+b)

lets define ab=k and a+b= n

f(k+n) =\= f(k) + f(n)

contradiction!
\end{solution}



\begin{solution}[by \href{https://artofproblemsolving.com/community/user/37259}{math154}]
	\begin{bolded}Note:\end{bolded} A moderator should move this to the Olympiad Algebra forum.

We only prove the nontrivial direction. Let $P(x,y)\implies f(xy+x+y)=f(xy)+f(x)+f(y)$. First
\[P(x,0)\implies f(0)=0,\\
P(x,1)\implies f(2x+1)=2f(x)+1,\\
P(x,-1)\implies f(x)=-f(-x).\]Thus we find
\begin{align*}
[2f(x)+1]-[2f(x-1)+1]
&= f(2x+1)-f(2x-1)\\
&=f(2x+1)+f(-2x+1)\\
&=[2f(x)+f(1)]+[2f(-x)+f(1)]=2f(1),
\end{align*}so
\[f(x)-f(x-1)=f(1)\]and
\[2f(x)+f(1)=f(2x+1)=f(2x)+f(1)\implies f(2x)=2f(x).\]Now
\[P(x,y)+P(x,-y)\implies f(xy+x+y)+f(-xy+x-y)=2f(x)=f(2x),\]whence
\[f(2x)+f(xy-x+y)=f(xy+x+y).\]Letting $u=2x$ and $v=xy-x+y$ (so that we cover all $(u,v)$ with $x=u\/2$ and $y=(u+2v)\/(u+2)$ when $u\ne-2$),
\[f(u)+f(v)=f(u+v).\]If $u=-2$, then
\[f(-2)+f(v)=f(v)-f(2)=f(v)-2f(1)=f(v-2)=f(u+v),\]so we're done.
\end{solution}
*******************************************************************************
-------------------------------------------------------------------------------

\begin{problem}[Posted by \href{https://artofproblemsolving.com/community/user/87}{hxtung}]
	Find all function $f:R \rightarrow R$ such that:
$f(f(x+y))=f(x+y)+f(x)f(y)-xy;\forall x,y \in R$.
	\flushright \href{https://artofproblemsolving.com/community/c6h6409}{(Link to AoPS)}
\end{problem}



\begin{solution}[by \href{https://artofproblemsolving.com/community/user/373}{matamata}]
	First time posting, hope its right...  :blush: 
First, we show f(0) = 0
Since f(f(0)) = f(0) + f(0)*f(0) and f(f(0)) = f(0) + f(x)*f(-x) + x^2 
f(0)*f(0) = f(x)*f(-x) + x^2 for all real x   Let f(0) = c, so f(c) = c + c^2
c^2 = f(x)*f(-x) + x^2   let x = c, so c^2 = f(c)*f(-c) + c^2, so either f(c) = 0 or f(-c) = 0.
If f(-c) = 0, then f(f(-c)) = f(-c) + f(0)*f(-c)   so f(0) = 0
If f(c) = 0, c^2 +c = 0 so c = 0 or c = -1.
If c = -1, then f(f(x)) = f(x) + f(x)*f(0) = 0  So f(x) = 0 for all x in f's range
But then 0 must be f's range, since f(some element) = 0, so f(0) = 0, and we have a contradiction.
Since f(0) = 0
f(f(x)) = f(x)  but f(x) = x for all reals only in f's range 
x+y = f(x+y) + f(x)*f(y) -xy  so if y = 0, f(x) = x, for all real x...
\end{solution}



\begin{solution}[by \href{https://artofproblemsolving.com/community/user/183}{Valiowk}]
	\begin{tcolorbox}x+y = f(x+y) + f(x)*f(y) -xy  so if y = 0, f(x) = x, for all real x...\end{tcolorbox}

I'm sorry, but I don't understand how you get this line.

Since $f(f(x)) = f(x) \forall x$ we have $f(x)f(y) = xy$.  Now the range of $f$ cannot be $\{0\}$ only, hence take any $x \neq 0$ in the range of $f$.  Then we have $xy = f(x)f(y) = xf(y)$ and hence $f(x) = x \forall x$.
\end{solution}



\begin{solution}[by \href{https://artofproblemsolving.com/community/user/373}{matamata}]
	ack! I knew I'd do something stupid like that...
What you did is how I solved it on paper, but while typing it up, I thought that "f(f(x)) = f(x) implies f(f(x+y)) = x+y, it's even simpler!", sorry  :blush:
\end{solution}



\begin{solution}[by \href{https://artofproblemsolving.com/community/user/143543}{dudeldai}]
	functional equations are a new topic for me ... so i have a question 

it's quite easy to proove that f has to be injective.
now we let f(x+y)=-f(x)*f(y)
and plugging in the initiial equation we get f(-f(x)*f(y))=-xy.
since f is injective -->f(x)*f(y)=x*y
and finally for x=y , f(x)^2=x^2 , f(x)=x.

is it possible ? please help me i'm tryn to get better in this stuff . 
thx
\end{solution}



\begin{solution}[by \href{https://artofproblemsolving.com/community/user/29428}{pco}]
	\begin{tcolorbox}functional equations are a new topic for me ... so i have a question 

it's quite easy to proove that f has to be injective.
now we let f(x+y)=-f(x)*f(y)
and plugging in the initiial equation we get f(-f(x)*f(y))=-xy.
since f is injective -->f(x)*f(y)=x*y
and finally for x=y , f(x)^2=x^2 , f(x)=x.

is it possible ? please help me i'm tryn to get better in this stuff . 
thx\end{tcolorbox}
1) Show us your proof that $f(x)$ is injective. It's true but it does not seem so simple (I had to go up to the solution $f(x)=x$ to prove injectivity)

2) what is the meaning of "let $f(x+y)=-f(x)*f(y)$" ?
If you mean that $f(x+y)=-f(x)f(y)$ $\forall x,y$, then it's false
If you mean that you choose some specific values $x,y$ such that  $f(x+y)=-f(x)f(y)$, then :
2.1. You must we aware that you did not prove that such $x,y$ exist
2.2. You must be aware that your conclusions will only be available for these specific values of $x,y$ and will not be general conclusions

3) How can you conclude from $f(-f(x)f(y))=-xy$ that $f(x)f(y)=xy$ ????? This is absolutely not a conclusion of the [not-yet-proved] injectivity.
Injection allow you to conclude from $f(a)=f(b)$ that $a=b$. That's all.

4) If you succeed to prove that $f^2(x)=x^2$, you can not immediately conclude $f(x)=x$. The only thing you could then conclude is that :
$\forall x$, either $f(x)=x$, either $f(x)=-x$
For example, the function $f(x)=(-1)^{\lfloor 32\sin x\rfloor}x$ is such that $f^2(x)=x^2$ $\forall x$
\end{solution}



\begin{solution}[by \href{https://artofproblemsolving.com/community/user/143543}{dudeldai}]
	thx for helping me! i still have a question.
So in point 2) you want to say that if i want to substitute a value for f
for example in my case f(x+y)=f(x)f(y)(i see now that it is totaly wrong) i have to show that it is true for 
every x,y  to make general conlcusions and finally find all possible functions or can i even find some specific
x,y  to find general conclusions?I mean when we try to find out more about a function most of the time we first try to make specific conclusions like f(1) and f(0) ecc.. but with these we can still make genral conclusions in the end..so my question : if i find an equation like f(x+y)=f(x)f(y) and i prove it for  some specific x,y .. plugging it in the initial  equation can i make general conclusions ??? 

i made some strange conclusions yesterday i was really tired .. sorry
\end{solution}



\begin{solution}[by \href{https://artofproblemsolving.com/community/user/29428}{pco}]
	If you choose $x,y$ with a given constraint, then your conclusion is valid only for those $x,y$.
\end{solution}



\begin{solution}[by \href{https://artofproblemsolving.com/community/user/29876}{ozgurkircak}]
	\begin{tcolorbox}Find all function $f:R \rightarrow R$ such that:

$f(f(x+y))=f(x+y)+f(x)f(y)-xy;\forall x,y \in R$.\end{tcolorbox}
Here is my solution: Replacing $y=0$ we get $f(f(x))=f(x)[1+f(0)]$ So for all $t \in Range$ we have $f(t)=ct$ where $c=1+f(0).$
The original equation forces c to be 1. So for all $t \in Range$ we have $f(t)=t.$
Now the original equation becomes 
$f(x+y)=f(x+y)+f(x)f(y)-xy \Longrightarrow f(x)f(y)=xy$ fixing y for a value s.t $f(y) \neq 0$ (we have such a point since $f \equiv 0$ is not a solution) we get $f(x)=ax$ and original equation gives $a=1.$
So $f(x)=x$ for all $x \in \mathbb{R}.$
\end{solution}



\begin{solution}[by \href{https://artofproblemsolving.com/community/user/29428}{pco}]
	\begin{tcolorbox}Here is my solution: Replacing $y=0$ we get $f(f(x))=f(x)[1+f(0)]$ So for all $t \in Range$ we have $f(t)=ct$ where $c=1+f(0).$
The original equation forces c to be 1. \end{tcolorbox}
Why this last sentence ?

Pluging $f(x)=cx$ $\forall x\in f(\mathbb R)$ in original equation, we get $(c-1)f(x+y)=f(x)f(y)-xy$ $\forall x,y\in\mathbb R$ and this does not allow to conclude $c=1$

Replacing $x,y$ by $f(x),f(y)$ in the above equation, we get $(c-1)f(f(x)+f(y))=(c^2-1)f(x)f(y)$ but this still does not allow to conclude $c=1$
\end{solution}



\begin{solution}[by \href{https://artofproblemsolving.com/community/user/29428}{pco}]
	\begin{tcolorbox}Find all function $f:R \rightarrow R$ such that:
$f(f(x+y))=f(x+y)+f(x)f(y)-xy;\forall x,y \in R$.\end{tcolorbox}
Let $P(x,y)$ be the assertion $f(f(x+y))=f(x+y)+f(x)f(y)-xy$

(a) : $P(1,1)$ $\implies$ $f(f(2))=f(2)+f(1)^2-1$
(b) : $P(2,0)$ $\implies$ $f(f(2))=f(2)+f(2)f(0$
(c) : $P(x,1)$ $\implies$ $f(f(x+1))=f(x+1)+f(x)f(1)-x$
(d) : $P(x,2)$ $\implies$ $f(f(x+2))=f(x+2)+f(x)f(2)-2x$
(e) : $P(x+1,0)$ $\implies$ $f(f(x+1))=f(x+1)+f(x+1)f(0)$
(f) : $P(x+1,1)$ $\implies$ $f(f(x+2))=f(x+2)+f(x+1)f(1)-x-1$

(a)$\times f(x)-$(b)$\times f(x)-$(c)$\times f(1)+$(d)$\times f(0)+$(e)$\times f(1)-$(f)$\times f(0)$ $\implies$ $f(x)=x(f(1)-f(0))+f(0)$

Plugging then back $f(x)=ax+b$ in original equation, we get $a=1$ and $b=0$ and so the unique solution $\boxed{f(x)=x}$
\end{solution}



\begin{solution}[by \href{https://artofproblemsolving.com/community/user/29876}{ozgurkircak}]
	thank you very much for your remark.
I now see that I made a mistake by taking granted if $a, b \in Range$ then $a+b \in Range$ which I didn't prove. And by considering the equation like a polynomial equation in $f(x)$ and $f(y).$
To complete my solution I need to prove two things:
$a,b \in Range \Longrightarrow a+b \in Range$ also there are an infinite number of elements in the Range.
\end{solution}



\begin{solution}[by \href{https://artofproblemsolving.com/community/user/152203}{borntobeweild}]
	A different approach:

Let $P(x,y)$ be the assertion.
$P(0,0) \implies f(f(0))=f(0)+f(0)^2$
$P(f(0),-f(0)) \implies f(f(0))=f(0)+f(f(0))f(-f(0))+f(0)^2$
Subtracting the first equation from the second gives:
$f(f(0))f(-f(0))=0$

Case $1$: $f(-f(0))=0$:
$P(0,-f(0)) \implies f(0)=0$

Case $2$: $f(f(0))=0$:
$P(0,f(0)) \implies f(0)=0$

Now as we have in general that $f(0)=0$, we take:
$P(x,0) \implies f(f(x))=f(x) \forall x$, which makes the original equation:
$f(x)f(y)=xy$

Let $y=x$, to get:
$f(x)^2=x^2$, so $\forall x$, either $f(x)=x$ or $f(x)=-x$.

Let $f(c)=-c$ for some $c$, and take:
$P(c,0) \implies c=-c$, so $c=0$

Therefore, $f(x)=x$
\end{solution}



\begin{solution}[by \href{https://artofproblemsolving.com/community/user/176302}{Andrax}]
	I know i'm reviving an old  thread but this is my first post , please proceed to look at this

plugging x,-x we get f(x)f(-x)=f(0)^2 -x^2
f(x)f(-x)=(f(0)-x)(f(0)+x)
so one of the solutions if f(x)=f(0)-x and f(-x)=f(0)+x , clearly f(x)=x satisfies this so it is a solution

is this considered one of the "solutions" and if this is an exercise worth 5 points how much points would i get with this trick? 
thanks in advance
\end{solution}



\begin{solution}[by \href{https://artofproblemsolving.com/community/user/199494}{IMI-Mathboy}]
	\[P(x;o) \Rightarrow f(f(x))=f(x)(c+1)  \text{here c=f(o)assume that} c\neq o then from original equation we have: f(x+y)c=f(x)f(y)-xy (1)  in(1) P(c;-c) \Rightarrow f(a)=o here a={c,-c} in (1) P(x;a) \Rightarrow f(x+a)=-frac{a}{c}x 1) case a=c \Rightarrow f(x)=-x-a here x\toa -2a=f(a)=o contradiction to our assume 2)case a=-c \Rightarrow f(x)=x-a checking it in original equation we get a=o this is also contradiction to our assume.So we have c=o from(1) we get f(x)f(y)=xy y\tob that f(b)%Error. "neqo" is a bad command.
\Rightarrow  f(x)=kx while checking we get k=1 So the solution is f(x)=x .\]
\end{solution}
*******************************************************************************
-------------------------------------------------------------------------------

\begin{problem}[Posted by \href{https://artofproblemsolving.com/community/user/3749}{pigfly}]
	Find all function $ f: \mathbb R\to \mathbb R$ satisfying the condition:
\[ f(f(x - y)) = f(x)\cdot f(y) - f(x) + f(y) - xy
\]
	\flushright \href{https://artofproblemsolving.com/community/c6h29669}{(Link to AoPS)}
\end{problem}



\begin{solution}[by \href{https://artofproblemsolving.com/community/user/581}{Anh Cuong}]
	Should it be:
  [tex] f(f(x-y))=f(x) \cdot f(y)-f(x)+f(y)-xy  [\/tex] ? :)
\end{solution}



\begin{solution}[by \href{https://artofproblemsolving.com/community/user/6984}{Kapi}]
	\begin{tcolorbox}Should it be:
  [tex] f(f(x-y))=f(x) \cdot f(y)-f(x)+f(y)-xy  [\/tex] ? :)\end{tcolorbox}
i have sloved this problem :my result: $f(x)=-x$ But my solution is long and not nice :( ; i will try to find another solution :)
\end{solution}



\begin{solution}[by \href{https://artofproblemsolving.com/community/user/6077}{nhat}]
	i have the solution for this problem
the first one:give$f(0)$=a;a in real
give $x=y=0$ then we have $f(a)=a^2$ 
give x=y then we have $f(a)=f(x)^2-x^2$ for all x in real
so we give $x=a$ then we have $2*a^2=a^4$ so $x=0$ or $x=sqrt(2)$ or $x=-sqrt(2)$
if $a=sqrt2$ then we easy to prove that $f(a^2)=a^3-a^2+a$ so give $x =a^2$ then we have the contradition 
it's similar for the case $a=-sqrt(2)$(contradition) so $a=0$ hence $f(x)=x$  or $f(x)=-x$ 
we esay to prove that two solution doesn't take place atthe same time so $f(x)=x$ fo all x in real or $f(x)=-x$ for all x in real are the solution
but $f(x)=x4 isn't the solution
then we have $f(x)=-x$ for all x in real is the function satisfy the condition  :D  :D  :D
\end{solution}



\begin{solution}[by \href{https://artofproblemsolving.com/community/user/7410}{TienVinh}]
	If a <sup>2<\/sup> =2 we choose y =0 then we can see it's wrong.
\end{solution}



\begin{solution}[by \href{https://artofproblemsolving.com/community/user/8390}{rustam}]
	that was an easy problem .
 Each guy here who is prepairing to IMO maneged it.
\end{solution}



\begin{solution}[by \href{https://artofproblemsolving.com/community/user/32886}{dgreenb801}]
	Let $ y = x$, then we find $ f(f(0)) = [f(x)]^2 - x^2$. Let $ f(f(0)) = C$, then $ f(x) = \pm \sqrt {x^2 + C}$. Of course, it cannot be both since f is a function, sometimes it will be plus and sometimes it will be minus.
Let $ f(0) = d$. Letting $ y = 0$, we have
$ f(f(x)) = df(x) - d - f(x)$
Letting $ x = 0$ and $ y = - x$, we have
$ f(f(x)) = df( - x) - d + f( - x)$
Let $ g = \frac {d - 1}{d + 1}$. Setting both expressions for $ f(f(x))$ equal, we find $ gf(x) = f( - x)$. But since $ f(x) = \pm \sqrt {x^2 + C}$, we find g must be either 1 or -1. $ g = 1$ gives no possible value for d, so $ g = - 1$ and $ d = 0$. So $ d=f(0)=0$ and $ C=f(f(0))=0$.  So f(x)= either x or -x.
But now, how do we prove $ f(x)$ can't be x on some values and -x on others? Nhat said it is easy, but never showed it.
\end{solution}



\begin{solution}[by \href{https://artofproblemsolving.com/community/user/143628}{MANMAID}]
	The full solution is too big. So I write some of these lines:
I get $f(f(0))=f(0)^2$,$f(x)+f(-x)=2f(0)$.
After that I got $f(x)^2=x^2+f(0)^2$ ,then $f(x)^2-f(-x)^2=0$, if $f(0)=0$,then we get $f(x)^2=x^2\Rightarrow{f(x)=\pm{x}}$
$f(x)=x$,this do not satisfy the eq. ,$f(x)=-x$ satisfies the eq. .
 If $f(x)=f(-x)$, & ,$f(0)\ne{0}$, then we can get $f(x)=\pm{1}$, which does not satisfy the main eq. Hence there does not exist any such $f$.
Hence  , $f(x)=-x$

:Edited:
\end{solution}



\begin{solution}[by \href{https://artofproblemsolving.com/community/user/29428}{pco}]
	\begin{tcolorbox}Find all function $ f: \mathbb R\to \mathbb R$ satisfying the condition:
\[ f(f(x - y)) = f(x)\cdot f(y) - f(x) + f(y) - xy
\]\end{tcolorbox}
I'm always very impressed by those users claiming "I have the solution but it's too long to be posted here" or "this a very easy problem that everbody managed and I'll not post the solution here 
I'm even more impressed when the claim is wrong (so that we cant point the flaw in the proof of the poster since he\/she hides it).

Here is a full solution of this rather simple equation :

Let $P(x,y)$ be the assertion $f(f(x-y))=f(x)f(y)-f(x)+f(y)-xy$

1) $f(x)^2=x^2$ $\forall x$
=============
Subtracting $P(0,0)$ from $P(x,x)$, we get $f(x)^2=x^2+f(0)^2$
An immediate consequence is that $f(-x)=\pm f(x)$ $\forall x$

Let then $x\ne 0$ : $f(f(x))=\pm f(f(-x))$ and :
If $f(f(x))=f(f(-x))$, subtracting $P(0,x)$ from $P(x,0)$ implies $f(x)=f(0)$ and so $f(x)^2=f(0)^2$ and so $x=0$, impossible.
So $f(f(x))=-f(f(-x))$ and then adding $P(0,x)$ to $P(x,0)$ implies $f(x)f(0)=0$ and so $f(0)=0$, else $f(x)=0$ $\forall x\ne 0$, which is not a solution.
Q.E.D

2) $f(x)=-x$ $\forall x>0$
================
We got from 1) above that $\forall x$, either $f(x)=x$, either $f(x)=-x$

Suppose now $\exists a,b>0$ such that $f(a)=a$ and $f(b)=-b$ :
$f(f(a-b))=\pm f(f(b-a))$ and :
If $f(f(a-b))=f(f(b-a))$, subtracting $P(b,a)$ from $P(a,b)$ implies $a=-b$ , impossible since $a,b>0$
So $f(f(a-b))=-f(f(b-a))$ and then adding $P(b,a)$ to $P(a,b)$ implies $ab=0$, impossible since $a,b>0$

So : either $f(x)=x$ $\forall x>0$, either $f(x)=-x$ $\forall x>0$

If $f(x)=x$ $\forall x>0$, $P(x,y)$ with $x>y>0$ implies contradiction.
So $f(x)=-x$ $\forall x>0$
Q.E.D.

3) $f(x)=-x$ $\forall x$
============
We got from 1) above that $\forall x$, either $f(x)=x$, either $f(x)=-x$

Suppose now $\exists a,b<0$ such that $f(a)=a$ and $f(b)=-b$ :
$f(f(a-b))=\pm f(f(b-a))$ and :
If $f(f(a-b))=f(f(b-a))$, subtracting $P(b,a)$ from $P(a,b)$ implies $a=-b$ , impossible since $a,b<0$
So $f(f(a-b))=-f(f(b-a))$ and then adding $P(b,a)$ to $P(a,b)$ implies $ab=0$, impossible since $a,b<0$

So : either $f(x)=x$ $\forall x<0$, either $f(x)=-x$ $\forall x<0$ and so :

Either $f(x)=-x$ $\forall x$, which indeed is a solution
Either $f(x)=-|x|$ $\forall x$ which is not a solution

Q.E.D.

Hence the unique solution $\boxed{f(x)=-x}$ $\forall x$
\end{solution}



\begin{solution}[by \href{https://artofproblemsolving.com/community/user/141397}{subham1729}]
	Denote $f(0) = k$. Substituting $x = y = 0$ into the given equation, we get $f(k) = k^2 (1)$
Then substituting $x = y$ and taking into account $(1)$, we have $f(k) =[f(x)]^2 - x^2$, or
$[f(x)]^2 = x^2 + k^2 (2)$ This shows that $[f(-x)]^2 = [f(x)]^2$, or
$[f(x) + f(-x)] .[f(x) - f(-x)] = 0$, . (3) Assume that there is a= 0 such that $f(a) = f(-a)$. Then substituting
$y = 0$ into the given equation one gets $f(f(x)) = kf(x) - f(x) - k$, ,and substituting $x = 0, y = -x$ one obtains $f(f(x)) = kf(-x) + f(-x) -k,$ which together yields $k[f(-x) -f(x)] + f(-x) + f(x) = 2k, (4)$ from which, by substituting $x = a$, we have $f(a) = k. (5)$ On the other hand, from (2) it follows that if $f(x) = f(y)$ then $x^2 = y^2$.Then by (5), the equality $f(a) = k = f(0)$ shows that $a = 0$, which contradicts to  $a= 0$. Thus $f(-x) = f(x)$ for all $x = 0$. Then (3) gives $k[f(x)-1] = 0$, which implies that $k = 0$ (otherwise $f(x) = 1$, contradicts to $f(-1) = f(1))$.So we arrive to $[f(x)]^2 = x^2.$ Now assume that there is $a = 0$ such that $f(a) = a$. Then $a =f(a) = -f(f(a)) = -f(a) =-a$, which implies that $a = 0$. This is impossible, as 
$a=0$. Hence, $f(x) = x$, Then from $[f(x)]^2 = x^2$ it follows that $f(x) = -x,$ Conversely, by direct verification we see that this solution satisfies the requirement of the problem. Thus the answer is $f(x) = -x.$
\end{solution}



\begin{solution}[by \href{https://artofproblemsolving.com/community/user/143628}{MANMAID}]
	\begin{tcolorbox}I'm always very impressed by those users claiming "I have the solution but it's too long to be posted here" or "this a very easy problem that everbody managed and I'll not post the solution here 
I'm even more impressed when the claim is wrong (so that we cant point the flaw in the proof of the poster since he\/she hides it).\end{tcolorbox}
  You are right , I overlooked it and I edited it.
Though you are right that I did not post the full solution, these lines that I have wrote is enough, and also my solution is smaller than yours.
\end{solution}



\begin{solution}[by \href{https://artofproblemsolving.com/community/user/29428}{pco}]
	\begin{tcolorbox}  You are right , I overlooked it and I edited it.
Though you are right that I did not post the full solution, these lines that I have wrote is enough, and also my solution is smaller than yours.\end{tcolorbox}
Congrats ! You are very smart
\end{solution}



\begin{solution}[by \href{https://artofproblemsolving.com/community/user/13967}{Karth}]
	Another solution:

Setting $x=y$, we get $f(f(0)) = f(x)^2 - x^2 \Rightarrow f(x) = \sqrt{x^2 + C_1}$ or $f(x) = -sqrt{x^2 + C_1}$, where $C_1 = f(f(0))$. Note that $C_1 = 0.$ Indeed, let $C_1 > 0.$ Note that $x = 0 \Rightarrow f(f(x)) = C_2 f(x) - f(x) + C_2$, where $C_2 = f(0)$. Further, $y = 0 \Rightarrow f(f(-y)) = C_2 f(y) + f(y) - C_2$. Since $C_1 > 0$, $f$ is an even function, so we have that $f(x) = f(-x) \Rightarrow f(f(x)) = f(f(-x)) \Rightarrow C_2 f(x) - f(x) + C_2 = C_2 f(x) + f(x) - C_2 \Rightarrow f(x) = f(0) = 0$, by simple substitution. This is a contradiction, since $|C_1| > 0 \Rightarrow f(x) \neq 0$. Substituting $C_1 = 0$, we get that $f(x) = x$, $f(x) = -x$, $f(x) = |x|$, or $f(x) = -|x|$. We can easily check that of these, only $f(x) = -x$ works. $\box$
\end{solution}
*******************************************************************************
-------------------------------------------------------------------------------

\begin{problem}[Posted by \href{https://artofproblemsolving.com/community/user/1991}{orl}]
	Let $ \mathbb{N}_0$ denote the set of nonnegative integers. Find all  functions $ f$ from $ \mathbb{N}_0$ to itself such that
\[ f(m + f(n)) = f(f(m)) + f(n)\qquad \text{for all} \; m, n \in \mathbb{N}_0.
\]
	\flushright \href{https://artofproblemsolving.com/community/c6h60429}{(Link to AoPS)}
\end{problem}



\begin{solution}[by \href{https://artofproblemsolving.com/community/user/3687}{aodeath}]
	Its not hard to guess that 

$f(x)=x$

works well. Actually, I just don't know if it's the only possible solution for that. It's quite reasonable to observe that an aplication of f sends 0 into itself. And by taking m=0, we opberve that

$f(f(n))=f(n), \forall n$

So i can garantee that there are more fixed points...don't know if that's a nice observation, because I can't conclude much about...but i wanted to discuss about this problem
\end{solution}



\begin{solution}[by \href{https://artofproblemsolving.com/community/user/26}{grobber}]
	There are definitely more solutions. I don't have the time to write a detailed solution, and it's not that hard anyway, but I believe the solutions can be described like this: 

Take some $r\in\mathbb N_0$, let $f$ map $1,2,\ldots,r-1$ onto some multiples of $r$ arbitrarily, put $f(kr)=kr,\ \forall k\in\mathbb N_0$, and extend the function to the other naturals by setting $f(u+r)=f(u)+r,\ \forall u\in\mathbb N_0$.
\end{solution}



\begin{solution}[by \href{https://artofproblemsolving.com/community/user/13603}{e.lopes}]
	Kalva Solution:

Setting $m = n = 0$, the given relation becomes: $f(f(0)) = f(f(0)) + f(0)$. Hence f(0) = 0. Hence also $f(f(0)) = 0$. Setting $m = 0$, now gives $f(f(n)) = f(n)$, so we may write the original relation as $f(m + f(n)) = f(m) + f(n)$. 

So $f(n)$ is a fixed point. Let $k$ be the smallest non-zero fixed point. If $k$ does not exist, then $f(n)$ is zero for all $n$, which is a possible solution. If $k$ does exist, then an easy induction shows that $f(qk) = qk$ for all non-negative integers $q$. Now if $n$ is another fixed point, write $n = kq + r$, with $0$ ≤ $r$ < $k$. 
Then $f(n) = f(r + f(kq)) = f(r) + f(kq) = kq + f(r)$. Hence $f(r) = r$, so $r$ must be zero. Hence the fixed points are precisely the multiples of $k$. 

But $f(n)$ is a fixed point for any $n$, so $f(n)$ is a multiple of $k$ for any $n$. 
Let us take $n_1, n_2, ... , n_{k-1}$ to be arbitrary non-negative integers and set $n_0 = 0$. 
Then the most general function satisfying the conditions we have established so far is: 

      $f(qk + r) = qk + n_rk$ for $0$ ≤$r$ < $k$. 

We can check that this satisfies the functional equation. 
Let $m = ak + r$, $n = bk + s$, with $0$ ≤ $r, s$ < $k$. 
Then $f(f(m)) = f(m) = ak + n_rk$, and $f(n) = bk + n_sk$, 
so $f(m + f(n)) = ak + bk + n_rk + n_sk$, and $f(f(m)) + f(n) = ak + bk + n_rk + n_sk$. 
So this is a solution and hence the most general solution.
\end{solution}



\begin{solution}[by \href{https://artofproblemsolving.com/community/user/10277}{Philip_Leszczynski}]
	I'm not sure about this... but it seems right.

[hide]

Put $m,n=0: f(f(0)) = f(f(0)) + f(0)$, so $f(0) = 0$.
Put $m=0: f(f(n)) = f(f(0)) + f(n)$, so $f(f(n)) = f(n)$.
So the original becomes $f(m+f(n)) = f(m) + f(n)$.
If we plug in $f(m)$ for $m$, we have $f(f(m)+f(n)) = f(f(m)) + f(n) = f(m) + f(n)$.
So if we have $m=n$, then $f(2f(m)) = 2f(m)$.
Since $2f(m)$ must be in the range of $f$ for all $m$, we can use a simple induction on $k$ to show that for all $k,m \in \mathbb{N}_0$, $f(kf(m)) = kf(m)$.
Let $f(1)=c$. Then $f(kf(1)) = f(kc) = kf(1) = kc$.
So $f(x)=cx$, where $c=f(1)$.
But then $f(f(x)) = cf(x) = c^2x > cx$. Contradiction if $c \ge 2$ (since $f(f(x)) = f(x)$)
So $c=0,1$.
But if $c=1$, then $f(x)=x$, and we obviously have a contradiction in the original. So $c=0$, and $f(x)=0$ for all x.

[\/hide]
\end{solution}



\begin{solution}[by \href{https://artofproblemsolving.com/community/user/13603}{e.lopes}]
	\begin{tcolorbox}I'm not sure about this... but it seems right.

[hide]

Put $m,n=0: f(f(0)) = f(f(0)) + f(0)$, so $f(0) = 0$.
Put $m=0: f(f(n)) = f(f(0)) + f(n)$, so $f(f(n)) = f(n)$.
So the original becomes $f(m+f(n)) = f(m) + f(n)$.
If we plug in $f(m)$ for $m$, we have $f(f(m)+f(n)) = f(f(m)) + f(n) = f(m) + f(n)$.
So if we have $m=n$, then $f(2f(m)) = 2f(m)$.
Since $2f(m)$ must be in the range of $f$ for all $m$, we can use a simple induction on $k$ to show that for all $k,m \in \mathbb{N}_0$, $f(kf(m)) = kf(m)$.
Let $f(1)=c$. Then $f(kf(1)) = f(kc) = kf(1) = kc$.
So $f(x)=cx$, where $c=f(1)$.
But then $f(f(x)) = cf(x) = c^2x > cx$. Contradiction if $c \ge 2$ (since $f(f(x)) = f(x)$)
So $c=0,1$.
But if $c=1$, then $f(x)=x$, and we obviously have a contradiction in the original. So $c=0$, and $f(x)=0$ for all x.

[\/hide]\end{tcolorbox}

wrong solution.  ;)
\end{solution}



\begin{solution}[by \href{https://artofproblemsolving.com/community/user/3687}{aodeath}]
	\begin{tcolorbox}
But then $f(f(x)) = cf(x) = c^2x > cx$. Contradiction if $c \ge 2$ (since $f(f(x)) = f(x)$)
So $c=0,1$.
But if $c=1$, then $f(x)=x$, and we obviously have a contradiction in the original. So $c=0$, and $f(x)=0$ for all x.
\end{tcolorbox}

This part here seems very strange to me... :huh: 
First because you weren't so clear about your contradiction [even though you didn't assume anything to contradict!]...second because $f(x)=x, \forall x\in \mathbb{N}_0$ works very well =]
\end{solution}



\begin{solution}[by \href{https://artofproblemsolving.com/community/user/72819}{Dijkschneier}]
	\begin{tcolorbox}
We can check that this satisfies the functional equation. 
Let $m = ak + r$, $n = bk + s$, with $0$ ≤ $r, s$ < $k$. 
Then $f(f(m)) = f(m) = ak + n_rk$, and $f(n) = bk + n_sk$, 
so $f(m + f(n)) = ak + bk + n_rk + n_sk$, and $f(f(m)) + f(n) = ak + bk + n_rk + n_sk$. 
So this is a solution and hence the most general solution.\end{tcolorbox}
When you are verifying, I think you should not make the assumption $f(f(m))=f(m)$.
If $f(qk+r)=qk + n_r k$, how do you prove that $f(f(m))=f(m)$ ?
Suppose otherwise that $f(f(m))=f(m)$. Then $qk+n_r k = f(qk+r) = f(f(qk+r))=f(qk+n_r k)=qk + n_{n_r k}k$, so  $n_r = n_{n_r k}$ : but why should they be equal ?
\end{solution}



\begin{solution}[by \href{https://artofproblemsolving.com/community/user/97340}{quantumbyte}]
	Can't we conclude from f(f(n))=f(n), that f(n)=n.
\end{solution}



\begin{solution}[by \href{https://artofproblemsolving.com/community/user/86504}{avatarofakato}]
	\begin{tcolorbox}Can't we conclude from f(f(n))=f(n), that f(n)=n.\end{tcolorbox}
Actually we can, but first we must show that $f$ is injective.
\end{solution}



\begin{solution}[by \href{https://artofproblemsolving.com/community/user/97340}{quantumbyte}]
	Can you explain to me how we could do that?
\end{solution}



\begin{solution}[by \href{https://artofproblemsolving.com/community/user/86504}{avatarofakato}]
	I don't think so. I have written this statement for the general case.
\end{solution}



\begin{solution}[by \href{https://artofproblemsolving.com/community/user/57654}{anchenyao}]
	I hope I am not reviving anything.

[hide="My Solution"]
First, set $m=n=0$. Thus, we have

$f(f(0))=f(f(0))+f(0)$. 

Thus $f(0)=0$

Then, set $m=0$. 

$f(f(n))=f(n)$

Next, let $m=n=f(b)$

$f(2f(b))=f(f(b))+f(b)$

$f(2f(b))=2f(b)$

Substituting $n=2f(b)$ and $m=f(b)$, we have

$f(3f(b))=f(2f(b))+f(f(b))=2f(b)+f(b)=3f(b)$.

By induction, we realize that 

$f(cf(b))=cf(b)=f(cn)=cf(n)$

Thus, if $n$ is in the range of $f(x)$, all integers of the form $cn$ for a positive integer $c$ is also in the range of $f(x)$. Thus, the range of $f(x)$ encompasses all integers.

From this, we get 

$f(f(n))=f(n)$

Letting $f(n)=k$,

$\fbox{f(k)=k}$.

This works when $f(b)\neq 0$, so the other solution is $\fbox{f(b)=0}$.[\/hide]
\end{solution}



\begin{solution}[by \href{https://artofproblemsolving.com/community/user/96840}{ACCCGS8}]
	This solution does not work.
\end{solution}



\begin{solution}[by \href{https://artofproblemsolving.com/community/user/140813}{MBGO}]
	Here's a new solution(i guess) by me :

suppose function is not constant(or else f=0) :

it is not hard to get f(0)=0, f(kf(n))=kf(n) (*),f(f(n))=f(n) (**), so if there exists some natural (or zero) i so that f(i)=1,then we will get
f(k)=k by (*), suppose the least \begin{bolded}natural \end{bolded}number which the function could produce is p, then by (*) we have f(kp)=kp, where k is a natural number, it's not hard to find that f(m+f(n))=f(f(m)) + f(n) is equal to f(m+f(n))=f(m)+f(n) by (**),plugging m=i<p,n=kp , we will get f(kp+i)=kp +f(i) ,so :

set n=kp+i,i<p,
f(f(n)=f(n) so f(kp+f(i))=kp+f(i),by the assumption of the minimality of function, we get f(i)=sp+j ,0<j<p
so f(kp+sp+j)=kp+sp+f(j)=kp+sp+j
so f(j)=j, contradiction with minimality,so in this case we get the least minimality is equal to 1 and so f(k)=k for all naturals,and it's Done.
.....
if j=0,then f(i)=sp for some arbitrary natural(or zero) s....the rest is Done.

\begin{italicized}typos fixed.\end{italicized}
\end{solution}



\begin{solution}[by \href{https://artofproblemsolving.com/community/user/105268}{McItran}]
	Is it true?

Let $m=n=0$. Then $f(0+f(0))=f(f(0))=f(f(0))+f(0)$. Then $f(0)=0$. Then put in $m=0$. We get $f(f(n))=f(f(0))+f(n)=f(n). Then $f(m+f(n))=f(m)+f(n). Assume that $a, b, c$ are arbitrary integer nonnegative numbers. Then consider $f(a+b+f(c))$. Using condition we get:
1) $f(a+b+f(c))=f(a)+f(b+f(c))=f(a)+f(b)+f(c)$
2) $f(a+b+f(c))=f(a+b)+f(c)$.
So $f(a+b)=f(a)+f(b)$. It's well-known Cauchy's functional equation. As we have integer numbers, $f(x)=kx$ (k is an arbitrary nonngative number). 
As $f(f(n))=f(n)$, $k^2n=kn$, so $k=0$ or $k=1$. So the only solvings are $f(x)=x$ and $f(x)=0$
\end{solution}



\begin{solution}[by \href{https://artofproblemsolving.com/community/user/140813}{MBGO}]
	you can only apply Cauchy's function if and only if f is continouse.
so you're not done.
\end{solution}



\begin{solution}[by \href{https://artofproblemsolving.com/community/user/31919}{tenniskidperson3}]
	\begin{tcolorbox}you can only apply Cauchy's function if and only if f is continouse.
so you're not done.\end{tcolorbox}

Actually, here it's ok because it's on the integers.  But you're right, McItran is not done.  His issue is here:

\begin{tcolorbox}1) $f(a+b+f(c))=f(a)+f(b+f(c))$\end{tcolorbox}

You cannot just take $a$ outside into its own function.  You need justification for that, and since you have none, you are wrong.  You can also check that you are wrong because $f(x)=x$ if $x$ is even and $f(x)=x-1$ if $x$ is odd also works.
\end{solution}



\begin{solution}[by \href{https://artofproblemsolving.com/community/user/140813}{MBGO}]
	@tenniskidperson3 : would you please check the end part of my solution? where i get in the last line : "f(i)=sp"?
\end{solution}



\begin{solution}[by \href{https://artofproblemsolving.com/community/user/173255}{soundouss}]
	i think i have a pretty idea why don't we generalize the exercise to be more free with it thing like working in R 
in fact it will be more easy to deal with the equation
if we prove that f(1)=1 we re done
\end{solution}



\begin{solution}[by \href{https://artofproblemsolving.com/community/user/31919}{tenniskidperson3}]
	\begin{tcolorbox}we will get f(kp+i)=kp +i\end{tcolorbox}

False, we will get $f(kp+i)=kp+f(i)$.

\begin{tcolorbox}we get f(i)=sp+j ,0<j<p\end{tcolorbox}

False again, we can have $j=0$.

\begin{tcolorbox}so f(j)=j, contradiction with minimality\end{tcolorbox}

... unless $j=0$ for all integers.
\end{solution}



\begin{solution}[by \href{https://artofproblemsolving.com/community/user/140813}{MBGO}]
	first one was just a typo,EDITED, second one i just devide the solution in two cases,the case where 0<j and j=0...i the only thing i doubt on it is when f(i)=sp,where i<p
\end{solution}



\begin{solution}[by \href{https://artofproblemsolving.com/community/user/31919}{tenniskidperson3}]
	OK, I didn't understand what you were writing.  It looks OK to me now.
\end{solution}



\begin{solution}[by \href{https://artofproblemsolving.com/community/user/105268}{McItran}]
	\begin{tcolorbox}[quote="MBGO"]you can only apply Cauchy's function if and only if f is continouse.
so you're not done.\end{tcolorbox}

Actually, here it's ok because it's on the integers.  But you're right, McItran is not done.  His issue is here:

\begin{tcolorbox}1) $f(a+b+f(c))=f(a)+f(b+f(c))$\end{tcolorbox}

You cannot just take $a$ outside into its own function.  You need justification for that, and since you have none, you are wrong.  You can also check that you are wrong because $f(x)=x$ if $x$ is even and $f(x)=x-1$ if $x$ is odd also works.\end{tcolorbox}

Oh, I'm sorry  :blush:
\end{solution}



\begin{solution}[by \href{https://artofproblemsolving.com/community/user/148207}{Particle}]
	[hide="Solution"]Suppose $P(m,n)\implies f(m+f(n))=f(f(m))+f(n)$
$P(0,0)\implies f(0)=0$ and $P(m,0)\implies f(m)=f(f(m))$. So the original equation turns into $f(m+f(n))=f(m)+f(n)$. From now we define this new equation to be $P(x,y)$. 

$P(f(m),m)\implies f(2f(m))=2f(m)$ and by induction $f(nf(m))=nf(m)\forall n\in \mathbb N_0\quad (1)$. Suppose $a$ is the gcd of the all elements in the image of $f$. Now pick $c,d$ such that $\gcd (f(c),f(d))=a$. Now the diophantine equation $f(c)x-f(d)y=a$ has solution in positive integers. So using (1) we get, $P(xf(c)-yf(d),yf(d))\implies f(xf(c))=f(a)+f(yf(d))\implies f(a)=a$. 

Now we can write the general solution: since $a|f(i)$, so define $f(i)=ac_i$ for $1\leq i<a$, $c_i\in \mathbb N_0$, $f(a)=a$ and for $i=ak+b>a$, define $f(i)=ak+f(b)$ with $b<a$. Very easy to prove this function works. (Can be done as other people did in the previous posts.) [\/hide]
[hide="Comment:"]I too fell for $f(n)=n$. After proving $f(a)=a$ I tried too desperately to prove $f(1)=1$, but was in vain. Finally I understood even if $f(1)=0$, we still get functions satisfying the equation. After that, boom! I killed it :D[\/hide]
\end{solution}



\begin{solution}[by \href{https://artofproblemsolving.com/community/user/183149}{JuanOrtiz}]
	Let $\Omega = \text{Im} (f) $. Throughout my solution, $a,b$ will denote elements of this set.

Notice that $f(m+a)=f(f(m))+a$, this is the statement. Then $f(a+b)=f(a)+b=f(b)+a$ and therefore $f(a)-a$ is constant $\forall a \in \Omega$. Also $a+c=f(a)=f(0+a)=f(f(0))+a=f(0)+a+c$ and so $f(0)=0$. Then if $a=0, c=0$ and so $f(a)=a \forall a \in \Omega$. Notice also $f(m+a)=f(m)+a$.

Notice that if $a,b \in \Omega$ then $f(a+b)=a+b$ so $\Omega$ is additive and if $b > a$ then $b=f(b)=f((b-a)+a)=f(b-a)+a$ then $\Omega$ is "subtractive". We obtain that $\Omega=\{0\}$ or the set of multiples of a positive integer $d$. In the first case, $f=0$ doesn't work. In the other case, we obtain $f(m+kd)=f(m)+kd$ and so the solution is:

Let $d$ be any positive integer and choose any $a_1,...,a_{d-1} \in \mathbb{N}_0$. Then $f(kd+i) = kd+a_i$ for $k \in \mathbb{N}_0$ and $1 \le i \le d-1$ and $f(kd)=kd$.
\end{solution}



\begin{solution}[by \href{https://artofproblemsolving.com/community/user/80125}{pi37}]
	Let $g$ be an arbitrary positive integer and let $a_1,a_2,\cdots a_{g-1}$ be an arbitrary sequence of nonnegative integers, with $a_0=0$. Any solution in the form
\[
f(pg+r)=pg+a_r
\]
for $0\le r<g$ can easily be shown to be valid. Now we show that any possible function is in this form. Indeed, $P(0,0)$ implies $f(0)=0$, and $P(0,n)$ implies $f(f(n))=f(n)$. Then
\[
f(m+f(n))=f(m)+f(n)
\]
so $f$ is periodic modulo any integer in its range. Let $g>0$ be the gcd of the range of $f$ (noting that the solution $f(x)=0$ identically has been accounted for in the solution set, and so we may assume otherwise). Then some linear combination of elements of the range of $f$ form any multiple of $g$, so in particular
\[
f(m+kg)=f(m)+kg
\]
for any $m,k$. This implies that $f$ is in the above form.
\end{solution}
*******************************************************************************
-------------------------------------------------------------------------------

\begin{problem}[Posted by \href{https://artofproblemsolving.com/community/user/6099}{keira_khtn}]
	A function f is called \begin{bolded}Darboux\end{bolded} if X is an interval implying f(X) is also an interval.
Find all Darboux funtion f s.t.:
$f(x+y)=f(x+f(y))$
	\flushright \href{https://artofproblemsolving.com/community/c6h61817}{(Link to AoPS)}
\end{problem}



\begin{solution}[by \href{https://artofproblemsolving.com/community/user/29428}{pco}]
	\begin{tcolorbox}A function f is called \begin{bolded}Darboux\end{bolded} if X is an interval implying f(X) is also an interval.
Find all Darboux funtion f s.t.:
$f(x+y)=f(x+f(y))$\end{tcolorbox}
Let $P(x,y)$ be the assertion $f(x+y)=f(x+f(y))$

Let $A=\{f(x)-x$ $\forall x\in\mathbb R\}$
Notation : in the following, the notation $|u,v|$ indicates an interval $(u,v)$ or $[u,v)$ or $(u,v]$ or $[u,v]$ (where $u,v$ may be numbers or $\pm \infty$)

1) $A$ is an additive subgroup of $\mathbb R$
=======================
$A$ is not empty.
$P(x-y,y)$ $\implies$ $f(x)=f(x-y+f(y))$ $\implies$ $(f(x)-x)-(fy)-y)$ $=f(x-y+f(y))-(x-y+f(y))$
So $a,b\in A$ $\implies$ $a-b\in A$
So $a-a=0\in A$
So $0-b=-b\in A$
So $a-(-b)=a+b\in A$
Q.E.D.

2) If $A=\{0\}$, then $f(x)=x$ $\forall x$
========================
This is just the definition of $A$ and $f(x)=x$ $\forall x$ is indeed a solution.
Q.E.D.

3) If $A\ne \{0\}$ is dense in $\mathbb R$, then $f(x)=c$ constant
======================================
$f(\mathbb R)$ is an interval since $f(x)$ is Darboux.
If this interval contains two distinct numbers $a<b$, then it contains $[a,b]$
$P(0,x)$ $\implies$ $f(f(x))=f(x)$ and so $f(x)=x$ $\forall x\in[a,b]$

Since $A$ is dense in $\mathbb R$, let then $u$ such that $0<f(u)-u<b-a$
$a\in[a,b]$ and so $f(a)=a$
$a+f(u)-u\in[a,b]$ and so $f(a+f(u)-u)=a+f(u)-u\ne f(a)$
But $P(a-u,u)$ $\implies$ $f(a)=f(a+f(u)-u)$ and so contradiction.

So no such distinct $a,b$ exist and $f(\mathbb R)=\{c\}$ which indeed is a solution.
Q.E.D.

4) If $A\ne\{0\}$ is not dense in $\mathbb R$, then no solution
===================================
If $A\ne \{0\}$ and not dense in $\mathbb R$, then $\exists \Delta>0$ such that $A=\Delta\mathbb Z$ (remember that $A$ is an additive subgroup of $\mathbb R$)

$\Delta\in A$ $\implies$ $\exists t$ such that $f(t)-t=\Delta$ and then $P(x-t,t)$ $\implies$ $f(x)=f(x+\Delta)$ $\forall x$

Let $f(\mathbb R)=|u,v|$
$P(0,x)$ $\implies$ $f(f(x))=f(x)$ and so $f(x)=x$ $\forall x\in|u,v|$

If $v-u>\Delta$, then :
$\exists a\in|u,v|$ such that$ a+\Delta\in|u,v|$ too
$a\in|u,v|$ $\implies$ $f(a)=a$
$a+\Delta\in|u,v|$ $\implies$ $f(a+\Delta)=a+\Delta$
But $f(a)=f(a+\Delta)$ 
and so contradiction.

So $v-u \le \Delta$

If $v-u<\Delta$, let $x\in(v,u+\Delta)$
$f(x)\in|u,v|$ and so $f(x)-x\in (0,-\Delta)$ which is impossible since $f(x)-x\in A=\Delta\mathbb Z$

So $v-u=\Delta$

But then $f\left(\left[u+\frac{2\Delta}3,u+\frac{4\Delta}3\right]\right)$ $=\left[u+\frac{2\Delta}3,u+\Delta\right)$ $\cup\{f(u+\Delta)\}$ $\cup\left(u,u+\frac{\Delta}3\right)$ is not an interval
And so no solution.
Q.E.D

5) Synthesis of solutions
=================
The only solutions are :
$f(x)=x$ $\forall x$
$f(x)=c$ $\forall x$ and for any $c\in\mathbb R$
\end{solution}
*******************************************************************************
-------------------------------------------------------------------------------

\begin{problem}[Posted by \href{https://artofproblemsolving.com/community/user/6912}{chess64}]
	Let $f$ be a function with the following properties:

1) $f(n)$ is defined for every positive integer $n$;
2) $f(n)$ is an integer;
3) $f(2)=2$;
4) $f(mn)=f(m)f(n)$ for all $m$ and $n$;
5) $f(m)>f(n)$ whenever $m>n$.

Prove that $f(n)=n$.
	\flushright \href{https://artofproblemsolving.com/community/c6h87616}{(Link to AoPS)}
\end{problem}



\begin{solution}[by \href{https://artofproblemsolving.com/community/user/5787}{ZetaX}]
	http://www.mathlinks.ro/Forum/viewtopic.php?t=55870 pwns it ;)
\end{solution}



\begin{solution}[by \href{https://artofproblemsolving.com/community/user/6912}{chess64}]
	Is it just me or does Kalva's solution (http://www.kalva.demon.co.uk\/canada\/casoln\/csol698.html) not make sense...
\end{solution}



\begin{solution}[by \href{https://artofproblemsolving.com/community/user/16858}{rjt}]
	\begin{tcolorbox}Is it just me or does Kalva's solution (http://www.kalva.demon.co.uk\/canada\/casoln\/csol698.html) not make sense...\end{tcolorbox}

Seems to make sense to me (maybe I'm missing something though).

My solution:
[hide]
It's easily shown that $f(1)=1$ and $f(4)=4$. 
$f(2)<f(3)<f(4)$
$2 < f(3) < 4$. Implies $f(3) = 3$

Now, let us assume that it is true for all $f(k)$ where $k\leq 2n$  
$f(2n+2)=f(2(n+1))=f(2)f(n+1)=2n+2$
$f(2n)<f(2n+1)<f(2n+2)$
$2n<f(2n+1)<2n+2$
Hence, $f(2n+1)=2n+1$, and by induction $f(n) = n$


[\/hide]
\end{solution}



\begin{solution}[by \href{https://artofproblemsolving.com/community/user/6912}{chess64}]
	Nice solution.
\end{solution}



\begin{solution}[by \href{https://artofproblemsolving.com/community/user/38022}{NapoleonXIV}]
	Clearly function is multiplicative.
Function is specifically defined  if we know values for $ p^k$ where $ p$ is a prime, and
 $ k\in N$

We know that for every $ n$ $ f(2^n)=f(2)*f(2{}^n{}^-{}^1)=f(2)*f(2)*f(2{}^n{}^-{}^2)=...=f^n(2)=2^n$

Does that mean that for every $ n$ $ f(n)=n$?
\end{solution}



\begin{solution}[by \href{https://artofproblemsolving.com/community/user/14052}{t0rajir0u}]
	\begin{tcolorbox}Function is specifically defined  if we know values for $ p^k$ where $ p$ is a prime\end{tcolorbox}

We can do better; a \begin{italicized}completely\end{italicized} multiplicative function (such as the one given) from the positive integers is defined by its values at the primes.  But note that for an odd prime $ p$ we have

$ f(p - 1) < f(p) < f(p + 1)$

and if we have shown (inductively) that $ f(q) = q$ for $ q$ a prime less than $ p$ then the fact that $ p - 1, p + 1$ are both composite allows us to conclude.  This is essentially rjt's solution.

Edit:  \begin{bolded}Follow-up:\end{bolded}  Is the condition that $ f(2) = 2$ necessary?  What if we only have $ f(2) \neq 0$?
\end{solution}



\begin{solution}[by \href{https://artofproblemsolving.com/community/user/126756}{panamath}]
	For any $n$, $f(2^{n}) = 2^{n}$... we have that between two powers of two the numbers are in order so are the $f's$ because $ f(m)>f(n) $ if $ m>n $ a simple strong induction will do it. We can also use that the groups {${2^{n-1} +1, ... , 2^{n} - 1}$} and {$f({2^{n-1} +1), ... , f(2^{n} - 1})$} are between the same numbers and in the same order, this fact plus the fact that they are all integers may be a complete solution, really don't know.
\end{solution}



\begin{solution}[by \href{https://artofproblemsolving.com/community/user/109704}{dien9c}]
	We have the stronger problem 
Find all $f: \mathbb{N} \to \mathbb{N}$ such that
1. $f(2)=2$
2. $f(mn)=f(m)f(n)$ if $\gcd(m,n)=1$
3. $f$ is increasing
\end{solution}



\begin{solution}[by \href{https://artofproblemsolving.com/community/user/140796}{mathbuzz}]
	[color=#FF0000]solution to the original problem, NOT THE STRONGER ONE[\/color]------f(2.1)=f(2).f(1) , so , 2=2f(1) , so f(1)=1 
now we use strong form induction
let P(n): f(n)=n.
let P(1) ,....,P(k) be true { base cases have already been checked}
f(k.k)=[f(k)]^2 , i.e. , f(k^2)=k^2 also f(k)=k
as per the problem condition , f(k)<f(k+1)<.....f(k^2-1)<f(k^2)
so , clearly , f(k+1)=k+1 , ......., f(k^2-1)=k^2-1 must hold
now our induction is complete. :D
\end{solution}



\begin{solution}[by \href{https://artofproblemsolving.com/community/user/29428}{pco}]
	\begin{tcolorbox}... f(k.k)=[f(k)]^2 ...\end{tcolorbox}
Unfortunately, property 2. is only valid when $\gcd(m,n)=1$ and so you cant use it for $m=n=k>1$
\end{solution}



\begin{solution}[by \href{https://artofproblemsolving.com/community/user/140796}{mathbuzz}]
	\begin{tcolorbox}[quote="mathbuzz"]... f(k.k)=[f(k)]^2 ...\end{tcolorbox}
Unfortunately, property 2. is only valid when $\gcd(m,n)=1$ and so you cant use it for $m=n=k>1$\end{tcolorbox}

why ?? the problem says that f(mn)=f(m)f(n) for all m ,n. s o ,please elaborate. i gave the solution to the original problem , not the stronger one.:oops:
\end{solution}



\begin{solution}[by \href{https://artofproblemsolving.com/community/user/29428}{pco}]
	Sorry, I did not see your sentence "solution to the original problem, NOT THE STRONGER ONE", although it was in red characters and great letters. :blush:
\end{solution}
*******************************************************************************
-------------------------------------------------------------------------------

\begin{problem}[Posted by \href{https://artofproblemsolving.com/community/user/9325}{mathematica}]
	Find all $f: \mathbb{N}\rightarrow \mathbb{N}$ such that for every $n\in \mathbb{N}$,

$f^{f(n)}(n)=n+1$, where $f^{f(n)}$ is the $f(n)$th iterate of $f$.
	\flushright \href{https://artofproblemsolving.com/community/c6h144434}{(Link to AoPS)}
\end{problem}



\begin{solution}[by \href{https://artofproblemsolving.com/community/user/13697}{mao_zai}]
	Hmmm. If I'm not wrong, there does not exist such such function $f$, assuming that $f^{(0)}$ is the identity function.

Claim 1: $f(n)\ne 0$ for all $n$. Else $n=f^{(0)}(n)=f^{f(n)}(n)=n+1$.

Claim 2: If $f(n)=1$, then $n=0$. Because $1=f(n)=f^{f(n)}(n)=n+1$.^

Hence, $f(n)\ge 2$ when $n\ge 1$.

Claim 3: $f$ is injective. Proof: WLOG $x$, $y\ge 1$. Hence, $f(x)=f(y)\ge 2$.
So, $f^{f(x)-1}$ and $f^{f(y)-1}$ are well-defined. Then $x+1=f^{f(x)-1}(f(x))=f^{f(y)-1}(f(y))=y+1$. QED.

Now, to determine $f(1)$. Let $f(1)=k+1$ , where $k\ge 1$.

Now, $f^{f(k)}(k)=f(f^{f(k)-1}(k))=k+1=f(1)$. By injectivity, $f^{f(k)-1}(k)=1$. ie.  $f(f^{f(k)-2}(k))=f(0)$ and again, $f^{f(k)-2}(k)=0$.

If $f(k)=2$, then $k=0$. Contradicting our assumption for $k$.
If $f(k)>2$, then $f(f^{f(k)-3}(k))=0$ contradicts Claim 1.

 :huh:
\end{solution}



\begin{solution}[by \href{https://artofproblemsolving.com/community/user/104682}{momo1729}]
	I believe the confusion comes from notation. Probably mathematica meant that $0\notin \mathbb{N}$.
\end{solution}



\begin{solution}[by \href{https://artofproblemsolving.com/community/user/29428}{pco}]
	\begin{tcolorbox}Find all $f: \mathbb{N}\rightarrow \mathbb{N}$ such that for every $n\in \mathbb{N}$,

$f^{f(n)}(n)=n+1$, where $f^{f(n)}$ is the $f(n)$th iterate of $f$.\end{tcolorbox}
Even with $\mathbb N$ being the set of positive integers, there are no solutions :

Obviously $f(n)$ is injective

If $f(a)=1$ for some $a>0$, then equation becomes for $n=a$ : $1=a+1$, impossible. So $1\notin f(\mathbb N)$

So $f(1)>1$ and seting $n=f(1)-1$ in equation gives $f^{f(f(1)-1)}(f(1)-1)=f(1)$ and so, since injective, $f^{f(f(1)-1)-1}(f(1)-1)=1$, impossible
\end{solution}
*******************************************************************************
-------------------------------------------------------------------------------

\begin{problem}[Posted by \href{https://artofproblemsolving.com/community/user/68025}{Pirkuliyev Rovsen}]
	Find all function $ f: \mathbb{R}\to\mathbb{R}$ such that there is $ f^{`}(0)$ following   $ xf(y)+yf(x)=a\underbrace {f( ...}_{n}(xy)...)$  Here  $ a>0$   



 :ninja:  :ninja:
	\flushright \href{https://artofproblemsolving.com/community/c6h322678}{(Link to AoPS)}
\end{problem}



\begin{solution}[by \href{https://artofproblemsolving.com/community/user/68025}{Pirkuliyev Rovsen}]
	Patrick, when I then could not solve this problem.Help me if you can
\end{solution}



\begin{solution}[by \href{https://artofproblemsolving.com/community/user/29428}{pco}]
	\begin{tcolorbox}Find all function $ f: \mathbb{R}\to\mathbb{R}$ such that there is $ f^{`}(0)$ following   $ xf(y)+yf(x)=a\underbrace {f( ...}_{n}(xy)...)$  Here  $ a>0$  \end{tcolorbox}

What does this mean :"there is $ f^{`}(0)$ following   $ xf(y)+yf(x)=a\underbrace {f( ...}_{n}(xy)...)$" :?:
\end{solution}



\begin{solution}[by \href{https://artofproblemsolving.com/community/user/68025}{Pirkuliyev Rovsen}]
	Hello Patrick.There exists a derivative$f^{'}(0)$,satisfies the equation...
\end{solution}



\begin{solution}[by \href{https://artofproblemsolving.com/community/user/29428}{pco}]
	\begin{tcolorbox}Find all function $ f: \mathbb{R}\to\mathbb{R}$ such that there is $ f^{`}(0)$ following   $ xf(y)+yf(x)=a\underbrace {f( ...}_{n}(xy)...)$  Here  $ a>0$   \end{tcolorbox}
$f(x)=0$ $\forall x$ is a solution. Let us from now look only for non allzero solutions.

Let $P(x,y)$ be the assertion $xf(y)+yf(x)=af^{[n]}(xy)$
Let $u=f'(0)$

$P(x,0)$ $\implies$ $xf(0)=af^{[n]}(0)$ and so $f(0)=0$

$P(xy,1)$ $\implies$ $xyf(1)+f(xy)=af^{[n]}(xy)$ and so $P(x,y)$ may be written as new assertion 
$Q(x,y)$ : $xf(y)+yf(x)=f(xy)+f(1)xy$

and so $\frac{f(xy)}{xy}-f(1)=\frac{f(x)}x-f(1)+\frac{f(y)}y-f(1)$ $\forall x,y\ne 0$

Let then $g(x)=\frac{f(x)}x-f(1)$ from $\mathbb R^*\to\mathbb R$ and we get $g(xy)=g(x)+g(y)$ $\forall x,y\ne 0$

This is a classical equation whose solution is $g(x)=h(\ln |x|)$ where $h(x)$ is any solution of Cauchy additive equation.

And so $f(x)=bx+xh(\ln |x|)$ $\forall x\ne 0$

But then, $\lim_{x\to 0}\frac{f(x)}x=u$ $\implies$  $\lim_{x\to 0}h(\ln|x|)=u-b$ and so $h(x)$ is bounded when $x\to-\infty$
So $h(x)$ is continuous and is $cx$ for some real $c$ and we need $c=0$ in order to have $h(x)$ bounded when $x\to-\infty$

So $f(x)=bx$ and, pluging back in original equation : $b=0$ or $b=\left(\frac 2a\right)^{\frac 1{n-1}}$ if $n>1$ or any value if $n=1$ and $a=2$


\begin{bolded}Hence the solutions\end{underlined}\end{bolded} :
$f(x)=0$ $\forall x$
$f(x)=\left(\frac 2a\right)^{\frac 1{n-1}}x$ $\forall x$  if $n>1$
$f(x)=bx$ if $a=2$ and $n=1$ for any real $b$
\end{solution}
*******************************************************************************
-------------------------------------------------------------------------------

\begin{problem}[Posted by \href{https://artofproblemsolving.com/community/user/128206}{eddy13579}]
	We say that a funnction $f:[0,\infty)\to[0,\infty)$ \begin{bolded}has property P\end{bolded} if $f(xf(y^2))=f(y) f(f(x^2)) $ for any $x,y\in[0,\infty)$. Prove that there \begin{bolded}are infinite functions with property P.\end{bolded}
	\flushright \href{https://artofproblemsolving.com/community/c6h457586}{(Link to AoPS)}
\end{problem}



\begin{solution}[by \href{https://artofproblemsolving.com/community/user/128206}{eddy13579}]
	Any ideas :D?
\end{solution}



\begin{solution}[by \href{https://artofproblemsolving.com/community/user/29428}{pco}]
	\begin{tcolorbox}We say that a funnction $f:[0,\infty)\to[0,\infty)$ \begin{bolded}has property P\end{bolded} if $f(xf(y^2))=f(y) f(f(x^2)) $ for any $x,y\in[0,\infty)$. Prove that there \begin{bolded}are infinite functions with property P.\end{bolded}\end{tcolorbox}
Choose for example, for any $a>1$ the function $f_a(x)$ defined as :
$f_a(1)=1$
$f_a(a^{2^n})=1$ $\forall n\in\mathbb Z$
$f_a(x)=0$ $\forall$ other $x$
\end{solution}
*******************************************************************************
-------------------------------------------------------------------------------

\begin{problem}[Posted by \href{https://artofproblemsolving.com/community/user/68025}{Pirkuliyev Rovsen}]
	Let function $f: \mathbb{R}\to\mathbb{R}$ satisfiying $f(f(x))=x^2,
f(0)=0$.Find $f(1)$


__________________________
Azerbaijan Land of the Fire  
	\flushright \href{https://artofproblemsolving.com/community/c6h458557}{(Link to AoPS)}
\end{problem}



\begin{solution}[by \href{https://artofproblemsolving.com/community/user/29428}{pco}]
	\begin{tcolorbox}Let function $f: \mathbb{R}\to\mathbb{R}$ satisfiying $f(f(x))=x^2,
f(0)=0$.Find $f(1)$\end{tcolorbox}

$f(x^2)=f(x)^2$ and so $f(1)=f(1)^2$ and so $f(1)\in\{0,1\}$. But since $f(0)=0$, we cant have $f(1)=0$ else $f(f(1))=f(0)=0\ne 1^2$

So, if such a function exists, we get $\boxed{f(1)=1}$  (and indeed such a function exists, for example $f(x)=|x|^{\sqrt 2}$)
\end{solution}
*******************************************************************************
-------------------------------------------------------------------------------

\begin{problem}[Posted by \href{https://artofproblemsolving.com/community/user/86345}{namdan}]
	Find all functions $f:\mathbb{R}\rightarrow \mathbb{R}$ satisfy
$f(fx)+y^2)=f(x)^2-f(x)f(y)+xy+x$
[hide="I thinks"]Suppose $\exists x_{1}, x_{2}\in \mathbb{R}$ such that $f(x_{1})=f(x_{2})$. 
We have $f(x_{1})=f(x_{2})$ so $f(f(x_{1})+y^2)=f(f(x_{2})+y^2)$ so $f(x_{1})^2- f(x_{1})f(y)+x_{1}y+x_{1}=f(x_{2})^2- f(x_{2})f(y)+x_{2}y+x_{2}$. So $x_{1}=x_{2} \forall x, y\in \mathbb{R}$. So $f$ is injective
Let $P(x,y)$ be the assertion $f(fx)+y^2)=f(x)^2-f(x)f(y)+xy+x$
$P(0,0)\Rightarrow f(f0)=0$
$P(f(0),y)\Rightarrow f(y^2)=yf(0)+f(0)$. So $yf(0)=0\forall  y\in \mathbb{R}$. So $f(0)=0$
Let $y:=-1$ in $f(y^2)=yf(0)+f(0)$ we have $f(1)=0$. So $f(1)=f(0)$, but $f$ is injective and $1\neq 0$, so not solution.[\/hide]
	\flushright \href{https://artofproblemsolving.com/community/c6h458728}{(Link to AoPS)}
\end{problem}



\begin{solution}[by \href{https://artofproblemsolving.com/community/user/29428}{pco}]
	\begin{tcolorbox}Find all functions $f:\mathbb{R}\rightarrow \mathbb{R}$ satisfy
$f(fx)+y^2)=f(x)^2-f(x)f(y)+xy+x$\end{tcolorbox}
Let $P(x,y)$ be the assertion $f(f(x)+y^2)=f(x)^2-f(x)f(y)+xy+x$

$P(1,1)$ $\implies$ $f(f(1)+1)=2$
$P(1,-1)$ $\implies$ $f(f(1)+1)=f(1)^2-f(1)f(-1)$ and so $f(1)(f(1)-f(-1))=2$ and so $f(1)-f(-1)\ne 0$

$P(x,1)$ $\implies$ $f(f(x)+1)=f(x)^2-f(x)f(1)+2x$
$P(x,-1)$ $\implies$ $f(f(x)+1)=f(x)^2-f(x)f(-1)$ and so $f(x)(f(1)-f(-1))=2x$ $\implies$ $f(x)=\frac 2{f(1)-f(-1)}x=ax$ for some $a$

Plugging back $f(x)=ax$ in original equation, we get that no $a$ fits and so no solution.
\end{solution}



\begin{solution}[by \href{https://artofproblemsolving.com/community/user/53448}{panos_lo}]
	We have $f(f(0))=0$. Putting $x=f(0)=c$, we get $f(y^2)=c(y+1)$ and from this (set $-y$ instead of $y$) we get $f(y^2)=0$, thus $f(1)=0$. Plugging $x=y=1$ the original equation gives $0=2$, absurd. Thus, no solution.
\end{solution}
*******************************************************************************
-------------------------------------------------------------------------------

\begin{problem}[Posted by \href{https://artofproblemsolving.com/community/user/139146}{Luigi19}]
	Find all functions $f:\mathbb{R} \to \mathbb{R}$ such that for all pairs $(x,y) \in \mathbb{R}$:

$f(xf(x)-yf(y))=(x+y)(f(x)-f(y))$
	\flushright \href{https://artofproblemsolving.com/community/c6h459129}{(Link to AoPS)}
\end{problem}



\begin{solution}[by \href{https://artofproblemsolving.com/community/user/29428}{pco}]
	\begin{tcolorbox}Find all functions $f:\mathbb{R} \to \mathbb{R}$ such that for all pairs $(x,y) \in \mathbb{R}$:

$f(xf(x)-yf(y))=(x+y)(f(x)-f(y))$\end{tcolorbox}
Let $P(x,y)$ be the assertion $f(xf(x)-yf(y))=(x+y)(f(x)-f(y))$
$f(x)=0$ $\forall x$ is a solution. So let us from now look for non all-zero solutions.
Let $u$ such that $f(u)\ne 0$

1) $f(x)=0$ $\iff$ $x=0$
=================
$P(0,0)$ $\implies$ $f(0)=0$
If $f(x)=0$ for some $x$, then :
$P(u,0)$ $\implies$ $f(uf(u))=uf(u)$
$P(u,x)$ $\implies$ $f(uf(u))=(u+x)f(u)$ $\implies$ $x=0$
Q.E.D.

2) $f(x)=x$
==========
$P(1,-1)$ $\implies$ $f(f(1)+f(-1))=0$ $\implies$ $f(-1)=-f(1)$
$P(x,1)$ $\implies$ $f(xf(x)-f(1))=(x+1)(f(x)-f(1))$
$P(x,-1)$ $\implies$ $f(xf(x)-f(1))=(x-1)(f(x)+f(1))$
So $(x+1)(f(x)-f(1))=(x-1)(f(x)+f(1))$ and so $f(x)=xf(1)$
Plugging this in original equation, we get $f(1)=1$
Q.E.D.

3) synthesis of solutions
=================
We got two solutions :
$f(x)=0$ $\forall x$
$f(x)=x$ $\forall x$
\end{solution}
*******************************************************************************
-------------------------------------------------------------------------------

\begin{problem}[Posted by \href{https://artofproblemsolving.com/community/user/139660}{pagla}]
	Find all functions f:R to R, such that  f(x)*f(y)=f(x+y) + x*y

.
	\flushright \href{https://artofproblemsolving.com/community/c6h459613}{(Link to AoPS)}
\end{problem}



\begin{solution}[by \href{https://artofproblemsolving.com/community/user/29428}{pco}]
	\begin{tcolorbox}Find all functions f:R to R, such that  f(x)*f(y)=f(x+y) + x*y

.\end{tcolorbox}
Let $P(x,y)$ be the assertion $f(x)f(y)=f(x+y)+xy$

$P(x,0)$ $\implies$ $f(x)(f(0)-1)=0$ and so, since $f(x)=0$ $\forall x$ is not a solution, we get $f(0)=1$

$P(1,-1)$ $\implies$ $f(1)f(-1)=0$

If $f(1)=0$, then $P(x-1,1)$ $\implies$ $f(x)=1-x$ which indeed is a solution
If $f(-1)=0$, then $P(x+1,-1)$ $\implies$ $f(x)=1+x$ which indeed is a solution

Hence the only two solutions :
$f(x)=1+x$ $\forall x$
$f(x)=1-x$ $\forall x$
\end{solution}



\begin{solution}[by \href{https://artofproblemsolving.com/community/user/84677}{andreass}]
	Another solution (longer of course):
Rearrange the original equation to $f(x+y)=f(x)f(y)-xy$. By taking it for three variables we get:
$f(x+y+z)=f(x+y)f(z)-(x+y)z=(f(x)f(y)-xy)f(z)-xz-yz$
$=f(x)f(y)f(z)-xyf(z)-xz-yz$
However by taking the other two variables first:
$f(x+y+z)=f(x+z)f(y)-(x+z)y=(f(x)f(z)-xz)f(y)-xy-yz$
$=f(x)f(y)f(z)-xzf(y)-xy-yz$
Now take the last sides of each equation. By eliminating $f(x)f(y)f(z)$ and $yz$ we get: $xyf(z)+xz=xzf(y)+xy \Rightarrow yf(z)+z=zf(y)+y$.
Now we try to separate $y$ from $z$.
$\Rightarrow yf(z)-y=zf(y)-z \Rightarrow y(f(z)-1)=z(f(y)-1) \Rightarrow \frac{f(y)-1}{y}=\frac{f(z)-1}{z}$
What this means is that $\frac{f(x)-1}{x}$ is a constant for all $x$ $\Rightarrow \frac{f(x)-1}{x}=k \Rightarrow f(x)-1=kx \Rightarrow f(x)=kx+1$
Substituting to the original equation we get $k=\pm 1$ and hence our only solutions are $f(x)\equiv 1+x$ and $f(x) \equiv 1-x$.
\end{solution}
*******************************************************************************
-------------------------------------------------------------------------------

\begin{problem}[Posted by \href{https://artofproblemsolving.com/community/user/111816}{waver123}]
	Find all $ f:\mathbb{R}\rightarrow\mathbb{R} $ such that $f(x+y^{2}) \geq (y+1)f(x)$.
	\flushright \href{https://artofproblemsolving.com/community/c6h459629}{(Link to AoPS)}
\end{problem}



\begin{solution}[by \href{https://artofproblemsolving.com/community/user/29428}{pco}]
	\begin{tcolorbox}Find all $ f:\mathbb{R}\rightarrow\mathbb{R} $ such that $f(x+y^{2}) \geq (y+1)f(x)$.\end{tcolorbox}
Let $P(x,y)$ be the assertion $f(x+y^2)\ge(y+1)f(x)$

$P(x-1,-1)$ $\implies$ $f(x)\ge 0$ $\forall x$

Let $y>0$ : $P(x+y^2,y)$ $\implies$ $f(x+2y^2)\ge (y+1)f(x+y^2)\ge (y+1)^2f(x)$
An immediate induction implies $f(x+ny^2)\ge (y+1)^nf(x)$

Setting $y=\frac 1k$ and $n=k^2$ in this expression, we get $f(x+1)\ge(1+\frac 1k)^{k^2}f(x)$ and so $f(x+1)(1+\frac 1k)^{-k^2}\ge f(x)\ge 0$

Setting $k\to+\infty$ in this inequality, we get $\boxed{f(x)=0}$ $\forall x$, which indeed is a solution.
\end{solution}
*******************************************************************************
-------------------------------------------------------------------------------

\begin{problem}[Posted by \href{https://artofproblemsolving.com/community/user/109774}{littletush}]
	Determine all real-valued functions $f$ on the set of real numbers satisfying
\[2f(x)=f(x+y)+f(x+2y)\]
for all real numbers $x$ and all non-negative real numbers $y$.
	\flushright \href{https://artofproblemsolving.com/community/c6h462017}{(Link to AoPS)}
\end{problem}



\begin{solution}[by \href{https://artofproblemsolving.com/community/user/29428}{pco}]
	\begin{tcolorbox}determine all real-valued functions $f$ on $R$ satisfying
$2f(x)=f(x+y)+f(x+2y)$
for all real numbers $x$ and $y\ge 0$.\end{tcolorbox}
Let $P(x,y)$ be the assertion $2f(x)=f(x+y)+f(x+2y)$

(a) : $P(2x,x)$ $\implies$ $2f(2x)=f(3x)+f(4x)$
(b) : $P(x,x)$ $\implies$ $2f(x)=f(2x)+f(3x)$
(c) : $P(0,x)$ $\implies$ $2f(0)=f(x)+f(2x)$
(d) : $P(0,2x)$ $\implies$ $2f(0)=f(2x)+f(4x)$

(a)-(b)+4(c)-(d) : $f(x)=f(0)$

And so $\boxed{f(x)=c}$ $\forall x\ge 0$ and $f(x)=$ any value $\forall x<0$ which indeed is a solution
\end{solution}



\begin{solution}[by \href{https://artofproblemsolving.com/community/user/89198}{chaotic_iak}]
	I guess because $y \ge 0$ is written in one LaTeX and not two (as in $y$ $\ge 0$), littletush intended it to be for all $x$ real and for all $y \ge 0$. Hence $f(x) = c$ for all $x$, not only for non-negative ones.

[color=#FF0000][mod: the problem statement has now been edited to be clearer][\/color]
\end{solution}



\begin{solution}[by \href{https://artofproblemsolving.com/community/user/121493}{AnhIsGod}]
	You can see here link: http://mathproblems123.wordpress.com\/2011\/04\/20\/romanian-tst-2011-problem-1\/
\end{solution}



\begin{solution}[by \href{https://artofproblemsolving.com/community/user/92753}{WakeUp}]
	here is my solution:

Comparing $P(x,y)$ and $P(x,2y)$ we get $f(x+y)+f(x+2y)=f(x+2y)+f(x+4y)$ so $f(x+y)=f(x+4y)$. Therefore when $x=-y$ we get $f(0)=f(3y)$ so when $x>0$, we have $f(x)=f(0)=c$. If $a$ is a negative real then $P(a,-a)$ implies $2f(a)=f(0)+f(-a)=2c$ so it follows $f(x)=c$ is constant, for any constant $c$.
\end{solution}



\begin{solution}[by \href{https://artofproblemsolving.com/community/user/96032}{prasanna1712}]
	Here's another solution:
$P(-t,t) \implies 2f(-t)=f(0)+f(t)$
$P(t,-t) \implies 2f(t)=f(0)+f(-t)$
Hence, $2f(-t)+f(0)+f(-t)=f(0)+f(t)+ 2f(t) \implies f(-t)=f(t)$
Putting this back into $P(-t,t)$ we get $f(t)=f(0)$ so
$ \boxed{f(x)=c} $ which is indeed a solution
\end{solution}



\begin{solution}[by \href{https://artofproblemsolving.com/community/user/152025}{shekast-istadegi}]
	$f(0)=a$ ,so $2f(x)=f(-x+x)+f(2x-x)$ so $f(x)=f(0)=a$
\end{solution}



\begin{solution}[by \href{https://artofproblemsolving.com/community/user/29428}{pco}]
	\begin{tcolorbox}$f(0)=a$ ,so $2f(x)=f(-x+x)+f(2x-x)$ so $f(x)=f(0)=a$\end{tcolorbox}
I dont understand :
From $2f(x)=f(x+y)+f(x+2y)$, what are the values you used for $x,y$ in order to get your equality ?
\end{solution}



\begin{solution}[by \href{https://artofproblemsolving.com/community/user/64716}{mavropnevma}]
	\begin{tcolorbox}Here's another solution:
$P(-t,t) \implies 2f(-t)=f(0)+f(t)$
$P(t,-t) \implies 2f(t)=f(0)+f(-t)$
Hence, $2f(-t)+f(0)+f(-t)=f(0)+f(t)+ 2f(t) \implies f(-t)=f(t)$
Putting this back into $P(-t,t)$ we get $f(t)=f(0)$ so
$ \boxed{f(x)=c} $ which is indeed a solution\end{tcolorbox}
In $P(x,y)$ one can only use $y\geq 0$, so both $P(-t,t)$ and $P(t,-t)$ cannot be simultaneously considered.
\end{solution}



\begin{solution}[by \href{https://artofproblemsolving.com/community/user/92753}{WakeUp}]
	A quicker way to finish off my solution:

After obtaining $f(x+y)=f(x+4y)$, compare the result of letting $x=-y$ and $x=-4y$. We obtain $f(3y)=f(0)=f(-3y)$ so $f(x)$ is clearly constant.
\end{solution}



\begin{solution}[by \href{https://artofproblemsolving.com/community/user/202408}{junioragd}]
	I think this is too easy for a TST even for a day $1$ number $1$.Plugg $x,y$ and then plugg $x,2y$ and obtain $f(x+y)=f(x+4y)$ and from this point it is obvious and it can be finished in milion ways,but still here is one:Plugg $x=-y$ to get $f(3x)=f(0)$ and $x=-4y$ to get $f(-3y)=f(0)$,so we have that $f(x)=c$ for any real $c$.
\end{solution}



\begin{solution}[by \href{https://artofproblemsolving.com/community/user/170877}{MathStudent2002}]
	Let $g(x) = f(x)-f(0)$, then observe that by subtracting $2f(0)$ from both sides, $2g(x) = g(x+y) + g(x+2y)$, call this $P(x,y)$. Observe that $g(0)= 0$.

Fix $y\ge 0$.

Then, taking $P(-y,y)$ yields $2g(-y) = g(y)$. Then, take $P(0,y)$ to get $g(y)=-g(2y) = -2g(-2y)$. Now, take $P(-2y,y)$ to get $2g(-2y) = g(-y)$. But $2g(-2y) = -g(y)$, so $g(y)+g(-y) = 0$; since $g(y) = 2g(-y)$, this yields $g(y) = g(-y) = 0$ for all $y\ge 0$, hence $g(x) = 0$ for all $x$, so $f(x) - f(0) = 0$, or $f(x) = f(0)$, hence $f$ is constant, which works. 
\end{solution}
*******************************************************************************
-------------------------------------------------------------------------------

\begin{problem}[Posted by \href{https://artofproblemsolving.com/community/user/122611}{oty}]
	Hey, I want to propose this problem. Consider two positive integers $m$ and $n$, with  $m > 1$ and $n>1$, ($m$ even).  Let $f$ be a function defined on $\mathbb{R}_+$, and which satisfies the following   conditions:

1) $f\left (\dfrac {x_1^m +\cdots+x_n^m} {n} \right) = \dfrac {f(x_1)^m +\cdots+f(x_n)^m} {n}$, for all $x_1, \ldots,x_n \in \mathbb{R}_+$; 
2) $f(2004)\neq  2004$ and $f(2006) \neq 0$. 

Compute  $f(2005)$. Thanks.
	\flushright \href{https://artofproblemsolving.com/community/c6h462127}{(Link to AoPS)}
\end{problem}



\begin{solution}[by \href{https://artofproblemsolving.com/community/user/122611}{oty}]
	Any idea ?
\end{solution}



\begin{solution}[by \href{https://artofproblemsolving.com/community/user/123624}{nyanks}]
	I know it. it is 2005
\end{solution}



\begin{solution}[by \href{https://artofproblemsolving.com/community/user/122611}{oty}]
	Do you know , where I can get his solution ? because I do not arrive to solve it .
\end{solution}



\begin{solution}[by \href{https://artofproblemsolving.com/community/user/29428}{pco}]
	\begin{tcolorbox}Hey, I want to propose this problem. Consider two positive integers $m$ and $n$, with  $m > 1$ and $n>1$, ($m$ even).  Let $f$ be a function defined on $\mathbb{R}_+$, and which satisfies the following   conditions:

1) $f\left (\dfrac {x_1^m +\cdots+x_n^m} {n} \right) = \dfrac {f(x_1)^m +\cdots+f(x_n)^m} {n}$, for all $x_1, \ldots,x_n \in \mathbb{R}_+$; 
2) $f(2004)\neq  2004$ and $f(2006) \neq 0$. 

Compute  $f(2005)$. Thanks.\end{tcolorbox}
Setting $x_k=x$ $\forall k$, we get $f(x^m)=f(x)^m$ and so equation may be written $f\left(\frac{\sum_{k=1}^nx_k^m}n\right)$ $=\frac{\sum_{k=1}^nf(x_k^m)}n$

So $f\left(\frac{\sum_{k=1}^nx_k}n\right)$ $=\frac{\sum_{k=1}^nf(x_k)}n$ $\forall x_k>0$

Setting $x_1=u+v$ and $x_2=1$, this becomes $f\left(\frac{u+v+1+\sum_{k=3}^nx_k}n\right)$ $=\frac{f(u+v)+f(1)+\sum_{k=3}^nf(x_k)}n$

Setting then $x_1=u$ and $x_2=v+1$, this becomes $f\left(\frac{u+v+1+\sum_{k=3}^nx_k}n\right)$ $=\frac{f(u)+f(v+1)+\sum_{k=3}^nf(x_k)}n$

And so $f(u+v)+f(1)=f(u)+f(v+1)$ $\forall u,v>0$
So $f(u+v)-f(u)=g(v)$ for some function $g(x)=f(x+1)-f(1)$
So $f(u+v)-f(v)=g(u)$ and so $f(u)-g(u)=f(v)-g(v)=c$ constant

And we get $f(x+y)=f(x)+f(y)-c$ $\forall x,y>0$

So $f(x)-c$ matches Cauchy equation and is lower bounded since $f(x^m)=f(x)^m$ and $m$ even imply $f(x)\ge 0$ $\forall x>0$

So $f(x)=ax+c$ for some $a,c$

But then $f(x^m)=f(x)^m$ implies $(a,c)\in\{(0,0),(0,1),(1,0)\}$

$(a,c)=(0,0)$ is impossible since $f(2006)\ne 0$
$(a,c)=(1,0)$ is impossible since $f(2004)\ne 2004$

Hence the unique solution $f(x)=1$ $\forall x$ which indeed is a solution

And so $\boxed{f(2005)=1}$
\end{solution}
*******************************************************************************
-------------------------------------------------------------------------------

\begin{problem}[Posted by \href{https://artofproblemsolving.com/community/user/139527}{lzh}]
	find all function:$R$ to $R$,such that for any real numbers $x,y$
(1)$f(x+y)=f(x)+f(y)$
(2)$f(f(x))=x$
	\flushright \href{https://artofproblemsolving.com/community/c6h463336}{(Link to AoPS)}
\end{problem}



\begin{solution}[by \href{https://artofproblemsolving.com/community/user/29428}{pco}]
	\begin{tcolorbox}find all function:$R$ to $R$,such that for any real numbers $x,y$
(1)$f(x+y)=f(x)+f(y)$
(2)$f(f(x))=x$\end{tcolorbox}
\begin{bolded}General solution\end{underlined}\end{bolded} :
Let $A,B$ two supplementary vector subspaces of the $\mathbb Q$ vectorspace $\mathbb R$
Let $a(x)$ from $\mathbb R\to A$ and $b(x)$ from $\mathbb R\to B$ the two projections of a real $x$ in $A$ and $B$ (so that $x=a(x)+b(x)$ in a unique way)

Then  $\boxed{f(x)=a(x)-b(x)}$

\begin{bolded}Proof that this indeed is a solution\end{underlined}\end{bolded} :
$a(x)$ and $b(x)$ are linear functions and so is their difference. So $f(x+y)=f(x)+f(y)$
$f(f(x))=a(a(x)-b(x))-b(a(x)-b(x))$ $=a(a(x))-a(b(x))-b(a(x))+b(b(x))$ $=a(x)-0-0+b(x)$ $=x$
Q.E.D.

\begin{bolded}Proof that any solution may be written in such a way (and so that this is a general solution)\end{underlined}\end{bolded} :
Let $f(x)$ a function such that $f(x+y)=f(x)+f(y)$ and $f(f(x))=x$
Let $A=\{x$ such that $f(x)=x\}$
Let $B=\{x$ such that $f(x)=-x\}$
Obviously $A$ and $B$ are two vector subspaces of the $\mathbb Q$ vectorspace $\mathbb R$ and  $A\cap B=\{0\}$

Let then $a(x)=\frac{x+f(x)}2$ and $b(x)=\frac{x-f(x)}2$ :

$f\left(\frac{x+f(x)}2\right)=\frac{x+f(x)}2$ and so $a(x)\in A$

$f\left(\frac{x-f(x)}2\right)=-\frac{x-f(x)}2$ and so $b(x)\in B$

We obviously have $x=a(x)+b(x)$ and so $A$ and $B$ are two supplementary vector subspaces

And $a(x)-b(x)=f(x)$
Q.E.D.
\end{solution}
*******************************************************************************
-------------------------------------------------------------------------------

\begin{problem}[Posted by \href{https://artofproblemsolving.com/community/user/3182}{Kunihiko_Chikaya}]
	Find all functions $ f : \mathbb{R} \mapsto \mathbb{R}$ such that $f(f(x+y)f(x-y))=x^2-yf(y)$ for all $ \ x,y \in \mathbb{R}.$
	\flushright \href{https://artofproblemsolving.com/community/c6h463346}{(Link to AoPS)}
\end{problem}



\begin{solution}[by \href{https://artofproblemsolving.com/community/user/64716}{mavropnevma}]
	Plugging $-y$ instead of $y$ yields $f(-y) = -f(y)$ for $y\neq 0$. For $y=0$ we get $f(f(x)^2) = x^2$, while for $x=0$ we get $f(f(y)^2) = yf(y)$ for all $y\neq 0$, if also $f(y)\neq 0$. Thus, $z^2 = zf(z)$ for all $z\neq 0$ with $f(z)\neq 0$, whence $f(z)=z$. It remains to ascertain the value of $f(0)$. Patrick, in the sequel, eliminated all other possibilities but $f(0)=0$.
\end{solution}



\begin{solution}[by \href{https://artofproblemsolving.com/community/user/29428}{pco}]
	\begin{tcolorbox}Find all functions $ f : \mathbb{R} \mapsto \mathbb{R}$ such that $f(f(x+y)f(x-y))=x^2-yf(y)$ for all $ \ x,y \in \mathbb{R}.$\end{tcolorbox}
Let $P(x,y)$ be the assertion $f(f(x+y)f(x-y))=x^2-yf(y)$
Let $a=f(0)$

$P(0,0)$ $\implies$ $f(a^2)=0$
$P(a^2,0)$ $\implies$ $a=a^4$ and so $a\in\{0,1\}$

If $a=1$ then $f(0)=1$ and $f(1)=0$ and then $P(2,1)$ $\implies$ $1=4$ impossible

So $a=0$ and $f(0)=0$
Then $P(x,x)$ $\implies$ $xf(x)=x^2$ and so $\boxed{f(x)=x}$ $\forall x$ which indeed is a solution
\end{solution}



\begin{solution}[by \href{https://artofproblemsolving.com/community/user/125018}{horizon}]
	by changing $x+y=u,x-y=v$, we have
$f(f(u)f(v))=(\frac{u+v}{2})^{2}-(\frac{u-v}{2})f(\frac{u-v}{2})$ by exchange $u$and $v$ we get $f(x)+f(-x)=0$
then return to the original equation,exchange $x$ and $y$ we have 
$-f(f(x-y)f(x+y))=y^{2}-xf(x)$,compare to the former equation,
we have $xf(x)-x^{2}=y^{2}-yf(y)$ so $xf(x)-x^{2}=0$ so for all non-zero $x$, $f(x)=x$,
then put $x=y$(not equal to $0$),we have $f(2xf(0))=0$ if $f(0)$ is not equal to $0$ then for all $x$ $f(x)=0$ a contradiction! so $f(0)=0$ 
then we get for all $x$ ,$f(x)=x$
\end{solution}



\begin{solution}[by \href{https://artofproblemsolving.com/community/user/82357}{applepi2000}]
	Let $P(x, y)$ be this assertion. Then $P(x, y), P(x, -y)$ together give $f(k)=-f(-k)$. Then $P(x, 0)$ implies $f(x)$ is surjective, so let $f(a)=0$, so $f(-a)=0$. We get $P(0, a)\implies f(0)=0$. Finally, $P(x, x)\implies xf(x)=x^2$, which combined with $f(0)=0$ gives $\boxed{f(x)=x\forall x}$.
\end{solution}



\begin{solution}[by \href{https://artofproblemsolving.com/community/user/160772}{hofamo}]
	[hide]its easy to know we have one z with f(z)=0.if Y=z and X=0 we have f(0)=0.if X=Y we have f(X)=X.[\/hide]
\end{solution}



\begin{solution}[by \href{https://artofproblemsolving.com/community/user/173477}{anhchangtoanhoc97}]
	Let $P(x,y)$ be the assertion $f(f(x+y)f(x-y))=x^2-yf(y)$
$P(0,y)$ $\Rightarrow $ $(f(f(y).f(-y))= -yf(y)$ (1)
$P(0,-y)$ $\Rightarrow $ $f(f(y).f(-y))=y.f(-y)$ (2)
(1),(2) $\Rightarrow $ $f(y)=-f(-y)$ so $f$ odd (*)
- $P(x,0)$ $\Rightarrow $ $f(f(x)^2)=x^2$ (3) . This meaning if $f(x)=f(y)$ $\Rightarrow$  $x^2=y^2$
For x =0, $f(f(0)^2)=0$
Now replacing x by $f(0)^2$ in (3):$ f(0)=f(0)^2$ $\Leftrightarrow $ $f(0)=0$ or $f(0)=1$
+ If $f(0)=0$,  $P(x,x)$  $\Rightarrow $ $f(0)=x^2-x.f(x)$  $\Rightarrow $ $\boxed{\text{f(x)=x}}$
+ If $f(0)=1$
$P(0,0)$ $\Rightarrow $ $f(1)=0$
$P(x,1)$ $\Rightarrow $  $f(f(x+1).f(x-1)=x^2$ ; f odd so $-f(f(1+x).f(1-x))=x^2 \Leftrightarrow -1+xf(x)=x^2 \Leftrightarrow  f(x)=x+\frac{1}{x}$, retry => contradict
Therefore $f(x)=x$
\end{solution}



\begin{solution}[by \href{https://artofproblemsolving.com/community/user/321670}{MNZ2000}]
	I think its a little simpler .Let $P(x, y)$ be this assertion. Then $P(x, y), P(x, -y)$ together give $f(k)=-f(-k)$ . $\boxed{1}$ $p(x,0)$ then give $f(f(x)^2)=x^2$ $\boxed{2}$ $p(0,-x)$ and $\boxed{1}$ give , $-f(f(x)^2)=-xf(x)$ $\boxed{3}$ With $2$ & $3$ give us: $\boxed{f(x)=x\forall x}$
\end{solution}



\begin{solution}[by \href{https://artofproblemsolving.com/community/user/159507}{MSTang}]
	Letting $x=y=0$ gives $f(f(0)^2) = 0$. Letting $x=0$ and $y=f(0)^2$ gives \[f\left(f\left(f(0)^2\right)f\left(-f(0)^2\right)\right) = -f(0)^2f\left(f(0)^2\right) \implies f(0) = 0.\] Letting $y=x$ gives $f(f(2x)f(0)) = 0 = x^2 - xf(x)$ for all $x$, so $f(x) = x$ for all $x \neq 0$. Since we know $f(0)=0$, the only solution is $f(x) = x$ for all $x$.
\end{solution}
*******************************************************************************
-------------------------------------------------------------------------------

\begin{problem}[Posted by \href{https://artofproblemsolving.com/community/user/68025}{Pirkuliyev Rovsen}]
	Find all functions $f,g: \mathbb{R}\to\mathbb{R}$ such that
$2f(x)+f(1-y)+g(x)-g(y)=3(x+1)^2-6y$


______________________________
Azerbaijan Land of the Fire 
	\flushright \href{https://artofproblemsolving.com/community/c6h463523}{(Link to AoPS)}
\end{problem}



\begin{solution}[by \href{https://artofproblemsolving.com/community/user/29428}{pco}]
	\begin{tcolorbox}Find all functions $f,g: \mathbb{R}\to\mathbb{R}$ such that
$2f(x)+f(1-y)+g(x)-g(y)=3(x+1)^2-6y$\end{tcolorbox}
Let $P(x,y)$ be the assertion $2f(x)+f(1-y)+g(x)-g(y)=3(x+1)^2-6y$

$P(x,x) $ $\implies$ e1 : $2f(x)+f(1-x)=3x^2+3$
$P(1-x,1-x) $ $\implies$ e2 : $2f(1-x)+f(x)=6-6x+3x^2$
2*e1-e2 : $3f(x)=3x^2+6x$ 

And so $f(x)=x^2+2x$
Then $P(x,0)$ $\implies$ $g(x)=x^2+2x+g(0)$


And so $\boxed{f(x)=x^2+2x\text{ and }g(x)=x^2+2x+a}$ which indeed is a solution
\end{solution}
*******************************************************************************
-------------------------------------------------------------------------------

\begin{problem}[Posted by \href{https://artofproblemsolving.com/community/user/127783}{Sayan}]
	Determine all function $f$ from the set of real numbers to itself such that for every $x$ and $y$,
\[f(x^2-y^2)=(x-y)(f(x)+f(y))\]
	\flushright \href{https://artofproblemsolving.com/community/c6h463664}{(Link to AoPS)}
\end{problem}



\begin{solution}[by \href{https://artofproblemsolving.com/community/user/29428}{pco}]
	\begin{tcolorbox}Determine all function $f$ from the set of real numbers to itself such that for every $x$ and $y$,
\[f(x^2-y^2)=(x-y)(f(x)+f(y))\]\end{tcolorbox}
Let $P(x,y)$ be the assertion $f(x^2-y^2)=(x-y)(f(x)+f(y))$
Let $a=f(1)$

$P(0,0)$ $\implies$ $f(0)=0$
$P(-1,0)$ $\implies$ $f(-1)=-a$
Comparing $P(x,1)$ and $P(x,-1)$, we get $(x-1)(f(x)+a)=(x+1)(f(x)-a)$ and so  $\boxed{f(x)=ax}$ $\forall x$ and for any real $a$, which indeed is a solution
\end{solution}



\begin{solution}[by \href{https://artofproblemsolving.com/community/user/201868}{NYY}]
	Let $P(x,y)$ be the assertion $f(x^2-y^2)=(x-y)(f(x)+f(y))$.
$P(0,0)=> f(0)=0$
$P(x,-x) => 2x(f(x)+f(-x))=0 => f(x)=f(-x)$
$P(x,-y) => P(x^2-y^2)=(x+y)(f(x)-f(y))$
=> $f(x)=ax$ .
\end{solution}



\begin{solution}[by \href{https://artofproblemsolving.com/community/user/105700}{ssilwa}]
	:O This is USAMO 2002.4

http://www.artofproblemsolving.com/Forum/viewtopic.php?p=337857&sid=c1bdb13e96665a23a63f347bb40b9fc4#p337857

Oh opps.. sorry, disregard
\end{solution}



\begin{solution}[by \href{https://artofproblemsolving.com/community/user/1430}{JBL}]
	No, that is not the same question.
\end{solution}



\begin{solution}[by \href{https://artofproblemsolving.com/community/user/223099}{MathPanda1}]
	$x=y$ into given implies $f(0)=0$.
$y=0$ into given implies $f(x^2)=xf(x)$. Changing $x$ to $-x$ in this gives $f$ is odd.
Changing $y$ to $-y$ in given implies 
$xf(x)-yf(y)+xf(y)-yf(x)=(x-y)(f(x)+f(y))=f(x^2-y^2)=
(x+y)(f(x)+f(-y))= xf(x)-yf(y)-xf(y)+yf(x)$
so $2xf(y)=2yf(x)$ or $\frac{f(x)}{x}=\frac{f(y)}{y}$ for all reals $x,y$. Therefore, $\frac{f(x)}{x}$ is a constant, so $f(x)=cx$ for some constant $c$, which indeed satisfies the problem.
\end{solution}
*******************************************************************************
-------------------------------------------------------------------------------

\begin{problem}[Posted by \href{https://artofproblemsolving.com/community/user/124745}{Uzbekistan}]
	Find all continous functions $f:R\rightarrow R$  that satisfy
$f(f(x))=f(x)+2x$
	\flushright \href{https://artofproblemsolving.com/community/c6h463707}{(Link to AoPS)}
\end{problem}



\begin{solution}[by \href{https://artofproblemsolving.com/community/user/64868}{mahanmath}]
	[url=http://www.artofproblemsolving.com/Forum/viewtopic.php?f=38&t=457745]Same idea[\/url]
\end{solution}



\begin{solution}[by \href{https://artofproblemsolving.com/community/user/124745}{Uzbekistan}]
	Another idea.
\end{solution}



\begin{solution}[by \href{https://artofproblemsolving.com/community/user/29428}{pco}]
	\begin{tcolorbox}Find all continous functions $f:R\rightarrow R$  that satisfy
$f(f(x))=f(x)+2x$\end{tcolorbox}
$f(x)$ is injective and continuous, and so monotonous.
If $f(x)$ has a finite limit $L$ when $x\to\infty$, then setting $x$ to this $\infty$ in $f(f(x))-f(x)=2x$ implies a contradiction (use continuity)
So $f(x)$ is surjective, and so bijective and it exists a function $f^{-1}(x)$

$f(x)=-x$ $\forall x$ is a solution.
Let us from now consider that $\exists a$ such that $f(a)+a\ne 0$

$f(f(f(a)))=f(f(a))+2f(a)=3f(a)+2a$ and so $\frac{f(f(f(a)))-f(a)}{f(f(a))-a}=2$ and so $f(x)$ is increasing.

Let then the sequence $x_0=f(a)$ and $x_1=a$ and $x_{n+1}=f^{-1}(x_n)$
We easily get $x_n=x_{n+1}+2x_{n+2}$ and a quick induction gives $x_n=\frac{(f(a)+a)2^{1-n}+(f(a)-2a)(-1)^n}3$

If $f(a)\ne 2a$, then $\lim_{n\to+\infty}\frac {f(x_{n+2})-f(x_{n+1})}{x_{n+2}-x_{n+1}}$ $=\lim_{n\to+\infty}\frac {x_{n+1}-x_n}{x_{n+2}-x_{n+1}}$ $=-1$
But this is impossible since we previously got that $f(x)$ was increasing. So $f(a)=2a$

And so $f(x)$ is continuous, increasing and $\forall x$, either $f(x)=-x$, either $f(x)=2x$ and so $f(x)=2x$ $\forall x$ which indeed is a solution.

Hence the only two solutions :
$f(x)=-x$ $\forall x$
$f(x)=2x$ $\forall x$
\end{solution}
*******************************************************************************
-------------------------------------------------------------------------------

\begin{problem}[Posted by \href{https://artofproblemsolving.com/community/user/121558}{Bigwood}]
	Find all $f$ from $\mathbb{Q}$ to it self such that
\[af(x)f(y)+f(x+y)=bf(x)f(y)+f(x)+f(y)\]
for all $x,y$ in  $\mathbb{Q}$. Let $a,b$ be reals.(not necessarily in $\mathbb{Q}$.)
Maybe easy...
	\flushright \href{https://artofproblemsolving.com/community/c6h464086}{(Link to AoPS)}
\end{problem}



\begin{solution}[by \href{https://artofproblemsolving.com/community/user/29428}{pco}]
	\begin{tcolorbox}Find all $f$ from $\mathbb{Q}$ to it self such that
\[af(x)f(y)+f(x+y)=bf(x)f(y)+f(x)+f(y)\]
for all $x,y$ in  $\mathbb{Q}$. Let $a,b$ be reals.(not necessarily in $\mathbb{Q}$.)
Maybe easy...\end{tcolorbox}
Are you sure about $a,b$ ? it seems we can simplify both sides of equation and write in RHS $(b-a)f(x)f(y)$
\end{solution}



\begin{solution}[by \href{https://artofproblemsolving.com/community/user/124745}{Uzbekistan}]
	Solution.    $P(x,y)\implies af(x)f(y)+f(x+y)=bf(x)f(y)+f(x)+f(y)$
 and then Put $a=b=\implies f(x+y)=f(x)+f(y)$ and this is Cauchy equation $\implies f(x)=cx$
\end{solution}



\begin{solution}[by \href{https://artofproblemsolving.com/community/user/89198}{chaotic_iak}]
	$a,b$ are givens.

If $f(0) \neq 0$, then $P(x,0)$ gives $f(x) = \dfrac{1}{a - b}$. For $a = b$, this simply shows $f(0) \neq 0$ is impossible and hence $f(0) = 0$.

More work later.
\end{solution}



\begin{solution}[by \href{https://artofproblemsolving.com/community/user/110524}{jatin}]
	\begin{tcolorbox}Solution.    $P(x,y)\implies af(x)f(y)+f(x+y)=bf(x)f(y)+f(x)+f(y)$
 and then Put $a=b=\implies f(x+y)=f(x)+f(y)$ and this is Cauchy equation $\implies f(x)=cx$\end{tcolorbox}
$a,b$ are some fixed reals I think. So you can't put $a=b$.
\end{solution}



\begin{solution}[by \href{https://artofproblemsolving.com/community/user/29428}{pco}]
	\begin{tcolorbox}Find all $f$ from $\mathbb{Q}$ to it self such that
\[af(x)f(y)+f(x+y)=bf(x)f(y)+f(x)+f(y)\]
for all $x,y$ in  $\mathbb{Q}$. Let $a,b$ be reals.(not necessarily in $\mathbb{Q}$.)
Maybe easy...\end{tcolorbox}
Ok, OP changed nothing (I was quite sure one side was $af(xy)$ or $bf(xy)$ instead of $af(x)f(y)$ or $bf(x)f(y)$) and so it seems the problem is correct.

Let $c=b-a$ and the problem is $f(x+y)=f(x)+f(y)+cf(x)f(y)$

If $c=0$, we get the classical $f(x)=f(1)x$

If $c\ne 0$, let $g(x)=cf(x)+1$ and we get $g(x+y)=g(x)g(y)$ whose solutions are
$g(x)=0$ $\forall x$
$g(x)=g(1)^x$ $\forall x$ where $g(1)>0$

It remains to check the constraint $f(x)\in\mathbb Q$ and only the first case may be in $\mathbb Q$ and we need $c\in\mathbb Q\setminus\{0\}$

\begin{bolded}Hence the result\end{underlined}\end{bolded} :
If $a=b$ then solutions are $f(x)=\alpha x$ for any $\alpha\in\mathbb Q$
If $a\ne b$ and $a-b\in\mathbb Q$, then the only solution is $f(x)=\frac 1{a-b}$
If $a-b\notin\mathbb Q$, then no solution.
\end{solution}
*******************************************************************************
-------------------------------------------------------------------------------

\begin{problem}[Posted by \href{https://artofproblemsolving.com/community/user/124745}{Uzbekistan}]
	If $2f(\frac{x}{x-1})-3f(\frac{3x-2}{2x+1})=\frac{13x-4}{2x-3x^2}$. Then $f(x)=?$
	\flushright \href{https://artofproblemsolving.com/community/c6h464531}{(Link to AoPS)}
\end{problem}



\begin{solution}[by \href{https://artofproblemsolving.com/community/user/29428}{pco}]
	\begin{tcolorbox}If $2f(\frac{x}{x-1})-3f(\frac{3x-2}{2x+1})=\frac{13x-4}{2x-3x^2}$. Then $f(x)=?$\end{tcolorbox}

What is the required domain for the function ?
What is the required domain for the functional equation ?
\end{solution}



\begin{solution}[by \href{https://artofproblemsolving.com/community/user/124745}{Uzbekistan}]
	PCO  what do you mean?
\end{solution}



\begin{solution}[by \href{https://artofproblemsolving.com/community/user/128206}{eddy13579}]
	\begin{tcolorbox}PCO  what do you mean?\end{tcolorbox}
A function must have a domain and a codomain
\end{solution}



\begin{solution}[by \href{https://artofproblemsolving.com/community/user/29428}{pco}]
	\begin{tcolorbox}PCO  what do you mean?\end{tcolorbox}
In a functional equation problem, the teacher has certainly given :

1) the domain (and generally too the codomain) of the function : is it a function from $\mathbb Z\to\mathbb C$, from $\mathbb R\to \mathbb R^+$, anything else ?

2) the domain of the functional equation : Is the statement you gave true $\forall x$?, $\forall x\in\{1,2\}$?; $\forall x\in$ domain of $f$?, $\forall x\in\mathbb R\setminus\{0,1\}$ ?

These are two important informations which may often change the solution.
\end{solution}



\begin{solution}[by \href{https://artofproblemsolving.com/community/user/124745}{Uzbekistan}]
	OH I am sorry $f:R\longrightarrow R$ and $x\neq \{1;-\frac{1}{2};0;\frac{2}{3}\}$
\end{solution}



\begin{solution}[by \href{https://artofproblemsolving.com/community/user/29428}{pco}]
	Then :
$2f(\frac{x}{x-1})-3f(\frac{3x-2}{2x+1})=\frac{13x-4}{2x-3x^2}$ $\forall x\notin\{-\frac 12,0,\frac 23,1\}$

Setting $x\to\frac x{x-1}$, this is equivalent to :
$2f(x)-3f(\frac{x+2}{3x-1})=-\frac{(9x+4)(x-1)}{x(x+2)}$ $\forall x\notin\{-2,0,\frac 13\}$

Setting then $x\to\frac{x+2}{3x-1}$, this is equivalent to :
$2f(\frac{x+2}{3x-1})-3f(x)=\frac{(2x-3)(3x+2)}{x(x+2)}$ $\forall x\notin\{-2,0,\frac 13\}$

Cancelling the $f(\frac{x+2}{3x-1})$ summand between the two last equalities, we get :
$-5f(x)=\frac{-2(9x+4)(x-1)+3(2x-3)(3x+2)}{x(x+2)}$ $=-\frac 5x$ $\forall x\notin\{-2,0,\frac 13\}$

And so $f(x)=\frac 1x$ $\forall x\notin\{-2,0,\frac 13\}$

And it's easy to check back that this indeed is a solution, whatever are the values of $f(-2),f(0),f(\frac 13)$
\end{solution}



\begin{solution}[by \href{https://artofproblemsolving.com/community/user/124745}{Uzbekistan}]
	[color=#FF0000][mod: no need to quote the entire post just above yours; and if you just want to thank - there is a Thanks button (which I see you in fact did use)][\/color]

It is correct nice solution PCO. Thanks
\end{solution}
*******************************************************************************
-------------------------------------------------------------------------------

\begin{problem}[Posted by \href{https://artofproblemsolving.com/community/user/88788}{phalkun}]
	Find all function $f$ mapping non-negative integers into non-negative integers and
such that $f(n)+f(f(n))=6n+4$ for all $n=0 , 1 , 2 , ...$.
	\flushright \href{https://artofproblemsolving.com/community/c6h464775}{(Link to AoPS)}
\end{problem}



\begin{solution}[by \href{https://artofproblemsolving.com/community/user/29428}{pco}]
	\begin{tcolorbox}Find all function $f$ mapping non-negative integers into non-negative integers and
such that $f(n)+f(f(n))=6n+4$ for all $n=0 , 1 , 2 , ...$.\end{tcolorbox}
It's easy to show with induction that $f^{[k]}(n)=$ $2^k\frac{f(n)+3n+4}5+$ $(-3)^k\frac{2n+1-f(n)}5-1$

If $f(n)<2n+1$ for some $n$, then, for odd $k$ great enough, we get $f^{[k]}(n)<0$, impossible.
If $f(n)>2n+1$ for some $n$, then, for even $k$ great enough, we get $f^{[k]}(n)<0$, impossible.

And so $\boxed{f(n)=2n+1}$ which indeed is a solution
\end{solution}
*******************************************************************************
-------------------------------------------------------------------------------

\begin{problem}[Posted by \href{https://artofproblemsolving.com/community/user/129317}{Dranzer}]
	Let $f:\mathbb{R} \to \mathbb{R}$ such that $f$ is continuous,$f(0)=1$ and $f(m+n+1)=f(m)+f(n)$ for all real $m$,$n$.Show that $f(x)=1+x$ for all real numbers $x$.Your help is appreciated;I managed to prove that for integers and for numbers of the form $\frac{1}{2^k}$ but would be grateful if someone could provide a proof for reals.

Thanks to chaotic_iak for pointing that typo out.I could not have a look again immediately after posting as my broadband connection is frequently down.
	\flushright \href{https://artofproblemsolving.com/community/c6h464832}{(Link to AoPS)}
\end{problem}



\begin{solution}[by \href{https://artofproblemsolving.com/community/user/89198}{chaotic_iak}]
	Assuming the likely statement that $f(0) = 1$ instead of $f(0) = 14$ (which then will invalidate the conclusion anyway), let $g(x) = f(x) - 1$ for all $x$. Note that because $f$ is continuous, $g$ is also continuous.

First, it's easy to see $f(1) = 2$ (substitution $m=n=0$), $g(0) = 0$ (definition of $g$), and $g(1) = 1$ (again definition of $g$). Also, we have:

$f(m+n+1) = f(m) + f(n)$
$f(m+n+1) - 1 = f(m) - 1 + f(n) - 1 + 1$
$g(m+n+1) = g(m) + g(n) + 1$
$g(m+n+1) = g(m) + g(n) + g(1)$ \begin{bolded}(1)\end{bolded}

Substituting to (1) $m=m,n=0$ gives:
$g(m+1) = g(m) + g(1)$ for all $m$

Hence we can state (1) as follows:
$g(m+n+1) = g(m) + (g(n) + g(1))$
$g(m+n+1) = g(m) + g(n+1)$
Denote $x=m,y=n+1$. Then

$g(x+y) = g(x) + g(y)$
which it a Cauchy functional equation. Because $g$ is continuous, the solution is $g(x) = ax$ for all real $x$ and some real $a$.

Substituting back to the equation $g(x) = f(x) - 1$ yields $f(x) = ax + 1$ for all real $x$ and some real $a$. But substituting to the original equation gives:
$f(m+n+1) = f(m) + f(n)$
$a(m+n+1) + 1 = am + 1 + an + 1$
$am + an + a + 1 = am + 1 + an + 1$
$a = 1$

Hence the solution $\boxed{f(x) = x+1}$ for all real $x$.
\end{solution}



\begin{solution}[by \href{https://artofproblemsolving.com/community/user/129317}{Dranzer}]
	Thanks!.A fine approach.But here's a hint that Dr Tao gave:

"First prove that integers, then for rationals and finally for real $x$".Can anyone please suggest how to finish off the problem that way?

As a matter of fact, proving that for integers turned out to be easy but when I tried proving it for rationals,I got really stuck.
\end{solution}



\begin{solution}[by \href{https://artofproblemsolving.com/community/user/29428}{pco}]
	\begin{tcolorbox}Thanks!.A fine approach.But here's a hint that Dr Tao gave:

"First prove that integers, then for rationals and finally for real $x$".Can anyone please suggest how to finish off the problem that way?

As a matter of fact, proving that for integers turned out to be easy but when I tried proving it for rationals,I got really stuck.\end{tcolorbox}
Let $g(x)=f(x-1)$ and the equation is $g((m+1)+(n+1))=g(m+1)+g(n+1)$

And so $g(1)=1$ and $g(m+n)=g(m)+g(n)$

You basically get from there :
$g(0)=0$
$g(kx)=kg(x)$ $\forall k\in\mathbb N$
$g(kx)=kg(x)$ $\forall k\in\mathbb Z$
$g(kx)=kg(x)$ $\forall k\in\mathbb Q$

And so $g(x)=xg(1)=x$ $\forall x\in\mathbb Q$
And continuity ends the problem : $g(x)=x$ $\forall x\in\mathbb R$
\end{solution}



\begin{solution}[by \href{https://artofproblemsolving.com/community/user/76247}{yugrey}]
	Note if a function $f(x)=g(x)$ when $x$ is rational and where $g(x)$ is continuous then $f(x)=g(x)$ for all real $x$.  This is because $h(x)=f(x)-g(x)$ for any $x$ can be bounded tightly (the rationals are dense) so that if it is not, then the slop eof $h(x)$ must be larger (in magnitude) than any real number, impossible.

EDIT: This technique is EXTREMELY useful in solving continuous functions.  I would even say that if you are given a function is continuous, there is a >90% chance that you will use this.
\end{solution}
*******************************************************************************
-------------------------------------------------------------------------------

\begin{problem}[Posted by \href{https://artofproblemsolving.com/community/user/124745}{Uzbekistan}]
	Find all functions 
\[
\begin{array}{l}
 f:N \to N \\ 
 2(f(m^2  + n^2 ))^3  = f^2 (m) \cdot f(n) + f^2 (n) \cdot f(m) \\ 
 \end{array}
\] for distinct m and n.
	\flushright \href{https://artofproblemsolving.com/community/c6h464961}{(Link to AoPS)}
\end{problem}



\begin{solution}[by \href{https://artofproblemsolving.com/community/user/124745}{Uzbekistan}]
	Do you have any idea?
\end{solution}



\begin{solution}[by \href{https://artofproblemsolving.com/community/user/106080}{a123}]
	\begin{tcolorbox}Find all functions 
\[
\begin{array}{l}
 f:N \to N \\ 
 2(f(m^2  + n^2 ))^3  = f^2 (m) \cdot f(n) + f^2 (n) \cdot f(m) \\ 
 \end{array}
\] for distinct m and n.\end{tcolorbox}
What  does $f^2 (m)$ mean?Double composition of f or square of $f(m)$?
\end{solution}



\begin{solution}[by \href{https://artofproblemsolving.com/community/user/29428}{pco}]
	\begin{tcolorbox}Find all functions 
\[
\begin{array}{l}
 f:N \to N \\ 
 2(f(m^2  + n^2 ))^3  = f^2 (m) \cdot f(n) + f^2 (n) \cdot f(m) \\ 
 \end{array}
\] for distinct m and n.\end{tcolorbox}
I consider that all exponents mean powers (and not composition). If so :


If $\exists $ prime $p$ such that $p|f(n)$ $\forall n$, then $\frac{f(n)}p$ is a solution too.
So let us from now consider that $\not\exists$ prime $p$ such that $p|f(n)$ $\forall n$

Let $P(m,n)$ be the assertion $2f^3(m^2+n^2)=f(m)f(n)\left(f(m)+f(n)\right)$

Let then odd prime $p$ such that $p|f(n)$. $\exists m\ne n$ such that $p\not|f(m)$ 
Then $P(m,n)$ implies $p|f^3(m^2+n^2)$ and so $p^3$ divides LHS and so $p^3|f(n)$

So if $p|f(n)$, we got that $p^3|f(n)$

But then, when we previously got $p|f^3(m^2+n^2)$, this means $p|f(m^2+n^2)$ and so $p^3|f(m^2+n^2)$ and so $p^9$ divides LHS and so $p^9|f(n)$

So if $p|f(n)$, we got that $p^9|f(n)$

It's easy to continue this method to prove that no odd $p$ can divide $f(n)$

Considering then the divisor $2$, If it exists two natural numbers $m,n$ such that $f(m)=2^u$ and $f(n)=2^v$ with $u<v$, we get that $2f^3(m^2+n^2)=2^{2u+v}(1+2^{v-u})$
But this is impossible since then it exists an odd prime $p$ dividing $1+2^{v-u}$ and so dividing $f(m^2+n^2)$

So $f(x)=1$ $\forall x$

Hence the result : $\boxed{f(x)=c}$ $\forall x$ and for any $c\in\mathbb N$, which indeed is a solution
\end{solution}
*******************************************************************************
-------------------------------------------------------------------------------

\begin{problem}[Posted by \href{https://artofproblemsolving.com/community/user/121558}{Bigwood}]
	Let $f$ be a function $\mathbb{N}$ to itself such that 
\[f(f(f(n)))+6n=7f(n)\]
holds for all $n$. Determine all possible values of $f(2012)$.
	\flushright \href{https://artofproblemsolving.com/community/c6h465544}{(Link to AoPS)}
\end{problem}



\begin{solution}[by \href{https://artofproblemsolving.com/community/user/121558}{Bigwood}]
	Very hard??  :roll:
\end{solution}



\begin{solution}[by \href{https://artofproblemsolving.com/community/user/121558}{Bigwood}]
	[hide="hint(as much as a solution)"]
Prove that It cannot be that $f(2012)=2011$.
[\/hide]
\end{solution}



\begin{solution}[by \href{https://artofproblemsolving.com/community/user/29428}{pco}]
	\begin{tcolorbox}Let $f$ be a function $\mathbb{N}$ to itself such that 
\[f(f(f(n)))+6n=7f(n)\]
holds for all $n$. Determine all possible values of $f(2012)$.\end{tcolorbox}
1) $f(2012)$ can take no integer value $<2012$
================================
$f(f(f(n)))>0$ and so $f(n)>\frac 67 n$
If $f(n)>cn$ for some $c>0$, then $f(f(f(n)))>c^3n$ and $f(n)>\frac{6+c^3}7n$

And so we can build a sequence $c_1=\frac 67$ and $c_{k+1}=\frac{6+c_k^3}7$ such that $f(n)>c_kn$ $\forall n,k$
And since $c_k$ is a convergent sequence whose limit is $1$, we get $f(n)\ge n$
Q.E.D.

2) $f(2012)$ can take any integer value $\ge 2012$
=================================
Let $a\ge 2012$
Let $f(n)$ defined as :

$f(2012+(a-2012)(2^k-1))=2012+(a-2012)(2^{k+1}-1)$ $\forall$ integer $k\ge 0$
$f(n)=n$ for all other positive integer values

It's easy to check that $f(n)$ fits all requirements and that $f(2012)=a$
Q.E.D.
\end{solution}
*******************************************************************************
-------------------------------------------------------------------------------

\begin{problem}[Posted by \href{https://artofproblemsolving.com/community/user/142467}{fun_ction}]
	find all functions $u:R\to R$ for which there exists a strictly monotonic $f:R\to R$ such that $f(xy)=f(x)u(y)+f(y)$
	\flushright \href{https://artofproblemsolving.com/community/c6h466126}{(Link to AoPS)}
\end{problem}



\begin{solution}[by \href{https://artofproblemsolving.com/community/user/29428}{pco}]
	\begin{tcolorbox}find all functions $u:R\to R$ for which there exists a strictly monotonic $f:R\to R$ such that $f(xy)=f(x)u(y)+f(y)$\end{tcolorbox}
Let $P(x,y)$ be the assertion $f(xy)=f(x)u(y)+f(y)$
Let $a$ such that $f(a)\ne 0$ (such $a$ exists since $f(x)$ is not constant).

$P(x,a)$ $\implies$ $f(ax)=f(x)u(a)+f(a)$
$P(a,x)$ $\implies$ $f(ax)=f(a)u(x)+f(x)$
And so $u(x)=\frac{u(a)-1}{f(a)}f(x)+1$ $=bf(x)+1$

Plugging this back in original equation, we get new assertion $Q(x,y)$ : $f(xy)=bf(x)f(y)+f(x)+f(y)$

If $b=0$, $Q(a,0)$ $\implies$ $f(a)=0$, impossible.
So $b\ne 0$ and then $u(x)=bf(x)+1$ is also stricly monotonic and $Q(x,y)$ may be written $u(xy)=u(x)u(y)$

Whose stricly monotonic solutions are classical :
Let $t>0$
$u(0)=0$
$u(x)=x^t$ $\forall x>0$
$u(x)=-(-x)^t$ $\forall x<0$

Which indeed are solutions (choosing for example $f(x)=u(x)-1$)
\end{solution}
*******************************************************************************
-------------------------------------------------------------------------------

\begin{problem}[Posted by \href{https://artofproblemsolving.com/community/user/141397}{subham1729}]
	Find all non-decreasing functions $f: \mathbb{R} \to \mathbb{R}$ such that $f(0)=0$, $f(1)=1$, and $f(a)+f(b)=f(a)f(b)+f(a+b-ab)$ for all $a<1<b$.
	\flushright \href{https://artofproblemsolving.com/community/c6h466326}{(Link to AoPS)}
\end{problem}



\begin{solution}[by \href{https://artofproblemsolving.com/community/user/29428}{pco}]
	\begin{tcolorbox}Find all non-decreasing functions $f: \mathbb{R} \to \mathbb{R}$ such that $f(0)=0$, $f(1)=1$, and $f(a)+f(b)=f(a)f(b)+f(a+b-ab)$ for all $a<1<b$.\end{tcolorbox}
Let $P(x,y)$ be the assertion $f(x)+f(y)=f(x)f(y)+f(x+y-xy)$ true $\forall x<1<y$

Let $g(x)=1-f(1-x)$ and we get $Q(x,y)$ : $g(xy)=g(x)g(y)$ true $\forall x>0>y$
and we have $g(0)=0$ and $g(1)=1$ anf $g(x)$ non-decreasing.
Since non decreasing, $g(-1)\le 0$

If $g(-1)=0$, then $Q(x,-1)$ $\implies$ $g(-x)=0$ $\forall x>0$ and we indeed got a solution, whatever is $g(x)$ for $x>0$ (if non decreasing and such that $g(1)=1$

If $g(-1)=a<0$ then $Q(x,-1)$ $\implies$ $g(-x)=ag(x)$ $\forall x>0$

Let $x,y>0$ : $Q(x,-y)$ $\implies$ $g(xy)=g(x)g(y)$ true $\forall x,y>0$
And so, since non decreasing : $g(x)=x^t$ $\forall x>0$ for some $t\ge 0$ and we indeed got a solution

Synthesis of solutions for $g(x)$
======================
1) 
Let $h(x)$ any non decreasing function from $\mathbb R_{\ge 0}\to\mathbb R_{\ge 0}$ such that $h(0)=0$ and $h(1)=1$. Then :
$g(x)=0$ $\forall x< 0$ and $g(x)=h(x)$ $\forall x\ge 0$ 

2) Let $a>0$ and $t\ge 0$, then :
$g(x)=x^t$ $\forall x>0$
$g(0)=0$
$g(x)=-a(-x)^t$ $\forall x<0$


Synthesis of solutions for $f(x)$
======================
1) 
Let $h(x)$ any non decreasing function from $(-\infty,1]\to(-\infty,1]$ such that $h(0)=0$ and $h(1)=1$. Then :
$f(x)=1$ $\forall x>1$ and $f(x)=h(x)$ $\forall x\le 1$ 

2) Let $a>0$ and $t\ge 0$, then :
$f(x)=1-(1-x)^t$ $\forall x<1$
$f(1)=1$
$f(x)=1+a(x-1)^t$ $\forall x>1$
\end{solution}
*******************************************************************************
-------------------------------------------------------------------------------

\begin{problem}[Posted by \href{https://artofproblemsolving.com/community/user/68712}{Layman conjecture}]
	2. Find all functions f : R to R such that

f(x + f(y)) = f(x + y) + f(y)
	\flushright \href{https://artofproblemsolving.com/community/c6h466398}{(Link to AoPS)}
\end{problem}



\begin{solution}[by \href{https://artofproblemsolving.com/community/user/68712}{Layman conjecture}]
	i must be missing something straightforward. 

i have proved the following:

f(0) = 0;

i know f = 0 and f(x) = 2x are solutions. I have proved this, but my proof is not complete. I still have to show the following:

if 'a' exists such that f(a) = 0 and a is not equal to 0, then :  f = 0 (It is constant and equal to 0) ; If i show this, I am done. 

All I need (and want) is a hint.
\end{solution}



\begin{solution}[by \href{https://artofproblemsolving.com/community/user/68712}{Layman conjecture}]
	never mind. i figured it out :) .
\end{solution}



\begin{solution}[by \href{https://artofproblemsolving.com/community/user/108692}{MariusBocanu}]
	Are you sure you didn't mean this http://www.artofproblemsolving.com/Forum/viewtopic.php?p=1165901&sid=abaf6f3f47a4dd0f23e03437ac464ad8#p1165901 ? If not, please post your solution.
\end{solution}



\begin{solution}[by \href{https://artofproblemsolving.com/community/user/29428}{pco}]
	\begin{tcolorbox}2. Find all functions f : R to R such that
f(x + f(y)) = f(x + y) + f(y)\end{tcolorbox}
1) General solution :
==============
1.1) Short result : $f(x)=x+g(x)$ where $g(x)$ is any involutive solution of additive Cauchy equation.

1.2) long result :
Let $A,B$ two supplementary vector subspaces of the $\mathbb Q-$vectorspace $\mathbb R$
Let $a(x)$ from $\mathbb R\to A$ and $b(x)$ from $\mathbb R\to B$ the two projections of $x$ in $A$ and $B$ (so that $x=a(x)+b(x)$ in a unique way)
Then $f(x)=2a(x)$

2) proof that these indeed are solutions
==========================
1.1 is a solution:
$f(x)$ is additive
$f(x+f(y))=x+f(y)+g(x+f(y))$ $=x+f(y)+g(x)+g(f(y))$ $=x+f(y)+g(x)+g(y)+g(g(y))$ $=x+f(y)+g(x)+g(y)+y$ $=f(x+y)+f(y)$
Q.E.D.

1.2 is a solution
$f(x)$ is additive
From $f(x)=2a(x)\in A$, we get $a(f(x))=f(x)=2a(x)$ and so $f(f(x))=2f(x)$
So $f(x+f(y))=f(x)+f(f(y))=f(x)+2f(y)=f(x+y)+f(y)$
Q.E.D.

3) proof that any solution may be written in these forms (and so that 1) is a general solution)
==============================================================
Let $f(x)$ any function such that assertion $P(x,y)$ : $f(x+f(y))=f(x+y)+f(y)$ is true
Let $a=f(0)$

$P(-a,0)$ $\implies$ $f(-a)=0$
$P(0,-a)$ $\implies$ $a=0$

Let then $g(x)=f(x)-x$
$g(0)=0$ and $P(x,y)$ becomes new assertion $Q(x,y)$ : $g(x+g(y))=g(x)+y$
$Q(0,x)$ $\implies$ $g(g(x))=x$
$Q(x,g(y))$ $\implies$ $g(x+y)=g(x)+g(y)$
So $g(x)$ is an involutive solution of additive Cauchy equation

Let $A=\{x$ such that $g(x)=x\}$. Since $g(x)$ is additive and $g(0)=0$, $A$ is a $\mathbb Q$ vectorspace
Let $B=\{x$ such that $g(x)=-x\}$. Since $g(x)$ is additive and $g(0)=0$, $A$ is a $\mathbb Q$ vectorspace
$A\cap B=\{0\}$

For any $x\in\mathbb R$ let $a(x)=\frac{x+g(x)}2$ and $b(x)=\frac{x-g(x)}2$

$a(x)$ is such that $g(a(x))=a(x)$ and so $a(x)\in A$

$b(x)$ is such that $g(b(x))=-b(x)$ and so $b(x)\in B$

And, since $x=a(x)+b(x)$, we get that $A,B$ are two supplementary vector subspaces of the $\mathbb Q-$vectorspace $\mathbb R$

And $g(x)=a(x)-b(x)$ and so $f(x)=x+a(x)-b(x)=2a(x)$
Q.E.D.
\end{solution}
*******************************************************************************
-------------------------------------------------------------------------------

\begin{problem}[Posted by \href{https://artofproblemsolving.com/community/user/139527}{lzh}]
	Find all function $f$:$Z$ to $Z$ such that:
$f(f(f(m))-f(f(n)))=m-n$ for all integers $m,n$
	\flushright \href{https://artofproblemsolving.com/community/c6h466988}{(Link to AoPS)}
\end{problem}



\begin{solution}[by \href{https://artofproblemsolving.com/community/user/29428}{pco}]
	\begin{tcolorbox}Find all function $f$:$Z$ to $Z$ such that:
$f(f(f(m))-f(f(n)))=m-n$ for all integers $m,n$\end{tcolorbox}
Let $P(x,y)$ be the assertion $f(f(f(x))-f(f(y)))=x-y$

$P(0,0)$ $\implies$ $f(0)=0$
$P(n,0)$ $\implies$ $f(f(f(n)))=n$
$P(f(m),f(n))$ $\implies$ $f(m-n)=f(m)-f(n)$,and so $f(n)=f(1)n$

Plugging back in original equation, we get $(f(1))^3=1$

And so the unique solution $\boxed{f(x)=x}$
\end{solution}



\begin{solution}[by \href{https://artofproblemsolving.com/community/user/142365}{Tima95}]
	\begin{tcolorbox}Find all function $f$:$Z$ to $Z$ such that:
$f(f(f(m))-f(f(n)))=m-n$ for all integers $m,n$\end{tcolorbox}
 We prove that the $ f(m)=f(k) <=> m=k $  We have that $ m-n=f(f(f(m)))-f(f(n))=f(f(f(k)))-f(f(n))=k-n <=> m=k $ 
 Substituting $ m=n=0 $ we have that $ f(f(f(0)))=f(f(0)) <=> f(f(0))=f(0) <=> f(0)=0 $. Substituting $ n=0 $ we have that $ f(f(f(m)))=m $. Substituting m=0 we have that $ f(f(n))=n $ . Then $ f(f(f(n)))=f(f(n)) <=> f(n)=n $
\end{solution}
*******************************************************************************
-------------------------------------------------------------------------------

\begin{problem}[Posted by \href{https://artofproblemsolving.com/community/user/88852}{myceliumful}]
	Find all functions $ f:\mathbb{C}\setminus \left \{ 0,1 \right \}\rightarrow \mathbb{C} $ such that\[f(z)+2f(\frac{1}{z})+3f(\frac{z}{z-1})=z\]
	\flushright \href{https://artofproblemsolving.com/community/c6h467160}{(Link to AoPS)}
\end{problem}



\begin{solution}[by \href{https://artofproblemsolving.com/community/user/29428}{pco}]
	\begin{tcolorbox}Find all functions $ f:\mathbb{C}\setminus \left \{ 0,1 \right \}\rightarrow \mathbb{C} $ such that\[f(z)+2f(\frac{1}{z})+3f(\frac{z}{z-1})=z\]\end{tcolorbox}
Let $A=\mathbb C\setminus\{0,1\}$
Let $u$ be the function from $A\to A$ defined as $u(z)=\frac 1z$
Let $v$ be the function from $A\to A$ defines as $v(z)=1-z$
Let $i$ be the function from $A\to A$ defines as $i(z)=z$
Let $P(z)$ be the assertion $f+2f\circ u+3f\circ u\circ v\circ u\equiv i$

Notice that $u\circ u=v\circ v=i$, that $u\circ v\circ u=v\circ u\circ v$

$P(z)$ $\implies$ $f+2f\circ u+3f\circ u\circ v\circ u\equiv i$
$P(u(z))$ $\implies$ $f\circ u+2f+3f\circ u\circ v\equiv u$
$P(v(z))$ $\implies$ $f\circ v+2f\circ u\circ v+3f\circ v\circ u\equiv v$
$P(u(v(z)))$ $\implies$ $f\circ u\circ v+2f\circ v+3f\circ u\equiv u\circ v$
$P(v(u(z)))$ $\implies$ $f\circ v\circ u+2f\circ u\circ v\circ u+3f\circ v\equiv  v\circ u$
$P(u(v(u(z)))$ $\implies$ $f\circ u\circ v\circ u+2f\circ v\circ u+3f\equiv u\circ v\circ u$

So we have a linear system of 6 equations with 6 variables $f,f\circ u,f\circ v,$ $f\circ u\circ v,$ $f\circ v\circ u,f\circ u\circ v\circ u$

And we can easily extract $f$ from this system and we get 

$f\equiv \frac{-3i+3u-5v+u\circ v+v\circ u+7u\circ v\circ u}{24}$

And so $\boxed{f(z)=\frac{2z^3+z^2+5z-2}{24z(z-1)}}$ which indeed is a solution
\end{solution}
*******************************************************************************
-------------------------------------------------------------------------------

\begin{problem}[Posted by \href{https://artofproblemsolving.com/community/user/88852}{myceliumful}]
	Let $ f:\mathbb{R}\rightarrow \mathbb{R} $ be such that \[f(x+y)+f(x-y)=2f(x)\cosh y\]
for all $ x,y\in \mathbb{R} $.
Find all functions $ f $.
	\flushright \href{https://artofproblemsolving.com/community/c6h467179}{(Link to AoPS)}
\end{problem}



\begin{solution}[by \href{https://artofproblemsolving.com/community/user/29428}{pco}]
	\begin{tcolorbox}Let $ f:\mathbb{R}\rightarrow \mathbb{R} $ be such that \[f(x+y)+f(x-y)=2f(x)\cosh y\]
for all $ x,y\in \mathbb{R} $.
Find all functions $ f $.\end{tcolorbox}
Let $P(x,y)$ be the assertion $f(x+y)+f(x-y)=2f(x)\cosh y$
Let $f(0)=a$

a: $P(x,y)$ $\implies$ $f(x+y)+f(x-y)=2f(x)\cosh y$
b: $P(y,x)$ $\implies$ $f(x+y)+f(y-x)=2f(y)\cosh x$
c: $P(0,x-y)$ $\implies$ $f(x-y)+f(y-x)=2a\cosh (x-y)$
a+b-c : new assertion $Q(x,y)$ : $f(x+y)=f(x)\cosh y+f(y)\cosh x-a\cosh(x-y)$

$Q(x,y+z)$ $\implies$ $f(x+y+z)=f(x)\cosh (y+z)+f(y+z)\cosh x-a\cosh(x-y-z)$

$\implies$ $f(x+y+z)$ $=f(x)\cosh (y+z)+$ $(f(y)\cosh z+f(z)\cosh y-a\cosh(y-z))\cosh x-a\cosh(x-y-z)$

$\implies$ $f(x+y+z)$ $=f(x)\cosh y\cosh z+f(y)\cosh x\cosh z+f(z)\cosh x\cosh y$ $+f(x)\sinh y\sinh z$ $-2a\cosh x\cosh y\cosh z +a\sinh x\sinh y\cosh z + a\sinh x\cosh y\sinh z$

Switching $x$ and $y$ in this last equation, we get :
$f(x+y+z)=f(y)\cosh x\cosh z+f(x)\cosh y\cosh z+f(z)\cosh x\cosh y$ $+f(y)\sinh x\sinh z$ $-2a\cosh x\cosh y\cosh z +a\sinh y\sinh x\cosh z + a\sinh y\cosh x\sinh z$

Subtracting, we get :
$f(x)\sinh y\sinh z+ a\sinh x\cosh y\sinh z$ $=f(y)\sinh x\sinh z+ a\sinh y\cosh x\sinh z$

Choosing $y,z\ne 0$ and dividing by $\sinh y\sinh z$, this becomes $f(x)=\frac{f(y)-a\cosh y}{\sinh y}\sinh x+ a\cosh x$

So $\boxed{f(x)=a\cosh x + b\sinh x}$, which indeed is a solution (which could also be written $\alpha e^x+\beta e^{-x}$)
\end{solution}
*******************************************************************************
-------------------------------------------------------------------------------

\begin{problem}[Posted by \href{https://artofproblemsolving.com/community/user/109774}{littletush}]
	Consider a function $f:\mathbb N\rightarrow \mathbb N$ such that for any two positive integers $x,y$, the equation $f(xf(y))=yf(x)$ holds. Find the smallest possible value of $f(2007)$.
	\flushright \href{https://artofproblemsolving.com/community/c6h467460}{(Link to AoPS)}
\end{problem}



\begin{solution}[by \href{https://artofproblemsolving.com/community/user/29428}{pco}]
	\begin{tcolorbox}Co0nsider a function $f:N\rightarrow N$ such that for any positive integers $x,y$,$f(xf(y))=yf(x)$.Find the smallest possible value of $f(2007)$.\end{tcolorbox}
Let $P(x,y)$ be the assertion $f(xf(y))=yf(x)$

Subtracting $P(f(x),1)$ from $P(f(1),x)$, we get $f(f(x))=f(f(1))x$ and so $f(x)$ is a bijection
$P(x,1)$ $\implies$ $f(xf(1))=f(x)$ and so, since bijective, $f(1)=1$ and so $f(f(x))=x$
$P(x,f(y))$ $\implies$ $f(xy)=f(x)f(y)$

So, it's immediate to show that the functional equation is equivalent to $f(xy)=f(x)f(y)$ and $f(f(x))=x$

This implies that $f(p)$ is prime for any prime $p$ and that $f(x)$ is a bijection from the primes into the primes and is fully defined over $\mathbb N$ by its values in the primes.

Since $2007=3^2223$, we get $f(2007)=f(3)^2f(223)$
If $f(3)=3$ and $f(223)=2$, then $f(2007)=18$
If $f(3)=2$, then $f(223)\ne 3$ and so $f(223)\ge 5$ and $f(2007)\ge 20>18$
If $f(3)\ge 5$, then $f(2007)\ge 25>18$

Hence the answer $\boxed{18}$
\end{solution}



\begin{solution}[by \href{https://artofproblemsolving.com/community/user/110552}{youarebad}]
	\begin{tcolorbox}This implies that $f(p)$ is prime for any prime $p$\end{tcolorbox}

How do you get this ?? I don't understand..  :blush:
\end{solution}



\begin{solution}[by \href{https://artofproblemsolving.com/community/user/64716}{mavropnevma}]
	So \begin{bolded}pco\end{bolded} proved that the functional equation is equivalent to $f(xy)=f(x)f(y)$ and $f(f(x))=x$; from the last follows that $f$ is bijective. He also proved, as a consequence, that $f(1)=1$.

This implies that $f(p)$ is prime for any prime $p$; assume it is not a prime, so $f(p) = ab$ with $a,b > 1$, then $p = f(f(p)) = f(ab) = f(a)f(b)$. Since $a,b \neq 1$ imply $f(a),f(b) \neq 1$, it contradicts $p$ being a prime.
\end{solution}



\begin{solution}[by \href{https://artofproblemsolving.com/community/user/274173}{Anar24}]
	\begin{tcolorbox}[quote="littletush"]Co0nsider a function $f:N\rightarrow N$ such that for any positive integers $x,y$,$f(xf(y))=yf(x)$.Find the smallest possible value of $f(2007)$.\end{tcolorbox}
Let $P(x,y)$ be the assertion $f(xf(y))=yf(x)$

Subtracting $P(f(x),1)$ from $P(f(1),x)$, we get $f(f(x))=f(f(1))x$ and so $f(x)$ is a bijection
$P(x,1)$ $\implies$ $f(xf(1))=f(x)$ and so, since bijective, $f(1)=1$ and so $f(f(x))=x$
$P(x,f(y))$ $\implies$ $f(xy)=f(x)f(y)$

So, it's immediate to show that the functional equation is equivalent to $f(xy)=f(x)f(y)$ and $f(f(x))=x$

This implies that $f(p)$ is prime for any prime $p$ and that $f(x)$ is a bijection from the primes into the primes and is fully defined over $\mathbb N$ by its values in the primes.

Since $2007=3^2223$, we get $f(2007)=f(3)^2f(223)$
If $f(3)=3$ and $f(223)=2$, then $f(2007)=18$
If $f(3)=2$, then $f(223)\ne 3$ and so $f(223)\ge 5$ and $f(2007)\ge 20>18$
If $f(3)\ge 5$, then $f(2007)\ge 25>18$

Hence the answer $\boxed{18}$\end{tcolorbox}

I have one question can't you solve the equation f (x)×f (y)=f (xy) in natural numbers 
\end{solution}



\begin{solution}[by \href{https://artofproblemsolving.com/community/user/274173}{Anar24}]
	You could find f(x)=x\begin{tcolorbox}[quote="littletush"]Co0nsider a function $f:N\rightarrow N$ such that for any positive integers $x,y$,$f(xf(y))=yf(x)$.Find the smallest possible value of $f(2007)$.\end{tcolorbox}
Let $P(x,y)$ be the assertion $f(xf(y))=yf(x)$

Subtracting $P(f(x),1)$ from $P(f(1),x)$, we get $f(f(x))=f(f(1))x$ and so $f(x)$ is a bijection
$P(x,1)$ $\implies$ $f(xf(1))=f(x)$ and so, since bijective, $f(1)=1$ and so $f(f(x))=x$
$P(x,f(y))$ $\implies$ $f(xy)=f(x)f(y)$

So, it's immediate to show that the functional equation is equivalent to $f(xy)=f(x)f(y)$ and $f(f(x))=x$

This implies that $f(p)$ is prime for any prime $p$ and that $f(x)$ is a bijection from the primes into the primes and is fully defined over $\mathbb N$ by its values in the primes.

Since $2007=3^2223$, we get $f(2007)=f(3)^2f(223)$
If $f(3)=3$ and $f(223)=2$, then $f(2007)=18$
If $f(3)=2$, then $f(223)\ne 3$ and so $f(223)\ge 5$ and $f(2007)\ge 20>18$
If $f(3)\ge 5$, then $f(2007)\ge 25>18$

Hence the answer $\boxed{18}$\end{tcolorbox}

4
\end{solution}



\begin{solution}[by \href{https://artofproblemsolving.com/community/user/29428}{pco}]
	\begin{tcolorbox}You could find f(x)=x
[... skipped useless full quoting of my previous post...]
4\end{tcolorbox}

What is the meaning of your post ???????
$f(x)=x$ is trivially a solution of functional equation but with this solution we have $f(2007)=2007$ which obviously is not the smallest value possible.

And what is the meaning of the isolated $4$ at the bottom of your post ?

\end{solution}



\begin{solution}[by \href{https://artofproblemsolving.com/community/user/274173}{Anar24}]
	\end{underlined}\end{underlined}\begin{tcolorbox}[quote=Anar24]You could find f(x)=x
[... skipped useless full quoting of my previous post...]
4\end{tcolorbox}

What is the meaning of your post ???????
$f(x)=x$ is trivially a solution of functional equation but with this solution we have $f(2007)=2007$ which obviously is not the smallest value possible.

And what is the meaning of the isolated $4$ at the bottom of your post ?\end{tcolorbox}
i am so sorry because i wrote 4 wrong.My question was that we can easily find that f(2007)=2007 so why question asks the smallest value of f(2007)????????
P.S.Apologize me for writing 4 Sorry :( \begin{bolded}\end{bolded}\end{underlined}

\end{solution}



\begin{solution}[by \href{https://artofproblemsolving.com/community/user/188154}{mishka1980}]
	If 0 is considered a natural number, we could potentially have $f(n)=0$.
\end{solution}



\begin{solution}[by \href{https://artofproblemsolving.com/community/user/88871}{GreenKeeper}]
	\begin{tcolorbox}If 0 is considered a natural number, we could potentially have $f(n)=0$.\end{tcolorbox}

This made me really laugh. I was one of the coordinators in this contest, and I remember that one contestant actually wrote this as a solution, and after the contest he argued that it's correct because he had some very old book that defined $\mathbb{N}$ as containing $0$.

One of the main organizers (who is quite a character) went completely ballistic, it was hilarious.  :rotfl:

Since then they don't use the term natural number in the problem statements in czech math olympiad, instead they always say positive integer.
\end{solution}



\begin{solution}[by \href{https://artofproblemsolving.com/community/user/313744}{MateoCV}]
	Don't you have to give a function that works and $f(2007)=18$? A function that works is $f(2^a223^bk)=2^b223^ak$ with $k$ odd and $223\not | k$
\end{solution}



\begin{solution}[by \href{https://artofproblemsolving.com/community/user/29428}{pco}]
	\begin{tcolorbox}My question was that we can easily find that f(2007)=2007 so why question asks the smallest value of f(2007)????????\end{tcolorbox}
There are infinitely multiplicative functions over the set of positive integers such that $f(f(x))=x$.
Just read my post.
Choose one such that $f(2)=223$ and $f(223)=2$ and you got the result $f(2007)=18$



\end{solution}
*******************************************************************************
-------------------------------------------------------------------------------

\begin{problem}[Posted by \href{https://artofproblemsolving.com/community/user/109774}{littletush}]
	Find all functions $f:\mathbb R^+ \rightarrow \mathbb R^+$ such that for all positive real numbers $x,y$,
\[x^2[f(x)+f(y)]=(x+y)f(yf(x)).\]
	\flushright \href{https://artofproblemsolving.com/community/c6h467480}{(Link to AoPS)}
\end{problem}



\begin{solution}[by \href{https://artofproblemsolving.com/community/user/29428}{pco}]
	\begin{tcolorbox}Find all functions $f:R^+ \rightarrow R^+$ such that for all positive real numbers $x,y$,
$x^2[f(x)+f(y)]=(x+y)f(yf(x))$.\end{tcolorbox}
Let $P(x,y)$ be the assertion $x^2(f(x)+f(y))=(x+y)f(yf(x))$

$P(x,x)$ $\implies$ $f(xf(x))=xf(x)$ and so $f(f(1))=f(1)$

$P(xf(x),f(1))$ $\implies$ $x^2f(x)^2=f(xf(x)f(1))$
$P(f(1),xf(x))$ $\implies$ $f(1)^2=f(xf(x)f(1))$

And so $f(x)=\frac{f(1)}x$ and, plugging this in original equation, we get $f(1)=1$

Hence the answer : $\boxed{f(x)=\frac 1x}$
\end{solution}
*******************************************************************************
-------------------------------------------------------------------------------

\begin{problem}[Posted by \href{https://artofproblemsolving.com/community/user/142949}{cuti20091996}]
	find all functions f: R$\rightarrow$R :
[size=150]f(f(x)-y)=f(x)+f(f(y)-f(-x))+x[\/size]
	\flushright \href{https://artofproblemsolving.com/community/c6h467658}{(Link to AoPS)}
\end{problem}



\begin{solution}[by \href{https://artofproblemsolving.com/community/user/29428}{pco}]
	\begin{tcolorbox}find all functions f: R$\rightarrow$R :
[size=150]f(f(x)-y)=f(x)+f(f(y)-f(-x))+x[\/size]\end{tcolorbox}
Let $P(x,y)$ be the assertion $f(f(x)-y)=f(x)+f(f(y)-f(-x))+x$

If $\exists u$ such that $f(u)=0$, then $P(u,-u)$ $\implies$ $u=-f(0)$

$P(0,f(0))$ $\implies$ $f(f(f(0))-f(0))=0$ $\implies$ $f(f(0))-f(0)=-f(0)$ $\implies$ $f(f(0))=0$ $\implies$ $f(0)=-f(0)$ $\implies$ $f(0)=0$

$P(0,x)$ $\implies$ $f(-x)=f(f(x))$ and $P(x,y)$ may be written $f(f(x)-y)=f(x)+f(f(y)-f(f(x)))+x$ and so $f(x)$ is injective

$P(0,x)$ $\implies$ $f(-x)=f(f(x))$ and so, since injective, $\boxed{f(x)=-x}$ which indeed is as solution
\end{solution}
*******************************************************************************
-------------------------------------------------------------------------------

\begin{problem}[Posted by \href{https://artofproblemsolving.com/community/user/128206}{eddy13579}]
	Determine all $f:\mathbb{Q}\to\mathbb{Q}$ such that:$f(x+y+f(x))=x+f(x)+f(y)$
	\flushright \href{https://artofproblemsolving.com/community/c6h467880}{(Link to AoPS)}
\end{problem}



\begin{solution}[by \href{https://artofproblemsolving.com/community/user/29428}{pco}]
	\begin{tcolorbox}Determine all $f:\mathbb{Q}\to\mathbb{Q}$ such that:$f(x+y+f(x))=x+f(x)+f(y)$\end{tcolorbox}
1) General solution :
===============

Let $A$ any additive subgroup of $\mathbb Q$
Let $\sim$ be the equivalence relation $x\sim y$ $\iff$ $x-y\in A$
Let $r(x)$ a function which associates to any rational a representant (unique per class) of its equivalence class
Let $h(x)$ any function from $\mathbb Q\to A$

Then $f(x)=h(r(x))+x-2r(x)$

2) proof that these functions indeed are solutions
===================================
$h(r(x))$ and $x-r(x)\in A$ and so $f(x)+x\in A$ (since $A$ is a group)
So $x+y+f(x)\sim y$ and so $r(x+y+f(x))=r(y)$
So $f(x+y+f(x))=h(r(x+y+f(x)))+x+y+f(x)-2r(x+y+f(x))$ $=h(r(y))+x+y+f(x)-2r(y)$ $=x+f(x)+f(y)$
Q.E.D.

3) proof that any solution may be written in this form (and so that it is indeed a general solution)
====================================================================
Let $f(x)$ a solution of the functional equation
Let $A=\{y\in\mathbb Q$ such that $f(x+y)=f(x)+y$ $\forall x\in\mathbb Q\}$

$0\in A$
Obviously $y\in A$ $\implies$ $-y\in A$
Obviously $y,z\in A$ $\implies$ $y+z\in A$
So $A$ is an additive subgroup of $\mathbb Q$

Let $\sim$ be the equivalence relation $x\sim y$ $\iff$ $x-y\in A$
Let $r(x)$ a function which associates to any rational a representant (unique per class) of its equivalence class

Since $f(x+y+f(x))=x+f(x)+f(y)$ $\forall x,y$, we get that $f(x)+x\in A$ $\forall x$

Then $x=r(x)+x-r(x)$ and $x-r(x)\in A$ implie $f(x)=f(r(x)+x-r(x))$ $=f(r(x))+x-r(x)$ $=f(r(x))+r(x)+x-2r(x)$
Defining then $h(x)=f(x)+x$ as a function from $\mathbb Q\to A$, we get $f(x)=h(r(x))+x-2r(x)$
Q.E.D.

4) An application : some examples of solutions
=================================

4.1) $A=\{0\}$
So $r(x)=x$ and $h(x)=0$ and the solution $\boxed{f(x)=-x}$

4.2) $A=\mathbb Q$
So $r(x)=c$ constant and the solution $\boxed{f(x)=x+d}$

3.3) $A=\mathbb Z$
We can choose then $r(x)=x-\lfloor x\rfloor$ and the solution $f(x)=\lfloor h(x-\lfloor x\rfloor)\rfloor -x+2\lfloor x\rfloor$, whatever is $h(x)$. Examples :

$\boxed{f(x)=\lfloor 150(x-\lfloor x\rfloor)\rfloor -x+2\lfloor x\rfloor}$

$\boxed{f(x)=\lfloor 3\sin(x-\lfloor x\rfloor)\rfloor -x+2\lfloor x\rfloor}$

And infinitely many other solutions changing $A$ and $h(x)$

And I wonder how such a problem could have been given in an olympiad contest or training session.
\end{solution}
*******************************************************************************
-------------------------------------------------------------------------------

\begin{problem}[Posted by \href{https://artofproblemsolving.com/community/user/128206}{eddy13579}]
	Determine all surjective functions $f:\mathbb{R}\to[1,\infty)$ which verify:
$f(x+f(y))=f(x)+2xf(y)+f^2(y)$
	\flushright \href{https://artofproblemsolving.com/community/c6h468004}{(Link to AoPS)}
\end{problem}



\begin{solution}[by \href{https://artofproblemsolving.com/community/user/29428}{pco}]
	\begin{tcolorbox}Determine all surjective functions $f:\mathbb{R}\to[1,\infty)$ which verify:
$f(x+f(y))=f(x)+2xf(y)+f^2(y)$\end{tcolorbox}
Let $P(x,y)$ be the assertion $f(x+f(y))=f(x)+2xf(y)+f^2(y)$
Let $a=f(0)$

$P(0,x)$ $\implies$ $f(f(x))=f^2(x)+a$ and so, since surjective : $f(x)=x^2+a$ $\forall x\ge 1$

Let then $x\in\mathbb R$ and $u\ge\max(1,1-x)$ and $v$ such that $f(v)=u$ (which exists since $u\ge 1$ and $f(x)$ is surjective).

$P(x,v)$ $\implies$ $f(x+u)=f(x)+2xu+u^2$
But $x+u\ge 1$ $\implies$ $f(x+u)=(x+u)^2+a$ and so $(x+u)^2+a=f(x)+2xu+u^2$

So $f(x)=x^2+a$ $\forall x$ and so $f(\mathbb R)=[a,+\infty)$
And since $f(R)=[1,+\infty)$, we get $a=1$

Hence the answer : $\boxed{f(x)=x^2+1}$ $\forall x$ which indeed is a solution
\end{solution}
*******************************************************************************
-------------------------------------------------------------------------------

\begin{problem}[Posted by \href{https://artofproblemsolving.com/community/user/132519}{siavosh}]
	find all injective and surjective function $ f:\mathbb{R}^{+}\rightarrow\mathbb{R}^{+} $ such that for each $ x,y\in\mathbb{R}^{+} $ 
                                          $\frac{1}{2}xf(x)+\frac{1}{2}yf^{-1}(y)\geq xy$
	\flushright \href{https://artofproblemsolving.com/community/c6h468410}{(Link to AoPS)}
\end{problem}



\begin{solution}[by \href{https://artofproblemsolving.com/community/user/29428}{pco}]
	\begin{tcolorbox}find all injective and surjective function $ f:\mathbb{R}^{+}\rightarrow\mathbb{R}^{+} $ such that for each $ x,y\in\mathbb{R}^{+} $ 
                                          $\frac{1}{2}xf(x)+\frac{1}{2}yf^{-1}(y)\geq xy$\end{tcolorbox}
Let $P(x,y)$ be the assertion $\frac 12xf(x)+\frac 12yf^{[-1]}(y)\ge xy$

$P(x,f(y))$ $\implies$ $f(x)-f(y)\ge (x-y)\frac{f(y)}x$ and $P(y,f(x))$ $\implies$ $f(x)-f(y)\le (x-y)\frac{f(x)}y$

So $(x-y)\frac{f(x)}y$ $\ge f(x)-f(y)\ge$ $(x-y)\frac{f(y)}x$

This implies that $f(x)$ is continuous, differentiable and that $\frac{f'(x)}{f(x)}=\frac 1x$

Hence the answer : $\boxed{f(x)=ax}$  for any $a>0$ which indeed are solutions
\end{solution}
*******************************************************************************
-------------------------------------------------------------------------------

\begin{problem}[Posted by \href{https://artofproblemsolving.com/community/user/125018}{horizon}]
	Find all continuous $f$:$R$ to $R$ such that for any reals $x,y$, we have
$f(x+yf(x))=f(x)f(y)$
	\flushright \href{https://artofproblemsolving.com/community/c6h468444}{(Link to AoPS)}
\end{problem}



\begin{solution}[by \href{https://artofproblemsolving.com/community/user/124745}{Uzbekistan}]
	I have similar some idea
$f(x)f(y)=2f(x+yf(x))$
solution:

Let $P(x,y)$ be the assertion $f(x)f(y)=2f(x+yf(x))$

Let $u,v>0$.
Let $a\in(0,u)$

Let $x=a>0$ and $y=\frac{u-a}{f(a)}>0$ and $z=\frac{2v}{f(x)f(y)}>0$

$f(x)f(y)=2f(x+yf(x))=2f(u)$ and so $f(x)f(y)f(z)=2f(u)f(z)=4f(u+zf(u))=4f(u+v)$

$f(y)f(z)=2f(y+zf(y))$ and so $f(x)f(y)f(z)=2f(x)f(y+zf(y))$ $=4f(x+(y+zf(y))f(x))$ $=4f(x+yf(x)+zf(x)f(y))$ $=4f(u+2v)$

And so $f(u+v)=f(u+2v)$ $\forall u,v>0$ and so $f(x)=f(y)$ $\forall x,y$ such that $2x>y>x>0$

And it's immediate from there to conclude $f(x)=f(y)$ $\forall x,y>0$

Hence the unique solution $\boxed{f(x)=2\forall x>0}$
\end{solution}



\begin{solution}[by \href{https://artofproblemsolving.com/community/user/29428}{pco}]
	\begin{tcolorbox}I have similar some idea
$f(x)f(y)=2f(x+yf(x))$\end{tcolorbox}

There is no $2$ in the original problem.
\end{solution}



\begin{solution}[by \href{https://artofproblemsolving.com/community/user/29428}{pco}]
	\begin{tcolorbox}Find all continuous $f$:$R$ to $R$ such that for any reals $x,y$, we have
$f(x+yf(x))=f(x)f(y)$\end{tcolorbox}
The only constant solutions are $f(x)=0$ $\forall x$ and $f(x)=1$ $\forall x$
Let us from now look only for non constant solutions.

Let $u$ such that $f(u)=v\notin\{0,1\}$. Such $u$ exists since $f(x)$ is continuous and non constant.
Let $P(x,y)$ be the assertion $f(x+yf(x))=f(x)f(y)$
Let $A=f(\mathbb R)$

$P(u,0)$ $\implies$ $v=vf(0)$ and so $f(0)=1$

1) $[0,+\infty)\in A$
============
1.1 $0\in A$ :
$P(u,\frac u{1-v})$ $\implies$ $f(\frac u{1-v})=0$ 
Q.E.D.

1.2. $a,b\in A$ $\implies$ $ab\in A$
This is an immediate consequence of $P(x,y)$ using $x$ such that $f(x)=a$ and $y$ such that $f(y)=b$
Q.E.D.

1.3. If $a,b\in A$ and $b\ne 0$, then $\frac ab\in A$
If $f(x)\ne 0$, then $P(x,\frac{y-x}{f(x)})$ $\implies$ $f(y)=f(x)f(\frac{y-x}{f(x)})$ and so $\frac{f(y)}{f(x)}=(\frac{y-x}{f(x)})$
Q.E.D.

1.4 $[0,+\infty)\in A$
$v\in A$ and so $v^2\in A$ (using 1.2)
So $[0,v^2]\in A$ (using continuity)
So $\frac {v^2}w\in A$ for any $w\in(0,v^2]$ (using 1.3)
setting $w\to 0+$, we get that we can find positive numberes as great as we want in $A$
And continuity ends the proof.
Q.E.D.

2) If $f(a)=f(b)=t\ne 0$ then $a=b$
=======================
$P(a,\frac{x-a}t)$ $\implies$ $f(x)=tf(\frac{x-a}t)$

$P(b,\frac{x-a}t)$ $\implies$ $f(x+b-a)=tf(\frac{x-a}t)$

And so $f(x+b-a)=f(x)$ and, if $b\ne a$, $f(x)$ is periodic, and so bounded, since continuous and so contradiction with 1) above
So $a=b$
Q.E.D.

3) non constant solutions :
==================
Let $x$ such that $f(x)\ne 0$
$P(x,u)$ $\implies$ $f(x+uf(x))=f(x)v$
$P(u,x)$ $\implies$ $f(u+xv))=f(x)v$

And so $f(x+uf(x))=f(u+xv)\ne 0$ and so (using 2) above) : $x+uf(x)=u+xv$ and so $f(x)=1+x\frac{v-1}u$

And so $f(x)=0$ or $f(x)=cx+1$ for any $x$ where $c=\frac{v-1}u$ is a constant $\ne 0$

So only three possible continuous non constant solutions :

$f(x)=1+cx$ $\forall x$ which indeed is a solution
$f(x)=1+cx$ $\forall x\ge -\frac 1c$ and $f(x)=0$ $\forall x\le -\frac 1c$ which indeed is a solution iff $c>0$
$f(x)=1+cx$ $\forall x\le -\frac 1c$ and $f(x)=0$ $\forall x\ge -\frac 1c$ which indeed is a solution iff $c<0$

4) synthesis of solutions
=================
So we found four continuous solutions :

$f(x)=0$ $\forall x$

$f(x)=1+cx$ $\forall x$ for any $c\in\mathbb R$ (notice that $c=0$ gives the constant solution $f(c)=1$)

$f(x)=1+cx$ $\forall x\ge -\frac 1c$ and $f(x)=0$ $\forall x\le -\frac 1c$ for any real $c>0$

$f(x)=1-cx$ $\forall x\le \frac 1c$ and $f(x)=0$ $\forall x\ge \frac 1c$ for any real $c>0$
\end{solution}



\begin{solution}[by \href{https://artofproblemsolving.com/community/user/124745}{Uzbekistan}]
	Oh PCO I am sorry  :blush:
\end{solution}
*******************************************************************************
-------------------------------------------------------------------------------

\begin{problem}[Posted by \href{https://artofproblemsolving.com/community/user/143105}{davidkomarovi}]
	We have a function f(x)=x^2+8x+56
Solve the equation: f(f(f(f(f(x)))))=2012
	\flushright \href{https://artofproblemsolving.com/community/c6h468614}{(Link to AoPS)}
\end{problem}



\begin{solution}[by \href{https://artofproblemsolving.com/community/user/29428}{pco}]
	\begin{tcolorbox}We have a function f(x)=x^2+8x+56
Solve the equation: f(f(f(f(f(x)))))=2012\end{tcolorbox}
$f(x)=(x+4)^2+40\ge 40$ and is increasing for $x\ge -4$
So $f(f(x))\ge f(40)=1976$
So $f(f(f(x)))\ge f(1976)>2012$

So no real solution.
\end{solution}
*******************************************************************************
-------------------------------------------------------------------------------

\begin{problem}[Posted by \href{https://artofproblemsolving.com/community/user/141397}{subham1729}]
	The function f : N → N is defined by f(n) = n + [√n]. Prove that
for any positive integer m the sequence m, f(m), f(f(m)), f(f(f(m))), . . .
contains infinitely many squares.
	\flushright \href{https://artofproblemsolving.com/community/c6h468872}{(Link to AoPS)}
\end{problem}



\begin{solution}[by \href{https://artofproblemsolving.com/community/user/29428}{pco}]
	\begin{tcolorbox}The function f : N → N is defined by f(n) = n + [√n]. Prove that
for any positive integer m the sequence m, f(m), f(f(m)), f(f(f(m))), . . .
contains infinitely many squares.\end{tcolorbox}
It's equivalent to prove that it contains at least one square.

If $n=p^2$, we get the result

If $n=p^2+k$ with $k\in[1,2p]$, then :
If $k<p+1$ then $f(n)=p^2+p+k$ and $f(f(n))=p^2+2p+k=(p+1)^2+(k-1)$ and $0\le k-1<k$
If $k=p+1$ then $f(n)=p^2+p+k=(p+1)^2$ is a square
If $k>p+1$ then $f(n)=p^2+p+k=(p+1)^2+(k-p-1)$ and $0<k-p-1<k$

And so, in at most $k$ such steps, we get a square.
Q.E.D.
\end{solution}
*******************************************************************************
-------------------------------------------------------------------------------

\begin{problem}[Posted by \href{https://artofproblemsolving.com/community/user/144414}{Masol}]
	Hey,

Find all function from IN* -> IN* such that :
$f(f(n))=n^{2}$

Good Luck
	\flushright \href{https://artofproblemsolving.com/community/c6h469041}{(Link to AoPS)}
\end{problem}



\begin{solution}[by \href{https://artofproblemsolving.com/community/user/29428}{pco}]
	\begin{tcolorbox}Hey,

Find all function from IN* -> IN* such that :
$f(f(n))=n^{2}$

Good Luck\end{tcolorbox}
General solution :

Let $C$ be the set of all non square positive integers.
Let $u(n)$ from $\mathbb N\to C$ and $v(n)$ from $\mathbb N\to \mathbb N\cup\{0\}$ uniquely defined such that $n=u(n)^{2^{v(n)}}$

Then $f(x)$ is defined as :
Let $A,B$ any split of $C$ in two equinumerous subsets and $g(x)$ any bijection from $A\to B$
If $u(n)\in A$, then $f(n)=\left(g(u(n))\right)^{2^{v(n)}}$
If $u(n)\in B$, then $f(n)=\left(g^{[-1]}(u(n))\right)^{2^{v(n)+1}}$
\end{solution}



\begin{solution}[by \href{https://artofproblemsolving.com/community/user/144414}{Masol}]
	Weird solution, wasn't expecting this :)
\end{solution}



\begin{solution}[by \href{https://artofproblemsolving.com/community/user/29428}{pco}]
	Notice that $u(n)$ and $v(n)$ are not defined when $n=1$

hence a little modification :
$f(1)=1$ (easy to prove)
and $f(n)$ as proposed in my previous post when $n>1$
\end{solution}



\begin{solution}[by \href{https://artofproblemsolving.com/community/user/144414}{Masol}]
	Ok!
And does there exist f IN* -> IN* such that $f(f(f(n)))=n^{3}$
\end{solution}



\begin{solution}[by \href{https://artofproblemsolving.com/community/user/29428}{pco}]
	\begin{tcolorbox}Ok!
And does there exist f IN* -> IN* such that $f(f(f(n)))=n^{3}$\end{tcolorbox}
You are welcome. Glad to have helped you.
You should try to understand my first answer and find yourself from it the solution of this new problem.
And tell your teacher \/ friend \/ little sister that his \/ her exercices are of a poor interest for your training since you will likely not find them in a real olympiad contest.

General solution :

Let $D$ be the set of all non perfect cubes integers
Let $u(n)$ from $\mathbb N\setminus\{1\}\to D$ and $v(n)$ from $\mathbb N\setminus\{1\}\to \mathbb N\cup\{0\}$ defined (in a unique manner) by $n=u(n)^{3^{v(n)}}$

Then $f(x)$ is defined as :
Let $A,B,C$ any split of $D$ in three equinumerous subsets and $g(x)$ any bijection from $A\to B$ and $h(x)$ any bijection from $B\to C$
$f(1)=1$
For $n>1$ :
If $u(n)\in A$, then $f(n)=\left(g(u(n))\right)^{3^{v(n)}}$
If $u(n)\in B$, then $f(n)=\left(h(u(n))\right)^{3^{v(n)}}$
If $u(n)\in C$, then $f(n)=\left(g^{[-1]}(h^{[-1]}(u(n)))\right)^{3^{v(n)+1}}$
\end{solution}



\begin{solution}[by \href{https://artofproblemsolving.com/community/user/144414}{Masol}]
	Thank's for help! 
But this is a problem from 2012 French TST and here is the full statement :
"let k>1 an integer. A function f : IN* -> IN* is called k-tastrophic when for any integer n>0, $f_{k}(n)=n^{k}$ where $f_{k}(n)=f(f(f...f(n))) \  k \ times$.
For which k does there exist a k-tastrophic function?"
\end{solution}



\begin{solution}[by \href{https://artofproblemsolving.com/community/user/29428}{pco}]
	\begin{tcolorbox}Thank's for help! 
But this is a problem from 2012 French TST and here is the full statement :
"let k>1 an integer. A function f : IN* -> IN* is called k-tastrophic when for any integer n>0, $f_{k}(n)=n^{k}$ where $f_{k}(n)=f(f(f...f(n))) \  k \ times$.
For which k does there exist a k-tastrophic function?"\end{tcolorbox}
With such a statement, problem is much simpler : you no longer need to find all the solutions but just one, and this is quite easier. For example :

Let integer $k>1$
Let $v_k(n)$ be the greatest power of $k$ which divides $n$
Let $g(x)$ from $\mathbb N\to\mathbb N$ defined as :
If $\left\lfloor\frac{n}{k^{v_k(n)+1}}\right\rfloor\equiv k-1\pmod k$, then $g(n)=kn-(k-1)k^{v_k(n)+2}$
If $\left\lfloor\frac{n}{k^{v_k(n)+1}}\right\rfloor\not\equiv k-1\pmod k$, then $g(n)=n+k^{v_k(n)+1}$

Then $g_k(n)=kn$

And then just define $f(n)$ as :
$f(1)=1$
For $n>1$ : $f(n)=$ $f\left(\prod p_i^{n_i}\right)=\prod p_i^{g(n_i)}$ where $\prod p_i^{n_i}$ is the prime decomposition of $n$

And so there exists a k-tastrophic function for any such $k$.
Notice too that the method I described in my previous posts gives the general form of all these functions.
\end{solution}



\begin{solution}[by \href{https://artofproblemsolving.com/community/user/144414}{Masol}]
	Thank you very much!
\end{solution}
*******************************************************************************
-------------------------------------------------------------------------------

\begin{problem}[Posted by \href{https://artofproblemsolving.com/community/user/94615}{Pedram-Safaei}]
	$ find \: all \: functions\,  f:\mathbb{N}\rightarrow \mathbb{N}\,  such \: that:f(2012f(n)+1390)=1^{2}+2^{2}+...+2012^2\:  for \,\: all\:  n\in \mathbb{N}.$
	\flushright \href{https://artofproblemsolving.com/community/c6h469218}{(Link to AoPS)}
\end{problem}



\begin{solution}[by \href{https://artofproblemsolving.com/community/user/29428}{pco}]
	\begin{tcolorbox}find all functions $f(n)$ from $\mathbb N\to\mathbb N$ such that:
$f(2012f(n)+1390)=n+(1^2+2^2+...+2012^2)$\end{tcolorbox}
The equation is $f(af(n)+b)=n+c$ with appropriate values for $a,b,c$

$f(n)$ is injective.

If $f(an)>c$, then we can write $f(af(f(an)-c)+b)=f(an)$ and so (since injective) $af(f(n)-c)+b=an$, impossible since $a\not|b$

So $f(an)\le c$ $\forall n\in\mathbb N$, which is impossible since $f(n)$ is injective.

So no solution to this equation.
\end{solution}



\begin{solution}[by \href{https://artofproblemsolving.com/community/user/94615}{Pedram-Safaei}]
	Yes My solution is the same your.what if 2012 replace by 1?
\end{solution}



\begin{solution}[by \href{https://artofproblemsolving.com/community/user/94615}{Pedram-Safaei}]
	\begin{tcolorbox}[quote="Pedram-Safaei"]find all functions $f(n)$ from $\mathbb N\to\mathbb N$ such that:
$f(2012f(n)+1390)=n+(1^2+2^2+...+2012^2)$\end{tcolorbox}
The equation is $f(af(n)+b)=n+c$ with appropriate values for $a,b,c$

$f(n)$ is injective.

If $f(an)>c$, then we can write $f(af(f(an)-c)+b)=f(an)$ and so (since injective) $af(f(n)-c)+b=an$, impossible since $a\not|b$

So $f(an)\le c$ $\forall n\in\mathbb N$, which is impossible since $f(n)$ is injective.

So no solution to this equation.\end{tcolorbox}
Yes My solution is the same your.what if 2012 replace by 1?
\end{solution}



\begin{solution}[by \href{https://artofproblemsolving.com/community/user/29428}{pco}]
	\begin{tcolorbox}Yes My solution is the same your.what if 2012 replace by 1?\end{tcolorbox}
So $f(f(n)+b)=n+c$ with appropriate values for $b,c$

Let then $g(n)=f(n)+b$ and we get $g(g(n))=n+b+c$ where $g(n)>b$ $\forall n$

General solution of this equation :
Let $A,B$ a split in two equinumerous subsets of $\{1,2,...,b+c\}$ such that $\{1,2,... b\}\in A$ (possible since $c>b$ and $b+c$ even)
Let $h(x)$ any bijection from $A\to B$

Let $n=a(n)(b+c)+r(n)$ where $a(n)\in\mathbb N\cup\{0\}$ and $r(n)\in\{1,2,...,b+c\}$

If $r(n)\in A$, then $f(n)=a(n)(b+c)+h(r(n))-b$
If $r(n)\in B$, then $f(n)=(a(n)+1)(b+c)+h^{-1}(r(n))-b$
\end{solution}
*******************************************************************************
-------------------------------------------------------------------------------

\begin{problem}[Posted by \href{https://artofproblemsolving.com/community/user/141397}{subham1729}]
	Let f : N - N be a function such that (x + y)f(x) <= x^2 + f(xy) + 110, for all x; y in N.
Determine the minimum and maximum values of f(23) + f(2011).
	\flushright \href{https://artofproblemsolving.com/community/c6h469528}{(Link to AoPS)}
\end{problem}



\begin{solution}[by \href{https://artofproblemsolving.com/community/user/29428}{pco}]
	\begin{tcolorbox}Let f : N - N be a function such that (x + y)f(x) <= x^2 + f(xy) + 110, for all x; y in N.
Determine the minimum and maximum values of f(23) + f(2011).\end{tcolorbox}
Let $P(x,y)$ be the assertion $(x+y)f(x)\le x^2+f(xy)+110$

1) any solution is such that $x-110\le f(x)\le x$ $\forall x$
=======================================
$P(x,1)$ $\implies$ $f(x)\le x+\frac{110}x$

Then $P(x,y)$ $\implies$ $(x+y)f(x)\le x^2+xy+\frac{110}{xy}+110$ and so $f(x)\le x+110\frac{xy+1}{xy(x+y)}$ and so, setting $y\to +\infty$ :

$f(x)\le x$ $\forall x$
So $f(1)=1$

$P(1,x)$ $\implies$ $f(x)\ge x-110$
Q.E.D.

2) any function from $\mathbb N\to\mathbb N$ such that $x-110\le f(x)\le x$ $\forall x$ is a solution
=========================================================
$(x+y)f(x)\le (x+y)x=x^2+xy\le x^2+f(xy)+110$
Q.E.D.

3) Answer to the question
==================
From 1) : $f(23)\le 23$ and $f(2011)\le 2011$ and so $f(23)+f(2011)\le 2034$ 
From 2) : $f(x)=x$ is a solution and so $f(23)+f(2011)= 2034$ may be reached.

From 1) : $f(23)\ge 1$ and $f(2011)\ge 2011-110= 1901$ and so $f(23)+f(2011)\ge 1902$
From 2) : $f(x)=\max(1,x-110)$ is a solution and so $f(23)+f(2011)= 1902$ may be reached.

Hence the answer $\boxed{1902\le f(23)+f(2011)\le 2034}$ and both bounds may be reached.
\end{solution}
*******************************************************************************
-------------------------------------------------------------------------------

\begin{problem}[Posted by \href{https://artofproblemsolving.com/community/user/114585}{anonymouslonely}]
	Find all polynomials such that $ f(x^{2})=f^{2}(x)+2f(x) $ and the function $ g $ defined on the real numbers and real valued such that $ g(x)=f(x) $ is bijective.
	\flushright \href{https://artofproblemsolving.com/community/c6h469554}{(Link to AoPS)}
\end{problem}



\begin{solution}[by \href{https://artofproblemsolving.com/community/user/31919}{tenniskidperson3}]
	Replace $x$ with $-x$ to see that $f^2(x)-2f(x)=f^2(-x)-2f(-x)$.  That is, $(f(x)-f(-x))(f(x)+f(-x)-2)=0$.  So one of the polynomials $f(x)-f(-x)$ or $f(x)+f(-x)-2$ has an infinite number of roots, and is thus identically 0.  It can't be $f(x)-f(-x)$ because $f$ is bijective from $\mathbb{R}$ to $\mathbb{R}$.  So that means that $f(x)+f(-x)-2=0$ for all $x$.  In particular, $f(0)=1$.  But letting $x=0$ in the original equation gives $f(0)=f^2(0)-2f(0)$ which means $1=1-2$ or $-1=1$, which is not true.  Therefore there are no such polynomials.

Unless $f^2(x)$ means $f(f(x))$ in which case the polynomial must have degree 2, but then it is still not bijective.  So there are no solutions either way you read the notation.
\end{solution}



\begin{solution}[by \href{https://artofproblemsolving.com/community/user/114585}{anonymouslonely}]
	yes...sorry.... I made a mistake. look at this now.
\end{solution}



\begin{solution}[by \href{https://artofproblemsolving.com/community/user/29428}{pco}]
	\begin{tcolorbox}Find all polynomials such that $ f(x^{2})=f^{2}(x)+2f(x) $ and the function $ g $ defined on the real numbers and real valued such that $ g(x)=f(x) $ is bijective.\end{tcolorbox}
So $f^2(x)+2f(x)=f^2(-x)+2f(-x)$ and so $(f(x)-f(-x))(f(x)+f(-x)+2)=0$ and since $f(x)$ is injective, $f(x)+f(-x)+2=0$

So $f(x)+1$ is an odd polynomial and we can write $f(x)=xh(x^2)-1$ where $h(x)\in\mathbb R[X]$

Plugging this in original equation, we get $h(x^4)=h(x^2)^2$ and so (since polynomial) $h(x^2)=h(x)^2$

This is a very classical equation whose only polynomial solutions are $h(x)=0$ and $h(x)=x^n$ (with $n\in\mathbb N\cup\{0\}$)

$h(x)=0$ implies $f(x)=-1$ which is not a solution

$h(x)=x^n$ implies $\boxed{f(x)=x^{2n+1}-1}$ which indeed is a solution
\end{solution}
*******************************************************************************
-------------------------------------------------------------------------------

\begin{problem}[Posted by \href{https://artofproblemsolving.com/community/user/24228}{yunxiu}]
	$n$ being a given integer, find all functions $f: \mathbb{Z} \to \mathbb{Z}$, such that for all integers $x,y$ we have $f\left( {x + y + f(y)} \right) = f(x) + ny$.
	\flushright \href{https://artofproblemsolving.com/community/c6h469660}{(Link to AoPS)}
\end{problem}



\begin{solution}[by \href{https://artofproblemsolving.com/community/user/29428}{pco}]
	\begin{tcolorbox}$n$ is a positive integer, find all function $f:Z \to Z$, satisty that for any integers $x,y$, $f\left( {x + y + f(y)} \right) = f(x) + ny$.\end{tcolorbox}
Let $P(x,y)$ be the assertion $f(x+y+f(y))=f(x)+ny$
Let $f(0)=a$

$P(0,0)$ $\implies$ $f(a)=a$
$P(a,0)$ $\implies$ $f(2a)=a$ 
$P(0,a)$ $\implies$ $a=0$

Let $A=\{y$ such that $f(x+y)=f(x)+f(y)$ $\forall x\in\mathbb Z\}$
Notice that $P(0,y)$ $\implies$ $f(y+f(y))=ny$ and so $x+f(x)\in A$ $\forall x$

$0\in A$
$y\in A$ $\implies$ $f(-y+y)=f(-y)+f(y)$ and so $f(-y)=-f(y)$ and so $f(x-y+y)=f(x-y)+f(y)$ and so $f(x-y)=f(x)+f(-y)$ and $-y\in A$
$y,z\in A$ $\implies$ $f(x+y+z)=f(x+y)+f(z)$ $=f(x)+f(y)+f(z)$ $=f(x)+f(y+z)$ and so $y+z\in A$

So $A$ is an additive subgroup of $\mathbb Z$ and so is $\{0\}$ or $\{ku\}_{k\in\mathbb Z}$ for some $u\in\mathbb N$

$A=\{0\}$ implies $x+f(x)=0$ and so $f(x)=-x$ which is not a solution
$A=\{ku\}_{k\in\mathbb Z}$ and $f(x+y)=f(x)+f(y)$ $\forall x,y\in A$ implies $f(nu)=nf(u)$ and so $f(x)=\frac{xf(u)}u$ $\forall x\in A$

So $f(x+f(x))=\frac{(x+f(x))f(u)}u=nx$ $\forall x\in\mathbb Z$

So $f(u)\ne 0$ and $f(x)=\frac{nu-f(u)}{f(u)}x=f(1)x=cx$ for some integer $c$

Plugging this back in ortiginal equation, we get $c(c+1)=n$

Hence the answer :
If $\exists c\in\mathbb Z$ such that $n=c(c+1)$, then $f(x)=cx$ and $f(x)=-(c+1)x$ are the only two solutions
If $\not\exists c\in\mathbb Z$ such that $n=c(c+1)$, then no solution
\end{solution}



\begin{solution}[by \href{https://artofproblemsolving.com/community/user/104062}{Hooksway}]
	To pco:I think your answer is a little wrong.You may pay more attention to the condition $n=0$.
\end{solution}



\begin{solution}[by \href{https://artofproblemsolving.com/community/user/29428}{pco}]
	\begin{tcolorbox}To pco:I think your answer is a little wrong.You may pay more attention to the condition $n=0$.\end{tcolorbox}
Thanks, but I think your remark is a little wrong.You may pay more attention to the statement "$n$ is a positive integer"
\end{solution}



\begin{solution}[by \href{https://artofproblemsolving.com/community/user/104062}{Hooksway}]
	\begin{tcolorbox}[quote="Hooksway"]To pco:I think your answer is a little wrong.You may pay more attention to the condition $n=0$.\end{tcolorbox}
Thanks, but I think your remark is a little wrong.You may pay more attention to the statement "$n$ is a positive integer"\end{tcolorbox}
The problem has been edited by yunxiu.So you have made no mistake before.
\end{solution}



\begin{solution}[by \href{https://artofproblemsolving.com/community/user/29428}{pco}]
	Indeed, OP silently modified his problem after your remark and my answer :(
If $n<0$, we easily get from my solution that no $c$ exists and so no solution.

If $n=0$, the equation is $f(x+y+f(y))=f(x)$

Considering the set $A=\{y$ such that $f(x+y)=f(x)$ $\forall x\}$, we easily get that this is an additive subgroup of $\mathbb Z$ and so is $\{0\}$ or $u\mathbb Z$

The case $A=\{0\}$ implies $\boxed{f(x)=-x}$ (since $f(x)+x\in A$) which indeed is a solution

The case $A=u\mathbb Z$ gives $f(x+u)=f(x)$ and $x+f(x)=g(x)u$ and so $g(x+u)=g(x)+1$

So $g(x)=\left\lfloor\frac xu\right\rfloor$ $+h\left(x-u\left\lfloor\frac xu\right\rfloor\right)$ where $h(x)$ is a function from $\{0,...,u-1\}\to\mathbb N$

And so :
$\boxed{f(x)=uh\left(x-u\left\lfloor\frac xu\right\rfloor\right)-\left(x-u\left\lfloor\frac xu\right\rfloor\right)}$

which indeed is a solution whatever are integer $u>0$ and function $h(x)$ from $\{0,...,u-1\}\to\mathbb Z$

Examples : 
$u=1$ gives $f(x)=c$

$u=2$ gives $f(2n)=2p$ and $f(2n+1)=2q+1$

$u=3$ gives $f(3n)=3p$ and $f(3n+1)=3q+2$ and $f(3n+2)=3r+1$

and so on
\end{solution}



\begin{solution}[by \href{https://artofproblemsolving.com/community/user/55721}{Thjch Ph4 Trjnh}]
	\begin{tcolorbox}$n$ is a given integer, find all function $f:Z \to Z$, satisty that for any integers $x,y$, $f\left( {x + y + f(y)} \right) = f(x) + ny$.\end{tcolorbox}
By induction, we have : $f(x+k(y+f(y))=f(x)+kny, \forall x,y,k\in Z$.
$\Rightarrow f(k(y+f(y))=f(0)+kny, \forall k,y\in Z$.
$\Rightarrow f((x+f(x))(1+f(1)))=f(0)+(x+f(x)).n.1=f(0)+(1+f(1)).n.x, \forall x\in Z$.
$\Rightarrow f(x)=f(1)x, \forall x\in Z$.
\end{solution}



\begin{solution}[by \href{https://artofproblemsolving.com/community/user/29428}{pco}]
	\begin{tcolorbox}[quote="yunxiu"]$n$ is a given integer, find all function $f:Z \to Z$, satisty that for any integers $x,y$, $f\left( {x + y + f(y)} \right) = f(x) + ny$.\end{tcolorbox}
By induction, we have : $f(x+k(y+f(y))=f(x)+kny, \forall x,y,k\in Z$.
$\Rightarrow f(k(y+f(y))=f(0)+kny, \forall k,y\in Z$.
$\Rightarrow f((x+f(x))(1+f(1)))=f(0)+(x+f(x)).n.1=f(0)+(1+f(1)).n.x, \forall x\in Z$.
$\Rightarrow f(x)=f(1)x, \forall x\in Z$.\end{tcolorbox}

And what about $n=0$ and $f(x)=1$ ?

And what about $n=0$ and $f(x)=\frac{5-(-1)^x}2$ ?

And, if $n\ne 0$ dont you think that some constraints exist on $f(1)$ ?
\end{solution}



\begin{solution}[by \href{https://artofproblemsolving.com/community/user/55721}{Thjch Ph4 Trjnh}]
	\begin{tcolorbox}[quote="Thjch Ph4 Trjnh"][quote="yunxiu"]$n$ is a given integer, find all function $f:Z \to Z$, satisty that for any integers $x,y$, $f\left( {x + y + f(y)} \right) = f(x) + ny$.\end{tcolorbox}
By induction, we have : $f(x+k(y+f(y))=f(x)+kny, \forall x,y,k\in Z$.
$\Rightarrow f(k(y+f(y))=f(0)+kny, \forall k,y\in Z$.
$\Rightarrow f((x+f(x))(1+f(1)))=f(0)+(x+f(x)).n.1=f(0)+(1+f(1)).n.x, \forall x\in Z$.
$\Rightarrow f(x)=f(1)x, \forall x\in Z$.\end{tcolorbox}

And what about $n=0$ and $f(x)=1$ ?

And what about $n=0$ and $f(x)=\frac{5-(-1)^x}2$ ?

And, if $n\ne 0$ dont you think that some constraints exist on $f(1)$ ?\end{tcolorbox}
I thought  $n>0$ and in this case then the solution same you.
\end{solution}



\begin{solution}[by \href{https://artofproblemsolving.com/community/user/133488}{Nostalgius}]
	\begin{tcolorbox}
The case $A=u\mathbb Z$ gives $f(x+u)=f(x)$ and $x+f(x)=g(x)u$ and so $g(x+u)=g(x)+1$
\end{tcolorbox}
\begin{bolded}@pco\end{bolded} Can you explain more about these quote please ? $x+f(x)=g(x)u$
\end{solution}



\begin{solution}[by \href{https://artofproblemsolving.com/community/user/29428}{pco}]
	\begin{tcolorbox}[quote="pco"]
The case $A=u\mathbb Z$ gives $f(x+u)=f(x)$ and $x+f(x)=g(x)u$ and so $g(x+u)=g(x)+1$
\end{tcolorbox}
\begin{bolded}@pco\end{bolded} Can you explain more about these quote please ? $x+f(x)=g(x)u$\end{tcolorbox}
From $f(x+y+f(y))=f(x)$ we get that $y+f(y)\in A$
From $A=u\mathbb Z$, we get that $u|y+f(y)$ $\forall y$ and so $\exists g(x)$ from $\mathbb Z\to\mathbb Z$ such that $x+f(x)=g(x)u$
\end{solution}



\begin{solution}[by \href{https://artofproblemsolving.com/community/user/24228}{yunxiu}]
	1) If $n = 0$, then for any integers $x,y$ we have $f\left( {x + y + f(y)} \right) = f(x)$.
	If any $y$, $y + f(y) \equiv 0$, then $f(y) =  - y$, which is a solution.
Now we suppose there exit a $y$ satisfies $y + f(y) \ne 0$, then $f(x)$ is a a periodic function. Let $T$ is the smallest positive period Of $f(x)$.For $x \in \left[ {1,T} \right]$, $f(x) =  - x + {k_x}T$, and$f(x + T) = f(x)$ for any $x \in Z$, which is an other solution.
2)If $n \ne 0$, let $f(0) = a$, take $x = y = 0$ we have$f(a) = a$;  take $[x = a, y = 0$ we have $f(2a) = a$; take $x = 0, y = a$ we have $a = f(2a) = a + na$, so $f(0) = a = 0$.
	Take $x = 0$ we have $f\left( {y + f(y)} \right) = ny$①, take $y = x + f(x)$ in ① we have
$n\left( {x + f(x)} \right) = f\left( {x + f(x) + f\left( {x + f(x)} \right)} \right) = f\left( {nx + x + f(x)} \right) = f(nx) + nx$
Hence for any integer $x$, $f(nx) = nf(x)$②.
If $f(x) = f(y)$, then $f(y) + nx = f\left( {y + x + f(x)} \right) = f\left( {x + y + f(y)} \right) = f(x) + ny$,
so $x = y$, $f$ is injective.
Let ${d_m} = \left[ {(m + 1) + f(m + 1)} \right] - \left[ {m + f(m)} \right]$, take $x = 0,y = m + 1$ we have$f\left( {m + 1 + f(m + 1)} \right) = n(m + 1)$.
Take $x = {d_m},y = m$ we have$f\left( {m + 1 + f(m + 1)} \right) = f\left( {{d_m} + m + f(m)} \right) = f({d_m}) + nm$, so $f({d_m}) = n$. Because $f$ is injective. ${d_m}$ is a constant, so $f(m)$ is a arithmetical series, so $f(m) = mc$, $c$ is a  constant. So we have 
$cx + ny = f(x) + ny = f\left( {x + y + f(y)} \right) = c\left( {x + y + f(y)} \right) = c\left( {x + y + cy} \right)$
Hence $n = c(c + 1)$, $n > 0$.

Above all if $n < 0$ there are no solution.
If $n = 0$, $f(y) =  - y$; or $x \in \left[ {1,T} \right]$, $f(x) =  - x + {k_x}T$, and $f(x + T) = f(x)$ for any $x \in Z$.
If $n > 0$, and $n = c(c + 1),c \in {Z^ + }$, then $f(x) = cx$, and $f(x) =  - (c + 1)x$ are two solutions.
If $n \ne c(c + 1)$ for any $c \in {Z^ + }$, there are no solution.
\end{solution}



\begin{solution}[by \href{https://artofproblemsolving.com/community/user/20399}{navid}]
	First Prove f Must be injecive , then by iserting ne variable  Count 
  f(x+y+z+f(z)+f(y)) twice!, conclude that f must be additive!,.....
\end{solution}



\begin{solution}[by \href{https://artofproblemsolving.com/community/user/29428}{pco}]
	\begin{tcolorbox}First Prove f Must be injecive , then by iserting ne variable  Count 
  f(x+y+z+f(z)+f(y)) twice!, conclude that f must be additive!,.....\end{tcolorbox}
And dont forget verifying your result.

For example $f(x)=2$ is a solution when $n=0$ but is neither injective, neither additive 
\end{solution}



\begin{solution}[by \href{https://artofproblemsolving.com/community/user/78444}{Babai}]
	Pco can u explain me please one thing??I understood that A is additive subgroup of Z but don't understand the very next line where you are saying that so this is {0} or {ku} .why did you write this??is this for cauchy's equation??
\end{solution}



\begin{solution}[by \href{https://artofproblemsolving.com/community/user/29428}{pco}]
	\begin{tcolorbox}Pco can u explain me please one thing??I understood that A is additive subgroup of Z but don't understand the very next line where you are saying that so this is {0} or {ku} .why did you write this??is this for cauchy's equation??\end{tcolorbox}
No, these are the only additive subgroups of $\mathbb Z$. Just consider in a given suggroup different from $\{0\}$ the littlest positive element. It's easy to show that it divides all the elements of the subgroup, hence the result.
\end{solution}



\begin{solution}[by \href{https://artofproblemsolving.com/community/user/20399}{navid}]
	My Approach works for $n\ne{0} $ ,I think  for$ n=0$ , it was same as problem priviously proposed for France TsT. BY THE WAY , if $n\ne{0}$ , we have:

Let$ f(z)=f(y)=t$ , we have:
 
$f(z)=f(-t+z+f(z))=f(-t)+nz=f(y)=f(-t+y+f(y))=f(-t)+ny ,.....$

other part of my solution seems to be obvious!
\end{solution}



\begin{solution}[by \href{https://artofproblemsolving.com/community/user/183149}{JuanOrtiz}]
	Let us have $g(x)=f(x)+x$. Firstly we can see that $f(x+f(0))=f(x)$ and since $f$ is not bounded (if $n\neq 0$) we obtain $f(0)=0$ and therefore $f(x+f(x))=nx \forall x$. Now we verify $g(x+g(y))=g(x)+g(y)+ny$ and so $g(g(y))=g(y)+ny$ and so $g(a)+g(b)=g(a+b)$ if $b \in \text{Img}(g)$. Let's take $b,c \in \text{Img}(g)$ and we verify $bg(c)=g(bc)=cg(b)$ and so $g(x)\/x$ is constant in $x \in \text{Img}(f)$ and since $g(g(x))=g(x)+nx$ then $g(g(x))\/g(x)=1+n(x\/g(x))$ is constant and so $x\/g(x)$ is constant and therefore it is linear and we easily finish.

For $n=0$ let $\Omega=\{x+f(x) : x \in \mathbb{Z} \}$ and so $f(x+e)=f(x)$ for all $e \in \Omega$ and by Bezout we get that $f$ is periodic with period $d = \text{gcd}(\Omega)$. Also, $d | x+f(x) \forall x$ by definition, so $f$ is of the form 

$f(x)=da_{x \text{ mod } d} - (x \text{ mod } d)$ where $a_0,...,a_{d-1}$ are any fixed integers 

(Note: $a \bmod b$ is equal to the residue of $a$ modulo $b$.) Notice that any function of this form works.
\end{solution}
*******************************************************************************
-------------------------------------------------------------------------------

\begin{problem}[Posted by \href{https://artofproblemsolving.com/community/user/134696}{bidruunbid}]
	Find all the functions $ f:\mathbb{R}\rightarrow\mathbb{R} $
$ f(f(x)) = 1-2x $
	\flushright \href{https://artofproblemsolving.com/community/c6h469717}{(Link to AoPS)}
\end{problem}



\begin{solution}[by \href{https://artofproblemsolving.com/community/user/29428}{pco}]
	\begin{tcolorbox}Find all the functions $ f:\mathbb{R}\rightarrow\mathbb{R} $
$ f(f(x)) = 1-2x $\end{tcolorbox}
Setting $f(x)=g(x-\frac 13)+\frac 13$, we get the equivalent equation $g(g(x))=-2x$

Notice then than no continuous solution exists since $-2x$ is decreasing.
A general form of non continuous solutions is easy to give.

Let $\sim$ be the equivalence relation in $\mathbb R^*$ such that $x\sim y$ $\iff$ $\exists n\in\mathbb Z$ such that $y=x(-2)^n$
Let $r(x)$ any function associating to a non zero real a representant (unique per class) of its equivalence class.
Let $A,B$ a split of $r(\mathbb R^*)$ in two equinumerous sets and $h(x)$ any bijection from $A\to B$
Let $n(x)$ the unique function from $\mathbb R^*\to \mathbb Z$ such that $x=r(x)(-2)^{n(x)}$
Then $g(x)$ may be defined as :

$g(0)=0$
$\forall x\ne 0$ such that $r(x)\in A$, then $g(x)=(-2)^{n(x)}h(r(x))$
$\forall x\ne 0$ such that $r(x)\in B$, then $g(x)=(-2)^{n(x)+1}h^{-1}(r(x))$
\end{solution}
*******************************************************************************
-------------------------------------------------------------------------------

\begin{problem}[Posted by \href{https://artofproblemsolving.com/community/user/3182}{Kunihiko_Chikaya}]
	Denote by $f(a,\ b)$ the number of all possible pairs $(a,\ b)$ of integers such that the equation in $x$: $x^{2n}+ax^n+b=0$ has $2n\ (n\in\mathbb{N^{+}})$ real solutions.

Find $\lim_{a\to\infty}\frac{f(a,\ b)}{a^3}.$

proposed by sunrise
	\flushright \href{https://artofproblemsolving.com/community/c6h469830}{(Link to AoPS)}
\end{problem}



\begin{solution}[by \href{https://artofproblemsolving.com/community/user/29428}{pco}]
	\begin{tcolorbox}Denote by $f(a,\ b)$ the number of pairs of integers $(a,\ b)$ such that the equation in $x$: $x^{2n}+ax^n+b=0$ has $2n\ (n\in\mathbb{N^{+}})$ real solutions.

Find $\lim_{a\to\infty}\frac{f(a,\ b)}{a^3}.$

proposed by sunrise\end{tcolorbox}
I'm confused by the usage of $(a,b)$  :

What is the meaning of "$u=f(1,3)$ is the number of pairs $(1,3)$ such that ... " ?????

With such definition :
 either $(1,3)$ is such that .... and then $f(1,3)=1$
 either $(1,3)$ is not such that .... and then $f(1,3)=0$

And in both cases, the required limit is $0$
\end{solution}



\begin{solution}[by \href{https://artofproblemsolving.com/community/user/29428}{pco}]
	\begin{tcolorbox}Denote by $f(a,\ b)$ the number of all possible pairs $(a,\ b)$ of integers such that the equation in $x$: $x^{2n}+ax^n+b=0$ has $2n\ (n\in\mathbb{N^{+}})$ real solutions.

Find $\lim_{a\to\infty}\frac{f(a,\ b)}{a^3}.$

proposed by sunrise\end{tcolorbox}
I'm sorry but each time I post a question, you silently edit the problem but your modification is meaningless for me ....

To be quite clear, could you explain exactly what is $f(a,b)$ and give us one or two examples :
With your definition, what is the value of $f(1,1)$ ?
With your definition, what is the value of $f(3,1)$ ?
With your definition, could you give us two integers $u,v$ such that $f(u,v)>1$ ?

Thanks helping me to understand your problem.
\end{solution}



\begin{solution}[by \href{https://artofproblemsolving.com/community/user/3182}{Kunihiko_Chikaya}]
	I'm very sorry, not to inform of you; I have made a slight modification of the expression of the problem.

Perhaps, when everyone sees the problem you might think such like this ;

Given $a,\ b$, consider the case that the equation with degree $2n\ (n\in{\mathbb{N^{+}}})$ in $x$ : $x^{2n}+ax^n+b=0$ has $2n$ real solutions.

However, let $x^{n}=t$, we can rewrite the equation as $t^2+at+b=0$, which has 2 solutions at best.

If $n$ is limited to the case of even, the real solutions has four at best, therefore for each $n$, the original equation can't have $2n$ real solutions.

That's why the problem seems to be incorrect.
\end{solution}



\begin{solution}[by \href{https://artofproblemsolving.com/community/user/29428}{pco}]
	\begin{tcolorbox}I'm very sorry, not to inform of you; I have made a slight modification of the expression of the problem.

Perhaps, when everyone sees the problem you might think such like this ;

Given $a,\ b$, consider the case that the equation with degree $2n\ (n\in{\mathbb{N^{+}}})$ in $x$ : $x^{2n}+ax^n+b=0$ has $2n$ real solutions.

However, let $x^{n}=t$, we can rewrite the equation as $t^2+at+b=0$, which has 2 solutions at best.

If $n$ is limited to the case of even, the real solutions has four at best, therefore for each $n$, the original equation can't have $2n$ real solutions.

That's why the problem seems to be incorrect.\end{tcolorbox}
Still quite quite unclear for me.

With your definition, what is the value of $f(1,1)$ ?
With your definition, what is the value of $f(3,1)$ ?
With your definition, could you give us two integers $u,v$ such that $f(u,v)>1$ ?

Thanks
\end{solution}



\begin{solution}[by \href{https://artofproblemsolving.com/community/user/3182}{Kunihiko_Chikaya}]
	Sorry, I can't understand the meaning of $f(a,\ b)$ as well.
\end{solution}



\begin{solution}[by \href{https://artofproblemsolving.com/community/user/29428}{pco}]
	\begin{tcolorbox}Sorry, I can't understand the meaning of $f(a,\ b)$.\end{tcolorbox}
Are you kidding us ?

You tried to define in your problem a function $f(x,y)$ from $\mathbb Z^2\to \mathbb N\cup\{0\}$ and the problem is to find $\lim_{x\to+\infty}\frac{f(x,y)}{x^3}$

And you dont understand the meaning of $f(x,y)$ ???

So how could we understand ?

And what is the meaning of $\lim_{x\to+\infty}\frac{f(x,y)}{x^3}$ is you (and so us) dont understand the meaning of $f(x,y)$

It's a joke ?
I give up.
\end{solution}



\begin{solution}[by \href{https://artofproblemsolving.com/community/user/3182}{Kunihiko_Chikaya}]
	Sorry, but I have been waiting for the proposer's comment and the solution.

If the original problem were rewritten as follows, then what would be the answer?

Let $m,\ n,\ a$ and $b$ be positive integers such that $a\leq m$.
Denote by $f(m)$ the number of $(a,\ b)$ such that $x^{2n}+ax^{n}+b=0$ has $2n$ distinct real solutions.
Determine the value of $n$ such that $f(m)\geq 1$, then for each $n$, find $\lim_{m\to\infty} \frac{f(m)}{m^3}.$
\end{solution}



\begin{solution}[by \href{https://artofproblemsolving.com/community/user/29428}{pco}]
	\begin{tcolorbox}Sorry, but I have been waiting for the proposer's comment and the solution.

If the original problem were rewritten as follows, then what would be the answer?

Let $m,\ n,\ a$ and $b$ be positive integers such that $a\leq m$.
Denote by $f(m)$ the number of $(a,\ b)$ such that $x^{2n}+ax^{n}+b=0$ has $2n$ distinct real solutions.
Determine the value of $n$ such that $f(m)\geq 1$, then for each $n$, find $\lim_{m\to\infty} \frac{f(m)}{m^3}.$\end{tcolorbox}
For $n>1$, we obviously have $f(m)=0$

For $n=1$, then $f(m)$ is the number of pairs $(a,b)$ such that $m^2\ge a^2>4b>0$ and so :
If $m<3$ : $f(m)=0$
If $m\ge 3$ : $f(m)=\sum_{a=3}^{m}\left\lfloor\frac{a^2-1}4\right\rfloor$

So $f(m)=\sum_{k=3,k\text{ odd}}^m\frac{k^2-1}4$ $+\sum_{k=3,k\text{ even}}^m\frac{k^2-4}4$

So $f(m)=\sum_{k=3}^m\frac{k^2}4$ $-\sum_{k=3,k\text{ odd}}^m\frac{1}4$ $-\sum_{k=3,k\text{ even}}^m\frac{4}4$

So $f(m)=\frac 14\left(\frac{m(m+1)(2m+1)}6-5\right)$ $-\frac 14\left(m+3\left\lfloor\frac m2\right\rfloor-5\right)$

So $\boxed{f(m)=\frac{m(m-1)(2m+5)}{24}-\frac 34\left\lfloor\frac m2\right\rfloor}$ when $n=1$ and $f(m)=0$ when $n>1$

And the required limit $\boxed{\lim_{m\to+\infty}\frac{f(m)}{m^3}=\frac 1{12}}$ when $n=1$ and $0$ when $n>1$
\end{solution}
*******************************************************************************
-------------------------------------------------------------------------------

\begin{problem}[Posted by \href{https://artofproblemsolving.com/community/user/8312}{olimpicu}]
	How demonstrate that $x+\sin x$ is increasing function, without using the notion of derived.
Thanks.
	\flushright \href{https://artofproblemsolving.com/community/c6h469863}{(Link to AoPS)}
\end{problem}



\begin{solution}[by \href{https://artofproblemsolving.com/community/user/3640}{Dr Sonnhard Graubner}]
	hello, if so, you must show that with $x_1<x_2$ follows $x_1+\sin(x_1)<x_2+\sin(x_2)$.
Sonnhard.
\end{solution}



\begin{solution}[by \href{https://artofproblemsolving.com/community/user/8312}{olimpicu}]
	That's right. You have any idea?
\end{solution}



\begin{solution}[by \href{https://artofproblemsolving.com/community/user/29428}{pco}]
	\begin{tcolorbox}How demonstrate that $x+\sin x$ is increasing function, without using the notion of derived.
Thanks.\end{tcolorbox}
Comparing for example chord and arc in a circle, it's easy to show that $|\sin x| < x $ $\forall x>0$

Then, if $x>y$, we get $\frac{x-y}2>|\sin\frac{x-y}2|$ $\ge |\sin\frac{x-y}2\cos \frac{x+y}2|$ $=\frac 12|\sin x-\sin y|$ 

So $x-y > \sin y - \sin x$ and so $x+\sin x > y+\sin y$

Q.E.D.
\end{solution}



\begin{solution}[by \href{https://artofproblemsolving.com/community/user/64716}{mavropnevma}]
	Among others, at the 2012 Romanian MO District stage for grade X, the following was asked (as easiest question)

[color=#0000FF]Let $f: [0,\infty) \to \mathbb{R}$ be non-constant, with the property $|f(x) - f(y)| \leq |\sin x - \sin y|$ for all $x,y \geq 0$. Prove that $f$ is bounded and periodic, while $g: [0,\infty) \to \mathbb{R}$ defined by $g(x) = x+f(x)$ is monotonous.[\/color]

Since $f(x) = \sin x$ clearly satisfies, we have the answer.
\end{solution}



\begin{solution}[by \href{https://artofproblemsolving.com/community/user/8312}{olimpicu}]
	Thanks . :-)
\end{solution}
*******************************************************************************
-------------------------------------------------------------------------------

\begin{problem}[Posted by \href{https://artofproblemsolving.com/community/user/10156}{Matematika}]
	Find all functions from the set of rational numbers to itself such that all following conditions are satisfied:
$f(0)=0$
$f(x)-x=f(-x)$
$f(f(y)-x)=f(f(x)-y)+y-x$
	\flushright \href{https://artofproblemsolving.com/community/c6h469887}{(Link to AoPS)}
\end{problem}



\begin{solution}[by \href{https://artofproblemsolving.com/community/user/139996}{Faustus}]
	If you put $x=0$ or $y=0$ in the 3rd relation(the equation is symmetric, so it doesn't make a difference) then you get, $f(f(x))=f(x)$
Thus, putting $f(x)=a$, we get $f(a)=a$(Remember f is $Q\to Q$, so we can do this)
but, $f(a)-a=f(-a)$
$\therefore f(-a)=0$ for all $x\in Q$
Thus, only soln. is $\boxed{f(x)=0}$
\end{solution}



\begin{solution}[by \href{https://artofproblemsolving.com/community/user/29428}{pco}]
	\begin{tcolorbox}
Thus, putting $f(x)=a$, we get $f(a)=a$(Remember f is $Q\to Q$, so we can do this)
\end{tcolorbox}

Do you mean that all functions from $\mathbb Q\to\mathbb Q$ are surjective ?

And, btw, check your conclusion against $f(x)=\frac{x+|x|}2$ :)
\end{solution}



\begin{solution}[by \href{https://artofproblemsolving.com/community/user/139996}{Faustus}]
	oh sorry, of course not!
Wrong proof!
Well, still an intuitive idea says, the trivial soln. is the only soln.
\end{solution}



\begin{solution}[by \href{https://artofproblemsolving.com/community/user/29428}{pco}]
	\begin{tcolorbox}oh sorry, of course not!
Wrong proof!
Well, still an intuitive idea says, the trivial soln. is the only soln.\end{tcolorbox}
$f(x)=0$ $\forall x$ is not a solution

And, according to me, $\frac{x+|x|}2$, which is a solution, is not a trivial one.
\end{solution}



\begin{solution}[by \href{https://artofproblemsolving.com/community/user/139996}{Faustus}]
	pco, thanks for guiding. $f(x)=0$ isn't a soln.
Can you show the proof and\/or how you landed up on $ \frac{x+|x|}{2} $
\end{solution}



\begin{solution}[by \href{https://artofproblemsolving.com/community/user/29428}{pco}]
	\begin{tcolorbox}pco, thanks for guiding. $f(x)=0$ isn't a soln.
Can you show the proof and\/or how you landed up on $ \frac{x+|x|}{2} $\end{tcolorbox}
I dont know if this is the unique solution but this is at least one solution.

$f(x)=x$ $\forall x\ge 0$ and $f(x)=0$ $\forall x\le 0$

$f(0)=0$ and so first constraint is OK

If $x\ge 0$ : $f(x)-x=f(-x)=0$
If $x\le 0$ : $f(x)-x=f(-x)=-x$
And so second constraint is OK

If $x\ge y\ge 0$ : $f(f(y)-x)=f(f(x)-y)+y-x=0$
If $y\ge x\ge 0$ : $f(f(y)-x)=f(f(x)-y)+y-x=y-x$
If $x\ge 0\ge y$ : $f(f(y)-x)=f(f(x)-y)+y-x=0$
If $y\ge 0\ge x$ : $f(f(y)-x)=f(f(x)-y)+y-x=y-x$
If $0\ge x$ and $0\ge y$ : $f(f(y)-x)=f(f(x)-y)+y-x=-x$
And so third constraint is OK


And thinking to this solution is a direct consequence of $f(f(x))=f(x)$ and $f(-f(x))=0$
\end{solution}



\begin{solution}[by \href{https://artofproblemsolving.com/community/user/29428}{pco}]
	In fact, I think that the only two solutions are $\frac{x+|x|}2$ and $\frac{x-|x|}2$

It's easy to check that these functions indeed are solutions but I'm still looking for a proof that they are the only one.
\end{solution}



\begin{solution}[by \href{https://artofproblemsolving.com/community/user/144544}{Oceansoul}]
	It has taken a while, but it should be all right.
Let $D$ be set of values of $f$. Of course for $x \in D: f(x)=x$ and $f(-x)=0$. From third condition we have:
 $f(f(y)+x)=f(f(-x)-y)+y+x$
 So for $x,y \in D: f(x+y)=x+y$. In particullary for $x \in D$ and $n \in N$ $nx \in D$. It means also, that $D \subset Q_-$ or $Q_+$. Because if $ \frac{m}{n} \in D$ and $\frac{-k}{l} \in D$ for natural $k,l$ then $mk \in D$ and $-mk \in D$ so one of fractions must be equal 0.

 \begin{bolded}1)\end{bolded} $D \subset Q_-$
 
 If for $x,y>0, f(x)=y$, $f(y)=y$, what denies $D \subset Q_-$.
 If $f(x)=-y$, then $f(x)$ and $f(-x)$ must be the same sign, so for some natural $m,n$:

  $f(x)= \frac{m}{n} f(-x)$ 

  $x+f(-x)=\frac{m}{n} f(-x)$

  $x=\frac{m-n}{n}f(-x)$

  $\frac{x}{f(-x)}=\frac{m-n}{n}<0$

  $\frac{m}{n}<1$ 
 
  $f(x)<f(-x)$
  
  $x<0$, what denies an assumption. So for sure $f(x)=0$, and $f(x)= \frac{x-|x|}{2}$.

 \begin{bolded}2)\end{bolded} $D \subset Q_+$
 For this cause, we have to notice that if function $f(x)$ satisfies conditions, also function $g(x)=-f(-x)$ does(quite easy to prove). So for any $x<0: f(x)=0$, and we have $f(x)= \frac{x+|x|}{2}$.
\end{solution}
*******************************************************************************
-------------------------------------------------------------------------------

\begin{problem}[Posted by \href{https://artofproblemsolving.com/community/user/89198}{chaotic_iak}]
	Find all functions $f : \mathbb{R} \rightarrow \mathbb{R}$ such that
\[f(x+y) + f(x)f(y) = f(xy) + (y+1)f(x) + (x+1)f(y)\]
for all $x,y \in \mathbb{R}$.
	\flushright \href{https://artofproblemsolving.com/community/c6h470214}{(Link to AoPS)}
\end{problem}



\begin{solution}[by \href{https://artofproblemsolving.com/community/user/91362}{goldeneagle}]
	\begin{tcolorbox}Find all functions $f : \mathbb{R} \rightarrow \mathbb{R}$ such that
\[f(x+y) + f(x)f(y) = f(xy) + (y+1)f(x) + (x+1)f(y)\]
for all $x,y \in \mathbb{R}$.\end{tcolorbox}

Steps of my solution: (Is it correct?  :maybe: )

Let $P(x,y)$ be the assertion  $f(x+y) + f(x)f(y) = f(xy) + (y+1)f(x) + (x+1)f(y)$ 

$P(0,0) \Rightarrow f(0)=0 $ or $f(0)=2$
 
a) $f(0)=2$ : $P(x,0) \Rightarrow f(x)=x+2$  but it dosen't work!!

b)$f(0)=0$  Define $f(1)=c$ so $P(1,1) \Rightarrow f(2)=5c-c^2$
$p(2,2,) \Rightarrow c=0$ or $c=2$ or $c=3$ or$c=5$

i)$f(1)=0 \Rightarrow $ 
$P(x,1) \Rightarrow f(x+1)=3f(x) \Rightarrow f(n)=0 \forall n \in \mathbb{Z}$
$P(x,-1) \Rightarrow f(-x)= \frac 13 f(x)$ (*)
$P(x,n) \Rightarrow f(nx)= (3^n-n-1)f(x) \Rightarrow f(x)=0 \forall x \in \mathbb{Q}$ 
$P(x,2) \Rightarrow f(2x)=6f(x)$ (**)
$P(x,-x) , P(x,x) ,* ,**\Rightarrow f(x)=0 $

ii)$f(1)=2$ 
 $P(x,1) \Rightarrow f(x+1)=f(x)+2x+2 \Rightarrow f(n)=n(n+1) \forall n \in \mathbb{Z}$
$P(x,-1) \Rightarrow f(-x)= f(x-1) =2x+f(x)$ (*)
$P(x,2) \Rightarrow f(2x)=4f(x)-2x$ (**)
$P(x,-x) , P(x,x) ,* ,**\Rightarrow f(x)=x(x+1) $

iii)$f(1)=3$
$P(x-1,1) \Rightarrow f(x)=3x$

iv) $f(1)=5$
$P(1,1) \Rightarrow f(2)=0$ 
$P(x-1,1) \Rightarrow f(x)=5x-2f(x-1) \Rightarrow f(-1)=0$
$P(x,-1) \Rightarrow f(-x)=\frac{5x-f(x)}{2}$ (*)
$P(x,2) \Rightarrow f(2x)=f(x)-5x$ (**)

$P(x,-x) , P(x,x) ,* , ** \Rightarrow f(x)(3x-1) = 5x(x+1) $ but $x=\frac 13 $ is contradictiuon!
So answers are $f(x)=0, f(x)=x(x+1), f(x)=3x$

\end{solution}



\begin{solution}[by \href{https://artofproblemsolving.com/community/user/29428}{pco}]
	\begin{tcolorbox}Find all functions $f : \mathbb{R} \rightarrow \mathbb{R}$ such that
\[f(x+y) + f(x)f(y) = f(xy) + (y+1)f(x) + (x+1)f(y)\]
for all $x,y \in \mathbb{R}$.\end{tcolorbox}
Let $P(x,y)$ be the assertion $f(x+y)+f(x)f(y)=f(xy)+(y+1)f(x)+(x+1)f(y)$
Let $f(0)=a$
Let $f(1)=b$

$P(0,0)$ $\implies$ $a(a-2)=0$
If $a=2$, then $P(x,0)$ $\implies$ $f(x)=x+2$ which is not a solution.
So $a=0$

$P(2,2)$ $\implies$ $f(2)(f(2)-6)=0$ and so $f(2)\in\{0,6\}$
$P(1,1)$ $\implies$ $f(2)=5b-b^2$ and so $b\in\{0,2,3,5\}$

1) $b=0$
$P(x,1)$ $\implies$ $f(x+1)=3f(x)$
Comparing then $P(x,y)$ and $P(x,y+1)$, we get $f(xy+x)=3f(xy)+(2y+1)f(x)$ which may be written $f(u+v)=3f(u)+(2\frac uv+1)f(v)$ $\forall u,v\ne 0$
Swiching $u,v$, we get $3f(u)+(2\frac uv+1)f(v)=3f(v)+(2\frac vu +1)f(u)$ and so $\frac{f(v)}v=-\frac{f(u)}u$ $\forall u\ne v$ and $u,v\ne 0$
Hence $f(x)=0$ $\forall x$ which indeed is a solution.

2) $b=2$
$P(x,1)$ $\implies$ $f(x+1)=f(x)+2x+2$
Comparing then $P(x,y)$ and $P(x,y+1)$, we get $f(xy+x)=f(xy)+f(x)(2y+1)-2xy$ which may be written $f(u+v)=f(u)+f(v)(2\frac uv+1)-2u$ $\forall u,v\ne 0$
Swiching $u,v$, we get $f(u)+f(v)(2\frac uv+1)-2u$ $=f(v)+f(u)(2\frac vu+1)-2v$ and so $\frac{f(v)}{v^2}-\frac 1v=\frac{f(u)}{u^2}-\frac 1u$ $\forall u,v\ne 0$
And so $f(x)=x^2+x$ which indeed is a solution

3) $b=3$
$P(x-1,1)$ $\implies$ $f(x)=3x$ which indeed is a solution

4) $b=5$
$P(-1,1)$ $\implies$ $f(-1)=0$
$P(x,1)$ $\implies$ $f(x+1)=-2f(x)+5x+5$
Comparing then $P(x,y)$ and $P(x,y+1)$, we get $f(xy+x)=-2f(xy)+(2y+1)f(x)-5xy$
Setting there $x=y=-1$, we get $f(0)=-2f(1)-5$ which is wrong.
So not a solution.

Hence the answers :
$f(x)=0$ $\forall x$
$f(x)=3x$ $\forall x$
$f(x)=x^2+x$ $\forall x$


\begin{bolded}edit \end{underlined}\end{bolded}: too late ! :)
\end{solution}



\begin{solution}[by \href{https://artofproblemsolving.com/community/user/91362}{goldeneagle}]
	\begin{tcolorbox}

\begin{bolded}edit \end{underlined}\end{bolded}: too late ! :)\end{tcolorbox}

Thanks PCO...Now i am sure about my solution!  :D
\end{solution}
*******************************************************************************
-------------------------------------------------------------------------------

\begin{problem}[Posted by \href{https://artofproblemsolving.com/community/user/89198}{chaotic_iak}]
	Let $P$ be a polynomial with real coefficients. Find all functions $f : \mathbb{R} \rightarrow \mathbb{R}$ such that there exists a real number $t$ such that
\[f(x+t) - f(x) = P(x)\]
for all $x \in \mathbb{R}$.
	\flushright \href{https://artofproblemsolving.com/community/c6h470219}{(Link to AoPS)}
\end{problem}



\begin{solution}[by \href{https://artofproblemsolving.com/community/user/29428}{pco}]
	\begin{tcolorbox}Let $P$ be a polynomial with real coefficients. Find all functions $f : \mathbb{R} \rightarrow \mathbb{R}$ such that there exists a real number $t$ such that
\[f(x+t) - f(x) = P(x)\]
for all $x \in \mathbb{R}$.\end{tcolorbox}
Choose any $t\ne 0$ and define $f(x)$ as any function you want over $[0,t)$ and by the induction formulas :
$f(x+t)=f(x)+P(x)$
$f(x-t)=f(x)-P(x)$

And you get all the solutions, built piece per piece.
\end{solution}



\begin{solution}[by \href{https://artofproblemsolving.com/community/user/96532}{dgrozev}]
	\begin{tcolorbox}
Choose any $t\ne 0$ and define $f(x)$ as any function you want over $[0,t)$ and by the induction formulas :
$f(x+t)=f(x)+P(x)$
$f(x-t)=f(x)-P(x)$

And you get all the solutions, built piece per piece.\end{tcolorbox}
I think it is just a typo, but it should be:
...
$f(x-t)=f(x)-P(x-t)$.
\end{solution}
*******************************************************************************
-------------------------------------------------------------------------------

\begin{problem}[Posted by \href{https://artofproblemsolving.com/community/user/132519}{siavosh}]
	Find all Functions $ f:\mathbb{R}\rightarrow\mathbb{R} $ such that 
$f(x^2+f(y))=f(x)+y$
	\flushright \href{https://artofproblemsolving.com/community/c6h470560}{(Link to AoPS)}
\end{problem}



\begin{solution}[by \href{https://artofproblemsolving.com/community/user/29428}{pco}]
	\begin{tcolorbox}Find all Functions $ f:\mathbb{R}\rightarrow\mathbb{R} $ such that 
$f(x^2+f(y))=f(x)+y$\end{tcolorbox}
Let $P(x,y)$ be the assertion $f(x^2+f(y))=f(x)+y$

Let $v\le\frac 14$ and $v\ne f(0)$ 
Let $u=f(v-f(0))$
Let $t$ any real root of $x^2-x+v=0$ 

$P(0,v-f(0))$ $\implies$ $f(u)=v$ 
$P(t,u)$ $\implies$ $u=0$ and so $v=f(u)=f(0)$, impossible 

And so no solution for this equation.
\end{solution}



\begin{solution}[by \href{https://artofproblemsolving.com/community/user/92753}{WakeUp}]
	Another idea:

$f$ is surjective so suppose $f(a)=0$. Then $P(0,a)\implies f(0)=f(0)+a$ i.e. $a=0$. Also $P(x,a)\implies f(x^2)=f(x)$ so clearly $f(x)=f(-x)$. Then $P(0,x)\implies f(f(x))=x$.

But then $x=f(f(x))=f(f(-x))=-x$, contradiction.
\end{solution}



\begin{solution}[by \href{https://artofproblemsolving.com/community/user/74510}{filipbitola}]
	Or how about:
$f$ is injective(obvious).
$y=0 \implies f(x^{2}+f(0))=f(x) \implies x^{2}-x$ is constant. Contradiction, so no solutions
\end{solution}
*******************************************************************************
-------------------------------------------------------------------------------

\begin{problem}[Posted by \href{https://artofproblemsolving.com/community/user/133488}{Nostalgius}]
	Find all function $: [0,\infty) \rightarrow [0,\infty)$ such that $ f(f(x)+y) = f(x) + f(y+1) $ for all $ x,y \ge 0 $
	\flushright \href{https://artofproblemsolving.com/community/c6h470638}{(Link to AoPS)}
\end{problem}



\begin{solution}[by \href{https://artofproblemsolving.com/community/user/29428}{pco}]
	\begin{tcolorbox}Find all function $: \mathbb{R^+} \rightarrow \mathbb{R^+}$ such that $ f[f(x)+y] = f(x) + f(y+1)) $ for all $ x,y $ in positive real\end{tcolorbox}
I suppose that brackets in LHS are parenthesis and that rightmost parenthesis is a typo. If so :

Let $P(x,y)$ be the assertion $f(f(x)+y)=f(x)+f(y+1)$

If $f(x)<x$ for some $x$, then $P(x,x-f(x))$ $\implies$ $f(x+1-f(x))=0$, impossible. So $f(x)\ge x$ $\forall x$

Let $x_i>0$

$P(x_1,f(x_2))$ $\implies$ $f(f(x_1)+f(x_2))=f(x_1)+f(f(x_2)+1)$ $=f(x_1)+f(x_2)+f(2)$
$P(x_1,f(x_2)+f(x_3))$ $\implies$ $f(f(x_1)+f(x_2)+f(x_3))$ $=f(x_1)+f(f(x_2)+f(x_3)+1)$ $=f(x_1)+f(x_2)+f(f(x_3)+2)$ $=f(x_1)+f(x_2)+f(x_3)+f(3)$

And it's easy to show that $f(\sum_{i=1}^n f(x_i))=\sum_{i=1}^n f(x_i)+f(n)$ $\forall n>1$

So $f(\sum_{i=1}^n f(x_i)+f(n)+f(y))=\sum_{i=1}^n f(x_i)+f(n)+f(y)+f(n+2)$

But $f(\sum_{i=1}^n f(x_i)+f(n)+f(y))$ $=f(f(\sum_{i=1}^n f(x_i))+f(y))$ $=f(\sum_{i=1}^n f(x_i))+f(y)+f(2)$ $=\sum_{i=1}^n f(x_i)+f(n)+f(y)+f(2)$

And so $f(n+2)=f(2)$ $\forall n>1$, in contradiction, when $n$ great enough (choose $n\ge f(2)-2$), with $f(x)\ge x$ $\forall x$ 

So no solution to this functional equation
\end{solution}



\begin{solution}[by \href{https://artofproblemsolving.com/community/user/133488}{Nostalgius}]
	\begin{bolded}@pco\end{bolded} sorry very much . I've made a mistake typo on the problem , positive real must be replace by non-negative real .
\end{solution}



\begin{solution}[by \href{https://artofproblemsolving.com/community/user/29428}{pco}]
	\begin{tcolorbox}Find all function $: [0,\infty) \rightarrow [0,\infty)$ such that $ f[f(x)+y] = f(x) + f(y+1) $ for all $ x,y \ge 0 $\end{tcolorbox}
Let $P(x,y)$ be the assertion $f(f(x)+y)=f(x)+f(y+1)$

1) $\exists u$ such that $f(u)=0$
====================
(Maybe this part could be shorter but I used a part from my previous post and did not look for shorter proof)
Let $x_i>0$

$P(x_1,f(x_2))$ $\implies$ $f(f(x_1)+f(x_2))=f(x_1)+f(f(x_2)+1)$ $=f(x_1)+f(x_2)+f(2)$
$P(x_1,f(x_2)+f(x_3))$ $\implies$ $f(f(x_1)+f(x_2)+f(x_3))$ $=f(x_1)+f(f(x_2)+f(x_3)+1)$ $=f(x_1)+f(x_2)+f(f(x_3)+2)$ $=f(x_1)+f(x_2)+f(x_3)+f(3)$

And it's easy to show that $f(\sum_{i=1}^n f(x_i))=\sum_{i=1}^n f(x_i)+f(n)$ $\forall n>1$

So $f(\sum_{i=1}^n f(x_i)+f(n)+f(y))=\sum_{i=1}^n f(x_i)+f(n)+f(y)+f(n+2)$

But $f(\sum_{i=1}^n f(x_i)+f(n)+f(y))$ $=f(f(\sum_{i=1}^n f(x_i))+f(y))$ $=f(\sum_{i=1}^n f(x_i))+f(y)+f(2)$ $=\sum_{i=1}^n f(x_i)+f(n)+f(y)+f(2)$

And so $f(n+2)=f(2)$ $\forall n>1$.

Choosing then $n\ge f(2)-2$, we get $f(n+2)\le n+2$ and then $P(n+2,n+2-f(n+2))$ $\implies$ $f(n+3-f(n+2))=0$ 
Q.E.D

2) $f(x+1)=f(x)$ and so $P(x,y)$ may be written as equivalent assertion $Q(x,y)$ : $f(f(x)+y)=f(x)+f(y)$
===============================================================================
Let $u$ such that $f(u)=0$
$P(u,x)$ $\implies$ $f(x)=f(x+1)$
Q.E.D.

3) $f(x)=0$ $\forall x\ge 0$
================
Let $x,y\ge 0$ and positive integers $n\ge f(x)-y$ and $m\ge f(y)-x$
$Q(x,y+n-f(x))$ $\implies$ $f(y+n)=f(x)+f(y+n-f(x))\ge f(x)$ and since $f(y+n)=f(y)$, we get $f(y)\ge f(x)$
$Q(y,x+m-f(y))$ $\implies$ $f(x+m)=f(y)+f(x+m-f(y))\ge f(y)$ and since $f(x+m)=f(x)$, we get $f(x)\ge f(y)$
So $f(x)=f(y)$ and $f(x)=c$ is constant
Plugging this back in original equation, we get $\boxed{f(x)=0}$ $\forall x$ which indeed is a solution.
\end{solution}



\begin{solution}[by \href{https://artofproblemsolving.com/community/user/133488}{Nostalgius}]
	Another way to approach the first step in \begin{bolded}pco\end{bolded} solution 

By Contradiction , Assume that $f(x) > 0$  $\forall n \geq 0 $ 
if there's $x_0$ that $f(x_0) \leq x_0 $ assert y by $x_0-f(x_0)$ x by $x_0$ in the equation , we get $f(x_o-f(x_0)+1) = 0$ Contradiction!
So $\forall x\geq 0 $ $x \geq f(x)$ $-(1)$

next, assert y by 0 , we get $f(f(x))=f(x)+f(1)$
assert x by $f(x)$ ,we get  $f(f(f(x))+y)=f(f(x))+f(y+1)$ $-(2)$
put $(1)$ into $(2)$ so $f(f(x)+f(1)+y)=f(x)+f(1)+f(y+1)$
but from asserting y by $f(1)+y$ , we get $f(f(x)+f(1)+y)=f(x)+f(f(1)+y+1)$
      and asserting x by 1 y by y+1 , we get $f(f(1)+y+1)=f(1)+f(y+2)$ 
we get $f(f(x)+f(1)+y)=f(x)+f(1)+f(y+2)$
so $f(y+1)=f(y+2) \forall y \geq 0$ and by easy induction $f(y+n)=f(y+1)$ $\forall n \in \mathbb{N}$ 
see $f(y+\left\lceil y+1\right\rceil) = f(y+1)$ but $f(y+\left\lceil y+1\right\rceil)>y+\left\lceil y+1\right\rceil>\left\lceil y+1\right\rceil$ contradiction ! 
so there's $x_0$ that $f(x_0)=0$
\end{solution}



\begin{solution}[by \href{https://artofproblemsolving.com/community/user/29126}{MellowMelon}]
	Now that you've added 0 back in, it seems there is at least one family of strange discontinuous solutions. Take a Hamel basis of $\mathbb{R}\/\mathbb{Q}$: $b_1 = 1, b_2, b_3, \ldots$. Set $f(b_1) = 0$ and $f(b_i) \in \{0, b_i\}$ (arbitrary choices allowed) for $i > 1$, and specify that $f$ is additive. This guarantees that $f(f(x)) = f(x)$ and $f(1) = 0$, so this satisfies the original equation:
\[f(f(x) + y) = f(f(x)) + f(y) = f(x) + f(y) + f(1) = f(x) + f(y+1).\]

EDIT: Woops, wrong; dropped the condition on the range. Thanks pco.
\end{solution}



\begin{solution}[by \href{https://artofproblemsolving.com/community/user/29428}{pco}]
	\begin{tcolorbox}Now that you've added 0 back in, it seems there is at least one family of strange discontinuous solutions. Take a Hamel basis of $\mathbb{R}\/\mathbb{Q}$: $b_1 = 1, b_2, b_3, \ldots$. Set $f(b_1) = 0$ and $f(b_i) \in \{0, b_i\}$ (arbitrary choices allowed) for $i > 1$, and specify that $f$ is additive. This guarantees that $f(f(x)) = f(x)$ and $f(1) = 0$, so this satisfies the original equation:
\[f(f(x) + y) = f(f(x)) + f(y) = f(x) + f(y) + f(1) = f(x) + f(y+1).\]\end{tcolorbox}
Unfortunately not : such a function can not have its image in $[0,+\infty)$ You'll have negative numbers.
\end{solution}
*******************************************************************************
-------------------------------------------------------------------------------

\begin{problem}[Posted by \href{https://artofproblemsolving.com/community/user/145315}{SMS}]
	Find the function: $f:{R^ + } \to {R^ + }$ satisfies : 
\[f\left( x \right).f\left( {yf\left( x \right)} \right) = f\left( {y + f\left( x \right)} \right);\forall x,y \in {R^ + }\]
	\flushright \href{https://artofproblemsolving.com/community/c6h470980}{(Link to AoPS)}
\end{problem}



\begin{solution}[by \href{https://artofproblemsolving.com/community/user/29428}{pco}]
	\begin{tcolorbox}Find the function: $f:{R^ + } \to {R^ + }$ satisfies : 
\[f\left( x \right).f\left( {yf\left( x \right)} \right) = f\left( {y + f\left( x \right)} \right);\forall x,y \in {R^ + }\]\end{tcolorbox}
Let $P(x,y)$ be the assertion $f(x)f(yf(x))=f(y+f(x))$

If $f(x)>1$ for some $x$, then $P(x,\frac{f(x)}{f(x)-1})$ $\implies$ $f(x)=1$, and so contradiction and $f(x)\le 1$ $\forall x$

So $2-f(2)>0$ and $P(2,2-f(2))$ $\implies$ $f(u)=1$ where $u=(2-f(2))f(2)$

Then $P(u,x)$ $\implies$ $f(x)=f(x+1)$
So, comparing $P(x,\frac y{f(x)})$ and $P(x,\frac y{f(x)}+1)$, we get $f(y)=f(y+f(x))$ and $P(x,y)$ becomes $f(x)f(yf(x))=f(y)$
Setting then $y=u$ in this equation, we get $\boxed{f(x)=1}$ $\forall x$ which indeed is a solution.
\end{solution}



\begin{solution}[by \href{https://artofproblemsolving.com/community/user/145315}{SMS}]
	I don't know this. 
\begin{tcolorbox}Setting then $y=u$ in this equation, we get $\boxed{f(x)=1}$ $\forall x$ which indeed is a solution.\end{tcolorbox}\end{tcolorbox}
I think : 
$y = u \to f\left( x \right)f\left( {uf\left( x \right)} \right) = f\left( u \right) \leftrightarrow f\left( x \right)f\left( {uf\left( x \right)} \right) = 1$
$ \Rightarrow f\left( x \right) = 1??$ How?
\end{solution}



\begin{solution}[by \href{https://artofproblemsolving.com/community/user/29428}{pco}]
	\begin{tcolorbox}I don't know this. 

I think : 
$y = u \to f\left( x \right)f\left( {uf\left( x \right)} \right) = f\left( u \right) \leftrightarrow f\left( x \right)f\left( {uf\left( x \right)} \right) = 1$
$ \Rightarrow f\left( x \right) = 1??$ How?\end{tcolorbox}
Since $f(t)\le 1$ $\forall t$, the equality $f(x)f(uf(x))=1$ is possible only if $f(x)=f(uf(x))=1$
\end{solution}



\begin{solution}[by \href{https://artofproblemsolving.com/community/user/108692}{MariusBocanu}]
	Maybe i've made a mistake, but writing $P(x,f(y))$ and $P(y,f(x))$ doesn't give us $f(x)=f(y)$ with x,y positive arbitrary chosen?
\end{solution}



\begin{solution}[by \href{https://artofproblemsolving.com/community/user/29428}{pco}]
	\begin{tcolorbox}Maybe i've made a mistake, but writing $P(x,f(y))$ and $P(y,f(x))$ doesn't give us $f(x)=f(y)$ with x,y positive arbitrary chosen?\end{tcolorbox}
No error at all ! You're right

Quite nice, simple and quick method to get the result $f(x)=1$ $\forall x$
Congrats :)
\end{solution}
*******************************************************************************
-------------------------------------------------------------------------------

\begin{problem}[Posted by \href{https://artofproblemsolving.com/community/user/104682}{momo1729}]
	Solve the following functional equation which is $f:\mathbb R\to\mathbb R$

$f\left(xf\left(y\right)\right) + f\left(f\left(x\right)+f\left(y\right)\right) = yf\left(x\right) + f\left(x+f\left(y\right)\right)$.
	\flushright \href{https://artofproblemsolving.com/community/c6h471789}{(Link to AoPS)}
\end{problem}



\begin{solution}[by \href{https://artofproblemsolving.com/community/user/29428}{pco}]
	\begin{tcolorbox}Solve the following functional equation which is $f:\mathbb R\to\mathbb R$
$f(xf(y))+f(f(x)+f(y))=yf(x)+f(x+f(y))$\end{tcolorbox}
Let $P(x,y)$ be the assertion $f(xf(y))+f(f(x)+f(y))=yf(x)+f(x+f(y))$
$f(x)=0$ $\forall x$ is a solution. So let us from now look only for non all-zero solutions and let then $u$ such that $f(u)\ne 0$

If $f(y_1)=f(y_2)$ for some $y_1,y_2$, then comparing $P(u,y_1)$ and $P(u,y_2)$ implies $y_1=y_2$ and so $f(x)$ is \begin{bolded}injective\end{bolded}.

$P(0,1)$ $\implies$ $f(f(0)+f(1))=f(f(1))$ and so, since injective, $f(0)=0$

$P(x,0)$ $\implies$ $f(f(x))=f(x)$ and so, since injective, $f(x)=x$ $\forall x$, which indeed is a solution

\begin{bolded}Hence the two solutions\end{underlined}\end{bolded} :
$f(x)=0$ $\forall x$
$f(x)=x$ $\forall x$
\end{solution}



\begin{solution}[by \href{https://artofproblemsolving.com/community/user/104682}{momo1729}]
	Thank you for the sol =)
I have come across another proof on [url=http://mathematicaltreasures.wordpress.com\/2012\/03\/22\/some-classical-problems\/]this blog[\/url].
I have trouble with this part :
\begin{tcolorbox}[img]http://s0.wp.com\/latex.php?latex=P%28%5Cfrac%7Bb%7D%7Bb-1%7D%2C+a%29%5Cimplies+af%28%5Cfrac%7Bb%7D%7Bb-1%7D%29%3Df%28f%28%5Cfrac%7Bb%7D%7Bb-1%7D%29%2Bf%28a%29%29%5Cimplies+f%28k%29%3D0&bg=ffffff&fg=333333&s=0[\/img] for some k\end{tcolorbox}

How do we have f(k)=0 for some k ? Thank you
\end{solution}



\begin{solution}[by \href{https://artofproblemsolving.com/community/user/29428}{pco}]
	\begin{tcolorbox}Thank you for the sol =)
I have come across another proof on [url=http://mathematicaltreasures.wordpress.com\/2012\/03\/22\/some-classical-problems\/]this blog[\/url].
I have trouble with this part :
\begin{tcolorbox}[img]http://s0.wp.com\/latex.php?latex=P%28%5Cfrac%7Bb%7D%7Bb-1%7D%2C+a%29%5Cimplies+af%28%5Cfrac%7Bb%7D%7Bb-1%7D%29%3Df%28f%28%5Cfrac%7Bb%7D%7Bb-1%7D%29%2Bf%28a%29%29%5Cimplies+f%28k%29%3D0&bg=ffffff&fg=333333&s=0[\/img] for some k\end{tcolorbox}

How do we have f(k)=0 for some k ? Thank you\end{tcolorbox}
I dont know. According to me, without more elements, this seems wrong.
\end{solution}



\begin{solution}[by \href{https://artofproblemsolving.com/community/user/104682}{momo1729}]
	Hello,
The full proof is here : http://mathematicaltreasures.wordpress.com\/2012\/03\/22\/some-classical-problems\/
Maybe there is some problem with it ?
\end{solution}



\begin{solution}[by \href{https://artofproblemsolving.com/community/user/82357}{applepi2000}]
	\begin{tcolorbox}Hello,
The full proof is here : http://mathematicaltreasures.wordpress.com\/2012\/03\/22\/some-classical-problems\/
Maybe there is some problem with it ?\end{tcolorbox}

I indeed omitted the case $f(0)=1$, which isn't hard to deal with:

[hide="This case"]
$P(0, 0)\implies f(2)+1=f(1)$, and $P(1, 0)\implies f(1)+f(f(1)+1)=f(2)\implies f(f(1)+1)=-1$.
But then we see that $P(0, 1)\implies -1=f(f(1))$ as well. But then we have that $P(a, f(1)), P(a, f(1)+1)$ with $f(a)\neq  0$ indeed gives $1=0$ (the same method \begin{bolded}pco\end{bolded} used above), a contradiction.  
[\/hide]

Else, let $a=0$. :)
\end{solution}
*******************************************************************************
-------------------------------------------------------------------------------

\begin{problem}[Posted by \href{https://artofproblemsolving.com/community/user/110012}{RealMatrik}]
	1. find $f:R\rightarrow R$ that satisfying the equation
\[f(x+y)+f(x-y)=2f(x)\cos y\]
2. find $f:Q\rightarrow R$ that satisfying the equation
\[f(x)+f(1-\frac{1}{x})=\log |x|;x\not=0,1\]
3. find $f:R\rightarrow R$ that satisfying the equation
\[f(x^2+f(y))=y+xf(x)\]
	\flushright \href{https://artofproblemsolving.com/community/c6h471903}{(Link to AoPS)}
\end{problem}



\begin{solution}[by \href{https://artofproblemsolving.com/community/user/115063}{PhantomR}]
	Does $\log x$ mean $\ln x $ or $\lg x$?
\end{solution}



\begin{solution}[by \href{https://artofproblemsolving.com/community/user/110012}{RealMatrik}]
	\begin{tcolorbox}Does $\log x$ mean $\ln x $ or $\lg x$?\end{tcolorbox}
sorry I mean $\log$ is $\lg$
\end{solution}



\begin{solution}[by \href{https://artofproblemsolving.com/community/user/29428}{pco}]
	\begin{tcolorbox}1. find $f:R\rightarrow R$ that satisfying the equation
\[f(x+y)+f(x-y)=2f(x)\cos y\]\end{tcolorbox}
Let $P(x,y)$ be the assertion $f(x+y)+f(x-y)=2f(x)\cos y$

(a) : $P(x-\frac{\pi}2,\frac{\pi}2)$ $\implies$ $f(x)+f(x-\pi)=0$

(b) : $P(\frac{\pi}2,x-\frac{\pi}2)$ $\implies$ $f(x)+f(\pi-x)=2f(\frac{\pi}2)\sin x$

(c) : $P(0,x-\pi)$ $\implies$ $f(x-\pi)+f(\pi-x)=-2f(0)\cos x$

(a)+(b)-(c) : $f(x)=f(\frac{\pi}2)\sin x +f(0)\cos x$

And so $\boxed{f(x)=a\sin x+b\cos x}$ which indeed is a solution
\end{solution}



\begin{solution}[by \href{https://artofproblemsolving.com/community/user/29428}{pco}]
	\begin{tcolorbox}3. find $f:R\rightarrow R$ that satisfying the equation
\[f(x^2+f(y))=y+xf(x)\]\end{tcolorbox}
Let $P(x,y)$ be the assertion $f(x^2+f(y))=y+xf(x)$

If $f(y_1)=f(y_2)$ for some $y_1,y_2$, then comparaison of $P(0,y_1)$ and $P(0,y_2)$ implies $y_1=y_2$ and so $f(x)$ is injective.

$P(0,x)$ $\implies$ $f(f(x))=x$

Comparaison of $P(x,0)$ and $P(f(x),0)$ implies $f(x^2+f(0))=f(f(x)^2+f(0))$ and so, since injective, $f^2(x)=x^2$

So, $\forall x$, either $f(x)=x$, either $f(x)=-x$

Suppose now $\exists x,y\ne 0$ such that $f(x)=x$ and $f(y)=-y$ : $P(x,y)$ $\implies$ $f(x^2-y)=y+x^2$ and so :
either $x^2-y=y+x^2$ and so $y=0$, impossible.
either $-x^2+y=y+x^2$ and so $x=0$, impossible.

So either $\boxed{f(x)=x}$ $\forall x$, either $\boxed{f(x)=-x}$ $\forall x$ which both are indeed solutions.
\end{solution}



\begin{solution}[by \href{https://artofproblemsolving.com/community/user/29428}{pco}]
	\begin{tcolorbox}2. find $f:Q\rightarrow R$ that satisfying the equation
\[f(x)-f(1-\frac{1}{x})=\log \abs{x};x\not=0,1\]\end{tcolorbox}
Functional equation can never be true for $x<0$
So no solution.
\end{solution}



\begin{solution}[by \href{https://artofproblemsolving.com/community/user/110012}{RealMatrik}]
	\begin{tcolorbox}[quote="RealMatrik"]2. find $f:Q\rightarrow R$ that satisfying the equation
\[f(x)-f(1-\frac{1}{x})=\log \abs{x};x\not=0,1\]\end{tcolorbox}
Functional equation can never be true for $x<0$
So no solution.\end{tcolorbox}
I'm sorry. I forgot absolute function in RHS. :(
\end{solution}



\begin{solution}[by \href{https://artofproblemsolving.com/community/user/110012}{RealMatrik}]
	sorry again I gave wrong equation to you
and the new equation I have gotten its solution
thx a lot for your solution
\end{solution}



\begin{solution}[by \href{https://artofproblemsolving.com/community/user/29428}{pco}]
	\begin{tcolorbox}2. find $f:Q\rightarrow R$ that satisfying the equation
\[f(x)+f(1-\frac{1}{x})=\log |x|;x\not=0,1\]\end{tcolorbox}
Let $P(x)$ be the assertion $f(x)+f(\frac{x-1}x)=\lg|x|$

(a) : $P(x)$ $\implies$ $f(x)+f(\frac{x-1}x)=\lg|x|$ $\forall x\ne 0,1$

(b) : $P(\frac{x-1}x)$ $\implies$ $f(\frac{x-1}x)+f(\frac 1{1-x})=\lg|x-1|-\lg|x|$ $\forall x\ne 0,1$

(c) : $P(\frac 1{1-x})$ $\implies$ $f(\frac 1{1-x})+f(x)=-\lg|x-1|$

(a)-(b)+(c) : $\boxed{f(x)=\lg|x|-\lg|x-1|}$ $\forall x\ne 0,1$ which indeed is a solution (and define $f(0)$ and $f(1)$ as you want).
\end{solution}
*******************************************************************************
-------------------------------------------------------------------------------

\begin{problem}[Posted by \href{https://artofproblemsolving.com/community/user/124126}{Rivero}]
	Find all functions $f: \mathbb{Z}\to \mathbb{Z}$ such that:
a) $f(0)=2$;
b) $f(x+f(x+2y))=f(2x)+f(2y)$ for all integer $x,y$.
	\flushright \href{https://artofproblemsolving.com/community/c6h471949}{(Link to AoPS)}
\end{problem}



\begin{solution}[by \href{https://artofproblemsolving.com/community/user/122637}{Diehard}]
	This looks more like an AIME problem. We easily see that $f(2)=4$ and that $f(x+2y)-x|f(2y)$. Setting $y=1$ gives $f(x+2)-x|f(2)=4$. Let $g(x)=f(x+2)-x$ and suppose $g(x)$ is not constant. Then by a well-known theorem of Schur, the set of primes dividing at least one of $g(1), g(2),...$ is infinite. Hence $g(x)=g(0)=4$ and $f(x)=x+2$.
\end{solution}



\begin{solution}[by \href{https://artofproblemsolving.com/community/user/122611}{oty}]
	I did a mistake ^^  .
\end{solution}



\begin{solution}[by \href{https://artofproblemsolving.com/community/user/64716}{mavropnevma}]
	Oh, oh ... so from $f(f(2x))=f(2x)+2$ for all $x\in \mathbb{Z}$ you infer $f(X)=X+2$ for all $X\in \mathbb{Z}$, as if you knew that all integers $X$ can be written as some $X=f(2x)$ ? Wherefrom this premonition ?
\end{solution}



\begin{solution}[by \href{https://artofproblemsolving.com/community/user/29126}{MellowMelon}]
	\begin{tcolorbox}We easily see ... that $f(x+2y)-x|f(2y)$.\end{tcolorbox}
This is not at all easy for me to see, which may or may not be my fault.

I'll work from oty's solution getting $f(2x) = 2x+2$ since that's the easy part anyway (EDIT: Derp, he edited it out; see pco's solution below instead). Suppose $f(m)$ is even for some odd $m$. Plug in $x = -f(m)+m, y = f(m)\/2$ to get $f(m) = -f(m)+2m+4$, or $f(m) = m+2$, contradiction of $f(m)$ even. So $f(m)$ is odd for all odd $m$, and $x = m, y = 0$ gives $f(m+f(m)) = 2m+4$. Since $m+f(m)$ is even, $m+f(m)+2 = 2m+4$, so $f(m) = m+2$. So we get $f(x) = x+2$ for all $x$.
\end{solution}



\begin{solution}[by \href{https://artofproblemsolving.com/community/user/122637}{Diehard}]
	\begin{tcolorbox}[quote="Diehard"]We easily see ... that $f(x+2y)-x|f(2y)$.\end{tcolorbox}
This is not at all easy for me to see, which may or may not be my fault.

\end{tcolorbox}

$f(x+2y)-x|f(x+f(x+2y))-f(2x)=f(2y)$.
\end{solution}



\begin{solution}[by \href{https://artofproblemsolving.com/community/user/122611}{oty}]
	Dear , \begin{bolded}mavropnevma\end{bolded} , $f$ is surjective .
\end{solution}



\begin{solution}[by \href{https://artofproblemsolving.com/community/user/29126}{MellowMelon}]
	Diehard: You seem to be assuming that $f$ is a polynomial or at least that $a - b \mid f(a) - f(b)$. But the problem doesn't give you that.

oty: How do you prove $f$ is surjective?
\end{solution}



\begin{solution}[by \href{https://artofproblemsolving.com/community/user/64716}{mavropnevma}]
	Dear \begin{bolded}oty\end{bolded}, even if you did prove, or at least mention, that $f$ was surjective, which you did not in your attempted proof, that only guarantees that any integer $X$ can be written as some $X = f(x)$, not  $X = f(2x)$.

In fact, when the answer is $f(x) = x+2$ for all $x$, your claim above is false; just try writing $3=f(2x)$ for some $x$ ...
\end{solution}



\begin{solution}[by \href{https://artofproblemsolving.com/community/user/29428}{pco}]
	\begin{tcolorbox}Find all functions $f: \mathbb{Z}\to \mathbb{Z}$ such that:
a) $f(0)=2$;
b) $f(x+f(x+2y))=f(2x)+f(2y)$ for all integer $x,y$.\end{tcolorbox}
Let $P(x,y)$ be the assertion $f(x+f(x+2y))=f(2x)+f(2y)$

$P(0,0)$ $\implies$ $f(2)=4$
$P(0,1)$ $\implies$ $f(4)=6$
And a simple induction using $P(0,x)$ gives $f(2x)=2x+2$ $\forall x\ge 0$

Let $x\ge 0$ : 
$P(-2x,2x)$ $\implies$ $f(-4x)=2-4x$
$P(-2x,x)$ $\implies$ $f(2-2x)=4-2x$
And so $f(x)=x+2$ $\forall$ even integer $x$ and $P(x,y)$ becomes $f(x+f(x+2y))=2x+2y+4$

If $f(x)$ is even for some odd $x$, then :
$P(x-f(x),\frac{f(x)}2)$ $\implies$ $f(x)=2x-f(x)+4$ and so $f(x)=x+2$, impossible since $x+2$ is odd.

So $f(x)$ is odd for all odd $x$ and then :
$P(1,\frac{x-1}2)$ $\implies$ $f(1+f(x))=x+5$ and since $f(x)+1$ is even, we get $x+5=f(1+f(x))=1+f(x)+2$ and so $f(x)=x+2$

And so $\boxed{f(x)=x+2}$ $\forall x\in\mathbb Z$

*\begin{bolded}edited\end{underlined}\end{bolded}* : Sorry MellowMelon. I've just seen that this is in fact your own proof. :oops:
\end{solution}



\begin{solution}[by \href{https://artofproblemsolving.com/community/user/122611}{oty}]
	yes , you are right  , dear \begin{bolded}mavropnevma\end{bolded} , thanks for the correction .
\end{solution}



\begin{solution}[by \href{https://artofproblemsolving.com/community/user/10045}{socrates}]
	http://www.artofproblemsolving.com/Forum/viewtopic.php?p=2212201#p2212201
http://www.artofproblemsolving.com/Forum/viewtopic.php?f=36&t=421417
\end{solution}
*******************************************************************************
-------------------------------------------------------------------------------

\begin{problem}[Posted by \href{https://artofproblemsolving.com/community/user/62647}{bengal}]
	Let f(n) be the number of ones when we put n into binary form (for example, f(5) = 2 because 5 = 101 base 2, etc.). Find $\frac{f(1)}{2^1} + \frac{f(2)}{2^2} + \frac{f(3)}{2^3} +\cdots$
	\flushright \href{https://artofproblemsolving.com/community/c6h472207}{(Link to AoPS)}
\end{problem}



\begin{solution}[by \href{https://artofproblemsolving.com/community/user/29428}{pco}]
	\begin{tcolorbox}Let f(n) be the number of ones when we put n into binary form (for example, f(5) = 2 because 5 = 101 base 2, etc.). Find $\frac{f(1)}{2^1} + \frac{f(2)}{2^2} + \frac{f(3)}{2^3} +\cdots$\end{tcolorbox}
Let $g(x)=\sum_{k=0}^{+\infty}\frac{f(k)}{x^k}$ with $x>1$
Note that $f(k)\le k$ $\forall k$ and so $g(x)\le \sum_{k=0}^{+\infty}\frac{k}{x^k}$ $=\frac x{(x-1)^2}$

$g(x)=\sum_{k=0}^{+\infty}\left(\frac{f(2k)}{x^{2k}}+\frac{f(2k+1)}{x^{2k+1}}\right)$ $=\sum_{k=0}^{+\infty}\left(\frac{f(k)}{x^{2k}}+\frac{f(k)+1}{x^{2k+1}}\right)$

$g(x)=\frac{x+1}x\sum_{k=0}^{+\infty}\frac{f(k)}{x^{2k}}$ $+\sum_{k=0}^{+\infty}\frac{1}{x^{2k+1}}$

$g(x)=\frac{x+1}xg(x^2)+\frac x{x^2-1}$

$\frac{(x-1)g(x)}x=\frac{(x^2-1)g(x^2)}{x^2}+\frac 1{x+1}$

$\frac{(x-1)g(x)}x=\frac{(x^{2^n}-1)g(x^{2^n})}{x^{2^n}}+\sum_{k=0}^{n-1}\frac 1{x^{2^k}+1}$

But $0<\frac{(x-1)g(x)}x\le\frac 1{x-1}$ and so $\lim_{x\to+\infty}\frac{(x-1)g(x)}x=0$

Setting then $n\to+\infty$ in the previous equality, we get $\frac{(x-1)g(x)}x=\sum_{k=0}^{+\infty}\frac 1{x^{2^k}+1}$

And so $g(x)=\frac x{x-1}\sum_{k=0}^{+\infty}\frac 1{x^{2^k}+1}$

And so the required quantity is $g(2)=\boxed{2\sum_{k=0}^{+\infty}\frac 1{2^{2^k}+1}\sim 1.19212634423564...}$ and I dont know if there is a closed form for this sum.
\end{solution}
*******************************************************************************
-------------------------------------------------------------------------------

\begin{problem}[Posted by \href{https://artofproblemsolving.com/community/user/82678}{hEatLove}]
	Find all function $f:\mathbb{R}\rightarrow\mathbb{R}$ such that for all real numbers $x,y$ satisfying this equation
$f(x+f(y))=f(x)+\dfrac{xf(4y)}{8}+f(f(y))$
	\flushright \href{https://artofproblemsolving.com/community/c6h472223}{(Link to AoPS)}
\end{problem}



\begin{solution}[by \href{https://artofproblemsolving.com/community/user/29428}{pco}]
	\begin{tcolorbox}Find all function $f:\mathbb{R}\rightarrow\mathbb{R}$ such that for all real numbers $x,y$ satisfying this equation
$f(x+f(y))=f(x)+\dfrac{xf(4y)}{8}+f(f(y))$\end{tcolorbox}
$f(x)=0$ $\forall x$ is a solution. So let us from now look for non allzero solutions

Let $P(x,y)$ be the asserion $f(x+f(y))=f(x)+\frac{xf(4y)}8+f(f(y))$
Let $u$ such that $f(u)\ne 0$

$P(0,0)$ $\implies$ $f(0)=0$

Subtracting $P(f(x),u)$ from $P(f(u),x)$, we get $f(4x)=8cf(x)$ where $c=\frac{f(4u)}{8f(u)}$

$P(x,y)$ becomes then new assertion $Q(x,y)$ : $f(x+f(y))=f(x)+cxf(y)+f(f(y))$

$Q(f(x),x)$ $\implies$ $f(2f(x))=cf(x)^2+2f(f(x))$
$Q(2f(x),x)$ $\implies$ $f(3f(x))=f(2f(x))+2cf(x)^2+f(f(x))$
$Q(3f(x),x)$ $\implies$ $f(4f(x))=f(3f(x))+3cf(x)^2+f(f(x))$
Adding these three lines, we get $f(4f(x))=6cf(x)^2+4f(f(x))$ and since we have also $f(4f(x))=8cf(f(x))$, we get $2(2c-1)f(f(x))=3cf(x)^2$
$c=\frac 12$ is impossible since the we'd have $f(x)=0$ $\forall x$ and so we got $\boxed{f(f(x))=\frac{3c}{2(2c-1)}f(x)^2}$

$Q(-f(x),x)$ $\implies$ $0=f(-f(x))-cf(x)^2+f(f(x))$
$Q(-2f(x),x)$ $\implies$ $f(-f(x))=f(-2f(x))-2cf(x)^2+f(f(x))$
$Q(-3f(x),x)$ $\implies$ $f(-2f(x))=f(-3f(x))-3cf(x)^2+f(f(x))$
$Q(-4f(x),x)$ $\implies$ $f(-3f(x))=f(-4f(x))-4cf(x)^2+f(f(x))$
Adding these four lines, we get $f(-4f(x))=10cf(x)^2-4f(f(x))$
But first line gives $f(-f(x))=cf(x)^2-f(f(x))$ and so $f(-4f(x))=8cf(-f(x))=8c^2f(x)^2-8cf(f(x))$ and so $10cf(x)^2-4f(f(x))=8c^2f(x)^2-8cf(f(x))$ and so $2(2c-1)f(f(x))=c(4c-5)f(x)^2$
$c=\frac 12$ is impossible since the we'd have $f(x)=0$ $\forall x$ and so we got $\boxed{f(f(x))=\frac{c(4c-5)}{2(2c-1)}f(x)^2}$

Comparing the two values we got for $f(f(x))$, we get $3c=c(4c-5)$ $\iff$ $c(c-2)=0$

$c=0$ is impossible since we'd then have $f(4x)=0$ $\forall x$ and so $c=2$

So we have :
$f(4x)=16f(x)$
$f(f(x))=f(x)^2$ and $f(-f(x))=f(x)^2$

$Q(-f(y),x)$ $\implies$ $f(f(x)-f(y))=f(x)^2-2f(x)f(y)+f(y)^2=(f(x)-f(y))^2$

It remains to see that 

$Q(\frac x{2f(u)}-\frac{f(u)}2,u)$ $\implies$ $f(\frac x{2f(u)}-\frac{f(u)}2+f(u))-f(\frac x{2f(u)}-\frac{f(u)}2)=x$
And so any $x$ may be written as $x=f(a)-f(b)$

And since we previously got $f(f(a)-f(b))=(f(a)-f(b))^2$, we now have $f(x)=x^2$ $\forall x$ which indeed is a solution.

\begin{bolded}Hence the only two solutions\end{underlined}\end{bolded} :
$f(x)=0$ $\forall x$
$f(x)=x^2$ $\forall x$
\end{solution}
*******************************************************************************
-------------------------------------------------------------------------------

\begin{problem}[Posted by \href{https://artofproblemsolving.com/community/user/145455}{faithjeff1993}]
	let f(x) and g(x) be quadratic function with real coefficient, for x>0,if g(x) is a integer, then f(x) is a integer, prove that there exist integers M,N, such that f(x)=Mg(x)+N
	\flushright \href{https://artofproblemsolving.com/community/c6h472459}{(Link to AoPS)}
\end{problem}



\begin{solution}[by \href{https://artofproblemsolving.com/community/user/145455}{faithjeff1993}]
	who can help me?
\end{solution}



\begin{solution}[by \href{https://artofproblemsolving.com/community/user/29428}{pco}]
	\begin{tcolorbox}let f(x) and g(x) be quadratic function with real coefficient, for x>0,if g(x) is a integer, then f(x) is a integer, prove that there exist integers M,N, such that f(x)=Mg(x)+N\end{tcolorbox}
Nobody can help you since your problem is obviously wrong.

Choose as counter-example $f(x)=x^2+x+1$ and $g(x)=x^2$
\end{solution}



\begin{solution}[by \href{https://artofproblemsolving.com/community/user/69901}{dinoboy}]
	@pco

Notice that $x = \sqrt{2}$ has $g(x)$ is an integer while $f(x)$ is not, so your counterexample does not work.
\end{solution}



\begin{solution}[by \href{https://artofproblemsolving.com/community/user/29428}{pco}]
	@dinoboy

You are quite right.

Sorry both for this wrong post.  :blush:
\end{solution}



\begin{solution}[by \href{https://artofproblemsolving.com/community/user/69901}{dinoboy}]
	First of all the condition $x > 0$ is useless.  It only introduces the complication that there are finitely many exceptions when we consider all real $x$.

First remark multiplying $g(x)$ by $-1$ does not change the result, nor does adding an integer to it (applying the same transformations on $f$ has the same result of not changing the truthfulness of the result). Furthermore, if we change $f(x), g(x)$ both into $f(x-a), g(x-a)$ for a real number $a$ the truthfulness is not changed.

Hence we shift $g$ to be symmetric about the origin and be of the form $g(x) = ax^2 + b$ for $a,b \ge 0$ and we can assume $f$ is a quadratic with positive leading coefficient which has no real roots. It follows for all sufficiently large $c$ a positive integer, we have $g(\pm \sqrt{(c-b)\/a})$ are integers and hence $f(\pm \sqrt{(c-b)\/a})$ are integers as well. Let $x_c = \sqrt{(c-b)\/a}$ and $y_c = f(x_c), z_c = f(-x_c)$ when $c$ is big enough. Let $f(x) = dx^2 + ex + f$. Remark that $z_n - y_n$ is always an integer.
But $z_n - y_n = 2ex_n$. Hence for all $x_n$, we need $2ex_n$ to be an integer. But this is obviously false unless $e \neq 0$ because $2ex_{n+1} - 2ex_n$ approaches $0$ as $n$ grows large, hence $e=0$.

Therefore we have reduced the problem to $g(x) = ax^2 + b, f(x) = dx^2 + f$. We need for all big enough integers $c$ that $\frac{d(c-b)}{a} + f$ is an integer. This means $\frac{d}{a}$ is an integer. Hence it suffices for $\frac{-db}{a} + f$ is an integer. By shifting $f(x)$ appropriately we can get it to equal $0$, as $f = (d\/a) \cdot b$. It follows $f(x) = g(x) \cdot (d\/a)$ where $d\/a$ is an integer, and hence we are done.

EDIT : Thanks bzprules, fixed the typo.
\end{solution}



\begin{solution}[by \href{https://artofproblemsolving.com/community/user/145455}{faithjeff1993}]
	elegant!
many thx!
\end{solution}



\begin{solution}[by \href{https://artofproblemsolving.com/community/user/60729}{GlassBead}]
	Oh, you have one small typo:
\begin{tcolorbox}
But $z_n - y-n = 2ex_n$. Hence for all $x_n$, we need $2ex_n$ to be an integer.
\end{tcolorbox}
This should be $y_n$.
\end{solution}
*******************************************************************************
-------------------------------------------------------------------------------

\begin{problem}[Posted by \href{https://artofproblemsolving.com/community/user/94615}{Pedram-Safaei}]
	find all functions $f:\mathbb{R}_+\rightarrow \mathbb{R}_+$ such that for every three positive reals $x,y,z$ we have:
$f(x+y+f(y))+f(x+z+f(z))=f(2f(z))+f(2x)+f(2y)$
	\flushright \href{https://artofproblemsolving.com/community/c6h472497}{(Link to AoPS)}
\end{problem}



\begin{solution}[by \href{https://artofproblemsolving.com/community/user/29428}{pco}]
	\begin{tcolorbox}find all functions $f:\mathbb{R}_+\rightarrow \mathbb{R}_+$ such that for every three positive reals $x,y,z$ we have:
$f(x+y+f(y))+f(x+z+f(z))=f(2f(z))+f(2x)+f(2y)$\end{tcolorbox}
Here is a rather ugly proof. I hope someone will post a simpler one.

Let $P(x,y,z)$ be the assertion $f(x+y+f(y))+f(x+z+f(z))=f(2f(z))+f(2x)+f(2y)$

Comparing $P(x,y,z)$ and $P(x,z,y)$, we get $f(2f(z))-f(2z)=f(2f(y))-f(2y)$ and so $f(2f(x))=f(2x)+c$ for some real $c$

So $P(x,y,z)$ becomes new assertion $Q(x,y,z)$ : $f(x+y+f(y))+f(x+z+f(z))=f(2x)+f(2y)+f(2z)+c$

1) $x\le f(x)\le x+\frac c2$ $\forall x$
======================
If $f(x)<x$ for some $x$, $Q(x-f(x),x,x)$ $\implies$ $0=f(2x-2f(x))+c$, impossible. And so $f(x)\ge x$ $\forall x$

If $f(x)>x$ for some $x$, $Q(f(x)-x,x,x)$ $\implies$ $f(2f(x)-2x)=c$

As a consequence, if $f(x)>x$, and since $f(2f(x)-2x)\ge 2f(x)-2x$, we get $f(x)\le x+\frac c2$

And so $x\le f(x)\le x+\frac c2$ $\forall x$
Q.E.D.

2) $\forall x$, either $f(x)=x$, either $f(x)=x+\frac c3$, either $f(x)=x+\frac c2$
====================================================
Suppose that $\forall x$ such that $f(x)>x$, we have $x+a_n\le f(x)\le x+b_n$ for some real $a_n,b_n$
Let $x$ such that $f(x)>x$. From $f(2f(x)-2x)=c$, we get :
either $y=2f(x)-2x$ is such that $f(y)=y$ and so  $f(x)=x+\frac c2$
either $y$ is such that $f(y)>y$ and so $y+a_n\le f(y)\le y+b_n$ and so $2f(x)-2x+a_n\le c\le 2f(x)-2x+b_n$
And so $x+\frac{c-b_n}2\le f(x)\le x+\frac{c-a_n}2$

Starting from $a_0=0$ and $b_0=\frac c2$, we build the sequences $a_{n+1}=\frac{c-b_n}2$ and $b_{n+1}=\frac{c-a_n}2$
These two sequences are convergent towards $\frac c3$
Q.E.D

3) $f(x)=x$ $\forall x$
==============
If $c=0$, then $f(x)=x$ $\forall x$
If $c>0$, then :
If $f(x)=x$ for some $x$, then $f(2f(x))=f(2x)$ and since we previously got $f(2f(x))=f(2x)+c$, we need $c=0$, impossible.
So $f(x)=x+\frac c2$ or $f(x)=x+\frac c3$ $\forall x$
If $f(x)=x+\frac c2$ for some $x$, then $f(2f(x)-2x)=c$ becomes $f(c)=c$, impossible.
So $f(x)=x+\frac c3$ $\forall x$, which is not a solution if $c\ne 0$

Hence the result $\boxed{f(x)=x}$ $\forall x$, which indeed is a solution.
\end{solution}



\begin{solution}[by \href{https://artofproblemsolving.com/community/user/29126}{MellowMelon}]
	Hopefully I've avoided any computational errors in this. My solution mimics pco's exactly up to determining $x \leq f(x) \leq x + \frac{c}{2}$ (even had the same variable name), so I'll continue from there.

Note that $c \geq 0$. Let $g(x) = f(x) - x$, so that $g$ is a function from positive reals to $[0, c\/2]$. Converting the original equation to $g$'s:
\[f(x+y+f(y)) + f(x+z+f(z)) = c + f(2x) + f(2y) + f(2z)\]
\[g(x+y+f(y)) + g(x+z+f(z)) + 2x+y+z+f(y)+f(z) = c + g(2x) + g(2y) + g(2z) + 2x + 2y + 2z\]
\[g(x+y+f(y)) + g(x+z+f(z)) + 2x+2y+2z+g(y)+g(z) = c + g(2x) + g(2y) + g(2z) + 2x + 2y + 2z\]
\[g(x+y+f(y)) + g(x+z+f(z)) + g(y)+g(z) = c + g(2x) + g(2y) + g(2z) \qquad (\ast)\]
Suppose that $c > 0$. Let $M = \sup_{x \in \mathbb{R}^+} g(x)$ and note $c \geq 2M$. Let $a$ be such that $g(2a) = M - \epsilon$ where $0 < \epsilon < \frac{M}{3}$. Then letting $x = y = z = a$ in $(\ast)$,
\[2g(2a+f(a)) + 2g(a) \leq 4M < 5M - 3\epsilon \leq c + 3M - 3\epsilon = c + 3g(2a).\]
But $(\ast)$ claims the LHS and RHS of this expression are equal, contradiction. So $c = 0$, $g(x)$ is identically 0, and $f(x) = x$ for all $x$.

Needed a little bit of ugliness to deal with the possibility of $g$ not assuming a maximum value...
\end{solution}



\begin{solution}[by \href{https://artofproblemsolving.com/community/user/31919}{tenniskidperson3}]
	\begin{tcolorbox}If $f(x)<x$ for some $x$, $Q(x-f(x),x,x)$ $\implies$ $0=f(2x-2f(x))+c$, impossible.\end{tcolorbox}

It's not immediate to me why this is a contradiction... Could you explain further, please?
\end{solution}



\begin{solution}[by \href{https://artofproblemsolving.com/community/user/29428}{pco}]
	\begin{tcolorbox}[quote="pco"]If $f(x)<x$ for some $x$, $Q(x-f(x),x,x)$ $\implies$ $0=f(2x-2f(x))+c$, impossible.\end{tcolorbox}

It's not immediate to me why this is a contradiction... Could you explain further, please?\end{tcolorbox}
$f(2f(x))=f(2x)+c$ shows that $c\ge 0$ else choosing $x$ such that $f(2x)$ is near of $\inf\{f(x)\}$ would lead to contradiction.

Then $f(2x-2f(x))+c>0$
\end{solution}
*******************************************************************************
-------------------------------------------------------------------------------

\begin{problem}[Posted by \href{https://artofproblemsolving.com/community/user/103227}{shohvanilu}]
	Given a quadeatial ponimial such that for any $a,b,c \in R$ $ f(a)=bc$  $f(b)=ac$  $f(c)=ab$. Then  find the value of $f(a+b+c)$.
	\flushright \href{https://artofproblemsolving.com/community/c6h472601}{(Link to AoPS)}
\end{problem}



\begin{solution}[by \href{https://artofproblemsolving.com/community/user/29428}{pco}]
	\begin{tcolorbox}Given a quadeatial ponimial such that for any $a,b,c \in R$ $ f(a)=bc$  $f(b)=ac$  $f(c)=ab$. Then  find the value of $f(a+b+c)$.\end{tcolorbox}
http://www.artofproblemsolving.com/Forum/viewtopic.php?f=36&t=472572
Posted five hours ago
\end{solution}



\begin{solution}[by \href{https://artofproblemsolving.com/community/user/127783}{Sayan}]
	Just wondering is it a tst problem! because this one is very easy
\end{solution}



\begin{solution}[by \href{https://artofproblemsolving.com/community/user/124745}{Uzbekistan}]
	It it is Regionalny olinpiad problem
\end{solution}



\begin{solution}[by \href{https://artofproblemsolving.com/community/user/103227}{shohvanilu}]
	\begin{tcolorbox}It it is Regionalny olinpiad problem\end{tcolorbox}
aldama
\end{solution}



\begin{solution}[by \href{https://artofproblemsolving.com/community/user/9882}{Virgil Nicula}]
	[color=darkred]\begin{bolded}PP\end{underlined}.\end{bolded} Let $\{a,b,c\}\subset\mathbb R$ and let $f(x)$ be a quadratic polynomial for which  $f(a)=bc\ ,\ f(b)=ac\ ,\ f(c)=ab$ . Find $f(a+b+c)$ .[\/color]

\begin{bolded}Proof\end{underlined}.\end{bolded} Denote $\left\{\begin{array}{c}
a+b+c=s_1\\\
ab+bc+ca=s_2\\\
abc=s_3\end{array}\right\|$ . Observe that $f(a)=bc=s_2-a(b+c)=s_2-a(s_1-a)$ a.s.o. 

Consider the quadratic polynomial $g(x)=f(x)-s_2+x(s_1-x)$ for which $g(a)=g(b)=g(c)=0$ . 

Therefore, $g(x)\equiv 0$ , i.e. $f(x)=x^2-s_1x+s_2$ $\implies$ $f(a+b+c)=f(s_1)=s_2=ab+bc+ca$ . 

\begin{bolded}Remark\end{underlined}.\end{bolded} $xf(x)=(x-a)(x-b)(x-c)+abc$ .
\end{solution}



\begin{solution}[by \href{https://artofproblemsolving.com/community/user/127386}{Abdurasul}]
	Dear shohvanilu this problem's solution $f(x+y+z)=xy+yz+xz$
\end{solution}
*******************************************************************************
-------------------------------------------------------------------------------

\begin{problem}[Posted by \href{https://artofproblemsolving.com/community/user/122611}{oty}]
	Let $a\in\mathbb{R}-\{0\}$. Find all functions $f: \mathbb{R}\to\mathbb{R}$ such that $f(a+x) = f(x) - x$ for all $x\in\mathbb{R}$.
	\flushright \href{https://artofproblemsolving.com/community/c6h472699}{(Link to AoPS)}
\end{problem}



\begin{solution}[by \href{https://artofproblemsolving.com/community/user/31919}{tenniskidperson3}]
	Define $f$ on $(a, 0]$ or $[0, a)$ however you wish.  Then use $f(a+x)=f(x)-x$ to extend it to the whole real line.  These are the only solutions.  It's also easy to verify that there can be no contradiction in this definition of $f$.
\end{solution}



\begin{solution}[by \href{https://artofproblemsolving.com/community/user/29428}{pco}]
	Yes, tenniskidperson3's solution is the general "piece per piece" solution.

In this case, we can also give a general form for the solution :

$f(x)=h\left(a\left\{\frac xa\right\}\right)$ $-\left\lfloor\frac xa\right\rfloor$ $\left(x-\left\lfloor\frac {x+a}a\right\rfloor\frac a2\right)$ where $h(x)$ is any function.
\end{solution}



\begin{solution}[by \href{https://artofproblemsolving.com/community/user/122611}{oty}]
	\begin{tcolorbox}Yes, tenniskidperson3's solution is the general "piece per piece" solution.

In this case, we can also give a general form for the solution :

$f(x)=h\left(a\left\{\frac xa\right\}\right)$ $-\left\lfloor\frac xa\right\rfloor$ $\left(x-\left\lfloor\frac {x+a}a\right\rfloor\frac a2\right)$ where $h(x)$ is any function.\end{tcolorbox}
Thank's a lot Dear \begin{bolded}pco \end{bolded} and \begin{bolded}tenniskidperson3\end{bolded} for the help , can you show me how you got this result please ?
\end{solution}



\begin{solution}[by \href{https://artofproblemsolving.com/community/user/29428}{pco}]
	That's just an extension of tenniskidperson3's remarks :

$f(x+a)=f(x)-x$
$f(x+2a)=f(x+a)-x-a=f(x)-2x-a$
...
Simple induction gives $f(x+na)=f(x)-nx-n(n-1)\frac a2$ $=f(x)-n((x+na)-(n+1)\frac a2)$

Writing then $x=a\left\{\frac xa\right\}+\left\lfloor\frac xa\right\rfloor a$ and using the previous line, we get 

$f(x)=f(a\left\{\frac xa\right\})$ $-\left\lfloor\frac xa\right\rfloor(x-(\left\lfloor\frac xa\right\rfloor+1)\frac a2)$

And it remains (as tenniskidperson3 said) to see that $a\left\{\frac xa\right\}$ is in $[0,a)$ or $(a,0]$ depending on sign of $a$ and that the function may take any value we want over this interval to get the form I gave.
\end{solution}



\begin{solution}[by \href{https://artofproblemsolving.com/community/user/122611}{oty}]
	Thank's a lot Dear \begin{bolded}pco\end{bolded} :) .
\end{solution}
*******************************************************************************
-------------------------------------------------------------------------------

\begin{problem}[Posted by \href{https://artofproblemsolving.com/community/user/122611}{oty}]
	Find all functions $f:\mathbb{R}\rightarrow\mathbb{R}$ such that $f(x)^2=1+xf(x+1) ,\forall{x\in{\mathbb{R}}}$.
	\flushright \href{https://artofproblemsolving.com/community/c6h472850}{(Link to AoPS)}
\end{problem}



\begin{solution}[by \href{https://artofproblemsolving.com/community/user/29428}{pco}]
	\begin{tcolorbox}Find all function : $f:\mathbb{R}\rightarrow\mathbb{R}$ such that : $(f(x))^2=1+xf(x+1) ,\forall{x\in{\mathbb{R}}}$\end{tcolorbox}
There are infinitely many solutions to this equation which may easily be built in piece per piece way :

Let $h(x)$ from $(0,1]\to\mathbb R$ any function such that $h(x)\le \frac 1{1-x}$ $\forall x\in(0,1)$

1. Choose $f(x)=h(x)$ $\forall x\in (0,1]$
2. Build all values in $(1,+\infty)$ by successive applications of $f(x+1)=\frac{f(x)^2-1}x$
3. Choose $f(0)=-1$ or $f(0)=+1$ as you want.
4. Build all values in $[-n,1-n)$ for $n=1\to+\infty$ by successive applications of the following rule :
For $x\in[-n,1-n)$ : 
If $\sqrt{1+xf(x+1)}>\frac 1{1-x}$ then choose $f(x)=-\sqrt{1+xf(x+1)}$
If $\sqrt{1+xf(x+1)}\le\frac 1{1-x}$ then choose $f(x)=-\sqrt{1+xf(x+1)}$ or $f(x)=\sqrt{1+xf(x+1)}$ as you want.

Note that the construction for $x<0$ is made so that $1+xf(x+1)\ge 0$ is always true.
Note also that if you forgot the [quite useless] constraint $f(x)\in\mathbb R[X]$, then the only solution would be $f(x)=x+1$
\end{solution}
*******************************************************************************
-------------------------------------------------------------------------------

\begin{problem}[Posted by \href{https://artofproblemsolving.com/community/user/60735}{hatchguy}]
	A function $f$ is defined on the positive integers by $f(1) = 1$ and for $n>1$, \[ f(n) = f( \lfloor\frac{2n-1}{3} \rfloor ) +  f( \lfloor\frac{2n}{3} \rfloor ) \] Is it true that $f(n) - f(n-1) \le n$ for all $n>1$?

Here $\lfloor x \rfloor$ denotes the greatest integer less than or equal to $x$.
	\flushright \href{https://artofproblemsolving.com/community/c6h472924}{(Link to AoPS)}
\end{problem}



\begin{solution}[by \href{https://artofproblemsolving.com/community/user/29428}{pco}]
	\begin{tcolorbox}A function $f$ is defined on the positive integers by $f(1) = 1$ and for $n>1$, \[ f(n) = f( \lfloor\frac{2n-1}{3} \rfloor ) +  f( \lfloor\frac{2n}{3} \rfloor ) \] Is it true that $f(n) - f(n-1) \le n$ for all $n>1$?

Here $\lfloor x \rfloor$ denotes the greatest integer less than or equal to $x$.\end{tcolorbox}
$f(3n)=f(2n-1)+f(2n)$
$f(3n+1)=2f(2n)$
$f(3n+2)=2f(2n+1)$

Writing $g(n)=f(n)-f(n-1)$, this becomes :
$g(3n)=g(2n)$
$g(3n+1)=g(2n)$
$g(3n+2)=2g(2n+1)$

Concentrating on the last line, which increases the ratio $\frac{g(n)}n$, we easily get a counter-example :
$g(3^5-1)=32g(31)$ $=32g(20)=64g(12)$ $=64g(8)=128g(5)$ $=256g(3)=256>3^5-1$

And so the assertion is wrong.
\end{solution}
*******************************************************************************
-------------------------------------------------------------------------------

\begin{problem}[Posted by \href{https://artofproblemsolving.com/community/user/82334}{bappa1971}]
	Determine all continuous functions $f:\mathbb{R}_{+}\rightarrow \mathbb{R}_{+}$ satisfy:

$f(xf(x)+yf(y))=xf(y)+yf(x)$
	\flushright \href{https://artofproblemsolving.com/community/c6h473050}{(Link to AoPS)}
\end{problem}



\begin{solution}[by \href{https://artofproblemsolving.com/community/user/29428}{pco}]
	\begin{tcolorbox}Determine all continuous functions $f:\mathbb{R}_{+}\rightarrow \mathbb{R}_{+}$ satisfy:

$f(xf(x)+yf(y))=xf(y)+yf(x)$\end{tcolorbox}
Let $P(x,y)$ be the assertion $f(xf(x)+yf(y))=xf(y)+yf(x)$
Let $g(x)=2xf(x)$

1) $g(x)$ is injective
=============
If $af(a)=bf(b)$ for some $a,b$, then :
$P(a,a)$ $\implies$ $f(2af(a))=2af(a)$
$P(a,b)$ $\implies$ $f(af(a)+bf(b))=af(b)+bf(a)$
And so $af(b)+bf(a)=2af(a)=af(a)+bf(b)$ and so $(a-b)(f(a)-f(b))=0$ 
So $a=b$ or $f(a)=f(b)$ and so again $a=b$ since $af(a)=bf(b)$
Q.E.D.

2) $g(x)$ is surjective
===============
$g(x)$ is continuous and injective, so monotonous.
If $\lim_{x\to+\infty}g(x)=L$, then $\lim_{x\to+\infty}f(x)=0$ and setting $x\to+\infty$ in $P(x,y)$ leads to contradicton.
So $\lim_{x\to+\infty}g(x)=+\infty$ and $g(x)$ is increasing.
If $\lim_{x\to 0}g(x)=L>0$, then $\lim_{x\to 0}f(x)=+\infty$ and setting $x\to 0$ in $P(x,y)$ leads to contradicton.
So $\lim_{x\to 0}g(x)=0$
Q.E.D.

3) No solution to this equation
=====================
$P(x,x)$ $\implies$ $f(g(x))=g(x)$ and so $g(g(x))=2g(x)^2$ and so, since surjective, $g(x)=2x^2$ and so $f(x)=x$, which is not a solution.
Q.E.D.
\end{solution}
*******************************************************************************
-------------------------------------------------------------------------------

\begin{problem}[Posted by \href{https://artofproblemsolving.com/community/user/61626}{StefanS}]
	Find all functions $f : \mathbb{R} \to \mathbb{Z}$ which satisfy the conditions: 

$f(x+y) < f(x) + f(y)$

$f(f(x)) = \lfloor {x} \rfloor + 2$
	\flushright \href{https://artofproblemsolving.com/community/c6h473827}{(Link to AoPS)}
\end{problem}



\begin{solution}[by \href{https://artofproblemsolving.com/community/user/29428}{pco}]
	\begin{tcolorbox}Find all functions $f : \mathbb{R} \to \mathbb{Z}$ which satisfy the conditions: 

$f(x+y) < f(x) + f(y)$

$f(f(x)) = \lfloor {x} \rfloor + 2$\end{tcolorbox}
From $f(f(x))=\lfloor x\rfloor+2$, we get $f(\lfloor x\rfloor+2)=f(x)+2$

Let then $f(0)=a$ and $f(1)=b$ and we get $f(2n)=2n+a$ and $f(2n+1)=2n+b$
$f(f(0))=2$ $\implies$ $f(a)=2$ and so $a$ is odd and $a+b=3$

$f(0+0)<f(0)+f(0)$ $\implies$ $a>0$
$f(1+1)<f(1)+f(1)$ $\implies$ $3a<4$
So $a=1$ and $b=2$ and $f(x)=x+1$ $\forall x\in\mathbb Z$

$f(\lfloor x\rfloor+2)=f(x)+2$ implies then $f(x)=\lfloor x\rfloor+1$ $\forall x\in\mathbb R$ which is not a solution since for example this implies $f(\frac 32+\frac 32)=f(\frac 32)+f(\frac 32)$

So \begin{bolded}no solution\end{underlined}\end{bolded}.
\end{solution}



\begin{solution}[by \href{https://artofproblemsolving.com/community/user/60735}{hatchguy}]
	Why is there no solution?

If $f(x) = \lfloor x \rfloor +1$ then $f(f(x)) = f( \lfloor x \rfloor +1 ) = \lfloor  \lfloor x \rfloor +1 \rfloor + 1 = \lfloor x \rfloor + 2$

and $f(x+y) < f(x) + f(y)$ is also true.
\end{solution}



\begin{solution}[by \href{https://artofproblemsolving.com/community/user/29428}{pco}]
	\begin{tcolorbox}Why is there no solution?

If $f(x) = \lfloor x \rfloor +1$ then $f(f(x)) = f( \lfloor x \rfloor +1 ) = \lfloor  \lfloor x \rfloor +1 \rfloor + 1 = \lfloor x \rfloor + 2$

and $f(x+y) < f(x) + f(y)$ is also true.\end{tcolorbox}
Have you simply read my post ?

I gave you a simple counter-example : with $f(x)=\lfloor x\rfloor +1$, choose $x=y=\frac 32$ and $f(x+y)=f(x)+f(y)$ and so $f(x+y)<f(x)+f(y)$ is not true.
\end{solution}



\begin{solution}[by \href{https://artofproblemsolving.com/community/user/60735}{hatchguy}]
	Is there a way I could question your solution without reading your post? 

I miscalculated your counter example and forgot that the inequality $\{x \} + \{y \} < \{x+y \} +1$ actually had an equality case.
\end{solution}



\begin{solution}[by \href{https://artofproblemsolving.com/community/user/74510}{filipbitola}]
	Here is my solution:
Assume $a\in\mathbb{Z}, a<x<a+1$
We will prove that $f(x)=f(a)$
$f(f(x))=\lfloor x\rfloor +2 \implies f(f(a))=a+2 \implies f(f(f(a)))=f(a)+2 \implies f(a+2)=f(a)+2$
Replacing $x=f(x)$
$f(f(f(x))=f(x)+2$ since $f(x)\in\mathbb{Z}$
$f(a+2)=f(x)+2 \implies f(a)+2=f(x)+2 \implies f(a)=f(x)$

In the first equation, substitute $x=y=t+\frac{1}{2}, t\in\mathbb{Z}$
$f(2t+1)<2f(t+\frac{1}{2}=2f(t)$

$x=5.5, y=0.5 \implies f(6)<f(5,5)+f(0.5)=f(5)+f(0)<2f(2)+f(0)=3f(0)+4$
Since $f(a+2)=f(a)+2 \implies f(6)=f(4)+2=f(2)+4=f(0)+6$
$f(0)+6<3f(0)+4\implies f(0)>1$

$x=1.5, y=0.5 \implies f(2)<f(1)+f(0)\implies f(0)+2<f(1)+f(0) \implies f(1)>2$

Using $f(0)>1, f(1)>2$ and $f(x+2)=f(x)+2 \forall x\in\mathbb{Z}$ by trivial induction we get that $f(n)>n+1 \forall n\in\mathbb{N}$
Now, $2=f(f(0))>f(0)+1>2$
Hence, no solutions.
\end{solution}



\begin{solution}[by \href{https://artofproblemsolving.com/community/user/94192}{MahyarB}]
	we can easily prove that for all integers such as $x$: $f(x)=x+1$ but i have no solutions for non-integers....
\end{solution}



\begin{solution}[by \href{https://artofproblemsolving.com/community/user/127104}{Nevergiveupbtw}]
	I think the problem use this symbol about celling function then it will have solution
\end{solution}



\begin{solution}[by \href{https://artofproblemsolving.com/community/user/29428}{pco}]
	\begin{tcolorbox}I think the problem use this symbol about celling function then it will have solution\end{tcolorbox}
Dont hesitate to post this solution.
\end{solution}



\begin{solution}[by \href{https://artofproblemsolving.com/community/user/127104}{Nevergiveupbtw}]
	\begin{tcolorbox}[quote="Nevergiveupbtw"]I think the problem use this symbol about celling function then it will have solution\end{tcolorbox}
Dont hesitate to post this solution.\end{tcolorbox}
But I can't solve with this hypothesis, can you?
\end{solution}



\begin{solution}[by \href{https://artofproblemsolving.com/community/user/29428}{pco}]
	\begin{tcolorbox}I think the problem use this symbol about celling function then it will have solution
...
But I can't solve with this hypothesis, can you?\end{tcolorbox}
Are you joking ? Just read my first solution (with floor function) and adapt it for ceiling function. It is quite elementary :

From $f(f(x))=\lceil x\rceil+2$, we get $f(\lceil x\rceil+2)=f(x)+2$

Let then $f(0)=a$ and $f(1)=b$ and we get $f(2n)=2n+a$ and $f(2n+1)=2n+b$
$f(f(0))=2$ $\implies$ $f(a)=2$ and so $a$ is odd and $a+b=3$

$f(0+0)<f(0)+f(0)$ $\implies$ $a>0$
$f(1+1)<f(1)+f(1)$ $\implies$ $3a<4$
So $a=1$ and $b=2$ and $f(x)=x+1$ $\forall x\in\mathbb Z$

$f(\lceil x\rceil+2)=f(x)+2$ implies then $f(x)=\lceil x\rceil+1$ $\forall x\in\mathbb R$ which obviously is a solution.
\end{solution}



\begin{solution}[by \href{https://artofproblemsolving.com/community/user/96840}{ACCCGS8}]
	Honestly pco you don't have to be so fierce - you are brilliant and very fast at maths, but others take longer to understand solutions.
\end{solution}



\begin{solution}[by \href{https://artofproblemsolving.com/community/user/29428}{pco}]
	\begin{tcolorbox}Honestly pco you don't have to be so fierce - you are brilliant and very fast at maths, but others take longer to understand solutions.\end{tcolorbox}
I wrote the solution for the case $\lfloor x\rfloor$
There is no need to be brilliant or clever or anything else for at least \begin{bolded}try \end{underlined}\end{bolded}the same proof in the case where the statement is with $\lceil x\rceil$

And it's the same proof exactly up the line previous to the last one.

I'm sure Nevergiveupbtw did not try at all.

The interest of posting solutions is to allow other to learn from them. So a minimal effort is needed (at least read). It seems to me that Nevergiveupbtw  did not make this minimal effort.
\end{solution}



\begin{solution}[by \href{https://artofproblemsolving.com/community/user/110552}{youarebad}]
	\begin{tcolorbox}so $a$ is odd and $a+b=3$\end{tcolorbox}

I don't understand how do you get this. Can u explain it to me ?
\end{solution}



\begin{solution}[by \href{https://artofproblemsolving.com/community/user/29428}{pco}]
	\begin{tcolorbox}[quote="pco"]so $a$ is odd and $a+b=3$\end{tcolorbox}

I don't understand how do you get this. Can u explain it to me ?\end{tcolorbox}
$f(a)=2$
If $a$ is even, then, since $f(2n)=2n+a$, we get $f(a)=2a$ and so $a=1$, odd. So contradiction and $a$ is odd.
Since $a$ is odd, and since $f(2n+1)=2n+b$, we get $f(a)=(a-1)+b$ and so $2=a+b-1$ and so $a+b=3$
\end{solution}
*******************************************************************************
-------------------------------------------------------------------------------

\begin{problem}[Posted by \href{https://artofproblemsolving.com/community/user/111767}{yaphets}]
	Find all continuous functions in $[0;1]$ satisfy: $f(xf(x))=f(x)$  $\forall x\in [0;1]$
	\flushright \href{https://artofproblemsolving.com/community/c6h473913}{(Link to AoPS)}
\end{problem}



\begin{solution}[by \href{https://artofproblemsolving.com/community/user/29428}{pco}]
	\begin{tcolorbox}Find all continuous functions in $[0;1]$ satisfy: $f(xf(x))=f(x)$  $\forall x\in [0;1]$\end{tcolorbox}
Do you mean that $f(x)$ is a continuous fonction from $[0,1]\to[0,1]$ ?
\end{solution}



\begin{solution}[by \href{https://artofproblemsolving.com/community/user/111767}{yaphets}]
	\begin{tcolorbox}[quote="yaphets"]Find all continuous functions in $[0;1]$ satisfy: $f(xf(x))=f(x)$  $\forall x\in [0;1]$\end{tcolorbox}
Do you mean that $f(x)$ is a continuous fonction from $[0,1]\to[0,1]$ ?\end{tcolorbox}
yes, it is continuous from $[0,1]\to[0,1]$
\end{solution}



\begin{solution}[by \href{https://artofproblemsolving.com/community/user/29428}{pco}]
	\begin{tcolorbox}Find all continuous functions from $[0,1]\to[0,1]$ satisfy: $f(xf(x))=f(x)$  $\forall x\in [0;1]$\end{tcolorbox}
Ok. If so :

Let $g(x)=xf(x)$
$g(x)$ is a continuous function from $[0,1]\to[0,1]$ such that :
$g(0)=0$
$g(x)\le x$
and functional equation becomes $g(g(x))=\frac{g(x)^2}x$ $\forall x\in(0,1]$

$g(x)$ is injective and so monotonous on any interval where $g(x)>0$
So continuity implies that $\exists t\ge 0$ such that $g(x)=0$ $\forall x\in[0,t]$ and $g(x)>0$ and increasing over $(t,1]$


If $g(a)=b\in(0,a)$ for some $1\ge a>b>0$, let then the sequence $u_n=b^{n-1}a^{2-n}$ for $n\in\mathbb N$ : 
$u_n$ is a decreasing sequence whose limit is $0$ and $g(u_k)=u_{k+1})$
As a consequence $t=0$ and $g(x)$ is increasing over $[0,1]$

Let then $x\in(0,u_1=a]$ : $\exists k\in\mathbb N$ such that $u_{k+1}<x\le u_k$ and so $g^{[n]}(u_{k+1})<g^{[n]}(x)\le g^{[n]}(u_{k})$
$\iff$ $u_{n+k+1}<g^{[n]}(x)\le u_{n+k}$

Simple induction gives $g^{[n]}(x)=\frac{(g(x))^n}{x^{n-1}}$ and so we got $b^{n+k}a^{1-n-k}<\frac{(g(x))^n}{x^{n-1}}\le b^{n+k-1}a^{2-n-k}$

$\iff$ $b^{k}a^{1-k}<\left(\frac{ag(x)}{bx}\right)^nx\le b^{k-1}a^{2-k}$
           
Setting $n\to+\infty$, we get $g(x)=\frac ba x$ $\forall x\le a$
And so $\exists c$ such that $g(x)=x$ or $g(x)=cx$ $\forall x\in[0,1]$

And continuity ends the problem : $g(x)=cx$ $\forall x$ and for any $c\in[0,1]$

Hence the result $\boxed{f(x)=c}$ $\forall x\in[0,1]$ and for any $c\in[0,1]$, which indeed is a solution.
\end{solution}



\begin{solution}[by \href{https://artofproblemsolving.com/community/user/112}{Diogene}]
	An other  point de vue   
It's easy to see, by induction,  that $\forall_n \ f(tf^n(t))=f(t)\in [0,1]$
Manifestly $f(t)=1$ is a solution. 
If  there is $| f(t)|<1$  then $\lim_{n\rightarrow \infty}f^n(t)=0$, so $f(0)=f(t)$
Now , because of continuity of $f$, we can conclude that  the solution is $f(t)=f(0)$ (constant). 

See you ..

 :cool:
\end{solution}



\begin{solution}[by \href{https://artofproblemsolving.com/community/user/29428}{pco}]
	\begin{tcolorbox}An other  point de vue   
It's easy to see, by induction,  that $\forall_n \ f(tf^n(t))=f(t)\in [0,1]$
Manifestly $f(t)=1$ is a solution. 
If  there is $| f(t)|<1$  then $\lim_{n\rightarrow \infty}f^n(t)=0$, so $f(0)=f(t)$
Now , because of continuity of $f$, we can conclude that  the solution is $f(t)=f(0)$ (constant). 

See you ..

 :cool:\end{tcolorbox}
Ahhhhhhhhh ! so simple !

I prefer your "point de vue" 
Thanks
\end{solution}
*******************************************************************************
-------------------------------------------------------------------------------

\begin{problem}[Posted by \href{https://artofproblemsolving.com/community/user/129317}{Dranzer}]
	Let $c$ be a fixed real number. Show that a root of the equation
\[x(x+1)(x+2)\cdots(x+2009)=c\]
can have multiplicity at most $2$. Determine the number of values of $c$ for which the equation has a root of multiplicity $2$.
	\flushright \href{https://artofproblemsolving.com/community/c6h474010}{(Link to AoPS)}
\end{problem}



\begin{solution}[by \href{https://artofproblemsolving.com/community/user/29428}{pco}]
	\begin{tcolorbox}Let $c$ be a fixed real number.Show that a root of the equation \[x(x+1)(x+2)...(x+2009)=c\] can have multiplicity at most 2.Determine the number of values of $c$ for which the equation has a root of multiplicity 2.\end{tcolorbox}
If $c=0$, we have $2010$ distinct roots of multiplicity $1$. So we can consider $c\ne 0$ and roots $\notin\{-2009,-2008,...,0\}$

Let $f(x)=\prod_{k=0}^{2009}(x+k)-c$

$f'(x)=\prod_{k=0}^{2009}(x+k)\sum_{k=0}^{2009}\frac 1{x+k}$ $\forall x\notin\{-2009,-2008,...,0\}$

$f"(x)=f'(x)\sum_{k=0}^{2009}\frac 1{x+k}$ $-\prod_{k=0}^{2009}(x+k)\sum_{k=0}^{2009}\frac 1{(x+k)^2}$ $\forall x\notin\{-2009,-2008,...,0\}$

If $r$ is a root of multiplicity $>2$, then $f(r)=f'(r)=f"(r)=0$. Then $\prod_{k=0}^{2009}(r+k)\sum_{k=0}^{2009}\frac 1{(r+k)^2}=0$, and so $c\sum_{k=0}^{2009}\frac 1{(r+k)^2}=0$ impossible.

And so no multiplicity $>2$

The number of double roots of $f(x)$ is the number of roots of $f'(x)$ and so we have $2009$ such roots.
It's then easy to see that each root $r$ has a matching root $r'=-2009-r$ for which we have the same value of $c$ and that no three roots give the same value of $c$ (previous result).

Hence the result $\boxed{1005}$ values of $c$
\end{solution}
*******************************************************************************
-------------------------------------------------------------------------------

\begin{problem}[Posted by \href{https://artofproblemsolving.com/community/user/68025}{Pirkuliyev Rovsen}]
	Prove that for any set of numbers $a_1,a_2,...a_{2012}$ there exists a function$f: \mathbb{R}\to\mathbb{R}$that satisfy
$a_1f(x)+a_2f(f(x))+...+a_{2012}\underbrace{f(f(f...f(x)...))}_{2012}=x$where$ a_{2012}>0$

____________________________
Azerbaijan Land of the Fire 
	\flushright \href{https://artofproblemsolving.com/community/c6h474030}{(Link to AoPS)}
\end{problem}



\begin{solution}[by \href{https://artofproblemsolving.com/community/user/29428}{pco}]
	\begin{tcolorbox}Prove that for any set of numbers $a_1,a_2,...a_{2012}$ there exists a function$f: \mathbb{R}\to\mathbb{R}$that satisfy
$a_1f(x)+a_2f(f(x))+...+a_{2012}\underbrace{f(f(f...f(x)...))}_{2012}=x$where$ a_{2012}>0$\end{tcolorbox}
Let $g(x)=\left(\sum_{k=1}^{2012}a_kx^k\right)-1$

$g(x)$ is continuous and such that $g(0)<0$ while $\lim_{x\to+\infty}g(x)=+\infty$ (since $a_{2012}>0$) and so $\exists \alpha>0$ such that $g(\alpha)=0$

Hence the result : just choose then $f(x)=\alpha x$
\end{solution}
*******************************************************************************
-------------------------------------------------------------------------------

\begin{problem}[Posted by \href{https://artofproblemsolving.com/community/user/108348}{AndrewROM}]
	Find all functions $ f: \Bbb{Z} \rightarrow \Bbb{Z} $ that satisfies:
\[ f(x+f(y))=y+f(x+2012) \]
for any integer numbers $ x $ and $ y $.

\begin{italicized}Lucian Dragomir\end{italicized}
	\flushright \href{https://artofproblemsolving.com/community/c6h474034}{(Link to AoPS)}
\end{problem}



\begin{solution}[by \href{https://artofproblemsolving.com/community/user/29428}{pco}]
	\begin{tcolorbox}Find all functions $ f: \Bbb{Z} \rightarrow \Bbb{Z} $ that satisfies:
\[ f(x+f(y))=y+f(x+2012) \]
for any integer numbers $ x $ and $ y $.\end{tcolorbox}
Let $P(x,y)$ be the assertion $f(x+f(y))=y+f(x+2012)$

Let $u=f(2011-f(2012))$ : $P(0,2011-f(2012))$ $\implies$ $f(u)=2011$

$P(x-2011,u)$ $\implies$ $f(x)=u+f(x+1)$ and so $f(x)=f(0)-ux$ 

Plugging then $f(x)=ax+b$ in original equation, we get the two solutions :
$f(x)=x+2012$ $\forall x$
$f(x)=2012-x$ $\forall x$
\end{solution}
*******************************************************************************
-------------------------------------------------------------------------------

\begin{problem}[Posted by \href{https://artofproblemsolving.com/community/user/122611}{oty}]
	Find all functions $f : \mathbb{N} \rightarrow \mathbb{N}$ such that $f(2m+2n)=f(m)f(n)$ for all natural numbers $m,n$.
	\flushright \href{https://artofproblemsolving.com/community/c6h474101}{(Link to AoPS)}
\end{problem}



\begin{solution}[by \href{https://artofproblemsolving.com/community/user/29428}{pco}]
	\begin{tcolorbox}Find all functions $f : \mathbb{N} \rightarrow \mathbb{N}$ such that $f(2m+2n)=f(m)f(n)$ for all natural numbers $m,n$.\end{tcolorbox}
Let $P(x,y)$ be the assertion $f(2x+2y)=f(x)f(y)$
Let $c=\frac{f(2)}{f(1)}$

$P(2,x)$ $\implies$ $f(2x+4)=f(2)f(x)=cf(1)f(x)$
$P(1,x+1)$ $\implies$ $f(2x+4)=f(1)f(x+1)$

And so $f(x+1)=cf(x)$ and $f(x)=f(1)c^{x-1}$

Plugging this in original equation, we get $c=f(1)=1$ and the unique solution $\boxed{f(x)=1}$ $\forall x\in\mathbb N$
\end{solution}
*******************************************************************************
-------------------------------------------------------------------------------

\begin{problem}[Posted by \href{https://artofproblemsolving.com/community/user/142365}{Tima95}]
	Let $n>1$
	\flushright \href{https://artofproblemsolving.com/community/c6h474133}{(Link to AoPS)}
\end{problem}



\begin{solution}[by \href{https://artofproblemsolving.com/community/user/29428}{pco}]
	\begin{tcolorbox}Let $n>1$ natural number. Find all functions $ f :\mathbb{R}\rightarrow\mathbb{R} $ such that $ f(x+y^{n})=f(x)+f^{n}(y) $\end{tcolorbox}
Let $P(x,y)$ be the assertion $f(x+y^n)=f(x)+f^n(y)$

$P(x,0)$ $\implies$ $f(0)=0$

Comparing $P(x^n,0)$ and $P(0,x)$, we get $f^n(x)=f(x^n)$ and so $P(x,y)$ becomes $f(x+y^n)=f(x)+f(y^n)$

So $f(x+y)=f(x)+f(y)$ $\forall x,\forall y\ge 0$
From there we get easily $f(-x)=-f(x)$ and so $f(x+y)=f(x)+f(y)$ $\forall x,y$

So $f(px)=pf(x)$ $\forall x,\forall p\in\mathbb Q$

Writing then $f((x+p)^n)=f^n(x+p)$, we get $\sum_{k=0}^n\binom nkp^kf(x^{n-k})=$ $\sum_{k=0}^n\binom nkp^kf(1)^kf(x)^{n-k}$

And so $\sum_{k=0}^n\binom nkp^k\left(f(x^{n-k})-f(1)^kf(x)^{n-k}\right)=0$ $\forall x,\forall p\in\mathbb Q$

Considering LHS as a polynomial in $p$ which is zero for any rational $p$, we get that this is the zero polynomial and so all coefficients are zero.

Looking then at coefficient of $p^{n-2}$, we get $f(x^2)=f(1)^{n-2}f^2(x)$
So $f(x)$ is bounded over $\mathbb R^+$ and so $f(x)=cx$ for some $c$.

Plugging this in original equation, we get the solutions :
$f(x)=0$ $\forall x$
$f(x)=x$ $\forall x$
$f(x)=-x$ $\forall x$ if $n$ is odd.
\end{solution}
*******************************************************************************
-------------------------------------------------------------------------------

\begin{problem}[Posted by \href{https://artofproblemsolving.com/community/user/122611}{oty}]
	Find all applications $f:\mathbb{R}\rightarrow \mathbb{R}$ such that : $f(x+y)f(x-y)=f(x^2)-f(y^2)$ ,$\forall{x,y\in{\mathbb{R}}}$
	\flushright \href{https://artofproblemsolving.com/community/c6h474192}{(Link to AoPS)}
\end{problem}



\begin{solution}[by \href{https://artofproblemsolving.com/community/user/29428}{pco}]
	\begin{tcolorbox}Find all applications $f:\mathbb{R}\rightarrow \mathbb{R}$ such that : $f(x+y)f(x-y)=f(x^2)-f(y^2)$ ,$\forall{x,y\in{\mathbb{R}}}$\end{tcolorbox}
$f(x)=0$ $\forall x$ is a solution. Let us from now look only for non allzero solutions.

Let $P(x,y)$ be the assertion $f(x+y)f(x-y)=f(x^2)-f(y^2)$
Let $u$ such that $f(u)\ne 0$

$P(0,0)$ $\implies$ $f(0)=0$
$P(x,0)$ $\implies$ $f(x^2)=f(x)^2$

1) $f(x)$ is an odd increasing function
==========================
$P(\frac{u+x}2,\frac{u-x}2)$ $\implies$ $f(u)f(x)=f(\frac{(u+x)^2}4)-f(\frac{(u-x)^2}4)$
$P(\frac{u-x}2,\frac{u+x}2)$ $\implies$ $f(u)f(-x)=f(\frac{(u-x)^2}4)-f(\frac{(u+x)^2}4)$
And so $f(-x)=-f(x)$

From $f(x^2)=f(x)^2$ we get $f(x)\ge 0$ $\forall x\ge 0$ and then $P(x,y)$ immediately gives that $f(x)$ is non decreasing.

If $f(v)=0$ for some $v\ne 0$, WLOG say $v>0$, then $P(x+v,x)$ $\implies$ $f(x+v)^2=f(x)^2$ and so, since non decreasing, $f(x)$ is constant and so is allzero, impossible.
So $f(x)=0$ $\iff$ $x=0$ and $f(x)$ is increasing.
Q.E.D.

2) $f(n)=n$ $\forall n\in\mathbb Z$
=====================
$P(1,0)$ $\implies$ $f(1)=f^2(1)$ and so $f(1)=1$ (since $f(x)$ is increasing)
Let $a=f(2)$

$P(2,1)$ $\implies$ $f(3)=a^2-1$
$P(3,1)$ $\implies$ $f(4)=a^3-2a$ and since $f(4)=f(2)^2$, we get $a^3-a^2-2a=0$ and so $a\in\{-1,0,2\}$ and so (since increasing) $a=2$
It's then easy to get $f(n)=n$ $\forall n\in\mathbb N$ thru induction.
And so, since odd, $f(n)=n$ $\forall n\in\mathbb Z$
Q.E.D.

3) $f(x+n)=f(x)+n$ $\forall x,\forall n\in\mathbb Z$
=================================
$P(\frac{x-1}2,\frac{x+1}2)$ $\implies$ $-f(x)=f(\frac{(x-1)^2}4)-f(\frac{(x+1)^2}4)$

$P(\frac{x+1}2,\frac{x+3}2)$ $\implies$ $-f(x+2)=f(\frac{(x+1)^2}4)-f(\frac{(x+3)^2}4)$

$P(\frac{x+3}2,\frac{x-1}2)$ $\implies$ $2f(x+1)=f(\frac{(x+3)^2}4)-f(\frac{(x-1)^2}4)$

Adding these three lines, we get $f(x+2)=2f(x+1)-f(x)$

And so $f(x+n)=nf(x+1)-(n-1)f(x)$ $\forall x,\forall n\in\mathbb Z$

Plugging this in $f(x+n)f(x-n)=f(x)^2-n^2$, we get $(f(x+1)-f(x))^2=1$ and so, since increasing, $f(x+1)=f(x)+1$
Q.E.D

4) $f(x)=x$ $\forall x$
=========
$P(x,y)$ $\implies$ $f(x+y)f(x-y)=f(x)^2-f(y)^2$
$P(x+1;y)$ $\implies$ $(f(x+y)+1)(f(x-y)+1)=(f(x)+1)^2-f(y)^2$
Subtracting implies $f(x+y)+f(x-y)=2f(x)$

From there, we get 
$f(nx)=nf(x)$ $\forall x,\forall n\in\mathbb Z$
$f(px)=pf(x)$ $\forall x,\forall p\in\mathbb Q$
$f(x)=xf(1)=x$ $\forall x\in\mathbb Q$
And so (since increasing) $f(x)=x$ $\forall x\in\mathbb R$, which indeed is a solution

5) Synthesis of solutions
===============
We got two solutions :
$f(x)=0$ $\forall x$
$f(x)=x$ $\forall x$
\end{solution}



\begin{solution}[by \href{https://artofproblemsolving.com/community/user/122611}{oty}]
	Amazing ! Dear \begin{bolded}pco\end{bolded} , congrats . :)
\end{solution}



\begin{solution}[by \href{https://artofproblemsolving.com/community/user/104682}{momo1729}]
	http://www.artofproblemsolving.com/Forum/viewtopic.php?f=36&t=455325&p=2559734#p2559734

By the way, would there be any changes if we defined the function from $\mathbb{C}$ to $\mathbb{C}$ instead of $\mathbb{R}$ ?
Thanks
\end{solution}



\begin{solution}[by \href{https://artofproblemsolving.com/community/user/29428}{pco}]
	\begin{tcolorbox}http://www.artofproblemsolving.com/Forum/viewtopic.php?f=36&t=455325&p=2559734#p2559734

By the way, would there be any changes if we defined the function from $\mathbb{C}$ to $\mathbb{C}$ instead of $\mathbb{R}$ ?
Thanks\end{tcolorbox}
Yes, it changes everything :
- impossible to speak about "increasing"
- $\mathbb Q$ is not dense in $\mathbb C$
...
\end{solution}
*******************************************************************************
-------------------------------------------------------------------------------

\begin{problem}[Posted by \href{https://artofproblemsolving.com/community/user/112281}{pr0likethis}]
	I posted the following in the preolympiad forum, but i think that was the wrong place. If i could get feedback as well as even perhaps a proven solution to the functional equation, that would be awesome

Hello, I was wondering some ways when solving a functional equation to PROVE that solutions you find are in fact the only solutions. For example, I was looking at a problem (I actually cannot find it now :( ) 
I think it was
$f(x+y)+f(x)f(y)=f(xy)+f(x)+f(y)+2xy$
But anyway I had 3 solutions: $f(x)=x^2, f(x)=2x, f(x)=-x$. I verified they were solutions by plugging into original function, but i was unsure of how to prove that these were the ONLY solutions. Any tips?
	\flushright \href{https://artofproblemsolving.com/community/c6h474274}{(Link to AoPS)}
\end{problem}



\begin{solution}[by \href{https://artofproblemsolving.com/community/user/29428}{pco}]
	Your question is strange.

Finding solutions of a functional equation is generally not a problem.
The problem IS to prove you found all of them.

So your question "how to prove that we found all the solutions" is simply "how to solve a functional equation"

And it's very difficult to give a general answer to a so general question.
Two classical ways :

1) from original properties, you use successive $\iff$ signs till the result and that's OK

2) more often, from original properties, you use successive $\implies$ signs till some result, which then is a mandatory form and you plug then back this result in the original equation in order to get what subset (maybe empty) of the mandatory form you got indeed is a result.
\end{solution}



\begin{solution}[by \href{https://artofproblemsolving.com/community/user/29428}{pco}]
	\begin{tcolorbox}$f(x+y)+f(x)f(y)=f(xy)+f(x)+f(y)+2xy$\end{tcolorbox}
Let $P(x,y)$ be the assertion $f(x+y)+f(x)f(y)=f(xy)+f(x)+f(y)+2xy$

$P(x,0)$ $\implies$ $(f(x)-2)f(0)=0$ and since $f(x)=2$ $\forall x$ is not a solution, we get $f(0)=0$
$P(2,2)$ $\implies$ $f(2)^2=2f(2)+8$ and so $f(2)\in\{-2,4\}$
$P(1,1)$ $\implies$ $f(2)+f(1)^2=3f(1)+2$ and so $f(1)\in\{-1,1,2,4\}$

Let $a=f(1)\in\{-1,1,2,4\}$

1) New assertion $Q(x,y)$ : $(2\frac xy+a-1)f(y)-2(a-1)x=(2\frac yx+a-1)f(x)-2(a-1)y$ $\forall x,y\ne 0$
====================================================================
$P(x,1)$ $\implies$ $f(x+1)=(2-a)f(x)+2x+a$
$P(x+y,1)$ $\implies$ $f(x+y+1)=(2-a)f(x+y)+2x+2y+a$

$P(x+1,y)$ $\implies$ $(2-a)f(x+y)+(2-a)f(x)f(y)+(2x+a-1)f(y)=f(xy+y)+(2-a)f(x)+2xy$
Subtracting $(2-a)\times P(x,y)$ from the above line, we get $f(xy+y)=(2-a)f(xy)+(2x+1)f(y)-2(a-1)xy$

And so $f(x+y)=(2-a)f(x)+(2\frac xy+1)f(y)-2(a-1)x$ $\forall x,\forall y\ne 0$
And, swapping $x,y$ : $f(x+y)=(2-a)f(y)+(2\frac yx+1)f(x)-2(a-1)y$ $\forall x\ne 0,\forall y$
Subtracting gives $(2\frac xy+a-1)f(y)-2(a-1)x=(2\frac yx+a-1)f(x)-2(a-1)y$ $\forall x,y\ne 0$
Q.E.D.

2) If $a=f(1)=-1$, then  $f(x)=-x$ $\forall x$
============================
$Q(x,1)$ $\implies$ $x-1=\frac{1-x}xf(x)$ $\forall x\ne 0$ and so $f(x)=-x$ $\forall x\notin\{0,1\}$
And so $f(x)=-x$ $\forall x$ which indeed is a solution (plugging back for verification)
Q.E.D.

3) If $a=f(1)=1$, then $f(x)=x^2$ $\forall x$
=============================
$Q(x,1)$ $\implies$ $x=\frac {f(x)}x$ $\forall x\ne 0$ and so $f(x)=x^2$ $\forall x\ne 0$
And so $f(x)=x^2$ $\forall x$ which indeed is a solution (plugging back for verification)
Q.E.D.

4) If $a=f(1)=2$, then $f(x)=2x$ $\forall x$
===========================
$Q(x,1)$ $\implies$ $2x+4=\frac{x+2}xf(x)$ $\forall x\ne 0$ and so $f(x)=2x$ $\forall x\notin\{-2,0\}$
$P(-1,-1)$ $\implies$ $f(-2)=-4$
And so $f(x)=2x$ $\forall x$ which indeed is a solution (plugging back for verification)
Q.E.D.

5) If $a=f(1)=4$, then no solution
===================
$P(1,1)$ $\implies$ $f(2)=-2$ 
$Q(2,1)$ $\implies$ $f(2)=\frac {11}2$
And so contradiction.
Q.E.D.

6) Synthesis of solutions
==================
We got three solutions:
$f(x)=-x$ $\forall x$
$f(x)=2x$ $\forall x$
$f(x)=x^2$ $\forall x$
\end{solution}
*******************************************************************************
-------------------------------------------------------------------------------

\begin{problem}[Posted by \href{https://artofproblemsolving.com/community/user/68025}{Pirkuliyev Rovsen}]
	Find all function $f: \mathbb{R}\to\mathbb{R}$ such that  $\frac{f(x)-f(y)}{x-y}=\frac{f(x)f(y)}{xy} $
$x{\neq}y$



_______________________________
Azerbaijan Land of the Fire 
	\flushright \href{https://artofproblemsolving.com/community/c6h474469}{(Link to AoPS)}
\end{problem}



\begin{solution}[by \href{https://artofproblemsolving.com/community/user/29428}{pco}]
	\begin{tcolorbox}Find all function $f: \mathbb{R}\to\mathbb{R}$ such that  $\frac{f(x)-f(y)}{x-y}=\frac{f(x)f(y)}{xy} $
$x{\neq}y$\end{tcolorbox}
I suppose we must add the condition $xy\ne 0$
Let $P(x,y)$ be the assertion $\frac{f(x)-f(y)}{x-y}=\frac{f(x)f(y)}{xy}$ $\forall x,y,x-y\ne 0$

If $f(u)=0$ for some $u\ne 0$, then $P(x,u)$ $\implies$ $f(x)=0$ $\forall x\notin\{u,0\}$ and so $f(x)=0$ $\forall x\ne 0$

Let us from now consider only functions such that $f(x)\ne 0$ $\forall x\ne 0$

$P(x,y)$ may be written $\frac 1{f(x)}-\frac 1x=\frac 1{f(y)}-y$ $\forall x,y,x-y\ne 0$

And so $f(x)=\frac x{ax+1}$ $\forall x\ne 0$

\begin{bolded}Hence the solutions\end{underlined}\end{bolded} :
$f(x)=0$ $\forall x\ne 0$ and $f(0)=c$ which indeed is a solution whatever is $c\in\mathbb R$

$f(x)=\frac {x}{ax+1}$ $\forall x\ne 0$ and $f(0)=c$ which indeed is a solution whatever are $a,c\in\mathbb R$
\end{solution}



\begin{solution}[by \href{https://artofproblemsolving.com/community/user/93837}{jjax}]
	I assume that we may not set $x=0$ or $y=0$ either? I'll restrict the domain of $f$ to the nonzero reals.

Case 1: $f(u)=0$ for some $u$. Then putting $y=u$ in the equation gives $f(x)=0$ for any $x$. Thus $f$ is the zero function.

Case 2: $f$ has no roots. Set $y=1$ and rearrange the equation (using for convenience $c=f(1)$)
$f(x)(x-cx+c)=cx$.
We now attempt to substitute $x= \frac{-c}{1-c}$ into the equation. Note that $x \neq 0$ since $c =f(1) \neq 0$.
Observe that the LHS becomes zero, and so the RHS is also zero. As $x$ is nonzero and $c$ is nonzero, there is a contradiction. So $x= \frac{-c}{1-c}$ is not valid. Thus $\frac{-c}{1-c}=0$ (not true) or $c=1$.
As $c=1$, we get $f(x)=x$. Clearly this too is a solution.

Thus the two solutions (on the restricted domain) are $f(x)=0$ and $f(x)=x$.

To pco: The second function you posted isn't defined at $x= \frac{-1}{a}$ for $a \neq 0$
\end{solution}



\begin{solution}[by \href{https://artofproblemsolving.com/community/user/29428}{pco}]
	\begin{tcolorbox}To pco: The second function you posted isn't defined at $x= \frac{-1}{a}$ for $a \neq 0$\end{tcolorbox}
Humpffffffffff :blush:  you're quite right. So $a=0$ and your solutions.
Thanks for this remark.
\end{solution}
*******************************************************************************
-------------------------------------------------------------------------------

\begin{problem}[Posted by \href{https://artofproblemsolving.com/community/user/146712}{PeykeNorouzi}]
	Find all strictly increasing functions $f:\mathbb R^+ \longrightarrow \mathbb R^+$ such that for all $x\in \mathbb R^+$

$f(\frac{x^2}{f(x)})=x$.
	\flushright \href{https://artofproblemsolving.com/community/c6h474473}{(Link to AoPS)}
\end{problem}



\begin{solution}[by \href{https://artofproblemsolving.com/community/user/95486}{diks94}]
	assuming the function is onto hence invertible
we get $f(x) f^{-1} (x) =x^2$ 
by mere inspection we know that f(X) should be of the form $ax+b$ also $a=1$ n $b=0$ for given condition hence only possible function would be $f(x)=x$
\end{solution}



\begin{solution}[by \href{https://artofproblemsolving.com/community/user/29428}{pco}]
	\begin{tcolorbox}by mere inspection we know that f(X) should be of the form $ax+b$\end{tcolorbox}
Could you explain this, please ?
\end{solution}



\begin{solution}[by \href{https://artofproblemsolving.com/community/user/95486}{diks94}]
	\begin{tcolorbox}[quote="diks94"]by mere inspection we know that f(X) should be of the form $ax+b$\end{tcolorbox}
Could you explain this, please ?\end{tcolorbox}
sorry pco its just an inspection ..i said that above
\end{solution}



\begin{solution}[by \href{https://artofproblemsolving.com/community/user/87322}{paul1703}]
	Source: Bulgaria 1996, I am sure it was posted before.I think you can find a solution in contests from around the world (Titu).
\end{solution}



\begin{solution}[by \href{https://artofproblemsolving.com/community/user/82319}{Andrei95}]
	Doing the substitution $ x \rightarrow \frac{x^2}{f(x)} $ we found that $ f(\frac{x^3}{f^2(x)})= \frac{x^2}{f(x)} $ 
Inductively we will find that $ f( \frac{x^n}{f^{n-1}(x))})= \frac{x^{n-1}}{f^{n-2}(x)} $
Now denote f(1)=k and we can easily compute that $ f(k^n)=k^{n+1} $ for any  $ n\in Z $
Let's take now the case f(x)<kx for some x (the case f(x)>kx will be done in the same way)
For every integer number t , we can find a sufficiently large N such that $ \frac{x^N}{f^{N-1}(x))}> \frac{1}{k^{N+t}} $ using that the function f is increasing we will find that the inequality is satisfied for every n so $ x >\frac{1}{k^t} $ for every integer t (contradiction) so the function will be f(x)=kx
And the only case left is f(1)=1 looking at $ f( \frac{x^n}{f^{n-1}(x))}) $ and $  f( \frac{y^n}{f^{n-1}(y))}) $ x>y and using the fact that f is increasing we found that $ \frac{x}{f(x)} $ is increasing for x>1 and decreasing for x<1. 
But $ \frac{x}{f(x)} $= $ \frac{ \frac{x^n}{f^n(x)}}{f( \frac{x^n}{f^n(x)})} $ from where it's easy to deduce that f(x)=x
\end{solution}
*******************************************************************************
-------------------------------------------------------------------------------

\begin{problem}[Posted by \href{https://artofproblemsolving.com/community/user/146712}{PeykeNorouzi}]
	Find all functions $f:\mathbb R\longrightarrow \mathbb R$ such that for all $x,y\in \mathbb R$

$f(x+yf(x))+f(xf(y)-y)=f(x)-f(y)+2xy$.
	\flushright \href{https://artofproblemsolving.com/community/c6h474474}{(Link to AoPS)}
\end{problem}



\begin{solution}[by \href{https://artofproblemsolving.com/community/user/29428}{pco}]
	\begin{tcolorbox}Find all functions $f:\mathbb R\longrightarrow \mathbb R$ such that for all $x,y\in \mathbb R$

$f(x+yf(x))+f(xf(y)-y)=f(x)-f(y)+2xy$.\end{tcolorbox}
Let $P(x,y)$ be the assertion $f(x+yf(x))+f(xf(y)-y)=f(x)-f(y)+2xy$

$P(0,0)$ $\implies$ $f(0)=0$
$P(0,x)$ $\implies$ $f(-x)=-f(x)$
If $f(x)=0$ $P(x,x)$ $\implies$ $x=0$ and so $f(x)=0\iff x=0$


$P(x,y)$ $\implies$ $f(x+yf(x))+f(xf(y)-y)=f(x)-f(y)+2xy$
$P(x,-y)$ $\implies$ $f(x-yf(x))-f(xf(y)-y)=f(x)-f(y)+2xy$
Adding these two lines, we get $f(x+yf(x))+f(x-yf(x))=2f(x)$

If $x\ne 0$, then $f(x)\ne 0$ and the substitution $y\to\frac y{f(x)}$ in previous line implies $f(x+y)+f(x-y)=2f(x)$, still true if $x=0$
Swapping then $x,y$ in this equation, we get $f(x+y)-f(x-y)=2f(y)$ and adding the two lines gives $f(x+y)=f(x)+f(y)$ $\forall x,y$ and so $f(x)$ is additive.

Then $P(x,y)$ becomes new assertion $Q(x,y)$ : $f(yf(x))+f(xf(y))=2xy$

$Q(1,1)$ $\implies$ $f(f(1))=1$ and then $Q(f(1),f(1))$ $\implies$ $f(1)^2=1$
$f(x)$ solution implies $-f(x)$ solution and so WLOG say $f(1)=1$

$Q(x,1)$ $\implies$ $f(f(x))=2x-f(x)$
$Q(x,x)$ $\implies$ $f(xf(x))=x^2$ and so $f(x)$ is surjective (remember it's an odd function)
$Q(f(x),f(x))$ $\implies$ $f(f(x)f(f(x)))=f(x)^2$ and so $f(f(x)^2)+f(x)^2=2x^2\ge 0$ and so, since surjective, $f(x^2)+x^2\ge 0$ $\forall x$

So $f(x)\ge -x$ $\forall x\ge 0$ and $f(x)$ is lower bounded on some non empty interval, and so is continuous and $f(x)=f(1)x=x$ which indeed is a solution.

\begin{bolded}Hence the two solutions\end{underlined}\end{bolded} :
$f(x)=x$ $\forall x$
$f(x)=-x$ $\forall x$
\end{solution}



\begin{solution}[by \href{https://artofproblemsolving.com/community/user/122611}{oty}]
	please , Howe you get : ''and so $f(f(x)^2)+f(x)^2=2x^2\ge 0$ ''  ? Thank you .
\end{solution}



\begin{solution}[by \href{https://artofproblemsolving.com/community/user/29428}{pco}]
	\begin{tcolorbox}please , Howe you get : ''and so $f(f(x)^2)+f(x)^2=2x^2\ge 0$ ''  ? Thank you .\end{tcolorbox}
$Q(f(x),f(x))$ $\implies$ $f(f(x)f(f(x)))=f(x)^2$

But we know that $f(f(x))=2x-f(x)$ (line above)

So $f(2xf(x)-f(x)^2)=f(x)^2$

So, since additive : $2f(xf(x))-f(f(x)^2)=f(x)^2$

And since we know (using $Q(x,x)$) that $f(xf(x))=x^2$, this becomes $2x^2-f(f(x)^2)=f(x)^2$

And so $f(f(x)^2)+f(x)^2=2x^2$
\end{solution}



\begin{solution}[by \href{https://artofproblemsolving.com/community/user/122611}{oty}]
	very smart , Thank you Dear $pco$ :) .
\end{solution}
*******************************************************************************
-------------------------------------------------------------------------------

\begin{problem}[Posted by \href{https://artofproblemsolving.com/community/user/122326}{FoolMath}]
	Let f be an function from N* to N* satisfying:
\[\left\{ {\begin{array}{*{20}{c}}
   {f(1) = 1}  \\
   {f(n) = n - f(f(n - 1))}  \\
\end{array}\forall n \in} N* \right.\]
	\flushright \href{https://artofproblemsolving.com/community/c6h474495}{(Link to AoPS)}
\end{problem}



\begin{solution}[by \href{https://artofproblemsolving.com/community/user/29428}{pco}]
	\begin{tcolorbox}Let f be an function from N* to N* satisfying:
\[\left\{ {\begin{array}{*{20}{c}}
   {f(1) = 1}  \\
   {f(n) = n - f(f(n - 1))}  \\
\end{array}\forall n \in} N* \right.\]\end{tcolorbox}

$f(n)<n$ and so $f(x)$ is uniquely defined by knowledge of $f(1)$.
So a unique solution.

And it's not very difficult (although a bit ugly) to show that $\boxed{f(x)=\left\lfloor\frac{-1+\sqrt 5}2(n+1)\right\rfloor}$ is a solution, so is the unique one.
\end{solution}



\begin{solution}[by \href{https://artofproblemsolving.com/community/user/122326}{FoolMath}]
	How do you find this solution![hide]I mean, can you show me the way you find it :) [\/hide]
\end{solution}



\begin{solution}[by \href{https://artofproblemsolving.com/community/user/29428}{pco}]
	\begin{tcolorbox}How do you find this solution![hide]I mean, can you show me the way you find it :) [\/hide]\end{tcolorbox}
I think there is no serious way to find it.
It's a crazy problem and I dont understand how it could be asked in a real olympiad training session :(

I personnaly encountered this problem in the past and just remembered the solution (no talent for that, sure :blush:)
\end{solution}
*******************************************************************************
-------------------------------------------------------------------------------

\begin{problem}[Posted by \href{https://artofproblemsolving.com/community/user/146712}{PeykeNorouzi}]
	Find all functions $f:\mathbb R \longrightarrow \mathbb R$ such that for all real numbers $x$ and $y$,

$f(x+y)f(f(x)-y)=xf(x)-yf(y)$
	\flushright \href{https://artofproblemsolving.com/community/c6h474512}{(Link to AoPS)}
\end{problem}



\begin{solution}[by \href{https://artofproblemsolving.com/community/user/122611}{oty}]
	let P(x,y) the assertion $f(x+y)f(f(x)-y)=xf(x)-yf(y)$ . $P(0,0)$ imply : $f(0)=0$ or $f(f(0))=0$ .if $f(f(0))=0$ , $P(0,x)$ imply $f(x)f(-x)=-xf(x)$ and so $f(x)=0$ or $f(x)=x$ $\forall{x\in{\mathbb{R}}}$  which verify the FE . if $f(0)=0$ , $P(0,-x)$ imply ;$f(-x)f(x)=xf(-x)$ and so $f(x)=x$ or $f(x)=0$ , $\forall{x\in{\mathbb{R}}}$ which verify the FE .
\end{solution}



\begin{solution}[by \href{https://artofproblemsolving.com/community/user/29428}{pco}]
	\begin{tcolorbox}let P(x,y) the assertion $f(x+y)f(f(x)-y)=xf(x)-yf(y)$ . $P(0,0)$ imply : $f(0)=0$ or $f(f(0))=0$ .if $f(f(0))=0$ , $P(0,x)$ imply $f(x)f(-x)=-xf(x)$ and so $f(x)=0$ or $f(x)=x$ $\forall{x\in{\mathbb{R}}}$  which verify the FE . if $f(0)=0$ , $P(0,-x)$ imply ;$f(-x)f(x)=xf(-x)$ and so $f(x)=x$ or $f(x)=0$ , $\forall{x\in{\mathbb{R}}}$ which verify the FE .\end{tcolorbox}
No, you made many errors :

Error 1 : 
'if $f(f(0))=0$ , $P(0,x)$ imply $f(x)f(-x)=-xf(x)$' is wrong : If if $f(f(0))=0$ , $P(0,x)$ imply $f(x)f(f(0)-x)=-xf(x))$

Error 2 (very classical, and you should check this error in eacb of your proofs) :
"$f(x)f(-x)=-xf(x)$ and so $f(x)=0$ or $f(x)=x$ $\forall{x\in{\mathbb{R}}}$ " is wrong :
$f(x)f(-x)=-xf(x)$ implies : $\forall x$, either $f(x)=0$, either $f(-x)=-x$  and you cant directly conclude from there that :
either $f(x)=0$ $\forall x$, either $f(-x)=-x$ $\forall x$, which is a quite different assertion.
For example $f(x)=0$ $\forall x\in\mathbb Q$ and $f(x)=x$ $\forall x\notin\mathbb Q$ matches the requirement $f(x)f(-x)=-xf(x)$

Error 3  (same as error 2)
$f(-x)f(x)=xf(-x)$ and so $f(x)=x$ or $f(x)=0$ , $\forall{x\in{\mathbb{R}}}$
\end{solution}



\begin{solution}[by \href{https://artofproblemsolving.com/community/user/29126}{MellowMelon}]
	I have a solution set of either $f(x) = x$, or $f(x) = 0$ for all nonzero $x$ and $f(0)$ allowed to be arbitrary. The first is an obvious solution. The second is a solution from the fact that $xf(x) = 0$ for all $x$ and $f(x+y)f(f(x)-y)$ also must be 0, as we can't have $x+y = 0$ and $f(x)-y = 0$ simultaneously.

Put in $y = 0$ to get $xf(x) = f(x)f(f(x))$. (1)
Put in $y = f(x)$ to get $f(0)f(f(x)+x) = xf(x)-f(x)f(f(x))$. By (1) the RHS is 0. Put in $y = -x$ to get $f(0)f(f(x)+x) = x(f(x)+f(-x))$. We just saw the LHS of this is 0, so for all nonzero reals $x$ we have $f(x) = -f(-x)$. (2)
Put in $x = 0$ and replace $y$ with $-y$ to get $f(-y)f(f(0)+y) = yf(-y)$. So either $f(-y) = 0$ or $f(f(0)+y) = y$ for every real $y$. (3)

We now obtain a partial form of injectivity. Suppose $f(a) = f(b) \neq 0$. From (1), $af(a) = f(a)f(f(a)) = f(b)f(f(b)) = bf(b)$, so $a = b$.

Suppose $x, f(x)$ are both nonzero. Then $f(x) = -f(-x)$ by (2), so $f(-x)$ is nonzero and $f(x+f(0)) = x$ by (3). Put in $y = f(0)$ to get $f(x+f(0))f(f(x)-f(0)) = xf(x)$; the $f(0)f(f(0))$ term vanishes by (1). This simplifies to $f(f(x)-f(0)) = f(x)$, so by the partial injectivity, $f(x) = x+f(0)$. Therefore, for every real $x$, one of the following is true: $f(x) = 0$ or $f(x) = x+f(0)$ (which includes the $x = 0$ case).

If for every nonzero real $x$ we have $f(x) = 0$, then $f(0)$ can be anything and we get a solution. So assume the existence of $a$ with $af(a) \neq 0$. Then letting $x = a$ in (1) and using $f(a) = a+f(0)$, we get $a = a+2f(0)$, meaning $f(0) = 0$ and $f(a) = a$. By (2), $f(a) = -f(-a)$. Finally, take any $b$ such that $f(b) = 0$. Then $x = b, y = a$ in the original equation gives $af(a+b) = a^2$, or $f(a+b) = a = f(a)$. By the partial injectivity $b = 0$. As a result we could have taken $a$ to be any nonzero real number, so we have $f(x) = x$ for all reals, the final solution.
\end{solution}



\begin{solution}[by \href{https://artofproblemsolving.com/community/user/29428}{pco}]
	\begin{tcolorbox}Find all functions $f:\mathbb R \longrightarrow \mathbb R$ such that for all real numbers $x$ and $y$,

$f(x+y)f(f(x)-y)=xf(x)-yf(y)$\end{tcolorbox}
Let $P(x,y)$ be the assertion $f(x+y)f(f(x)-y)=xf(x)-yf(y)$

1) If $f(0)\ne 0$, the solution is $f(x)=0$ $\forall x\ne 0$ and $f(0)=c$ where $c$ is any real different from zero
============================================================================
$P(x,0)$ $\implies$ $f(x)(f(f(x))-x)=0$
$P(x,f(x))$ $\implies$ $f(x+f(x))f(0)=-f(x)(f(f(x))-x)=0$ and so $f(x+f(x))=0$ $\forall x$
$P(x,-x)$ $\implies$ $f(0)f(f(x)+x)=x(f(x)+f(-x))$ and so $f(-x)=-f(x)$ $\forall x\ne 0$

Let $x\ne 0$ such that $f(x)\ne 0$ : $P(x+f(x),x)$ $\implies$ $f(2x+f(x))=x$
Let then $u=2x+f(x)$ so that $f(u)=x\ne 0$ : $P(u,0)$ $\implies$ $f(f(u))=u$ and so $f(x)=2x+f(x)$, impossible
So $f(x)=0$ $\forall x\ne 0$
And so $f(x)=0$ $\forall x\ne 0$ and $f(0)=c$ which indeed is a solution.
Q.E.D.

2) If $f(0)=0$, the two solutions are $f(x)=0$ $\forall x$ and $f(x)=x$ $\forall x$
======================================================
$P(x,-x)$ $\implies$ $x(f(x)+f(-x))=0$ and so $f(-x)=-f(x)$ $\forall x$
$P(0,x)$ $\implies$ $f(x)(f(x)-x)=0$ and so $\forall x$, either $f(x)=0$, either $f(x)=x$

If $\exists x,y\ne 0$ such that $f(x)=0$ and $f(y)=y$, then $P(x,y)$ $\implies$ $f(x+y)=y$, impossible since $f(x+y)=0$ or $x+y$ and so $y=0$ or $x=0$

And so :
Either $f(x)=0$ $\forall x$ which indeed is a solution
either $f(x)=x$ $\forall x$ which indeed is a solution

3) Synthesis of solutions :
=================
$f(x)=0$ $\forall x\ne 0$ and $f(0)=c$ where $c$ is any real
$f(x)=x$ $\forall x$


*\begin{bolded}edit\end{underlined}\end{bolded}* : too late :)
\end{solution}



\begin{solution}[by \href{https://artofproblemsolving.com/community/user/122611}{oty}]
	\begin{tcolorbox}[quote="oty"]let P(x,y) the assertion $f(x+y)f(f(x)-y)=xf(x)-yf(y)$ . $P(0,0)$ imply : $f(0)=0$ or $f(f(0))=0$ .if $f(f(0))=0$ , $P(0,x)$ imply $f(x)f(-x)=-xf(x)$ and so $f(x)=0$ or $f(x)=x$ $\forall{x\in{\mathbb{R}}}$  which verify the FE . if $f(0)=0$ , $P(0,-x)$ imply ;$f(-x)f(x)=xf(-x)$ and so $f(x)=x$ or $f(x)=0$ , $\forall{x\in{\mathbb{R}}}$ which verify the FE .\end{tcolorbox}
No, you made many errors :

Error 1 : 
'if $f(f(0))=0$ , $P(0,x)$ imply $f(x)f(-x)=-xf(x)$' is wrong : If if $f(f(0))=0$ , $P(0,x)$ imply $f(x)f(f(0)-x)=-xf(x))$

Error 2 (very classical, and you should check this error in eacb of your proofs) :
"$f(x)f(-x)=-xf(x)$ and so $f(x)=0$ or $f(x)=x$ $\forall{x\in{\mathbb{R}}}$ " is wrong :
$f(x)f(-x)=-xf(x)$ implies : $\forall x$, either $f(x)=0$, either $f(-x)=-x$  and you cant directly conclude from there that :
either $f(x)=0$ $\forall x$, either $f(-x)=-x$ $\forall x$, which is a quite different assertion.
For example $f(x)=0$ $\forall x\in\mathbb Q$ and $f(x)=x$ $\forall x\notin\mathbb Q$ matches the requirement $f(x)f(-x)=-xf(x)$
Error 3  (same as error 2)
$f(-x)f(x)=xf(-x)$ and so $f(x)=x$ or $f(x)=0$ , $\forall{x\in{\mathbb{R}}}$\end{tcolorbox}
sorry , Thank you a lot Dear \begin{bolded}pco\end{bolded} , for the correction  :) .
\end{solution}
*******************************************************************************
-------------------------------------------------------------------------------

\begin{problem}[Posted by \href{https://artofproblemsolving.com/community/user/51901}{KittyOK}]
	Find all $f:\mathbb{R} \to \mathbb{R}$ such that $f(x)f(x+y)=(f(y)f(x-y))^2 e^{y+4}$ for all $x,y\in \mathbb{R}$.
	\flushright \href{https://artofproblemsolving.com/community/c6h474810}{(Link to AoPS)}
\end{problem}



\begin{solution}[by \href{https://artofproblemsolving.com/community/user/29428}{pco}]
	\begin{tcolorbox}Find all $f:\mathbb{R} \to \mathbb{R}$ such that $f(x)f(x+y)=(f(y)f(x-y))^2 e^{y+4}$ for all $x,y\in \mathbb{R}$.\end{tcolorbox}
If $f(0)=0$, then setting $y=0$ in original equation gives $f(x)=0$ $\forall x$ which indeed is a solution.
So let us from now consider $f(0)\ne 0$

Setting then $x=y=0$, we get $f(0)=\pm e^{-2}$ and since $f(x)$ solution implies $-f(x)$ solution, WLOG say $f(0)=e^{-2}$

Let then $g(x)=f(x)e^{2-x}$ and the equation becomes $g(x)g(x+y)=g(y)^2g(x-y)^2$ with $g(0)=1$
Let $P(x,y)$ be the assertion $g(x)g(x+y)=g(y)^2g(x-y)^2$

$P(0,x)$ $\implies$ $g(x)=g(x)^2g(-x)^2$
$P(0,-x)$ $\implies$ $g(-x)=g(-x)^2g(x)^2=g(x)$ and so $g(x)=g(x)^4$ and so $\forall x$, either $g(x)=0$, either $g(x)=1$

Let then $A=\{x$ such that $g(x)=0\}$ and $B=\{x$ such that $g(x)=1\}$
We got up to now that $0\in B$ and that $x,-x$ always are in the same set.

Let $a_1,a_2\in A$ : $P(a_1,a_2)$ is always true
Let $a\in A,b\in B$ : $P(a,-b)$ $\implies$ $a+b\in A$
Let $a\in A,b\in B$ : $P(b,a)$ $\implies$ $a+b\in A$
Let $b_1,b_2\in B$ : $P(b_1,b_2)$ $\implies$ $g(b_1+b_2)=g(b_1-b_2)$ and so $b_1+b_2$ and $b_1-b_2$ both in the same set.
But $b_1+b_2\in A$ $\implies$ $(b_1+b_2)+(-b_2)\in A$, impossible and so $b_1+b_2,b_1-b_2\in B$

So we got five rules :
r1 : $0\in B$
r2 : $a\in A$ $\implies$ $-a\in A$
r3 : $b\in B$ $\implies$ $-b\in B$
r4 : $a\in A,b\in B$ $\implies$ $a+b\in A$
r5 : $b_1,b_2\in B$ $\implies$ $b_1+b_2\in B$
Notice that r1+r3+r5 imply r2+r4 and it's easy to see that these necessary conditions are sufficient.

\begin{bolded}Hence the solutions\end{underlined}\end{bolded} :
S1 : $f(x)=0$ $\forall x$

Let $B$ any additive subgroup of $\mathbb R$. We can define two solutions based upon any such $B$ :

S2 : $f(x)=e^{x-2}$ $\forall x\in B$ and $f(x)=0$ $\forall x\notin B$
S3 : $f(x)=-e^{x-2}$ $\forall x\in B$ and $f(x)=0$ $\forall x\notin B$
\end{solution}
*******************************************************************************
-------------------------------------------------------------------------------

\begin{problem}[Posted by \href{https://artofproblemsolving.com/community/user/127386}{Abdurasul}]
	For all functions f:R_R such that f(x-y)(x+y)=f(x+y)(x-y)
	\flushright \href{https://artofproblemsolving.com/community/c6h474833}{(Link to AoPS)}
\end{problem}



\begin{solution}[by \href{https://artofproblemsolving.com/community/user/29428}{pco}]
	\begin{tcolorbox}For all functions f:R_R such that f(x-y)(x+y)=f(x+y)(x-y)\end{tcolorbox}
Setting $x=\frac{u+1}2$ and $y=\frac{1-u}2$, we get $f(u)=f(1)u$ $\forall u$ which indeed is a solution, whatever is $f(1)$

Hence the solutions $\boxed{f(x)=ax}$ $\forall x$ and for any real $a$
\end{solution}
*******************************************************************************
-------------------------------------------------------------------------------

\begin{problem}[Posted by \href{https://artofproblemsolving.com/community/user/110678}{alimathematics}]
	Find all functions $f:\mathbb{R^+} \to \mathbb{R^+}$ such that :

$\forall x,y\in \mathbb{R^+}$ $f(\frac{x-y}{f(y)})=\frac{f(x)}{f(y)}$
	\flushright \href{https://artofproblemsolving.com/community/c6h475122}{(Link to AoPS)}
\end{problem}



\begin{solution}[by \href{https://artofproblemsolving.com/community/user/29428}{pco}]
	\begin{tcolorbox}Find all functions $f:\mathbb{R^+} \to \mathbb{R^+}$ such that :

$\forall x,y\in \mathbb{R^+}$ $f(\frac{x-y}{f(y)})=\frac{f(x)}{f(y)}$\end{tcolorbox}
I suppose we must add the constraint that functionnal equation is true only $x,y>0$ \begin{bolded}and \end{underlined}\end{bolded}$x>y$, else obviously no solution.

If so, let $P(x,y)$ be the assertion $f(\frac{x-y}{f(y)})=\frac{f(x)}{f(y)}$

If $f(x)<1$ for some $x$, then $P(\frac x{1-f(x)},x)$ $\implies$ $f(x)=1$, impossible. So $f(x)\ge 1$ $\forall x$

$P(x+y,x)$ $\implies$ $f(x+y)=f(x)f(\frac y{f(x)})\ge f(x)$ and so $f(x)$ is non decreasing.

If $f(t)=1$ for some $t$, then $P(x+t,t)$ $\implies$ $f(x+t)=f(x)$ and so, since non decreasing, $f(x)$ is constant and the solution $f(x)=1$ $\forall x$
If $f(x)>1$ $\forall x$, then $f(x)$ is increasing and so injective. Then :

$P(xf(1)+1,1)$ $\implies$ $f(x)f(1)=f(xf(1)+1)$
$P(f(x)+x,x)$ $\implies$ $f(x)f(1)=f(f(x)+x)$
So $f(xf(1)+1)=f(f(x)+x)$ and so, since injective : $xf(1)+1=f(x)+x$ and so $f(x)=(f(1)-1)x+1$ and so $f(x)=ax+1$ for $a>0$, which indeed is a solution.

Hence the result : $\boxed{f(x)=ax+1}$ $\forall x$ and for any real $a\ge 0$ (the case $a=0$ giving he previous solution $f(x)=1$ $\forall x$
\end{solution}
*******************************************************************************
-------------------------------------------------------------------------------

\begin{problem}[Posted by \href{https://artofproblemsolving.com/community/user/94615}{Pedram-Safaei}]
	suppose that $f(n)$ be the number of prime numbers that are not greater than $n$.does there exist infinitely many $n$ such that $f(n)$ divide $n$.
	\flushright \href{https://artofproblemsolving.com/community/c6h475159}{(Link to AoPS)}
\end{problem}



\begin{solution}[by \href{https://artofproblemsolving.com/community/user/29428}{pco}]
	\begin{tcolorbox}suppose that $f(n)$ be the number of prime numbers that are not greater than $n$.does there exist infinitely many $n$ such that $f(n)$ divide $n$.\end{tcolorbox}
Suppose $a=kf(a)$ for some $a\ge 2$ and some $k>1$
Consider the function $g(n)=n-(k+1)f(n)$ and we get :

$g(a)=-f(a)<0$ and $\lim_{n\to +\infty}g(n)=+\infty$
Let then $A=\{x\ge a$ such that $g(y)<0$ $\forall y\in[a,x]\}$
A is a finite set and let $m=\max(A)$ :

$g(m)<0$ and $g(m+1)\ge 0$ but either $g(n+1)=g(n)+1$, either $g(n+1)=g(n)-k$ and so $g(m+1)=0$ and so $f(m+1)|m+1$ with $m+1>a$

And starting with $a=2$ gives the result : there exists infinitely many $n$ such that $f(n)$ divides $n$

Moreover, $\frac n{f(n)}$ can take any integer value $\ge 2$ we want.
\end{solution}
*******************************************************************************
-------------------------------------------------------------------------------

\begin{problem}[Posted by \href{https://artofproblemsolving.com/community/user/60735}{hatchguy}]
	Find all functions $f$ defined on the non-negative reals such that the following holds

1. $f(x) \ge 0$ for all non-negative reals $x$.

2. $f(1) = \frac{1}{2}$

3. $f(yf(x))f(x) = f(x+y)$ for all non negative reals $x,y$.
	\flushright \href{https://artofproblemsolving.com/community/c6h475206}{(Link to AoPS)}
\end{problem}



\begin{solution}[by \href{https://artofproblemsolving.com/community/user/74510}{filipbitola}]
	[hide]In the last equation, plugging $y=0$ we get that:
$f(0)f(x)=f(x)$ and since $f(1)\neq 0 \implies f(0)=1$
$x=1 \implies f(y+1)=\frac{f(\frac{y}{2})}{2}$
Let for some $x>1, f(x)>1$
Then, $y=\frac{x}{f(x)-1} \implies f(x)=1$ contradiction. So $f(x)\leq 1 \forall x>1$
Similarly, $f(x)\geq 1 \forall x<1$
Let for some $b\geq a, f(b)=f(a)$

Since $yf(a)=yf(b) \implies  f(a+y)=f(b+y)$ meaning that the function is periodic with period $b-a$.

\begin{bolded}Case 1:\end{bolded} $\exists x_{0}<1 f(x_{0})>1$
Since $f(x_{0})=f(x_{0}+k(b-a)), \forall k\in\mathbb{N}$ if $b\neq a$ for sufficiently large $k, x_{0}+k(b-a)>1$ which is a contradiction.

Hence the function is injective. 
So, $ f(y+1)=\frac{f(\frac{y}{2})}{2} \implies  f(2y+1)=\frac{f(y)}{2}$.
Placing $y=\frac{1}{f(x)}$ in the original equation $\implies \frac{f(x)}{2}=f(x+\frac{1}{f(x)}) \implies f(x+\frac{1}{f(x)})=f(2x+1)$ Since $f$ is injective, $f(x)=\frac{1}{x+1}$. Checking in the original equation we see that it is a solution.

\begin{bolded}Case 2:\end{bolded} $f(x)=1 \forall x<1$
Since the period is $b-a$ where $f(a)=f(b)$ we can choose, for example $b-a=\frac{1}{2}$ and that would imply that $f(\frac{1}{2})=f(1)\implies 1=\frac{1}{2}$. Hence contradiction.

Hence, the only solution to this equation is $\boxed{f(x)=\frac{1}{x+1}}$[\/hide]

Unfortunately it seems that I have some mistakes... I'll fix them tomorrow, it's late now.
Thanks, hatchguy for noticing the mistakes...
\end{solution}



\begin{solution}[by \href{https://artofproblemsolving.com/community/user/60735}{hatchguy}]
	There is something wrong with your proof.

First of all, you should consider that case where $f(\frac{xf(x)}{f(x)-1}) = 0$.

Also, you claim that $f(x) = \frac{1}{x+1}$ which is clearly always less than $1$ so I don't know how you proved $f(x) \ge 1$ for all $x< 1$.  However this means case $1$ should not be considered.

You can't fix the period in case $2$.
\end{solution}



\begin{solution}[by \href{https://artofproblemsolving.com/community/user/29428}{pco}]
	\begin{tcolorbox}Find all functions $f$ defined on the non-negative reals such that the following holds

1. $f(x) \ge 0$ for all non-negative reals $x$.

2. $f(1) = \frac{1}{2}$

3. $f(yf(x))f(x) = f(x+y)$ for all non negative reals $x,y$.\end{tcolorbox}
Let $P(x,y)$ be the assertion $f(yf(x))f(x)=f(x+y)$

If $f(x_n)=2^{-2^n}$ for some $x_n$, then $P(x_n,2^{2^n}x_n)$ $\implies$ $f(x_n(1+2^{2^n}))=2^{-2^{n+1}}$
Starting with $x_0=1$, we get $f(\prod_{k=0}^n(1+2^{2^k})=2^{-2^{n+1}}>0$

If $f(t)=0$ for some $t\ge 0$, then $P(t,x-t)$ $\implies$ $f(x)=0$ $\forall x\ge t$, in contradiction with the previous sentence (since $\prod_{k=0}^n(1+2^{2^k}\to+\infty$)
So $f(x)>0$ $\forall x\ge 0$

If $f(t)>1$ for some $t$, then $P(t,\frac t{f(t)-1})$ $\implies$ $f(t)=1$ (since $f(x)>0$ $\forall x$), impossible. So $f(x)\le 1$ $\forall x$

Then $P(x,y)$ implies $f(x+y)\le f(x)$ and $f(x)$ is non increasing

If $f(t)=1$ for some $t>0$, then $P(t,x)$ $\implies$ $f(x+t)=f(x)$ and so, since non decreasing, $f(x)$ is constant and so must be $\frac 12$, which is not a solution. So $f(x)<1$ $\forall x>0$

Then $P(x,y)$ with $y>0$ implies $f(x+y)< f(x)$ and $f(x)$ is decreasing and so injective.

$P(x,\frac 1{f(x)})$ $\implies$ $f(x+\frac 1{f(x)})=\frac{f(x)}2$

$P(1,2x)$ $\implies$ $f(2x+1)=\frac{f(x)}2$

So $f(2x+1)=f(x+\frac 1{f(x)})$ and so, since injective, $2x+1=x+\frac 1{f(x)}$

And so $\boxed{f(x)=\frac 1{x+1}}$ $\forall x$ which indeed is a solution.
\end{solution}



\begin{solution}[by \href{https://artofproblemsolving.com/community/user/74510}{filipbitola}]
	\begin{tcolorbox}

Unfortunately it seems that I have some mistakes... I'll fix them tomorrow, it's late now.
Thanks, hatchguy for noticing the mistakes...\end{tcolorbox}

Okay, here is the (hopefully correct) corrected version of the solution:
Let for some $x>0, f(x)>1$. Then placing $y=\frac{x}{f(x)-1}$ we get that ${{f(\frac{xf(x)}{f(x)-1}})f(x)=f(\frac{xf(x)}{f(x)-1}})$ which implies that:
${f(\frac{xf(x)}{f(x)-1}})=0$. Now, placing ${x=f(\frac{xf(x)}{f(x)-1}})$ in the original equation, we get that $f(x+y)=0 \forall y\geq 0$ which can be written as: $f(t)=0 \forall t\geq \frac{xf(x)}{f(x)-1}$ where $f(x)>1$.
However, using $f(y)=2f(2y+1)$ we have $\frac{1}{2}=f(1)=2f(3)=4f(7)=...=2^{s}f(m)$ for some $s,m \in\mathbb{N}, m>\frac{xf(x)}{f(x)-1}$. Hence $\frac{1}{2}=0$, contradiction.

So, $f(x)\leq 1 \forall x\geq 0$

$f(x+y)=f(x)f(yf(x))\leq f(x) \implies f$ is non-increasing.
Assume $\exists y_{0}\neq 0, f(y_{0})=1$
$x=y=y_{0} \implies f(2y_{0})=1$. Continuing this process, we get that $f(2^{n}y_{0})=1$.  However, for sufficiently large $n$ we have that $2^{n}y_{0}>1$ implying that $1=f(2^{n}y_{0})\leq f(1)=\frac{1}{2}$, contradiction.
Hence, $f$ is strictly decreasing and therefore, injective.
So, $ f(y+1)=\frac{f(\frac{y}{2})}{2} \implies  f(2y+1)=\frac{f(y)}{2}$.
Placing $y=\frac{1}{f(x)}$ in the original equation $\implies \frac{f(x)}{2}=f(x+\frac{1}{f(x)}) \implies f(x+\frac{1}{f(x)})=f(2x+1)$ Since $f$ is injective, $f(x)=\frac{1}{x+1}$. Checking in the original equation we see that it is a solution.


Darn, too late...
\end{solution}



\begin{solution}[by \href{https://artofproblemsolving.com/community/user/60735}{hatchguy}]
	Very nice. My solution is somehow similar to yours, however I only wrote like $\frac{3}{4}$ of it during the exam.

I got $f(x) \le 1$ and $f$ is non-increasing as above.

If there exists a real number $a$ such that $f(a) =1$ we get that $f(x+a) = f(x)$ hence $f$ has period $a$. However $f$ is non-increasing and unless we have $f(x) = k$ for all $x$ (which can't happen since $f(1) =\frac{1}{2}$ and $f(0) = 1$) we must have $a = 0$. Hence, $1 > f(yf(x)) = \frac{f(x+y)}{f(x)}$ and therefore $f$ is strictly non-increasing. We can finish as above.
\end{solution}



\begin{solution}[by \href{https://artofproblemsolving.com/community/user/104682}{momo1729}]
	We can find the general solution to 3) [url=http://www.artofproblemsolving.com/Forum/viewtopic.php?f=36&t=373453]here[\/url] (and if you understand French, have a look at the last page [url=http://www.animath.fr\/IMG\/pdf\/OFM_2011-2012-envoi2-corrige.pdf]here[\/url], which is more or less "filipbitola" 's solution). This functional equation is not new...
\end{solution}
*******************************************************************************
-------------------------------------------------------------------------------

\begin{problem}[Posted by \href{https://artofproblemsolving.com/community/user/122611}{oty}]
	Suppose that $f : \mathbb R^+ \to \mathbb R^+$ is a decreasing function such that
\[f(x+y)+f(f(x)+f(y))=f(f(x+f(y))+f(y+f(x))), \quad \forall x,y \in \mathbb R^+.\]
Prove that $f(x) = f^{-1}(x).$
	\flushright \href{https://artofproblemsolving.com/community/c6h475500}{(Link to AoPS)}
\end{problem}



\begin{solution}[by \href{https://artofproblemsolving.com/community/user/29428}{pco}]
	\begin{tcolorbox}Suppose that $f : \mathbb R^+ \to \mathbb R^+$ is a decreasing function such that
\[f(x+y)+f(f(x)+f(y))=f(f(x+f(y))+f(y+f(x))), \quad \forall x,y \in \mathbb R^+.\]
Prove that $f(x) = f^{-1}(x).$\end{tcolorbox}
$f(x)$ is decreasing and so we can define two decreasing positive functions $h(x)\ge f(x)\ge  g(x)$ :
$g(x)=\lim_{z\to x^+}f(z)$
$h(x)=\lim_{z\to x^-}f(z)$

$f(f(x+f(y))+f(y+f(x)))>f(x+y)$ and so $f(x+f(y))+f(y+f(x))<x+y$
Setting $x,y\to 0^+$ in this inequality shows that $f(x)$ may be as little as we want and so that $\lim_{x\to +\infty}f(x)=0$
And so $\lim_{x\to 0^+}f(x)=+\infty$

$\lim_{y\to 0^+}f(x+y)=g(x)$
$\lim_{y\to 0^+}f(f(x)+f(y))=0$
And so $\lim_{y\to 0^+}(f(x+y)+f(f(x)+f(y)))=g(x)$

$\lim_{y\to 0+}f(x+f(y))=0$
$\lim_{y\to 0+}f(f(x)+y)=g(f(x))$
And so $\lim_{y\to 0^+}(f(x+f(y))+f(f(x)+y))=g(f(x))$
And so $\lim_{y\to 0^+}f(f(x+f(y))+f(f(x)+y))$ exists (since $LHS$ has a limit) and so $\in\{g(g(f(x))),h(g(f(x)))\}$

So : $\forall x$, either $g(x)=g(g(f(x)))$, either $g(x)=h(g(f(x)))$ and since $h(x)\ge g(x)$ $\forall x$, we get in both cases $g(x)\ge g(g(f(x)))$ 
And so $g(f(x))\ge x$ $\forall x$
And so $f(f(x))\ge x$ $\forall x$
And so (moving $x\to f(x)$ : $f(f(f(x)))\ge f(x)$ and so (since decreasing) $f(f(x))\le x$

And so $f(f(x))=x$ $\forall x$
Q.E.D.
\end{solution}



\begin{solution}[by \href{https://artofproblemsolving.com/community/user/122611}{oty}]
	Thank you Dear \begin{bolded}pco\end{bolded} , very nice :)
\end{solution}
*******************************************************************************
-------------------------------------------------------------------------------

\begin{problem}[Posted by \href{https://artofproblemsolving.com/community/user/104682}{momo1729}]
	The problem statement is very simple.
Construct a bijection between $\mathbb{N}_{0}$ the set of nonnegative integers and $\mathbb{Q}$.
	\flushright \href{https://artofproblemsolving.com/community/c6h475504}{(Link to AoPS)}
\end{problem}



\begin{solution}[by \href{https://artofproblemsolving.com/community/user/29428}{pco}]
	\begin{tcolorbox}The problem statement is very simple.
Construct a bijection between $\mathbb{N}_{0}$ the set of nonnegative integers and $\mathbb{Q}$.\end{tcolorbox}
Let $a(x)$ any bijection from $\mathbb Z\to\mathbb N_0$ such that $a(0)=0$ ( [hide="For example"]
$\forall x\ge 0$ : $a(x)=2x$
$\forall x<0$ : $a(x)=-1-2x$
[\/hide])

Any rational $x\notin\{-1,0,1\}$ may be written in a unique manner as $x=\pm\prod p_i^{z_i}$ where $z_i\in\mathbb Z^*$ and $p_i$ are distinct prime natural numbers.

Choose then $f(0)=0,f(1)=a(1),f(-1)=a(-1)$ and $f(\pm\prod p_i^{z_i})=a(\pm \prod p_i^{a(z_i)})$
\end{solution}



\begin{solution}[by \href{https://artofproblemsolving.com/community/user/104682}{momo1729}]
	Thanks.  [color=#FFFFFF](nice)[\/color]
\end{solution}
*******************************************************************************
-------------------------------------------------------------------------------

\begin{problem}[Posted by \href{https://artofproblemsolving.com/community/user/103356}{Vladislao}]
	Let $f: \mathbb{R} \to \mathbb{R}$ such that satisfies:

i) If $x<y$ then $f(x)>f(y)$.

ii) $\forall x$ the equality $x+f(x)=f(x+f(x))$ holds.

Show that for all $x$, $f(f(x))=x$.
	\flushright \href{https://artofproblemsolving.com/community/c6h475813}{(Link to AoPS)}
\end{problem}



\begin{solution}[by \href{https://artofproblemsolving.com/community/user/29428}{pco}]
	\begin{tcolorbox}Let $f: \mathbb{R} \to \mathbb{R}$ such that satisfies:

i) If $x<y$ then $f(x)>f(y)$.

ii) $\forall x$ the equality $x+f(x)=f(x+f(x))$ holds.

Show that for all $x$, $f(f(x))=x$.\end{tcolorbox}
If $f(u)=u$ for some $u\ne 0$, then ii) implies $f(2u)=2u$ and so $\frac{f(2u)-f(u)}{2u-u}=1>0$  in contradiction with i)

So $f(u)=u$ $\implies$ $u=0$

And then ii) implies $f(x)+x=0$ $\forall x$ and so $\boxed{f(x)=-x}$ $\forall x$, which indeed is a solution

And so $f(f(x))=x$, as required.
\end{solution}
*******************************************************************************
-------------------------------------------------------------------------------

\begin{problem}[Posted by \href{https://artofproblemsolving.com/community/user/87206}{safa698}]
	Find all $ n\geq 1000 $ satisfy the conditions.
$ f, h:\mathbb{Z}\to\mathbb{Z} $
$h({0}) =0$, $h({2n}) = h({n})$,  $h({2n+1}) = h({n})+1$,
$f({0}) =0$, $f({2n}) =  f({n})+ h({n})$, $f({2n+1}) = f({2n})+1$
and $f({n}) = 12$.
	\flushright \href{https://artofproblemsolving.com/community/c6h475840}{(Link to AoPS)}
\end{problem}



\begin{solution}[by \href{https://artofproblemsolving.com/community/user/29428}{pco}]
	\begin{tcolorbox}Find all $ n\geq 1000 $ satisfy the conditions.
$ f, h:\mathbb{Z}\to\mathbb{Z} $
$h({0}) =0$, $h({2n}) = h({n})$,  $h({2n+1}) = h({n})+1$,
$f({0}) =0$, $f({2n}) =  f({n})+ h({n})$, $f({2n+1}) = f({2n})+1$
and $f({n}) = 12$.\end{tcolorbox}
Each non negative integer may be written as $\sum a_k2^k$ where $a_k\in\{0,1\}$ and all $k$ distincts.

It's easy to establish that $h(\sum a_k2^k)=\sum a_k$ and $f(\sum a_k2^k)=\sum a_k(k+1)$

If $n\in[1000,1024)$, binary representation of $n$ is $\overline{11111xxxxx}_2$ and so $f(n)\ge 10+9+8+7+6>12$ and so no solution

If $n\in[1024,2048)$, we get $f(n)=11+f(n-1024)$ and so $f(n-1024)=1$ and the solution $n=1025$

If $n\in[2048,4096)$, we get $f(n)=12+f(n-2048)$ and so $f(n-2048)=0$ and the solution $n=2048$

If $n\ge 4096$, $f(n)>12$

Hence the answer : $\boxed{n\in\{1025,2048\}}$
\end{solution}
*******************************************************************************
-------------------------------------------------------------------------------

\begin{problem}[Posted by \href{https://artofproblemsolving.com/community/user/104682}{momo1729}]
	Find all functions $f$ from $\mathbb{N}$ the set of positive integers to itself such that for all positive integers $m$ and $n$ we have :
$f(m^2+f(n))=f(m)^2+n$
	\flushright \href{https://artofproblemsolving.com/community/c6h476037}{(Link to AoPS)}
\end{problem}



\begin{solution}[by \href{https://artofproblemsolving.com/community/user/29428}{pco}]
	\begin{tcolorbox}Find all functions $f$ from $\mathbb{N}$ the set of positive integers to itself such that for all positive integers $m$ and $n$ we have :
$f(m^2+f(n))=f(m)^2+n$\end{tcolorbox}
Let $P(x,y)$ be the assertion $f(x^2+f(y))=f(x)^2+y$

$f(n)$ is injective
If $f(y)>f(x)^2$ then $P(x,f(y)-f(x)^2)$ $\implies$ $f(x^2+f(f(y)-f(x)^2))=f(y)$ and so $x^2+f(f(y)-f(x)^2)=y$ and so $y>x^2$

So $y\le x^2$ implies $f(y)\le f(x)^2$ and so, since we have $x^2$ such $y$, we need (remember $f$ is injective) $f(x)^2\ge x^2$
And so $f(x)\ge x$

So $P(1,x)$ implies $f(1)^2+x\ge 1+f(x)$ and $x+a\ge f(x)\ge x$ $\forall x$ and for some $a=f(1)^2-1$

Then $P(x,1)$ $\implies$ $x^2+f(1)+a\ge f(x)^2+1\ge x^2+f(1)$ and so $x^2+b\ge f(x)^2\ge x^2$ $\forall x$ and for some $b=f(1)+a-1$

But $x^2+b\ge f(x)^2\ge x^2$ implies $f(x)=x$ for $x>\frac{b-1}2$

Choose then $x$ such that $x^2\ge x>\frac{b-1}2$ : $f(x)=x$ and $f(x^2+f(y))=x^2+f(y)$ and so $P(x,y)$ implies $x^2+f(y)=x^2+y$

 hence the result : $\boxed{f(x)=x}$ $\forall x$, which indeed is a solution
\end{solution}
*******************************************************************************
-------------------------------------------------------------------------------

\begin{problem}[Posted by \href{https://artofproblemsolving.com/community/user/68025}{Pirkuliyev Rovsen}]
	Determine function $f: \mathbb{Z}\to\mathbb{Z}$ such that $3f(f(x))-8f(x)+4x=0$


__________________________
Azerbaijan Land of the Fire 
	\flushright \href{https://artofproblemsolving.com/community/c6h476249}{(Link to AoPS)}
\end{problem}



\begin{solution}[by \href{https://artofproblemsolving.com/community/user/29428}{pco}]
	\begin{tcolorbox}Determine function $f: \mathbb{Z}\to\mathbb{Z}$ such that $3f(f(x))-8f(x)+4x=0$\end{tcolorbox}
So $f^{[n+2]}(x)=\frac 83f^{[n+1]}(x)-\frac 43f^{[n]}(x)$ and so $f^{[n]}(x)=\frac{3f(x)-2x}42^n+\frac{6x-3f(x)}4\left(\frac 23\right)^n$

And since $f^{[n]}(x)\in\mathbb Z$, we get $6x-3f(x)=0$ and so $\boxed{f(x)=2x}$ which indeed is a solution.
\end{solution}



\begin{solution}[by \href{https://artofproblemsolving.com/community/user/96264}{Carolstar9}]
	\begin{tcolorbox}$f^{[n+2]}(x)=\frac 83f^{[n+1]}(x)-\frac 43f^{[n]}(x)$\end{tcolorbox}How did you write this step?
\end{solution}



\begin{solution}[by \href{https://artofproblemsolving.com/community/user/29428}{pco}]
	\begin{tcolorbox}[quote="pco"]$f^{[n+2]}(x)=\frac 83f^{[n+1]}(x)-\frac 43f^{[n]}(x)$\end{tcolorbox}How did you write this step?\end{tcolorbox}
$f(f(x))=\frac 83f(x)-\frac 43x$ $\forall x\in\mathbb Z$

replacing $x$ by $f^{[n]}(x)$ is this expression gives $f^{[n+2]}(x)=\frac 83f^{[n+1]}(x)-\frac 43f^{[n]}(x)$
\end{solution}
*******************************************************************************
-------------------------------------------------------------------------------

\begin{problem}[Posted by \href{https://artofproblemsolving.com/community/user/126067}{mihhai}]
	Determine function $f:\mathbb{R}\to\mathbb{R}$ with

$2f(x)+f(-x)= \{\begin{array}{l}-x-3; 1\ge x \\x+3; 1<x\\ \end{array}$
	\flushright \href{https://artofproblemsolving.com/community/c6h476391}{(Link to AoPS)}
\end{problem}



\begin{solution}[by \href{https://artofproblemsolving.com/community/user/29428}{pco}]
	\begin{tcolorbox}Determine function $f:\mathbb{R}\to\mathbb{R}$ with

$2f(x)+f(-x)= \{\begin{array}{l}-x-3; 1\ge x \\x+3; 1<x\\ \end{array}$\end{tcolorbox}
1) $x>1$ and so $-x\le 1$
$x>1$ $\implies$ $2f(x)+f(-x)=x+3$
$-x\le 1$ $\implies$ $2f(-x)+f(x)=x-3$
Twice the first line less the second line gives then $f(x)=\frac x3+3$

2) $1\ge x\ge -1$ and so $-x\le 1$
$x\le 1$ $\implies$ $2f(x)+f(-x)=-x-3$
$-x\le 1$ $\implies$ $2f(-x)+f(x)=x-3$
Twice the first line less the second line gives then $f(x)=-x-1$

3) $-1>x$ and so $-x>1$
$x\le 1$ $\implies$ $2f(x)+f(-x)=-x-3$
$-x> 1$ $\implies$ $2f(-x)+f(x)=-x+3$
Twice the first line less the second line gives then $f(x)=-\frac x3-3$

\begin{bolded}Hence the answer\end{underlined}\end{bolded} :
$\forall x\in(-\infty,-1)$ : $f(x)=-\frac x3-3$

$\forall x\in[-1,+1]$ : $f(x)=-x-1$

$\forall x\in (1,+\infty)$ : $f(x)=\frac x3+3$

And it's not very difficult to check back that this indeed is a solution
\end{solution}
*******************************************************************************
-------------------------------------------------------------------------------

\begin{problem}[Posted by \href{https://artofproblemsolving.com/community/user/29034}{newsun}]
	Given $ a\in \mathbb{R}$ find all functions f such that $ f(\frac{3}{a-x})=2f(x)-1$

[color=#FF0000][Mod: Give your topics a better title than "help."][\/color]
	\flushright \href{https://artofproblemsolving.com/community/c6h476651}{(Link to AoPS)}
\end{problem}



\begin{solution}[by \href{https://artofproblemsolving.com/community/user/29428}{pco}]
	\begin{tcolorbox}Given $ a\in \mathbb{R}$ find all functions f such that $ f(\frac{3}{a-x})=2f(x)-1$\end{tcolorbox}
What are the domain and codomain of $f(x)$ and the domain of functional equation ?
\end{solution}



\begin{solution}[by \href{https://artofproblemsolving.com/community/user/29034}{newsun}]
	In $\mathbb{R}$
\end{solution}



\begin{solution}[by \href{https://artofproblemsolving.com/community/user/29428}{pco}]
	\begin{tcolorbox}In $\mathbb{R}$\end{tcolorbox}
So no solution since $a\in$ domain of functional equation and LHS in functional equation is not defined for $x=a$
\end{solution}



\begin{solution}[by \href{https://artofproblemsolving.com/community/user/29034}{newsun}]
	What about other case
\end{solution}



\begin{solution}[by \href{https://artofproblemsolving.com/community/user/29428}{pco}]
	\begin{tcolorbox}What about other case\end{tcolorbox}
Pleaaaaaase give a real problem with all informations :

Find all functions $f(x)$ from "Set 1"$\to$"Set 2" such that :

   $f(\frac{3}{a-x})=2f(x)-1$  $\forall x\in$"Set 3"

If "Set 1,2,3" are free and we can choose anything we want, this is a stupid problem.
I'll choose "Set 3"$=\{a-1\}$ and any function such that $f(3)=2f(a-1)-1$
\end{solution}



\begin{solution}[by \href{https://artofproblemsolving.com/community/user/29034}{newsun}]
	Domain is $\mathbb{R} \setminus \{a\}$.
\end{solution}



\begin{solution}[by \href{https://artofproblemsolving.com/community/user/29428}{pco}]
	So let us consider the problem :
Let $a\in \mathbb R$

Find all functions $f(x)$ from $\mathbb R\to\mathbb R$ such that $f\left(\frac 3{a-x}\right)=2f(x)-1$ $\forall x\ne a$

For easier writing, let $g(x)=f(x)-1$ from $\mathbb R\to\mathbb R$ and $h(x)=\frac 3{a-x}$ bijection from $\mathbb R\setminus\{a\}\to\mathbb R\setminus\{0\}$ and the problem is :

Find all functions $g(x)$ from $\mathbb R\to\mathbb R$ such that $g(h(x))=2g(x)$ $\forall x\ne a$

It's easy to see that the relation $x\sim y$ $\iff$ $\exists n\in\mathbb Z$ such that $y=h^{[n]}(x)$ is an equivalence relation in $\mathbb R$

Let then $r(x)$ any choice function assocating to any real $x$ a representant (unique per class) of its equivalence class.

We can consider three kinds of equivalence classes depending on $a,x$ :

Type 1\end{underlined} : Infinite classes.
there exists then a unique integer $z(x)$ for any $x$ in the class such that $x=h^{[z(x)]}(r(x))$ 
Then $g(x)$ is fully defined by knowledge of $g(r(x))$ thru $g(x)=2^{z(x)}g(r(x))$

Type 2\end{underlined} : finite classes containing both $0$ and $a$ 
This case occurs when $\exists n\in\mathbb N$ such that $h^{n}(0)=a$ (for example $a=\sqrt 3$)
Then again there exists a unique integer $z(x)$ for any $x$ in the class such that $x=h^{[z(x)]}(r(x))$ 
Then $g(x)$ is fully defined by knowledge of $g(r(x))$ thru $g(x)=2^{z(x)}g(r(x))$

Type 3\end{underlined} : finite classes cyclic.
This case occurs when $\exists u$ in the class and $n\in\mathbb N$ such that $h^{[n]}(u)=u$
Then obviously $g(x)=0$ $\forall x\in$ class.

\begin{bolded}hence the answer\end{bolded}\end{underlined} :
Let$\sim$ be the equivalence relation $x\sim y$ $\iff$ $\exists n\in\mathbb Z$ such that $y=h^{[n]}(x)$ 
Let $r(x)$ any choice function associating to any real $x$ a representant (unique per class) of its equivalence class.
Let $k(x)$ any function from $\mathbb R\to\mathbb R$
Let $A=\{x$ such that equivalence class of $x$ is of type 1 or type 2$\}$
Let $z(x)$ from $A\to\mathbb Z$ the function such that $x=h^{[z(x)]}(r(x))$
Then :

$\forall x\in A$ : $g(x)=2^{z(x)}k(r(x))$
$\forall x\notin A$ : $g(x)=0$

Notice that for some values of $a$ (for example $a=0$), $A$ is a finite set and so $f(x)=1$ for all but a finite set of values.
\end{solution}



\begin{solution}[by \href{https://artofproblemsolving.com/community/user/29034}{newsun}]
	[color=#FF0000][mod: what is the purpose to completely quote the (quite large) post just above yours?][\/color]

Dear Pco, i tried but i can not understand well this solution. Did you have another explanation for these problem like this? 
You are very kind, thanks in advanced!
\end{solution}



\begin{solution}[by \href{https://artofproblemsolving.com/community/user/64716}{mavropnevma}]
	You have to be more precise. Which is the\begin{bolded} first \end{bolded}thing you do not understand in that solution?
\end{solution}



\begin{solution}[by \href{https://artofproblemsolving.com/community/user/183430}{lafius}]
	Hello Pco,
Thanks for the solution, I really appreciate your posts !
\end{solution}



\begin{solution}[by \href{https://artofproblemsolving.com/community/user/64716}{mavropnevma}]
	He gets $g(h(x)) = f(h(x)) - 1 = f\left(\frac 3{a-x}\right) - 1 = (2f(x)-1) - 1 = 2(f(x) - 1) = 2g(x)$.

Before rushing to ask questions, it is recommended to spend a couple seconds (maybe pen in hand), trying to see if it is not easy, see obvious ...
\end{solution}
*******************************************************************************
-------------------------------------------------------------------------------

\begin{problem}[Posted by \href{https://artofproblemsolving.com/community/user/51901}{KittyOK}]
	Find all continuous functions $f: \mathbb{R^+} \to \mathbb{R}^+$ such that $f(xf(y)+yf(x))=f(f(xy))$ for all $x,y \in \mathbb{R^+}$.
	\flushright \href{https://artofproblemsolving.com/community/c6h476675}{(Link to AoPS)}
\end{problem}



\begin{solution}[by \href{https://artofproblemsolving.com/community/user/29428}{pco}]
	\begin{tcolorbox}Find all continuous functions $f: \mathbb{R^+} \to \mathbb{R}^+$ such that $f(xf(y)+yf(x))=f(f(xy))$ for all $x,y \in \mathbb{R^+}$.\end{tcolorbox}
Where is this problem coming from ?
It does not seem to be a serious real problem :(

There are infinitely many solutions and I dont know how we can give a general form.

Example of solution : 
$f(x)=\max(1,\frac 1{x^2})$

$f(x)=\max(1,\frac 1{2x}+\frac{\sqrt{x^2+8}}6)$ $\forall x\le 1$ and $f(x)=1$ $\forall x\ge 1$

This family of solution is in fact :
Choose $a,c>0$ such that $c\ge a$ and $c\ge \frac 12$
Choose $f(x)=c$ $\forall x\ge a$ and $f(x)$ as you want for $x\in(0,a]$ with the only constraints :
$f(x)\ge \max(a,\frac a{2x})$
$f(x)$ continuous and $f(a)=c$
\end{solution}



\begin{solution}[by \href{https://artofproblemsolving.com/community/user/51901}{KittyOK}]
	The answer given is $ f(x) =\begin{cases}c &\mbox{if }x\geq a\\ a+h(x) &\mbox{if }a\geq x\geq\frac{r}{2a}\\ \frac{r}{2x}+h(x) &\mbox{if }\frac{r}{2a}\geq x\end{cases} $ when $a,c,r$ are constants such that $2c^2 \ge 2ca \ge r \ge a \ge 0$ and $h: (0,a] \to [0, \infty)$ is any continuous function with $h(a)=c-a$.
\end{solution}



\begin{solution}[by \href{https://artofproblemsolving.com/community/user/29428}{pco}]
	\begin{tcolorbox}The answer given is $ f(x) =\begin{cases}c &\mbox{if }x\geq a\\ a+h(x) &\mbox{if }a\geq x\geq\frac{r}{2a}\\ \frac{r}{2x}+h(x) &\mbox{if }\frac{r}{2a}\geq x\end{cases} $ when $a,c,r$ are constants such that $2c^2 \ge 2ca \ge r \ge a \ge 0$ and $h: (0,a] \to [0, \infty)$ is any continuous function with $h(a)=c-a$.\end{tcolorbox}
I have some doubts ....

Let $f(x)=\frac{4x+1+|4x-1|}{16x}$ which is a solution.

Let us try to find $a,c,r,h(x)$ to fit your general form.

Obviously $c=\frac 12$ and $r\ge a\ge \frac 14$ and so $4r\ge 1$

Looking at $x\to 0$, third line gives $\frac r{2x}+h(x)=\frac 1{8x}$ and so on this interval $h(x)=\frac{1-4r}{8x}$  and so $4r=1$ and $r=\frac 14$

But then $r\ge a$ implies $a\le \frac 14$

And $a\ge\frac r{2a}$ implies $a\ge\frac 1{2\sqrt 2}>\frac 14$ and so contradiction

So no suitable $a,c,r,h(x)$ for the solution $f(x)=\frac{4x+1+|4x-1|}{16x}$
\end{solution}
*******************************************************************************
-------------------------------------------------------------------------------

\begin{problem}[Posted by \href{https://artofproblemsolving.com/community/user/146961}{104ntb}]
	Find all continuous functions $g:\mathbb{R^{+}}\rightarrow\mathbb{R^{+}}$ such that for any
positive real numbers $a,b,c$ satisfying $ab+bc+ca=1$,we have $\sum_{cyc}g(a)g(b)=1$.
	\flushright \href{https://artofproblemsolving.com/community/c6h476687}{(Link to AoPS)}
\end{problem}



\begin{solution}[by \href{https://artofproblemsolving.com/community/user/29428}{pco}]
	\begin{tcolorbox}Find all continuous functions $g:\mathbb{R^{+}}\rightarrow\mathbb{R^{+}}$ such that for any
positive real numbers $a,b,c$ satisfying $ab+bc+ca=1$,we have $\sum_{cyc}g(a)g(b)=1$.\end{tcolorbox}
$ab+bc+ca=1$ with $a,b,c>0$ $\iff$ $\arctan(a)+\arctan(b)+\arctan(c)=\frac{\pi}2$

Let then $f(x)$ from $(0,\frac{\pi}2)\to(0,\frac{\pi}2)$ defined as $f(x)=\arctan(g(\tan x))$ :

$f(x)$ is continous and such that $a+b+c=\frac{\pi}2$ $\implies$ $f(a)+f(b)+f(c)=\frac{\pi}2$

Let then $a,b\in(0,\frac{\pi}2)$ such that $a+b<\frac{\pi}2$

$a+b+(\frac{\pi}2-a-b)=\frac{\pi}2$ and so $f(a)+f(b)+f(\frac{\pi}2-a-b)=\frac{\pi}2$

$\frac{a+b}2+\frac{a+b}2+(\frac{\pi}2-a-b)=\pi$ and so $2f(\frac{a+b}2)+f(\frac{\pi}2-a-b)=\frac{\pi}2$

So $f(a)+f(b)=2f(\frac{a+b}2)$ $\forall a,b\in(0,\frac{\pi}2)$ such that $a+b<\frac{\pi}2$

Continuity gives then $f(x)=\alpha x+\beta$ $\forall x\in(0,\frac{\pi}4)$

Let $x\in(0,\frac{\pi}2)$ : $x+\frac{\frac{\pi}2-x}2+\frac{\frac{\pi}2-x}2=\frac{\pi}2$ and so $f(x)+2f(\frac{\frac{\pi}2-x}2)=\frac{\pi}2$ and, since $\frac{\frac{\pi}2-x}2\in(0,\frac{\pi}4)$, we previously got $f(\frac{\frac{\pi}2-x}2)=\alpha\frac{\frac{\pi}2-x}2+\beta$

And so $f(x)=\frac{\pi}2-2(\alpha\frac{\frac{\pi}2-x}2+\beta)$ $\forall x\in(0,\frac{\pi}2)$

And so $f(x)=\alpha x+(1-\alpha)\frac{\pi}6$

The constraint $f(x)\in(0,\frac{\pi}2)$ implies $\alpha\in[-\frac 12,1]$

Hence the solutions : 

$\boxed{g(x)=\tan(\alpha \arctan(x)+(1-\alpha)\frac{\pi}6)}$ $\forall x\in(0,+\infty)$  and for any $\alpha\in[-\frac 12,1]$
\end{solution}
*******************************************************************************
-------------------------------------------------------------------------------

\begin{problem}[Posted by \href{https://artofproblemsolving.com/community/user/68025}{Pirkuliyev Rovsen}]
	Find all functions $f: \mathbb{Z}\to\mathbb{Z}$ such that $f(x+f(f(y))=-f(f(x+1))-y$.


__________________________________
Azerbaijan Land of the Fire 
	\flushright \href{https://artofproblemsolving.com/community/c6h477422}{(Link to AoPS)}
\end{problem}



\begin{solution}[by \href{https://artofproblemsolving.com/community/user/29428}{pco}]
	\begin{tcolorbox}Find all functions $f: \mathbb{Z}\to\mathbb{Z}$ such that $f(x+f(f(y))=-f(f(x+1))-y$\end{tcolorbox}
Parenthesis mismatch in $LHS$. :(

1) if real problem is $f(x+f(f(y)))=-f(f(x+1))-y$ :
===================================
Let $P(x,y)$ be the assertion $f(x+f(f(y)))=-f(f(x+1))-y$
$f(x)$ is injective.

$P(0,x)$ $\implies$ $f(f(f(x))=-f(f(1))-x$

$P(f(x)-1,0)$ $\implies$ $f(f(x)-1+f(f(0)))=-f(f(f(x)))$ $=x+f(f(1))$
$P(f(x+1)-1,1)$ $\implies$ $f(f(x+1)-1+f(f(1)))=-f(f(f(x+1)))-1$ $=x+f(f(1))$

So $f(f(x)-1+f(f(0)))=f(f(x+1)-1+f(f(1)))$ and, since injective, $f(x+1)=f(x)+f(f(0))-f(f(1))$

And so $f(x)=ax+b$ for some $a,b$. Plugging this back in original equation we get $\boxed{f(x)=-x-1}$

2) if real problem is $f(x)+f(f(y))=-f(f(x+1))-y$ :
=====================================
Let $P(x,y)$ be the assertion $f(x)+f(f(y))=-f(f(x+1))-y$

$P(x,0)$ $\implies$ $f(x)+f(f(0))=-f(f(x+1))$
$P(0,x+1)$ $\implies$ $f(0)+f(f(x+1))=-f(f(1))-x-1$

Subtracting the two lines, we get $f(x)=x+1+f(f(1))+f(0)-f(f(0))$ and so $f(x)=x+a$ for some $a$
Plugging this back in original equation we get \begin{bolded}no solution\end{underlined}\end{bolded}.
\end{solution}
*******************************************************************************
-------------------------------------------------------------------------------

\begin{problem}[Posted by \href{https://artofproblemsolving.com/community/user/68025}{Pirkuliyev Rovsen}]
	Problem:Find the functions $f: \mathbb{R}\to\mathbb{R}$,$f$ continuous,and such that
$f(0)=0,f(1)=1,f(x^3){\ge}x^2f(x),f(2x){\leq}x+f(x)$.
	\flushright \href{https://artofproblemsolving.com/community/c6h477424}{(Link to AoPS)}
\end{problem}



\begin{solution}[by \href{https://artofproblemsolving.com/community/user/29428}{pco}]
	\begin{tcolorbox}Problem:Find the functions $f: \mathbb{R}\to\mathbb{R}$,$f$ continuous,and such that
$f(0)=0,f(1)=1,f(x^3){\ge}x^2f(x),f(2x){\leq}x+f(x)$.\end{tcolorbox}
Let $f(x)=g(x)+x$ and the problem becomes : $g(0)=g(1)=0$ and $g(x^3)\ge x^2g(x)$ and $g(2x)\le g(x)$

$g(x)\le g(\frac x2)$ implies $g(x)\le g(\frac x{2^n})$ and setting $n\to +\infty$ there and using continuity, we get $g(x)\le 0$ $\forall x$

Let $x\ne 0$ : $g(x^3)\ge x^2g(x)$ $\implies$ $\frac{g(x^3)}{|x|^3}\ge \frac{g(x)}{|x|}$ $\implies$ $\frac{g(x)}{|x|}\ge \frac{g(x^{\frac 13})}{|x|^{\frac 13}}$
And so $\frac{g(x)}{|x|}\ge \frac{g(x^{\frac 1{3^n}})}{|x|^{\frac 1{3^n}}}$
Setting $n\to +\infty$ there and using continuity, we get $g(x)\ge 0$ $\forall x\ne 0$

So $g(x)=0$ $\forall x$ 

And so $\boxed{f(x)=x}$ $\forall x$, which indeed is a solution.
\end{solution}
*******************************************************************************
-------------------------------------------------------------------------------

\begin{problem}[Posted by \href{https://artofproblemsolving.com/community/user/68025}{Pirkuliyev Rovsen}]
	Determine function $f: \mathbb{R}\to\mathbb{R}$ such that $f^{2n-1}(x)+x^{2n}f(x)=x^{2n-1}+x^{2n+1}$,where $n\in{N}$



_______________________________
Azerbaijan Land of the Fire 
	\flushright \href{https://artofproblemsolving.com/community/c6h477434}{(Link to AoPS)}
\end{problem}



\begin{solution}[by \href{https://artofproblemsolving.com/community/user/29428}{pco}]
	\begin{tcolorbox}Determine function $f: \mathbb{R}\to\mathbb{R}$ such that $f^{2n-1}(x)+x^{2n}f(x)=x^{2n-1}+x^{2n+1}$,where $n\in{N}$\end{tcolorbox}
Is  $f^{2n-1}(x)$ real power or composition power ?
\end{solution}



\begin{solution}[by \href{https://artofproblemsolving.com/community/user/68025}{Pirkuliyev Rovsen}]
	Hello,Pco.Yes $(f(x))^{2n-1}=f^{2n-1}(x)$
\end{solution}



\begin{solution}[by \href{https://artofproblemsolving.com/community/user/29428}{pco}]
	\begin{tcolorbox}Determine function $f: \mathbb{R}\to\mathbb{R}$ such that $f^{2n-1}(x)+x^{2n}f(x)=x^{2n-1}+x^{2n+1}$,where $n\in{N}$\end{tcolorbox}
So, this is real power and not composition power.

So $f(x)$ is a root of equation $y^{2n-1}+x^{2n}y-x^{2n-1}-x^{2n+1}=0$

It's easy to see that $LHS$ is an odd degree polynomial whose derivative is always $\ge 0$ and so has exactly one zero. And since $y=x$ is a trivial solution, this is the only one.

Hence the answer : $\boxed{f(x)=x}$
\end{solution}
*******************************************************************************
-------------------------------------------------------------------------------

\begin{problem}[Posted by \href{https://artofproblemsolving.com/community/user/117753}{Dragonboy}]
	Find all functions $f$ defined on real to real numbers  such that
$f(x^2+y^2)=f(x^2)+f(y^2)+2f(x)f(y) \forall x, y \in  R$
	\flushright \href{https://artofproblemsolving.com/community/c6h477552}{(Link to AoPS)}
\end{problem}



\begin{solution}[by \href{https://artofproblemsolving.com/community/user/29428}{pco}]
	\begin{tcolorbox}Find all functions $f$ defined on real to real numbers  such that
$f(x^2+y^2)=f(x^2)+f(y^2)+2f(x)f(y) \forall x, y \in  R$\end{tcolorbox}
Here is a rather ugly solution (dont hesitate to post a prettier one)
The only constant solutions are $f(x)=0$ $\forall x$ and $f(x)=-\frac 12$ $\forall x$
Let us from now look only for non constant solutions.

Let $P(x,y)$ be the assertion $f(x^2+y^2)=f(x^2)+f(y^2)+2f(x)f(y)$
Let $a$ such that $f(a)\ne 0$
Let $b$ such that $f(b)\ne -\frac 12$

$P(b,0)$ $\implies$ $f(0)=0$
Comparing then $P(x,a)$ and $P(-x,a)$, we get $f(x)=f(-x)$ and $f(x)$ is an even function.

$P(x,\sqrt{y^2+z^2})$ $\implies$ $f(x^2+y^2+z^2)=f(x^2)+f(y^2)+f(z^2)+2f(y)f(z)+2f(x)f(\sqrt{y^2+z^2})$
Switching $x$ and $y$ in above equality and subtracting, we get $f(y)f(z)+f(x)f(\sqrt{y^2+z^2})=f(x)f(z)+f(y)f(\sqrt{x^2+z^2})$
Setting $y=a$ in this equality, we get $f(\sqrt{x^2+z^2})=f(z)+f(x)h(z)$ for some function $h$ depe,ding only on $z$

Switching $x,z$ in this last equality, we get $f(z)+f(x)h(z)=f(x)+f(z)h(x)$
Setting $z=a$ in this equality, we get $h(x)=cf(x)+1$ for some real $c$

And so $f(\sqrt{x^2+y^2})=cf(x)f(y)+f(x)+f(y)$

If $c=0$, this becomes $f(\sqrt{x^2+y^2})=f(x)+f(y)$ and so $g(x)=f(\sqrt {|x|})$ is such that $g(x^2+y^2)=g(x^2)+g(y^2)$
So $f(x)=h(x^2)$ where $h(x)$ is any solution of additive Cauchy equation.
Plugging this back in original equation, we get $h(x^2y^2)=h(x^2)h(y^2)$ and so $h(x)\ge 0$ $\forall x\ge 0$ and so $h(x)=x$
And the solution $f(x)=x^2$.

If $c\ne 0$, the equation may be written $cf(\sqrt{x^2+y^2})+1=(cf(x)+1)(cf(y)+1)$
And so $g(x)=cf(\sqrt{|x|})+1$ is such that $g(x^2+y^2)=g(x^2)g(y^2)$
So $g(x+y)=g(x)g(y)$ $\forall x,y\ge 0$ and $g(-x)=g(x)$
It's then easy to get $g(x)=e^{d|x|}$ and $f(x)=\frac{e^{dx^2}-1}c$
Plugging this back in original equation, we get no solution (except $f(x)=0$ $\forall x$)

\begin{bolded}Hence the answer\end{underlined}\end{bolded} :
$f(x)=0$ $\forall x$
$f(x)=-\frac 12$ $\forall x$
$f(x)=x$ $\forall x$
\end{solution}



\begin{solution}[by \href{https://artofproblemsolving.com/community/user/117753}{Dragonboy}]
	\begin{tcolorbox}

So $g(x+y)=g(x)g(y)$ $\forall x,y\ge 0$ and $g(-x)=g(x)$
It's then easy to get $g(x)=e^{d|x|}$ and $f(x)=\frac{e^{dx^2}-1}c$
\end{tcolorbox}
My solution is quite like you, but different here. I have shown $g(x)=e^{d|x|}$ and $f(x)=\frac{e^{dx^2}-1}c$ for all rational $x$ and then plugging some value, i have shown it's not possible. But would you please show me how to extend this for all real not using any continuity or bound or monotonicity in this problem. Please help if it's possible.... I'll be grateful  if anyone give a different solution...
\end{solution}
*******************************************************************************
-------------------------------------------------------------------------------

\begin{problem}[Posted by \href{https://artofproblemsolving.com/community/user/88852}{myceliumful}]
	Find all functions $ f:\mathbb{Z}\rightarrow \mathbb{Z} $ such that \[f(x+y+f(y))=f(x)+2y\]
for all integers $ x,y $.
	\flushright \href{https://artofproblemsolving.com/community/c6h477857}{(Link to AoPS)}
\end{problem}



\begin{solution}[by \href{https://artofproblemsolving.com/community/user/10045}{socrates}]
	Assume $f(a)=f(b)$. Comparing $P(a,b)$ and $P(b,a)$ we get $a=b,$ that is f is injective.
So $P(0,0)$ gives $f(0)=0.$

Moreover, $P(0,y)$ gives $f(x+f(x))=2x$ so $f,$ as well as, $x+f(x)$ is onto.

So, 
$ f(x+y+f(y))=f(x)+2y=f(x)+f(y+f(y))\implies f(x+y)=f(x)+f(y), \ \forall x,y .$

This is the Cauchy equation with solution $f(x)=cx.$

Plugging in the original equation we get $c\in \{-2,1\}.$
\end{solution}



\begin{solution}[by \href{https://artofproblemsolving.com/community/user/29428}{pco}]
	\begin{tcolorbox}Assume $f(a)=f(b)$. Comparing $P(a,b)$ and $P(b,a)$ we get $a=b,$ that is f is injective.
So $P(0,0)$ gives $f(0)=0.$

Moreover, $P(0,y)$ gives $f(x+f(x))=2x$ so $f,$ as well as, $x+f(x)$ is onto.\end{tcolorbox}

Why ? 
All you proved here is that $f(x)$ can take any even value, and not any integer value.

BTW, one of your solutions, $f(x)=-2x$ is not onto :oops:
\end{solution}



\begin{solution}[by \href{https://artofproblemsolving.com/community/user/29428}{pco}]
	\begin{tcolorbox}Find all functions $ f:\mathbb{Z}\rightarrow \mathbb{Z} $ such that \[f(x+y+f(y))=f(x)+2y\]
for all integers $ x,y $.\end{tcolorbox}
Equation implies $f(x+n(y+f(y)))=f(x)+2ny$

Setting there $n=z+f(z)$, this becomes $f(x+(z+f(z))(y+f(y)))=f(x)+2y(z+f(z))$

Swapping $y,z$ and subtracting, we get $yf(z)=zf(y)$

Setting $z=1$, we get $f(y)=yf(1)$

Plugging back in original equation, we get the two solutions :
$f(x)=x$ $\forall x$
$f(x)=-2x$ $\forall x$
\end{solution}



\begin{solution}[by \href{https://artofproblemsolving.com/community/user/10045}{socrates}]
	\begin{tcolorbox}[quote="socrates"]Assume $f(a)=f(b)$. Comparing $P(a,b)$ and $P(b,a)$ we get $a=b,$ that is f is injective.
So $P(0,0)$ gives $f(0)=0.$

Moreover, $P(0,y)$ gives $f(x+f(x))=2x$ so $f,$ as well as, $x+f(x)$ is onto.\end{tcolorbox}

Why ? 
All you proved here is that $f(x)$ can take any even value, and not any integer value.

BTW, one of your solutions, $f(x)=-2x$ is not onto :oops:\end{tcolorbox}

You are right  :blush: 

I 'll try to fix my solution...
\end{solution}
*******************************************************************************
-------------------------------------------------------------------------------

\begin{problem}[Posted by \href{https://artofproblemsolving.com/community/user/68025}{Pirkuliyev Rovsen}]
	Find all functions $f: \mathbb{R}\to\mathbb{R}$ such that $f(x-y)-xf(y)\leq1-x$.


____________________________________
Azerbaijan Land of the Fire 
	\flushright \href{https://artofproblemsolving.com/community/c6h478022}{(Link to AoPS)}
\end{problem}



\begin{solution}[by \href{https://artofproblemsolving.com/community/user/29428}{pco}]
	\begin{tcolorbox}Find all functions $f: \mathbb{R}\to\mathbb{R}$ such that $f(x-y)-xf(y)\leq1-x$.\end{tcolorbox}
Let $P(x,y)$ be the assertion $f(x-y)-xf(y)\le 1-x$ $\iff$ $f(x-y)-1\le x(f(y)-1)$

$P(\frac 12,\frac 12-x)$ $\implies$ $f(x)-1\le \frac{f(\frac 12-x)-1}2$

$P(\frac 12,x)$ $\implies$ $f(\frac 12-x)-1\le \frac{f(x)-1}2$

And so $f(x)-1\le \frac{f(x)-1}4$ and so $f(x)\le 1$ $\forall x$



$P(\frac 32,\frac 32-x)$ $\implies$ $f(x)-1\le \frac{3(f(\frac 32-x)-1)}2$

$P(\frac 32,x)$ $\implies$ $f(\frac 32-x)-1\le \frac{3(f(x)-1)}2$

And so $f(x)-1\le \frac{9(f(x)-1)}4$ and so $f(x)\ge 1$ $\forall x$

And so $\boxed{f(x)=1}$ $\forall x$, which indeed is a solution.
\end{solution}



\begin{solution}[by \href{https://artofproblemsolving.com/community/user/108692}{MariusBocanu}]
	Nice solution, pco, but you can begin easier just writing $P(0,y)$, you will get $f(-y)\le 1$ for every $y$.
\end{solution}



\begin{solution}[by \href{https://artofproblemsolving.com/community/user/120756}{qua96}]
	\begin{tcolorbox}[quote="Pirkuliyev Rovsen"]Find all functions $f: \mathbb{R}\to\mathbb{R}$ such that $f(x-y)-xf(y)\leq1-x$.\end{tcolorbox}
Let $P(x,y)$ be the assertion $f(x-y)-xf(y)\le 1-x$ $\iff$ $f(x-y)-1\le x(f(y)-1)$

$P(\frac 12,\frac 12-x)$ $\implies$ $f(x)-1\le \frac{f(\frac 12-x)-1}2$

$P(\frac 12,x)$ $\implies$ $f(\frac 12-x)-1\le \frac{f(x)-1}2$

And so $f(x)-1\le \frac{f(x)-1}4$ and so $f(x)\le 1$ $\forall x$



$P(\frac 32,\frac 32-x)$ $\implies$ $f(x)-1\le \frac{3(f(\frac 32-x)-1)}2$

$P(\frac 32,x)$ $\implies$ $f(\frac 32-x)-1\le \frac{3(f(x)-1)}2$

And so $f(x)-1\le \frac{9(f(x)-1)}4$ and so $f(x)\ge 1$ $\forall x$

And so $\boxed{f(x)=1}$ $\forall x$, which indeed is a solution.\end{tcolorbox}

Patrick, could you explain to me how to you solve it ??
\end{solution}



\begin{solution}[by \href{https://artofproblemsolving.com/community/user/29428}{pco}]
	\begin{tcolorbox}Patrick, could you explain to me how to you solve it ??\end{tcolorbox}
I dont understand your question. Is there something in my solution that you dont understand ?
\end{solution}
*******************************************************************************
-------------------------------------------------------------------------------

\begin{problem}[Posted by \href{https://artofproblemsolving.com/community/user/68025}{Pirkuliyev Rovsen}]
	Find a function different from function identical so that:
$f_n(x)=f(x)$ if  $ n=2k+1$. $f_n(x)=x$ if  $ n=2k.$ where $f_n=f{\circ}f{\circ}..{\circ}f$


_______________________________
Azerbaijan Land of the Fire 
	\flushright \href{https://artofproblemsolving.com/community/c6h478024}{(Link to AoPS)}
\end{problem}



\begin{solution}[by \href{https://artofproblemsolving.com/community/user/107075}{hadikh}]
	\begin{tcolorbox}Find a function different from function identical so that:
$f_n(x)=f(x)$ if  $ n=2k+1$. $f_n(x)=x$ if  $ n=2k.$ where $f_n=f{\circ}f{\circ}..{\circ}f$
\end{tcolorbox}

All functions $f$ s.t. $f(f(x))=x$.
For example $f(x)=-x+a$ where $a\in \mathbb{R}$
\end{solution}



\begin{solution}[by \href{https://artofproblemsolving.com/community/user/89198}{chaotic_iak}]
	With $k = 2$ to the second, we have $f(f(x)) = x$ for all $x$. It's easy to see (one way is by induction) that the other conditions become redundant, hence all functions satisfying $f(f(x)) = x$ and not $f(x) = x$ satisfy the condition. (It's like very random; assign $f(0) = a$ and hence $f(a) = 0$, then continue to 1 assuming $a \neq 1$; $f(1) = b$ and $f(b) = 1$, then continuing to the next unused number; and so on.)
\end{solution}



\begin{solution}[by \href{https://artofproblemsolving.com/community/user/29428}{pco}]
	\begin{tcolorbox}...then continue to 1 assuming $a \neq 1$; $f(1) = b$ and $f(b) = 1$, then continuing to the next unused number; and so on.)\end{tcolorbox}
Just notice that this method works fine if domain is countable (for example $\mathbb N,\mathbb Z,\mathbb Q$ but must be adapted if the domain is uncountable ($\mathbb R$) : the concept of "next unused number" is quite not clear then.

The general solution of functional equation "$f(x)$ from $\mathbb R\to\mathbb R$ and $f(f(x))=x$ $\forall x$" may be written in the form :

Let $A,B,C$ a split of $\mathbb R$ (union is $\mathbb R$ and pairwise intersections are empty) such that $B$ and $C$ are equinumerous.
Let $g(x)$ any bijection from $B\to C$.
Then $f(x)$ may be defined as :
$\forall x\in A$ : $f(x)=x$
$\forall x\in B$ : $f(x)=g(x)$
$\forall x\in C$ : $f(x)=g^{-1}(x)$

The solution $f(x)=x$ $\forall x$ may be obtanined with $(A,B,C,g)=(\mathbb R,\emptyset,\emptyset, Id)$

The solution $f(x)=a-x$ $\forall x$ may be be obtanined with $(A,B,C,g)=$ $(\{\frac a2\},(-\infty,\frac a2),(\frac a2,+\infty),a-x)$
...
\end{solution}
*******************************************************************************
-------------------------------------------------------------------------------

\begin{problem}[Posted by \href{https://artofproblemsolving.com/community/user/68025}{Pirkuliyev Rovsen}]
	Find all $f: \mathbb{R_+}\to\mathbb{R_+}$continuous functions of nonconstant, such that
$f(x)f(f(x^2))+f(f(x))f(x^2)=(x^2+x)f(x)f(x^2)$


__________________________________
Azerbaijan Land of the Fire 
	\flushright \href{https://artofproblemsolving.com/community/c6h478421}{(Link to AoPS)}
\end{problem}



\begin{solution}[by \href{https://artofproblemsolving.com/community/user/29428}{pco}]
	\begin{tcolorbox}Find all $f: \mathbb{R_+}\to\mathbb{R_+}$continuous functions of nonconstant, such that
$f(x)f(f(x^2))+f(f(x))f(x^2)=(x^2+x)f(x)f(x^2)$\end{tcolorbox}

Let $g(x)=\frac{f(f(x))}{f(x)}-x$ : $g(x)$ is a continuous function from $\mathbb R^+\to\mathbb R$

Dividing both sides of functional equation by $f(x)f(x^2)$, we get $g(x^2)=-g(x)$ and continuity immediately gives $g(x)=0$ $\forall x$

So functional equation is equivalent to $f(f(x))=xf(x)$ where $f(x)$ is a continuous function from $\mathbb R^+\to\mathbb R^+$

From there, look at http://www.artofproblemsolving.com/Forum/viewtopic.php?p=1692652#p1692652

\begin{bolded}And the only two solutions are\end{underlined}\end{bolded}

$f(x)=x^{\frac{1-\sqrt 5}2}$ $\forall x>0$

$f(x)=x^{\frac{1+\sqrt 5}2}$ $\forall x>0$
\end{solution}
*******************************************************************************
-------------------------------------------------------------------------------

\begin{problem}[Posted by \href{https://artofproblemsolving.com/community/user/120756}{qua96}]
	Find all continuous function $f : (0; +\infty) \to  (0; +\infty) $ such that: If a,b,c are three adges of a triangle then : $ \frac{f(a+b-c) + f(b+c-a) + f(c+a-b)}{3} = f(\sqrt{\frac{ab+bc+ca}{3}}) $
	\flushright \href{https://artofproblemsolving.com/community/c6h478462}{(Link to AoPS)}
\end{problem}



\begin{solution}[by \href{https://artofproblemsolving.com/community/user/29428}{pco}]
	\begin{tcolorbox}Find all continuous function $f : (0; +\infty) \to  (0; +\infty) $ such that: If a,b,c are three adges of a triangle then : $ \frac{f(a+b-c) + f(b+c-a) + f(c+a-b)}{3} = f(\sqrt{\frac{ab+bc+ca}{3}}) $\end{tcolorbox}
Here is a rather ugly solution, but I found nothing prettier.

\begin{bolded}Lemma \end{underlined}\end{bolded}: 
If $g(x)$ is a continuous function from $(0,+\infty)\to[0,+\infty)$ such that $g(ax)=bg(x)$ and $g(cx)=dg(x)$ $\forall x>0$ for some $a,b,c,d>1$ such that $\frac{\ln a}{\ln c}\notin \mathbb Q$ and $\frac{\ln a}{\ln c}\ne \frac{\ln b}{\ln d}$, then : $g(x)=0$ $\forall x>0$
[hide="proof of lemma"]==================== start of proof of lemma =======================
If $g(u)>0$ $\forall x>0$, we can define a continuous function $h(x)=\ln(g(e^x))$ from $\mathbb R\to \mathbb R$
$g(ax)=bg(x)$ becomes $h(x+\ln a)=h(x)+\ln b$ and so $h(x)-\frac{\ln b}{\ln a}x$ is periodic (with period $\ln a$) and so bounded.
Same, $g(cx)=dg(x)$ implies $h(x)-\frac{\ln d}{\ln c}x$ is bounded

But $h(x)-\frac{\ln b}{\ln a}x$ and $h(x)-\frac{\ln d}{\ln c}x$ can be both bounded only if $\frac{\ln b}{\ln a}= \frac{\ln d}{\ln c}$, which is forbidden in the lemma statement

So $\exists u>0$ such that $g(u)=0$ and then $g(ua^nc^m)=0$ $\forall m,n\in\mathbb Z$
But $\frac{\ln a}{\ln c}\notin \mathbb Q$ implies $\left\{n\frac{\ln a}{\ln c}\right\}$ is dense in $[0,1]$

So $n\frac{\ln a}{\ln c}+m$ is dense in $\mathbb R$

So $n\ln a+m\ln c$ is dense in $\mathbb R$

So $\ln u+n\ln a+m\ln c$ is dense in $\mathbb R$

So $ua^nc^m$ is dense in $\mathbb R^+$

And continuity implies then $g(x)=0$ $\forall x>0$
Q.E.D.
====================  end of proof of lemma  =======================[\/hide]

\begin{bolded}Main proof\end{underlined}\end{bolded}f :
Let $P(a,b,c)$ be the assertion $f(a+b-c)+f(b+c-a)+f(c+a-b)=3f(\sqrt{\frac{ab+bc+ca}3})$ where $a,b,c>0$ and $a+b+c>2\max(a,b,c)$

$P(x,x,xy)$ $\implies$ new assertion $Q(x,y)$ : $f(x(2-y))+2f(xy)=3f(x\sqrt{\frac{2y+1}3})$ $\forall x>0$, $\forall y\in (0,2)$

Setting $x\to 0+$ in $Q(1,x)$, we get $f(2)+2\lim_{x\to 0^+}f(x)=3f(\frac 1{\sqrt 3})$ and so $\lim_{x\to 0^+}f(x)$ exists.
Let then $a=\lim_{x\to 0^+}f(x)$ and $g(x)=|f(x)-a|$, continuous function from $(0,+\infty)\to[0,+\infty)$

Setting $y\to 0^+$ in $Q(x\sqrt 3,y)$, we get $g(2\sqrt 3x)=3g(x)$ $\forall x>0$

Setting $y\to 2^-$ in $Q(x\sqrt{\frac 35},y)$, we get $g(2\sqrt{\frac 35}x)=\frac 32g(x)$ $\forall x>0$

It's easy to check that :
$2\sqrt 3,3,2\sqrt{\frac 35},\frac 32 >1$

$\frac{\ln 2\sqrt 3}{\ln 2\sqrt{\frac 35}}\notin\mathbb Q$

$\frac{\ln 2\sqrt 3}{\ln 2\sqrt{\frac 35}}\ne \frac{\ln 3}{\ln\frac 32}$

So we can apply lemma and get $g(x)=0$ $\forall x>0$ and so $\boxed{f(x)=a}$ $\forall x>0$ which indeed is a solution, whatever is $a>0$
\end{solution}
*******************************************************************************
-------------------------------------------------------------------------------

\begin{problem}[Posted by \href{https://artofproblemsolving.com/community/user/46213}{CoBa_c_Kacka}]
	Find all functions $f:\mathbb{Z}\rightarrow \mathbb{Z}$ satisfying the following:
$i)$ $f(1)=1$;
$ii)$ $f(m+n)(f(m)-f(n))=f(m-n)(f(m)+f(n))$ for all $m,n \in \mathbb{Z}$.
	\flushright \href{https://artofproblemsolving.com/community/c6h478497}{(Link to AoPS)}
\end{problem}



\begin{solution}[by \href{https://artofproblemsolving.com/community/user/29428}{pco}]
	\begin{tcolorbox}Find all functions $f:\mathbb{Z}\rightarrow \mathbb{Z}$ satisfying the following:
$i)$ $f(1)=1$;
$ii)$ $f(m+n)(f(m)-f(n))=f(m-n)(f(m)+f(n))$ for all $m,n \in \mathbb{Z}$.\end{tcolorbox}
Let $P(x,y)$ be the assertion $f(x+y)(f(x)-f(y))=f(x-y)(f(x)+f(y))$

1) $f(2)\in\{-1,0,2\}$
=====================
Let $a=f(2)$

$P(1,1)$ $\implies$ $f(0)=0$

$P(2,1)$ $\implies$ $f(3)(a-1)=a+1$ and so $a\ne 1$ and $f(3)=\frac{a+1}{a-1}$

$P(3,1)$ $\implies$ $f(4)=a^2$

$P(3,2)$ $\implies$ $f(5)(\frac{a+1}{a-1}-a)=\frac{a+1}{a-1}+a$ and so $f(5)=\frac{a^2+1}{-a^2+2a+1}$

Suppose $a\ne -1$
$P(4,1)$  $\implies$ $f(5)(a^2-1)=\frac{a+1}{a-1}(a^2+1)$ and so $f(5)=\frac{a^2+1}{(a-1)^2}$

So $\frac{a^2+1}{(a-1)^2}$ $=\frac{a^2+1}{-a^2+2a+1}$ $\iff$ $a(a-2)=0$
Q.E.D.

2) If $f(2)=-1$
===========
Paragraph 1 gives $\{f(0),f(1),f(2),f(3),f(4),f(5)\}$ $=\{0,1,-1,0,1,-1\}$

It's then easy to establish with induction using $P(x,3)$ that $f(3p+1)=1$ and $f(3p+2)=-1$ $\forall p\in\mathbb Z$
Then $P(3p-1,1)$ $\implies$ $f(3p)=0$
And so :
$f(3p)=0$ $\forall p\in\mathbb Z$
$f(3p+1)=1$ $\forall p\in\mathbb Z$
$f(3p+2)=-11$ $\forall p\in\mathbb Z$
which indeed is a solution

3) If $f(2)=0$
===========
Paragraph 1 gives $\{f(0),f(1),f(2),f(3),f(4),f(5)\}$ $=\{0,1,0,-1,0,1\}$

It's then easy to establish with induction using $P(x,2)$ that $f(2p+1)=(-1)^p$ $\forall p\in\mathbb Z$
Then $P(2p-1,2p+1)$ $\implies$ $f(4p)=0$
And $P(4p+1,1)$ $\implies$ $f(4p+2)=0$
And so :
$f(2p)=0$ $\forall p\in\mathbb Z$
$f(2p+1)=(-1)^p$ $\forall p\in\mathbb Z$
which indeed is a solution

4) If $f(2)=2$
==========
It's then easy to get with induction using $P(n,1)$ that $f(x)=x$ $\forall x\ge 0$
$P(0,x)$ gives then $f(-x)=-f(x)$
And so the solution $f(x)=x$ $\forall x$

5) Synthesis of solutions 
=================
We got exactly three solutions :

S1 : $f(x)=x$ $\forall x\in\mathbb Z$

S2 : $f(2p)=0$ and $f(2p+1)=(-1)^p$ $\forall p\in\mathbb Z$

S3 : $f(3p)=0$ and $f(3p+1)=1$ and $f(3p+2)=-1$ $\forall p\in\mathbb Z$
\end{solution}
*******************************************************************************
-------------------------------------------------------------------------------

\begin{problem}[Posted by \href{https://artofproblemsolving.com/community/user/68025}{Pirkuliyev Rovsen}]
	Determine function ${f: \mathbb{R}\to\mathbb(-\infty;1)}$  , $f(1)=-1$ such that $f(x+y)=f(x)+f(y)-f(x)f(y)$.


____________________________________
Azerbaijan Land of the Fire 
	\flushright \href{https://artofproblemsolving.com/community/c6h478824}{(Link to AoPS)}
\end{problem}



\begin{solution}[by \href{https://artofproblemsolving.com/community/user/29428}{pco}]
	\begin{tcolorbox}Determine function ${f: \mathbb{R}\to\mathbb(-\infty;1)}$  , $f(1)=-1$ such that $f(x+y)=f(x)+f(y)-f(x)f(y)$\end{tcolorbox}
Writing $g(x)=1-f(x)$, the problem is :
Find all functions $g(x)$ from $\mathbb R\to\mathbb R^+$ such that $g(1)=2$ and $g(x+y)=g(x)g(y)$

And so $g(x)=e^{h(x)\frac{\ln 2}{h(1)}}$ where $h(x)$ is any solution of Cauchy additive equation such that $h(1)\ne 0$

Hence the answer : $\boxed{f(x)=1-2^{h(x)}}$ $\forall x$ where $h(x)$ is any solution of Cauchy additive equation such that $h(1)=1$
\end{solution}
*******************************************************************************
-------------------------------------------------------------------------------

\begin{problem}[Posted by \href{https://artofproblemsolving.com/community/user/121558}{Bigwood}]
	Determine all the function $f$ from $\mathbb{Q}$ to itself which satisfies\[f(f(x+y)-y)+y=f(f(x))+f(y)\]for all $x,y$ in $\mathbb{Q}$.
	\flushright \href{https://artofproblemsolving.com/community/c6h478852}{(Link to AoPS)}
\end{problem}



\begin{solution}[by \href{https://artofproblemsolving.com/community/user/29428}{pco}]
	\begin{tcolorbox}Determine all the function $f$ from $\mathbb{Q}$ to itself which satisfies\[f(f(x+y)-y)+y=f(f(x))+f(y)\]for all $x,y$ in $\mathbb{Q}$.\end{tcolorbox}
Let $P(x,y)$ be the assertion $f(f(x+y)-y)+y=f(f(x))+f(y)$

$P(0,0)$ $\implies$ $f(0)=0$
$P(0,x)$ $\implies$ $f(f(x)-x)=f(x)-x$
$P(f(x)-x,x-f(x))$ $\implies$ $f(x-f(x))=x-f(x)$
$P(x-f(x),f(x))$ $\implies$ $f(f(x))=2f(x)-x$ and so $f(x)$ is injective.

If $f(u)=u$, then $P(x,u)$ $\implies$ $f(f(x+u)-u)=f(f(x))$ and so, since injective,  $f(x+u)=f(x)+u$ 
Setting then $u=f(x+y)-x-y$ in the previous line, we get $f(f(x+y)-y)=f(x)+f(x+y)-x-y$
Subtracting $P(x,y)$ from the above line and using $f(f(x))=2f(x)-x$, we get $f(x+y)=f(x)+f(y)$

So $f(x)=cx$ (remember we are in $\mathbb Q$) 

Plugging this in original equation, we get $c=1$ and so $\boxed{f(x)=x}$ $\forall x\in\mathbb Q$
\end{solution}
*******************************************************************************
-------------------------------------------------------------------------------

\begin{problem}[Posted by \href{https://artofproblemsolving.com/community/user/144185}{drEdrE}]
	Let $f:\mathbb{R} \rightarrow \mathbb{Z}$ be a monotone function such that
i) $f(x)=x$, for every integer $x$;
ii) $f(x+y) \ge f(x)+f(y)$, for every reals $x,y$.
Find $f$.
	\flushright \href{https://artofproblemsolving.com/community/c6h478972}{(Link to AoPS)}
\end{problem}



\begin{solution}[by \href{https://artofproblemsolving.com/community/user/29428}{pco}]
	\begin{tcolorbox}Let $f:\mathbb{R} \rightarrow \mathbb{Z}$ be a function such that
i) $f(x)=x$, for every integer $x$;
ii) $f(x+y) \ge f(x)+f(y)$, for every reals $x,y$.
Find $f$.\end{tcolorbox}
Are you sure there is no missing information ?
It's not very difficult to show that $f(x+n)=f(x)+n$ and so that (ii) need only to be proved $\forall x,y\in(0,1)$
With these conditions, it's easy to build infinitely many solutions and I did not find any general form for these solutions up to now.


Some examples of solutions :
$f(x)=\lfloor x\rfloor$

$f(x)=x$ $\forall x\in\mathbb Z$ and $f(x)=\lfloor x\rfloor-5$ $\forall x\notin\mathbb Z$

$f(x)=x$ $\forall x\in\mathbb Z$ and $f(x)=\lfloor x\rfloor+\left\lfloor 10\sin 2\pi x\right\rfloor-30$ $\forall x\notin\mathbb Z$

These solutions are built with the following (non general)  method:
Choose any non negative integer $n$ and any function $a(x)$ from $(0,1)\to\{-2n,-2n,-2n+1,...,-n-1,-n\}$
Define then $f(x)$ as :

$f(x)=x$ $\forall x\in\mathbb Z$
$f(x)=\lfloor x\rfloor+a(\{x\})$ $\forall x\notin\mathbb Z$

And these are certainly not all the solutions.
\end{solution}



\begin{solution}[by \href{https://artofproblemsolving.com/community/user/144185}{drEdrE}]
	oh, i am really sorry :D the function is monotone
\end{solution}



\begin{solution}[by \href{https://artofproblemsolving.com/community/user/29428}{pco}]
	\begin{tcolorbox}Let $f:\mathbb{R} \rightarrow \mathbb{Z}$ be a function such that
i) $f(x)=x$, for every integer $x$;
ii) $f(x+y) \ge f(x)+f(y)$, for every reals $x,y$.
Find $f$.\end{tcolorbox}
And also, since all words are important :(  :(  :
\begin{tcolorbox}oh, i am really sorry :D the function is monotone\end{tcolorbox}
$f(x+n)\ge f(x)+f(n)=f(x)+n$
$f(x)\ge f(x+n)+f(-n)=f(x+n)-n$
So $f(x+n)=f(x)+n$

So it's enough to define $f(x)$ over $[0,1)$ and since $f(0)=0$, $f(1)=1$, $f(x)\in\mathbb Z$ and $f(x)$ monotonous, $\exists  a\in[0,1)$ such that $f(x)=0$  $\forall x\in[0,a)$ and $f(x)=1$ $\forall x\in(a,1]$

So the only possible solutions are $f(x)=\lfloor x+b\rfloor$ and $f(x)=\lceil x+b\rceil$ for some $b\in[0,1)$

Plugging back in original equation, we get the only solution $\boxed{f(x)=\lfloor x\rfloor}$

and, please, make at least the effort of copying all the words when you post a problem, even those which seems to you useless.
\end{solution}



\begin{solution}[by \href{https://artofproblemsolving.com/community/user/144185}{drEdrE}]
	sorry, it was a mistake. you don't have to be that sharp :D
\end{solution}



\begin{solution}[by \href{https://artofproblemsolving.com/community/user/64716}{mavropnevma}]
	Yes, he does. It only takes you 5 seconds to check all constraints are present, but maybe it takes him 10 minutes, half an hour, or even more, to try and solve an impossible problem.
\end{solution}



\begin{solution}[by \href{https://artofproblemsolving.com/community/user/29428}{pco}]
	\begin{tcolorbox}sorry, it was a mistake. you don't have to be that sharp :D\end{tcolorbox}
You're welcome. Glad to have helped you.
\end{solution}
*******************************************************************************
-------------------------------------------------------------------------------

\begin{problem}[Posted by \href{https://artofproblemsolving.com/community/user/144185}{drEdrE}]
	Let $ f:\mathbb{R} \rightarrow \mathbb{R}$ be a function such that $f=ax^2+bx+c$. Prove that  $|A| \neq 3 $, where $A =\{x\in\mathbb{R} | f(f(x))=x\}$ . ( I denoted |A| the number of elements from A.)
	\flushright \href{https://artofproblemsolving.com/community/c6h478974}{(Link to AoPS)}
\end{problem}



\begin{solution}[by \href{https://artofproblemsolving.com/community/user/140796}{mathbuzz}]
	if i am not misreading it , it is easy
note that f(f(x))-x=0 is a 4th degree equation. with all coefficients. being real ( it is quite clear from the fact that the domain of f and range of f are R ,according to the problem)so, if n(A)=3 , then , there are 3 real roots and another non real one. then the sum of the roots would be non real ,which leads to a clear contradiction. 
\end{solution}



\begin{solution}[by \href{https://artofproblemsolving.com/community/user/144185}{drEdrE}]
	but what if the 4th root is equal to the 3rd, or to the 2nd\/ 1st ?
you must prove that the multiplicity order of each root is 1
\end{solution}



\begin{solution}[by \href{https://artofproblemsolving.com/community/user/29428}{pco}]
	\begin{tcolorbox}Let $ f:\mathbb{R} \rightarrow \mathbb{R}$ be a function such that $f=ax^2+bx+c$. Prove that  $|A| \neq 3 $, where $A =\{x\in\mathbb{R} | f(f(x))=x\}$ . ( I denoted |A| the number of elements from A.)\end{tcolorbox}
If $a=0$, then $f(f(x))-x=0$ $\iff$ $(b+1)((b-1)x+c)=0$ and $|A|\in\{0,1,+\infty\}$ and so $|A|\ne 3$
If $a\ne 0$ :

Let $B=\{x\in\mathbb R$ such that $f(x)=x\}$. $|B|\le 2$
Let $C=\{x\in\mathbb R$ such that $f(f(x))=x$ and $f(x)\ne x\}$. If $u\in C$, then $f(u)\ne u\in C$ and so $|C|$ is finite and even
$B\cap C=\emptyset$ and $B\cup C=A$ and so $|A|=|B|+|C|$

If $|C|=0$, then $|A|=|B|\le 2$
If $|C|\ne 0$, let $a,b=f(a)\ne a\in C$ : $f(a)-a=b-a$ and $f(b)-b=a-b$ and so $f(x)-x$ is a degree 2 polynomial and changes its sign over $\mathbb R$ and so $|B|=2$ and so $|A|$ is even

Hence the result
\end{solution}
*******************************************************************************
-------------------------------------------------------------------------------

\begin{problem}[Posted by \href{https://artofproblemsolving.com/community/user/72731}{goodar2006}]
	Let $g(x)$ be a polynomial of degree at least $2$ with all of its coefficients positive. Find all functions $f:\mathbb R^+ \longrightarrow \mathbb R^+$ such that
\[f(f(x)+g(x)+2y)=f(x)+g(x)+2f(y) \quad \forall x,y\in \mathbb R^+.\]

\begin{italicized}Proposed by Mohammad Jafari\end{italicized}
	\flushright \href{https://artofproblemsolving.com/community/c6h479267}{(Link to AoPS)}
\end{problem}



\begin{solution}[by \href{https://artofproblemsolving.com/community/user/29428}{pco}]
	\begin{tcolorbox}$g(x)$ is a polynomial of degree at least $2$ with all of it's coefficients positive. Find all functions $f:\mathbb R^+ \longrightarrow \mathbb R^+$ such that for all $x,y\in \mathbb R^+$

                                                                                  $f(f(x)+g(x)+2y)=f(x)+g(x)+2f(y)$\end{tcolorbox}
It's rather easy to show that $f(x)=x$ is the only continuous solution.
Could you kindly confirm us that the problem must be solved without continuity constraint ?
\end{solution}



\begin{solution}[by \href{https://artofproblemsolving.com/community/user/64868}{mahanmath}]
	\begin{tcolorbox}
Could you kindly confirm us that the problem must be solved without continuity constraint ?\end{tcolorbox}
Yes , It's can be solved without any other restrictions . :)
\end{solution}



\begin{solution}[by \href{https://artofproblemsolving.com/community/user/115063}{PhantomR}]
	Does $0\in \mathbb{R_+}$ in this problem?
\end{solution}



\begin{solution}[by \href{https://artofproblemsolving.com/community/user/72731}{goodar2006}]
	\begin{tcolorbox}Does $0\in \mathbb{R_+}$ in this problem?\end{tcolorbox}

I only translated what was written on the exam paper, but I think the answer to your question is no.
\end{solution}



\begin{solution}[by \href{https://artofproblemsolving.com/community/user/93837}{jjax}]
	\begin{tcolorbox}$g(x)$ is a polynomial of degree at least $2$ with all of its coefficients positive. Find all functions $f:\mathbb R^+ \longrightarrow \mathbb R^+$ such that for all $x,y\in \mathbb R^+$

                                                                                  $f(f(x)+g(x)+2y)=f(x)+g(x)+2f(y)$\end{tcolorbox}

Let $P(x,y):f(f(x)+g(x)+2y)=f(x)+g(x)+2f(y)$.

We will show that for some $c>0$ and $M>0$, for all $x>M$ we have $f(x)+c=f(x+c)$.
[hide]Consider a positive number $a$. Clearly, as $g$ is unbounded, there is some $b$ such that $g(b)>f(a)+g(a)$, so $f(b)+g(b)>f(a)+g(a)$.
Comparing $P(a,y),P(b,y)$ and eliminating the term $2f(y)$ gives
$f(f(b)+g(b)+2y)-f(f(a)+g(a)+2y)=f(b)+g(b)-f(a)-g(a)$.
Thus, for $c=f(b)+g(b)-f(a)-g(a)>0$, we have $f(x)+c=f(x+c)$ for all sufficiently large $x>M$.[\/hide]

Let $d=g(x_0 +c)-g(x_0)$, for some $x_0 >M$.

We show that $f(r)+d=f(r+d)$ for all reals $r$.
[hide]$P(x_0+c,y):f(f(x_0+c)+g(x_0+c)+2y)=f(x_0+c)+g(x_0+c)+2f(y)$.
For $y>M$ this becomes $f(f(x_0)+g(x_0+c)+2y)=f(x_0)+g(x_0+c)+2f(y)$.
Subtracting from this $P(x_0,y)$ gives $f(f(x_0)+g(x_0+c)+2y)-f(f(x_0)+g(x_0)+2y))=g(x_0+c)-g(x_0)$.
Then $f(z+d)-f(z)=d$ for all sufficiently large $z>N$.
Now consider any positive real $r$. Choose $X$ large enough such that $f(X)+g(X)>g(X)>N$.
Comparing $P(X,r+d)$ and $P(X,r)$ gives $f(r)+d=f(r+d)$.[\/hide]

Since some interval $[k, \infty )$ is contained in $ \{ g(x+c)-g(x), x>M \}$, we have
$f(r)+d=f(r+d)$ for any $d \geq  k$.
Consider any $p>0$. Since $f(r+p)+k=f(r+p+k)=f(r)+p+k$, we obtain that $f(r+p)=f(r)+p$ for any $r,p>0$.
Thus $f(x)=x+a$ for some $a$. A quick check shows that $f(x)=x$ is the only solution.

My thoughts:
[hide]This solution is truly distasteful; it's a rehash of the solution for China TST 2011 Quiz 2 - D1 - P1.
[url]http://www.artofproblemsolving.com/Forum/viewtopic.php?f=38&t=398427[\/url]
It's a common trick to obtain $f(x+c)=f(x)+b$ for constants $c,b$ for all $x$ then spam it everywhere.
In this particular solution, the number inside the $f$ (in this case, it's $c$) is fudged using the polynomial $g$.[\/hide]
\end{solution}



\begin{solution}[by \href{https://artofproblemsolving.com/community/user/72731}{goodar2006}]
	Congrats! Awesome!

During the exam, only three people (out of sixteen!) could solve it!
\end{solution}



\begin{solution}[by \href{https://artofproblemsolving.com/community/user/94615}{Pedram-Safaei}]
	put $h(x)=f(x)+g(x)$.because $g$ is a polynomial so $h$ is not constant now if $r,s$ are two positive reals such that we have $h(r)-h(s)=2T$ for some positive $T$ we have:$h(s)+2y+2T=h(r)+2y$ so with take an $f$ we have $f(y+T)=f(y)+T$ so for natural $n$ ,$f(y+nT)=f(y)+nT$ now for an arbitrary $x$ choose $y$ such that $y+h(x)=nT$ for some $n\geq2$ so we have $2f(y)+h(x)=f(2y+h(x))=f(y+nT)=f(y)+nT$ so $f(y)=y$.
now because $h(x+T)-h(x)=T+g(x+T)-g(x)$ so there exist $c$ such that it is surjective after $c$ so for any $l\geqq$ we have $f(y+l)=f(y)+l$ now with use the fact that $f$ has a fixed point we have $f(y)=y$ for any positive $y$.
\end{solution}



\begin{solution}[by \href{https://artofproblemsolving.com/community/user/82334}{bappa1971}]
	Let $P(x,y)$ be the given assertion. Obviously, $h(x)=f(x)+g(x)$ is not a constant function,

Outline of my solution:
Take some $x, y$ such that $h(x)>h(z)$ and take $d=\frac{h(x)-h(z)}{2}$
Then, $P(x,y)$ and $P(z,y+d)$ imply $f(y+d)=f(y)+d$
So, the function $f(x)-x$ is periodic and hence bounded.
That is, $|f(x)-x|<M$ for some real $M$.

But, $f(y)-y=\frac{f(2y+h(x))-(2y+h(x))}{2}=\frac{f(4y+3h(x))-(4y+3h(x))}{4}$
$=\cdots =\frac{f(2^n y + (2^n - 1)h(x)) - (2^n y + (2^n - 1)h(x)) }{2^n}$
So, $|f(y)-y|<\frac{M}{2^n}$  for large $n$.
So, $f(y)=y$, done!
\end{solution}



\begin{solution}[by \href{https://artofproblemsolving.com/community/user/29428}{pco}]
	\begin{tcolorbox}So, the function $f(x)-x$ is periodic and hence bounded.\end{tcolorbox}
"periodic implies bounded" is true only if you have continuity (there are a lot of periodic unbounded non continuous functions)
And we dont have.
\end{solution}



\begin{solution}[by \href{https://artofproblemsolving.com/community/user/334227}{reveryu}]
	\begin{tcolorbox}
now because $h(x+T)-h(x)=T+g(x+T)-g(x)$ so there exist $c$ such that it is surjective after $c$ so for any $l\geqq$ we have $f(y+l)=f(y)+l$ now with use the fact that $f$ has a fixed point we have $f(y)=y$ for any positive $y$.\end{tcolorbox}

Can anyone gives a detailed explanation for these lines please?


\end{solution}



\begin{solution}[by \href{https://artofproblemsolving.com/community/user/243741}{anantmudgal09}]
	This is a really nice problem! :)

\begin{tcolorbox}Let $g(x)$ be a polynomial of degree at least $2$ with all of its coefficients positive. Find all functions $f:\mathbb R^+ \longrightarrow \mathbb R^+$ such that
\[f(f(x)+g(x)+2y)=f(x)+g(x)+2f(y) \quad \forall x,y\in \mathbb R^+.\]

\begin{italicized}Proposed by Mohammad Jafari\end{italicized}\end{tcolorbox}

Let $h(x) \overset{\text{def}}{:=} f(x)-x$, $p(x) \overset{\text{def}}{:=} x+g(x)$ and $t(x) \overset{\text{def}}{:=} h(x)+p(x)$ for all $x>0$; then we obtain $$h(h(x)+p(x)+2y)=2h(y)$$ for all $x,y>0$. Swap $x$ with a variable $z>0$; then $$h(h(x)+p(x)+2y)=h(h(z)+p(z)+2y)=2h(y)$$ for all $x,y,z>0$. Notice that if $h$ is injective, then $f(x)=-g(x)+c$ which is false since $g(x) \rightarrow \infty$ as $x \rightarrow \infty$ while $f(x)>0$ always holds. 

Pick $a,b$ with $t(a) \ne t(b)$ and set $c=|t(a)-t(b)|$ then for all $x>N=\min(t(a), t(b), 0)$ we have $h(x)=h(x+c)$. Now put $y=\varepsilon$ and $x$ to be large enough so that $g(x)>N$ (hence $t(x) \ge p(x)-x=g(x)>N$), and substitute $y$ with $\varepsilon+c$; then $f(c+\varepsilon)=f(\varepsilon)$. Thus, $h$ is periodic with period $c$; i.e. $h(x+c)=h(x)$ for all $x>0$.

Now notice that $h(x+c)+g(x+c)-h(x)-g(x)=g(x+c)-g(x)$ is a period of $h$, for all $x>0$. Thus, all points in an entire interval are periods of $h$ proving $h$ is the constant function. Clearly, $h(h(1)+p(1)+2)=2h(1)$ shows that $h \equiv 0$. Hence $f$ is the identity function and it clearly works! $\blacksquare$
\end{solution}
*******************************************************************************
-------------------------------------------------------------------------------

\begin{problem}[Posted by \href{https://artofproblemsolving.com/community/user/112260}{robinson123}]
	Find all functions  $ f: R \to R $ satisfying 
$ f(\frac{xf(y)}{2})+f(\frac{yf(x)}{2})=4xy, \forall x,y \in R $
	\flushright \href{https://artofproblemsolving.com/community/c6h479408}{(Link to AoPS)}
\end{problem}



\begin{solution}[by \href{https://artofproblemsolving.com/community/user/29428}{pco}]
	\begin{tcolorbox}Find all functions  $ f: R \to R $ satisfying 
$ f(\frac{xf(y)}{2})+f(\frac{yf(x)}{2})=4xy, \forall x,y \in R $\end{tcolorbox}
For easier writing, let $g(x)=\frac{f(x)}2$ and $P(x,y)$ be the assertion $g(xg(y))+g(yg(x))=2xy$

1) $g(x)$ is an odd injective function and $g(0)=0$
===================================
If $g(a)=g(b)=c$ then :
(1) : $P(a,a)$ $\implies$ $g(ac)=a^2$
(2) : $P(b,b)$ $\implies$ $g(bc)=b^2$
(3) : $P(a,b)$ $\implies$ $g(ac)+g(bc)=2ab$
(1)+(2)-(3) $\implies$ $a=b$
So $g(x)$ is injective

$P(0,0)$ $\implies$ $g(0)=0$ and so (since injective) $g(x)=0$ $\iff$ $x=0$

Comparing $P(x,x)$ and $P(-x,-x)$ and using injectivity, we get $xg(x)=-xg(-x)$ and so (since $g(0)=0$) $g(-x)=-g(x)$ $\forall x$ and $g(x)$ is an odd function.
Q.E.D.

2) $g(x)=x$ $\forall x$ or $g(x)=-x$ $\forall x$
==========================
Let $x\ne 0$ :
$P(x,x)$ $\implies$ $g(xg(x))=x^2$

$P(xg(x),\frac 1x)$ $\implies$ $g(xg(x)g(\frac 1x))+g(\frac 1xg(xg(x)))=2g(x)$ $\implies$ $g(xg(x)g(\frac 1x))=g(x)$

And so, since injective : $g(\frac 1x)=\frac 1{g(x)}$
Setting $x=1$ in above equation, we get $g(1)\in\{-1,1\}$

$P(x,\frac 1x)$ $\implies$ $g(\frac x{g(x)})+\frac 1{g(\frac x{g(x)})}=2$ and so $g(\frac x{g(x)})=1$

If $g(1)=1$, we get $g(\frac x{g(x)})=g(1)$ and so (since injective) $g(x)=x$ $\forall x\ne 0$ and so $g(x)=x$ $\forall x$, which indeed is a solution

If $g(1)=-1$, we get $g(\frac x{g(x)})=-g(1)$ and so (since odd and injective) $g(x)=-x$ $\forall x\ne 0$ and so $g(x)=-x$ $\forall x$, which indeed is a solution
Q.E.D.

3) synthesis of solutions
=================
Hence the two solutions : $\boxed{f(x)=2x}$ $\forall x$ and $\boxed{f(x)=-2x}$ $\forall x$
\end{solution}



\begin{solution}[by \href{https://artofproblemsolving.com/community/user/114130}{61plus}]
	why does $ g(xg(x)g(\frac{1}{x}))+g(\frac{1}{x}g(xg(x)))=2g(x) $$ \implies $ $ g(xg(x)g(\frac{1}{x}))=g(x) $
\end{solution}



\begin{solution}[by \href{https://artofproblemsolving.com/community/user/29428}{pco}]
	\begin{tcolorbox}why does $ g(xg(x)g(\frac{1}{x}))+g(\frac{1}{x}g(xg(x)))=2g(x) $$ \implies $ $ g(xg(x)g(\frac{1}{x}))=g(x) $\end{tcolorbox}
Because $g(xg(x))=x^2$ (previous line)  and so $g(\frac{1}{x}g(xg(x)))=g(\frac{1}{x}x^2)=g(x)$

So $ g(xg(x)g(\frac{1}{x}))+g(\frac{1}{x}g(xg(x)))=2g(x) $ becomes :

$ g(xg(x)g(\frac{1}{x}))+g(x)=2g(x) $ and so $ g(xg(x)g(\frac{1}{x}))=g(x) $
\end{solution}



\begin{solution}[by \href{https://artofproblemsolving.com/community/user/114130}{61plus}]
	oh thanks i forgot about the first line
\end{solution}
*******************************************************************************
-------------------------------------------------------------------------------

\begin{problem}[Posted by \href{https://artofproblemsolving.com/community/user/133488}{Nostalgius}]
	Find all function $f :\mathbb{R} \rightarrow \mathbb{R}$ which satisfied $f(f(x)+xf(y))=3f(x)+4xy$ $\forall x,y \in \mathbb{R}$
	\flushright \href{https://artofproblemsolving.com/community/c6h479690}{(Link to AoPS)}
\end{problem}



\begin{solution}[by \href{https://artofproblemsolving.com/community/user/29428}{pco}]
	\begin{tcolorbox}Find all function $f :\mathbb{R} \rightarrow \mathbb{R}$ which satisfied $f(f(x)+xf(y))=3f(x)+4xy$ $\forall x,y \in \mathbb{R}$\end{tcolorbox}
Let $P(x,y)$ be the assertion $f(f(x)+xf(y))=3f(x)+4xy$

If $f(a)=f(b)$ for some $a,b$, comparaison of $P(1,a)$ and $P(1,b)$ implies $a=b$ and so $f(x)$ is injective
$P(\frac 14,y-3f(\frac 14))$ $\implies$ $f(\text{something})=y$ and so $f(x)$ is surjective.

If $f(0)=0$, $P(x,0)$ $\implies$ $f(f(x))=3f(x)$ and so, since surjective, $f(x)=3x$ $\forall x$, which is not a soluton.

If $f(0)\ne 0$, let then $u\ne 0$ such that $f(u)=0$
$P(u,0)$ $\implies$ $f(uf(0))=0=f(u)$ and so, since injective, $f(0)=1$
$P(0,0)$ $\implies$ $f(1)=3$
$P(-1,-1)$ $\implies$ $f(-1)=-1$
$P(-1,u)$ $\implies$ $u=-\frac 12$

$P(x,-\frac 12)$ $\implies$ $f(f(x))=3f(x)-2x$

Let then $x\in\mathbb R$ and $v$ such that $f(v)=-(x+1)$
$P(-1,v)$ $\implies$ $v=-\frac{f(x)+3}4$ and so $f(-\frac{f(x)+3}4)=-x-1$ $\forall x$

So $f(-x-1)=f(f(-\frac{f(x)+3}4))$ $=3f(-\frac{f(x)+3}4)-2(-\frac{f(x)+3}4)$ $=3(-x-1)+\frac{f(x)+3}2$ $=-3x-\frac 32+\frac{f(x)}2$

So  (a) : $2f(-x-1)=-6x-3+f(x)$
Replacing $x$ by $-x-1$ in this last equation, we get (b) : $2f(x)=6x+3+f(-x-1)$
(a) + 2(b) gives $\boxed{f(x)=2x+1}$ $\forall x$ which indeed is a solution
\end{solution}



\begin{solution}[by \href{https://artofproblemsolving.com/community/user/115063}{PhantomR}]
	Set $x=1$ to obtain $f(f(1)+f(y))=3f(1)+4y, \ \forall y\in\mathbb{R}$ \begin{bolded}(1)\end{bolded}. Since the left hand side takes all real values (as $3f(1)$ is constant), our function is surjective. Now take $m,n$ such that $f(m)=f(n)$. Then (1) implies $3f(1)+4m=3f(1)+4n \Leftrightarrow m=n$, from where we deduce that $f$ is injective. Hence $f$ is bijective.

So $!\exists a$ such as $f(a)=0$. Put $x=a$ in the original equation to obtain $f(af(y))=4ay$ \begin{bolded}(2)\end{bolded}. Now consider $y_1$ for which $f(y_1)=1$. (2) gives $4ay_1=0 \Leftrightarrow ay_1=0$.

\begin{bolded}Case 1:\end{bolded} $a=0$.
This implies $f(0)=0$. If we take $y=0$ in the original equation we'll get $f(f(x))=3f(x)$ and since $f(x)$ is surjective we have $f(y)=3y,\forall y\in\mathbb{R}$, which is not a solution (check!).

\begin{bolded}Case 2:\end{bolded} $y_1=0$.
Equivalently, we have $f(0)=1$.


Mr. \begin{bolded}pco\end{bolded}, do you know any way to continue from here?
\end{solution}



\begin{solution}[by \href{https://artofproblemsolving.com/community/user/133488}{Nostalgius}]
	\begin{bolded}@pco\end{bolded} I'm so impressed your solution . I've a question , How could you know the step following there ? Can you tell me your idea ? Thank you !
\begin{tcolorbox}
Let then $x\in\mathbb R$ and $v$ such that $f(v)=-(x+1)$
$P(-1,v)$ $\implies$ $v=-\frac{f(x)+3}4$ and so $f(-\frac{f(x)+3}4)=-x-1$ $\forall x$
\end{tcolorbox}
\end{solution}



\begin{solution}[by \href{https://artofproblemsolving.com/community/user/29428}{pco}]
	\begin{tcolorbox}\begin{bolded}@pco\end{bolded} I'm so impressed your solution . I've a question , How could you know the step following there ? Can you tell me your idea ? Thank you !\end{tcolorbox}
Thanks.
In fact, we had :
$f(x)+xf(y)=z$ $\iff$ $f(z)=3f(x)+4xy$

$f(y)=\frac{z-f(x)}x$ $\iff$ $y=\frac{f(z)-3f(x)}{4x}$

$f(\frac{f(z)-3f(x)}{4x})=\frac{z-f(x)}x$

And applying then $f(f(t))=3f(t)-2t$ :

$f(\frac{z-f(x)}x)=\frac{6z-3f(x)-f(z)}{2x}$

And I tried different values of $x$ to see if some interesting result was possible :

$x=1$ $\implies$ $f(z-3)=\frac{6z-9-f(z)}{2}$

$x=-\frac 12$ $\implies$ $f(-2z)=f(z)-6z$

$x=-1$ $\implies$ $f(-z-1)=\frac{6z+3-f(z)}{-2}$

And indeed $x=-1$ gave an interesting result (although I tested $-1$ very late)
\end{solution}
*******************************************************************************
-------------------------------------------------------------------------------

\begin{problem}[Posted by \href{https://artofproblemsolving.com/community/user/89198}{chaotic_iak}]
	Find all functions $f : \mathbb{R}^+ \rightarrow \mathbb{R}^+$ satisfying $f(x)^{f(y)} = f(x+y)$ for all $x,y \in \mathbb{R}^+$.
	\flushright \href{https://artofproblemsolving.com/community/c6h480075}{(Link to AoPS)}
\end{problem}



\begin{solution}[by \href{https://artofproblemsolving.com/community/user/29428}{pco}]
	\begin{tcolorbox}Find all functions $f : \mathbb{R}^+ \rightarrow \mathbb{R}^+$ satisfying $f(x)^{f(y)} = f(x+y)$ for all $x,y \in \mathbb{R}^+$.\end{tcolorbox}
Let $P(x,y)$ be the assertion $f(x)^{f(y)}=f(x+y)$

Comparing $P(x,y)$ and $P(y,x)$, we get $f(x)^{f(y)}=f(y)^{f(x)}$ and so $\frac{\ln f(x)}{f(x)}=c$ constant. and so $f(x)$ can take at most two values.

If $f(x)$ takes one value $u$, we get $u^u=u$ and so $u=1$, which indeed is a solution

If $f(x)$ takes two different values $u,v$, choosing $x,y$ such that $f(x)=f(x+y)$ we get hat one of $u,v$ is $1$

Let then $a$ such that $f(a)=1$ :
$P(a,x-a)$ $\implies$ $f(x)=1$ $\forall x\ge a$
$P(x,a)$ $\implies$ $f(x)=f(x+a)$
And so $f(x)=1$ $\forall x$

Hence the answer $\boxed{f(x)=1}$ $\forall x$
\end{solution}
*******************************************************************************
-------------------------------------------------------------------------------

\begin{problem}[Posted by \href{https://artofproblemsolving.com/community/user/89198}{chaotic_iak}]
	Find all functions $f : \mathbb{R} \rightarrow \mathbb{R}^+$ satisfying $f(x)^{f(y)} = f(x+y)$ for all $x,y \in \mathbb{R}$.
	\flushright \href{https://artofproblemsolving.com/community/c6h480076}{(Link to AoPS)}
\end{problem}



\begin{solution}[by \href{https://artofproblemsolving.com/community/user/29428}{pco}]
	\begin{tcolorbox}Find all functions $f : \mathbb{R} \rightarrow \mathbb{R}^+$ satisfying $f(x)^{f(y)} = f(x+y)$ for all $x,y \in \mathbb{R}$.\end{tcolorbox}
Let $P(x,y)$ be the assertion $f(x)^{f(y)}=f(x+y)$


Comparing $P(x,y)$ and $P(y,x)$, we get $f(x)^{f(y)}=f(y)^{f(x)}$ and so $\frac{\ln f(x)}{f(x)}=c$ constant. and so $f(x)$ can take at most two values.

If $f(x)$ takes one value $u$, we get $u^u=u$ and so $u=1$, which indeed is a solution

If $f(x)$ takes two different values $u,v$, choosing $x,y$ such that $f(x)=f(x+y)$ we get hat one of $u,v$ is $1$

Let then $a$ such that $f(a)=1$ :
$P(a,x-a)$ $\implies$ $f(x)=1$ $\forall x$

Hence the answer $\boxed{f(x)=1}$ $\forall x$
\end{solution}



\begin{solution}[by \href{https://artofproblemsolving.com/community/user/84677}{andreass}]
	For $x=y=0 \Rightarrow f(0)^{f(0)}=f(0) \Rightarrow f(0)=1$.
Now for $y=-x\Rightarrow f(x)^{f(-x)}=f(x-x)=f(0)=1 \Rightarrow f(-x)=0$, rejected, or $f(x)=1$ which is our solution $\forall x \in \mathbb{R}$.
\end{solution}
*******************************************************************************
-------------------------------------------------------------------------------

\begin{problem}[Posted by \href{https://artofproblemsolving.com/community/user/89198}{chaotic_iak}]
	Find all functions $f : \mathbb{R}^+ \rightarrow \mathbb{R}^+$ satisfying $f(x)^{f(y)} = f(x^y)$ for all $x,y \in \mathbb{R}^+$.
	\flushright \href{https://artofproblemsolving.com/community/c6h480079}{(Link to AoPS)}
\end{problem}



\begin{solution}[by \href{https://artofproblemsolving.com/community/user/29428}{pco}]
	\begin{tcolorbox}Find all functions $f : \mathbb{R}^+ \rightarrow \mathbb{R}^+$ satisfying $f(x)^{f(y)} = f(x^y)$ for all $x,y \in \mathbb{R}^+$.\end{tcolorbox}
$f(x)=1$ $\forall x$ is a solution. Let us from now look only for non all-one solutions.

Let $P(x,y)$ be the assertion $f(x)^{f(y)}=f(x^y)$
Let $u>0$ such that $f(u)\ne 1$

$P(u^x,y)$ $\implies$ $f(u^x)^{f(y)}=f(u^{xy})$ and so $f(u)^{f(x)f(y)}=f(u^{xy})$ 

$P(u,xy)$ $\implies$ $f(u)^{f(xy)}=f(u^{xy})$

And so $f(u)^{f(x)f(y)}=f(u)^{f(xy)}$ and so $f(xy)=f(x)f(y)$ $\forall x,y>0$ and so $f(x)=e^{h(\ln x)}$ for some $h(x)$ from $\mathbb R\to\mathbb R$ solution of additive Cauchy equation.

Plugging this back in original equation, we get $h(\ln x)e^{h(\ln y)}=h(y\ln x)$ $\forall x,y>0$

And so  $\forall x,y$ $h(x)e^{h(y)}=h(xe^y)$

This implies $h(e^x)e^{h(y)}=h(e^{x+y})=h(e^y)e^{h(x)}$ and so $h(e^x)e^{-h(x)}=c$ constant.
So sign of $h(x)$ is constant over $\mathbb R^+$, and so $h(x)=cx$ (since either upper bounded or lower bounded on a non empty open interval).

Plugging this back in original equation, we get $c=1$ and so $f(x)=x$ which indeed is a solution.

\begin{bolded}Hence the solutions\end{underlined}\end{bolded} :

$f(x)=1$ $\forall x$
$f(x)=x$ $\forall x$
\end{solution}



\begin{solution}[by \href{https://artofproblemsolving.com/community/user/84677}{andreass}]
	Another solution: as pco noted above $f(xy)=f(x)f(y)$.
Now $f(u^{x+y})=f(u)^{f(x+y)}$.
However from above $f(u^{x+y})=f(u^xu^y)=f(u^x)f(u^y)=f(u)^{f(x)}f(u)^{f(y)}=f(u)^{f(x)+f(y)}$ 
i.e. $f(x+y)=f(x)+f(y) \forall x,y \in \mathbb{R}^+$.
Obviously for $n \in \mathbb{N}$ $f(nx)=f(x+x+x+.....+x)=f(x)+f(x)+....+f(x)=nf(x)$.
Also we easily note that $mf(nx)=nf(mx) \Rightarrow f(nx)=\frac{n}{m} f(mx)$ substituting $y=nx$ gives $f(y)=\frac{n}{m} f(\frac{m}{n} x) \Rightarrow f(qx)=qf(x) \forall x \in \mathbb{R}^+, q \in \mathbb{Q}^+$.
$f(1)=1$ therefore $f(q)=q \forall q \in \mathbb{Q}^+$ and it is left to generalise it for reals.
Now \begin{tcolorbox}Find all functions $f : \mathbb{R}^+ \rightarrow \mathbb{R}^+$\end{tcolorbox}
hence from $f(x+y)=f(x)+f(y)$, since $f(y)>0$, follows $f(x+y)>f(x)$ but also $x+y>x$ hence from $a>b$ follows $f(a)>f(b)$, that is, the function is an increasing one.
Let $q$ and $p$ be two positive rational number and $r$ a positive real such that $q<r<p$ hence $f(q)<f(r)<f(p) \Rightarrow q<f(r)<p$. Now we can get $q$ and $p$ as close to $r$ as we want, up to the point that they are infinitesimally smaller and larger than it respectively hence $f(r)$ is bounded to be equal to $r$ and we have alternatively proved the second solution $f(x)\equiv x$.
\end{solution}



\begin{solution}[by \href{https://artofproblemsolving.com/community/user/29428}{pco}]
	\begin{tcolorbox} [quote="chaotic_iak"]Find all functions $f : \mathbb{R}^+ \rightarrow \mathbb{R}^+$\end{tcolorbox}
hence from $f(x+y)=f(x)+f(y)$, since $f(y)>0$, follows $f(x+y)>f(x)$ but also $x+y>x$ hence from $a>b$ follows $f(a)>f(b)$, that is, the function is an increasing one.
Let $q$ and $p$ be two positive rational number and $r$ a positive real such that $q<r<p$ hence $f(q)<f(r)<f(p) \Rightarrow q<f(r)<p$. Now we can get $q$ and $p$ as close to $r$ as we want, up to the point that they are infinitesimally smaller and larger than it respectively hence $f(r)$ is bounded to be equal to $r$ and we have alternatively proved the second solution $f(x)\equiv x$.\end{tcolorbox}

But we dont have $f(x+y)=f(x)+f(y)$. We have $f(xy)=f(x)f(y)$ 

And if you take $f(x)=e^{g(\ln x)}$, then we have $g(x+y)=g(x)+g(y)$ but we no longer have immediately $g(x)$ from $\mathbb R^+\to\mathbb R^+$
\end{solution}



\begin{solution}[by \href{https://artofproblemsolving.com/community/user/84677}{andreass}]
	\begin{tcolorbox}
But we dont have $f(x+y)=f(x)+f(y)$. We have $f(xy)=f(x)f(y)$ 
\end{tcolorbox}

\begin{tcolorbox}
Now $f(u^{x+y})=f(u)^{f(x+y)}$.
However from above $f(u^{x+y})=f(u^xu^y)=f(u^x)f(u^y)=f(u)^{f(x)}f(u)^{f(y)}=f(u)^{f(x)+f(y)}$ 


i.e. $f(x+y)=f(x)+f(y) \forall x,y \in \mathbb{R}^+$\end{underlined}.

\end{tcolorbox}

I showed $f(x+y)=f(x)+f(y)$ but i hope it's correct :)
\end{solution}



\begin{solution}[by \href{https://artofproblemsolving.com/community/user/93837}{jjax}]
	Russian mathematical Olympiad 93, Grade 11 Q3.
[url]http://www.imomath.com\/othercomp\/Rus\/RusMO93.pdf[\/url]
\end{solution}
*******************************************************************************
-------------------------------------------------------------------------------

\begin{problem}[Posted by \href{https://artofproblemsolving.com/community/user/94121}{swanswan}]
	Does there exist a function f, with both codomain and domain being the set of natural numbers, such that the function  is surjective and n divides f(1)+f(2)+...+f(n) for all natural number n?
	\flushright \href{https://artofproblemsolving.com/community/c6h480330}{(Link to AoPS)}
\end{problem}



\begin{solution}[by \href{https://artofproblemsolving.com/community/user/29428}{pco}]
	\begin{tcolorbox}Does there exist a function f, with both codomain and domain being the set of natural numbers, such that the function  is surjective and n divides f(1)+f(2)+...+f(n) for all natural number n?\end{tcolorbox}
Yes, choose for example $f(2n)=3n$ and $f(2n-1)=n$ $\forall n\in\mathbb N$
\end{solution}



\begin{solution}[by \href{https://artofproblemsolving.com/community/user/94121}{swanswan}]
	what if f is bijective?
\end{solution}



\begin{solution}[by \href{https://artofproblemsolving.com/community/user/29428}{pco}]
	You are welcome, glad to have helped you.

Does the problem you got in your exam \/ training session say "surjective" or "bijective" ?
\end{solution}



\begin{solution}[by \href{https://artofproblemsolving.com/community/user/64716}{mavropnevma}]
	The answer is YES, even for $f$ bijective, and is a quite famous problem. Denote $S_n = \sum_{k=1}^n f(k)$. Start with $f(1) = 1$; if $f$ is built up to $f(n)$, take $m$ the least positive integer not yet used as a value for $f$, and consider the system of congruences $S_n + x \equiv 0 \pmod{n+1}$ and $S_n + x + m \equiv 0 \pmod{n+2}$. Since $\gcd(n+1,n+2) = 1$, this system has, by the Chinese Remainder Theorem, a positive integer solution $x > \max\{S_n,m\}$. Take now $f(n+1) = x$ and $f(n+2) = m$. This ensures both the injectivity and surjectivity of the function $f$ that we build.
\end{solution}



\begin{solution}[by \href{https://artofproblemsolving.com/community/user/29428}{pco}]
	Thanks mavropnevma.
Here is another simple algorithm to build such a bijective solution, using a second function $g(n)$  :

$f(1)=1$ and $g(1)=1$
$f(2)=3$ and $g(2)=2$
$\forall n>1$ :
If $g(n)\in f([1,n])$ and $g(n)=f(n)$, then $f(n+1)=f(n-1)+3$ and $g(n+1)=g(n)+1$
If $g(n)\in f([1,n])$ and $g(n)\ne f(n)$, then $f(n+1)=f(n)+2$ and $g(n+1)=g(n)+1$
If $g(n)\not\in f([1,n])$, then $f(n+1)=g(n)$ and $g(n+1)=g(n)$
\end{solution}
*******************************************************************************
-------------------------------------------------------------------------------

\begin{problem}[Posted by \href{https://artofproblemsolving.com/community/user/92794}{lypnol}]
	Find all functions $f:\mathbb{R}^{*}\to\mathbb{R}$  that satisfy: 
$\left(\forall  (x,y) \in \mathbb{R}^{*}\times\mathbb{R}^{*}\right)  :f(xy)=\frac{1}{x}f(y)+yf(x)$
where: $\left(\mathbb{R}^{*}=\mathbb{R}-\{0\}\right)$
	\flushright \href{https://artofproblemsolving.com/community/c6h480461}{(Link to AoPS)}
\end{problem}



\begin{solution}[by \href{https://artofproblemsolving.com/community/user/29428}{pco}]
	\begin{tcolorbox}Find all functions $f:\mathbb{R}^{*}\to\mathbb{R}$  that satisfy: 
$\left(\forall  (x,y) \in \mathbb{R}^{*}\times\mathbb{R}^{*}\right)  :f(xy)=\frac{1}{x}f(y)+yf(x)$
where: $\left(\mathbb{R}^{*}=\mathbb{R}-\{0\}\right)$\end{tcolorbox}
Let $P(x,y)$ be the assertion $f(xy)=\frac 1xf(y)+yf(x)$

$P(x,2)$ $\implies$ $f(2x)=\frac 1xf(2)+2f(x)$

$P(2,x)$ $\implies$ $f(2x)=\frac 12f(x)+xf(2)$

Subtracting, we get $f(x)=\frac{2f(2)}3(x-\frac 1x)$ and so 

$\boxed{f(x)=a\left(x-\frac 1x\right)}$ $\forall x\ne 0$ which indeed is a solution whatever is $a\in\mathbb R$
\end{solution}
*******************************************************************************
-------------------------------------------------------------------------------

\begin{problem}[Posted by \href{https://artofproblemsolving.com/community/user/122611}{oty}]
	Find numerical sequence $(u_{n})$ ,$n\in N$ s.t : [list]1-$(\forall n\in N): u_{n}\geq{1}$ [\/list] , [list]2-$(\forall n\in N): (n+1)u_{n}\leq{n.u_{n+1}}$[\/list] , [list]3-$(\forall n,m\in N):u_{nm}=u_{n}.u_{m}$[\/list]
	\flushright \href{https://artofproblemsolving.com/community/c6h480787}{(Link to AoPS)}
\end{problem}



\begin{solution}[by \href{https://artofproblemsolving.com/community/user/29428}{pco}]
	\begin{tcolorbox}Find numerical sequence $(u_{n})$ ,$n\in N$ s.t : [list]1-$(\forall n\in N): u_{n}\geq{1}$ [\/list] , [list]2-$(\forall n\in N): (n+1)u_{n}\leq{n.u_{n+1}}$[\/list] , [list]3-$(\forall n,m\in N):u_{nm}=u_{n}.u_{m}$[\/list]\end{tcolorbox}
Let $f(x)$ from $\mathbb N\to\mathbb R$ defined as $f(n)=\frac{u_n}n$

(2) implies that $f(x)$ is non decreasing.
(3) implies that $f(xy)=f(x)f(y)$

(1) implies $f(x)>0$ and then, setting $y=1$ in (3) implies $f(1)=1$

If $f(u)=1$ for some $u>1$, then (3) implies$ f(u^n)=1=f(1)$ and then (2) implies $f(x)=1$ $\forall x$ and $u_n=n$ $\forall n$ which indeed is a solution.

Let us from now look for solutions where $f(x)>1$ $\forall x>1$ 

Let $x,y> 1$ and $m,n\in\mathbb N$ such that $\frac mn\ge \frac{\log x}{\log y}$

We get $y^m\ge x^n$ and so $f(y^m)\ge f(x^n)$ and so $f(y)^m\ge f(x)^n$ and so $m\log f(y)\ge n\log f(x)$ and so, since $f(y)>1$ : $\frac mn\ge \frac{\log f(x)}{\log f(y)}$

So we got $\frac mn\ge \frac{\log x}{\log y}$ $\implies$ $\frac mn\ge \frac{\log f(x)}{\log f(y)}$ $\forall$ integers $m,n>1$

So $\frac{\log f(x)}{\log f(y)}\le \frac{\log x}{\log y}$ $\forall x,y>1$

So $\frac{\log f(x)}{\log x}\le \frac{\log f(y)}{\log y}$ $\forall x,y>1$

So $\frac{\log f(x)}{\log x}$ is constant and $f(x)=x^t$ $\forall x>1$ for some $t>0$ (this condition in order to have $f(x)$ non decreasing and $f(x)\ne 1$ $\forall x>1$)

So $u_n=n^{t+1}$ which indeed is a solution.

Hence the answer $\boxed{u_n=n^a}$ $\forall n\in\mathbb N$ and for any real $a\ge 1$ (we added $a=1$ in order to get back the first solution $u_n=n$)
\end{solution}
*******************************************************************************
-------------------------------------------------------------------------------

\begin{problem}[Posted by \href{https://artofproblemsolving.com/community/user/82904}{gold46}]
	Let $f:\mathbb{R} \rightarrow \mathbb{R}$ be function such that for any $x,y$ reals $f\left (\frac{x+y}{2}\right )=\frac{f(x)+f(y)}{2}$. Then prove that for all positive integers  $n , x_1 , \ldots , x_n \in \mathbb{R}$ we have
$f\left (\frac{x_1+\cdots+x_n}{n}\right )=\frac{f(x_1)+\cdots+f(x_n)}{n}$.
	\flushright \href{https://artofproblemsolving.com/community/c6h480877}{(Link to AoPS)}
\end{problem}



\begin{solution}[by \href{https://artofproblemsolving.com/community/user/64716}{mavropnevma}]
	Prove by induction on $k\geq 1$ that $f\left (\frac{x_1+\cdots+x_n}{n}\right )=\frac{f(x_1)+\cdots+f(x_n)}{n}$ for $n=2^k$ (the base case for $k=1$ is given). Now, for some $n$, take $k$ such that $n<2^k$, and take $x_{n+1} = \cdots = x_{2^k} = \frac {x_1+\cdots + x_n} {n}$. Plug in the known identity for $2^k$, and watch how it turns into the identity for $n$. This type of double-barreled induction goes back to Cauchy.
\end{solution}



\begin{solution}[by \href{https://artofproblemsolving.com/community/user/29428}{pco}]
	Going to Cauchy is more direct : 

Let $P(x,y)$ be the assertion $f(\frac{x+y}2)=\frac{f(x)+f(y)}2$
$f(x)$ solution implies $f(x)+c$ solution and so WLOG say $f(0)=0$

$P(2x,0)$ $\implies$ $f(2x)=2f(x)$
$P(2x,2y)$ $\implies$ $f(x+y)=f(x)+f(y)$

Hence the result
\end{solution}



\begin{solution}[by \href{https://artofproblemsolving.com/community/user/62502}{PolyaPal}]
	Let me provide some details.  I will use both forward and backward induction a la Cauchy.

Let $P_{n}$ be the claim that  $ f\left (\frac{x_{1}+\cdots+x_{n}}{n}\right )=\frac{f(x_{1})+\cdots+f(x_{n})}{n} .$  It is clear that $P_{1}$ and $P_{2}$ are true.  Now we will show that $P_{n}\Rightarrow P_{n-1}.$

Let $y=\frac{x_{1}+x_{2}+\cdots+x_{n-1}}{n-1}.$  Assuming $P_{n}$ is true, we can see that

$ f\left (\frac{x_{1}+\cdots+x_{n-1}+y}{n}\right )=\frac{1}{n}f(x_{1})+\cdots+\frac{1}{n}f(x_{n-1})+\frac{1}{n}f(y). $

But the left-hand side is $f(y)$, implying that $n\cdot f(y)= f(x_{1})+\cdots+f(x_{n-1})+f(y)$ and

$f(y)=\frac{1}{n-1}\cdot\left(f(x_{1})+\cdots+f(x_{n-1})\right).$

Now we establish that $P_{n}\Rightarrow P_{2n}.$

Let  $A=f \left(\frac {x_{1}+\cdots+x_{n}+x_{n+1}+\cdots+x_{2n}}{2n}\right)=f \left(\frac{u+v}{2}\right)$,
where $u=\frac{x_{1}+\cdots+x_{n}}{n}$ and $v=\frac{x_{n+1}+\cdots+x_{2n}}{n}.$

Since $f\left(\frac{u+v}{2}\right)= \frac{1}{2}\left(f(u)+f(v)\right)$, we have

$A= \frac{1}{2} \left(f(u)+f(v)\right)=\frac{1}{2}\cdot\left( f\left(\frac{x_{1}+\cdots+x_{n}}{n}\right)+f\left(\frac{x_{n+1}+\cdots+x_{2n}}{n}\right)\right)=\frac{1}{2n}\cdot\left(f(x_{1})+\cdots+f(x_{n})+f(x_{n+1})+\cdots+f(x_{2n})\right),$

where we have twice used the assumption that $P_{n}$ is true.

Thus the induction is complete and the desired result (\begin{italicized}Jensen's functional equation \end{italicized}) is proved.
\end{solution}



\begin{solution}[by \href{https://artofproblemsolving.com/community/user/140796}{mathbuzz}]
	Well, WLOG assume that $f$ is not constant (if so, the result is trivial). The functional equation given is [color=#FF0000]Jensen's functional equation and its solutions are well known to be linear functions, namely $f(x)=cx$ [\/color]. Then the rest is trivial :D
[hide]here we just look for tame solutions .untame solutions are  given in PSS[\/hide]
\end{solution}



\begin{solution}[by \href{https://artofproblemsolving.com/community/user/64716}{mavropnevma}]
	\begin{bolded}Only \end{bolded}if it is known to be continuous, or any of the weaker conditions that make the linear functions be the only solution. Any additive function (solution to the Cauchy equation $f(x+y) = f(x) + f(y)$) trivially fulfills, and there are some "untame" such functions (built with Hamel bases) that are not linear.
\end{solution}



\begin{solution}[by \href{https://artofproblemsolving.com/community/user/104682}{momo1729}]
	See chapter 3 here : http://www.eleves.ens.fr\/home\/kortchem\/olympiades\/Cours\/Inegalites\/tin2006.pdf for a concise proof and equivalent formulations.
\end{solution}
*******************************************************************************
-------------------------------------------------------------------------------

\begin{problem}[Posted by \href{https://artofproblemsolving.com/community/user/120756}{qua96}]
	Problem: Find all natural number n satisfy exist $f : R^{+} --> R^{+}$ such that:
$f(x+y) \geq y.f_{n}(x)$ for all $x,y > 0$, here $ f_{n}(x) = f(f_{n-1}(x))$ with $n \geq 1; f_{0}(x) = x.$

p\/s:The problem was created by Dr.Pham Van Quoc  :D
	\flushright \href{https://artofproblemsolving.com/community/c6h481138}{(Link to AoPS)}
\end{problem}



\begin{solution}[by \href{https://artofproblemsolving.com/community/user/29428}{pco}]
	\begin{tcolorbox}Problem: Find all natural number n satisfy exist $f : R^{+} --> R^{+}$ such that:
$f(x+y) \geq y.f_{n}(x)$ for all $x,y > 0$, here $ f_{n}(x) = f(f_{n-1}(x))$ with $n \geq 1; f_{0}(x) = x.$\end{tcolorbox}
Let $P(x,y)$ be the assertion $f(x+y)\ge yf_n(x)$

If $n=1$, such a function exists : choose for example $f(x)=e^x$

If $n>1$ :
$P(1,x)$ $\implies$ $f(x+1)\ge f_n(1)x$ and so $\lim_{x\to+\infty}f(x)=+\infty$  and so $\exists x_0>0$ such that $a=f_n(x_0)>1$

So $P(x_0,x-x_0)$ $\implies$ $f(x)\ge ax-ax_0$ $\forall x>x_0$ and so, since $a>1$,  $\exists x_1$ such that $f(x)>x+1$ $\forall x>x_1$
So $f_{n-1}(x)>x+1$ $\forall x>x_1$ (here we need $n>1$)

But If $f_{n-1}(x)\ge x$ for some $x$, then $P(x,f_{n-1}(x)-x)$ $\implies$ $1\ge f_{n-1}(x)-x$ and so $f_{n-1}(x)\le x+1$ $\forall x>0$
So the contradiction and so no such function when $n>1$

Hence the result : $\boxed{n=1}$
\end{solution}
*******************************************************************************
-------------------------------------------------------------------------------

\begin{problem}[Posted by \href{https://artofproblemsolving.com/community/user/148441}{ts0_9}]
	Determine all pairs of positive real numbers $(a, b)$ for which there exists a function  $ f:\mathbb{R^{+}}\rightarrow\mathbb{R^{+}} $ satisfying for all positive real numbers $x$ the equation
$ f(f(x))=af(x)- bx $
	\flushright \href{https://artofproblemsolving.com/community/c6h481143}{(Link to AoPS)}
\end{problem}



\begin{solution}[by \href{https://artofproblemsolving.com/community/user/29428}{pco}]
	\begin{tcolorbox}Determine all pairs of positive real numbers $(a, b)$ for which there exists a function  $ f:\mathbb{R^{+}}\rightarrow\mathbb{R^{+}} $ satisfying for all positive real numbers $x$ the equation
$ f(f(x))=af(x)- bx $\end{tcolorbox}
If quadratic $x^2-ax+b$ has at least a real root $u$, then $u>0$ and $f(x)=ux$ is such a function.

If quadratic $x^2-ax+b$ has two non real roots $re^{it}$ and $re^{-it}$, with $t\ne 0,\pi$ then it's easy to show that $f^{[n]}(x)=\frac {r^{n-1}}{\sin t}(f(x)\sin nt-rx\sin(n-1)t)$
And obviously we can find $n$ such that $\sin nt<0<\sin(n-1)t$ if $\sin t >0$ or   $\sin nt>0>\sin(n-1)t$ if $\sin t <0$and such $n$ imply contradiction ($f^{[n]}(x)<0$)

Hence the answer : $\boxed{a^2-4b\ge 0}$
\end{solution}



\begin{solution}[by \href{https://artofproblemsolving.com/community/user/184873}{amatysten}]
	Just another solution.

1) Let's [color=#0000FF]assume[\/color] that $a^2-4b<0\Rightarrow b$ must be $>0$ and if $a\le 0$ we get a contradiction from $ff(x)=af(x)-bx$.

2.1) If $a>0$ then consider set $M^+=\{k\ge 0|\forall x f(x)\ge kx \}\ne \emptyset$.
If $M^+$ is unbounded from above then there're no such functions.

2.2) For any $k\in M^+ af(x)-bx=ff(x)\ge kf(x)$. If $k\ge a \Rightarrow -bx\ge (k-a)f(x) \Rightarrow b\le 0$. It contradicts our [color=#0000FF]assumption[\/color].
Thus, $f(x)>\frac b {a-k}x$.  $k'=\frac b {a-k}>k$, since in our [color=#0000FF]assumption[\/color] $k^2-ak+b>0$.
So, for any $k\in M^+ \exists k'\in M^+, k'>k \Rightarrow M^+$ is infinite.

2.3) Let's consider an increasing sequence from $M^+ :k_1<k_2<\dots$ As an increasing bounded sequence it has a limit $k$.
For any fixed $x$ $\forall i f(x)\ge k_i x \Rightarrow f(x) \ge kx \Rightarrow k\in M^+$. It means $M^+$ must include it's supremum.
But from 2.2 we saw it can't happen, since with any $k$   $M^+$ includes $k'>k$. It means no such function exists.

3) If $a^2-4b\ge 0$ an example can be found in the form $f(x)=kx$, where $k$ is a root of $k^2-ak+b=0$.
All pairs $(a,b)$ such that $a^2-4b\ge 0$ is the answer.
\end{solution}
*******************************************************************************
-------------------------------------------------------------------------------

\begin{problem}[Posted by \href{https://artofproblemsolving.com/community/user/82904}{gold46}]
	Find all $f:\mathbb{R}^+ \rightarrow \mathbb{R}$ function such that for any $a,b\in \mathbb{R} , a , b \not=0 : f(\frac{a^2+ab+b^2}{3})=f(a^2)+f(b^2)$.
	\flushright \href{https://artofproblemsolving.com/community/c6h481225}{(Link to AoPS)}
\end{problem}



\begin{solution}[by \href{https://artofproblemsolving.com/community/user/29428}{pco}]
	\begin{tcolorbox}Find all $f:\mathbb{R}^+ \rightarrow \mathbb{R}$ function such that for any $a,b\in \mathbb{R}: f(\frac{a^2+ab+b^2}{3})=f(a^2)+f(b^2)$.\end{tcolorbox}
In this forum $0\notin\mathbb R^+$ and so functional equation can not be true $\forall a,b\in\mathbb R$ since for $a=b=0$ neither LHS, neither RHS is defined.

So, either functional equation is true only for $a,b\ne 0$, either $f(x)$ is supposed from $\mathbb R_{\ge 0}\to\mathbb R$

Thanks for any precision you could give us.
\end{solution}



\begin{solution}[by \href{https://artofproblemsolving.com/community/user/82904}{gold46}]
	okay it's edited. . thank u ;D
\end{solution}



\begin{solution}[by \href{https://artofproblemsolving.com/community/user/29428}{pco}]
	\begin{tcolorbox}Find all $f:\mathbb{R}^+ \rightarrow \mathbb{R}$ function such that for any $a,b\in \mathbb{R} , a , b \not=0 : f(\frac{a^2+ab+b^2}{3})=f(a^2)+f(b^2)$.\end{tcolorbox}
Let $P(x,y)$ be the assertion $f(\frac{x^2+xy+y^2}3)=f(x^2)+f(y^2)$ where $x,y\ne 0$

Consider the system where $a,b$ are two distinct positive real numbers :
$x^2+xy+y^2=3a$
$x^2-xy+y^2=3b$

$\iff$ 
$(x+y)^2=\frac 32(3a-b)$
$(x-y)^2=\frac 32(3b-a)$

So this system has solution in non zero integers $x,y$ as soon as $0<\frac a3\le b\le 3a$ and $a\ne b$

Comparing then $P(x,y)$ and $P(x,-y)$ with the solutions of this system, we get $f(a)=f(b)$

So $f(x)$ is constant over any interval $[u,9u]$ and so is constant over $\mathbb R^+$

hence the unique solution $\boxed{f(x)=0}$ $\forall x>0$
\end{solution}
*******************************************************************************
-------------------------------------------------------------------------------

\begin{problem}[Posted by \href{https://artofproblemsolving.com/community/user/75382}{djb86}]
	Find all function pairs $(f,g)$ where each $f$ and $g$ is a function defined on the integers and with values, such that, for all integers $a$ and $b$, 
\[f(a+b)=f(a)g(b)+g(a)f(b)\\
g(a+b)=g(a)g(b)-f(a)f(b).\]
	\flushright \href{https://artofproblemsolving.com/community/c6h481341}{(Link to AoPS)}
\end{problem}



\begin{solution}[by \href{https://artofproblemsolving.com/community/user/29428}{pco}]
	\begin{tcolorbox}Find all function pairs $(f,g)$ where each $f$ and $g$ is a function defined on the integers and with values, such that, for all integers $a$ and $b$, 
\[f(a+b)=f(a)g(b)+g(a)f(b)\\
g(a+b)=g(a)g(b)-f(a)f(b).\]\end{tcolorbox}
If $f(1)=g(1)=0$ :
Equations imply, setting $b=1$ : $f(a+1)=g(a+1)=0$ and so $f(x)=g(x)=0$ $\forall x$ which indeed is a solution.

If $f(1)=0$ and $g(1)\ne 0$ :
Equations imply, setting $b=1$ : 
$f(a+1)=f(a)g(1)$ 
$g(a+1)=g(a)g(1)$ 
And so $f(x)=0$ $\forall x$ and $g(x)=a^x$ for some $a\ne 0$, which indeed is a solution.

If $f(1)\ne 0$ and $g(1)=0$ :
Equations imply, setting $b=1$ : $f(a+1)=g(a)f(1)$ and $g(a+1)=-f(a)f(1)$
And so $f(2n)=0$ and $f(2n+1)=(-1)^na^{2n+1}$ and $g(2n)=(-1)^na^{2n}$ and $g(2n+1)=0$ for some $a\ne 0$, which indeed is a solution.

If $f(1)\ne 0$ and $g(1)\ne 0$
Setting $b=1$ in first equation, we get $g(a)=\frac{f(a+1)-f(a)g(1)}{f(1)}$
Plugging this in second equation, we get $f(a+2)=2g(1)f(a+1)-f(a)(f(1)^2+g(1)^2)$

This is a classical linear recurrence with order two and characteristic equation has negative discriminant.
So $f(n)=ur^n\cos(an+b)$ for some constant $u\ne 0$ and $r>0$ and $a\ne k\pi$ and $b$ and we easily get from there 
$g(n)=vr^n\cos(an+c)$ with $v\ne 0$ too.

Plugging this in original equations, we get :
$\cos(a(x+y)+b)=v\cos(ax+b)\cos(ay+c)+v\cos(ax+c)\cos(ay+b)$
$\cos(a(x+y)+c)=v\cos(ax+c)\cos(ay+c)-\frac{u^2}v\cos(ax+b)\cos(ay+b)$

Setting $x=y=0$ gives $\cos b=0$ and so Wlog say $b=\frac{\pi}2$ (changing $u\to -u$ is necessary) and equations become :
$\sin(a(x+y))=v\sin(ax)\cos(ay+c)+v\cos(ax+c)\sin(ay)$
$\cos(a(x+y)+c)=v\cos(ax+c)\cos(ay+c)-\frac{u^2}v\sin(ax)\sin(ay)$

Setting $x=y=0$ gives $\cos c=0$ or $\cos c=\frac 1v$
If $\cos c=0$, Wlog say $c=\frac{\pi}2$ (changing $v\to -v$ is necessary) and equations become :
$\sin(a(x+y))=-2v\sin(ax)\sin(ay)$
$\sin(a(x+y))=(\frac{u^2}v-v)\sin(ax)\sin(ay)$
And no solution (remember $a\ne k\pi$ since roots of quadratic are not real numbers)

So $\cos c=\frac 1v$ and equations become :
$\sin(a(x+y))\cos c=\sin(ax)\cos(ay+c)+\cos(ax+c)\sin(ay)$
$\cos(a(x+y)+c)\cos c=\cos(ax+c)\cos(ay+c)-u^2\cos^2 c\sin(ax)\sin(ay)$
Setting $y=-x$ in first equation, we get $c=k\pi$ and $u^2=1$
And so $f(n)=\pm r^n\sin(an)$ and $g(n)=r^n\cos(an)$ which indeed are solutions.

\begin{bolded}Hence the answer\end{underlined}\end{bolded} :
$f(x)=g(x)=0$ $\forall x$
$f(x)=0$ $\forall x$ and $g(x)=r^x$ for some $r\ne 0$
$f(2n)=0$ and $f(2n+1)=(-1)^nr^{2n+1}$ and $g(2n)=(-1)^nr^{2n}$ and $g(2n+1)=0$ for some $r\ne 0$
$f(x)=\pm r^x\sin(ax)$ and $g(x)=r^x\cos(ax)$ for some $a\ne k\pi$ and $r>0$
\end{solution}
*******************************************************************************
-------------------------------------------------------------------------------

\begin{problem}[Posted by \href{https://artofproblemsolving.com/community/user/75382}{djb86}]
	The function $f$ is increasing and convex (i.e. every straight line between two points on the graph of $f$ lies above the graph) and satisfies $f(f(x))=3^x$ for all $x\in\mathbb{R}$. If $f(0)=0.5$ determine $f(0.75)$ with an error of at most $0.025$. The following are corrent to the number of digits given:

\[3^{0.25}=1.31607,\quad 3^{0.50}=1.73205,\quad 3^{0.75}=2.27951.\]
	\flushright \href{https://artofproblemsolving.com/community/c6h481391}{(Link to AoPS)}
\end{problem}



\begin{solution}[by \href{https://artofproblemsolving.com/community/user/29428}{pco}]
	Let $a=f(\frac 34)$

$f(0)=\frac 12$

$f(\frac 12)=f(f(0))=3^0=1$

$f(1)=f(f(\frac 12))=\sqrt 3$

$f(\sqrt 3)=f(f(1))=3$

Line between points $(\frac 12,1)$ and $(1,\sqrt 3)$ of the graph has equation $y=2(\sqrt 3-1)x+2-\sqrt 3$
Since $\frac 34\in(\frac 12,1)$ and $f(x)$ is convex, we have $f(\frac 34)=a<2(\sqrt 3-1)\frac 34+2-\sqrt 3$ $=\frac{1+\sqrt 3}2$

Line between points $(1,\sqrt 3)$ and $(\sqrt 3,1)$ of the graph has equation $y=\sqrt 3x$
Since $f(x)$ is increasing, $f(\frac 34)\in(f(\frac 12),f(1))=(1,\sqrt 3)$
Since $f(\frac 34)\in(1,\sqrt 3)$ and $f(x)$ is convex, we have $f(f(\frac 34))<\sqrt 3f(\frac 34)$ and so $a>\frac{3^{\frac 34}}{\sqrt 3}=3^{\frac 14}$

So $\sqrt[4]3<a<\frac{1+\sqrt 3}2$

Using numeric values given in the problem statement, we get $\boxed{1.31607<a<1.366025}$ which is a suitable result since interval has width $<2\times 0.025$
\end{solution}
*******************************************************************************
-------------------------------------------------------------------------------

\begin{problem}[Posted by \href{https://artofproblemsolving.com/community/user/120756}{qua96}]
	Solve the functional equation:
$f: R \to R$
$f(x^2) = f^2(x)$ 
and $f(x + 1) = f(x) + 1$
	\flushright \href{https://artofproblemsolving.com/community/c6h481766}{(Link to AoPS)}
\end{problem}



\begin{solution}[by \href{https://artofproblemsolving.com/community/user/29428}{pco}]
	\begin{tcolorbox}Problem: Solve function equation:
$f: R \to R$
$f(x^2) = f^2(x)$ 
and $f(x + 1) = f(x) + 1$\end{tcolorbox}
If $f^2(x)=(f(x))^2$ or $f(f(x))$ ?
\end{solution}



\begin{solution}[by \href{https://artofproblemsolving.com/community/user/120756}{qua96}]
	@ pco: It is  $f^2(x)=(f(x))^2$
\end{solution}



\begin{solution}[by \href{https://artofproblemsolving.com/community/user/64716}{mavropnevma}]
	So, best, just write it $f(x)^2$, with no doubt whatsoever about its meaning :)
\end{solution}



\begin{solution}[by \href{https://artofproblemsolving.com/community/user/86443}{roza2010}]
	Solution is $f(x)=x$ (obvious)
\end{solution}



\begin{solution}[by \href{https://artofproblemsolving.com/community/user/29428}{pco}]
	\begin{tcolorbox}Solve the functional equation:
$f: R \to R$
$f(x^2) = f^2(x)$ 
and $f(x + 1) = f(x) + 1$\end{tcolorbox}
Are you sure there is no missing information (like continuity, for example) ?
It's quite easy and classical to get $f(x)=x$ $\forall x\in\mathbb Q$ but I've some doubts about the possibility to get further :?:
Proving for example that $f(\pi)=\pi$ seems quite difficult.

And roza2010, maybe you can give once a proof of your assertions (although they are statistically half time wrong)
\end{solution}



\begin{solution}[by \href{https://artofproblemsolving.com/community/user/86443}{roza2010}]
	I like impulsive answer: proof is a tecnical problem... Guess is >75% of work ...
\end{solution}



\begin{solution}[by \href{https://artofproblemsolving.com/community/user/64716}{mavropnevma}]
	Nothing impulsive in guessing $f(x) = x$ verifies the equation. Some guesses are obvious - and so not worth mentioning; the only issue in such cases is proving there is no other possibility.
\end{solution}



\begin{solution}[by \href{https://artofproblemsolving.com/community/user/29428}{pco}]
	\begin{tcolorbox}Solve the functional equation:
$f: R \to R$
$f(x^2) = f^2(x)$ 
and $f(x + 1) = f(x) + 1$\end{tcolorbox}
First equation implies $f(x)\ge 0$ $\forall x\ge 0$
If $f(x)-x\le -1$ for some $x$, then $f(x)\le x-1$ and so $f(x-\lfloor x\rfloor)\le x-\lfloor x\rfloor-1<0$, impossible (see previous line) and so $f(x)> x-1$ $\forall x$

$f(x+n)=x+(f(x)-x)+n$ $\forall x$, $\forall n\in\mathbb Z$
$f((x+n)^2)=(x+n)^2+(f(x)-x)^2+2(x+n)(f(x)-x)$
And $f((x+n)^2)>(x+n)^2-1$ implies $(f(x)-x)^2+2(x+n)(f(x)-x)+1>0$ $\forall x$, $\forall n\in\mathbb Z$

And so $f(x)-x=0$ else, choosing $n\to\infty$ with sign opposite to sign of $f(x)-x$, we get a contradicton.

Hence the unique solution $\boxed{f(x)=x}$ $\forall x$, which indeed is a solution.
\end{solution}
*******************************************************************************
-------------------------------------------------------------------------------

\begin{problem}[Posted by \href{https://artofproblemsolving.com/community/user/61129}{gus135791}]
	f:R->R

Find all functinal equation f(x) satifying the followings:
(1)  f(f(x))=x
(2) f(x+y)=f(x)+f(y)

please help
	\flushright \href{https://artofproblemsolving.com/community/c6h481793}{(Link to AoPS)}
\end{problem}



\begin{solution}[by \href{https://artofproblemsolving.com/community/user/29428}{pco}]
	\begin{tcolorbox}f:R->R

Find all functinal equation f(x) satifying the followings:
(1)  f(f(x))=x
(2) f(x+y)=f(x)+f(y)\end{tcolorbox}
Very classical. General solution is :

Let $A,B$ be two supplementary vector subspaces of the $\mathbb Q$-vectorspace $\mathbb R$
Let $a(x)$ from $\mathbb R\to A$ be the projection of $x$ in $A$ along $B$ and $b(x)$ from $\mathbb R\to B$ be the projection of $x$ in $B$ along $A$ so that $x=a(x)+b(x)$ in a unique manner

Then $f(x)=a(x)-b(x)$
\end{solution}



\begin{solution}[by \href{https://artofproblemsolving.com/community/user/76247}{yugrey}]
	Also, I would like to point out that this is equivalent to 

$f(f(x)+f(y))=x+y$.

If $f(f(x))=x$ and $f(x+y)=f(x)+f(y)$, $f(f(x)+f(y))=f(f(x+y))=x+y$.

Also, if $f(f(x)+f(y))=x+y$, then let $y=0$ so $f(f(y)+f(0))=y$.

Note if $f(x)=f(z)$, $f(f(x)+f(y))=f(f(z)+f(y))=x+y=z+y$ and $x=z$, so this injective.  It is obviously surjective as well (as $f(f(y)+f(0))=y$, so it is bijective.

Then l take $f(y)=0$, so $y$ is a constant, and $f(f(x))=x+c$.

Also, $f(f(f(x)))=f(x+c)=f(x)+c$.

Take $x=0$, so $f(c)=f(0)+c$, and $f(c)=0$.  Thus, $f(0)+c=0$ and $f(0)=f(f(c))=c+c=2c$.

Thus, $c=0$, $f(f(x))=x$.

Also, $f(f(x+y))=x+y$, so $f(f(x+y))=f(f(x)+f(y))$ as they are both $x+y$.

Then $f(x+y)=f(x)+f(y)$ by injectivity.
\end{solution}



\begin{solution}[by \href{https://artofproblemsolving.com/community/user/67669}{minhphuc.v}]
	$f(0)=0$ and $f(m)=mf(1)\; ;\;f(\frac{1}{n})=\frac{1}{n}f(1)\;;f(\frac{m}{n})=\frac{m}{n}f(1)$
So $f(x)=xf(1)$  and $x=f(f(x))=f(xf(1))=f(1)f(x)=x(f(1)^2) \Rightarrow f(1)=1\; or\; f(1)=-1$
So $f(x) =x \;or f(x) =-x $
\end{solution}



\begin{solution}[by \href{https://artofproblemsolving.com/community/user/29428}{pco}]
	\begin{tcolorbox}So $f(x)=xf(1)$ \end{tcolorbox}
Wrong. You only proved $f(x)=xf(1)$ $\forall x\in\mathbb Q$ and not for all real numbers ... .
\end{solution}



\begin{solution}[by \href{https://artofproblemsolving.com/community/user/67669}{minhphuc.v}]
	yes!  I think that it is a continuous function
\end{solution}



\begin{solution}[by \href{https://artofproblemsolving.com/community/user/143628}{MANMAID}]
	My solution is similar to \begin{bolded}minhphuc.v\end{bolded}, and I think it is right.
$f(x+y)=f(x)+f(y)$ then ,$f(x)=cx$ , $c$ is a constant. Now $f(f(x))=x$ ,then $f(f(x))=cf(x)=x$ ,or, $c^2x=x$ ,or, $x(c+1)(c-1)=0$ ,So $f(x+0)=f(x)+f(0)$ ,or , $f(0)=0$ ,then ,$c=\pm{1}$, when $x\neq{0}$,.
Hence $f(x)=\pm{x}$ 
                                                                                                                                                                                        (Though it is a bad solution)
\end{solution}



\begin{solution}[by \href{https://artofproblemsolving.com/community/user/29428}{pco}]
	\begin{tcolorbox}My solution is similar to \begin{bolded}minhphuc.v\end{bolded}, and I think it is right.
$f(x+y)=f(x)+f(y)$ then ,$f(x)=cx$ , \end{tcolorbox}
And, as minhphuc.v's solution, it's also wrong.

$f(x+y)=f(x)+f(y)$ does not imply $f(x)=cx$ without supplementary constraints (as continuous, or monotonous, ...)
\end{solution}



\begin{solution}[by \href{https://artofproblemsolving.com/community/user/143628}{MANMAID}]
	\begin{tcolorbox}[quote="MANMAID"]My solution is similar to \begin{bolded}minhphuc.v\end{bolded}, and I think it is right.
$f(x+y)=f(x)+f(y)$ then ,$f(x)=cx$ , \end{tcolorbox}
And, as minhphuc.v's solution, it's also wrong.

$f(x+y)=f(x)+f(y)$ does not imply $f(x)=cx$ without supplementary constraints (as continuous, or monotonous, ...)\end{tcolorbox}

$f(x+y)=f(x)+f(y)$ is well known function whose solution is  $f(x)=cx$, so I thought not to prove it. 
AND  for the line "My solution is similar to \begin{bolded}minhphuc.v\end{bolded}, and I think it is right" sorry for this line ,I just lazy, so wrote it.
\end{solution}



\begin{solution}[by \href{https://artofproblemsolving.com/community/user/29428}{pco}]
	\begin{tcolorbox}
$f(x+y)=f(x)+f(y)$ is well known function whose solution is  $f(x)=cx$, so I thought not to prove it. 
AND  for the line "My solution is similar to \begin{bolded}minhphuc.v\end{bolded}, and I think it is right" sorry for this line ,I just lazy, so wrote it.\end{tcolorbox}
$f(x+y)=f(x)+f(y)$ is a well known functional equation (called CAUCHY functional equation) whose solutions set includes infinitely many other functions than $f(x)=cx$, as soon as Axiom of Choice is accepted (which is generally the situation).

In order to restrict solutions set to linear functions, you must add some constraints, for example (all following constraints are equivalent) :
$f(x)$ is continuous at least at one point
$f(x)$ is monotonous
$f(x)$ is lower bounded on some non empty open interval $(a,b)$
$f(x)$ is upper bounded on some non empty open interval $(a,b)$
\end{solution}



\begin{solution}[by \href{https://artofproblemsolving.com/community/user/64716}{mavropnevma}]
	This is a pointless battle with one side paying no attention to the valid, classical arguments the other is providing. Insisting that the only solution is $f(x) = cx$, especially as it's wrong, brings no illumination whatsoever, so I take the opportunity to lock this topic.
\end{solution}
*******************************************************************************
-------------------------------------------------------------------------------

\begin{problem}[Posted by \href{https://artofproblemsolving.com/community/user/89198}{chaotic_iak}]
	Suppose a function $f : \mathbb{Z}^+ \rightarrow \mathbb{Z}^+$ satisfies $f(f(n)) + f(n+1) = n+2$ for all positive integer $n$. Prove that $f(f(n)+n) = n+1$ for all positive integer $n$.
	\flushright \href{https://artofproblemsolving.com/community/c6h481881}{(Link to AoPS)}
\end{problem}



\begin{solution}[by \href{https://artofproblemsolving.com/community/user/29428}{pco}]
	\begin{tcolorbox}Suppose a function $f : \mathbb{Z}^+ \rightarrow \mathbb{Z}^+$ satisfies $f(f(n)) + f(n+1) = n+2$ for all positive integer $n$. Prove that $f(f(n)+n) = n+1$ for all positive integer $n$.\end{tcolorbox}
1) $f(2)=2$ and $f(3)=2$ and $f(4)=3$
===========================
$f(2)=3-f(f(1))$ and so $f(2)\in\{1,2\}$

If $f(2)=1$, then $f(3)=4-f(f(2))=4-f(1)$ implies $f(1)\in\{1,2,3\}$ but then :
$f(1)=1$ $\implies$ $1=f(2)=3-f(f(1))=2$, impossible
$f(1)=2$ $\implies$ $1=f(2)=3-f(f(1))=3-f(2)=2$, impossible
$f(1)=3$ $\implies$ $1=f(2)=3-f(f(1))=3-f(3)$ and so $f(3)=2$, in contradiction with $f(3)=4-f(f(2))=4-f(1)=1$

So $f(2)=2$
$f(3)=4-f(f(2))=2$
$f(4)=5-f(f(3))=5-f(2)=3$
Q.E.D.

2) $f(1)=1$ and $\forall n\ge 2$ : $n\ge f(n)\ge 2$
===============================
$f(n+1)=n+2-f(f(n))\le n+1$ and so $f(n)\le n$ $\forall n\ge 2$

Let then $n\ge 2$ : if $f(n)\ge 2$, we get $f(f(n))\le f(n)\le n$ and so $f(n+1)=n+2-f(f(n))\ge 2$
And since $f(2)\ge 2$, we get $f(n)\ge 2$ $\forall n\ge 2$
But $2=f(2)=3-f(f(1))$ implies $f(f(1))=1$ and so $f(1)<2$ and so $f(1)=1$
Q.E.D.

3) $\forall n\ge 2$ $f(n)\in\{f(n-1),f(n-1)+1\}$
==============================
Suppose $f(k)\in\{f(k-1),f(k-1)+1\}$ $\forall k\in[2,n]$ with $n>ge 2$
If $f(n)=f(n-1)$, then $f(n+1)=n+2-f(f(n))=1+(n+1-f(f(n-1)))=1+f(n)$
If $f(n)=f(n-1)+1$ : since $f(n-1)+1\in[2,n]$, we get $f(f(n-1)+1)\in\{f(f(n-1)),f(f(n-1))+1\}$ and so :
If $f(f(n-1)+1)=f(f(n-1))$ : $f(n+1)=n+2-f(f(n))$ $=n+2-f(f(n-1)+1)$ $=n+2-f(f(n-1))$ $=1+f(n)$
If $f(f(n-1)+1)=f(f(n-1))+1$ : $f(n+1)=n+2-f(f(n))$ $=n+2-f(f(n-1)+1)$ $=n+1-f(f(n-1))$ $=f(n)$
And so $f(n+1)\in\{f(n),f(n)+1\}$

And since $f(k)\in\{f(k-1),f(k-1)+1\}$ $\forall k\in[2,2]$, the induction gives the required result.
Q.E.D.

4) $f(f(n)+n)=n+1$ $\forall n$
====================
Simple induction :
Property is true for $n=1$ : $f(f(1)+1)=f(2)=2$

If property is true for $n$, then :
$f(n+1)\in\{f(n),f(n)+1\}$

If $f(n+1)=f(n)$, then :
$f(f(n+1)+n+1)=f(n+1)+n+2-f(f(f(n+1)+n))$ $=f(n+1)+n+2-f(f(f(n)+n))$ $=f(n+1)+n+2-f(n+1)$ $=n+2$

If $f(n+1)=f(n)+1$, then :
$f(f(n)+n+1)=f(n)+n+2-f(f(f(n)+n))$ $=f(n)+n+2-f(n+1)=n+1$
$f(f(n+1)+n+1)=f(n+1)+n+2-f(f(f(n+1)+n))$ $=f(n+1)+n+2-f(f(f(n)+n+1))$ $=f(n+1)+n+2-f(n+1)=n+2$

And so $f(f(n+1)+n+1)=n+2$
End of induction

And so $\boxed{f(f(n)+n)=n+1}$ $\forall n\in\mathbb Z^+$

Q.E.D.

5)Alternative direct [magic] path
======================
It's possible to show that there is a unique function $f(x)$ matching all requirements and that $\boxed{f(x)=1+\left\lfloor\frac{2x}{1+\sqrt 5}\right\rfloor}$ matches all the requirements and so is this unique function.

It remains then casework to show $f(f(n)+n)=n+1$ from the direct expression of $f(x)$

(note \end{underlined}: solution based on benimath's solution for a similar problem here : http://www.artofproblemsolving.com/Forum/viewtopic.php?p=2133084#p2133084   )
\end{solution}
*******************************************************************************
-------------------------------------------------------------------------------

\begin{problem}[Posted by \href{https://artofproblemsolving.com/community/user/142879}{ionbursuc}]
	Be function$f:\mathbb{R}\to \mathbb{R}$ satisfying the properties:     
     1)  $f\left( {{\log }_{a}}b \right)\cdot f\left( {{\log }_{b}}c \right)={{\log }_{f\left( a \right)}}f\left( c \right)$ $,\left( \forall  \right)a,b,c>0,a\ne 1,b\ne 1$;
     2)  $f\left( x \right)>0$, $\left( \forall  \right)x>0$;
     3)  $f\left( x \right)\ne 1$ ,$\left( \forall  \right)x\ne 1$.
It requires:
    a) Prove that $f\left( xy \right)=f\left( x \right)\cdot f\left( y \right)$,  $\left( \forall  \right)x,y\in \mathbb{R}$;
    b) ) Prove that $f\left( x+y \right)=f\left( x \right)+f\left( y \right)$, $\left( \forall  \right)x,y\in \mathbb{R}$.
	\flushright \href{https://artofproblemsolving.com/community/c6h482250}{(Link to AoPS)}
\end{problem}



\begin{solution}[by \href{https://artofproblemsolving.com/community/user/29428}{pco}]
	\begin{tcolorbox}Be function$f:\mathbb{R}\to \mathbb{R}$ satisfying the properties:     
     1)  $f\left( {{\log }_{a}}b \right)\cdot f\left( {{\log }_{b}}c \right)={{\log }_{f\left( a \right)}}f\left( c \right)$ $,\left( \forall  \right)a,b,c>0,a\ne 1,b\ne 1$;
     2)  $f\left( x \right)>0$, $\left( \forall  \right)x>0$;
     3)  $f\left( x \right)\ne 1$ ,$\left( \forall  \right)x\ne 1$.
It requires:
    a) Prove that $f\left( xy \right)=f\left( x \right)\cdot f\left( y \right)$,  $\left( \forall  \right)x,y\in \mathbb{R}$;
    b) ) Prove that $f\left( x+y \right)=f\left( x \right)+f\left( y \right)$, $\left( \forall  \right)x,y\in \mathbb{R}$.\end{tcolorbox}
Let $P(x,y,z)$ be the assertion $f(\log_xy)f(\log_yz)=\log_{f(x)}f(z)$ where $x,y,z>0$, $x\ne 1$, $y\ne 1$

$P(2,2,2)$ $\implies$ $f(1)^2=1$ and so $f(1)=1$ (using 2)
$P(2,2,1)$ $\implies$ $f(0)=0$
$P(e^y,e^x,e^x)$ with $x,y\ne 0$ $\implies$ new assertion $Q(x,y)$ : $f(\frac xy)=\frac{\log f(e^x)}{\log f(e^y)}$ $\forall x,y\ne 0$

Comparing $Q(x,1)$ and $Q(1,x)$, we get $f(\frac 1x)=\frac 1{f(x)}$ $\forall x\ne 0$

$Q(xy,y)$ $\implies$ $f(x)=\frac{\log f(e^{xy})}{\log f(e^y)}$ $\forall x,y\ne 0$
$Q(y,1)$ $\implies$ $f(y)=\frac{\log f(e^y)}{\log f(e)}$ $\forall y\ne 0$
$Q(1,xy)$ $\implies$ $f(\frac 1{xy})=\frac{\log f(e)}{\log f(e^{xy})}$ $\forall x,y\ne 0$

Multiplying these three lines gives $f(x)f(y)f(\frac 1{xy})=1$ $\forall x,y\ne 0$

And so (using previous result) $f(xy)=f(x)f(y)$ $\forall x,y\ne 0$
And since we previously got $f(0)=0$, we have $\boxed{f(xy)=f(x)f(y)}$ $\forall x,y$

$Q(x,1)$ becomes then $f(x)=\frac{\log f(e^x)}{\log f(e)}$ $\forall x\ne 0$ and so 

And so $f(e^x)=e^{af(x)}$ $\forall x\ne 0$ and for some real $a\ne 0$ ($a=0$ would be in contradiction with 3)

So $f(e^{x+y})=e^{af(x+y)}$
But $f(e^{x+y})=f(e^x)f(e^y)=e^{a(f(x)+f(y))}$

And since $a\ne 0$, we get $f(x+y)=f(x)+f(y)$ $\forall x,y\ne 0$

And since $f(0)=0$, we get $\boxed{f(x+y)=f(x)+f(y)}$ $\forall x,y$
\end{solution}



\begin{solution}[by \href{https://artofproblemsolving.com/community/user/142879}{ionbursuc}]
	Be function$f:\mathbb{R}\to \mathbb{R}$ satisfying the properties:     
     1)  $f\left( {{\log }_{a}}b \right)\cdot f\left( {{\log }_{b}}c \right)={{\log }_{f\left( a \right)}}f\left( c \right)$ $,\left( \forall  \right)a,b,c>0,a\ne 1,b\ne 1$;
     2)  $f\left( x \right)>0$, $\left( \forall  \right)x>0$;
     3)  $f\left( x \right)\ne 1$ ,$\left( \forall  \right)x\ne 1$.
It requires:
    Prove that $f\left( x \right)=x$,  $\left( \forall  \right)x\in \mathbb{R}.$
\end{solution}



\begin{solution}[by \href{https://artofproblemsolving.com/community/user/29428}{pco}]
	We had thru previous post $f(x+y)=f(x)+f(y)$ and so $f(x)=f(1)x$ $\forall x\in\mathbb Q$ (Cauchy)
And since $f(x)>0$ $\forall x>0$, the equation $f(x+y)=f(x)+f(y)$ implies that $f(x)$ is increasing.
So $f(x)=f(1)x$ $\forall x$

And since $f(1)^2=f(1)$ and $f(1)>0$, we get $f(1)=1$ and $f(x)=x$ $\forall x$
\end{solution}
*******************************************************************************
-------------------------------------------------------------------------------

\begin{problem}[Posted by \href{https://artofproblemsolving.com/community/user/110786}{mastergeo}]
	Prove that if $c>1$, there exists a continuous function $f:\mathbb{R}^+\to\mathbb{R}^+$ such that: \[f\left(f(x)+\frac{1}{f(x)}\right)=x+c.\]
	\flushright \href{https://artofproblemsolving.com/community/c6h482296}{(Link to AoPS)}
\end{problem}



\begin{solution}[by \href{https://artofproblemsolving.com/community/user/29428}{pco}]
	\begin{tcolorbox}Prove that if $c>1$, there exists a continuous function $f:\mathbb{R}\to\mathbb{R}$ such that: \[f\left(f(x)+\frac{1}{f(x)}\right)=x+c.\]\end{tcolorbox}
If such a function exists, it's a bijection and so $\exists x_0$ such that $f(x_0)=0$ and functional equation can not be true (since not defined) for this $x_0$

So no such function.
\end{solution}



\begin{solution}[by \href{https://artofproblemsolving.com/community/user/110786}{mastergeo}]
	\begin{tcolorbox}[quote="mastergeo"]Prove that if $c>1$, there exists a continuous function $f:\mathbb{R}\to\mathbb{R}$ such that: \[f\left(f(x)+\frac{1}{f(x)}\right)=x+c.\]\end{tcolorbox}
If such a function exists, it's a bijection and so $\exists x_0$ such that $f(x_0)=0$ and functional equation can not be true (since not defined) for this $x_0$

So no such function.\end{tcolorbox}
I'm sorry it's my typo, I've already edited
\end{solution}



\begin{solution}[by \href{https://artofproblemsolving.com/community/user/29428}{pco}]
	\begin{tcolorbox}Prove that if $c>1$, there exists a continuous function $f:\mathbb{R}^+\to\mathbb{R}^+$ such that: \[f\left(f(x)+\frac{1}{f(x)}\right)=x+c.\]\end{tcolorbox}
Let $a\in(1,c)$.
Let the functional equation :
$h(x)$ is a continuous increasing bijection from $[a,+\infty)\to[c,+\infty)$ such that $h(h(x))=x+\frac 1x+c$ $\forall x\ge a$

It's easy to show that infinitely many such $h(x)$ exist
[hide="how ?"]
Let $u(x)=x+\frac 1x+c$
Just choose $h(x)$ as any increasing continous function on $[a,c]$ such that $h(a)=c$ and $h(c)=u(a)$
Then use $h(h(x))=u(x)$ in order to define $h(x)$ over $[h(a),h(c)]=[c,u(a)]$
Then use $h(h(x))=u(x)$ in order to define $h(x)$ over $[h(h(a)),h(h(c))]=[u(a),u(c)]$
...
[\/hide]

Then $g(x)=h(x)-c$ is a continuous increasing bijection from $[a,+\infty)\to[0,+\infty)$ such that $g(g(x)+c)=x+\frac 1x$ $\forall x\ge a$

Then $f(x)=g^{-1}(x)$ is a continuous increasing bijection from $[0,+\infty)\to[a,+\infty)$ such that $f^{-1}(f^{-1}(x)+c)=x+\frac 1x$ $\forall x\ge a$
$\iff$ $f^{-1}(x)+c=f(x+\frac 1x)$ $\forall x\ge a$
$\iff$ $x+c=f(f(x)+\frac 1{f(x)})$ $\forall x\ge 0$

Q.E.D.
\end{solution}



\begin{solution}[by \href{https://artofproblemsolving.com/community/user/110786}{mastergeo}]
	Thanks pco, that's a nice solution :D
\end{solution}
*******************************************************************************
-------------------------------------------------------------------------------

\begin{problem}[Posted by \href{https://artofproblemsolving.com/community/user/120756}{qua96}]
	Problem: Find all $f$satisfy $f:\mathbb{N.N}\rightarrow\mathbb{N}$ and
$i)f(a,b)=f(b,a)$

$ii)f(b,f(a,b))=a$

$iii)$ If $f(a,b)>c$ then $f(b,c)<a$
Here $\mathbb{N}$ is the set of all natural numbers ( In my country $0 \in \mathbb{N}$)
	\flushright \href{https://artofproblemsolving.com/community/c6h482395}{(Link to AoPS)}
\end{problem}



\begin{solution}[by \href{https://artofproblemsolving.com/community/user/29428}{pco}]
	\begin{tcolorbox}Problem: Find all $f$satisfy $f:\mathbb{N.N}\rightarrow\mathbb{N}$ and
$i)f(a,b)=f(b,a)$

$ii)f(b,f(a,b))=a$

$iii)$ If $f(a,b)>c$ then $f(b,c)<a$
Here $\mathbb{N}$ is the set of all natural numbers ( In my country $0 \in \mathbb{N}$)\end{tcolorbox}
Let $b\in\mathbb N_0$ fixed and $h(x)=f(x,b)=f(b,x)$

ii) becomes $h(h(x))=x$ $\forall x\in\mathbb N_0$ and so $h(x)$ is a bijection.
iii) becomes $h(x)>y$ $\implies$ $h(y)<x$ $\forall x,y\in\mathbb N_0$ which implies (setting $x\to h(x)$) :
$x>y$ $\implies$ $h(y)<h(x)$ $\forall x,y\in\mathbb N_0$ and so $h(x)$ is increasing

So $h(x)$ is an increasing bijection from $\mathbb N_0\to\mathbb N_0$ and so is $x$

So $f(a,b)=a$ $\forall a,b$, which is impossible since then $f(a,b)=f(b,a)=b$ $\forall a,b,$

So\begin{bolded} no solution\end{underlined}\end{bolded}.
\end{solution}
*******************************************************************************
-------------------------------------------------------------------------------

\begin{problem}[Posted by \href{https://artofproblemsolving.com/community/user/30710}{huyhoang}]
	Find all $f:\mathbb{N}\to \mathbb{N}$ such that $f(f(n))=n+2$.
	\flushright \href{https://artofproblemsolving.com/community/c6h483104}{(Link to AoPS)}
\end{problem}



\begin{solution}[by \href{https://artofproblemsolving.com/community/user/29428}{pco}]
	\begin{tcolorbox}Find all $f:\mathbb{N}\to \mathbb{N}$ such that $f(f(n))=n+2$.\end{tcolorbox}
So $f(n+2)=f(n)+2$ and so :
$f(2n+1)=2n+f(1)$
$f(2n+2)=2n+f(2)$

Pluging this in $f(f(n))=n+2$, we get that $f(1)$ is even and that $f(1)+f(2)=5$

Hence the two solutions :

1) $f(1)=2$
$f(2n+1)=2n+2$
$f(2n+2)=2n+3$
Which may be written $\boxed{f(x)=x+1}$

2) $f(1)=4$
$f(2n+1)=2n+4$
$f(2n+2)=2n+1$
Which may be written $\boxed{f(x)=x+1-2(-1)^n}$
\end{solution}



\begin{solution}[by \href{https://artofproblemsolving.com/community/user/340066}{aopsx}]
	\begin{tcolorbox}Pluging this in $f(f(n))=n+2$, we get that $f(1)$ is even and that $f(1)+f(2)=5$.\end{tcolorbox}
Sir Pco, I don't understand this step. How do you obtain $f(1)$ is even and $f(1)+f(2)=5$?
$\boxed{f(x)=x+1-2(-1)^n}$
\begin{tcolorbox}  $\boxed{f(x)=x+1-2(-1)^n}$  \end{tcolorbox}
And $f(0)$ doesn't exist since $f:\mathbb{N} \rightarrow \mathbb{N}$
\end{solution}



\begin{solution}[by \href{https://artofproblemsolving.com/community/user/29428}{pco}]
	\begin{tcolorbox}How do you obtain $f(1)$ is even and $f(1)+f(2)=5$?\end{tcolorbox}
As I said, just by plugging this in original equation :
If $f(1)$ is odd, then $f(2n+1)$ is odd and so $f(f(2n+1))=f(2n+1)+f(1)-1=2n+2f(1)-1$
And so $2n+3=2n+2f(1)-1$ and $f(1)=2$, which is not odd. So  contradiction
And so $f(1)$ is even.

So $f(2n+1)$ is even and so $f(f(2n+1))=f(2n+1)+f(2)-2=2n+f(1)+f(2)-2$
And so $2n+3=2n+f(1)+f(2)-2$
Which is $f(1)+f(2)=5$

\begin{tcolorbox}$\boxed{f(x)=x+1-2(-1)^n}$And $f(0)$ doesn't exist since $f:\mathbb{N} \rightarrow \mathbb{N}$\end{tcolorbox}
Is it a question ?
What is the link with my post ?
I never spoke about $f(0)$ ...



\end{solution}
*******************************************************************************
-------------------------------------------------------------------------------

\begin{problem}[Posted by \href{https://artofproblemsolving.com/community/user/30710}{huyhoang}]
	Let $T$ be an even number. Find $f:\mathbb{N} \to \mathbb{N}$ such that $f(f(n))=n+T$ and $f(n+1)>f(n)$
	\flushright \href{https://artofproblemsolving.com/community/c6h483106}{(Link to AoPS)}
\end{problem}



\begin{solution}[by \href{https://artofproblemsolving.com/community/user/29428}{pco}]
	\begin{tcolorbox}Let $T$ be an even number. Find $f:\mathbb{N} \to \mathbb{N}$ such that $f(f(n))=n+T$ and $f(n+1)>f(n)$\end{tcolorbox}
So $f(n+T)=f(n)+T$ and $f(x)>T$ $\forall x>T$

And since any integer $>T$ is in $f(\mathbb N)$, the condition $f(n+1)>f(n)$ implies $f(n+1)=f(n)+1$ $\forall n>T$ (else the number $f(n)+1\not\in f(\mathbb N)$)

So $f(n)=n+a$ $\forall n>T$ and $a=\frac T2$

So $f(n+T)=n+T+\frac T2$ since $n+T>T$ 
But $f(n+T)=f(n)+T$

So $\boxed{f(n)=n+\frac T2}$ $\forall n\in\mathbb N$
\end{solution}
*******************************************************************************
-------------------------------------------------------------------------------

\begin{problem}[Posted by \href{https://artofproblemsolving.com/community/user/141397}{subham1729}]
	Find all functions $f:\mathbb{Q}\to \mathbb{Q}$ such that $f(x+f(y))=f(x)f(y)$.
	\flushright \href{https://artofproblemsolving.com/community/c6h483238}{(Link to AoPS)}
\end{problem}



\begin{solution}[by \href{https://artofproblemsolving.com/community/user/29428}{pco}]
	\begin{tcolorbox}Find all functions $f:\mathbb{Q}\to \mathbb{Q}$ such that $f(x+f(y))=f(x)f(y)$.\end{tcolorbox}
Let $P(x,y)$ be the assertion $f(x+f(y))=f(x)f(y)$

If $f(a)=0$ for some $a$, then $P(x,a)$ $\implies$ $\boxed{f(x)=0}$ $\forall x$ which indeed is a solution.
So let us from now consider $f(x)\ne 0$ $\forall x$

$P(-f(0),0)$ $\implies$ $f(-f(0))=1$
$P(x,-f(0))$ $\implies$ $f(x+1)=f(x)$ and so $f(x+n)=f(x)$ $\forall x\in\mathbb Q,n\in\mathbb Z$

A simple induction from $P(x,y)$ gives $f(x+nf(y))=f(x)(f(y))^n$

Let then $f(y)=\frac pq$ with $q\in\mathbb N$

$f(x+qf(y))=f(x)f(y)^q$ becomes $f(x+p)=f(x)\left(\frac pq\right)^q$ and since $f(x+p)=f(x)$, we get $\left(\frac pq\right)^q=1$

And so $\frac pq\in\{-1,1\}$ and $f(x)\in\{-1,1\}$ $\forall x$
But if $\exists b$ such that $f(b)=-1$, then $P(x+1,b)$ $\implies$ $f(x)=-f(x+1)$, impossible since $f(x)=f(x+1)$ and $f(x)\ne 0$

And so $\boxed{f(x)=1}$ $\forall x$ which indeed is a solution.
\end{solution}
*******************************************************************************
-------------------------------------------------------------------------------

\begin{problem}[Posted by \href{https://artofproblemsolving.com/community/user/115131}{BakyX}]
	Find all real functions satisfying $f(xf(y))+y+f(x)=f(x+f(y))+yf(x)$ for all real $x,y$.
	\flushright \href{https://artofproblemsolving.com/community/c6h483271}{(Link to AoPS)}
\end{problem}



\begin{solution}[by \href{https://artofproblemsolving.com/community/user/29428}{pco}]
	\begin{tcolorbox}Find all real functions satisfying $f(xf(y))+y+f(x)=f(x+f(y))+yf(x)$ for all real $x,y$.\end{tcolorbox}
Let $P(x,y)$ be the assertion $f(xf(y))+y+f(x)=f(x+f(y))+yf(x)$

If $f(1)\ne 1$ : $P(\frac{f(1)}{f(1)-1},1)$ $\implies$ $f(\frac{f(1)^2}{f(1)-1})+1=f(\frac{f(1)^2}{f(1)-1})$, impossible

So $f(1)=1$ and $P(x,1)$ $\implies$ $f(x+1)=f(x)+1$
So $f(0)=0$ and $P(0,x)$ $\implies$ $f(f(x))=x$ and $f(x)$ is bijective

$P(x,y+1)$ $\implies$ $f(xf(y)+x)+y=f(x+f(y))+yf(x)$
And, subtracting $P(x,y)$, we get $f(xf(y)+x)=f(xf(y))+f(x)$

Since $f(x)$ is surjective, this implies $f(x+y)=f(x)+f(y)$ $\forall x\ne 0$ and so $f(x+y)=f(x)+f(y)$ $\forall x,y$

Then $P(x,f(x))$ becomes $f(x^2)=f(x)^2$ and so $f(x)\ge 0$ $\forall x\ge 0$ and so (since $f(x+y)=f(x)+f(y)$) $f(x)$ is non decreasing.

Monotonous solutions of Cauchy's equation are linear and so $f(x)=cx$ and, pluging in original equation, we get $c=1$

And so the unique solution $\boxed{f(x)=x}$ $\forall x$
\end{solution}



\begin{solution}[by \href{https://artofproblemsolving.com/community/user/115131}{BakyX}]
	Why is $f(x)$ non decreasing, please ?
\end{solution}



\begin{solution}[by \href{https://artofproblemsolving.com/community/user/29428}{pco}]
	\begin{tcolorbox}Why is $f(x)$ non decreasing, please ?\end{tcolorbox}
$f(y)\ge 0$ $\forall y\ge 0$ and so $f(x+y)\ge f(x)$ $\forall y\ge 0$
\end{solution}
*******************************************************************************
-------------------------------------------------------------------------------

\begin{problem}[Posted by \href{https://artofproblemsolving.com/community/user/149837}{ropro01}]
	Define $f(n)=n+[\sqrt{n}]$ for $n\in\mathbb{Z}^+$, where $[x]$ denotes the largest integer not greater than $x$.
Furthermore, define $f^{(k)}(n)$ recursively by $f^{(0)}(n)=n$ and $f^{(k+1)}(n)=f\left(f^{(k)}(n)\right)$ for $k\ge 0$.

Prove that for every positive integer $m$, there is a nonnegative integer $l$ such that $f^{(l)}(m)$ is a perfect square.
	\flushright \href{https://artofproblemsolving.com/community/c6h483540}{(Link to AoPS)}
\end{problem}



\begin{solution}[by \href{https://artofproblemsolving.com/community/user/29428}{pco}]
	\begin{tcolorbox}Define $f(n)=n+[\sqrt{n}]$ for $n\in\mathbb{Z}^+$, where $[x]$ denotes the largest integer not greater than $x$.
Furthermore, define $f^{(k)}(n)$ recursively by $f^{(0)}(n)=n$ and $f^{(k+1)}(n)=f\left(f^{(k)}(n)\right)$ for $k\ge 0$.

Prove that for every positive integer $m$, there is a nonnegative integer $l$ such that $f^{(l)}(m)$ is a perfect square.\end{tcolorbox}
Let $m\in\mathbb N$ 

Let $m=n^2+p$ with $p\in[0,2n]$

If $p=0$, we got the result
If $p=n+1$, then $f(m)=(n^2+p)+n=(n+1)^2$ and we got the result.
If $p>n+1$, then $f(m)=(n^2+p)+n=(n+1)^2+(p-(n+1))$ and note that $0<p-(n+1)<p$
If $p<n+1$, then $f(m)=(n^2+p)+n<(n+1)^2$ and $f(f(m))=(n^2+p+n)+n=(n+1)^2+(p-1)$ and note that $0\le p-1<p$

In the two last cases, the distance between $f(m)$ or $f(f(m))$ and the greatest perfect square below them is stricly decreasing and so will eventually reach $0$

Q.E.D.
\end{solution}
*******************************************************************************
-------------------------------------------------------------------------------

\begin{problem}[Posted by \href{https://artofproblemsolving.com/community/user/89198}{chaotic_iak}]
	Find all functions $f : \mathbb{R} \rightarrow \mathbb{R}$ such that
\[f(yf(x) + x) \le f(f(y)) + x^2f(xy)\]
for all $x,y \in \mathbb{R}$.
	\flushright \href{https://artofproblemsolving.com/community/c6h483681}{(Link to AoPS)}
\end{problem}



\begin{solution}[by \href{https://artofproblemsolving.com/community/user/29428}{pco}]
	\begin{tcolorbox}Find all functions $f : \mathbb{R} \rightarrow \mathbb{R}$ such that
\[f(yf(x) + x) \le f(f(y)) + x^2f(xy)\]
for all $x,y \in \mathbb{R}$.\end{tcolorbox}
I dont think that there is a general form for the solutions of this equation.

Besides constant functions $f(x)=c\ge 0$, we can find infinitely many very different solutions. For example (limiting to continous functions) :

$f(x)=a\min(e^{a^2-x^2},1)$ where $a$ is any positive real.

$f(x)=1+\lim_{y\to x}\left(\frac 1{(y-1)^2}-\frac{|y(y-2)|}{(y-1)^2}\right)$

And these are just a very very limited subset of solutions (I let you find the key of these examples :) )
\end{solution}
*******************************************************************************
-------------------------------------------------------------------------------

\begin{problem}[Posted by \href{https://artofproblemsolving.com/community/user/121558}{Bigwood}]
	$f,g$ are functions both from $\mathbb{R}$ to itself which satisfies
\[f(y+xg(y))=g(xy)+f(g(y))\]
for all $(x,y)\in\mathbb{R}^2$. Determine all such pairs of functions.
	\flushright \href{https://artofproblemsolving.com/community/c6h484044}{(Link to AoPS)}
\end{problem}



\begin{solution}[by \href{https://artofproblemsolving.com/community/user/29428}{pco}]
	\begin{tcolorbox}$f,g$ are functions both from $\mathbb{R}$ to itself which satisfies
\[f(y+xg(y))=g(xy)+f(g(y))\]
for all $(x,y)\in\mathbb{R}^2$. Determine all such pairs of functions.\end{tcolorbox}
Let $P(x,y)$ be the assertion $f(y+xg(y))=g(xy)+f(g(y))$

$P(1,0)$ $\implies$ $g(0)=0$

If $g(a)=0$ for some $a\ne 0$, then $P(\frac xa,a)$ $\implies$ $g(x)=f(a)-f(0)$ and so $g(x)$ is constant and so zero and then $f(x)=g(x)=0$ $\forall x$

If $g(x)\ne 0$ $\forall x\ne 0$, then $P(\frac{g(x)-x}{g(x)},x)$ $\implies$ $g(x\frac{g(x)-x}{g(x)})=0$ $\implies$ $g(x)=x$ $\forall x\ne 0$

So $g(x)=x$ $\forall x$, then $P(-1,x)$ $\implies$ $f(x)=x$ which indeed is a solution.

\begin{bolded}Hence the two solutions\end{underlined}\end{bolded} :
$f(x)=g(x)=0$ $\forall x$
$f(x)=g(x)=x$ $\forall x$
\end{solution}



\begin{solution}[by \href{https://artofproblemsolving.com/community/user/141397}{subham1729}]
	Put $x=(g(y)-y\/g(y)$
get $g(g(y)-y\/g(y).x)=0$
so we get $a$ s.t $g(a)=0$
$g(ax)=f(a)-f(0)$
so either $a=0$ or $g(x)$ is constant and= $0$
so we get $g(x)=f(x)=0$
or $g(x)=x,f(x)=x+1-f(1)$ put $x=1$ we get $f(1)=1$
Thus we get all solutions.
\end{solution}



\begin{solution}[by \href{https://artofproblemsolving.com/community/user/93837}{jjax}]
	Both the above posts have all the key steps, which are the same as mine.
But both posts are a little off the mark.

The correct set of solutions is:
$f(x)=c$ and $g(x)=0$ for all $x$, for some constant $c$.
$f(x)=x+c$ and $g(x)=x$ for all $x$, for some constant $c$.

If requested, I can post a complete solution but what's above is already nearly right.
\end{solution}



\begin{solution}[by \href{https://artofproblemsolving.com/community/user/29428}{pco}]
	Oops, you're perfectly right, thanks :blush:
\end{solution}
*******************************************************************************
-------------------------------------------------------------------------------

\begin{problem}[Posted by \href{https://artofproblemsolving.com/community/user/82904}{gold46}]
	Find all $f:\mathbb{R}^+ \rightarrow \mathbb{R}^+$  functions that satisfy $2f\left (\sqrt{\frac{x^2+xy+y^2}{3}}\right )=f(x)+f(y)$ for all $x,y\in \mathbb{R}^+$.
	\flushright \href{https://artofproblemsolving.com/community/c6h484381}{(Link to AoPS)}
\end{problem}



\begin{solution}[by \href{https://artofproblemsolving.com/community/user/82904}{gold46}]
	any idea ?
\end{solution}



\begin{solution}[by \href{https://artofproblemsolving.com/community/user/29428}{pco}]
	Up to now, I only proved that $f(x)$ is non decreasing and continuous.
I think that the only solution is likely $f(x)=c$ but no proof till now :oops:
\end{solution}



\begin{solution}[by \href{https://artofproblemsolving.com/community/user/29428}{pco}]
	\begin{tcolorbox}Find all $f:\mathbb{R}^+ \rightarrow \mathbb{R}^+$  functions that satisfy $2f\left (\sqrt{\frac{x^2+xy+y^2}{3}}\right )=f(x)+f(y)$ for all $x,y\in \mathbb{R}^+$.\end{tcolorbox}
$f(x)=c$ $\forall x$ is a solution. So let us from now look only for non constant solutions.

Let $P(x,y)$ be the assertion $2f\left(\sqrt{\frac{x^2+xy+y^2}3}\right)=f(x)+f(y)$

1) $f(x)$ is non decreasing
==================
Let $y>x$
Suppose that $f(y)<f(x)$. Then :
Let the sequence $a_0=x$ and $a_1=y$ and $a_{n+2}=\frac{-a_n+\sqrt{12a_{n+1}^2-3a_n^2}}2$
It's easy to check that the sequence $a_n$ is increasing.

$P(a_n,a_{n+2})$ $\implies$ $2f(a_{n+1})=f(a_n)+f(a_{n+2})$ and so $f(a_n)=f(a_0)+n(f(a_1)-f(a_0))$ $=f(x)+n(f(y)-f(x))$
And so, for $n$ great enough, $f(a_n)<0$, which is impossible.
So $y>x$ $\implies$ $f(y)\ge f(x)$
Q.E.D.

2) $f(x)$ is increasing and so injective
==========================
Let $b>a$ so that $f(b)\ge f(a)$, according to 1) above.
Suppose that $f(b)=f(a)$
Let the sequence $a_0=b$ and $a_{n+1}=\frac{-a+\sqrt{12a_{n}^2-3a^2}}2$
It's easy to check that the sequence $a_n$ is increasing and that $\lim_{n\to+\infty}a_n=+\infty$

$P(a,a_{n+1})$ $\implies$ $2f(a_{n})=f(a)+f(a_{n+1})$ and so $f(a_n)=f(a)$ $\forall n$
And since $f(x)$ is non decreasing and $a_n\to+\infty$, we get $f(x)=f(a)$ $\forall x\ge a$

Let then $u>0$. Choosing $x$ great enough such such that $x\ge a$ and $\sqrt{\frac{u^2+ux+x^2}3}\ge a$, $P(u,x)$ implies $f(u)=f(a)$
And so $f(x)=f(a)$ $\forall x$, which is impossible since we are currently considering only non constant solutions.
So $f(b)>f(a)$
Q.E.D.

3) No non constant solutions
====================
$P(1,22)$ $\implies$ $f(13)=\frac{f(1)+f(22)}2$

$P(1,13)$ $\implies$ $f(\sqrt{61})=\frac{3f(1)+f(22)}4$

$P(13,22)$ $\implies$ $f(\sqrt{313})=\frac{f(1)+3f(22)}4$

$P(\sqrt{61},\sqrt{313})$ $\implies$ $f(\sqrt{\frac {374+\sqrt{19093}}3})$ $=\frac{f(1)+f(22)}2$ $=f(13)$

And so, since injective, $\sqrt{\frac {374+\sqrt{19093}}3}=13$, which is false
Q.E.D.


Hence the solution : $\boxed{f(x)=c}$ $\forall x>0$ and for any $c>0$
\end{solution}
*******************************************************************************
-------------------------------------------------------------------------------

\begin{problem}[Posted by \href{https://artofproblemsolving.com/community/user/87206}{safa698}]
	Find all function 
$ f:\mathbb{R}\rightarrow\mathbb{R} $ satisfies 
$ f(x)^2+2yf(x)+f(y)=f(y+f(x))$.
	\flushright \href{https://artofproblemsolving.com/community/c6h484393}{(Link to AoPS)}
\end{problem}



\begin{solution}[by \href{https://artofproblemsolving.com/community/user/29428}{pco}]
	\begin{tcolorbox}Find all function 
$ f:\mathbb{R}\rightarrow\mathbb{R} $ satisfies 
$ f(x)^2+2yf(x)+f(y)=f(y+f(x))$.\end{tcolorbox}
$f(x)=0$ $\forall x$ is a solution. So let us from now look only for non allzeroes solutions.

Let $P(x,y)$ be the assertion $f(x)^2+2yf(x)+f(y)=f(y+f(x))$
Let $a=f(0)$
Let $u,v$ such that $f(u)=v\ne 0$

$P(x,0)$ $\implies$ $f(f(x))=f(x)^2+a$
$P(x,-f(x))$ $\implies$ $f(-f(x))=f(x)^2+a$
$P(x,-f(y))$ $\implies$ $f(f(x)-f(y))=(f(x)-f(y))^2+a$

$P(u,\frac{x}{2v}-\frac v2)$ $\implies$ $x=f(\frac{x}{2v}+\frac v2)-f(\frac{x}{2v}-\frac v2)$ and so any real may be written as $x=f(r)-f(s)$

And since $f(f(r)-f(s))=(f(r)-f(s))^2+a$ $\forall r,s$, we get $f(x)=x^2+a$ $\forall x$ which indeed is a solution

\begin{bolded}Hence the solutions\end{underlined}\end{bolded} :
$f(x)=0$ $\forall x$
$f(x)=x^2+a$ $\forall x$
\end{solution}



\begin{solution}[by \href{https://artofproblemsolving.com/community/user/117753}{Dragonboy}]
	\begin{tcolorbox}
$P(u,\frac{x}{2v}-\frac v2)$ $\implies$ $x=f(\frac{x}{2v}+\frac v2)-f(\frac{x}{2v}-\frac v2)$ and so any real may be written as $x=f(r)-f(s)$

\end{tcolorbox}
Is it possible to share your idea behind this substitution, I mean to say why any one will think about substituting $y$, by $\frac{x}{2v}-\frac v2$.  I'm really curious. :)
\end{solution}



\begin{solution}[by \href{https://artofproblemsolving.com/community/user/29428}{pco}]
	I started from my aim and just wrote in left part of equation $f(x)^2+2yf(x)=z$

And so $y=\frac{z-f(x)^2}{2f(x)}$

And I choosed $x$ such that $f(x)\ne 0$, so $x=u$ and $y=\frac {z-f(u)^2}{2f(u)}$ $=\frac z{2f(u)}-\frac{f(u)}2$ $=\frac z{2v}-\frac v2$
\end{solution}
*******************************************************************************
-------------------------------------------------------------------------------

\begin{problem}[Posted by \href{https://artofproblemsolving.com/community/user/141397}{subham1729}]
	Find all functions $f:N-N$ such that 

$2(f(m^2+n^2))^3=f(m)f(n)(f(m)+f(n))$
	\flushright \href{https://artofproblemsolving.com/community/c6h484435}{(Link to AoPS)}
\end{problem}



\begin{solution}[by \href{https://artofproblemsolving.com/community/user/29428}{pco}]
	\begin{tcolorbox}Find all functions $f:N-N$ such that 

$2(f(m^2+n^2))^3=f(m)f(n)(f(m)+f(n))$\end{tcolorbox}
Let $x\ne y$. If $f(x)\ne f(y)$, Wlog say $f(x)>f(y)$

Then functional equation implies $f(x)>f(x^2+y^2)>f(y)$ and we could continue this process in infinitely many steps, which is impossible since all these numbers are natural numbers and so there are finitely many possible values in $(f(y),f(x))$

So $f(x)=f(y)$ $\forall x,y$ and the unique solution $\boxed{f(x)=c}$ $\forall x$ and for any $c\in\mathbb N$ (and it's easy to check that these indeed are solutions)
\end{solution}
*******************************************************************************
-------------------------------------------------------------------------------

\begin{problem}[Posted by \href{https://artofproblemsolving.com/community/user/67669}{minhphuc.v}]
	Find all continuous functions $f: R \rightarrow R $ satisfying $\{f(x+y)\}=\{f(x)+f(y)\}$ for every $x,y \in R$ and 
$\{t\}=t-[t]$
	\flushright \href{https://artofproblemsolving.com/community/c6h484446}{(Link to AoPS)}
\end{problem}



\begin{solution}[by \href{https://artofproblemsolving.com/community/user/29428}{pco}]
	\begin{tcolorbox}Find all continuous functions $f: R \rightarrow R $ satisfying $\{f(x+y)\}=\{f(x)+f(y)\}$ for every $x,y \in R$ and 
$\{t\}=t-[t]$\end{tcolorbox}
So $f(x+y)-f(x)-f(y)\in\mathbb Z$ and continuity implies $f(x+y)-f(x)-f(y)=b$ with $b$ constant $\in\mathbb Z$

So $(f(x+y)+b)=(f(x)+b)+(f(y)+b)$ and continuity implies :

$\boxed{f(x)=ax-b}$ $\forall x$ and for any $a\in\mathbb R$ and $b\in\mathbb Z$ and it's easy to check back that these indeed are solutions.
\end{solution}
*******************************************************************************
-------------------------------------------------------------------------------

\begin{problem}[Posted by \href{https://artofproblemsolving.com/community/user/110552}{youarebad}]
	find all function $f : \mathbb{N} \rightarrow \mathbb{N}$, such that

$f(f(n) + f(m) + 1) = n + m + 2011$, for all $n, m \in \mathbb{N}$
	\flushright \href{https://artofproblemsolving.com/community/c6h484624}{(Link to AoPS)}
\end{problem}



\begin{solution}[by \href{https://artofproblemsolving.com/community/user/29428}{pco}]
	\begin{tcolorbox}find all function $f : \mathbb{N} \rightarrow \mathbb{N}$, such that

$f(f(n) + f(m) + 1) = n + m + 2011$, for all $n, m \in \mathbb{N}$\end{tcolorbox}
Let $P(x,y)$ be the assertion $f(f(x)+f(y)+1)=x+y+2011$

$f(x)$ is injective

Comparing $P(x,y+1)$ and $P(x+1,y)$, we get $f(f(x)+f(y+1)+1)=f(f(x+1)+f(y)+1)$ and so $f(y+1)-f(y)=f(x+1)-f(x)$

So $f(x+1)=f(x)+c$ and so $f(x)=cx+f(1)-c$

Pluging $f(x)=ax+b$ in original equation, we get $(a,b)=(1,670)$ or $(-1,-2012)$

Hence the unique solution $\boxed{f(x)=x+670}$
\end{solution}



\begin{solution}[by \href{https://artofproblemsolving.com/community/user/110552}{youarebad}]
	\begin{tcolorbox}$f(x)$ is injective\end{tcolorbox}

How to prove that $f(x)$ is injective ?
\end{solution}



\begin{solution}[by \href{https://artofproblemsolving.com/community/user/64716}{mavropnevma}]
	From $f(f(x)+f(y)+1)=x+y+2011$ for all $x,y$, assume $f(x_1) = f(x_2)$, so $x_1+y+2011 = f(f(x_1)+f(y)+1)=f(f(x_2)+f(y)+1)=x_2+y+2011$, whence $x_1=x_2$ and so $f(x)$ is injective. Quite easy, classical and from just definition(s).
\end{solution}



\begin{solution}[by \href{https://artofproblemsolving.com/community/user/29428}{pco}]
	Let $a,b$ such that $f(a)=f(b)=c$

$P(a,1)$ $\implies$ $f(f(1)+c+1)=a+2012$
$P(b,1)$ $\implies$ $f(f(1)+c+1)=b+2012$
Subtracting implies $a=b$

* \begin{bolded}edited \end{underlined}\end{bolded}* : too late :)
\end{solution}
*******************************************************************************
-------------------------------------------------------------------------------

\begin{problem}[Posted by \href{https://artofproblemsolving.com/community/user/6248}{matholympicman}]
	Find all f:R---->R that satisfy  f( f(x)^2 + f(y)) = x f(x) + y
	\flushright \href{https://artofproblemsolving.com/community/c6h484647}{(Link to AoPS)}
\end{problem}



\begin{solution}[by \href{https://artofproblemsolving.com/community/user/107451}{Learner94}]
	Let $P(x,y)$ be the statement. $P(0,y)$ gives $f((f(0))^2 + f(y) ) = y$. Which shows that $f$ is bijective. Let $f(0) = k$ and $f(t) = 0$. $P(0,0)$ gives $f(k^2 + k) = 0 = f(t)$ so $k^2 + k = t$. Now $P(t,t)$ gives $k=t$. So $t^2 + t = k^2 + k = t$, so $t=k=0$. Now $P(x,0)$ gives $f(f(x)^2 ) = xf(x)$. (*) Note that $P(0,y)$ gives $f(f(y))= y$. So $P(f(x), 0)$ gives $f(x^2) = xf(x)$.So $f(f(x)^2) = xf(x) = f(x^2)$ implying $f(x)^2 = x^2$. Now let $f(t_1) = t_1$ and $f(t_2) = -t_2$. $P(t_1,t_2)$ gives $f(t_1 ^2 - t_2) = t_2 + t_1 ^2$, which is false unless one of $t_1$ or $t_2$ is zero. This gives that $f(x) = x$  for all $x \in \mathbb{R}$ or $f(x) = - x$ for all $x \in \mathbb{R}$.

Edited mistake.
\end{solution}



\begin{solution}[by \href{https://artofproblemsolving.com/community/user/29428}{pco}]
	\begin{tcolorbox}Find all f:R---->R that satisfy  f( f(x)^2 + f(y)) = x f(x) + y\end{tcolorbox}
Let $P(x,y)$ be the assertion $f(f(x)^2+f(y))=xf(x)+y$

$f(x)$ is clearly bijective.
Let then $a$ such that $f(a)=0$ 

$P(a,x)$ $\implies$ $f(f(x))=x$
$P(x,a)$ $\implies$ $f(f(x)^2)=xf(x)+a$
$P(f(x),a)$ $\implies$ $f(x^2)=xf(x)+a$

So $f(f(x)^2)=f(x^2)$ and so (since injective) $f(x)^2=x^2$ and so $\forall x$, either $f(x)=x$, either $f(x)=-x$ and $a=0$

Suppose then $\exists u,v\ne 0$ such that $f(u)=u$ and $f(v)=-v$

$P(v,u)$ $\implies$ $f(v^2+u)=-v^2+u$ and so :
either $v^2+u=-v^2+u$ and so $v=0$, impossible
either $-v^2-u=-v^2+u$ and so $u=0$, impossible

\begin{bolded}And so the two solutions\end{underlined}\end{bolded} :
either $f(x)=x$ $\forall x$, which indeed is a solution
either $f(x)=-x$ $\forall x$, which indeed is a solution
\end{solution}



\begin{solution}[by \href{https://artofproblemsolving.com/community/user/103150}{Djurre}]
	\begin{bolded}Problem\end{bolded}
Find all $f:\mathbb{R}\to\mathbb{R}$ that satisfy $f(f(x)^2+f(y))=xf(x)+y$.

[hide="Solution"]
I assume that that it is $f(f(x)^2+f(y))=xf(x)+y\quad\forall x,y\in\mathbb{R}$.
Let $P(x,y)$ be the assertion of $f(f(x)^2+f(y))=xf(x)+y$.

==== $f$ is bijective and $f(f(0))=0$
$P(0,y)\Longrightarrow f(f(0)^2+f(y))=y\quad\forall y\in\mathbb{R}$
So $f\circ f $ is bijective, so $f$ is bijective. Therefore $\exists a\in\mathbb{R}$ sucht that $f(a)=0$.
$P(a,a)\Longrightarrow f(0)=a$
Combining the previous results we conclude that $f(f(0))=0$.

==== $f(0)=0$ and $f(f(x))=x\quad\forall x\in\mathbb{R}$
$P(0,f(0))\Longrightarrow f(f(0)^2+f(f(0)))=f(0)$
Because $f$ is injective and $f(f(0))=0$ we see that $f(0)^2=0$, so $f(0)=0$.
From $P(0,x)$ it follows inmediatly that $f(f(x))=x\quad\forall x\in\mathbb{R}$

==== $f(x)^2=x^2$
$P(f(x),0)\Longrightarrow f(f(f(x))^2+f(0))=f(x)f(f(x))\quad\forall x\in\mathbb{R}$
So $f(x^2)=xf(x)\quad\forall x\in\mathbb{R}$
If we combine the abovel line with $P(x,0)$ we see that
$f(x^2)=xf(x)=f(f(x)^2)\quad\forall x\in\mathbb{R}$
But we know that $f$ is injective, so $x^2=f(x)^2\quad\forall x\in\mathbb{R}$.
So $f(x)=x$ or $f(x)=-x$.

Suppose there exists $b,c\in\mathbb{R}_{\neq 0}$ (because we already know that $f(0)=0$) with $f(b)=-b$ 
and $f(c)=c$. With $P(b,c)\Longrightarrow f(b^2+c)=-b^2+c$ we see that there are two options:
- $b^2+c=-b^2+c$ so $b=0$ or
- $-b^2-c=-b^2+c$ so $c=0$.
In both cases we find that $b$ or $c$ equals $0$, contradiction. 

So the only two functions with satisfy the given functional equation are $f(x)=-x\quad\forall x\in\mathbb{R}$ and $f(x)=x\quad\forall x\in\mathbb{R}$. If we control the functions they satisfy also.
$\Box$

[\/hide]
\end{solution}



\begin{solution}[by \href{https://artofproblemsolving.com/community/user/103150}{Djurre}]
	\begin{tcolorbox}Let $P(x,y)$ be the statement. $P(0,y)$ gives $f((f(0))^2 + f(y) ) = y$. Which shows that $f$ is bijective. Let $f(0) = k$ and $f(t) = 0$. $P(0,0)$ gives $f(k^2 + k) = 0 = f(t)$ so $k^2 + k = t$. Now $P(t,t)$ gives $k=t$. So $t^2 + t = k^2 + k = t$, so $t=k=0$. Now $P(x,0)$ gives $f(f(x)^2 ) = xf(x)$. (*) Note that $P(0,y)$ gives $f(f(y))= y$. So $P(f(x), 0)$ gives $f(x^2) = x^2$. Setting $x= f(x)$ in this equation we get $f( (f(x)^2) = f(x)^2 $. So (*)  now becomes $f(x)^2 = xf(x)$ which by bijectivity of $f(x)$ gives $f(x) = x$ for all real $x$.\end{tcolorbox}

First there's a flaw in your solution, because your don't get $f(x)=-x\quad\forall x\in\mathbb{R}$ as a solution.
Second I don't get the following step: "So $P(f(x),0)$ gives us that $f(x^2)=x^2$.
\end{solution}



\begin{solution}[by \href{https://artofproblemsolving.com/community/user/107451}{Learner94}]
	Yeah that's a mistake, I have fixed it.
\end{solution}



\begin{solution}[by \href{https://artofproblemsolving.com/community/user/47092}{utsab001}]
	\begin{tcolorbox}[quote="matholympicman"]Find all f:R---->R that satisfy  f( f(x)^2 + f(y)) = x f(x) + y\end{tcolorbox}
Let $P(x,y)$ be the assertion $f(f(x)^2+f(y))=xf(x)+y$

$f(x)$ is clearly bijective.
\end{tcolorbox}
Dear pco can you please show me how to prove f(x) is injective ? :( i could prove surjectivity..and the rest is fine..
\end{solution}



\begin{solution}[by \href{https://artofproblemsolving.com/community/user/93837}{jjax}]
	Suppose $f(a)=f(b)$.
Then $xf(x)+a=f(f(x)^2+f(a))=f(f(x)^2+f(b))=xf(x)+b$.
Thus $a=b$.

It's a common idea to prove injectivity when you see $x$ or $y$ outside $f$s.
\end{solution}



\begin{solution}[by \href{https://artofproblemsolving.com/community/user/103150}{Djurre}]
	From $P(0,y)$ we get that $f(f(0)^2+f(y))=y\quad\forall y\in\mathbb{R}$. Let $a,b\in\mathbb{R}$ and $f(a)=f(b)$ we will know prove that $a=b$. We know that
$f(f(0)^2+f(a))-f(f(0)^2+f(b))=a-b$
With $f(a)=f(b)$ we see that $f(f(0)^2+f(a))=f(f(0)^2+f(b))$ so $a=b$. Hence the injectivity of $f$ is proven.
\end{solution}
*******************************************************************************
-------------------------------------------------------------------------------

\begin{problem}[Posted by \href{https://artofproblemsolving.com/community/user/110552}{youarebad}]
	Find all injective function $f : \mathbb{N} \rightarrow \mathbb{N}$ that satisfy :

$f(f(m) + f(n)) = f(f(m)) + f(n)$, $f(1) = 2, f(2) = 4$.

Is the answer is :

$f(1) = 2$
$f(n) = n + 2$, for all $n \in \mathbb{N}$ except $1$

Is there another answer ? :blush:
	\flushright \href{https://artofproblemsolving.com/community/c6h484810}{(Link to AoPS)}
\end{problem}



\begin{solution}[by \href{https://artofproblemsolving.com/community/user/29428}{pco}]
	\begin{tcolorbox}Find all injective function $f : \mathbb{N} \rightarrow \mathbb{N}$ that satisfy :

$f(f(m) + f(n)) = f(f(m)) + f(n)$, $f(1) = 2, f(2) = 4$.

Is the answer is :

$f(1) = 2$
$f(n) = n + 2$, for all $n \in \mathbb{N}$ except $1$

Is there another answer ? :blush:\end{tcolorbox}
Claim : this is indeed the only solution to the problem.

Let $P(x,y)$ be the assertion $f(f(x)+f(y))=f(f(x))+f(y)$
Let $A=f(\mathbb N)$

Comparing $P(x,1)$ and $P(1,x)$, we get $f(f(x))=f(x)+2$ and so $f(x)=x+2$ $\forall x\in A$

$a\in A$ $\implies$ $a+2\in A$ and so :
Since $2\in A$, all even numbers are in $A$ and so $f(n)=n+2$ $\forall $ even $n$
$f(3)$ can not be even since $f(x)$ is injective and all even numbers are image of $1$ or an even number.
So $A$ contains odd numbers and let then $u$ be the smallest odd number in $A$
Note that any odd number greater than $u$ is also in $A$

So $f(n)=n+2$ $\forall$ odd $n\ge u$
If $u\ge 7$, then :
$f(3),f(5)$ must be odd (as we seen previously)
$f(3),f(5)$ cant be $\ge u+2$ since $f(x)$ is injective and all odd numbers $\ge u+2$ are image of an odd number $\ge u$
So $f(3)=f(5)=u$, which is impossible since injective

So $u\in\{1,3,5\}$ but $f(1)=2\ne 1+2$ and so $u\ne 1$

So $u\in\{3,5\}$

$u=3$ implies $f(n)=n+2$ $\forall n\ne 1$ but then $3\notin A$ and so $u\ne 3$

So $u=5$ and $f(n)=n+2$ $\forall n\notin\{1,3\}$
But then $f(3)=u=5$, as we previously seen 
and so $f(n)=n+2$ $\forall n\ne 1$ and $f(1)=2$

And it's easy to check back that this is indeed a solution.
Q.E.D.
\end{solution}
*******************************************************************************
-------------------------------------------------------------------------------

\begin{problem}[Posted by \href{https://artofproblemsolving.com/community/user/110552}{youarebad}]
	Determine all injective function  $f : \mathbb{R} \rightarrow \mathbb{R}$ which satisfy the functional equation 

$(x - y)f(x+y) - (x+y)f(x-y) = 4xy(x^2-y^2)$
	\flushright \href{https://artofproblemsolving.com/community/c6h484811}{(Link to AoPS)}
\end{problem}



\begin{solution}[by \href{https://artofproblemsolving.com/community/user/29428}{pco}]
	\begin{tcolorbox}Determine all injective function  $f : \mathbb{R} \rightarrow \mathbb{R}$ which satisfy the functional equation 

$(x - y)f(x+y) - (x+y)f(x-y) = 4xy(x^2-y^2)$\end{tcolorbox}
Let $P(x,y)$ be the assertion $(x-y)f(x+y)-(x+y)f(x-y)=4xy(x^2-y^2)$

$P(\frac{x+1}2,\frac {x-1}2)$ $\implies$ $f(x)=x^3+x(f(1)-1)$

Pluging back $f(x)=x^3+ax$ in original equation, we see that this indeed is a solution whatever is $a$.

It remains to check the (rather stupid and useless) injectivity constraint and we get $a\ge 0$

Hence the answer : $\boxed{f(x)=x^3+ax}$ $\forall x$ and whatever is $a\ge 0$
\end{solution}
*******************************************************************************
-------------------------------------------------------------------------------

\begin{problem}[Posted by \href{https://artofproblemsolving.com/community/user/123332}{Mathlover20}]
	If $f:R\rightarrow R$ be a function such that $f(x^2)=f(x)^2$ and $f(x+1)=f(x)+1$.

Determine the function $f$ and prove that it is an odd function.
	\flushright \href{https://artofproblemsolving.com/community/c6h484819}{(Link to AoPS)}
\end{problem}



\begin{solution}[by \href{https://artofproblemsolving.com/community/user/86443}{roza2010}]
	Its $\boxed{f(x)=x}$
\end{solution}



\begin{solution}[by \href{https://artofproblemsolving.com/community/user/29428}{pco}]
	\begin{tcolorbox}If $f:R\rightarrow R$ be a function such that $f(x^2)=f(x)^2$ and $f(x+1)=f(x)+1$.

Determine the function $f$ and prove that it is an odd function.\end{tcolorbox}
From $f(x^2)=f(x)^2$ we get $f(x)\ge 0$ $\forall x\ge 0$

From $f(x+1)=f(x)+1$, we get $f(x+n)=f(x)+n$ $\forall x\in\mathbb R$, $\forall n\in\mathbb Z$

So $f(x)=f(\lfloor x\rfloor+(x-\lfloor x\rfloor))$ $=f(x-\lfloor x\rfloor)+\lfloor x\rfloor$ $\ge \lfloor x\rfloor$ $>x-1$

So $f((x+n)^2)>(x+n)^2-1$ $\forall x\in\mathbb R$, $\forall n\in\mathbb Z$

But $f((x+n)^2)=f(x+n)^2=f(x)^2+2nf(x)+n^2$

So $f(x)^2+2nf(x)+n^2$ $>(x+n)^2-1$

So $2n(f(x)-x)>x^2-f(x)^2-1$

If $f(x)\ne x$, and choosing $|n|$ great enough and $n$ with opposite sign to $f(x)-x$, we get $LHS$ as small as we want and so contradiction.

So $\boxed{f(x)=x}$ $\forall x$, which indeed is a solution.
And which is indeed an odd function.
\end{solution}



\begin{solution}[by \href{https://artofproblemsolving.com/community/user/96840}{ACCCGS8}]
	This is sort of similar to Q1 of the Functional Equations Marathon! See http://www.artofproblemsolving.com/Forum/viewtopic.php?f=36&t=350187

This problem is a nice variation of Q1 of that marathon.
\end{solution}
*******************************************************************************
-------------------------------------------------------------------------------

\begin{problem}[Posted by \href{https://artofproblemsolving.com/community/user/67669}{minhphuc.v}]
	Find all function $f: \mathbb{R} \rightarrow  \mathbb{R}$ satisfying
1. $f$ is a surjective function;
2. $f$ is monotone increasing;
3. $f(f(x))=f(x)+12x$ for all $x \in \mathbb{R}$.
	\flushright \href{https://artofproblemsolving.com/community/c6h484843}{(Link to AoPS)}
\end{problem}



\begin{solution}[by \href{https://artofproblemsolving.com/community/user/29428}{pco}]
	\begin{tcolorbox}Find all function $f: R \rightarrow  R$ satisfy
1. $f$ is a surjective function
2. $f$ is a monotone increase
3. $f(f(x))=f(x)+12x$ for all $x \in R$\end{tcolorbox}
From 3., we get that $f(x)$ is injective and so, using 1., that it is bijective.

From 3., we get $f^{[n]}(x)=\frac{f(x)+3x}74^n-\frac{f(x)-4x}7(-3)^n$ $\forall n\in\mathbb N_0$

Since bijective, this is true $\forall n\in \mathbb Z$

From 2., we get that $f^{[n]}(x)$ is increasing too.

Notice that $f^{[n+1]}(x)=f^{[n]}(x)$ for some $x$ $\implies$ (injectivity) $f(x)=x$ $\implies$ (using 3.) $x=0$

So $\frac{f^{[n+2]}(x)-f^{[n+1]}(x)}{f^{[n+1]}(x)-f^{[n]}(x)}>0$ $\forall x\ne 0$, $\forall n\in\mathbb Z$

So $\frac{3\frac{f(x)+3x}74^{n+1}+4\frac{f(x)-4x}7(-3)^{n+1}}{3\frac{f(x)+3x}74^{n}+4\frac{f(x)-4x}7(-3)^{n}}$ $>0$ $\forall x\ne 0$, $\forall n\in\mathbb Z$

But, if $f(x)-4x\ne 0$, setting $n\to-\infty$ in the above expression gives in $LHS$ a limit $-3<0$

So $f(x)=4x$ $\forall x\ne 0$ and so $f(0)=0$, since monotonous

So $\boxed{f(x)=4x}$ $\forall x$ which indeed is a solution.
\end{solution}
*******************************************************************************
-------------------------------------------------------------------------------

\begin{problem}[Posted by \href{https://artofproblemsolving.com/community/user/110552}{youarebad}]
	Find all $f : \mathbb{R} \rightarrow \mathbb{R}$ such that :

$f(f(y) + x) = f(f(y) - x) + 4xf(y)$ for all $x, y \in \mathbb{R}$
	\flushright \href{https://artofproblemsolving.com/community/c6h485016}{(Link to AoPS)}
\end{problem}



\begin{solution}[by \href{https://artofproblemsolving.com/community/user/29428}{pco}]
	\begin{tcolorbox}Find all $f : \mathbb{R} \rightarrow \mathbb{R}$ such that :

$f(f(y) + x) = f(f(y) - x) + 4xf(y)$ for all $x, y \in \mathbb{R}$\end{tcolorbox}
$f(x)=0$ $\forall x$ is a solution. So let us from now look only for non allzero solutions.

Let $P(x,y)$ be the assertion $f(f(y)+x)=f(f(y)-x)+4xf(y)$
Let $a=f(0)$
Let $u,v$ such that $f(u)=v\ne 0$

$P(\frac x{8v},u)$ $\implies$ $x=2f(v+\frac x{8v})-2f(v-\frac x{8v})$ and so any real $x$ may be written $x=2f(r)-2f(s)$ for some real $r,s$

$P(f(s),s)$ $\implies$ $f(2f(s))=a+4f(s)^2$
$P(2f(s)-f(r),r)$ $\implies$ $f(2f(r)-2f(s))=a+(2f(r)-2f(s))^2$

And so $f(x)=x^2+a$ $\forall x$ which indeed is a solution, whatever is $a$

\begin{bolded}Hence the solutions\end{underlined}\end{bolded} :
$f(x)=0$ $\forall x$
$f(x)=x^2+a$ $\forall x$ and for any $a\in\mathbb R$
\end{solution}
*******************************************************************************
-------------------------------------------------------------------------------

\begin{problem}[Posted by \href{https://artofproblemsolving.com/community/user/47092}{utsab001}]
	Find all such functions $f:\mathbb{R} \to \mathbb{R} \text{satisfies the condition}$  $2f(x^3)-f^2(x)\geq 1$.
	\flushright \href{https://artofproblemsolving.com/community/c6h485220}{(Link to AoPS)}
\end{problem}



\begin{solution}[by \href{https://artofproblemsolving.com/community/user/98555}{dr_Civot}]
	We can prove that $f(x)\geq 1,\forall x\in\mathbb{R}$. Maybe that can help.
\end{solution}



\begin{solution}[by \href{https://artofproblemsolving.com/community/user/106080}{a123}]
	\begin{tcolorbox}Find all such functions $f:\mathbb{R} \to \mathbb{R} \text{satisfies the condition}$  $2f(x^3)-f^2(x)\geq 1$.\end{tcolorbox}
What does $f^2(x)$ mean? Square or double composition ?
\end{solution}



\begin{solution}[by \href{https://artofproblemsolving.com/community/user/29428}{pco}]
	\begin{tcolorbox}Find all such functions $f:\mathbb{R} \to \mathbb{R} \text{satisfies the condition}$  $2f(x^3)-f^2(x)\geq 1$.\end{tcolorbox}
If $f^2(x)$ means $(f(x))^2$ and not $f(f(x))$, then there are infinitely many solutions which may be built piece per piece.
(and I'm surprised such a problem occured in a real olympiad contest or trainind session :?: )

As dr_Civot said, it's easy to prove that $f(x)\ge 1$ $\forall x$

1) It's easy to build piece per piece over $(1,+\infty)$
Let $a_n=2^{3^n}$ for $n\in\mathbb Z$

Choose $f(x)$ as any values you want in $[1,+\infty)$ for $x\in [a_0,a_1)$

For any $n\ge 0$, build $f(x)$ over $[a_{n+1},a_{n+2})$ as any values you want with the only constraint $f(x)\ge\frac{(f(\sqrt[3]x))^2+1}2$
(note that $x\in[a_{n+1},a_{n+2})$ means $\sqrt[3]x\in[a_{n},a_{n+1})$ and so this is a valid induction construction)

For any $n\le 0$, build $f(x)$ over $[a_{n-1},a_{n})$ as any values you want with the only constraint $1\le f(x)\le\sqrt{2f(x^3)-1}$
(note that $x\in[a_{n-1},a_{n})$ means $x^3\in[a_{n},a_{n+1})$ and so this is a valid induction construction, considering that $2f(x^3)-1\ge 1$)

2) It's easy to build piece per piece over $(0,1)$
Let $a_n=2^{-3^n}$ for $n\in\mathbb Z$

Choose $f(x)$ as any values you want in $[1,+\infty)$ for $x\in [a_1,a_0)$
And apply the same method than in 1) to build $f(x)$ over $(0,1)$

3) It's easy to build piece per piece over $(-\infty,-1)$
Let $a_n=-2^{3^n}$ for $n\in\mathbb Z$

Choose $f(x)$ as any values you want in $[1,+\infty)$ for $x\in [a_1,a_0)$
And apply the same method than in 1) to build $f(x)$ over $(-\infty,-1)$

4) It's easy to build piece per piece over $(-1,0)$
Let $a_n=-2^{-3^n}$ for $n\in\mathbb Z$

Choose $f(x)$ as any values you want in $[1,+\infty)$ for $x\in [a_0,a_1)$
And apply the same method than in 1) to build $f(x)$ over $(-1,0)$

5) $f(-1)=f(0)=f(1)=1$
For these three values, we have $x^3=x$ and so $(f(x))^2-2f(x)+1\le 0$ $\iff$ $(f(x)-1)^2\le 0$
Q.E.D.
\end{solution}



\begin{solution}[by \href{https://artofproblemsolving.com/community/user/29428}{pco}]
	\begin{tcolorbox}Find all such functions $f:\mathbb{R} \to \mathbb{R} \text{satisfies the condition}$  $2f(x^3)-f^2(x)\geq 1$.\end{tcolorbox}
If $f^2(x)$ means $f(f(x))$ and not $(f(x))^2$, then there are infinitely many solutions.

For example : 

$f(x)=17$

$f(x)=2+(\sin x)^2$

$f(x)=\frac{4\pi+2\arctan x}{\pi}$

$f(x)=2+\left\{e^{3x}+\sqrt{\frac{x^2+1}2}\right\}$

...

I let you guess the common condition of these four examples (this condition gives a subset of all solutions).

And once again, I'm surprised that such an exercice occured in a real olympiad contest or training session.

But maybe it's just a crazy invented problem. 
\end{solution}



\begin{solution}[by \href{https://artofproblemsolving.com/community/user/47092}{utsab001}]
	Many many thanks \begin{bolded}pco\end{bolded} for such a detailed venture and yea i should have mentioned that here $f^2(x)\equiv (f(x))^2$ but then it would not be possible for me to get your beautiful solutions..
Actually in the original problem ,some more conditions were there like continuity,but i wanted it to do in a more general way and as im a bit noob in functional equations,i thought it might be better to have some guidance from experienced solvers like pco.
Unfortunately i don't remember the source of this problem. :o
\end{solution}



\begin{solution}[by \href{https://artofproblemsolving.com/community/user/29428}{pco}]
	In my opinion, it's not fair to post a real problem after removing some constraints just in order to get a possible general solution :(

If people who created the problem included some constraints, it's generally because these constraints are important for solving the problem in the current context (for example olympiad-level).

And when you post without the constraints in the Algebra Unsolved Problem, you claim to students reading your problem "Hey men, here is a problem whose solution is not known by me but about which I'm sure it exists an olympiad-level solution"
And, this is not the truth ... .

It would be much more honest to give the real problem, with the real constraints and to add at the end of the problem a sentence like "besides this real problem, I wonder if a more general solution, without the constraints blah-blah-blah exist"
\end{solution}



\begin{solution}[by \href{https://artofproblemsolving.com/community/user/47092}{utsab001}]
	I admit about the fact that i did an unfair job though it was not due to any kind of dishonesty rather curiosity and i believe ,there are problems on functional equations came in previous IMO's which are much more tougher than this one.
\end{solution}
*******************************************************************************
-------------------------------------------------------------------------------

\begin{problem}[Posted by \href{https://artofproblemsolving.com/community/user/110552}{youarebad}]
	Find all function $f : \mathbb{R} \rightarrow \mathbb{R}$ such that $f(f(x)^2 + y) = x^2 + f(y)$ for all $x, y \in \mathbb{R}$
	\flushright \href{https://artofproblemsolving.com/community/c6h485330}{(Link to AoPS)}
\end{problem}



\begin{solution}[by \href{https://artofproblemsolving.com/community/user/29428}{pco}]
	\begin{tcolorbox}Find all function $f : \mathbb{R} \rightarrow \mathbb{R}$ such that $f(f(x)^2 + y) = x^2 + f(y)$ for all $x, y \in \mathbb{R}$\end{tcolorbox}
Let $P(x,y)$ be the assertion $f(f(x)^2+y)=x^2+f(y)$

If $f(x)=0$ for some $x$, then $P(x,0)$ $\implies$ $x=0$
$P(y,-f(y)^2)$ $\implies$ $f(-f(y)^2)=f(0)-y^2$
$P(x,-f(y)^2)$ $\implies$ $f(f(x)^2-f(y)^2)=x^2-y^2+f(0)$ and so $f(x)$ is surjective and $f(0)=0$ (see two lines above)

$P(x,0)$ $\implies$ $f(f(x)^2)=x^2$ and so $P(x,y)$ implies $f(x+y)=f(x)+f(y)$ $\forall x\ge 0$ $\forall y$ (remember $f(x)$ is surjective)
So $f(-x)=-f(x)$ $\forall x$ and $f(x+y)=f(x)+f(y)$ $\forall x,y$

And $f(f(x)^2)=x^2$ plus surjectivity implies that $f(x)\ge 0$ $\forall x\ge 0$

So $f(x)=cx$ and plugging this in original equation, we get $c=1$ and the unique solution $\boxed{f(x)=x}$ $\forall x$
\end{solution}



\begin{solution}[by \href{https://artofproblemsolving.com/community/user/114585}{anonymouslonely}]
	if $ f(a)=f(b) $ then $ a=b $.
putting $ x,-x $ we obtain that $ f^{2}(x)=f^{2}(-x) $ so $ f(x)+f(-x)=0 $.
so $ f^{2}(x)+y=f^{2}(a)+b $ is equivalent with $ x^{2}+f(y)=a^{2}+f(b) $.
if $ y,b $ satisfies the first equality then also $ y+r,b+r $ satisfies it. 
so $ f(y)-f(b)=f(y+r)-f(b+r) $.
that means that $ f(x)+f(y)=f(x+y)+f(0) $ for every $ x,y $.
for $ x=y=0 $ in the initial equality we obtain that $ f(0)=0 $.
so $ f(x)+f(y)=f(x+y) $. so $ f(f^{2}(x))=x^{2} $.
that means that $ f $ is surjective.
that+the initial equality mean that $ f $ is increasing and $ f(x+y)=f(x)+f(y) $.
that means that $ f(x)=ax $ and solving we obtain $ a=1 $.
\end{solution}



\begin{solution}[by \href{https://artofproblemsolving.com/community/user/29428}{pco}]
	\begin{tcolorbox}if $ f(a)=f(b) $ then $ a=b $.\end{tcolorbox}
How ?
\end{solution}



\begin{solution}[by \href{https://artofproblemsolving.com/community/user/114585}{anonymouslonely}]
	oh...sorry...you're right. my solution has a gasp now.
\end{solution}



\begin{solution}[by \href{https://artofproblemsolving.com/community/user/112260}{robinson123}]
	\begin{tcolorbox}[quote="youarebad"]Find all function $f : \mathbb{R} \rightarrow \mathbb{R}$ such that $f(f(x)^2 + y) = x^2 + f(y)$ for all $x, y \in \mathbb{R}$\end{tcolorbox}
Let $P(x,y)$ be the assertion $f(f(x)^2+y)=x^2+f(y)$

If $f(x)=0$ for some $x$, then $P(x,0)$ $\implies$ $x=0$
$P(y,-f(y)^2)$ $\implies$ $f(-f(y)^2)=f(0)-y^2$
$P(x,-f(y)^2)$ $\implies$ $f(f(x)^2-f(y)^2)=x^2-y^2+f(0)$ and so $f(x)$ is surjective and $f(0)=0$ (see two lines above\end{tcolorbox}
I don't understand... I saw 2 lines above but I still don't understand. Can you show me...?
\end{solution}



\begin{solution}[by \href{https://artofproblemsolving.com/community/user/29428}{pco}]
	$f(x)$ is surjective, so $\exists u$ such that $f(u)=0$ and then (see the line two lines above) $u=0$ and so $f(0)=0$
\end{solution}



\begin{solution}[by \href{https://artofproblemsolving.com/community/user/112260}{robinson123}]
	\begin{tcolorbox}$f(x)$ is surjective, so $\exists u$ such that $f(u)=0$ and then (see the line two lines above) $u=0$ and so $f(0)=0$\end{tcolorbox}
I mean why f(x) is surjective?...
\end{solution}



\begin{solution}[by \href{https://artofproblemsolving.com/community/user/114585}{anonymouslonely}]
	because $ x^{2}-y^{2}+f(0) $ is surjective.
\end{solution}



\begin{solution}[by \href{https://artofproblemsolving.com/community/user/304072}{ht2000}]
	\begin{tcolorbox}[quote="youarebad"]Find all function $f : \mathbb{R} \rightarrow \mathbb{R}$ such that $f(f(x)^2 + y) = x^2 + f(y)$ for all $x, y \in \mathbb{R}$\end{tcolorbox}
Let $P(x,y)$ be the assertion $f(f(x)^2+y)=x^2+f(y)$

If $f(x)=0$ for some $x$, then $P(x,0)$ $\implies$ $x=0$

But how we know that there exist x such that f(x)=0 ??? 

\end{solution}



\begin{solution}[by \href{https://artofproblemsolving.com/community/user/198450}{wu2481632}]
	As mentioned many times in the thread, pco proves surjectivity in the next two lines.
\end{solution}



\begin{solution}[by \href{https://artofproblemsolving.com/community/user/306507}{KereMath}]
	Easyyyy question
\end{solution}
*******************************************************************************
-------------------------------------------------------------------------------

\begin{problem}[Posted by \href{https://artofproblemsolving.com/community/user/110552}{youarebad}]
	Find all function $f : \mathbb{N} \rightarrow \mathbb{N}$ such that $f(m^2 + f(n)) = (f(m))^2 + n$ for all $m , n \in \mathbb{N}$
	\flushright \href{https://artofproblemsolving.com/community/c6h485331}{(Link to AoPS)}
\end{problem}



\begin{solution}[by \href{https://artofproblemsolving.com/community/user/29428}{pco}]
	\begin{tcolorbox}Find all function $f : \mathbb{N} \rightarrow \mathbb{N}$ such that $f(m^2 + f(n)) = (f(m))^2 + n$ for all $m , n \in \mathbb{N}$\end{tcolorbox}
Let $P(x,y)$ be the assertion $f(x^2+f(y))=f(x)^2+y$

$P(x,y^2+f(z))$ $\implies$ $f(x^2+f(y)^2+z)=f(x)^2+y^2+f(z)$ and so $f(z+k(x^2+f(y)^2))=f(z)+k(f(x)^2+y^2)$

Setting $k=f(x)^2+y^2$, we get $f(z+(y^2+f(x)^2)(x^2+f(y)^2))=f(z)+(f(x)^2+y^2)^2$
Swapping $x,y$, we get $f(z+(y^2+f(x)^2)(x^2+f(y)^2))=f(z)+(f(y)^2+x^2)^2$

And so $f(x)^2+y^2=f(y)^2+x^2$ $\forall x,y\in\mathbb N$
And so $f(x)^2=x^2+a$ $\forall x\in\mathbb N$ and for some $a\in\mathbb Z$

But the only integer $a$ such that $n^2+a$ is a perfect square $\forall n\in\mathbb N$ is obviously $a=0$, and so $f(x)^2=x^2$

Hence the answer : $\boxed{f(x)=x}$ $\forall x$, which indeed is a solution (remember $f(x)\in\mathbb N$)
\end{solution}



\begin{solution}[by \href{https://artofproblemsolving.com/community/user/109704}{dien9c}]
	We also have the general problem
\begin{bolded}IMO 1992.\end{bolded} Find all function $f : \mathbb{R} \rightarrow \mathbb{R}$ such that $f(m^2 + f(n)) = (f(m))^2 + n$ for all $m , n \in \mathbb{R}$
The only answer is $f(x)=x$
\end{solution}
*******************************************************************************
-------------------------------------------------------------------------------

\begin{problem}[Posted by \href{https://artofproblemsolving.com/community/user/109704}{dien9c}]
	Prove that there exits fuction $f: \mathbb{N} \to \mathbb{N}$ such that
1. $f(1)=2$
2. $f(f(n))=f(n)+n$
3. $f(n+1) > f(n)$
	\flushright \href{https://artofproblemsolving.com/community/c6h485786}{(Link to AoPS)}
\end{problem}



\begin{solution}[by \href{https://artofproblemsolving.com/community/user/141397}{subham1729}]
	As it's a strictly increasing function so $n=f(f(n))-f(n)\geq (f(n)-n)$
So $2n\geq f(n)>n$.....so $f(f(n))>2n,f(f(f(n)))>3n,f(f(f(f(n))))>5n$
Take for some $n, f(n)=n+r$....so $f(f(f(f(n))))=n+4r>5n$
So we get $f(n)>2n$....a contradiction,,,,,,so no function exist.
\end{solution}



\begin{solution}[by \href{https://artofproblemsolving.com/community/user/114585}{anonymouslonely}]
	why is true your first inequality?
\end{solution}



\begin{solution}[by \href{https://artofproblemsolving.com/community/user/141397}{subham1729}]
	\begin{tcolorbox}why is true your first inequality?\end{tcolorbox}

Note $f(n+1)-f(n)\geq 1$
so $f(n+2)-f(n+1)+f(n+1)-f(n)=f(n+2)-f(n) \geq 2$
.
.
.
.
.
.
\end{solution}



\begin{solution}[by \href{https://artofproblemsolving.com/community/user/29428}{pco}]
	\begin{tcolorbox}Prove that there exits fuction $f: \mathbb{N} \to \mathbb{N}$ such that
1. $f(1)=2$
2. $f(f(n))=f(n)+n$
3. $f(n+1) > f(n)$\end{tcolorbox}
$f(n)=2+\left\lfloor\frac{(1+\sqrt 5)(n-1)}2\right\rfloor$ is a solution.
\end{solution}



\begin{solution}[by \href{https://artofproblemsolving.com/community/user/109704}{dien9c}]
	\begin{tcolorbox}As it's a strictly increasing function so $n=f(f(n))-f(n)\geq (f(n)-n)$
So $2n\geq f(n)>n$.....so $f(f(n))>2n,f(f(f(n)))>3n,f(f(f(f(n))))>5n$
Take for some $n, f(n)=n+r$....so $f(f(f(f(n))))=n+4r>5n$
So we get $f(n)>2n$....a contradiction,,,,,,so no function exist.\end{tcolorbox}
You wrong here 
$f^4(n)=5n+4r \ge 5n$ is of course
\end{solution}



\begin{solution}[by \href{https://artofproblemsolving.com/community/user/114585}{anonymouslonely}]
	to subham1729: I know that $ f(n)>n $ but you used twice this inequality and once you changed its sign.
\end{solution}
*******************************************************************************
-------------------------------------------------------------------------------

\begin{problem}[Posted by \href{https://artofproblemsolving.com/community/user/114130}{61plus}]
	Find all functions $f:\mathbb{R}\to\mathbb{R}$ so that $(x+y)(f(x)-f(y))=(x-y)f(x+y)$ for all $x,y$ that belongs to $\mathbb{R}$.
	\flushright \href{https://artofproblemsolving.com/community/c6h486615}{(Link to AoPS)}
\end{problem}



\begin{solution}[by \href{https://artofproblemsolving.com/community/user/29428}{pco}]
	\begin{tcolorbox}Find all functions $f:\mathbb{R}\to\mathbb{R}$ so that $(x+y)(f(x)-f(y))=(x-y)f(x+y)$ for all $x,y$ that belongs to $\mathbb{R}$.\end{tcolorbox}
Let $P(x,y)$ be the assertion $(x+y)(f(x)-f(y))=(x-y)f(x+y)$

$P(1,0)$ $\implies$ $f(0)=0$

Let $x\ne 0$
$P(\frac x2-1,\frac x2+1)$ $\implies$ $f(\frac x2-1)-f(\frac x2+1)=-\frac 2xf(x)$

$P(\frac x2+1,-\frac x2)$ $\implies$ $f(\frac x2+1)-f(-\frac x2)=(x+1)f(1)$

$P(-\frac x2,\frac x2-1)$ $\implies$ $f(-\frac x2)-f(\frac x2-1)=(x-1)f(-1)$

Adding these three lines, we get $f(x)=\frac{f(1)+f(-1)}2x^2+\frac{f(1)-f(-1)}2x$ $\forall x\ne 0$, still true when $x=0$

And so $\boxed{f(x)=ax^2+bx}$ $\forall x$ which indeed is a solution whatever are $a,b\in\mathbb R$
\end{solution}



\begin{solution}[by \href{https://artofproblemsolving.com/community/user/167924}{utkarshgupta}]
	Very weird !
Singapore TST 2008
http://www.artofproblemsolving.com/Forum/viewtopic.php?p=1249246&sid=b744f9044d3b4254aef08d5b7cbd41d0#p1249246
\end{solution}
*******************************************************************************
-------------------------------------------------------------------------------

\begin{problem}[Posted by \href{https://artofproblemsolving.com/community/user/67223}{Amir Hossein}]
	Find all strictly ascending functions $f$ such that for all $x\in \mathbb R$,
\[f(1-x)=1-f(f(x)).\]
	\flushright \href{https://artofproblemsolving.com/community/c6h486623}{(Link to AoPS)}
\end{problem}



\begin{solution}[by \href{https://artofproblemsolving.com/community/user/29428}{pco}]
	\begin{tcolorbox}Find all strictly ascending functions $f$ such that for all $x\in \mathbb R$,
\[f(1-x)=1-f(f(x)).\]\end{tcolorbox}
Let $P(x)$ be the assertion $f(1-x)=1-f(f(x))$

$P(1-f(x))$ $\implies$ $f(f(x))=1-f(f(1-f(x)))$
$P(x)$ $\implies$ $f(f(x))=1-f(1-x)$
So $f(f(1-f(x)))=f(1-x)$ $\implies$ (since increasing, and so injective) $f(1-f(x))=1-x$

$P(f(x))$ $\implies$ $f(1-f(x))=1-f(f(f(x)))$ and so (using previous conclusion) $f(f(f(x)))=x$ $\forall x$

Then, if $f(a)>a$ for some $a$, we get $f(f(a))>f(a)>a$ and $f(f(f(a)))>f(f(a))>f(a)>a$, impossible
Same, if $f(a)<a$ for some $a$, we get $f(f(a))<f(a)<a$ and $f(f(f(a)))<f(f(a))<f(a)<a$, impossible

Hence the unique solution $\boxed{f(x)=x}$ $\forall x$ which indeed is a solution.
\end{solution}
*******************************************************************************
-------------------------------------------------------------------------------

\begin{problem}[Posted by \href{https://artofproblemsolving.com/community/user/51901}{KittyOK}]
	Determine all functions $f:\mathbb{R} \to \mathbb{R}$ such that \[f(x+y)f(f(x)-y)=xf(x)-yf(y)\] for every $x,y \in \mathbb{R}$.
	\flushright \href{https://artofproblemsolving.com/community/c6h487187}{(Link to AoPS)}
\end{problem}



\begin{solution}[by \href{https://artofproblemsolving.com/community/user/151851}{Mathematicalx}]
	You can easily see that for,$f(x)=0$ equation holds.
\end{solution}



\begin{solution}[by \href{https://artofproblemsolving.com/community/user/151851}{Mathematicalx}]
	İf you take $y=0$ you can also get $f(f(x))=x$
\end{solution}



\begin{solution}[by \href{https://artofproblemsolving.com/community/user/151851}{Mathematicalx}]
	And if you take $y=x$ you can get (and with $f(2x)$ not equals to 0) $f(f(x)-x)=0$
\end{solution}



\begin{solution}[by \href{https://artofproblemsolving.com/community/user/140796}{mathbuzz}]
	trivially, a constant solution f(x)=0 clearly holds. let's seek non constant solutions
putting y=0 yields f(x). f(f(x))=x.f(x) for all x in R ,. so , 3 cases appear --------------- [color=#0040FF]case--1[\/color]  f(f(x))=x for all x in Ri.e. f is a bijection and f itself is its inverse function. so , f must be linear (evident from curves-- because we get the graph of the inverse function of f when we reflect the graph of f(x) w.r.t the line y=x, then some modification is enough to derive it intitutively.    ) .so , f(x)=ax+b . putting it into the main equation , we can get a & b. then the rest is trivial. [color=#0040FF]case 2[\/color]-- f(x0=0 for all x , which has been dealt before. [color=#0040FF]case--3[\/color] f(x)=0 for some x & f(f(x))=x for some x. here also , if we know that the function is continuous , then it turns  again into f(x)=0 for all x or f(x) is linear .[ otherwise f will become discontinuous . it can be checked from graphs.]. then we can do it in a similar way , but what if x is not continuous ?? :( . i have some ideas , but can't do it.

@pco--- thanks for pointing my flaw. :blush: . please give a complete solution.
\end{solution}



\begin{solution}[by \href{https://artofproblemsolving.com/community/user/29428}{pco}]
	\begin{tcolorbox}İf you take $y=0$ you can also get $f(f(x))=x$\end{tcolorbox}
Wrong

\begin{tcolorbox}putting y=0 yields f(f(x))=x i.e. f is a bijection and ...\end{tcolorbox}
Wrong

Setting $y=0$ implies $f(x)f(f(x))=xf(x)$ and so $\forall x$, either $f(x)=0$, either $f(f(x))=x$
Which is quite different of $\forall x$ $f(f(x))=x$
You should both consider the case where $f(x)=0$ for some $x$ and $f(f(x))=x$ for all other $x$ (and so $f(x)$ is not constant)
\end{solution}



\begin{solution}[by \href{https://artofproblemsolving.com/community/user/29428}{pco}]
	\begin{tcolorbox}Determine all functions $f:\mathbb{R} \to \mathbb{R}$ such that \[f(x+y)f(f(x)-y)=xf(x)-yf(y)\] for every $x,y \in \mathbb{R}$.\end{tcolorbox}
Let $P(x,y)$ be the assertion $f(x+y)f(f(x)-y)=xf(x)-yf(y)$

Let $A=f^{-1}(\{0\})$
$P(0,0)$ $\implies$ $f(0)f(f(0))=0$ and so $A\ne\emptyset$


1) $\forall x\ne 0$ : $f(-x)=-f(x)$
=====================
(a) : $P(x,0)$ $\implies$ $f(x)f(f(x))=xf(x)$
(b) : $P(x,f(x))$ $\implies$ $f(x+f(x))f(0)=xf(x)-f(x)f(f(x))$
(c) : $P(x,-x)$ $\implies$ $f(0)f(f(x)+x)=xf(x)+xf(-x)$
(a)-(b)+(c) : $x(f(x)+f(-x))=0$
Q.E.D.

2) If $A=\{0\}$, then the only solution is $f(x)=x$ $\forall x$
========================================
Let $x\ne 0$ : $P(x,x)$ $\implies$ $f(2x)f(f(x)-x)=0$ and so $f(x)=x$
Q.E.D.

3) If $A\ne\{0\}$, then the only solution is $f(x)=0$ $\forall x\ne 0$ with $f(0)=c$ where $c$ is any real
=======================================================================
Let $a\ne 0\in A$
$P(x,a-x)$ $\implies$ $xf(x)=(a-x)f(a-x)$ $\implies$ (using 1) above) $xf(x)=(x-a)f(x-a)$ $\forall x\ne a$, still true when $x=a$ and so :
New assertion $Q(x,y)$ : $xf(x)=(x-y)f(x-y)$ $\forall x\in\mathbb R$, $\forall y\in A$

Let $x\ne 0$ : $P(a,x)$ $\implies$ (using 1) above) $f(x)(f(x+a)-x)=0$ and then :
If $f(x)\ne 0$, then $f(x+a)=x$ and $Q(x+a,a)$ $\implies$ $f(x)=x+a$

So $\forall x\ne 0$ : either $f(x)=0$, either $f(x)=x+a$

$Q(0,a)$ $\implies$ $-a\in A$ and so, changing $a\to -a$ in the above result, we get :
$\forall x\ne 0$ : either $f(x)=0$, either $f(x)=x+a=x-a$, impossible

So $f(x)=0$ $\forall x\ne 0$ which indeed is a solution, whatever is $f(0)$
Q.E.D.
\end{solution}
*******************************************************************************
-------------------------------------------------------------------------------

\begin{problem}[Posted by \href{https://artofproblemsolving.com/community/user/152586}{browneyes2001}]
	Let $f:A \to A$ where $A$ is a finite set such as: $f(f(x))=2011f(x)-2010x$. Find $ f $.
	\flushright \href{https://artofproblemsolving.com/community/c6h487358}{(Link to AoPS)}
\end{problem}



\begin{solution}[by \href{https://artofproblemsolving.com/community/user/29428}{pco}]
	\begin{tcolorbox}Let $f:A \to A$ where $A$ is a finite set such as: $f(f(x))=2011f(x)-2010x$. Find $ f $.\end{tcolorbox}
Infinitely many solutions. For example :

$f(x)=\sqrt{|\sin x|}$ and $A=\emptyset$

$f(x)=x^3-16x^2-15x-17$ and $A=\{17\}$

...
\end{solution}



\begin{solution}[by \href{https://artofproblemsolving.com/community/user/152586}{browneyes2001}]
	\begin{tcolorbox}[quote="browneyes2001"]Let $f:A \to A$ where $A$ is a finite set such as: $f(f(x))=2011f(x)-2010x$. Find $ f $.\end{tcolorbox}
Infinitely many solutions. For example :

$f(x)=\sqrt{|\sin x|}$ and $A=\emptyset$

$f(x)=x^3-16x^2-15x-17$ and $A=\{17\}$

...\end{tcolorbox}

The set as I said before is finite, but you have to solve the functional equation without finding the set. Just notice that the function is bijective and then solve.... :)
\end{solution}



\begin{solution}[by \href{https://artofproblemsolving.com/community/user/89198}{chaotic_iak}]
	I suppose that means the set $A$ is given. Although that would mean messy stuffs like given before for dealing with "special cases" ($f(x)$ is any random thing when $A = \{\}$, $f(x)$ has the property that $f(x) = x$ for the single element of $A$ when $|A| = 1$, and like that).
\end{solution}



\begin{solution}[by \href{https://artofproblemsolving.com/community/user/29428}{pco}]
	In fact, assuming that domain of functional equation is $A$ too, and looking at sequence $a_0=x$ and $a_1=f(x)$ and $a_{n+2}=2011a_{n+1}-2010a_n$, it's immediate to see that if $a_0\ne a_1$, then the set $\{a_n\}\subseteq A$ has infinite cardinal.

And so $f(x)=x$ $\forall x\in A$
\end{solution}



\begin{solution}[by \href{https://artofproblemsolving.com/community/user/64716}{mavropnevma}]
	... this so being because, $\lambda^2 - 2011\lambda + 2010 = (\lambda-1)(\lambda - 2010) = 0$ being the characteristic polynomial, the general term writes as $a_n = \alpha + \beta 2010^n$, producing infinitely many values unless $\beta = 0$, when the sequence is constant, and so $f(x) = a_1 = \alpha = a_0 = x$ ...
\end{solution}
*******************************************************************************
-------------------------------------------------------------------------------

\begin{problem}[Posted by \href{https://artofproblemsolving.com/community/user/122611}{oty}]
	Find all function $f: \{\ A|A\in Q , A\geq{1}\}\ \to Q$ such that : $f(xy^2)=f(4x)f(y)+\frac{f(8x)}{f(2y)}$ , $\forall x,y \in A$
	\flushright \href{https://artofproblemsolving.com/community/c6h487443}{(Link to AoPS)}
\end{problem}



\begin{solution}[by \href{https://artofproblemsolving.com/community/user/29428}{pco}]
	How do you know that this equation is nice since you did not succeed solving it :?:

\begin{tcolorbox}Find all function $f: \{\ A|A\in Q , A\geq{1}\}\ \to Q$ such that : $f(xy^2)=f(4x)f(y)+\frac{f(8x)}{f(2y)}$ , $\forall x,y \in A$\end{tcolorbox}
There is some confusion about usage of letter $A$ which seems to be a rational number in left part of statement and a set in right part of statement.
I'll consider that the statement is :

Let $A=\mathbb Q\cap[1,+\infty)$
Find all functions $f$ from $A\to\mathbb Q$ such that $f(xy^2)=f(4x)f(y)+\frac{f(8x)}{f(2y)}$ $\forall x,y\in A$

Let $P(x,y)$ be the assertion $f(xy^2)=f(4x)f(y)+\frac{f(8x)}{f(2y)}$ $\forall$ rational numbers $x,y\ge 1$

$P(x,y)$ implies that $f(x)\ne 0$ $\forall$ rational number $x\ge 2$

$P(x,2)$ $\implies$ $f(8x)=f(4x)f(4)(1-f(2))$ and so $f(2x)=af(x)$ $\forall x\ge 4$ and for some rational number $a\ne 0$


Let $x,y\ge 4$ : 
$P(x,y)$ $\implies$ $f(xy^2)=a^2f(x)f(y)+a^2\frac{f(x)}{f(y)}$
$P(x,2y)$ $\implies$ $f(xy^2)=af(x)f(y)+\frac{f(x)}{af(y)}$
Subtracting and simplifying, we get $(a-1)(a^2f(y)^2+a^2+a+1)=0$ and so $a=1$ (the other part of the product is always positive)

So $f(2x)=f(x)$ $\forall x\ge 4$

Let $x\ge 4$ : $P(x,4)$ implies then $f(x)=f(x)\left(f(4)+\frac{1}{f(4)}\right)$, which is impossible 

So \begin{bolded}no solution\end{underlined}\end{bolded} for this equation.
\end{solution}
*******************************************************************************
-------------------------------------------------------------------------------

\begin{problem}[Posted by \href{https://artofproblemsolving.com/community/user/35129}{Zhero}]
	Determine all strictly increasing functions $f: \mathbb{N}\to\mathbb{N}$ satisfying $nf(f(n))=f(n)^2$ for all positive integers $n$.

\begin{italicized}Carl Lian and Brian Hamrick.\end{italicized}
	\flushright \href{https://artofproblemsolving.com/community/c6h487455}{(Link to AoPS)}
\end{problem}



\begin{solution}[by \href{https://artofproblemsolving.com/community/user/29428}{pco}]
	\begin{tcolorbox}Determine all strictly increasing functions $f: \mathbb{N}\to\mathbb{N}$ satisfying $nf(f(n))=f(n)^2$ for all positive integers $n$.\end{tcolorbox}
Let $P(n)$ be the assertion $nf(f(n))=f(n)^2$
Notation : in the following $f(n)^k$ is $f(n)$ raised to power $k$ while $f^{k}(n)$ is $f(f(f(... n ...))$, the $k^{th}$ composition.

It's easy to show with induction that $f^{k}(n)=\frac{f(n)^k}{n^{k-1}}$

Setting $k$ great enough, we get that $n|f(n)$ and so we can define a function $g(n)=\frac{f(n)}n$ from $\mathbb N\to\mathbb N$.

Since $f^{k}(n)=g(n)^kn$ and $f^{k}$ is increasing, we get $g(m)m^{\frac 1k}>g(n)n^{\frac 1k}$ $\forall m>n$ and $\forall k\in\mathbb N$

Setting $k\to+\infty$ in this inequality, we get that $g(n)$ is non decreasing.

But $P(n)$ implies $g(ng(n))=g(n)$ $\forall n$ and so $g(ng(n)^k)=g(n)$ and so :
If $g(n)=c>1$ for some $n$, then, since non decreasing, $g(m)=c$ $\forall m\ge n$ and so :
either $g(n)=1$ $\forall n$
either $g(n)=1$ $\forall n\in[1,a)$ and $g(n)=c$ $\forall n\ge a$ for some positive integers $a$ and $c>1$

\begin{bolded}Hence the two solutions\end{underlined}\end{bolded} :

First case gives the solution $f(n)=n$, which indeed is a solution.

Second case gives the solution $f(n)=n$ $\forall n< a$ and $f(n)=cn$ $\forall n\ge a$, for any positive integers $a$ and $c>1$, which indeed is a solution.
\end{solution}



\begin{solution}[by \href{https://artofproblemsolving.com/community/user/159507}{MSTang}]
	Sorry if I'm missing something ... why isn't $f(n) \equiv an$ ($a \in \mathbb{Z}^+$) a solution?
\end{solution}



\begin{solution}[by \href{https://artofproblemsolving.com/community/user/164606}{FlakeLCR}]
	That is counted when $a=1$ in case 2.
\end{solution}
*******************************************************************************
-------------------------------------------------------------------------------

\begin{problem}[Posted by \href{https://artofproblemsolving.com/community/user/35129}{Zhero}]
	Find all functions $f: \mathbb{R} \to \mathbb{R}$ such that $f(x+y) = \max(f(x),y) + \min(f(y),x)$.

\begin{italicized}George Xing.\end{italicized}
	\flushright \href{https://artofproblemsolving.com/community/c6h487457}{(Link to AoPS)}
\end{problem}



\begin{solution}[by \href{https://artofproblemsolving.com/community/user/29428}{pco}]
	\begin{tcolorbox}Find all functions $f: \mathbb{R} \to \mathbb{R}$ such that $f(x+y) = \max(f(x),y) + \min(f(y),x)$.\end{tcolorbox}
Let $P(x,y)$ be the assertion $f(x+y)=\max(f(x),y)+\min(f(y),x)$
Let $a=f(0)$

Let $x\le a$ and $x\ne 0$ : $P(x,0)$ $\implies$ $f(x)=\max(f(x),0)+x$ and so $f(x)<0$ and $f(x)=x$
So $a\le 0$ and $f(x)=x$ $\forall x\le a$ and $x\ne 0$

If $f(x)<0$ $\forall x>0$, then, choosing $x,y>0$, $P(x,y)$ implies $f(x+y)=y+f(y)=x+f(x)$ and so $f(x)=u-x$ and $f(x+y)=u=u-(x+y)$, impossible.
So $\exists b>0$ such that $f(b)\ge 0$ and then $P(b,0)$ $\implies$ $f(b)=f(b)+a$ and so $a=0$

Let $x\ge 0$ : $P(x,0)$ $\implies$ $f(x)=\max(f(x),0)$ and so $f(x)\ge 0$

So $f(0)=0$ and $f(x)=x$ $\forall x\le 0$ and $f(x)\ge 0$ $\forall x\ge 0$

Let then $x\ge 0$ : $P(x,-x)$ $\implies$ $0=f(x)-x$

And so $\boxed{f(x)=x}$ $\forall x$, which indeed is a solution.
\end{solution}



\begin{solution}[by \href{https://artofproblemsolving.com/community/user/121558}{Bigwood}]
	My solution is similar to pro's.
Let $a=f(0),x\neq 0$. We just consider these two lines;
$f(x)=max(a,x)+min(f(x),0)$---(1), $f(x)=min(f(x),0)+max(a,x)$---(2)
1. $f(x)\ge 0,x\ge a$ follows $f(x)=x,f(0)=0$
2. $f(x)\ge 0, x\le a$ follows $x=0$, contradiction.
3. $f(x)\le 0, x\ge a$ follows $x=0$ again a contradiction.
4. $f(x)\le 0, x\le a$ follows $f(x)=x,f(0)=0$.
Then We get $f(x)=x$ for all $x\neq 0$. Then $a=max(x,-x)+min(-x,x)=0$ follows, thus $\boxed{f(x)=x \forall x}$.
\end{solution}



\begin{solution}[by \href{https://artofproblemsolving.com/community/user/89198}{chaotic_iak}]
	Let $P(x,y)$ be the preposition $f(x+y) = \max(f(x),y) + \min(f(y),x)$. Recall that $\max(a,b) + \min(a,b) = a + b$ and $\max(a,b) - min(a,b) = |a-b|$.

$P(x,0) + P(0,x)$: $f(x) = x + f(0)$
$P(x,0) - P(0,x)$: $|f(x)| = |f(0) - x|$

Substituting the first into the second and squaring, we get $4xf(0) = 0$. Taking nonzero $x$ gives $f(0) = 0$ and hence $f(x) = x$ for all $x$.
\end{solution}



\begin{solution}[by \href{https://artofproblemsolving.com/community/user/159507}{MSTang}]
	Swapping $x$ and $y$, we have \[\max(f(x),y) + \min(f(y),x) = \max(f(y),x) + \min(f(x), y)\] so \[\max(f(x),y) - \min(f(x),y) = \max(f(y), x) - \min(f(y), x)\] that is, \[|f(x) - y| = |f(y) - x|\] for all $x, y$. Taking $y=f(x)$, we get $f(f(x)) = x$, so $f$ is bijective. Taking $y = 0$, we get $|f(x)| = |f(0) - x|$, so for each $x$, either $f(x) = f(0) - x$ or $f(x) = x - f(0)$.

Suppose that $f(a) = f(0) - a$ for some $a$. Taking $y=x=a$ in the original equation, we have $f(2a) = f(a) + a = f(0)$. But $f(2a) \in \{f(0) - 2a, 2a - f(0)\}$, so either $f(0) = f(0) - 2a \implies a=0$, or $f(0) = 2a - f(0) \implies a = f(0)$.

Thus, for all $a \not \in \{0, f(0)\}$, we have $f(a) = a - f(0)$. Specifically, if $f(0) \neq 0$, then taking $a=2f(0)$, we have $f(2f(0)) = f(0)$. Since $f$ is injective, we have $2f(0) = 0$, contradicting $f(0) \neq 0$. Thus $f(0) = 0$. It is now easy to conclude $f(x) = x$ for all $x$.
\end{solution}



\begin{solution}[by \href{https://artofproblemsolving.com/community/user/247657}{Ankoganit}]
	Let $P(x,y)$ be the given assertion. Note that for $a,b\in \mathbb R$, we have $\max(a,b)+\min(a,b)=a+b$. Now\begin{align*}P(x,0)+P(0,x)\implies &2f(x)&&=\left(\max(f(x),0)+\min(f(x),0)\right)+\left(\max(f(0)+x)+\min(f(0)+x)\right)\\
& &&=(f(x)+0)+(f(0)+x)\\
\implies & f(x) &&=f(0)+x.\end{align*}
Let's say $f(0)=c$, then $f(x)=x+c\quad \forall x\in\mathbb R$. Now $P(2c,0)$ says $2c+0+c=\max(3c,0)+\min(c,2c)$. If $c>0$, then this becomes $3c=3c+c$, which is bad; if $c<0$, then this gives $3c=0+2c$, which is not good either. So $c=0$ and we have $\boxed{f(x)=x\; \forall x\in\mathbb R}$; which is easily seen to work. $\blacksquare$
\end{solution}



\begin{solution}[by \href{https://artofproblemsolving.com/community/user/67223}{Amir Hossein}]
	This is an old problem posted on AoPS on 2011: https:\/\/artofproblemsolving.com\/community\/c6h406771
\end{solution}
*******************************************************************************
-------------------------------------------------------------------------------

\begin{problem}[Posted by \href{https://artofproblemsolving.com/community/user/13967}{Karth}]
	Find all functions $f:\mathbb{R}\rightarrow\mathbb{R}$ such that
\[
f(x^2+f(y))=y+\left(f(x)\right)^2, \forall x,y\in\mathbb{R} (*)
\]

I'm trying to get better at proof-writing. Could you rate my solution from 1-7 (as an IMO grader would) and provide feedback on how I could improve my writing style? Thanks! :)

[hide="Solution"]The only feasible function is $f(x) = x$. The proof follows below.

We first show that $f(0) = 0$. Let $C = f(0)$. Substituting $x = 0$, we have that
\[
f(f(y)) = y + C^2 (**)
\]
More generally, note that
\[
f^{2k}(C)= f^{2k-2}(C^2 + C) = \cdots = f^2((2k-4)C^2 + C) = (2k-3)C^2 + C
\]
However, we also have that $f(C) = f(f(0)) = C^2, f^2(C) = C + C^2, f^3(C) = 2C^2, \cdots, f^{2k}(C) = kC^2 + C$. We thus have that $\forall k, (2k-3)C^2 + C = kC^2 + C \Rightarrow C = 0 \Rightarrow f(0) = 0$.

Next, we show that $f:\mathbb{R}\rightarrow\mathbb{R}$ is a bijection. Equation $(**)$ becomes $f(f(y)) = y$. Let $f(a) = b$ for some $a, b$. We must have that $f(f(a)) = a \Rightarrow f(b) = a$. As a result, $f$ must be an injection, i.e. $f(x_1) = f(x_2) \Rightarrow x_1 = x_2$; furthermore, note that $f$ is a bijection, since $f^{-1}(a) = f(a)$.

We next show that $f$ is monotonically increasing. Consider $a, b \in \mathbb{R}$, where without loss of generality, $a > b$. There must exist some $t$ such that $a = b + t^2$ for $t\in \mathbb{R}$. We have that $f(a) = f(t^2 + b) = f^{-1}(b) + f(t)^2 = f(b) + f(t)^2 > f(b)$, since $t \neq 0, f(t) > 0$.

Finally, we prove that $f(x) = x$ is the only solution to the functional equation. Consider some $a\in \mathbb{R}$, and let $f(a) = b$. For the sake of contradiction, suppose that $a \neq b$. Without loss of generality, let $a < b$. Since $f(b) = a$, we have that $f(a) > f(b)$, a contradiction, since we proved earlier that $a < b \Rightarrow f(a) < f(b)$. Therefore, we must have that $a = b$.

Thus, $f(x) = x$ is the only function that satisfies the provided functional, as desired. $\Box$[\/hide]

EDIT: Thanks to pco for pointing out an error in my proof. Modified accordingly.
EDIT 2: The modified proof doesn't work either, I just realized... an injection doesn't have to be monotone (e.g. if the function isn't continuous). I'll repost once I've figured out how to patch together this solution.
EDIT 3: Updated! This should be right now.
	\flushright \href{https://artofproblemsolving.com/community/c6h487789}{(Link to AoPS)}
\end{problem}



\begin{solution}[by \href{https://artofproblemsolving.com/community/user/29428}{pco}]
	\begin{tcolorbox}We have that $a < b \Rightarrow f(a) > f(b)$, implying that $f$ is monotonically decreasing. \end{tcolorbox}
This is wrong.
You proved that $a<b\implies f(a) > f(b)$ only in the case where $a,b$ are such that $f(a)=b$ and $f(b)=a$, not in general case. So you cant conclude that $f(x)$ is decreasing.
\end{solution}



\begin{solution}[by \href{https://artofproblemsolving.com/community/user/13967}{Karth}]
	\begin{tcolorbox}[quote="Karth"]We have that $a < b \Rightarrow f(a) > f(b)$, implying that $f$ is monotonically decreasing. \end{tcolorbox}
This is wrong.
You proved that $a<b\implies f(a) > f(b)$ only in the case where $a,b$ are such that $f(a)=b$ and $f(b)=a$, not in general case. So you cant conclude that $f(x)$ is decreasing.\end{tcolorbox}

In case 1, we have that $f(f(a)) = a$ is true. Suppose that $f(a) = b$. This means that $f(b) = a$ must be true (substitute $f(a) = b$ in $f(f(a)) = a$). In other words, $f(a) = b$ and $f(b) = a$ are given.
\end{solution}



\begin{solution}[by \href{https://artofproblemsolving.com/community/user/29428}{pco}]
	\begin{tcolorbox}In case 1, we have that $f(f(a)) = a$ is true. Suppose that $f(a) = b$. This means that $f(b) = a$ must be true (substitute $f(a) = b$ in $f(f(a)) = a$). In other words, $f(a) = b$ and $f(b) = a$ are given.\end{tcolorbox}
Yes. If $f(a)=b$, then $f(b)=a$

So you proved that when $a=f(b)$, then $a<b$ $\implies$ $f(a)>f(b)$
But you did not prove that $a<b$ $\implies$ $f(a)>f(b)$ in other cases.
So you cant conclude that $f$ is decreasing just from this.
\end{solution}



\begin{solution}[by \href{https://artofproblemsolving.com/community/user/29428}{pco}]
	\begin{tcolorbox}Find all functions $f:\mathbb{R}\rightarrow\mathbb{R}$ such that $f(x^2+f(y))=y+\left(f(x)\right)^2, \forall x,y\in\mathbb{R}$\end{tcolorbox}
No need to compute $f(0)$ first. Here is a simple solution (in addition to those already in the thread pointed by AnhIsGod) :
Let $P(x,y)$ be the assertion $f(x^2+f(y))=y+f(x)^2$. $f(x)$ is bijective.
Let $a=f(0)$ and $u$ such that $f(u)=0$

$P(0,y-a^2)$  $\implies$ $f(f(y-a^2))=y$
$P(0,f(y-a^2))$ $\implies$ $f(y)=f(y-a^2)+a^2$
$P(x,u)$ $\implies$ $f(x^2)=u+f(x)^2$ (note that this implies that $f(x)$ is lower bounded on $\mathbb R^+$)

$P(x,f(y-a^2))$ $\implies$ $f(x^2+y)=f(y-a^2)+f(x)^2=f(y)+f(x^2)-u-a^2$

Writing $g(x)=f(x)-u-a^2$, we get $g(x+y)=g(x)+g(y)$ $\forall x\ge 0,\forall y$
So $g(0)=0$ and $g(-x)=-g(x)$ and so $g(x+y)=g(x)+g(y)$ $\forall x,y$.

But we know that $g(x)$ is lower bounded on $\mathbb R^+$ and so $g(x)=cx$ and $f(x)=cx+d$ for some real $c,d$

Pluging this back in original equation, we get $c=1$ and $d=0$ and so the unique solution $\boxed{f(x)=x}$ $\forall x$
\end{solution}



\begin{solution}[by \href{https://artofproblemsolving.com/community/user/13967}{Karth}]
	Yep, I know that there's an IMO thread per problem with solutions. I just wanted to see if I could come up with my own solution and write it up well. I've updated my solution. Could you do one last pass over it? I think I've got it now, and it's a lot cleaner.

Thanks again for your help!
\end{solution}



\begin{solution}[by \href{https://artofproblemsolving.com/community/user/29428}{pco}]
	\begin{tcolorbox} More generally, note that $f^{2k}(C)= f^{2k-2}(C^2 + C) = \cdots = f^2((2k-4)C^2 + C) = (2k-3)C^2 + C$\end{tcolorbox}
I dont understand how you get this.
For me, the "$\cdots$" directly gives $f^{2k}(C)=kC^2+C$ (your second result)
\end{solution}



\begin{solution}[by \href{https://artofproblemsolving.com/community/user/320348}{K.titu}]
	\begin{tcolorbox}[quote="Karth"]Find all functions $f:\mathbb{R}\rightarrow\mathbb{R}$ such that $f(x^2+f(y))=y+\left(f(x)\right)^2, \forall x,y\in\mathbb{R}$\end{tcolorbox}
No need to compute $f(0)$ first. Here is a simple solution (in addition to those already in the thread pointed by AnhIsGod) :
Let $P(x,y)$ be the assertion $f(x^2+f(y))=y+f(x)^2$. $f(x)$ is bijective.
Let $a=f(0)$ and $u$ such that $f(u)=0$

$P(0,y-a^2)$  $\implies$ $f(f(y-a^2))=y$
$P(0,f(y-a^2))$ $\implies$ $f(y)=f(y-a^2)+a^2$
$P(x,u)$ $\implies$ $f(x^2)=u+f(x)^2$ (note that this implies that $f(x)$ is lower bounded on $\mathbb R^+$)

$P(x,f(y-a^2))$ $\implies$ $f(x^2+y)=f(y-a^2)+f(x)^2=f(y)+f(x^2)-u-a^2$

Writing $g(x)=f(x)-u-a^2$, we get $g(x+y)=g(x)+g(y)$ $\forall x\ge 0,\forall y$
So $g(0)=0$ and $g(-x)=-g(x)$ and so $g(x+y)=g(x)+g(y)$ $\forall x,y$.

But we know that $g(x)$ is lower bounded on $\mathbb R^+$ and so $g(x)=cx$ and $f(x)=cx+d$ for some real $c,d$

Pluging this back in original equation, we get $c=1$ and $d=0$ and so the unique solution $\boxed{f(x)=x}$ $\forall x$\end{tcolorbox}

Hello pco 
Could you please explain that how from $f(x^2)=u+f(x)^2$ we get that $f$ is lower bounded?
Thanks in advance 
\end{solution}



\begin{solution}[by \href{https://artofproblemsolving.com/community/user/29428}{pco}]
	\begin{tcolorbox}Hello pco 
Could you please explain that how from $f(x^2)=u+f(x)^2$ we get that $f$ is lower bounded?
Thanks in advance\end{tcolorbox}
$f(x)\ge u$ $\forall x\in\mathbb R^+$ and so $f(x)$ is lowerbounded over $\mathbb R^+$, as I wrote.



\end{solution}



\begin{solution}[by \href{https://artofproblemsolving.com/community/user/320348}{K.titu}]
	Tnx  :cool: 
\end{solution}



\begin{solution}[by \href{https://artofproblemsolving.com/community/user/335975}{Taha1381}]
	\begin{tcolorbox}[quote=K.titu]Hello pco 
Could you please explain that how from $f(x^2)=u+f(x)^2$ we get that $f$ is lower bounded?
Thanks in advance\end{tcolorbox}
$f(x)\ge u$ $\forall x\in\mathbb R^+$ and so $f(x)$ is lowerbounded over $\mathbb R^+$, as I wrote.\end{tcolorbox}

How do you prove if $g$ is lower bounded on positive reals and additive then $g$ is linear?I read sth similar before but that prove the lemma when the lower bound is $0$.
\end{solution}



\begin{solution}[by \href{https://artofproblemsolving.com/community/user/29428}{pco}]
	\begin{tcolorbox}How do you prove if $g$ is lower bounded on positive reals and additive then $g$ is linear?I read sth similar before but that prove the lemma when the lower bound is $0$.\end{tcolorbox}
It is well known that the graph of a non linear additive function is dense in $\mathbb R^2$
And so if an additive function is either lowerbounded, either upper bounded on a non empty open interval, its graph can no longer be dense in $\mathbb R^2$ and si it is a continuous linear function.



\end{solution}



\begin{solution}[by \href{https://artofproblemsolving.com/community/user/335975}{Taha1381}]
	\begin{tcolorbox}[quote=Taha1381]How do you prove if $g$ is lower bounded on positive reals and additive then $g$ is linear?I read sth similar before but that prove the lemma when the lower bound is $0$.\end{tcolorbox}
It is well known that the graph of a non linear additive function is dense in $\mathbb R^2$
And so if an additive function is either lowerbounded, either upper bounded on a non empty open interval, its graph can no longer be dense in $\mathbb R^2$ and si it is a continuous linear function.\end{tcolorbox}

Could you give a written proof as I know if $|f(x)|<M$ for any $x \in [a,b]$ then $f$ is linear but then we have both upper and lower bound.
\end{solution}



\begin{solution}[by \href{https://artofproblemsolving.com/community/user/29428}{pco}]
	As I wrote, the graph of a non linear additive function is dense in $\mathbb R^2$. This is the major property from which all other properties are derived.

\end{solution}



\begin{solution}[by \href{https://artofproblemsolving.com/community/user/29428}{pco}]
	Normally, the simplest thing now would be :
- either to ask your teacher (this is just a course question when studying Cauchy functional equation, in my opinion)
- either use the search function (it has been proved many times in this forum)

Here is a reminder of the simplest proof I know (note that I already posted it in this forum ....) :

Let $f(x)$ an additive nonlinear function from $\mathbb R\to\mathbb R$
Let $c=f(1)$ and $a\ne 0$ such that $f(a)=ba$ with $b\ne c$ (which exists since non linear)

Let any point $M(u,v)$ in the plane

Let $p_n$ any sequence of rational numbers whose limit is $\frac{bu-v}{b-c}$

Let $q_n$ any sequence of rational numbers whose limit is $\frac{v-cu}{(b-c)a}$

Let $x_n=p_n+aq_n$ : this sequence of real numbers has limit $u$
And $f(x_n)=cp_n+baq_n$ is a sequence of real numbers with limit $v$

And so the sequence of points $(x_n,f(x_n))$ has limit $M(u,v)$

And so we always can find a point of the graph as near as we want of any point $M(u,v)$ of the plane.
Q.E.D.

\end{solution}



\begin{solution}[by \href{https://artofproblemsolving.com/community/user/335975}{Taha1381}]
	\begin{tcolorbox}Normally, the simplest thing now would be :
- either to ask your teacher (this is just a course question when studying Cauchy functional equation, in my opinion)
- either use the search function (it has been proved many times in this forum)

Here is a reminder of the simplest proof I know (note that I already posted it in this forum ....) :

Let $f(x)$ an additive nonlinear function from $\mathbb R\to\mathbb R$
Let $c=f(1)$ and $a\ne 0$ such that $f(a)=ba$ with $b\ne c$ (which exists since non linear)

Let any point $M(u,v)$ in the plane

Let $p_n$ any sequence of rational numbers whose limit is $\frac{bu-v}{b-c}$

Let $q_n$ any sequence of rational numbers whose limit is $\frac{v-cu}{(b-c)a}$

Let $x_n=p_n+aq_n$ : this sequence of real numbers has limit $u$
And $f(x_n)=cp_n+baq_n$ is a sequence of real numbers with limit $v$

And so the sequence of points $(x_n,f(x_n))$ has limit $M(u,v)$

And so we always can find a point of the graph as near as we want of any point $M(u,v)$ of the plane.
Q.E.D.\end{tcolorbox}

Thank you so much just a little question why the interval has to be open in the theorem?
\end{solution}



\begin{solution}[by \href{https://artofproblemsolving.com/community/user/29428}{pco}]
	\begin{tcolorbox}Thank you so much just a little question why the interval has to be open in the theorem?\end{tcolorbox}
There is no "open interval" in the property "The graph of non linear additiv functions is dense in $\mathbb R^2$"

And the mention "non empty open interval" in the subproperty "either upperbounded, either lowerbounded" is just to avoid some people uses the property on interval $[a,a]$ for example (since all functions are bounded on such intervals).



\end{solution}
*******************************************************************************
-------------------------------------------------------------------------------

\begin{problem}[Posted by \href{https://artofproblemsolving.com/community/user/5820}{N.T.TUAN}]
	Find all $f:\mathbb{R}\to\mathbb{R}$ such that
\[
f(x)f(x+y)=f^2(y)f^2(x-y)e^{y+4}\,\,\,\forall x,y\in\mathbb{R}.
\]
	\flushright \href{https://artofproblemsolving.com/community/c6h488026}{(Link to AoPS)}
\end{problem}



\begin{solution}[by \href{https://artofproblemsolving.com/community/user/29428}{pco}]
	\begin{tcolorbox}Find all $f:\mathbb{R}\to\mathbb{R}$ such that
\[
f(x)f(x+y)=f^2(y)f^2(x-y)e^{y+4}\,\,\,\forall x,y\in\mathbb{R}.
\]\end{tcolorbox}
I suppose that notation $f^2(y)$ means $(f(y))^2$ and not $f(f(y))$ (generally convention on this forum is the contrary)
$f(x)=0$ $\forall x$ is a solution. So let us from now look only for non allzero solutions.
$f(x)$ solution implies $-f(x)$ solution. So Wlog consider $f(0)\ge 0$

Let $P(x,y)$ be the assertion $f(x)f(x+y)=f(y)^2f(x-y)^2e^{y+4}$
Let $A=f^{[-1]}(\{0\})$
Let $B=\mathbb R\setminus A$

1) $B$ is an additive subgroup of $\mathbb R$
========================
Since we consider that $f(x)$ is not the allzero function, then $\exists u\in B$
$P(u,0)$ $\implies$ $f(0)=e^{-2}$ and so $0\in B$
If $x\in B$, then $P(x,-x)$ $\implies$ $f(x)=f(-x)^2f(2x)^2e^{-x+6}$ and so $-x\in B$
If $x,y\in B$, then $-x,-y\in B$ and $P(-x,x+y)$ $\implies$ $f(-x)f(y)=f(x+y)^2f(-2x-y)^2e^{x+y+4}$ and so $x+y\in B$
Q.E.D.

2) $f(x)=e^{x-2}$ $\forall x\in B$
==================
Let $x\in B$ so that $-x\in B$ too.
$P(0,x)$ $\implies$ $1=f(x)f(-x)^2e^{x+6}$ $\implies$ $f(-x)^2=(f(x))^{-1}e^{-x-6}$
$P(0,-x)$ $\implies$ $1=f(-x)f(x)^2e^{-x+6}$ $\implies$ $f(-x)^2=(f(x))^{-4}e^{2x-12}$
Subtracting, we get $f(x)^{3}=e^{3x-6}$ and so $f(x)=e^{x-2}$
Q.E.D.

3) The function $f(x)=e^{x-2}$ $\forall x\in$additive subgroup of $\mathbb R$ and $f(x)=0$ elsewhere is indeed a solution
===========================================================================
Let $B$ any additive subgroup of $\mathbb R$
Let $f(x)$ such that $f(x)=e^{x-2}$ $\forall x\in B$ and $f(x)=0$ $\forall x\notin B$

If $x,y\in B$, then $x+y,x-y\in B$ (since additive subgroup) and then :
$f(x)f(x+y)=e^{x-2}e^{x+y-2}$ $=e^{2x+y-4}$
$f(y)^2f(x-y)^2e^{y+4}=e^{2y-4}e^{2x-2y-4}e^{y+4}$ $=e^{2x+y-4}$
And so $f(x)f(x+y)=f(y)^2f(x-y)^2e^{y+4}$

If $x,y\notin B$, then $f(x)=f(y)=0$ and :
$f(x)f(x+y)=0$
$f(y)^2f(x-y)^2e^{y+4}=0$
And so $f(x)f(x+y)=f(y)^2f(x-y)^2e^{y+4}$

If $x\in B$ and $y\notin B$, then $x+y\notin B$ (else $x\in B$ and $x+y\in B$ would imply $y=(x+y)-x\in B$) and so :
$f(x)f(x+y)=0$
$f(y)^2f(x-y)^2e^{y+4}=0$
And so $f(x)f(x+y)=f(y)^2f(x-y)^2e^{y+4}$

If $x\notin B$ and $y\in B$, then $x-y\notin B$ (else $y\in B$ and $x-y\in B$ would imply $x=(x-y)+y\in B$) and so :
$f(x)f(x+y)=0$
$f(y)^2f(x-y)^2e^{y+4}=0$
And so $f(x)f(x+y)=f(y)^2f(x-y)^2e^{y+4}$

Q.E.D

4) Synthesis of solutions :
==================
We got three families of solutions :

4.1 : $f(x)=0$ $\forall x$

4.2 : Let $G$ any additive subgroup of $\mathbb R$ : $f(x)=e^{x-2}$ $\forall x\in G$ and $f(x)=0$ $\forall x\notin G$

4.3 : Let $G$ any additive subgroup of $\mathbb R$ : $f(x)=-e^{x-2}$ $\forall x\in G$ and $f(x)=0$ $\forall x\notin G$
\end{solution}
*******************************************************************************
-------------------------------------------------------------------------------

\begin{problem}[Posted by \href{https://artofproblemsolving.com/community/user/139296}{ablyrise}]
	Let $\alpha$ be a real number,find all funtions $f:\mathbb{R}\to\mathbb{R}$  such as 
$f\left(x^2+f\left(y\right)\right)=\alpha y+\left(f\left(x\right)\right)^2,\forall x,y\in\mathbb{R}$
	\flushright \href{https://artofproblemsolving.com/community/c6h488035}{(Link to AoPS)}
\end{problem}



\begin{solution}[by \href{https://artofproblemsolving.com/community/user/29428}{pco}]
	\begin{tcolorbox}Let $\alpha$ be a real number,find $f:\mathbb{R}\to\mathbb{R}$  such as 
$f\left(x^2+f\left(y\right)\right)=\alpha y+f^2\left(x\right),\forall x,y\in\mathbb{R}$\end{tcolorbox}
Is $f^2(x)=(f(x))^2$ or $f(f(x))$ ?
Is problem you got from your teacher or in your contest asking for just one $f(x)$ or all $f(x)$ ?
\end{solution}



\begin{solution}[by \href{https://artofproblemsolving.com/community/user/29428}{pco}]
	\begin{tcolorbox}Let $\alpha$ be a real number,find $f:\mathbb{R}\to\mathbb{R}$  such as 
$f\left(x^2+f\left(y\right)\right)=\alpha y+f^2\left(x\right),\forall x,y\in\mathbb{R}$\end{tcolorbox}
Assuming first easy case : $\alpha\ne 0$ and $f^2(x)=f(x)^2$
Let $P(x,y)$ be the assertion $f(x^2+f(y))=\alpha y+f(x)^2$
Let $f(0)=a$
$f(x)$ is bijective and so let $u$ such that $f(u)=0$

$P(0,y)$ $\implies$ $f(f(y))=\alpha y+a^2$
$P(0,f(y))$ $\implies$ $f(\alpha y+a^2)=\alpha f(y)+a^2$
$P(x,u)$ $\implies$ $f(x^2)=\alpha u+f(x)^2$. Note that this implies that $f(x)$ is lower bounded over $\mathbb R_{\ge 0}$

$P(x,f(y))$ $\implies$ $f(x^2+\alpha y+a^2)=\alpha f(y)+f(x)^2$ $=f(\alpha y+a^2)-a^2+f(x^2)-\alpha u$
And so $f(x+y)=f(x)+f(y)-a^2-\alpha u$ $\forall x\ge 0,\forall y$

Writing $g(x)=f(x)-a^2-\alpha u$, we get $g(x+y)=g(x)+g(y)$ $\forall x\ge 0,\forall y$
So $g(0)=0$ and $g(-x)=-g(x)$ and so $g(x+y)=g(x)+g(y)$ $\forall x,y$
And since $g(x)$ is lower bounded over $\mathbb R_{\ge 0}$, this implies $g(x)=cx$ and $f(x)=cx+d$ for some real $c,d$

Pluging this in original equation, we get :
\begin{bolded}1)\end{bolded} no solution if $\alpha\ne 1$
\begin{bolded}2)\end{bolded} Unique solution $f(x)=x$ if $\alpha=1$

The case $\alpha=0$ is the real difficulty of this problem and there are infinitely many solutions, some being quite weird.
For example : $f(x)=(-1)^{\left\lfloor(x+1)\sin e^{-x-1}\right\rfloor}$ 
I hope there is no mistake in your problem and that the case $\alpha=0$ was really present in the exercise you got in your olympiad contest or training session.
I'll look again a bit for the solution in this case.
\end{solution}



\begin{solution}[by \href{https://artofproblemsolving.com/community/user/139296}{ablyrise}]
	I edited.Your solution is good.Can you solve a other funtion equation which is like as above equation?
Let $\alpha$ be a real number,find all funtions such as
$f\left(x^2+f\left(y\right)\right)=y+\alpha\left(f\left(x\right)\right)^2,\forall x,y\in\mathbb{R}$
\end{solution}



\begin{solution}[by \href{https://artofproblemsolving.com/community/user/29428}{pco}]
	\begin{tcolorbox}I edited.Your solution is good.Can you solve a other funtion equation which is like as above equation?
Let $\alpha$ be a real number,find all funtions such as
$f\left(x^2+f\left(y\right)\right)=y+\alpha\left(f\left(x\right)\right)^2,\forall x,y\in\mathbb{R}$\end{tcolorbox}
Have you simply tried to solve this equation after having read my solution to your first problem ? 
\begin{bolded}This is exactly the same solution\end{underlined}\end{bolded}.
You should read the solutions users give you and learn from them.
And, btw, dont hesitate to post the solution your teacher will give you for previous problem when $\alpha=0$. I'm interested in.

 
Let $P(x,y)$ be the assertion $f(x^2+f(y))=y+\alpha f(x)^2$
Let $a=f(0)$
$f(x)$ is bijective and so let $u$ such that $f(u)=0$

$P(0,y)$ $\implies$ $f(f(y))=y+\alpha a^2$
$P(0,f(y))$ $\implies$ $f(y+\alpha a^2)=f(y)+\alpha a^2$
$P(x,u)$ $\implies$ $f(x^2)=u+\alpha f(x)^2$. Note that this implies that $f(x)$ is either lower bounded, either upper bounded, either both, over $\mathbb R_{\ge 0}$ (depending on sign of $\alpha$)

$P(x,f(y))$ $\implies$ $f(x^2+y+\alpha a^2)=f(y)+\alpha f(x)^2$ $=f(y+\alpha a^2)-\alpha a^2+f(x^2)-u$
And so $f(x+y)=f(x)+f(y)-\alpha a^2-u$ $\forall x\ge 0,\forall y$

Writing $g(x)=f(x)-\alpha a^2-u$, we get $g(x+y)=g(x)+g(y)$ $\forall x\ge 0$, $\forall y$
So $g(0)=0$ and $g(-x)=-g(x)$ and so $g(x+y)=g(x)+g(y)$ $\forall x,y$
And since $g(x)$ is either lower bounded, either upper bounded, either both, over $\mathbb R_{\ge 0}$, this implies $g(x)=cx$ and $f(x)=cx+d$ for some real $c,d$

Pluging this in original equation, we get :
\begin{bolded}1)\end{bolded} If $\alpha =1$, a unique solution $f(x)=x$ $\forall x$
\begin{bolded}2)\end{bolded} If $\alpha=-1$, a unique solution $f(x)=-x$ $\forall x$
\begin{bolded}3)\end{bolded} If $\alpha\notin\{-1,+1\}$, no solution.
\end{solution}
*******************************************************************************
-------------------------------------------------------------------------------

\begin{problem}[Posted by \href{https://artofproblemsolving.com/community/user/109704}{dien9c}]
	Find all function $f: \mathbb{Q} \to \mathbb{Q}$ such that
\[f(x+f(y))=f(x)f(y)\]
	\flushright \href{https://artofproblemsolving.com/community/c6h488077}{(Link to AoPS)}
\end{problem}



\begin{solution}[by \href{https://artofproblemsolving.com/community/user/140796}{mathbuzz}]
	let p(x,y) be the assertion of f(x+f(y))=f(x).f(y).if there exists a rational number r such that $f(r)=0$ , then we can get that
P(x,r) implies that f(x)=f(x).0=0 for all x in Q. so , in this case , we get [color=#0000FF]f(x) =0[\/color] which is indeed a solution.the same solution is the only solution when f(x) is onto. 

if there is no r in Q such that f(r)=0 , then , f(f(y))=f(0)f(y). now , chose a number s in Q , such that there exists h in Q such that f(h)=s. , then f(s)=cs where c=f(0).but i cant complete it now , please help. :(
\end{solution}



\begin{solution}[by \href{https://artofproblemsolving.com/community/user/29428}{pco}]
	\begin{tcolorbox}Find all function $f: \mathbb{Q} \to \mathbb{Q}$ such that
\[f(x+f(y))=f(x)f(y)\]\end{tcolorbox}
$f(x)=0$ $\forall x$ is a solution. So let us from now look only for non allzero solutions.
Let $P(x,y)$ be the assertion $f(x+f(y))=f(x)f(y)$
Let $u$ such that $f(u)\ne 0$

If $f(x)=0$ for some $x$, then $P(u,x)$ $\implies$ $f(u)=0$, impossible. So $f(x)\ne 0$ $\forall x$

$P(-f(0),0)$ $\implies$ $f(-f(0))=1$
$P(x,-f(0))$ $\implies$ $f(x+1)=f(x)$ and so $f(x+n)=f(x)$ $\forall n\in\mathbb Z$

Simple induction from $P(x,y)$ gives $f(x+nf(y))=f(x)f(y)^n$ $\forall x,y\in\mathbb Q$, $\forall n\in\mathbb N$
Let $f(y)=\frac pq$ with $q>0$. Using the above formula with $n=q$, we get $f(x+p)=f(x)\left(\frac pq\right)^q$
And since $f(x+p)=f(x)$, we get $f(y)=\frac pq=\pm 1$

But $f(y)=\pm 1$ $\forall y$ implies that $LHS$ in $P(x,y)$ is $f(x)$ and then $f(y)=1$ $\forall y$, which indeed is a solution.

\begin{bolded}Hence the two solutions\end{underlined}\end{bolded} :
$f(x)=0$ $\forall x$
$f(x)=1$ $\forall x$
\end{solution}



\begin{solution}[by \href{https://artofproblemsolving.com/community/user/44887}{Mathias_DK}]
	\begin{tcolorbox}Find all function $f: \mathbb{Q} \to \mathbb{Q}$ such that
\[f(x+f(y))=f(x)f(y)\]\end{tcolorbox}
If $f(a) = 0$ for some $a$ then $f(x)=f(x+f(a))=f(x)f(a)=0$ for all $x$. So either $\forall x: f(x) = 0$ or $\forall x: f(x) \neq 0$ - the former clearly being a solution. So assume $f(x) \neq 0$ for all $x$.

First prove the lemma: $f(x+nf(y)) = f(x)(f(y))^n$ for all $n \in \mathbb{Z}$. Then let $z,w$ be any two rational numbers and let $n \in \mathbb{N}, n \neq 0$ be such that $nf(z), nf(w) \in \mathbb{Z}$. Hence:
\[ f(x)(f(z))^{nf(w)} = f(x+nf(z)f(w)) = f(x)(f(w))^{nf(z)} \]
From here we conclude that (remember to consider the case where $f(z)$ or $f(w)$ are negative):
\[ \forall z,w: \frac{\log |f(z)|}{f(z)} = \frac{\log |f(w)|}{f(w)} \]
Since the image of $f$ is a subgroup of $(\mathbb{Q}-\{0\}, \cdot)$ and the condition above implies that it is finite we conclude that the image of $f$ is either $\{1\}$ or $\{-1,1\}$. $f(x) = 1$ for all $x$ is clearly a solution so consider the last case. Let $a$ be such that $f(a) = 1$, then $f(x+f(a)) = f(x)f(a)$ or $f(x+1) = f(x)$. Let $b$ be such that $f(b) = -1$. Then $f(x+1+f(b)) = f(x+1)f(b)$ or $f(x) = -f(x+1)$. But then $f(x) = -f(x)$, or $f(x) = 0$ which we assumed wasn't the case, so the image of $f$ cannot be $\{-1,1\}$. Hence there are two solutions:
$f(x) = 0$ for all $x$
$f(x) = 1$ for all $x$
\end{solution}
*******************************************************************************
-------------------------------------------------------------------------------

\begin{problem}[Posted by \href{https://artofproblemsolving.com/community/user/109704}{dien9c}]
	Find all $f: \mathbb{N} \to \mathbb{N}$ such that
a. $f$ is stricly increasing
b.  $f(mn)=f(m)f(n)$
	\flushright \href{https://artofproblemsolving.com/community/c6h488133}{(Link to AoPS)}
\end{problem}



\begin{solution}[by \href{https://artofproblemsolving.com/community/user/123332}{Mathlover20}]
	\begin{tcolorbox}Find all $f: \mathbb{N} \to \mathbb{N}$ such that
a. $f$ is stricly increasing
b.  $f(mn)=f(m)f(n)$\end{tcolorbox}

This functioal equation is \begin{bolded}so-well-known\end{bolded} and known as \begin{bolded}Cauchy's Functional Equation\end{bolded}.
As it is strictly increasing, one immediate solution is...

$f(m)=m^a$, where $a$ is greater than $0$.
\end{solution}



\begin{solution}[by \href{https://artofproblemsolving.com/community/user/140272}{caoquyetthang}]
	\begin{tcolorbox}[quote="dien9c"]Find all $f: \mathbb{N} \to \mathbb{N}$ such that
a. $f$ is stricly increasing
b.  $f(mn)=f(m)f(n)$\end{tcolorbox}

This functioal equation is \begin{bolded}so-well-known\end{bolded} and known as \begin{bolded}Cauchy's Functional Equation\end{bolded}.
As it is strictly increasing, one immediate solution is...

$f(m)=m^a$, where $a$ is greater than $0$.\end{tcolorbox}
sorry but if $f(2)=2$ $\Longrightarrow f(x)=x$ and if $f(2)=3$ no funtion sastisfies
\end{solution}



\begin{solution}[by \href{https://artofproblemsolving.com/community/user/123332}{Mathlover20}]
	\begin{tcolorbox}[quote="Mathlover20"][quote="dien9c"]Find all $f: \mathbb{N} \to \mathbb{N}$ such that
a. $f$ is stricly increasing
b.  $f(mn)=f(m)f(n)$\end{tcolorbox}

This functioal equation is \begin{bolded}so-well-known\end{bolded} and known as \begin{bolded}Cauchy's Functional Equation\end{bolded}.
As it is strictly increasing, one immediate solution is...

$f(m)=m^a$, where $a$ is greater than $0$.\end{tcolorbox}
sorry but if $f(2)=2$ $\Longrightarrow f(x)=x$ and if $f(2)=3$ no funtion sastisfies\end{tcolorbox}

I think not that.
Obviously, $f(x)=x$ is a solution to the functional equation.

When, $f(x)=3$, as you said, we have $a=log_2 3$. And this also satisfies the equation.

I should have mentioned a thing that $a\ge 1$ because we are told to find increasing function.
\end{solution}



\begin{solution}[by \href{https://artofproblemsolving.com/community/user/29428}{pco}]
	\begin{tcolorbox}Find all $f: \mathbb{N} \to \mathbb{N}$ such that
a. $f$ is stricly increasing
b.  $f(mn)=f(m)f(n)$\end{tcolorbox}
We have $f(1)=1$ and $f(x^y)=f(x)^y$
Note that $x>1$ implies $f(x)>1$

Let then $a,b\in\mathbb N\setminus\{1\}$
Let $p,q\in\mathbb N$ such that $\frac{\ln a}{\ln b}>\frac pq$
This implies $a^q>b^p$ and so $f(a^q)>f(b^p)$ and so $f(a)^q>f(b)^p$ and so $\frac{\ln f(a)}{\ln f(b)}>\frac pq$

And since this is true whatever are $p,q$ such that $\frac{\ln a}{\ln b}>\frac pq$, we get $\frac{\ln f(a)}{\ln f(b)}\ge \frac{\ln a}{\ln b}$

Swapping $a,b$, ge get  $\frac{\ln f(a)}{\ln f(b)}=\frac{\ln a}{\ln b}$ and so $\boxed{f(x)=x^c}$, which indeed is a solution whenever $c\in\mathbb N$
\end{solution}



\begin{solution}[by \href{https://artofproblemsolving.com/community/user/123332}{Mathlover20}]
	\begin{tcolorbox}
Swapping $a,b$, ge get  $\frac{\ln f(a)}{\ln f(b)}=\frac{\ln a}{\ln b}$ and so $\boxed{f(x)=x^c}$, which indeed is a solution whenever $c\in\mathbb N$\end{tcolorbox}

Excuse me \begin{bolded}pco\end{bolded}, is it necessary that $ c\in\mathbb{N} $?
\end{solution}



\begin{solution}[by \href{https://artofproblemsolving.com/community/user/29428}{pco}]
	\begin{tcolorbox}Excuse me \begin{bolded}pco\end{bolded}, is it necessary that $ c\in\mathbb{N} $?\end{tcolorbox}
Yes, it is, since OP demanded that $f(\mathbb N)\subseteq\mathbb N$
\end{solution}
*******************************************************************************
-------------------------------------------------------------------------------

\begin{problem}[Posted by \href{https://artofproblemsolving.com/community/user/139296}{ablyrise}]
	you solve ex:find all contiuous functions $f:\mathbb{R}\to\mathbb{R}$ so that
$f\left(f\left(f\left(x\right)\right)\right)=x,\forall x\in\mathbb{R}$
	\flushright \href{https://artofproblemsolving.com/community/c6h488303}{(Link to AoPS)}
\end{problem}



\begin{solution}[by \href{https://artofproblemsolving.com/community/user/29428}{pco}]
	\begin{tcolorbox}you solve ex:find all contiuous functions $f:\mathbb{R}\to\mathbb{R}$ so that
$f\left(f\left(f\left(x\right)\right)\right)=x,\forall x\in\mathbb{R}$\end{tcolorbox}
$f(x)$ is bijective, and so monotonous since continuous, and so increasing (else $LHS$ is decreasing while $RHS$ is increasing).
If $f(x)>x$ for some $x$, then, since increasing, $f(f(x))>f(x)>x$ and then, since increasing, $x=f(f(f(x)))>f(f(x)>f(x)>x$, impossible.
If $f(x)<x$ for some $x$, then, since increasing, $f(f(x))<f(x)<x$ and then, since increasing, $x=f(f(f(x)))<f(f(x)<f(x)<x$, impossible.

So $\boxed{f(x)=x}$ $\forall x$, which indeed is a solution.
\end{solution}
*******************************************************************************
-------------------------------------------------------------------------------

\begin{problem}[Posted by \href{https://artofproblemsolving.com/community/user/1991}{orl}]
	Determine all pairs $(f,g)$ of functions from the set of real numbers to itself that satisfy \[g(f(x+y)) = f(x) + (2x + y)g(y)\] for all real numbers $x$ and $y$.

\begin{italicized}Proposed by Japan\end{italicized}
	\flushright \href{https://artofproblemsolving.com/community/c6h488535}{(Link to AoPS)}
\end{problem}



\begin{solution}[by \href{https://artofproblemsolving.com/community/user/29428}{pco}]
	\begin{tcolorbox}Determine all pairs $(f,g)$ of functions from the set of real numbers to itsels that satisfy \[g(f(x+y)) = f(x) + (2x + y)g(y)\] for all real numbers $x$ and $y.$\end{tcolorbox}
Let $P(x,y)$ be the assertion $g(f(x+y))=f(x)+(2x+y)g(y)$

Let $x\ne 0$ :
$P(x,0)$ $\implies$ $g(f(x))=f(x)+2xg(0)$
$P(0,x)$ $\implies$ $g(f(x))=f(0)+xg(x)$
Subtracting, we get $g(x)=\frac{f(x)-f(0)}x+2g(0)$ $\forall x\ne 0$

$P(x,y)$ $\implies$ $g(f(x+y))=f(x)+(2x+y)g(y)$
$P(x+y,0)$ $\implies$ $g(f(x+y))=f(x+y)+(2x+2y)g(0)$
Subtracting, we get $f(x+y)=f(x)+(2x+y)g(y)-(2x+2y)g(0)$

Considering $y\ne 0$ and using previous result, this becomes $f(x+y)=f(x)+(2x+y)\frac{f(y)-f(0)}y+2xg(0)$
Considering $x\ne 0$ and swapping $x,y$, this becomes $f(x+y)=f(y)+(2y+x)\frac{f(x)-f(0)}x+2yg(0)$

Considering $x,y\ne 0$ and subtracting, we get $f(x)=x^2(\frac{f(y)-f(0)}{y^2}+\frac{g(0)}y)-g(0)x+f(0)$

Setting $y=1$ in the above line, we get $f(x)=x^2(f(1)-f(0)+g(0))-g(0)x+f(0)$ $\forall x\ne 0$

Plugging this in the equality $g(x)=\frac{f(x)-f(0)}x+2g(0)$ $\forall x\ne 0$ we previously got, we get then :
$g(x)=x(f(1)-f(0)+g(0))+g(0)$ $\forall x\ne 0$

Plugging this in original equation, we get two possibilities :
$f(x)=g(x)=0$ $\forall x\ne 0$
$f(x)=x^2+c$ and $g(x)=x$ $\forall x\ne 0$

It's then easy to check that we need the same values for $x=0$ and we get the\begin{bolded} two families of solutions\end{underlined}\end{bolded} :
$f(x)=g(x)=0$ $\forall x$
$f(x)=x^2+c$ and $g(x)=x$ $\forall x$
\end{solution}



\begin{solution}[by \href{https://artofproblemsolving.com/community/user/152831}{Orin}]
	Let $P(x,y)$ be the assertion $g(f(x+y))=f(x)+(2x+y)g(y)$
$P(x,y)-P(y,x) \Rightarrow f(x)-f(y)=(2y+x)g(x)-(2x+y)g(y)$.
Let $Q(x,y)$ be the assertion $f(x)-f(y)=(2y+x)g(x)-(2x+y)g(y)$
$Q(1,0) \Rightarrow f(1)-f(0)=g(1)-2g(0)$.....(1)
$Q(x,0)-Q(x,1) \Rightarrow f(1)-f(0)=2xg(1)+g(1)-2xg(0)-2g(x)$.....(2)
From (1),(2) we get $g(x)=(g(1)-g(0))x+g(0)$
now we have two cases.
Case1 $\Rightarrow g(1)-g(0)=0$
Then $g(x)=g(0)=c \forall x \in \mathbb {R}$,where $c$ is a constant.
Plugging $g(x)=c$ in $P(x,y) \Rightarrow f(x)+2cx+cy=c \forall x,y \in \mathbb {R}$
which implies $c=0,f(x)=0$ whcih is indeed a solution.
Case2 $\Rightarrow g(1)-g(0) \neq 0$
We may write $g(x)=ax+b;a \in \mathbb {R}-\{0\},b \in \mathbb {R}$.
Plugging $g(x)=ax+b$ in $P(x,-x) \Rightarrow af(0)+b=f(x)+xg(-x)$
or,$f(x)=ax^2-bx+af(0)+b$,let $af(0)+b=c$
Then $f(x)=ax^2-bx+c$
Plugging these values in $P(0,x)$
$\Rightarrow a^2x^2-abx+ac+b=ax^2+bx+c \forall x \in \mathbb {R}$,Let it be $R(x)$.
$R(0) \Rightarrow ac+b=c$.....(3)
$R(1)-R(-1) \Rightarrow b(a-1)=0$
Again we have 2 cases.
Case2.1:$a=1$,so from (3) we get $b=0$
so $f(x)=x^2+c,g(x)=x$ which indeed satisfy $P(x,y)$.
Case2.2:$b=0$,so from (3) we get $c(a-1)=0$
If $a=1$ it becomes case 2.1
So $c=0$ implying $f(x)=ax^2,g(x)=ax$
Plugging these values in $P(x,0) \Rightarrow a=0$ or $a=1$
if $a=1;f(x)=x^2,g(x)=x$,which is a subset of solutions of Case2.1
else if $a=0;f(x)=0,g(x)=0$.
=============================================
Synthesis of solutions:
1.$f(x)=0,g(x)=0 \forall x \in \mathbb {R}$
2,$f(x)=x^2+c,g(x)=x \forall x \in \mathbb {R}$.
\end{solution}



\begin{solution}[by \href{https://artofproblemsolving.com/community/user/177206}{omkarkamat}]
	I have another way of solving the problem but don't know if it's right.

Plug in (x,y)=(x,-2x) to get g(f(-x))=f(x)     for all x in R
And (x,y) = (y,-2y) to get g(f(-y)=f(y)        for all y in R

Case 1: f(x) is even.
If this is the case, then g(x) = x for all x in R
The original equation now becomes f(x + y)=f(x)+y(2x+y)
If we take (x,y)=(0,y) we get f(y)=y^2. WLOG let f(0) =c. Where can is a constant,
Then f(y)=y^2 +c for all y in R.

Now we prove that either f(x) is a linear function or a constant function.
We find that f(x)-f(y)=(2y+x)g(x)-(2x+y)g(y)
Plugging in (x,y) = (x,0) ,(1,x),(0,1) we obtain

F(x)-f(0)=xg(x)-2xg(0)
F(1)-f(x)=(2x+1)g(1)-(x+2)g(x)
F(0)-f(1)=2g(0)-g(1)

Taking the sum and dividing by two yields
G(x)=x(g(1)-g(0))+g(0) which proves that it has to a constant or a linear function.

We first examine the case in which it is a constant if such a function exists.
We take f(x)=c1 and this implies that g(x)=c2. Plugging into the equation we see that the only constant that works is 0. 

Since this constant works, there is no such linear function that we need to examine and hence the solutions are 

F(x)= y^2 +c, f(x)=0 and g(x)=x, g(x)=0
\end{solution}



\begin{solution}[by \href{https://artofproblemsolving.com/community/user/198687}{Legend-crush}]
	Let $P(x,y)\Leftrightarrow g(f(x+y))=f(x)+(2x+y)g(y)$
$P(x,0)\Rightarrow  g(f(x))=f(x)+2xg(0)\  (*)$
$P(0,x) \Rightarrow  g(f(x))=f(x)+2xg(0) \ (**)$
from (*) and (**) we get \[f(x)=f(0)+x(g(x)-2g(0))\]
replacing in the functionnal equation we get \[P(x,y) \Rightarrow  (x+y)g(x+y)=x(g(x)-2g(0))+(2x+y)g(y)\]
since the LHS is symetric in x,y ,  we have  \[x(g(x)-2g(0))+(2x+y)g(y)=y(g(y)-2g(0))+(2y+x)g(x)\]
which yields to $(\forall x,y\neq 0) \ \frac{g(x)-g(0)}{x}=\frac{g(y)-g(0)}{y}$ hence \[(\exists a,b\in \mathbb{R}) (\forall x\in \mathbb{R}) g(x)=ax+b\]
it also follows that $f(x)=ax^2-bx+f(0)$ forall real x.
replacing in the functionnal equation we get : 
1.$f=g=0 $
2. $f(x)=x^2+c$ and $g(x)=x \forall x$
\end{solution}



\begin{solution}[by \href{https://artofproblemsolving.com/community/user/167924}{utkarshgupta}]
	\begin{bolded}Solution\end{bolded}
Let $P(x,y)$ be the assertion $g(f(x+y)) = f(x) + (2x + y)g(y)$.
$$P(x,0) \implies g(f(x))=f(x)+2xg(0)$$
$$P(0,x) \implies g(f(x))=f(0)+xg(x)$$
$$\implies xg(x)+f(0)=f(x)+2xg(0)$$
$$\implies f(x)=xg(x)+f(0)-2xg(0)$$ 

Replacing this in $P(x,y)$
$$g(f(x+y))=xg(x)+f(0)-2xg(0)+(2x+y)g(y)$$
Here setting $x=z$, $y=x+y-z$
$$g(f(x+y))=zg(z)+f(0)-2zg(0)+(x+y+z)g(x+y-z)$$

Comparing and setting $y=z=1$
$$xg(x)-2xg(0)+2xg(1)=2g(0)+(x+2)g(x)$$
$$\implies g(x)=Ax+B$$

Now the question is easy 


Thus the solutions are $\boxed {f(x)=0=g(x)}$ or
 $\boxed {f(x)=x^2+C}$ and $\boxed {g(x)=x}$
\end{solution}



\begin{solution}[by \href{https://artofproblemsolving.com/community/user/201484}{AMN300}]
	Let $P(x, y)$ be the given assertion.
\[ P(x, 1) \implies g(f(x+1)) = f(x) + (2x+1) g(1) \]
\[ P(1, x) \implies g(f(x+1)) = f(1) + (x+2) g(x) \]
\[ P(x, 0) \implies g(f(x)) = f(x) + 2x g(0) \]
\[ P(0, x) \implies g(f(x)) = f(0) + x g(x) \]
The first two equations mean \[ f(x) + (2x+1) g(1) = f(1) + (x+2) g(x) = g(f(x+1)) \]
The second two mean \[ f(x) + 2x g(0) = f(0) + x g(x) = g(f(x)) \]
Subtracting the two equations we have \[ 2x(g(1)-g(0)) + g(1) = f(1) - f(0) + 2 g(x) \]
So $g(x)$ is linear. Plug in $g(x)=ax+b$. Then,
\[ P(x, 0) \implies mf(x) + b = f(x) + 2xb \]
If $m=1$ we have $b=0$. We see that this means $f(x) = x^2+c$ upon plugging in.
If $m \neq 1$ we simplify to $f(x) = \frac{b(2x-1)}{m-1}$. $P(0, x) \implies mf(x)=f(0)+xg(x)$ so plugging in we see that we need $b=m=0$.
So the solutions are $\boxed{f(x) \equiv 0, g(x) \equiv 0}$ or $\boxed{f(x) \equiv x^2+c, g(x) \equiv x}$.
\end{solution}



\begin{solution}[by \href{https://artofproblemsolving.com/community/user/247657}{Ankoganit}]
	We do not use the notation $P(x,y)$ for the given assertion. Swapping $x,y$ in the given equation and then comparing it with the given yields: $$f(x)+(2x+y)g(y)=f(y)+(2y+x)g(x).\qquad (\star)$$Setting $y=0$ in $(\star)$ gives $$f(x)=xg(x)-2xg(0)+f(0).\qquad (\smiley )$$Now plug this definition of $f(x)$ into $(\star)$ to get: $$-xg(0)+xg(y)=-yg(0)+yg(x).$$For nonzero $x,y$, this becomes $$\frac{g(y)-g(0)}{y}=\frac{g(x)-g(0)}{x}\implies g(x)=kx+g(0)\forall x\in\mathbb R\setminus \{0\}.$$ Here $k$ is some constant. Note that this automatically holds for $x=0$, so no need to worry about that $\mathbb R\setminus \{0\}$ bit. So $g(x)$ is of the form $ax+b$, and using $(\smiley )$ lets us infer that $f(x)=ax^2-bx+c$. Now plugging these back into the original equation, we see that only $\boxed{g(x)\equiv f(x)\equiv 0}$ and $\boxed{g(x)\equiv x, \; f(x)\equiv x^2+c}$ work, so these are the only solutions. $\blacksquare$
\end{solution}



\begin{solution}[by \href{https://artofproblemsolving.com/community/user/208265}{john111111}]
	My solution:
Let $P(x,y)$ be the assertion that $g(f(x+y))=f(x)+(2x+y)g(y)$
$P(0,0) \Rightarrow g(f(0))=f(0)$
$P(-x,2x) \Rightarrow g(f(x))=f(-x)$
$P(x,0) \Rightarrow g(f(x))=f(x)+2xg(0)$ (1).
Suppose that $g(0) \neq 0$.Then (1) easily implies that f is 1-1.
$P(-f(x),f(x)) \Rightarrow g(f(0))=f(-f(x))-f(x)g(f(x)) \Rightarrow f(-f(x))=f(x)f(-x)+f(0)$                                                                                                  .Putting in this $-x$ where $x$ is we obtain
$f(-f(x))=f(-f(-x)$ so from 1-1 $f(x)=f(-x) \Rightarrow x=-x \Rightarrow x=0$, a contradiction. So $g(0)=0$ and thus from (1) $g(f(x))=f(x)$ (2)
$P(0,x) \Rightarrow g(f(x))=f(0)+xg(x) \Rightarrow f(x)=f(0)+xg(x)$ (3).
Using (2) and (3) in the initial relation we have $f(x+y)=f(x)+(2x+y)g(y) \Rightarrow (x+y)g(x+y)+f(0)=xg(x)+f(0)+(2x+y)g(y) \Rightarrow (x+y)g(x+y)=xg(x)+(2x+y)g(y)$.                                                                                                                                   Considering here $xy \neq 0$ by swapping $x,y$ we have $xg(x)+(2x+y)g(y)=yg(y)+(2y+x)g(x) \Rightarrow xg(y)=yg(x) \Rightarrow \frac {g(x)} {x}=\frac {g(y)} {y}=c \Rightarrow g(x)=cx$ for every $x \in \mathbb{R}$
(since $g(0)=0$) and substituting in (2) we have $f(x)(c-1)=0$.
If $f(x)=0$ for all real $x$ then (3) gives $g(x)=0$ for all real $x$ (since $g(0)=0$) which is indeed a solution.
If $c=1$ we have that $g(x)=x$ and from (3) we obtain $f(x)=f(0)+x^2$ which is a solution,whatever $f(0)$ is.
\end{solution}



\begin{solution}[by \href{https://artofproblemsolving.com/community/user/328042}{ak12sr99}]
	Let $P(x,y)$ denote the given functional equation. Then $P(x,0)$ and $P(0,x)$ yield $$g(f(x))=f(x)+2xg(0)=f(0)+xg(x) \implies f(x)=f(0)-2xg(0)+xg(x) ....(1).$$ Next $P(x,y)$ and $P(x+y,0)$ yield $f(x+y)+2(x+y)g(0)=f(x)+(2x+y)g(y)$ which, in conjunction with $(1)$, becomes $$(f(0)-2(x+y)g(0)+(x+y)g(x+y))+2(x+y)g(0)=(f(0)-2xg(0)+xg(x))+(2x+y)g(y) \iff \frac{g(x)-g(0)}{x}=\frac{g(y)-g(0)}{y}=\lambda$$ after which it's just substitution.
\end{solution}
*******************************************************************************
-------------------------------------------------------------------------------

\begin{problem}[Posted by \href{https://artofproblemsolving.com/community/user/1991}{orl}]
	Determine all pairs $(f,g)$ of functions from the set of positive integers to itself that satisfy \[f^{g(n)+1}(n) + g^{f(n)}(n) = f(n+1) - g(n+1) + 1\] for every positive integer $n$. Here, $f^k(n)$ means $\underbrace{f(f(\ldots f)}_{k}(n) \ldots ))$.

\begin{italicized}Proposed by Bojan Bašić, Serbia\end{italicized}
	\flushright \href{https://artofproblemsolving.com/community/c6h488536}{(Link to AoPS)}
\end{problem}



\begin{solution}[by \href{https://artofproblemsolving.com/community/user/29428}{pco}]
	$LHS\ge 2$ and so $RHS\ge 2$ and so $f(n+1)\ge g(n+1)+1\ge 2$ $\forall n\ge 1$ and so $f(n)\ge 2$ $\forall n\ge 2$
So $LHS\ge 3$ $\forall n\ge 2$ and so $f(n+1)\ge g(n+1)+2\ge 3$ $\forall n\ge 2$ and so $f(n)\ge 3$ $\forall n\ge 3$
So $LHS\ge 4$ $\forall n\ge 3$ and so $f(n+1)\ge g(n+1)+3\ge 4$ $\forall n\ge 3$ and so $f(n)\ge 4$ $\forall n\ge 4$
...
And we easily get thru induction $f(n)\ge n$ $\forall n\ge 1$

So $f(n+1)=f^{g(n)}(f(n))+g^{f(n)}(n)+g(n+1)-1$ $\ge f(n)+1$ and $f(n)$ is increasing.

If $f(n)\ge n+1$ for some $n$, then $f(f(n))\ge f(n+1)$ and so $f^{g(n)+1}(n)\ge f(n+1)$ and then $LHS\ge f(n+1)+1$ while $RHS \le f(n+1)$, impossible

So $f(n)=n$ $\forall n$ and equation is $g^{n}(n)=2-g(n+1)$ and so $g(n+1)=1$ and $g(n)=1$ $\forall n\ge 2$
Setting $n=1$ in equation, we get then $g(1)=2-g(2)=1$

\begin{bolded}Hence the solution\end{underlined}\end{bolded} : $f(n)=n$ and $g(n)=1$ $\forall n\in\mathbb N$, which indeed is a solution.
\end{solution}



\begin{solution}[by \href{https://artofproblemsolving.com/community/user/31917}{daniel73}]
	Nice solution, Patrick!

A relatively standard technique that can also be applied here is as follows: assume first that the minimum that $f$ takes is for some $N>1$, take $n=N-1$ in the proposed equation, and reach a contradiction, hence $f(1)<f(n)$ for all $n\geq2$. Now assume that $f(1)<f(2)<\dots<f(N)$ for some $N\geq1$, and $f(n)>f(N)$ for all $n>N$ (clearly true for $N=1$). It clearly follows that $f(n)\geq n$ for all $n\leq N$, and $f(n)>N$ for all $n>N$. Now work your way similarly to a contradiction if the minimum of $f(n)$ for $n\geq N+1$ happens for some $M>N+1$ (again taking $n=M-1$ in the proposed equation).  By induction, we conclude that $f$ is strictly increasing, and consequently $f(n)\geq n$, while for all $n>m$, $f(n)-n\geq f(m)-m$. And the (IMHO) hard part of the problem is thus concluded...

I have seen this technique somewhere before (can't remember where), applied on a different type of functional equation, but the idea behind the process for showing that $f$ is strictly increasing is similar. But your approach is much shorter and much more elegant!

Edit: now I remember where I saw this "uglier" technique, some solution to problem 6 in IMO1977... Patrick's approach would not have worked in that problem, I think, but it works perfectly in this one!
\end{solution}



\begin{solution}[by \href{https://artofproblemsolving.com/community/user/237384}{Wolowizard}]
	\begin{bolded}My solution:\end{bolded}\end{underlined}\\
\begin{bolded}Lemma 1:\end{bolded}\\
If $f(k)$ is minimum of $f$ than $k=1$.\\
\begin{bolded}Proof:\end{bolded}\\
Let $k\in \mathbb N$ be a positive integer such that $f(k)$ is minimum of $f$.\\
We have $f(n+1)-f^{g(n)+1}(n)=g(n+1)-1+g^{f(n)}(n)\ge 1$ so if $k>1$ plugging in $n=k-1$ we have that \\
$ f^{g(k-1)+1}(k-1)\le f(k)-1$ which is contradiction so $k=1$. \\
__________________________________________________________\\
\begin{bolded} Lemma 2:\end{bolded}\\
If $f(a)=1$ than $a=1$.\\
\begin{bolded}Proof:\end{bolded}\\
If we have $f(a)=1$ it implies $f(a)\le f(n)$ for all $n\in\mathbb N$ so we have $a=1$ by \begin{bolded} Lemma 1\end{bolded}.\\
__________________________________________________________\\
Let $k' \in \mathbb N$ be a positive integer such that $f(k')\le f(n)$ for all $n\neq 1$. We have that $k'>1$ so plugging in $n=k'-1$ we have\\
$f(k')-f^{g(k'-1)+1}(k'-1)=g(k')-1+g^{f(k'-1)}(k'-1)\ge 1$ so we have that $f(k')> f^{g(k'-1)+1}(k'-1)$ which implies \\
$ f^{g(k'-1)}(k'-1)=1$ so by \begin{bolded}Lemma 2\end{bolded} applied consequently we have $f(k'-1)=1$ and $k'-1=1$ which implies \\
$f(1)=1$ and $k'=2$.\\
________________________________________________________\\
\begin{bolded} Induction hypothesis:\end{bolded}\\
$f(k)=k$ and $f(A)=k \implies A=k$ for all $k\le n$.\\
Proof for $n+1$:\\
Let $a_{n+1}$ be positive integer such that $f(j)<f(a_{n+1})\le f(m)$ for all $m>n, j\le n$($n+1$'st smallest value of $f$(since $f$ is not constat it exists)).\\
Plugging in $n=a_{n+1}-1$ we have \\
$f(a_{n+1})-f^{g(a_{n+1}-1)+1}(a_{n+1}-1)=g(a_{n+1})-1+g^{f(a_{n+1}-1)}(a_{n+1}-1)\ge 1$ so we have $ f^{g(a_{n+1}-1)+1}(a_{n+1}-1)<f(a_{n+1})$ so \\
$ f^{g(a_{n+1}-1)}(a_{n+1}-1)=j$ for some $1\le j\le n$ so by IH we have $f(a_{n+1}-1)=j$ and $a_{n+1}=j+1$ so we have since $a_{n+1}>n$ \\
$j=n$ so $a_{n+1}=n+1$. \\
Now let $a_{n+2}$ be $n+2$'nd minimum of $f$ that is $f(j)<f(a_{n+2})\le f(m)$ for all $m>n+1, j\le n+1$.\\
Plugging in $a_{n+2}-1$ we have \\
$f(a_{n+2})-f^{g(a_{n+2}-1)+1}(a_{n+2}-1)\ge 1$ and it implies $ f^{g(a_{n+2}-1)}(a_{n+2}-1)=n+1$(it has to be a number from $n+1,n,...,1$ but if it's $j<n+1$ that it would be $a_{n+2}=j+1$ which is contradiction). But now we have $n+1= f^{g(a_{n+2}-1)}(a_{n+2}-1)>n>n-1>...>1$ so we have that  $ f^{g(a_{n+2}-1)}(a_{n+2}-1)$ is $n+1$ st minimum so actually $f(a_{n+1})=f(n+1)= f^{g(a_{n+2}-1)}(a_{n+2}-1)=n+1$.\\
So we have proved both statements of IH therefore it's true for all $n\in \mathbb N$.\\
______________________________________\\
Now we have $g^{n}(n)+g(n+1)=2$ so $g(n)=1$ for all $n$.\\
So the only solution is $f(n)=n,g(n)=1$ for all $n$ which indeed is s solution since $n+1-1+1=n+1$.



\end{solution}
*******************************************************************************
-------------------------------------------------------------------------------

\begin{problem}[Posted by \href{https://artofproblemsolving.com/community/user/68025}{Pirkuliyev Rovsen}]
	Determine all functions $f:(0;+\infty)\to\mathbb{R}$ satisfying $1)f(2)=-4$
  $2)f(xy)=x^2f(y)+y^2f(x)$ for all $x,y{\in}(0;+\infty)$.



______________________________________
Azerbaijan Land of the Fire 
	\flushright \href{https://artofproblemsolving.com/community/c6h489222}{(Link to AoPS)}
\end{problem}



\begin{solution}[by \href{https://artofproblemsolving.com/community/user/29428}{pco}]
	\begin{tcolorbox}Determine all functions $f:(0;+\infty)\to\mathbb{R}$ satisfying $1)f(2)=-4$
  $2)f(xy)=x^2f(y)+y^2f(x)$ for all $x,y{\in}(0;+\infty)$.\end{tcolorbox}
Let $h(x)$ from $\mathbb R\to\mathbb R$ defined as $h(x)=f(e^{x})e^{-2x}$. Problem becomes $h(x+y)=h(x)+h(y)$ $\forall x,y\in\mathbb R$ and $h(\ln 2)=-1$. 

And so infinitely many solutions : $\boxed{f(x)=x^2h(\ln x)}$ where $h(x)$ is any solution of additive Cauchy equation such that $h(\ln 2)=-1$.
\end{solution}
*******************************************************************************
-------------------------------------------------------------------------------

\begin{problem}[Posted by \href{https://artofproblemsolving.com/community/user/72235}{Goutham}]
	Let $f:\mathbb{R}\longrightarrow \mathbb{R}$ be a function such that $f(x+y+xy)=f(x)+f(y)+f(xy)$ for all $x, y\in\mathbb{R}$. Prove that $f$ satisfies $f(x+y)=f(x)+f(y)$ for all $x, y\in\mathbb{R}$.
	\flushright \href{https://artofproblemsolving.com/community/c6h489338}{(Link to AoPS)}
\end{problem}



\begin{solution}[by \href{https://artofproblemsolving.com/community/user/29428}{pco}]
	\begin{tcolorbox}Let $f:\mathbb{R}\longrightarrow \mathbb{R}$ be a function such that $f(x+y+xy)=f(x)+f(y)+f(xy)$ for all $x, y\in\mathbb{R}$. Prove that $f$ satisfies $f(x+y)=f(x)+f(y)$ for all $x, y\in\mathbb{R}$.\end{tcolorbox}
Let $P(x,y)$ be the assertion $f(x+y+xy)=f(x)+f(y)+f(xy)$

$P(0,0)$ $\implies$ $f(0)=0$
$P(x,-1)$ $\implies$ $f(-x)=-f(x)$

Let $x,y\ne -1$
$P(x,\frac y{x+1})$ $\implies$ $f(x+y)=f(x)+f(\frac y{x+1})+f(\frac {xy}{x+1})$

$P(x,-\frac y{x+1})$ $\implies$ $f(x-y)=f(x)-f(\frac y{x+1})-f(\frac {xy}{x+1})$

$P(y,\frac x{y+1})$ $\implies$ $f(x+y)=f(y)+f(\frac x{y+1})+f(\frac {xy}{y+1})$

$P(y,-\frac x{y+1})$ $\implies$ $-f(x-y)=f(y)-f(\frac x{y+1})-f(\frac {xy}{y+1})$

Adding these four lines, we get $f(x+y)=f(x)+f(y)$ $\forall x,y\ne -1$

It remains to get rid of the constraint $x,y\ne -1$ :
Let $x\ne 0$ : $x-1\ne -1$ and $1\ne -1$ and so $f((x-1)+1)=f(x-1)+f(1)$ and so $f(x-1)=f(x)+f(-1)$, still true when $x=0$.
So $f(x+y)=f(x)+f(y)$ $\forall x\ne -1,\forall y$
So $f(x+y)=f(x)+f(y)$ $\forall x,\forall y\ne -1$
And $f(1+1)=f(1)+f(1)$ and so $f(-1-1)=f(-1)+f(-1)$ and so $f(x+y)=f(x)+f(y)$ when $x=y=-1$ 


So $f(x+y)=f(x)+f(y)$ $\forall x,y$
Q.E.D.
\end{solution}



\begin{solution}[by \href{https://artofproblemsolving.com/community/user/142879}{ionbursuc}]
	Generalization 1
Let $a\in {{\mathbb{R}}^{*}}$ and $f:\mathbb{R}\to \mathbb{R}$ be a function such that $f(x+y+axy)=f(x)+f(y)+f(axy)$ for all $x, y\in\mathbb{R}$. Prove that $f$ satisfies $f(x+y)=f(x)+f(y)$ for all $x, y\in\mathbb{R}$.
\end{solution}



\begin{solution}[by \href{https://artofproblemsolving.com/community/user/142879}{ionbursuc}]
	Generalization 2
Let $a,b\in {{\mathbb{R}}^{*}},a+b\ne 0$ and $f:\mathbb{R}\to \mathbb{R}$ be a function such that $f(ax+by+xy)=af(x)+bf(y)+f(xy)$ for all $x, y\in\mathbb{R}$. Prove that $f$ satisfies $f(x+y)=f(x)+f(y)$ for all $x, y\in\mathbb{R}$.
\end{solution}



\begin{solution}[by \href{https://artofproblemsolving.com/community/user/72235}{Goutham}]
	Ah, it is in fact from [url=http://www.artofproblemsolving.com/Forum/viewtopic.php?f=149&t=366996&]LL[\/url].
\end{solution}



\begin{solution}[by \href{https://artofproblemsolving.com/community/user/273464}{div5252}]
	Isn't it very simple
It can be clearly seen that $f(x)=0$
$f(x+y+xy+0)=f(x+y)+f(xy)+f(0)$
or $f(x+y+xy)=f(x+y)+f(xy)$
Subtracting the above and given equation
we get $f(x+y)=f(x)+f(y)$
\end{solution}



\begin{solution}[by \href{https://artofproblemsolving.com/community/user/247657}{Ankoganit}]
	\begin{tcolorbox}
$f(x+y+xy+0)=f(x+y)+f(xy)+f(0)$
\end{tcolorbox}

Why?


\end{solution}



\begin{solution}[by \href{https://artofproblemsolving.com/community/user/273464}{div5252}]
	oh i thought we could divide it into any 3 sums
\end{solution}



\begin{solution}[by \href{https://artofproblemsolving.com/community/user/293525}{uraharakisuke_hsgs}]
	\begin{bolded}My solution : \end{underlined}\end{bolded}
We have : $f(x+y+xy) = f(x)+f(y)+f(xy)$ $(1)$
Plug in $x=y=0$ $\rightarrow$ $f(0) = 0$
Plug in $ y = -1$ $\rightarrow$ $f(-x) = -f(x)$
Plug in $ y = 1 $ $\rightarrow$ $f(2x+1) = 2f(x) + 1$ $(2)$
From $(1)$ and $(2)$ $\rightarrow$ $f(2(xy+x+y)+1) = 2f(xy+x+y) + f(1) = 2[f(x)+f(y)+f(xy)] +f(1)$
On the other hand we have : 
$f(2(xy+x+y)+1) = f(2x+1+y(2x+1) +y) \\ = f(2x+1) + f(y(2x+1)) + f(y) \\ =2f(x) + f(1) + f(y) + f(2xy + y)$
$\rightarrow$ $f(2xy + y) = f(y) + 2f(xy)$ $(3)$
Plug in $x = \frac{-1}{2} $ to $(3)$ we have :
$f(y) = -2f(\frac{-y}{2}) = 2f(\frac{y}{2})$ 
Combine this with $(3)$ , it follows that $f(2xy+y) = f(y) + f(2xy)$
With $u,v \in\mathbb{R} , v = 0$ we have $f(u+v) = f(u) + f(v)$
With $u,v \in\mathbb{R} ; u , v \ne 0$ let $x = \frac{u}{2v} ; y = v$ \\ $\rightarrow$ $f(u+v) = f(u) + f(v)$
So $f(x+y) = f(x) + f(y)$ $\forall x,y$
\end{solution}
*******************************************************************************
-------------------------------------------------------------------------------

\begin{problem}[Posted by \href{https://artofproblemsolving.com/community/user/68025}{Pirkuliyev Rovsen}]
	Determine all continuous functions $f: \mathbb{R}\to\mathbb{R}$ such that 
$f(x)+f(y)+f(z)+f(x+y+z)=f(x+y)+f(y+z)+f(z+x)$


_________________________________
Azerbaijan Land of the Fire 
	\flushright \href{https://artofproblemsolving.com/community/c6h489364}{(Link to AoPS)}
\end{problem}



\begin{solution}[by \href{https://artofproblemsolving.com/community/user/29428}{pco}]
	\begin{tcolorbox}Determine all continuous functions $f: \mathbb{R}\to\mathbb{R}$ such that 
$f(x)+f(y)+f(z)+f(x+y+z)=f(x+y)+f(y+z)+f(z+x)$\end{tcolorbox}
Let $P(x,y,z)$ be the assertion $f(x)+f(y)+f(z)+f(x+y+z)$ $=f(x+y)+f(y+z)+f(z+x)$

$P(0,0,0)$ $\implies$ $f(0)=0$

$P(x,-x,(n+1)x)$ $\implies$ $f((n+2)x)=2f((n+1)x)-f(nx)+f(x)+f(-x)$

This induction easily implies $f(nx)=\frac{f(x)+f(-x)}2n^2+\frac{f(x)-f(-x)}2$    $\forall n\in\mathbb Z$

And also, setting $x\to -x$ : $f(-nx)=\frac{f(x)+f(-x)}2n^2-\frac{f(x)-f(-x)}2$ 

Adding, we get $f(nx)+f(-nx)=n^2(f(x)+f(-x))$
So $f(x)+f(-x)=x^2(f(1)+f(-1))$ $\forall x\in\mathbb Q$
And continuity implies then $f(x)+f(-x)=2ax^2$ $\forall x$ and for some $a\in\mathbb R$

Subtracting, instead of adding, we get $f(nx)-f(-nx)=n(f(x)-f(-x))$
So $f(x)-f(-x)=x(f(1)-f(-1))$ $\forall x\in\mathbb Q$
And continuity implies then $f(x)-f(-x)=2bx$ $\forall x$ and for some $b\in\mathbb R$

So $\boxed{f(x)=ax^2+bx}$ $\forall x$, which indeed is a solution, whatever are $a,b\in\mathbb R$
\end{solution}
*******************************************************************************
-------------------------------------------------------------------------------

\begin{problem}[Posted by \href{https://artofproblemsolving.com/community/user/153347}{KazemSepehrinia}]
	If       f(x) -3 f(x-1) =2-2(x^2)

and f(0)=3

then f(99) =...
	\flushright \href{https://artofproblemsolving.com/community/c6h489507}{(Link to AoPS)}
\end{problem}



\begin{solution}[by \href{https://artofproblemsolving.com/community/user/29428}{pco}]
	\begin{tcolorbox}If       f(x) -3 f(x-1) =2-2(x^2)

and f(0)=3

then f(99) =...\end{tcolorbox}
So we have to find $a_{99}$ in the sequence $a_0=3$ and $a_{n+1}=3a_n+2-2(n+1)^2$

Let $b_n=(a_n+1)3^{-n}$ and we get $b_0=4$ and $b_{n+1}=b_n-2(n+1)^23^{-(n+1)}$ and so $b_n=4-2\sum_{k=0}^nk^23^{-k}$

Starting from $f(x)=\sum_{k=1}^nx^k=\frac{x^{n+1}-x}{x-1}$ (extended with continuity at $x=1$), we get :

$xf'(x)=\sum_{k=1}^nkx^{k}=\frac{nx^{n+2}-(n+1)x^{n+1}+x}{(x-1)^2}$

$x(xf'(x))'=\sum_{k=1}^nk^2x^{k}=\frac{n^2x^{n+3}-(2n^2+2n-1)x^{n+2}+(n+1)^2x^{n+1}-x^2-x}{(x-1)^3}$

And so $\sum_{k=0}^nk^23^{-k}=$ $\frac{n^23^{-n-3}-(2n^2+2n-1)3^{-n-2}+(n+1)^23^{-n-1}-3^{-2}-3^{-1}}{(3^{-1}-1)^3}$

So $\sum_{k=0}^nk^23^{-k}=$ $\frac{3^{n+1}-(n^2+3n+3)}{2\times 3^n}$

And $b_n=\frac{n^2+3n+3+3^n}{3^n}$

And $\boxed{f(n)=n^2+3n+2+3^n}$ $\forall $ integer $n\ge 0$ and $f(99)=10100+3^{99}$
\end{solution}
*******************************************************************************
-------------------------------------------------------------------------------

\begin{problem}[Posted by \href{https://artofproblemsolving.com/community/user/10045}{socrates}]
	Determine all functions $f,g,h:\Bbb{R}\to\Bbb{R}$ such that $f(h(g(x)) + y) + g(z + f(y)) = h(y) + g(y + f(z)) + x$ for all $x,y,z\in \Bbb{R}.$
	\flushright \href{https://artofproblemsolving.com/community/c6h489518}{(Link to AoPS)}
\end{problem}



\begin{solution}[by \href{https://artofproblemsolving.com/community/user/29428}{pco}]
	\begin{tcolorbox}Determine all functions $f,g,h:\Bbb{R}\to\Bbb{R}$ such that $f(h(g(x)) + y) + g(z + f(y)) = h(y) + g(y + f(z)) + x$ for all $x,y,z\in \Bbb{R}.$\end{tcolorbox}
Let $P(x,y,z)$ be the assertion $f(h(g(x))+y)+g(z+f(y))=h(y)+g(y+f(z))+x$

If $g(a)=g(b)$, subtracting $P(a,0,0)$ from $P(b,0,0)$ gives $a=b$ and so $g(x)$ is injective.

Subtracting $P(0,x,x)$ from $P(0,x,0)$, we get $g(f(x))=g(x+f(0))$ and so, since $g(x)$ is injective,  $f(x)=x+a$ 

$P(0,x,0)$ becomes then $h(x)=x+a+h(g(0))$ and so $h(x)=x+b$

$P(x,0,0)$ becomes then $g(x)=x-a$

Hence the solutions, where $a,b$ are any two real numbers :
$f(x)=x+a$ $\forall x$
$g(x)=x-a$ $\forall x$
$h(x)=x+b$ $\forall x$
Which indeed fit
\end{solution}
*******************************************************************************
-------------------------------------------------------------------------------

\begin{problem}[Posted by \href{https://artofproblemsolving.com/community/user/138558}{Ali-mes}]
	Find all functions defined from the set of reals to itself, such that: 

$f(x)^2+x.f(x)+f(y)=f(xf(x)+x^2+y) $ for all real numbers $x$ and $y$ .
	\flushright \href{https://artofproblemsolving.com/community/c6h489583}{(Link to AoPS)}
\end{problem}



\begin{solution}[by \href{https://artofproblemsolving.com/community/user/29428}{pco}]
	\begin{tcolorbox}Find all functions defined from the set of reals to itself, such that: 

$f(x)^2+x.f(x)+f(y)=f(xf(x)+x^2+y) $ for all real numbers $x$ and $y$ .\end{tcolorbox}
Continuity would help a lot (and give solutions $f(x)=0$ $\forall x$, $f(x)=x$ $\forall x$ and $f(x)=-x$ $\forall x$).

Could you confirm us that the statement you posted is the exact problem you got in your olympiad contest or training session ?
\end{solution}



\begin{solution}[by \href{https://artofproblemsolving.com/community/user/10045}{socrates}]
	\begin{tcolorbox}[quote="Ali-mes"]Find all functions defined from the set of reals to itself, such that: 

$f(x)^2+x.f(x)+f(y)=f(xf(x)+x^2+y) $ for all real numbers $x$ and $y$ .\end{tcolorbox}
Continuity would help a lot (and give solutions $f(x)=0$ $\forall x$, $f(x)=x$ $\forall x$ and $f(x)=-x$ $\forall x$).

Could you confirm us that the statement you posted is the exact problem you got in your olympiad contest or training session ?\end{tcolorbox}


Pco, can you present your solution with the continuity constraint?
Thank you!
\end{solution}



\begin{solution}[by \href{https://artofproblemsolving.com/community/user/29428}{pco}]
	\begin{tcolorbox}Find all functions defined from the set of reals to itself, such that: 

$f(x)^2+x.f(x)+f(y)=f(xf(x)+x^2+y) $ for all real numbers $x$ and $y$ .\end{tcolorbox}
Considering that we have the constraint "continuous" :
Let $g(x)=xf(x)+x^2$ and $A=g(\mathbb R)$
Let $P(x,y)$ be the assertion $f(g(x)+y)=f(x)^2+xf(x)+f(y)$

$P(0,0)$ $\implies$ $f(0)=0$
$P(x,0)$ $\implies$ $f(g(x))=f(x)^2+xf(x)$ and so $f(g(x)+y)=f(g(x))+f(y)$

So $f(x+y)=f(x)+f(y)$ $\forall x$, $\forall y\in A$
So (setting $x\to -y$) : $f(-y)+f(y)=0$ $\forall y\in A$
And (setting $x\to x-y$) : $f(x-y)=f(x)-f(y)=f(x)+f(-y)$ $\forall x$, $\forall y\in A$

Since $0\in A$ and $g(x)$ continuous, then :
Either $A=\{0\}$ and so $xf(x)+x^2=0$ $\forall x$ and so $f(x)=-x$ $\forall x$ (remember we already know that $f(0)=0$), which indeed is a solution
Either $\exists a\ne 0$ such that $[0,a]$ (or $[a,0]$) $\subseteq A$ and so $f(x+y)=f(x)+f(y)$ $\forall x$, $\forall y\in [-|a|,+|a|]$ 
Its immediate to get  thru induction that $f(x+ny)=f(x)+nf(y)$ $\forall x$, $\forall y\in [-|a|,+|a|]$ 

And so $f(x+y)=f(x)+f(y)$ $\forall x,y$ and continuity gives then $\boxed{f(x)=ax}$ which indeed is a solution whatever is $a$.

And BTW, I missed some solutions when I wrote in my previous post $f(x)=0,x,-x$ :oops:
\end{solution}



\begin{solution}[by \href{https://artofproblemsolving.com/community/user/138558}{Ali-mes}]
	Hello ! I've got this EF while trying to solve another one !! 

How about when $ f(f(x)+x)=-f(x) $ forall real number $x$ ??
\end{solution}



\begin{solution}[by \href{https://artofproblemsolving.com/community/user/29428}{pco}]
	\begin{tcolorbox}Hello ! I've got this EF while trying to solve another one !! \end{tcolorbox}
Forum usage :

\begin{italicized}Unsolved category\end{underlined}\end{italicized} : "Problems you couldn't solve and \begin{bolded}to which you know that there is a solution\end{underlined}\end{bolded} (i.e. a problem from a contest, etc.) but you don't know it."

\begin{italicized}Proposed and own category\end{underlined}\end{italicized} : "Problems that you have already solved and you are interested in second opinions or solutions."

\begin{italicized}Open category\end{underlined}\end{italicized} : "An open question is a question that has no known solution up to this moment, and it is not known wheter the problem has one or not."
\end{solution}



\begin{solution}[by \href{https://artofproblemsolving.com/community/user/29428}{pco}]
	\begin{tcolorbox}How about when $ f(f(x)+x)=-f(x) $ forall real number $x$ ??\end{tcolorbox}
Infinitely many solutions : $f(x)=g(x)-x$ where $g(x)$ is any involutive function.
\end{solution}



\begin{solution}[by \href{https://artofproblemsolving.com/community/user/138558}{Ali-mes}]
	\begin{tcolorbox}[quote="Ali-mes"]How about when $ f(f(x)+x)=-f(x) $ forall real number $x$ ??\end{tcolorbox}
Infinitely many solutions : $f(x)=g(x)-x$ where $g(x)$ is any involutive function.\end{tcolorbox}

You've misunderstood me, I meant : 
Find all functions from the set of reals to itself such that forall real numbers $x$ and $y$ we have: 

$f(x)^2+x.f(x)+f(y)=f(xf(x)+x^2+y) $ AND $f(f(x)+x)=-f(x)$ .
\end{solution}



\begin{solution}[by \href{https://artofproblemsolving.com/community/user/29428}{pco}]
	\begin{tcolorbox}Find all functions defined from the set of reals to itself, such that: 

$f(x)^2+x.f(x)+f(y)=f(xf(x)+x^2+y) $ for all real numbers $x$ and $y$ .
$f(f(x)+x)=-f(x)$ $\forall x$\end{tcolorbox}
Setting $g(x)=f(x)+x$, the system becomes :

$g(xg(x)+y)=g(x)^2+g(y)$
$g(g(x))=x$

We get immediately from the first $g(0)=0$ and $g(xg(x))=g(x)^2$
Setting $x\to g(x)$ in this last equality, we get $g(x)^2=x^2$

So $\forall x$, either $g(x)=x$, either $g(x)=-x$

Suppose now that $\exists x,y\ne 0$ such that $g(x)=x$ and $g(y)=-y$
First equation implies $g(x^2+y)=x^2-y$ and so :
either $x^2+y=x^2-y$ and so $y=0$, impossible
either $-(x^2+y)=x^2-y$ and so $x=0$ impossible

So either $g(x)=x$ $\forall x$, either $g(x)=-x$ $\forall x$, which both indeed are solutions

\begin{bolded}Hence the two solutions\end{underlined}\end{bolded} :
$f(x)=0$ $\forall x$
$f(x)=-2x$ $\forall x$
\end{solution}
*******************************************************************************
-------------------------------------------------------------------------------

\begin{problem}[Posted by \href{https://artofproblemsolving.com/community/user/109704}{dien9c}]
	Find all function $f:\mathbb{N} \to \mathbb{N}$ such that 
1. $f(1)=1$
2. $f(n)f(n+2)=f(n+1)^2+p$
Where $p$ is odd prime.
	\flushright \href{https://artofproblemsolving.com/community/c6h489590}{(Link to AoPS)}
\end{problem}



\begin{solution}[by \href{https://artofproblemsolving.com/community/user/141397}{subham1729}]
	It's easy to see $\frac {f(n+3)+f(n+1)}{f(n+2)}=f(2)+\frac {p+1}{f(2)}=c$-------$1$
Now $f(2)|p(p+1)$ it's easy to see $GCD (f(2),p)=1$
So $LHS$ of $1$ is an integer.
So, solving for $f(n+1)$ we get $c^2-4$ is a square....so $f(2)=0$....so no function exist.
\end{solution}



\begin{solution}[by \href{https://artofproblemsolving.com/community/user/29428}{pco}]
	\begin{tcolorbox}....so no function exist.\end{tcolorbox}
And what about for example $p=3$ and $f(n)=\left(2+\frac 9{\sqrt{21}}\right)\left(\frac{5-\sqrt{21}}2\right)^n$ $+\left(2-\frac 9{\sqrt{21}}\right)\left(\frac{5+\sqrt{21}}2\right)^n$ which seems to be an existing solution (amongst infinitely many others) ?
\end{solution}



\begin{solution}[by \href{https://artofproblemsolving.com/community/user/152025}{shekast-istadegi}]
	at first we have $gcd(f(n),f(n+1)$ is $1$ or $p$.
case (1)suppose that $f(n)$ dose not divisible by $p$ so we have $gcd(f(n),f(n+1)=1$ and $f(n)\mid f(n+1)^2+p,f(n+1)\mid f(n)^2+p$ and so $f(n)f(n+1)\mid f(n+1)^{3}+pf(n+1)$ and $f(n)f(n+1)\mid f(n)^{3}+pf(n)$ $\Rightarrow f(n)f(n+1)\mid (f(n+1)-f(n))\times (f(n+1)^2+f(n)^2+p)$ we let$ f(n)=a,f(n+1)=b$ we have $gcd(ad,a-b)=1$ then $ab\mid a^2+b^2+p$ but $gcd(a,b)=1$ so its easy to see $p\mid a$ it is contradiction by our impose.
case (2) for all $n$ we have  $p\mid f(n)$ .then we have $2\leq v_{p}(f(n).f(n+2))=v_{p}(f(n+1)^2+p)\leq 1$ this is  contradiction too.so there is no function.
\end{solution}



\begin{solution}[by \href{https://artofproblemsolving.com/community/user/29428}{pco}]
	Excuse me, but i did not fully understand your post.  Do you mean, as subham1729 claimed, that there are no solution ?

If so, just look at my previous post : I gave one solution (it's easy to check that this indeed is a solution) amongst infinitely many other.

In fact, for example $f(1)=f(2)=1$ and $f(n+2)=(p+2)f(n+1)-f(n)$ is basically a solution.
\end{solution}



\begin{solution}[by \href{https://artofproblemsolving.com/community/user/29428}{pco}]
	Instead of checking yourself your proof against my counter-example to your claim, you required me thru pm to check myself. :(
Here is a simple check

My counter example gives :
$p=3$
$f(1)=1$
$f(2)=1$
$f(3)=4$
$f(4)=19$
$f(5)=91$
First, let us check that at least these starting values fit :
$f(1)f(3)=4$ and $f(2)^2+3=4$ and so $f(1)f(3)=f(2)^2+3$
$f(2)f(4)=19$ and $f(3)^2+3=19$ and so $f(2)f(4)=f(3)^2+3$
$f(3)f(5)=364$ and $f(4)^2+3=364$ and so $f(3)f(5)=f(4)^2+3$
This is not a proof that my example works $\forall n$ (which is easy to prove, btw), but this is enough to show the hole in your proof .

let us then have a look on your proof :
For all these values, we get $\gcd(f(n),f(n+1))=1$ and so we are in your first case.
Just then look at your last assertion :

\begin{tcolorbox}...then $ab\mid a^2+b^2+p$ but $gcd(a,b)=1$ so its easy eo see $p\mid a$ ...\end{tcolorbox}
$n=1$ $\implies$ $a=1$ and $b=1$ and indeed $ab|a^2+b^2+3$ and $\gcd(a,b)=1$ but your easy seeing $3|1$ is unfortunately wrong.
$n=2$ $\implies$ $a=1$ and $b=4$ and indeed $ab|a^2+b^2+3$ and $\gcd(a,b)=1$ but your easy seeing $3|1$ is unfortunately wrong.
$n=3$ $\implies$ $a=4$ and $b=19$ and indeed $ab|a^2+b^2+3$ and $\gcd(a,b)=1$ but your easy seeing $3|4$ is unfortunately wrong.
$n=4$ $\implies$ $a=19$ and $b=91$ and indeed $ab|a^2+b^2+3$ and $\gcd(a,b)=1$ but your easy seeing $3|19$ is unfortunately wrong.
...
\end{solution}



\begin{solution}[by \href{https://artofproblemsolving.com/community/user/29428}{pco}]
	\begin{tcolorbox}Find all function $f:\mathbb{N} \to \mathbb{N}$ such that 
1. $f(1)=1$
2. $f(n)f(n+2)=f(n+1)^2+p$
Where $p$ is odd prime.\end{tcolorbox}
Here is general solution :

1) a mandatory form ...
================
Let $f(2)=a$
$f(n)f(n+2)-f(n+1)^2=f(n+1)f(n+3)-f(n+2)^2$ and so $\frac{f(n+3)+f(n+1)}{f(n+2)}=\frac{f(n+2)+f(n)}{f(n+1)}$

So $\frac{f(n+2)+f(n)}{f(n+1)}=c$ $\iff$ $f(n+2)=cf(n+1)-f(n)$ for some $c\in\mathbb Q^+$ and $\forall n\ge 1$

So $f(3)=ac-1$ and since $f(1)f(3)=f(2)^2+p$, we get $ac-1=a^2+p$ and so $c=a+\frac{p+1}a$. Then :

$f(1)=1$
$f(2)=a$
$f(3)=a^2+p$
$f(4)=a^3+2pa+\frac{p(p+1)}a$ and so $a|p(p+1)$
$f(5)=a^4+3pa^2+3p^2+2p+\frac{p(p+1)^2}{a^2}$ and so $a^2|p(p+1)^2$

If $a=kp$ then $a^2|p(p+1)^2$ implies $p|(p+1)^2$, impossible, and so $\gcd(a,p)=1$ and $a|p+1$ and so $c=a+\frac{p+1}a$ is an integer $\ge 2$

So, we got up to now that :
$f(2)=a$ where $a|p+1$ 
$f(n+2)=cf(n+1)-f(n)$ where $c=a+\frac{p+1}a$

Let us now check that this mandatory condition is enough

2) ... which is sufficient
================
Let $p$ prime and $a$ a positive integer such that $a|p+1$ and let $c=a+\frac{p+1}a$, positive integer $\ge 2$
Let the sequence $f(n)$ defined as :
$f(1)=1$ and $f(2)=a$ and $f(n+2)=cf(n+1)-f(n)$ $\forall n\ge 1$

Immediate induction gives $f(n)\in\mathbb Z$ $\forall n$
$f(n+2)-f(n+1)=(c-1)f(n+1)-f(n)\ge f(n+1)-f(n)$ and so an immediate induction gives $f(n+1)-f(n)\ge f(2)-f(1)\ge 0$
So $f(n)$ is non decreasing and $f(n)\in\mathbb N$ $\forall n$

$c=\frac{f(n+3)+f(n+1)}{f(n+2)}=\frac{f(n+2)+f(n)}{f(n+1)}$ and so $f(n+3)f(n+1)-f(n+2)^2=f(n+2)f(n)-f(n+1)^2$

Immediate induction implies then $f(n+2)f(n)-f(n+1)^2=f(3)f(1)-f(2)^2$ $=ca-1-a^2$ $=p$
Hence the result.

3) hence all the solutions
=================
We have exactly as many solutions as positive divisors of $p+1$ (and so at least $3$ : $1,2,p+1$) (note that, for a given $p$, we dont have infinitely many solutions, as I previously claimed :oops:)

Let $a$ any positive divisor of $p+1$
$f(1)=1$ and $f(2)=a$ and $f(n+2)=(a+\frac{p+1}a)f(n+1)-f(n)$ $\forall n\ge 1$

For those who prefer a closed form, we get :
Let $c=a+\frac{p+1}a$

$f(n)=\frac 12\left(c-a+\frac{2-c(c-a)}{\sqrt{c^2-4}}\right)\left(\frac{c+\sqrt{c^2-4}}2\right)^n$ $+\frac 12\left(c-a-\frac{2-c(c-a)}{\sqrt{c^2-4}}\right)\left(\frac{c-\sqrt{c^2-4}}2\right)^n$
\end{solution}
*******************************************************************************
-------------------------------------------------------------------------------

\begin{problem}[Posted by \href{https://artofproblemsolving.com/community/user/142879}{ionbursuc}]
	a)Demonstrate that the function \[f:\mathbb{R}\to \mathbb{R},\,\,f(x)=\left| x \right|\] satisfies relation:
   \[\left| f(x)+y \right|+\left| f(x)-y \right|=\left| x+y \right|+\left| x-y \right|,\left( \forall  \right)x,y\in \mathbb{R}(1)\]                                                     
 b). Determine all functions \[f:\mathbb{R}\to \mathbb{R}\] that satisfy the relation (1).
	\flushright \href{https://artofproblemsolving.com/community/c6h489991}{(Link to AoPS)}
\end{problem}



\begin{solution}[by \href{https://artofproblemsolving.com/community/user/29428}{pco}]
	\begin{tcolorbox}a)Demonstrate that the function \[f:\mathbb{R}\to \mathbb{R},\,\,f(x)=\left| x \right|\] satisfies relation:
$\left| f(x)+y \right|+\left| f(x)-y \right|=\left| x+y \right|+\left| x-y \right|,\left( \forall  \right)x,y\in \mathbb{R}(1)$                                                 \end{tcolorbox}
If $x\ge 0$ then $| f(x)+y|+| f(x)-y |$ $=| x+y|+|x-y |$ 
If $x\le 0$ then $| f(x)+y|+| f(x)-y |$ $=|- x+y|+|-x-y |$ $=| x+y|+|x-y |$ 
Q.E.D.

\begin{tcolorbox}Determine all functions \[f:\mathbb{R}\to \mathbb{R}\] that satisfy the relation (1).\end{tcolorbox}
Just set $y=0$ in the equation and you get $|f(x)|=|x|$ and so $f(x)=e(x)x$ where $e(x)$ is some fonction from $\mathbb R\to\{-1,+1\}$
And it's easy to check that this indeed is a solution whatever is $e(x)$ from $\mathbb R\to\{-1,+1\}$
\end{solution}



\begin{solution}[by \href{https://artofproblemsolving.com/community/user/142879}{ionbursuc}]
	Generalization 1.
Let functions \[h:[0,\infty )\to \mathbb{R},\,\,g:\mathbb{R}\to \mathbb{R}\] with $h$ injective.
 a) Demonstrate that the function \[f:\mathbb{R}\to \mathbb{R},\,\,f(x)=\left| g(x) \right|\] satisfies relation :
   \[h\left( \left| f(x)+y \right| \right)+h\left( \left| f(x)-y \right| \right)=h\left( \left| g(x)+y \right| \right)+h\left( \left| g(x)-y \right| \right),\,\,\,\left( \forall  \right)x,y\in \mathbb{R}. (1)\]
b) Determine all functions \[f:\mathbb{R}\to \mathbb{R}\] that satisfy the relation (1).
\end{solution}



\begin{solution}[by \href{https://artofproblemsolving.com/community/user/142879}{ionbursuc}]
	[hide]Generalization 2.
Let functions \[h:\left[ 0,\infty  \right)\to \mathbb{R},\,\,g:\mathbb{R}\to \mathbb{R}\] with $h$ injective and   \[s:\mathbb{R}\to \mathbb{R},\,\,s(x)=1\ \ \ ,\,\,if\,\ x\ge 0,s(x)=-1,\,\,if\ \ x<0\]
a) Demonstrate that the function \[f:\mathbb{R}\to \mathbb{R},\,\,f(x)=\left| g(x) \right|\] satisfies relation:
   \[\left( 1 \right)\ \ ah\left( \left| f(x)+y \right| \right)+bh\left( \left| f(x)-y \right| \right)=\left( \frac{a+b}{2}+\frac{a-b}{2}s\left( g(x) \right) \right)h\left( \left| g(x)+y \right| \right)+\]
\[+\left( \frac{a+b}{2}+\frac{b-a}{2}s\left( g(x) \right) \right)h\left( \left| g(x)-y \right| \right),\,\left( \forall  \right)x,y\in \mathbb{R}\,,\] 
   where \[a,b\in (0,\infty ).\]
b)Determine all functions \[f:\mathbb{R}\to \mathbb{R}\] that satisfy the relation (1).[\/hide]
\end{solution}



\begin{solution}[by \href{https://artofproblemsolving.com/community/user/29428}{pco}]
	You are welcome, glad to have helped you.

\begin{tcolorbox}Let functions \[h:\left[ 0,\infty  \right)\to \mathbb{R},\,\,g:\mathbb{R}\to \mathbb{R}\] with $h$ injective and   \[s:\mathbb{R}\to \mathbb{R},\,\,s(x)=1\ \ \ ,\,\,if\,\ x\ge 0,s(x)=-1,\,\,if\ \ x<0\]
a) Demonstrate that the function \[f:\mathbb{R}\to \mathbb{R},\,\,f(x)=\left| g(x) \right|\] satisfies relation:
   \[\left( 1 \right)\ \ ah\left( \left| f(x)+y \right| \right)+bh\left( \left| f(x)-y \right| \right)=\left( \frac{a+b}{2}+\frac{a-b}{2}s\left( g(x) \right) \right)h\left( \left| g(x)+y \right| \right)+\]
\[+\left( \frac{a+b}{2}+\frac{b-a}{2}s\left( g(x) \right) \right)h\left( \left| g(x)-y \right| \right),\,\left( \forall  \right)x,y\in \mathbb{R}\,,\] 
   where \[a,b\in (0,\infty ).\]
\end{tcolorbox}
If $g(x)\ge 0$, then $LHS=ah(|g(x)+y|)+bh(|g(x)-y|)$ and $RHS=ah(|g(x)+y|)+bh(|g(x)-y|)$
If $g(x)< 0$, then $LHS=ah(|-g(x)+y|)+bh(|-g(x)-y|)$ and $RHS=bh(|g(x)+y|)+ah(|g(x)-y|)$
Q.E.D.

\begin{tcolorbox}b)Determine all functions \[f:\mathbb{R}\to \mathbb{R}\] that satisfy the relation (1).\end{tcolorbox}
Just set $y=0$ in equation and you get $(a+b)h(|f(x)|)=(a+b)h(|g(x)|)$ and so, since $a+b>0$ and $h(x)$ injective, $|f(x)|=|g(x)|$ and so $f(x)=e(x)g(x)$ for some $e(x)$ from $\mathbb R\to\{-1,+1\}$
And it's easy to check that this indeed is a solution, whatever is $e(x)$ from $\mathbb R\to\{-1,+1\}$


And, btw, according to me, it's more a third repetition than a generalization 2.
Change your teacher.
\end{solution}



\begin{solution}[by \href{https://artofproblemsolving.com/community/user/142879}{ionbursuc}]
	[hide]Generalization 3.
Let functions \[h:\left[ 0,\infty  \right)\to \mathbb{R},\,\,g:\mathbb{R}\to \mathbb{R}\] with $h$ injective and   \[s:\mathbb{R}\to \mathbb{R},\,\,s(x)=1\ \ \ ,\,\,if\,\ x\ge 0,s(x)=-1,\,\,if\ \ x<0\]. Let \[a,b,c,d\in \mathbb{R}\] with \[a+b\ne 0,\,c\ne 0,\,d\ne 0.\]

a) Demonstrate that the function \[f:\mathbb{R}\to \mathbb{R},\,\,f(x)=\left| g(x) \right|\] satisfies relation:
    
   \[\left( 1 \right)ah\left( \left| f(x)+cy \right| \right)+bh\left( \left| f(x)-dy \right| \right)=\left( \frac{a+b}{2}+\frac{a-b}{2}s\left( g(x) \right) \right)\cdot \]
\[h\left( \left| g(x)+y\left( \frac{c+d}{2}+\frac{c-d}{2}s\left( g(x) \right) \right) \right| \right)+\]\[\left( \frac{a+b}{2}+\frac{b-a}{2}s\left( g(x) \right) \right)\cdot \]
\[\cdot h\left( \left| g(x)-y\left( \frac{c+d}{2}+\frac{d-c}{2}s\left( g(x) \right) \right) \right| \right)\,,\left( \forall  \right)\,x,y\in \mathbb{R}.\]
b)Determine all functions \[f:\mathbb{R}\to \mathbb{R}\] that satisfy the relation (1).[\/hide]
\end{solution}



\begin{solution}[by \href{https://artofproblemsolving.com/community/user/29428}{pco}]
	I think that now you should be able to solve your - not very interesting - teacher's problems alone :

1) 
If $g(x)\ge 0$, we get that $LHS=RHS$
If $g(x)<0$, we get that $LHS=RHS$
Q.E.D.

2)
Set $y=0$ and you get, since $a+b\ne 0$ and $h(x)$ injective, that $f(x)=e(x)g(x)$ for some function $e(x)$  from $\mathbb R\to\{-1,+1\}$
And it is easy to check that this indeed is a solution, whatever is $e(x)$  from $\mathbb R\to\{-1,+1\}$

And, btw, according to me, this is more a fourth repetition than a generalization 3.
\end{solution}



\begin{solution}[by \href{https://artofproblemsolving.com/community/user/145671}{gdana}]
	Pco to give too much.
You do not have the courage to answer challenge made here
http://www.artofproblemsolving.com/Forum/viewtopic.php?f=834&t=488500
\end{solution}



\begin{solution}[by \href{https://artofproblemsolving.com/community/user/29428}{pco}]
	\begin{tcolorbox}Pco to give too much.
You do not have the courage to answer challenge made here
http://www.artofproblemsolving.com/Forum/viewtopic.php?f=834&t=488500\end{tcolorbox}
Yes, I know.
I have so few courage :blush:
I'm really sorry.
\end{solution}



\begin{solution}[by \href{https://artofproblemsolving.com/community/user/64716}{mavropnevma}]
	\begin{tcolorbox}Heyyyyy where is PCO's solutionnnnn?\end{tcolorbox}
This is probably what you refer to, \begin{bolded}gdana\end{bolded}. So, poor \begin{bolded}pco\end{bolded}, lacking the courage, you think, to answer that challenge ... To solve a poor, trivial problem, that \begin{bolded}pco\end{bolded} will probably have ten of them for breakfast, and still leave the table hungry ... Very poor opinion you have on the master of functional equations - maybe you're just too new on the forum to have witnessed his many solutions, to most difficult problems. But time is never lost, you can start perusing them now :)
\end{solution}



\begin{solution}[by \href{https://artofproblemsolving.com/community/user/142879}{ionbursuc}]
	Generalization 4                                                  
 b). Let $\varepsilon \ge 0$ and $A=\mathbb{R}-B,B=\left\{ x\in \mathbb{R}|-\varepsilon <x<\varepsilon  \right\}$.Determine all functions 
	\[f:A\to \mathbb{R}\]
 that satisfy the relation \[\left| f(x)+y \right|+\left| f(x)-y \right|=\left| x+y \right|+\left| x-y \right|,\left( \forall  \right)x,y\in A\]
\end{solution}



\begin{solution}[by \href{https://artofproblemsolving.com/community/user/64716}{mavropnevma}]
	It follows, in the same style as above, that $f(x) = e(x)x$ for $|x| > \varepsilon$ and arbitrary $e: \{x \in \mathbb{R} \mid |x| > \varepsilon\}\to \{-1,1\}$, and arbitrary $f(\varepsilon), f(-\varepsilon) \in [-\varepsilon,\varepsilon]$.
This is because taking $y=x\in A$ we have $|f(x)+x| + |f(x)-x| = 2|x|$; now we immediately get that if $f(x) \in A$ then taking $y=f(x)$ we obtain $2|f(x)| = |x+f(x)| + |x-f(x)| = 2|x|$, so $|f(x)| = |x|$. 
If $f(x)\not \in A$ and $|x| > \varepsilon$, that means $|f(x)| < \varepsilon$; assume $x > \varepsilon$ and now taking $\varepsilon \leq y < x$ we get $|f(x)+y| + |f(x)-y| = 2y$, but $|x+y| + |x-y| = 2x$, thus $y=x$, absurd. Similarly if $x < -\varepsilon$.
So it only remains the case when $f(x)\not \in A$ and $|x| = \varepsilon$, and it is seen that in this case we may take $f(\pm \varepsilon)$ arbitrarily in  $[-\varepsilon,\varepsilon]$.
\end{solution}
*******************************************************************************
-------------------------------------------------------------------------------

\begin{problem}[Posted by \href{https://artofproblemsolving.com/community/user/68025}{Pirkuliyev Rovsen}]
	Determine all functions $f: \mathbb[-1;1]\to\mathbb[-1;1]$ such that for all $x\in[-1;1]$
$1)3x+2f(x)\in[-1;1] ,2)f(3x+2f(x))=-x$.


___________________________________________
Azerbaijan Land of the Fire 
	\flushright \href{https://artofproblemsolving.com/community/c6h490124}{(Link to AoPS)}
\end{problem}



\begin{solution}[by \href{https://artofproblemsolving.com/community/user/86443}{roza2010}]
	are conditions 1) and 2) independent, or not ?
\end{solution}



\begin{solution}[by \href{https://artofproblemsolving.com/community/user/29428}{pco}]
	\begin{tcolorbox}Determine all functions $f: \mathbb[-1;1]\to\mathbb[-1;1]$ such that for all $x\in[-1;1]$
$1)3x+2f(x)\in[-1;1] ,2)f(3x+2f(x))=-x$.\end{tcolorbox}
First condition implies $f(x)\in[1-a-ax,a-1-ax]$ with $a=\frac 32>1$

Then second condition implies $-x\in[1-a-a(3x+2f(x)),a-1-a(3x+2f(x))]$ and so $f(x)\in[1-b-bx,b-1-bx]$ with $b=\frac {3a-1}{2a}>1$

And so $f(x)\in[1-a_n-a_nx,a_n-1-a_nx]$ for all $a_n$ of the sequence $a_1=\frac 32$ and $a_{n+1}=\frac {3a_n-1}{2a_n}$

It's easy to see that $a_n$ is a decreasing positive sequence whose limit is $1$

And so $f(x)\in[-x,-x]$ and $\boxed{f(x)=-x}$ which indeed is a solution.
\end{solution}
*******************************************************************************
-------------------------------------------------------------------------------

\begin{problem}[Posted by \href{https://artofproblemsolving.com/community/user/114608}{MLDV}]
	Find all functions $f : \mathbb{R} \to \mathbb{R}$ such that $(x-y)f(z)+(y-z)f(x)+(z-x)f(y) = 0$ for all $x,y,z\in\mathbb{R}$, that satisfy $x+y+z=0.$
	\flushright \href{https://artofproblemsolving.com/community/c6h490152}{(Link to AoPS)}
\end{problem}



\begin{solution}[by \href{https://artofproblemsolving.com/community/user/29428}{pco}]
	\begin{tcolorbox}Find all functions $f : \mathbb{R} \to \mathbb{R}$ such that $(x-y)f(z)+(y-z)f(x)+(z-x)f(y) = 0$ for all $x,y,z\in\mathbb{R}$, that satisfy $x+y+z=0.$\end{tcolorbox}
$f(x)$ solution implies $f(x)-ax^3-bx-c$ solution and so WLOG say $f(0)=f(1)=f(2)=0$

Let $P(x,y)$ be the equivalent (just write $z=-x-y$) assertion $(x-y)f(-x-y)+(x+2y)f(x)-(2x+y)f(y)=0$ $\forall x,y$

$P(x,0)$ $\implies$ $x(f(-x)+f(x))=0$ and so $f(-x)=-f(x)$ $\forall x\ne 0$, still true when $x=0$

$P(x,y)$ $\implies$ $(x-y)f(x+y)=(x+2y)f(x)-(2x+y)f(y)$

$P(x,2)$ $\implies$ $(x-2)f(x+2)=(x+4)f(x)$ and so $f(x+2)=\frac{x+4}{x-2}f(x)$ $\forall x\ne 2$
$P(x,1)$ $\implies$ $(x-1)f(x+1)=(x+2)f(x)$
$P(x+1,1)$ $\implies$ $xf(x+2)=(x+3)f(x+1)$ and so $f(x+2)=\frac{(x+3)(x+2)}{x(x-1)}f(x)$ $\forall x\notin\{0,1\}$

And so $f(x)\left(\frac{x+4}{x-2}-\frac{(x+3)(x+2)}{x(x-1)}\right)=0$ $\forall x\notin\{0,1,2\}$

Which becomes $\frac{12f(x)}{x(x-1)(x-2)}=0$ $\forall x\notin\{0,1,2\}$ and so $f(x)=0$ $\forall x$ which indeed is a solution.

Hence the answer : $\boxed{f(x)=ax^3+bx+c}$ $\forall x$ and whatever are $a,b,c\in\mathbb R$
\end{solution}
*******************************************************************************
-------------------------------------------------------------------------------

\begin{problem}[Posted by \href{https://artofproblemsolving.com/community/user/139608}{ArianitZeqiri}]
	if \[f:\mathbb{R}^+\rightarrow \mathbb{R}\] satisfies :\[f(xf(y))=yf(x)\]
for all \[x,y\epsilon \mathbb{R}^+\].
	\flushright \href{https://artofproblemsolving.com/community/c6h490565}{(Link to AoPS)}
\end{problem}



\begin{solution}[by \href{https://artofproblemsolving.com/community/user/29428}{pco}]
	\begin{tcolorbox}if \[f:\mathbb{R}^+\rightarrow \mathbb{R}\] satisfies :\[f(xf(y))=yf(x)\]
for all \[x,y\epsilon \mathbb{R}^+\].\end{tcolorbox}
Let $P(x,y)$ be the assertion $f(xf(y))=yf(x)$

If $f(1)=0$, then :
$P(1,x)$ $\implies$ $f(f(x))=0$ and so $f(0)=0$
$P(x,1)$ $\implies$ $f(x)=f(0)=0$ $\forall x$ which indeed is a solution

If $f(1)\ne 0$, then :
$P(1,x)$ $\implies$ $f(f(x))=xf(1)$ and $f(x)$ is an injection.
$P(1,1)$ $\implies$ $f(f(1))=f(1)$ and so, since injective, $f(1)=1$ and so $f(f(x))=x$
$P(x,f(y)$ $\implies$ $f(xy)=f(x)f(y)$

And so we just have to find involutive solutions of $f(xy)=f(x)f(y)$ with $f(1)=1$

This is a classical problem whose solution is $f(x)=e^{g(\ln x)}$ where $g(x)$ any involutive solution of additive Cauchy equation.
\end{solution}



\begin{solution}[by \href{https://artofproblemsolving.com/community/user/112260}{robinson123}]
	\begin{tcolorbox}[quote="ArianitZeqiri"]if \[f:\mathbb{R}^+\rightarrow \mathbb{R}\] satisfies :\[f(xf(y))=yf(x)\]
for all \[x,y\epsilon \mathbb{R}^+\].\end{tcolorbox}
Let $P(x,y)$ be the assertion $f(xf(y))=yf(x)$

If $f(1)=0$, then :
$P(1,x)$ $\implies$ $f(f(x))=0$ and so $f(0)=0$
$P(x,1)$ $\implies$ $f(x)=f(0)=0$ $\forall x$ which indeed is a solution

If $f(1)\ne 0$, then :
$P(1,x)$ $\implies$ $f(f(x))=xf(1)$ and $f(x)$ is an injection.
$P(1,1)$ $\implies$ $f(f(1))=f(1)$ and so, since injective, $f(1)=1$ and so $f(f(x))=x$
$P(x,f(y)$ $\implies$ $f(xy)=f(x)f(y)$

And so we just have to find involutive solutions of $f(xy)=f(x)f(y)$ with $f(1)=1$

This is a classical problem whose solution is $f(x)=e^{g(\ln x)}$ where $g(x)$ any involutive solution of additive Cauchy equation.\end{tcolorbox}

I have a question, if  some $ y $ such that $ f(y)<0 $ then the function is not definite?
\end{solution}



\begin{solution}[by \href{https://artofproblemsolving.com/community/user/89198}{chaotic_iak}]
	Yes. $f(x) > 0$ for all $x$, otherwise plugging $y$ to be the offending value ($f(y) \le 0$), we get an undefined value on LHS.

pco made a mistake as $f$ is defined from positive reals and hence it's not allowed to define $f(0)$. Also it means $f(x) = 0$ is not a solution (LHS becomes undefined).
\end{solution}



\begin{solution}[by \href{https://artofproblemsolving.com/community/user/29428}{pco}]
	\begin{tcolorbox}Yes. $f(x) > 0$ for all $x$, otherwise plugging $y$ to be the offending value ($f(y) \le 0$), we get an undefined value on LHS.

pco made a mistake as $f$ is defined from positive reals and hence it's not allowed to define $f(0)$. Also it means $f(x) = 0$ is not a solution (LHS becomes undefined).\end{tcolorbox}
Quite quite exact. :oops:

Hence the unique solution $f(x)=e^{g(\ln x)}$ $\forall x>0$ where $g(x)$ is any involutive solution of additive Cauchy equation.

Thanks for the correction
\end{solution}
*******************************************************************************
-------------------------------------------------------------------------------

\begin{problem}[Posted by \href{https://artofproblemsolving.com/community/user/78444}{Babai}]
	Find all functions $f: [0,\infty) \rightarrow [0,\infty $ such that $f\left (\frac{x+f(x)}{2}+y\right )=2x-f(x)+f(f(y))$.
	\flushright \href{https://artofproblemsolving.com/community/c6h491257}{(Link to AoPS)}
\end{problem}



\begin{solution}[by \href{https://artofproblemsolving.com/community/user/103150}{Djurre}]
	Is $ f: R^{+}U{{0}}\rightarrow R^{+}U{{0}} $ the same as $f:\mathbb{R}_{>0}\to\mathbb{R}_{>0}$?
\end{solution}



\begin{solution}[by \href{https://artofproblemsolving.com/community/user/29428}{pco}]
	\begin{tcolorbox}Find all functions from $f: R^{+}U{{0}}\rightarrow R^{+}U{{0}}$ such that $f(\frac{x+f(x)}{2}+y)=2x-f(x)+f(f(y))$\end{tcolorbox}
Let $P(x,y)$ be the assertion $f(\frac{x+f(x)}2+y)=2x-f(x)+f(f(y))$, true $\forall, x,y\ge 0$
Let $a=f(0)$
Let $g(x)=\frac{x+f(x)-a}2$ and $h(x)=2x+a-f(x)$

$P(0,x)$ $\implies$ $f(\frac a2+x)=-a+f(f(x))$
So $P(x,y)$ implies new assertion $Q(x,y)$ : $f(\frac{x+f(x)}2+y)=2x-f(x)+a+f(y+\frac a2)$, true $\forall, x,y\ge 0$

Let $t\ge 3a$ : $Q(\frac{t-a}2,\frac{t-2a}2)$ $\implies$  $f(\frac{3t-5a}4+\frac{f(\frac{t-a}2)}2)=t$ and $\frac{3t-5a}4\ge a\ge\frac a2$
And so $\forall t\ge 3a$, $\exists x\ge \frac a2$ such that $f(x)=t$

Let $x\ge \frac a2$ : 
$Q(y,x-\frac a2)$ $\implies$ $f(x+g(y))=f(x)+h(y)$
Notice that $y\ge a$ implies $g(y)\ge 0$ and so $f(x+ng(y))=f(x)+nh(y)$ and so $h(y)\ge 0$
Comparing then $P(0,x)$ and $P(0,x+g(y))$, we get $f(f(x)+h(y))=f(f(x))+h(y)$ and so $f(x+h(y))=f(x)+h(y)$ $\forall x\ge 3a$

So $f(x+g(y))=f(x+h(y))$ $\forall x\ge 3a,\forall y\ge 0$
So $f(x+|g(y)-h(y)|)=f(x)$ $\forall x\ge 3a+\min(g(y),h(y))$, $\forall y\ge a$

For $x$ great enough, comparing then $P(x,0)$ and $P(x+2|g(y)-h(y)|,0)$, we get $g(y)=h(y)$ $\forall y\ge a$ and so $f(x)=x+a$ $\forall x\ge a$

Plugging this in $P(0,x)$, we get $a=0$ and so $\boxed{f(x)=x}$ $\forall x\ge a=0$, which indeed is a solution.
\end{solution}
*******************************************************************************
-------------------------------------------------------------------------------

\begin{problem}[Posted by \href{https://artofproblemsolving.com/community/user/92753}{WakeUp}]
	Let $k>1$ be an integer. A function $f:\mathbb{N^*}\to\mathbb{N^*}$ is called $k$-\begin{italicized}tastrophic\end{italicized} when for every integer $n>0$, we have $f_k(n)=n^k$ where $f_k$ is the $k$-th iteration of $f$:
\[f_k(n)=\underbrace{f\circ f\circ\cdots \circ f}_{k\text{ times}}(n)\]
For which $k$ does there exist a $k$-tastrophic function?
	\flushright \href{https://artofproblemsolving.com/community/c6h491800}{(Link to AoPS)}
\end{problem}



\begin{solution}[by \href{https://artofproblemsolving.com/community/user/29428}{pco}]
	\begin{tcolorbox}Let $k>1$ be an integer. A function $f:\mathbb{N^*}\to\mathbb{N^*}$ is called $k$-\begin{italicized}tastrophic\end{italicized} when for every integer $n>0$, we have $f_k(n)=n^k$ where $f_k$ is the $k$-th iteration of $f$:
\[f_k(n)=\underbrace{f\circ f\circ\cdots \circ f}_{k\text{ times}}(n)\]
For which $k$ does there exist a $k$-tastrophic function?\end{tcolorbox}
I suppose that $N^*=\mathbb N$ is the set of all positive integers.

There exists a k-tastrophic function for any integer $k>1$.
Unlike the funny claim in previous post, here is the proof :

Let integer $k>1$
Let $v_k(n)$ be the greatest power of $k$ which divides $n$

Let $A$ be the set of all positive integers which are not divisible by $k$.
Since $k>1$, $A$ is an infinite set and so we may split it in $k$ equinumerous subsets $A_1,A_2,...A_k$
Let $\{g_i(n)\}_{i=1}^{k-1}$ a set of bijections from $A_i\to A_{i+1}$
Let $h(n)$ bijection from $A_k\to A_1$ defined as $h(n)=g_{1}^{-1}(g_{2}^{-1}(...g_{k-1}^{-1}(n)...))$

Let $g(x)$ from $\mathbb N\to\mathbb N$ defined as :
If $\frac n{k^{v_k(n)}}\in A_i$ with $i\ne k$, then $g(n)=k^{v_k(n)}g_i\left (\frac n{k^{v_k(n)}}\right )$
If $\frac n{k^{v_k(n)}}\in A_k$, then $g(n)=k^{v_k(n)+1}h\left (\frac n{k^{v_k(n)}}\right )$

Then $g_k(n)=kn$

And then just define $f(n)$ as :
$f(1)=1$
For $n>1$ : $f(n)=$ $f\left(\prod p_i^{n_i}\right)=\prod p_i^{g(n_i)}$ where $\prod p_i^{n_i}$ is the prime decomposition of $n$

Q.E.D.
\end{solution}
*******************************************************************************
-------------------------------------------------------------------------------

\begin{problem}[Posted by \href{https://artofproblemsolving.com/community/user/150451}{Algie}]
	Does there exist a function $ f :\mathbb{R}\to\mathbb{R} $ , satisfying:
\begin{italicized}a)\end{italicized} $ f(x+f(y))\geq x+yf(x) $ for all $ x,y\in\mathbb{R}? $
\begin{italicized}b)\end{italicized} $ f(x+f(y))\leq x+yf(x) $ for all $ x,y\in\mathbb{R}? $
	\flushright \href{https://artofproblemsolving.com/community/c6h494261}{(Link to AoPS)}
\end{problem}



\begin{solution}[by \href{https://artofproblemsolving.com/community/user/62176}{mathe760}]
	Hello, I am not sure, but perhaps one could apply some of the ideas used here: [url]http://www.artofproblemsolving.com/Forum/viewtopic.php?f=36&t=348176&p=1867661&hilit=functional+inequality#p1867661[\/url] ?
\end{solution}



\begin{solution}[by \href{https://artofproblemsolving.com/community/user/29428}{pco}]
	\begin{tcolorbox}Does there exist a function $ f :\mathbb{R}\to\mathbb{R} $ , satisfying:
\begin{italicized}a)\end{italicized} $ f(x+f(y))\geq x+yf(x) $ for all $ x,y\in\mathbb{R}? $\end{tcolorbox}
Yes : choose for example $f(x)=e^x$ 

The inequation is then equivalent to $e^{x+e^y}\ge x+ye^x$

$\iff$ $e^x\left(e^{e^y}-y\right)\ge x$

$\iff$ $e^{e^y}-y\ge xe^{-x}$ which is true since :

$e^{e^y}-y\ge e^y+1-y\ge y+1+1-y=2$ $\forall y$

$xe^{-x}\le \frac 1e<2$ $\forall x$ (this function has a global maximum $\frac 1e$ when $x=1$)
\end{solution}



\begin{solution}[by \href{https://artofproblemsolving.com/community/user/10045}{socrates}]
	\begin{tcolorbox}Does there exist a function $ f :\mathbb{R}\to\mathbb{R} $ , satisfying:
\begin{italicized}b)\end{italicized} $ f(x+f(y))\leq x+yf(x) $ for all $ x,y\in\mathbb{R}? $\end{tcolorbox}


\end{solution}
*******************************************************************************
-------------------------------------------------------------------------------

\begin{problem}[Posted by \href{https://artofproblemsolving.com/community/user/153828}{mortezaMashini}]
	Determine all functions  $f : \mathbb{N} \rightarrow \mathbb{N} $ such that  $f(f(n))=n+k$.

$k \in  \mathbb{N}$
	\flushright \href{https://artofproblemsolving.com/community/c6h495097}{(Link to AoPS)}
\end{problem}



\begin{solution}[by \href{https://artofproblemsolving.com/community/user/153828}{mortezaMashini}]
	i think the only answer is  $f(n)=n+\frac{k}{2}$  and $k$ is an even number.

i start like this...$f$ is an one-to-one function. since if $f(m)=f(n)$ then 

                                                                                    $f(f(m))=f(f(n)) \Rightarrow m+k=n+k \Rightarrow m=n$

now we can write

                                                                    $f(n)+k=f(f(f(n)))=f(n+k) \Rightarrow f(n+k)=f(n)+k$

im stuck here.actually i did nothing..!!
\end{solution}



\begin{solution}[by \href{https://artofproblemsolving.com/community/user/127783}{Sayan}]
	See here
http://www.artofproblemsolving.com/Forum/viewtopic.php?highlight=1987&t=2749
\end{solution}



\begin{solution}[by \href{https://artofproblemsolving.com/community/user/29428}{pco}]
	\begin{tcolorbox}Determine all functions  $f : \mathbb{N} \rightarrow \mathbb{N} $ such that  $f(f(n))=n+k$.

$k \in  \mathbb{N}$\end{tcolorbox}
Let $A=\{1,2,...k\}\cap f(\mathbb N)$ and $B=\{1,2,...k\}\setminus A$
$f$ is injective and $f(B)=A$ and so $|A|=|B|=\frac k2$ and so $k$ must be even.

And then any bijection from $A\to B$ gives a solution and we get the general solution :

Let $A,B$ a split of $\{1,2,...,k\}$ in two equinumerous sets and let $g(x)$ any bijection from $A\to B$, then define $f(x)$ as :

Let $n-1=u(n)k+v(n)-1$ be the euclidian division of $n-1$ by $k$ where $v(n)\in\{1,2,...,k\}$
If $v(n)\in A$ : $f(n)=u(n)k+g(v(n))$
If $v(n)\in B$ : $f(n)=(u(n)+1)k+g^{-1}(v(n))$

And so $\frac{k!}{\left(\frac k2\right)!}$ such functions whenever $k$ is even.
\end{solution}
*******************************************************************************
-------------------------------------------------------------------------------

\begin{problem}[Posted by \href{https://artofproblemsolving.com/community/user/78444}{Babai}]
	Find All function from $N$ to $N_{0}$ such that $f(1)=0$ and $f(mn^2)=f(mn)+f(n)$ for m,n natural numbers.

[See it's easy to guess the function but i can't prove this  is the only one.PCO pls help]
	\flushright \href{https://artofproblemsolving.com/community/c6h495099}{(Link to AoPS)}
\end{problem}



\begin{solution}[by \href{https://artofproblemsolving.com/community/user/29428}{pco}]
	\begin{tcolorbox}Find All function from $N$ to $N_{0}$ such that $f(1)=0$ and $f(mn^2)=f(mn)+f(n)$ for m,n natural numbers.

[See it's easy to guess the function but i can't prove this  is the only one.PCO pls help]\end{tcolorbox}
Let $P(x,y)$ be the assertion $f(xy^2)=f(xy)+f(y)$ true $\forall x,y\in\mathbb N$

(a) : $P(1,xy)$ $\implies$ $f(x^2y^2)=2f(xy)$
(b) : $P(x^2,y)$ $\implies$ $f(x^2y^2)=f(x^2y)+f(y)$ 
(c) : $P(y,x)$ $\implies$ $f(x^2y)=f(xy)+f(x)$

(a)-(b)-(c) :  $f(xy)-f(x)-f(y)=0$ which is obviously also a sufficient condition.

So the problem is equivalent to $f(1)=0$ and $f(mn)=f(m)+f(n)$ which is classical and has infinitely many solutions :

Choose as you want the image $f(p_i)=q_i$ for all prime numbers $p_i$ with any $q_i\in\mathbb N_0$ and then :

$f(1)=0$
$f(\prod p_i^{n_i})=\sum n_iq_i$
\end{solution}



\begin{solution}[by \href{https://artofproblemsolving.com/community/user/114585}{anonymouslonely}]
	steps:
1. $ f(pq)=f(p)+f(q) $ for $ p,q $ primes 
3. $ f(p_{1}^{a_{1}}...p_{t}^{a_{t}})=a_{1}f(p_{1})+...+a_{t}f(p_{t}) $ (induction on $ s=a_{1}+...+a_{t} $) where $ p_{i} $ are primes.
\end{solution}
*******************************************************************************
-------------------------------------------------------------------------------

\begin{problem}[Posted by \href{https://artofproblemsolving.com/community/user/300168}{finn123}]
	Can anyone give me documents or maybe solutions or this problem: 
Find all $f:\mathbb{R}\rightarrow\mathbb{R}$ with f is a continuous function and $ f(f(x) = a. f(x) + b.x$ $\forall$ $x\in\mathbb{R}$ and a,b be given numbers.
I think this equation has many satisfy functions depend on a,b so please help me with this.
	\flushright \href{https://artofproblemsolving.com/community/c6h1344617}{(Link to AoPS)}
\end{problem}



\begin{solution}[by \href{https://artofproblemsolving.com/community/user/29428}{pco}]
	\begin{tcolorbox}Can anyone give me documents or maybe solutions or this problem: 
Find all $f:\mathbb{R}\rightarrow\mathbb{R}$ with f is a continuous function and $ f(f(x) = a. f(x) + b.x$ $\forall$ $x\in\mathbb{R}$ and a,b be given numbers.
I think this equation has many satisfy functions depend on a,b so please help me with this.\end{tcolorbox}
I hope that if you encounter a problem "solve in $\mathbb Z$ equation $x+17=3$" and another problem "solve in $\mathbb R$ equation $x^2+2x+1=0$", you will not ask in the forum, claiming that it is a real olympiad exercise : "solve in $\mathbb C$ equation $\sum_{k=0}^na_kx^k$ where $a_i\in\mathbb C$"

Here is a rather quick answer to your general question. It has been written first for problem :
"Find continuous function $f:\mathbb{R} \to \mathbb{R}$ such that
$a f(f(x)) = b f(x) + cx$"

So you can skip cases $1,2,3$ (dealing with case $a=0$ and go directly to case $4$.

\begin{bolded}Some personal comments\end{underlined}\end{bolded} :
I did not check all the typos (too tiring) and I hope you'll excuse them
I dont claim that all my proofs are the shortest.
This post is too long to be posted in one unique post. So I'll try to post in multiple posts.

Maybe some subcases could have been merged.
Subcase 7.2 is not finished. It seems to me that it is one of the most complex and it has infinitely many solutions.

I hope you'll read this post line per line.

\begin{bolded}Part one of two\end{underlined}\end{bolded}

\begin{bolded}Case 1\end{underlined}\end{bolded} : If $a=b=c=0$ : any continuous function is solution

\begin{bolded}Case 2\end{underlined}\end{bolded} : If $a=b=0$ and $c\ne 0$ : no solution

\begin{bolded}Case 3\end{underlined} \end{bolded}: If $a=0$ and $b\ne 0$ : unique solution is $f(x)=-\frac cbx$ $\forall x$

\begin{bolded}Case 4\end{underlined}\end{bolded} : If $a\ne 0$ and $b=c=0$, equation is $f(f(x))=0$ and so $f(x)=0$ $\forall x\in f(\mathbb R)$ and, since continuous, $f(\mathbb R)$ is an interval and so : 

\begin{bolded}Subcase 4.1\end{bolded} : $f(\mathbb R)=\mathbb R$ and the unique solution $f(x)=0$ $\forall x$

\begin{bolded}Subcase 4.2\end{bolded} : $f(\mathbb R)=[a,+\infty)$. with $a\le 0$ (since obviously $0\in f(\mathbb R)$)
Then we get infinitely many solutions :
Let $a\le 0$
Let $h(x)$ any continuous surjection from $(-\infty,a]to[a,+\infty)$ such that $h(a)=0$
Then $f(x)=h(x)$ $\forall x<a$ and $f(x)=0$ $\forall x\ge a$

\begin{bolded}Subcase 4.3\end{bolded} : $f(\mathbb R)=(a,+\infty)$. with $a< 0$ (since obviously $0\in f(\mathbb R)$
Then we get infinitely many solutions :
Let $a<0$
Let $h(x)$ any continuous surjection from $(-\infty,a]to(a,+\infty)$ such that $h(a)=0$
Then $f(x)=h(x)$ $\forall x<a$ and $f(x)=0$ $\forall x\ge a$

\begin{bolded}Subcase 4.4 \end{bolded}: $f(\mathbb R)=(-\infty,a]$. with $a\ge 0$ (since obviously $0\in f(\mathbb R)$
Then we get infinitely many solutions :
Let $a\ge 0$
Let $h(x)$ any continuous surjection from $[a,+\infty)to(-\infty,a]$ such that $h(a)=0$
Then $f(x)=h(x)$ $\forall x>a$ and $f(x)=0$ $\forall x\le a$

\begin{bolded}Subcase 4.5\end{bolded} : $f(\mathbb R)=(-\infty,a)$. with $a>0$ (since obviously $0\in f(\mathbb R)$
Then we get infinitely many solutions :
Let $a>0$
Let $h(x)$ any continuous surjection from $[a,+\infty)to(-\infty,a)$ such that $h(a)=0$
Then $f(x)=h(x)$ $\forall x>a$ and $f(x)=0$ $\forall x\le a$

\begin{bolded}Subcase 4.6\end{bolded} : $f(\mathbb R)=[a,b]$ with $a\le 0\le b$ (since obviously $0\in f(\mathbb R)$
Then we get infinitely many solutions :
Let $a\le 0\le b$
Let $h(x)$ any continuous function from $\mathbb R\to\mathbb R$ such that $h(\mathbb R\setminus(a,b))=[a,b]$ and $h(a)=h(b)=0$.
Then $f(x)=h(x)$ $\forall x\notin[a,b]$ and $f(x)=0$ $\forall x\in[a,b]$

\begin{bolded}Subcase 4.7\end{bolded} : $f(\mathbb R)=[a,b)$ with $a\le 0< b$ (since obviously $0\in f(\mathbb R)$
Then we get infinitely many solutions :
Let $a\le 0< b$
Let $h(x)$ any continuous function from $\mathbb R\to\mathbb R$ such that $hf(\mathbb R\setminus(a,b))=[a,b)$ and $h(a)=h(b)=0$.
Then $f(x)=h(x)$ $\forall x\notin[a,b]$ and $f(x)=0$ $\forall x\in[a,b]$

\begin{bolded}Subcase 4.8\end{bolded} : $f(\mathbb R)=(a,b]$ with $a< 0\le b$ (since obviously $0\in f(\mathbb R)$
Then we get infinitely many solutions :
Let $a< 0\le b$
Let $h(x)$ any continuous function from $\mathbb R\to\mathbb R$ such that $h(\mathbb R\setminus(a,b))=(a,b]$ and $h(a)=h(b)=0$.
Then $f(x)=h(x)$ $\forall x\notin[a,b]$ and $f(x)=0$ $\forall x\in[a,b]$

\begin{bolded}Subcase 4.9\end{bolded} : $f(\mathbb R)=(a,b)$ with $a< 0< b$ (since obviously $0\in f(\mathbb R)$
Then we get infinitely many solutions :
Let $a< 0< b$
Let $h(x)$ any continuous function from $\mathbb R\to\mathbb R$ such that $h(\mathbb R\setminus(a,b))=(a,b)$ and $h(a)=h(b)=0$.
Then $f(x)=h(x)$ $\forall x\notin[a,b]$ and $f(x)=0$ $\forall x\in[a,b]$

\begin{bolded}Case 5\end{underlined}\end{bolded} : If $a\ne 0$ and $b=0$ and $c\ne 0$, equation is $f(f(x))=tx$ with $t=\frac ca\ne 0$
So $f(x)$ is bijective and so, since continuous, monotonous.
So $f(f(x))$ is an increasing function.

\begin{bolded}Subcase 5.1\end{bolded} : If $t<0$ ($\iff ac<0$) then  no solution since $LHS$ is increasing while RHS is decreasing

\begin{bolded}Subcase 5.2\end{bolded} : if $1>t>0$ ($\iff$ $a>c>0$ or $a<c<0$), this is a classical equation with infinitely many solutions which may be built piece par piece.

5.2.1) increasing solutions\end{underlined} :
$\forall x>0$ :
Let $a\in(t,1)$
Let $h_1(x)$ any continuous increasing bijection from $[a,1]\to[t,a]$
Define $f(x)$ as :
$\forall x\in(a,1]$ : $f(x)=h_1(x)$
$\forall x\in(t,a]$ : $f(x)=th_1^{[-1]}(x)$
$\forall x\in(0,t]\cup(1,+\infty)$ : $f(x)=t^{\lfloor\log_t x\rfloor}f(xt^{-\lfloor\log_t x\rfloor})$

$f(0)=0$

$\forall x<0$
Let $b\in(-1,-t)$
Let $h_2(x)$ any continuous increasing bijection from $[-1,b]\to[b,-t]$
Define $f(x)$ as :
$\forall x\in[-1,b)$ : $f(x)=h_2(x)$
$\forall x\in[b,-t)$ : $f(x)=th_2^{[-1]}(x)$
$\forall x\in(-\infty,-1)\cup[-t,0)$ : $f(x)=t^{\lfloor\log_t -x\rfloor}f(xt^{-\lfloor\log_t -x\rfloor})$

5.2.2) decreasing solutions\end{underlined} :
Let $a<0$
Let $h(x)$ be any continuous decreasing bijection from $[t,1]\to[a,ta]$
$\forall x\ge 0$ : $f(x)=t^{\lfloor\log_t x\rfloor}h(xt^{-\lfloor\log_t x\rfloor})$
$\forall x<0$ : $f(x)=th^{[-1]}(x)$

\begin{bolded}Subcase 5.3\end{bolded} : If $t=1$ ($\iff a=c$), then equation is $f(f(x))=x$ and is very classical :

5.3.1) increasing solutions\end{underlined}
It's well known that equation $f(f(x))=x$ has a unique continuous increasing solution $f(x)=x$ $\forall x$

5.3.2) decreasing solutions\end{underlined}
A general form is :
Let $h(x)$ be any continuous decreasing bijection from $[0,+\infty)\to(-\infty,0]$
$\forall x\ge 0$ : $f(x)=h(x)$
$\forall x<0$ : $f(x)=h^{[-1]}(x)$

\begin{bolded}Subcase 5.4\end{bolded} : if $t>1$ ($\iff$ $c>a>0$ or $c<a<0$), this is a classical equation with infinitely many solutions which may be built piece par piece.

5.4.1) increasing solutions\end{underlined} :
$\forall x>0$ :
Let $a\in(1,t)$
Let $h_1(x)$ any continuous increasing bijection from $[1,a]\to[a,t]$
Define $f(x)$ as :
$\forall x\in[1,a)$ : $f(x)=h_1(x)$
$\forall x\in[a,t)$ : $f(x)=th_1^{[-1]}(x)$
$\forall x\in(0,1)\cup[t,+\infty)$ : $f(x)=t^{\lfloor\log_t x\rfloor}f(xt^{-\lfloor\log_t x\rfloor})$

$f(0)=0$

$\forall x<0$
Let $b\in(-t,-1)$
Let $h_2(x)$ any continuous increasing bijection from $[b,-1]\to[-t,b]$
Define $f(x)$ as :
$\forall x\in(b,-1]$ : $f(x)=h_2(x)$
$\forall x\in(-t,b]$ : $f(x)=th_2^{[-1]}(x)$
$\forall x\in(-\infty,-t]\cup(-1,0)$ : $f(x)=t^{\lfloor\log_t -x\rfloor}f(xt^{-\lfloor\log_t -x\rfloor})$

5.4.2) decreasing solutions\end{underlined} :
Let $a<0$
Let $h(x)$ be any continuous decreasing bijection from $[1,t]\to[ta,a]$
$\forall x\ge 0$ : $f(x)=t^{\lfloor\log_t x\rfloor}h(xt^{-\lfloor\log_t x\rfloor})$
$\forall x<0$ : $f(x)=th^{[-1]}(x)$

\begin{bolded}Case 6\end{underlined}\end{bolded} : If $a\ne 0$ and $b=\ne 0$ and $c=0$, equation is $f(f(x))=tf(x)$ with $t=\frac ba\ne 0$
So $f(x)=tx$ $\forall x\in f(\mathbb R)$ 
Since continuous, $f(\mathbb R)$ is an interval and so :

\begin{bolded}Subcase 6.1\end{bolded} : if $f(\mathbb R)=\mathbb R$, we get $f(x)=tx$ $\forall x$ which indeed is a solution

\begin{bolded}Subcase 6.2\end{bolded} : if $f(\mathbb R)=[a,+\infty)$ 
This means $ta\ge a$ (since $f(a)=ta\in f(\mathbb R)$) and the solutions :

\begin{bolded}Subcase 6.2.1\end{bolded} : if $t>1$, then $a\ge 0$ and :
Let $a\ge 0$
Let $h(x)$ be any continuous function from $(-\infty,a]\to[a,+\infty)$ such that $h(a)=ta$ and $[a,ta]\subseteq h((-\infty,a])$ then 
$f(x)=h(x)$ $\forall x\le a$
$f(x)=tx$ $\forall x>a$

\begin{bolded}Subcase 6.2.2\end{bolded} : if $t=1$, then no constraint on $a$ and :
Let $a \in\mathbb R$
Let $h(x)$ be any continuous function from $(-\infty,a]\to[a,+\infty)$ such that $h(a)=a$. Then :
$f(x)=h(x)$ $\forall x\le a$
$f(x)=x$ $\forall x>a$

\begin{bolded}Subcase 6.2.3\end{bolded} : if $1>t>0$, then $a\le 0$ and :
Let $a\le 0$
Let $h(x)$ be any continuous function from $(-\infty,a]\to[a,+\infty)$ such that $h(a)=ta$ and $[a,ta]\subseteq h((-\infty,a])$ then 
$f(x)=h(x)$ $\forall x\le a$
$f(x)=tx$ $\forall x>a$

\begin{bolded}Subcase 6.2.4\end{bolded} : if $t<0$
Choosing some $f(x)$ positive and great enough, we get that $tf(x)<a$ and so can not be in $f(\mathbb R)$
So no solution

\begin{bolded}Subcase 6.3\end{bolded} : if $f(\mathbb R)=(a,+\infty)$
We get $f(x)=tx$ $\forall x>a$ and so (continuity) $f(a)=ta$ and so $ta>a$ and the solutions :

\begin{bolded}Subcase 6.3.1\end{bolded} : if $t>1$, then $a>0$ and :
Let $a>0$
Let $h(x)$ be any continuous function from $(-\infty,a]\to(a,+\infty)$ such that $h(a)=ta$ and $(a,ta]\subseteq h((-\infty,a])$ then 
$f(x)=h(x)$ $\forall x\le a$
$f(x)=tx$ $\forall x>a$

\begin{bolded}Subcase 6.3.2\end{bolded} : if $t=1$, then $ta>a$ is impossible and so no solution

\begin{bolded}Subcase 6.3.3\end{bolded} : if $1>t>0$, then $a< 0$ and :
Let $a<0$
Let $h(x)$ be any continuous function from $(-\infty,a]\to(a,+\infty)$ such that $h(a)=ta$ and $(a,ta]\subseteq h((-\infty,a])$ then 
$f(x)=h(x)$ $\forall x\le a$
$f(x)=tx$ $\forall x>a$

\begin{bolded}Subcase 6.3.4\end{bolded} : if $t<0$
Choosing some $f(x)$ positive and great enough, we get that $tf(x)\le a$ and so can not be in $f(\mathbb R)$
So no solution

\begin{bolded}Subcase 6.4\end{bolded} : if $f(\mathbb R)=(-\infty,a]$
This means $ta\le a$ (since $f(a)=ta\in f(\mathbb R)$) and the solutions :

\begin{bolded}Subcase 6.4.1\end{bolded}: if $t>1$, then $a\le 0$ and :
Let $a\le 0$ 
Let $h(x)$ any continuous function from $[a,+\infty)\to(-\infty,a]$ such that $h(a)=ta$ and $[ta,a]\subseteq h([a,+\infty))$. Then :
$f(x)=tx$ $\forall x\le a$
$f(x)=h(x)$ $\forall x>a$

\begin{bolded}Subcase 6.4.2\end{bolded}: if $t=1$, then no constraint on $a$ and :
Let $a\in\mathbb R$ 
Let $h(x)$ any continuous function from $[a,+\infty)\to(-\infty,a]$ such that $h(a)=a$. Then :
$f(x)=x$ $\forall x\le a$
$f(x)=h(x)$ $\forall x>a$

\begin{bolded}Subcase 6.4.3\end{bolded} : if $1>t>0$, then $a\ge 0$ and :
Let $a\ge 0$ 
Let $h(x)$ any continuous function from $[a,+\infty)\to(-\infty,a]$ such that $h(a)=ta$ and $[ta,a]\subseteq h([a,+\infty))$. Then :
$f(x)=tx$ $\forall x\le a$
$f(x)=h(x)$ $\forall x>a$

\begin{bolded}Subcase 6.4.4\end{bolded} : if $t<0$
Choosing some $f(x)$ negative and small enough, we get that $tf(x)> a$ and so can not be in $f(\mathbb R)$
So no solution

\begin{bolded}Subcase 6.5\end{bolded} : if $f(\mathbb R)=(-\infty,a)$
We get $f(x)=tx$ $\forall x<a$ and so (continuity) $f(a)=ta$ and so $ta<a$ and the solutions :

\begin{bolded}Subcase 6.5.1\end{bolded}: if $t>1$, then $a< 0$ and :
Let $a<0$ 
Let $h(x)$ any continuous function from $[a,+\infty)\to(-\infty,a)$ such that $h(a)=ta$ and $[ta,a)\subseteq h([a,+\infty))$. Then :
$f(x)=tx$ $\forall x\le a$
$f(x)=h(x)$ $\forall x>a$

\begin{bolded}Subcase 6.5.2\end{bolded}: if $t=1$, then $ta<a$ is impossible and so no solution

\begin{bolded}Subcase 6.5.3\end{bolded} : if $1>t>0$, then $a>0$ and :
Let $a>0$ 
Let $h(x)$ any continuous function from $[a,+\infty)\to(-\infty,a)$ such that $h(a)=ta$ and $[ta,a)\subseteq h([a,+\infty))$. Then :
$f(x)=tx$ $\forall x\le a$
$f(x)=h(x)$ $\forall x>a$

\begin{bolded}Subcase 6.5.4\end{bolded} : if $t<0$
Choosing some $f(x)$ negative and small enough, we get that $tf(x)\ge a$ and so can not be in $f(\mathbb R)$
So no solution

\begin{bolded}Subcase 6.6 \end{bolded}: if $f(\mathbb R)=[a,b]$
Then $a\le ta\le b$ and $a\le tb\le b$

\begin{bolded}Subcase 6.6.1\end{bolded} : if $t>1$, then $a\le ta\le b$ and $a\le tb\le b$ imply $a=b=0$ and so $f(x)=0$ $\forall x$, which indeed is a solution

\begin{bolded}Subcase 6.6.2\end{bolded} : if $t=1$, we get the solutions :
Let $a\le b$
Let $h_1(x)$ any continuous function from $(-\infty,a]\to[a,b]$ such that $h_1(a)=a$
Let $h_2(x)$ any continuous function from $[b,+\infty]\to[a,b]$ such that $h_2(b)=b$
Then :
$\forall x<a$ : $f(x)=h_1(x)$
$\forall x\in[a,b]$ : $f(x)=x$
$\forall x>b$ : $f(x)=h_2(x)$

\begin{bolded}Subcase 6.6.3\end{bolded} : if $1>t>0$, then $a\le ta\le b$ and $a\le tb\le b$ imply $a\le 0\le b$ and :
Let $a\le 0\le b$
Let $h(x)$ be any continuous function from $\mathbb R\to[a,b]$ such that :
$h(a)=ta$
$h(b)=tb$
$[a,ta]\cup[tb,b]\subseteq h((-\infty,a]\cup[b,+\infty))$
Then :
$\forall x\in[a,b]$ : $f(x)=tx$
$\forall x\notin[a,b]$ : $f(x)=h(x)$

\end{solution}



\begin{solution}[by \href{https://artofproblemsolving.com/community/user/29428}{pco}]
	\begin{tcolorbox}Can anyone give me documents or maybe solutions or this problem: 
Find all $f:\mathbb{R}\rightarrow\mathbb{R}$ with f is a continuous function and $ f(f(x) = a. f(x) + b.x$ $\forall$ $x\in\mathbb{R}$ and a,b be given numbers.
I think this equation has many satisfy functions depend on a,b so please help me with this.\end{tcolorbox}

\begin{bolded}Part two of two\end{underlined}\end{bolded}

\begin{bolded}Subcase 6.6.4\end{bolded} : if $-1\le t<0$, then $a\le ta\le b$ and $a\le tb\le b$ imply $a\le 0$ and $\frac at\ge b\ge ta$ and :
Let $a\le 0$ and $b\in[ta,\frac at]$
Let $h(x)$ be any continuous function from $\mathbb R\to[a,b]$ such that :
$h(a)=ta$
$h(b)=tb$
$[a,tb]\cup[ta,b]\subseteq h((-\infty,a]\cup[b,+\infty))$
Then :
$\forall x\in[a,b]$ : $f(x)=tx$
$\forall x\notin[a,b]$ : $f(x)=h(x)$

\begin{bolded}Subcase 6.6.5 \end{bolded}: if $t<-1$, then $a\le ta\le b$ and $a\le tb\le b$ imply $a=b=0$ and the solution $f(x)=0$ $\forall x$ which indeed is a solution.

\begin{bolded}Subcase 6.7\end{bolded} : if $f(\mathbb R)=[a,b)$
Then $f(x)=tx$ $\forall x\in [a,b)$ and so $tx\in[a,b)$ $\forall x\in[a,b)$
And so continuity implies $a\le ta< b$ and $a\le tb< b$

\begin{bolded}Subcase 6.7.1\end{bolded} : if $t\ge 1$, then $a\le ta< b$ and $a\le tb< b$ imply no solution

\begin{bolded}Subcase 6.7.2\end{bolded} : if $1>t>0$, then $a\le ta< b$ and $a\le tb< b$ imply $a\le 0< b$ and :
Let $a\le 0< b$
Let $h(x)$ be any continuous function from $\mathbb R\to[a,b)$ such that :
$h(a)=ta$
$h(b)=tb$
$[a,ta]\cup[tb,b)\subseteq h((-\infty,a]\cup[b,+\infty))$
Then :
$\forall x\in[a,b]$ : $f(x)=tx$
$\forall x\notin[a,b]$ : $f(x)=h(x)$

\begin{bolded}Subcase 6.7.3\end{bolded} : if $-1< t<0$, then $a\le ta< b$ and $a\le tb< b$ imply $a<0$ and $\frac at\ge b> ta$ and :
Let $a<0$ and $b\in(ta,\frac at]$
Let $h(x)$ be any continuous function from $\mathbb R\to[a,b)$ such that :
$h(a)=ta$
$h(b)=tb$
$[a,tb]\cup[ta,b)\subseteq h((-\infty,a]\cup[b,+\infty))$
Then :
$\forall x\in[a,b]$ : $f(x)=tx$
$\forall x\notin[a,b]$ : $f(x)=h(x)$

\begin{bolded}Subcase 6.7.4\end{bolded} : if $t\le -1$, then $a\le ta< b$ and $a\le tb< b$ imply no solution

\begin{bolded}Subcase 6.8\end{bolded} : if $f(\mathbb R)=(a,b]$
Then $f(x)=tx$ $\forall x\in (a,b]$ and so $tx\in(a,b]$ $\forall x\in(a,b]$
And so continuity implies $a< ta\le b$ and $a< tb\le b$

\begin{bolded}Subcase 6.8.1\end{bolded} : if $t\ge 1$, then $a< ta\le b$ and $a< tb\le b$ imply no solution

\begin{bolded}Subcase 6.8.2\end{bolded} : if $1>t>0$, then $a< ta\le b$ and $a< tb\le b$ imply $a<0\le b$ and :
Let $a< 0\le b$
Let $h(x)$ be any continuous function from $\mathbb R\to[a,b]$ such that :
$h(a)=ta$
$h(b)=tb$
$(a,ta]\cup[tb,b]\subseteq h((-\infty,a]\cup[b,+\infty))$
Then :
$\forall x\in[a,b]$ : $f(x)=tx$
$\forall x\notin[a,b]$ : $f(x)=h(x)$

\begin{bolded}Subcase 6.8.3\end{bolded} : if $-1<t<0$, then $a< ta\le b$ and $a< tb\le b$ imply $a< 0$ and $\frac at> b\ge ta$ and :
Let $a< 0$ and $b\in[ta,\frac at)$
Let $h(x)$ be any continuous function from $\mathbb R\to(a,b]$ such that :
$h(a)=ta$
$h(b)=tb$
$(a,tb]\cup[ta,b]\subseteq h((-\infty,a]\cup[b,+\infty))$
Then :
$\forall x\in[a,b]$ : $f(x)=tx$
$\forall x\notin[a,b]$ : $f(x)=h(x)$

\begin{bolded}Subcase 6.8.4\end{bolded} : if $t\le -1$, then $a< ta\le b$ and $a< tb\le b$ imply no solution

\begin{bolded}Subcase 6.9\end{bolded} : if $f(\mathbb R)=(a,b)$
Then $f(x)=tx$ $\forall x\in (a,b)$ and so $tx\in(a,b)$ $\forall x\in(a,b)$
And so continuity implies $a< ta< b$ and $a< tb< b$

\begin{bolded}Subcase 6.9.1\end{bolded} : if $t\ge 1$, then $a< ta<b$ and $a< tb< b$ imply no solution

\begin{bolded}Subcase 6.9.2\end{bolded} : if $1>t>0$, then $a< ta<b$ and $a< tb< b$ imply $a< 0< b$ and :
Let $a< 0< b$
Let $h(x)$ be any continuous function from $\mathbb R\to(a,b)$ such that :
$h(a)=ta$
$h(b)=tb$
$(a,ta]\cup[tb,b)\subseteq h((-\infty,a]\cup[b,+\infty))$
Then :
$\forall x\in[a,b]$ : $f(x)=tx$
$\forall x\notin[a,b]$ : $f(x)=h(x)$

\begin{bolded}Subcase 6.9.3\end{bolded} : if $-1< t<0$, then $a< ta<b$ and $a< tb< b$ imply $a< 0$ and $\frac at> b> ta$ and :
Let $a<0$ and $b\in(ta,\frac at)$
Let $h(x)$ be any continuous function from $\mathbb R\to(a,b)$ such that :
$h(a)=ta$
$h(b)=tb$
$(a,tb]\cup[ta,b)\subseteq h((-\infty,a]\cup[b,+\infty))$
Then :
$\forall x\in[a,b]$ : $f(x)=tx$
$\forall x\notin[a,b]$ : $f(x)=h(x)$

\begin{bolded}Subcase 6.9.4\end{bolded} : if $t\le -1$, then $a< ta<b$ and $a< tb< b$ imply no solution


\begin{bolded}Case  7\end{underlined}\end{bolded} : $a,b,c\ne 0$ and the equation is $f(f(x))=uf(x)+vx$ with $u,v\ne 0$

$f(x)$ is injective and so, since continuous, monotonous.
If $\lim_{x\to+\infty}f(x)=L$, then setting $x\to+\infty$ in functional equation and using continuity gives contradiction
So $\lim_{x\to+\infty}f(x)=\pm\infty$
If $\lim_{x\to-\infty}f(x)=L$, then setting $x\to-\infty$ in functional equation and using continuity gives contradiction
So $\lim_{x\to-\infty}f(x)=\pm\infty$

And since $f(x)$ is monotonous, we get that $f(\mathbb R)=\mathbb R$ and so $f(x)$ is a bijection.

\begin{bolded}Subcase 7.1\end{bolded} : $v>0$
let $x\in\mathbb R$ and the sequence $a_n$ defined as :
$a_0=x$
$a_1=f(x)$
$a_{n+2}=ua_{n+1}+va_n$ $\forall n\ge 0$
Notice than $a_n=f^{[n]}(x)$
Since $v>0$, the characteristic equation $x^2-ux-v$ has two distinct real roots $r_1>0>r_2$

And so $f^{[n]}(x)=\frac{(f(x)-r_2x)r_1^n-(f(x)-r_1x)r_2^n}{r_1-r_2}$

Since $f(x)$ is a bijection, then $f^{-1}(x)$ exists and it is easy to show that the above expression is true $\forall n\in\mathbb Z$

\begin{bolded}Subcase 7.1.1\end{bolded} : $u+v\ne 1$ 
$u+v\ne 1$ $\implies$ $r_1\ne 1$
$u\ne 0$ $\implies$ $r_2\ne -r_1$

If the equation $f(x)=x$ has real root $r$, then functional equation implies $r=ur+vr$ and so $r=0$
So, if $x\ne 0$, $f^{k+1}(x)\ne f^k(x)$ $\forall k$

For $x\ne 0$, we can then define $\Delta_n(x)=\frac{f^{n+2}(x)-f^{n+1}(x)}{f^{n+1}(x)-f^{n}(x)}$
$\Delta_n(x)\ne 0$
Since $f(x)$ is monotonous, $\Delta_n(x)$ has a constant sign, for any values of $n\in\mathbb Z$ and $x\in\mathbb R^*$

$\Delta_n(x)=$ $\frac{(f(x)-r_2x)(r_1-1)r_1^{n+1}-(f(x)-r_1x)(r_2-1)r_2^{n+1}}{(f(x)-r_2x)(r_1-1)r_1^{n}-(f(x)-r_1x)(r_2-1)r_2^{n}}$

Let $x\ne 0$ such that $f(x)\ne r_1x$ and $f(x)\ne r_2x$
We know that $|r_1|\ne |r_2|$ and so :
If $|r_1|>|r_2| : $\lim_{n\to +\infty}\Delta_n(x)=r_1$ and $\lim_{n\to -\infty}\Delta_n(x)=r_2$
If $|r_1|<|r_2| : $\lim_{n\to +\infty}\Delta_n(x)=r_2$ and $\lim_{n\to -\infty}\Delta_n(x)=r_1$
In both cases $\Delta_n(x)$ does not have a constant sign.

So $\forall x\ne 0$, either $f(x)=r_1x$, either $f(x)=r_2x$ and continuity + monotonicity imply :
either $f(x)=r_1x$ $\forall x$ which indeed is a solution
either $f(x)=r_2x$ $\forall x$ which indeed is a solution

Hence two solutons in this subcase

\begin{bolded}Subcase 7.1.2\end{bolded} : $u+v=1$
So $r_1=1$ and $r_2=u-1=-v<0$
$f(x)=x$ is a solution.
Let us from now in this subcase look for other solutions (different from $f(x)=x$ $\forall x$)

Let then $x$ such that $f(x)\ne x$ : $\\frac{f(f(x))-f(x)}{f(x)-x}=-v<0$ and so $f(x)$ must be decreasing.

Since $f(x)$ is continuous and decreasing, then equation $f(x)=x$ has a unique root $r$.
Let then $g(x)=f(x+r)-r$ : it's easy to check that $g(x)$ is such that $g(g(x))=ug(x)+vx$ and $g(0)=0$

So WLOG consider from now that $f(0)=0$ and so $f(x)\ne 0$ $\forall x\ne 0$
Since $f(x)$ is decreasing, $f^{n}(x)$ is increasing for even $n$ and decreasing for odd $n$
And since $f^{n}(0)=0$, we can conclude

$\forall x\ne 0$ : $\frac{f^{n}(x)}x$ is nonzero and has same sign as $(-1)^n$

But $\frac{f^{n}(x)}x$ $=\frac{f(x)-r_2x-(f(x)-x)r_2^n}{(1-r_2)x}$

Since $r_2\ne -1$, this quantity has limit  $\frac{f(x)-r_2x-}{(1-r_2)x}$ when $n$ is set to $+\infty$ if $|r_2|<1$ or $-\infty$ if $|r_2|>1$
And so $f(x)=r_2x$, else this quantity can no longer has same sign as $(-1)^n$ when $n$ is set to the appropriate $\infty$
So $f(x)=-vx$ $\forall x$, which indeed is a solution

And so the solutions in this subcase :
$f(x)=x$ $\forall x$
$f(x)=c-vx$ $\forall x$

\begin{bolded}Subcase 7.2\end{bolded} : $v<0$ and $u^2+4v>0$
\begin{bolded}[color=red]This case is to studied again[\/color].\end{bolded}
We again obviously have the two solutions $f(x)=r_1x$ and $f(x)=r_2x$, both $r_1,r_2$ having the same sign
But there are in some cases a lot of other solutions.
Look for example at $f(f(x))=5f(x)-6x$ : it's possible to buid piece per piece infinitely many solutions such that for example $f(x)\in(2x,3x)$ $\forall x>0$


\begin{bolded}Subcase 7.3 \end{bolded}: $v<0$ and $u^2+4v=0$
\begin{bolded}Subcase 7.3.1\end{bolded} : $u\ne 2$
let $x\in\mathbb R$ and the sequence $a_n$ defined as :
$a_0=x$
$a_1=f(x)$
$a_{n+2}=ua_{n+1}+va_n$ $\forall n\ge 0$
Notice than $a_n=f^{[n]}(x)$
Since $u^2+4v=0$, the characteristic equation $x^2-ux-v$ has one double real root $r=\frac u2\notin\{0,1\}$

And so $f^{[n]}(x)=r^{n-1}(nf(x)-r(n-1)x)$

Since $f(x)$ is a bijection, then $f^{-1}(x)$ exists and it is easy to show that the above expression is true $\forall n\in\mathbb Z$

Note that $u^2+4v=0$ and $u\ne 2$ implu $u+v\ne 1$
If the equation $f(x)=x$ has real root $z$, then functional equation implies $z=uz+vz$ and so $z=0$
So, if $x\ne 0$, $f^{k+1}(x)\ne f^k(x)$ $\forall k$

For $x\ne 0$, we can then define $\Delta_n(x)=\frac{f^{n+2}(x)-f^{n+1}(x)}{f^{n+1}(x)-f^{n}(x)}$
$\Delta_n(x)\ne 0$
Since $f(x)$ is monotonous, $\Delta_n(x)$ has a constant sign, for any values of $n\in\mathbb Z$ and $x\in\mathbb R^*$

$\Delta_n(x)=$ $r\frac{(n+1)(r-1)(f(x)-rx)+r(f(x)-x)}{n(r-1)(f(x)-rx)+r(f(x)-x)}$

If $f(x)\ne rx$ for some $x\ne 0$, and since $r\ne 1$, we get $(r-1)(f(x)-rx)\ne 0$ and so $\exists n\in\mathbb Z$ such that the two parts of the fraction have opposite signs.
And since, for $n$ great enough, these two parts have same sign, we get that $\Delta_n(x)$ can not have a constant sign.
So $f(x)=rx$ $\forall x$, whch indeed is a solution.

\begin{bolded}Subcase 7.3.2\end{bolded} : $u=2$ and $v=-1$
Equation is $f(f(x))=2f(x)-x$

Writing $g(x)=f(x)-x$, the equation is $g(x+g(x))=g(x)$

$g(x)=0$ $\forall x$ is a solution and let us from now look for non all zero solutions.
If $g(x)$ is solution, then $-g(-x)$ is solution too and so Wlog say $g(p)=q>0$ for some $p$

Let $A=\{x\ge p$ such that $g(x)=g(p)=q\}$

From $g(x+g(x))=g(x)$, we get $g(x+ng(x))=g(x)$ and so $p+nq\in A$ $\forall n\in\mathbb N\cup\{0\}$

If $A$ is not dense in $[p,+\infty)$, let then $a,b\in A$ such that $p\le a<b$ and $(a,b)\cap A=\emptyset$. (existence of $a,b$ needs continuity of $g(x)$)

Let then $y\in(a,b)$. So $g(y)\ne q$ 
Consider then $y-a+n(g(y)-q)$ for $n\in\mathbb N$
Since $g(y)\ne q$, this quantity, for $n$ great enough is out of $[-q,+q]$ and so let $m>0$ such that $y-a+m(g(y)-q)\notin[-q,+q]$ and so such that $y+mg(y)\notin[a+(m-1)q,a+(m+1)q]$

Looking at the continuous function $h(x)=x+mg(x)$, we get :
$h(a)=a+mq\in(a+(m-1)q,a+(m+1)q)$
$h(y)=y+mg(y)\notin[a+(m-1)q,a+(m+1)q]$

So (using continuity of $h(x)$), $\exists z\in(a,y)$ such that $h(z)=a+(m-1)q$ or $h(z)=a+(m+1)q$
But then $g(h(z))=q$ and so $g(z+mg(z))=g(z)=q$, impossible since $z\in(a,b)$ and $(a,b)\cap A=\emptyset$.

So $A$ is dense in $[p,+\infty)$

Then continuity of $g(x)$ implies $g(x)=q$ $\forall x\ge p$.
Let then any $w<p$ : If $g(w)>0$, then $\exists n\in\mathbb N$ such that $w+ng(w)>p$ and so $g(w)=q$. So $\forall x<p$ : either $g(x)=q$, either $g(x)\le 0$ and continuity gives the conclusion $g(x)=q$ $\forall x$

So $g(x)=c$ and $f(x)=x+c$ which indeed is a solution.

\begin{bolded}Subcase 7.4 \end{bolded}: $v<0$ and $u^2+4v<0$
let $x\in\mathbb R$ and the sequence $a_n$ defined as :
$a_0=x$
$a_1=f(x)$
$a_{n+2}=ua_{n+1}+va_n$ $\forall n\ge 0$
Notice than $a_n=f^{[n]}(x)$
Since $u^2+4v<0$, the characteristic equation $x^2-ux-v$ has two distinct complex roots $re^{it}$ and $re^{-it}$ with $r>0$ and $t\in(0,\pi)$

And so $f^{[n]}(x)=r^{n-1}\frac{f(x)\sin nt-rx\sin(n-1)t}{\sin t}$

Since $f(x)$ is a bijection, then $f^{-1}(x)$ exists and it is easy to show that the above expression is true $\forall n\in\mathbb Z$

Note that $u^2+4v<0$ implies $u+v<1$
If the equation $f(x)=x$ has real root $r$, then functional equation implies $r=ur+vr$ and so $r=0$
So, if $x\ne 0$, $f^{k+1}(x)\ne f^k(x)$ $\forall k$

For $x\ne 0$, we can then define $\Delta_n(x)=\frac{f^{n+2}(x)-f^{n+1}(x)}{f^{n+1}(x)-f^{n}(x)}$
$\Delta_n(x)\ne 0$
Since $f(x)$ is monotonous, $\Delta_n(x)$ has a constant sign, for any values of $n\in\mathbb Z$ and $x\in\mathbb R^*$

$f^{k+1}(x)-f^{k}(x)$ $=r^{k}\frac{f(x)\sin (k+1)t-rx\sin kt}{\sin t}$ $-r^{k-1}\frac{f(x)\sin kt-rx\sin(k-1)t}{\sin t}$
$=\frac{r^{k-1}}{\sin t}$ $(rf(x)\sin (k+1)t-r^2x\sin kt-f(x)\sin kt+rx\sin(k-1)t)$

$=cr^k\sin(kt+d)$ for some $c\ne 0,d$ depending on $x$

So $\Delta_n(x)=$ $r\frac{\sin (n+1)t+d}{\sin nt+d}$

But this quantity can not be of constant sign $\forall n$ (since $t\in(0,\pi)$)

Hence no solution


\end{solution}



\begin{solution}[by \href{https://artofproblemsolving.com/community/user/201911}{sansae}]
	Wow... you did a really big thing..   awesome!!
+) i like your profile picture
\end{solution}
*******************************************************************************

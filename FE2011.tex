-------------------------------------------------------------------------------

\begin{problem}[Posted by \href{https://artofproblemsolving.com/community/user/1598}{Arne}]
	Suppose $f$ is a real functions satisfying $f(x + 19) \leq f(x) + 19$ and $f(x + 94) \geq f(x) + 94$ for all real $x$. Prove that $f(x + 1) = f(x) + 1$ for all real $x$.
	\flushright \href{https://artofproblemsolving.com/community/c6h6517}{(Link to AoPS)}
\end{problem}



\begin{solution}[by \href{https://artofproblemsolving.com/community/user/998}{Peter Scholze}]
	again a problem i can solve  :D 
we have $f(x+1)+94\leq f(x+95)=f(x+5\cdot 19)\leq f(x)+95$, so $f(x+1)\leq f(x)+1$. but since $f(x+94)\geq f(x)+94$, we must always have inequality.

Peter
\end{solution}



\begin{solution}[by \href{https://artofproblemsolving.com/community/user/5820}{N.T.TUAN}]
	General problem 

Suppose $ f$ is a real functions satisfying $ f(x+a) \leq f(x)+a$ and $ f(x+b) \geq f(x)+b$ for all real $ x$, here $ a,b\in\mathbb N$ and $ \gcd (a,b)=1$. Prove that $ f(x+1) = f(x)+1$ for all real $ x$.
\end{solution}



\begin{solution}[by \href{https://artofproblemsolving.com/community/user/18420}{aviateurpilot}]
	\begin{tcolorbox}General problem 

Suppose $ f$ is a real functions satisfying $ f(x+a) \leq f(x)+a$ and $ f(x+b) \geq f(x)+b$ for all real $ x$, here $ a,b\in\mathbb N$ and $ \gcd (a,b)=1$. Prove that $ f(x+1) = f(x)+1$ for all real $ x$.\end{tcolorbox}

we use $ f(n+ax)\le f(x)+ax$ and $ f(n)\ge f(n-b)+b\ge f(n-2b)+2b\ge ...\ge f(n-yb)+yb$
we have $ gcd(a,b)=1$ so:
$ \exists (x,y)\in N^{*}^{2}: \ ax-by=1$
so, $ f(n+1)=f(n+ax-by)\le f(n-by)+ax\le f(n)-by+ax\le f(n)+1$
and $ \exists (x,y)\in N^{*}^{2}: \-ax+by=1$
so, $ f(n+1)=f(n+ax-by)\ge f(n-by)+ax\ge f(n)-by+ax\ge f(n)+1$
then $ f(n+1)=f(n)+1$
\end{solution}



\begin{solution}[by \href{https://artofproblemsolving.com/community/user/94870}{Thamdu}]
	\begin{tcolorbox}Suppose $f$ is a real functions satisfying $f(x + 19) \leq f(x) + 19$ and $f(x + 94) \geq f(x) + 94$ for all real $x$. Prove that $f(x + 1) = f(x) + 1$ for all real $x$.\end{tcolorbox}
This is easy and old problem.
\end{solution}



\begin{solution}[by \href{https://artofproblemsolving.com/community/user/29428}{pco}]
	\begin{tcolorbox}[quote="Arne"]Suppose $f$ is a real functions satisfying $f(x + 19) \leq f(x) + 19$ and $f(x + 94) \geq f(x) + 94$ for all real $x$. Prove that $f(x + 1) = f(x) + 1$ for all real $x$.\end{tcolorbox}
This is easy and old problem.\end{tcolorbox}

You really should avoid such posts :
- claiming that a problem is easy is useless : if the solution is given and easy, everybody knows it. If the solution is given and difficult, it seems to be false. If the solution is not given and you claim it's easy without giving the solution, you look like a liar.
- claiming that the problem is old is useless : everybody can see that the original post was more than six years old ...
\end{solution}



\begin{solution}[by \href{https://artofproblemsolving.com/community/user/80099}{Kenne}]
	\begin{tcolorbox}[quote="Arne"]Suppose $f$ is a real functions satisfying $f(x + 19) \leq f(x) + 19$ and $f(x + 94) \geq f(x) + 94$ for all real $x$. Prove that $f(x + 1) = f(x) + 1$ for all real $x$.\end{tcolorbox}
This is easy and old problem.\end{tcolorbox}

What was that? I agree totally Patrick.
\end{solution}
*******************************************************************************
-------------------------------------------------------------------------------

\begin{problem}[Posted by \href{https://artofproblemsolving.com/community/user/3749}{pigfly}]
	Find all functions $f: \mathbb{Z} \mapsto \mathbb{Z}$ satisfying the condition: $f(x^3 +y^3 +z^3 )=f(x)^3+f(y)^3+f(z)^3.$
	\flushright \href{https://artofproblemsolving.com/community/c6h14908}{(Link to AoPS)}
\end{problem}



\begin{solution}[by \href{https://artofproblemsolving.com/community/user/246}{pbornsztein}]
	This one has been used in our correspondance program last year.
See here for a solution (dossier 4 pb#2, in french but it doesn't matter here) :

http://www.animath.fr\/tutorat\/dossier_03044sol.pdf

Pierre.
\end{solution}



\begin{solution}[by \href{https://artofproblemsolving.com/community/user/168}{Mircea Lascu}]
	This problem was published in a book called "200 de ecuatii functionale pe N, Z, Q", in 2003 at GIL Publishing House ( http://www.gil.ro ). It can be found at page 27, problem nr. 139, author: Drimbe Onucu Mihai.
\end{solution}



\begin{solution}[by \href{https://artofproblemsolving.com/community/user/5246}{indybar}]
	In French it's hard to understand. Can somebody translate (briefly)?
\end{solution}



\begin{solution}[by \href{https://artofproblemsolving.com/community/user/10512}{mathmanman}]
	Well, now it's a bit late, but tomorrow I'll put the translation here ;)


EDIT : Ok, here goes :
First, they notice that $f(x) = 0$, $f(x) = x$ and $f(x) = -x$ are solutions, and, they want to show that these are the only ones.
Now, let $x=y=z=0$, we thus get : $f(0) = 3f(0)^3$., which leads to $f(0)=0$..
Then, if we let $y=-x$ and $z=0$, we get $f(-x)^3 = -f(x)^3$, which show that $f$ is an odd function.
With $(x,y,z) = (1,0,0)$, we have $f(1) = f(1)^3$, hence $f(1) = 0, 1 \ or -1$.
Moreover, with $(x,y,z)=(1,1,0)$ and $(x,y,z)=(1,1,1)$, we get that $f(2) = 2f(1)$ and $f(3) = 3f(1)$.
Thus, to conclude, we'd have to show that : $f(x) = xf(1)$ for all x.
To show this, we'll use the following lemma :
\begin{bolded}Lemma\end{bolded} : For all integer $x \ge 4$, $x^3$ can be written as a sum of five cubes of integers strictly less than $x$ in absolute value.
\begin{bolded}Proof\end{bolded} : I'll let this to the reader  :P  (if you can't prove it though, just ask and I'll translate that part too)

Then, thanks to this lemma, we can show via induction that $f(x) = xf(1)$, for all $x \in \mathbb{N}$ (and so, in $\mathbb{Z}$ too, since $f$ is odd).
And, eventually, we have that the only functions satisfying the asked relation are : 
$f(x) = 0$, $f(x) = x$ and $f(x) = -x$.

:)
\end{solution}



\begin{solution}[by \href{https://artofproblemsolving.com/community/user/10341}{mumath}]
	\begin{tcolorbox}Find all functions $f: \mathbb{Z} \mapsto \mathbb{Z}$ satisfying the condition: $f(x^3 +y^3 +z^3 )=f(x)^3+f(y)^3+f(z)^3.$\end{tcolorbox}if the added property:x,y,z are distinc,can you solve the problem so easily?
\end{solution}



\begin{solution}[by \href{https://artofproblemsolving.com/community/user/5246}{indybar}]
	Thanks, mathmanman!
\end{solution}



\begin{solution}[by \href{https://artofproblemsolving.com/community/user/957}{nmtruong1986}]
	Who has the solution for the expanding: Given m,n integer numbers. Find out all integral functions $f$ satisfying:
\[ f(\sum_{k=1}^nx_k^m)=\sum_{k=1}^nf^m(x_k)  \]
For every integer numbers $x_1,x_2,...,x_n$

And for each integer m, what is the smallest n for which, there are finite function $f$.
\end{solution}



\begin{solution}[by \href{https://artofproblemsolving.com/community/user/5820}{N.T.TUAN}]
	This is a problem from American Mathematic Monthly, year 1997 (if i no wrong  :ninja: )  :D
\begin{tcolorbox}This one has been used in our correspondance program last year.
See here for a solution (dossier 4 pb#2, in french but it doesn't matter here) :

http://www.animath.fr\/tutorat\/dossier_03044sol.pdf

Pierre.\end{tcolorbox}
\end{solution}



\begin{solution}[by \href{https://artofproblemsolving.com/community/user/30264}{lasha}]
	If $ x = y = z = 0$, we have: $ f(0) = 3f(0)^{3}$, so $ f(0) = 0$ is only integer solution.
If $ y = z = 0$, $ f(x^{3}) = f(x)^{3}$. So, $ f(y^{3}) = f(y)^{3}$.
If $ z = 0$, $ f(x^{3}+y^{3}) = f(x)^{3}+f(y)^{3}= f(x^{3})+f(y^{3})$. So, $ f(a+b) = f(a)+f(b)$, for any integers $ (a,b)$.
So, $ f(x) = f(1)+f(x-1) = 2f(1)+f(x-2) = ... = xf(1)$. By substituting $ f(1) = k$, where $ k$ is integer number, we get that $ f(x) = kx$ is solution for any integer number $ k$.
\end{solution}



\begin{solution}[by \href{https://artofproblemsolving.com/community/user/22804}{nayel}]
	\begin{tcolorbox}...$ f(x^{3}+y^{3}) = f(x)^{3}+f(y)^{3}= f(x^{3})+f(y^{3})$. So, $ f(a+b) = f(a)+f(b)$, for any integers $ (a,b)$...\end{tcolorbox}

I think this part is wrong. $ f(a+b)=f(a)+f(b)$ only for $ a,b$ perfect cubes. :(
\end{solution}



\begin{solution}[by \href{https://artofproblemsolving.com/community/user/30264}{lasha}]
	Yes you're right. I made mistake. :(
\end{solution}



\begin{solution}[by \href{https://artofproblemsolving.com/community/user/88984}{craft_man}]
	\begin{tcolorbox}Well, now it's a bit late, but tomorrow I'll put the translation here ;)


EDIT : Ok, here goes :
First, they notice that $f(x) = 0$, $f(x) = x$ and $f(x) = -x$ are solutions, and, they want to show that these are the only ones.
Now, let $x=y=z=0$, we thus get : $f(0) = 3f(0)^3$., which leads to $f(0)=0$..
Then, if we let $y=-x$ and $z=0$, we get $f(-x)^3 = -f(x)^3$, which show that $f$ is an odd function.
With $(x,y,z) = (1,0,0)$, we have $f(1) = f(1)^3$, hence $f(1) = 0, 1 \ or -1$.
Moreover, with $(x,y,z)=(1,1,0)$ and $(x,y,z)=(1,1,1)$, we get that $f(2) = 2f(1)$ and $f(3) = 3f(1)$.
Thus, to conclude, we'd have to show that : $f(x) = xf(1)$ for all x.
To show this, we'll use the following lemma :
\begin{bolded}Lemma\end{bolded} : For all integer $x \ge 4$, $x^3$ can be written as a sum of five cubes of integers strictly less than $x$ in absolute value.
\begin{bolded}Proof\end{bolded} : I'll let this to the reader  :P  (if you can't prove it though, just ask and I'll translate that part too)

Then, thanks to this lemma, we can show via induction that $f(x) = xf(1)$, for all $x \in \mathbb{N}$ (and so, in $\mathbb{Z}$ too, since $f$ is odd).
And, eventually, we have that the only functions satisfying the asked relation are : 
$f(x) = 0$, $f(x) = x$ and $f(x) = -x$.

:)\end{tcolorbox}
can u translate the lemma for me, mathmanman?
thanks.
\end{solution}



\begin{solution}[by \href{https://artofproblemsolving.com/community/user/90621}{Love_Math1994}]
	From AMM 1999
and appear in Venkatachala book in 2002 
\end{solution}
*******************************************************************************
-------------------------------------------------------------------------------

\begin{problem}[Posted by \href{https://artofproblemsolving.com/community/user/1991}{orl}]
	Find all functions $f: \mathbb{R} \rightarrow \mathbb{R}$, satisfying \[
f(xy)(f(x) - f(y)) = (x-y)f(x)f(y)
\] for all $x,y$.
	\flushright \href{https://artofproblemsolving.com/community/c6h17450}{(Link to AoPS)}
\end{problem}



\begin{solution}[by \href{https://artofproblemsolving.com/community/user/1991}{orl}]
	Please post your solutions. This is just a solution template to write up your solutions in a nice way and formatted in LaTeX. But maybe your solution is so well written that this is not required finally. For more information and instructions regarding the ISL\/ILL problems please look here: [url=http://www.mathlinks.ro/Forum/viewtopic.php?t=15623]introduction for the IMO ShortList\/LongList project[\/url] and regarding[url=http://www.mathlinks.ro/Forum/viewtopic.php?t=15623]solutions[\/url]  :)
\end{solution}



\begin{solution}[by \href{https://artofproblemsolving.com/community/user/26}{grobber}]
	Assume $f(1)=0$. Take $y=1$. We get $f^2(x)=0,\ \forall x$, so $f(x)=0,\ \forall x$. This is a solution, so we can take it out of the way: assume $f(1)\ne 0$. 

$y=1\Rightarrow f(x)[f(x)-f(1)]=(x-1)f(x)f(1)$. We either have $f(x)=0$ or $f(x)=f(1)x$, so for every $x$, $f(x)\in\{0,f(1)x\}$. In particular, $f(0)=0$.

Assume $f(y)=0$. We get $f(x)f(xy)=0,\ \forall x$. This means that $f(a),f(b)\ne 0\Rightarrow f(\frac ab)\ne 0\ (*)$ ($\frac ab$ is defined because $f(b)\ne 0\Rightarrow b\ne 0$). Assume now that $x\ne y$ and $f(x),f(y)\ne 0$. We get $f(x)=f(1)x,\ f(y)=f(1)y$, and after replacing everything we get $f(xy)=f(1)xy\ne 0$, so $x\ne y,\ f(x),f(y)\ne 0\Rightarrow f(xy)\ne 0\ (**)$. Assume now $f(x)\ne 0$. From $(*)$ we get $f(\frac 1x)\ne 0$, and after applying $(*)$ again to $a=x,b=\frac 1x$ we get $f(x^2)\ne 0\ (***)$. We can now see that $(**),(***)$ combine to $f(x),f(y)\ne 0\Rightarrow f(xy)\ne 0\ (\#)$.

Let $G=\{x\in\mathbb R|f(x)\ne 0\}$. $(*)$ and $(\#)$ simply say that $(G,\ \cdot)$ is a subgroup of $(\mathbb R^{*},\ \cdot)$.

Conversely, let $G$ be a subgroup of the multiplicative group $\mathbb R^*$. Take $f(x)=\{\begin{array}{c}f(1)x,\ x\in G\\0,\ x\not \in G\end{array}$. It's easy to check the condition $f(xy)[f(x)-f(y)]=(x-y)f(x)f(y)$.
\end{solution}



\begin{solution}[by \href{https://artofproblemsolving.com/community/user/102876}{Pascal96}]
	Please tell me if there is anything wrong with my solution:
Put y=1. If $f(1)$ = 0, f is identically 0. Otherwise f(x) = f(1)x = kx for any real k.
\end{solution}



\begin{solution}[by \href{https://artofproblemsolving.com/community/user/29428}{pco}]
	\begin{tcolorbox}Please tell me if there is anything wrong with my solution:
Put y=1. If $f(1)$ = 0, f is identically 0. Otherwise f(x) = f(1)x = kx for any real k.\end{tcolorbox}
Yes, just look at the previous post which gives a lot of other solutions.

And since you just give the result and not the content of your own proof, we can not show you where is your error. The only thing we can say is that your proof is wrong.
\end{solution}



\begin{solution}[by \href{https://artofproblemsolving.com/community/user/112281}{pr0likethis}]
	I have the same solution as pascal, but i'll post it so you can point out what's wrong:
let our assertion be $P(x,y)$. 
$P(x,1): f(x)(f(x)-f(1))=(x-1)f(x)f(1)$
Therefore, either $f(x)=0$ or $f(x)-f(1)=(x-1)f(1)$
Therefore, $f(x)=0$ is a solution
Now assume $f(x) \not=0$
then $f(x)-f(1)=xf(1)-f(1) \implies f(x)=xf(1)$
Plugging in $x=1$, we see $f(1)=f(1)$, so we have no further restrictions on $f(1)$
Therefore, for all $c \in \mathbb{R}$, there is an $f(x)=cx$.
Furthermore, $f(x)=0$ is merely $f(x)=cx, c=0$, so all solutions are in the form $f(x)=cx$ for all $c \in \mathbb{R} \blacksquare$

You said this is wrong; why?
\end{solution}



\begin{solution}[by \href{https://artofproblemsolving.com/community/user/69901}{dinoboy}]
	It's wrong because you only have $f(x) = xf(1)$ or $f(x) = 0$ for each $x \in \mathbb{R}$. You have to do a little more work to restrict which $x$ you have fall under the first case and which fall under the second.
\end{solution}



\begin{solution}[by \href{https://artofproblemsolving.com/community/user/112281}{pr0likethis}]
	Oh, tired me might realize; Is it because this function is not continuous?
\end{solution}



\begin{solution}[by \href{https://artofproblemsolving.com/community/user/195619}{DrMath}]
	First, suppose $f(0)\neq 0$. Then $f(0)\cdot [f(x)-f(0)] = x\cdot f(x)\cdot f(0)$. Taking $x=1$ we get $f(1)-f(0)=f(1)$, contradiction. Thus $f(0)=0$.

Now, suppose there was a value of $x$ such that $f(x)\neq 0$. Then $f(x)\cdot [f(x)-f(1)] = (x-1)\cdot f(x)\cdot f(1)$, or $f(x)=x\cdot f(1)$. Clearly $f(1)\neq 0$, or this gives a contradiction. Moreover, if $f$ is a solution, then $c\cdot f$ is a solution for any constant $c$. Then we either have $f(1)=0$, in which case $f\equiv 0$, or $f(1)=1$, in which case $f(x)$ is either $0$ or $x$.

Let $S$ be the set of values for which $f(x)=x$. If $S$ is nonempty, then $1\in S$. Moreover note that $0\not\in S$. Now, suppose $|S|>1$. We claim that $S$ is a subgroup of $\mathbb{R}^*$. To do this, we need to show that $S$ is closed under multiplication, and that $x\in S\rightarrow 1\/x\in S$.

Suppose $x\in S$ but $1\/x\not\in S$. Clearly $x\neq -1$. Then $f(1\/x)=0$. It follows that $f(1)\cdot [f(x)-f(1\/x)] = 0$. But $f(x)=x$ and $f(1\/x)=0$, so $f(1)\cdot x = 0$, implying $x=0$, contradiction as $0\not\in S$. Thus for every $x\in S$, we have $1\/x\in S$.

Now we need to check for closure under multiplication. First, if $x,y$ are distinct and in $S$, then $f(xy)\cdot (x-y) = (x-y)\cdot x\cdot y$, so $f(xy)=xy$. It remains to check that for $x\in S$, $f(x^2)=x^2$. This is clear if $x=1$. But note that $f(x)\cdot [f(x^2)-f(1\/x)] = (x^2-1\/x)\cdot f(x^2)\cdot f(1\/x)$. If $f(x^2)=0$, then it follows that $f(x)\cdot (0-1\/x)=0$, or $f(x)=0$, contradiction! Thus $f(x^2)=x^2$. It follows that $S$ is also closed under multiplication.

Thus, the solutions are of the form $\boxed{f(x)=f(1)\cdot x \mid x\in S; f(x) = 0 \mid  x\not\in S}$ when $S$ is a subgroup of $\mathbb{R}^*$. It is easy to check that these solutions work. 
\end{solution}
*******************************************************************************
-------------------------------------------------------------------------------

\begin{problem}[Posted by \href{https://artofproblemsolving.com/community/user/13}{enescu}]
	Prove that there is no function $f:\mathbb {Z}\rightarrow \mathbb {Z}$ such that $f(f(x))=x+1, \forall x\in \mathbb {Z}.$
Prove that there are infinitely many functions $g:\mathbb {Z}\rightarrow \mathbb {Z}$ such that $g(g(x))=-x, \forall x\in \mathbb {Z}.$
	\flushright \href{https://artofproblemsolving.com/community/c6h28101}{(Link to AoPS)}
\end{problem}



\begin{solution}[by \href{https://artofproblemsolving.com/community/user/6798}{Severius}]
	1. $f(0)=k, f(k)=f(f(0))=1, f(1)=f(f(k))=k+1,$ and then if $f(n)=k+n, f(k+n)=f(f(n))=n+1, f(n+1)=f(f(k+n))=k+n+1$, then $f(a)=k+a$ by induction for non-negative $a$. But then either $f(k)=2k \neq f(f(0))=1$ for $k$ positive, or $f(k+n)=n+1=2k+n$, so $2k=1$, again impossible.
\end{solution}



\begin{solution}[by \href{https://artofproblemsolving.com/community/user/285}{harazi}]
	Mr enescu, you should probably post the extremely beautiful problem that the two teachers proposed in the TST in the same year, related to iterated functions that give linear functions. That problem is a real challenge: nobody solved it. Unfortunately, I do not remember very well the details of the statement, so please post it.
\end{solution}



\begin{solution}[by \href{https://artofproblemsolving.com/community/user/13}{enescu}]
	I will do that. In fact, it was problem 4, 2nd selection test 1991.
\end{solution}



\begin{solution}[by \href{https://artofproblemsolving.com/community/user/285}{harazi}]
	Thanks and sorry for missing the right year.  :)
\end{solution}
*******************************************************************************
-------------------------------------------------------------------------------

\begin{problem}[Posted by \href{https://artofproblemsolving.com/community/user/13}{enescu}]
	Let $n\ge 2$ be a positive integer and let $a,b\in \mathbb {Z}$ such that $a\ne 0$. Consider the function $u:\mathbb {Z}\rightarrow \mathbb {Z}$, $u(x)=ax+b.$
Prove that if $a\ne 1$ then there exist infinitely many functions $f:\mathbb {Z}\rightarrow \mathbb {Z}$ such that $f^{(n)}=u$.
If $a=1$ prove that one can find $b$ such that there is no function   $f:\mathbb {Z}\rightarrow \mathbb {Z}$ such that $f^{(n)}=u$.
(Here, $f^{(n)}=\underset{n\text{ times}}{\underbrace{f\circ f\circ \ldots \circ f}}$)
	\flushright \href{https://artofproblemsolving.com/community/c6h28117}{(Link to AoPS)}
\end{problem}



\begin{solution}[by \href{https://artofproblemsolving.com/community/user/26}{grobber}]
	This might be wrong, I'm not sure right now, but, at first sight, it looks like there are no such functions when $a=1$ and $n\not|\ b$ (this concerns the second question of the problem). We can always choose $b=1$, but I was trying to do more (find all such $b$ for given $n$, maybe).
\end{solution}



\begin{solution}[by \href{https://artofproblemsolving.com/community/user/285}{harazi}]
	It is so bad that nobody is interested in this problem, though we have seen quite lots of particular cases till now and we gave all our interest in them. As usual, nice problems remain in darkness. This one worths efforts...
\end{solution}



\begin{solution}[by \href{https://artofproblemsolving.com/community/user/26}{grobber}]
	Consider the equivalence relation $\sim_a$ defined on $\mathbb Z$ as follows: let $g(x)=ax+b,\ \forall x\in\mathbb Z$, and put $x\sim_ay$ iff there exists a natural $m\ge 0$ s.t. $y=g^{(m)}(x)$ or $x=g^{(m)}(y)$ (where $g^{(m)}$ is the $m$-fold composition of $g$ with itself, and $g^{(0)}$ is defined to be the identity function).

Suppose now we have shown that $\sim_a$ has infinitely many classes $C_x,\ x\in\mathbb Z$ ($C_x$ being the class of $x$). We can then find infinitely many permutations of the set of classes consisting of disjoint $n$-cycles. For each class pick an element as follows: if the class is finite take any element, and if it's infinite take the "smallest" element $x$, in the sense that there's no $y\ne x$ s.t. $x=ay+b$ (it's easy to check that such an element must exist in all infiniet classes of $\sim_a$ when $a\ne 1$). Now, for each $n$-cycle $(C_{x_1},C_{x_2},\ldots,C_{x_n})$ in our permutation, where $x_i$ are the chosen elements from $C_{x_i}$, put $f(x_i)=x_{i+1},\ \forall i\in\overline{1,n-1}$, and extend this function s.t. $f(ax+b)=af(x)+b,\ \forall x$.

All that's left is proving that if $a\ne 1$, then $\sim_a$ has infinitely many classes:

If $a=-1$, then all the classes have at most two elements, and we're done, while if $|a|\ge 2$, all the classes have density zero, so it's obvious that $\mathbb Z$ canot be covered by finitely many of them.
\end{solution}



\begin{solution}[by \href{https://artofproblemsolving.com/community/user/7309}{Bojan Basic}]
	\begin{tcolorbox}Now, for each $n$-cycle $(C_{x_1},C_{x_2},\ldots,C_{x_n})$ in our permutation, where $x_i$ are the chosen elements from $C_{x_i}$, put $f(x_i)=x_{i+1},\ \forall i\in\overline{1,n-1}$, and extend this function s.t. $f(ax+b)=af(x)+b,\ \forall x$.\end{tcolorbox}
And what do you put for $f(x_n)$? I suppose you mean $f(x_n)=ax_1+b$ but I think that won't work if there are both finite and infinite classes in the chosen $n$-cycle.
\end{solution}



\begin{solution}[by \href{https://artofproblemsolving.com/community/user/26}{grobber}]
	You're right, that's what I meant, but that can easily be fixed. If $|a|\ge 2$ there is at most one finite class (this happens when $1-a|b$, and it's the class of $\frac b{1-a}$, which has only one member); just keep that class fixed and perform the permutations on the other classes (and you can also work it out easily if $a=-1$). I hope I didn't miss anything else.
\end{solution}
*******************************************************************************
-------------------------------------------------------------------------------

\begin{problem}[Posted by \href{https://artofproblemsolving.com/community/user/6005}{K09}]
	Find all functions $f:\mathbb{R} \to \mathbb{R}$ satisfying the equation \[
	f(x^2+y^2+2f(xy)) = (f(x+y))^2.
\] for all $x,y \in \mathbb{R}$.
	\flushright \href{https://artofproblemsolving.com/community/c6h33734}{(Link to AoPS)}
\end{problem}



\begin{solution}[by \href{https://artofproblemsolving.com/community/user/7254}{erdos}]
	Please guys, you should learn how to $LATEX$   ;) 
Let's $LATEX$ it so it can become clearer :   
Find all the functions $f: \mathbb R \rightarrow \mathbb R$ satisfying : 
$f(x^2+y^2+2f(xy))=(f(x+y))^2$ for all real numbers  $x,y$
\end{solution}



\begin{solution}[by \href{https://artofproblemsolving.com/community/user/7254}{erdos}]
	Take $x=y=0$ and put $k=f(0)$ we get $f(2k)=k^2$
take $y=0$ we get $f(x^2+2k)=(f(x))^2=(f(-x))^2$ for all $x \in \mathbb R$
So for $x=k^2$ we get $f((-k)^2)=f(k^2)$ or $f((-k)^2)=-f(k^2)$
Suppose $f((-k)^2)=f(k^2)$, then take $x=y=k$ we get $f(2k^2+2f(k^2))=k^4$
now take $x=-y=k$ we get $f(2k^2+2f((-k)^2))=k^2$
so $k^4=k^2$ which means that $k \in \{0,-1,1\}$
we can check that $k=-1$ and $k=1$ are not possible cases.
So $f(0)=0$
Take $y=0$ in the original functional equation we get $f(x^2)=(f(x))^2$ for all $x \in \mathbb R$
and this is has been solved before, we find that $f(x)=x$ for all $x \in \mathbb R$
Have I made a mistake somewhere ? :?
\end{solution}



\begin{solution}[by \href{https://artofproblemsolving.com/community/user/7738}{pleurestique}]
	\begin{tcolorbox}Take $x=y=0$ and put $k=f(0)$ we get $f(2k)=k^2$
take $y=0$ we get $f(x^2+2k)=(f(x))^2=(f(-x))^2$ for all $x \in \mathbb R$
So for $x=k^2$ we get $f((-k)^2)=f(k^2)$ or $f((-k)^2)=-f(k^2)$
Suppose $f((-k)^2)=f(k^2)$, then take $x=y=k$ we get $f(2k^2+2f(k^2))=k^4$
now take $x=-y=k$ we get $f(2k^2+2f((-k)^2))=k^2$
so $k^4=k^2$ which means that $k \in \{0,-1,1\}$
we can check that $k=-1$ and $k=1$ are not possible cases.
So $f(0)=0$
Take $y=0$ in the original functional equation we get $f(x^2)=(f(x))^2$ for all $x \in \mathbb R$
and this is has been solved before, we find that $f(x)=x$ for all $x \in \mathbb R$
Have I made a mistake somewhere ? :?\end{tcolorbox}

for one thing, you're writing $ f((-k)^2) = f(k^2) $ or $ f((-k)^2) = -f(k^2) $ where you really mean $ f(-k^2) = f(k^2) $ or $ f(-k^2) = -f(k^2) $.  and you're missing the case where $ f(-k^2) = -f(k^2) $
\end{solution}



\begin{solution}[by \href{https://artofproblemsolving.com/community/user/7254}{erdos}]
	Oh yeeh , I have totally forgotten about it  :blush:
So my solution posted above is not totally correct. This sounds a hard problem. Does anybody have an idea about it ?
\end{solution}



\begin{solution}[by \href{https://artofproblemsolving.com/community/user/7738}{pleurestique}]
	there's also the fact that $ f $ can be identically $ 0 $ or identically $ 1 $.  heh, this turned out to be a harder problem than i thought... (or maybe it's actually very easy and i'm just blind?)
\end{solution}



\begin{solution}[by \href{https://artofproblemsolving.com/community/user/146}{ali}]
	f(x^2)=f(x)^2
It was claimed above that this equation can be solved here! 
I don't think so. Not unless some sort of assumption on continuty is given.
\end{tcolorbox}
\end{solution}



\begin{solution}[by \href{https://artofproblemsolving.com/community/user/7254}{erdos}]
	I think that even the Most experienced people on this forum haven't found a solution yet. Please can anyone help ?? :?
\end{solution}



\begin{solution}[by \href{https://artofproblemsolving.com/community/user/285}{harazi}]
	Could you please let the people think? I really do not think it is the end of the world if we do not solve it in 4 days.
\end{solution}



\begin{solution}[by \href{https://artofproblemsolving.com/community/user/7254}{erdos}]
	OK. I am really sorry.  :blush:  :blush:  :blush:
\end{solution}



\begin{solution}[by \href{https://artofproblemsolving.com/community/user/7340}{AYMANE}]
	Thank you harazi, I spend 3 hours to find this solution.
 :) 

Put f(0)=a

we have f(x <sup>2<\/sup> +y <sup>2<\/sup> +2f(xy))= [f(x+y)] <sup>2<\/sup> 

Put now X=x+y and Y=xy then the equation becomes :

         f(X <sup>2<\/sup> -2Y+2f(Y))=f(X) <sup>2<\/sup> 

Put X=0 Then   f(2f(Y)-2Y)=f(0) <sup>2<\/sup> =a <sup>2<\/sup> 

Put now g(x)= 2f(x)-2x   then f(x)= 1\/2g(x)+x

Composing we have :  f(g(x))=1\/2g(g(x))+g(x)=a <sup>2<\/sup> (*)

Linear equation with constant coefficient:

So g(x) must be in the forme  : g(x)=Ax+B

Reporting this to (*) and computing we mill have:

   A(A+2)=0 and B(A+2)=2a <sup>2<\/sup> 

1°case:   A=0  then B=a <sup>2<\/sup>  and f(x)=x+1\/2a <sup>2<\/sup> 
   
or [f(x)] <sup>2<\/sup> =[f(-x)] <sup>2<\/sup> so a=0 

And f(x)=x.

2°case: A=-2  then a=0  and g(x)=-2x+B then f(x)=B\/2 or f(0)=a=0

Then f(x) =0.


Correct me if not ;)
\end{solution}



\begin{solution}[by \href{https://artofproblemsolving.com/community/user/7738}{pleurestique}]
	you're also missing the possibility $ f(x) \equiv 1$.
and there are a couple of other points in your solution i'm unclear about...
\end{solution}



\begin{solution}[by \href{https://artofproblemsolving.com/community/user/146}{ali}]
	First of all X=x+y and Y=xy implies that X^2>=4Y.
So if you let X=0 then Y has to be negative. Furthermore 
How can you possibly say that g(g(x)) + g(x) is linear and thus....
This again brings me to a point in one of the last posts:
f(x^2)=f(x)^2 was said to imply f(x)=x^2, where as it is completely wrong. One can construct many funmctions satisfying that equation.
The solution needs to be more detalied...
\end{solution}



\begin{solution}[by \href{https://artofproblemsolving.com/community/user/146}{ali}]
	Just to be explicit:
take the equation g(g(x)) + 2g(x) =2a^2 that you derived.
x has to be negative.(I explained this above)
So you can't really use the above equation, Because you don't know how g(x) behaves when x>0.
\end{solution}



\begin{solution}[by \href{https://artofproblemsolving.com/community/user/7340}{AYMANE}]
	for Ali


IF WE HAVE  g(g(x))+2g(x)=2a <sup>2<\/sup> then g(x)=Ax+B

For this take x(0)=x and x(n+1)=g(x(n)) then x(n+1)+2x(n)=a <sup>2<\/sup> 

take y(n)=x(n)-2\/3a <sup>2<\/sup>  then y(n+1)+2y(n)=0 and y(0)=x-2\/3a <sup>2<\/sup> 

The caracteristical equation is : x+2=0 and go on :?
\end{solution}



\begin{solution}[by \href{https://artofproblemsolving.com/community/user/7546}{jivko777}]
	As far as I know, the problems from SL 2004 are still being kept secret ... Well, I guess the idea of the SLs is that they could be used for selection tests by the IMO trainers. Am I wrong? Or do you mean there may be a leak of information?

Regards,
jivko
\end{solution}



\begin{solution}[by \href{https://artofproblemsolving.com/community/user/285}{harazi}]
	Look on the forums and you will see how well they are kept.  ;)
\end{solution}



\begin{solution}[by \href{https://artofproblemsolving.com/community/user/7546}{jivko777}]
	Hi harazi,


Thanks for posting in this topic ... actually some of the functional equations, created by you, are among my favourite ones  ;)  Could you please tell us something about "the process of creation" of functional equations\/inequalities? ... I mean in general, not Fedor's equation in particular, since we all understood that in fact it shouldn't be here  :) .

Many thanks!
jivko
PS Posting SL2004 questions now is not a bad idea in general, but I think that we all should respect somehow IMO and everything related to it.
\end{solution}



\begin{solution}[by \href{https://artofproblemsolving.com/community/user/285}{harazi}]
	The truth is that none of my functional equations will ever reach the level of this beautiful problem, that killed me: I spent two days on it and nothing, I could not even get the set of solutions. About process of creation, I have no idea, they just come to my mind. It is nothing special, I think, especially when the result is not so interesting or spectacular, which is the case with 99,99% of my problems. For the rest, well, that's another story... ;)
\end{solution}



\begin{solution}[by \href{https://artofproblemsolving.com/community/user/7546}{jivko777}]
	Well, one of your functional equations is one of my favourites, but I can not discuss it right now ... for reasons which will be revealed later (may be by some of the british members of the forum ;) )

I will come back to this topic later 

Thanks harazi! Fedor's problem is really beautiful.
jivko
\end{solution}



\begin{solution}[by \href{https://artofproblemsolving.com/community/user/270}{Fedor Petrov}]
	It is easy to invent such equation. I just take the identity, say $x^2+y^2+2xy=(x+y)^2$, and then add function $f$ in some places. Mostly we get something trivial, or something like Cauchey equation $f(x+y)=f(x)+f(y)$, but after few attempts I got this problem. It has taken some time to solve it, of course. Few hours for sure, I do not remember exactly.

I do not consider this problem very much beautiful, since it is not natural at all.
\end{solution}



\begin{solution}[by \href{https://artofproblemsolving.com/community/user/154}{Myth}]
	\begin{tcolorbox}It is easy to invent such equation. I just take the identity, say $x^2+y^2+2xy=(x+y)^2$, and then add function $f$ in some places.\end{tcolorbox}
He-he!!! You know, EXACTLY the same thing I said to Jivko some days ago  
\end{solution}



\begin{solution}[by \href{https://artofproblemsolving.com/community/user/270}{Fedor Petrov}]
	No, Misha, I did not know. Else I would refer correctly. Sorry.
\end{solution}



\begin{solution}[by \href{https://artofproblemsolving.com/community/user/154}{Myth}]
	Don't mind. It was my private conversation with Jivko. He asked me how to invent an functional equation.
\end{solution}



\begin{solution}[by \href{https://artofproblemsolving.com/community/user/7546}{jivko777}]
	Thanks, Fedor! You have reffered correctly. I asked Myth the same question in a private conversation, but not about your functional equation, I asked him about functional equations in general and he replied exactly in the same way as you did.  :) 


Have you created any other functional equations? I mean functional equations which are not in SL 2004\/2005. If so, it would be great if you share them with us ... if you don't mind of course.


Best regards!
Zhivko
\end{solution}



\begin{solution}[by \href{https://artofproblemsolving.com/community/user/20415}{who}]
	my sol uses limit idea :!: 
 all sol are F(x)=x ,F(x)=0 and all functions of the form F(x)=1 or -1,according to x does not belong to X and it does belong, :) X is proper subset of(-infinity,-2\3)
\end{solution}



\begin{solution}[by \href{https://artofproblemsolving.com/community/user/20415}{who}]
	, :| ,sorry i didnt see sol ofn bojan basic
\end{solution}



\begin{solution}[by \href{https://artofproblemsolving.com/community/user/49556}{xxp2000}]
	This is an interesting f.e. problem. 
I am sorry I did not notice there are three pages following this post until I submitted my message. Pls forget about this proof if you are familiar with this post.

Let $ a = f(0)$.
$ P(x,0): f(x^2 + 2a) = f(x)^2$.(*)
Suppose $ a < 0$, we let $ x = \sqrt { - 2a}$ and (*) gives $ a = f(x)^2\ge0$. Contradiction!
Hence $ a\ge0$.
Plug this into the RHS of the original formula,
$ f(x^2 + y^2 + 2f(xy)) = f(x^2 + y^2 + 2xy + 2a)$. (**)

If $ f(x) = x + a,\forall x$, it is easy to verify $ a = 0$ and $ f(x) = x$ is one solution.
From now on, we assume there exists $ b_0$ with $ c_0 = f(b_0)\ne b_0 + a$.

1) Let $ c = f(b)$ for arbitrary $ b\ne0$, then $ f(u + 2c - 2b) = f(u + 2a),\forall u\ge 2|b| + 2b$
When $ u\ge 2|b| + 2b$, we can always solve $ (x + y)^2 = u$ and $ xy = b$
Then we can plug in (**) and show the claim.
Hence, $ 2c - 2b - 2a$ is a period for $ f(x)$ when $ x\ge 2|b| + 2b + 2a$.
In particular, we see $ T_0 = |2c_0 - 2b_0 - 2a_0| > 0$ is a period when $ x\ge N_0 = 4|b_0| + 2a$.

2) $ f(x) = C$ when $ x\ge2a$, where $ C = 0$ or $ C = 1$.
Let $ b\in [N_0,N_0 + 1]$ be arbitrary with $ c = f(b)$. 
Then we have $ f(b^2 + 2a) = c^2$ and $ f((b + T_0)^2 + 2a) = f(b + T_0)^2 = c^2$.
By 1), when $ x\ge 4b + 4T_0 + 2a$, both $ 2c^2 - 2b^2 - 6a$ and $ 2c^2 - 2(b + T_0)^2 - 6a$ are periods.
So $ [2c^2 - 2b^2 - 6a] - [ 2c^2 - 2(b + T_0)^2 - 6a] = 2T_0^2 + 4bT_0$ is also a period.
Now let $ b_0 > b_1$ be arbitrary on $ [N_0,N_0 + 1]$, we see for $ x \ge N_1 = 4N_0 + 4 + 4T_0 + 2a$,
$ [ 2T_0^2 + 4b_0T_0] - [ 2T_0^2 + 4b_1T_0] = 4(b_0 - b_1)T_0$ is a period.
In other words, any number in between $ 0$ and $ 4T_0$ is a period of $ f(x)$ when $ x \ge N_1$.
It is easy to see that $ f(x) = C$ for some constant $ C$ when $ x\ge N_1$.
Then use (*), we have $ C^2 = C$. So $ C = 0$ or $ C = 1$.
Again from (*), we know $ f( - x) = \pm f(x)$.
For any $ x\ge0$, we can find $ b\le - N_1$ and $ x + 2c - 2b\ge N_1$ because $ c = 0,\pm1$.
By 1), $ f(x + 2a) = f(x + 2c - 2b) = C$. This shows the claim.

3) If $ a = 0$, $ f(x) = 0$.
By 2), $ f(x) = C,\forall x\ge0$. Since $ f(0) = 0$, $ f(x) = 0$ for all $ x\ge0$. (*) implies $ f( - x) = \pm f(x) = 0$.

4) If $ a > 0$, $ f(x) = 1,\forall x\ge - \frac23;f(x) = \pm1,\forall x < - \frac23$.
By (*), $ f(2a) = a^2$. By 2), $ a^2 = C$. Since $ a > 0$, we have $ f(x) = 1$ when $ x\ge2$.
Now LHS of (*) is 1,  $ f(x) = \pm 1,\forall x$.
Now we look back the original equation. 
If $ f(b) = - 1$ for some $ b = xy$, we have $ f(x^2 + y^2 - 2) = 1$. 
In other words, $ f(u) = 1,\forall u\ge2|b| - 2$.
Once $ 2|b| - 2\le b$, we will get contradiction.
Hence, $ b < - \frac23$. It is not hard to see that is sufficient for the equation to hold.

Conclusion:
1) $ f(x) = x$.
2) $ f(x) = 0$.
3) $ f(x) = 1,\forall x\ge - \frac23;f(x) = \pm1,\forall x < - \frac23$.
\end{solution}



\begin{solution}[by \href{https://artofproblemsolving.com/community/user/67223}{Amir Hossein}]
	See [url=http://www.artofproblemsolving.com/Forum/viewtopic.php?p=2170259]here[\/url] for official solution.
\end{solution}



\begin{solution}[by \href{https://artofproblemsolving.com/community/user/243405}{ThE-dArK-lOrD}]
	Substitute $y=0$ gives us $f(x^2+2f(0))=f(x)^2$ for all real number $x$. 
This also gives $f(x)^2=f(-x)^2$ for all real number $x$.
Note that for any real number $a,b$ that $a^2\geq 4b$, there exists real numbers $x,y$ that $x+y=a$ and $xy=b$.
From $f(x^2+y^2+2f(xy)) =f(x+y)^2$ for all real numbers $x,y$, we get that 
$$f(a^2-2b+2f(b))=f(a)^2=f(a^2+2f(0))$$ for all real numbers $a,b$ that $a^2\geq 4b$.
This can be simplify to 
$$P(x,y): f(x-2y+2f(y))=f(x+2f(0))$$ for all real numbers $x,y$ that $x\geq \max \{ 0,4y\}$.
Also, we've 
$$Q(x,y): f(x^2-y^2+2f(-xy))=f(x-y)^2$$ for all real numbers $x,y$.

If $f(y)=y+f(0)$ for all real number $y$, substitute in $f(x^2+y^2+2f(xy))=f(x+y)^2$ gives us 
$$x^2+y^2+2xy+3f(0)=(x+y+f(0))^2$$ for all real numbers $x,y$.
This easily gives $f(x)=x$ for all real number $x$, which clearly is one of the solution.

Now, we’ll consider only the case when there exists real number $k$ that $f(k)\neq k+f(0)$.
Let $M=2f(k)-2k-2f(0)$. Note that $M\neq 0$.
From $P(x-2f(0),k)$, we’ve $f(x+M)=f(x)$ for all real number $x\geq 2f(0)+\max \{ 0,4k\}$.
If $M>0$, we get $f(x+|M|)=f(x)$ for all real number $x\geq 2f(0)+\max \{ 0,4k\}$.
If $M<0$, we get $f(x)=f(x-M)=f(x+|M|)$ for all real number $x-M\geq 2f(0)+\max \{ 0,4k\}\iff x\geq M+2f(0)+\max \{ 0,4k\}$.
In both cases, we have $f(x)=f(x+|M|)$ for all real number $x\geq 2f(0)+\max \{ 0,4k\}$.
We can prove by induction that $f(x)=f(x+v|M|)$ for all positive integer $v$ and real number $x\geq 2f(0)+\max \{ 0,4k\}$.

For any real numbers $c,d$ that $f(c)=f(c+d)$, we’ve the following results:
1) $f(c^2+2f(0))=f(c^2+2cd+d^2+2f(0))$ 
This follows from $f(c^2+2f(0))= f(c)^2=f(c+d)^2=f(c^2+2cd+d^2+2f(0))$.
2) $f(x)=f(x-2d)$ for all real number $x\geq 2f(c)-2c+\max \{ 0,4c,4c+4d\}$
This follows from $P(x,c),P(x,c+d)$, which gives us $f(x-2c+2f(c))=f(x-2c-2d+2f(c))$ for all real number $x\geq \max \{ 0,4c,4c+4d\}$. 
3) $f(x)=f(x-2(2cd+d^2))$ for all real number $x\geq 2f(c)^2-2(c^2+2f(0))+\max \{ 0,4c^2+8f(0),4c^2+8f(0)+8cd+4d^2\}$.
This is true by applying the result 1) combine with 2) and note that $f(c^2+2f(0))=f(c)^2$.

Back to the problem. 
In this paragraph, we'll show that $f(\ell )=f(\ell -\Delta)$ for all real numbers $\Delta \in (0,|M|)$ and $\ell \geq 2f(0)+\max \{ 0, 4k\} +|M|$.
From $f(x)=f(x+|M|)$ for all real number $x\geq 2f(0)+\max \{ 0,4k\}$, using result 3), we get that 
$$f( t)=f(t-2(2x|M|+M^2))$$ for all real number $t\geq 2f(x)^2-2(x^2+2f(0)) +\max \{ 0,4x^2+8f(0), 4x^2+8f(0) +8x|M|+4M^2 \}$ where $x\geq 2f(0)+\max \{ 0,4k\}$.
There exists real number $h$ that $2(2h|M|+M^2) =\Delta $.
There also exists positive integer $n$ that $h+n\geq 2f(0)+\max \{ 0,4k\}$.
So, we’ve $$f(t)=f(t-2(2(h+n)|M|+M^2))=f(t-\Delta -4n|M|)$$ for all real number $t\geq 2f(h+n)^2-2((h+n)^2+2f(0))+ \max \{ 0,4(h+n)^2+8f(0),4(h+n)^2+8f(0)+8(h+n)|M|+4M^2\}$.
For any real number $\ell \geq 2f(0)+\max \{ 0,4k\}+|M|$, there exists positive integer $m>4n$ that 
$$\ell +m|M|\geq 2f(h+n)^2-2((h+n)^2+2f(0))+ \max \{ 0,4(h+n)^2+8f(0),4(h+n)^2+8f(0)+8(h+n)|M|+4M^2\}.$$
So, $f(\ell +m|M|)=f(\ell -\Delta -4n|M|+m|M|)=f(\ell -\Delta +(m-4n)|M|)$.
Since $\ell -\Delta >\ell -|M|\geq 2f(0)+\max \{ 0,4k\}$ and $\ell \geq 2f(0)+\max \{ 0,4k\}+|M|\geq 2f(0) +\max \{ 0,4k\}$, we get
$$f(\ell +m|M|)=f(\ell)\text{ and } f(\ell -\Delta +(m-4n)|M|)=f(\ell -\Delta).$$
Hence, $f(\ell )=f(\ell -\Delta)$, which we aim at the beginning of this paragraph.

Using the result in preceding paragraph, together with $f(x)=f(x+|M|)$ for all real number $x\geq 2f(0)+\max \{ 0,4k\}$, we get that $f(x)$ must be constant for all real number $x\geq 2f(0)+\max \{ 0,4k\}$.
For convenient, we write it as $f(x)=C$ for all real number $x\geq D$. Note that we can force $D$ to be positive.
Since there exists real number $r$ that $r$ and $r^2+2f(0))$ are greater than $D$, we get $C=f(r^2+2f(0))=f(r)^2=C^2$.
Hence, $C\in \{ 0,1\}$. So we've two following cases:

1) $C=0$. 
Since $f(x)=0$ for all real number $x\geq D$ and $f(x)^2=f(-x)^2$ for all real number $x$, we get that $f(x)=0$ for all real number $x$ that $|x|\geq D$.
For any real number $e$, there exists positive real number $u$ that $u(u+e)>D$ and $(u+e)^2+u^2>D$.
The former gives $f(-u(u+e))=0$ and so $f((u+e)^2+u^2+2f(-u(u+e)))=f((u+e)^2+u^2)=0$.
From $Q(u+e,u)$, we get that $0=f(e)^2\implies f(e)=0$.
Hence, $f(x)=0$ for all real number $x$, which clearly is one of the solution.

2) $C=1$.
Since $f(x)=1$ for all real number $x\geq D$ and $f(x)^2=f(-x)^2$ for all real number $x$, we get that $f(x)\in \{ -1,1\}$ for all real number $x$ that $|x|\geq D$.
For any real number $e$, there exists positive real number $u$ that $u(u+e)>D$ and $(u+e)^2+u^2-2>D$.
The former gives $f(-u(u+e))\in \{ -1,1\} $ and so $(u+e)^2+u^2+2f(-u(u+e)) \geq (u+e)^2+u^2-2>D$.
Hence, $f((u+e)^2+u^2+2f(-u(u+e))) =1$.
From $Q(u+e,u)$, we get that $1=f(e)^2\implies f(e)\in \{ -1,1\}$.
In other words, $f(x)\in \{ -1,1\}$ for all real number $x$.
Let $A=\{ x\in \mathbb{R}\mid f(x)=1\}$ and $B=\{ x\in \mathbb{R}\mid f(x)=-1\}$. Note that they form a partition of $\mathbb{R}$.
From $f(x^2+y^2+2f(xy)) =f(x+y)^2$ for all real numbers $x,y$, we get that 
$$f(x^2-2y+2f(y))=f(x)^2=1$$ for all real numbers $x,y$ that $x^2\geq 4y$.
This gives us $f(x-2y+2f(y))=1$ for all real number $x\geq \max \{ 0,4y\}$.
Using this equation, we get that
$$\begin{cases}
 x-2-2y\in A  \ \text{for all} \ x,y\in \mathbb{R}\text{ that }x\geq \max \{ 0,4y\} ,y\in B\\
 x+2-2y \in A  \ \text{for all} \ x,y\in \mathbb{R}\text{ that }x\geq \max \{ 0,4y\} ,y\in A
\end{cases} $$
In case that $y>0$, we may (and will) substitute $x$ by $4y+z$ where $z$ is real number that $z\geq 0$.
So, we can rewrite the condition in the following four cases
\begin{align*}
1) & \ x-2y+2\in A \ \text{for all} \ x,y\in \mathbb{R}\text{ that }x\geq 0,y\leq 0,y\in A \\
2) & \ x+2y+2\in A \ \text{for all} \ x,y\in \mathbb{R}\text{ that }x\geq 0,y\geq 0,y\in A \\
3) & \ x-2y-2\in A \ \text{for all} \ x,y\in \mathbb{R}\text{ that }x\geq 0,y\leq 0,y\in B \\
4) & \ x+2y-2\in A \ \text{for all} \ x,y\in \mathbb{R}\text{ that }x\geq 0,y\geq 0,y\in B.
\end{align*}
If there exists real number $0\leq w\leq 2$ that $w\in B$, using condition 4), we get that $w<2w-2 \implies w>2$, contradiction.
So, $x\in A$ for all real number $x$ that $0\leq x\leq 2$.
Using condition 2), we can extend it to $x\in A$ for all real number $x$ that $x\geq 0$.
If there exists real number $\frac{-2}{3} \leq w\leq 0$ that $w\in B$, using condition 3), we get that $w<-2w-2\implies w<\frac{-2}{3}$, contradiction.
So, $x\in A$ for all real number $x$ that $\frac{-2}{3}\leq x\leq 0$.
Hence, now we’ve $f(x)=1$ for all $x\geq \frac{-2}{3}$.
This means there exists subset $\mathcal{S}$ of $\{ x\in \mathbb{R}\mid x<\frac{-2}{3} \}$ that 
$$f(x)=\begin{cases} 
-1& , \text{if} \ x\in \mathcal{S}\\
1& , \text{otherwise}
\end{cases}.$$
Not hard to verify that this function is one of the solution for all subset $\mathcal{S}$ of $\{ x\in \mathbb{R}\mid x<\frac{-2}{3} \}$.

Finally, we have three type of solutions to the problem: 
1. $f(x)=0$ for all $x\in \mathbb{R}$, 
2. $f(x)=x$ for all $x\in \mathbb{R}$, and 
3. $f(x)=\begin{cases} 
-1& , \text{if} \ x\in \mathcal{S}\\
1& , \text{otherwise}
\end{cases}$ for any subset $\mathcal{S}$ of $\{ x\in \mathbb{R}\mid x<\frac{-2}{3} \}$.
\end{solution}
*******************************************************************************
-------------------------------------------------------------------------------

\begin{problem}[Posted by \href{https://artofproblemsolving.com/community/user/9954}{Rushil}]
	Let $f$ be a real-valued function defined for all real numbers, such that for some $a>0$ we have \[ f(x+a)={1\over2}+\sqrt{f(x)-f(x)^2} \] for all $x$.
Prove that $f$ is periodic, and give an example of such a non-constant $f$ for $a=1$.
	\flushright \href{https://artofproblemsolving.com/community/c6h59591}{(Link to AoPS)}
\end{problem}



\begin{solution}[by \href{https://artofproblemsolving.com/community/user/6825}{spider_boy}]
	i think you can solve by plugging x->x+a,x->x+2a etc.
\end{solution}



\begin{solution}[by \href{https://artofproblemsolving.com/community/user/15581}{Davron}]
	Directly from the equality given: f(x+a) ≥ 1\/2 for all x, and hence f(x) ≥ 1\/2 for all x. 

So f(x+2a) = 1\/2 + √( f(x+a) - f(x+a)2 ) = 1\/2 + √f(x+a) √(1 - f(x+a)) = 1\/2 + √(1\/4 - f(x) + f(x)2) = 1\/2 + (f(x) - 1\/2) = f(x). So f is periodic with period 2a. 

We may take f(x) to be arbitrary in the interval [0,1). For example, let f(x) = 1 for 0 ≤ x < 1, f(x) = 1\/2 for 1 ≤ x < 2. Then use f(x+2) = f(x) to define f(x) for all other values of x.

kalva
\end{solution}



\begin{solution}[by \href{https://artofproblemsolving.com/community/user/88295}{ivanbart-15}]
	Since $\sqrt{f(x)-(f(x))^{2}}\geq 0$, $f(x+a)\geq \frac{1}{2}$ and $f(x)\geq \frac{1}{2}$. 
Squaring the first equation, we obtain:
$(f(x+a))^{2}-f(x+a)+\frac{1}{4}=f(x)-(f(x))^{2}$, which is equivalent to: $|f(x)-\frac{1}{2}|=\sqrt{f(x+a)-(f(x+a))^{2}}$.  (*)

Plugging $x+a$ instead of $x$ into the original equation, we get $f(x+2a)-\frac{1}{2}=\sqrt{f(x+a)-(f(x+a))^{2}}$. Backing this into (*) and applying the condition $f(x)\geq \frac{1}{2}$, we get: $ f(x)-\frac{1}{2}=f(x+2a)-\frac{1}{2} $, or equivalently $f(x)=f(x+2a)$.
\end{solution}
*******************************************************************************
-------------------------------------------------------------------------------

\begin{problem}[Posted by \href{https://artofproblemsolving.com/community/user/1991}{orl}]
	Prove that there is no function $f$ from the set of non-negative integers into itself such that $f(f(n))=n+1987$ for all $n$.
	\flushright \href{https://artofproblemsolving.com/community/c6h60761}{(Link to AoPS)}
\end{problem}



\begin{solution}[by \href{https://artofproblemsolving.com/community/user/13603}{e.lopes}]
	One Nice Solution:


Let $N$ be the set of non-negative integers. Put $A = N - f(N)$
(the set of all $n$ such that we cannot find $m$ with $f(m) = n$).
Put $B = f(A)$. 

Note that $f$ is injective because if $f(n) = f(m)$, then $f(f(n)) = f(f(m))$ so $m = n$.
We claim that $B = f(N) - f(f(N))$.
Obviously $B$ is a subset of $f(N)$ and if $k$ belongs to $B$,
then it does not belong to $f(f(N))$ since $f$ is injective.
Similarly, a member of $f(f(N))$ cannot belong to $B$. 

Clearly $A$ and $B$ are disjoint. 
They have union $N - f(f(N))$ which is ${0, 1, 2, ... , 1986}$.
But since $f$ is injective they have the same number of elements, 
which is impossible since ${0, 1, ... , 1986}$ has an odd number of elements. 

 :)
\end{solution}



\begin{solution}[by \href{https://artofproblemsolving.com/community/user/5787}{ZetaX}]
	Solved for the more general case:
http://www.mathlinks.ro/Forum/viewtopic.php?highlight=1987&t=2749
\end{solution}



\begin{solution}[by \href{https://artofproblemsolving.com/community/user/30692}{smartjenny}]
	what if you:

   Take the derivative of $ f(f(n))$ which is one and therefore derivative of $ f(n)$ is also 1, which means $ f$ is a linear function, and since f must have integer coefficients in the form of $ an+b$, plug it in $ f(f(n))$ and see that b cannot be integer, a contradiction.
\end{solution}



\begin{solution}[by \href{https://artofproblemsolving.com/community/user/15697}{shape_shift2005}]
	$ f: \set{N^*} \to \set{N^*}$
$ f'(x)=?$
 :!:  :?:
\end{solution}



\begin{solution}[by \href{https://artofproblemsolving.com/community/user/29126}{MellowMelon}]
	You can't differentiate this function. That would be $ \lim_{h \rightarrow 0} \frac{f(x+h) - f(x)}{h}$, and the concept of a limit doesn't make sense on the natural numbers or any other discrete set.
\end{solution}



\begin{solution}[by \href{https://artofproblemsolving.com/community/user/40935}{Davi Medeiros}]
	Define $a(n)=f(n)-n$. Then, from $f(f(n))=n+1987$ we get $(f(f(n))-f(n))+(f(n)-n)=1987$, or $a(f(n))+a(n)=1987$ for every $n$ natural.

If we replace $n$ by $f(n)$ in $a(f(n))+a(n)=1987$, we get $a(f(f(n)))+a(f(n))=1987=a(f(n))+a(n)$, and then $a(n+1987)=a(n)$ for every natural $n$. Thus, $a$ is an 1987-periodic funcion.

Now, let $a_1=a(1), a_2=a(2),...,a_{1987}=a(1987)$ be the 1987 possible values of $a(n)$ (Since $a$ is periodic, this is possible). We must pair the 1987 $a_i$ in the following mannner: $a_i$ is paired with $a_j$ if $a_i +a_j =1987$ (we can always do this, because $a(f(n))+a(n)=1987$ for every $n$ natural). If there's two or more $j$ satisfying the pairment, anyone serve. So, we can divide 1987 numbers in pairs, and this is a contradiction, because 1987 is odd. Finally, the problem is solved.
\end{solution}



\begin{solution}[by \href{https://artofproblemsolving.com/community/user/108687}{mmmath}]
	$f(x)$must be one-to-one.$f(f(n))=n+1987$so $f(f(n))$ will miss 1987 values.
suppose that $f(n)$ miss exactly $k$ distinct values $c_1,c_2,...,c_k$so 
$f(f(n))$will miss $c_1,c_2,...,c_k,f(c_1),f(c_2),....,f(c_k)$so if $f(m)=w$so $w\ne f(c_1),f(c_2),....,f(c_k),m\ne c_1,c_2,...,c_k$so there is $n$ such that $f(n)=m$ hence $f(f(n))=w$so $f(f(n))$ misses $2k$ values.
so it's contradiction.
\end{solution}



\begin{solution}[by \href{https://artofproblemsolving.com/community/user/70157}{varun.tinkle}]
	Wrong solution thanx pco for ur help
\end{solution}



\begin{solution}[by \href{https://artofproblemsolving.com/community/user/29428}{pco}]
	\begin{tcolorbox}Assuming that a function exsists....   suppouse.... the function is 
$ ax+b $ ...
Similarly we can check for higher degrees of equation........\end{tcolorbox}

But .... you cant check ALL the possible functions to achieve the result!

And what's the meaning of "degrees of equation" ? What is the degree of the function $f(n)=1+\left\lfloor e^{175\sin(n)}\right\rfloor$ ?

Dont forget that there exists infinitely many non polynomial functions
\end{solution}



\begin{solution}[by \href{https://artofproblemsolving.com/community/user/72900}{mstf9696}]
	good solution
\end{solution}



\begin{solution}[by \href{https://artofproblemsolving.com/community/user/173719}{murom4869}]
	build a $Z_{1987}$ -> $Z_{1987}$ function
\end{solution}



\begin{solution}[by \href{https://artofproblemsolving.com/community/user/274173}{Anar24}]
	i can't find pco's solution who can help me?\begin{tcolorbox}Wrong solution thanx pco for ur help\end{tcolorbox}


\end{solution}



\begin{solution}[by \href{https://artofproblemsolving.com/community/user/334597}{darkeagle}]
	I'm interested to know what is the number of function which verified $f(f(x))=x+k$ where k is even.
\end{solution}



\begin{solution}[by \href{https://artofproblemsolving.com/community/user/29428}{pco}]
	\begin{tcolorbox}I'm interested to know what is the number of function which verified $f(f(x))=x+k$ where k is even.\end{tcolorbox}
I suppose you mean "functions from $\mathbb N\to\mathbb N$"

If so, just look at the general solution and count.

You'll find $\frac{k!}{\frac k2!}$



\end{solution}



\begin{solution}[by \href{https://artofproblemsolving.com/community/user/64049}{grupyorum}]
	$f(\cdot)$ is clearly injective. Let $I\subseteq [1986]\triangleq \{0,1,2,\dots,1986\}$ be a set, such that, $i\in I\iff f(i)<1987$; and $I^c$ be the complement of $I$, with respect to $[1986]$.Clearly, since, $f(f(i))=i+1987 \geq 1987$, we get, $i\in I \implies f(i)\in I^c$.Since, $f(\cdot)$ is an injection, we must have, $|I|\leq |I^c|$.

Next, take an $i\in I^c$, and let, $f(i)=1987k+\ell$, where, $0\leq \ell <1987$. Using, $f(f(\ell))=\ell+1987$, with $f(\ell)$, in place of $\ell$, we have, $f(\ell+1987)=f(\ell)+1987$, and thus, via induction, $f(\ell+1987k)=f(\ell)+1987k$. Hence,
$$
f(f(i))=f(1987k+\ell)=1987k+f(\ell)=i+1987 \implies f(\ell)+1987(k-1)=i.
$$
Since, $i<1987$, the only possibility,  is to have, $k=1$, and, $f(\ell)=i$. Therefore, if, $i\in I^c$, then, $f(i)=1987+\ell$, where, $f(\ell)=i$. In particular, $\ell = f(i)-1987 \in I$; therefore, $i\in I^c \implies f(i)-1987 \in I$. Clearly, $t\mapsto f(t)-1987$ is also injective, and therefore, $|I^c|\leq |I|$. Combining this with what was previously obtained, we get, $|I|=|I^c|$; while, $I\cap I^c=\emptyset$, and, $I\cup I^c=[1986]$, and since, $[1986]$ has odd cardinality, this is impossible.
\end{solution}
*******************************************************************************
-------------------------------------------------------------------------------

\begin{problem}[Posted by \href{https://artofproblemsolving.com/community/user/45762}{FelixD}]
	Find all functions $ f: \mathbb{R} \to \mathbb{R}$, such that
\[ f(xf(y)) + f(f(x) + f(y)) = yf(x) + f(x + f(y))\]
holds for all $ x$, $ y \in \mathbb{R}$, where $ \mathbb{R}$ denotes the set of real numbers.
	\flushright \href{https://artofproblemsolving.com/community/c6h303622}{(Link to AoPS)}
\end{problem}



\begin{solution}[by \href{https://artofproblemsolving.com/community/user/29428}{pco}]
	\begin{tcolorbox}Find all functions $ f: \mathbb{R} \to \mathbb{R}$, such that
\[ f(xf(y)) + f(f(x) + f(y)) = yf(x) + f(x + f(y))\]
holds for all $ x$, $ y \in \mathbb{R}$, where $ \mathbb{R}$ denotes the set of real numbers.\end{tcolorbox}

Let $ P(x,y)$ be the assertion $ f(xf(y))+f(f(x)+f(y))=yf(x)+f(x+f(y))$
Let $ f(0)=a$

$ f(x)=0$ $ \forall x$ is a solution. Let us from now consider $ \exists c$ such that $ f(c)\neq 0$

If $ f(y_1)=f(y_2)$, subtracting $ P(c,y_1)$ from $ P(c,y_2)$ implies $ 0=f(c)(y_2-y_1)$ and so $ y_1=y_2$ and $ f(x)$ is injective.

$ P(0,1)$ $ \implies$ $ a+f(a+f(1))=a+f(f(1))$, so $ f(a+f(1))=f(f(1))$ and so, since $ f(x)$ is injective, $ a=0$

Then $ P(x,0)$ $ \implies$ $ f(f(x))=f(x)$ and, since $ f(x)$ is injective, $ f(x)=x$

Hence two solutions :
$ f(x)=0$ $ \forall x$
$ f(x)=x$ $ \forall x$
\end{solution}



\begin{solution}[by \href{https://artofproblemsolving.com/community/user/109774}{littletush}]
	let us suppose that f is non-constant($f(x)\equiv 0 $is a trivial solution)let $x=0$,$f(0)+f(f(y)+f(0))=yf(0)+f(f(y))$
let $y=0,f(xf(0))+f(f(x)+f(0))=f(x+f(0))$
it's trivial that f is injective.
let $y=1$,then $f(f(1)+f(0))=f(f(1))$,hence $f(0)=0$
so $f(f(x))=f(x)$ hence $f(x)=x$.
\end{solution}



\begin{solution}[by \href{https://artofproblemsolving.com/community/user/29428}{pco}]
	\begin{tcolorbox}let us suppose that f is non-constant($f(x)\equiv 0 $is a trivial solution)let $x=0$,$f(0)+f(f(y)+f(0))=yf(0)+f(f(y))$
let $y=0,f(xf(0))+f(f(x)+f(0))=f(x+f(0))$
it's trivial that f is injective.
let $y=1$,then $f(f(1)+f(0))=f(f(1))$,hence $f(0)=0$
so $f(f(x))=f(x)$ hence $f(x)=x$.\end{tcolorbox}
Very interesting brand new solution !
Congrats !
\end{solution}



\begin{solution}[by \href{https://artofproblemsolving.com/community/user/234290}{getrektm9}]
	is this solution ok?
Let $P(x,y)$ be assertion of given equation.We can easily prove that the function is an injection(like in the post before).
$P(x,x)$ ->$f(xf(x))+f(2f(x))=xf(x)+f(x+f(x))$
$P(f(x),x)$ ->$f(f(x)x)+f(2x)=f(x)x+f(f(x)+x)=f(xf(x))+f(2f(x))$ which implies $f(2x)=f(2f(x))$ so since the function is an injection we get $f(x)=x$ which is solution.
Hence the only solutions are $f(x)=0,f(x)=x$
\end{solution}



\begin{solution}[by \href{https://artofproblemsolving.com/community/user/360187}{jikonmlime}]
	\begin{tcolorbox}let us suppose that f is non-constant($f(x)\equiv 0 $is a trivial solution)let $x=0$,$f(0)+f(f(y)+f(0))=yf(0)+f(f(y))$
let $y=0,f(xf(0))+f(f(x)+f(0))=f(x+f(0))$
it's trivial that f is injective.
let $y=1$,then $f(f(1)+f(0))=f(f(1))$,hence $f(0)=0$
so $f(f(x))=f(x)$ hence $f(x)=x$.\end{tcolorbox}

How did u get that $f$ is injective?
\end{solution}
*******************************************************************************
-------------------------------------------------------------------------------

\begin{problem}[Posted by \href{https://artofproblemsolving.com/community/user/46488}{Raja Oktovin}]
	Find all functions $ f: \mathbb{R} \rightarrow \mathbb{R}$ satisfying \[ f(x^3+y^3)=xf(x^2)+yf(y^2)\] for all real numbers $ x$ and $ y$.
\begin{italicized}Hery Susanto, Malang\end{italicized}
	\flushright \href{https://artofproblemsolving.com/community/c6h311512}{(Link to AoPS)}
\end{problem}



\begin{solution}[by \href{https://artofproblemsolving.com/community/user/60946}{matrix41}]
	Set $ y=0$ then

$ f(x^3)=xf(x^2)$ , so $ f(y^3)=yf(y^2)$ , then

$ f(x^3+y^3)=xf(x^2)+yf(y^2)=f(x^3)+f(y^3)$ $ \rightarrow$ $ f(x)+f(y)=f(x+y)$

simple induction gives : $ f(x_1+x_2+...+x_n)=f(x_1)+f(x_2)+f(x_3)+...+f(x_n)$  so $ \forall \ n\in\mathbb{R}$ 

then $ f(nx)=nf(x)$ and thus by setting $ x=1$ we have $ f(n)=nf(1)$, since $ f(1)$ is a constant then by assuming $ f(1)=a$ , gives $ f(n)=an$ with $ a\in\mathbb{R}$ 

done
\end{solution}



\begin{solution}[by \href{https://artofproblemsolving.com/community/user/34380}{math10}]
	\begin{tcolorbox}Set $ y = 0$ then

$ f(x^3) = xf(x^2)$ , so $ f(y^3) = yf(y^2)$ , then

$ f(x^3 + y^3) = xf(x^2) + yf(y^2) = f(x^3) + f(y^3)$ $ \rightarrow$ $ f(x) + f(y) = f(x + y)$

simple induction gives : $ f(x_1 + x_2 + ... + x_n) = f(x_1) + f(x_2) + f(x_3) + ... + f(x_n)$  so $ \forall \ n\in\mathbb{R}$ then $ f(nx) = nf(x)$ \end{tcolorbox}
Why for all $ n \in R$  :?: 

\begin{tcolorbox}Set $ y = 0$ then

$ f(x^3) = xf(x^2)$ , so $ f(y^3) = yf(y^2)$ , then

$ f(x^3 + y^3) = xf(x^2) + yf(y^2) = f(x^3) + f(y^3)$ $ \rightarrow$ $ f(x) + f(y) = f(x + y)$

simple induction gives : $ f(x_1 + x_2 + ... + x_n) = f(x_1) + f(x_2) + f(x_3) + ... + f(x_n)$  so $ \forall \ n\in\mathbb{R}$ 
then $ f(nx) = nf(x)$ and thus by setting $ x = 1$ we have $ f(n) = nf(1)$, since $ f(1)$ is a constant then by assuming $ f(1) = a$ , gives $ f(n) = an$ with $ a\in\mathbb{R}$ 

done\end{tcolorbox}
you only solve equation on $ N$ not $ R$
We have:$ f(x + y) = f(x) + f(y)$ and $ f(x^3) = xf(x^2)$,$ (x) = - f( - x)$,$ f(kx) = kf(x)$ for all $ k \in N$
We have:
 $ f((x + 1)^3 + (x - 1)^3) = (x + 1)f(x^2 + 2x + 1) + (x - 1)f(x^2 - 2x + 1) = 2xf(x^2) + 2xf(1) + 4f(x)$
and $ f((x + 1)^3 + (x - 1)^3) = f(2x^3 + 6x) = 2xf(x^2) + 6f(x)$
So $ f(x) = xf(1)$ for all $ x \in R$
\end{solution}



\begin{solution}[by \href{https://artofproblemsolving.com/community/user/60946}{matrix41}]
	Sorry there are some mistakes in my solution

Let me write my full solution

First from my first post, we have 

$ f(x_1)+f(x_2)+...+f(x_n)=f(x_1+x_2+x_3+...+x_n)$ with $ n\in\mathbb{N}$ so $ f(nx)=nf(x)$ $ \forall n\in\mathbb{N}$

and then by setting $ x=m+1$ and $ y=m-1$ it gives

$ f((m+1)^3+(m-1)^3)=f((m^3+3m^3+3m+1)+(m^3-3m^3+3m-1))=f(2m^3+6m)=f(2m^3)+f(6m)=2f(m^3)+6f(m)$

and

$ f((m+1)^3+(m-1)^3)=(m+1)f(m^2+2x+1)+(m-1)f(x^2-2x+1)=(m+1)\left(f(m^2)+f(2m)+f(1)\right)+(m-1)\left(f(m^2)+f(-2m)+f(1)\right)=2mf(m^2)+4f(m)+2mf(1)$

since $ 2mf(m^2)=2f(m^3)$ , so $ f((m+1)^3+(m-1)^3)=2mf(m^2)+4f(m)+2mf(1)=2f(m^3)+4f(m)+2mf(1)$

$ 2f(m^3)+6f(m)=f((m+1)^3+(m-1)^3)=2f(m^3)+4f(m)+2mf(1)$

$ 2f(m)=2mf(1)$ $ \rightarrow$ $ f(m)=mf(1)$ $ \forall m\in\mathbb{R}$

please check....

EDITED : Sorry I didn't know that my solution is similar to math10's solution  :oops:
\end{solution}



\begin{solution}[by \href{https://artofproblemsolving.com/community/user/109774}{littletush}]
	not so hard.
let $y=0$,then
$f(x^3)=xf(x^2)$
so $f(x^3+y^3)=f(x^3)+f(y^3)$
hence f satisfies Cauchy's  function,so for rational x,$f(x)=x$
hence $f(1)=1$
by letting $x=x+1$ we get
$f((x+1)^3)=(x+1)f((x+1)^2)$
hence $2f(x^2)=(2x-1)f(x)+x$
let $x=x+1$
we get $2(f(x^2)+2f(x)+1)=(2x+1)(f(x)+1)+x+1$
hence $f(x)=x$.
\end{solution}



\begin{solution}[by \href{https://artofproblemsolving.com/community/user/29428}{pco}]
	\begin{tcolorbox} ... hence f satisfies Cauchy's  function,\end{tcolorbox}Right
\begin{tcolorbox}so for rational x,$f(x)=x$\end{tcolorbox}Wrong : $f(x)=ax$
\end{solution}



\begin{solution}[by \href{https://artofproblemsolving.com/community/user/109774}{littletush}]
	\begin{tcolorbox}[quote="littletush"] ... hence f satisfies Cauchy's  function,\end{tcolorbox}Right
\begin{tcolorbox}so for rational x,$f(x)=x$\end{tcolorbox}Wrong : $f(x)=ax$\end{tcolorbox}
oh gush!such a terrible mistake!
a can be any real number.
\end{solution}



\begin{solution}[by \href{https://artofproblemsolving.com/community/user/215362}{nawaites}]
	So which solution is correct?????
\end{solution}



\begin{solution}[by \href{https://artofproblemsolving.com/community/user/260346}{Takeya.O}]
	It is easy to show that ∃$a\in \mathbb R$ s.t. $f(x)=ax(\forall x\in \mathbb Q)$.

If we show that $f$ is
-continuous at some point
-monotonous
-either upperbounded or lowerbounded on some open interval

$f(x)=ax(\forall x\in \mathbb R)$ :P

Anyone has idea? :?
\end{solution}



\begin{solution}[by \href{https://artofproblemsolving.com/community/user/29428}{pco}]
	\begin{tcolorbox}Find all functions $ f: \mathbb{R} \rightarrow \mathbb{R}$ satisfying \[ f(x^3+y^3)=xf(x^2)+yf(y^2)\] for all real numbers $ x$ and $ y$.
\begin{italicized}Hery Susanto, Malang\end{italicized}\end{tcolorbox}

Let $P(x,y)$ be the assertion $f(x^3+y^3)=xf(x^2)+yf(y^2)$
Let $a=f(1)$

$P(0,0)$ $\implies$ $f(0)=0$
$P(x,0)$ $\implies$ $f(x^3)=xf(x^2)$
And so $f(x^3+y^3)=f(x^3)+f(y^3)$ and so $f(x)$ is additive.

So $f(px)=pf(x)$ $\forall x$ and $\forall p\in\mathbb Q$

Let $x\in\mathbb R$ and $k\in\mathbb Q$
$P(x+k,0)$ $\implies$ $f(x^3+3kx^2+3k^2x+k^3)=(x+k)f(x^2+2kx+k^2)$

Which may be written $k^2(f(x)-ax)+2k(f(x^2)-xf(x))=0$

Considering this as a polynomial in $k$ with infinitely many roots (any rational), we get thet it must be the zero polynomial and so, looking at coefficient of $k^2$ :

$\boxed{f(x)=ax\text{  }\forall x}$ which indeed is a solution, whatever is $a\in\mathbb R$


\end{solution}



\begin{solution}[by \href{https://artofproblemsolving.com/community/user/260515}{anhtaitran}]
	My solution:Plug x=y=0 so f(0)=0.
Plug x=0 so f(x^3)=xf(x^2).(1)
So f(x^3+y^3)=f(x^3)+f(y^3) for every x;y real.
So f is addictive.
Now we will caculate f( (x+1)^3+(x-1)^3) in 2 ways.
f( (x+1)^3+(x-1)^3) = (x+1)f((x+1)^2)+(x-1)f((x-1)^2).
                               =(x+1)[f(x^2)+2f(x)+f(1)]+(x-1)(f(x^2)-2f(x)+f(1)].
                              =2xf(x^2)+2xf(1)+4f(x).(2)
On the other hand,
f( (x+1)^3+(x-1)^3) =f(2x^3+6x)=2f(x^3)+6f(x).(3)
By (1);(2) and (3) we have :
f(x)=cx(c=f(1)).

\end{solution}



\begin{solution}[by \href{https://artofproblemsolving.com/community/user/260346}{Takeya.O}]
	@pco,@anhtaitran
What a nice solution! :w00t: Are you a FE master? 
\end{solution}



\begin{solution}[by \href{https://artofproblemsolving.com/community/user/337737}{Evenprime123}]
	\begin{tcolorbox}
And so $f(x^3+y^3)=f(x^3)+f(y^3)$ and so $f(x)$ is additive.

So $f(px)=pf(x)$ $\forall x$ and $\forall p\in\mathbb Q$
\end{tcolorbox}

How do I conclude this? Induction would work on \(\mathbb N \) but what can I do here?
\end{solution}



\begin{solution}[by \href{https://artofproblemsolving.com/community/user/29428}{pco}]
	\begin{tcolorbox}[quote=pco]
And so $f(x^3+y^3)=f(x^3)+f(y^3)$ and so $f(x)$ is additive.

So $f(px)=pf(x)$ $\forall x$ and $\forall p\in\mathbb Q$
\end{tcolorbox}

How do I conclude this? Induction would work on \(\mathbb N \) but what can I do here?\end{tcolorbox}
You should consider this as a well-known property of additive functions.

If you want to prove it :
$f(2x)=f(x)+f(x)=2f(x)$
$f(3x)=f(2x)+f(x)=3f(x)$
And, with induction : $f(nx)=nf(x)$ $\forall x$, $\forall n\in\mathbb N$

So $f(x)=f(q\frac xq)=qf(\frac xq)$ and so $f(\frac xq)=\frac 1qf(x)$ $\forall x$, $\forall q\in\mathbb N$

So $f(\frac pqx)=pf(\frac xq)=\frac pqf(x)$ and so $f(px)=pf(x)$ $\forall x$, $\forall p\in\mathbb Q^+$

It remains to remember that $f(-x)=-f(x)$ and $f(0)=0$ and so 
$f(px)=pf(x)$ $\forall x$, $\forall p\in\mathbb Q$

\end{solution}



\begin{solution}[by \href{https://artofproblemsolving.com/community/user/180207}{GeronimoStilton}]
	[color=#f00]redacted[\/color]
\end{solution}



\begin{solution}[by \href{https://artofproblemsolving.com/community/user/29428}{pco}]
	\begin{tcolorbox}Some progress:
...
... so I can't finish the problem.\end{tcolorbox}
Sorry, but what is the interest of this post ?
Just publicly claim that you are working on this problem ? And what about your progress about your sister's garden ? and what is the last film you liked ?
Just request for some help ? : read the thread ! there is already a full solution (see post #10)
Just bring some new informations to the community ? but this contribution already has been given at least thrice in the current thread (and btw is the beginning of the full solution in the post #10)



\end{solution}
*******************************************************************************
-------------------------------------------------------------------------------

\begin{problem}[Posted by \href{https://artofproblemsolving.com/community/user/89157}{duck1606}]
	Find all function $f: \mathbb R^{+} \to \mathbb R^{+}$ such that \[f(x) \cdot f(yf(x))=f(x+y)\] for all $x,y>0$.
	\flushright \href{https://artofproblemsolving.com/community/c6h373453}{(Link to AoPS)}
\end{problem}



\begin{solution}[by \href{https://artofproblemsolving.com/community/user/29428}{pco}]
	\begin{tcolorbox}Find all function $f:R^{+}\rightarrow R^{+}$ such that: $f(x).f(yf(x))=f(x+y)$ with $x,y>0$\end{tcolorbox}

Let $P(x,y)$ be the assertion $f(x)f(yf(x))=f(x+y)$
$f(x)=1$ $\forall x$ is obviously a solution. So let us from now look for non all-one solutions.
Let then $u>0$ such that $f(u)\ne 1$


1) $f(x)\le 1$ $\forall x>0$
=================
If $\exists a>0$ such that $f(a)>1$, then :

$P(a,\frac{a}{f(a)-1})$ $\implies$ $f(a)f(\frac{af(a)}{f(a)-1})$ $=f(\frac{af(a)}{f(a)-1})$ and so $f(a)=1$ and contradiction

Q.E.D

2) $f(x)$ is injective
==============
As a consequence of 1) above : $f(u)<1$
If $\exists a>0$ and $\Delta>0$ such that $f(a)=f(a+\Delta)$ :
Comparing $P(a,x)$ and $P(a+\Delta,x)$, we get $f(x+a)=f(x+a+\Delta)$ and so $f(x)=f(x+\Delta)$ $\forall x>a$
So $f(x)=f(x+n\Delta)$ $\forall x>a$, $\forall n\in\mathbb N\cup\{0\}$

Let then $n$ great enough such that $\frac{(n\Delta-u)f(u)}{1-f(u)}>a$
Let $y=\frac{n\Delta-u}{1-f(u)}$ such that $yf(u)>a$ and $y+u=yf(u)+n\Delta$

$P(u,y)$ $\implies$ $f(u)f(yf(u))=f(y+u)=f(yf(u)+n\Delta)=f(yf(u))$ and so $f(u)=1$, contradiction
Q.E.D.

3) $f(x)<1$ $\forall x$
===============
If $\exists a>0$ such that $f(a)=1$, then $P(a,x)$ $\implies$ $f(x)=f(x+a)$, impossible, since $f(x)$ is injective
Q.E.D.

4) $f(x)=\frac 1{ax+1}$
================
$P(x,\frac 1{f(x)})$ $\implies$ $f(x)f(1)=f(x+\frac 1{f(x)})$

$\frac 1{f(x)}>1$. So : $P(1,x+\frac 1{f(x)}-1)$ $\implies$ $f(1)f((x+\frac 1{f(x)}-1)f(1))=f(x+\frac 1{f(x)})$

And so $f(x)f(1)=f(1)f((x+\frac 1{f(x)}-1)f(1))$ and so, since injective :

$x=(x+\frac 1{f(x)}-1)f(1)$ and so $f(x)=\frac 1{x(\frac 1{f(1)}-1)+1}$ which may be written $f(x)=\frac 1{ax+1}$ with $a>0$
which indeed is a solution.

5) Synthesis of solutions
=================
We got :
$f(x)=1$ $\forall x>0$
$f(x)=\frac 1{ax+1}$ with $a>0$

And so the family of solutions $\boxed{f(x)=\frac 1{ax+1}}$ $\forall x>0$ with $a\ge 0$
\end{solution}



\begin{solution}[by \href{https://artofproblemsolving.com/community/user/104682}{momo1729}]
	A somewhat shorter solution can be found in the last page of this document (in French) : http://www.animath.fr\/IMG\/pdf\/OFM_2011-2012-envoi2-corrige.pdf
\end{solution}
*******************************************************************************
-------------------------------------------------------------------------------

\begin{problem}[Posted by \href{https://artofproblemsolving.com/community/user/96202}{Askeladden}]
	Find all functions $f: \mathbb R \to \mathbb R$ such that for all reals $x$ and $y$,
\[ f(x^2+y^2)=f(x^2)+f(y^2)+2f(x)f(y).\]
	\flushright \href{https://artofproblemsolving.com/community/c6h384406}{(Link to AoPS)}
\end{problem}



\begin{solution}[by \href{https://artofproblemsolving.com/community/user/29428}{pco}]
	\begin{tcolorbox}Find all functions f:R-->R such as:
$ f(x^2+y^2)=f(x^2)+f(y^2)+2f(x)f(y)$

Good luck\end{tcolorbox}
Let $P(x,y)$ the assertion $f(x^2+y^2)=f(x^2)+f(y^2)+2f(x)f(y)$

$P(x,0)$ $\implies$ $f(0)(1+2f(x))=0$ and so, if $f(0)\ne 0$, then $f(x)=-\frac 12$ $\forall x$ which indeed is a solution.
Let us from now consider that $f(0)=0$

$f(x)=0$ $\forall x$ is also a solution and let us from now consider that $\exists u$ such that $f(u)\ne 0$

Comparing $P(u,x)$ and $P(u,-x)$, we get $f(-x)=f(x)$ and $f(x)$ is an even function and we can write $f(x)=g(x^2)$

Notice then that $P(x,y)$ becomes $P'(x,y)$ : $g(x^2+2xy+y^2)=g(x^2)+g(y^2)+2g(x)g(y)$ $\forall x,y\ge 0$

1) $\exists h(x)$ such that we have new assertion $Q(x,y)$ : $g(x+y)=g(x)h(y)+g(y)$ $\forall x,y\ge 0$
=====================================================================================
$P(x,\sqrt{y^2+z^2})$ $\implies$ $f(x^2+y^2+z^2)=f(x^2)+f(y^2+z^2)+2f(x)f(\sqrt{y^2+z^2})$ $=f(x^2)+f(y^2)+f(z^2)+2f(x)f(\sqrt{y^2+z^2})+2f(y)f(z)$

Same, : $P(y,\sqrt{x^2+z^2})$ $\implies$ $f(x^2+y^2+z^2)=f(x^2)+f(y^2)+f(z^2)+2f(y)f(\sqrt{x^2+z^2})+2f(x)f(z)$

And so $f(x)f(\sqrt{y^2+z^2})+f(y)f(z)$ $=f(y)f(\sqrt{x^2+z^2})+f(x)f(z)$

Which implies $g(x^2)g(y^2+z^2)+g(y^2)g(z^2)$ $=g(y^2)g(x^2+z^2)+g(x^2)g(z^2)$

And so $g(x)g(y+z)+g(y)g(z)$ $=g(y)g(x+z)+g(x)g(z)$ $\forall x,y,z\ge 0$

$\implies$ $g(x)(g(y+z)-g(z))=g(y)(g(x+z)-g(z))$

Let $y=u^2$ such that $g(y)=f(u)\ne 0$. the above equation becomes $g(x+z)-g(z)=g(x)\frac{g(u^2+z)-g(z)}{g(u^2)}$

And so $g(x+z)-g(z)=g(x)h(z)$ for some function $h(x)$
Q.E.D.

2) The only possibilities are $h(x)=0$ and $h(x)=1$
==================================================
$Q(x,u^2)$ $\implies$ $g(x+u^2)=g(x)h(u^2)+g(u^2)$
$Q(y^2,x)$ $\implies$ $g(x+u^2)=g(u^2)h(x)+g(x)$
Subtracting, we get ${g(u^2)(h(x)-1)=g(x)(h(u^2)-1}$

If $h(u^2)=1$, we get $h(x)=1$ $\forall x$ and so $g(x+y)=g(x)+g(y)$
Then $P'(x,y)$ implies $g(xy)=g(x)g(y)$
And this double equation is very classical and has a unique non constant solution $g(x)=x$ and so $f(x)=x^2$ which indeed is a solution

If $h(u^2)=\ne 1$, we get $g(x)=a(h(x)-1)$
But then :

$Q(u^2,x+y)$ $\implies$ $g(u^2+x+y)=g(u^2)h(x+y)+g(x+y)$ $=g(u^2)h(x+y)+g(x)h(y)+g(y)$
$Q(u^2+x,y)$ $\implies$ $g(u^2+x+y)=g(u^2+x)h(y)+g(y)$ $=g(u^2)h(x)h(y)+g(x)h(y)+g(y)$
And so $h(x+y)=h(x)h(y)$ $\forall x,y\ge 0$

This is a classical equation which gives : either $h(x)=0$ $\forall x$, either $h(x)=e^{c(x)}$ where $c(x)$ is any solution of Cauchy equation.

$h(x)=0$ $\forall x$ implies $g(x)$ is constant and so $f(x)$ is constant and so the two solutions we already know $f(x)=0$ or $f(x)=-\frac 12$

$h(x)=e^{c(x)}$ : we get $g(x)=a(e^{c(x)}-1)$
Plugging this in $P'(x,y)$ implies $g(x)=0$ and so $f(x)=0$
Notice that this point is not obvious at all and I had a lot of difficulties to show it. I could give the directions I used  if somebody asked.

3) Synthesis of solutions :
==========================
We found three solutions :
$f(x)=0$ $\forall x$
$f(x)=-\frac 12$ $\forall x$
$f(x)=x^2$ $\forall x$
\end{solution}
*******************************************************************************
-------------------------------------------------------------------------------

\begin{problem}[Posted by \href{https://artofproblemsolving.com/community/user/93909}{magical}]
	Find all functions $f:(1, +\infty ) \to \mathbb R$ which satisfy
\[f(x)-f(y)=(y-x)f(xy)\] for all $x,y >1 $.
	\flushright \href{https://artofproblemsolving.com/community/c6h384559}{(Link to AoPS)}
\end{problem}



\begin{solution}[by \href{https://artofproblemsolving.com/community/user/29428}{pco}]
	\begin{tcolorbox}Find $f:(1, +\infty ) \rightarrow R $ satisfy
$f(x)-f(y)=(y-x)f(xy)$ for all $x,y >1 $\end{tcolorbox}
Let $P(x,y)$ be the assertion $f(x)-f(y)=(y-x)f(xy)$
Let $a>2$ and $x\in(\frac {a^2}4,4a^2)$

$P(\frac 2a\sqrt x,\frac a2\sqrt x)$ $\implies$ $f(\frac 2a\sqrt x)-f(\frac a2\sqrt x)$ $=(\frac a2-\frac 2a)\sqrt xf(x)$

$P(\frac a2\sqrt x,\frac{2a}{\sqrt x})$ $\implies$ $f(\frac a2\sqrt x)-f(\frac{2a}{\sqrt x})$ $=(\frac {2a}x-\frac a2)\sqrt xf(a^2)$

$P(\frac{2a}{\sqrt x},\frac 2a\sqrt x)$ $\implies$ $f(\frac{2a}{\sqrt x})-f(\frac 2a\sqrt x)$ $=(\frac 2a-\frac{2a}x)\sqrt xf(4)$

Adding these three lines, we get $(\frac a2-\frac 2a)f(x)+(\frac {2a}x-\frac a2)f(a^2)+(\frac 2a-\frac{2a}x)f(4)=0$

And so $f(x)=\frac{\alpha(a)}x+\beta(a)$ $\forall x\in(\frac {a^2}4,4a^2)$

So $\alpha(a)$ and $\beta(a)$ are constant and $f(x)=\frac{\alpha}x+\beta$ 

Plugging this in original equation, we get $\beta=0$ and so $\boxed{f(x)=\frac{\alpha}x}$
\end{solution}
*******************************************************************************
-------------------------------------------------------------------------------

\begin{problem}[Posted by \href{https://artofproblemsolving.com/community/user/31067}{ridgers}]
	Find all the continuous functions $f: \mathbb R^+\to \mathbb R$ such that \[f(x)+f(y)=f(\sqrt[m]{x^m+y^m}),\] for all $x,y \in R^+$, where $m$ is a natural number.
	\flushright \href{https://artofproblemsolving.com/community/c6h384702}{(Link to AoPS)}
\end{problem}



\begin{solution}[by \href{https://artofproblemsolving.com/community/user/29428}{pco}]
	\begin{tcolorbox}Find all the continuous functions $f:R^+\rightarrow R$ such that $f(x)+f(y)=f(\sqrt[m]{x^m+y^m})$ where $x,y \in R^+$ and $m$  is a natural number.\end{tcolorbox}
Setting $f(x)=g(x^m)$, we get $g(x^m)+g(y^m)=g(x^m+y^m)$ and so $g(x)+g(y)=g(x+y)$ $\forall x,y>0$
Since $g(x)$ is continuous, we get $g(x)=ax$ and so $\boxed{f(x)=ax^m}$
\end{solution}
*******************************************************************************
-------------------------------------------------------------------------------

\begin{problem}[Posted by \href{https://artofproblemsolving.com/community/user/92964}{dyta}]
	Let $n \in \mathbb{N}$. Find all functions $f : \mathbb{R}^+ \rightarrow \mathbb{R}^+$ such that $f(x^n)=f(x)^n$ for all $x>0$.
	\flushright \href{https://artofproblemsolving.com/community/c6h384854}{(Link to AoPS)}
\end{problem}



\begin{solution}[by \href{https://artofproblemsolving.com/community/user/29428}{pco}]
	\begin{tcolorbox}Let $n \in \mathbb{N}$. Find all of function $f : \mathbb{R}^+ \rightarrow \mathbb{R}^+$ such that $f(x^n)=f(x)^n$\end{tcolorbox}

Not an olympiad exercise.

If $n=1$, any function from $\mathbb R^+\to \mathbb R^+$ is solution.

If $n>1$, a general solution is :
$\forall x>1$ : $f(x)=h(x)^{n^{\left\lfloor\frac{\ln(\ln x)-\ln(\ln 2)}{\ln n}\right\rfloor}}$ 

$f(1)=1$

$\forall x<1$ : $f(x)=k(x)^{n^{\left\lfloor\frac{\ln(-\ln x)-\ln(\ln 2)}{\ln n}\right\rfloor}}$ 

Where :
$h(x)$ is any function from $[2,2^n)\to\mathbb R^+$
$k(x)$ is any function from $(2^{-n},\frac 12]\to\mathbb R^+$
\end{solution}



\begin{solution}[by \href{https://artofproblemsolving.com/community/user/68429}{MathUniverse}]
	How did you find these general solutions?

Thanks in advance.
\end{solution}



\begin{solution}[by \href{https://artofproblemsolving.com/community/user/29428}{pco}]
	\begin{tcolorbox}How did you find these general solutions?

Thanks in advance.\end{tcolorbox}
That's quite classical.

For $x>1$ for example, consider the sequence $a_k=2^{n^k}$ for $k\in\mathbb Z$

Let $k\in\mathbb Z$ : $\forall x\in[a_k,a_{k+1})$, an immediate induction gives $f(x)=f(x^{n^{-k}})^{n^k}$ 

And $x^{n^{-k}}\in[2,2^n)$ and so $f(x)$ for $x>1$ is uniquely defined by the knowledge of $f(x)$ over $[2,2^n)$

Then, consider $x>1$ :
$a_k\le x<a_{k+1}$ $\iff$ $2^{n^k}\le x<2^{2^{k+1}}$ $\iff$ $n^k\le \frac{\ln x}{\ln 2}<n^{k+1}$

$\iff$ $k\ln n\le \ln(\ln x)-\ln(\ln 2)<(k+1)\ln n$ $\iff$ $k\le\frac{\ln(\ln x)-\ln(\ln 2)}{\ln n}<k+1$

$\iff$ $k=\left\lfloor\frac{\ln(\ln x)-\ln(\ln 2)}{\ln n}\right\rfloor$

And so, using $f(x)=f(x^{n^{-k}})^{n^k}$, we get $f(x)=f(x^{n^{-\left\lfloor\frac{\ln(\ln x)-\ln(\ln 2)}{\ln n}\right\rfloor}})^{n^\left\lfloor\frac{\ln(\ln x)-\ln(\ln 2)}{\ln n}\right\rfloor}$ 

So,choosing any $h(x)$ from $[2,2^n)\to\mathbb R^+$, we get :

$\forall x>1$ : $f(x)=h(x^{n^{-\left\lfloor\frac{\ln(\ln x)-\ln(\ln 2)}{\ln n}\right\rfloor}})^{n^\left\lfloor\frac{\ln(\ln x)-\ln(\ln 2)}{\ln n}\right\rfloor}$ 

\begin{bolded}which is slightly different from my previous post (I forgot the power inside the parenthesis). Sorry \end{bolded}\end{underlined} :oops:

And same for $x<1$
\end{solution}
*******************************************************************************
-------------------------------------------------------------------------------

\begin{problem}[Posted by \href{https://artofproblemsolving.com/community/user/83439}{Zeus93}]
	Given $ f: \mathbb Z\to \mathbb R$ which satisfies $f(n) = n-3 $ if $n\ge 1000$ and $f(n) = f(f(n+5))$ if $n <1000$. Find the value of $f(94)$.
	\flushright \href{https://artofproblemsolving.com/community/c6h384856}{(Link to AoPS)}
\end{problem}



\begin{solution}[by \href{https://artofproblemsolving.com/community/user/29428}{pco}]
	\begin{tcolorbox}Given $ f: Z\rightarrow R$ which satisfies 
$f(n) = n-3 $ if $n\ge 1000$ and $f(n) = f(f(n+5))$ if $n <1000$ . Find the value of $f(94) $\end{tcolorbox}
$f(999)=f(f(1004))=f(1001)=998$
$f(998)=f(f(1003))=f(1000)=997$
$f(997)=f(f(1002)=f(999)=998$

and so $f(f(998))=998$

$f(94)=f(f(99)=f(f(f(104)))=...=f^{[p+1]}(94+5p)$ $\forall p\le 181$ (so that $94+5p<1000$)

So $f(94)=f^{[182]}(999)=f^{[181]}(998)=f^{[179]}(998)=...=f^{[3]}(998)=f(998)=997$

So $\boxed{f(94)=997}$

\begin{bolded}Nota 1\end{bolded}\end{underlined}: According to me, we should normally prove that this mandatory value fits, that's to say we should prove the existence of $f(x)$. I did not prove that because the problem begins with "given f..." and so I considered that existence of $f(x)$ is supposed by the problem statement.

\begin{bolded}Nota 2 (later) \end{bolded}\end{underlined}: Btw, it's easy to show that such a function exists :
Choose $f(n)=n-3$ $\forall n\ge 1000$ and $f(n)=\frac{1995-(-1)^n}2$ $\forall n<1000$
(and this is the unique one, but uniqueness is not mandatory in order to conclude the proof)
\end{solution}
*******************************************************************************
-------------------------------------------------------------------------------

\begin{problem}[Posted by \href{https://artofproblemsolving.com/community/user/83439}{Zeus93}]
	Find all increasing function $f: \mathbb N \rightarrow \mathbb N$ satisfy:
\[f(y(f(x))=x^2f(xy), \quad \forall x\in \mathbb N.\]
	\flushright \href{https://artofproblemsolving.com/community/c6h385174}{(Link to AoPS)}
\end{problem}



\begin{solution}[by \href{https://artofproblemsolving.com/community/user/29428}{pco}]
	\begin{tcolorbox}Find all increasing function $f: N^* \rightarrow N^*$ satisfy:
$f(y(f(x))=x^2f(xy) \forall x\in N^*$\end{tcolorbox}
Let $P(x,y)$ be the assertion $f(yf(x))=x^2f(xy)$

$P(1,x)$ $\implies$ $f(f(1)x)=f(x)$ and so $f(1)=1$ since $f(x)$ is increasing (and so injective)

$P(x,1)$ $\implies$ $f(f(x))=x^2f(x)$
$P(x,x^2)$ $\implies$ $f(x^2f(x))=x^2f(x^3)$ anbd so $f(f(f(x)))=x^2f(x^3)$
$P(f(x),1)$ $\implies$ $f(f(f(x)))=f(x)^2f(f(x))=x^2f(x)^3$
And so $f(x^3)=f(x)^3$

$P(xy,x^2)$ $\implies$ $f(x^2f(xy))=x^2y^2f(x^3y)$ and so $f(f(yf(x)))=x^2y^2f(x^3y)$ 
$P(yf(x),1)$ $\implies$ $f(f(yf(x)))=y^2f(x)^2f(yf(x))=x^2y^2f(x)^2f(xy)$
And so $f(x^3y)=f(x)^2f(xy)$

Setting $y\to y^3$ in this last equality, we get $f(x^3y^3)=f(x)^2f(xy^3)=f(x)^2f(y)^2f(xy)$

And since $f(x^3y^3)=f(xy)^3$, we get $f(xy)^3=f(x)^2f(y)^2f(xy)$ and so $f(xy)=f(x)f(y)$

It's a classical result that increasing solutions of $f(xy)=f(x)f(y)$ are $f(x)=x^n$ for some $n>0$

Plugging this in original equation, we get $n=2$ and the solution $\boxed{f(x)=x^2}$
\end{solution}
*******************************************************************************
-------------------------------------------------------------------------------

\begin{problem}[Posted by \href{https://artofproblemsolving.com/community/user/27047}{mathlink}]
	Find all functions $f: \mathbb R\to \mathbb R$ such that
\[f(x^n+2f(y))=(f(x))^n +y+f(y), \quad \forall x, y \in \mathbb R,\quad n \in \mathbb Z_{\geq 2}.\]
	\flushright \href{https://artofproblemsolving.com/community/c6h385331}{(Link to AoPS)}
\end{problem}



\begin{solution}[by \href{https://artofproblemsolving.com/community/user/93464}{taiiyama}]
	sorry i just got one solution which is f(x)=x,
this is what i can say by seeing your question .i will see what path this requires :) 
i hope i will get more solutions
\end{solution}



\begin{solution}[by \href{https://artofproblemsolving.com/community/user/29428}{pco}]
	\begin{tcolorbox}Find all funtion $f: R\to R$ such that
$f(x^n+2f(y))=(f(x))^n +y+f(y) \forall x, y \in R$
(n is an integer greater than 1)\end{tcolorbox}
Let $P(x,y)$ be the assertion $f(x^n+2f(y))=f(x)^n+y+f(y)$

1) If $n$ is even
============
If $f(a)=f(b)$, then comparing $P(0,a)$ and $P(0,b)$ implies $a=b$ and so $f(x)$ is injective.
Comparing $P(x,0)$ and $P(-x,0)$ implies $f(x)^n=f(-x)^n$ and so $f(-x)=-f(x)$ since $f(x)$ is injective.
As a consequence, $f(0)=0$
$P(x,0)$ $\implies$ $f(x^n)=f(x)^n$ and $P(x,y)$ may be written $f(x^n+2f(y))=f(x^n)+y+f(y)$
So $f(x+2f(y))=f(x)+y+f(y)$ $\forall x\ge 0$
Setting $y\to -y$ in this equation, we get $f(x-2f(y))=f(x)-y-f(y)$ and so $f(-x+2f(y))=f(-x)+y+f(y)$

So new assertion $Q(x,y)$ : $f(x+2f(y))=f(x)+y+f(y)$ $\forall x$
$Q(x,-x)$ $\implies$ $f(x-2f(x))=-x$ and so $f(x)$ is surjective.

$Q(0,y)$ $\implies$ $f(2f(y))=y+f(y)$ and so $Q(x,y)$ may be written $f(x+2f(y))=f(x)+f(2f(y))$ and surjectivity implies $f(x+y)=f(x)+f(y)$

This is Cauchy's equation and since $f(x^n)=f(x)^n$, we know that $f(x)>0$ $\forall x>0$ and $f(x)$ is increasing and the only solution is $f(x)=ax$ and plugging this in original equation, we get $\boxed{f(x)=x}$

2) If $n$ is odd
===========
2.1) $f(x+y)=f(x)+f(y)-f(0)$
--------------------------------------------
$P(x,y)$ $\implies$ $f(x^n+2f(y))=f(x)^n+y+f(y)$
$P(x,0)$ $\implies$ $f(x^n+2f(0))=f(x)^n+f(0)$
Subtracting these two lines, we get $f(x^n+2f(y))=f(x^n+2f(0))+y+f(y)-f(0)$

And, since $n$ is odd : $f(x+2f(y))=f(x+2f(0))+y+f(y)-f(0)$
So $f(x+2f(y)-2f(0))=f(x)+y+f(y)-f(0)$

Let then $g(x)=f(x)-f(0)$. This last equation becomes $Q(x,y)$ : $g(x+2g(y))=g(x)+y+g(y)$ true $\forall x,y$

$Q(-x-2g(-x),-x)$ $\implies$ $g(-x-2g(-x))=x$ and so $g(x)$ is surjective.

$Q(x,y)$ $\implies$ $g(x+2g(y))=g(x)+y+g(y)$
$Q(0,y)$ $\implies$ $g(2g(y))=y+g(y)$
Subtracting these two lines, we get $g(x+2g(y))=g(x)+g(2g(y))$ and so, since surjective : $g(x+y)=g(x)+g(y)$
And so $f(x+y)=f(x)+f(y)-f(0)$
Q.E.D.

2.2) $f(0)=0$
-----------------
(I'm a little bit ashame of this ugly proof and hope someone will find something prettier)
$P(x,0)$ $\implies$ $f(x^n+2f(0))=f(x)^n+f(0)$
$f(x^n+2f(0))=f(x^n)+f(2f(0))-f(0)$ and so $f(x^n)=f(x)^n+c$ where $c=2f(0)-f(2f(0))$

$f(x+y)=f(x)+f(y)-f(0)$ $\implies$ $f(px)=pf(x)-(p-1)f(0)$ $\forall x,\forall p\in\mathbb Z$

So $f(p^nx^n)=p^nf(x^n)-(p^n-1)f(0)=p^n(f(x)^n+c)-(p^n-1)f(0)$

But $f(p^nx^n)=f(px)^n+c=(pf(x)-(p-1)f(0))^n+c$

and so $p^n(f(x)^n+c)-(p^n-1)f(0)=(pf(x)-(p-1)f(0))^n+c$

And so the polynomial $S(z)=z^n(f(x)^n+c)-(z^n-1)f(0)-(zf(x)-(z-1)f(0))^n-c$ has infinitely many roots (since $S(p)=0$ $\forall p\in\mathbb Z$), and so is the zero polynomial.

So coefficient of $z$ in this polynomial is 0 and so $f(0)^{n-1}(f(x)-f(0))=0$ and since $f(x)=f(0)$ $\forall x$ is not a solution, we get $f(0)=0$ and $c=0$

2.3) $f(x)=x$
--------------------
So the problem is :
$f(x+y)=f(x)+f(y)$
$f(x^n)=f(x)^n$
$2f(f(x))=x+f(x)$

From the second, we get $f(1)\in\{-1,0,1\}$ but using the first and third, it only remains $f(1)=1$ and so $f(p)=p$ $\forall p\in\mathbb Z$
Then $f((x+p)^n)=f(\sum_{k=0}^n\binom nkx^kp^{n-k})=\sum_{k=0}^n\binom nkf(x^k)p^{n-k}$

But $f((x+p)^n)=(f(x)+p)^n=\sum_{k=0}^n\binom nkf(x)^kp^{n-k}$

And so $\sum_{k=0}^n\binom nk(f(x)^k-f(x^k))p^{n-k}=0$

So the polynomial $T(z)=\sum_{k=0}^n\binom nk(f(x)^k-f(x^k))z^{n-k}$ has infinitely many roots and so is all zero.

Since $n$ is odd and $n>1$, we get $n>2$ and we can look at coefficient of $z^{n-2}$ and we get $f(x^2)=f(x)^2$

And now, we get that $f(x)>0$ $\forall x>0$ and so $f(x)$ is increasing and so $f(x)=xf(1)=x$ which indeed is a solution.

Hence the result : $\boxed{f(x)=x}$
\end{solution}



\begin{solution}[by \href{https://artofproblemsolving.com/community/user/27047}{mathlink}]
	\begin{tcolorbox}

1) If $n$ is even
============
If $f(a)=f(b)$, then comparing $P(0,a)$ and $P(0,b)$ implies $a=b$ and so $f(x)$ is injective.
Comparing $P(x,0)$ and $P(-x,0)$ implies [size=150]$f(x)^n=f(-x)^n$ [color=#FF0000]and so[\/color] $f(-x)=-f(x)$ [\/size]since $f(x)$ is injective.
As a consequence, $f(0)=0$
\end{tcolorbox}

Your idea is great. But there is a small mistake:
 With $f^n(0)=f^n(-0)$ and f is injective, you can not have $f(-0)=-f(0)$. 
But it is easy to prove f(0)=0 ( another way). 
Thank u!
\end{solution}



\begin{solution}[by \href{https://artofproblemsolving.com/community/user/29428}{pco}]
	\begin{tcolorbox}[quote="pco"]

1) If $n$ is even
============
If $f(a)=f(b)$, then comparing $P(0,a)$ and $P(0,b)$ implies $a=b$ and so $f(x)$ is injective.
Comparing $P(x,0)$ and $P(-x,0)$ implies [size=150]$f(x)^n=f(-x)^n$ [color=#FF0000]and so[\/color] $f(-x)=-f(x)$ [\/size]since $f(x)$ is injective.
As a consequence, $f(0)=0$
\end{tcolorbox}

Your idea is great. But there is a small mistake:
 With $f^n(0)=f^n(-0)$ and f is injective, you can not have $f(-0)=-f(0)$. 
But it is easy to prove f(0)=0 ( another way). 
Thank u!\end{tcolorbox}

You're right, thanks, but your red quotation is misplaced.
$f(x)^n=f(-x)^n$  and $n$ even and $f(x)$ injective indeed implies $f(-x)=-f(x)$, but only for $x\ne 0$ and the wrong part is "As a consequence, $f(0)=0$" :)
\end{solution}



\begin{solution}[by \href{https://artofproblemsolving.com/community/user/87195}{SCP}]
	\begin{tcolorbox}[quote="mathlink"]Find all funtion $f: R\to R$ such that
$f(x^n+2f(y))=(f(x))^n +y+f(y) \forall x, y \in R$
(n is an integer greater than 1)\end{tcolorbox}
Let $P(x,y)$ be the assertion $f(x^n+2f(y))=f(x)^n+y+f(y)$

1) If $n$ is even
============
If $f(a)=f(b)$, then comparing $P(0,a)$ and $P(0,b)$ implies $a=b$ and so $f(x)$ is injective.
Comparing $P(x,0)$ and $P(-x,0)$ implies $f(x)^n=f(-x)^n$ and so $f(-x)=-f(x)$ since $f(x)$ is injective.
As a consequence, $f(0)=0$
$P(x,0)$ $\implies$ $f(x^n)=f(x)^n$ and $P(x,y)$ may be written $f(x^n+2f(y))=f(x^n)+y+f(y)$
So $f(x+2f(y))=f(x)+y+f(y)$ $\forall x\ge 0$
Setting $y\to -y$ in this equation, we get $f(x-2f(y))=f(x)-y-f(y)$ and so $f(-x+2f(y))=f(-x)+y+f(y)$

So new assertion $Q(x,y)$ : $f(x+2f(y))=f(x)+y+f(y)$ $\forall x$
$Q(x,-x)$ $\implies$ $f(x-2f(x))=-x$ and so $f(x)$ is surjective.

$Q(0,y)$ $\implies$ $f(2f(y))=y+f(y)$ and so $Q(x,y)$ may be written $f(x+2f(y))=f(x)+f(2f(y))$ and surjectivity implies $f(x+y)=f(x)+f(y)$

This is Cauchy's equation and since $f(x^n)=f(x)^n$, we know that $f(x)>0$ $\forall x>0$ and $f(x)$ is increasing and the only solution is $f(x)=ax$ and plugging this in original equation, we get $\boxed{f(x)=x}$

2) If $n$ is odd
===========
2.1) $f(x+y)=f(x)+f(y)-f(0)$
--------------------------------------------
$P(x,y)$ $\implies$ $f(x^n+2f(y))=f(x)^n+y+f(y)$
$P(x,0)$ $\implies$ $f(x^n+2f(0))=f(x)^n+f(0)$
Subtracting these two lines, we get $f(x^n+2f(y))=f(x^n+2f(0))+y+f(y)-f(0)$

And, since $n$ is odd : $f(x+2f(y))=f(x+2f(0))+y+f(y)-f(0)$
So $f(x+2f(y)-2f(0))=f(x)+y+f(y)-f(0)$

Let then $g(x)=f(x)-f(0)$. This last equation becomes $Q(x,y)$ : $g(x+2g(y))=g(x)+y+g(y)$ true $\forall x,y$

$Q(-x-2g(-x),-x)$ $\implies$ $g(-x-2g(-x))=x$ and so $g(x)$ is surjective.

$Q(x,y)$ $\implies$ $g(x+2g(y))=g(x)+y+g(y)$
$Q(0,y)$ $\implies$ $g(2g(y))=y+g(y)$
Subtracting these two lines, we get $g(x+2g(y))=g(x)+g(2g(y))$ and so, since surjective : $g(x+y)=g(x)+g(y)$
And so $f(x+y)=f(x)+f(y)-f(0)$
Q.E.D.

2.2) $f(0)=0$
-----------------
(I'm a little bit ashame of this ugly proof and hope someone will find something prettier)
$P(x,0)$ $\implies$ $f(x^n+2f(0))=f(x)^n+f(0)$
$f(x^n+2f(0))=f(x^n)+f(2f(0))-f(0)$ and so $f(x^n)=f(x)^n+c$ where $c=2f(0)-f(2f(0))$

$f(x+y)=f(x)+f(y)-f(0)$ $\implies$ $f(px)=pf(x)-(p-1)f(0)$ $\forall x,\forall p\in\mathbb Z$

So $f(p^nx^n)=p^nf(x^n)-(p^n-1)f(0)=p^n(f(x)^n+c)-(p^n-1)f(0)$

But $f(p^nx^n)=f(px)^n+c=(pf(x)-(p-1)f(0))^n+c$

and so $p^n(f(x)^n+c)-(p^n-1)f(0)=(pf(x)-(p-1)f(0))^n+c$

And so the polynomial $S(z)=z^n(f(x)^n+c)-(z^n-1)f(0)-(zf(x)-(z-1)f(0))^n-c$ has infinitely many roots (since $S(p)=0$ $\forall p\in\mathbb Z$), and so is the zero polynomial.

So coefficient of $z$ in this polynomial is 0 and so $f(0)^{n-1}(f(x)-f(0))=0$ and since $f(x)=f(0)$ $\forall x$ is not a solution, we get $f(0)=0$ and $c=0$

2.3) $f(x)=x$
--------------------
So the problem is :
$f(x+y)=f(x)+f(y)$
$f(x^n)=f(x)^n$
$2f(f(x))=x+f(x)$

From the second, we get $f(1)\in\{-1,0,1\}$ but using the first and third, it only remains $f(1)=1$ and so $f(p)=p$ $\forall p\in\mathbb Z$
Then $f((x+p)^n)=f(\sum_{k=0}^n\binom nkx^kp^{n-k})=\sum_{k=0}^n\binom nkf(x^k)p^{n-k}$

But $f((x+p)^n)=(f(x)+p)^n=\sum_{k=0}^n\binom nkf(x)^kp^{n-k}$

And so $\sum_{k=0}^n\binom nk(f(x)^k-f(x^k))p^{n-k}=0$

So the polynomial $T(z)=\sum_{k=0}^n\binom nk(f(x)^k-f(x^k))z^{n-k}$ has infinitely many roots and so is all zero.

Since $n$ is odd and $n>1$, we get $n>2$ and we can look at coefficient of $z^{n-2}$ and we get $f(x^2)=f(x)^2$

And now, we get that $f(x)>0$ $\forall x>0$ and so $f(x)$ is increasing and so $f(x)=xf(1)=x$ which indeed is a solution.

Hence the result : $\boxed{f(x)=x}$\end{tcolorbox}

How long do you have to think about such a functional equation?
\end{solution}



\begin{solution}[by \href{https://artofproblemsolving.com/community/user/29428}{pco}]
	\begin{tcolorbox}How long do you have to think about such a functional equation?\end{tcolorbox}

A long time (some hours) .... :)
But I think there surely is a short elegant solution that I did not see.
\end{solution}



\begin{solution}[by \href{https://artofproblemsolving.com/community/user/307628}{lebathanh}]
	can anybody show f(0)=0 (I think it is important)
\end{solution}



\begin{solution}[by \href{https://artofproblemsolving.com/community/user/307628}{lebathanh}]
	I can show it : assume not exist u s.t f(u)=0 .I sub (x,-f(x)^n) in F.E(case n even): f(x^n-2f(-f(x)^n)) =f(-(f(x)^n) because f(x) not equal 0 with x is on R then f is odd associate f injective then x^n=f(f(x)^n) then sub x=0 => f(f(0)^n)=0 (contract) => exist u: f(u)=0 if u is not equal 0 then f(-u)=-f(u)=0 => f(u)=f(-u) => u=0 (contract)
\end{solution}
*******************************************************************************
-------------------------------------------------------------------------------

\begin{problem}[Posted by \href{https://artofproblemsolving.com/community/user/31067}{ridgers}]
	Find all functions $f:\mathbb R \to \mathbb R$ differentiable on real numbers that have the following property: for every $3$ real numbers $x_1,x_2$, and $x_3$ that form an arithmetic progression, $f(x_1),f(x_2)$, and $f(x_3)$ also form an arithmetic progression.
	\flushright \href{https://artofproblemsolving.com/community/c6h385634}{(Link to AoPS)}
\end{problem}



\begin{solution}[by \href{https://artofproblemsolving.com/community/user/29428}{pco}]
	\begin{tcolorbox}Find all the function $f(x)$ differentiable in real numbers that have the following property:  For every 3 values $x_1,x_2,x_3$ that for an arithmetic progression also the values $f(x_1),f(x_2),f(x_3)$ form an arithmetic progression.\end{tcolorbox}
So $f(x+y)-f(x)=f(x)-f(x-y)$ $\iff$ $f(x)=\frac{f(x+y)+f(x-y)}2$ $\iff$ $f(\frac{x+y}2)=\frac{f(x)+f(y)}2$

This is a very classical equation, solved many many times in this forum, whose continuous solutions are only $\boxed{f(x)=ax+b}$
\end{solution}
*******************************************************************************
-------------------------------------------------------------------------------

\begin{problem}[Posted by \href{https://artofproblemsolving.com/community/user/93909}{magical}]
	Find all functions $f: \mathbb R \to \mathbb R$ such that for all reals $x$ and $y$,
\[f(x+y)+f(x)f(y)=f(xy)+f(x)+f(y).\]
	\flushright \href{https://artofproblemsolving.com/community/c6h385670}{(Link to AoPS)}
\end{problem}



\begin{solution}[by \href{https://artofproblemsolving.com/community/user/29428}{pco}]
	\begin{tcolorbox}Find $f:R \to R$ satisfy:
$f(x+y)+f(x)f(y)=f(xy)+f(x)+f(y)$\end{tcolorbox}
I'm surprised that you continue to require some help without at least answering questions asked by those who try to help you. Please answer my question in http://www.artofproblemsolving.com/Forum/viewtopic.php?f=38&t=385038

I'll try again to help you on this problem but ...

Let $P(x,y)$ be the assertion $f(x+y)+f(x)f(y)=f(xy)+f(x)+f(y)$

$P(x,0)$ $\implies$ $f(0)(f(x)-2)=0$ and so $f(0)=0$ or $f(x)=2$ $\forall x$
The function $f(x)=2$ $\forall x$ is indeed a solution and we'll consider from now that $f(0)=0$

Let $a=f(1)$
$P(x,1)$ $\implies$ $f(x+1)=(2-a)f(x)+a$

1) If $a\ne 1$ : $f(x)=0$ $\forall x$
Notice that $a\ne 2$, else $P(-1,1)$ would imply $f(0)=2$, impossible.
Notice too that $a\ne 3$, else this would imply $f(x+1)=3-f(x)$ and $f(x+2)=f(x)$ and comparaison of $P(x,y)$ and $P(x,y+2)$ would imply $f(x)$ constant, which is impossible if $f(1)=3$

$f(x+1)=(2-a)f(x)+a$ $\implies$ $f(x+n)=(2-a)^n(f(x)+\frac a{1-a})-\frac a{1-a}$ and so $f(n)=((2-a)^n-1)\frac a{1-a})$
It's immediate to see, plugging this in original equation, that this can be a solution only if $a=0$, else RHS contains a summand $(2-a)^{xy}$ which cant be cancelled.

So $a=0$ and $f(x+1)=2f(x)$
$P(x,y+1)$ $\implies$ $2f(x+y)+2f(x)f(y)=f(xy+x)+f(x)+2f(y)$
Subtracting $2P(x,y)$ from this equality, we get $f(xy+x)=f(xy)+f(x)$ and so $f(x+y)=f(x)+f(y)$
Then $P(x,y)$ implies $f(xy)=f(x)f(y)$
And the system $f(x+y)=f(x)+f(y)$ and $f(xy)=f(x)f(y)$ is very classical and has two solutions $f(x)=0$ and $f(x)=x$
And since $f(1)=0$, we get $f(x)=0$ $\forall x$, which indeed is a solution.

2) If $a=1$ : $f(x)=x$ $\forall x$

$f(x+1)=f(x)+1$
$P(x,y+1)$ $\implies$ $f(x+y)+1+f(x)(f(y)+1)=f(xy+x)+f(x)+f(y)+1$
Subtracting $P(x,y)$ from this equation, we get $f(xy+x)=f(xy)+f(x)$ and so $f(x+y)=f(x)+f(y)$
Then $P(x,y)$ implies $f(xy)=f(x)f(y)$
And the system $f(x+y)=f(x)+f(y)$ and $f(xy)=f(x)f(y)$ is very classical and has two solutions $f(x)=0$ and $f(x)=x$
And since $f(1)=1$, we get $f(x)=x$ $\forall x$, which indeed is a solution.

Synthesis of solutions :
$f(x)=0$ $\forall x$
$f(x)=2$ $\forall x$
$f(x)=x$ $\forall x$
\end{solution}



\begin{solution}[by \href{https://artofproblemsolving.com/community/user/141397}{subham1729}]
	Sorry to revive , but it's a nice problem.

\begin{tcolorbox}Find $f:R \to R$ satisfy:
$f(x+y)+f(x)f(y)=f(xy)+f(x)+f(y)$\end{tcolorbox}

$P(x,y): f(x+y)+f(x)f(y)=f(xy)+f(x)+f(y)$
$P(x,y+z): f(x+y+z)=f(x) + f(y + z) + f(xy + xz)-f(x)f(y + z)
=f(x) + f(y) + f(z) + f(xy) + f(yz) + f(zx) + f(x)f(y)f(z)-f(x)f(y)-f(y)f(z)-f(z)f(x)+f(x^2yz)-f(xy)f(xz)-f(x)f(yz).$

From $P(y,x+z),P(x,y+z)$ we get $f(x^2yz)-f(xy)f(xz)-f(x)f(yz) = f(xy^2z)-f(xy)f(yz)-f(y)f(xz).$
Put $y=1$ and we get $f(x^2z) = (a-1)f(xz) + f(x)f(xz),$ with $a=2-f(1)-----(1)$

$P(x,xz)$ and $(1)$ gives $f(x + xz) = af(xz) + f(x).$

Putting $z=0$ we get $af(0)=0$ now if $a=0$ then setting $x=1$ gives $f(z+1)=f(1)=2 \forall z$.
So $f(0)=0$ now by setting $z=-1$ we've $f(x)=-af(-x)$ replacing $x$ by $-x$ we get $f(x)=a^2f(x)$

Now it's very easy to see $a$ can't be $-1$. So if $a=1$ then we've $f(x+xz)=f(x)+f(xz)$
Putting $xz=y$ we get $f(x+y)=f(x)+f(y)$ and now $P(x,y)$ implies $f(xy)=f(x)f(y)$
Which is well known Cauchy equation and hence $f(x)=x \forall x$.

So now if $a^2$ isn't $1$ then $f(x)=2 \forall x$.
\end{solution}
*******************************************************************************
-------------------------------------------------------------------------------

\begin{problem}[Posted by \href{https://artofproblemsolving.com/community/user/86713}{zZzZzZzZz}]
	Find all functions $f: \mathbb R \to \mathbb R$ such that for all reals $x$ and $y$,
\[f(x^2-y^2)=xf(x)-yf(y).\]
	\flushright \href{https://artofproblemsolving.com/community/c6h385721}{(Link to AoPS)}
\end{problem}



\begin{solution}[by \href{https://artofproblemsolving.com/community/user/29428}{pco}]
	\begin{tcolorbox}Find $f : R \to R $ satisfy $f(x^2-y^2)=xf(x)-yf(y)$\end{tcolorbox}
Let $P(x,y)$ be the assertion $f(x^2-y^2)=xf(x)-yf(y)$

$P(0,0)$ $\implies$ $f(0)=0$
$P(x,0)$ $\implies$ $f(x^2)=xf(x)$
$P(0,x)$ $\implies$ $f(-x^2)=-xf(x)$ 

So $f(-x)=-f(x)$

$P(x,y)$ may then be written $f(x^2-y^2)=f(x^2)-f(y^2)$ and so $f(x-y)=f(x)-f(y)$ $\forall x,y\ge 0$

Writing this $f(x-y)=f(x)+f(-y)$ or $f(-x+y)=f(-x)+f(y)$, this implies $f(x+y)=f(x)+f(y)$ $\forall x,y$ with opposite signs
But then, if $x$ and $y$ have same signs, then $-x$ and $x+y$ have opposite signs and $f(x+y-x)=f(-x)+f(x+y)$ and so, again, $f(x+y)=f(x)+f(y)$

So $f(x+y)=f(x)+f(y)$ $\forall x,y$

Then $f(x^2)=xf(x)$ implies $f((x+1)^2)=(x+1)f(x+1)$ and so $f(x^2+2x+1)=(x+1)(f(x)+f(1))$ and so :
$f(x^2)+2f(x)+f(1)=xf(x)+xf(1)+f(x)+f(1)$ and since we know that $f(x^2)=xf(x)$, it remains $f(x)=xf(1)$ which indeed is a solution.

Hence the answer : $\boxed{f(x)=ax}$
\end{solution}
*******************************************************************************
-------------------------------------------------------------------------------

\begin{problem}[Posted by \href{https://artofproblemsolving.com/community/user/3182}{Kunihiko_Chikaya}]
	Consider all functions $ f$ defined on positive integers and taking positive integer values such that
\[(x+y)f(x)\leq x^2+f(xy)+110\] for all positive integers $x$ and $y$. 

Find the possible maximum and minimum values of $f(23)+f(2011)$.
	\flushright \href{https://artofproblemsolving.com/community/c6h385793}{(Link to AoPS)}
\end{problem}



\begin{solution}[by \href{https://artofproblemsolving.com/community/user/29428}{pco}]
	\begin{tcolorbox}Consider all functions $ f$, defined on positive integers and taking positive integers such that
$(x+y)f(x)\leq x^2+f(xy)+110$ for any positive integer $ x,\ y$.
Find the possible maximum and minimum values of $f(23)+f(2011)$.\end{tcolorbox}
Let $P(x,y)$ be the assertion $(x+y)f(x)\le x^2+f(xy)+110$

$P(x,1)$ $\implies$ $f(x)\le x+\frac{110}x$

$P(1,x)$ $\implies$ $xf(1)+f(1)-111\le f(x)$

And so $xf(1)+f(1)-111\le x+\frac{110}x$. 
Setting $x\to+\infty$ in this last inequation, we get $f(1)=1$

And so $x-110\le f(x)\le x+\frac{110}x$

So $(x+y)f(x)\le x^2+f(xy)+110\le x^2+xy+\frac{110}{xy}+110$

And so $f(x)\le x+\frac{110}{xy(x+y)}+\frac {110}{x+y}$ 
Setting $y\to +\infty$ in this last equation, we get $f(x)\le x$

And so $x-110\le f(x)\le x$ and, since $f(x)\ge 1$ : $\max(x-110,1)\le f(x)\le x$

So $1+1901\le f(23)+f(2011)\le 23+2011$ and $1902\le f(23)+f(2011)\le 2034$

And since $f(x)=x$ is indeed a solution, $2034$ as upper bound may be reached.
And since $f(x)=\max(1,x-110)$ is indeed a solution, $1902$ as lower bound may be reached.
[hide="why ?"]If $x\le 111$ and $xy\le 111$, the inequation is $x+y\le x^2+111$ obviously true
If $x\le 111$ and $xy>111$, the inequation is $x+y\le x^2+xy$ obviously true
If $x>111$ and so $xy>111$, the inequation is $(x+y)(x-110)\le x^2+xy$ obviously true.[\/hide]

Hence the answer $\boxed{1902\le f(23)+f(2011)\le 2034}$
\end{solution}



\begin{solution}[by \href{https://artofproblemsolving.com/community/user/3182}{Kunihiko_Chikaya}]
	That's correct.
\end{solution}
*******************************************************************************
-------------------------------------------------------------------------------

\begin{problem}[Posted by \href{https://artofproblemsolving.com/community/user/92753}{WakeUp}]
	Find all functions $f$ of two variables, whose arguments $x,y$ and values $f(x,y)$ are positive integers, satisfying the following conditions (for all positive integers $x$ and $y$):
\begin{align*} f(x,x)& =x,\\ f(x,y)& =f(y,x),\\ (x+y)f(x,y)& =yf(x,x+y).\end{align*}
	\flushright \href{https://artofproblemsolving.com/community/c6h385986}{(Link to AoPS)}
\end{problem}



\begin{solution}[by \href{https://artofproblemsolving.com/community/user/29428}{pco}]
	\begin{tcolorbox}Find all functions $f$ of two variables, whose arguments $x,y$ and values $f(x,y)$ are positive integers, satisfying the following conditions (for all positive integers $x$ and $y$):
\begin{align*} f(x,x)& =x,\\ f(x,y)& =f(y,x),\\ (x+y)f(x,y)& =yf(y,x+y).\end{align*}\end{tcolorbox}
Hello WakeUp! could you confirm us that there is a closed form for the solution and that you'll be able to post it when asked ?

I have some doubts about existence of such closed form.
Some results I've got up to now :

a) there exist a unique solution
b) $f(ax,ay)=af(x,y)$
c) $f(1,n)=n!$

And also, for example :
$f(x,nx+1)=x!(1+\frac 1x)(2+\frac 1x)(3+\frac 1x)...(n+\frac 1x)$

And I wonder what closed form could give the above result :?:
\end{solution}



\begin{solution}[by \href{https://artofproblemsolving.com/community/user/92753}{WakeUp}]
	\begin{tcolorbox}
Hello WakeUp! could you confirm us that there is a closed form for the solution and that you'll be able to post it when asked ?

I have some doubts about existence of such closed form.
Some results I've got up to now :

a) there exist a unique solution
b) $f(ax,ay)=af(x,y)$
c) $f(1,n)=n!$

And also, for example :
$f(x,nx+1)=x!(1+\frac 1x)(2+\frac 1x)(3+\frac 1x)...(n+\frac 1x)$

And I wonder what closed form could give the above result :?:\end{tcolorbox}

Hi pco, much to my embarrassment there was a typo that is now fixed :blush:  I'm sure you can complete your solution now.
\end{solution}



\begin{solution}[by \href{https://artofproblemsolving.com/community/user/81000}{Aksenov239}]
	\begin{tcolorbox}Find all functions $f$ of two variables, whose arguments $x,y$ and values $f(x,y)$ are positive integers, satisfying the following conditions (for all positive integers $x$ and $y$):
\begin{align*} f(x,x)& =x,\\ f(x,y)& =f(y,x),\\ (x+y)f(x,y)& =yf(x,x+y).\end{align*}\end{tcolorbox}
Firstly, we can prove, that $\all a > 0 : f(x, ax) = af(x, x)$.
Proof : 
$(x + (a - 1)x)f(x, (a - 1)x) = (a - 1)xf(x,ax); \Rightarrow f(x, ax) = \frac{a}{a - 1}f(x, (a - 1)x) =$
$=af(x, x) = ax$.

After that:
$f(x, px + q) = \frac{f(px, px + q)}{p} = \frac{(px + q)f(px, q)}{pq}$. (We take - $x' = px$ and $y' = q$;)
$\frac{(px + q)f(px, q)}{pq} = \frac{px + q}{q}f(q, x)$.

Let's watch on new function $g$ : $g(x, y) = \frac{xy}{f(x, y)}$.

$\frac{xy}{f(x, y)} = \frac{x(px + q)}{f(x, px + q)} = \frac{x(px + q)}{\frac{px + q}{q}f(q, x)} = \frac{qx}{f(q, x)} = ... = \frac{d(kd)}{f(d, kd)} = d$. (Where d = (x, y).)

From that - we can see, that $\frac{xy}{f(x, y)} = (x, y); \Rightarrow f(x, y) = [x, y]$!

That's all!  :) !
\end{solution}
*******************************************************************************
-------------------------------------------------------------------------------

\begin{problem}[Posted by \href{https://artofproblemsolving.com/community/user/92753}{WakeUp}]
	Let $\mathbb{R}$ be the set of all real numbers. Find all functions $f:\mathbb{R}\rightarrow\mathbb{R}$ satisfying for all $x,y\in\mathbb{R}$ the equation $f(x)+f(y)=f(f(x)f(y))$.
	\flushright \href{https://artofproblemsolving.com/community/c6h385996}{(Link to AoPS)}
\end{problem}



\begin{solution}[by \href{https://artofproblemsolving.com/community/user/29428}{pco}]
	\begin{tcolorbox}Let $\mathbb{R}$ be the set of all real numbers. Find all functions $f:\mathbb{R}\rightarrow\mathbb{R}$ satisfying for all $x,y\in\mathbb{R}$ the equation $f(x)+f(y)=f(f(x)f(y))$.\end{tcolorbox}
Let $P(x,y)$ be the assertion $f(x)+f(y)=f(f(x)f(y))$
Let $A=f(\mathbb R)$

Let $u\in A$
Functional equation implies that $a,b\in A\implies a+b\in A$ and so $nu\in A$ $\forall n\in\mathbb N$

Let $x_1$ such that $f(x_1)=u$
Let $x_2$ such that $f(x_2)=2u$
Let $x_4$ such that $f(x_4)=4u$

$P(x_1,x_4)$ $\implies$ $5u=f(4u^2)$
$P(x_2,x_2)$ $\implies$ $4u=f(4u^2)$
So $4u=5u$ and $u=0$

So $A=\{0\}$ and so $\boxed{f(x)=0}\forall x$ which indeed is a solution.
\end{solution}



\begin{solution}[by \href{https://artofproblemsolving.com/community/user/93044}{nguyenhung}]
	* Assume that there is \begin{bolded}not\end{bolded} exist a real number $a$ such that $f(a)=0$. let $y=b$ ($f(b) \ne 0$), we get $f(x)+q=f(qf(x))$. So, we can prove that $f(x)$ is a one-to-one function. Hence $\exists t, f\left( t \right) = 0$, which is not true, because of our assumption.
* Otherwise, $\exists t, f\left( t \right) = 0$. Let $y=t$, we get $f(x)=f(0)=k:const$
From the condition, we infer $k=0$ so $\boxed{f\left( x \right) = 0}$
\end{solution}



\begin{solution}[by \href{https://artofproblemsolving.com/community/user/29428}{pco}]
	\begin{tcolorbox}* Assume that there is \begin{bolded}not\end{bolded} exist a real number $a$ such that $f(a)=0$. let $y=b$ ($f(b) \ne 0$), we get $f(x)+q=f(qf(x))$. So, we can prove that $f(x)$ is a one-to-one function. Hence $\exists t, f\left( t \right) = 0$, which is not true, because of our assumption.
\end{tcolorbox}
I'm sorry, but I understood nothing to this step.
1) you seem to conclude from "one to one" property that $\exists t, f\left( t \right) = 0$. I dont see why. "one to one" means injective.  And I dont see how injectivty gives you the claimed result :oops:

2) I dont see how you can conclude anything from $f(x)+q=f(qf(x))$ where $x\in\mathbb R$ and $q\in f(\mathbb R)$ : neither injection, neither surjection.
\end{solution}



\begin{solution}[by \href{https://artofproblemsolving.com/community/user/93044}{nguyenhung}]
	\begin{tcolorbox}I'm sorry, but I understood nothing to this step.
1) you seem to conclude from "one to one" property that $\exists t, f\left( t \right) = 0$. I dont see why. "one to one" means injective.  And I dont see how injectivty gives you the claimed result :oops:

2) I dont see how you can conclude anything from $f(x)+q=f(qf(x))$ where $x\in\mathbb R$ and $q\in f(\mathbb R)$ : neither injection, neither surjection.\end{tcolorbox}

Oh.. yes. You're right. My idea has a lot of troubles.  :wacko:
\end{solution}
*******************************************************************************
-------------------------------------------------------------------------------

\begin{problem}[Posted by \href{https://artofproblemsolving.com/community/user/67223}{Amir Hossein}]
	Determine all real functions $f(x)$ that are defined and continuous on the interval $(-1, 1)$ and that satisfy the functional equation
\[f(x+y)=\frac{f(x)+f(y)}{1-f(x) f(y)} \qquad (x, y, x + y \in (-1, 1)).\]
	\flushright \href{https://artofproblemsolving.com/community/c6h386060}{(Link to AoPS)}
\end{problem}



\begin{solution}[by \href{https://artofproblemsolving.com/community/user/29428}{pco}]
	\begin{tcolorbox}Determine all real functions $f(x)$ that are defined and continuous on the interval $(-1, 1)$ and that satisfy the functional equation
\[f(x+y)=\frac{f(x)+f(y)}{1-f(x) f(y)} \qquad (x, y, x + y \in (-1, 1)).\]\end{tcolorbox}
We easily get $f(0)=0$
Let $u(x)=\arctan(f(x))$ : $u(x)$ is continuous and the equation becomes $\tan(u(x+y))=\tan(u(x)+u(y))$

So $u(x+y)=u(x)+u(y)+k(x,y)\pi$ with $k(x,y)\in\mathbb Z$
But continuity of $u(x)$ plus the fact that $u(0)=0$ implies $u(x+y)=u(x)+u(y)$ $\forall x,y,x+y\in(-1,1)$

It's rather easy, even with the constraint $x,y,x+y\in(-1,1)$ to conclude $u(x)=ax$ $\forall x\in(-1,1)$

Hence the answer : $\boxed{f(x)=\tan(ax)}$ for any $a$ such that $|a|\le \frac{\pi}2$ in order to have continuity.
\end{solution}
*******************************************************************************
-------------------------------------------------------------------------------

\begin{problem}[Posted by \href{https://artofproblemsolving.com/community/user/97012}{tuanhoangnhi}]
	Find all functions $f: \mathbb Z \to \mathbb Z$ such that $f(-1) = f(1)$ and for all integers $x$ and $y$,
\[f(x) + f(y) = f(x+2xy) + f(y-2xy).\]
	\flushright \href{https://artofproblemsolving.com/community/c6h386360}{(Link to AoPS)}
\end{problem}



\begin{solution}[by \href{https://artofproblemsolving.com/community/user/29428}{pco}]
	\begin{tcolorbox}find all functions f : Z ----->Z such that f(-1) = f(1) and
f(x) + f(y) = f(x+2xy) + f(y-2xy) for all integers x,y.\end{tcolorbox}
Let $f(1)=f(-1)=a$
Let $P(x,y)$ be the assertion $f(x)+f(y)=f(x+2xy)+f(y-2xy)$

Let $A=\{x$ such that $f(x)=f(-x)=a\}$
$1\in A$

1) $x\in A$ $\implies$ $2x+1\in A$
==================================
Let $x\in A$
$P(-x,-1)$ $\implies$ $f(-x)+f(-1)=f(x)+f(-1-2x)$ $\implies$ $f(-2x-1)=a$
$P(1,x)$ $\implies$ $f(1)+f(x)=f(1+2x)+f(-x)$ $\implies$ $f(2x+1)=a$
So $f(2x+1)=f(-2x-1)=a$ and so $2x+1\in A$
Q.E.D.

2) $f(x)=f(-x)$ $\implies$ $f(x-1)=f(1-x)$
=========================================
Let $x$ such that $f(x)=f(-x)$
$P(1,-x)$ $\implies$ $f(1)+f(-x)=f(1-2x)+f(x)$ and so $f(1-2x)=a$
$P(1-x,-1)$ $\implies$ $f(1-x)+f(-1)=f(x-1)+f(1-2x)$ and so $f(1-x)=f(x-1)$
Q.E.D

3) $f(x)=f(-x)$ $\forall x$ and $f(2x+1)=a$ $\forall x$
======================================
From 1) and since $1\in A$, we deduce $1\in A$, $3\in A$, $7\in A$, ..., $2^n-1\in A$ ...
So we can find in $A$ numbers as great as we want.
Using then 2) as many times as we want, we get thet $f(x)=f(-x)$ $\forall x$
Then $P(1,x)$ $\implies$ $f(1)+f(x)=f(1+2x)+f(-x)$ $\implies$ $f(2x+1)=a$
Q.E.D.

4)$f((2k+1)x)=f(x)$ $\forall x,k$
=================================
$P(x,2k+1)$ $\implies$ $f(x)+f(2k+1)=f(x(4k+3))+f((2k+1)(1-2x))$ and so, using 3) : $f(x)=f(x(4k+3))$
$P(-x,-2k-1)$ $\implies$ $f(-x)+f(-2k-1)=f(x(4k+1))+f(-(2k+1)(2x+1))$ and so, using 2) and 3) : $f(x)=f(-x)=f(x(4k+1))$
So $f(x)=f(x(2k+1))$
Q.E.D.

5) General solution
===================
From $f(x)=f(x(2k+1))$, we get that $f(x)=h(v_2(x))$
And since $v_2(x)=v_2(x(2y+1))$ and $v_2(y)=v_2(y(1-2x))$, we get that any $h(x)$ is a solution.
Hence the answer :

$\boxed{f(x)=h(v_2(x))}$ where $h(x)$ is any function from $\mathbb N\cup\{0\}\to \mathbb Z$
\end{solution}
*******************************************************************************
-------------------------------------------------------------------------------

\begin{problem}[Posted by \href{https://artofproblemsolving.com/community/user/27047}{mathlink}]
	Determine all functions $f: \mathbb R \to \mathbb R$ with the property
\[f(f(x)+y)=2x+f(f(y)-x), \quad \forall x,y \in \mathbb R.\]
	\flushright \href{https://artofproblemsolving.com/community/c6h386509}{(Link to AoPS)}
\end{problem}



\begin{solution}[by \href{https://artofproblemsolving.com/community/user/72731}{goodar2006}]
	see ISL 2002 for a solution
\end{solution}



\begin{solution}[by \href{https://artofproblemsolving.com/community/user/97349}{FBI__}]
	Can you post full solution  :maybe:  ?
\end{solution}



\begin{solution}[by \href{https://artofproblemsolving.com/community/user/29428}{pco}]
	\begin{tcolorbox}Determine all functions $f:R \to R$ with the property
$f(f(x)+y)=2x+f(f(y)-x) \forall x,y \in R$\end{tcolorbox}
Here is a solution :

Let $P(x,y)$ be the assertion $f(f(x)+y)=2x+f(f(y)-x)$
Let $f(0)=a$

1) $f(x)$ is surjective
=======================
$P(x,-f(x))$ $\implies$ $a-2x=f(f(-f(x))-x)$
Choosing $x=\frac{a-u}2$, we get $u=f(\text{something})$
Q.E.D.

2) $f(x)$ is injective
======================
Let $y_1,y_2$ such that $f(y_1)=f(y_2)$
Comparing $P(x,y_1)$ and $P(x,y_2)$, we get $f(y_1+f(x))=f(y_2+f(x))$ and, since $f(x)$ is surjective, $f(x+y_1)=f(x+y_2)$ $\forall x$

So $f(x)=f(x+T)$ $\forall x$ where $T=y_2-y_1$
Then $P(x+T,y)$ $\implies$ $f(f(x)+y)=2(x+T)+f(f(y)-x)$ and so $T=0$
Q.E.D

3) $f(x)=x+a$
=============
$P(0,x)$ $\implies$ $f(x+a)=f(f(x))$ and so, since injective : $f(x)=x+a$ which indeed is a solution.

Hence the answer : $\boxed{f(x)=x+a}$ $\forall x$
\end{solution}



\begin{solution}[by \href{https://artofproblemsolving.com/community/user/97349}{FBI__}]
	\begin{tcolorbox}[quote="mathlink"]Determine all functions $f:R \to R$ with the property
$f(f(x)+y)=2x+f(f(y)-x) \forall x,y \in R$\end{tcolorbox}

Then $P(x+T,y)$ $\implies$ $f(f(x)+y)=2(x+T)+f(f(y)-x)$ 
\end{tcolorbox}
I dont understand this line ?
Thank you, pco.
\end{solution}



\begin{solution}[by \href{https://artofproblemsolving.com/community/user/29428}{pco}]
	\begin{tcolorbox}[quote="pco"][quote="mathlink"]Determine all functions $f:R \to R$ with the property
$f(f(x)+y)=2x+f(f(y)-x) \forall x,y \in R$\end{tcolorbox}

Then $P(x+T,y)$ $\implies$ $f(f(x)+y)=2(x+T)+f(f(y)-x)$ 
\end{tcolorbox}
I dont understand this line ?
Thank you, pco.\end{tcolorbox}
$P(x+T,y)$ $\implies$ $f(f(x+T)+y)=2(x+T)+f(f(y)-x-T)$

But $f(x+T)=f(x)$ and so $f(f(x+T)+y)=f(f(x)+y)$
And, same, $f(f(y)-x-T)=f(f(y)-x)$

And so $f(f(x+T)+y)=2(x+T)+f(f(y)-x-T)$ becomes $f(f(x)+y)=2(x+T)+f(f(y)-x)$
\end{solution}
*******************************************************************************
-------------------------------------------------------------------------------

\begin{problem}[Posted by \href{https://artofproblemsolving.com/community/user/68025}{Pirkuliyev Rovsen}]
	Find all increasing functions $f:\mathbb{R}\to\mathbb{R}$ such that \[f(x+f(y))=f(x+y)+1\] for all $x, y \in \mathbb R$.
	\flushright \href{https://artofproblemsolving.com/community/c6h386667}{(Link to AoPS)}
\end{problem}



\begin{solution}[by \href{https://artofproblemsolving.com/community/user/29428}{pco}]
	\begin{tcolorbox}Find all increasing function 
$f: \mathbb{R}\to\mathbb{R}$ such that $f(x+f(y))=f(x+y)+1$. \end{tcolorbox}
Let $P(x,y)$ be the assertion $f(x+f(y))=f(x+y)+1$

$P(x,0)$ $\implies$ $f(x+f(0))=f(x)+1$
$P(0,x)$ $\implies$ $f(f(x))=f(x)+1$
Subtracting these two lines, we get : $f(f(x))=f(x+f(0))$

$f(x)$ increasing implies $f(x)$ injective and so $f(x)=x+f(0)$
Plugging this in original equation, we get $f(0)=1$ and so the unique solution $\boxed{f(x)=x+1}$

Nota \end{underlined}: I understood "increasing" as usual on this forum as "strictly increasing". If this is not the case, it's easy to prove that "non decreasing" implies the same conclusion.
\end{solution}



\begin{solution}[by \href{https://artofproblemsolving.com/community/user/44083}{jgnr}]
	Suppose there exist $a\ne b$ such that $f(a)=f(b)$. Then $f(x+f(a))=f(x+f(b))$ for all $x$, so $f(x+a)=f(x+b)$. So $f$ is periodic. Since $f$ is also increasing, it must be constant, which is impossible since $f(x+f(y))-f(x+y)=1$. Therefore $f$ is injective. We have $f(x+f(y))=f(x+y)+1=f(y+f(x))$, so $x+f(y)=y+f(x)$, $f(x)-x=f(y)-y$ for all $x,y$. So $f(x)-x=c$ is constant, we get $f(x)=x+c$. Substitute back to the original equation, we get $x+y+2c=x+y+c+1$, so $c=1$, therefore $f(x)=x+1$.
\end{solution}
*******************************************************************************
-------------------------------------------------------------------------------

\begin{problem}[Posted by \href{https://artofproblemsolving.com/community/user/77076}{jaydoubleuel}]
	Find all functions $f: \mathbb R \to \mathbb R$ such that for all reals $x$ and $y$,
\[f(xy+f(x))=xf(y)+f(x).\]
	\flushright \href{https://artofproblemsolving.com/community/c6h386678}{(Link to AoPS)}
\end{problem}



\begin{solution}[by \href{https://artofproblemsolving.com/community/user/44358}{crazyfehmy}]
	Here is a solution.

Take $y=0, \: f(f(x))=f(x)+xf(0)$

If $f(0) \neq 0,$ then $f$ is one to one, so $f(f(0))=f(0)$ implies $f(0)=0$

Therefore $f(0)=0$ and $f(f(x))=f(x)$

Take $y=f(k), \: f(xf(k)+f(x))=xf(f(k))+f(x)=xf(k)+f(x)$

Take $x=f(k), \: f(yf(k)+f(k))=f(y)f(k)+f(k)$

Now take $x=y=k, \: f(xf(x)+f(x))=xf(x)+f(x)=[f(x)]^2+f(x)$

Therefore for a real number $x_0, \: f(x_0)=x_0$ or $f(x_0)=0$

We shall show that there is no any other solution different from $f(x)=0$ and $f(x)=x.$

Obviously, these two are solutions of this equation. Now suppose that $f(x_0)=0, \: f(y_0)=y_0$ where $x_0,y_0$ are non-zero real numbers. 

Take $x=y_0, \: y=x_0, \: f(x_0y_0+y_0)=y_0$

So, either $y_0=0$ or $y_0=x_0y_0+y_0$ and therefore either $x_0$ or $y_0$ must be zero which is a contradiction. 

Hence, we are done.
\end{solution}



\begin{solution}[by \href{https://artofproblemsolving.com/community/user/29428}{pco}]
	\begin{tcolorbox}find all functions $f : R \longrightarrow R$ such that
$f(xy+f(x))=xf(y)+f(x)$\end{tcolorbox}
Let $P(x,y)$ be the assertion $f(xy+f(x))=xf(y)+f(x)$

Let $x\ne 0$ : 
$P(x,1-\frac{f(x)}x)$ $\implies$ $f(1-\frac{f(x)}x)=0$
Then $P(1-\frac{f(x)}x,1)$ $\implies$ $f(1)(1-\frac{f(x)}x)=0$

If $f(1)\ne 0$, this implies $f(x)=x$ $\forall x\ne 0$ and $P(1,0)$ $\implies$ $f(0)=0$ and $f(x)=x$ $\forall x$ which indeed is a solution.

If $f(1)=0$ :
If $f(a)=f(b)$, comparaison of $P(a,b)$ and $P(b,a)$ implies $(a-b)f(a)=0$ and so either $a=b$, either $f(a)=f(b)=0$

Then $P(x,1)$ $\implies$ $f(x+f(x))=f(x)$ and so :
either $x+f(x)=x$ and so $f(x)=0$
either $f(x+f(x))=f(x)=0$

And so $f(x)=0$ $\forall x$ which indeed is a solution.

Hence the answer : two solutions :
$f(x)=0$ $\forall x$
$f(x)=x$ $\forall x$
\end{solution}



\begin{solution}[by \href{https://artofproblemsolving.com/community/user/93909}{magical}]
	\begin{tcolorbox}find all functions $f : R \longrightarrow R$ such that
$f(xy+f(x))=xf(y)+f(x)$\end{tcolorbox}

Let $x=y=0$ we have $f(f(0))=f(0) (*)$
Let $y=0$ we have $f(f(x)=f(x)+xf(0)$
If $ f(0) \neq 0$   then easy we have $f$ : injective
$(*) \Rightarrow f(0)=0$ dissatisfied!
So $f(0)=0$
$\Rightarrow f(f(x))=f(x) $
$\Rightarrow f(xy+f(x))=f(f(xy+f(x)))=f(xf(y)+f(x))$
If $f \neq const $ $\Rightarrow xy+f(x)=xf(y)+f(x) \Rightarrow f(x)=x$ $ \forall x \in R$
If $ f = const =c \Rightarrow c=0 \Rightarrow f(x)=0$ $ \forall x \in R $
[hide]I hope it is true :oops: [\/hide]
\end{solution}



\begin{solution}[by \href{https://artofproblemsolving.com/community/user/29428}{pco}]
	\begin{tcolorbox} $\Rightarrow f(xy+f(x))=f(f(xy+f(x)))=f(xf(y)+f(x))$
If $f \neq const $ $\Rightarrow xy+f(x)=xf(y)+f(x)$\end{tcolorbox}
Why ?
You should first show injectivity to get this conclusion.
\end{solution}



\begin{solution}[by \href{https://artofproblemsolving.com/community/user/328042}{ak12sr99}]
	Let $P(x,y)$ be the given statement. Then $P(x,0), P(f(x),x)$ and $P(x,f(x))$ are respectively (1) $f(f(x))=f(x)$; (2) $f(f(x)(x+1)) = f(x)(f(x)+1)$ and (3) $f((x+1)f(x)) = (x+1)f(x)$, whence $(2)$ and $(3)$ yield $f(x) = 0$ or $f(x)=x$ for all $x \in \mathbb{R}$. Now we define $K(f) := \{r \in \mathbb{R}| f(r)=0\}$ (clearly $0 \in K(f)$). 

We consider two cases:

Case $1$: $f(-1)=0$. Then $P(x,-1) \implies f(f(x)-x)=f(x)$ so if $\exists x \neq 0$ s.t. $x \not\in K(f)$ then we obtain $x=f(x)=f(f(x)-x)=f(x-x)=f(0)=0$, a contradiction.

Case $2$: $f(-1)=-1$. Then $P(f(x),y)$ and $(1)$ together yield $f(f(x)(y+1)) = f(x)((f(y)+1)$ (the general form of $(2)$ above), wherein $x=-1$ yields $f(-x-1)=-f(x)-1$. So if $\exists r \in K(f)$ such that $r \neq 0$, then $f(-r-1)=-1=f(-1)   ...(*)$, whereas since $f(x) \in \{0, x\}$ $\forall x \in \mathbb{R}$, $f$ must be one-one for those real numbers for which it is not zero so that $(*)$ is actually a contradiction, yielding $K(f)=\{ 0 \} \implies f(x)=x$ $\forall x \in \mathbb{R}$.

The above cases yield $f(x) = 0$ and $f(x)=x$ $\forall x \in \mathbb{R}$ as the only solutions.

\end{solution}
*******************************************************************************
-------------------------------------------------------------------------------

\begin{problem}[Posted by \href{https://artofproblemsolving.com/community/user/33535}{wangsacl}]
	Find all functions $f:\mathbb{R}\rightarrow\mathbb{R}$ which satisfy the equality,
\[f(x+f(y))=f(x-f(y))+4xf(y)\]
for any $x,y\in\mathbb{R}$.
	\flushright \href{https://artofproblemsolving.com/community/c6h386820}{(Link to AoPS)}
\end{problem}



\begin{solution}[by \href{https://artofproblemsolving.com/community/user/29428}{pco}]
	\begin{tcolorbox}Find all functions $f:\mathbb{R}\rightarrow\mathbb{R}$ such that for all $x,y\in\mathbb{R}$, then
\[f(x+f(y))=f(x-f(y))+4xf(y)\]
where $\mathbb{R}$ denote the set of real numbers.\end{tcolorbox}
Let $P(x,y)$ be the assertion $f(x+f(y))=f(x-f(y))+4xf(y)$

The function $f(x)=0$ $\forall x$ is a solution. Let us from now look for non all zero solutions.
Let $u$ such that $f(u)\ne 0$

$P(\frac x{8f(u)},u)$ $\implies$ $f(\frac x{8f(u)}+f(u))=f(\frac x{8f(u)}-f(u))+\frac x2$

And so $x=2f(\frac x{8f(u)}+f(u))-2f(\frac x{8f(u)}-f(u))$ and so any real $x$ may be written as $2f(y)-2f(z)$ for some $y,z$

$P(-f(z),z)$ $\implies$ $f(0)=f(-2f(z))-4f(z)^2$
$P(f(y)-2f(z),y)$ $\implies$ $f(2f(y)-2f(z))=f(-2f(z))+4f(y)^2-8f(y)f(z)$
Subtracting these two lines, we get : $f(2f(y)-2f(z))=(2f(y)-2f(z))^2+f(0)$

And so $f(x)=x^2+f(0)$ $\forall x$ which indeed is a solution.

Hence the two solutions :
$f(x)=0$ $\forall x$
$f(x)=x^2+a$ $\forall x$
\end{solution}



\begin{solution}[by \href{https://artofproblemsolving.com/community/user/64716}{mavropnevma}]
	Notice the uncanny resemblance (if not utter equivalence) with Problem 3 at the 2007 Balkan Mathematical Olympiad.
See [url]http://www.artofproblemsolving.com/Forum/viewtopic.php?p=827178&sid=0d0afe7c258254c48c003fe03fe58e44#p827178[\/url].
\end{solution}



\begin{solution}[by \href{https://artofproblemsolving.com/community/user/19461}{Utkirstudios}]
	Can anybody post other problems of first day?
Thanks!!!
\end{solution}



\begin{solution}[by \href{https://artofproblemsolving.com/community/user/30264}{lasha}]
	$f(x+f(y)) = f(x-f(y))+4xf(y) $. (1)
First note that $f(x)=0$ for any $x$ is solution to the given equation. Suppose there exist some $u$ such that $f(u)$ doesn't equal to $0$. Then, $4xf(u)$ can get equal to any real number, but by (1) it's the difference of two f-s. It means that any real can be expressed as a $f(a)-f(b)$. (2)
 Denote $f(0)=c$. Than, setting $x=f(y)$ in (1), we get: 
$f(2f(y))=(2f(y))^{2}+c $ (3).
Furthermore, setting $f(x)$ instead of $x$ in (1), $f(f(x)-f(y)=f(f(x)+f(y))-4f(x)f(y)$. So, as the right hand side is symmetric, $f(f(x)-f(y))=f(f(y)-f(x))$. Hence, $f$ is odd, because $f(x)-f(y)$ gets any real value by (2).
Denote by $A$ the set of all $a$, satisfying $f(a)=a^{2}+c$.
From (3), $2f(x)$ belong to $A$ and as $f$ is odd, also $-2f(x)$ belong to that set. 
From (1), it's clear that if $a-b=2f(x)$ for some $x,a,b$, than $f(a)-f(b)=a^{2}-b^{2}$. So, if $a$ is in $A$, than so does $a+2f(x)$. From the last fact, as $-2f(a)$ is in$A$ for any $a$, than for any $b$, $-2f(a)+2f(b)=2(f(b)-f(a))$ is in $A$. But $f(b)-f(a)$ gets any real value, as well as $2(f(b)-f(a))$. So, $A=R$. It means, from the definition of $A$, that $f(x)=x^{2}+c$ holds for any $x$. 
Finally, the only solutions are $f(x)=0$ and $f(x)=x^{2}+c$. 
P.S. The problem is not easy, but the approach is quite well known :)
\end{solution}



\begin{solution}[by \href{https://artofproblemsolving.com/community/user/74511}{MatjazL}]
	\begin{tcolorbox}
P.S. The problem is not easy, but the approach is quite well known :)\end{tcolorbox}

What about this approach is well known?

(I don't see what the main idea, that is said to be well known is?)
\end{solution}



\begin{solution}[by \href{https://artofproblemsolving.com/community/user/93837}{jjax}]
	The main idea is that the expression $f(a)-f(b)$ may take on any real value.
Such a step is certainly not easy to motivate. Indeed, it is the key step of IMO 1999 Q6 (so it's quite well known). However, after seeing a solution using this method, the method is easy to mimic for similar questions.
\end{solution}



\begin{solution}[by \href{https://artofproblemsolving.com/community/user/111816}{waver123}]
	If we put $f(x)=x^2 + g(x)$, then we get LHS=RHS for all $x,y$. 

Does that mean anything?
\end{solution}



\begin{solution}[by \href{https://artofproblemsolving.com/community/user/64716}{mavropnevma}]
	If you do that, you get $g(x+y^2 + g(y)) = g(x-y^2 - g(y))$. It remains for you to prove that this functional equation has as only solutions $g(x)=-x^2$ (leading to $f \equiv 0$), and $g(x) = c$ (constant, leading to $f(x) = x^2+c$).
\end{solution}



\begin{solution}[by \href{https://artofproblemsolving.com/community/user/30264}{lasha}]
	\begin{tcolorbox}[quote="lasha"]
P.S. The problem is not easy, but the approach is quite well known :)\end{tcolorbox}

What about this approach is well known?

(I don't see what the main idea, that is said to be well known is?)\end{tcolorbox}



Sorry for a late response, I've just came across it :) Anyway, jjax anwered to you correctly. That approach has been used in IMO 1999\/6 (But not only, I've solved several problems with that, just don't remember the sources).
\end{solution}



\begin{solution}[by \href{https://artofproblemsolving.com/community/user/149800}{liimr}]
	\begin{tcolorbox}If we put $f(x)=x^2 + g(x)$, then we get LHS=RHS for all $x,y$. 

Does that mean anything?\end{tcolorbox}

solution with $g(x)$ is pretty good, you may see in
http://www.artofproblemsolving.com/Forum/viewtopic.php?p=827178&sid=f14af484eddd9e2dc421aec7daa5ed21#p827178

for solution by \begin{bolded}Anto\end{bolded}
\end{solution}



\begin{solution}[by \href{https://artofproblemsolving.com/community/user/214306}{DanDumitrescu}]
	I don't know if it's good but i think to create the function $g(x)=f(x)-x^2$ we can write $4xf(y)=(x+f(y))^2-(x-f(y))^2$ and we have that $g(x+f(y))=g(x-f(y)) $ and we can put y=0 and note that f(0)=a and we have that $f(k)=k^2+g(f(0))=k^2+f(a)-a^2$
\end{solution}
*******************************************************************************
-------------------------------------------------------------------------------

\begin{problem}[Posted by \href{https://artofproblemsolving.com/community/user/96252}{laurentblance}]
	Find all maps $ f : \mathbb{R} \to \mathbb{R}$ such that $f(x^2)=f(x)^2$ and $f(x+1)=f(x)+1$ for all $x \in \mathbb{R}$.
	\flushright \href{https://artofproblemsolving.com/community/c6h386824}{(Link to AoPS)}
\end{problem}



\begin{solution}[by \href{https://artofproblemsolving.com/community/user/29428}{pco}]
	\begin{tcolorbox}Find all map$ f : \mathbb{R} \to \mathbb{R}$ such that :

$\forall x \in \mathbb{R}, f(x^2)=f(x)^2$
$\forall x \in \mathbb{R}, f(x+1)=f(x)+1 $\end{tcolorbox}
1) $f(n)=n$ $\forall n\in\mathbb Z$
==============
$f(0^2)=f(0)^2$ and so $f(0)=0$ or $f(0)=1$
If $f(0)=1$, we get $f(1)=2$ and then $f(1^2)\ne f(1)^2$
So $f(0)=0$ and $f(x+1)=f(x)+1$ implies $f(x+n)=f(x)+n$ and so $f(n)=n$
Q.E.D.

2) $f(-x)=-f(x)$ $\forall x$
================
From the first equation, we get that $f(-x)^2=f(x)^2$ and so either $f(-x)=f(x)$, either $f(-x)=-f(x)$
Suppose then that $f(-x)=f(x)$ for some $x$.
Then $f(-x-n)=f(-x)-n=f(x)-n=f(x+n)-2n$ $\forall n\in\mathbb Z$

But $f(-x-n)$ is either $f(x+n)$, either $-f(x+n)$ and so $f(x+n)-2n=-f(x+n)$ $\forall n\ne 0$ and $f(x+n)=n$ and so $f(x)=0$
But then $f(-x)=f(x)=0$ implies $f(-x)=-f(x)$
Q.E.D.

3) $[x]\le f(x)\le [x]+1$
==================
$f(x^2)=f(x)^2$ implies $f(x)\ge 0$ $\forall x\ge 0$
$f(-x^2)=-f(x^2)=-f(x)^2$ implies $f(x)\le 0$ $\forall x\le 0$
$x\ge [x]$ $\implies$ $x-[x]\ge 0$ and so $f(x-[x])\ge 0$ and so $f(x)-[x]\ge 0$ and so $f(x)\ge [x]$
So $f(-x)\ge[-x]$ and so $-f(x)\ge[-x]$ and $f(x)\le -[-x]\le[x]+1$
Q.E.D.

4) $f(x)=x$ $\forall x$
===========
If $f(x)=x+a$ with $a>0$, then $f(x+n)=x+n+a$
Choose then $n$ such that $x+n>\frac 1{2a}$

We get $f((x+n)^2)=f(x+n)^2=(x+n)^2+2a(x+n)+a^2>(x+n)^2+1$ in contradiction with 3) above.

If $f(x)=x+a$ with $a<0$, then $f(-x)=f(-x+(-a))$ and the three lines above show again the contradiction.
Q.E.D

And since $\boxed{f(x)=x}$ indeed is a solution, this is the unique one.
\end{solution}



\begin{solution}[by \href{https://artofproblemsolving.com/community/user/96252}{laurentblance}]
	Great ! 

But  where do the idea come from for 3 and 4 ?
\end{solution}
*******************************************************************************
-------------------------------------------------------------------------------

\begin{problem}[Posted by \href{https://artofproblemsolving.com/community/user/92753}{WakeUp}]
	Prove that there is no function $f:(0,\infty )\rightarrow (0,\infty)$ such that
\[f(x+y)\ge f(x)+yf(f(x)) \]
for every $x,y\in (0,\infty )$.
	\flushright \href{https://artofproblemsolving.com/community/c6h386866}{(Link to AoPS)}
\end{problem}



\begin{solution}[by \href{https://artofproblemsolving.com/community/user/64716}{mavropnevma}]
	Posted before (in several versions) - do a search.
\end{solution}



\begin{solution}[by \href{https://artofproblemsolving.com/community/user/29428}{pco}]
	\begin{tcolorbox}Prove that there is no function $f:(0,\infty )\rightarrow (0,\infty)$ such that
\[f(x+y)\ge f(x)+yf(f(x)) \]
for every $x,y\in (0,\infty )$.\end{tcolorbox}
Adaptation of one of the multiple previously posted solutions :

Let $P(x,y)$ be the assertion $f(x+y)\ge f(x)+yf(f(x))$

1) $f(x)< x+1$ $\forall x>0$
===================
If $f(x)\le x$, we get obviously $f(x)< x+1$
If $f(x)>x$ for some $x$, then $P(x,f(x)-x)$ $\implies$ $f(f(x))\ge f(x)+(f(x)-x)f(f(x))>(f(x)-x)f(f(x))$ $\implies$ $f(x)< x+1$
Q.E.D.

2) $f(f(x))\le 1$ $\forall x>0$
====================
If $f(f(x))>1$ for some $x$, then Let $y>\frac{x+1}{f(f(x))-1}$ so that $yf(f(x))>x+y+1$ : 
$P(x,y)$ $\implies$ $f(x+y)\ge f(x)+yf(f(x))>x+y+1$ and so contradiction with 1) above.
Q.E.D.

3) $\exists M>0$ such that $f(x)<M$ $\forall x>0$
==================================
Suppose that $\forall a>0,\exists u_a>0$ such that $f(u_a)>a$
Let $x>0$ and $a>x$ and $u_a>0$ such that $f(u_a)>a>x$
$P(x,f(u_a)-x)$ $\implies$ $f(f(u_a))\ge f(x)+(f(u_a)-x)f(f(x))>(a-x)f(f(x))$ $\implies$ (using 2) above) $1>(a-x)f(f(x))$

So $0<f(f(x))<\frac 1{a-x}$ $\forall a>x$ which is impossible
Q.E.D.

4) No such function exists
==================
$P(x,y)$ $\implies$ $M>yf(f(x))$ and so $0<f(f(x))<\frac My$ $\forall y$ which is impossible
Q.E.D.
\end{solution}



\begin{solution}[by \href{https://artofproblemsolving.com/community/user/92753}{WakeUp}]
	\begin{tcolorbox}Posted before (in several versions) - do a search.\end{tcolorbox}

Apologies for this, I searched but out of the matches that came up, they were either unsolved or slightly different (such as http://www.artofproblemsolving.com/Forum/viewtopic.php?f=36&t=307863).
\end{solution}



\begin{solution}[by \href{https://artofproblemsolving.com/community/user/64716}{mavropnevma}]
	\begin{tcolorbox}Prove that there is no function $f:(0,\infty )\rightarrow (0,\infty)$ such that
\[f(x+y)\ge f(x)+yf(f(x)) \]
for every $x,y\in (0,\infty )$.\end{tcolorbox}
The problem, as such, was used in some Italian math competition. A strengthened version, for the relation $f(x+y)\ge yf(f(x))$, was asked at the 2009 IMAR (Romanian Institute for Mathematics) competition. Detailed proof(s), generalization, and discussions to be found in the \begin{bolded}RMC 2009 \end{bolded}brochure.
\end{solution}
*******************************************************************************
-------------------------------------------------------------------------------

\begin{problem}[Posted by \href{https://artofproblemsolving.com/community/user/67223}{Amir Hossein}]
	Find the least non-negative integer $n$ for which there exists a function $f : \mathbb Z \to [0, +\infty)$ which satisfies the following conditions:

(i) For all integers $x$ and $y$, $f(xy)=f(x)f(y)$, and

(ii) $2 f(x^2+y^2)-f(x)-f(y) \in [0, 1, \ldots, n]$ for all $x,y \in \mathbb Z$.

Also, find all such functions for $n$.
	\flushright \href{https://artofproblemsolving.com/community/c6h386878}{(Link to AoPS)}
\end{problem}



\begin{solution}[by \href{https://artofproblemsolving.com/community/user/64868}{mahanmath}]
	\begin{tcolorbox}Find the least non-negative integer such that there exists a function $f : \mathbb Z \to [0, +\infty)$ which satisfies the conditions

\begin{bolded}(i)\end{bolded} $f(xy)=f(x)f(y),$ and

\begin{bolded}(ii)\end{bolded} $2 f(x^2+y^2)-f(x)-f(y) \in [0, 1, \ldots, n].$ For all integers $x,y.$

And then, find all functions for $n.$\end{tcolorbox}

[hide="Hint"]
$n=1$
Look at the primes $p$ which $4 \mid p+1$ [\/hide]
\end{solution}



\begin{solution}[by \href{https://artofproblemsolving.com/community/user/29428}{pco}]
	\begin{tcolorbox}[quote="amparvardi"]Find the least non-negative integer such that there exists a function $f : \mathbb Z \to [0, +\infty)$ which satisfies the conditions

\begin{bolded}(i)\end{bolded} $f(xy)=f(x)f(y),$ and

\begin{bolded}(ii)\end{bolded} $2 f(x^2+y^2)-f(x)-f(y) \in [0, 1, \ldots, n].$ For all integers $x,y.$

And then, find all functions for $n.$\end{tcolorbox}

$n=1$
Look at the primes $p$ which $4 \mid p+1$ \end{tcolorbox}
Wrong.
Least $n$ is $\boxed{n=0}$ : choose $f(x)=0$

And finding all $f$ for $n=0$ is easy :

$2f(z^2x^2+z^2y^2)=f(zx)+f(zy)$ and so $2f(x^2+y^2)f(z)^2=(f(x)+f(y))f(z)$

And since $2f(x^2+y^2)f(z)^2=(f(x)+f(y))f(z)^2$, we get $(f(x)+f(y))f(z)(f(z)-1)=0$ $\forall x,y,z$

And so : $\forall x$ : either $f(x)=0$, either $f(x)=1$

Suppose now that $f(x)=0$ for some $x$ and $f(y)=1$ for some $y$ and we get $f(x^2+y^2)=\frac 12$, impossible.

And so : either $\boxed{f(x)=0}$ $\forall x$, either $\boxed{f(x)=1}$ $\forall x$ which indeed are solutions.
\end{solution}
*******************************************************************************
-------------------------------------------------------------------------------

\begin{problem}[Posted by \href{https://artofproblemsolving.com/community/user/92753}{WakeUp}]
	Define a function $f:\mathbb{N}\rightarrow\mathbb{N}_0$ by $f(1)=0$ and
\[f(n)=\max_j\{ f(j)+f(n-j)+j\}\quad\forall\, n\ge 2 \]
Determine $f(2000)$.
	\flushright \href{https://artofproblemsolving.com/community/c6h387704}{(Link to AoPS)}
\end{problem}



\begin{solution}[by \href{https://artofproblemsolving.com/community/user/29428}{pco}]
	\begin{tcolorbox}Define a function $f:\mathbb{N}\rightarrow\mathbb{N}_0$ by $f(1)=0$ and
\[f(n)=\max_j\{ f(f)+f(n-j)+j\}\quad\forall\, n\ge 2 \]
Determine $f(2000)$.\end{tcolorbox}
In the max expression :
1) what is the range of $j$ ?
2) what is the meaning of $f(f)$ ?
\end{solution}



\begin{solution}[by \href{https://artofproblemsolving.com/community/user/92753}{WakeUp}]
	\begin{tcolorbox}[quote="WakeUp"]Define a function $f:\mathbb{N}\rightarrow\mathbb{N}_0$ by $f(1)=0$ and
\[f(n)=\max_j\{ f(f)+f(n-j)+j\}\quad\forall\, n\ge 2 \]
Determine $f(2000)$.\end{tcolorbox}
In the max expression :
1) what is the range of $j$ ?
2) what is the meaning of $f(f)$ ?\end{tcolorbox}
Woops! Edited.

But I don't know what the range of $j$ is. This is the exact wording of the problem and it had me confused also; but I'm guessing $j\in\mathbb{N}$ and $j<n$. But this is only a guess!
\end{solution}



\begin{solution}[by \href{https://artofproblemsolving.com/community/user/29428}{pco}]
	Ok, then. Let us consider $f(n)=\max_{k\in[1,n-1]}(f(k)+f(n-k)+k)$

Choosing $k=n-1$, we get $f(n)\ge f(n-1)+n-1$ and so $f(n)\ge (n-1)+(n-2)+...+1+f(1)=\frac{n(n-1)}2$

And it is easy to see that $f(n)=\frac{n(n-1)}2$ indeed fits :

$f(k)+f(n-k)+k=\frac{k(k-1)}2+\frac{(n-k)(n-k-1)}2+k$ $=\frac{n(n-1)}2-k((n-1)-k)\le f(n)$ with equality when $k=n-1$

So $\boxed{f(2000)=1999000}$
\end{solution}



\begin{solution}[by \href{https://artofproblemsolving.com/community/user/51452}{efoski1687}]
	Dear mathlinkers,

    You can find the problems from Taiwan Olympiad 2000 at the book of Titu: Mathematical Olympiads Contests Around the World 2000-2001 . I have looked for the statement of j and the true range of j is:
 
      0 < j < (n+1)\/2 . So the solution of pco is false with this statement and for solution the author used a very complicated method. The answer is
 f(2000)=10864. 
 
        A hint for the solution from the book:

      For each positive integer n, we consider the binary
representation of n. Consider the substrings of the representation
formed by removing at least one digit from the left side of the
representation, such that the substring so formed begins with a 1.
We call the decimal values of these substrings the tail-values of n.
Also, for each 1 that appears in the binary representation of n, if it
represents the number 2^k, let (2^k).(k\/2) be a place-value of n. Let g(n) be the sum of the tail- and place-values of n. We prove by induction on n that f(n) = g(n). For convenience, let g(0) = 0.
It is clear that g(1) = 0....
\end{solution}



\begin{solution}[by \href{https://artofproblemsolving.com/community/user/197352}{InfiniteHorizon}]
	One of the best problems I encountered in my Math Olympiad career (which was ~8 years ago).

Claim: $f(2n)=2f(n)+n$ and $f(2n+1)=f(n)+f(n+1)+n$.

Proof: Go by induction; suppose this is true for smaller $n$. We know $f(2n)$ is the maximum of
\[f(2k)+f(2n-2k)+2k,\,\,\,1\leq k\leq n\/2,\] and
\[f(2k-1)+f(2n-2k+1)+2k-1,\,\,\,1\leq k\leq (n+1)\/2.\]
The first expression equals\[2(f(k)+f(n-k)+k)+n\leq 2f(n)+n\] by induction hypothesis and the definition of $f$.
As for the second, if $k=1$ we get \[f(2n-1)+1=f(n)+f(n-1)+n< 2f(n)+n\] since by definition we have $f(n)>f(n-1)$; if $k=(n+1)\/2$ we get exactly $2f(n)+n$ again, and when $2\leq k\leq n\/2$ we get\[f(k)+f(k-1)+f(n-k)+f(n-k+1)+(2k+n-2),\] which equals\[(f(k)+f(n-k)+k)+(f(k-1)+f(n-k+1)+(k-1))+(n-1)\leq 2f(n)+n-1.\]

Similarly, $f(2n+1)$ is the maximum of
\[f(2k)+f(2n-2k+1)+2k,\,\,\,1\leq k\leq n\/2,\]and
\[f(2k+1)+f(2n-2k)+(2k+1),\,\,\,0\leq k\leq (n-1)\/2.\] The first expression equals
\[2f(k)+f(n-k)+f(n-k+1)+n+2k,\] which then equals
\[(f(k)+f(n-k)+k)+(f(k)+f(n+1-k)+k)+n\leq f(n)+f(n+1)+n.\] For the second, if $k=0$ we get\[f(2n)+1=2f(n)+n+1\leq f(n)+f(n+1)+n\] due to what we just proved;  if $2\leq k\leq (n-1)\/2$ we get
\[f(k)+f(k+1)+2f(n-k)+(n+2k+1),\] which is just\[(f(k)+f(n-k)+k)+(f(k+1)+f(n-k)+(k+1))+n\leq f(n)+f(n+1)+n.\]

Finally:
\[f(1)=0,f(2)=1,f(3)=2,f(4)=4,f(7)=9,f(8)=12,f(15)=28,f(16)=32,f(31)=75,f(32)=80,f(62)=181,f(63)=186,f(125)=429,f(250)=983,f(500)=2216,f(1000)=4932,f(2000)=10864.\]
\end{solution}
*******************************************************************************
-------------------------------------------------------------------------------

\begin{problem}[Posted by \href{https://artofproblemsolving.com/community/user/67223}{Amir Hossein}]
	Find all functions $f : \mathbb R \to \mathbb R$ such that 
\[f\left(x+xy+f(y)\right)= \left( f(x)+\frac 12 \right) \left( f(y)+\frac 12 \right) \qquad \forall x,y \in \mathbb R.\]
	\flushright \href{https://artofproblemsolving.com/community/c6h387796}{(Link to AoPS)}
\end{problem}



\begin{solution}[by \href{https://artofproblemsolving.com/community/user/28136}{Rofler}]
	$y=-1$ gives $f(f(-1))=(f(x)+\frac{1}{2})(f(-1)+\frac{1}{2})$.
If $f(-1)$ is not equal to $-\frac{1}{2}$, then $f(x)$ must be constant for all $x$. If $f$ is a constant $k$, then the functional equation is equivalent to
$k=(k+\frac{1}{2})(k+\frac{1}{2})$
$k=k^2+k+\frac{1}{4}$
$0=k^2+\frac{1}{4}$, which is impossible, contradiction! Therefore, we must have $f(-1)=-\frac{1}{2}$.

$y=-1$ gives then $f(f(-1))=0$, so $f(-\frac{1}{2})=0$.
$(x,y)=(0,-\frac{1}{2})$ gives
$f(0)=(f(0)+\frac{1}{2})(f(-\frac{1}{2})+\frac{1}{2})=\frac{1}{2}(f(0)+\frac{1}{2})$
$2f(0)=f(0)+\frac{1}{2}$
$f(0)=\frac{1}{2}$

If there is a $k \ne -1$ such that $f(k)=-\frac{1}{2}$, then $y=k$ gives $f((1+k)x-\frac{1}{2})=0$, which by varying $x$ causes $f(x)=0$ for all $x$, contradicting there being no non-constant solution.

Therefore, $-1$ is the unique $x$ s.t. $f(x)=-\frac{1}{2}$.

$y=0$ gives $f(x+\frac{1}{2})=f(x)+\frac{1}{2}$, which implies $f(x)-\frac{1}{2}=f(x-\frac{1}{2})$.
$x=-1$ gives $f(f(y)-y-1)=0$, so $-\frac{1}{2}=f(f(y)-y-1)-\frac{1}{2}=f(f(y)-y-\frac{3}{2})$.
This implies that $f(y)-y-\frac{3}{2}=-1$, since $-1$ is the unique $x$ s.t. $f(x)=-\frac{1}{2}$.
Therefore, $f(y)=y+\frac{3}{2}-1=y+\frac{1}{2}$.

Plugging this solution back in, $x+y+xy+1=(x+1)(y+1)$ which is true.

Therefore, $f(x)=x+\frac{1}{2}$ is the only solution.

Cheers,

Rofler
\end{solution}



\begin{solution}[by \href{https://artofproblemsolving.com/community/user/29428}{pco}]
	\begin{tcolorbox}Find all functions $f : \mathbb R \to \mathbb R$ such that 
\[f\left(x+xy+f(y)\right)= \left( f(x)+\frac 12 \right) \left( f(y)+\frac 12 \right) \qquad \forall x,y \in \mathbb R.\]\end{tcolorbox}
Let $P(x,y)$ be the assertion $f(x+xy+f(y))=\left(f(x)+\frac 12\right)\left(f(y)+\frac 12\right)$

1) $f(x+y)=f(x)+f(y)-\frac 12$ $\forall x,y$ and $f(x)=x+\frac 12$ $\forall x\in\mathbb Q$
=============================================================
$P(x,-1)$ $\implies$ $f(f(-1))=\left(f(x)+\frac 12\right)\left(f(-1)+\frac 12\right)$
If $f(-1)\ne -\frac 12$, this implies $f(x)=c$ constant but, plugging this in original equation, we get $c=(c+\frac 12)^2$ and so no solution.

So $f(-1)=-\frac 12$ and $f(f(-1))=0$ and so $f(-\frac 12)=0$

$P(0,-\frac 12)$ $\implies$ $f(0)=\frac 12\left(f(0)+\frac 12\right)$ and so $f(0)=\frac 12$

$P(x,0)$ $\implies$ $f(x+\frac 12)=f(x)+\frac 12$

$P(x+\frac 12,y)$ $\implies$ $f(x+xy+f(y)+\frac y2)+\frac 12=\left(f(x)+1\right)\left(f(y)+\frac 12\right)$
Subtracting $P(x,y)$ from this equation, we get :

$f(x+xy+f(y)+\frac y2)=f(x+xy+f(y))+\frac 12f(y)-\frac 14$

If $y\ne -1$, and setting then in this equation $x=\frac{u-f(y)}{y+1}$, this becomes $f(u+\frac y2)=f(u)+\frac 12f(y)-\frac 14$, still true for $y=-1$

Setting $u=0$ in this equality, we get $f(\frac y2)=\frac 12f(y)+\frac 14$ and so the equality becomes :

$f(u+\frac y2)=f(u)+f(\frac y2y)-\frac 12$ and so $f(x+y)=f(x)+f(y)-\frac 12$
So $g(x)=f(x)-\frac 12$ is such that $g(x+y)=g(x)+g(y)$ and so $g(x)=xg(1)$ $\forall x\in\mathbb Q$
So $f(x)=ax+\frac 12$ $\forall x\in\mathbb Q$ and, setting $x=-1$ in this equality, we get $a=1$
Q.E.D

2) $f(x)$ is injective
============
If $f(y_1)=f(y_2)$ with $T=y_1-y_2\ne 0$ :
$f(x+y_1)=f(x)+f(y_1)-\frac 1=f(x)+f(y_2)-\frac 12=f(x+y_2)$ and so $f(x+T)=f(x)$

$P(x+T,y)$ $\implies$ $f(x+xy+f(y)+Ty)=f(x+xy+f(y))$
If $y\ne -1$, and setting then in this equation $x=\frac{u-f(y)}{y+1}$, this becomes $f(u+Ty)=f(u)$, still true for $y=-1$
And this implies that $f(x)=c$ is constant, which is not a solution.
So such $y_1,y_2$ dont exist and $f(x)$ is injective.
Q.E.D

3) $f(x)=x+\frac 12$ $\forall x$
====================
Comparing $P(x,0)$ and $P(0,x)$, we get $f(x+\frac 12)=f(f(x))$ and injectivity implies then $f(x)=x+\frac 12$
Q.E.D

And since this mandatory value indeed is a solution, we get the result : $\boxed{f(x)=x+\frac 12}$ $\forall x$

*\begin{bolded}edited\end{bolded}\end{underlined}* : too late :)
\end{solution}



\begin{solution}[by \href{https://artofproblemsolving.com/community/user/177508}{mathuz}]
	I have a solution!
$P(x,-1)$ $\rightarrow $ $f(f(-1))=(f(x)+\frac{1}{2})(f(-1)+\frac{1}{2}). $  If $f(-1)\not= -\frac{1}{2} $ then  $f(x)=const $.  Let  $f(x)=c$, so  $c=(c+0,5)^2$  contradiction.
Hence, $f(-1)=-\frac{1}{2},$ $f(-\frac{1}{2})=0$ and $f(0)=\frac{1}{2}.$
Put on original equation of $P(x,0)$, we have  $f(x+\frac{1}{2})=f(x)+\frac{1}{2}.$
If $f(a)=-\frac{1}{2}$ then $a=-1.$  Let $y\not=-1$, then 
$P(\frac{-0,5-f(y)}{1+y}, y)$ $ \rightarrow $  $f(y)=y+\frac{1}{2}.$
it's true at $y=-1.$ 
\end{solution}
*******************************************************************************
-------------------------------------------------------------------------------

\begin{problem}[Posted by \href{https://artofproblemsolving.com/community/user/78444}{Babai}]
	1. Find all functions $f: \mathbb R \to \mathbb R$ such that for all reals $x$ and $y$,
\[f(x-y+f(y))=f(x)+f(y)\]

2. Find all functions $f: \mathbb R \to \mathbb R$ such that for all reals $x$ and $y$,
\[f(f(x) +y)=f(x^2 -y) + 4f(x)y.\]

3. Find all functions $f: \mathbb R \to \mathbb R$ such that for all reals $x$ and $y$,
\[f(xf(y)) +f(f(x)+f(y))=yf(x) +f(x+f(y)).\]
	\flushright \href{https://artofproblemsolving.com/community/c6h388552}{(Link to AoPS)}
\end{problem}



\begin{solution}[by \href{https://artofproblemsolving.com/community/user/41723}{Wickedestjr}]
	[hide="Solution to #1"]Letting $x = y$, we get:

$f(x - x + f(x)) = f(x) + f(x)$

$f(f(x)) = 2f(x)$

Thus, in general, $f(x) = 2x$. Trying this out, we get:

$f(x - y + f(y)) = f(x) + f(y)$

$f(x - y + 2y) = 2x + 2y$

$2x + 2y = 2x + 2y$

Which is obviously true.[\/hide]
\end{solution}



\begin{solution}[by \href{https://artofproblemsolving.com/community/user/29428}{pco}]
	\begin{tcolorbox} Letting $x = y$, we get:

$f(x - x + f(x)) = f(x) + f(x)$

$f(f(x)) = 2f(x)$

Thus, in general, $f(x) = 2x$. \end{tcolorbox}

You cant take this conclusion.
The only thing you can conclude is that $f(x)=x$ $\forall x\in f(\mathbb R)$ and if $f(\mathbb R)\ne\mathbb R$, you got nothing.

Infinitely many non surjective functions different from $2x$ match the equation $f(f(x))=2f(x)$
\end{solution}



\begin{solution}[by \href{https://artofproblemsolving.com/community/user/49798}{lightest}]
	\begin{tcolorbox}Find all functions $ f:R\rightarrow R $ such that:

$1.f(x-y+f(y))=f(x)+f(y) $

    for real $x,y$\end{tcolorbox}

[hide="Partial Solution"]

From (1), letting  $x=y$ we get $f(f(y))=2f(y)$ (4)

In (1), changing $y$ to $f(y)$ and using (2) we get $f(x+f(y))=f(x)+2f(y)$   (5)

In (1), changing $x$ to $x+y$ we get $f(x+f(y))=f(x+y)+f(y)$    (6)

Comparing (5) and (6) we get the Cauchy from $f(x+y)=f(x)+f(y)$   (7)

(7) and (4) imply (1): $f(x-y+f(y))=f(x)-f(y)+f(f(y))=f(x)+f(y)$
[\/hide]
\end{solution}



\begin{solution}[by \href{https://artofproblemsolving.com/community/user/29126}{MellowMelon}]
	I believe noncontinuous solutions to #1 exist. Working from the previous post, use AC to take a basis of $\mathbb{R}$ over $\mathbb{Q}$ and send every element of this basis to some rational number under $f$. Extend to all reals with the additive condition. Also set it up so that $f(x) = 2x$ for all rational $x$ (easy if a rational number is part of the aforementioned basis). I believe such a function has range equal to $\mathbb{Q}$, and so $f(f(x)) = 2f(x)$.
\end{solution}



\begin{solution}[by \href{https://artofproblemsolving.com/community/user/49798}{lightest}]
	\begin{tcolorbox}Find all functions $ f:R\rightarrow R $ such that:

$2.f(f(x) +y)=f(x^2 -y) + 4f(x)y$   ...... (2)

    for real $x,y$\end{tcolorbox}

[hide="Solution to 2"]

From (2), letting $y=-f(x)$ we get $f(0)=f(x^2+f(x))-4f(x)^2$  ...... (2.1)

From (2), letting $y=x^2$ we get $f(x^2+f(x))=f(0)+4f(x)x^2$   ...... (2.2)

Comparing (2.1) and (2.2), we have $f(x)x^2 = f(x)^2$  ...... (2.3)

(2.3) implies $f(0)=0$ ...... (2.4), and also if $f(x) \neq 0$ then $f(x)=x^2$   ...... (2.5)

One more useful property is $f(x)=f(-x)$ ...... (2.6), which follows simply from letting $x=0$ in (2) and using (2.4). 

$f(x) \equiv 0$ is a trivial solution of (2), so we assume that there exists a nonzero $f(c)=c^2 >0$. From (2.6), we can simply assume that $c>0$.

Now if there exists $z$ such that $f(z)=0$, from (2) we have $f(f(z)+c)=f(z^2-c)+4f(z)c$, so $f(z^2-c)=c^2>0$, therefore $(z^2-c)^2 = c^2$, which implies whether $z^2=0$ or $z^2=-2c$, but the latter contradicts with our assumption that $c>0$. Therefore only $z=0$ satisfies $f(z)=0$.

Conclusively, all the possible solutions of (2) are $f(x) \equiv 0$ or $f(x) \equiv x^2$.


[\/hide]
\end{solution}



\begin{solution}[by \href{https://artofproblemsolving.com/community/user/49798}{lightest}]
	\begin{tcolorbox}I believe noncontinuous solutions to #1 exist. Working from the previous post, use AC to take a basis of $\mathbb{R}$ over $\mathbb{Q}$ and send every element of this basis to some rational number under $f$. Extend to all reals with the additive condition. Also set it up so that $f(x) = 2x$ for all rational $x$ (easy if a rational number is part of the aforementioned basis). I believe such a function has range equal to $\mathbb{Q}$, and so $f(f(x)) = 2f(x)$.\end{tcolorbox}

I believe so. By the way, I have long been doubting that the existence of a discontinuous solution of Cauchy's functional equation does not need any choice-axiom.
\end{solution}



\begin{solution}[by \href{https://artofproblemsolving.com/community/user/29126}{MellowMelon}]
	[hide="Solution to 3"]Suppose that $f(0) = 1$. Put in $y = 0$ to get $f(f(x)+1) = f(x+1) - f(x)$ (1). In the original equation, put in $x = 0$ to get $1 + f(f(y)+1) = y+f(f(y))$. Combining this with (1), we get $f(x+1) + 1 = x + f(x) + f(f(x))$ (2). Replace $x$ with $f(x)$ in (2) and apply (1) to get $f(x+1) + 1 = 2f(x) + f(f(x)) + f(f(f(x)))$. Substitute (2) in to get $x = f(x) + f(f(f(x)))$ (3). Putting $x = 0$ into (3) gives $f(f(1)) = -1$. Putting $x = 1$ into (3) gives $1 = f(1) + f(-1)$. But putting $x = 0, y = f(1)$ into the original equation gives $2 = f(1) + f(-1)$. 1 does not equal 2. Contradiction.

So we must have $f(0) \neq 1$. This means that we can find an $a$ such that $af(0) = a+f(0)$. Put in $x = a, y = 0$ to get $f(f(a)+f(0)) = 0$. Let $c = f(a) + f(0)$. Put in $x = y = c$ to get $f(0) = 0$. Put $y = 0$ into the original equation to get $f(f(x)) = f(x)$ (4). Now replace $x$ with $f(x)$ in the original equation; using (4), we eventually reach $f(f(x)f(y)) = yf(x)$. But now replacing $y$ with $f(y)$ and using (4) gives $f(f(x)f(y)) = f(y)f(x)$. So by substitution $yf(x) = f(y)f(x)$ or $f(x)(y-f(y)) = 0$. From here it is clear that either $f(x) = 0$ or $f(x) = x$ for all $x$, the only two solutions.[\/hide]
\end{solution}



\begin{solution}[by \href{https://artofproblemsolving.com/community/user/93837}{jjax}]
	A simpler solution to q3:

let P(x,y) denote the proposition that
$f(xf(y))+f(f(x)+f(y))=yf(x)+f(x+f(y))$.

Clearly the zero function is a solution. Let us consider functions that are not everywhere zero.

There exists k so that f(k) is nonzero. Suppose$ f(a)=f(b).$
P(k,a) and P(k,b) together show that $af(k)=bf(k)$. Thus a=b and the function is injective.

$P(0,1): f(0)+f(f(0)+f(1))=f(0)+f(f(1))$
Injectivity gives $f(0)=0.$
$P(x,0): f(f(x))=f(x)$.
Injectivity gives $f(x)=x$ for all x.

Thus the two solutions are $f(x)=0, f(x)=x$.
\end{solution}
*******************************************************************************
-------------------------------------------------------------------------------

\begin{problem}[Posted by \href{https://artofproblemsolving.com/community/user/89681}{mitkkkk}]
	Find all functions $f: \mathbb R \to \mathbb R$ which satisfy for all reals $x$ and $y$ the following equation:
\[f(f(x-y)) =  f(x) - f(y) + f(x) \cdot f(y) -xy \]
	\flushright \href{https://artofproblemsolving.com/community/c6h388724}{(Link to AoPS)}
\end{problem}



\begin{solution}[by \href{https://artofproblemsolving.com/community/user/64716}{mavropnevma}]
	For $x=y$ one gets $f(f(0)) = f(x)^2 - x^2$, so $f(x)^2 = x^2 + f(f(0))$. Then $c = f(0)^2 = f(f(0))$. 
It easily follows that this value $c$ is either $0$ or $2$.

Now try in turn the cases $c = 0$ and $c =2$, and define $A = \{x \in \mathbb{R} \mid f(x) = \sqrt{x^2 + c}\}$ and $B = \{x \in \mathbb{R} \mid f(x) = -\sqrt{x^2 + c}\}$. The challenge is to see if any other solution than $c=0$, $f(x) = x$ (i.e. $A = \{x \in \mathbb{R} \mid x\geq 0 \}$, $B = \{x \in \mathbb{R} \mid x\leq 0 \}$) may exist.
\end{solution}



\begin{solution}[by \href{https://artofproblemsolving.com/community/user/89681}{mitkkkk}]
	sorry, i cant understand your post
\end{solution}



\begin{solution}[by \href{https://artofproblemsolving.com/community/user/29428}{pco}]
	\begin{tcolorbox}my friend gave me a problem and up to now, i still dont know its solution 
hope you will help me on this
\begin{bolded}find all functions f: R -> R satisfy
f(f(x-y)) =  f(x) - f(y) + f(x).f(y) -xy 
for real x,y \end{bolded}\end{tcolorbox}
Let $P(x,y)$ be the assertion $f(f(x-y))=f(x)-f(y)+f(x)f(y)-xy$

$P(0,0)$ $\implies$ $f(f(0))=f(0)^2$
$P(x,x)$ $\implies$ $f(x)^2=x^2+f(f(0))$ and so $f(x)^2=x^2+f(0)^2$

$P(x,0)$ $\implies$ $f(f(x))=f(x)(f(0)+1)-f(0)$
squaring, we get $f(x)^2+f(0)^2=(f(x)(f(0)+1)-f(0))^2$ and so $f(x)f(0)(f(x)(f(0)+2)-2(f(0)+1))=0$
If $f(x)\ne 0$ $\forall x$ then $f(x)f(0)\ne 0$ and so $f(x)(f(0)+2)=2(f(0)+1)$ $\forall x$ and so $f(x)=c$ which is not a solution
So $\exists u$ such that $f(u)=0$ and then $f(u)^2=u^2+f(0)^2$ implies $u=0$ and $f(0)=0$

$P(x,0)$ $\implies$ $f(f(x))=f(x)$
$P(1,1)$ $\implies$ $f(1)^2=1$
If $f(1)=1$ , then $P(1,1-x)$ $\implies$ $f(f(x))=x$ and since $f(f(x))=f(x)$, we get $f(x)=x$ $\forall x$ which indeed is a solution
If $f(1)=-1$, then $P(x+1,1)$ $\implies$ $f(f(x))=-x$ and since $f(f(x))=f(x)$, we get $f(x)=-x$ $\forall x$ which is not a solution

Hence the unique answer : $\boxed{f(x)=x}$ $\forall x$
\end{solution}



\begin{solution}[by \href{https://artofproblemsolving.com/community/user/89681}{mitkkkk}]
	thanks pco and your post
\end{solution}



\begin{solution}[by \href{https://artofproblemsolving.com/community/user/89681}{mitkkkk}]
	pco, would you mind explaining it for me?  :(  it's a hard :blush:  :blush: 
\begin{tcolorbox}If $ f(x)\ne 0 $ then $f(x)f(0)\ne 0$ and so $ f(x)(f(0)+2)=2(f(0)+1) \forall x $ and so $f(x)=c $which is not a solution\end{tcolorbox}
is $c$ = $2(f(0)+1) \/ (f(0)+2) $ ? but if $ f(0)= -2$ then $c$=?
if not, $c$= ? and why \begin{bolded}it's  not  a  solution\end{bolded}
\end{solution}



\begin{solution}[by \href{https://artofproblemsolving.com/community/user/29428}{pco}]
	\begin{tcolorbox}pco, would you mind explaining it for me?  :(  it's a hard :blush:  :blush: 
\begin{tcolorbox}If $ f(x)\ne 0 $ then $f(x)f(0)\ne 0$ and so $ f(x)(f(0)+2)=2(f(0)+1) \forall x $ and so $f(x)=c $which is not a solution\end{tcolorbox}
is $c$ = $2(f(0)+1) \/ (f(0)+2) $ ? but if $ f(0)= -2$ then $c$=?
if not, $c$= ? and why \begin{bolded}it's  not  a  solution\end{bolded}\end{tcolorbox}
If $f(0)=-2$, the equation $f(x)(f(0)+2)=2(f(0)+1)$ becomes $0=-2$, impossible.

So $f(0)\ne -2$ and $f(x)(f(0)+2)=2(f(0)+1)$ $\implies$ $f(x)=\frac{2(f(0)+1)}{f(0)+2}$ constant
\end{solution}
*******************************************************************************
-------------------------------------------------------------------------------

\begin{problem}[Posted by \href{https://artofproblemsolving.com/community/user/72235}{Goutham}]
	Find a function $f(x)$ defined for all real values of $x$ such that for all $x$,
\[f(x+ 2) - f(x) = x^2 + 2x + 4,\]
and if $x \in [0, 2)$, then $f(x) = x^2.$
	\flushright \href{https://artofproblemsolving.com/community/c6h388769}{(Link to AoPS)}
\end{problem}



\begin{solution}[by \href{https://artofproblemsolving.com/community/user/44358}{crazyfehmy}]
	\begin{tcolorbox}Find a function $f(x)$ defined for all real values of $x$ such that for all $x$,
\[f(x+ 2) - f(x) = x^2 + 2x + 4,\]
and if $x \in [0, 2)$, then $f(x) = x^2.$\end{tcolorbox}
We can easily prove using induction that 

$x \in [2n,2n+2) \: \Longrightarrow \: f(x)=(n+1)x^2-2n(n+2)x+\frac{4n(n+1)(n+2)}{3}$ for all $n \in \mathbb{Z}.$
\end{solution}



\begin{solution}[by \href{https://artofproblemsolving.com/community/user/29428}{pco}]
	\begin{tcolorbox}Find a function $f(x)$ defined for all real values of $x$ such that for all $x$,
\[f(x+ 2) - f(x) = x^2 + 2x + 4,\]
and if $x \in [0, 2)$, then $f(x) = x^2.$\end{tcolorbox}
Let $f(x)=g(x)+\frac{x^3+8x}6$ and the problem becomes :

$g(x)=\frac{-x^3+6x^2-8x}6$ $\forall x\in[0,2)$
$g(x+2)=g(x)$

And so $g(x)=g(x-2\left\lfloor\frac x2\right\rfloor)$ $=\frac{-(x-2\left\lfloor\frac x2\right\rfloor)^3+6(x-2\left\lfloor\frac x2\right\rfloor)^2-8(x-2\left\lfloor\frac x2\right\rfloor)}6$

And so $\boxed{f(x)=\frac{x^3+8x}6+\frac{-(x-2\left\lfloor\frac x2\right\rfloor)^3+6(x-2\left\lfloor\frac x2\right\rfloor)^2-8(x-2\left\lfloor\frac x2\right\rfloor)}6}$
\end{solution}
*******************************************************************************
-------------------------------------------------------------------------------

\begin{problem}[Posted by \href{https://artofproblemsolving.com/community/user/61832}{LJK}]
	Find all functions $f,g : \mathbb{R} \to \mathbb{R}$ that satisfy
\[f(x+y)=g\left(\frac1x+\frac1y\right){(xy)}^{2008}\]
For all non-zero real numbers $x$ and $y$.
	\flushright \href{https://artofproblemsolving.com/community/c6h388782}{(Link to AoPS)}
\end{problem}



\begin{solution}[by \href{https://artofproblemsolving.com/community/user/64868}{mahanmath}]
	\begin{tcolorbox}Find functions $f,g : \mathbb{R} \to \mathbb{R}$ that :
\[f(x+y)=g(\frac1x+\frac1y){(xy)}^{2008}\]\end{tcolorbox}
Just find or find all ? :maybe: 

$f(x)=g(x) = x^{2008}$ works for nonzero $x$ .

In general $f(x) = x^{2008} . F(x) , g(x)= x^{2008} .G(x) $ works when $F(x+y)=G(\frac{1}{x} + \frac{1}{y})$
\end{solution}



\begin{solution}[by \href{https://artofproblemsolving.com/community/user/29428}{pco}]
	\begin{tcolorbox}Find functions $f,g : \mathbb{R} \to \mathbb{R}$ that :
\[f(x+y)=g(\frac1x+\frac1y){(xy)}^{2008}\]\end{tcolorbox}
First, notice that once again the problem is badly stated. Obviously the functional equation cant be respected for any $x,y$ : $xy=0$ should be excluded :(

Let $P(x,y)$ be the assertion $f(x+y)=g(\frac 1x+\frac 1y)(xy)^{2008}$

1) $f(x)=g(1)x^{2008}$ $\forall x\notin[0,4)$
========================
Let $u\notin [0,4)$. 
Let $x$ a real root of the equation $x^2-ux+u=0$.
Clearly $x\ne 0$ and let $y=\frac ux$ so that $xy=x+y=u$
$P(x,y)$ $\implies$ $f(u)=g(1)u^{2008}$
Q.E.D.

2) $g(x)=g(1)x^{2008}$ $\forall x\ne 0$
==========================
Let $u\ne 0$
Let $a=4$ if $u<0$ and $a=-4$ if $u>0$ so that $au<0$
Let $x$ a real root of the equation $ux^2-aux+a=0$
Clearly $x\ne 0$ and let $y=\frac a{ux}$ so that $x+y=a$ and $\frac 1x+\frac 1y=u$
$P(x,y)$ $\implies$ $f(a)=g(u)(xy)^{2008}$
Since $a\notin[0,4)$, we also have $f(a)=g(1)a^{2008}$

So $g(1)a^{2008}=g(u)(xy)^{2008}$ and so $g(u)=g(1)(\frac a{xy})^{2008}$ $=g(1)u^{2008}$
Q.E.D

3) $f(x)=g(1)x^{2008}$ $\forall x\ne 0$
=======================
Just consider $x,y\ne 0$ and $x+y\ne 0$ and plug the result from 2) above in $P(x,y)$ and you get $f(x+y)=g(1)(x+y)^{2008}$
Q.E.D

4) $f(0)=g(0)=0$
================
Let $x\ne 0$ : $P(x,-x)$ $\implies$ $f(0)=g(0)x^{4016}$
Hence the result.

\begin{bolded}Hence the solution \end{bolded}\end{underlined}:
$f(x)=g(x)=ax^{2008}$ $\forall x$
\end{solution}



\begin{solution}[by \href{https://artofproblemsolving.com/community/user/61832}{LJK}]
	I'm really sorry
I edited my above post
\end{solution}
*******************************************************************************
-------------------------------------------------------------------------------

\begin{problem}[Posted by \href{https://artofproblemsolving.com/community/user/92753}{WakeUp}]
	Find all functions $f:\mathbb{Z}_{>0}\rightarrow\mathbb{Z}_{>0}$ that satisfy the following two conditions:
[list]$\bullet\ f(n)$ is a perfect square for all $n\in\mathbb{Z}_{>0}$
$\bullet\ f(m+n)=f(m)+f(n)+2mn$ for all $m,n\in\mathbb{Z}_{>0}$.[\/list]
	\flushright \href{https://artofproblemsolving.com/community/c6h388897}{(Link to AoPS)}
\end{problem}



\begin{solution}[by \href{https://artofproblemsolving.com/community/user/29428}{pco}]
	\begin{tcolorbox}Find all functions $f:\mathbb{Z}_{>0}\rightarrow\mathbb{Z}_{>0}$ that satisfy the following two conditions:
[list]$\bullet\ f(n)$ is a perfect square for all $n\in\mathbb{Z}_{>0}$
$\bullet\ f(m+n)=f(m)+f(n)+2mn$ for all $m,n\in\mathbb{Z}_{>0}$.[\/list]\end{tcolorbox}
Second line shows that $f(x)$ is strictly increasing.
Let $f(x)=g(x)^2$ with $g(x)>0$ strictly increasing too.

So $g(m+1)\ge g(m)+1$

So $g(m+1)^2=g(m)^2+g(1)^2+2m\ge g(m)^2+2g(m)+1$ and so $g(m)\le m+\frac{g(1)^2-1}2$

So $g(m+1)>g(m)+1$ can occur only a finite number of times and so $g(m)=m+c$ $\forall m$ great enough.

Pluging this in original equation, we get $c=0$ and so $g(m)=m$ $\forall m$ great enough and so $g(m)=m$ $\forall m$ since positive and strictly increasing.

Hence the unique answer $\boxed{f(n)=n^2}$ which indeed is a solution
\end{solution}



\begin{solution}[by \href{https://artofproblemsolving.com/community/user/47645}{lajanugen}]
	$f(n+1)-f(n)=2n+f(1)$
And hence, $f(n)=n(n+c) ; c=f(1)-1$
Since $4f(n)=(2n+c)^2-c^2$ and RHS cannot be a square for large $n$ unless $c$ is zero, we conclude $c=0$ and hence the only solution $f(n)=n^2$
\end{solution}



\begin{solution}[by \href{https://artofproblemsolving.com/community/user/92753}{WakeUp}]
	$f(2)=f(1)+f(1)+2=2f(1)+2$
$f(3)=f(2)+f(1)+4=3f(1)+6$
$f(4)=f(3)+f(1)+6=4(f(1)+3)$

Since $f(4)$ is square, so is $4(f(1)+3)$ which means $f(1)+3$ is a square. But $f(1)$ is a square also, and the only squares that differ by $3$ are $1^2$ and $2^2$ so that $f(1)=1$.

To prove that $f(n)=n^2$ for all $n$, the base case is done, and for the inductive step note that $f(m+1)=f(m)+f(1)+2m=m^2+2m+1=(m+1)^2$.

So $f(n)=n^2$ for all $n$.

lajanugen's solution doesn't make any sense.
\end{solution}



\begin{solution}[by \href{https://artofproblemsolving.com/community/user/64716}{mavropnevma}]
	Maybe a closer look at lajanugen's solution will help. Let us take it step-by-step.

For $m=1$ we get $f(n+1)-f(n)=2n+f(1)$, for all $n\geq 1$. This is a linear recurrence relation for the sequence $a_{n+1} = a_n + (2n+f(1))$, by taking $a_n = f(n)$. Denote $b_n = n^2 +n(f(1)-1)$; it is immediately checked that we also have $b_{n+1} = b_n + (2n + f(1))$. Subtracting the two relations yields $a_{n+1} - b_{n+1} = a_{n} - b_{n}$, thus constantly equal to $a_1 - b_1 = 0$.

And hence, $f(n)=n(n+c)$, for $c=f(1)-1$.

Since then $4f(n)=(2n+c)^2-c^2$ (quite easy to check), and $f(n)$ is given to be a perfect square $x_n^2$, it follows $(2n+c)^2 - x_n^2 = c^2$ for all $n\geq 1$, impossible if $c\neq 0$, since the difference between two unequal squares indefinitely increases when one of them indefinitely increases.

We conclude $c=0$ and hence the only solution $f(n)=n^2$. It remains to be seen that this indeed verifies the conditions, which it does.

So I suggest, dear WakeUp, to wake up and revise your rating of lajanugen's post!
\end{solution}



\begin{solution}[by \href{https://artofproblemsolving.com/community/user/92753}{WakeUp}]
	OK! I've woken up! I never meant that it was incorrect, but that it was extraordinarily vague.
\end{solution}



\begin{solution}[by \href{https://artofproblemsolving.com/community/user/47645}{lajanugen}]
	thankyou very much mavropnevma, for taking your time to comment on my solution
\end{solution}



\begin{solution}[by \href{https://artofproblemsolving.com/community/user/99076}{RSM}]
	I did not read the solutions.
So I don't know whether this logic has been posted before
It is easy to see that
f(n)=n(n-1+f(1))
Now choose n to be a prime p
Clearly p divides f(1)-1
For any prime this holds
So a finite number is divisible by infinitely many primes
This implies f(1)=1 :roll:
\end{solution}



\begin{solution}[by \href{https://artofproblemsolving.com/community/user/109774}{littletush}]
	let $f(1)=c^2$,then$f(m+1)=f(m)+2m+c^2$
hence $f(m)=f(1)+(m-1)c^2+2[1+...+(m-1)]=m(c^2+m-1)$
if $c>1$ let m be a prime
greater then $c^2-1$,then $p||f(m)$ contradiction!
hence $c=1,f(m)=m^2$.
\end{solution}
*******************************************************************************
-------------------------------------------------------------------------------

\begin{problem}[Posted by \href{https://artofproblemsolving.com/community/user/90274}{will_energetic}]
	Does there exist a function $f: \mathbb R \to \mathbb R$ such that
\[{\frac{f(x)+f(y)}{2}}\ge f\left(\frac{x+y}{2}\right)+|x-y|\]
holds for all $x,y \in \mathbb R$?
	\flushright \href{https://artofproblemsolving.com/community/c6h389157}{(Link to AoPS)}
\end{problem}



\begin{solution}[by \href{https://artofproblemsolving.com/community/user/90622}{myro111}]
	\begin{tcolorbox}$f: R\to R$:

${\frac{f(x)+f(y)}{2}}\ge f(\frac{x+y}{2})+|x-y|, \forall x,y \in R$\end{tcolorbox}


Is your problem ${\frac{f(x)+f(y)}{2}}\ge f(\frac{x+y}{2})+|x-y|, \forall x,y \in R$ and do we need to find all functions or what?
\end{solution}



\begin{solution}[by \href{https://artofproblemsolving.com/community/user/90274}{will_energetic}]
	\begin{tcolorbox}[quote="will_energetic"]$f: R\to R$:

${\frac{f(x)+f(y)}{2}}\ge f(\frac{x+y}{2})+|x-y|, \forall x,y \in R$\end{tcolorbox}


Is your problem ${\frac{f(x)+f(y)}{2}}\ge f(\frac{x+y}{2})+|x-y|, \forall x,y \in R$ and do we need to find all functions or what?\end{tcolorbox}

I editted!
\end{solution}



\begin{solution}[by \href{https://artofproblemsolving.com/community/user/90622}{myro111}]
	\begin{tcolorbox}$f: R\to R$:

${\frac{f(x)+f(y)}{2}}\ge f(\frac{x+y}{2})+|x-y|, \forall x,y \in R$

exist functions $f$ or not?\end{tcolorbox}
\end{solution}



\begin{solution}[by \href{https://artofproblemsolving.com/community/user/90622}{myro111}]
	Does anyone know how to solve this??
\end{solution}



\begin{solution}[by \href{https://artofproblemsolving.com/community/user/29428}{pco}]
	\begin{tcolorbox}$f: R\to R$:

${\frac{f(x)+f(y)}{2}}\ge f(\frac{x+y}{2})+|x-y|, \forall x,y \in R$

exist functions $f$ or not?\end{tcolorbox}
Let $P(x,y)$ be the assertion $\frac{f(x)+f(y)}2\ge f(\frac{x+y}2)+|x-y|$

$P(x+2y,x)$ $\implies$ $f(x+2y)+f(x)\ge 2f(x+y)+4|y|$ and so $f(x+2y)-f(x+y)\ge f(x+y)-f(x)+4|y|$

$P(x+3y,x+y)$ $\implies$ $f(x+3y)-f(x+2y)\ge f(x+2y)-f(x+y)+4|y|$ $\ge f(x+y)-f(x)+8|y|$
...
And so $f(x+ky)-f(x+(k-1)y)\ge f(x+y)-f(x)+4(k-1)|y|$

Summing all these inequalities for $k=1,2,...,n$, we get $f(x+ny)-f(x)\ge n(f(x+y)-f(x))+4|y|(1+2+3+...+(n-1))$

And so (I1) : $f(x+ny)-f(x)\ge n(f(x+y)-f(x))+2n(n-1)|y|$

Setting $y=\frac 1n$ in (I1), we get $f(x+1)-f(x)\ge n(f(x+\frac 1n)-f(x))+2(n-1)$

And so (I2) : $f(x+\frac 1n)-f(x)\le \frac{f(x+1)-f(x)}{n}-2\frac{n-1}n$

Setting now $y=-\frac 1n$ in (I1), we get $f(x-1)-f(x)\ge n(f(x-\frac 1n)-f(x))+2(n-1)$

And so (I3) : $f(x-\frac 1n)-f(x)\le \frac{f(x-1)-f(x)}{n}-2\frac{n-1}n$

Adding (I2) and (I3) and dividing by two we get  : 

$\frac{f(x+\frac 1n)+f(x-\frac 1n)}2\le f(x)+\frac{f(x+1)+f(x-1)-2f(x)}{2n}-2\frac{n-1}n$

But $P(x-\frac 1n,x+\frac 1n)$ $\implies$ $\frac{f(x+\frac 1n)+f(x-\frac 1n)}2\ge f(x)+\frac 2n$ and so :

$f(x)+\frac 2n\le f(x)+\frac{f(x+1)+f(x-1)-2f(x)}{2n}-2\frac{n-1}n$

$\iff$ $\frac 2n\le \frac{f(x+1)+f(x-1)-2f(x)}{2n}-2\frac{n-1}n$

Setting $n\to +\infty$ in this inequality implies $0\le -2$, impossible.

So $\boxed{\text{no such }f(x)}$
\end{solution}



\begin{solution}[by \href{https://artofproblemsolving.com/community/user/90622}{myro111}]
	\begin{tcolorbox}[quote="will_energetic"]$f: R\to R$:

${\frac{f(x)+f(y)}{2}}\ge f(\frac{x+y}{2})+|x-y|, \forall x,y \in R$

exist functions $f$ or not?\end{tcolorbox}
Let $P(x,y)$ be the assertion $\frac{f(x)+f(y)}2\ge f(\frac{x+y}2)+|x-y|$

$P(x+2y,x)$ $\implies$ $f(x+2y)+f(x)\ge 2f(x+y)+4|y|$ and so $f(x+2y)-f(x+y)\ge f(x+y)-f(x)+4|y|$

$P(x+3y,x+y)$ $\implies$ $f(x+3y)-f(x+2y)\ge f(x+2y)-f(x+y)+4|y|$ $\ge f(x+y)-f(x)+8|y|$
...
And so $f(x+ky)-f(x+(k-1)y)\ge f(x+y)-f(x)+4(k-1)|y|$

Summing all these inequalities for $k=1,2,...,n$, we get $f(x+ny)-f(x)\ge n(f(x+y)-f(x))+4|y|(1+2+3+...+(n-1))$

And so (I1) : $f(x+ny)-f(x)\ge n(f(x+y)-f(x))+2n(n-1)|y|$

Setting $y=\frac 1n$ in (I1), we get $f(x+1)-f(x)\ge n(f(x+\frac 1n)-f(x))+2(n-1)$

And so (I2) : $f(x+\frac 1n)-f(x)\le \frac{f(x+1)-f(x)}{n}-2\frac{n-1}n$

Setting now $y=-\frac 1n$ in (I1), we get $f(x-1)-f(x)\ge n(f(x-\frac 1n)-f(x))+2(n-1)$

And so (I3) : $f(x-\frac 1n)-f(x)\le \frac{f(x-1)-f(x)}{n}-2\frac{n-1}n$

Adding (I2) and (I3) and dividing by two we get  : 

$\frac{f(x+\frac 1n)+f(x-\frac 1n)}2\le f(x)+\frac{f(x+1)+f(x-1)-2f(x)}{2n}-2\frac{n-1}n$

But $P(x-\frac 1n,x+\frac 1n)$ $\implies$ $\frac{f(x+\frac 1n)+f(x-\frac 1n)}2\ge f(x)+\frac 2n$ and so :

$f(x)+\frac 2n\le f(x)+\frac{f(x+1)+f(x-1)-2f(x)}{2n}-2\frac{n-1}n$

$\iff$ $\frac 2n\le \frac{f(x+1)+f(x-1)-2f(x)}{2n}-2\frac{n-1}n$

Setting $n\to +\infty$ in this inequality implies $0\le -2$, impossible.

So $\boxed{\text{no such }f(x)}$\end{tcolorbox}

Nice solution pco,how do you just think of taking $P(x+2y,x)$ and ...?
Thank you.
\end{solution}



\begin{solution}[by \href{https://artofproblemsolving.com/community/user/29428}{pco}]
	\begin{tcolorbox} Nice solution pco,how do you just think of taking $P(x+2y,x)$ and ...?
Thank you.\end{tcolorbox}

I just tried to simplify and replaced $(x,y)$ by $(x,x+z)$ and then decided to write $z=2t$ in order to avoid the $\frac 12$ :oops:
\end{solution}



\begin{solution}[by \href{https://artofproblemsolving.com/community/user/13}{enescu}]
	This is USAMO 2000, see [url]http://www.artofproblemsolving.com/Forum/viewtopic.php?p=299244&sid=a2b46b962cf748f1381c7ba518bfec8c#p299244[\/url]
\end{solution}
*******************************************************************************
-------------------------------------------------------------------------------

\begin{problem}[Posted by \href{https://artofproblemsolving.com/community/user/93044}{nguyenhung}]
	For any real number $x$, denoting $f(x)$ is the maximum value of $y=\sqrt{t^2+2t+2}$ on $[x-2 \; , \; x]$.

a) Prove $f(x)$ is an even function.

b) Prove that the sequence $[{a_n}]$ with $a_n=\left\{ {f\left( n \right)} \right\}$ has finite limit and compute that limit.
	\flushright \href{https://artofproblemsolving.com/community/c6h389158}{(Link to AoPS)}
\end{problem}



\begin{solution}[by \href{https://artofproblemsolving.com/community/user/29428}{pco}]
	\begin{tcolorbox}For any real number $x$, denoting $f(x)$ is the maximum value of $y=\sqrt{t^2+2t+2}$ on $[x-2 \; , \; x]$.

a) Prove $f(x)$ is an even function.

b) Prove that the sequence $[{a_n}]$ with $a_n=\left\{ {f\left( n \right)} \right\}$ has finite limit and compute that limit.\end{tcolorbox}
$g(t)=t^2+2t+2$ is positive and continuous and has no local maximum and so $\max_{t\in[a,b]}g(t)=\max(g(a),g(b))$

And since $g(x)=x^2+2x+2$ and $g(x-2)=x^2-2x+2$, we get $\max_{t\in[x-2,x]}g(t)=x^2+2|x|+2$

And so $f(x)=\sqrt {x^2+2|x|+2}$ is indeed an even function.

$f(n)=\sqrt{n^2+2n+2}=n+1+(\sqrt{n^2+2n+2}-(n+1))$ $=n+1+\frac 1{\sqrt{n^2+2n+2}+n+1}$

$\frac 1{\sqrt{n^2+2n+2}+n+1}\in[0,1)$ and so $\{f(n)\}=\frac 1{\sqrt{n^2+2n+2}+n+1}$ and its limit is $0$
\end{solution}
*******************************************************************************
-------------------------------------------------------------------------------

\begin{problem}[Posted by \href{https://artofproblemsolving.com/community/user/61896}{Mateescu Constantin}]
	Find all continuous functions $f:(1,2)\to\mathbb R$ so that: \[f\left(\frac {3xy-4x-4y+6}{2xy-3x-3y+5}\right)=f(x)+f(y),\quad\forall\ x,y\in (1,2).\]
	\flushright \href{https://artofproblemsolving.com/community/c6h389227}{(Link to AoPS)}
\end{problem}



\begin{solution}[by \href{https://artofproblemsolving.com/community/user/29428}{pco}]
	\begin{tcolorbox}Find all continuous functions $f:(1,2)\to\mathbb R$ so that : $f\left(\frac {3xy-4x-4y+6}{2xy-3x-3y+5}\right)=f(x)+f(y)\ ,\ \forall\ x,y\in (1,2)$ .\end{tcolorbox}
Let $g(x)=f(\frac{x+2}{x+1})$ : $g(x)$ is a continuous function from $\mathbb R^+\to\mathbb R$

We get $f(x)=g(\frac{2-x}{x-1})$ and the functional equation becomes $g(\frac{2-x}{x-1}\times\frac{2-y}{y-1})$ $=g(\frac{2-x}{x-1})$ $+g(\frac{2-y}{y-1})$ $\forall x,y\in(1,2)$

And so $g(xy)=g(x)+g(y)$ $\forall x,y\in\mathbb R^+$

And the continuous solutions of this equation are well known : $g(x)=a\ln x$

Hence the solution : $\boxed{f(x)=a\ln(2-x)-a\ln(x-1)}$
\end{solution}
*******************************************************************************
-------------------------------------------------------------------------------

\begin{problem}[Posted by \href{https://artofproblemsolving.com/community/user/99076}{RSM}]
	Find a bijection between $\mathbb Q$ and $\mathbb Z$.
	\flushright \href{https://artofproblemsolving.com/community/c6h389308}{(Link to AoPS)}
\end{problem}



\begin{solution}[by \href{https://artofproblemsolving.com/community/user/29428}{pco}]
	\begin{tcolorbox}Find a bijection between Q and Z.\end{tcolorbox}
One very classical :

Let us first build a bijection from $\mathbb N\to\mathbb Z^*$, set of non zero integers :
$h(2n)=n$
$h(2n-1)=-n$

Consider then $f(x)$ from $\mathbb Z\to\mathbb Q$ as :
$f(0)=0$
$\forall x>0$ : Write $x=\prod p_i^{n_i}$ with $p_i$ prime numbers and $n_i>0$. Then $f(x)=\prod p_i^{h(n_i)}$
$\forall x<0$ : $f(x)=-f(-x)$

And you got a bijection.
\end{solution}



\begin{solution}[by \href{https://artofproblemsolving.com/community/user/99076}{RSM}]
	Thanks pco
\end{solution}
*******************************************************************************
-------------------------------------------------------------------------------

\begin{problem}[Posted by \href{https://artofproblemsolving.com/community/user/99194}{letrongquang1995}]
	Suppose that the function $f: \mathbb N \to \mathbb N$ is such that $f(m) \neq f(n)$ for any two positive integers $m$ and $n$ with $m \neq n$, and
\[f(f(n))\leq \frac{n+4f(n)}{5}\]
for all positive integers $n$. Prove that $f(n)=n$.
	\flushright \href{https://artofproblemsolving.com/community/c6h389309}{(Link to AoPS)}
\end{problem}



\begin{solution}[by \href{https://artofproblemsolving.com/community/user/99076}{RSM}]
	Thanks pco for his correction
My solution was wrong
\end{solution}



\begin{solution}[by \href{https://artofproblemsolving.com/community/user/29428}{pco}]
	\begin{tcolorbox}Write f(n)=n+g(n)
g(n) is a function from Z to Z\end{tcolorbox}
No, $g(n)$ is a function from $\mathbb Z^+\to\mathbb Z$
\begin{tcolorbox}Substituting this we get
g(n)+g(n+g(n))<=0\end{tcolorbox}
No, we get $\frac{g(n)}5+g(n+g(n))\le 0$

...
\end{solution}



\begin{solution}[by \href{https://artofproblemsolving.com/community/user/29428}{pco}]
	\begin{tcolorbox}$f:Z^{+}\to Z^+$ satisfy
1, $f(m)\neq f(n)$ for two  positive interger $m\neq n$
2, $f(f(n))\leq \frac{n+4f(n)}{5}$ for all positive interger $n$
Prove that $f(n)=n$\end{tcolorbox}
If, for some $n$ we have $f(n)<n$, then the equation implies $f(f(n))<\frac{n+4n}5$ and so $f(f(n))<n$

And so $f^{[k]}(n)<n$ $\forall k>0$ (exponent is used for composition of functions)
Then, since we just have $n-1$ values $<n$, we need to have $f^{[k_1]}(n)=f^{[k_2]}(n)$ for some $k_1>k_2$
Using then injectivity property, we get $f^{[k_1-k_2]}(n)=n$, in contradiction with $f^{[k]}(n)<n$ $\forall k>0$

So $f(n)\ge n$ $\forall n$

So $f(n)\le f(f(n))\le \frac{n+4f(n)}5$ and so $f(n)\le\frac{n+4f(n)}5$ and so $f(n)\le n$

And so $\boxed{f(n)=n}$
\end{solution}



\begin{solution}[by \href{https://artofproblemsolving.com/community/user/95245}{CPT_J_H_Miller}]
	Since $f$ is an one-one function, an inverse exists. So can we simply substitute in 
$n=f^{-1}(n)$, which would give us $f(n)$ $\le$ $n$, and thus $f(n)=n$?
\end{solution}



\begin{solution}[by \href{https://artofproblemsolving.com/community/user/29428}{pco}]
	\begin{tcolorbox}Since $f$ is an one-one function, an inverse exists. So can we simply substitute in 
$n=f^{-1}(n)$, which would give us $f(n)$ $\le$ $n$, and thus $f(n)=n$?\end{tcolorbox}
The function is injective (one to one) but nothing allow to immediately conclude that it is surjective.
So we dont know if $f^{-1}(n)$ may be defined over $\mathbb N$
\end{solution}
*******************************************************************************
-------------------------------------------------------------------------------

\begin{problem}[Posted by \href{https://artofproblemsolving.com/community/user/78444}{Babai}]
	Find all functions $f: \mathbb R \to \mathbb R$ such that $f(x)^2 +2yf(x)+f(y)=f(y+f(x))$ for all reals $x$ and $y$.
	\flushright \href{https://artofproblemsolving.com/community/c6h389373}{(Link to AoPS)}
\end{problem}



\begin{solution}[by \href{https://artofproblemsolving.com/community/user/61832}{LJK}]
	$f(y)+f(x)$ or $f(y+f(x))$
\end{solution}



\begin{solution}[by \href{https://artofproblemsolving.com/community/user/44358}{crazyfehmy}]
	\begin{tcolorbox}Find all functions $f:R\rightarrow R $ such that $f(x)^2 +2yf(x)+f(y)=f(y+f(x)$ for all real $x,y$
[I suppose that the solutions are:$f(x)=0$ or$f(x)=x^2$ but can't solve it]\end{tcolorbox}
Here is a solution.

Firstly, taking $y=0,$ we get $f(f(x))=[(f(x)]^2+f(0)$ for all $x \in \mathbb{R}.$

$f(y+f(x))-f(y)=f(x)(f(x)+2y).$ If $f \not\equiv 0,$ then there exists a real number $x_0$ such that $f(x_0) \neq 0.$

So, $f(y+f(x_0))-f(y)=f(x_0)(f(x_0)+2y).$ Hence, for all real numbers $r,$ there exists a real number $y$ such that

$f(y+f(x_0))-f(y)=f(x_0)(f(x_0)+2y)=r$ since $f(x_0) \neq 0.$ 

So, if $f \not\equiv 0,$ then $f(x)-f(y)$ is surjective.

Now, taking $y=f(z)-f(x)$ in the original equation, we get

$-[f(x)]^2+2f(x)f(z)+f(f(z)-f(x))=f(f(z))=[f(z)]^2+f(0),$ therefore

$f(f(z)-f(x))=[f(z)-f(x)]^2+f(0)$ and since $f(z)-f(x)$ is surjective, then we get

$f(x)=x^2+f(0)$ for all $x \in \mathbb{R}.$

By substituting, we can easily prove that $f(x)=x^2+c$ satisfies the equation. So, all solutions are:

$f(x)=0 \: , \forall x \in \mathbb{R}$ and 

$f(x)=x^2+c \: , \forall x \in \mathbb{R}$ where $c$ is an arbitrary real constant.
\end{solution}



\begin{solution}[by \href{https://artofproblemsolving.com/community/user/29428}{pco}]
	\begin{tcolorbox}Find all functions $f:R\rightarrow R $ such that $f(x)^2 +2yf(x)+f(y)=f(y+f(x)$ for all real $x,y$
[I suppose that the solutions are:$f(x)=0$ or$f(x)=x^2$ but can't solve it]\end{tcolorbox}
I suppose RHS is $f(y+f(x))$

Let $P(x,y)$ be the assertion $f(x)^2+2yf(x)+f(y)=f(y+f(x))$

$f(x)=0$ $\forall x$ is a solution and we'll from now look for non all zero solutions.
Let $u$ such that $f(u)\ne 0$

$P(u,\frac{t-f(u)^2}{2f(u)})$ $\implies$ $t=f(\frac{t-f(u)^2}{2f(u)}+f(u))-f(\frac{t-f(u)^2}{2f(u)})$
and so any real $t$ may be written as $f(a)-f(b)$ for some $a,b\in\mathbb R$

$P(b,-f(b))$ $\implies$ $-f(b)^2+f(-f(b))=f(0)$
$P(a,-f(b))$ $\implies$ $f(a)^2-2f(a)f(b)+f(-f(b))=f(f(a)-f(b))$

Subtracting these two lines, we get $f(f(a)-f(b))=(f(a)-f(b))^2+f(0)$ and so $f(t)=t^2+f(0)$ $\forall t$ which indeed is a solution.

Hence the answer :
$f(x)=0$ $\forall x$
$f(x)=x^2+a$ $\forall x$

*edit\end{underlined}* : too late :)
\end{solution}
*******************************************************************************
-------------------------------------------------------------------------------

\begin{problem}[Posted by \href{https://artofproblemsolving.com/community/user/97691}{shikha1128}]
	1. If $f(x)$ is a polynomial of degree $n$ such that \[f(0)=0, f(1)=\frac{1}{2}, \ldots, f(n)=\frac{n}{(n+1)},\] then find $f(n+1)$.

2. Let $f(x)=1+\frac{2}{x}$ and denote by $f_{n}(x)$ the composition of $f$ with itself $n$ times. Find the maximum number of real roots of $f_{n}(x)$.
	\flushright \href{https://artofproblemsolving.com/community/c6h389393}{(Link to AoPS)}
\end{problem}



\begin{solution}[by \href{https://artofproblemsolving.com/community/user/29428}{pco}]
	\begin{tcolorbox}1. If f(x) is a polynomial of degree n such that $f(0)=0, f(1)=\frac{1}{2},......f(n)=\frac{n}{(n+1)}$, then find $f(n+1).$\end{tcolorbox}
Let $P(x)=(x+1)f(x)-x$
$P(x)$ has degree $n+1$ and $P(x)=0$ for $x=0,1,2,...,n$
So $P(x)=ax(x-1)(x-2)...(x-n)$

Setting $x=-1$, we get $1=a(-1)^{n+1}(n+1)!$ and so $a=\frac{(-1)^{n+1}}{(n+1)!}$

And so $f(x)=\frac{(n+1)!x+(-1)^{n+1}x(x-1)(x-2)...(x-n)}{(n+1)!(x+1)}$

And $\boxed{f(n+1)=\frac{n+1+(-1)^{n+1}}{n+2}}$
\end{solution}



\begin{solution}[by \href{https://artofproblemsolving.com/community/user/29428}{pco}]
	\begin{tcolorbox} 2. Let $f(x)=1+\frac{2}{x}$ and $f_{n}(x)=fff.......f(x)$ then find the maximum number of real roots of $f_{n}(x)$.\end{tcolorbox}
Let $g(x)=\frac 2{x-1}$

It's easy to check that $g^{[n]}(0)$ is defined $\forall n$  : $g^{[n]}(0)=\frac 3{1-(-2)^{n+1}}-1$

and so $f_n(x)=0$ $\iff$ $x=g^{[n]}(0)=\frac 3{1-(-2)^{n+1}}-1$

And so exactly one real root for any $n$
\end{solution}



\begin{solution}[by \href{https://artofproblemsolving.com/community/user/97691}{shikha1128}]
	\begin{tcolorbox}$1=a(-1)^{n+1}(n+1)!$ and so $a=\frac{(-1)^{n+1}}{(n+1)!}$\end{tcolorbox}

Sir a should be $\frac{1}{(-1)^{n+1}(n+1)!}$na??
\end{solution}



\begin{solution}[by \href{https://artofproblemsolving.com/community/user/29428}{pco}]
	\begin{tcolorbox}[quote="pco"]$1=a(-1)^{n+1}(n+1)!$ and so $a=\frac{(-1)^{n+1}}{(n+1)!}$\end{tcolorbox}

Sir a should be $\frac{1}{(-1)^{n+1}(n+1)!}$na??\end{tcolorbox}

$\frac 1{(-1)^{n+1}}=(-1)^{n+1}$ :)
\end{solution}



\begin{solution}[by \href{https://artofproblemsolving.com/community/user/97691}{shikha1128}]
	:oops: 
thnx for the solution!!!
\end{solution}
*******************************************************************************
-------------------------------------------------------------------------------

\begin{problem}[Posted by \href{https://artofproblemsolving.com/community/user/51470}{Potla}]
	Find all functions $f:\mathbb{R}\to \mathbb R$ satisfying
\[f(x+y)f(x-y)=\left(f(x)+f(y)\right)^2-4x^2f(y),\]
For all $x,y\in\mathbb R$.
	\flushright \href{https://artofproblemsolving.com/community/c6h390192}{(Link to AoPS)}
\end{problem}



\begin{solution}[by \href{https://artofproblemsolving.com/community/user/87195}{SCP}]
	\begin{tcolorbox}Let $f:\mathbb{R}\to \mathbb R$ be a functional equation satisfying
\[f(x+y)f(x-y)=\left(f(x)+f(y)\right)^2-4x^2f(y),\]
For all $x,y\in\mathbb R,$ where $\mathbb R$ denotes the set of all real numbers.\end{tcolorbox}

$0,0$ gives $f(0)=0$ 
$0,y$ gives $f(y)f(-y)=f^2(y)$ and after $2$ cases, there holds always $f(y)=f(-y)$
Then $y,-y$ gives $0=f(0)=4f^2(y)-4y^2f(y)$
and so if $y \not 0$ we see $f(y)=0$ or $f(y)=y^2$
easy to check these are only solutions.
\end{solution}



\begin{solution}[by \href{https://artofproblemsolving.com/community/user/29428}{pco}]
	\begin{tcolorbox}Let $f:\mathbb{R}\to \mathbb R$ be a functional equation satisfying
\[f(x+y)f(x-y)=\left(f(x)+f(y)\right)^2-4x^2f(y),\]
For all $x,y\in\mathbb R,$ where $\mathbb R$ denotes the set of all real numbers.\end{tcolorbox}
Let $P(x,y)$ be the assertion $f(x+y)f(x-y)=(f(x)+f(y))^2-4x^2f(y)$

$P(0,0)$ $\implies$ $f(0)^2=4f(0)^2$ $\implies$ $f(0)=0$
$P(x,x)$ $\implies$ $f(x)(f(x)-x^2)=0$ and so $\forall x$ : either $f(x)=0$, either $f(x)=x^2$

Suppose now that $\exists a\ne 0, b\ne 0$ such that $f(a)=a^2$ and $f(b)=0$
$P(a,b)$ $\implies$ $f(a+b)f(a-b)=a^4\ne 0$
So $f(a+b)\ne 0$ and so $f(a+b)=(a+b)^2$
Same, $f(a-b)\ne 0$ and so $f(a-b)=(a-b)^2$
And our equality becomes $(a^2-b^2)^2=a^4$ and so, since $b\ne 0$ : $b^2=2a^2$
And this is obviously impossible, else :
Choose $c\notin\{0,b,-b,a,-a\}$ :
If $f(c)=0$, we get (using $a,c$ instead of $a,b$ in lines above : $c^2=2a^2=b^2$, impossible
If $f(c)=c^2$, we get, (using $c,b$ instead pf $a,b$ in lines above : $b^2=2c^2$ and so $c^2=a^2$, impossible.

And so :
either $f(x)=0$ $\forall x$ which indeed is a solution
either $f(x)=x^2$ $\forall x$ which indeed is a soluton.

@SCP : you got "$\forall x$ : either $f(x)=0$, either $f(x)=x^2$" and not "either $f(x)=0$ $\forall x$ , either $f(x)=x^2$ $\forall x$ " and you miss the second part of the proof.
\end{solution}



\begin{solution}[by \href{https://artofproblemsolving.com/community/user/72235}{Goutham}]
	\begin{tcolorbox}Let $f:\mathbb{R}\to \mathbb R$ be a functional equation satisfying
\[f(x+y)f(x-y)=\left(f(x)+f(y)\right)^2-4x^2f(y),\]
For all $x,y\in\mathbb R,$ where $\mathbb R$ denotes the set of all real numbers.\end{tcolorbox}

$y=0\Longrightarrow f(0)=0$ or $f(x)=\frac{4x^2-f(0)}{2}$. Suppose $f(x)=\frac{4x^2-f(0)}{2}$. $x=0$ implies that $f(0)=0\Longrightarrow f(x)=2x^2$ which when substituted back proves that it is not a solution. Now, $f(0)=0$. $x=0\Longrightarrow f(x)=0$ for all $x$ or $f$ is even. If $f$ is even, $x=-y$ implies that $f(x)=0$ for all $x$ or $f(x)=x^2$ which are solutions.
\end{solution}



\begin{solution}[by \href{https://artofproblemsolving.com/community/user/29428}{pco}]
	\begin{tcolorbox} ... $x=-y$ implies that $f(x)=0$ for all $x$ or $f(x)=x^2$ which are solutions.\end{tcolorbox}

Wrong : $x=-y$ implies that for all $x$ : $f(x)=0$  or $f(x)=x^2$, which is quite different from what you wrote.
\end{solution}



\begin{solution}[by \href{https://artofproblemsolving.com/community/user/64716}{mavropnevma}]
	\begin{tcolorbox}Suppose now that $\exists a\ne 0, b\ne 0$ such that $f(a)=a^2$ and $f(b)=0$
$P(a,b)$ $\implies$ $f(a+b)f(a-b)=a^4$\end{tcolorbox}
We can continue in a much simpler way. We also have $P(b,a)$ $\implies$ $f(b+a)f(b-a)=a^4 - 4b^2a^2$.

But $f$ is an even function (*) in both cases, so $a^4 = a^4 - 4b^2a^2$, whence $ab = 0$, absurd.

(*) Take $b=0$ in the above relation(s) (we can do this, since $f(0) = 0$); so $f(a) = a^2$, but from the second, $f(a)f(-a) = a^4$, therefore also $f(-a) = a^2$. This means that the set $A$ of the numbers $a$ with $f(a) = a^2$ is such that $A = -A$, so also the set $B$ of the numbers $b$ with $f(b) = 0$ is such that $B = -B$, and my assertion is true.
Even simpler, for $x=0$ (and knowing $f(0)=0$), from the initial equation we get $f(y)f(-y) = f(y)^2$, so $f$ is even.
\end{solution}



\begin{solution}[by \href{https://artofproblemsolving.com/community/user/67107}{sankha012}]
	[CONTENT DELETED]
\end{solution}



\begin{solution}[by \href{https://artofproblemsolving.com/community/user/66411}{rationalist}]
	I would like to introduce the correction in the question posted.
Find all functions $f:\mathbb{R}\to \mathbb R$ be a functional equation satisfying
\[f(x+y)f(x-y)=\left(f(x)+f(y)\right)^2-4x^2f(y),\]
For all $x,y\in\mathbb R,$ where $\mathbb R$ denotes the set of all real numbers.
[color=#FF0000][moderator edit: thank you, edited. :)][\/color]
I did something like SCP but i got
$f(x)f(x)= x^2f(x)$. where x for all x $\in$ reals. as i wrote it in the examination.
now we have two cases f(x) = 0 forms a function satisgying the given functional equation and f(x) = x^2 forms another function. 
One of the biggest statement which can cause wrong conclusion seems to be from $f(x)f(x)= x^2f(x)$.
As per respected 'pco' i forgot to prove the second part as well. :(  :mad:
\end{solution}



\begin{solution}[by \href{https://artofproblemsolving.com/community/user/99076}{RSM}]
	I want to add one more condition with the functional equation problem that x is not equal to y.
Here is its solution:-

Denote the equation $ f(x+y)f(x-y)=(f(x)+f(y))^2-4x^2f(y) $ as $(1)$.
Exchanging x and y, this leads to
$ f(x+y)f(y-x)=(f(x)+f(y))^2-4y^2f(x) \ \ \ \ \ (2) $

Putting y=0 in (1), we have
$ f(x)^2=(f(x)+c)^2-4cx^2 \ \ \text{where} \ \ c=f(0); \ \ \ \ (3) $

Or,
$ f(x)f(-x)=(f(x)+c)^2 \ \ \ \ (4) $

Subtracting $ (3) $ from $ (4) $, we have
$ f(x)\{f(-x)-f(x)\}=4cx^2 \ \ \ \ \ (5); $

And subtracting $ (1) $ from $ (2) $, we have
$ f(x+y)[f(x-y)-f(y-x)]=4y^2f(x)-4x^2f(y). $

Applying $ (5) $ here and putting $ y=0 $ and after simplifying this we will get $ c=0 $ or f is odd.

Case I: c=0.
In this case, by (4) we get f(x)=0 or f is even (this means f is even for all x).

Comparing (1) and (2) we have $ \frac{f(a)}{a^2}=\frac{f(b)}{b^2}. $

Since f is even, this holds for every two elements of $ \mathbb R $ except $ 0 $ .

So $ f(x)=mx^2 \forall x\in\mathbb R. $

Substituting this in (1), we get m=0 or 1.

Case II: f is odd.
Substituting this in $ (5) $ we obtain

$ f(x)=nx $ for some constant n

Now, it is easy to check that this is inconsistent with the given conditions.


Thanks Potla for Latexifying this.
\end{solution}



\begin{solution}[by \href{https://artofproblemsolving.com/community/user/67107}{sankha012}]
	After getting $f(x)=0$ or $f(x)=x^2$ and omitting the solution $f(x)=0$ for all real $x$.It suffices to prove that $f(x)\neq 0$ if $x\neq 0$.Let $y$ be a non-zero real such that $f(y)=0$ and $x$ be a non-zero real such that $f(x)=x^2$.
These imply,$f(x+y)f(x-y)=(f(x))^2>0$(*).So both factors of the RHS are non-zero.so $f(x\pm y)=(x\pm y)^2$.Substituting this,
(*) reduces to $y^2=2x^2$.But this makes the number of $y$'s and $x$'s finite.Contradiction.
\end{solution}



\begin{solution}[by \href{https://artofproblemsolving.com/community/user/97134}{jgthegreat}]
	\begin{tcolorbox}[quote="Potla"]Let $f:\mathbb{R}\to \mathbb R$ be a functional equation satisfying
\[f(x+y)f(x-y)=\left(f(x)+f(y)\right)^2-4x^2f(y),\]
For all $x,y\in\mathbb R,$ where $\mathbb R$ denotes the set of all real numbers.\end{tcolorbox}

$y=0\Longrightarrow f(0)=0$ or $f(x)=\frac{4x^2-f(0)}{2}$. Suppose $f(x)=\frac{4x^2-f(0)}{2}$. $x=0$ implies that $f(0)=0\Longrightarrow f(x)=2x^2$ which when substituted back proves that it is not a solution.(I forgot to substitute it back in the examination  :( ) Now, $f(0)=0$. $x=0\Longrightarrow f(x)=0$ for all $x$ or $f$ is even. If $f$ is even, $x=-y$ implies that $f(x)=0$ for all $x$ or $f(x)=x^2$ which are solutions.\end{tcolorbox}

Sorry Goutham.
Sai Mali, Anshul and me discussed the problem and found f(x)=x^2 and not 0.
\end{solution}



\begin{solution}[by \href{https://artofproblemsolving.com/community/user/87195}{SCP}]
	\begin{tcolorbox}[quote="Goutham"][quote="Potla"]Let $f:\mathbb{R}\to \mathbb R$ be a functional equation satisfying
\[f(x+y)f(x-y)=\left(f(x)+f(y)\right)^2-4x^2f(y),\]
For all $x,y\in\mathbb R,$ where $\mathbb R$ denotes the set of all real numbers.\end{tcolorbox}

$y=0\Longrightarrow f(0)=0$ or $f(x)=\frac{4x^2-f(0)}{2}$. Suppose $f(x)=\frac{4x^2-f(0)}{2}$. $x=0$ implies that $f(0)=0\Longrightarrow f(x)=2x^2$ which when substituted back proves that it is not a solution.(I forgot to substitute it back in the examination  :( ) Now, $f(0)=0$. $x=0\Longrightarrow f(x)=0$ for all $x$ or $f$ is even. If $f$ is even, $x=-y$ implies that $f(x)=0$ for all $x$ or $f(x)=x^2$ which are solutions.\end{tcolorbox}

Sorry Goutham.
Sai Mali, Anshul and me discussed the problem and found f(x)=x^2 and not 0.\end{tcolorbox}
$f=0$ is of course a solution also.
\end{solution}



\begin{solution}[by \href{https://artofproblemsolving.com/community/user/72235}{Goutham}]
	\begin{tcolorbox}Sorry Goutham.
Sai Mali, Anshul and me discussed the problem and found f(x)=x^2 and not 0.\end{tcolorbox}
Maybe you missed something. Because it is obvious from the first glance that $f(x)=0$ for all $x$ is a solution.
\end{solution}



\begin{solution}[by \href{https://artofproblemsolving.com/community/user/89144}{sumanguha}]
	Let put x=y to get 
$f(2x)f(0)=4f(x)^2-4x^2f(x)$
now set x=0 to see
$f(0)^2=4f(0)^2$
so, $f(0)=0$

so put it back to get

$ 0=4f(x)^2-4x^2f(x)$
$f(x)[f(x)-x^2]=0$

so, $ f(x)=0 $ or $ x^2$

now let there exist $x_{1}\neq 0, x_{2}\neq 0$ s.t. $f(x_{1})=0,f(x_{2})\neq 0$
i.e. $f(x_{1})=0,f(x_{2})=x_{2}^2$

then set $y=x_{1},x=x_{2}$ to get

$f(x_{2}+x_{1})f(x_{2}-x_{1})=x_{2}^4$

since $x_{2}\neq 0$ so,$f(x_{2}+x_{1})\neq 0, f(x_{2}-x_{1})\neq 0$

 so,$f(x_{2}+x_{1})=(x_{2}+x_{1})^2, f(x_{2}-x_{1})=(x_{2}-x_{1})^2$

so, $(x_{2}^2-x_{1}^2)^2=x_{2}^4$

or, $x_{1}^2(x_{1}^2+2x_{2}^2)=0$

contradiction as $x_{1}\neq 0, x_{2}\neq 0$ .

so, either $f(x)=0$ for all $x\neq 0$ or $f(x)\neq0$ for all $x\neq 0$ i.e.

$f(x)=0$ for $x\neq 0$ or $f(x)=x^2$ for $x\neq 0$ .

Now see this are indeed solutions.

so only possible solutions are

$f(x)=x^2$ and $f(x)=0$
\end{solution}



\begin{solution}[by \href{https://artofproblemsolving.com/community/user/33910}{CDP100}]
	\begin{tcolorbox}
or, $x_{1}^2(x_{1}^2+2x_{2}^2)=0$
\end{tcolorbox}

I think it must be $x_{1}^2(x_{1}^2-2x_{2}^2)=0$!
\end{solution}



\begin{solution}[by \href{https://artofproblemsolving.com/community/user/89144}{sumanguha}]
	\begin{tcolorbox} I think it must be $ x_{1}^{2}(x_{1}^{2}-2x_{2}^{2})=0 $!\end{tcolorbox}

Yes, you are right.

and then a modified arguement would be.

let there is $ x_{1}\neq 0, x_{2}\neq 0 $ s.t. $ f(x_{1})=0,f(x_{2})\neq 0 $

Then by corrected arguement $x_{1}=\pm\sqrt{2} x_{2}$

Then we get either infinitely many  (1) $z_{i} \neq 0$ s.t. $f_(z_{i})=0$ 

                         or infinitely many (2) $z_{i} \neq 0$ s.t. $f_(z_{i}) \neq 0$

if (1) is true then $x_{2}=\pm\frac{1}{\sqrt{2}}z_{i}$ for all $i$ 

contradiction since $z_{i}$ are distinct.

if (2) is true then $x_{1}=\pm\sqrt{2}z_{i}$ for all $i$ 

contradiction since $z_{i}$ are distinct.

so, contradiction.
\end{solution}



\begin{solution}[by \href{https://artofproblemsolving.com/community/user/204311}{Onlygodcanjudgeme}]
	$ f(x+y) \cdot f(x-y) = [f(x) + f(y)]^2 - 4 \cdot x^2 \cdot f(y)   $
  $ (x,y) = (0,0) $ then we take that $ f(0) =0 $
  $ (x.y) = (x,x) $ then we take that $ f(x) =0 $ or  $ f(x) = x^2 $
\end{solution}



\begin{solution}[by \href{https://artofproblemsolving.com/community/user/29428}{pco}]
	\begin{tcolorbox}...
  $ (x.y) = (x,x) $ then we take that $ f(x) =0 $ or  $ f(x) = x^2 $\end{tcolorbox}
No, you should read the entire thread, before posting exactly the same error than above :

 $ (x.y) = (x,x) $ implies $f(x)(f(x)-x^2)=0$ and so : "$\forall x$, either $f(x)=0$, either $f(x)=x^2$"

Which is quite different from your own conclusion ("either $f(x)=0$ $\forall x$, either $f(x)=x^2$ $\forall x$")
For example the fonction $f(x)=\left(\frac{x+|x|}2\right)^2$ is such that $f(x)(f(x)-x^2)=0$ $\forall x$

So you miss some work again before conclusion.
\end{solution}



\begin{solution}[by \href{https://artofproblemsolving.com/community/user/213285}{Kezer}]
	Just skimmed the thread, seems like my solution hasn't been posted yet.

Set $x=y=0$, that yields $f(0)=0$. Now set $x=0$, that gives \[ f(y) f(-y) = f(y)^2 \qquad \iff \qquad f(y) \left(f(y)-f(-y) \right) = 0. \] So $f(y)=0$ or $f(y)=f(-y)$. Let $r \in \mathbb{R}$ be some number for which $f(r)=0$, assuming such a number exists, for $f(-r)$ we can either have $f(-r)=0$ or $f(-r)=f(r)=0$, in both cases $f(r)=f(-r)$. So for each $x \in \mathbb{R}$ we have $f(x)=f(-x)$. If such a $r$ as above doesn't exist, the conclusion is immediate.
Now exploit the symmetry. As $f(x-y)=f(y-x)$ with $f(x)=f(-x)$ we can see \[ \left(f(x)+f(y) \right)^2-4x^2f(y) = f(x+y)f(x-y)=f(y+x)f(y-x) = \left(f(y)+f(x) \right)^2 - 4y^2f(x). \] Thus, we have \[ 4x^2f(y) = 4y^2f(x) \qquad \iff \qquad x^2f(y) = y^2f(x). \] Set $y=1$ to get $f(x) = x^2 f(1)$, hence $f(x)=cx^2$ for some constant $c \in \mathbb{R}$. Putting that into the equation yields $c=0$ or $c=1$ and $f(x)=0$ and $f(x)=x^2$ are indeed solutions.
\end{solution}
*******************************************************************************
-------------------------------------------------------------------------------

\begin{problem}[Posted by \href{https://artofproblemsolving.com/community/user/87506}{mattermatters}]
	Let $f: \mathbb{N} \to \mathbb{N}$ be a strictly increasing function such that $f(f(n))=3n$ for all natural numbers $n$. Find $f(2001)$.
	\flushright \href{https://artofproblemsolving.com/community/c6h390679}{(Link to AoPS)}
\end{problem}



\begin{solution}[by \href{https://artofproblemsolving.com/community/user/29428}{pco}]
	\begin{tcolorbox}Let F: N--->N be a strictly increasing function such that f(f(n))=3n for all natural numbers n. Find f(2001).\end{tcolorbox}
Notice that $f(f(n))=3n$ $\implies$ $f(3n)=3f(n)$

Let $a(n)=f(n+1)-f(n)>0$

Since strictly increasing, $f(n)\ge n$, so $f(f(1))=3\ge f(1)$ and so $f(1)\in\{1,2,3\}$
$f(1)=1$ $\implies$ $f(f(1))=f(1)=1$, impossible
$f(1)=3$ $\implies$ $3=f(f(1))=f(3)$ and $f(1)=f(3)=3$, impossible
So $f(1)=2$ and so $f(2)=f(f(1))=3$and $f(3)=3f(1)=6$
So $a(1)=1$ and $a(2)=3$

$f(f(n))=3n$
$f(f(n)+a(n))=f(f(n+1))=3n+3$
And since there is at most two values between $3n$ and $3n+3$ and $f(x)$ is strictly increasing, then $\boxed{a(n)\in\{1,2,3\}}$

Suppose now that exists $n$ such that $a(n)=2$ and let $u$ be the smallest positive integer such that $a(u)=2$
Let $m$ such that $3m\le u<3m+3$
Since $a(1)=1$ and $a(2)=3$, we get $m>0$
$f(3m)=3f(m)$ and $f(3m+3)=3f(m+1)=3f(m)+3a(m)=3f(m)+3\/6\/9$
But since $f(3m+3)=f(3m)+a(3m)+a(3m+1)+a(3m+2)$ and one of the three quantities $a(3m),a(3m+1),a(3m+2)$ is $a(u)=2$, the only possibility is $f(3m+3)=3f(m)+6$
And so $f(m+1)=f(m)+2$ with $0<m<3m\le u$ and so contradiction since $u$ is the smallest such integer.
So $\boxed{a(n)\in\{1,3\}}$

$f(n+1)=f(n)+a(n)$ $\implies$ $f(3n+3)=3f(n+1)=3f(n)+3a(n)$
So we have $f(3n)=3f(n)$ and $f(3n+3)=3f(n)+3a(n)=3f(n)+a(3n)+a(3n+1)+a(3n+2)$
So $3a(n)=a(3n)+a(3n+1)+a(3n+2)$ and so, since $a(n)\in\{1,3\}$ : $\boxed{a(3n+2)=a(3n+1)=a(3n)=a(n)}$
And so $a(n)=a($leftmost digit of base$_3$ representation of $n)$

So $a(n)=1$ $\forall n\in[3^p,2\times 3^p)$ and $a(n)=3$ $\forall n\in[2\times 3^p,3^{p+1})$

We also have $f(3^p)=3^pf(1)=2\times 3^p$

And so :
$\forall x\in[3^p,2\times 3^p)$ : $f(x)=2\times 3^p+(x-3^p)=x+3^p$
$\forall x\in[2\times 3^p,3^{p+1})$ : $f(x)=f(2\times 3^p)+3(x-2\times 3^p)$ $=f(3^p)+3^p+3(x-2\times 3^p)$ $=3x-3^{p+1}$

And so, \begin{bolded}in order to get f(x) :\end{bolded}\end{underlined}
If base3 leftmost digit of $x$ is $1$, just move it to $2$
If base3 leftmost digit of $x$ is $2$, move it to $1$ and add a zero digit to the right.
and it's easy to check that this indeed is a solution

So $f(2001)=f(2202010_3)=12020100_3=\boxed{3816}$
\end{solution}



\begin{solution}[by \href{https://artofproblemsolving.com/community/user/72819}{Dijkschneier}]
	Here is my solution :
[hide]
1) $f(f(n))=3n \implies 3f(n)=f(f(f(n)))=f(3n) \implies f(3^k n)=3^k f(n)$, $\forall n,k \in \mathbb{N}$
2) Because f is strictly increasing, then $f(n) \geq n, \forall n\in \mathbb{N}$
3) Suppose that for a certain n, we have f(n)=n. Then $f(f(n))=3n \implies n=3n$, which is impossible
Hence : $f(n)>=n+1, \forall n \in \mathbb{N}$
4) $3n=f(f(n))\geq f(n)+1 \implies 3n-1 \geq f(n)$
For n=1, we get $f(1)\leq 2$, which, together with $f(1) \geq 2$, gives f(1)=2
Hence, f(f(1))=3, that is, f(2)=3
5) Using -1-, we get $f(3^n)=3^nf(1)=2\cdot 3^n$ and $f(2\cdot 3^n) = 3^n f(2) = 3^{n+1}$, $\forall n \in \mathbb{N}$
And because there are $3^n$ integers between $3^n$ and $2 \cdot 3^n$, and $3^n$ integers between $2\cdot 3^n$ and $3^{n+1}$, and f is strictly increasing, then : $f(m)=2\cdot 3^n + (m - 3^n) = 3^n + m, \forall n \in \mathbb{N}, \forall m \in \{3^n, 3^n+1, 3^n+2, ..., 2 \cdot 3^n\}$
6) $\forall  m \in \{2 \cdot 3^n,  2 \cdot 3^n+1, 2 \cdot 3^n+2, ...,3^{n+1}\}$, we have $m-3^n \in \{3^n, 3^n+1, 3^n+2, ..., 2 \cdot 3^n\}$, and so by -5-, we get $f(m-3^n)=3^n + (m-3^n)=m$, hence $3(m-3^n)=f(f(m-3^n))=f(m)$
So :  $\forall  m \in \{2 \cdot 3^n,  2 \cdot 3^n+1, 2 \cdot 3^n+2, ..., 3^{n+1}\}$, $f(m)=3(m-3^n)$
Now f is defined for every positive integer, so it is easy to conclude.
[\/hide]
\end{solution}
*******************************************************************************
-------------------------------------------------------------------------------

\begin{problem}[Posted by \href{https://artofproblemsolving.com/community/user/73386}{mousavi}]
	Find all injective functions $f: \mathbb N\to \mathbb R$ such that$f(1)=2,f(2)=4$, and
\[f(f(m)+f(n))=f(f(m))+f(n)\]
for all positive integers $n$.
	\flushright \href{https://artofproblemsolving.com/community/c6h390689}{(Link to AoPS)}
\end{problem}



\begin{solution}[by \href{https://artofproblemsolving.com/community/user/77832}{abhinavzandubalm}]
	\begin{tcolorbox}find all  injective functions $f:N\rightarrow R$ such that:
$f(1)=2,f(2)=4,f(f(m)+f(n))=f(f(m))+f(n),  (m,n\in N)$\end{tcolorbox}
We Can Easily Notice By Replacing $m$ And $n$ .
That 
$f(f(m))-f(m)=c$
Putting $m=1$
We get $c=2$
Hence We Have $f(1)=2,f(2)=4,f(4)=6,f(6)=8,........f(2n)=2n+2~~\forall n\ge1$
If Someone Could Give The Proof For Other Numbers That 
$f(2n+1)=2n+2~~\forall n\ge 0$ 
Or Any Other Solution I Would Be Obliged.
\end{solution}



\begin{solution}[by \href{https://artofproblemsolving.com/community/user/73386}{mousavi}]
	\begin{tcolorbox}[quote="mousavi"]find all  injective functions $f:N\rightarrow R$ such that:
$f(1)=2,f(2)=4,f(f(m)+f(n))=f(f(m))+f(n),  (m,n\in N)$\end{tcolorbox}
We Can Easily Notice By Replacing $m$ And $n$ .
That 
$f(f(m))-f(m)=c$
Putting $m=1$
We get $c=2$
Hence We Have $f(1)=2,f(2)=4,f(4)=6,f(6)=8,........f(2n)=2n+2~~\forall n\ge1$
If Someone Could Give The Proof For Other Numbers That 
$[color=\#BF0000]f(2n+1)=2n+2[\/color]~~\forall n\ge 0$ 
Or Any Other Solution I Would Be Obliged.\end{tcolorbox}
$f$ is injective is it possible $f(2n)=2n+2,f(2n+1)=2n+2$??!!!!
\end{solution}



\begin{solution}[by \href{https://artofproblemsolving.com/community/user/29428}{pco}]
	\begin{tcolorbox}find all  injective functions $f:N\rightarrow R$ such that:
$f(1)=2,f(2)=4,f(f(m)+f(n))=f(f(m))+f(n),  (m,n\in N)$\end{tcolorbox}
Let $P(x,y)$ be the assertion  $f(f(x)+f(y))=f(f(x))+f(y)$

$P(x,1)$ $\implies$ $f(f(x)+2)=f(f(x))+2$
$P(1,x)$ $\implies$ $f(f(x)+2)=4+f(x)$
And so $f(f(x))=f(x)+2$ $\forall x$

From this, ,we get that if $f(n)=n+2$, we also have $f(n+2)=n+4$ and, since $f(2)=4$, we get thru immediate induction $f(2n)=2n+2$ $\forall n\in\mathbb N$

So all even numbers are taken as images of $1$ or even numbers. So $f(2n+1)$ is odd for any $n>0$
Since $f(f(n))=f(n)+2$, the set $A=\{x>1$ odd integer such that $f(x)=x+2\}$ is not empty and let $a=\min(A)$

If $a>3$, we know that $\forall$ odd $x\in[3,a)$, $f(x)\in A$ and so $f(x)\ge a$
But if $f(x)\ge a+2$, then we have $f(x)-2\in A$ and $f(f(x)-2)=f(x)$, impossible.
So the only possibility could be $f(x)=a$ and, since injective, $a=5$ (else $[3,a)$ would contain at least two odd integers which both have same image $a$.
But $a=5$ implies $f(3)=5$, impossible.

So $a=3$

And the function is $f(1)=2$ and $f(n)=n+2$ $\forall n>1$ and it is easy to check that this indeed is a solution.
\end{solution}



\begin{solution}[by \href{https://artofproblemsolving.com/community/user/77832}{abhinavzandubalm}]
	\begin{tcolorbox}[quote="mousavi"]find all  injective functions $f:N\rightarrow R$ such that:
$f(1)=2,f(2)=4,f(f(m)+f(n))=f(f(m))+f(n),  (m,n\in N)$\end{tcolorbox}
We Can Easily Notice By Replacing $m$ And $n$ .
That 
$f(f(m))-f(m)=c$
Putting $m=1$
We get $c=2$
Hence We Have $f(1)=2,f(2)=4,f(4)=6,f(6)=8,........f(2n)=2n+2~~\forall n\ge1$
If Someone Could Give The Proof For Other Numbers That 
$f(2n+1)=2n+2~~\forall n\ge 0$ 
Or Any Other Solution I Would Be Obliged.\end{tcolorbox}
As An Easier Solution We Can Assume That As $f(f(m))$ Is Defined For All $m$, Therefore $f(m)\in N$
Therefore The Function Is $f:N \to N$.
I Have Already Proved That $f(2n)=2n+2$
Therefore $2n+2<f(2n+1)<2n+4$
Therefore $f(2n+1)=2n+3$.
Hence $f(1)=2,f(m)=m+2~\forall~m>1\in~N$
\end{solution}



\begin{solution}[by \href{https://artofproblemsolving.com/community/user/29428}{pco}]
	\begin{tcolorbox} [I Have Already Proved That $f(2n)=2n+2$
Therefore $2n+2<f(2n+1)<2n+4$\end{tcolorbox}

Why ?
Function is injective does not mean it is monotonous
\end{solution}
*******************************************************************************
-------------------------------------------------------------------------------

\begin{problem}[Posted by \href{https://artofproblemsolving.com/community/user/72235}{Goutham}]
	Let $f(x)$ be a function such that every straight line has the same number of intersection points with the graph $y = f(x)$ and with the graph $y = x^2$. Prove that $f(x) = x^2.$
	\flushright \href{https://artofproblemsolving.com/community/c6h390697}{(Link to AoPS)}
\end{problem}



\begin{solution}[by \href{https://artofproblemsolving.com/community/user/29428}{pco}]
	\begin{tcolorbox}Let $f(x)$ be a function such that every straight line has the same number of intersection points with the graph $y = f(x)$ and with the graph $y = x^2$. Prove that $f(x) = x^2.$\end{tcolorbox}
If $f(a)<a^2$ for some $a$, it's easy to build a straight line from $(a,f(a))$ not encountering $y=x^2$ (choose for example the parallel to tangent at $a$ to parabola), and so contradiction.

So $f(a)\ge a^2$ and graph of $f(x)$ is "above" parabola $y=x^2$

If $f(a)>a^2$ for some $a$, the straight line which is tangent at the parabola at $(a,a^2)$ encounters the parabola in one point and is "below" the parabola anywhere else and so never encounters $y=f(x)$, and so contradiction.

So $f(a)\le a^2$

So $f(x)=x^2$ $\forall x$
\end{solution}
*******************************************************************************
-------------------------------------------------------------------------------

\begin{problem}[Posted by \href{https://artofproblemsolving.com/community/user/3182}{Kunihiko_Chikaya}]
	Find all functions $ f : \mathbb{R} \to\mathbb{R}$ such that $f(f(x)-f(y))=f(f(x))-2x^2f(y)+f(y^2)$ for all $x,y \in \mathbb{R}.$
	\flushright \href{https://artofproblemsolving.com/community/c6h391060}{(Link to AoPS)}
\end{problem}



\begin{solution}[by \href{https://artofproblemsolving.com/community/user/29428}{pco}]
	\begin{tcolorbox}Find all functions $ f : \mathbb{R} \mapsto \mathbb{R}$ such that $f(f(x)-f(y))=f(f(x))-2x^2f(y)+f(y^2)$ for all $ \ x,y \in \mathbb{R}.$\end{tcolorbox}
Let $P(x,y)$ be the assertion $f(f(x)-f(y))=f(f(x))-2x^2f(y)+f(y^2)$

Let $A=\{x$ such that $f(x)=0\}$
$P(0,0)$ $\implies$ $f(f(0))=0$ and so $f(0)\in A$
If $x\in A$, then $P(x,x)$ $\implies$ $f(x^2)=0$ and so $x^2\in A$

Let $a\in A$. Then $P(a,x)$ $\implies$ $f(-f(x))=f(0)-2a^2f(x)+f(x^2)$
So, if $a\ne 0$, and setting $a,a^2,a^4,...$ in the above equation, we get $f(x)=0$ $\forall x$ which indeed is a solution.

So we'll from now consider $A=\{0\}$ and so $f(x)=0$ $\iff$ $x=0$

$P(x,x)$ $\implies$ $f(f(x))=2x^2f(x)-f(x^2)$
Plugging this in original equation, we get 
new assertion $Q(x,y)$ : $f(f(x)-f(y))=2x^2(f(x)-f(y))+f(y^2)-f(x^2)$

Looking at $Q(x,y)$, we get that $f(x)=f(y)$ $\implies$ $f(x^2)=f(y^2)$
Looking then at $f(f(x))=2x^2f(x)-f(x^2)$, we get that $f(x)=f(y)$ $\implies$ $x^2=y^2$ and so $y=\pm x$

$P(0,x)$ $\implies$ $f(-f(x))=f(x^2)$ and, using previous line : $f(x)=\pm x^2$

And so : $\forall x$ : either $f(x)=x^2$, either $f(x)=-x^2$

Suppose now that $\exists a\ne 0$ such that $f(a)=a^2$ and $b\ne 0$ such that $f(b)=-b^2$
$P(a,a)$ $\implies$ $f(a^2)=a^4$
$P(b,b)$ $\implies$ $f(-b^2)=-2b^4-f(b^2)$ and so $f(b^2)=f(-b^2)=-b^4$
$P(a,b)$ $\implies$ $f(a^2+b^2)=a^4+2a^2b^2-b^4$
And since either $f(a^2+b^2)=(a^2+b^2)^2$, either $f(a^2+b^2)=-(a^2+b^2)^2$, we get $a=0$ or $b=0$, impossible.

And so either $f(x)=x^2$ $\forall x$, either $f(x)=-x^2$ $\forall x$ which both are solutions.

\begin{bolded}Hence the answer \end{bolded}\end{underlined}:
$f(x)=0$ $\forall x$
$f(x)=x^2$ $\forall x$
$f(x)=-x^20$ $\forall x$
\end{solution}



\begin{solution}[by \href{https://artofproblemsolving.com/community/user/29386}{mszew}]
	\begin{tcolorbox}So, if $a\ne 0$, and setting $a,a^2,a^4,...$ in the above equation, we get $f(x)=0$ $\forall x$ \end{tcolorbox}
Why is that? What happened with $a=1$?
\end{solution}



\begin{solution}[by \href{https://artofproblemsolving.com/community/user/29428}{pco}]
	\begin{tcolorbox}[quote="pco"]So, if $a\ne 0$, and setting $a,a^2,a^4,...$ in the above equation, we get $f(x)=0$ $\forall x$ \end{tcolorbox}
Why is that? What happened with $a=1$?\end{tcolorbox}
Hummmmf, I'm a bit too tired.
You're right.
I'll try to modify my proof :oops:
\end{solution}



\begin{solution}[by \href{https://artofproblemsolving.com/community/user/48552}{ocha}]
	Here's the minor patch

As pco showed, $|A| =1$, else $(f\equiv 0 \forall x)$. So if $f(1)=0$ then $f(0)\neq 0$. But $P(0,0) \Rightarrow f(f(0))=0$, so $f(0)=1$. Then $P(1,0) \Rightarrow f(-f(0))=0$. So $f(0)=-1$ contradiction. So $f(x)=0 \Leftrightarrow x=0$
\end{solution}



\begin{solution}[by \href{https://artofproblemsolving.com/community/user/95461}{TheIronChancellor}]
	I found the same 3 solutions as patrick .

To make things easier, prove that f is injective.
\end{solution}



\begin{solution}[by \href{https://artofproblemsolving.com/community/user/64868}{mahanmath}]
	\begin{tcolorbox}I found the same 3 solutions as patrick .

To make things easier, prove that f is injective.\end{tcolorbox}

But $x^2 -1$ is also a solution !!

If I would be able to check the case when $f(1)=f(-1) = 1$ I`ll complet my 4 pages proof !!
\end{solution}



\begin{solution}[by \href{https://artofproblemsolving.com/community/user/95461}{TheIronChancellor}]
	At last, f is not injective?

If I put x^2 -1 in the first equation I get f(0) = -1 , which means what?
\end{solution}



\begin{solution}[by \href{https://artofproblemsolving.com/community/user/90062}{vntbqpqh234}]
	f(x)=0 is a root. if f(x) not equal 0 then
We have the  notes that
 If $f(x)=f(z)$ then $x^{2}=z^{2}$(*)
With $x=y=0$ we have $f(f(0))=0$
hence with $y=f(0)$ then $f([f(0)]^{2})=0$
all for me $f(f(0))=f([f(0)]^{2})$ hence $[f(0)]^{2}=[f(0)]^{4}$
then $f(0)=0,-1,1$
-First $f(0)=0$ then with $x=0$ we have $f(-f(y))=f(y^{2})$
have (*) for us $f[(y)]^{2}=y^{4}$
we have f(x)=$x^{2}$ or $-x^{2}$
-Second, $f(0)=1$ then $f(1)=0$ similar we have $f(1-f(y))=f(y^{2}$ and $f(0)=1$ all then
 $f(x)=1-x^{2}$
-Third,$ f(0)=-1$ similar have $f(x)=x^{2}-1$
\end{solution}



\begin{solution}[by \href{https://artofproblemsolving.com/community/user/48552}{ocha}]
	Let $P(x,y)$ be the proposition. $f\equiv 0$ is a solution, so from now on assume that $f$ is not identically zero.

\begin{bolded}Step 1 \end{bolded} 
let $A = \{u \, : \, f(u) = 0\}$. If $0 \neq a \in A$ then
$P(0,a) \Rightarrow f(f(0))=f(f(0))+f(a^2) \Rightarrow a^2 \in A$
$P(a,x) \Rightarrow f(-f(x))=f(0)-2a^2f(x) + f(x^2) \forall a \in A$

The second equation imples that the set $A^2 = \{a^2 \, : \, a \in A\}$ has at most one element. So $A=\{0\}$ or $A\subseteq \{-1,1\}$.

\begin{bolded}Step 2\end{bolded}
$P(0,0) \Rightarrow  f(f(0))=0$
$P(0,y) \Rightarrow  f(f(0)-f(y)) = f(y^2)$
$P(f(0)-f(y),x) \Rightarrow f(f(y^2)-f(x)) = f(f(y^2)) - 2(f(0)-f(y))^2f(x) + f(x^2)$
$P(y^2,x) \Longrightarrow f(f(y^2)-f(x)) = f(f(y^2)) - 2y^4 f(x) + f(x^2)$

Subtracting the last two equations gives $(f(y)-f(0))^2 = y^4$, hence $f(y) = \epsilon(y)y^2 + f(0)$, where $\epsilon(y) \in \{-1,0,1\}$ for all $y$ and $f(0)\in \{-1,0,1\}$ (because $f(f(0))=0$)

\begin{bolded}Step 3\end{bolded} 
Let's look at $\epsilon(x)$. On the LHS of $P(x,y)$ we have $f(f(x)-f(y))$ which becomes $\pm (\pm x^2 \pm y^2)^2 +c$. Regardless of how we choose the $\pm$ signs we always have the coefficient of $x^4$ and $y^4$ equal. Now on the RHS, the $x^4$ coefficient comes from $f(f(x))$ while the $y^4$ coefficient comes from $f(y^2)$. If $\epsilon(f(f(x)))\neq \epsilon (f(y^2))$, then the LHS and RHS are not equivalent so $(x,y)$ must be the solution to a two variable quadratic. Then if we fix $x$ and vary $y$, we can only have $\epsilon(f(f(x)))\neq \epsilon (f(y^2))$ for at most two values of $y$ (because a quadratic has at most two roots). But since we can also vary $x$, we find that $\epsilon(y^2)$ is constant for all $y\in \mathbb{R}$. Similarly, to have equality on the LHS and RHS we must also have $\epsilon(y^2) = \epsilon(f(x)-f(y))$, and $f(x)-f(y)$ can take any real value, because if $f(0)-f(y)=\pm y^2$ then $f(y)-f(0)=\mp y^2$. So $\epsilon(x)$ is constant for all $x\in \mathbb{R}$.

\begin{bolded}Step 4\end{bolded}
It is left to check a finite number of solutions: $f(x)=\pm x + c$ where $c\in \{-1,0,1\}$. From which we find the only solutions are $f(x)=0,x^2,-x^2,x^2-1$ and $-x^2+1$.
\end{solution}
*******************************************************************************
-------------------------------------------------------------------------------

\begin{problem}[Posted by \href{https://artofproblemsolving.com/community/user/93909}{magical}]
	Find all polynomials $f(x) \in \mathbb{R}[x] $ such that 
\[f(x)f(2x^2)=f(2x^3+x^2), \quad \forall  x\in \mathbb{R}.\]
	\flushright \href{https://artofproblemsolving.com/community/c6h391333}{(Link to AoPS)}
\end{problem}



\begin{solution}[by \href{https://artofproblemsolving.com/community/user/29428}{pco}]
	\begin{tcolorbox}Find all polynomial $f(x) \in \mathbb{R}[x] $ satisfy: $f(x)f(2x^2)=f(2x^3+x^2)$ $\forall x\in \mathbb{R}$\end{tcolorbox}
Constant polynomials which are solutions are $f(x)=0$ and $f(x)=1$

If $f(x)$ is a non constant polynomial, let $ax^p$ with $a\ne 0$ be the smallest degree summand of $f(x)$ :
Smallest degree summand in LHS is $2^pa^2x^{3p}$
Smallest degree summand in RHS is $ax^{2p}$
And so $p=0$ and $a=1$

Let then $bx^q$ with $b\ne 0$ and $q>0$ the smallest degree summand of $f(x)-1$ so that $f(x)$ ends with $bx^q+1$

The two smallest degree summands of LHS are $bx^q+1$
The two smallest degree summands of RHS are $bx^{2q}+1$
And so contradiction since $b\ne 0$ and $q>0$

Hence the only two solutions :
$f(x)=0$ $\forall x$
$f(x)=1$ $\forall x$
\end{solution}



\begin{solution}[by \href{https://artofproblemsolving.com/community/user/175682}{nkalosidhs}]
	\begin{tcolorbox}Find all polynomial $f(x) \in \mathbb{R}[x] $ such that 
\[f(x)f(2x^2)=f(2x^3+x^2) \quad \forall  x\in \mathbb{R}.\]\end{tcolorbox}
I have got another solution, but I am not sure if it is correct.
Let suppose that $P(x)$ is a non-constant polynomial and $P(x)=f(x)$.
Let $degP(x)=n$.
For $x=0$ the equation becomes:
$P(0)^2=P(0)$
which means that either $P(0)=0$ either $P(0)=1$.
So,if $P(0)=0$ then:
Suppose that $P(x)=x^k*Q(x)$ where $k$ has got the maximum value and $degQ(x)=n-k$ and $Q(0)\neq0$
Then the equation becomes:
$2^kx^kQ(x)Q(2x^2)=(2x+1)Q(2x^3+x^2)$
Setting $x=0$ we obtain $Q(0)=0$ a contradiction!
This means that 0 is not a root of the Polynomial P.
Let $r$ be a root of $P$. Then $P(r)=0$.
Setting $x=r$ we get that $P(2r^3+r^2)=0$ which means that $2r^3+r^2$ is also a root of $P$.
But for $r>0$ we get that $2r^3+r^2>=r$, which means that there infinitely many zeros for $P$!
So we have that $2r^3+r^2=r$, which means that $2r^2+r-1=0$. So we have that $r=1\/2$ (this means that the only positive root is 1\/2)
We can write the polynomial P in this way:
$P(x)=(x-\frac{1}{2})*S(x)$ where $degS(x)=n-1$
We get from the equation:
$(x-1)(2x^2-1)S(x)S(2x^2)=(2x^3+x^2-2)S(2x^3+x^2)$
Setting $x=1$ we get that:
$S(3)=0$
a contradiction as 1\/2 is the only positive solution!
So $P(x)=c$ where $c=0$ or $1$
\end{solution}



\begin{solution}[by \href{https://artofproblemsolving.com/community/user/242048}{phaniraj}]
	why Q(0)  should not be equal to 0

\end{solution}
*******************************************************************************
-------------------------------------------------------------------------------

\begin{problem}[Posted by \href{https://artofproblemsolving.com/community/user/31067}{ridgers}]
	Given the function $f: \mathbb R \to \mathbb R$ such that $f(x+1)+f(x-1)=\sqrt{2}f(x)$ for all real $x$. Prove that $f$ is a periodic function.
	\flushright \href{https://artofproblemsolving.com/community/c6h391357}{(Link to AoPS)}
\end{problem}



\begin{solution}[by \href{https://artofproblemsolving.com/community/user/29428}{pco}]
	\begin{tcolorbox}Given the function $ f:R\to R $ such that $f(x+1)+f(x-1)=\sqrt{2}f(x)$, prove that is is a periodic function.\end{tcolorbox}
$f(x+1)=\sqrt 2f(x)-f(x-1)$
$f(x+2)=\sqrt 2f(x+1)-f(x)=f(x)-\sqrt 2f(x-1)$
$f(x+3)=\sqrt 2f(x+2)-f(x+1)=-f(x-1)$

And so $f(x+4)=-f(x)$ and $f(x+8)=f(x)$ $\forall x$
Q.E.D.
\end{solution}
*******************************************************************************
-------------------------------------------------------------------------------

\begin{problem}[Posted by \href{https://artofproblemsolving.com/community/user/55393}{makar}]
	Let $f(x)$ be a real valued continuous function such that \[f(x-y)=f(x)f(y)-f(a-x)f(a+y)\] for all $x,y \in \mathbb{R}$, where $a$ is a given constant and $f(0)=1$. Find all possible such $f(x)$.

 :)

[hide="My working:"]
Let $P(x,y):f(x-y)=f(x)f(y)-f(a-x)f(a+y)\ \forall\ x,\ y\ \in \mathbb{R}$

$P(0,0):f(0)=f(0)f(0)-f(a)f(a)\implies 1=1-f^2(a)\implies f(a)=0$

$P(0,x):f(0-x)=f(0)f(x)-f(a-0)f(a+x)\implies f(-x)=f(x)\implies f(x)$ is an even function.

$P(a,x):f(a-x)=f(a)f(x)-f(a-a)f(a+x)$

$\implies \boxed{f(a-x)=-f(a+x)}\implies f(2a-x)=-f(x)$

$\implies f(x)=-f(2a+x)\implies \boxed{f(x)=f(4a+x)}$

$\implies f(x)$ is a periodic function with one of the period $4|a|$

$P(x,-y):f(x+y)=f(x)f(-y)-f(a-x)f(a-y)=f(x)f(y)+f(a-x)f(a+y)$

$P(x,y)+P(x,-y):\boxed{\boxed{f(x+y)+f(x-y)=2f(x)f(y)}}$  

The final equation after lots of manipulation gives  
 
$f(x)=\{\begin{array}{cc}{\cos bx} & {-1\le f(x)\le1}\\ \\{\frac{\alpha^x+\alpha^{-x}}{2}}&{1\le f(x)}\\ \\0\end{array}$

Is it correct? While checking the parent equation, I got stuck due to constant $a$. Should i choose $b$ or $\alpha$ so that it satisfy parent equation? I am puzzled please help me.  :(  :wallbash:       

[hide="Actual problem asked by a mathlinker was as follows:"] 
A real valued function f(x) satisfies the functional equation f(x-y)=f(x)f(y)-f(a-x)f(a+y) where a is a given constant and f(0)=1, then find the value of f(2a-x),( not a specific numerical value).
[\/hide]
[\/hide]
	\flushright \href{https://artofproblemsolving.com/community/c6h391364}{(Link to AoPS)}
\end{problem}



\begin{solution}[by \href{https://artofproblemsolving.com/community/user/29428}{pco}]
	\begin{tcolorbox}Let $f(x)$ be a real valued continuous function such that $f(x-y)=f(x)f(y)-f(a-x)f(a+y)\ \forall\ x,\ y\ \in \mathbb{R}$ where $a$ is a given constant and $f(0)=1$. Find all possible $f(x).$\end{tcolorbox}
Let $P(x,y)$ be the assertion $f(x-y)=f(x)f(y)-f(a-x)f(a+y)$

If $a=0$, $P(0,0)$ $\implies$ $f(0)=0$ and so contradiction. So $a\ne 0$

Let $f(x)=g(\frac xa)$ and we get the new assertion $Q(x,y)$ : $g(x-y)=g(x)g(y)-g(1-x)g(1+y)$

$Q(0,0)$ $\implies$ $g(x)g(1)=0$ and so $g(1)=0$
$Q(0,x)$ $\implies$ $g(-x)=g(x)$
$Q(1,x)$ $\implies$ $g(1-x)=-g(x+1)$
$Q(x,x)$ $\implies$ $1=g(x)^2-g(1-x)g(1+x)$ $=g(x)^2+g(1-x)^2$ and $g(x)\in[-1,+1]$
$Q(x,-x)$ $\implies$ $g(2x)=g(x)g(-x)-g(1-x)g(1-x)$ $=2g(x)^2-1$

Let then $A=\{x>0$ such that $g(x)=0\}$ 
$A$ is non empty since $1\in A$ and let $u=\inf(A)$. Continuity implies $g(u)=0$ and we have $g(x)\in(0,1]$ $\forall x\in[0,u)$

$g(u)=0$ and $g(x)=2g(\frac x2)^2-1$ implies then $g(\frac u2)=\cos\frac{\pi}4$
And an immediate induction implies $g(\frac{u}{2^n})=\cos\frac{\pi}{2^{n+1}}$

Using then $g(x+y)=g(x)g(y)-g(1-x)g(1-y)$ and $g(x)^2+g(1-x)^2=1$, it's easy to get $g(\frac p{2^q}u)=\cos(\frac p{2^q}\frac{\pi}2)$ $\forall p,q\in\mathbb N$ such that $p\le 2^q$

Continuity gives then $g(x)=\cos\frac{\pi}{2u}x$ $\forall x\in[0,u]$

And it's easy to extend this to $g(x)=\cos\frac{\pi}{2u}x$ $\forall x$
Using then $g(1)=0$, we get $u=\frac 1{2n+1}$ with $n\in\mathbb Z$

Hence the general solution $\boxed{f(x)=\cos\frac{(2n+1)\pi}{2a}x}$ $\forall x$ which indeed is a solution
\end{solution}
*******************************************************************************
-------------------------------------------------------------------------------

\begin{problem}[Posted by \href{https://artofproblemsolving.com/community/user/68555}{trbst}]
	Find all functions $f: \mathbb{R}\to\mathbb{R}$ which are continuous at $1$ and satisfy \[f(9x-8)-2f(3x-2)+f(x)=4x-4\] for all real $x$.
	\flushright \href{https://artofproblemsolving.com/community/c6h391547}{(Link to AoPS)}
\end{problem}



\begin{solution}[by \href{https://artofproblemsolving.com/community/user/29428}{pco}]
	\begin{tcolorbox}Find $f$ : $\mathbb{R}\to\mathbb{R}$ continuous in 1 so that : $f(9x-8)-2f(3x-2)+f(x)=4x-4$ .\end{tcolorbox}
let $g(x)=f(x+1)-(x+1)$ so that $f(x)=x+g(x-1)$ and the equation becomes $g(9x-9)-2g(3x-3)+g(x-1)=0$

And so $g(9x)-2g(3x)+g(x)=0$ which may be written $g(9x)-g(3x)=g(3x)-g(x)$ and so $g(3x)-g(x)=g(3^{1-n}x)-g(3^{-n}x)$

Setting $n\to+\infty$ and using continuity, we get $g(3x)=g(x)$ and so $g(x)=g(3^{-n}x)$

Setting $n\to+\infty$ and using continuity, we get $g(x)=g(0)$

And so $\boxed{f(x)=x+a}$ which indeed is a solution.
\end{solution}
*******************************************************************************
-------------------------------------------------------------------------------

\begin{problem}[Posted by \href{https://artofproblemsolving.com/community/user/68555}{trbst}]
	Let $f: \mathbb{R}\to\mathbb{R}$ be a function such that \[2^{1+f(x+1)}\ge 2^{f(x)}+2^{f(x+2)}\] for all $x\in\mathbb{R}$.

Show that if $\lim_{x\to\infty} f(x)$ exists, then the function $f$ is constant.
	\flushright \href{https://artofproblemsolving.com/community/c6h391548}{(Link to AoPS)}
\end{problem}



\begin{solution}[by \href{https://artofproblemsolving.com/community/user/29428}{pco}]
	\begin{tcolorbox}Let $f$ : $\mathbb{R}\to\mathbb{R}$ a function such that : $2^{1+f(x+1)}\ge 2^{f(x)}+2^{f(x+2)}$ forall $x\in\mathbb{R}$ .

Show that if exists $\lim_{x\to\infty} f(x)$ then the function $f$ is constant.\end{tcolorbox}
For easier writing, let $g(x)=2^{f(x)}$ so that the inequation is $2g(x+1)\ge g(x)+g(x+2)$

So $g(x+1)-g(x)\ge g(x+2)-g(x+1)$ and :
$g(x+n)-g(x)\le n(g(x+1)-g(x))$ $\forall n\in\mathbb N$
$g(x-n)-g(x)\le n(g(x)-g(x+1))$ $\forall n\in\mathbb N$

If $g(x+1)>g(x)$, setting $n\to+\infty$ in second line shows that $g(x)<0$ at some moment, which is impossible
If $g(x+1)<g(x)$, setting $n\to+\infty$ in first line shows that $g(x)<0$ at some moment, which is impossible
So $g(x+1)=g(x)$

So $g(x+n)=g(x)$ and $g(x)=\lim_{n\to+\infty}g(x+n)=$ constant.

And so $f(x)=c$
Q.E.D.
\end{solution}
*******************************************************************************
-------------------------------------------------------------------------------

\begin{problem}[Posted by \href{https://artofproblemsolving.com/community/user/67223}{Amir Hossein}]
	Find all functions $f : \mathbb R \to \mathbb R$ such that
\[f(x^2) \biggl( f(x)^2 + f\left( \frac{1}{y^2} \right) \biggr) = 1+f\left( \frac{1}{xy} \right) \quad \forall x,y \in \mathbb R \setminus\{0\} .\]
	\flushright \href{https://artofproblemsolving.com/community/c6h391584}{(Link to AoPS)}
\end{problem}



\begin{solution}[by \href{https://artofproblemsolving.com/community/user/29428}{pco}]
	\begin{tcolorbox}Find all functions $f : \mathbb R \to \mathbb R$ such that
\[f(x^2) \biggl( f(x)^2 + f\left( \frac{1}{y^2} \right) \biggr) = 1+f\left( \frac{1}{xy} \right) \quad \forall x,y \in \mathbb R .\]\end{tcolorbox}
I suppose we must read $\forall x,y\in\mathbb R\setminus\{0\}$

Let $P(x,y)$ be the assertion $f(x^2)(f(x)^2+f(\frac 1{y^2}))=1+f(\frac 1{xy})$
Let $a=f(1)$

$P(1,\frac 1x)$ $\implies$ $f(x)=a^3+af(x^2)-1$ and so $f(x)$ is even

$P(x,\frac 1x)$ $\implies$ $f(x^2)(f(x)^2+f(x^2))=1+a$
Plugging in this line the value $f(x)=a^3+af(x^2))-1$ (see two lines above), we get :
$f(x^2)((a^3+af(x^2)-1)^2+f(x^2))=1+a$

And so $f(x^2)$ is root of the equation $a^2X^3+(2a^4-2a+1)X^2+(a^3-1)^2X-(a+1)=0$

$P(1,1)$ $\implies$ $(a+1)(a^2-1)=0$ and so $a=\pm 1$

If $a=-1$, the cubic is $X(X+1)(X+4)=0$
The equation $f(x)=a^3+af(x^2)-1$ becomes $f(x)=-f(x^2))-2$
If $f(x^2)=0$ for some $x$, then $f(x)=-2$, impossible
If $f(x^2)=-4$ for some $x$, then $f(x)=2$, impossible
So $f(x)=-1$ $\forall x\ne 0$ which indeed is a solution.

If $a=1$, the cubic is $(X-1)(X^2+2X+2)=0$ and so $f(x)=1$ $\forall x\ne 0$ which indeed is a solution.

Hence the answer :
$f(x)=-1$ $\forall x\ne 0$
$f(x)=+1$ $\forall x\ne 0$
\end{solution}
*******************************************************************************
-------------------------------------------------------------------------------

\begin{problem}[Posted by \href{https://artofproblemsolving.com/community/user/96259}{qoqoreaqoqores}]
	Find all functions $f: \mathbb R \to \mathbb R$ such that for all reals $x$ and $y$,
\[f(x^3+y^3)=(x+y)(f(x)^2+f(x)f(y)+f(y)^2).\]
	\flushright \href{https://artofproblemsolving.com/community/c6h391700}{(Link to AoPS)}
\end{problem}



\begin{solution}[by \href{https://artofproblemsolving.com/community/user/29428}{pco}]
	\begin{tcolorbox}plz prove this problem.


Find all f.

\begin{italicized}f(x^3+y^3)=(x+y){f(x)^2+f(x)f(y)+f(y)^2}\end{italicized}\end{tcolorbox}
Let $P(x,y)$ be the assertion $f(x^3+y^3)=(x+y)(f(x)^2+f(x)f(y)+f(y)^2)$

$P(0,0)$ $\implies$ $f(0)=0$
$P(1,0)$ $\implies$ $f(1)=f(1)^2$ and so $f(1)=0$ or $f(1)=1$

If $f(1)=0$ :
Notice first that $f(x)^2+f(x)f(y)+f(y)^2\ge 0$ and equality only when $f(x)=f(y)=0$
$P(x,\sqrt[3]{1-x^3})$ $\implies$ $0=(x+\sqrt[3]{1-x^3})(f(x)^2+f(x)f(\sqrt[3]{1-x^3})+f(\sqrt[3]{1-x^3})^2)$
And so $f(x)=0$ $\forall x$ which indeed is a solution.

If $f(1)=1$
$P(1,1)$ $\implies$ $f(2)=6$
$P(x,0)$ $\implies$ $f(x^3)=xf(x)^2$ and so $f(x)$ is zero or have same sign than $x$
Let $x>0$
$P(\sqrt[3]x,0)$ $\implies$ $f(x)=\sqrt[3]xf(\sqrt[3]x)^2$ and so $f(\sqrt[3]x)=\sqrt{\frac{f(x)}{\sqrt[3]x}}$
$P(\sqrt[3]x,1)$ $\implies$ $f(x+1)=(\sqrt[3]x+1)(\frac{f(x)}{\sqrt[3]x}+\sqrt{\frac{f(x)}{\sqrt[3]x}}+1)$ $>f(x)(1+\frac 1{\sqrt[3] x})$

And so $f(8)>f(2) (1+\frac 1{\sqrt[3]2})$ $(1+\frac 1{\sqrt[3]3})$ $(1+\frac 1{\sqrt[3]4})$ $(1+\frac 1{\sqrt[3]5})$ $(1+\frac 1{\sqrt[3]6})$ $(1+\frac 1{\sqrt[3]7}) >100$

But $P(2,0)$ $\implies$ $f(8)=72$
So contradiction and no solution with $f(1)=1$

Hence the unique solution : $\boxed{f(x)=0}$ $\forall x$
\end{solution}
*******************************************************************************
-------------------------------------------------------------------------------

\begin{problem}[Posted by \href{https://artofproblemsolving.com/community/user/86514}{cpn}]
	Find all functions $ f:\mathbb R \to \mathbb R$ such that $xf(y)+yf(x)=(x+y)f(x)f(y)$ for all reals $x$ and $y$.
	\flushright \href{https://artofproblemsolving.com/community/c6h391858}{(Link to AoPS)}
\end{problem}



\begin{solution}[by \href{https://artofproblemsolving.com/community/user/93909}{magical}]
	\begin{tcolorbox}Find all function $ f:\mathbb R \to \mathbb R$ such that $xf(y)+yf(x)=(x+y)f(x)f(y)$\end{tcolorbox}

Put 
$y:=1$ we have $f(x)[xf(1)+f(1)-1]=xf(1)$
$x=y=1$ we have $f(1)=(f(1))^2 $

If $f(1)=0 \Rightarrow f(x)=0$

If $f(1)=1 \Rightarrow f(x)=1$

So $f(x)=0$ and $f(x)=1$ $\forall x \in \mathbb R$
\end{solution}



\begin{solution}[by \href{https://artofproblemsolving.com/community/user/29428}{pco}]
	\begin{tcolorbox}Find all function $ f:\mathbb R \to \mathbb R$ such that $xf(y)+yf(x)=(x+y)f(x)f(y)$\end{tcolorbox}
Let $P(x,y)$ be the assertion $xf(y)+yf(x)=(x+y)f(x)f(y)$

$f(x)=0$ $\forall x$ is a solution and let us from now look onlu for non all-zero solutions.

1) if $f(0)\ne 0$ :
$P(x,0)$ $\implies$ $xf(0)=xf(x)f(0)$ and so $f(x)=1$ $\forall x\ne 0$
And the function $f(0)=a$ and $f(x)=1$ $\forall x\ne 0$ is indeed a solution.

2) $f(0)=0$ and $f(u)\ne 0$ for some $u\ne 0$
If $f(y)=0$ for some $y\ne 0$, $P(u,y)$ becomes $yf(u)=0$ and so contradiction. So $f(x)=0$ $\iff$ $x=0$

Let then $x,y\ne 0$ and $g(x)=\frac x{f(x)}$. $P(x,1)$ becomes : $g(x)=x+1-g(1)$
Setting  $x=1$ in this equation, we get $g(1)=1$ and $g(x)=x$ and $f(x)=1$ which indeed is a solution.

Hence the two solutions :
$f(x)=0$ $\forall x$
$f(x)=1$ $\forall x\ne 0$ and $f(0)=a$ where $a$ is any real.
\end{solution}



\begin{solution}[by \href{https://artofproblemsolving.com/community/user/89144}{sumanguha}]
	set $x=y $ to get $2xf(x)=2xf(x)^2$
so, for $x\neq 0$  we have $ f(x)[f(x)-1]=0$

so, for $x\neq 0$ either $f(x)=0 $ or $1$.

now let there is $x_{1}\neq 0, x_{2}\neq 0$ s.t. $f(x_{1})=0,f(x_{2})=1$

then set $x=x_{1},y=x_{2} $ to see $x_{1}=0$ contradiction.

so, either $f(x)=0$ for all $x\neq 0$
or $f(x)=1$ for all $x\neq 0$

consider $f(x)=0$ for $x\neq 0$

then set $x\neq 0$ and $y=0$

to get$xf(0)=0$ so, $f(0)=0$

so then see $f(x)=0 $ for all x is a solution.

for the other case see $f(x)=1 $ for $x\neq 0$ and $f(0)=c$

satisfy the equation.


so the solutions are

$f(x)=1 $ for $x\neq 0$ and $f(0)=c$ where $c\in \mathbb R$
and
$f(x)=0$ for all $x\in \mathbb R$
\end{solution}
*******************************************************************************
-------------------------------------------------------------------------------

\begin{problem}[Posted by \href{https://artofproblemsolving.com/community/user/93044}{nguyenhung}]
	Find all functions $f:\mathbb{Z} \to \mathbb{Z}$ such that \[f\left( {m + f\left( n \right)} \right) = f\left( m \right) - n, \quad  \forall m,n \in \mathbb{Z}.\]
	\flushright \href{https://artofproblemsolving.com/community/c6h391861}{(Link to AoPS)}
\end{problem}



\begin{solution}[by \href{https://artofproblemsolving.com/community/user/29428}{pco}]
	\begin{tcolorbox}Find all function $f:\mathbb{Z} \to \mathbb{Z}$ such that $f\left( {m + f\left( n \right)} \right) = f\left( m \right) - n,   \forall m,n \in \mathbb{Z}$\end{tcolorbox}
Let $P(x,y)$ be the assertion $f(x+f(y))=f(x)-y$

$f(x)$ is injective and so $P(0,0)$ $\implies$ $f(f(0))=f(0)$ and so $f(0)=0$
$P(0,x)$ $\implies$ $f(f(x))=-x$
$P(x,f(y))$ $\implies$ $f(x-y)=f(x)-f(y)$ and so $f(x)=xf(1)$
Plugging this in original equation, we get $f(1)^2=-1$

So no solution.
\end{solution}
*******************************************************************************
-------------------------------------------------------------------------------

\begin{problem}[Posted by \href{https://artofproblemsolving.com/community/user/93909}{magical}]
	Find all functions $f: \mathbb R \to \mathbb R$ which satisfy for all $x, y \in \mathbb R$,
\[f(2x+f(y))=f(2x)+xf(2y)+f(f(y)).\]
	\flushright \href{https://artofproblemsolving.com/community/c6h391863}{(Link to AoPS)}
\end{problem}



\begin{solution}[by \href{https://artofproblemsolving.com/community/user/29428}{pco}]
	\begin{tcolorbox}Find all functional $f:\mathbb{R}\to\mathbb{R}$ satisfy
$f(2x+f(y))=f(2x)+xf(2y)+f(f(y)),\forall x,y\in\mathbb{R} $\end{tcolorbox}
Let $P(x,y)$ be the assertion $f(2x+f(y))=f(2x)+xf(2y)+f(f(y))$

$P(0,0)$ $\implies$ $f(0)=0$
$f(x)=0$ $\forall x$ is a solution and let us from now look for non all-zero solutions. Let then $u$ such that $f(u)\ne 0$

1) any real x may be written as $x=f(u)-f(v)$ for some real $u,v$
===========================================
$P(\frac{x-f(f(\frac u2))}{f(u)},\frac u2)$ $\implies$ $f(2\frac{x-f(f(\frac u2))}{f(u)}+f(\frac u2))$ $-f(2\frac{x-f(f(\frac u2))}{f(u)})$ $=x$
Q.E.D.

2) $f(2x)=4f(x)$ and $f(f(x))=f(-f(x))=f(x)^2$
=====================================
Maybe there exist a shorter proof for this part :oops:
$P(\frac{f(x)}2,u)$ $\implies$ $f(f(x)+f(u))=f(f(x))+\frac 12f(x)f(2u)+f(f(u))$
$P(\frac{f(u)}2,x)$ $\implies$ $f(f(u)+f(x))=f(f(u))+\frac 12f(u)f(2x)+f(f(x))$
And so $f(2x)=af(x)$ with $a=\frac{f(2u)}{f(u)}$ and $a\ne 0$ (else $f(x)=0$ $\forall x$)

$P(\frac{f(x)}2,x)$ $\implies$ $f(2f(x))=2f(f(x))+\frac 12f(x)f(2x)$ and so $(a-2)f(f(x))=\frac a2f(x)^2$
If $a=2$, this means $f(x)=0$ $\forall x$, impossible in this part of the proof.
So $a\ne 2$ and $f(f(x))=\frac a{2a-4}f(x)^2$

$P(-\frac{f(x)}2,x)$ $\implies$ $0=f(-f(x))-\frac a2f(x)^2+f(f(x))$ $=f(-f(x))-\frac{a(a-3)}{2a-4}f(x)^2$
And so $f(-f(x))=\frac{a(a-3)}{2a-4}f(x)^2$

$P(-f(x),x)$ $\implies$ $f(-f(x))=f(-2f(x))-f(x)f(2x)+f(f(x))$ $=af(-f(x))-af(x)^2+f(f(x))$ $=af(-f(x))-\frac{a(2a-5)}{2a-4}f(x)^2$
And so $(a-1)f(-f(x))=\frac{a(2a-5)}{2a-4}f(x)^2$ and so $a\ne 1$ and $f(-f(x))=\frac{a(2a-5)}{(a-1)(2a-4)}f(x)^2$

Comparing the two expressions we got for $f(-f(x))$, we get $\frac{a(a-3)}{2a-4}=\frac{a(2a-5)}{(a-1)(2a-4)}$ and so $a=4$ (the solutions $a=0,2$ have already be discarded).

So $f(2x)=4f(x)$ and $f(f(x))=f(-f(x))=f(x)^2$ 
Q.E.D.

3) $f(x)=x^2$ $\forall x$
===============
$P(-\frac{f(y)}2,x)$ $\implies$ $f(f(x)-f(y))=f(-f(y))-\frac 12f(y)f(2x)+f(f(x))$ $=f(x)^2-2f(x)f(y)+f(y)^2$
And so $f(f(x)-f(y))=(f(x)-f(y))^2$
Using then the result of 1) above, we get $f(x)=x^2$ $\forall x$ which indeed is a solution.
Q.E.D.

4) Synthesis of results
================
We got two solutions :
$f(x)=0$ $\forall x$
$f(x)=x^2$ $\forall x$
\end{solution}



\begin{solution}[by \href{https://artofproblemsolving.com/community/user/93909}{magical}]
	This is hard problem.

Thank you very much!
\end{solution}
*******************************************************************************
-------------------------------------------------------------------------------

\begin{problem}[Posted by \href{https://artofproblemsolving.com/community/user/31160}{canada}]
	Find all functions $f:\mathbb{R}\to\mathbb{R}$ such that for all $a,b\in\mathbb{R}$ the following holds: \[a\geq f(b) \iff  f(a)\geq f(b).\]
	\flushright \href{https://artofproblemsolving.com/community/c6h392074}{(Link to AoPS)}
\end{problem}



\begin{solution}[by \href{https://artofproblemsolving.com/community/user/29428}{pco}]
	\begin{tcolorbox}Find all functions $f:\mathbb{R}\rightarrow\mathbb{R}$ such that for all $a,b\in\mathbb{R}$ the following holds \[a\geq f(b)\ \ \Leftrightarrow\ \ f(a)\geq f(b)\]\end{tcolorbox}
Setting $a=f(x)$ and $b=x$, we get $f(x)\ge f(x)$ $\iff$ $f(f(x))\ge f(x)$
Setting $a=f(x)$ and $b=f(x)$, we get $f(x)\ge f(f(x))$ $\iff$ $f(f(x))\ge f(f(x))$
And so $f(f(x))=f(x)$

Setting $a=x$ and $b=x$, we get $x\ge f(x)$ $\iff$ $f(x)\ge f(x)$ and so $f(x)\le x$

Then $x\ge y$ $\implies$ $x\ge f(y)$ $\implies$ $f(x)\ge f(y)$ and so $f(x)$ is non decreasing.

So we got the necessary conditions :
$f(x)$ is non decreasing
$f(x)\le x$
$f(f(x))=f(x)$

It's then easy to see that these conditions are sufficient :
$a\ge f(b)$ implies (non decreasing) $f(a)\ge f(f(b))=f(b)$ 
$f(a)\ge f(b)$ implies $a\ge f(a)\ge f(b)$ and so $a\ge f(b)$

So the answer is $\boxed{\text{any non decreasing function such that }f(f(x))=f(x)\le x\text{    }\forall x\in\mathbb R}$

And infinitely many such functions exist. For example :
$f(x)=x$
$f(x)=\lfloor x\rfloor$
$f(x)=n\left\lfloor\frac xn\right\rfloor$
$f(x)=\frac{x-|x|}2$
...

\begin{bolded}We can also define $f(x)$ in a more general manner as \end{bolded}\end{underlined}:
Let $A\subseteq\mathbb R$ any subset of $\mathbb R$ not lower bounded. Then : $\boxed{f(x)=\sup\left( A\cap(-\infty,x]\right)}$
\end{solution}
*******************************************************************************
-------------------------------------------------------------------------------

\begin{problem}[Posted by \href{https://artofproblemsolving.com/community/user/61513}{Obel1x}]
	Find all continuous functions $f:\mathbb{R}^{+} \to \mathbb{R}$ such that for all $x,y >0$,
\[ f(x)+f(y)=f\Big( \sqrt[m]{x^m+y^m}\Big),\]
where $m \in \mathbb{N}$.
	\flushright \href{https://artofproblemsolving.com/community/c6h392299}{(Link to AoPS)}
\end{problem}



\begin{solution}[by \href{https://artofproblemsolving.com/community/user/29428}{pco}]
	\begin{tcolorbox}Find all continuous functions $f:\mathbb{R}^{+} \to \mathbb{R}$ such that $\forall x,y \ge 0$ holds:
\[ f(x)+f(y)=f\Big( \sqrt[m]{x^m+y^m}\Big)\]
where $m \in \mathbb{N}$.\end{tcolorbox}
Let $f(x)=g(x^m)$ and the equation becomes $g(x^m)+g(y^m)=g(x^m+y^m)$ and so $g(x)+g(y)=g(x+y)$ and so $g(x)=cx$ and $\boxed{f(x)=cx^m}$
\end{solution}
*******************************************************************************
-------------------------------------------------------------------------------

\begin{problem}[Posted by \href{https://artofproblemsolving.com/community/user/39663}{M\u00eendril\u0103 Claudiu}]
	Find all functions $f : \mathbb{R} \longmapsto \mathbb{R}$ such that \[f(x^3+y^3)=xf(x^2)+yf(y^2), \quad \forall x,\ y\in\mathbb{R}.\]
	\flushright \href{https://artofproblemsolving.com/community/c6h393100}{(Link to AoPS)}
\end{problem}



\begin{solution}[by \href{https://artofproblemsolving.com/community/user/29428}{pco}]
	\begin{tcolorbox}Fiind the functions $f : \mathbb{R} \longmapsto \mathbb{R}$  such that \[f(x^3+y^3)=xf(x^2)+yf(y^2)\ \forall x,\ y\in\mathbb{R}\]

Romanian National Mathematical Olympiad, 2009\end{tcolorbox}
Let $P(x,y)$ be the assertion $f(x^3+y^3)=xf(x^2)+yf(y^2)$
Let $f(1)=a$

$P(x,0)$ $\implies$ $f(x^3)=xf(x^2)$ and the equation implies $f(x^3+y^3)=f(x^3)+f(y^3)$ and so $f(x+y)=f(x)+f(y)$

From this, we get $f(px)=pf(x)$ $\forall x\in\mathbb R$ $\forall p\in\mathbb Q$

$P(x+p,0)$ $\implies$ $f(x^3+3px^2+3p^2x+p^3)$ $=(x+p)f(x^2+2px+p^2)$
$\implies$ $f(x^3)+3pf(x^2)+3p^2f(x)+f(p^3)$ $=xf(x^2)+2pxf(x)+xp^2a+pf(x^2)+2p^2f(x)+pf(p^2)$
$\implies$ $2p(f(x^2)-xf(x))+p^2(f(x)-ax)=0$

And since this is true $\forall p\in\mathbb Q$, we get $f(x^2)=xf(x)$ and $f(x)=ax$ which indeed is a solution.

Hence the answer : $\boxed{f(x)=ax}$ $\forall x$ and for any $a\in\mathbb R$
\end{solution}



\begin{solution}[by \href{https://artofproblemsolving.com/community/user/39663}{M\u00eendril\u0103 Claudiu}]
	Thank you, Patrick! :-)
\end{solution}
*******************************************************************************
-------------------------------------------------------------------------------

\begin{problem}[Posted by \href{https://artofproblemsolving.com/community/user/100575}{soulhunter}]
	Let $F: [0,1] \to \mathbb R$ be an increasing function such that for all $x \in [0, 1]$,
\[F\left(\frac{x}{3}\right)=\frac{F(x)}{2} \quad \text{and} \quad F(1-x)=1-F(x).\]
Find $ F(173\/1993)$ and $ F(1\/13)$.
	\flushright \href{https://artofproblemsolving.com/community/c6h393366}{(Link to AoPS)}
\end{problem}



\begin{solution}[by \href{https://artofproblemsolving.com/community/user/31915}{Batominovski}]
	Note that $F(0)=0$ and $F(1)=1$.  Write each $x\in(0,1)$ in its ternary representation, say:
\[x=\left(\overline{0.a_1a_2a_3\ldots}\right)_3\,,\]
where $a_i \in \{0,1,2\}$ for all $i$.  Suppose that $a_i=1$ for some $i$; let $k$ is the minimum such $i$.  It can be proven that
\[F(x)=\left(\overline{0.b_1b_2b_3 \ldots b_{k-1}a_k}\right)_2\,,\]
where $b_i=\frac{a_i}{2}$ for each $i=1,2,\ldots,k-1$.  We can also use the monotonicity condition to show that if $a_i\in\{0,2\}$ for all $i$, then
\[F(x)=\left(\overline{0.b_1b_2b_3 \ldots }\right)_2\,,\]
where $b_i=\frac{a_i}{2}$ for every $i=1,2,3,\ldots$.

Knowing that $\frac{173}{1993}\approx\left(0.0021\right)_3$ and $\frac{1}{13}=\left(0.\dot{0}0\dot{2}\right)_3$, we get
\[F\left(\frac{173}{1993}\right) = \left(0.0011\right)_2 = \frac{3}{16}\]
and
\[F\left(\frac{1}{13}\right) = \left(0.\dot{0}0\dot{1}\right)_2=\sum_{i=1}^\infty \frac{1}{8^i} = \frac{1}{7}\,.\]
\end{solution}



\begin{solution}[by \href{https://artofproblemsolving.com/community/user/99076}{RSM}]
	To find $ F(\frac {1}{13}) $ you don't need the condition that F is monotonic.
It can be done in this way:-
$ 1-F(\frac {1}{13})=F(\frac {12}{13})=2F(\frac {4}{13})=2-2F(\frac {9}{13})=2-8F(\frac {1}{13}) $

$ 7F(\frac {1}{13})=1 $

However, @Batominovski's solution is great.
\end{solution}



\begin{solution}[by \href{https://artofproblemsolving.com/community/user/100575}{soulhunter}]
	@RSM tats the way i solved the second part and stuck on that as i can't solve the first part!! lolzz
\end{solution}



\begin{solution}[by \href{https://artofproblemsolving.com/community/user/29428}{pco}]
	This the nice Cantor function.
For complementary informations, see http://en.wikipedia.org\/wiki\/Cantor_function
\end{solution}
*******************************************************************************
-------------------------------------------------------------------------------

\begin{problem}[Posted by \href{https://artofproblemsolving.com/community/user/89818}{gauman}]
	Find all functions $f: \mathbb R^+ \to \mathbb R^+$ which satisfy for all $x, y \in \mathbb R^+$,
\[f(x^3+y)=f(x)^3 + \frac{f(xy)}{f(x)}.\]
	\flushright \href{https://artofproblemsolving.com/community/c6h393579}{(Link to AoPS)}
\end{problem}



\begin{solution}[by \href{https://artofproblemsolving.com/community/user/29428}{pco}]
	\begin{tcolorbox}Find all $f:R+ \to R+$: $f(x^3+y)=f(x)^3 + \frac{f(xy)}{f(x)}$\end{tcolorbox}
Let $P(x,y)$ be the assertion $f(x^3+y)=f(x)^3+\frac {f(xy)}{f(x)}$
Let $f(1)=a$

The key point here is to show that $f(1)=1$ and I hope that some simpler path than mine exists.


$P(x,y)$ implies $f(x^3+y)>f(x)^3$ and so $P(1,x-1)$ $\implies$ $f(x)>a^3$ $\forall x>1$
But then $P(\sqrt[3]{\frac {x+1}2},\frac {x-1}2)$ $\implies$ $f(x)>f(\sqrt[3]{\frac {x+1}2})^3>a^9$ $\forall x>1$
And an immediate induction shows that $f(x)>a^{3^n}$ $\forall x>1$ and so $\boxed{a\le 1}$

$P(x,1)$ $\implies$ $f(x^3+1)=f(x)^3+1>1$ and so $f(x)>1$ $\forall x>1$
$P(1,x)$ $\implies$ $f(x+1)=a^3+\frac{f(x)}a$ and so (using previous line) $a^3+\frac{f(x)}a>1$
So $f(x)>a-a^4$ $\forall x$

Let $x<1$ : $P(x,1-x^3)$ $\implies$ $a>f(x)^3$ and so $f(x)=b<1$
$P(x,y)$ implies $f(x^3+y)>\frac{f(xy)}{f(x)}$ and so $f(xy)<f(x)f(x^3+y)$
Setting $y=x-x^3$ in this inequality, we get $f(x^2-x^4)<b^2<1$
Using this process as many times we want, we can find some values $x_n<1$ such that $f(x_n)<b^{2^n}$ and so $f(x)$ may be as near of zero as we want.
But then $f(x)>a-a^4$ implies $a-a^4\le 0$ and so $\boxed{a\ge 1}$

And so $a=1$
$P(1,x)$ $\implies$ $f(x+1)=f(x)+1$

So $P(x,y+1)$ $\implies$ $f(x^3+y)+1=f(x)^3+\frac {f(xy+x)}{f(x)}$
Subtracting $P(x,y)$ from this line, we get : $f(xy+x)=f(xy)+f(x)$ and so $f(x+y)=f(x)+f(y)$

So $f(x+y)>f(x)$ and we have a Cauchy's equation with increasing solution and so $f(x)=xf(1)=x$, which indeed is a solution.

Hence the answer : $\boxed{f(x)=x}$ $\forall x$
\end{solution}



\begin{solution}[by \href{https://artofproblemsolving.com/community/user/55393}{makar}]
	\begin{tcolorbox}Find all $f:R+ \to R+$: $f(x^3+y)=f(x)^3 + \frac{f(xy)}{f(x)}$\end{tcolorbox}

Let $f(x)=xg(x),\ g(x)>0$

$\implies (x^3+y)g(x^3+y)=x^3g(x)^3 + \frac{yg(xy)}{g(x)}$

$\implies x^3(g(x)-g(x)^4)=y(g(xy)-g(x)g(x^3+y))$

LHS is independent of $y\implies g(x)-g(x)^4=0\implies g(x)=1$

$\implies f(x)=x$
\end{solution}



\begin{solution}[by \href{https://artofproblemsolving.com/community/user/29428}{pco}]
	\begin{tcolorbox} $\implies (x^3+y)g(x^3+y)=x^3g(x)^3 + \frac{yg(xy)}{g(x)}$

$\implies x^3(g(x)-g(x)^4)=y(g(xy)-g(x)g(x^3+y))$\end{tcolorbox}


:oops: how do you come from the first line to the second one ? I dont understand at all.
\end{solution}



\begin{solution}[by \href{https://artofproblemsolving.com/community/user/55393}{makar}]
	\begin{tcolorbox}[quote="makar"] $\implies (x^3+y)g(x^3+y)=x^3g(x)^3 + \frac{yg(xy)}{g(x)}$

$\implies x^3(g(x)-g(x)^4)=y(g(xy)-g(x)g(x^3+y))$\end{tcolorbox}


:oops: how do you come from the first line to the second one ? I dont understand at all.\end{tcolorbox}


What a fool i am :fool:  

Its my bad sorry :oops:
\end{solution}
*******************************************************************************
-------------------------------------------------------------------------------

\begin{problem}[Posted by \href{https://artofproblemsolving.com/community/user/46121}{u2tommyf}]
	Determine $f:(0, \infty) \rightarrow (0, \infty)$ such that for any $x \in (0, \infty)$ we have:

$f(f(x)-x)=6x$.
	\flushright \href{https://artofproblemsolving.com/community/c6h393617}{(Link to AoPS)}
\end{problem}



\begin{solution}[by \href{https://artofproblemsolving.com/community/user/29428}{pco}]
	\begin{tcolorbox}Determine $f:(0, \infty) \rightarrow (0, \infty)$ such that for any $x \in (0, \infty)$ we have:

$f(f(x)-x)=6x$.\end{tcolorbox}
In order to have the functional equation verified, we need to have $f(x)-x>0$ and so $f(x)>x$ $\forall x$
Then $f(f(x)-x)>f(x)-x$ and so $f(x)<7x$

Suppose now $a_nx>f(x)>b_nx$ $\forall x>0$

$a_n(f(x)-x)>f(f(x)-x)>b_n(f(x)-x$ and so $(1+\frac 6{b_n})x>f(x)>(1+\frac 6{a_n})x$

And so we can build the sequences :
$a_1=7$
$b_1=1$
$a_{n+1}=1+\frac 6{b_n}$
$b_{n+1}=1+\frac 6{a_n}$

And it's easy to show that these two sequences are convergent towards $3$ and, since $a_nx>f(x)>b_nx$ $\forall x>0$, the only possible solution must be $f(x)=3x$ which indeed is a solution.

Hence the answer : $\boxed{f(x)=3x}$ $\forall x$
\end{solution}



\begin{solution}[by \href{https://artofproblemsolving.com/community/user/46121}{u2tommyf}]
	\begin{tcolorbox}
$a_n(f(x)-x)>f(f(x)-x)>b_n(f(x)-x$ and so $(1+\frac 6{b_n})x>f(x)>(1+\frac 6{a_n})x$
\end{tcolorbox}
How exactly did you do this?

Edit:  I got it, thanks anyway.
\end{solution}



\begin{solution}[by \href{https://artofproblemsolving.com/community/user/29428}{pco}]
	\begin{tcolorbox}[quote="pco"]
$a_n(f(x)-x)>f(f(x)-x)>b_n(f(x)-x$ and so $(1+\frac 6{b_n})x>f(x)>(1+\frac 6{a_n})x$
\end{tcolorbox}
How exactly did you do this?\end{tcolorbox}
I suppose obviously that $a_n,b_n>0$. Then :

$a_n(f(x)-x)>f(f(x)-x)=6x$
$a_nf(x)-a_nx>6x$
$a_nf(x)>a_nx+6x$
$f(x)>(1+\frac 6{a_n})x$

$6x=f(f(x)-x)>b_n(f(x)-x)$
$6x>b_nf(x)-b_nx$ 
$b_nx+6x>b_nf(x)$ 
$(1+\frac6{b_n})x>f(x)$
\end{solution}
*******************************************************************************
-------------------------------------------------------------------------------

\begin{problem}[Posted by \href{https://artofproblemsolving.com/community/user/60219}{shixian105}]
	Find all bijections $f$ such that $f^{-1}(g(f(x)))$ is a quadratic polynomial for all quadratic polynomials $g(x)$.
	\flushright \href{https://artofproblemsolving.com/community/c6h393948}{(Link to AoPS)}
\end{problem}



\begin{solution}[by \href{https://artofproblemsolving.com/community/user/29428}{pco}]
	\begin{tcolorbox}"Find all bijections $f$ such that $f^{-1}(g(f(x)))$ is a quadratic polynomial $\forall g(x)$ quadratic polynomial"\end{tcolorbox}
Thanks for the confirmation. If so :

First it's obvious that $f(x)$ is a bijection from $\mathbb R\to\mathbb R$

Let then $g_a(x)=x^2-ax+1$ and $h_a(x)=u(a)x^2+v(a)x+w(a)=f^{-1}(g_a(f(x)))$ and let $t(a)=-\frac{v(a)}{u(a)}$

Let $a\in\mathbb R$ and $x\ne \frac {t(a)}2$ : $h_a(t(a)-x)=h_a(x)$ and so $g_a(f(t(a)-x))=g_a(f(x))$
and since $t(a)-x\ne x$, we get $f(t(a)-x)\ne f(x)$ and so $f(t(a)-x)=a-f(x)$ $\forall x\ne  \frac {t(a)}2$

But this equation implies $f(x)\ne \frac a2$ $\forall x\ne\frac{t(a)}2$ ($f$ is a bijection) and so $f(\frac{t(a)}2)=\frac a2$ (else $f$ would not be surjective)

And so the functional equation $f(t(y)-x)+f(x)=y$ $\forall x,y$ (assertion $P(x,y)$)

$P(0,x)$ $\implies$ $f(t(x))=x-f(0)$ and so $t(x)$ is a bijection (since $f(x)$ is)
$P(x,t^{-1}(x+y))$ $\implies$ $f(x)+f(y)=t^{-1}(x+y)$
$P(x+y,t^{-1}(x+y))$ $\implies$ $f(x+y)+f(0)=t^{-1}(x+y)$

And so $f(x)+f(y)=f(x+y)+f(0)$

Consider then $g(x)=x^2$ and $f^{-1}(f(x)^2)=ax^2+bx+c$.
This implies $f(x)^2=f(ax^2+bx+c)$ and so $f(x)\ge 0$ on some non empty open segment $(u,v)$

The conjonction $f(x)+f(y)=f(x+y)+f(0)$ and $f(x)$ lower bounded on some non empty open interval is a classical condition of continuity in Cauchy's equation and implies $f(x)=ax+b$ for some $a,b$

And it's easy to check back that any of these function is indeed a solution as soon as $a\ne 0$

Hence the answer : $\boxed{f(x)=ax+b}$ for any $a\ne 0$
\end{solution}
*******************************************************************************
-------------------------------------------------------------------------------

\begin{problem}[Posted by \href{https://artofproblemsolving.com/community/user/74529}{bigbang195}]
	Find all continuous functions $f: \mathbb R \to \mathbb R$ which satisfy for all $x, y \in \mathbb R$,
\[(x+y)(f(x)-f(y))=(x-y)f(x+y).\]
	\flushright \href{https://artofproblemsolving.com/community/c6h394338}{(Link to AoPS)}
\end{problem}



\begin{solution}[by \href{https://artofproblemsolving.com/community/user/29428}{pco}]
	\begin{tcolorbox}Find the $f:\mathbb{R} \to \mathbb{R}$ such that $f$ is a continuous function and satisfy :
$(x+y)(f(x)-f(y))=(x-y)f(x+y)$\end{tcolorbox}
Let $P(x,y)$ be the assertion $(x+y)(f(x)-f(y))=(x-y)f(x+y)$

$P(1,0)$ $\implies$ $f(0)=0$

Let $x\ne 0$ :
$P(x-1,x+1)$ $\implies$ $f(x-1)-f(x+1)=-\frac{f(2x)}x$

$P(x+1,2-x)$ $\implies$ $f(x+1)-f(2-x)=\frac {2x-1}{3}f(3)$

$P(2-x,x-1)$ $\implies$ $f(2-x)-f(x-1)=(3-2x)f(1)$

Adding these three lines, we get $f(2x)=x^2\left(\frac{2f(3)}3-2f(1)\right)+x\left(3f(1)-\frac{f(3)}3\right)$ and so $f(x)=ax^2+bx$ for some $a,b$ $\forall x\ne 0$

This equality is still true for $x=0$ and, plugging back in original equation, we see that this indeed is a solution whatever are $a,b$

Hence the answer : $\boxed{f(x)=ax^2+bx}$ $\forall x$, for any real $a,b$

And, btw, no need for continuity constraint.
\end{solution}



\begin{solution}[by \href{https://artofproblemsolving.com/community/user/74529}{bigbang195}]
	\begin{tcolorbox}[quote="bigbang195"]Find the $f:\mathbb{R} \to \mathbb{R}$ such that $f$ is a continuous function and satisfy :
$(x+y)(f(x)-f(y))=(x-y)f(x+y)$\end{tcolorbox}
Let $P(x,y)$ be the assertion $(x+y)(f(x)-f(y))=(x-y)f(x+y)$

$P(1,0)$ $\implies$ $f(0)=0$

Let $x\ne 0$ :
$P(x-1,x+1)$ $\implies$ $f(x-1)-f(x+1)=-\frac{f(2x)}x$

$P(x+1,2-x)$ $\implies$ $f(x+1)-f(2-x)=\frac {2x-1}{3}f(3)$

$P(2-x,x-1)$ $\implies$ $f(2-x)-f(x-1)=(3-2x)f(1)$

Adding these three lines, we get $f(2x)=x^2\left(\frac{2f(3)}3-2f(1)\right)+x\left(3f(1)-\frac{f(3)}3\right)$ and so $f(x)=ax^2+bx$ for some $a,b$ $\forall x\ne 0$

This equality is still true for $x=0$ and, plugging back in original equation, we see that this indeed is a solution whatever are $a,b$

Hence the answer : $\boxed{f(x)=ax^2+bx}$ $\forall x$, for any real $a,b$

And, btw, no need for continuity constraint.\end{tcolorbox}

thanks you very much,pco.
I do not understand what made you think about moving on two variable equations functions  to a variable equation function and then use it to complete a great way, Can you share secrets  :D
\end{solution}



\begin{solution}[by \href{https://artofproblemsolving.com/community/user/29428}{pco}]
	\begin{tcolorbox} thanks you very much,pco.
I do not understand what made you think about moving on two variable equations functions  to a variable equation function and then use it to complete a great way, Can you share secrets  :D\end{tcolorbox}
It's a rather classical test to do when you have $f(x)-f(y)=something$ :
Add the three lines built with $x,y$ then $y,z$ then $z,x$

In our case $f(x)-f(y)=\frac{x-y}{x+y}f(x+y)$ (I dont care in this explanation about zeroes values)

So $\frac{x-y}{x+y}f(x+y)+\frac{y-z}{y+z}f(y+z)+\frac{z-x}{z+x}f(z+x)=0$

It remains to make one of the $f(x)$ variable and the other constants :

$x+y=u$
$y+z=a$
$z+x=b$

And so :
$x=\frac {u+b-a}2$

$y=\frac{u+a-b}2$

$z=\frac {a+b-u}2$

And choose $a,b,u$ in order to avoid fractions :) and simplify writing.
\end{solution}
*******************************************************************************
-------------------------------------------------------------------------------

\begin{problem}[Posted by \href{https://artofproblemsolving.com/community/user/27047}{mathlink}]
	Find all functions $f: \mathbb R \to \mathbb R$ which satisfy
\[[f(x)-f(y)]^2=f(f(x))-2x^2f(y)+f(y^2), \quad \forall x,y \in \mathbb R.\]
	\flushright \href{https://artofproblemsolving.com/community/c6h395730}{(Link to AoPS)}
\end{problem}



\begin{solution}[by \href{https://artofproblemsolving.com/community/user/29428}{pco}]
	\begin{tcolorbox}Find all funtion $f:R \to R$ such that: $[f(x)-f(y)]^2=f(f(x))-2x^2f(y)+f(y^2) \forall x,y \in R$

One solution is $f(x)=x^2$\end{tcolorbox}
Let $P(x,y)$ be the assertion $(f(x)-f(y))^2=f(f(x))-2x^2f(y)+f(y^2)$

$P(0,y)$ $\implies$ $(f(0)-f(y))^2=f(f(0))+f(y^2)$

Subtracting this equation from $P(x,y)$, we get :
$f(x)^2-f(0)^2-f(f(x))+f(f(0))=(2f(x)-2f(0)-2x^2)f(y)$

If $2f(x)-2f(0)-2x^2\ne 0$ for some $x$, we get $f(y)=c$ and plugging in the original equation, we get $c=0$

If $2f(x)-2f(0)-2x^2=0$ $\forall x$, we get $f(x)=x^2+f(0)$ and, plugging in original equation, we get $f(0)^2+2f(0)=0$

\begin{bolded}And so the three solutions \end{bolded}\end{underlined}:
$f(x)=0$ $\forall x$
$f(x)=x^2$ $\forall x$
$f(x)=x^2-2$ $\forall x$
\end{solution}



\begin{solution}[by \href{https://artofproblemsolving.com/community/user/35756}{m.candales}]
	$[f(x)-f(y)]^2 = f(f(x)-2x^2f(y)+f(y^2)$
	$x=0,y=0: f(f(0)) = -f(0)$
	$x=0:\ \ f(y^2) = [f(y)-f(0)]^2-f(f(0)) = [f(y)-f(0)]^2 + f(0)\ (*)$
	$y=0:\ \ f(f(x)) = [f(x)-f(0)]^2 + 2x^2f(0) - f(0)\ (**)$ 
	Substituting (*) and (**) in the original equation we get:
	$[f(y)-f(0)][f(x)-x^2-f(0)] = 0$
	Then $f(x) = f(0)\ \forall x\in\mathbb{R}$ or $f(x) = x^2+f(0)\ \forall x\in\mathbb{R}$
	In the first case we have $-f(0) = f(f(0)) = f(0)\Rightarrow f(0) = 0\Rightarrow f(x) = 0\ \forall x\in\mathbb{R}$
	In the second case we have $f(0)^2 + 2f(0) = 0$ (by substituting $f(x) = x^2+f(0)$ in the original equation)
	Then $f(0) = 0$ or $-2$. Then $f(x) = x^2$ or $f(x) = x^2-2$
\end{solution}
*******************************************************************************
-------------------------------------------------------------------------------

\begin{problem}[Posted by \href{https://artofproblemsolving.com/community/user/88788}{phalkun}]
	Let $f(x)=\frac{x^{2}+2}{2x+1}$. Compute $f^n(x)$, which is the composition of $f$ with itself $n$ times.
	\flushright \href{https://artofproblemsolving.com/community/c6h395886}{(Link to AoPS)}
\end{problem}



\begin{solution}[by \href{https://artofproblemsolving.com/community/user/64716}{mavropnevma}]
	Take $g(x) = \dfrac {x+2} {x-1}$. It is easily checked that $g(g(x)) = x$ and $g(f(x)) = g(x)^2$. By simple induction then $g(F_n(x)) = g(x)^{2^n}$, and so $F_n(x) = g\left (g(x)^{2^n}\right )$.
\end{solution}



\begin{solution}[by \href{https://artofproblemsolving.com/community/user/29428}{pco}]
	\begin{tcolorbox}Let $f(x)=\frac{x^{2}+2}{2x+1}$
Compute $F_{n}(x)=f(f(f(....f(f(x))....) ) ) $, where there are $n$ pairs of brackets.\end{tcolorbox}
Let $h(x)=\frac{x^2+1}{2x}$ and $g(x)=\frac {2x+1}{3}$ so that $f(x)=g^{-1}(h(g(x))$ and so $f^n(x)=g^{-1}(h^n(g(x))$

Writing $h^n(x)=\frac {P_n(x)}{Q_n(x)}$, we get $\frac{P_{n+1}(x)}{Q_{n+1}(x)}=\frac{P_n(x)^2+Q_n(x)^2}{2P_n(x)Q_n(x)}$

Choosing then  $P_{n+1}(x)=P_n(x)^2+Q_n(x)^2$ and $Q_{n+1}(x)=2P_n(x)Q_n(x)$ we get :

$P_{n+1}-Q_{n+1}=(P_n-Q_n)^2=(P_0-Q_0)^{2^n}$ $=(x-1)^{2^n}$ and $P_{n+1}+Q_{n+1}=(P_n+Q_n)^2$ $=(P_0+Q_0)^{2^n}=(x+1)^{2^n}$

And so $h^n(x)=\frac{(x+2)^{2^n}+(x-1)^{2^n}}{(x+2)^{2^n}-(x-1)^{2^n}}$

and so $\boxed{f^n(x)=\frac 32\frac{(x+2)^{2^n}+(x-1)^{2^n}}{(x+2)^{2^n}-(x-1)^{2^n}}-\frac 12}$
\end{solution}



\begin{solution}[by \href{https://artofproblemsolving.com/community/user/29428}{pco}]
	\begin{tcolorbox}Take $g(x) = \dfrac {x+2} {x-1}$. It is easily checked that $g(g(x)) = x$ and $g(f(x)) = g(x)^2$. By simple induction then $g(F_n(x)) = g(x)^{2^n}$, and so $F_n(x) = g\left (g(x)^{2^n}\right )$.\end{tcolorbox}
Hummmmf ! Quite nice !
:) :oops:
\end{solution}
*******************************************************************************
-------------------------------------------------------------------------------

\begin{problem}[Posted by \href{https://artofproblemsolving.com/community/user/51470}{Potla}]
	Suppose $f:\mathbb{R} \to \mathbb{R}$ is a function such that 
\[|f(x+y)-f(x)-f(y)|\le 1\ \ \ \text{for all} \ \  x, y \in\mathbb R.\]
Prove that there is a function $g:\mathbb{R}\to\mathbb{R}$ such that $|f(x)-g(x)|\le 1$ and $g(x+y)=g(x)+g(y)$ for all $x,y \in\mathbb R.$
	\flushright \href{https://artofproblemsolving.com/community/c6h396244}{(Link to AoPS)}
\end{problem}



\begin{solution}[by \href{https://artofproblemsolving.com/community/user/29428}{pco}]
	\begin{tcolorbox}Suppose $f:\mathbb{R} \to \mathbb{R}$ is a function such that 
\[|f(x+y)-f(x)-f(y)|\le 1\ \ \ \text{for all} \ \  x, y \in\mathbb R.\]
Prove that there is a function $g:\mathbb{R}\to\mathbb{R}$ such that $|f(x)-g(x)|\le 1$ and $g(x+y)=g(x)+g(y)$ for all $x,y \in\mathbb R.$\end{tcolorbox}
Let $x\in\mathbb R$ and the sequence $a_n=2^{-n}f(2^nx)$

We get $|f(2^nx+2^nx)-2f(2^nx)|\le 1$ and so $|a_{n+1}-a_n|\le 2^{-n-1}$

So, considering $n>p$ : $|a_n-a_p|\le 2^{-n-1}+2^{-n}+...+2^{-p-1}<2^{-p}$ And so $a_n$ is a Cauchy sequence and so is convergent.

Let then $g(x)=\lim_{n\to +\infty}2^{-n}f(2^nx)$

$|f(2^n(x+y))-f(2^nx)-f(2^ny)|\le 1$ and so $|2^{-n}f(2^n(x+y))-2^{-n}f(2^nx)-2^{-n}f(2^ny)|\le 2^{-n}$
Setting $n\to+\infty$ in this inequality, we get $g(x+y)=g(x)+g(y)$

From $|a_n-a_p|<2^{-p}$, we also get $|a_n-a_0|<1$ and so $|a_n-f(x)|<1$
Setting $n\to+\infty$ in this inequality, we get $|g(x)-f(x)|\le 1$
Q.E.D.
\end{solution}
*******************************************************************************
-------------------------------------------------------------------------------

\begin{problem}[Posted by \href{https://artofproblemsolving.com/community/user/64716}{mavropnevma}]
	I cannot resist the temptation of posting this nice problem (Ilya Bogdanov told me it was used in an edition of the Kolmogorov Cup; I know two essentially different solutions).

[color=#0000FF]Prove there exists a (strictly) increasing real function $f : \mathbb{R} \to \mathbb{R} \setminus \mathbb{Q}$.[\/color]
	\flushright \href{https://artofproblemsolving.com/community/c6h396641}{(Link to AoPS)}
\end{problem}



\begin{solution}[by \href{https://artofproblemsolving.com/community/user/43269}{JoeBlow}]
	Define $g:[0,1)\to [0,1)\setminus \mathbb{Q}$ as follows: For $x=\sum_{k=1}^\infty a_k2^{-k}$ where $a_k \in \lbrace 0,1 \rbrace$ and $\lim_{k\to \infty}a_k\neq 1$, define $g(x)=\sum_{k=1}^\infty (a_k+1)2^{-(k+1)^2}$.  For any $x$ the binary representation of $g(x)$ never becomes periodic (*), so $g(x)$ is irrational.  Moreover, $g(x)>g(y)$ if and only if $x>y$.  Taking $f(x)=\lfloor x \rfloor + g\left(\lbrace x \rbrace \right)$ gives a strictly increasing function from $\mathbb{R}$ to $\mathbb{R}\setminus \mathbb{Q}$ (where $\lbrace \cdot \rbrace$ denotes the fractional part).

(*) More precisely: For every pair $k,N\in \mathbb{N}$, there exists an $n>N$ such that $a_{n+k}\neq a_n$.
\end{solution}



\begin{solution}[by \href{https://artofproblemsolving.com/community/user/64716}{mavropnevma}]
	The cute solution. Enumerate $\mathbb{Q}$ as $\{r_1,r_2,\ldots,r_n,\ldots \}$. Define $f(x) = \sum_{r_n < x} \dfrac {1} {10^{n!}}$. Clearly $f$ is increasing (and discontinuous at all rational numbers). Not only $f(x)$ is irrational at all real numbers (no periodicity), but it takes transcendental values, by the Liouville criterion!
\end{solution}



\begin{solution}[by \href{https://artofproblemsolving.com/community/user/102382}{my_name_is_math}]
	sorry, but how can you enumerate Q since betwen any  two elements there exist an infinity more???
\end{solution}



\begin{solution}[by \href{https://artofproblemsolving.com/community/user/29428}{pco}]
	\begin{tcolorbox}sorry, but how can you enumerate Q since betwen any  two elements there exist an infinity more???\end{tcolorbox}
$\mathbb Q$ is countable, and so we always can enumerate $\mathbb Q$

There are a lot of bijections between $\mathbb N$ and $\mathbb Q$

For example \end{underlined}:
Let $f(n)$ a bijection from $\mathbb N\to \mathbb Z^*$ (for example $f(n)=(-1)^{n+1}\left\lfloor\frac{n+1}2\right\rfloor$)

Let $g(n)$ from $\mathbb Z\to\mathbb Q$ defined as :
$g(0)=0$
$g(\prod p_i^{n_i})=\prod p_i^{f(n_i)}$ where $p_i$ are primes and $n_i\in\mathbb Z$
$g(-\prod p_i^{n_i})=-\prod p_i^{f(n_i)}$ where $p_i$ are primes and $n_i\in\mathbb Z$
$g(n)$ is a bijection from $\mathbb Z\to\mathbb Q$

We then just have to add a bijection $h(n)$ from $\mathbb N\to\mathbb Z$ (for example $h(n)=(-1)^{n+1}\left\lfloor\frac{n}2\right\rfloor$)

and $g(h(n))$ is a bijection from $\mathbb N\to\mathbb Q$

And a lot of other (this one is my preferred one)
\end{solution}
*******************************************************************************
-------------------------------------------------------------------------------

\begin{problem}[Posted by \href{https://artofproblemsolving.com/community/user/44083}{jgnr}]
	Given a positive integer $n\ge2$. Find all functions $f:\mathbb{R}\rightarrow\mathbb{R}$ such that \[f(x-f(y))=f(x+y^n)+f(f(y)+y^n)\] for any $x,y\in\mathbb{R}$.
	\flushright \href{https://artofproblemsolving.com/community/c6h398427}{(Link to AoPS)}
\end{problem}



\begin{solution}[by \href{https://artofproblemsolving.com/community/user/29428}{pco}]
	Hello Johan !

Could you give us your solution for this problem, please ?
I got the trivial solutions $f(x)=0$ and $f(x)=-x^n$  but I'm unable to find some other or to show there are no other  :oops:
\end{solution}



\begin{solution}[by \href{https://artofproblemsolving.com/community/user/44083}{jgnr}]
	[hide="please check my solution"]Let $H=\{f(x)+x^n|x\in\mathbb{R}\}$. The condition is equivalent to $f(x)=f(x+y)+f(y),\forall x\in\mathbb{R},y\in H$. Let $P(x,y)$ be the assertion that $f(x)=f(x+y)+f(y)$.

$P(0,x),x\in H\implies f(x)=\frac{f(0)}2$

Let $f(0)=2c$, then $P(x,y),x\in\mathbb{R},y\in H$ becomes $f(x)=f(x+y)+c$.

$P(x,x),x\in H\implies f(2x)=0$

So $x\in H\implies f(2x)+(2x)^n=(2x)^n\in H$.

Obviously $H$ is nonempty. If $H$ does not contain any nonzero element, then $\boxed{f(x)=-x^n\forall x\in\mathbb{R}}$, which is one solution. Assume $H$ contains a nonzero element $a$.

$P((x+a)^n-x^n,f(x)+x^n)\implies f((x+a)^n-x^n)=f((x+a)^n+f(x))+c$

But we also have $f(x)=f(x+a)+c$ (because $a\in H$), so $f((x+a)^n-x^n)=f((x+a)^n+f(x+a)+c)+c$

Since $(x+a)^n+f(x+a)\in H$, then $f((x+a)^n+f(x+a)+c)=f(c)-c$, therefore $f((x+a)^n-x^n)=f(c)$, which is a constant value. Note that $(x+a)^n-x^n$ attains all real numbers on an infinite interval.

Now $S$ clearly contains infinitely many elements. Let $b\in H$ such that $b\ne(2b)^n$, which clearly exists. We have $P(x,b),P(x,(2b)^n)\implies f(x+b)=f(x+(2b)^n)$, so $f$ is periodic. Since $f$ is periodic and constant on an infinite interval, then it must be constant. We can easily check that $\boxed{f(x)=0\forall x\in\mathbb{R}}$ is the only constant function satisfying the given conditions.[\/hide]
\end{solution}



\begin{solution}[by \href{https://artofproblemsolving.com/community/user/29428}{pco}]
	\begin{tcolorbox}please check my solution \end{tcolorbox}
Thanks for your answer.

According to me, this is quite OK (except a typo : $S$ instead of $H$ at one place) and quite nice. :)

I particularly enjoyed the part where you established that $f((x+a)^n-x^n)$ is constant. This was the missing  - major -  piece in my own work.

Congrats !
\end{solution}



\begin{solution}[by \href{https://artofproblemsolving.com/community/user/93837}{jjax}]
	let $P(x,y)$ be the proposition that $f(x-f(y)) = f(x+y^n)+f(f(y)+y^n)$
Let $f(0)=a$.
$P(f(x),x): f(x^n+f(x)) = a\/2$.
$P(x,0): f(x) = f(x-a)-a\/2$.
Applying this twice yields $f(y)+a = f(y-2a)$.
$P(x,y-2a): f(x-f(y-2a)) = f(x+(y-2a)^n) + a\/2$
$f(x-f(y)-a) = f(x+(y-2a)^n) + a\/2$
$f(x-f(y)) = f(x+(y-2a)^n)$
$P(x,y): f(x-f(y)) = f(x+y^n)+a\/2 = f(x+y^n-a)$
Thus $f(x+(y-2a)^n)=f(x+y^n-a)$.
If a is nonzero, then varying y shows that f is a constant function and thus the zero function.

If a is zero,$f(x-f(y)) = f(x+y^n)$.
Clearly if $f(k)+k^n$ is zero for all k, then we are done.
Else, $f(k)+k^n$ is nonzero for some k, and the function is periodic. (k is not zero)
$f(x+y^n)=f(x-f(y))=f(x-f(y+k)) = f(x+(y+k)^n)$.
Varying y shows that f is constant and thus zero.

Thus the solutions are $f(x)=0$ and $f(x) = -x^n$
\end{solution}



\begin{solution}[by \href{https://artofproblemsolving.com/community/user/67223}{Amir Hossein}]
	\begin{tcolorbox}
Could you give us your solution for this problem, please ?
\end{tcolorbox}

It is China TST 2011, Second Quiz, First Day, Problem 1. :)
\end{solution}
*******************************************************************************
-------------------------------------------------------------------------------

\begin{problem}[Posted by \href{https://artofproblemsolving.com/community/user/96252}{laurentblance}]
	1. Find all functions $f:\mathbb{R} \to \mathbb{R}$ continuous at $0$ such that \[f(x)=f \left( \frac{x}{2}\right)+\frac{x}{(x+1)(x+2)}\] holds for all reals $x$. 

2. Find all continuous functions $f: \mathbb R \to \mathbb R$ such that for all $x, y \in \mathbb R$, we have \[f(x+y)=f(x)f(y)-g(x)g(y)\] and \[ g(x+y)=f(x)g(y)+f(y)g(x).\]
	\flushright \href{https://artofproblemsolving.com/community/c6h398607}{(Link to AoPS)}
\end{problem}



\begin{solution}[by \href{https://artofproblemsolving.com/community/user/64716}{mavropnevma}]
	1. By iteration, we get $f(x) = f(x\/2^n) + x\left (\dfrac {1} {x+1} - \dfrac {1} {x+2^n} \right ) \to f(0) + \dfrac {x} {x+1}$.

2. Seems to have as solutions some pair of trigonometric functions (but there may be some sign error?).
\end{solution}



\begin{solution}[by \href{https://artofproblemsolving.com/community/user/96252}{laurentblance}]
	For the first, your right. 

Then for the second I made a sign error, you have to read :

"2 - Find all $(f,g)$ continuous such $\forall (x,y)\in \mathbb{R}^2, f(x+y)=f(x)f(y)-g(x)g(y) ; g(x+y)=f(x)g(y)+f(y)g(x)$."

Sorry for the double problem in only one post. And sorry for the section ...
\end{solution}



\begin{solution}[by \href{https://artofproblemsolving.com/community/user/64716}{mavropnevma}]
	Well, I guessed that, right? I think it is known the main solution is $(f,g) = (k\cos,k\sin)$.

EDIT. Right, pco! Wrong place for the constant ... That includes the constant pair $(1,0)$, but there is also the trivial solution $(0,0)$.
\end{solution}



\begin{solution}[by \href{https://artofproblemsolving.com/community/user/29428}{pco}]
	\begin{tcolorbox}Well, I guessed that, right? I think it is known the solution is $(f,g) = (k\cos,k\sin)$.\end{tcolorbox}

Not exactly : $(f,g) = (\cos kx, \sin kx)$ :)
\end{solution}



\begin{solution}[by \href{https://artofproblemsolving.com/community/user/96252}{laurentblance}]
	Yes, and what is your method ? Do you use D'allambert equation ?
\end{solution}



\begin{solution}[by \href{https://artofproblemsolving.com/community/user/29428}{pco}]
	\begin{tcolorbox}That includes the constant pair $(1,0)$, but there is also the trivial solution $(0,0)$.\end{tcolorbox}
And also $(e^{ax},0)$
\end{solution}



\begin{solution}[by \href{https://artofproblemsolving.com/community/user/29428}{pco}]
	\begin{bolded}The full set of solutions is \end{bolded}\end{underlined}:

$S0$ : $f(x)=g(x)=0$ $\forall x$
$S1$ : $f(x)=e^{ax}$ and $g(x)=0$ $\forall x$
$S2$ : $f(x)=e^{ax}\cos bx$ and $g(x)=e^{ax}\sin bx$ $\forall x$
$S3$ : $f(x)=e^{ax}\cos bx$ and $g(x)=-e^{ax}\sin bx$ $\forall x$

[hide="(long) proof"]It's immediate to get from these two equations that $f(x+y)^2+g(x+y)^2=(f(x)^2+g(x)^2)$ $(f(y)^2+g(y)^2)$
So the function $h(x)=f(x)^2+g(x)^2$ is continuous and such that $h(x+y)=h(x)h(y)$
This is a well known functional equation whose only continuous solutions are $h(x)=0$ and $h(x)=e^{ux}$

If $h(x)=0$, we get the solution $f(x)=g(x)=0$ and solution $S0$ above.
If $h(x)=e^{ux}$, let then $f(x)=e^{\frac{ux}2}c(x)$ and $g(x)=e^{\frac{ux}2}s(x)$ and the system of equations becomes :

$c(x+y)=c(x)c(y)-s(x)s(y)$
$s(x+y)=c(x)s(y)+c(y)s(x)$
$c(x)^2+s(x)^2=1$

Let $P(x,y)$ be the assertion $c(x+y)=c(x)c(y)-s(x)s(y)$
Let $Q(x,y)$ be the assertion $s(x+y)=c(x)s(y)+c(y)s(x)$


$Q(0,0)$ $\implies$ $s(0)=2c(0)s(0)$
If $s(0)\ne 0$, we get $c(0)=\frac 12$ and then $P(0,0)$ $\implies$ $s(0)^2=-\frac 14$, impossible.

So $s(0)=0$ and $P(0,0)$ implies $c(0)=c(0)^2$ and so $c(0)=1$ since $c(0)^2+s(0)^2=1$

1) let us consider that $c(x)>0$ $\forall x$
$P(x,x)$ $\implies$ $c(2x)=c(x)^2-s(x)^2=2c(x)^2-1$
Notice that $c(x)^2+s(x)^2$ implies $|c(x)\le 1$
If $|a_0|<1$, the sequence $a_{n+1}=2a_n^2-1$ always have some negative terms.

So $|c(x)|=1$ and continuity gives $c(x)=1$ and $s(x)=0$ $\forall x$ which indeed is a solution and gives the solution $S1$ above for original problem.

2) Let us consider that $c(x)\le 0$ for some $x$
Then, since $c(0)=1$ and $c(x)$ is continuous, $\exists b\ne 0$ such that $c(b)=0$

Then $s(b)=\pm 1$ and $Q(b,-b)$ $\implies$ $c(-b)=0$

Let then $A=\{x\ge 0$ such that $c(x)=0\}$.
This set is non empty since $|b|\in A$ and so $\exists a=\inf A$
Continuity of $c(x)$ implies $c(a)=0$ and so $a\ne 0$

So $c(0)=1$ and $c(a)=0$ and $c(x)>0$ $\forall x\in[0,a)$

$c(a)=0$ $\implies$ $s(a)=\pm 1$
Since $(c(x),s(x))$ solution implies $(c(x),-s(x))$ solution too, wlog say $s(a)=1$

If $c(t)=1$ for some $t\in(0,a)$, then $P\frac t2,\frac t2)$ $\implies$ $c(\frac t2)=1$ and so $c(\frac t{2^n})=1$
But $P(x,t)$ $\implies$ $c(x+t)=c(x)$ $\forall x$ and so $c(x)$ is periodic with periods as little as we want, and so is constant (since continuous), in contradiction with $c(a)=0$

So $c(x)\in(0,1)$ $\forall x\in(0,a)$ and $s(x)\in(0,1)$ $\forall x\in(0,a)$

From there, using $c(2x)=2c(x)^2-1$ and starting from $c(a)=\cos\frac{\pi}2$, we get $c(\frac a{2^n})=\cos\frac{\pi}{2^{n+1}}$ and $s(\frac a{2^n})=\sin\frac{\pi}{2^{n+1}}$

Using then $c(x+y)=c(x)c(y)-s(x)s(y)$, we get $c(\frac {ma}{2^n})=\cos\frac{m\pi}{2^{n+1}}$ for any $m\le 2^n$
And continuity gives $c(x)=\cos \frac{\pi x}{2a}$ and $s(x)=\sin \frac{\pi x}{2a}$ $\forall x\in[0,a]$

Then $P(x,a)$ and $Q(x,a)$ $\implies$ $c(x)=\cos \frac{\pi x}{2a}$ and $s(x)=\sin \frac{\pi x}{2a}$ $\forall x\in[0,2a]$

So $c(2a)=-1$ and $s(2a)=0$ and :
$P(x,2a)$ $\implies$ $c(x+2a)=-c(x)$
$Q(x,2a)$ $\implies$ $s(x+2a)=-s(x)$

And so $c(x)=\cos \frac{\pi x}{2a}$ and $s(x)=\sin \frac{\pi x}{2a}$ $\forall x$
And this gives the two solutions $S_2$ and $S_3$

\begin{bolded}The end\end{bolded}\end{underlined}
[\/hide]
\end{solution}



\begin{solution}[by \href{https://artofproblemsolving.com/community/user/96252}{laurentblance}]
	Thank you pco :)
\end{solution}
*******************************************************************************
-------------------------------------------------------------------------------

\begin{problem}[Posted by \href{https://artofproblemsolving.com/community/user/94813}{gobi}]
	Find all functions $f:\mathbb{R} \rightarrow \mathbb{R}$ such that $f(x+f(y))=f(x)+2xy^2 +y^2 f(y) $ for all $x,y \in \mathbb{R} $.
	\flushright \href{https://artofproblemsolving.com/community/c6h399024}{(Link to AoPS)}
\end{problem}



\begin{solution}[by \href{https://artofproblemsolving.com/community/user/29428}{pco}]
	\begin{tcolorbox}Find all functions $f:\mathbb{R} \rightarrow \mathbb{R}$ such that $f(x+f(y))=f(x)+2xy^2 +y^2 f(y) $ for all $x,y \in \mathbb{R} $\end{tcolorbox}
Let $P(x,y)$ be the assertion $f(x+f(y))=f(x)+2xy^2+y^2f(y)$
Let $f(0)=a$

$P(0,0)$ $\implies$ $f(a)=a$
$P(0,a)$ $\implies$ $a=a+a^3$
and so $a=0$

(a) : $P(f(x),1)$ $\implies$ $f(f(x)+f(1))=f(f(x))+2f(x)+f(1)$
(b) : $P(f(1),x)$ $\implies$ $f(f(1)+f(x))=f(f(1))+2f(1)x^2+x^2f(x)$
(c) : $P(0,x)$ $\implies$ $f(f(x))=x^2f(x)$
(d) : $P(0,1)$ $\implies$ $f(f(1))=f(1)$
(a)-(b)+(c)-(d) : $f(x)=f(1)x^2$

And plugging this back in original equation, we get $f(1)^2=1$

Hence the two solutions :
$f(x)=x^2$
$f(x)=-x^2$
\end{solution}
*******************************************************************************
-------------------------------------------------------------------------------

\begin{problem}[Posted by \href{https://artofproblemsolving.com/community/user/86021}{Headhunter}]
	Find all functions $f: \mathbb R \to \mathbb R$ which satisfy for all $x, y \in \mathbb R $,
\[f(x+y)f( f(x)-y )=xf(x)-yf(y).\]
	\flushright \href{https://artofproblemsolving.com/community/c6h399531}{(Link to AoPS)}
\end{problem}



\begin{solution}[by \href{https://artofproblemsolving.com/community/user/29428}{pco}]
	\begin{tcolorbox}Hello.

Find $f(x)$ $~$ satisfying that $~$ $f(x+y)f( f(x)-y )=xf(x)-yf(y)$ $~$ for all real numbers $x$, $y$\end{tcolorbox}
Let $P(x,y)$ be the assertion $f(x+y)f(f(x)-y)=xf(x)-yf(y)$

1) If $f(x_1)=f(x_2)\ne 0$, then $x_1=x_2$
========================
$P(x_1,0)$ $\implies$ $f(x_1)f(f(x_1))=x_1f(x_1)$ $\implies$ $f(f(x_1))=x_1$
$P(x_2,0)$ $\implies$ $f(x_2)f(f(x_2))=x_2f(x_2)$ $\implies$ $f(f(x_2))=x_2$
And so, since $f(x_1)=f(x_2)$, we get $f(f(x_1))=f(f(x_2))$ and so $x_1=x_2$
Q.E.D.

2) If $\exists u\ne 0$ such that $f(u)\ne 0$
==========================
2.1) it exists a unique $v$ such that $f(v)=0$
----------------------------------------------
$P(0,0)$ $\implies$ $f(0)f(f(0))=0$ and so it exists at least one $v$ such that $f(v)=0$

$P(x,0)$ $\implies$ $f(x)f(f(x))=xf(x)$
$P(x,f(x))$ $\implies$ $f(0)f(f(x)+x)=xf(x)-f(x)f(f(x))$ $=0$ (using previous line)
$P(x,-x)$ $\implies$ $f(0)f(f(x)+x)=xf(x)+xf(-x)$
Comparing these two lines, we get $x(f(x)+f(-x))=0$ and so $f(x)+f(-x)=0$ $\forall x\ne 0$

So $f(-u)=-f(u)$

If $f(v)=0$ for some $v$, then :
$P(v,u)$ $\implies$ $f(v+u)f(-u)=-uf(u)$ $\implies$ $f(v+u)=u$
So, if $f(v_1)=f(v_2)=0$, we get $f(v_1+u)=f(v_2+u)\ne 0$ and so, using 1) : $v_1=v_2$
Q.E.D.

2.2) $f(x)=x$ $\forall x$
-----------------------
$P(x,0)$ $\implies$ $f(x)f(f(x))=xf(x)$
$P(x,f(x)-x)$ $\implies$ $f(f(x))f(x)=xf(x)-(f(x)-x)f(f(x)-x)$
Comparing these two lines, we get $(f(x)-x)f(f(x)-x)=0$ and so :
either $f(x)-x=0$
either $f(f(x)-x)=0$ and so $f(x)-x=v$
And so : $\forall x$ : either $f(x)=x$, either $f(x)=x+v$

Let then $x$ such that $f(x)=x+v$
If $x=v$, this means $0=2v$ and so $v=0$
If $x\ne v$, then $P(x,0)$ $\implies$ $f(x)f(f(x))=xf(x)$ and so $f(f(x))=x$ and so $f(x+v)=x$
But $f(x+v)=x+v$ or $f(x+v)=(x+v)+v$ and so, again, $v=0$

3) If $f(x)=0$ $\forall x\ne 0$
=================
$xf(x)=yf(y)=0$ $\forall x,y$ and $P(x,y)$ is $f(x+y)f(f(x)+y)=0$
If $x,y\ne 0$ $f(x)=0$ and $f(x)+y\ne 0$ and so $f(f(x)+y)=0$ and the equation is OK
If $x\ne 0$ and $y=0$, then $x+y\ne 0$ and so $f(x+y)=0$ and the equation is OK
If $x=0$ and $y\ne 0$, then $x+y\ne 0$ and so $f(x+y)=0$ and the equation is OK
If $x=y=0$ and $f(0)\ne 0$, then $f(f(x)+y)=f(f(0))=0$ and the equation is OK
Hence we got a solution whatever is $f(0)$

4) synthesis of solutions
=================
We got two solutions :
$f(x)=x$ $\forall x$
$f(0)=a$ any real and $f(x)=0$ $\forall x\ne 0$
\end{solution}



\begin{solution}[by \href{https://artofproblemsolving.com/community/user/92964}{dyta}]
	[url]http://www.artofproblemsolving.com/Forum/viewtopic.php?f=36&t=379993[\/url]
\end{solution}



\begin{solution}[by \href{https://artofproblemsolving.com/community/user/29428}{pco}]
	\begin{tcolorbox}[url]http://www.artofproblemsolving.com/Forum/viewtopic.php?f=36&t=379993[\/url]\end{tcolorbox}
Sorry, dyta, but these are different problems (with different solutions)

Current problem is $f(x+y)f( f(x)-y)=xf(x)-yf(y)$
 
Pointed problem is $f(x+y)(f(f(x))-y)=xf(x)-yf(y)$

second factor in LHS is different
\end{solution}



\begin{solution}[by \href{https://artofproblemsolving.com/community/user/86021}{Headhunter}]
	to pco.

Thanks for the great solution. I got it by your favor.
\end{solution}
*******************************************************************************
-------------------------------------------------------------------------------

\begin{problem}[Posted by \href{https://artofproblemsolving.com/community/user/87322}{paul1703}]
	Let $f$ be a continuous and injective functions $f: \mathbb R \to \mathbb R$ such that $f(1)=1$ and $f(2x-f(x))=x$ for all reals $x$. Prove that $f(x)=x$.
	\flushright \href{https://artofproblemsolving.com/community/c6h399730}{(Link to AoPS)}
\end{problem}



\begin{solution}[by \href{https://artofproblemsolving.com/community/user/29428}{pco}]
	\begin{tcolorbox}Let f be a contiuous and injective function $R$  $->$  $R$ ;  $f(1)=1$  ;          $f(2x-f(x))=x$.  Prove that $f(x)=x$.\end{tcolorbox}
So $f(x)$ is strictly monotonous.
If $f(x)$ is decreasing, then $2x-f(x)$ is increasing and $f(2x-f(x))$ is decreasing, which is wrong.

So $f(x)$ is increasing.

If $f(a)>a$, then $2a-f(a)<a$ and $f(2a-f(a))<f(a)$ and so $f(a)>a$, impossible
If $f(a)<a$, then $2a-f(a)>a$ and $f(2a-f(a))>f(a)$ and so $f(a)<a$, impossible

So $\boxed{f(x)=x}$ $\forall x$, which indeed is a solution.
\end{solution}



\begin{solution}[by \href{https://artofproblemsolving.com/community/user/73386}{mousavi}]
	\begin{tcolorbox}[quote="paul1703"]Let f be a contiuous and injective function $R$  $->$  $R$ ;  $f(1)=1$  ;          $f(2x-f(x))=x$.  Prove that $f(x)=x$.\end{tcolorbox}
So $f(x)$ is strictly monotonous.
If $f(x)$ is decreasing, then $2x-f(x)$ is increasing and $f(2x-f(x))$ is decreasing, which is wrong.

So $f(x)$ is increasing.

If $f(a)>a$, then $2a-f(a)<a$ and $f(2a-f(a))<f(a)$ and so $f(a)>a$, impossible
If $f(a)<a$, then $2a-f(a)>a$ and $f(2a-f(a))>f(a)$ and so $f(a)<a$, impossible

So $\boxed{f(x)=x}$ $\forall x$, which indeed is a solution.\end{tcolorbox}

for $f(a)<a$ why $f(2a-f(a))=a>f(a)$ is impossible?
\end{solution}



\begin{solution}[by \href{https://artofproblemsolving.com/community/user/29428}{pco}]
	\begin{tcolorbox}[quote="pco"][quote="paul1703"]Let f be a contiuous and injective function $R$  $->$  $R$ ;  $f(1)=1$  ;          $f(2x-f(x))=x$.  Prove that $f(x)=x$.\end{tcolorbox}
So $f(x)$ is strictly monotonous.
If $f(x)$ is decreasing, then $2x-f(x)$ is increasing and $f(2x-f(x))$ is decreasing, which is wrong.

So $f(x)$ is increasing.

If $f(a)>a$, then $2a-f(a)<a$ and $f(2a-f(a))<f(a)$ and so $f(a)>a$, impossible
If $f(a)<a$, then $2a-f(a)>a$ and $f(2a-f(a))>f(a)$ and so $f(a)<a$, impossible

So $\boxed{f(x)=x}$ $\forall x$, which indeed is a solution.\end{tcolorbox}

for $f(a)<a$ why $f(2a-f(a))=a>f(a)$ is impossible?\end{tcolorbox}

:oops: In fact both lines are completely wrong.
I'm sorry (too early in the morning here)
Thanks for the remark
\end{solution}



\begin{solution}[by \href{https://artofproblemsolving.com/community/user/29126}{MellowMelon}]
	The condition says that $f(\text{something}) = x$, so $f$ is surjective and hence bijective. Let $g$ be the inverse of $f$. The equation is equivalent to $f(x) + g(x) = 2x$.

Suppose $f(a) = a + d$ for $d > 0$. Then $f(a) + g(a) = 2a$ implies $f(a-d) = a$, and likewise $f(a+d) = g(a+d) = 2(a+d)$ implies $f(a+d) = a+2d$. Applying these two repeatedly shows $f(a+kd) = a + (k+1)d$ for all integers $k$. Take the maximal integer $k$ (possibly negative) such that $a + kd < 1$, so $a + (k+1)d \geq 1$. Then $f(a+kd) \geq 1 = f(1)$, but $a+kd < 1$, so this is a contradiction of $f$ strictly increasing.

Note that if $f$ is a solution to the equation, then its inverse $g$ also works. So if $f(a) = a-d$ for $d > 0$, then $g(a-d) = a$, and we can apply the exact same argument above to get a contradiction.

So the only possibility is that $f(x) = x$ for all $x$; otherwise we'd fall into one of the (impossible) cases above.
\end{solution}
*******************************************************************************
-------------------------------------------------------------------------------

\begin{problem}[Posted by \href{https://artofproblemsolving.com/community/user/86345}{namdan}]
	Find all $f:\mathbb{R}\rightarrow \mathbb{R}$ which satisfy
\[f(x^5+f(y))=y+(f(x))^5\]
for all $x, y \in \mathbb R$.
	\flushright \href{https://artofproblemsolving.com/community/c6h399997}{(Link to AoPS)}
\end{problem}



\begin{solution}[by \href{https://artofproblemsolving.com/community/user/29428}{pco}]
	\begin{tcolorbox}Find all $f:\mathbb{R}\rightarrow \mathbb{R}$ satisfy:
$f(x^5+f(y))=y+(f(x))^5$\end{tcolorbox}
Let $P(x,y)$ be the assertion $f(x^5+f(y))=y+f(x)^5$
$f(x)$ is obviously bijective and let then $a,u$ such that :
$f(0)=a$
$f(y)=0$

$P(0,x)$ $\implies$ $f(f(x))=x+a^5$
$P(x,u)$ $\implies$ $f(x^5)=f(x)^5+u$
And so $P(x,f(y))$ may be written $f(x^5+y+a^5)=f(x^5)+f(y)-u$ and so $f((x+a^5)+(y+a^5)-a^5)-u$ $=f((x+a^5)-a^5)-u$ $+f((y+a^5)-a^5)-u$

Setting then $f(x)=g(x+a^5)+u$, we get $g(x+y)=g(x)+g(y)$

So $f(x)=g(x)+g(a^5)+u$ where $g(x)$ is solution of Cauchy's equation.

Setting $x=0$, we get $g(a^5)+u=a$ and $f(x)=g(x)+a$

$P(0,x)$ $\implies$ $f(f(x))=x+a^5$ which becomes $g(g(x))=x+a^5-g(a)-a$


and since $g(g(x+y))=g(g(x))+g(g(y))$, we get $a^5-g(a)-a=0$ and $g(g(x))=x$
So $g(a)=g(g(0))=0$ and $f(x)=g(x)+a$ implies $f(a)=a$ and so $f(f(0))=f(0)$ and so, since injective, $f(0)=0$ and $a=u=0$

So $f(x)=g(x)$ is additive and $f(0)=0$ and then :
$P(x,0)$ $\implies$ $f(f(x))=x$
$P(0,x)$ $\implies$ $f(x^5)=f(x)^5$

Notice then that $f(x)$ solution implies $-f(x)$ solution
Since $f(1)\ne 0$ (bijection) and $f(1)=f(1)^5$, wlog say then $f(1)=1$

Replacing $x$ with $x+py$ where $p\in\mathbb Q$ in this last equation and using $f(py)=pf(y)$, we get :

$f(x^5)+5pf(yx^4)+10p^2f(y^2x^3)+10p^3f(y^3x^2)+5p^4f(y^4x)+p^5f(y^5)$ $=f(x)^5+5pf(y)f(x)^4$ $+10p^2f(y)^2f(x)^4+$ $10p^3f(y)^3f(x)^2+5p^4f(y)^4f(x)$ $+p^5f(y)^5$

And this in fact is a polynomial in $p$ which is zero for all rational $p$ and so which is the zero polynomial.

Looking then at the coefficient of different degrees, we get :
$f(x^5)=f(x)^5$ (we already knew this)
$f(x^4y)=f(x)^4f(y)$
$f(x^3y^2)=f(x)^3f(y)^2$
Using $f(1)=1$ and $f(-1)=-1$ we get $f(xy)=f(x)f(y)$ $\forall x,y$

And the system $f(x+y)=f(x)+f(y)$ plus $f(xy)=f(x)f(y)$ plus $f(1)=1$ is simple and well known and gives $f(x)=x$

Hence the two solutions (which indeed are solutions) :
$f(x)=x$ $\forall x$
$f(x)=-x$ $\forall x$
\end{solution}



\begin{solution}[by \href{https://artofproblemsolving.com/community/user/86345}{namdan}]
	Thanks you very much, pco. :)
\end{solution}
*******************************************************************************
-------------------------------------------------------------------------------

\begin{problem}[Posted by \href{https://artofproblemsolving.com/community/user/86345}{namdan}]
	Find all functions $f:\mathbb{Q}\to\mathbb{Q}$ which satisfy
\[f(x+y)+f(x-y)=2f(x)+2f(y),\quad \forall x, y\in \mathbb{Q}.\]
	\flushright \href{https://artofproblemsolving.com/community/c6h400122}{(Link to AoPS)}
\end{problem}



\begin{solution}[by \href{https://artofproblemsolving.com/community/user/31915}{Batominovski}]
	Sketch of Proof:
1) $x,y:=0$ gives $f(0)=0$.
2) $x:=0$ gives $f(+y)=f(-y)$.
3) $y:=x$ gives $f(2x)=2^2f(x)$.
4) Induction on $n \in \mathbb{N}$ to show $f(nx)=n^2f(x)$.
5) From 4), if $m,n \in \mathbb{Z}$, $n \neq 0$, $f\left(\left(\frac{m}{n}\right)x\right) = m^2f\left(\frac{x}{n}\right) = m^2\left(\frac{f(x)}{n^2}\right) = \left(\frac{m}{n}\right)^2f(x)$.
6) From 5), there exists $a$ s.t. $f(r)=ar^2$ for every $r \in \mathbb{Q}$.
\end{solution}



\begin{solution}[by \href{https://artofproblemsolving.com/community/user/60946}{matrix41}]
	Setting  $f(x)=(g(x))^2)$ easily found $g(x)=\pm xg(1)$ for $x\in\mathbb{N}$ simply changes it to $x\in\mathbb{Q}$ , So $f(x)=0$ or $f(x)=cx^2$ $\forall x\in\mathbb{Q}$ where $c\in\mathbb{Q}$ as constant value
\end{solution}



\begin{solution}[by \href{https://artofproblemsolving.com/community/user/86345}{namdan}]
	Find all $f:\mathbb{Q}\rightarrow \mathbb{Q}$ satisfy :
$f(f(x)+y)=x+f(y), \forall x, y\in \mathbb{Q}$
\end{solution}



\begin{solution}[by \href{https://artofproblemsolving.com/community/user/29428}{pco}]
	\begin{tcolorbox}Find all $f:\mathbb{Q}\rightarrow \mathbb{Q}$ satisfy :
$f(f(x)+y)=x+f(y), \forall x, y\in \mathbb{Q}$\end{tcolorbox}
Better to create a new topic for a new problem.

Let $P(x,y)$ be the assertion $f(f(x)+y)=x+f(y)$

$P(x,0)$ $\implies$ $f(f(x))=x+f(0)$
$P(f(x),y)$ $\implies$ $f(x+y+f(0))=f(x)+f(y)$

Writing $f(x)=g(x+f(0))$, this becomes $g((x+f(0))+(y+f(0)))=g(x+f(0))+g(y+f(0))$
So $g(x+y)=g(x)+g(y)$ and $g(x)=g(1)x$ $\forall x\in\mathbb Q$ and so $f(x)=g(1)(x+f(0))$

So $f(x)=ax+b$ and, plugging this in original equation, we get $b=0$ and $a^2=1$

Hence the two solutions :
$f(x)=x$ $\forall x$
$f(x)=-x$ $\forall x$
\end{solution}



\begin{solution}[by \href{https://artofproblemsolving.com/community/user/86345}{namdan}]
	\begin{tcolorbox}[quote="namdan"]Find all $f:\mathbb{Q}\rightarrow \mathbb{Q}$ satisfy :
$f(f(x)+y)=x+f(y), \forall x, y\in \mathbb{Q}$\end{tcolorbox}
Better to create a new topic for a new problem.

Let $P(x,y)$ be the assertion $f(f(x)+y)=x+f(y)$

$P(x,0)$ $\implies$ $f(f(x))=x+f(0)$
$P(f(x),y)$ $\implies$ $f(x+y+f(0))=f(x)+f(y)$

Writing $f(x)=g(x+f(0))$, this becomes $g((x+f(0))+(y+f(0)))=g(x+f(0))+g(y+f(0))$
So $g(x+y)=g(x)+g(y)$ and $g(x)=g(1)x$ $\forall x\in\mathbb Q$ and so $f(x)=g(1)(x+f(0))$

So $f(x)=ax+b$ and, plugging this in original equation, we get $b=0$ and $a^2=1$

Hence the two solutions :
$f(x)=x$ $\forall x$
$f(x)=-x$ $\forall x$\end{tcolorbox}
Thanks you! pco :)
\end{solution}



\begin{solution}[by \href{https://artofproblemsolving.com/community/user/86345}{namdan}]
	\begin{tcolorbox}Sketch of Proof:
4) Induction on $n \in \mathbb{N}$ to show $f(nx)=n^2f(x)$.
\end{tcolorbox}
Can you help me prove this.
\end{solution}



\begin{solution}[by \href{https://artofproblemsolving.com/community/user/29428}{pco}]
	\begin{tcolorbox}[quote="Batominovski"]Sketch of Proof:
4) Induction on $n \in \mathbb{N}$ to show $f(nx)=n^2f(x)$.
\end{tcolorbox}
Can you help me prove this.\end{tcolorbox}
Just set $(x,y)\to(nx,x)$ in the original equation and you get $f((n+1)x)+f((n-1)x)=2f(nx)+2f(x)$

And so, if $f(kx)=k^2f(x)$ $\forall k\le n$ : $f((n+1)x)+(n-1)^2f(x)=2n^2f(x)+2f(x)$ and so $f((n+1)x)=(n+1)^2f(x)$

And start of induction is basic : $f(0\times x)=0^2f(x)$ and $f(1\times x)=1^2f(x)$
\end{solution}
*******************************************************************************
-------------------------------------------------------------------------------

\begin{problem}[Posted by \href{https://artofproblemsolving.com/community/user/87322}{paul1703}]
	Find all continuous functions $f: \mathbb R^+ \to \mathbb R^+$ which satisfy for all $x, y \in \mathbb R^+$ the following equation:
\[f\left(x+\frac{1}{x}\right)+f\left(y+\frac{1}{y}\right)=f\left(x+\frac{1}{y}\right)+f\left(y+\frac{1}{x}\right).\]
	\flushright \href{https://artofproblemsolving.com/community/c6h400169}{(Link to AoPS)}
\end{problem}



\begin{solution}[by \href{https://artofproblemsolving.com/community/user/29428}{pco}]
	\begin{tcolorbox}Find all contiuous function $f:R->R$ satisfying:
$f(x+\frac{1}{x})+f(y+\frac{1}{y})=f(x+\frac{1}{y})+f(y+\frac{1}{x})$\end{tcolorbox}
PLEASE, give the full problem statement !!!!
You gave the domain of $f(x)$ but not the domain of the functional equation.
If we suppose then that these two domains are the same, there are obviously no solutions since the functional equation can not be true for $x=0$, for example.
:(
\end{solution}



\begin{solution}[by \href{https://artofproblemsolving.com/community/user/87322}{paul1703}]
	im really sorry it's $R+->R+$
\end{solution}



\begin{solution}[by \href{https://artofproblemsolving.com/community/user/29428}{pco}]
	\begin{tcolorbox}im really sorry it's $R+->R+$\end{tcolorbox}
Do you mean :
The domain of $f(x)$ is $\mathbb R^+$ (set of positive real numbers)
And the codomain of $f(x)$ is $\mathbb R^+$
And the domain of functional equation is $(\mathbb R^+)^2$
?
\end{solution}



\begin{solution}[by \href{https://artofproblemsolving.com/community/user/87322}{paul1703}]
	The function maps the positive real numbers to the positive real numbers.The function is not bidimensional, it is a normal functional equation.
\end{solution}



\begin{solution}[by \href{https://artofproblemsolving.com/community/user/29428}{pco}]
	\begin{tcolorbox}The function maps the positive real numbers to the positive real numbers.The function is not bidimensional, it is a normal functional equation.\end{tcolorbox}
Do you know the difference between a function and a functional equation ?

The function is $f(x)$ and the functional equation is the equality $f(x+\frac 1x)+f(y+\frac 1y)=f(x+\frac 1y)+f(y+\frac 1x)$

The domain of the function is the set of values for which $f(x)$ is defined
The domain of the functional equation is the set of values for which the equation is true.


For example, in the problem :
Find all functions $f(x)$ from $\mathbb R\to\mathbb R$ for which :
$f(x)+f(\frac 1y)=x+\frac 1y$ $\forall x,\forall y\ne 0$ :
The domain of the function is $\mathbb R$
The codomain of the function is $\mathbb R$
The domain of the functional equation is $\mathbb R\times\mathbb R^*$

In a functional equation problem, all these three data must be given.
\end{solution}



\begin{solution}[by \href{https://artofproblemsolving.com/community/user/87322}{paul1703}]
	I knew the difference of function, and functional equation, I meant the problem is a normal functional equation on one variable ( i thought by (RxR) you meant on 2 variables)
the domanin of the function is R+
the codomaine is R+ in this particular function
\end{solution}



\begin{solution}[by \href{https://artofproblemsolving.com/community/user/29428}{pco}]
	\begin{tcolorbox}I knew the difference of function, and functional equation, I meant the problem is a normal functional equation on one variable ( i thought by (RxR) you meant on 2 variables)
the domanin of the function is R+
the codomaine is R+ in this particular function\end{tcolorbox}
Ok, so the domain of the functional equation is not given, although I ask you for this information since the beginning

So I'll consider that the functional equation is true only $\forall x=y$

Then any continuous function from $\mathbb R^+\to\mathbb R^+$ is solution.

This was a quite easy functional equation problem.
I'm glad to have helped you.
\end{solution}



\begin{solution}[by \href{https://artofproblemsolving.com/community/user/87322}{paul1703}]
	=)))) now i got what you meant=( the functional equation is satisfied for every y and x from R+ i'm really  really sorry, wy every contiunous function is a solution?
\end{solution}



\begin{solution}[by \href{https://artofproblemsolving.com/community/user/29428}{pco}]
	\begin{tcolorbox}Find all contiuous function $f:R+->R+$ satisfying:
$f(x+\frac{1}{x})+f(y+\frac{1}{y})=f(x+\frac{1}{y})+f(y+\frac{1}{x})$ for every x,y from R+\end{tcolorbox}
Consider then $a,b>0$ such that $a\ne b$ and $ab\ge 4$

Consider the system :
$x\ge \sqrt{\frac ab}$ and $y\ge\sqrt{\frac ba}$
$x+\frac 1y=a$
$y+\frac 1x=b$
Ths system always have a unique real solution

Let then $u=x+\frac 1x$ and $v=y+\frac 1y$
It's easy to see that :
$f(a)+f(b)=f(u)+f(v)$
$a+b=u+v$
$|u-v|<|a-b|$
$u\ne v$ and $uv\ge 4$

And so we can create a sequence $(a,b)\to (u,v)$, repeating the process
It's easy to see that the two numbers have their difference tending towards 0 and so have the same limit $\frac {a+b}2$

and so, since continuous, $f(a)+f(b)=2f(\frac{a+b}2)$  $\forall a,b>0$ such that $a\ne b$ and $ab\ge 4$

This is a classical functional equation which implies easily (continuity again) $f(x)=cx+d$ $\forall x\ge 2$

Using then the functonal equation with for example $y\ge \frac 12$, we get $x+\frac 1x,y+\frac 1y, x+\frac 1y\ge 2$ and so $f(y+\frac 1x)=c(y+\frac 1x)+d$ and so $f(x)=cx+d$ $\forall x>\frac 12$

And it's easy to use similar steps as many times as we want to get $f(x)=cx+d$ $\forall x>0$

And this indeed is a solution as soon as $c\ge 0$ and $d\ge 0$ or $c=0$ and $d>0$

Hence the answer : $\boxed{f(x)=ax+b}$ $\forall x>0$ and for any ($a>0$ and $b\ge 0$) or ($a=0$ and $b>0$)
\end{solution}
*******************************************************************************
-------------------------------------------------------------------------------

\begin{problem}[Posted by \href{https://artofproblemsolving.com/community/user/86345}{namdan}]
	Find all $f:\mathbb{N}\rightarrow \mathbb{N}$ that satisfy
\[f(f(m+n))=f(m)+f(n), \quad \forall m, n\in \mathbb{N}.\]
	\flushright \href{https://artofproblemsolving.com/community/c6h401032}{(Link to AoPS)}
\end{problem}



\begin{solution}[by \href{https://artofproblemsolving.com/community/user/29428}{pco}]
	\begin{tcolorbox}Find all $f:\mathbb{N}\rightarrow \mathbb{N}$ satisfy:
$f(f(m+n))=f(m)+f(n), \forall m, n\in \mathbb{N}$\end{tcolorbox}
So $f(m+a)+f(n-a)=f(m)+f(n)$ and so $f(m+a)=f(m)+h(a)$ for some $h$

Then $f(m+n)=f(m)+h(n)=f(n)+h(m)$ and so $h(m)-f(m)=c$ and $f(m+n)=f(m)+f(n)+c$

So $f(n)=an+b$ for some $a,b$ and, plugging this in original equation, we get $a=1$ and any $b\in\mathbb N\cup\{0\}$

Hence the answer $\boxed{f(x)=x+a-1}$ $\forall x$ and for any $a\in\mathbb N$
\end{solution}



\begin{solution}[by \href{https://artofproblemsolving.com/community/user/86345}{namdan}]
	Thanks you very much! pco :)
Can solve this by induction Mathematics?
\end{solution}
*******************************************************************************
-------------------------------------------------------------------------------

\begin{problem}[Posted by \href{https://artofproblemsolving.com/community/user/86345}{namdan}]
	Find all functions $f:\mathbb{Q}\to\mathbb{R}$ that satisfy
\[f(x+y)=f(x)f(y)-f(xy)+1,\quad \forall x, y\in \mathbb{Q}.\]
	\flushright \href{https://artofproblemsolving.com/community/c6h401194}{(Link to AoPS)}
\end{problem}



\begin{solution}[by \href{https://artofproblemsolving.com/community/user/29428}{pco}]
	\begin{tcolorbox}Find all $f:\mathbb{Q}\rightarrow \mathbb{R}$ satisfy:
$f(x+y)=f(x)f(y)-f(xy)+1, \forall x, y\in \mathbb{Q}$\end{tcolorbox}
Let $P(x,y)$ be the assertion $f(x+y)=f(x)f(y)-f(xy)+1$

$P(0,0)$ $\implies$ $f(0)=1$

If $f(1)=1$, then $P(x-1,1)$ $\implies$ $f(x)=1$ $\forall x$ which indeed is a solution.

If $f(1)=2$, then :
$P(x,1)$ $\implies$ $f(x+1)=f(x)+1$ and so $f(x+n)=f(x)+n$ and $f(n)=n+1$
Then $P(q,\frac pq)$ $\implies$ $q+f(\frac pq)=(q+1)f(\frac pq)-(p+1)+1$ and so $f(\frac pq)=\frac pq+1$
And so $f(x)=x+1$ $\forall x$, which indeed is a solution.

If $f(1)=a+1$ with $a\notin\{0,1\}$, then :
$P(x,1)$ $\implies$ $f(x+1)=af(x)+1$ and so $f(x+n)=a^nf(x)+\frac{a^n-1}{a-1}$ and $f(n)=\frac{a^{n+1}-1}{a-1}$
Then $P(q,\frac pq)$ $\implies$ $f(\frac pq)=\frac{a^{p+1}-a}{a^q-1}+1$
Setting for example $(p,q)=(2,1)$ and then $(p,q)=(4,2)$, we get that no such $a$ fits.

\begin{bolded}Hence the two solutions :\end{bolded}\end{underlined}
$f(x)=1$  $\forall x$
$f(x)=x+1$ $\forall x$
\end{solution}



\begin{solution}[by \href{https://artofproblemsolving.com/community/user/86345}{namdan}]
	Thanks you, pco! :)
\end{solution}



\begin{solution}[by \href{https://artofproblemsolving.com/community/user/105386}{Bertus}]
	My Solution :
Let $P(x,y)$ the assertion : $f(x+y)=f(x)f(y)-f(xy)+1$
$P(0,0): (f(0)-1)^{2}=0 \Rightarrow f(0)=1$$
P(x,-x): f(-x^{2})=f(x).f(-x)$, so taking $x=1$ we get etiher $f(1)=1$ either $f(-1)=0$
[color=#FF0000]- Case 1 :[\/color] $f(1)=1$
$P(x,1): f(x+1)=-f(x)+1$, and so taking $x=0$ we get : $0=f(1)=-f(0)+1=1$, which is impossible.
[color=#FF0000]- Case 2 :[\/color] $f(-1)=0$
Consider the function $g$ such that $\forall x \in \mathbb{R} : g(x)=f(x)-1$; we then get the assertion $Q(x,y) : g(x+y)=g(x)+g(y)-g(xy)+g(x)g(y)$ 
$Q(x,1): g(x+1)=g(1)g(x)+g(1)$, and so for every non-negative integer we get that g is on the form $x|\Rightarrow cx+d$( since $g(0)=0$ and we multiply by the same factors, otherwise we can proof it directly by induction if necessary) and so immediatly $f$ is in the same form too.
Plugging in $Q(x,y)$ we get either $g(x)=0$ or $g(x)=x$ for every $x \in \mathbb{N}$.
If $g$ is zero function in $\mathbb{N}$, then :
$Q(x,1): g(x+1)=0 \forall x \in \mathbb{Q}$ and so $g$ is zero function, which give us $\forall x \in \mathbb{Q} : f(x)=1$ which is indeed a solution.
Otherwise, if $\forall x \in \mathbb{N} : g(x)=x$, so $\forall x \in \mathbb{N} f(x)=x+1$ then $f(1)=2$ 
$P(x,-1): f(x)=x+1 \forall x \in \mathbb{Z-}$
$P(x,1):f(x+1)=f(x)+1$ and so by induction $f(x+k)=f(x)+k$ for every integer $k$ and rational $x$.
Then we get $f(qx)=q(f(x)-1) \forall (x,q) \in \mathbb{Z}*\mathbb{Q}$
Hence let $x=\frac{p}{q}$ such that $(p,q) \in \mathbb{N}.\mathbb{Z^{*}}$
$p=f(p)=f(qx)=q(f(x)-1)$
Finally $f(x)=x+1 \forall x \in \mathbb{Q}$
Conversly, the functions :
- $f(x)=1 \forall x \in \mathbb{Q}$
- $f(x)=x+1 \forall x \in \mathbb{Q}$
satisfy the FE.
\end{solution}



\begin{solution}[by \href{https://artofproblemsolving.com/community/user/72819}{Dijkschneier}]
	\begin{tcolorbox}$Q(x,1): g(x+1)=g(1)g(x)+g(1)$, and so for every non-negative integer we get that g is on the form $x|\Rightarrow cx+d$\end{tcolorbox}
No, this is not true, unfortunately.
\end{solution}



\begin{solution}[by \href{https://artofproblemsolving.com/community/user/105386}{Bertus}]
	\begin{tcolorbox}[quote="Bertus"]$Q(x,1): g(x+1)=g(1)g(x)+g(1)$, and so for every non-negative integer we get that g is on the form $x|\Rightarrow cx+d$\end{tcolorbox}
No, this is not true, unfortunately.\end{tcolorbox}
Yes indeed.
The function defined in $\mathbb{N}$ will depend on the variable which is $(f(1)-1)^{n}$, but we can check that since thsi term is constant that $f(1)=1$ or $f(1)=2$. Anyway, i think the unique way to solve it is the method of pco :)
\end{solution}



\begin{solution}[by \href{https://artofproblemsolving.com/community/user/72819}{Dijkschneier}]
	Yeah. Nevertheless, if the idea of setting f(1)=a+1 doesn't come to our mind, we can prove that $f(1) \in \{0,1,2\}$
Assume f(0)=1 and f(-1)=0 (case 2) and we want to find out t=f(1).
P(x,y) : f(x+y)=f(x)f(y)-f(xy)+1
P(1,1) ==> f(2)=f(1)²-f(1)+1
P(-1,-1) ==> f(-2)=-f(1)+1
P(-2,-2) ==> f(-4)=f(-2)²-f(4)+1
P(2,-2) ==> f(-4)=f(2)f(-2)
P(2,2) ==> 2f(4)=f(2)²+1
We have : 2f(-4)=2f(-2)²-2f(4)+2=2(1-f(1))²-(1+f(2)²)+2 = 2(1-t)²-(1+(t²-t+1)²)+2 
And : 2f(-4)=2f(2)f(-2)=2(t²-t+1)(1-t)
Equating : 2(1-t)²-(1+(t²-t+1)²)+2 = 2(t²-t+1)(1-t)
<==> $t \in \{0,1,2\}$
And then we should prove (as pco did more generally) that $f(1)=0$ leads to a contradiction.
\end{solution}



\begin{solution}[by \href{https://artofproblemsolving.com/community/user/105386}{Bertus}]
	\begin{tcolorbox}Yeah. Nevertheless, if the idea of setting f(1)=a+1 doesn't come to our mind, we can prove that $f(1) \in \{0,1,2\}$
Assume f(0)=1 and f(-1)=0 (case 2) and we want to find out t=f(1).
P(x,y) : f(x+y)=f(x)f(y)-f(xy)+1
P(1,1) ==> f(2)=f(1)²-f(1)+1
P(-1,-1) ==> f(-2)=-f(1)+1
P(-2,-2) ==> f(-4)=f(-2)²-f(4)+1
P(2,-2) ==> f(-4)=f(2)f(-2)
P(2,2) ==> 2f(4)=f(2)²+1
We have : 2f(-4)=2f(-2)²-2f(4)+2=2(1-f(1))²-(1+f(2)²)+2 = 2(1-t)²-(1+(t²-t+1)²)+2 
And : 2f(-4)=2f(2)f(-2)=2(t²-t+1)(1-t)
Equating : 2(1-t)²-(1+(t²-t+1)²)+2 = 2(t²-t+1)(1-t)
<==> $t \in \{0,1,2\}$
And then we should prove (as pco did more generally) that $f(1)=0$ leads to a contradiction.\end{tcolorbox}
Nice proof :).
\end{solution}
*******************************************************************************
-------------------------------------------------------------------------------

\begin{problem}[Posted by \href{https://artofproblemsolving.com/community/user/87322}{paul1703}]
	Let $f:\mathbb{N}\to \mathbb{N}$ be a function satisfying:
\[f(f(n))=4n-3 \quad \text{and} \quad f(2^n)=2^{n+1}-1\]
for all positive integers $n$. Find $ f(1993)$. Can you find explicitly the value of $ f(2007)$? What values can $f(1997)$ take?
	\flushright \href{https://artofproblemsolving.com/community/c6h401323}{(Link to AoPS)}
\end{problem}



\begin{solution}[by \href{https://artofproblemsolving.com/community/user/29428}{pco}]
	\begin{tcolorbox}Let $f:\mathbb{N}->\mathbb{N}$ be a function satisfying:
 $f(f(n))=4n-3$
 $(2^n)=2^{n+1}-1$, for all natural n
Find $ f(1993)$, can you find explicietly the value $ f(2007)$? what values can $f(1997)$ take?\end{tcolorbox}
I suppose that third line must be read $f(2^n)=2^{n+1}-1$

Let $\mathbb N_0=\mathbb N\cup\{0\}$
Let $g(n)$ from $\mathbb N_0\to\mathbb N_0$ defined as $g(n)=f(n+1)-1$. The equation is then $g(g(n))=4n$ whose general solution is :

Let $A,B$ two equinumerous sets whose intersection is empty and whose union is the set of all natural numbers not divisible by $4$.
Let $h(x)$ any bijection from $A\to B$ and $h^{-1}(x)$ it's inverse function.

Then $g(x)$ may be defined as :
$g(0)=0$
$\forall x\in A$ : $g(x)=h(x)$
$\forall x\in B$ : $g(x)=4h^{-1}(x)$
$\forall x\in\mathbb N\setminus(A\cup B)$ : $g(x)=4^{v_4(x)}g(x4^{-v_4(x)})$


The constraint $f(2^n)=2^{n+1}-1$ becomes $g(2^n-1)=2^{n+1}-2$ and so we just have to add to the previous general solution the constraints :
$2^n-1\in A$ $\forall n\in\mathbb N$
$2^n-2\in B$ $\forall n>1\in\mathbb N$
$h(2^n-1)=2^{n+1}-2$

Then $f(1993)=g(1992)+1=4g(498)+1$ and since $498$ is not divisible by $4$ and is not in the form $2^n-1$ neither $2^{n+1}-1$, we get nearly no constraint for $g(498)$ :
We can put $498$ in $A$ and then $g(498)\in B$ may be any value not divisible by $4$ and not in the form $2^{n+1}-2$
We can put $498$ in $B$ and then $g(498)=4u$ where $u$ is any number not divisible by $4$ and not in the form $2^n-1$

And the same conclusions may be obtained for $g(2006)$ and $g(1996)$.
\end{solution}
*******************************************************************************
-------------------------------------------------------------------------------

\begin{problem}[Posted by \href{https://artofproblemsolving.com/community/user/103547}{iijoclu}]
	We have $f,g : [a,b] \rightarrow \mathbb{R}$, two functions with the following properties:

i) $g$ is strictly increasing on $[a,b]$
ii) $|f(x) - f(y)| \leq |g(x) - g(y)|, \forall x,y \in [a,b]$
iii) $f(a) = g(a)$ and $f(b)=g(b)$

Prove that $f(x) = g(x), \forall x,y \in [a,b]$.
	\flushright \href{https://artofproblemsolving.com/community/c6h401453}{(Link to AoPS)}
\end{problem}



\begin{solution}[by \href{https://artofproblemsolving.com/community/user/29428}{pco}]
	\begin{tcolorbox}We have $f,g : [a,b] \rightarrow \mathbb{R}$, two functions with the following properties:

i) $g$ is strictly increasing on $[a,b]$
ii) $|f(x) - f(y)| \leq |g(x) - g(y)|, \forall x,y \in [a,b]$
iii) $f(a) = g(a)$ and $f(b)=g(b)$

Prove that $f(x) = g(x), \forall x,y \in [a,b]$.\end{tcolorbox}
Let $y=a$ : $f(x)-f(a)\le |f(x)-f(a)|\le |g(x)-g(a)|=g(x)-g(a)$ and so $f(x)\le g(x)$

So $f(x)\le g(x)\le g(b)=f(b)$ and so $f(x)-f(b)\le 0$

Let $y=b$ : $|f(x)-f(b)|\le |g(x)-g(b)|$ and so $f(b)-f(x)\le g(b)-g(x)$ and so $f(x)\ge g(x)$

Hence the result.
\end{solution}
*******************************************************************************
-------------------------------------------------------------------------------

\begin{problem}[Posted by \href{https://artofproblemsolving.com/community/user/61896}{Mateescu Constantin}]
	Find all continuous functions $f:\mathbb{R}\to\mathbb{R}$ so that \[f(f(x))=f(x)+x,\quad \forall\ x\in\mathbb{R}.\]
	\flushright \href{https://artofproblemsolving.com/community/c6h401454}{(Link to AoPS)}
\end{problem}



\begin{solution}[by \href{https://artofproblemsolving.com/community/user/29428}{pco}]
	\begin{tcolorbox}Find all continuous functions $f:\mathbb{R}\to\mathbb{R}$ so that : $f(f(x))=f(x)+x\ ,\ \forall\ x\in\mathbb{R}$ .\end{tcolorbox}
Let $x\in\mathbb R$ and the sequence $a_0=x$ and $a_{n+1}=f(a_n)$
We get $a_0=x$ and $a_1=f(x)$ and $a_{n+2}=a_{n+1}+a_n$.

Let $r_1<r_2$ be the two real roots of equation $x^2-x-1=0$. We get $a_n=\frac{(f(x)-r_2x)r_1^n-(f(x)-r_1x)r_2^n}{r_1-r_2}$

$f(x)$ is injective.
It's easy to see that $f(x)$ is neither upper bounded, neither lower bounded and so $f(x)$ is a bijection from $\mathbb R\to\mathbb R$

So the equality $a_n=\frac{(f(x)-r_2x)r_1^n-(f(x)-r_1x)r_2^n}{r_1-r_2}$ is true also for $n<0$

Setting $x=0$ in the equation, we get $f(f(0))=f(0)$ and so $f(0)=0$, since injective.
$f(x)$ is injective  and continuous, and so monotonous and so $\frac{f(x)-f(0)}{x-0}$ has a constant sign and so $\frac{a_{n+1}}{a_n}$ has a constant sign.

So $\frac{(f(x)-r_2x)r_1^{n+1}-(f(x)-r_1x)r_2^{n+1}}{(f(x)-r_2x)r_1^n-(f(x)-r_1x)r_2^n}$ has a constant sign.

If $f(x)$ is decreasing and $f(x)-r_1x\ne 0$, then the above quantity has limit $r_2>0$ when $n\to +\infty$, in contradiction with the fact $f(x)$ decreasing. So the only continuous decreasing solution may be $f(x)=r_1x$ which indeed is a solution.

If $f(x)$ is increasing and $f(x)-r_2x\ne 0$, then the above quantity has limit $r_1<0$ when $n\to -\infty$, in contradiction with the fact $f(x)$ increasing. So the only continuous increasing solution may be $f(x)=r_2x$ which indeed is a solution.

Hence the only solutions :
$f(x)=\frac{1+\sqrt 5}2x$

$f(x)=-\frac{\sqrt 5 -1}2x$
\end{solution}
*******************************************************************************
-------------------------------------------------------------------------------

\begin{problem}[Posted by \href{https://artofproblemsolving.com/community/user/68555}{trbst}]
	Find all differentiable functions $f,g:\mathbb R \to \mathbb R$ with \[xf(y)+yf'(x)\ge xg'(y)+yg(x)\] for all $x,y\in \mathbb R$ .
	\flushright \href{https://artofproblemsolving.com/community/c6h401614}{(Link to AoPS)}
\end{problem}



\begin{solution}[by \href{https://artofproblemsolving.com/community/user/29428}{pco}]
	\begin{tcolorbox}Find all differentiable functions $f,g:R\to R$ with $xf(y)+yf'(x)\ge xg'(y)+yg(x)$ for all $x,y\in R$ .\end{tcolorbox}
For $x\ne 0$, let $u(x)=\frac{f(x)-g'(x)}x$ and $v(x)=\frac{g(x)-f'(x)}x$

The inequation may be written (E1) : $xyu(y)\ge xyv(x)$

Let $x,y>0$, E1 becomes $u(y)\ge v(x)$ and so $\exists a$ such that $u(x)\ge a$ $\forall x>0$ and $v(x)\le a$ $\forall x>0$

Let $x,y<0$, E1 becomes $u(y)\ge v(x)$ and so $\exists b$ such that $u(x)\ge b$ $\forall x<0$ and $v(x)\le b$ $\forall x<0$

Let $x>0,y<0$, E1 becomes $u(y)\le v(x)$ and so $\exists c$ such that $u(x)\le c$ $\forall x<0$ and $v(x)\ge c$ $\forall x>0$

Let $x<0,y>0$, E1 becomes $u(y)\le v(x)$ and so $\exists d$ such that $u(x)\le d$ $\forall x>0$ and $v(x)\ge d$ $\forall x<0$

This implies :
$\forall x>0$ : $d\ge u(x)\ge a$ and $a\ge v(x)\ge c$ and so $d\ge a\ge c$
$\forall x<0$ : $c\ge u(x)\ge b$ and $b\ge v(x)\ge d$ and so $c \ge b\ge d$

So $a=b=c=d$ and $u(x)=a$ and $v(x)=a$ $\forall x\ne 0$

So $f(x)-g'(x)=ax$ and $g(x)-f'(x)=ax$ $\forall x$ (equality at $x=0$ is from continuity)

So $f(x)=ax+a+f''(x)$

This is a simple equation whose general solution is : $f(x)=ax+a+\alpha e^x+\beta e^{-x}$
Which implies $g(x)=f'(x)+ax=ax+a+\alpha e^x-\beta e^{-x}$ which indeed are solutions.

\begin{bolded}Hence the answer \end{bolded}\end{underlined}:
$f(x)=ax+a+\alpha e^x+\beta e^{-x}$ $\forall x$ and for any real $a,\alpha,\beta$
$g(x)=ax+a+\alpha e^x-\beta e^{-x}$ $\forall x$
\end{solution}



\begin{solution}[by \href{https://artofproblemsolving.com/community/user/31915}{Batominovski}]
	\begin{tcolorbox}
So $f(x)=ax+a+f''(x)$

This is a simple equation whose general solution is : $f(x)=ax+a+\alpha e^x+\beta e^{-x}$
Which implies $g(x)=f'(x)+ax=ax+a+\alpha e^x-\beta e^{-x}$ which indeed are solutions.
\end{tcolorbox}

I'm afraid you can't assume that $f''$ exists.  A differentiable function need not be twice differentiable.  However, you can approach the same answer via defining $P:=f+g$ and $Q:=f-g$.  You can see that $P(x)-P'(x) = 2ax$ and $Q(x)+Q'(x)=0$.  Therefore, $P(x)=2a+2ax+2\alpha\exp(+x)$ and $Q(x)=2\beta \exp(-x)$ for some constants $\alpha$ and $\beta$.  We then finally obtain your conclusions.
\end{solution}



\begin{solution}[by \href{https://artofproblemsolving.com/community/user/64716}{mavropnevma}]
	\begin{tcolorbox}So $f(x)-g'(x)=ax$ and $g(x)-f'(x)=ax$ $\forall x$ (equality at $x=0$ is from continuity)\end{tcolorbox}
So pco obtained $f'(x) = g(x) - ax$, and as both $g$ and $x \to ax$ are differentiable, it follows $f'$ is differentiable. Similarly for $g'$. In fact it turns out by iterating that both $f$ and $g$ are $C^{\infty}$.
\end{solution}
*******************************************************************************
-------------------------------------------------------------------------------

\begin{problem}[Posted by \href{https://artofproblemsolving.com/community/user/78376}{babbomammo}]
	Find all functions $f: \mathbb R^{+} \to \mathbb R^{+}$ such that \[f(xyz)+f(x)+f(y)+f(z)=f(\sqrt{xy})f(\sqrt{yz})f(\sqrt{zx})\]  for positive reals $x,y$, and $z$ and also $f(x) < f(y)$ for $1 \leq x <y$.
	\flushright \href{https://artofproblemsolving.com/community/c6h402541}{(Link to AoPS)}
\end{problem}



\begin{solution}[by \href{https://artofproblemsolving.com/community/user/29428}{pco}]
	\begin{tcolorbox}Find all functions $f : R^{+}\rightarrow R^{+}$  such that $f(xyz)+f(x)+f(y)+f(z)=f(\sqrt{xy})f(\sqrt{yz})f(\sqrt{zx})$  for positive reals $x,y,z$ and also $f(x) < f(y)$ for
$1 \leq x <y$\end{tcolorbox}
Let $P(x,y)$ be the assertion $f(xyz)+f(x)+f(y)+f(z)=f(\sqrt{xy})f(\sqrt{yz})f(\sqrt{zx})$

$P(1,1,1)$ $\implies$ $4f(1)=f(1)^3$ and so $f(1)=2$

$P(x^2,1,1)$ $\implies$ $f(x^2)=f(x)^2-2$

$P(x^2,y^2,1)$ $\implies$ $f(x^2y^2)+f(x^2)+f(y^2)+2=f(xy)f(x)f(y)$
And so, using $f(x^2)=f(x)^2-2$ for $x^2y^2,x^2$ and $y^2$ :

$f(xy)^2-f(xy)f(x)f(y)+f(x)^2+f(y)^2-4=0$

The discriminant of this quadratic in $f(xy)$ is $(f(x)^2-4)(f(y)^2-4)$
And since we now that $f(x)>2$ $\forall x>1$, we get that $f(x)\ge 2$ $\forall x>0$

Let then $u(x)\ge 1$ such that $f(x)=u(x)+\frac 1{u(x)}$ (which always exists since $f(x)\ge 2$)

The above quadratic implies $u(xy)=u(x)u(y)$ or $u(xy)=\frac{u(x)}{u(y)}$ or $u(xy)=\frac{u(y)}{u(x)}$

Using the fact that $f(x)$ is increasing for $x\ge 1$ and so $u(x)$ is increasing too, we get that $u(xy)=u(x)u(y)$ $\forall x,y\ge 1$

So $u(x)=x^a$ with $a>0$ $\forall x\ge 1$

Plugging this back in original equation, we get that any real $a>0$ fits and so $f(x)=x^a+x^{-a}$ $\forall x\ge 1$

$P(x,\frac 1x,1)$ $\implies$ $f(x^2)+f(\frac 1{x^2})+4=2f(x)f(\frac 1x)$
And so, using $f(x^2)=f(x)^2-2$ for $x^2$ and $\frac 1{x^2}$ :

$(f(x)-f(\frac 1x))^2=0$ and so $f(\frac 1x)=f(x)$

So $\boxed{f(x)=x^a+x^{-a}}$ $\forall x$ and for any real $a\ne 0$ which indeed is a solution.
\end{solution}



\begin{solution}[by \href{https://artofproblemsolving.com/community/user/105802}{tom_damrong}]
	any alternative solution? :)
\end{solution}
*******************************************************************************
-------------------------------------------------------------------------------

\begin{problem}[Posted by \href{https://artofproblemsolving.com/community/user/107185}{mymath7}]
	Find all function $ f: \mathbb R\to \mathbb R$ satisfying the condition
\[f(y+f(x))=f(x)f(y)+f(f(x))+f(y)-xy\]
for all $x, y \in \mathbb R$.
	\flushright \href{https://artofproblemsolving.com/community/c6h403064}{(Link to AoPS)}
\end{problem}



\begin{solution}[by \href{https://artofproblemsolving.com/community/user/73386}{mousavi}]
	$(x,y):(0,0)\Longrightarrow f(0)(f(0+1)=0$

===========================================================
a)$f(0)=-1\Longrightarrow y=0 \Longrightarrow f(x)=-1$

===========================================================

b)$f(0)=0$

1)it's obvious that $f$ is injective.
===========================================================

$(x,y):(f(1),1)\Longrightarrow f(1+f(f(1)))=f(f(1))f(1)+f(f(f(1))) $ and $(x,y):(1,f(f(1)))\Longrightarrow f(f(f(1))+1)=f(1)f(f(f(1)))+f(f(f(1)))$

$\Longrightarrow f(1)=1$

===========================================================
$y=>f(y)\Longrightarrow f(x)f(f(y))=xf(y)=f(y)f(f(x))-yf(x) , y=1 \Longrightarrow f(f(x))=2f(x)-x $ (2)

$x=1\Longrightarrow f(y+1)=2f(x)-x+1$ (3)

$\Longrightarrow f(y+f(x))=f(x)f(y)+2f(x)-x+f(y)-xy$ 

$(x,y):(x,y+1)\Longrightarrow f(y+f(x)+1)=2f(x)f(y)-f(x)y+f(x)+2f(x)-2x+2f(y)-y+1-xy$ (*)

by (3) :
$f(y+f(x)+1)=2f(y+f(x))-y-f(x)+1=2f(x)f(y)+4f(x)-2x+2f(y)-2xy+y-f(x)+1$ (**)

$\Longrightarrow f(x)y=xy\Longrightarrow f(x)=x$
\end{solution}



\begin{solution}[by \href{https://artofproblemsolving.com/community/user/29428}{pco}]
	\begin{tcolorbox} a)$f(0)=-1\Longrightarrow y=0 \Longrightarrow f(x)=-1$
\end{tcolorbox}
Which is not a solution

\begin{tcolorbox}$(x,y):(f(1),1)\Longrightarrow f(1+f(f(1)))=f(f(1))f(1)+f(f(f(1))) $ and $(x,y):(1,f(f(1)))\Longrightarrow f(f(f(1))+1)=f(1)f(f(f(1)))+f(f(f(1)))$

$\Longrightarrow f(1)=1$\end{tcolorbox}
You made an error. When $(x,y):(1,f(f(1)))$, then $LHS$ is $f(f(f(1))+f(1))$ and not $f(f(f(1))+1)$
\end{solution}



\begin{solution}[by \href{https://artofproblemsolving.com/community/user/29428}{pco}]
	\begin{tcolorbox}Find all function $ f: \mathbb R\to \mathbb R$ satisfying the condition:

\[f(y+f(x))=f(x)f(y)+f(f(x))+f(y)-xy\]\end{tcolorbox}
Let $P(x,y)$ be the assertion $f(y+f(x))=f(x)f(y)+f(f(x))+f(y)-xy$

$f(x)=-1$ $\forall x$ is not a solution and so let $v$ such that $f(v)\ne -1$
$P(v,0)$ $\implies$ $f(0)(f(v)+1)=0$ and so $f(0)=0$

$f(x)=0$ $\forall x$ is not a solution and so let $u$ such that $f(u)\ne 0$

$P(x,f(u))$ $\implies$ $f(f(x)+f(u))=f(x)f(f(u))+f(f(x))+f(f(u))-xf(u)$
$P(u,f(x))$ $\implies$ $f(f(x)+f(u))=f(u)f(f(x))+f(f(x))+f(f(u))-uf(x)$
Subtracting, we get $f(f(x))+x=f(x)\frac{f(f(u))+u}{f(u)}$

and so $f(f(x))=af(x)-x$ for some $a\in\mathbb R$

So we can rewrite $P(x,y)$ as new assertion $Q(x,y)$ : $f(y+f(x))=f(x)f(y)+af(x)-x+f(y)-xy$

$Q(y,-1)$ $\implies$ $f(f(y)-1)=f(y)(f(-1)+a)+f(-1)$ $=cf(y)+d$

$Q(x,f(y)-1)$ $\implies$ $f(f(x)+f(y)-1)=f(x)(cf(y)+d)+af(x)-x+cf(y)+d-x(f(y)-1)$ and so :

$f(f(x)+f(y)-1)=cf(x)f(y) +(a+d)f(x)+(c-x)f(y)+d$
Swapping $x,y$, we get $f(f(x)+f(y)-1)=cf(x)f(y) +(a+d)f(y)+(c-y)f(x)+d$
Subtracting : $(a+d-c+y)f(x)=(a+d-c+x)f(y)$

Setting $y=0$ in this line, we get $a+d-c=0$ and so $yf(x)=xf(y)$ $ \forall x,y$

Setting $y=1$ in this expression, we get $f(x)=xf(1)$

Plugging in original equation, we get $f(1)=\pm 1$

\begin{bolded}And so the two solutions \end{bolded}\end{underlined}:
$f(x)=x$ $\forall x$
$f(x)=-x$ $\forall x$
\end{solution}



\begin{solution}[by \href{https://artofproblemsolving.com/community/user/107185}{mymath7}]
	Very nice soln, pco. Almost the same as mine :) :)
\end{solution}
*******************************************************************************
-------------------------------------------------------------------------------

\begin{problem}[Posted by \href{https://artofproblemsolving.com/community/user/90103}{Winner2010}]
	Find all $f:\mathbb{R}\rightarrow\mathbb{R}$ that satisfy

$f(x-f(y)+y) = f(x)-f(y)$

all real numbers $x,y$.
	\flushright \href{https://artofproblemsolving.com/community/c6h403165}{(Link to AoPS)}
\end{problem}



\begin{solution}[by \href{https://artofproblemsolving.com/community/user/29428}{pco}]
	\begin{tcolorbox}Find all $f:\mathbb{R}\rightarrow\mathbb{R}$ that satisfy

$f(x-f(y)+y) = f(x)-f(y)$

all real numbers $x,y$.\end{tcolorbox}
Let $g(x)=f(x)-x$ and the equation becomes assertion $P(x,y)$ : $g(x-g(y))=g(x)-y$

This implies that $g(x)$ is a bijection. So $\exists u$ suvh that $g(u)=0$. $P(x,u)$ implies then $u=0$

$P(g(x),x)$ $\implies$ $g(g(x))=x$
$P(x,g(y))$ $\implies$ $g(x-y)=g(x)-g(y)$
So $g(x)$ is any involutive solution of Cauchy's equation.

And it'simmediate to verify that this is indeed a solution.

\begin{bolded}Hence the answer \end{bolded}\end{underlined}: $f(x)=x+g(x)$ where $g(x)$ is any involutive solution of Cauchy's equation

Notice that we have infinitely many solutions.
The only continuous solutions are $f(x)=0$ $\forall x$ and $f(x)=2x$ $\forall x$
\end{solution}



\begin{solution}[by \href{https://artofproblemsolving.com/community/user/72819}{Dijkschneier}]
	I think pco meant $g(x)=x-f(x)$.
Here are the details :
[hide]
P(x,x) ==> f(2x-f(x))=0
Hence there exists a real c such that f(c)=0
P(x,c) ==> f(x+c)=f(x)
P(0,y) ==> f(y-f(y)+c)=f(0)-f(y) ==> f(y-f(y))=f(0)-f(y)
P(c,y) ==> f(y-f(y))=-f(y)
In particular : f(0)=0
Let g(x)=x-f(x) <==> f(x)=x-g(x)
f(y-f(y))=-f(y) ==> f(g(x))=g(x)-x ==> g(x)-g(g(x))=g(x)-x ==> g(g(x))=x
Therefore, g is a bijection, and hence, the original equation may be rewritten as : f(x+y)=f(x)+f(y), or : g(x+y)=g(x)+g(y)
So g is an involutive solution to Cauchy's euqation.[\/hide]
\end{solution}



\begin{solution}[by \href{https://artofproblemsolving.com/community/user/74510}{filipbitola}]
	P(c,y) is not equal to f(y-f(y))=-f(y)
\end{solution}



\begin{solution}[by \href{https://artofproblemsolving.com/community/user/74510}{filipbitola}]
	\begin{tcolorbox}[quote="pco"][quote="Winner2010"]Find all $f:\mathbb{R}\rightarrow\mathbb{R}$ that satisfy

$f(x-f(y)+y) = f(x)-f(y)$

all real numbers $x,y$.\end{tcolorbox}
Let $g(x)=f(x)-x$ and the equation becomes assertion $P(x,y)$ : $g(x-g(y))=g(x)-y$
...
\end{tcolorbox}

$g(x-g(y)) \neq  g(x)-y$\end{tcolorbox}

modularmarc101: Please check your math. According to my calculations, Patrick's is correct.
\end{solution}



\begin{solution}[by \href{https://artofproblemsolving.com/community/user/29428}{pco}]
	\begin{tcolorbox}[quote="pco"][quote="Winner2010"]Find all $f:\mathbb{R}\rightarrow\mathbb{R}$ that satisfy

$f(x-f(y)+y) = f(x)-f(y)$

all real numbers $x,y$.\end{tcolorbox}
Let $g(x)=f(x)-x$ and the equation becomes assertion $P(x,y)$ : $g(x-g(y))=g(x)-y$
...
\end{tcolorbox}

$g(x-g(y)) \neq  g(x)-y$\end{tcolorbox}

I'm sorry, but I checked my post and, according to me (and filipbitola too, thanks) , everything is correct.
\end{solution}



\begin{solution}[by \href{https://artofproblemsolving.com/community/user/29428}{pco}]
	Notice that the general solution for "involutive solutions of Cauchy's equation" may also be written as :

Let $A,B$ two supplementary subvectorspaces of the $\mathbb Q$-vectorspace $\mathbb R$
Let $a(x)$  and $b(x)$ the projections of $x$ in $A$ and $B$ so that $x=a(x)+b(x)$ with $a(x)\in A$ and $b(x)\in B$

Then $g(x)=a(x)-b(x)$

1) proof that any such $g(x)$ is an involutive solution of Cauchy's equation and so this is a solution
===============================================================
$a(x)$ and $b(x)$ are additive and so $g(x)$ is solution of Cauchy's equation.
$a(a(x))=a(x)$ and $a(b(x))=0$ and $a(a(x)-b(x))=a(x)$
$b(a(x))=0)$ and $b(b(x))=b(x)$ and $b(a(x)-b(x))=-b(x)$
And so $g(g(x))=a(x)+b(x)=x$
Q.E.D.

2) proof that any solution may be written in this form and so it's a general solution
=====================================================
Let $A=\{x$ such that $g(x)=x\}$
Let $B=\{x$ such that $g(x)=-x\}$
Obviously, since $g(x)$ is additive, $A,B$ are subvectorspaces of the $\mathbb Q$-vectorspace $\mathbb R$
$A\cap B=\{0\}$

Since $g(g(x))=x$, we get that $g(x+g(x))=x+g(x)$ and so $a(x)=\frac{x+g(x)}2\in A$
Since $g(g(x))=x$, we get that $g(x-g(x))=g(x)-x$ and so $b(x)=\frac{x-g(x)}2\in B$

And since $a(x)+b(x)=x$, we conclude that $A,B$ are supplementary subvectorspaces.

And we clearly have $g(x)=a(x)-b(x)$
Q.E.D.
\end{solution}



\begin{solution}[by \href{https://artofproblemsolving.com/community/user/72819}{Dijkschneier}]
	\begin{tcolorbox}P(c,y) is not equal to f(y-f(y))=-f(y)\end{tcolorbox}
In fact, it is : I just missed another detail  :blush: 
Of course, it doesn't matter since pco's solution was correct from the beginning.
\end{solution}
*******************************************************************************
-------------------------------------------------------------------------------

\begin{problem}[Posted by \href{https://artofproblemsolving.com/community/user/92753}{WakeUp}]
	Find all monotonic functions $u:\mathbb{R}\rightarrow\mathbb{R}$ which have the property that there exists a strictly monotonic function $f:\mathbb{R}\rightarrow\mathbb{R}$ such that
\[f(x+y)=f(x)u(x)+f(y) \]
for all $x,y\in\mathbb{R}$.

\begin{italicized}Vasile Pop\end{italicized}
	\flushright \href{https://artofproblemsolving.com/community/c6h403521}{(Link to AoPS)}
\end{problem}



\begin{solution}[by \href{https://artofproblemsolving.com/community/user/29428}{pco}]
	\begin{tcolorbox}Find all monotonic functions $u:\mathbb{R}\rightarrow\mathbb{R}$ which have the property that there exists a strictly monotonic function $f:\mathbb{R}\rightarrow\mathbb{R}$ such that
\[f(x+y)=f(x)u(x)+f(y) \]
for all $x,y\in\mathbb{R}$.\end{tcolorbox}

Let $P(x,y)$ be the assertion $f(x+y)=f(x)u(x)+f(y)$

Subtracting $P(x,0)$ from $P(x,y)$, we get $f(x+y)=f(x)+f(y)-f(0)$ and so, since strictly increasing, $f(x)=ax+b$ with $a>0$

And so $x=(x+\frac ba)u(x)$

Setting $x=-\frac ba$, we get $b=0$ and so the solution :

$u(x)=1$ $\forall x\ne 0$ and $u(0)=c$ any real
\end{solution}



\begin{solution}[by \href{https://artofproblemsolving.com/community/user/72819}{Dijkschneier}]
	Here is my solution :
[hide]Let u be a monotonic function which has that property.
P(x,y) : f(x+y) = f(x)u(x) + f(y)
P(0,0) ==> f(0)=f(0)u(0)+f(0)
==> f(0)=0 or u(0)=0
P(x,0) ==> f(x)=f(x)u(x)+f(0)
==> f(x)(1-u(x))=f(0)
If f(0)=0, then f(x)(1-u(x))=0, and because f is strictly monotonic, u(x) will always equal 1 except possibly in one point, and because it is monotonic, then we can conclude that it always equal 1.
Conversely, if $\forall x \in \mathbb{R} : u(x)=1$, then take f to be a monotonic solution to Cauchy's equation, to see that it works.

If $f(0)\neq 0$ and u(0)=0
Furthermore, we have by symetry : f(x)u(x)+f(y) = f(y)u(y)+f(x)
==> f(x)(u(x)-1) = f(y)(u(y)-1)
And so the function h(x)=f(x)(u(x)-1) is constant. (= c)
Suppose it is the allzero function. Then, because f is strictly monotonic, u(x) will always equals 1 except possibly in one point, and because it is monotonic, then we can conclude that it always equals 1.
But this is a contradiction with u(0)=0.

Now suppose that h it is not the allzero function. Then f(x) is never equal to zero, so it implies u(x)-1 = c\/f(x), u(x)=c\/f(x) + 1
Hence : f(x+y)=c+f(x)+f(y)
==> f(x+y) + c = (f(x)+c) + (f(y)+c)
==> g(x+y)=g(x)+g(y), where g(x)=f(x)+c (note that g is also monotonic)
==> g(x)=ax
==> f(x)=ax-c
But then, f(c\/a)=0, and so we have a contradiction

Hence the unique solution :\end{underlined}
$\forall x \in \mathbb{R} : u(x)=1$[\/hide]
\end{solution}



\begin{solution}[by \href{https://artofproblemsolving.com/community/user/72819}{Dijkschneier}]
	Dear pco, if you define $u(x)=1 \forall x\neq 0$ and $u(0)=c$, and $c\neq 1$, is u still monotonic ?  :maybe:
\end{solution}



\begin{solution}[by \href{https://artofproblemsolving.com/community/user/64716}{mavropnevma}]
	pco's solution emphasizes the weakness of this problem; the monotonicity of $u$ is never used, except at the very last moment, when having established that a function $u$ obeying the conditions must be $u(x)=1$ for all non-zero $x$, with arbitrary $u(0)$, hence in order to be monotonic needing having $u(0)$.

No wonder Patrick forgot to enforce this last requirement  :)
\end{solution}



\begin{solution}[by \href{https://artofproblemsolving.com/community/user/29428}{pco}]
	Thanks all for the remarks.
Indeed, since it was useless for finding solutions (just to reduce the result),  I really forgot the constraint $u(x)$ monotonic. :oops:
Sorry for the error.
\end{solution}
*******************************************************************************
-------------------------------------------------------------------------------

\begin{problem}[Posted by \href{https://artofproblemsolving.com/community/user/67218}{Fersolve}]
	Does there exists function, $f:\mathbb{N} \longrightarrow \mathbb{N}$, where $f(f(n))=n+2009$ for all $n \in \mathbb{N}$.
	\flushright \href{https://artofproblemsolving.com/community/c6h403771}{(Link to AoPS)}
\end{problem}



\begin{solution}[by \href{https://artofproblemsolving.com/community/user/29428}{pco}]
	\begin{tcolorbox}I'd like to see the different approaches to this problem :-D : 

Does there exists function, $f:\mathbb{N} \longrightarrow \mathbb{N}$, where $f(f(n))+n=2009 ,\forall n \in \mathbb{N}$

@pco, thanks, edited.\end{tcolorbox}

Obviously, no such functions exist : just choose $n=2010$ and the equation becomes $f(f(2010))=-1$, impossible since $f(x)$ is a function from $\mathbb N\to\mathbb N$ and so is $>0$
\end{solution}



\begin{solution}[by \href{https://artofproblemsolving.com/community/user/67218}{Fersolve}]
	OMG. :oops: THAT"S ANOTHER TYPO! :P -CHANGED NOW!
\end{solution}



\begin{solution}[by \href{https://artofproblemsolving.com/community/user/29428}{pco}]
	\begin{tcolorbox}I'd like to see the different approaches to this problem :-D : 

Does there exists function, $f:\mathbb{N} \longrightarrow \mathbb{N}$, where $f(f(n))=n+2009 ,\forall n \in \mathbb{N}$

@pco, thanks, edited.\end{tcolorbox}
With this double correction, this is a very classical problem without solution (since $2009$ is odd).

Simplest way to see this is to consider $A=f(\mathbb N)\cup[1,2009]$ and $B=(\mathbb N\cup[1,2009])\setminus A$ and to show that $A,B$ must be equinumerous, which is impossible.
\end{solution}



\begin{solution}[by \href{https://artofproblemsolving.com/community/user/67218}{Fersolve}]
	Right. I've got a similar solution too. Are there any other ones? :)
\end{solution}



\begin{solution}[by \href{https://artofproblemsolving.com/community/user/64716}{mavropnevma}]
	For such a function $f$ to exist, the relation $f(f(n))= n+2009$ makes $f$ injective. For each $n\in\mathbb{N}$ define $A_n =\{f^{2k}(n) = n + 2009k \mid k=0,1,\ldots\}$, where $f^m$ is the $m$-th iterate of $f$ (so $f^0(x) = x, f^1(x) = f(x), f^2(x) = (f\circ f)(x) = f(f(x)), \ldots$.

Notice that $f(A_n) = A_{f(n)}$, so $f(f(A_n)) = f(A_{f(n)}) = A_{n+2009} \subset A_n$, for all $n$. Also notice that we cannot have $f(n) \in A_n$, since then $f(n) = f^{2k}(n) = n+ 2009k$ for some $k \geq 0$, but then also $n + 2009 = f(f(n)) = f(f^{2k}(n)) = f^{2k}(f(n)) = f^{2k}(n+2009k) = n + 2009\cdot 2k$, absurd. 

This allows us to define $g: \mathbb{Z}_{2009} \to \mathbb{Z}_{2009}$ by $g(x) \equiv f(x) \not \equiv x \pmod {2009}$, so that $g(g(x)) \equiv x \pmod{2009}$. Thus $g\circ g = \textrm{id}$, with $g(x) \neq x$ for all $x$, impossible.
\end{solution}
*******************************************************************************
-------------------------------------------------------------------------------

\begin{problem}[Posted by \href{https://artofproblemsolving.com/community/user/30710}{huyhoang}]
	Let $f, g, h$ be three functions such that $f, g, h: \mathbb{R} \to \mathbb{R}$ and for all real $x$,
\[\left\{ \begin{gathered}
  f(x) = \frac{1}
{2}\left( {h(x + 1) + h(x - 1)} \right) \hfill \\
  g(x) = \frac{1}
{2}\left( {h(x + 1) + h(x - 4)} \right) \hfill \\ 
\end{gathered}  \right.\]
Express $h$ in terms of $f, g$
	\flushright \href{https://artofproblemsolving.com/community/c6h403948}{(Link to AoPS)}
\end{problem}



\begin{solution}[by \href{https://artofproblemsolving.com/community/user/29428}{pco}]
	\begin{tcolorbox}Let $f, g, h$ be three functions such that $f, g, h: \mathbb{R} \to \mathbb{R}$ and \[\left\{ \begin{gathered}
  f(x) = \frac{1}
{2}\left( {h(x + 1) + h(x - 1)} \right) \hfill \\
  g(x) = \frac{1}
{2}\left( {h(x + 1) + h(x - 4)} \right) \hfill \\ 
\end{gathered}  \right.\]
Express $h$ in terms of $f, g$\end{tcolorbox}
Let $e_1(x)$ be the assertion $h(x+1)+h(x-1)=2f(x)$
Let $e_2(x)$ be the assertion $h(x+1)+h(x-4)=2g(x)$

Subtracting $e_2(x+2)$ from $e_1(x+2)$, we get the new assertion :
$e_3(x)$ : $h(x+1)-h(x-2)=2f(x+2)-2g(x+2)$

Subtracting $e_3(x+2)$ from $e_1(x+2)$, we get the new assertion :
$e_4(x)$ : $h(x+1)+h(x)=2f(x+2)-2f(x+4)+2g(x+4)$

Subtracting $e_1(x+1)$ from $e_4(x+1)$, we get the new assertion :
$e_5(x)$ : $h(x+1)-h(x)=2f(x+3)-2f(x+5)+2g(x+5)-2f(x+1)$

Eliminating then $h(x+1)$ between $e_4(x)$ and $e_5(x)$, we get :

$\boxed{h(x)=f(x+1)+f(x+2)-f(x+3)-f(x+4)+f(x+5)+g(x+4)-g(x+5)}$
\end{solution}
*******************************************************************************
-------------------------------------------------------------------------------

\begin{problem}[Posted by \href{https://artofproblemsolving.com/community/user/83976}{sartt}]
	Let $P(a)$ be the largest prime positive divisor of $a^2 + 1$. Prove that exist infinitely many positive integers $a, b, c$ such that $P(a)=P(b)=P(c)$.

\begin{italicized}A. Golovanov\end{italicized}
	\flushright \href{https://artofproblemsolving.com/community/c6h404321}{(Link to AoPS)}
\end{problem}



\begin{solution}[by \href{https://artofproblemsolving.com/community/user/29428}{pco}]
	\begin{tcolorbox}Let $P(a)$ be a largest positive divisor of $a^2 + 1$. Prove that exist infinitely many positive integers $a, b, c$ such that $P(a)=P(b)=P(c)$.
(A. Golovanov).\end{tcolorbox}
Maybe I misunderstand, but I think that $P(a)=a^2+1$ and so there are not infinitely many positive integers $a, b, c$ such that $P(a)=P(b)=P(c)$.
\end{solution}



\begin{solution}[by \href{https://artofproblemsolving.com/community/user/83976}{sartt}]
	Thank, I'm so inattentive.
\end{solution}



\begin{solution}[by \href{https://artofproblemsolving.com/community/user/43727}{RaleD}]
	Unless I do something wrong, the solution will be like this:
Suppose contradiction:
We know that there are infinitely solutions of $a^2+1=2(b^2+1)$ so $a^2+1$ and $b^2+1$ should have same greatest prime factor. Now, let this prime be $p$; there is solution $p|x^2+1$ (we choose the smallest one) less than $p\/2$; so it means $p$ is greatest prime factor of $x^2+1$. So $x=b$. Now bigger is solution $y=p-b$, and also the biggest prime of $y^2+1$ is $p$; that means $y=a$. Now it is enough to find infinitely $a, b$ such $a^2-2b^2=1$ and $a+b$ is composite. The general formula is
$x_{n+1}=3x_n+4y_n$
$y_{n+1}=3y_n+2x_n$; and we can take $x_0=1, y_0=0$.
Now every pair $(x_n,y_n)$ produces exactly one $(x_{n+1},y_{n+1}$ $\pmod 5$, and every is produced by exactly one pair( again $\pmod 5$). So pair $(1,0) \pmod 5$ is produced infinitely times; then $5|x_{n+1}+y_{n+1}$.
\end{solution}



\begin{solution}[by \href{https://artofproblemsolving.com/community/user/35129}{Zhero}]
	For any integer $n$, let $c$ be the smallest positive even integer such that $c^2 + 1$ has a prime factor larger than $n$ (such an integer necessarily exists because the set of prime factors of the polynomial $Q(n) = 4n^2 + 1$ is infinite.) Since $p | (2p - c)^2 + 1$ and $2p-c$ is even, $2p - c \leq c$, so $p > c$. 

Let $q = P((p-c)^2 + 1)$. Since $p | (p-c)^2 + 1$, $q \leq p$. If $q > p$, then $pq | (p-c)^2 + 1$, so $(p-c)^2 + 1 \geq pq > p^2$, which is impossible, so we must have $q = p$. 

Let $r = P((c+p)^2 + 1)$. Suppose for the sake of contradiction that $r > p$. Note that $r | (c+p-r)^2 + 1$. Because $c+p-r$ is even and less than $c$, by the minimality of $c$ we must have $c+p-r < 0$, that is, $r > c+p$. Since $2pr | (c+p)^2 + 1$, 
\[ (c+p)^2 + 1 \geq 2pr \geq 2p^2 + 2pc \implies c^2 + 1 \geq p^2 \geq (c+1)^2 > c^2 + 2c + 1, \]
which is impossible. 

Thus, $r \leq p$, so $p = P(c^2 + 1) = P((p-c)^2 + 1) = P((c+p)^2 + 1)$. Since $c$ can grow arbitrarily large, our proof is complete.
\end{solution}



\begin{solution}[by \href{https://artofproblemsolving.com/community/user/110389}{polya78}]
	There are an infinite number of integral solutions of $A^2 + 1 = 2B^2$. (Examine $(1+\sqrt2)^{2k+1}$).

So if $p =P(A)$, then $p|B$.  (A and B are both odd, so $4\nmid (A^2 +1)$, and p must be odd.)  So, $p\lneq A$.  Let m and n be the two solutions of $x^2 +1\equiv 0 \bmod p$, with $m,n \lneq p$.  Then we have that m,n, and A are all distinct and $P(A)=P(m)=P(n) = p.$
\end{solution}



\begin{solution}[by \href{https://artofproblemsolving.com/community/user/127581}{sahadian}]
	1)lemma1: we now that for any $(p=4k+1)$ we have a number $t<P$ that $p\mid t^2+1$ (It is easy to prove this lemma because $((\frac{p-1}{2})!)^2\equiv -1)$
2)lemma2: there is infinitely $P=4k+1$ that we can write in the form $P=5q+3$ or$ p=5q+2$ 
consider infinitely  $ p=4k+1$ and $p=5q+2 $ or $ 5q+3$  we now that we have a number $ t<p$ that $p\mid t^2+1$ so $p\mid (p-t)^2+1$ and p is the greatest prime divisor of $t^2+1$ and $(p-t)^2+1$ 
consider $(P+t)^2+1$ and $(2p-t)^2+1$  it is easy to show that one of this to number divisible by 5 
WLOG Suppose that  $5\mid(P+t)^2+1$  
$p\mid t^2+1$ so $p\mid(P+t)^2+1$  so $5p\mid(P+t)^2+1$ and we know that $\frac{(p+t)^2+1}{5p}<p$ so p is the greatest divisor of $(p+t)^2+1$
so p is the gratest divisor of $(p+t)^2+1$ and  $t^2+1$ and $(p-t)^2+1$
\end{solution}



\begin{solution}[by \href{https://artofproblemsolving.com/community/user/16261}{Rust}]
	You need at least 3 different numbers $a<b<c$, suth that $P(a)=P(b)=P(c)=p$.
Let $p>5$ is prime form $p=1\mod 4$.  Then exist unique $a<p, a-odd$, suth that $p|a^2+1$.  Let $b=p-a$. Because $\frac{a^2+1}{p}<p, \frac{b^2+1}{p}<p$ we get $P(a)=p=P(b)$.
Consider numbers $x=p+a,2p-a, 2p+a$.  If one of $x$ give residue $\pm 2\mod 5$, then take $c=x$. Because $5|c^2+1$ and for case $x=2p+a$ $10|x^2+1$, we get $P(c)=p$.
\end{solution}



\begin{solution}[by \href{https://artofproblemsolving.com/community/user/69901}{dinoboy}]
	OK I'm confused that nobody posted this simple solution

Note that every prime $p \equiv 1 \pmod{4}$ has at least two solutions : just pick the two integers $a, p-a < p$ such that $a^2 + 1 \equiv 0 \pmod{p}$. Thus it suffices to find infinitely many integers $k$ such that $P(k) < k$.

This is easy. Note that $(2a^2)^2 + 1 = 4a^4 + 1 = (1 + 2a^2 + 2a)(1 + 2a^2 - 2a)$. Now just pick $a \equiv 1 \pmod{5}$ to get $P(2a^2) < 2a^2$ for $a > 1$ so we're done.
\end{solution}



\begin{solution}[by \href{https://artofproblemsolving.com/community/user/76247}{yugrey}]
	Note that $P(a^2+a+1)=max(P(a),P(a+1))$ and $P(a^2-a+1)=max(P(a-1),P(a))$.

This is because $(a^2+1)((a+1)^2+1)=(a^2+a+1)^2+1$.

Now, if $P(a-1)$ is at most $P(a)$ and $P(a+1)$ is also at most $P(a)$, then $P(a)=P(a^2-a+1)=P(a^2+a+1)$.

If there are infinitely many $a$ such that $P(a)$ is at least both $P(a-1)$ and $P(a+1)$, then just choose $b=a^2-a+1$ and $c=a^2+a+1$ for these $a$ and there are infinitely many solutions.

Otherwise, for all $a>K$ for some $k>9000$, $P(a)$ is monotonically increasing.

However, $max(P(K+9000),P(K+9001))$ is $P((K+9000)^2+9001)$ but this is a contradiction so we are done.
\end{solution}



\begin{solution}[by \href{https://artofproblemsolving.com/community/user/151851}{Mathematicalx}]
	\begin{tcolorbox}OK I'm confused that nobody posted this simple solution\end{tcolorbox}
Note that $k=2a$ is not a constant. So your claim is absolutely false dinoboy.
\end{solution}



\begin{solution}[by \href{https://artofproblemsolving.com/community/user/69901}{dinoboy}]
	How is that relevant at all to my proof?
\end{solution}



\begin{solution}[by \href{https://artofproblemsolving.com/community/user/151851}{Mathematicalx}]
	Dear dinoboy, how did you get result while $k=2a$ is growing?
\end{solution}



\begin{solution}[by \href{https://artofproblemsolving.com/community/user/69901}{dinoboy}]
	I don't see how "$k=2a$ is growing" is relevant at all to my proof. I show there exist infinitely many $k$ such that $P(k) < k$. Let these values be $k_1, k_2,...$ Then by choosing $a,P(k_i)-a$ to be the two square roots to $-1$ modulo $P(k_i)$ we get $P(a) = P(p-a) = P(k_i)$ and these are all distinct as $a \neq P(k_i)-a$ and $a, P(k_i) - a < P(k_i) < k_i$ so we are done. These are all pretty easy steps that are hardly worth filling in.
\end{solution}



\begin{solution}[by \href{https://artofproblemsolving.com/community/user/187815}{JH_KOR}]
	dinoboy,

how are you sure to P(a) = P(k_i) ?

can we find infinitely many k_i  that the largest prime of a^2 + 1  equals the largest prime of k_i ^2 + 1 ?

 i'm sorry for LateX illiteracy
\end{solution}



\begin{solution}[by \href{https://artofproblemsolving.com/community/user/46927}{fattypiggy123}]
	@above: $P(a) = P(k_i)$ because one has $a < P(k_i)$ since $a$ is a residue mod $P(k_i)$ and hence one has $a^2 + 1 < P(k_i)^2$. In particular, $a^2 + 1$ cannot have another prime factor larger than $P(k_i)$. The same argument applies for $(P(k_i) - a)^2 + 1$ and finally these two numbers are different from $P(k_i)$ since one has $a < P(k_i) < k_i$. 
\end{solution}



\begin{solution}[by \href{https://artofproblemsolving.com/community/user/345008}{Kayak}]
	Please check my solution :help:

Let $M(a)$ denote the maximum prime divisor of $a^2+1$. Let $S(p)$ denote the smallest integer $a$ such that $a^2+1 \equiv 0 \mod p$, where $p$ is a prime $\equiv 1 \mod 4$

\begin{bolded}Claim \end{bolded}: For every prime $p \equiv 1 \mod 4$, there exists another prime $q \geq p$ with $q \equiv 1 \mod 4$ and an integer $r > q$ such that $M(r) = q$.
[hide = Proof] Note that as $p$ divides $(p+S(p))^2+1$, if $M(p+S(p)) = p$, then we're done by setting $q = p$ and $r  = p+S(p)$. Otherwise, let $a := \frac{(p+S(p))^2 + 1}{p}$. Note that $p < a < 2.25p$. We split it into two cases.

(Note that the factorization of $a$ must be one of the two cases, otherwise we would have all prime factors of $a$ smaller than or equal to $p$)
\begin{bolded} Case - 0 \end{bolded}: $a = 2 * prime := 2 * s $
[hide = Trivial] By some trivial bounding, we have $s < p+S(p)$, hence we choose $q= s$, $r = p+S(p)$.[\/hide]
\begin{bolded} Case - 1 \end{bolded}: $a = prime$
[hide = Bash] By some trivial bounding, we have $a > p+S(p)$. Defining $$ f(p) := a = \frac{(p+S(p))^2 + 1}{p} = p + 2S(p) + \frac{S(p)^2 + 1}{p}$$ , we prove that eventually one term in the sequence $\{p, f(p), f(f(p)), f(f(f(p))), \cdots, \}$ becomes composite, hence getting rid of this case. As $ \frac{a}{2} < p+S(p) < a$, we have $a - S(f(p)) = p +  S(p)$, hence we get the recurrence $$S(f(p)) = a+p-S(p) = S(p) + \frac{(S(p))^2 + 1}{p}$$
The "proof" is trivial, we just run a python [hide=code] [code=python]d = {1: 1, 2: 3, 3: 2, 4: 4} #basically stores the inverse modulo 5
def chain(a,b): #Starts with p, S(p), and prints the steps until f(f(...f(p)...) is divisible by 5, thus composite
	i = 0
	print(a,b)
	while i < 27:
		if a == 0:
			print("--------------")
			return i
		else:
			k = (a + 2*b + (b**2 + 1)*d[a]) % 5
			b = ( b + (b**2 + 1)*d[a]) % 5
			i = i + 1
			a = k
			print(a,b)

for a in range(1, 5): #all possible values of p mod 5 are 1,2,3,4
	for b in range(5): #all possible values of S(p) mod 5 are 0,1,2,3,4,5
		chain(a,b)[\/code][\/hide] and see that eventually $f(\cdots f(p) \cdots )$ is divisible by $5$ [\/hide] [\/hide]
 (The question is equivalent to proving the claim since you already get two integers $0< b = S(q) < \frac{q}{2} < c = q-S(q) < q$ such that $M(b) = M(c) = q$, for free)
\end{solution}



\begin{solution}[by \href{https://artofproblemsolving.com/community/user/345008}{Kayak}]
	\begin{tcolorbox}Unless I do something wrong, the solution will be like this:
Suppose contradiction:
We know that there are infinitely solutions of $a^2+1=2(b^2+1)$ so $a^2+1$ and $b^2+1$ should have same greatest prime factor. Now, let this prime be $p$; there is solution $p|x^2+1$ (we choose the smallest one) less than $p\/2$; so it means $p$ is greatest prime factor of $x^2+1$. So $x=b$. Now bigger is solution $y=p-b$, and also the biggest prime of $y^2+1$ is $p$; that means $y=a$. Now it is enough to find infinitely $a, b$ such $a^2-2b^2=1$ and $a+b$ is composite. \end{tcolorbox}

Sorry, but I don't understand how you get $x = b$ and $p-x = a$. For example, we have $239^2 + 1 = 2(169^2+1)$. But highest prime factor of $239^2+1$ is $13$, and highest prime factor of $70^2+1$ is $29$ :ewpu:


\end{solution}
*******************************************************************************
-------------------------------------------------------------------------------

\begin{problem}[Posted by \href{https://artofproblemsolving.com/community/user/43536}{nguyenvuthanhha}]
	Find all pair of functions $ f,g : \mathbb{N}\cup \{0\} \to\mathbb{N} \cup \{0\}$ such that
\[ f(n) + f(n + g(n)) = f(n+1), \quad \forall n \in \mathbb{N} \cup \{0\}.\]
	\flushright \href{https://artofproblemsolving.com/community/c6h404408}{(Link to AoPS)}
\end{problem}



\begin{solution}[by \href{https://artofproblemsolving.com/community/user/29428}{pco}]
	\begin{tcolorbox}Let $ \mathbb{N} $ be the set of all nonegative integers 

   Find all pair of functions $ f : \mathbb{N} \mapsto \mathbb{N} \ ; \ g : \mathbb{N} \mapsto \mathbb{N}$ such that :

 $ f(n) + f(n + g(n)) = f(n+1) \ \  \forall n \in \mathbb{N}$\end{tcolorbox}
I'll use the notation $\mathbb N_0$ in the following for the required set of all nonnegative integers.

$f(n)=0$ $\forall n$ is a solution.
Let us from now look for non allzero solutions.

$f(n+1)-f(n)=f(n+g(n))\ge 0$ and so $f(n)$ is a non decreasing function.
Let then $a=\min\{x\in\mathbb N_0$ such that $f(x)\ne 0\}$. $a$ exists since $f(n)$ is not allzero in this part of the analysis.

If $g(n)>0$ for some $n$, then $f(n+g(n))\ge f(n+1)=f(n)+f(n+g(n))$ and so $f(n)=0$
So $g(n)=0$ $\forall n\ge a$
So $f(n+1)=2f(n)$ $\forall n\ge a$
So $f(n)=2^{n-a}f(a)$ $\forall n\ge a$ and $f(n)$ is injective for $n\ge a$

If $a>0$, we get $f(a)=f(a-1)+f(a-1+g(a-1))=f(a-1+g(a-1))$
So $g(a-1)>0$ (else $f(a)=f(a-1)=0$) and so $a=a-1+g(a-1)$ and $g(a-1)=1$

If $a>1$, let $n<a-1$ : $f(n+1)=f(n)+f(n+g(n))$ becomes $f(n+g(n))=0$ and so $g(n)\in[0,a-n-1]$

\begin{bolded}Hence the solutions \end{bolded}\end{underlined}(it's easy to check that these mandatory conditions are sufficient) :

\begin{bolded}S1\end{bolded}\end{underlined} : $f(n)=0$ $\forall n$ and any $g(n)$

\begin{bolded}S2 \end{bolded}\end{underlined}: $f(n)=c2^n$ and $g(n)=0$ $\forall n$ with $c$ any positive integer.

\begin{bolded}S3\end{bolded}\end{underlined} : 
$f(0)=0$ and $f(n)=c2^{n-1}$ $\forall n>0$ with $c$ any positive integer.
$g(0)=1$ and $g(n)=0$ $\forall n>0$

\begin{bolded}S4\end{bolded}\end{underlined} :
Let $a>1$ and $c>0$ two integers
$f(n)=0$ $\forall n\in[0,a)$ and $f(n)=c2^{n-a}$ $\forall n\ge a$
$g(n)$ is any value in $[0,a-n-1]$ $\forall n\in[0,a-1)$
$g(a-1)=1$
$g(n)=0$ $\forall n\ge a$
\end{solution}
*******************************************************************************
-------------------------------------------------------------------------------

\begin{problem}[Posted by \href{https://artofproblemsolving.com/community/user/104682}{momo1729}]
	Find all functions $f:\mathbb{R}\rightarrow \mathbb{R}$ such that for all $x,y, \in \mathbb{R}$,
 \[xf(x+xy)=xf(x)+f(x^{2})\cdot f(y).\]
	\flushright \href{https://artofproblemsolving.com/community/c6h404589}{(Link to AoPS)}
\end{problem}



\begin{solution}[by \href{https://artofproblemsolving.com/community/user/29428}{pco}]
	\begin{tcolorbox}Find all functions $f:\mathbb{R}\rightarrow \mathbb{R}$ such that for all $x,y$ in $\mathbb{R}$,
 $xf(x+xy)=xf(x)+f(x^{2}).f(y)$\end{tcolorbox}
Let $P(x,y)$ be the assertion $xf(x+xy)=xf(x)+f(x^2)f(y)$

$P(0,0)$ $\implies$ $f(0)=0$
If $f(1)=0$, then $P(1,x-1)$ $\implies$ $f(x)=0$ which indeed is a solution
Let us from now consider that $f(1)=a\ne 0$

If $a\ne 1$, $P(1,x)$ $\implies$ $f(x+1)=af(x)+a$ and we easily get $f(n)=a\frac{a^n-1}{a-1}$ $\forall n\in\mathbb N$
Plugging this expression in $P(m,n)$, we see that this is not a solution (rather ugly, I think).

So $a=1$ and $P(1,x)$ $\implies$ $f(x+1)=f(x)+1$ and so $f(n)=n$ and $f(x+n)=f(x)+n$

$P(x,-1)$ $\implies$ $f(x^2)=xf(x)$
Plugging this in $P(x,y)$, we get $xf(x(y+1))=xf(x)(f(y)+1)=xf(x)f(y+1)$

And so $f(xy)=f(x)f(y)$

$P(x,y)$ becomes then $xf(x)f(y+1)=xf(x)+f(x)^2f(y)$ $\iff$ $xf(x)(f(y)+1)=xf(x)+f(x)^2f(y)$

And so, setting $y=1$ : $f(x)(f(x)-x)=0$ and so $\forall x$, either $f(x)=0$, either $f(x)=x$
But, if for some $x\ne 0$, we have $f(x)=0$, then $f(x+1)=f(x)+1$ implies $f(x+1)=1$ which is impossible since either $f(x+1)=x+1\ne 1$, either $f(x+1)=0\ne 1$

So $f(x)=x$ $\forall x$, which indeed is a solution.

\begin{bolded}Hence the answer \end{bolded}\end{underlined}: 
$f(x)=0$ $\forall x$
$f(x)=x$ $\forall x$
\end{solution}



\begin{solution}[by \href{https://artofproblemsolving.com/community/user/81769}{arshakus}]
	$ f:\mathbb{R}\rightarrow\mathbb{R} $
$ xf(x+xy)=xf(x)+f(x^{2}).f(y) $
$y=-1=>0=xf(x)+f(x^2)f(-1)$
$x=-1=>f(-1)=f(1)f(-1)=> f(-1)=0$ or $f(1)=1$
1) $f(-1)=0=>xf(x)=0=>f(x)=0$ where $x!=0$ if $x=0=>f(0)=0=>f(x)=0 \forall x\in\mathbb{R}$
2) $f(1)=1=>x=1=>f(x+1)=f(x)+1$
$f(-1)=-1$
$y=-1=>xf(x)=f(x^2)=>xf(x+xy)=xf(x)(f(y)+1)=>f(x+xy)=f(x)f(y+1)=>y=x-1=>f^2(x=f(x^2)=>xf(x)=f^2(x)=>f(x)(f(x)-x)=0$ if there exists such nonzero $a,b$ that $f(a)=0=>x=a=>f(a+ay)=0 \forall y\in\mathbb{R}$ if there exists such nonzero $b$ hat $f(b)=b=>y=\frac  {b-a} {a}=>f(b)=0$ which is condraction.)

Thus the answer is
$f(x)=0 \forall x \in\mathbb{R}$
$f(x)=x \forall x \in\mathbb{R}$
\end{solution}



\begin{solution}[by \href{https://artofproblemsolving.com/community/user/152203}{borntobeweild}]
	[hide="Another solution to this nice problem"]Copying pco's wonderful notation (did you make it up, pco?) we let $P$ be the given assertion. We then get:

$P(0,0): f(0)=0$
$P(1,-1): f(1)(f(-1)+1)=0\implies f(1)=0\,\, \text{or} \,f(-1)=-1$

We now have two cases:

Case $1$ (The easy one): $f(1)=0$

$P(1,x-1): f(x)=0$, which is a solution.

Case $2$ (The harder one): $f(-1)=-1$

$P(-1,-1): -f(1)+1=0\implies f(1)=1$
$P(1,x): f(x+1)=f(x)+1$
$P(1,x-1): f(x-1)=f(x)-1$

Now consider only $x\neq 0$

$P(x,\frac{1}{x}): xf(x+1)=xf(x)+f(x^2)f(\frac{1}{x})\implies f(x^2)f(\frac{1}{x})=x$

This gives:

$f(x+xy)=f(x)+\frac{f(y)}{f(\frac{1}{x})}$

Calling this assertion $Q$ for convenience, we have:

$Q(x,-1): f(x)-\frac{1}{f(\frac{1}{x})}=0\implies f(\frac{1}{x})=\frac{1}{f(x)}\implies f(x^2)=xf(x)$

We now take:

$P(x,x): xf(x^2+x)=xf(x)+f(x^2)f(x)\implies f(x^2+x)=f(x)^2+f(x)$
$P(x^2,\frac{1}{x}): x^2f(x^2+x)=x^2f(x^2)+f(x^4)f(\frac{1}{x})\implies f(x^2+x)=xf(x)+x$

Combining these, we get:

$f(x)^2+f(x)=xf(x)+x\implies \forall x, f(x)=x \,\,\text{or}\, f(x)=-1$

But as $f(x+1)=f(x)+1$, we get $f(x)=-1$ only if $x=-1$, so $f(x)=x$

Therefore, our solutions are $\boxed{f(x)=0, f(x)=x}$[\/hide]
\end{solution}



\begin{solution}[by \href{https://artofproblemsolving.com/community/user/170079}{MMEEvN}]
	$P(0,0) \Rightarrow f(0)=0$
$P(x,-1) \Rightarrow xf(x)=-f(x^2)f(-1)$
$P(x,x-1) \Rightarrow xf(x^2)=xf(x)+f(x^2)f(x-1) \Rightarrow xf(x^2)=-f(x^2)f(-1)+f(x^2)f(x-1) \Rightarrow f(x^2)=0$ or $f(x-1)=x+c$ which is  $f(x^2)=0$ or $f(x-1)=x-1$ proving that a mixed function does not occur is not hard
\end{solution}
*******************************************************************************
-------------------------------------------------------------------------------

\begin{problem}[Posted by \href{https://artofproblemsolving.com/community/user/33535}{wangsacl}]
	Find all non-decreasing functions $f:\mathbb{R}\rightarrow\mathbb{R}$ such that
(1) $f(0)=0, f(1)=1$, and
(2) $f(a)+f(b)=f(a)f(b)+f(a+b-ab)$ for all reals $a$ and $b$ such that $a<1<b$.
	\flushright \href{https://artofproblemsolving.com/community/c6h404926}{(Link to AoPS)}
\end{problem}



\begin{solution}[by \href{https://artofproblemsolving.com/community/user/16261}{Rust}]
	Let $g(x)=1-f(1-x)$, then $g(0)=0,g(1)=1$- nondecreasing and
$g(xy)=g(x)g(y)$. Therefore $g(x)=x^a, a>0$ and $f(x)=1-(1-x)^a.$
\end{solution}



\begin{solution}[by \href{https://artofproblemsolving.com/community/user/29428}{pco}]
	\begin{tcolorbox}Let $g(x)=1-f(1-x)$, then $g(0)=0,g(1)=1$- nondecreasing and
$g(xy)=g(x)g(y)$. Therefore $g(x)=x^a, a>0$ and $f(x)=1-(1-x)^a.$\end{tcolorbox}
To be more precise, it's a common error to say that general solution of $g(xy)=g(x)g(y)$ is $x^a$

This function $x^a$ is not defined for $x<0$ when $a\notin \mathbb Z$

The general solutions of $g(xy)=g(x)g(y)$ are :
$g(x)=0$
$g(x)=|x|^a$ and $g(0)=0$
$g(x)=sign(x)|x|^a$ and $g(0)=0$

In our case, the $g(0)=1$ condition cancels the first solution and the "non decreasing" condition cancels the second and implies $a\ge 0$ in the third, giving two general solutions (distinguishing $a=0$) :
$g(x)=sign(x)|x|^a$ and $g(0)=0$ with $a>0$ (not exactly the one you gave)
$g(x)=sign(x)$ and $g(0)=0$

And so two solutions here for $f(x)$ :
$f(x)=1-sign(1-x)|1-x|^a$ and $f(1)=1$ for any $a>0$
$f(x)=1-sign(1-x)$ and $f(1)=1$
\end{solution}



\begin{solution}[by \href{https://artofproblemsolving.com/community/user/44083}{jgnr}]
	http://www.artofproblemsolving.com/Forum/viewtopic.php?t=17477
\end{solution}



\begin{solution}[by \href{https://artofproblemsolving.com/community/user/1430}{JBL}]
	\begin{tcolorbox}The general solutions of $g(xy)=g(x)g(y)$ are :
$g(x)=0$
$g(x)=|x|^a$ and $g(0)=0$
$g(x)=sign(x)|x|^a$ and $g(0)=0$\end{tcolorbox}  Actually there are even more than this (if you believe in the axiom of choice).
\end{solution}



\begin{solution}[by \href{https://artofproblemsolving.com/community/user/29428}{pco}]
	\begin{tcolorbox}[quote="pco"]The general solutions of $g(xy)=g(x)g(y)$ are :
$g(x)=0$
$g(x)=|x|^a$ and $g(0)=0$
$g(x)=sign(x)|x|^a$ and $g(0)=0$\end{tcolorbox}  Actually there are even more than this (if you believe in the axiom of choice).\end{tcolorbox}

I believe :) ...

And you are quite right :oops:, thanks

These are only continuous solutions (with the condition $a>0$)
\end{solution}
*******************************************************************************
-------------------------------------------------------------------------------

\begin{problem}[Posted by \href{https://artofproblemsolving.com/community/user/81769}{arshakus}]
	1. Find all function $f: \mathbb R^{+} \to \mathbb R^{+}$ such that \[f(x) \cdot f(yf(x))=f(x+y)\]for all $x,y>0$.

2. Find all function $f: \mathbb R^{\geq 0} \to \mathbb R^{\geq 0}$ such that \[f(x) \cdot f(yf(x))=f(x+y)\]for all $x,y \geq 0$.
	\flushright \href{https://artofproblemsolving.com/community/c6h405083}{(Link to AoPS)}
\end{problem}



\begin{solution}[by \href{https://artofproblemsolving.com/community/user/31915}{Batominovski}]
	\begin{tcolorbox}1. Find all function $f: \mathbb R^{+} \to \mathbb R^{+}$ such that \[f(x) \cdot f(yf(x))=f(x+y)\]for all $x,y>0$.\end{tcolorbox}


Apparently, if $k \geq 0$, then $f(x) = \frac{1}{1+kx}$ is a solution.  I have no proof that there are no other functions that satisfy the equation, and I think there might be other solutions.

If $f:\mathbb{R}_{\geq 0} \to \mathbb{R}_{\geq 0}$, then for any $\alpha \geq 0$, \[f(x)=\{
\begin{array}{ll}
\frac{\alpha}{\alpha-x} & \text{if }0\leq x < \alpha
\\
0 & \text{if }x \geq \alpha\,
\end{array}\] is also a solution.
\end{solution}



\begin{solution}[by \href{https://artofproblemsolving.com/community/user/81769}{arshakus}]
	but it was only answer))))lol
I need solution)):-D
\end{solution}



\begin{solution}[by \href{https://artofproblemsolving.com/community/user/29428}{pco}]
	\begin{tcolorbox}$f:R^{+}->R^{+}$
$f(x)f(yf(x))=f(x+y)$
determine $f$.\end{tcolorbox}
See http://www.artofproblemsolving.com/Forum/viewtopic.php?f=36&t=373453
\end{solution}



\begin{solution}[by \href{https://artofproblemsolving.com/community/user/31915}{Batominovski}]
	To Patrick (or whomever it may interest), do you have some time to deal with $f:\mathbb{R}_{\geq 0} \to \mathbb{R}_{\geq 0}$ case?  I don't know if there is an olympiad-level solution to this modification.  There is, however, an IMO problem similar to this problem (with some additional constraints which I can't recall it from the top of my head).  It would be great if you could solve this modified problem.
\end{solution}



\begin{solution}[by \href{https://artofproblemsolving.com/community/user/29428}{pco}]
	\begin{tcolorbox}2. Find all function $f: \mathbb R^{\geq 0} \to \mathbb R^{\geq 0}$ such that \[f(x) \cdot f(yf(x))=f(x+y)\]for all $x,y \geq 0$.\end{tcolorbox}

\begin{italicized}\begin{bolded}Modified problem where the function if from $\mathbb R_{\ge 0}\to\mathbb R_{\ge 0}$\end{bolded}\end{italicized}

Let $P(x,y)$ be the assertion $f(x)f(yf(x))=f(x+y)$

$P(0,0)$ $\implies$ $f(0)\in\{0,1\}$
If $f(0)=0$ then $P(0,x)$ $\implies$ $f(x)=0$ $\forall x$ which indeed is a solution.

Let us from now consider $f(0)=1$

If $f(x)>0$ $\forall x>0$, then :
The previous posts imply $f(x)=\frac 1{1+ax}$ for some $a\ge 0$ and for any $x>0$ 
And since $f(0)=1$, this formula is true again for $x=0$ and it's easy to see that this indeed is a solution.

If $\exists u>0$ such that $f(u)=0$, then $P(u,x)$ $\implies$ $f(u+x)=0$ $\forall x\ge 0$
Let then $a=\inf\{x>0$ such that $f(x)=0\}$

If $a=0$, we get $f(x)=0$ $\forall x>0$ and it's immediate to see that this indeed is a solution (including the fact that $f(0)=1$).

If $a>0$, we get $f(x)=0$ $\forall x>a$ and $f(x)>0$ $\forall x<a$

Consider now $x<a$ and $x+y>a$ : $P(x,y)$ $\implies$ $f(yf(x))=0$ and so $yf(x)\ge a$
So $f(x)\ge \frac ay$ $\forall y\in(a-x,+\infty)$
So $f(x)\ge \frac a{a-x}$ $\forall x\in(0,a)$

Consider now $x<a$ and $x+y<a$ with $y\ne 0$ : $P(x,y)$ $\implies$ $f(yf(x))\ne 0$ and so $yf(x)\le a$
So $f(x)\le \frac ay$ $\forall y\in(0,a-x)$
So $f(x)\le \frac a{a-x}$ $\forall x\in(0,a)$

So we got a mandatory condition : $f(x)=\frac a{a-x}$ $\forall x\in (0,a)$, still true for $x=0$
Then $P(\frac a2,\frac a2)$ $\implies$ $f(a)=0$ and we got the function :
$f(x)=\frac a{a-x}$ $\forall x\in[0,a)$ and $f(x)=0$ $\forall x\ge a$ which indeed is a solution.


\begin{bolded}Hence the solutions \end{bolded}\end{underlined}:
S1 : $f(x)=0$ $\forall x$

S2 : $f(x)=0$ $\forall x>0$ and $f(0)=1$

S3 : $f(x)=\frac 1{1+ax}$ $\forall x$ and for any $a\ge 0$

S4 : $f(x)=\frac a{a-x}$ $\forall x\in[0,a)$ and $f(x)=0$ $\forall x>a$ for any $a>0$
\end{solution}
*******************************************************************************
-------------------------------------------------------------------------------

\begin{problem}[Posted by \href{https://artofproblemsolving.com/community/user/109054}{doper}]
	Determine all such funtions $f,g,h$ from $\mathbb R^+$ to itself such that $f(g(h(x))+y)+h(z+f(y))=g(y)+h(y+f(z))+x$ for all $x,y,z>0$.
	\flushright \href{https://artofproblemsolving.com/community/c6h405320}{(Link to AoPS)}
\end{problem}



\begin{solution}[by \href{https://artofproblemsolving.com/community/user/29428}{pco}]
	\begin{tcolorbox}Determine all such funtions $f,g,h$ from $R^+$ to itself, that $f(g(h(x))+y)+h(z+f(y))=g(y)+h(y+f(z))+x$.\end{tcolorbox}
I supposed that the domain of functional equation is the same than domain of functions (better to indicate both domains).

Let $P(x,y,z)$ be the assertion $f(g(h(x))+y)+h(z+f(y))=g(y)+h(y+f(z))+x$
$P(x,y,y)$ $\implies$ $f(g(h(x))+y)=g(y)+x$ and so $h(x)$ is injective

Subtracting $P(x,y,y)$ from $P(x,y,z)$, we get $h(z+f(y))=h(y+f(z))$ and so, since $h(x)$ is injective :
$z+f(y)=y+f(z)$ and so $f(x)=x+a$ for some $a\ge 0$

Plugging this in $P(1,x,x)$, we get $g(h(1))+x+a=g(x)+1$ and so $g(x)=x+b$ for some $b\ge 0$

Plugging $f(x)=x+a$ and $g(x)=x+b$ in original equation, we get $h(x)=x-a$ and so $a=0$

Hence the solutions : $\boxed{(f,g,h)=(x,x+b,x)}$ for any real $b\ge 0$


And, btw :
a) Could you, please, give us the two solutions you promised us in the other posts.
b) What is DMO ?
\end{solution}
*******************************************************************************
-------------------------------------------------------------------------------

\begin{problem}[Posted by \href{https://artofproblemsolving.com/community/user/96756}{JustN}]
	Given $ f:\mathbb{R}^{\geq 0}\to\mathbb{R}^{\geq 0} $ and for all $x,y\in \mathbb{R}^{\geq 0})$,
\[f(x^2)+f(y)=f(x^2+y+xf(4y)).\]
a) Prove that if $a<b$ then $f(a)\leq\ f(b)$.
b) Find all such functions $f$.
	\flushright \href{https://artofproblemsolving.com/community/c6h405554}{(Link to AoPS)}
\end{problem}



\begin{solution}[by \href{https://artofproblemsolving.com/community/user/29428}{pco}]
	\begin{tcolorbox}Given $ f:\mathbb{R}_{0}\to\mathbb{R}_{0} $ and $ (\forall x,y\in \mathbb{R}_{0})$.
$f(x^2)+f(y)=f(x^2+y+xf(4y))$.
a) Prove that: if $a<b$ then $f(a)\leq\ f(b)$.
Where $ \mathbb{R}_{0}$ is the set of non negative reals.\end{tcolorbox}
Let $y\ge 0$ and $t>0$
The quadratic $x^2+xf(4y)-t$ has a non negative discriminant and so two real roots, whose product is $<0$ and so one positive real root.

And since $f(y+t)=f(y)+f(x^2)\ge f(y)$, we get the required result.
\begin{tcolorbox}Given $ f:\mathbb{R}_{0}\to\mathbb{R}_{0} $ and $ (\forall x,y\in \mathbb{R}_{0})$.
$f(x^2)+f(y)=f(x^2+y+xf(4y))$.
b) Find all such functions $f$.

Where $ \mathbb{R}_{0}$ is the set of non negative reals.\end{tcolorbox}
The previous result shows that $f(x)$ is non decreasing.
Let $P(x,y)$ be the assertion $f(x^2)+f(y)=f(x^2+y+xf(4y))$

$P(0,0)$ $\implies$ $f(0)=0$

If $\exists u>0$ such that $f(u)=0$, then $P(\sqrt u,u)$ $\implies$ $f(2u+\sqrt uf(4u))=0$
And since $f(x)$ is non decreasing, this implies $f(2u)=0$ and so $f(2^nu)=0$ and so $f(x)=0$ $\forall x$, which indeed is a solution.

Let us from now look for non allzero solutions.
We just got that $f(x)>0$ $\forall x>0$

Looking back at the proof of a) above, we can now say :
Let $y\ge 0$ and $t>0$
The quadratic $x^2+xf(4y)-t$ has a non negative discriminant and so two real roots, whose product is $<0$ and so one positive real root.

And since $f(y+t)=f(y)+f(x^2)>f(y)$, we get now that $f(x)$ is increasing, and so \begin{bolded}injective\end{bolded}\end{underlined}.

$P(x,1)$ $\implies$ $f(x^2)+f(1)=f(x^2+1+xf(4))$
$P(1,x^2)$ $\implies$ $f(x^2)+f(1)=f(x^2+1+f(4x^2))$

And so $f(x^2+1+xf(4))=f(x^2+1+f(4x^2))$ and so, since injective : $xf(4)=f(4x^2)$

So $f(4x^2)=f(4)x$ and so $f(x)=a\sqrt x$ where $a>0$ which indeed is a solution

And since the previous solution $f(x)=0$ $\forall x$ may be obtained thru this formula when setting $a=0$, we get the set of solutions :

$\boxed{f(x)=a\sqrt x}$ $\forall x$ and for any real $a\ge 0$
\end{solution}
*******************************************************************************
-------------------------------------------------------------------------------

\begin{problem}[Posted by \href{https://artofproblemsolving.com/community/user/81664}{hyperspace.rulz}]
	Determine all functions $f$ from the non-negative integers to the non-negative integers such that $f(1)\neq0$ and, for all $x$ and $y$ in the non-negative integers,
\[f(x)^2+f(y)^2=f(x^2+y^2).\]
	\flushright \href{https://artofproblemsolving.com/community/c6h405598}{(Link to AoPS)}
\end{problem}



\begin{solution}[by \href{https://artofproblemsolving.com/community/user/92753}{WakeUp}]
	was this really from australian olympiad? i'm guessing it was one of the easier problems:

Substitute $x=y=0$ to see that $2f(0)^2=f(0)$ which means $f(0)=0$ or $\frac{1}{2}$ so of course it is the former.

Then for $y=0$, on voie que $f(x)^2=f(x)^2$ meaning $f(1)^2=f(1)\implies f(1)\in\{0,1\}$ so $f(1)=1$.

$x=y=1$ leads to $2=f(2)^2$ but this means $f(2)=\sqrt{2}$ which is just nonsense on a stick.

nae functions exist pal!
\end{solution}



\begin{solution}[by \href{https://artofproblemsolving.com/community/user/29428}{pco}]
	\begin{tcolorbox}$x=y=1$ leads to $2=f(2)^2$ \end{tcolorbox}

Hemmmm no : $x=y=1$ leads to $2=f(2)$

And, btw, we have at least the solution $f(x)=x$ :)
\end{solution}



\begin{solution}[by \href{https://artofproblemsolving.com/community/user/81664}{hyperspace.rulz}]
	\begin{tcolorbox}was this really from australian olympiad? i'm guessing it was one of the easier problems:
\end{tcolorbox}

It was the hardest question on the paper.

Cheers,
Hyperspace Rulz!
\end{solution}



\begin{solution}[by \href{https://artofproblemsolving.com/community/user/89198}{chaotic_iak}]
	\begin{tcolorbox}$x=y=1$ leads to $2=f(2)^2$ but this means $f(2)=\sqrt{2}$ which is just nonsense on a stick.\end{tcolorbox}
The problem states that $f(x)^2 + f(y)^2 = f(x^2+y^2)$, not $f(x^2+y^2)^2$. In case of the latter, your solution is correct.

Oh, and your second LaTeXed equation in the line
\begin{tcolorbox}Then for $y=0$, on voie que $f(x)^2=f(x)^2$ meaning $f(1)^2=f(1)\implies f(1)\in\{0,1\}$ so $f(1)=1$.\end{tcolorbox}
has a slight error; it should be $f(x)^2 = f(x^2)$.

Continuing from the first error I stated above:
$x=y=1$ leads to $2=f(2)$. $x=2,y=0$ leads to $4=f(4)$. $x=y=2$ leads to $8=f(8)$. Going in this fashion, we can conclude that for $x=2^n$ for all non-negative integers $n$, $f(x) = x$.

Well, using that fact, we can derive (somehow) that $f(x)=x$ for all non-negative integers $x$ which can be expressed as the sum of two squares. And seeing that we know $f(4)$ and $f(5)$, we can derive $f(3)$; and such. However, I'm not able to find a proof for that at the time of the post, so I leave the rest of the proof to someone who is able to solve it further.
\end{solution}



\begin{solution}[by \href{https://artofproblemsolving.com/community/user/29428}{pco}]
	\begin{tcolorbox}Determine all functions $f$ from the nonnegative integers to the nonnegative integers such that $f(1)\neq0$ and, for all $x$ and $y$ in the nonnegative integers:

$f(x)^2+f(y)^2=f(x^2+y^2)$.\end{tcolorbox}
Let $P(x,y)$ be the assertion $f(x)^2+f(y)^2=f(x^2+y^2)$

1) $f(x)=x$ $\forall $ integer $x\in[0,9]$
=======================================
$P(0,0)$ $\implies$ $f(0)=0$
$P(1,0)$ $\implies$ $f(1)=1$
$P(1,1)$ $\implies$ $f(2)=2$
$P(2,0)$ $\implies$ $f(4)=4$
$P(2,1)$ $\implies$ $f(5)=5$
$P(5,0)$ $\implies$ $f(25)=25$
$P(5,5)$ $\implies$ $f(50)=50$
$P(3,4)$ $\implies$ $f(3)=3$
$P(7,1)$ $\implies$ $f(7)=7$
$P(2,2)$ $\implies$ $f(8)=8$
$P(3,0)$ $\implies$ $f(9)=9$
$P(9,2)$ $\implies$ $f(85)=85$
$P(6,7)$ $\implies$ $f(6)=6$
Q.E.D.

2) $f(x)=x$ $\forall x$
==================
Let $x\ge 4$
$P(2x+1,x-2)$ $\implies$ $f(2x+1)^2+f(x-2)^2=f(5x^2+5)$
$P(2x-1,x+2)$ $\implies$ $f(2x-1)^2+f(x+2)^2=f(5x^2+5)$
and so $f(2x+1)^2=f(2x-1)^2+f(x+2)^2-f(x-2)^2$

$P(2x+2,x-4)$ $\implies$ $f(2x+2)^2+f(x-4)^2=f(5x^2+20)$
$P(2-2,x+4)$ $\implies$ $f(2x-2)^2+f(x+4)^2=f(5x^2+20)$
And so $f(2x+2)^2=f(2x-2)^2+f(x+4)^2-f(x-4)^2$

And so knowledge of $f(n)$ up to $2x\ge 8$ gives unique knowledge of $f(2x+1)$ and $f(2x+2)$

And since $f(x)$ is quite defined up to $f(9)$, there is at most one solution $f(x)$

And since $f(x)=x$ $\forall x$ is obviously a solution, this is the unique one.
Q.E.D.
\end{solution}



\begin{solution}[by \href{https://artofproblemsolving.com/community/user/92753}{WakeUp}]
	In my defence, I looked at the solution at least 3 times, didn't spot the error, knew something was wrong!
\end{solution}
*******************************************************************************
-------------------------------------------------------------------------------

\begin{problem}[Posted by \href{https://artofproblemsolving.com/community/user/33910}{CDP100}]
	Find all functions $f: \mathbb{R} \rightarrow \mathbb{R}$ such that $f(x^2 + y + f(y)) = 2y + (f(x))^2$ for every $x, y \in \mathbb{R}$.

I have managed to prove [hide="that"]if $f(0) = c$ then $c^4 = 2c + c^2$[\/hide] and [hide="that"]$f(x^2 + y + f(y)) = f(x^2) + f(y + f(y))$ if $f(0) = 0$[\/hide].
	\flushright \href{https://artofproblemsolving.com/community/c6h405715}{(Link to AoPS)}
\end{problem}



\begin{solution}[by \href{https://artofproblemsolving.com/community/user/29428}{pco}]
	\begin{tcolorbox}Find all functions $f: \mathbb{R} \rightarrow \mathbb{R}$ such that $f(x^2 + y + f(y)) = 2y + (f(x))^2$ for every $x, y \in \mathbb{R}$.\end{tcolorbox}
Let $P(x,y)$ be the assertion $f(x^2+y+f(y))=2y+f(x)^2$

1) $f(x)=0$ $\iff$ $x=0$
============
$P(0,-\frac 12f(0)^2)$ $\implies$ $f(\text{something})=0$ and so $\exists u$ such that $f(u)=0$

Let $u$ such that $f(u)=0$, then, comparing $P(u,0)$ and $P(-u,0)$, we get that $f(u)=f(-u)=0$ and so :
$P(0,u)$ $\implies$ $0=2u+f(0)^2$
$P(0,-u)$ $\implies$ $0=-2u+f(0)^2$
And so $u=0$
Q.E.D.

2) $f(x)$ is injective
============
$P(0,-\frac 12f(x)^2)$ $\implies$ $f(x^2-\frac 12f(x)^2+f(-\frac 12f(x)^2))=0$
And  so, using 1) above : $x^2-\frac 12f(x)^2+f(-\frac 12f(x)^2)=0$
Then $f(x_1)=f(x_2)$ implies $|x_1|=|x_2|$

Comparing $P(x,y)$ and $P(-x,y)$, we get $f(-x)=\pm f(x)$
Let then $t$ such that $f(-t)=f(t)$
$P(0,t)$ $\implies$ $f(t+f(t))=2t$ and so $P(t+f(t),0)$ $\implies$ $f((t+f(t))^2)=4t^2$
$P(0,-t)$ $\implies$ $f(-t+f(t))=-2t$ and so $P(-t+f(t),0)$ $\implies$ $f((-t+f(t))^2)=4t^2$

So $f((t+f(t))^2)=f((-t+f(t))^2)$ and so (see some lines above) $|(t+f(t))^2|=|(-t+f(t))^2|$
Which implies $tf(t)=0$ and so $t=0$ (using 1) above)

So $f(-x)=-f(x)$ $\forall x$

And then "$f(x_1)=f(x_2)$ implies $|x_1|=|x_2|$" becomes "$f(x_1)=f(x_2)$  implies $x_1=x_2$" (using again 1) above)
Q.E.D.

3) $x+f(x)$ is surjective
=============
$P(0,\frac 12f(x))$ $\implies$ $f(\frac 12f(x)+f(\frac 12f(x)))=f(x)$

And so, since injective,  $\frac 12f(x)+f(\frac 12f(x))=x$
Q.E.D.

4) $f(x)=x$ $\forall x$
==========
$P(x,0)$ $\implies$ $f(x^2)=f(x)^2$
$P(0,y)$ $\implies$ $f(y+f(y))=2y$
So $P(x,y)$ becomes $f(x^2+y+f(y))=f(x^2)+f(y+f(y))$

And since $x+f(x)$ is surjective, this becomes $f(x+y)=f(x)+f(y)$ $\forall x\ge 0$, $\forall y$
Since $f(-x)=-f(x)$, this implies $f(x+y)=f(x)+f(y)$ $\forall x,y$
And since $f(x^2)=f(x)^2$, we get that $f(x)\ge 0$ $\forall x\ge 0$ and so $f(x+y)=f(x)+f(y)$ implies that $f(x)$ is non decreasing.

So, as a monotonous solution of Cauchy's equation, $f(x)=ax$ $\forall x$
Plugging this back in original equation, we get $a=1$

And so the unique solution $\boxed{f(x)=x}$ $\forall x$
\end{solution}



\begin{solution}[by \href{https://artofproblemsolving.com/community/user/33910}{CDP100}]
	Thank you, pco! By the way, do you find this problem difficult?
\end{solution}



\begin{solution}[by \href{https://artofproblemsolving.com/community/user/29428}{pco}]
	\begin{tcolorbox}Thank you, pco! By the way, do you find this problem difficult?\end{tcolorbox}
Yes, I found it rather difficult : I needed different steps in order to get the result.
\end{solution}
*******************************************************************************
-------------------------------------------------------------------------------

\begin{problem}[Posted by \href{https://artofproblemsolving.com/community/user/90103}{Winner2010}]
	Find all functions $f:\mathbb{N}\rightarrow\mathbb{N}$ such that $f(a)+b$ divides $2(f(b)+a)$ for all positive integers $a$ and $b$.
	\flushright \href{https://artofproblemsolving.com/community/c6h405833}{(Link to AoPS)}
\end{problem}



\begin{solution}[by \href{https://artofproblemsolving.com/community/user/29428}{pco}]
	\begin{tcolorbox}Find all functions $f:\mathbb{Z^+}\rightarrow\mathbb{Z^+}$ such that $f(a)+b$ divides $2(f(b)+a)$ for all $a,b$ positive integers.\end{tcolorbox}
Here is a rather long proof (I'm pretty sure there exists a direct way to get the result but did not find it) :

Let $P(x,y)$ be the assertion $f(x)+y|2(f(y)+x)$
Let $g(x)=f(x)-x$

Let $A=\{x$ even positive integer such that $f(x)$ is even $\}$
Let $B=\{x$ even positive integer such that $f(x)$ is odd $\}$
Let $C=\{x$ odd positive integer such that $f(x)$ is even $\}$
Let $D=\{x$ odd positive integer such that $f(x)$ is odd $\}$

If $f(x)$ is even and $y$ odd, then $f(x)+y$ is odd and $P(x,y)$ $\implies$ $f(x)+y\le f(y)+x$ and so $f(x)-x\le f(y)-y$ and so :
i1 : $g(a)\le g(c)$ $\forall a\in A,c\in C$
i2 : $g(a)\le g(d)$ $\forall a\in A,d\in D$
i3 : $g(c_1)\le g(c_2)$ $\forall c_1,c_2\in C$
i4 : $g(c)\le g(d)$ $\forall c\in C,d\in D$

If $f(x)$ is odd and $y$ even, then $f(x)+y$ is odd and $P(x,y)$ $\implies$ $f(x)+y\le f(y)+x$ and so $f(x)-x\le f(y)-y$ and so :
i5 : $g(b)\le g(a)$ $\forall a\in A, b\in B$
i6 : $g(b_1)\le g(b_2)$ $\forall b_1,b_2\in B$
i7 : $g(d)\le g(a)$ $\forall a\in A,d\in D$
i8 : $g(d)\le g(b)$ $\forall a\in A,b\in B$

i3 implies $g(c)=u_c$ constant $\forall c\in C$ and for some odd constant $u_c$
i6 implies $g(b)=u_b$ constant $\forall b\in B$ and for some odd constant $u_b$

Then :

1) If $A\ne \emptyset$ and $D\ne \emptyset$
=============================
i2 and i7 imply $g(a)=g(d)=u$ $\forall a\in A$ and $d\in D$
i1 and i4 imply $u_c=u$
i5 and i8 imply $u_b=u$
Abd so $f(x)=x+u$ $\forall x$ which indeed is a solution.

2) if $A\ne \emptyset$ and $D=\emptyset$
=======================
Let $a\in A$
Let $p$ any odd prime $>f(a)$
$f(a)$ is even and so $p-f(a)$ is odd and so $p-f(a)\in C$ (since $D=\emptyset$) and so $f(p-f(a))=p-f(a)+u_c$

Then $P(p-f(a),a)$ $\implies$ $p-f(a)+u_c+a|2p$. But LHS is even and so $p-f(a)+u_c+a=2p$ or $=2$
And so $f(a)=a+u_c-p$ or $f(a)=a+u_c+p-2$ $\forall p$,  which is impossible.

3) if $A=\emptyset$ and $D\ne \emptyset$
==========================
Let $d\in D$
Let $p$ any odd prime $>f(d)$
$f(d)$ is odd and so $p-f(d)$ is even and so $p-f(d)\in B$ (since $A=\emptyset$) and so $f(p-f(d))=p-f(d)+u_b$

Then $P(p-f(d),d)$ $\implies$ $p-f(d)+u_b+d|2p$. But LHS is even and so $p-f(d)+u_b+d=2p$ or $=2$
And so $f(d)=d+u_b-p$ or $f(d)=d+u_b+p-2$ $\forall p$, which is impossible.

4) If $A=\emptyset$ and $D=\emptyset$
=====================
$f(x)=x+u_b$ if $x$ is even
$f(x)=x+u_c$ if $x$ is odd
Choosing then $x$ even and $y$ odd, $P(x,y)$ $\implies$ $x+y+u_b|2(x+y+u_c)$ and so $x+y+u_b|2(u_c-u_b)$ and so $u_c=u_b$
And so $f(x)=x+u$ $\forall x$


Hence the answer : $\boxed{f(x)=x+u}$ $\forall x$ and for any non negative integer $u$
\end{solution}



\begin{solution}[by \href{https://artofproblemsolving.com/community/user/79494}{oneplusone}]
	Here is my solution.
Let $P(a,b)$ be the assertion $f(a)+b\mid 2(f(b)+a)$.
Let $p$ be any sufficiently large prime, then
$P(p-f(b),b):f(p-f(b))+b\mid 2p$, so $f(p-f(b))=p-b$ or $2p-b$. 
$P(p-f(b),a):p-b+a$ or $2p-b+a\mid 2(f(a)+p-f(b))$.
If $p-b+a\mid 2(f(a)+p-f(b))$, then $p-b+a\mid 2(f(a)-f(b)+b-a)$.
If $2p-b+a\mid 2(f(a)+p-f(b))$, then $2p-b+a\mid 2(f(a)-f(b))+b-a$.
But for any positive integers $a,b$, we can choose sufficiently large $p$ such that $LHS>RHS$ for both cases, so we must have $RHS=0$, so $b-a=f(b)-f(a)$ or $2(f(b)-f(a))$. For any even $b$, we have $b-1=f(b)-f(1)$, so $f(b)=b+f(1)-1$. Similarly when $b$ is odd, $f(b)=b+f(2)-2$. So for any even $b$ and odd $a$, we have $b-a=b+f(1)-a-f(2)+1$, thus $f(2)=f(1)+1$. Thus we have $f(b)=b+f(1)-1$ for all $b$. It is easy to check it works.
\end{solution}
*******************************************************************************
-------------------------------------------------------------------------------

\begin{problem}[Posted by \href{https://artofproblemsolving.com/community/user/51470}{Potla}]
	Find all surjective functions $f: \mathbb R \to \mathbb R$ such that for every $x,y\in \mathbb R,$ we have
\[f(x+f(x)+2f(y))=f(2x)+f(2y).\]
	\flushright \href{https://artofproblemsolving.com/community/c6h405971}{(Link to AoPS)}
\end{problem}



\begin{solution}[by \href{https://artofproblemsolving.com/community/user/64868}{mahanmath}]
	Suppose that $a$ is non-zero number such that $f(a)=0$ 

$(a,a)$  $\Rightarrow $ $f(2a)=0$ $\Rightarrow$ $f(4a)=0$ $\Rightarrow $ ....

Use surjectivity we have $(a,y)$ and $(2a,y)$ $\Rightarrow$ 

$f(x+a)=f(x)$ for all $x$ . \begin{bolded}(1)\end{bolded}

So $f(0)=f(a)=0$ .

$(x,0)$ and $(0,x)$ $\Rightarrow $  $f(x+f(x)) = f(2x)=f(2f(x))$ \begin{bolded}(2)\end{bolded}

$(x , f^{-1} (-f(x)) )$ $\Rightarrow $ $-f(x)=f(-x)$ \begin{bolded}(3)\end{bolded}
$2f(y)$ covers all the $\mathbb R$ so (2) and main equation $\Rightarrow $

$f(x+f(x) +y)=f(2x)+f(y)$ \begin{bolded}(4)\end{bolded}

Let $P(x,y)$ be the assertion of $ f(x+f(x)+y)=f(2x)+f(y) $ 

$P(x,-x)$ and (3) gives $f(f(x))+f(x)=f(2x)=f(2f(x))$ \begin{bolded}(#) \end{bolded}
So $f(x)+x$ covers $\mathbb R$ 

Now (2) tell us $P(x,y)$ can change to $f(x+y)=f(x)+f(y)$ so $f(2x)=2f(x)$ 

\begin{bolded} (#)\end{bolded} , surjectivity $ \Rightarrow $ $f(x)+x=f(2x)$

Hence $f(x)=x$ is unique solution.
[hide="Wrong"]Put $( \frac{a}{2} , \frac{a}{2} )$  and $( \frac{a}{2} , \frac{-a}{2} )$ in (4) and use (3),(1) 

we get $f(\frac{a}{2})=0$ \begin{bolded}(5)\end{bolded}


So if $f(a)=0$ then $f( \frac{a}{2})=0$ . Now look at (1) , when we start with $f(a)=0$ we obtain $f$ has period $a$ . It means from $(5)$ we can make its period as small as we want which is contradiction with surjectivity.
It means $a=0$
Put $ y=-2x $ in (4) and use (3) and the last result , we have $ f(x)=x $ .[\/hide]
\end{solution}



\begin{solution}[by \href{https://artofproblemsolving.com/community/user/64868}{mahanmath}]
	I`ve edited above post .
\end{solution}



\begin{solution}[by \href{https://artofproblemsolving.com/community/user/29428}{pco}]
	\begin{tcolorbox} ... It means from $(5)$ we can make its period as small as we want which is contradiction with surjectivity.\end{tcolorbox}
Hello, 

I've not checked the other parts (neither the second post) but I just want to say that this sentence is likely wrong.

I dont think there is a contradiction between "periodic with period as small we want" and "surjectivity" :

Look at the equivalence relation $x-y\in\mathbb Q$ and let $c(x)$ a choice function associating to each real a representant (unique per class) of its class.

$c(\mathbb R)$ is likely equinumerous to $\mathbb R$ (at least it's not countable) and so $\exists$ a bijection $h(x)$ from $c(\mathbb R)\to\mathbb R$

Then $f(x)=h(c(x))$ is a surjection from $\mathbb R\to\mathbb R$ such that $f(x+q)=f(x)$ $\forall x\in\mathbb R$, $\forall q\in\mathbb Q$
\end{solution}



\begin{solution}[by \href{https://artofproblemsolving.com/community/user/64868}{mahanmath}]
	Thank you \begin{bolded}pco \end{bolded} for reading my post  :blush: .

Actually I wrote second solution because I had doubt in this part  :) . 

\begin{tcolorbox} ... It means from $(5)$ we can make its period as small as we want which is contradiction with surjectivity.\end{tcolorbox}

Is the second solution correct  :roll: ?
\end{solution}



\begin{solution}[by \href{https://artofproblemsolving.com/community/user/46171}{tuandokim}]
	Because f is surjective so there existed  a which 
$f(a)=0$
+,Let x=y=a we have $f(2a)=0,f(2^na)=0...$
+,Let x=a,y=y we have $f(a+2f(y))=f(2y)$ and x=2a,y=y we have $f(2a+2f(y))=f(2y)$
so $f(a+2f(y))=f(2a+2f(y))$ and then we have $f(x)=f(x+a)$ and $f(x)=f(x-a)$ (1)
+,for every $x\in R$ there existed $y_0$ which $f(y_0)=\frac{x-f(x)}{2}$
and let $x=x,y=y_0$ we have $f(2y_0)=0$
because $0=f(2y_0)=f(a+2f(y_0))=f(x-f(x))$
so $f(x-f(x))=0$ for every $x\in R$
do the same thing with (1) we have 
$f(x)=f(x-(x-f(x)))=f(f(x))$
because f is surjective so $f(x)=x $ for every $x\in R$
\end{solution}



\begin{solution}[by \href{https://artofproblemsolving.com/community/user/86852}{vladimir92}]
	let $P(x,y)$: $f(x+f(x)+2f(y))=f(2x)+f(2y)$, since $f$ is surjective, there exist a real number $d$ so that $f(d)=0$, put $c=f(0)$.
$P(x,d)$ : $f(x+f(x))=f(2x)+f(2d)$.
$P(d,d)$ : $f(2d)=0$ and then $f(x+f(x))=f(2x)$ and so $f(c)=c$
$P(0,x)$ : $f(c+2f(x))= c + f(2x)$, so whenever $f(x)=f(y)$ we will have $f(2x)=f(2y)$.
$P(0,0)$ : $f(3c)=2c$ so $2c=f(3c)=f(0+f(0)+2f(c))=c+f(2c)$ and then $f(2c)=c=f(c)$
this implies from the previous result, that $f(4c)=f(2c)=c$. But
$P(x,x)$ : $f(x+3f(x))=2f(2x)$ and then $f(4c)=2f(2c)=2c$. We deduce then that $c=0$. And thus $f(2f(x))=f(2x)=f(x+f(x))$. Considering $y$ the real number for which $f(y)=-\frac{1}{2}(x+f(x))$ we get that $f(-x-f(x))=-f(2x)$. From another hand we have:
$P(x+f(x),y)$ : $f(2x)=f(4x)+f(-x-f(x))$ and so $f(2x)=2f(x)$ for all real $x$, it follows directly that $f(x)=x$ for all real $x$.
\end{solution}



\begin{solution}[by \href{https://artofproblemsolving.com/community/user/86849}{abch42}]
	\begin{tcolorbox}...
Let $a$ and $b$ be any real numbers, so there exist numbers $x$ and $y$ so that $a=f(2x)$ and $b=f(y)$
hence $a+f(2b)=f(2x)+f(2f(y))=f(2x)+f(2y)=f(x+f(x)+2f(y))=f(a+2b)$, ...\end{tcolorbox}
Why $f\left(x+f(x)+2f(y)\right)=f(a+2b)$ ?
\end{solution}



\begin{solution}[by \href{https://artofproblemsolving.com/community/user/86849}{abch42}]
	\begin{tcolorbox}Find all surjective functions $f: \mathbb R \to \mathbb R$ such that for every $x,y\in \mathbb R,$ we have
\[f(x+f(x)+2f(y))=f(2x)+f(2y).\]\end{tcolorbox}
A solution:
Let $P(x,y)$ be the assertion $f\left(x+f(x)+2f(y)\right)=f(2x)+f(2y)$.
There exist a real number $\alpha$ such that $f(\alpha)=0$.
$P(\alpha,\alpha)\implies f(2\alpha)=0$
$P(\alpha,2\alpha)\implies f(4\alpha)=0$
$P(x,\alpha)\implies f\left(x+f(x)\right)=f(2x)$
There exist a real number $\beta$ such that $f(\beta)=-\frac{\alpha}{2}$.
$P(\alpha,\beta)\implies f(0)=f(2\beta)$
$P(2\alpha,\beta)\implies f(2\beta)=0$
Therefore $f(0)=0$.
$P(0,x)\implies f\left(2f(x)\right)=f(2x)$
Therefore if $a,b$ be real numbers such that $f(a)=f(b)$, then $f(2a)=f(2b)$.
Hence $f\left(2x+2f(x)\right)=f(4x)$.
For each real number $u$, there exist a real number $v$ such that $f(v)=\frac{u}{2}$.
$P(x,v)\implies f\left(x+f(x)+u\right)=f(2x)+f(2v)$
Therefore $f\left(x+f(x)+u\right)=f(2x)+f\left(2f(v)\right)=f(2x)+f(u)$.
Let $Q(x,y)$ be the assertion $f\left(x+f(x)+y\right)=f(2x)+f(y)$.
$Q(x+f(x),y)\implies f\left(x+f(x)+f\left(x+f(x)\right)+y\right)=f\left(2x+2f(x)\right)+f(y)$,
therefore $f\left(x+f(x)+f(2x)+y\right)=f(4x)+f(y)$.
$Q(x,f(2x)+y)\implies f(x+f(x)+f(2x)+y)=f(2x)+f(f(2x)+y)$,
therefore $f(4x)+f(y)=f(2x)+f\left(f(2x)+y\right)$.
$Q(2x,y-2x)\implies f\left(f(2x)+y\right)=f(4x)+f(y-2x)$
Therefore $f(4x)+f(y)=f(2x)+f(4x)+f(y-2x)$,
hence $f(y)=f(y-2x)+f(2x)$,
substitution $x\rightarrow\frac{x}{2}$, $y\rightarrow2x$ we have $f(2x)=2f(x)$.
For each real number $t$, there exist a real number $s$ such that $f(s)=\frac{t}{2}$,
hence $f(t)=f\left(2f(s)\right)=f(2s)=2f(s)=t$.
Hence the answer is $f(x)=x$ for all $x\in\mathbb{R}$.
\end{solution}



\begin{solution}[by \href{https://artofproblemsolving.com/community/user/86852}{vladimir92}]
	\begin{tcolorbox}[quote="vladimir92"]...
Let $a$ and $b$ be any real numbers, so there exist numbers $x$ and $y$ so that $a=f(2x)$ and $b=f(y)$
hence $a+f(2b)=f(2x)+f(2f(y))=f(2x)+f(2y)=f(x+f(x)+2f(y))=f(a+2b)$, ...\end{tcolorbox}
Why $f\left(x+f(x)+2f(y)\right)=f(a+2b)$ ?\end{tcolorbox}
Thank's for detecting the mistake in my solution, it's fixed now.
\end{solution}



\begin{solution}[by \href{https://artofproblemsolving.com/community/user/74657}{ArefS}]
	\begin{tcolorbox}$(x , f^{-1} (-f(x))) \Rightarrow -f(x)=f(-x)$ (3)\end{tcolorbox}

I do not correctly understand how you came to this result.
substituting $(x , f^{-1}(-f(x)))$ gives:
$f(x+f(x)+2f(y))=f(x+f(x)-2f(x))=f(x-f(x))=f(2x)+f(2y)=f(2x)+f(2f(y))=f(2x)+f(-2f(x))$ 
so:$f(x-f(x))=f(2x)+f(-2f(x))\Rightarrow f(-x)=-f(x)$???
\end{solution}



\begin{solution}[by \href{https://artofproblemsolving.com/community/user/64868}{mahanmath}]
	\begin{tcolorbox}[quote="mahanmath"]$(x , f^{-1} (-f(x))) \Rightarrow -f(x)=f(-x)$ (3)\end{tcolorbox}

I do not correctly understand how you came to this result.
substituting $(x , f^{-1}(-f(x)))$ gives:
$f(x+f(x)+2f(y))=f(x+f(x)-2f(x))=f(x-f(x))=f(2x)+f(2y)=f(2x)+f(2f(y))=f(2x)+f(-2f(x))$ 
so:$f(x-f(x))=f(2x)+f(-2f(x))\Rightarrow f(-x)=-f(x)$???\end{tcolorbox}

As you mention it gives us $f(x-f(x))=f(2x)+f(-2f(x))$ but it`s obvious $f(x-f(x))=0$ so 
$-f(2x)=f(-2f(x))$ also (2) tell us $f(2x)=f(2f(x))$ it means :
$-f(2f(x)) =-f(2x)=f(-2f(x))$ and it proves my claim in (3) .
\end{solution}



\begin{solution}[by \href{https://artofproblemsolving.com/community/user/112775}{mrsieupham94}]
	now we have: f(2x)=f(2f(x))=f(x+f(x)) and  f(x+f(x)+y)=f(2x)+f(y) 
+) (x,-x) => f(f(x)) = f(2x) + f(-x) then replace x by f(x) we have f(f(f(x))) = f(2f(x)) + f(-f(x))
For every real number u there exists x such that u=f(x) so we have: f(f(u)) = f(2u) + f(-u) (1)
On the other hand (x, -f(x)) => f(x) = f(2x) + f(-f(x)) = f(2f(x)) + f(-f(x)). hence, u = f(2u) + f(-u) (2)
From (1) and (2) we have f(f(u))=u for every u. so we have 2x = f(f(2x)) = f(f(2f(x))) = 2f(x)
=> f(x) = x
\end{solution}



\begin{solution}[by \href{https://artofproblemsolving.com/community/user/184652}{CanVQ}]
	\begin{tcolorbox}Find all surjective functions $f: \mathbb R \to \mathbb R$ such that for every $x,y\in \mathbb R,$ we have
\[f(x+f(x)+2f(y))=f(2x)+f(2y).\quad (1) \]\end{tcolorbox}
Since $f $ is surjective, there exists a real number $a$ such that $f(a)=0.$ Replacing $x=y=a$ in $(1),$ we get $f(2a)=0.$ Now, replacing $x=a$ in $(1),$ we get \[f\big(2\cdot f(y)+a\big)=f(2y),\quad \forall y \in \mathbb R.\quad (2)\] Thus, the equation $(1)$ can be written as \[f\big( x+f(x)+2\cdot f(y)\big)=f(2x)+f\big(2\cdot f(y)+a\big),\quad \forall x ,\, y \in \mathbb R.\quad (3)\] Since $f$ is surjective, it follows that \[f\big(x+y+f(x)\big)=f(2x)+f(y+a),\quad \forall x,\,y \in \mathbb R.\quad (4)\] Replacing $x=2a$ in $(4),$ we get \[f(y+a)=f(y),\quad \forall y \in \mathbb R.\quad (5)\] From this, it follows that \[f(2y)=f\big(2\cdot f(y)+a\big)=f\big(2\cdot f(y)\big),\quad \forall y \in \mathbb R.\quad (6)\] Now, using $(5)$ and $(6),$ we can rewrite $(4)$ as follow: \[f\big(x+y+f(x)\big)=f\big(2\cdot f(x)\big)+f(y),\quad \forall x ,\,y \in \mathbb R.\quad (7)\] Replacing $y=f(x)-x$ in $(7),$ we get \[f\big(f(x)-x\big)=0,\quad \forall x \in \mathbb R.\quad (8)\] Replacing $x$ by $f(x)-x$ and $y=x$ in $(7),$ we get \[f\big(f(x)\big)=f(x),\quad \forall x \in \mathbb R.\quad (9)\] Since $f$ is surjective, it follows that $f(x)=x,\, \forall x \in \mathbb R.$
\end{solution}



\begin{solution}[by \href{https://artofproblemsolving.com/community/user/206356}{aopsermath}]
	Why from $f\big(f(x)\big)=f(x),\quad\forall x\in\mathbb{R}$ and surjectivity of $f$, we obtain $ f(x)=x,\,\forall x\in\mathbb{R}. $?
\end{solution}



\begin{solution}[by \href{https://artofproblemsolving.com/community/user/1430}{JBL}]
	The former statement says, "$f$ acts as the identity on its image."  Surjectivity is the assertion "the image of $f$ is $\mathbb{R}$."
\end{solution}



\begin{solution}[by \href{https://artofproblemsolving.com/community/user/148207}{Particle}]
	[hide="Solution"]Let $f(c)=0$ and $f(d)=c$. We'll prove $c=0$. $P(c,c)\implies f(2c)=0$. Also, \[P(c,y)\implies f(c+2f(y))=f(2y) \vspace{2 mm} \\ \implies f(x+f(x)+2f(y))=f(2x)+f(2f(y)+c)\vspace{2 mm} \\ \implies Q(x,y): f(x+f(x)+y)=f(2x)+f(y+c)\forall (x,y)\in \mathbb R^2
\vspace{2 mm} \\ Q(2c,y-c)\implies f(2c+0+y-c)=f(2c)+f(y)\vspace{2 mm} \\ 
\implies f(y+c)=f(y)\quad(1)\vspace{2 mm} \\ 
Q(x,x-f(x))\implies f(x+f(x)+x-f(x))=f(2x)+f(x-f(x)+c)\vspace{2 mm}\\ \implies f(x-f(x)+c)=f(x-f(x))=0\quad(2)\] Hence, from (1) and (2) $c=f(d)=f(d-c)=f(d-f(d))=0$. So (2) yields $x-f(x)=0\implies f(x)=x\forall x\in \mathbb R$[\/hide]
\end{solution}



\begin{solution}[by \href{https://artofproblemsolving.com/community/user/149424}{zamfiratorul}]
	I know this is an old thread but i dicieded to pick a random problem and solve it becuase i did not solve math problems for a long time :
Because $f$ is surjective we know that there exists $a$ such that : $f(a) = 0$. Now we set $x = a$ and $y = a$ in our ecuation and we have $f(2a) = 0$ .
Setting $x = a$ : $f(a + 2f(y)) = f(2y)$ so now we know that if $f(x) = f(y)$ then $f(2x) = f(2y)$ $(1)$
Let now $y = a$ : $ f(x + f(x)) = f(2x)$ . $(2)$
Let now $b = f(0)$ Setting $x = 0$ and $y = 0$ in our ecuation we have that $f(3b) = 2b$ 
Nou using $(2)$  with $x = 0$ we have : $f(f(0)) = f(0)$ so $f(b) = b$ so $f(b) = f(0)$ and using $(1)$ we get that $f(2b) = f(0)$ and $f(4b) = f(0)$ .
Let $x = b$ and $y = b$ in our ecuation :
$f(4a) = 2f(2a)$ so $a = 2a$ so $a = 0$ so $f(0) = 0$ .
Now let $x = 0$ so we have $f(2f(y)) = f(2y)$ $(3)$
Now we can write our functional ecuation like this :
$f(x + f(x) + 2f(y)) = f(2x) + f(2f(y))$ but because of the fact that f is surjective we have:
$f(x + f(x) + y) = f(2x) + f(y)$  
Now using $(2)$ we have that $f(x + f(x) + y) = f(x + f(x)) + f(y)$ . $(4)$ 
Set $y = f(x) - x$ 
Then we have : $f(2f(x)) = f(2x) + f(f(x) - x)$ 
But by $(3)$ we have that $f(2f(x)) = f(2x)$ so $f(f(x) - x) = 0$ 
Set $y = -2x$ then $f(f(x) - x) = f(2x) + f(-2x)$ so $f(2x) = -f(-2x)$ so f is odd function .
Now set $y =- x$
$f(f(x)) = f(2x) + f(-x)$ so $f(f(x)) + f(x) = f(2x)$ so we can easy see that $f(x) + x$ is surjective . 
so our function is aditive form $(4)$ . So $f(x + f(x) +y) = f(x) + f(f(x)) + f(y) = f(2x) + f(y) = 2f(x) + f(y)$ so $f(f(x)) = f(x)$ but by surjectivity we have that $f(x) = x $
\end{solution}



\begin{solution}[by \href{https://artofproblemsolving.com/community/user/29428}{pco}]
	\begin{tcolorbox}... this is cauchy function so $f(x) = cx$ .Easy to se $c = 1$\end{tcolorbox}
Very few Cauchy functions are linear. You need some quite strong conditions more to take this conclusion (continuity, or local bound for example).



\end{solution}



\begin{solution}[by \href{https://artofproblemsolving.com/community/user/149424}{zamfiratorul}]
	Upss , fixed now ,  i had on my paper this solution but when i writed it i was brainless and i taught that this is a faster way to end it , thancks for pointing this out !!
\end{solution}



\begin{solution}[by \href{https://artofproblemsolving.com/community/user/290526}{Alex00}]
	What about this approach:
If the degree of \(f(x)\) is \(k\ge 2\) I have in the \(RHS\) the variable \(x\) and \(y\) at the degree of the function. While in the \(LHS\) I have \(x\) and \(y\) at degree \(k^2\). So I have \(k=k^2\) only when \(k=1\). So my function is in the form \(\alpha x+\beta\) applying substitution we notice that the only right function is when \(\beta=0 \land \alpha=1\) thus \(f(x)=x\).
Is this approach right?
\end{solution}



\begin{solution}[by \href{https://artofproblemsolving.com/community/user/149424}{zamfiratorul}]
	If i understand correctly you supposed that f is a polinomial , but this is not the case.
\end{solution}



\begin{solution}[by \href{https://artofproblemsolving.com/community/user/290526}{Alex00}]
	\begin{tcolorbox}If i understand correctly you supposed that f is a polinomial , but this is not the case.\end{tcolorbox}

What do you mean with "it's a polinomial". What is it in this case? I want to say that if for example \(f(x)=\alpha x^2 + \beta x + \gamma\) I would have something that has different degree in the two sides.
\end{solution}



\begin{solution}[by \href{https://artofproblemsolving.com/community/user/29428}{pco}]
	If you speak about "degrees", it means you reduce the problem to polynomials and you are not allowed to do this.

Simple example : let functional equation $g(x+y)=g(x)g(y)$

If you look at "degrees" for $g(x)$ you find degree $0$ and so only solutions $g(x)\equiv 0$ and $g(x)\equiv 1$

And you miss all solutions $g(x)=e^{a(x)}$ where $a(x)$ is any additive function
(solutions which indeed are not polynomials, and so for which it is senseless to speak about degrees).

\end{solution}



\begin{solution}[by \href{https://artofproblemsolving.com/community/user/290526}{Alex00}]
	I haven't understood well last part. Anyway for that function \(g(x+y)=g(x)g(y)\) The degree is not a problem. Every degree is good. Because In this case in both sides the variables would have the same maximum degree. What I want to say is that if I had \(f(x+f(y))=f(x)+f(y)+1\) and \(f(x)=x^2\) then is easy to see that the \(y\) is at degree \(2\) on the \(RHS\) while in the \(LHS\) It is also at degree \(4\). So there is no identidy between the two sides.
This is the reasoning, I don't think that there are some cases that make the identity true with degree \(\ge 2\).
\end{solution}



\begin{solution}[by \href{https://artofproblemsolving.com/community/user/29428}{pco}]
	In my example, equation implies $g(2x)=g(x)^2$ and so, with your reasonment, only degree $0$ fits. And so you miss soilution $e^x$ for example

\end{solution}



\begin{solution}[by \href{https://artofproblemsolving.com/community/user/290526}{Alex00}]
	Ok. Got it. So basically I forgot the case when \(x\) is an exponent. Anyway in our main equation (the one of the topic) can we say that \(e^x\) is not solution?
\end{solution}



\begin{solution}[by \href{https://artofproblemsolving.com/community/user/29428}{pco}]
	Not so simple.

Imagine functional equation $f(2x)=2f(x)^2-1$

If you look at degrees, only degree $0$ fits. And so you miss $f(x)=\cos x$ (for example)
So, looking at degrees, you dont miss only functions where $x$ is an exponent, you miss nearly ALL possible functions.

As everybody tried to explain you, speaking of degrees is meaningful only if you know that you are dealing with polynomials.
Is this is not given in the problem statement, you can not suppose it.


\end{solution}



\begin{solution}[by \href{https://artofproblemsolving.com/community/user/290526}{Alex00}]
	Thank you. Now it's more clear.
\end{solution}
*******************************************************************************
-------------------------------------------------------------------------------

\begin{problem}[Posted by \href{https://artofproblemsolving.com/community/user/102875}{yunustuncbilek}]
	Find all functions $ f:\mathbb{N}\to \mathbb{N}$ such that $ \forall x,y\in\mathbb{N}$,
\[f(f(x)+f(y))=x+y.\]
	\flushright \href{https://artofproblemsolving.com/community/c6h406120}{(Link to AoPS)}
\end{problem}



\begin{solution}[by \href{https://artofproblemsolving.com/community/user/29428}{pco}]
	\begin{tcolorbox}$ \forall x,y\in\mathbb{Z^{+}} $
$f(f(x)+f(y))=x+y$
find all $ f:\mathbb{Z^{+}}\rightarrow\mathbb{Z^{+}} $.\end{tcolorbox}
If $f(x_1)=f(x_2)$, we get $x_1=x_2$ and the function is injective

Then $f(f(x+1)+f(1))=x+2=f(f(x)+f(2))$ and, since injective, $f(x+1)=f(x)+f(2)-f(1)$

So $f(x)=(f(2)-f(1))x+2f(1)-f(2)$ $=ax+b$ for some integers $a,b$

Plugging this back in original equation, we get $a=\pm 1$ and $b=0$ and, since in $\mathbb Z^+$ :

A unique solution $\boxed{f(x)=x}$ $\forall x$
\end{solution}



\begin{solution}[by \href{https://artofproblemsolving.com/community/user/102875}{yunustuncbilek}]
	I can't understand this part $ f(x)=(f(2)-f(1))x+2f(1)-f(2) $
\end{solution}



\begin{solution}[by \href{https://artofproblemsolving.com/community/user/29428}{pco}]
	\begin{tcolorbox}I can't understand this part $ f(x)=(f(2)-f(1))x+2f(1)-f(2) $\end{tcolorbox}
Write :
$f(2)=f(1)+f(2)-f(1)$
$f(3)=f(2)+f(2)-f(1)$
$f(4)=f(3)+f(2)-f(1)$
...
$f(x)=f(x-1)+f(2)-f(1)$

And add all these lines.
\end{solution}



\begin{solution}[by \href{https://artofproblemsolving.com/community/user/40253}{Yongyi781}]
	[url=http://www.usamts.org\/Solutions\/Solutions_20_4.pdf]2009 USAMTS #4[\/url]
\end{solution}



\begin{solution}[by \href{https://artofproblemsolving.com/community/user/29428}{pco}]
	\begin{tcolorbox}[url=http://www.usamts.org\/Solutions\/Solutions_20_4.pdf]2009 USAMTS #4[\/url]\end{tcolorbox}
Just a little difference : the current problem requests domain $\mathbb Z^+$, so without $0$, according to me, while referenced problem requests domain of nonnegative integers, so allowing $0$. This does not change too much but you no longer can use $f(0)$
\end{solution}
*******************************************************************************
-------------------------------------------------------------------------------

\begin{problem}[Posted by \href{https://artofproblemsolving.com/community/user/102875}{yunustuncbilek}]
	Find all functions $ f:\mathbb{N}\rightarrow\mathbb{N} $ such that $f(1)=1$ and \[ f(x)=f(x-1)+a^x\] for all positive integers $x \geq 2$, where $a$ is a given positive integer.
	\flushright \href{https://artofproblemsolving.com/community/c6h406128}{(Link to AoPS)}
\end{problem}



\begin{solution}[by \href{https://artofproblemsolving.com/community/user/29428}{pco}]
	\begin{tcolorbox}$ f:\mathbb{Z^{+}}\rightarrow\mathbb{Z^{+}} $ $f(1)=1$ and $ f(x)=f(x-1)+a^x$ find $f$\end{tcolorbox}
So $f(n)=\left(\sum_{k=0}^na^k\right)-a$

So : 
If $a=1$ : $f(n)=n$

If $a\ne 1$ : $f(n)=\frac{a^{n+1}-1}{a-1}-a$
\end{solution}
*******************************************************************************
-------------------------------------------------------------------------------

\begin{problem}[Posted by \href{https://artofproblemsolving.com/community/user/102875}{yunustuncbilek}]
	Find all functions $f: \mathbb R \to \mathbb R$ which satisfy for all $x, y \in \mathbb R$,
\[f(x)^2=f(x+y)f(x-y).\]
	\flushright \href{https://artofproblemsolving.com/community/c6h406129}{(Link to AoPS)}
\end{problem}



\begin{solution}[by \href{https://artofproblemsolving.com/community/user/29428}{pco}]
	\begin{tcolorbox}$f^2(x)=f(x+y)f(x-y)$ find all $f:\mathbb{R}\rightarrow\mathbb{R} $ functions\end{tcolorbox}
Let $P(x,y)$ be the assertion $f(x)^2=f(x+y)f(x-y)$

if $f(u)=0$ for some $u$, then $P(x,u-x)$ $\implies$ $f(x)=0$  and we get the allzero solution.

So let us consider from now that $f(x)\ne 0$ $\forall x$

$P(\frac x2,\frac x2)$ $\implies$ $\frac{f(x)}{f(0)}=\frac{f(\frac x2)^2}{f(0)^2}$ and so $\frac{f(x)}{f(0)}>0$ $\forall x$

Let then $g(x)=\ln \frac{f(x)}{f(0)}$ : we get the new assertion $Q(x,y)$ : $2g(x)=g(x+y)+g(x-y)$ with $g(0)=0$

$Q(x,x)$ $\implies$ $2g(x)=g(2x)$ and so the equation is $g(2x)=g(x+y)+g(x-y)$

And so $g((x+y)+(x-y))=g(x+y)+g(x-y)$ and so $g(x+y)=g(x)+g(y)$

And so $g(x)$ is any solution of Cauchy equation.

\begin{bolded}Hence the solutions \end{bolded}\end{underlined}:

$\boxed{f(x)=a\cdot e^{h(x)}}$ $\forall x$ and for any real $a$ and any $h(x)$ solution of Cauchy equation, which indeed is a solution

Notice that $a=0$ gives the allzero solution
\end{solution}
*******************************************************************************
-------------------------------------------------------------------------------

\begin{problem}[Posted by \href{https://artofproblemsolving.com/community/user/90103}{Winner2010}]
	Find all functions $f:\mathbb{N}\rightarrow\mathbb{N}$ such that $f(a)+b$ divides $(f(b)+a)^2$ for all positive integers $a$ and $$.
	\flushright \href{https://artofproblemsolving.com/community/c6h406324}{(Link to AoPS)}
\end{problem}



\begin{solution}[by \href{https://artofproblemsolving.com/community/user/29428}{pco}]
	\begin{tcolorbox}Find all functions $f:\mathbb{Z^+}\rightarrow\mathbb{Z^+}$ such that $f(a)+b$ divides $(f(b)+a)^2$ for all $a,b$ positive integers.\end{tcolorbox}
Let $P(x,y)$ be the assertion $f(x)+y|(f(y)+x)^2$

Let $x>0$ and $p>f(x)$ prime. $P(p-f(x),x)$ $\implies$ $f(p-f(x))+x|p^2$ and so $f(p-f(x))\in\{p-x,p^2-x\}$

Let $A_x=\{p$ prime integers $>f(x)$ such that $f(p-f(x))=p^2-x\}$

For $p\in A_x$ : $P(p-f(x),y)$ $\implies$ $p^2-x+y|(f(y)+p-f(x))^2$

And so (subtracting $LHS$ from $RHS$) : $p^2+y-x|x-y+(f(y)-f(x))(2p+f(y)-f(x))$

But, for $p$ great enough, $|LHS|>|RHS|$ and $RHS$ cant be zero for any $y$ and any $p$ and so impossibility

So $A_x$ is upper bounded and $\exists N_x$ such that $\forall p>N_x$ $f(p-f(x))=p-x$

Then, For $p>N_x$ : $P(p-f(x),y)$ $\implies$ $p-x+y|(f(y)+p-f(x))^2$
And so (subtracting $LHS^2$ from $RHS$) : $p+y-x|(f(y)-f(x)-y+x)(2p+f(y)-f(x)+y-x)$
And (subtracting $2(f(y)-f(x)-y+x)LHS$ from $RHS$= : $p+y-x|(f(y)-f(x)-y+x)^2$

But, for $p$ great enough, $|LHS|>|RHS|$ and so $RHS$ must be zero for any $y$ and so $f(y)-y=f(x)-x$

So $\boxed{f(x)=x+a}$ $\forall x$ and for any $a\in\mathbb Z_{\ge 0}$ which indeed is a solution
\end{solution}
*******************************************************************************
-------------------------------------------------------------------------------

\begin{problem}[Posted by \href{https://artofproblemsolving.com/community/user/68025}{Pirkuliyev Rovsen}]
	Find all functions $f:\mathbb{R}\to\mathbb{R}$ such that for every $x,y\in \mathbb{R}$,
\[f(x+f(y))=f(x-f(y))+4xf(y).\]
	\flushright \href{https://artofproblemsolving.com/community/c6h406379}{(Link to AoPS)}
\end{problem}



\begin{solution}[by \href{https://artofproblemsolving.com/community/user/29428}{pco}]
	\begin{tcolorbox}Find all functions $f: \mathbb{R}\to\mathbb{R}$ such that for every $x,y\in{R}$
$f(x+f(y))=f(x-f(y))+4xf(y)$\end{tcolorbox}
Let $P(x,y)$ be the assertion $f(x+f(y))=f(x-f(y))+4xf(y)$

$f(x)=0$ $\forall x$ is a solution and let us from now look for non allzero solutions.
Let $u$ such that $f(u)\ne 0$
Let $A=\{2f(x)$ $\forall x\in\mathbb R\}$

$P(\frac x{8f(u)},u)$ $\implies$ $x=2f(\frac x{8f(u)}+f(u))-2f(\frac x{8f(u)}-f(u))$
So any $x\in\mathbb R$ may be written as $x=a-b$ where $a,b\in A$

Let then $g(x)=f(x)-x^2$
Let $a=2f(y)\in A$
$P(x+f(y),y)$ $\implies$ $f(x+a)=f(x)+2ax+a^2$ and so $g(x+a)=g(x)$ $\forall x\in\mathbb R$, $\forall a\in A$

So $g(x-b)=g(x)$ $\forall x\in\mathbb R$, $\forall b\in A$

So $g(x+a-b)=g(x-b)=g(x)$ $\forall x\in\mathbb R$, $\forall a,b\in A$

And since we already proved that any real may be written as $a-b$ with $a,b\in A$, we get $g(x+y)=g(x)$ $\forall x,y\in\mathbb R$ and so $g(x)=c$

Hence the two solutions :
$f(x)=0$ $\forall x$
$f(x)=x^2+c$ $\forall x$ and for any $c\in\mathbb R$, which indeed is a solution.
\end{solution}
*******************************************************************************
-------------------------------------------------------------------------------

\begin{problem}[Posted by \href{https://artofproblemsolving.com/community/user/109919}{abrar-zulkamar}]
	Given $f,g : \mathbb{R} \rightarrow \mathbb{R}$. Such that $f(x+ g(y)) = 3x + y + 12$ for all reals $x$ and$y$. Find the value of $g(2004) +f(2004)$.
	\flushright \href{https://artofproblemsolving.com/community/c6h406483}{(Link to AoPS)}
\end{problem}



\begin{solution}[by \href{https://artofproblemsolving.com/community/user/29428}{pco}]
	\begin{tcolorbox}Given $f,g : \mathbb{R} \rightarrow \mathbb{R}$. Such that $f(x+ g(y)) = 3x + y + 12$. For all $x$, $y$ real. Find the value of $g(2004 +f(2004)$\end{tcolorbox}
Parenthesis mismatch.

Must we read  $g(2004) +f(2004)$ or $g(2004 +f(2004))$ ?
\end{solution}



\begin{solution}[by \href{https://artofproblemsolving.com/community/user/109919}{abrar-zulkamar}]
	\begin{tcolorbox}Must we read  $g(2004) +f(2004)$ or $g(2004 +f(2004))$ ?\end{tcolorbox}

$g(2004 +f(2004)$
\end{solution}



\begin{solution}[by \href{https://artofproblemsolving.com/community/user/29428}{pco}]
	\begin{tcolorbox}[quote="pco"]Must we read  $g(2004) +f(2004)$ or $g(2004 +f(2004))$ ?\end{tcolorbox}

$g(2004 +f(2004)$\end{tcolorbox}
It's a joke ?
\end{solution}



\begin{solution}[by \href{https://artofproblemsolving.com/community/user/29428}{pco}]
	\begin{tcolorbox}Given $f,g : \mathbb{R} \rightarrow \mathbb{R}$. Such that $f(x+ g(y)) = 3x + y + 12$. For all $x$, $y$ real. Find the value of $g(2004 +f(2004)$\end{tcolorbox}
Let $P(x,y)$ be the assertion $f(x+g(y))=3x+y+12$

$P(x-g(0),0)$ $\implies$ $f(x)=3x-3g(0)+12$ and so $f(x)=3x+c$ for some real $c$

$P(-g(x),x)$ $\implies$ $c=-3g(x)+x+12$ and so $g(x)=\frac x3+4-\frac c3$

And it's easy to check that this couple $(f,g)=(3x+c,\frac x3+4-\frac c3)$ is indeed a solution whatever is the real $c$.

Then $\boxed{g(2004)+f(2004)=6684+\frac {2c}3}$

And $\boxed{g(2004+f(2004))=2008}$

Choose what you want (I guess it's the second one)

And I dont know what value could have $g(2004+f(2004)$
\end{solution}
*******************************************************************************
-------------------------------------------------------------------------------

\begin{problem}[Posted by \href{https://artofproblemsolving.com/community/user/68025}{Pirkuliyev Rovsen}]
	Find all functions $f:\mathbb{R}\to\mathbb{R}$ such that $f(x+y)=\max(f(x),y)+\min(x,f(y))$ for all reals $x$ and $y$.
	\flushright \href{https://artofproblemsolving.com/community/c6h406771}{(Link to AoPS)}
\end{problem}



\begin{solution}[by \href{https://artofproblemsolving.com/community/user/29428}{pco}]
	\begin{tcolorbox}Find all functions $f: \mathbb{R}\to\mathbb{R}$ such that $f(x+y)=max(f(x),y)+min(x,f(y))$\end{tcolorbox}
Let $P(x,y)$ be the assertion $f(x+y)=\max(f(x),y)+\min(x,f(y))$

(a) : $P(x,0)$ $\implies$ $f(x)=\max(f(x),0)+\min(x,f(0))$
(b) : $P(0,x)$ $\implies$ $f(x)=\min(0,f(x))+\max(f(0),x)$

Using the fact that $\max(u,v)+\min(u,v)=u+v$, the sum (a)+(b) implies  $f(x)=x+f(0)$ 

Then $P(0,f(0))$ $\implies$ $f(0)=\min(0,2f(0))$ and so $f(0)=0$

Hence the unique solution : $\boxed{f(x)=x}$ $\forall x$, which indeed is a solution.
\end{solution}
*******************************************************************************
-------------------------------------------------------------------------------

\begin{problem}[Posted by \href{https://artofproblemsolving.com/community/user/86167}{alphabeta1729}]
	If $f : \mathbb{R} \to \mathbb{R}$ satisfies
\[f \left( \frac{x+y}{2} \right) = \frac{f(x)+f(y)}{2}, \quad \forall x,y \in \mathbb{R},\]
and $f(0)=3=f'(0)$, then find $f(x)$.
	\flushright \href{https://artofproblemsolving.com/community/c6h406831}{(Link to AoPS)}
\end{problem}



\begin{solution}[by \href{https://artofproblemsolving.com/community/user/29428}{pco}]
	\begin{tcolorbox}If $f : \mathbb{R} \to \mathbb{R}$
 
$f \left( \frac{x+y}{2} \right) = \frac{f(x)+f(y)}{2} \qquad \forall x,y \in \mathbb{R}$

and $f(0)=3=f'(0)$. \begin{bolded}Then find\end{bolded} $f(x)$\end{tcolorbox}
It's very classical that $f(\frac{x+y}2)=\frac{f(x)+f(y)}2$ + continuity implies $f(x)=ax+b$

Adding then the constraints $f(0)=f'(0)=3$, we get the result $\boxed{f(x)=3x+3}$
\end{solution}



\begin{solution}[by \href{https://artofproblemsolving.com/community/user/61082}{Pain rinnegan}]
	\begin{tcolorbox}[quote="alphabeta1729"]If $f : \mathbb{R} \to \mathbb{R}$
 
$f \left( \frac{x+y}{2} \right) = \frac{f(x)+f(y)}{2} \qquad \forall x,y \in \mathbb{R}$

and $f(0)=3=f'(0)$. \begin{bolded}Then find\end{bolded} $f(x)$\end{tcolorbox}
It's very classical that $f(\frac{x+y}2)=\frac{f(x)+f(y)}2$ + continuity implies $f(x)=ax+b$

Adding then the constraints $f(0)=f'(0)=3$, we get the result $\boxed{f(x)=3x+3}$\end{tcolorbox}

How do you know $f$ is continuous?
\end{solution}



\begin{solution}[by \href{https://artofproblemsolving.com/community/user/29428}{pco}]
	\begin{tcolorbox} How do you know $f$ is continuous?\end{tcolorbox}
It seems from the problem statement that the function is differentiable, so continuous.
\end{solution}



\begin{solution}[by \href{https://artofproblemsolving.com/community/user/61082}{Pain rinnegan}]
	I think the OP has to give some clarifications because the problem (as stated) suggests only that $f$ is differentiable (=> continuous) at $0$.
\end{solution}



\begin{solution}[by \href{https://artofproblemsolving.com/community/user/31915}{Batominovski}]
	\begin{tcolorbox}I think the OP has to give some clarifications because the problem (as stated) suggests only that $f$ is differentiable (=> continuous) at $0$.\end{tcolorbox}

But Cauchy-type functional equations do not distinguish between continuity at a point and everywhere continuity.  So, Patrick's solution is correct.
\end{solution}



\begin{solution}[by \href{https://artofproblemsolving.com/community/user/64716}{mavropnevma}]
	No need for anything extra. Take $g(x) = f(x) - f(0)$, so $g(0) = 0$. Then $g$ satisfies the same functional equation as $f$ does. Henceforth $g(x) = g\left( \dfrac {2\cdot 0 + 2x} {2}\right ) = \dfrac {g(0) + g(2x)} {2} = \dfrac {g(2x)} {2}$, so $g(2x) = 2g(x)$.

Now, $g(x+y) = g\left( \dfrac {2x + 2y} {2}\right ) = \dfrac {g(2x) + g(2y)} {2} = g(x) + g(y)$. When $f$ is continuous at $0$ then so is $g$, and it is enough for the Cauchy relation satisfied by $g$ to imply $g$ is linear (continuity at $0$ propagates into continuity everywhere by the Cauchy relation).
\end{solution}
*******************************************************************************
-------------------------------------------------------------------------------

\begin{problem}[Posted by \href{https://artofproblemsolving.com/community/user/86167}{alphabeta1729}]
	Find all functions $f:\mathbb{R} \to \mathbb{R}$ differentiable at $0$ which satisfy the following conditions:
i) $(x+y^{2n+1})=f(x)+(f(y))^{2n+1}, \quad \forall x,y \in \mathbb{R} \text{ and }  \forall n \in \mathbb{N}$, and
ii) $f'(0) \ge 0$.
	\flushright \href{https://artofproblemsolving.com/community/c6h407378}{(Link to AoPS)}
\end{problem}



\begin{solution}[by \href{https://artofproblemsolving.com/community/user/29428}{pco}]
	\begin{tcolorbox}If $f:\mathbb{R} \to \mathbb{R}$ be a function satisfying :

$i)\; f(x+y^{2n+1})=f(x)+[f(y)]^{2n+1} \quad \forall x,y \in \mathbb{R} \text{ and } n \in \mathbb{N}$

$ii)\; f'(0) \ge 0$

The find $f(x)$.\end{tcolorbox}
Question 1 : does "$\forall x,y \in \mathbb{R} \text{ and } n \in \mathbb{N}$" means "$\forall x,y \in \mathbb{R} \text{ and } \forall n \in \mathbb{N}$" ?

Question 2 : is the function supposed to be diffrentiable all over $\mathbb R$ or only at $0$ ?

Question 3 : does "$[f(y)]$" mean "$(f(y))$" or "$\left\lfloor f(y)\right\rfloor$" ?
\end{solution}



\begin{solution}[by \href{https://artofproblemsolving.com/community/user/86167}{alphabeta1729}]
	\begin{tcolorbox}[quote="alphabeta1729"]If $f:\mathbb{R} \to \mathbb{R}$ be a function satisfying :

$i)\; f(x+y^{2n+1})=f(x)+[f(y)]^{2n+1} \quad \forall x,y \in \mathbb{R} \text{ and } n \in \mathbb{N}$

$ii)\; f'(0) \ge 0$

The find $f(x)$.\end{tcolorbox}
Question 1 : does "$\forall x,y \in \mathbb{R} \text{ and } n \in \mathbb{N}$" means "$\forall x,y \in \mathbb{R} \text{ and } \forall n \in \mathbb{N}$" ?

Question 2 : is the function supposed to be diffrentiable all over $\mathbb R$ or only at $0$ ?

Question 3 : does "$[f(y)]$" mean "$(f(y))$" or "$\left\lfloor f(y)\right\rfloor$" ?\end{tcolorbox}

Answer 1 : Edited

Answer 2 : It is given to be differentiable at $x=0$

Answer 3 : Edited
\end{solution}



\begin{solution}[by \href{https://artofproblemsolving.com/community/user/29428}{pco}]
	\begin{tcolorbox}If $f:\mathbb{R} \to \mathbb{R}$ be a function satisfying :

$i)\; f(x+y^{2n+1})=f(x)+(f(y))^{2n+1} \quad \forall x,y \in \mathbb{R} \text{ and }  \forall n \in \mathbb{N}$

$ii)\; f'(0) \ge 0$

Then find $f(x)$.\end{tcolorbox}
Thanks for the answers.

Let $P(x,y,n)$ be the assertion $f(x+y^{2n+1})=f(x)+f(y)^{2n+1}$

$P(0,0,1)$ $\implies$ $f(0)=0$
Let $y\ne 0$ : $P(x,\sqrt[3]y,1)$ $\implies$ $f(x+y)=f(x)+f(\sqrt[3]y)^3$ and so $\frac{f(x+y)-f(x)}y=\left(\frac{f(\sqrt[3]y)}{\sqrt[3]y}\right)^3$

Using then the fact that $f(x)$ is differentiable at $0$ and setting $y\to 0$, we get that $f(x)$ is continuous and differentiable everywhere and that $f'(x)$ is constant.

So $f(x)=ax$

Plugging this in original equation, we get $a\in\{-1,0,1\}$

Using $f'(0)\ge 0$, it remains only two solutions :
$f(x)=0$ $\forall x$
$f(x)=x$ $\forall x$
\end{solution}
*******************************************************************************
-------------------------------------------------------------------------------

\begin{problem}[Posted by \href{https://artofproblemsolving.com/community/user/67223}{Amir Hossein}]
	For $n \in \mathbb N$, let $f(n)$ be the number of positive integers $k \leq n$ that do not contain the digit $9$. Does there exist a positive real number $p$ such that $\frac{f(n)}{n} \geq p$ for all positive integers $n$?
	\flushright \href{https://artofproblemsolving.com/community/c6h407851}{(Link to AoPS)}
\end{problem}



\begin{solution}[by \href{https://artofproblemsolving.com/community/user/29428}{pco}]
	\begin{tcolorbox}For $n \in \mathbb N$, let $f(n)$ be the number of positive integers $k \leq n$ that do not contain the digit $9$. Does there exist a positive real number $p$ such that $\frac{f(n)}{n} \geq p$ for all positive integers $n$?\end{tcolorbox}
It's easy to see that $f(10^n)=9^n$

So, we get $a_k=\frac{f(10^k)}{10^k}=\left(\frac 9{10}\right)^k$ and $\lim_{k\to+\infty}a_k=0$ and so $\boxed{\text{no such }p}$
\end{solution}



\begin{solution}[by \href{https://artofproblemsolving.com/community/user/5787}{ZetaX}]
	Actually, the sum of inverses of these numbers (without digit $9$) converges.
\end{solution}



\begin{solution}[by \href{https://artofproblemsolving.com/community/user/29428}{pco}]
	\begin{tcolorbox}Actually, the sum of inverses of these numbers (without digit $9$) converges.\end{tcolorbox}
Yes, and limit is $22.920676619...$ and rather difficult to get with a computer (simple direct computation needs to go at least till $\frac 1{10^{20}}$ in order to reach $22$  and so is meaningless  :) )
\end{solution}
*******************************************************************************
-------------------------------------------------------------------------------

\begin{problem}[Posted by \href{https://artofproblemsolving.com/community/user/110882}{colosimo}]
	Determine all pairs of functions $f,g:\mathbb{R}\rightarrow\mathbb{R}$ such that for any $x,y\in \mathbb{R}$,
\[f(x)f(y)=g(x)g(y)+g(x)+g(y).\]
	\flushright \href{https://artofproblemsolving.com/community/c6h408195}{(Link to AoPS)}
\end{problem}



\begin{solution}[by \href{https://artofproblemsolving.com/community/user/29428}{pco}]
	\begin{tcolorbox}Determine all pairs of functions $f,g:\mathbb{R}\rightarrow\mathbb{R}$ such that for any $x,y\in \mathbb{R}$,

\[f(x)f(y)=g(x)g(y)+g(x)+g(y).\]\end{tcolorbox}
Let $P(x,y)$ be the assertion $f(x)f(y)=g(x)g(y)+g(x)+g(y)$

If $g(0)=-1$, then $P(0,0)$ $\implies$ $f(0)^2=-1$, impossible. And so $g(0)\ne -1$ 

Then $P(x,0)$ $\implies$ $g(x)=\frac{f(x)f(0)-g(0)}{g(0)+1}$ and so $g(x)=af(x)+b$ for some real $a,b$

$P(x,x)$ becomes then $(a^2-1)f(x)^2+2a(b+1)f(x)+b^2+2b=0$ and so $f(x)$ may take at most \end{underlined} two values $u,v$ (the roots of the quadratic)
If we indeed have two values and if $f(x)=u$ for some $x$ and $f(y)=v$ for some $y$, then :
e1 : $P(x,x)$ $\implies$ $(a^2-1)u^2+2a(b+1)u+b^2+2b=0$
e2 : $P(y,y)$ $\implies$ $(a^2-1)v^2+2a(b+1)v+b^2+2b=0$
e3 : $P(x,y)$ $\implies$ $(a^2-1)uv+a(b+1)(u+v)+b^2+2b=0$
e1+e2-2e3 : $(a^2-1)(u-v)^2=0$
And so either $a^2=1$, either $u=v$ and in both cases $f(x)$ may take only one value.

So $f(x)=c$ $\forall x$ and, since $g(x)=af(x)+b$, we get $g(x)=d$

\begin{bolded}Hence the answer \end{bolded}\end{underlined}:
$f(x)=c$ $\forall x$
$g(x)=d$ $\forall x$
for any real $c,d$ such that $c^2=d(d+2)$
\end{solution}



\begin{solution}[by \href{https://artofproblemsolving.com/community/user/109774}{littletush}]
	let $x=y$
then$f^2(x)=g^2(x)+2g(x)=(g(x)+1)^2-1$
hence $(g(x)+1)^2=f^2(x)+1$
by the original equation,
$(f(x)f(y)+1)^2=(g(x)+1)^2(g(y)+1)^2=(f^2(x)+1)(f^2(y)+1)$
by Carlson's ineq $LHS\le RHS$
so the equality holds,hence
$f(x)=f(y)$ for all (x,y)
so f is constant
let $f(x)=c$
then$(g(x)+1)(g(y)+1)=c^2+1$
for any (x,y)
so g is constant hence
$g(x)=\sqrt{c^2+1}-1$ or $-\sqrt{c^2+1}-1$.
\end{solution}



\begin{solution}[by \href{https://artofproblemsolving.com/community/user/113192}{AlexanderMusatov}]
	I hope I haven't done any silly mistake. Substitute $ h(x)=g(x)+1 $ and we get $f(x)f(y)+1=h(x)h(y)$. Plugging $x=y$ we get $ h(x)^2-f(x)^2=1 $ so we can set $ h(x)=\cosh e(x) $ and $ f(x)=\sinh e(x) $ where $ e(x) $ is a real function. Then, we get $ 1= \cosh e(x) \cosh e(y) - \sinh e(x) \sinh e(y) $ by using the hyperbolic addition formula $ \cosh (x-y)=\cosh (x)\cosh (y) -\sinh(x) \sinh(y) $ we get $ 1= \cosh (e(x)-e(y)) $ from which we get $ 1=\cosh (e(x)-e(0)) $ therefore $ e(x) $ is a constant, thus $ f(x) $ and $ g(x) $ are constants.
\end{solution}



\begin{solution}[by \href{https://artofproblemsolving.com/community/user/60735}{hatchguy}]
	There is actually a direct solution...

Let $x=y$ we get $f(x)^2 = g(x)^2 + 2g(x) \implies f(x)^2+1 = (g(x)+1)^2$

The original equation is equivalent to \[f(x)f(y) +1 = (g(x) +1)(g(y)+1) \implies (f(x)f(y)+1)^2= (f(x)^2+1)(f(y)^2+1)\] which after expanding implies $ f(x) = f(y)$ and we can finish as above.
\end{solution}
*******************************************************************************
-------------------------------------------------------------------------------

\begin{problem}[Posted by \href{https://artofproblemsolving.com/community/user/92753}{WakeUp}]
	Let a function $g:\mathbb{N}_0\to\mathbb{N}_0$ satisfy $g(0)=0$ and $g(n)=n-g(g(n-1))$ for all $n\ge 1$. Prove that:

a) $g(k)\ge g(k-1)$ for any positive integer $k$.
b) There is no $k$ such that $g(k-1)=g(k)=g(k+1)$.
	\flushright \href{https://artofproblemsolving.com/community/c6h408646}{(Link to AoPS)}
\end{problem}



\begin{solution}[by \href{https://artofproblemsolving.com/community/user/29428}{pco}]
	\begin{tcolorbox}Let a function $g:\mathbb{N}_0\to\mathbb{N}_0$ satisfy $g(0)=0$ and $g(n)=n-g(g(n-1))$ for all $n\ge 1$. Prove that:

a) $g(k)\ge g(k-1)$ for any positive integer $k$.
b) There is no $k$ such that $g(k-1)=g(k)=g(k+1)$.\end{tcolorbox}
First notice that $g(n)\le n$ $\forall n\in\mathbb N_0$
Let us then prove with induction that $g(n+1)-g(n)\in\{0,1\}$ $\forall n\in\mathbb N_0$

$g(0)=0$
$g(1)=1-g(g(0))=1$
$g(2)=2-g(g(1))=1$

and so $g(k+1)-g(k)\in\{0,1\}$ $\forall k\in[0,1]$

Suppose now $g(k+1)-g(k)\in\{0,1\}$ $\forall k\in[0,n-1]$ for some $n\ge 2\in\mathbb N$
$g(n+1)-g(n)=1-(g(g(n))-g(g(n-1)))$
We know that $g(n)-g(n-1)\in\{0,1\}$ and so :
If $g(n)-g(n-1)=0$, we get $g(g(n))-g(g(n-1))=0$ and so $g(n+1)-g(n)=1$
If $g(n)-g(n-1)=1$, we get $g(g(n))-g(g(n-1))=g(g(n-1)+1)-g(g(n-1))\in\{0,1\}$ (since $g(n-1)\le n-1$ and using then the induction property)
And so $g(n+1)-g(n)=1-(g(g(n))-g(g(n-1)))\in\{0,1\}$
End of induction step

And so $g(n+1)\ge g(n)$ $\forall n\in\mathbb N_0$ and part $a)$ is proved.

Part b) is quite simple :
If $g(n)=g(n-1)$, then $g(g(n))=g(g(n-1))$ and so $g(n+1)-g(n)=n+1-g(g(n))-n+g(g(n-1))=1$ and so $g(n+1)\ne g(n)$
Q.E.D.


\begin{bolded}Notice \end{bolded}\end{underlined}: this sequence is the well known sequence $g(n)=\left\lfloor\frac{2(n+1)}{1+\sqrt 5}\right\rfloor$
\end{solution}
*******************************************************************************
-------------------------------------------------------------------------------

\begin{problem}[Posted by \href{https://artofproblemsolving.com/community/user/67223}{Amir Hossein}]
	Let $E$ be the set of all bijective mappings from $\mathbb R$ to $\mathbb R$ satisfying
\[f(t) + f^{-1}(t) = 2t, \qquad \forall t \in \mathbb R,\]
where $f^{-1}$ is the mapping inverse to $f$. Find all elements of $E$ that are monotonic mappings.
	\flushright \href{https://artofproblemsolving.com/community/c6h409307}{(Link to AoPS)}
\end{problem}



\begin{solution}[by \href{https://artofproblemsolving.com/community/user/29428}{pco}]
	\begin{tcolorbox}Let $E$ be the set of all bijective mappings from $\mathbb R$ to $\mathbb R$ satisfying
\[f(t) + f^{-1}(t) = 2t, \qquad \forall t \in \mathbb R,\]
where $f^{-1}$ is the mapping inverse to $f$. Find all elements of $E$ that are monotonic mappings.\end{tcolorbox}
$f(x)$ strictly (since bijective) monotonic implies $f^{-1}(x)$ strictly monotonic in the same direction (both increasing or both decreasing) and since their sum is increasing, we get that $f(x)$ is increasing.

Suppose now that $f(x)-x$ is not constant. Let then $u\ne v$ such that $f(u)-u=a>b=f(v)-v$

Using $f(x)+f^{-1}(x)=2x$, it's easy to show that $f(u+na)=u+(n+1)a$ and $f(v+nb)=v+(n+1)b$ $\forall n\in\mathbb Z$

Let then $n=\left\lfloor\frac{v-u}{a-b}\right\rfloor$ so that $n+1>\frac{v-u}{a-b}\ge n$ :
$\frac{v-u}{a-b}\ge n$ $\implies$ $v-u\ge na-nb$ 
$\implies$ $v+nb\ge u+na$
$\implies$ $f(v+nb)\ge f(u+na)$ (since increasing) 
$\implies$ $v+(n+1)b\ge u+(n+1)a$ 
$\implies$ $\frac{v-u}{a-b}\ge n+1$
And so contradiction.

So $f(x)-x$ is constant and $\boxed{f(x)=x+c}$ $\forall x$, and for any real $c$ 
And it's easy to check back that these functions indeed are solutions.
\end{solution}
*******************************************************************************
-------------------------------------------------------------------------------

\begin{problem}[Posted by \href{https://artofproblemsolving.com/community/user/10045}{socrates}]
	Determine all monotone functions $f:\mathbb{R}_0^+ \rightarrow \mathbb{R}$ such that  for all $x,y \geq 0$,
\[ f(x+y)-f(x)-f(y)=f(xy+1)-f(xy)-f(1),\] and \[f(3)+3f(1)=3f(2)+f(0).\]
	\flushright \href{https://artofproblemsolving.com/community/c6h410339}{(Link to AoPS)}
\end{problem}



\begin{solution}[by \href{https://artofproblemsolving.com/community/user/29428}{pco}]
	\begin{tcolorbox}Determine all monotone functions $f:\mathbb{R}_0^+ \rightarrow \mathbb{R}$ such that 

$\displaystyle  f(x+y)-f(x)-f(y)=f(xy+1)-f(xy)-f(1),$ for all $x,y \geq 0$ and 

$f(3)+3f(1)=3f(2)+f(0).$



Source: [url]http://www.mathnet.or.kr\/mathnet\/olympiad_file\/UkrMO_2005(30).pdf[\/url]\end{tcolorbox}
If $f(x)$ is solution, then so is $f(x)+a$ and so Wlog say $f(1)=1$ 

Let $P(x,y)$ be the assertion $f(x+y)-f(x)-f(y)=f(xy+1)-f(xy)-1$

Let $m,n,p\in\mathbb N$  and let $g(x)=f(\frac xp)$
Comparing $P(\frac {2m}p,\frac np)$ and $P(\frac {2n}p,\frac mp)$, we get :
$g(2m+n)-g(2m)-g(n)=g(2n+m)-g(2n)-g(m)$

1) Let us look for all solutions of the following problem :
"Find all functions $ g(x)$ from $ \mathbb N\to\mathbb R$ such that : $ g(2x + y) - g(2x) - g(y) = g(2y + x) - g(2y) - g(x)$ $ \forall x,y\in\mathbb N$"

The set $ \mathbb S$ of solutions is a $ \mathbb R$-vector space.
Setting $ y = 1$, we get $ g(2x + 1) = g(2x) + g(1) + g(x + 2) - g(2) - g(x)$
Setting $ y = 2$, we get $ g(2x + 2) = g(2x) + g(2) + g(x + 4) - g(4) - g(x)$
From these two equations, we see that knowledge of $ g(1),g(2),g(3),g(4)$ and $ g(6)$ gives knowledge of $ g(x)$ $ \forall x\in\mathbb N$ and so dimension of $ \mathbb S$ is at most $ 5$.
But the $ 5$ functions below are independant solutions :
$ g_1(x) = 1$
$ g_2(x) = x$
$ g_3(x) = x^2$
$ g_4(x) = 1$ if $ x = 0\pmod 2$ and $ g_4(x) = 0$ if $ x\neq 0\pmod 2$
$ g_5(x) = 1$ if $ x = 0\pmod 3$ and $ g_5(x) = 0$ if $ x\neq 0\pmod 3$
And the general solution is $ g(x) = a\cdot x^2 + b\cdot x + c + d\cdot g_4(x) + e\cdot g_5(x)$

2) back to our problem
So $f(\frac xp)=a_px^2+b_px+c_p+d_pg_4(x)+e_pg_5(x)$ $\forall x\in\mathbb N$

Choosing $x=kp$, we get $f(k)=a_pk^2p^2+b_pkp+c_p+d_pg_4(kp)+e_pg_5(kp)$ and so $a_pp^2=a$ and $b_pp=b$ for some real $a,b$
Choosing $x=2kp$, $x=3kp$ and $x=6kp$, we get $c_p=c$ and $d_p=e_p=0$

So $f(\frac xp)=a\frac {x^2}{p^2}+b\frac xp+c$ $\forall x\in\mathbb N$

And so $f(x)=ax^2+bx+c$ $\forall x\in\mathbb Q^+$

$f(x)$ monotonous implies then $a=0$ or $\frac ba\ge 0$

$f(x)$ monotonous implies then $f(x)=ax^2+bx+c$ $\forall x\in\mathbb R^+$

$f(3)+3f(1)=3f(2)+f(0)$ implies then $f(x)=ax^2+bx+c$ $\forall x\in\mathbb R_0^+$
and it's easy to check back that this mandatory form indeed is a solution.

\begin{bolded}Hence the answer \end{bolded}\end{underlined}:
$\boxed{f(x)=ax^2+bx+c}$ $\forall x\ge 0$ and for any real $a,b,c$ such that $ab\ge 0$
\end{solution}
*******************************************************************************
-------------------------------------------------------------------------------

\begin{problem}[Posted by \href{https://artofproblemsolving.com/community/user/32935}{soruz}]
	Find all surjective functions $ f:\mathbb{N}\to\mathbb{N} $ if $ f (n) \ge n+(-1)^n, \forall n \in \mathbb{N} $.
	\flushright \href{https://artofproblemsolving.com/community/c6h410614}{(Link to AoPS)}
\end{problem}



\begin{solution}[by \href{https://artofproblemsolving.com/community/user/29428}{pco}]
	\begin{tcolorbox}Find all surjective functions $ f:\mathbb{N}\to\mathbb{N} $ if $ f (n) \ge n+(-1)^n, \forall n \in \mathbb{N} $.\end{tcolorbox}
Let $S_n$ be the set of natural numbers solutions of the equation $x+(-1)^x\le n$ :
Obviously, this set is the set of all even numbers $\le n-1$ and all odd numbers $\le n+1$ and so :

$S_{2p}=\{1,2,3,...,2p-1,2p+1\}$
$S_{2p+1}=\{1,2,3,...,2p+1\}$

So $S_1=\{1\}$ and so $f(1)=1$

We clearly have $f^{-1}([1,n])\subseteq\bigcup_{k\in[1,n]}S_k$
So $f^{-1}([1,2p])\subseteq \{1,2,3,...,2p-1,2p+1\}$
And $f^{-1}([1,2p+1])\subseteq \{1,2,3,...,2p+1\}$

So $|f^{-1}([1,n])|=n$ and this implies that $f^{-1}(\{n\})=f^{-1}([1,n])\setminus f^{-1}([1,n-1])$

\begin{bolded}Hence the unique solution \end{bolded}\end{underlined}:
$f(1)=1$
$f(2p)=2p+1$ $\forall p\ge 1$
$f(2p+1)=2p$ $\forall p\ge 1$
\end{solution}



\begin{solution}[by \href{https://artofproblemsolving.com/community/user/84677}{andreass}]
	We otherwise use induction which is fairly simple.
First easily prove $f(1)=1, f(2)=3, f(3)=2$. 
Then assume that for some $k \in \mathbb{N} , \forall n \in \mathbb{N}, 1 \le n \le k, f(2k)=2k+1$ and $f(2k+1)=2k$.
Then the equations $f(x)=2k+2$ and $f(y)=2k+3$ both have solutions since the function is surjective and obviously $x \neq y$.
However $\forall n \ge k+2, f(2n) \ge 2n+1 \ge 2k+5$ and $f(2n+1) \ge 2n+1-1\ge 2k+4$ hence $x,y \in \{2k+2,2k+3\}$. But we know $f(2k+2) \ge 2k+3$ therefore $x \neq 2k+2 \Rightarrow x=2k+3 \Rightarrow y=2k+2$.
And by induction we have proved that the solution to the functional equation is:
$ f(n)=\begin{cases}1 &\textrm{ if } n=1\\ n+1 &\textrm{ if } n \equiv 0\pmod 2,\\ n-1 &\textrm{ if } n \equiv 1\pmod 2\end{cases} $
\end{solution}



\begin{solution}[by \href{https://artofproblemsolving.com/community/user/210672}{Chirantan}]
	\begin{tcolorbox}We otherwise use induction which is fairly simple.
First easily prove $f(1)=1, f(2)=3, f(3)=2$. 
Then assume that for some $k \in \mathbb{N} , \forall n \in \mathbb{N}, 1 \le n \le k, f(2k)=2k+1$ and $f(2k+1)=2k$.
Then the equations $f(x)=2k+2$ and $f(y)=2k+3$ both have solutions since the function is surjective and obviously $x \neq y$.
However $\forall n \ge k+2, f(2n) \ge 2n+1 \ge 2k+5$ and $f(2n+1) \ge 2n+1-1\ge 2k+4$ hence $x,y \in \{2k+2,2k+3\}$. But we know $f(2k+2) \ge 2k+3$ therefore $x \neq 2k+2 \Rightarrow x=2k+3 \Rightarrow y=2k+2$.
And by induction we have proved that the solution to the functional equation is:
$ f(n)=\begin{cases}1 &\textrm{ if } n=1\\ n+1 &\textrm{ if } n \equiv 0\pmod 2,\\ n-1 &\textrm{ if } n \equiv 1\pmod 2\end{cases} $\end{tcolorbox}

How do I proove $f(1)=1, f(2)=3, f(3)=2$.
\end{solution}



\begin{solution}[by \href{https://artofproblemsolving.com/community/user/305697}{AE-TheRocket}]
	But as the function is surjective , $0$ must have an inverse which seems to be omitted in both solutions! am i wrong ?
\end{solution}



\begin{solution}[by \href{https://artofproblemsolving.com/community/user/29428}{pco}]
	\begin{tcolorbox}But as the function is surjective , $0$ must have an inverse which seems to be omitted in both solutions! am i wrong ?\end{tcolorbox}

As usual in this international forum, the set $\mathbb N$ is considered as the set of all positive integers (and so $0\notin\mathbb N$).

We all know that in some countries (mine, for example, and likely yours), the convention is that $0\in\mathbb N$. But since convention is different in each country, a kind of commun choice must be done on such an international forum.

The best thing to do would generally be to add the definition of $\mathbb N$ in each problem using it. 
\end{solution}



\begin{solution}[by \href{https://artofproblemsolving.com/community/user/348216}{i_am_not_the_one_who}]
	How about trigonometric functions like x+cos(pi*x) and x-sin((2x-1)pi\/2)? Do we usually neglect solutions of this type?
\end{solution}



\begin{solution}[by \href{https://artofproblemsolving.com/community/user/29428}{pco}]
	\begin{tcolorbox}How about trigonometric functions like x+cos(pi*x) and x-sin((2x-1)pi\/2)? Do we usually neglect solutions of this type?\end{tcolorbox}

Considering $\mathbb N$ as the set of positive integers, none of these two functions (which are the same, in fact) is a surjective function from $\mathbb N\to\mathbb N$, as required.
\end{solution}



\begin{solution}[by \href{https://artofproblemsolving.com/community/user/339220}{omarius}]
	please how can we prove that f(1)=1 ?

\end{solution}



\begin{solution}[by \href{https://artofproblemsolving.com/community/user/330078}{Delray}]
	\begin{tcolorbox}please how can we prove that f(1)=1 ?\end{tcolorbox}

We have that $f(1)\geq 0$. Similarly, $f(2)\geq3$ and $f(3)\geq 2$. Since $f$ is surjective, there must be some value $x$, such that $f(x)=1$, and since one is the only possible value in this range, we must have that $f(1)=1$.
\end{solution}



\begin{solution}[by \href{https://artofproblemsolving.com/community/user/339220}{omarius}]
	\begin{tcolorbox}[quote="soruz"]Find all surjective functions $ f:\mathbb{N}\to\mathbb{N} $ if $ f (n) \ge n+(-1)^n, \forall n \in \mathbb{N} $.\end{tcolorbox}
Let $S_n$ be the set of natural numbers solutions of the equation $x+(-1)^x\le n$ :
Obviously, this set is the set of all even numbers $\le n-1$ and all odd numbers $\le n+1$ and so :

$S_{2p}=\{1,2,3,...,2p-1,2p+1\}$
$S_{2p+1}=\{1,2,3,...,2p+1\}$

So $S_1=\{1\}$ and so $f(1)=1$

We clearly have $f^{-1}([1,n])\subseteq\bigcup_{k\in[1,n]}S_k$
So $f^{-1}([1,2p])\subseteq \{1,2,3,...,2p-1,2p+1\}$
And $f^{-1}([1,2p+1])\subseteq \{1,2,3,...,2p+1\}$

So $|f^{-1}([1,n])|=n$ and this implies that $f^{-1}(\{n\})=f^{-1}([1,n])\setminus f^{-1}([1,n-1])$

\begin{bolded}Hence the unique solution \end{bolded}\end{underlined}:
$f(1)=1$
$f(2p)=2p+1$ $\forall p\ge 1$
$f(2p+1)=2p$ $\forall p\ge 1$\end{tcolorbox}

can you please clarify this solution? it's beautiful but i haven't managed yes to get it fully 
thank you
\end{solution}



\begin{solution}[by \href{https://artofproblemsolving.com/community/user/29428}{pco}]
	\begin{tcolorbox}can you please clarify this solution? it's beautiful but i haven't managed yes to get it fully \end{tcolorbox}
Dont hesitate to indicate what is the first line you dont understand.


\end{solution}



\begin{solution}[by \href{https://artofproblemsolving.com/community/user/368111}{Anis2017}]
	i havn't understood nothing
can you explain more please?
\end{solution}
*******************************************************************************
-------------------------------------------------------------------------------

\begin{problem}[Posted by \href{https://artofproblemsolving.com/community/user/110786}{mastergeo}]
	1. Find all pairs of functions $f,g: \mathbb{Z} \to \mathbb{Z}$ such that: 
\[f(g(x)+y)=g(f(y)+x)\]
holds for arbitrary integers $x$ and $y$ and $g$ is injective.

2. Find all function $f: \mathbb{R} \to \mathbb{R}$ such that:
\[f(f(x)+y)=f(x^2-y^2)+4f(x).y,\quad \forall x,y \in \mathbb{R}.\]
	\flushright \href{https://artofproblemsolving.com/community/c6h411155}{(Link to AoPS)}
\end{problem}



\begin{solution}[by \href{https://artofproblemsolving.com/community/user/67223}{Amir Hossein}]
	I'm going to make a set of problems for Functional Equations, and I saw the second problem [url=http://www.artofproblemsolving.com/Forum/viewtopic.php?t=328125]here[\/url]. :)

Anyways, please don't post more than one problem in each topic.
\end{solution}



\begin{solution}[by \href{https://artofproblemsolving.com/community/user/111614}{iamnot}]
	\begin{tcolorbox}1. Find all pairs of functions $f,g: \mathbb{Z} \to \mathbb{Z}$ such that: 
\[f(g(x)+y)=g(f(y)+x)\]
holds for arbitrary integers $x,y$ and $g$ is injective.\end{tcolorbox}
I didn't notice why we need $\mathbb{Z}$ instead of $\mathbb{R}$ in this problem, but anyway.

$f(g(x)+y)=g(f(y)+x)\Leftrightarrow g(f(g(x)+y)+z)=g(g(f(y)+x)+z)\Leftrightarrow$

$\Leftrightarrow f(g(z)+g(x)+y)=g(g(f(y)+x)+z) \Leftrightarrow$

$\Leftrightarrow g(f(g(z)+y)+x)=g(g(f(y)+x)+z)\implies$

$\implies f(g(z)+y)+x=g(f(y)+x)+z \Leftrightarrow g(f(y)+z)+x=g(f(y)+x)+z.$ 

Put $z=-f(y):$
$g(0)+x+f(y)=g(f(y)+x)$

Put $x=-f(y)+t:$
$g(t)=t+g(0)=t+c.$

Our statement now looks as follows
$f(x+y+c)=x+f(y)+c.$

Put $x=-c-y:$
$f(y)=y+f(0).$

\begin{bolded}Answer: $f(x)=x+c_1, g(x)=x+c_2$\end{bolded}
\end{solution}



\begin{solution}[by \href{https://artofproblemsolving.com/community/user/111614}{iamnot}]
	\begin{tcolorbox}I saw the second problem [url=http://www.artofproblemsolving.com/Forum/viewtopic.php?t=328125]here[\/url].\end{tcolorbox}
Are you sure that it's the same problem?
It is your link:
\begin{tcolorbox}Find all functions $ f: \mathbb R \to \mathbb R$ such that 
\[f(f(x)+y)=f(x^{2}-y)+4yf(x) \qquad \forall x,y \in \mathbb R.\]\end{tcolorbox}
It is the given problem:
\begin{tcolorbox}Find all function $f: \mathbb{R} \to \mathbb{R}$ such that:
\[f(f(x)+y)=f(x^2-y^2)+4yf(x), \forall x,y \in \mathbb{R}\]\end{tcolorbox}
\end{solution}



\begin{solution}[by \href{https://artofproblemsolving.com/community/user/29428}{pco}]
	\begin{tcolorbox}2. Find all function $f: \mathbb{R} \to \mathbb{R}$ such that:
\[f(f(x)+y)=f(x^2-y^2)+4f(x).y, \forall x,y \in \mathbb{R}\]\end{tcolorbox}
$f(x)=0$ $\forall x$ is a solution.
Let us from now look for non allzero solutions.
Let $P(x,y)$ be the assertion $f(f(x)+y)=f(x^2-y^2)+4f(x)y$
Let $f(u)=v\ne 0$

1) Any real may be written as $x=f(a)-f(b)$ for some $a,b\in\mathbb R$
=====================================================
(a) : $P(u,\frac{x}{8v})$ $\implies$ $f(u+\frac{x}{8v})=f(u^2-(\frac{x}{8v})^2)+\frac x2$
(b) : $P(u,-\frac{x}{8v})$ $\implies$ $f(u-\frac{x}{8v})=f(u^2-(\frac{x}{8v})^2)-\frac x2$
(a)-(b) : $x=f(u+\frac{x}{8v})-f(u-\frac{x}{8v})$
Q.E.D.

2) $f(x)$ is even 
==============
(a) : $P(x,f(y))$ $\implies$ $f(f(x)+f(y))=f(x^2-f(y)^2)+4f(x)f(y)$
(b) : $P(x,-f(y))$ $\implies$ $f(f(x)-f(y))=f(x^2-f(y)^2)-4f(x)f(y)$
(c) : $P(y,f(x))$ $\implies$ $f(f(x)+f(y))=f(y^2-f(x)^2)+4f(x)f(y)$
(d) : $P(y,-f(x))$ $\implies$ $f(f(y)-f(x))=f(y^2-f(x)^2)-4f(x)f(y)$
(a)-(b)-(c)+(d) : $f(f(x)-f(y))=f(f(y)-f(x))$
Q.E.D. (using 1) )

3) If $f(x)=x$ for some $x$ implies $x=0$
===================================
$P(x,-x)$ $\implies$ $f(f(x)-x)=f(0)-4xf(x)$
If $f(x)=x$, this becomes $f(0)=f(0)-4x^2$
Q.E.D.

4) $f(0)=0$
==========
$P(0,0)$ $\implies$ $f(f(0))=f(0)$ and so, using 3) : $f(0)=0$
Q.E.D.

5) No non allzero solution
=======================
$P(0,u)$ $\implies$ $f(u)=f(-u^2)=f(u^2)$
$P(u,0)$ $\implies$ $f(f(u))=f(u^2)$
And so $f(f(u))=f(u)$ and so, using 3) : $f(u)=0$ and so contradiction


Hence the unique solution : $\boxed{f(x)=0}$ $\forall x$
\end{solution}
*******************************************************************************
-------------------------------------------------------------------------------

\begin{problem}[Posted by \href{https://artofproblemsolving.com/community/user/110786}{mastergeo}]
	1. Prove that for all functions $f(x): \mathbb{Q} \to\mathbb{Q}$, there exist injections $g,h: \mathbb{Q} \to \mathbb{Q}$ such that $f(x)=g(x)+h(x)$ for all $x \in \mathbb{Q}$.

2. Prove that for all injective functions $f(x): \mathbb{Q} \to\mathbb{Q}$, there exist injections $g,h: \mathbb{Q} \to \mathbb{Q}$ such that $f(x)=g(x)\cdot h(x)$ for all $x \in \mathbb{Q}$ if and only if $|f^{-1}(\{0\})|\le 2$.
	\flushright \href{https://artofproblemsolving.com/community/c6h411686}{(Link to AoPS)}
\end{problem}



\begin{solution}[by \href{https://artofproblemsolving.com/community/user/29428}{pco}]
	\begin{tcolorbox}
2. Prove that for all injective functions $f(x): \mathbb{Q} \to\mathbb{Q}$, there exist injections $g,h: \mathbb{Q} \to \mathbb{Q}$ such that $f(x)=g(x)\cdot h(x)$ for all $x \in \mathbb{Q}$ if and only if $|f^{-1}(\{0\})|\le 2$.\end{tcolorbox}


I claim that the condition $|f^{-1}(\{0\})|\le 2$ is mandatory and sufficient to get the required property.

1) If $|f^{-1}(\{0\})|>2$ the property is wrong :
===============================
Let $a,b,c$ three different rational numbers such that $f(a)=f(b)=f(c)=0$.
If the property is true, then $g(a)h(a)=g(b)h(b)=g(c)h(c)=0$ and at least two of $\{g(a),g(b),g(c)\}$ or of $\{h(a),h(b),h(c)\}$ must be zero.
And so $g(x)$ and $h(x)$ cant be both injective.
Q.E.D.

2) If $|f^{-1}(\{0\})|\le 2$ the property is true :
=================================
Let $\mathbb P$ be the set of prime integers.
$\forall p\in\mathbb P,x\in\mathbb Q^*$, let $v_p(x)$ from $\mathbb Q^*\to\mathbb Z$ the power of $p$ (zero, positive or negative) in prime representation of $x$.
Let $m(x)$ from $\mathbb Q^*\to\mathbb Z$ such that $m(x)=\min_{p\in\mathbb P}v_p(x)$

Let $a(x)$ an injection from $\mathbb Q^*\to\mathbb P$

Let $r_u(x)$ a bijection from $f^{-1}(\{f(u)\})\to[1,|f^{-1}(\{f(u)\})|]$ and $b(x)=r_x(x)$
$b(x)$ is just the rank (in some arbitrary order) of $x$ in the set of all $z$ such that $f(z)=f(x)$
Notice too that this has sense even if $f^{-1}(\{f(u)\})$ is infinite since it is countable (since $\subseteq\mathbb Q$)

Definition of $g(x),h(x)$
==============
If $f(x)=0$, then $b(x)\in\{1,2\}$ and then :
If $f(x)=0$ and $b(x)=1$ : $g(x)=1$ and $h(x)=0$
If $f(x)=0$ and $b(x)=2$ : $g(x)=0$ and $h(x)=1$

If $f(x)\ne 0$ :
$g(x)=a(f(x))^{b(x)+\max(0,v_{a(f(x))}(f(x))-\min(0,m(f(x))))}$
$h(x)=\frac{f(x)}{g(x)}$ (which has sense since $g(x)\ne 0$)

Obviously $f(x)=g(x)h(x)$ $\forall x$


Proof that $g(x)$ is injective
================
Notice that :
$b(x)>0$ (remember $b(x)$ is just the rank (in some arbitrary order) of $x$ in the set of all $z$ such that $f(z)=f(x)$)
$\max(0,v_{a(f(x))}-\min(0,m(f(x))))\ge 0$ 
and so $b(x)+\max(0,v_{a(f(x))}(f(x))-\min(0,m(f(x))))>0$ and $g(x)>1$ when $f(x)\ne 0$

If $g(x)=0$, then $f(x)=0$ and $b(x)=2$ and so $x=r_0^{-1}(2)$ and so $x$ is unique.
If $g(x)=1$, then $f(x)=0$ and $b(x)=1$ and so $x=r_0^{-1}(1)$ and so $x$ is unique.

If $g(x)\notin\{0,1\}$, then $g(x)=p^k$ for some prime $p$ and some integer $k\ge 1$

We immediately get then $f(x)=a^{-1}(p)$ and then $b(x)=k-\max(0,v_p(f(x))-\min(0,m(f(x))))$
So knowledge of $g(x)\notin\{0,1\}$ gives unique knowledge of $f(x)$ and $b(x)$

Choosing then $t$ as any element of $f^{-1}(\{f(x)\})$, we get $x=r_t^{-1}(b(x))$ uniquely determined.
(in fact we know $f(x)$ and the rank of $x$ in the set of all $z$ such that $f(z)=f(x)$ and so we know $x$)
And so $g(x)$ is injective.
Q.E.D.


Proof that $h(x)$ is injective
=================
Notice that if $f(x)\ne 0$ then $v_{a(f(x))}(h(x))=v_{a(f(x))}(f(x))-b(x)-\max(0,v_{a(f(x))}(f(x))-\min(0,m(f(x))))$ and so :

Since $b(x)\ge 1$ and $\max(0,v_{a(f(x))}-\min(0,m(f(x))))$ $\ge v_{a(f(x))}(f(x))-\min(0,m(f(x)))$ $\ge v_{a(f(x))}(f(x))$, then :
$v_{a(f(x))}(h(x))<0$ and $g(x)\notin\{0,1\}$

If $h(x)=0$, then $f(x)=0$ and $b(x)=1$ and so $x=r_0^{-1}(1)$ and so $x$ is unique.
If $h(x)=1$, then $f(x)=0$ and $b(x)=2$ and so $x=r_0^{-1}(2)$ and so $x$ is unique.
If $h(x)\notin\{0,1\}$, then $f(x)\ne 0$

Consider the definition of $h(x)$ when $f(x)\ne 0$ :
$\max(0,v_{a(f(x))}(f(x))-\min(0,m(f(x))))\ge v_{a(f(x))}(f(x))-\min(0,m(f(x)))$

So $v_{a(f(x))}(h(x))=v_{a(f(x))}(f(x))-b(x)-\max(0,v_{a(f(x))}(f(x))-\min(0,m(f(x))))$ $\le \min(0,m(f(x)))-b(x)$ $<\min(0,m(f(x)))\le m(f(x))$
And so $v_{a(f(x))}(h(x))< m(f(x))$
Let then any prime $q\ne p=a(f(x))$ : $v_q(h(x))=v_q(f(x))\ge m(f(x)) > v_{a(f(x))}(h(x))$

And so, looking at the prime decomposition of $h(x)$, it's immediate to find $a(f(x))$ : it's the unique prime $p$ whose power $k$ is the littlest.
Knowledge of $p=a(f(x))$ gives immediate knowledge of $f(x)$
and so we get :
$k=v_{a(f(x))}(h(x))=v_p(f(x))-b(x)-\max(0,v_p(f(x))-\min(0,m(f(x))))$

And so $b(x)=v_p(f(x))-\max(0,v_p(f(x))-\min(0,m(f(x))))-k$

So knowledge of $h(x)\notin\{0,1\}$ gives unique knowledge of $f(x)$ and $b(x)$

Choosing then $t$ as any element of $f^{-1}(\{f(x)\})$, we get $x=r_t^{-1}(b(x))$ uniquely determined.
(in fact we know $f(x)$ and the rank of $x$ in the set of all $z$ such that $f(z)=f(x)$ and so we know $x$)

And so $h(x)$ is injective
Q.E.D.
\end{solution}



\begin{solution}[by \href{https://artofproblemsolving.com/community/user/64716}{mavropnevma}]
	\begin{tcolorbox}1. Prove that for all functions $f(x): \mathbb{Q} \to\mathbb{Q}$, there exist injections $g,h: \mathbb{Q} \to \mathbb{Q}$ such that $f(x)=g(x)+h(x)$ for all $x \in \mathbb{Q}$.
\end{tcolorbox}


The solution is easy (and classical). Enumerate $\mathbb{Q}$ as $\{r_1,r_2,\ldots,r_n,\ldots\}$. Define $g(r_1), h(r_1)$ arbitrary such that $g(r_1)+h(r_1) = f(r_1)$. Assume $g(r_k), h(r_k)$ have been successfully built such that $g(r_k)+h(r_k) = f(r_k)$ for all $1\leq k \leq n$, and no value repeats in $G_n = \{g(r_k) \mid 1\leq k \leq n\}$ and  $H_n = \{h(r_k) \mid 1\leq k \leq n\}$. Take again $g(r_{n+1}), h(r_{n+1})$ arbitrary such that $g(r_{n+1})+h(r_{n+1}) = f(r_{n+1})$, and also $g(r_{n+1}) \not \in G_n$, $h(r_{n+1}) \not \in H_n$ (possible, since those sets are finite). This inductive process builds injective functions $g$, $h$, with the required property.
\end{solution}
*******************************************************************************
-------------------------------------------------------------------------------

\begin{problem}[Posted by \href{https://artofproblemsolving.com/community/user/112907}{BarneyStinson}]
	Find all functions $f: \mathbb R \to \mathbb R$ such that for $x \in \mathbb R \setminus \{0,1\}$:
\[f\left(\frac {1}{x} \right)+ f(1-x) = x.\]
	\flushright \href{https://artofproblemsolving.com/community/c6h411718}{(Link to AoPS)}
\end{problem}



\begin{solution}[by \href{https://artofproblemsolving.com/community/user/29428}{pco}]
	\begin{tcolorbox}Hi!

Find all functions  $f:R\rightarrow R$ such that for $x \in R 	\setminus${0,1}:

$f(\frac {1}{x} )+ f(1-x) = x $

Thanks.\end{tcolorbox}
Let $P(x)$ be the assertion $f(\frac 1x)+f(1-x)=x$

(a) : $P(\frac 1x)$ $\implies$ $f(x)+f(\frac {x-1}x)=\frac 1x$

(b) : $P(1-x)$ $\implies$ $f(\frac 1{1-x})+f(x)=1-x$

(c) : $P(\frac x{x-1})$ $\implies$ $f(\frac{x-1}x)+f(\frac 1{1-x})=\frac x{x-1}$

(a)+(b)-(c) : $\boxed{f(x)=\frac 1{2x}-\frac x2-\frac 1{2(x-1)}}$ $\forall x\notin\{0,1\}$ and $f(0),f(1)$ taking any value we want.
And it's easy to check back that this indeed is a solution.
\end{solution}



\begin{solution}[by \href{https://artofproblemsolving.com/community/user/112907}{BarneyStinson}]
	Thank you :)
\end{solution}
*******************************************************************************
-------------------------------------------------------------------------------

\begin{problem}[Posted by \href{https://artofproblemsolving.com/community/user/67223}{Amir Hossein}]
	Find all functions $f: \mathbb R \to \mathbb R$ such that
\[f(x+f(y))=f(y^2+3)+2x\cdot f(y)+f(x)-3, \quad \forall x,y \in \mathbb R.\]
	\flushright \href{https://artofproblemsolving.com/community/c6h412431}{(Link to AoPS)}
\end{problem}



\begin{solution}[by \href{https://artofproblemsolving.com/community/user/29428}{pco}]
	\begin{tcolorbox}Find all functions $f: \mathbb R \to \mathbb R$ such that
\[f(x+f(y))=f(y^2+3)+2x\cdot f(y)+f(x)-3, \qquad \forall x,y \in \mathbb R.\]\end{tcolorbox}
Let $P(x,y)$ be the assertion $f(x+f(y))=f(y^2+3)+2xf(y)+f(x)-3$
Let $f(0)=a$

$P(x,y)$ may be written $f(x+f(y))-f(x)=(f(y^2+3)-3)+2xf(y)$
So, since $f(x)=0$ $\forall x$ is not a solution, we get that any real $x$ may be written $x=f(u)-f(v)$ for some $u,v$

Let $g(x)=f(x)-x^2-a$.
$P(x,y)$ becomes $g(x+f(y))=g(x)+f(y^2+3)-f(y)^2-3$
$P(0,y)$ becomes $g(f(y))=f(y^2+3)-f(y)^2-3$
Subtracting, we get new assertion $Q(x,y)$ : $g(x+f(y))=g(x)+g(f(y))$

(a) : $Q(x-f(z),y)$ $\implies$ $g(x+f(y)-f(z))=g(x-f(z))+g(f(y))$
(b) : $Q(x-f(z),z)$ $\implies$ $g(x)=g(x-f(z))+g(f(z))$
(c) : $Q(f(y)-f(z),z)$ $\implies$ $g(f(y))=g(f(y)-f(z))+g(f(z))$
(a)-(b)+(c) : $g(x+f(y)-f(z))-g(x)=g(f(y)-f(z))$

And since any real may be written as $f(y)-f(z)$, we get $g(x+y)=g(x)+g(y)$

And so we get $f(x)=x^2+a+g(x)$ where $g(x)$ is some solution of additive Cauchy equation.

Plugging this in $P(0,x)$ : $f(f(x))=f(x^2+3)+a-3$, we get :

$a^2+g(x)^2+2ax^2+2x^2g(x)+2ag(x)+g(a)+g(g(x))-6x^2-6-g(3)-a=0$

Replacing in the above line $x\to px$ with $p\in\mathbb Q$ and remembering that $g(px)=pg(x)$, we get :
$a^2+p^2g(x)^2$ $+2ax^2p^2+2x^2g(x)p^3$ $+2ag(x)p+g(a)+g(g(x))p$ $-6x^2p^2-6-g(3)-a=0$

And this is a polynomial in $p$ which is zero for any $p\in\mathbb Q$ and so this is the null polynomial.
So coefficient of $p^3$ is zero and so $g(x)=0$ $\forall x$

So $f(x)=x^2+a$ and plugging this in original equation, we easily get $a=3$

Hence the unique solution $\boxed{f(x)=x^2+3}$
\end{solution}



\begin{solution}[by \href{https://artofproblemsolving.com/community/user/67223}{Amir Hossein}]
	Thank you very much! :)

Just one question: \begin{tcolorbox}So, since $f(x)=0 \forall x$ is not a solution, we get that any real $x$ may be written $x=f(u)-f(v)$ for some $u,v$.\end{tcolorbox}
why is this?
\end{solution}



\begin{solution}[by \href{https://artofproblemsolving.com/community/user/29428}{pco}]
	\begin{tcolorbox}Thank you very much!\end{tcolorbox}You're welcome :)
\begin{tcolorbox}Just one question: [quote="pco"]So, since $f(x)=0 \forall x$ is not a solution, we get that any real $x$ may be written $x=f(u)-f(v)$ for some $u,v$.\end{tcolorbox}
why is this?\end{tcolorbox}
We have $f(x+f(y))-f(x)=(f(y^2+3)-3)+2xf(y)$

Choose $y$ such that $f(y)\ne 0$ and then you can choose $x$ such that $RHS=(f(y^2+3)-3)+2xf(y)=z$
$RHS$ becomes $z$
$LHS$ is in the form $f(u)-f(v)$
\end{solution}



\begin{solution}[by \href{https://artofproblemsolving.com/community/user/67223}{Amir Hossein}]
	You're right, thanks. :)
\end{solution}



\begin{solution}[by \href{https://artofproblemsolving.com/community/user/15024}{Farenhajt}]
	If we put $y=0$, then $f(x+C)-f(x)=Ax+B$, where $C=f(0), A=2f(0), B=f(3)-3$

So by the known theoretic results, $f$ is a quadratic polynomial. Putting $f(x)=px^2+qx+r$ and solving for the coefficients, we get $f(x)=x^2+3$
\end{solution}



\begin{solution}[by \href{https://artofproblemsolving.com/community/user/31915}{Batominovski}]
	\begin{tcolorbox}If we put $y=0$, then $f(x+C)-f(x)=Ax+B$, ... So by the known theoretic results, $f$ is a quadratic polynomial.\end{tcolorbox}

I don't think so.  I can only conclude that $f(x)$ is a quadratic polynomial in $x$ for all $x$ which is an integer multiple of $C$.  I think you even need a lot more than continuity to conclude that $f$ is a polynomial.  Basically, on the interval $\big[0,|C|\big)$, you can define $f$ as you wish, and the rest follows from the restriction $f(x+C)-f(x)=Ax+B$.
\end{solution}



\begin{solution}[by \href{https://artofproblemsolving.com/community/user/29428}{pco}]
	\begin{tcolorbox}If we put $y=0$, then $f(x+C)-f(x)=Ax+B$, where $C=f(0), A=2f(0), B=f(3)-3$

So by the known theoretic results, $f$ is a quadratic polynomial. Putting $f(x)=px^2+qx+r$ and solving for the coefficients, we get $f(x)=x^2+3$\end{tcolorbox}
Very well known theoretic result, indeed :)

Example of this famous theorem : $A=B=0$ and so $f(x+C)-f(x)=0$ implies $f(x)$ is a quadratic.

So the only periodic functions in the world are quadratics.
Cheers.
\end{solution}



\begin{solution}[by \href{https://artofproblemsolving.com/community/user/15024}{Farenhajt}]
	Point taken by both of you.

So assume $f(0)=0\implies f(x)=f(3)+f(x)-3\implies f(3)=3$. Then $f(x+3)=f(12)+6x$ which can't be satisfied for $x=9$.

Thus $f(0)\neq 0$, hence $A\neq 0$ and we don't have the degenerate case $A=B=0$.
\end{solution}



\begin{solution}[by \href{https://artofproblemsolving.com/community/user/29428}{pco}]
	\begin{tcolorbox}Point taken by both of you.

So assume $f(0)=0\implies f(x)=f(3)+f(x)-3\implies f(3)=3$. Then $f(x+3)=f(12)+6x$ which can't be satisfied for $x=9$.

Thus $f(0)\neq 0$, hence $A\neq 0$ and we don't have the degenerate case $A=B=0$.\end{tcolorbox}
Dont insist, your well known theoretic result is wrong. Even in non degenerate case :

Choose for example $f(x)=x^2+3+\sin(2\pi x)$ which was not a quadratic in my old studies and is such that $f(x+1)-f(x)=2x+1$
\end{solution}
*******************************************************************************
-------------------------------------------------------------------------------

\begin{problem}[Posted by \href{https://artofproblemsolving.com/community/user/110552}{youarebad}]
	Given two function $f, g : \mathbb{R} \to\mathbb{R}$, such that $f(x+g(y))=3x+y+12$ for all $x, y \in \mathbb R$. Find the value of $g(2004+f(2004))$.
	\flushright \href{https://artofproblemsolving.com/community/c6h412703}{(Link to AoPS)}
\end{problem}



\begin{solution}[by \href{https://artofproblemsolving.com/community/user/29428}{pco}]
	\begin{tcolorbox}Given two function $f, g : \mathbb{R} \rightarrow \mathbb{R}$, such that $f(x+g(y))=3x+y+12$ for all $x, y \in R$. Find the value of $g(2004+f(2004))$\end{tcolorbox}
Let $P(x,y)$ be the assertion $f(x+g(y))=3x+y+12$

$P(x-g(0),0)$ $\implies$ $f(x)=3x-3g(0)+12$
$P(-g(x),x)$ $\implies$ $f(0)=-3g(x)+x+12$

So $f(x)=3x+a$ and $g(x)=\frac x3+b$ with $a+3b=12$ which indeed are solutions

Then $g(x+f(x))=g(4x+a)=\frac{4x}3+\frac {a+3b}3$ $=\frac{4x}3+4$

And so $\boxed{g(2004+f(2004))=2676}$
\end{solution}



\begin{solution}[by \href{https://artofproblemsolving.com/community/user/110552}{youarebad}]
	\begin{tcolorbox}So $f(x)=3x+a$ and $g(x)=\frac x3+b$ with $a+3b=12$ which indeed are solutions\end{tcolorbox}

How to get it ??
\end{solution}



\begin{solution}[by \href{https://artofproblemsolving.com/community/user/29428}{pco}]
	\begin{tcolorbox}[quote="pco"]So $f(x)=3x+a$ and $g(x)=\frac x3+b$ with $a+3b=12$ which indeed are solutions\end{tcolorbox}

How to get it ??\end{tcolorbox}
Huhh ?
Did you read my post ?

Two lines above, I have $f(x)=3x-3g(0)+12$. So just call $12-3g(0)=a$ and you get $f(x)=3x+a$

One line above, I have $f(0)=-3g(x)+x+12$ and so $g(x)=\frac x3+4-\frac {f(0)}3$. So just call $4-\frac {f(0)}3=b$ and you get $g(x)=\frac x3+b$

And since $f(0)=a$ and $g(0)=b$, both relations $12-3g(0)=a$ and $4-\frac {f(0)}3=b$ are $a+3b=12$
\end{solution}



\begin{solution}[by \href{https://artofproblemsolving.com/community/user/98017}{Redox}]
	great solution
\end{solution}
*******************************************************************************
-------------------------------------------------------------------------------

\begin{problem}[Posted by \href{https://artofproblemsolving.com/community/user/88788}{phalkun}]
	Find all functions $f: \mathbb R \to \mathbb R$ such that \[(f(x) \cdot f(y))^2=f(x+y)\cdot f(x-y)\] for all reals $x$ and $y$.
	\flushright \href{https://artofproblemsolving.com/community/c6h412704}{(Link to AoPS)}
\end{problem}



\begin{solution}[by \href{https://artofproblemsolving.com/community/user/29428}{pco}]
	\begin{tcolorbox}Find all functions $f$ such that $[f(x).f(y)]^2=f(x+y).f(x-y)$ ( $x ,y$ Reals )\end{tcolorbox}
Are you sure that the problem statement you got in your contest does not contain some supplementary constraint (continuity, for example) ?

As is, we have at least infinitely many solutions :
$f(x)=0$ $\forall x$
$f(x)=e^{ah(x)^2}$ where $h(x)$ is any solution of Cauchy equation
$f(x)=-e^{ah(x)^2}$ where $h(x)$ is any solution of Cauchy equation
And also any product of such solutions
\end{solution}



\begin{solution}[by \href{https://artofproblemsolving.com/community/user/31915}{Batominovski}]
	\begin{tcolorbox}
Are you sure that the problem statement you got in your contest does not contain some supplementary constraint (continuity, for example) ?
\end{tcolorbox}

Is the problem nicely solvable if continuity is assumed?
\end{solution}



\begin{solution}[by \href{https://artofproblemsolving.com/community/user/29428}{pco}]
	\begin{tcolorbox} Is the problem nicely solvable if continuity is assumed?\end{tcolorbox}
Nicely, I dont know :) . But it is solvable :

Let $P(x,y)$ be the assertion $f(x)^2f(y)^2=f(x+y)f(x-y)$

$f(x)=0$ $\forall x$ is a solution and let us from now look for non all-zero solutions.
Let $u$ such that $f(u)\ne 0$

$P(u,0)$ $\implies$ $f(u)^2f(0)^2=f(u)^2$ and so $f(0)=\pm 1$
$f(x)$ solution implies $-f(x)$ solution and so wlog say $f(0)=+1$

If $f(t)=0$ for some $t\ne 0$, then $P(\frac t2,\frac t2)$ $\implies$ $f(\frac t2)^4=f(t)$ and so $f(\frac t2)=0$ and so $f(\frac t{2^n})=0$ $\forall n\in\mathbb N$
So continuity would imply $f(0)=0$, impossible.

So $f(x)>0$ $\forall x$ and we can write $f(x)=e^{g(x)}$ for some continuous function $g(x)$ such that :
$g(0)=0$
New assertion $Q(x,y)$ : $2g(x)+2g(y)=g(x+y)+g(x-y)$ $\forall x,y$

Let $x\in\mathbb R$ and the sequence $a_n=g(nx)$ with $a_0=0$
$Q((n+1)x,x)$ $\implies$ $a_{n+2}=2a_{n+1}-a_n+2a_1$ whose solution is $a_n=a_1n^2$

So $g(nx)=n^2g(x)$ $\forall x,\forall n\in\mathbb N$
It's immediate to show that this is still true for $n\in\mathbb Z$

$g(p)=p^2g(1)$ $\forall p\in\mathbb Z$ and so $p^2g(1)=g(q\frac pq)=q^2g(\frac pq)$

So $g(x)=x^2g(1)$ $\forall x\in\mathbb Q$ and continuity again gives $g(x)=ax^2$ $\forall x\in\mathbb R$

\begin{bolded}Hence the continuous solutions of the equation \end{bolded}\end{underlined} (it's easy to check back that they indeed are solutions) :
$f(x)=0$ $\forall x$
$f(x)=e^{ax^2}$ $\forall x\in\mathbb R$ and for any real $a$ 
$f(x)=-e^{ax^2}$ $\forall x\in\mathbb R$ and for any real $a$
\end{solution}
*******************************************************************************
-------------------------------------------------------------------------------

\begin{problem}[Posted by \href{https://artofproblemsolving.com/community/user/30710}{huyhoang}]
	Find all functions $f: \mathbb{R}\to [0;+\infty)$ such that: 
\[f(x^2+y^2)=f(x^2-y^2)+f(2xy)\]
for all real numbers $x$ and $y$.

\begin{italicized}Laurentiu Panaitopol\end{italicized}
	\flushright \href{https://artofproblemsolving.com/community/c6h412717}{(Link to AoPS)}
\end{problem}



\begin{solution}[by \href{https://artofproblemsolving.com/community/user/29428}{pco}]
	\begin{tcolorbox}
Find all functions $f: \mathbb{R}\to [0;+\infty)$ such that: 
\[f(x^2+y^2)=f(x^2-y^2)+f(2xy)\]
for all real numbers $x$ and $y$.\end{tcolorbox}
Let $P(x,y)$ be the assertion $f(x^2+y^2)=f(x^2-y^2)+f(2xy)$

$P(0,0)$ $\implies$ $f(0)=0$
$P(0,x)$ $\implies$ $f(x^2)=f(-x^2)$ and so $f(x)$ is even.

Let $x\ge y\ge z\ge 0$

(a) : $P(\sqrt{\frac {x+y}2},\sqrt{\frac{x-y}2})$ $\implies$ $f(x)=f(y)+f(\sqrt{x^2-y^2})$

(b) : $P(\sqrt{\frac {y+z}2},\sqrt{\frac{y-z}2})$ $\implies$ $f(y)=f(z)+f(\sqrt{y^2-z^2})$

(c) : $P(\sqrt{\frac {x+z}2},\sqrt{\frac{x-z}2})$ $\implies$ $f(x)=f(z)+f(\sqrt{x^2-z^2})$

(a)+(b)-(c) : $f(\sqrt{x^2-z^2})=f(\sqrt{x^2-y^2})+f(\sqrt{y^2-z^2})$

Writing $f(x)=g(x^2)$, this becomes $g(x+y)=g(x)+g(y)$ $\forall x,y\ge 0$
And since $g(x)\ge 0$, we get $g(x)=ax$ and so $f(x)=ax^2$ $\forall x\ge 0$ and for some $a\ge 0$

And since $f(x)$ is even, we get $\boxed{f(x)=ax^2}$ $\forall x$ and for any real $a\ge 0$ which indeed is a solution.
\end{solution}



\begin{solution}[by \href{https://artofproblemsolving.com/community/user/30710}{huyhoang}]
	nice solution, dear pco :D
\end{solution}



\begin{solution}[by \href{https://artofproblemsolving.com/community/user/92753}{WakeUp}]
	Also posted here: 

http://www.artofproblemsolving.com/Forum/viewtopic.php?f=36&t=425331
http://www.artofproblemsolving.com/Forum/viewtopic.php?f=38&t=51585
\end{solution}
*******************************************************************************
-------------------------------------------------------------------------------

\begin{problem}[Posted by \href{https://artofproblemsolving.com/community/user/86849}{abch42}]
	Find all functions $f:\mathbb{R}\rightarrow\mathbb{R}$ such that
\[f\left(x+f(y)\right)=f(x+xy)+yf(1-x)\]
for all real numbers $x$ and $y$.
	\flushright \href{https://artofproblemsolving.com/community/c6h412726}{(Link to AoPS)}
\end{problem}



\begin{solution}[by \href{https://artofproblemsolving.com/community/user/44083}{jgnr}]
	Substitute $y=0$, we get $f(x+f(0))=f(x)$. If $f(0)\ne0$, then $f$ has a period of $f(0)$. Substitute $x=1$, we get $f(1+f(y))=f(1+y)+yf(0)$. Take $y\rightarrow y+f(0)$, we get $yf(0)=(y+f(0))f(0)$, which gives $f(0)=0$, a contradiction. So $f(0)=0$.

(i) $f(1)\ne0$

Substitute $x=0$, we get $f(f(y))=yf(1)$. So $f$ is bijective. Substitute $x=1$, we get $f(1+f(y))=f(1+y)$, so $1+f(y)=1+y$ and hence $f(y)=y$, which is indeed a solution.

(ii) $f(1)=0$

Substitute $x=0$, we get $f(f(y))=0$. Substitute $x=1,y=f(a)$, we get $0=f(1+f(a))$. Substitute $x=1,y=a$, we get $0=f(1+a)$, so $f$ is constant zero, which is another solution.
\end{solution}



\begin{solution}[by \href{https://artofproblemsolving.com/community/user/29428}{pco}]
	\begin{tcolorbox}Find all functions $f:\mathbb{R}\rightarrow\mathbb{R}$ such that
$f\left(x+f(y)\right)=f(x+xy)+yf(1-x)$
for all real numbers $x$ and $y$.\end{tcolorbox}
Let $P(x,y)$ be the assertion $f(x+f(y))=f(x+xy)+yf(1-x)$

1) If $f(1)\ne 0$
==========
$P(0,x)$ $\implies$ $f(f(x))=f(0)+xf(1)$ and so $f(x)$ is injective.
$P(0,0)$ $\implies$ $f(f(0))=f(0)$ and so $f(0)=0$ (since injective)

Let then $x\ne 0$ : $P(\frac{f(x)}x,x)$ $\implies$ $f(1-\frac{f(x)}x)=0$ and so $1-\frac{f(x)}x=0$ (since injective)
So $f(x)=x$ $\forall x$ which indeed is a solution.

2) If $f(1)=0$
===========
$P(0,0)$ $\implies$ $f(f(0))=f(0)$
$P(1,f(0))$ $\implies$ $f(0)^2=0$ and so $f(0)=0$
$P(0,x)$ $\implies$ $f(f(x))=0$

$P(1,f(x-1))$ $\implies$ $f(f(x-1)+1)=0$
$P(1,x-1)$ $\implies$ $f(f(x-1)+1)=f(x)$

And so $f(x)=0$ $\forall x$ which indeed is a solution.

\begin{bolded}Hence the solutions \end{bolded}\end{underlined}:
$f(x)=x$ $\forall x$
$f(x)=0$ $\forall x$
\end{solution}
*******************************************************************************
-------------------------------------------------------------------------------

\begin{problem}[Posted by \href{https://artofproblemsolving.com/community/user/74510}{filipbitola}]
	Find all functions $f:\mathbb{Q}^{+}\to\mathbb{Q}^{+}$ such that for all $x,y \in \mathbb{Q}^+$
\[ f(f^{2}(x)y)=x^{3}f(xy).\]
Here, $ f^{2}(x) $ means $ f(x) \cdot f(x) $.
	\flushright \href{https://artofproblemsolving.com/community/c6h412941}{(Link to AoPS)}
\end{problem}



\begin{solution}[by \href{https://artofproblemsolving.com/community/user/29428}{pco}]
	\begin{tcolorbox}Find all functions $f:\mathbb{Q}^{+}\to\mathbb{Q}^{+}$ such that for all $x,y$ in $\mathbb{Q}$
$ f(f^{2}(x)y)=x^{3}f(xy) $
Here $ f^{2}(x) $ means $ f(x)*f(x) $\end{tcolorbox}
Let $P(x,y)$ be the assertion $f(f^2(x)y)=x^3f(xy)$

$P(x,1)$ $\implies$ $f(f^2(x))=x^3f(x)$ and so $f(x)$ is injective.

$P(x,f^2(y))$ $\implies$ $f(f^2(x)f^2(y))=x^3f(xf^2(y))$
$P(y,x)$ $\implies$ $f(f^2(y)x)=y^3f(xy)$
$P(xy,1)$ $\implies$ $x^3y^3f(xy)=f(f^2(xy))$

Multiplying these lines (and since no factor may be zero), we get $f(f^2(x)f^2(y))=f(f^2(xy))$ and so, since injective and positive :
$f(xy)=f(x)f(y)$

$P(x,y)$ becomes then $(f(f(x)))^2=x^3f(x)$ and $f(xy)=f(x)f(y)$

Setting $g_1(x)=xf(x)$, this is equivalent to $(g_1(g_1(x))^2=g_1^5(x)$ and $g_1(xy)=g_1(x)g_1(y)$

From there we get that $g_1(x)$ must always be the square of a rational and so it exists a function $g_2(x)$ from $\mathbb Q^+\to\mathbb Q^+$ such that :
$g_1(x)=g_2(x)^2$ and so :
$(g_2(g_2(x))^4=g_2^{5}(x)$ and $g_2(xy)=g_2(x)g_2(y)$

And this may be repeated infinitely, building a sequence of multiplicative functions $g_n(x)$ such that :
$g_{n-1}(x)=g_{n}^2(x)$ and $(g_n(g_n(x)))^{2n}=g_n^5(x)$

And so the only possibility is $g_n(x)=1$ $\forall x$ and $g(x)=1$ and so $\boxed{f(x)=\frac 1x}$ which indeed is a solution.
\end{solution}



\begin{solution}[by \href{https://artofproblemsolving.com/community/user/74510}{filipbitola}]
	\begin{tcolorbox}Multiplying these lines (and since no factor may be zero), we get $ f(f^{2}(x)f^{2}(y))=f(f^{2}(xy)) $
 and so, since injective and positive: $f(xy)=f(x)f(y)$
\end{tcolorbox}
I don't quite understand this. Can you please explain in detail?
Thanks
\end{solution}



\begin{solution}[by \href{https://artofproblemsolving.com/community/user/29428}{pco}]
	\begin{tcolorbox}[quote]Multiplying these lines (and since no factor may be zero), we get $ f(f^{2}(x)f^{2}(y))=f(f^{2}(xy)) $
 and so, since injective and positive: $f(xy)=f(x)f(y)$
\end{tcolorbox}
I don't quite understand this. Can you please explain in detail?
Thanks\end{tcolorbox}
If you agree with $f(x)$ injective and $ f(f^{2}(x)f^{2}(y))=f(f^{2}(xy)) $ then, since $f(u)=f(v)$ implies $u=v$, we get $ f^{2}(x)f^{2}(y)=f^{2}(xy) $

And since $f(x)>0$ $\forall x$, we can just take square root and we get $f(x)f(y)=f(xy)$
Q.E.D.
\end{solution}



\begin{solution}[by \href{https://artofproblemsolving.com/community/user/74510}{filipbitola}]
	Thanks a lot :D
\end{solution}



\begin{solution}[by \href{https://artofproblemsolving.com/community/user/54529}{Martin N.}]
	\begin{tcolorbox}And this may be repeated infinitely, building a sequence of multiplicative functions $g_n(x)$ such that :
$g_{n-1}(x)=g_{n}^2(x)$ and $(g_n(g_n(x)))^{2n}=g_n^5(x)$

And so the only possibility is $g_n(x)=1$ $\forall x$ and $g(x)=1$ and so $\boxed{f(x)=\frac 1x}$ which indeed is a solution.\end{tcolorbox}
I do not really understand this conclusion...and I think that there must be a mistake in there because $f(x)=x^{\frac{3}{2}}$ is a solution, too.
\end{solution}



\begin{solution}[by \href{https://artofproblemsolving.com/community/user/64716}{mavropnevma}]
	There was another thread mentioning that when $f$ was defined and taking values on positive reals, rather than rationals, there were other solutions, including the one mentioned by you - but check the domain and codomain of $f$.
\end{solution}



\begin{solution}[by \href{https://artofproblemsolving.com/community/user/54529}{Martin N.}]
	Ah ok, I did not think about that...thank you :)
\end{solution}



\begin{solution}[by \href{https://artofproblemsolving.com/community/user/72819}{Dijkschneier}]
	Take y=1 to see that $f(f(x)^2)=x^3f(x)$ and hence injective, and $y=f(z)^2$ to see that $f(f(x)^2f(z)^2)=(xz)^3f(xz)=f(f(xz)^2)$ and using the fact that f is injective and taking the square root we get that f is multiplicative.
Therefore, it is sufficient to define f for prime numbers p.
From $f(f(x)^2)=x^3f(x)$, we get $f(f(x))^2 = x^3 f(x)$ and by induction : $f^n(x)=f(x)^{u_n}x^{v_n}$ where $u_n$ and $v_n$ are 2 sequences defined as follow : 
$u_n = -\frac{2}{5}(-1)^n + \frac{9}{10}(\frac{3}{2})^n$
$v_n = \frac{3}{5}(-1)^n + \frac{9}{10}(\frac{3}{2})^n$
As $f^n(p)=f(p)^{u_n}p^{v_n} \in \mathbb{Q}$ for prime p, we must have all the q-adic valuations to be integers.
Since for two fixed primes p and q ($p \neq q$), $v_q(f^n(p)) = v_q(f(p)^{u_n})=u_n v_q(f(p))$ can't be an integer for all n unless $v_q(f(p))=0$ (because as n grows, the denominator of $u_n$ is a great power of 2), then $f(p)$ has just (possibly) p in its prime factorization. Let k be its p-adic valuation (k may be positive or negative). We need to have $ku_n + v_n \in \mathbb{Z}$ for all n, that is, $-\frac{2}{5}(-1)^n(k-\frac{3}{2}) + \frac{9}{10}(\frac{3}{2})^n(k+1) \in \mathbb{Z}$, and after multiplication by 10, we see that these numbers belong to Z if and only if k=-1, which gives $f(p)=\frac{1}{p}$ for prime numbers p, and hence $f(x)=\frac{1}{x}$ for every positive rational number.
\end{solution}



\begin{solution}[by \href{https://artofproblemsolving.com/community/user/87206}{safa698}]
	$ x^3y^3f(xy)=f(f(x)^2f(y)^2) $

$ \implies $

$ x^3f(x)=f(f(x\/y)^2f(y)^2)=f(f(x)^2) $

$ \implies $

$ f(x\/y)f(y)=f(x) $
$ x=y  \implies  f(x)f(y)=f(xy) $
How can i continue after then?
\end{solution}



\begin{solution}[by \href{https://artofproblemsolving.com/community/user/53945}{antimonyarsenide}]
	After $f(x)f(y)=f(xy)$, $f(f(x))^2=x^3 f(x)$ implies $f(xf(x))^2=(xf(x))^3$. Let $xf(x)=n$. If $n\ne 1$, let $n=n_0^{2^k}$ where $n_0$ is a rational that's not the square of any rational. Then $f(n)^2=n^3 \implies f(n_0)=n_0^{3\/2}$, an irrational, contradiction, so $n=1$. That is, $f(x)=\frac{1}{x}$ for all $x$.
\end{solution}



\begin{solution}[by \href{https://artofproblemsolving.com/community/user/148207}{Particle}]
	[hide="Solution"]It is very easy to prove $f$ is multiplicative and $f(f(x)^2)=f(f(x))^2=x^3f(x)$. The challenge is to continue after that. So I'll post only that part.

Call a rational $n$-\begin{italicized}th power\end{italicized} if it is of the form $\left (\frac p q \right )^n$ with $p,q\in N$. Note that $\left [\frac 1 x f(f(x))\right ]^2=xf(x)\implies xf(x)$ is a square. So $f(x)f(f(x))$ is also a square. Let $n=1$. Now note that \[[f(x)f(f(x))]^2=(xf(x))^3\quad (1)\]
$xf(x)$ is a $2^n$-th power. So RHS is a $3\cdot2^n$-th power. But LHS is a $2^{n+1}$-th power. So to be equality in (1), both sides must be $3\cdot 2^{n+1}$-th power. But it implies RHS is a $\frac {3\cdot 2^{n+1}}3=2^{n+1}$-th power. Continuing this argument, we'll get $xf(x)$ is a $2^i$-th power for every $i\in N$. This is possible only if $xf(x)=1\implies f(x)=\frac 1 x$.[\/hide]
\end{solution}
*******************************************************************************
-------------------------------------------------------------------------------

\begin{problem}[Posted by \href{https://artofproblemsolving.com/community/user/70520}{hvaz}]
	Find all functions $f: \mathbb{R} \to \mathbb{R} $ such that $f(x + f(y)) = 2f(xf(y))$ for all reals $x$ and $y$.
	\flushright \href{https://artofproblemsolving.com/community/c6h413312}{(Link to AoPS)}
\end{problem}



\begin{solution}[by \href{https://artofproblemsolving.com/community/user/29428}{pco}]
	\begin{tcolorbox}Find all functions $f: \mathbb{R} \to \mathbb{R} $ such that $f(x + f(y)) = 2f(xf(y))$\end{tcolorbox}
Let $P(x,y)$ be the assertion $f(x+f(y))=2f(xf(y))$

$f(x)=1$ $\forall x$ is not a solution and so $\exists u$ such that $f(u)\ne 1$

$P(\frac{f(u)}{f(u)-1},u)$ $\implies$ $f(v)=0$ with $v=\frac{f(u)^2}{f(u)-1}$

$P(0,v)$ $\implies$ $f(0)=0$ and then $P(x,v)$ $\implies$ $\boxed{f(x)=0}$ $\forall x$ which indeed is a solution.
\end{solution}
*******************************************************************************
-------------------------------------------------------------------------------

\begin{problem}[Posted by \href{https://artofproblemsolving.com/community/user/68025}{Pirkuliyev Rovsen}]
	Find all continuous functions $f: \mathbb{R}\to\mathbb{R}$ such that \[f(x+y)=f(x)+f(y)+xy(x+y)(x^2+xy+y^2)\] for all $x,y \in \mathbb R$.
	\flushright \href{https://artofproblemsolving.com/community/c6h413826}{(Link to AoPS)}
\end{problem}



\begin{solution}[by \href{https://artofproblemsolving.com/community/user/29428}{pco}]
	\begin{tcolorbox}Find all continuous function $f: \mathbb{R}\to\mathbb{R}$ such that $f(x+y)=f(x)+f(y)+xy(x+y)(x^2+xy+y^2)$.\end{tcolorbox}
Let $g(x)=f(x)-\frac{x^5}5$ and the equation becomes $g(x+y)=g(x)+g(y)$ and so $g(x)=ax$ since continuous

Hence the solutions : $\boxed{f(x)=\frac{x^5}5+ax}$ $\forall x$ and for any real $a$
\end{solution}
*******************************************************************************
-------------------------------------------------------------------------------

\begin{problem}[Posted by \href{https://artofproblemsolving.com/community/user/67223}{Amir Hossein}]
	Find all functions $f :\mathbb R \to \mathbb R$ such that for all real $x, y$
\[f(f(x)^2 + f(y)) = xf(x) + y.\]
	\flushright \href{https://artofproblemsolving.com/community/c6h414395}{(Link to AoPS)}
\end{problem}



\begin{solution}[by \href{https://artofproblemsolving.com/community/user/29428}{pco}]
	\begin{tcolorbox}Find all functions $f :\mathbb R \to \mathbb R$ such that for all real $x, y$
\[f(f(x)^2 + f(y)) = xf(x) + y.\]\end{tcolorbox}
Let $P(x,y)$ be the assertion $f(f(x)^2+f(y))=xf(x)+y$
Let $f(0)=a$

$P(0,0)$ $\implies$ $f(a^2+a)=0$ and then  $P(a^2+a,x)$ $\implies$ $f(f(x))=x$ and $f(x)$ is bijective and involutive.

Then $P(f(1),a)$ $\implies$ $f(1)=f(1)+a$ and so $a=0$

$P(f(x),f(y))$ $\implies$ $f(x^2+y)=xf(x)+f(y)$
$P(f(x),0)$ $\implies$ $f(x^2)=xf(x)$
Subtracting, we get $f(x^2+y)=f(x^2)+f(y)$

So $f(x+y)=f(x)+f(y)$ $\forall x\ge 0,\forall y$ and it's immediate to conclude $f(x+y)=f(x)+f(y)$ $\forall x,y$.

$P(f(x),0)$ $\implies$ $f(x^2)=xf(x)$
$P(f(x+1),0)$ $\implies$ $f(x^2+2x+1)=(x+1)f(x+1)$

Subtracting, we get $2f(x)+f(1)=xf(1)+f(x)+f(1)$ and so $f(x)=xf(1)$ $\forall x$

Plugging back in original equation, we get two solutions :
$f(x)=x$ $\forall x$
$f(x)=-x$ $\forall x$
\end{solution}



\begin{solution}[by \href{https://artofproblemsolving.com/community/user/208472}{strujabog}]
	Let there be $y_1$ different from $y$,such that $f(y_1)=f(y)$.Then we have: $y+xf(x)=f(f(x)^2+f(y))=f(f(x)^2+f(y_1))=xf(x)+y_1$ which yields $y=y_1$.Clearly the function is an injection.Also let $P(x,y)$ be assertion of $f(f(x)^2+f(y))=f(x)x+y$.
$P(f(x),y)$-->$f(x^2+f(y))=xf(x)+y=f(f(x)^2+f(y))$-->$x^2+f(y)=f(x)^2+f(y)$-->$f(x)^2=x^2$-->$f(x)=x,-x$.
Checking both solutions we see both of them work,so only solutions are $f(x)=x,f(x)=-x$
\end{solution}



\begin{solution}[by \href{https://artofproblemsolving.com/community/user/29428}{pco}]
	\begin{tcolorbox}...$f(x)^2=x^2$-->$f(x)=x,-x$.
Checking both solutions we see both of them work,so only solutions are $f(x)=x,f(x)=-x$\end{tcolorbox}
Classical error : 
$f(x)^2=x^2$ implies "$\forall x$, either $f(x)=x$, either $f(x)=-x$"
and not "either $f(x)=x$ $\forall x$, either $f(x)=-x$ $\forall x$"
\end{solution}



\begin{solution}[by \href{https://artofproblemsolving.com/community/user/208472}{strujabog}]
	oh thank you pco
ill try to fix it
\end{solution}



\begin{solution}[by \href{https://artofproblemsolving.com/community/user/260346}{Takeya.O}]
	Let $P(x,y)$ be $f(f(x)^2+f(y))=xf(x)+y$.

Obviously $f$ is \begin{bolded}bijective\end{bolded}.Then ∃$a$ s.t. $f(a)=0$.$P(a,y)\rightarrow f(f(y))=y$.$P(f(x),y)\rightarrow f(x^2+f(y))=xf(x)+y=f(f(x)^2+f(y))$.Since $f$ is injective,$x^2+f(y)=f(x)^2+f(y)$.Thus $f(x)=\pm{x}$.

We suppose that ∃$b\neq 0$,∃$c\neq 0$ s.t. $f(b)=b,f(c)=-c$.$P(b,c)\rightarrow f(b^2-c)=b^2+c$.Then $\pm{b^2-c}=b^2+c$.If $b^2-c=b^2+c\rightarrow c=0$ which is absurd.If $-(b^2-c)=b^2+c\rightarrow b=0$ which is absurd.Therefore
$\boxed{\forall x\in \mathbb R:f(x)=x, \forall x\in \mathbb R:f(x)=-x}$ which satisfy the condition.$\blacksquare$ :coolspeak:
\end{solution}



\begin{solution}[by \href{https://artofproblemsolving.com/community/user/288210}{tenplusten}]
	$P(0,x)$ gives us $f$ is bijective. So there is real number $a$ such that $f(a)=0$.$P(a,x)$ $\Longrightarrow$ $f(f(x))=x$ 
$P(0,x)$ $\Longrightarrow$ $f((f(0))^2+f(x))=x=f(f(x))$ from injectivity $f(0)=0$ and from $P(x,0)$and $P(f(x),0$ we get $(f(x))^2=x^2$ Lets assume that there is $a,b$ such that $f(a)=a$ $f(b)=-b$ ($a,b\neq 0$) $P(a,b)$ $\Longrightarrow$ $b-a^2=a^2+b$ contradict to $a\neq 0$ So $f(x)=x$ $f(x)=-x$
\end{solution}



\begin{solution}[by \href{https://artofproblemsolving.com/community/user/305092}{fighter}]
	how nice problem here is my solution;

assume f(x) = f(y).

or, f(f(x)^2 + f(y)) = f(f(x)^2 + f(x))

or, x*f(x) + x = x*f(x) + y

or, x = y this means the function is injective;

assume that, f(t) = 0;

putting x = y = t,

f(0) = t;

also, putting x = y = 0,

f(t^2 + t) = 0 = f(t);

from injectivity,

t = 0;

so, f(0) = 0;

putting x = 0 in main equation,

f(f(y)) = y;

setting x = f(x) and y = f(y),

f(x^2 + y) = x*f(x) + y     (1)

putting y = 0 at (1),

f(x^2) = x*f(x);

also, putting y = 0 at main equation,

f(f(x)^2) = x*f(x);

so, f(f(x)^2) = f(x^2)

from injectivity, f(x)^2 = x^2;

or, f(x) = x or f(x) = -x;

if f(1) = 1 we get by checking values f(x) = x for all real x;

if f(1) = -1 we get by checking values f(x) = -x for all real x;

by easy checking , these two are solutions
\end{solution}



\begin{solution}[by \href{https://artofproblemsolving.com/community/user/305092}{fighter}]
	takeya.O what a nice solution!
\end{solution}



\begin{solution}[by \href{https://artofproblemsolving.com/community/user/260346}{Takeya.O}]
	@fighter
Thanks,my brother! :coool:
\end{solution}



\begin{solution}[by \href{https://artofproblemsolving.com/community/user/260346}{Takeya.O}]
	\begin{tcolorbox}assume that, f(t) = 0\end{tcolorbox}

Brother!
I think that you should show that $f=0$ has root(zero point). 

\end{solution}



\begin{solution}[by \href{https://artofproblemsolving.com/community/user/305092}{fighter}]
	is my solution right I'm in little confution
\end{solution}



\begin{solution}[by \href{https://artofproblemsolving.com/community/user/260346}{Takeya.O}]
	Brother! :D

You should show that $f$ is surjective for assuming that $f(t)=0$. :coool:
\end{solution}
*******************************************************************************
-------------------------------------------------------------------------------

\begin{problem}[Posted by \href{https://artofproblemsolving.com/community/user/36998}{sandu2508}]
	Find all function $f:\mathbb{R}\to\mathbb{R}$ that satisfy the relation
\[(x-2)f(y)+f(y+2f(x))=f(x+yf(x))\]
for all real numbers $x$ and $y$.
	\flushright \href{https://artofproblemsolving.com/community/c6h414538}{(Link to AoPS)}
\end{problem}



\begin{solution}[by \href{https://artofproblemsolving.com/community/user/114445}{jejungchv}]
	If $f(0)=0$ ,$P(0,y)$ $\implies$ $f(x)$ =0 is a solution
And other wise, $P(x,0)$ $\implies$  $(x-2)f(0)+f(2f(x))=f(x)$
-Thus  $f$  is a bijective and $f(2)=1$,$P(x,\frac{x-2f(x)}{1-f(x)})$ $\implies$ $(x-2)f(\frac{x-2f(x)}{1-f(x)})=$ 0,
Then we have $(f(\frac{x-2f(x)}{1-f(x)})$=0\begin{bolded}  (1)\end{bolded}
-$P(3,\frac{3}{f(3)-1})$  $\implies$  $f(\frac{3}{f(3)-1}+2f(3))=0$ \begin{bolded} (2)\end{bolded}
From \begin{bolded}(1)\end{bolded} and\begin{bolded} (2)\end{bolded},and $f$  is injective,we have  $\frac{x-2f(x)}{1-f(x)}=\frac{3}{f(3)-1}+2f(3)$
We have $f(x)=x-1$
\end{solution}



\begin{solution}[by \href{https://artofproblemsolving.com/community/user/97349}{FBI__}]
	I think $f(x)=x-1$ is not true.

Setting $x:=y:=0$ we have $f(0)=0$
Setting $x:=0$ we have $-2f(y)+f(y)=f(0)=0$
So $f(x)=0$ $ \forall x \in \mathbb{R}$
\end{solution}



\begin{solution}[by \href{https://artofproblemsolving.com/community/user/29428}{pco}]
	\begin{tcolorbox} Setting $x:=y:=0$ we have $f(0)=0$\end{tcolorbox}
Setting $x=y=0$, I find $f(2f(0))=3f(0)$.
How did you get $f(0)=0$ from there ?
\end{solution}



\begin{solution}[by \href{https://artofproblemsolving.com/community/user/10045}{socrates}]
	\begin{tcolorbox}Find all function $f:\mathbb{R}\to\mathbb{R}$ that satisfy the relation

$(x-2)f(y)+f(y+2f(x))=f(x+yf(x))$

for all real numbers $x, y$\end{tcolorbox}
\end{solution}



\begin{solution}[by \href{https://artofproblemsolving.com/community/user/67223}{Amir Hossein}]
	This is still unsolved. BUMP.
\end{solution}



\begin{solution}[by \href{https://artofproblemsolving.com/community/user/402732}{Arkmmq}]
	\begin{tcolorbox}This is still unsolved. BUMP.\end{tcolorbox}

If $f(0)=0$ then set$ x=0$ to find that $f(y)=0$ which is a solution.
Now suppose that $f(0)\neq 0$ .
Set $y=0 \rightarrow f $is injective.
Set $y=\frac{2f(x)-x}{f(x)-1} $
$\rightarrow (x-2)f(\frac{2f(x)-x}{f(x)-1}=0$
and since $f$ is injective then $\frac{2f(x)-x}{f(x)-1}=constant=c\neq 2$.
So $f(x)=\frac{x-c}{2-c}$ substitute to find that $c=1$ and $f(x)=x-1$.
\end{solution}



\begin{solution}[by \href{https://artofproblemsolving.com/community/user/29428}{pco}]
	This is a nearly perfect proof.

Some precisions in order to have more rigor :
You should indicate that your conclusion $(2-c)f(x)=x-c$ is only valid is $x\ne 2$ and if $f(x)\ne 1$
Injectivity implies then $f(x)=\frac{x-c}{2-c}$ for all real except maybe two of them ($2$ and maybe at most another real $u$ such that $f(u)=1$)

You get then $f(x)=x-1$ except when $x=2$ or when $f(x)=1$
And these two missing cases need (easily) to be managed in one or two more lines.
\end{solution}
*******************************************************************************
-------------------------------------------------------------------------------

\begin{problem}[Posted by \href{https://artofproblemsolving.com/community/user/112168}{sir_hoang}]
	Find all functions $f: \mathbb R \to \mathbb R$ such that for all reals $x$ and $y$,
\[f(f(x - y)) = f(x)f(y) + f(x) - f(y) - xy.\]
	\flushright \href{https://artofproblemsolving.com/community/c6h414539}{(Link to AoPS)}
\end{problem}



\begin{solution}[by \href{https://artofproblemsolving.com/community/user/79323}{VIPMaster}]
	The only solution that I could find to your equation was $f(x) = |x|$. Here is the proof (hopefully there isn't anything wrong with it):

First, let us plug in $x = y = 0.$ This gives us the equality $f(f(0)) = (f(0))^2.$ Hold on to this, because we will need it later.

Now let us plug in $x = y = a,$ where $a$ is some constant. This gives us that $f(f(0)) = (f(x))^2 - x^2.$ We can rearrange this to get that $f(x) = \sqrt{f(f(0)) + x^2}.$ Now that we are so close, we need to find a value for $f(f(0)).$

We'll start by plugging in $f(0)$ into the definition of the function. This gives us that $f(f(0)) = f(\sqrt{f(f(0)) + (f(0))^2}).$ Because we know that $f(f(0)) = (f(0))^2,$ we can substitute that expression for $f(f(0)) = f(\sqrt{2(f(0))^2).}$ Then we know that $f(0) = \sqrt{2}f(0).$ Clearly, the only solution to this is $f(0) = 0.$ Then our original expression for the function becomes:
$\boxed{f(x) = \sqrt{x^2} = |x|}$

After typing this up, I feel as if there is something seriously wrong with this solution, but I'm not really sure what. If anyone has any ideas, please let me know.

--VIPMaster
\end{solution}



\begin{solution}[by \href{https://artofproblemsolving.com/community/user/74510}{filipbitola}]
	Well, you don't have injectivity so you can't deduce that $f(0)=0$
Also, let $a=f(f(0))$. Then $f(x)=\sqrt{a+x^{2}}$ is a solution by itself, but you need to prove that there doesn't exist a peace-wise function, such that $f(m)>0$ and $f(n)<0$ for some $m,n\in\mathbb{R}$.
Since you only wanted to know if your solution is correct, I am only giving you the answer: partially.
I haven't checked if there is any constraint to $a$, nor have I checked for peace-wise functions(it's very late where I come from), so I guess you're on your own. :D
\end{solution}



\begin{solution}[by \href{https://artofproblemsolving.com/community/user/29428}{pco}]
	\begin{tcolorbox}Find all functions $f:R \to R$ satisfy the following equation
$f(f(x - y)) = f(x)f(y) + f(x) - f(y) - xy$\end{tcolorbox}
Let $P(x,y)$ be the assertion $f(f(x-y))=f(x)f(y)+f(x)-f(y)-xy$
Let $f(0)=a$

Notice that the summand $xy$ in RHS implies that $f(x)$ can not be bounded.

$P(x,0)$ $\implies$ $f(f(x))=(a+1)f(x)-a$
And so (squaring) : $f(f(x))^2=(a+1)^2f(x)^2-2a(a+1)f(x)+a^2$
$P(f(x),f(x))$ $\implies$ $f(f(x))^2=f(x)^2+f(a)$

And so $(a+1)^2f(x)^2-2a(a+1)f(x)+a^2=f(x)^2+f(a)$
And since $P(0,0)$ implies $a^2=f(a)$, we get : $af(x)((a+2)f(x)-2(a+1))=0$

Setting $x=0$ in this last equality, we get $a^2(a^2-2)=0$ and so $a=0$ or $a^2=2$

If $a^2=2$, then $af(x)((a+2)f(x)-2(a+1))=0$ implies $f(x)\in\{0,2\frac {a+1}{a+2}\}$ bounded, in contradiction with original equation.
So $a=0$ and $P(x,x)$ $\implies$ $f(x)^2=x^2$ $\forall x$

Let then $x,y\notin\{0,1\}$ such that $f(x)=x$ and $f(y)=-y$ :
If $f(f(x-y))=x-y$, $P(x,y)$ becomes $xy=y$, impossible
If $f(f(x-y))=y-x$, $P(x,y)$ becomes $xy=x$, impossible
So :
either $f(x)=x$ $\forall x\ne 1$
either $f(x)=-x$ $\forall x\ne 1$

If $f(x)=x$ $\forall x\ne 1$, then $P(3,1)$ $\implies$ $2=3f(1)+3-f(1)-3$ and so $f(1)=1$ and so $f(x)=x$ $\forall x$
If $f(x)=-x$ $\forall x\ne 1$ then $P(2,0)$ $\implies$ $2=-2$, impossible

Hence the unique solution : $\boxed{f(x)=x}$ $\forall x$ which indeed is a solution.
\end{solution}
*******************************************************************************
-------------------------------------------------------------------------------

\begin{problem}[Posted by \href{https://artofproblemsolving.com/community/user/86849}{abch42}]
	Does the equation
$x+f\left(y+f(x)\right)=y+f\left(x+f(y)\right)$
have a continuous solution $f:\mathbb{R}\rightarrow\mathbb{R}$?
	\flushright \href{https://artofproblemsolving.com/community/c6h414783}{(Link to AoPS)}
\end{problem}



\begin{solution}[by \href{https://artofproblemsolving.com/community/user/29428}{pco}]
	\begin{tcolorbox}Does the equation
$x+f\left(y+f(x)\right)=y+f\left(x+f(y)\right)$
have a continuous solution $f:\mathbb{R}\rightarrow\mathbb{R}$?\end{tcolorbox}
Let $P(x,y)$ be the assertion $x+f(y+f(x))=y+f(x+f(y))$
Let $g(x)=f(x)-x$
$P(x,y)$ becomes new assertion $Q(x,y)$ : $x+g(x)+g(x+y+g(x))=y+g(y)+g(x+y+g(y))$
From this equation, we get that $g(x)$ is injective and so, since continuous, monotonous.

$Q(x,-x)$ $\implies$ $x+g(x)+g(g(x))=-x+g(-x)+g(g(-x))$ and so $x+g(x)+g(g(x))$ is an even function.
But if $g(x)$ is increasing, $x+g(x)+g(g(x))$ is increasing, so injective, and so cant be even.
So $g(x)$ is decreasing.
Looking at $Q(x,y)$, we immediately get then that $\lim_{x\to-\infty}g(x)=+\infty$ and $\lim_{x\to+\infty}g(x)=-\infty$ (il any of these limits was a finite value, $Q(x,y)$ would lead to contradiction : one side infinite, the other finite).

Writing $Q(x,y)$ as $f(x)+g(y+f(x))=f(y)+g(x+f(y))$, we get that $f(x)$ is injective too, and so monotonous.
Writing $Q(x,y)$ as $-y+f(y+f(x))=f(y)+g(x+f(y))$, we get that $\lim_{x\to+\infty}f(x)=-\infty$ and $\lim_{x\to-\infty}f(x)=-\infty$, in contradiction with the fact that $f(x)$ is monotonous.

\begin{bolded}So no such continuous solution.\end{bolded}\end{underlined}
\end{solution}
*******************************************************************************
-------------------------------------------------------------------------------

\begin{problem}[Posted by \href{https://artofproblemsolving.com/community/user/67223}{Amir Hossein}]
	Find all functions $f : \mathbb R \to \mathbb R$ such that for all reals $x, y, z$ it holds that
\[f(x + f(y + z)) + f(f(x + y) + z) = 2y.\]
	\flushright \href{https://artofproblemsolving.com/community/c6h415093}{(Link to AoPS)}
\end{problem}



\begin{solution}[by \href{https://artofproblemsolving.com/community/user/29428}{pco}]
	\begin{tcolorbox}Find all functions $f : \mathbb R \to \mathbb R$ such that for all reals $x, y, z$ it holds that
\[f(x + f(y + z)) + f(f(x + y) + z) = 2y.\]\end{tcolorbox}
Let $P(x,y,z)$ be the assertion $f(x+f(y+z))+f(f(x+y)+z)=2y$
Let $f(0)=a$

$P(x-a,\frac {a-x}2,\frac {x-a}2)$ $\implies$ $f(x)+f(f(\frac {x-a}2)+\frac {x-a}2)=a-x$

$P(\frac{x-a}2,0,\frac{x-a}2)$ $\implies$ $f(\frac{x-a}2+f(\frac{x-a}2))=0$

And so $\boxed{f(x)=a-x}$ which indeed is a solution, whatever is the real $a$
\end{solution}



\begin{solution}[by \href{https://artofproblemsolving.com/community/user/74510}{filipbitola}]
	Plugging in $x=z=0$ we get that:
$f(f(y))=y$
Plugging $x=y=0$ we get that:
$f(z+f(0))=-z$
So $f(x)=-x-c \forall x\in\mathbb{R}$
By plugging in the original equation we see that there is no limitation on $c$
So $\boxed{f(x)=-x-c}\forall x\in\mathbb{R}$ where $c=f(0)$
\end{solution}
*******************************************************************************
-------------------------------------------------------------------------------

\begin{problem}[Posted by \href{https://artofproblemsolving.com/community/user/112643}{proglote}]
	Find all functions $f : \mathbb{R} \to \mathbb{R}$ such that
\[f(xf(y) + f(x)) = 2f(x) + xy, \quad \forall x,y \in \mathbb{R}.\]
	\flushright \href{https://artofproblemsolving.com/community/c6h415230}{(Link to AoPS)}
\end{problem}



\begin{solution}[by \href{https://artofproblemsolving.com/community/user/29428}{pco}]
	\begin{tcolorbox}Find all functions $f : \mathbb{R} \to \mathbb{R}$ such that

$f(xf(y) + f(x)) = 2f(x) + xy \ \forall x,y \in \mathbb{R}.$\end{tcolorbox}
Let $P(x,y)$ be the assertion $f(xf(y)+f(x))=2f(x)+xy$

If $f(a)=f(b)$, comparing $P(1,a)$ and $P(1,b)$ implies $a=b$ and $f(x)$ is an injection.
$P(1,x-2f(1))$ $\implies$ $f(f(x-2f(1))+f(1))=x$ and $f(x)$ is a surjection
Let then $u,v$ such that $f(u)=0$ and $f(v)=1$ : $P(u,v)$ $\implies$ $0=uv$ and so either $f(0)=0$, either $f(0)=1$

If $f(0)=0$, then $P(x,0)$ $\implies$ $f(f(x))=2f(x)$ and so, since surjective, $f(x)=2x$ which is not a solution
So $f(0)=1$

Let then $x\ne 0$ and $y$ such that $f(y)=-\frac{f(x)}x$ (which exists since $f(x)$ is surjective)
$P(x,y)$ $\implies$ $y=\frac{1-2f(x)}x$ and so :
(i) : $f(\frac{1-2f(x)}x)=-\frac{f(x)}x$ $\forall x\ne 0$

$P(x,-\frac{f(x)}x)$ $\implies$ $f(xf(-\frac{f(x)}x)+f(x))=f(x)$ and so, since injective, $xf(-\frac{f(x)}x)+f(x)=x$ and so :
(ii) : $f(-\frac{f(x)}x)=1-\frac{f(x)}x$ $\forall x\ne 0$

$P(-1,-1)$ $\implies$ $f(-1)=0$
$P(x,-1)$ $\implies$ $f(f(x))=2f(x)-x$
Setting $x\to \frac{1-2f(x)}x$ in this expression and, using (i) and (ii), we get :
$f(f(\frac{1-2f(x)}x))=2f(\frac{1-2f(x)}x)-\frac{1-2f(x)}x$
$f(-\frac{f(x)}x)=-2\frac{f(x)}x-\frac{1-2f(x)}x$
$1-\frac{f(x)}x=-2\frac{f(x)}x-\frac{1-2f(x)}x$
$x-f(x)=-2f(x)-(1-2f(x))$
$f(x)=x+1$ $\forall x\ne 0$
And since $f(0)=1=0+1$, we get $\boxed{f(x)=x+1}$ $\forall x$, which indeed is a solution
\end{solution}



\begin{solution}[by \href{https://artofproblemsolving.com/community/user/112643}{proglote}]
	Thanks, this must require a lot of creativity :)
\end{solution}
*******************************************************************************
-------------------------------------------------------------------------------

\begin{problem}[Posted by \href{https://artofproblemsolving.com/community/user/106722}{ambus}]
	Find all functions $\mathbb{R}\to\mathbb{R}$ such that for all $x,y\in\mathbb{R}$,
\[ f(f(x)+y)=f(x^{2}-y)+ 4yf(x).\]
	\flushright \href{https://artofproblemsolving.com/community/c6h415843}{(Link to AoPS)}
\end{problem}



\begin{solution}[by \href{https://artofproblemsolving.com/community/user/114445}{jejungchv}]
	\begin{tcolorbox}Find all functions $\mathbb{R}\to\mathbb{R}$ such that $\forall x,y\in\mathbb{R}$
                                   $     f(f(x)+y)=f(x^{2}-y)+ 4yf(x)$\end{tcolorbox}
Take $y=\frac{x^2-f(x)}{2}$ we have $4(\frac{x^2-f(x)}{2})f(x)$=0
 there fore $f(x)=x^2$ or $f(x)=0$
\end{solution}



\begin{solution}[by \href{https://artofproblemsolving.com/community/user/16261}{Rust}]
	If $f(x_1)=f(x_2)=z, x_1^2\not =x_1^2$, then $f(x_1^2-y)=f(z+y)-4yz=f(x_2^2-y)$ for all y. It mean $x_1^2-x_2^2$ is period. If $T$ is period, then $f(x)=f(x+T)\to T^2+2xT$ is period for all x and $f(x)\equiv const\equiv 0$ if $T\not =0$. 
$y=0$ $f(f(x))=f(x^2)$. We prove, that if $f(x)\not =\pm x^2$ for some x, then $f(x)\equiv 0$.
If $f(x_1^2)=x_1^2\not =0,f(x_2^2)=-x_2^2$ then for $y=x_2^2-x_1^2, x=x_1,x_2$ gives contradition.
Therefore $f(x)=cx^2$, were $c=0,\pm 1$.
\end{solution}



\begin{solution}[by \href{https://artofproblemsolving.com/community/user/29428}{pco}]
	@jejungchv : you missed the classical trap of such problem. You proved that $\forall x$, either $f(x)=0$, either $f(x)=x^2$ and not that either $f(x)=0$ $\forall x$, either $f(x)=x^2$ $\forall x$
You should prove (which is not difficult, but is mandatory) that there is no solution where $f(x)=0$ for some non zero values and $f(x)=x^2$ for some other non zero values.

@Rust : obviously, $f(x)=-x^2$ is not a solution.
\end{solution}



\begin{solution}[by \href{https://artofproblemsolving.com/community/user/86169}{Octav}]
	We shall prove that if $f(a)=f(b)$ for some real $a, b$, then either $a=\pm b$ or $f\equiv 0$.
Consider that $f(a)=f(b)$ for some real $a, b$ for which the desired result does not hold. Plugging into the equation $x=a$ we obtain $f(f(a)+y)-4yf(a)=f(a^2-y)$.
But the same term is equal to $f(f(b)+y)-4yf(b)$ because $f(a)=f(b)$, which furthermore equals $f(b^2-y)$, so, by choosing whatever $y$ we desire, we get that the function $f$ is periodic (the period is $T=|a^2-b^2|\neq 0$).
Now we put $y=T$, $y=0$ and get $f(f(x)+T)-f(x^2-T)=4Tf(x)=f(f(x))-f(x^2)=0$ by periodicity, so $f(x)=0$ for all $x$.

We will now prove $f(x)=f(-x)$. 
By plugging in $x=y=0$ we get $f(f(0))=f(0)$ and by using our previous result we get $f(0)=0$. 
And now, by plugging in $x=0$ we obtain the desired result.

By plugging in $y=0$ we have that $f(f(x))=f(x^2)$, and by our first result we get that $f(x)=\pm x^2$. Suppose that we have some $x\neq 0$ for which $f(x)=-x^2$. By plugging it in into the initial equation we get $f(-x^2+y)=f(x^2-y)-4yx^2$, and by our second result we get that $yx^2=0$, since $-x^2+y=-(x^2-y)$. Taking some $y\neq 0$ we get a contradiction.

Thus we obtain $f(x)=x^2$.
\end{solution}
*******************************************************************************
-------------------------------------------------------------------------------

\begin{problem}[Posted by \href{https://artofproblemsolving.com/community/user/112643}{proglote}]
	Find all functions $ f : \mathbb{R} \to \mathbb{R}$ such that $ f(ab)= f(a+b)$ for all irrational $a$ and $b$ .
	\flushright \href{https://artofproblemsolving.com/community/c6h415870}{(Link to AoPS)}
\end{problem}



\begin{solution}[by \href{https://artofproblemsolving.com/community/user/79323}{VIPMaster}]
	Well, let's try this:
Let $a$ equal some irrational number, and let $b = \frac{1}{a}$
Then we have that $f(1) = f(\frac{a^2 + 1}{a})$
Now, since $a$ can be any irrational number, $f(x)$ is most definitely not an injective function, so we can't use any assumptions based on that.
We should now try what happens when $b = \frac{2}{a}$ or $\frac{3}{a}$, etc.
This gives us that $f(n) = \frac{a^2 + n}{a},$ where $a$ is any irrational number, and $n$ is any rational number except 0.
This looks promising, but I'm not quite sure what to do from here.
Any help would be appreciated.

--VIPMaster
\end{solution}



\begin{solution}[by \href{https://artofproblemsolving.com/community/user/29428}{pco}]
	\begin{tcolorbox}Find all functions $ f : \mathbb{R} \to \mathbb{R}$ such that $ f(ab)= f(a+b)$ for all $a, b$ irrational.\end{tcolorbox}
1) $f(u)=f(0)$ $\forall u\in\mathbb Q$
=====================
Let $u\in\mathbb Q$
Let $y$ any irrational number such that $y^2\notin \mathbb Q$
$u^2+4y^2\notin\mathbb Q$ and so $\sqrt{u^2+4y^2}\notin\mathbb Q$ and so :
$x_1=\frac{u-\sqrt{u^2+4y^2}}2\notin\mathbb Q$ and $x_2=\frac{u+\sqrt{u^2+4y^2}}2\notin\mathbb Q$

$y,-y\notin\mathbb Q$ $\implies$ $f(y(-y))=f(y-y)$ $\implies$ $f(0)=f(-y^2)$
$x_1,x_2\notin\mathbb Q$ $\implies$ $f(x_1x_2)=f(x_1+x_2)$ $\implies$ $f(-y^2)=f(u)$
And so $f(u)=f(0)$ $\forall u\in\mathbb Q$
Q.E.D.

2) $f(u)=f(0)$ $\forall u\notin\mathbb Q$
=====================
Let $u\notin\mathbb Q$
Let $y$ any irrational number such that $\sqrt{u^2+4y^2}\in \mathbb Q$ 
$\sqrt{u^2+4y^2}\in\mathbb Q$ and so :
$x_1=\frac{u-\sqrt{u^2+4y^2}}2\notin\mathbb Q$ and $x_2=\frac{u+\sqrt{u^2+4y^2}}2\notin\mathbb Q$

$y,-y\notin\mathbb Q$ $\implies$ $f(y(-y))=f(y-y)$ $\implies$ $f(0)=f(-y^2)$
$x_1,x_2\notin\mathbb Q$ $\implies$ $f(x_1x_2)=f(x_1+x_2)$ $\implies$ $f(-y^2)=f(u)$
And so $f(u)=f(0)$ $\forall u\in\mathbb Q$
Q.E.D.


Hence the answer : $\boxed{f(x)=c}$ constant
\end{solution}



\begin{solution}[by \href{https://artofproblemsolving.com/community/user/112643}{proglote}]
	Thanks, very creative solution!
\end{solution}
*******************************************************************************
-------------------------------------------------------------------------------

\begin{problem}[Posted by \href{https://artofproblemsolving.com/community/user/86626}{enigmation}]
	Find all functions $f: \mathbb R \to \mathbb R$ such that for all reals $x$ and $y$,
\[f(x + f(y)) = f(x - f(y)) + 4xf(y).\]
	\flushright \href{https://artofproblemsolving.com/community/c6h415924}{(Link to AoPS)}
\end{problem}



\begin{solution}[by \href{https://artofproblemsolving.com/community/user/51248}{professordad}]
	[hide="Maybe Wrong"]
Let $f(y) = a$; then we get the equation $f(x + a) = f(x - a) + 4ax$, which is $f(x + a) = f(x - a) + (x + a)^2 - (x - a)^2$.  Subtracting $(x + a)^2$ gives $f(x + a) - (x + a)^2 = f(x - a) - (x - a)^2$.  Letting $x + a = m$ and $x - a = n$ yields $f(m) - m^2 = f(n) - n^2$.  Let this be $c$; we find that any $f(x)$ such that $f(x) = \boxed{x^2 + c}$ works.

EDIT: whoops forgot to check.  

Plugging in $x^2 + c$ gives us $f(x + (y^2 + c))$ on the LHS and $f(x - (y^2 + c))^2 + 4x(y^2 + c)$ on the RHS.  So basically we must have $(x + (y^2 + c))^2 - (x - (y^2 + c))^2 = 4x(y^2 + c)$, which is true.
[\/hide]

Let's see how fast pco or BigSams finds an error with this
\end{solution}



\begin{solution}[by \href{https://artofproblemsolving.com/community/user/29428}{pco}]
	\begin{tcolorbox} $f(x + a) - (x + a)^2 = f(x - a) - (x - a)^2$.  Letting $x + a = m$ and $x - a = n$ yields $f(m) - m^2 = f(n) - n^2$.  Let this be $c$; we find that any $f(x)$ such that $f(x) = \boxed{x^2 + c}$ works.\end{tcolorbox}
You just proved that $f(x)-x^2$ is periodic with period $2a$ and then said that a constant function is periodic with any period and can be a solution.
So you found a family of solutions.
It remains to prove (which is the real problem) if these are all the solutions(which is unfortunately wong :) ).
\end{solution}



\begin{solution}[by \href{https://artofproblemsolving.com/community/user/29428}{pco}]
	\begin{tcolorbox}hi all

Find all functions f : R ! R which satisfy the equality,
f(x + f(y)) = f(x − f(y)) + 4xf(y)
for any x, y 2 R (Here R denote the set of real numbers)

enigmation\end{tcolorbox}
A classical solution :
$f(x)=0$ $\forall x$ is a solution.
Let us from now look for non allzero solutions :

Let $P(x,y)$ be the assertion $f(x+f(y))=f(x-f(y))+4xf(y)$
Let $t$ such that $f(t)\ne 0$

Let $u\in\mathbb R$ : $P(\frac{u}{8f(t)},t)$ $\implies$ $u=2f(\frac{u}{8f(t)}+f(t))-2f(\frac{u}{8f(t)}-f(t))$

Let us call $a=\frac{u}{8f(t)}+f(t)$ and $b=\frac{u}{8f(t)}-f(t)$ so that $2f(a)-2f(b)=u$

$P(2f(a)-f(b),b)$ $\implies$ $f(2f(a))=f(2f(a)-2f(b))+8f(a)f(b)-4f(b)^2$
$P(f(a),a)$ $\implies$ $f(2f(a))=f(0)+4f(a)^2$

Subtracting these two lines, we get $f(2f(a)-2f(b))=f(0)+(2f(a)-2f(b))^2$ and so $f(u)=f(0)+u^2$ $\forall u$ which indeed is a solution.

\begin{bolded}Hence the only solutions \end{bolded}\end{underlined}:
$f(x)=0$ $\forall x$
$f(x)=x^2+c$ $\forall x$ and for any real $c$
\end{solution}
*******************************************************************************
-------------------------------------------------------------------------------

\begin{problem}[Posted by \href{https://artofproblemsolving.com/community/user/95616}{mrugesh}]
	If $ f(x) $ is a continuous function and $ f(f(x)) = 1 + x $  then find $ f(x) $.
	\flushright \href{https://artofproblemsolving.com/community/c6h416061}{(Link to AoPS)}
\end{problem}



\begin{solution}[by \href{https://artofproblemsolving.com/community/user/29428}{pco}]
	\begin{tcolorbox}If $ f(x) $ is a continuous function and $ f(f(x)) = 1 + x $  then find $ f(x) $.\end{tcolorbox}
$f(x)$ is a continuous bijection and so is monotonic.
If $f(x)$ is decreasing, then $\exists u$ such that $f(u)=u$ but then $f(f(u))=u\ne u+1$ and so $f(x)$ is increasing.

If $f(x)\le x$ for some $x$, then $f(f(x))\le f(x)\le x$ and so $f(f(x))\ne x+1$. So $f(x)>x$ $\forall x$
If $f(x)\ge x+1$ for some $x$, then $f(f(x))\ge f(x+1)$ and so $f(x+1)\le x+1$, impossible (see previous line)

So $f(x)$ is a continuous increasing function such that $x<f(x)<x+1$ $\forall x$

Let then $f(0)=a\in(0,1)$
$f(a)=f(f(0))=1$ and so $f([0,a))=[a,1)$
Using then $f(x)=1+f^{-1}(x)$, we get that knowledge of $f(x)$ in $[0,a)$ implies knowledge of $f(x)$ in $[a,1)$
Using then $f(x+1)=f(x)+1$, we get that knowledge of $f(x)$ in $[0,1)$ implies knowledge of $f(x)$ in $\mathbb R$

So $f(x)$ is full defined by its values over $[0,a)$

And obviously, the only constraints for these values are : increasing, continuous, and $f(a)=1$

\begin{bolded}Hence the solutions \end{bolded}\end{underlined}:
Let any $a\in(0,1)$
Let any continuous increasing bijection $h(x)$ from $[0,a]\to[a,1]$
$h^{-1}(x)$ is a continuous increasing bijection from $[a,1]\to[0,a]$

Define $f(x)$ as :
$\forall x\in[0,a)$ : $f(x)=h(x)$
$\forall x\in[a,1)$ : $f(x)=1+h^{-1}(x)$
$\forall x\notin[0,1)$ : $f(x)=f(\{x\})+\lfloor x\rfloor$

And so obviously infinitely many solutions (the simplest is trivially $x+\frac 12$)
\end{solution}



\begin{solution}[by \href{https://artofproblemsolving.com/community/user/29428}{pco}]
	\begin{tcolorbox}If $ f(x) $ is a continuous function and $ f(f(x)) = 1 + x $  then find $ f(x) $.\end{tcolorbox}
Just for complementary info : here is a rather nice general family of solutions :

Let $u(x)$ any increasing continuous bijection from $[0,1]\to[0,1]$

Let $h(x)=\lfloor x\rfloor+u(\{x\})$
$h(x)$ is an increasing continuous bijection from $\mathbb R\to\mathbb R$

Then $f(x)=h^{-1}(h(x)+\frac 12)$ is a continuous solution of the functional equation $f(f(x))=x+1$

The problem is that I'm not sure that this is a general solution (I mean that I'm not sure that all solutions may be obtained in this form).
My previous post gives all the solutions.
\end{solution}
*******************************************************************************
-------------------------------------------------------------------------------

\begin{problem}[Posted by \href{https://artofproblemsolving.com/community/user/49267}{n-naoufal}]
	Let $a$ and $b$ be reals numbers, $b<0$. Let $f: \mathbb R \to \mathbb R$ be a function satisfying: 
\[f(f(x))=a+bx\]
for all reals $x$. Prove that $f$ has infinitely many discontinuities.
	\flushright \href{https://artofproblemsolving.com/community/c6h416090}{(Link to AoPS)}
\end{problem}



\begin{solution}[by \href{https://artofproblemsolving.com/community/user/29428}{pco}]
	\begin{tcolorbox}Let $a$ and $b$ be reals numbers, $b<0$. Let $f$ be a function from the real line $R$ into $R$ and satisfying: 
$(x \in R)$,  $f(f(x))=a+bx$
Prove that $f$ has infinitly discontinuities.\end{tcolorbox}
Writing $f(x)=g(x-\frac a{1-b})+\frac a{1-b}$, the equation becomes $g(g(x))=bx$

If $b=-1$, see nice proof at http://www.artofproblemsolving.com/Forum/viewtopic.php?f=67&t=113408
If $b\ne -1$ :

From $g(g(x))=bx$, we get $g(bx)=bg(x)$ and $g(0)=0$
Notice that $g(x)$ is a bijection and so $g(x)=0$ $\iff$ $x=0$

Let $u>0$ and $v=g(u)\ne 0$
If $v>0$, then $g(v)=bu<0$ and so there is a discontinuity in $[u,v]$ (or $[v,u]$) else we would have some $t\in(u,v)$ or $(v,u)$ such that $g(t)=0$, impossible
If $v<0$, then $g(v)=bu<0$ and $g(bu)=bv>0$ so there is a discontinuity in $[v,bu]$ (or $[bu,v]$) else we would have some $t\in(v,bu)$ or $(bu,v)$ such that $g(t)=0$, impossible

So there is at least a discontinuity $x_0\ne 0$
Since $f(bx)=bf(x)$, a discontinuity point at $x_0$ implies a discontinuity point at $bx_0$ and so, since $b\ne -1$ and $x_0\ne 0$, infinitely many discontinuity points.
Q.E.D.
\end{solution}
*******************************************************************************
-------------------------------------------------------------------------------

\begin{problem}[Posted by \href{https://artofproblemsolving.com/community/user/114445}{jejungchv}]
	Find all functions $f: \mathbb R \to \mathbb R$ which satisfy for all $x, y \in \mathbb R$,
\[f(x+y+f(y))=f(f(x))+2y.\]
	\flushright \href{https://artofproblemsolving.com/community/c6h416138}{(Link to AoPS)}
\end{problem}



\begin{solution}[by \href{https://artofproblemsolving.com/community/user/29428}{pco}]
	\begin{tcolorbox}Let $\mathbb{R}\rightarrow \mathbb{R}$
Find all function $f(x)$ satisfying
    $f(x+y+f(y)=f(f(x))+2y$\end{tcolorbox}
There is a missing parenthesis in LHS and, since I understood that we always are allowed in modern Olympiads to modify the statement as we want in order to find a more interesting problem,  I interpreted this LHS as $f(x+y+f(y))$ (other possible interpretations : $f(x+y)+f(y)$ or $f(x)+y+f(y)$ for example)
Let $P(x,y)$ be the assertion $f(x+y+f(y))=f(f(x))+2y$

If $f(a)=f(b)=u$ for some $a,b$, then :
$P(a,b)$ $\implies$ $f(a+b+u)=f(u)+2b$
$P(b,a)$ $\implies$ $f(b+a+u)=f(u)+2a$
And so $a=b$ and $f(x)$ is injective.

$P(x,0)$ $\implies$ $f(x+f(0))=f(f(x))$ and so, since injective : $x+f(0)=f(x)$ which indeed is a solution, whatever is $f(0)$

Hence the answer : $\boxed{f(x)=x+c}$ $\forall x$ and for any real $c$
\end{solution}
*******************************************************************************
-------------------------------------------------------------------------------

\begin{problem}[Posted by \href{https://artofproblemsolving.com/community/user/61626}{StefanS}]
	Find all functions $ f : \mathbb{R} \rightarrow \mathbb{R}$ such that $~$ $ f(x+y) + xy = f(x)f(y)$ for all reals $x$ and $y$.
	\flushright \href{https://artofproblemsolving.com/community/c6h416198}{(Link to AoPS)}
\end{problem}



\begin{solution}[by \href{https://artofproblemsolving.com/community/user/29428}{pco}]
	\begin{tcolorbox}Find all functions $~$ $ f : \mathbb{R} \rightarrow \mathbb{R}$ $~$ such that $~$ $ f(x+y) + xy = f(x)f(y).$\end{tcolorbox}
Let $P(x,y)$ be the assertion $f(x+y)+xy=f(x)f(y)$

$P(x,1)$ $\implies$ $f(x+1)+x=f(x)f(1)$ and so $f(x+1)=f(1)f(x)-x$
$P(x+1,1)$ $\implies$ $f(x+2)+x+1=f(x+1)f(1)$ and so $f(x+2)=f(1)f(x+1)-x-1$ and so $f(x+2)=f(1)^2f(x)-x(f(1)+1)-1$
$P(x,2)$ $\implies$ $f(x+2)+2x=f(x)f(2)$ and so $f(x+2)=f(2)f(x)-2x$

So $f(1)^2f(x)-x(f(1)+1)-1=f(2)f(x)-2x$ and $(f(1)^2-f(2))f(x)=x(f(1)-1)+1$

$f(1)^2-f(2)=0$ would imply $x(f(1)-1)+1=0$ $\forall x$, which is impossible

So $f(x)=ax+b$ for some $a,b$ and plugging this in original equation, we get $a=\pm1$ and $b=1$

\begin{bolded}Hence the solutions \end{bolded}\end{underlined}:
$f(x)=x+1$ $\forall x$
$f(x)=1-x$ $\forall x$
\end{solution}



\begin{solution}[by \href{https://artofproblemsolving.com/community/user/61626}{StefanS}]
	[hide="My Solution"]

$ \boxed { f(x+y) + xy = f(x)f(y) } \; \cdots \, (1) $

$ y: \, = 0 \quad \stackrel {(1)} {\Longrightarrow} \quad \boxed { f(x) = f(x)f(0) } \; \cdots \, (2) $

$ f \equiv 0 \quad \stackrel {(1)} {\Longleftrightarrow} \quad xy = 0 \; \forall \; x, y \in \mathbb{R} \; \# \quad \Longrightarrow \quad f \not\equiv 0 \quad \stackrel {(2)} {\Longrightarrow}  \quad \boxed { f(0) = 1 } \; \cdots \, (3) $

$ y: \, = -x \quad \stackrel {(1)} {\Longrightarrow} \quad 1 - x^2 = f(x)f(-x) \quad \stackrel {x: \, 1} {\Longrightarrow} \quad 0 = f(1)f(-1) \quad \Downarrow $  

$ \Longrightarrow \quad \boxed{ f(1) \; \vee \; f(-1) = 0 } $

$~$

$ i) \; f(1) = 0 $

$ y: \, = 1 \quad \stackrel {(1)} {\Longrightarrow} \quad f(x+1) + x = 0 \quad \stackrel {x: \, = x-1} {\Longrightarrow} \quad \boxed { f(x) = 1 - x } \; \cdots \, (4) $

$~$

$ ii) \; f(-1) = 0 $

$ y: \, = -1 \quad \stackrel {(1)} {\Longrightarrow} \quad f(x-1) - x = 0 \quad \stackrel {x: \, = x+1} {\Longrightarrow} \quad \boxed { f(x) = 1 + x } \; \cdots \, (5) $

$~$

$ \stackrel {(4) \; \wedge \; (5)} {\Longrightarrow} $ $~$ $~$ the only possible solutions are: $~$ $~$ $ f(x) = 1 - x \quad \wedge \quad f(x) = 1 + x$

By substituting $~$ $ (4) \; \wedge \; (5) $ $~$ in $~$ $ (1)$ $~$ we prove that both of them are solutions.

Hence, $~$ $ f(x) = 1 \pm x \quad \blacksquare $[\/hide]
\end{solution}



\begin{solution}[by \href{https://artofproblemsolving.com/community/user/74510}{filipbitola}]
	Let $P(x,y)$ be the assertion $f(x+y)+xy=f(x)f(y)$
$P(x,0)\implies f(x)=f(x)f(0)$
If $f(0)=0\implies f(x)=0$ which is not a solution.
So $f(0)\neq 0$
$P(0,0)\implies f(0)=1$
$P(x-y,y)\implies f(x)+xy-y^{2}=f(x)f(y)f(-y)+xyf(y)$
Let $G(x,y)$ be the assertion $f(x)+xy-y^{2}=f(x)f(y)f(-y)+xyf(y)$
$G(x,-y)$ ^ $G(x,y)\implies f(-y)=2-f(y)$
$P(x,-x)\implies x^{2}=(f(x)-1)^{2}$
Considering cases we get that $f(x)=x+1$ and $f(x)=1-x$.
The peace-wise case doesn't produce any solutions(can be easily checked).
\end{solution}
*******************************************************************************
-------------------------------------------------------------------------------

\begin{problem}[Posted by \href{https://artofproblemsolving.com/community/user/81416}{swaqar}]
	Let $f: \mathbb R^+\to \mathbb R^+$ be a continuous strictly decreasing function such that :
\[f(x+y) + f(f(x)+f(y)) = f(f(x+f(y)) + f(y+f(x))), \quad \forall x,y\in\mathbb R^+.\]
Prove that $f(f(x))=x$ for all $x\in\mathbb R^+$.
	\flushright \href{https://artofproblemsolving.com/community/c6h416312}{(Link to AoPS)}
\end{problem}



\begin{solution}[by \href{https://artofproblemsolving.com/community/user/29428}{pco}]
	\begin{tcolorbox}Let $ f : \mathbb{R} \longmapsto \mathbb{R} $ be an increasing function such that for all positive reals $ x $ and $ y $, the equation , 

\[ f( x+ y) + f ( f(x) + f(y) ) = f ( f(x+ f(y)) + f(   y + f(x)) ) \] 

holds. 

Prove that  $  f( f( x)) = x$  .\end{tcolorbox}
Are you sure that the statement is correct ?
In http://www.artofproblemsolving.com/Forum/viewtopic.php?f=37&t=51456
And in http://www.artofproblemsolving.com/Forum/viewtopic.php?f=36&t=325436

The problem (IRAN 1997 too) seems the same but :
a) function is supposed decreasing and not increasing
b) function is supposed continuous while it's not in your problem
c) function is supposed from $\mathbb R^+\to\mathbb R^+$ and not from $\mathbb R\to\mathbb R$

Was there two so similar problems the same year in the same countryn?
Or did you just omit "useless" conditions ?
\end{solution}



\begin{solution}[by \href{https://artofproblemsolving.com/community/user/81416}{swaqar}]
	Exactly the same suspicions as you have ..... 
http://www.imomath.com\/tekstkut\/funeqn_mr.pdf .... 
problem sixteen and page fifteen ....
\end{solution}



\begin{solution}[by \href{https://artofproblemsolving.com/community/user/29428}{pco}]
	This problem from the pdf is certainly wrong.

Choose for example $f(x)=x$ $\forall x>0$ and $f(x)=x-1$ $\forall x\le 0$ and $f(f(x))=x$ is wrong $\forall x\le 0$
\end{solution}



\begin{solution}[by \href{https://artofproblemsolving.com/community/user/115063}{PhantomR}]
	@pco: Actually, I think the counterexample is wrong because the problem says "for all positive real numbers $x$ and $y$". Please correct me if I'm wrong.

EDIT: @pco: Ah, you're very right! I apologize for being an idiot.
\end{solution}



\begin{solution}[by \href{https://artofproblemsolving.com/community/user/29428}{pco}]
	\begin{tcolorbox}@pco: Actually, I think the counterexample is wrong because the problem says "for all positive real numbers $x$ and $y$". Please correct me if I'm wrong.\end{tcolorbox}
No.
The problem says :
The domain of the function is $\mathbb R$
The domain of the functional equation is $\mathbb R^+$
Nothing about the domain of $f(f(x))=x$ but likely the domain of the function, so $\mathbb R$

So my counterexample is OK.
\end{solution}



\begin{solution}[by \href{https://artofproblemsolving.com/community/user/81416}{swaqar}]
	\begin{tcolorbox}@pco: Actually, I think the counterexample is wrong because the problem says "for all positive real numbers $x$ and $y$". Please correct me if I'm wrong.\end{tcolorbox}

hmmm .... you mean that problem asks for $  f ( f ( x ) ) = x $ for only positive $ x $ then ....
\end{solution}



\begin{solution}[by \href{https://artofproblemsolving.com/community/user/81416}{swaqar}]
	okay .... seems like then this problem is wrong or something not interesting to work with .... can you post the ' correct ' problem which indeed has a solution by introducing or changing some conditions ....
\end{solution}



\begin{solution}[by \href{https://artofproblemsolving.com/community/user/29428}{pco}]
	"Correct problem" seems to be :

Let $f(x)$ a continuous strictly decreasing function from $\mathbb R^+\to \mathbb R^+$ such that :
$f(x+y) + f(f(x)+f(y)) = f(f(x+f(y)) + f(y+f(x)))$ $\forall x,y\in\mathbb R^+$
Prove that $f(f(x))=x$ $\forall x\in\mathbb R^+$

[hide="easy proof"]
$f(x)$ from $\mathbb R^+\to \mathbb R^+$, continuous, strictly decreasing $\implies$ equation $f(x)=x$ has a unique root $a>0$
Setting $y=a$ in the functional equation implies $f(x+a) + f(f(x)+a) = f(f(x+a) + f(f(x)+a))$
And so $f(x+a) + f(f(x)+a)$ is also root of $f(X)=X$ and so is $a$ :
$f(x+a) + f(f(x)+a)=a$
Setting $x\to f(x)$ in this expression, we get $f(f(x)+a) + f(f(f(x))+a)=a$
And so $f(f(f(x))+a)=f(x+a)$ and, since injective (since strictly decreasing) : $\boxed{f(f(x))=x}$
Q.E.D
and, btw, such a function exists : choose for example $f(x)=\frac 1x$
[\/hide]
\end{solution}



\begin{solution}[by \href{https://artofproblemsolving.com/community/user/74510}{filipbitola}]
	\begin{tcolorbox}
$f(x)$ from $\mathbb R^+\to \mathbb R^+$, continuous, strictly decreasing $\implies$ equation $f(x)=x$ has a unique root $a>0$
\end{tcolorbox}
Can you explain where this comes from?(I'm not that familiar with function theory) :)
\end{solution}



\begin{solution}[by \href{https://artofproblemsolving.com/community/user/29428}{pco}]
	\begin{tcolorbox}[quote="pco"]
$f(x)$ from $\mathbb R^+\to \mathbb R^+$, continuous, strictly decreasing $\implies$ equation $f(x)=x$ has a unique root $a>0$
\end{tcolorbox}
Can you explain where this comes from?(I'm not that familiar with function theory) :)\end{tcolorbox}
$\lim_{x\to 0}(f(x)-x)>0$ (since $f(x)$ decreasing and positive)

$\lim_{x\to+\infty}(f(x)-x)<0$ (since $f(x)$ decreasing )

$f(x)-x$ is continuous and so $f(x)-x=0$ has solutions (intermediate value theorem)

But $f(a)-a=0$ and $f(b)-b=0$ with wlog $a<b$ would imply $f(a)<f(b)$ with $a<b$, in contradiction with "strictly decreasing"

So $f(x)-x=0$ has exactly one solution.
\end{solution}
*******************************************************************************
-------------------------------------------------------------------------------

\begin{problem}[Posted by \href{https://artofproblemsolving.com/community/user/81416}{swaqar}]
	Find all real values of $\alpha$ for which there exists exactly one function $f: \mathbb R \to \mathbb R$ such that
\[ f ( x^{2} + y + f( y ) )  = (f(x))^{2} + \alpha y  \]
holds for all $x,y \in \mathbb R$.
	\flushright \href{https://artofproblemsolving.com/community/c6h416315}{(Link to AoPS)}
\end{problem}



\begin{solution}[by \href{https://artofproblemsolving.com/community/user/29428}{pco}]
	\begin{tcolorbox}Find those values of the real parameter $ \alpha $ such that there exists only one function $ f $ from reals to reals satisfying the following functional equation : 

\[ f ( x^{2} + y + f( y ) )  = (f(x))^{2} + \alpha y .  \]\end{tcolorbox}
Let $P(x,y)$ be the assertion $f(x^2+y+f(y))=f(x)^2+\alpha y$
Let $f(0)=a$

If $\alpha=0$, then we get at least the two solutions $f(x)=0$ $\forall x$ and $f(x)=1$ $\forall x$. So $\alpha\ne 0$

Since $\alpha\ne 0$, $P(0,\frac {x-a^2}{\alpha})$ $\implies$ $f(\frac {x-a^2}{\alpha}+f(\frac {x-a^2}{\alpha}))=x$ and so $\boxed{f(x)\text{ is surjective}}$.
Comparing $P(x,y)$ and $P(-x,y)$, we get $f(-x)^2=f(x)^2$ and so $\forall x$ : either $f(-x)=-f(x)$, either $f(-x)=f(x)$

Let $x>0$ and $b$ such that $f(b)=-x$ : $P(\sqrt x1,b)$ $\implies$ $-x=f(\sqrt x)^2+\alpha b$ and so $b=-\frac{x+f(\sqrt x)^2}{\alpha}\ne 0$
So there is a unique $b\ne 0$ such that $f(b)=-x$ and so $f(-b)$ cant be equal to $f(b)$ and so $f(-b)=x=-f(b)$
$P(0,b)$ $\implies$ $f(b+f(b))=a^2+\alpha b$
$P(0,-b)$ $\implies$ $f(-b-f(b))=a^2-\alpha b$
And since $f(-b-f(b))=\pm f(b+f(b))$, we get $a^2+\alpha b=\pm(a^2-\alpha b)$ and so $a=\boxed{f(0)=0}$ (since $b\ne 0$)

If $f(u)=f(v)=w<0$, then the previous lines proved that $a=b$ ($=-\frac{-w+f(\sqrt {-w})^2}{\alpha}\ne 0$)
If $f(u)=f(v)=w>0$, then $\exists$ unique $t$ such that $f(t)=-w$ and $f(-t)=w$ and so $u=\pm t$ but $f(t)=-w$ and so $u=v=-t$
If $f(u)=0$, then the previous lines proved that there is a unique $b$ such that $f(b)=0$ and since $f(0)=0$, we get $b=0$

So $\boxed{f(x)\text{ is an odd bijection}}$.

$P(0,y)$ $\implies$ $f(y+f(y))=\alpha y$
$P(x,0)$ $\implies$ $f(x^2)=f(x)^2$

And so $P(x,y)$ becomes $f(x^2+y+f(y))=f(x^2)+f(y+f(y))$
And since $f(x+f(x))=\alpha x$ and $f(x)$ is bijective, we get  that $x+f(x)$ is bijective too

And so $f(x^2+y+f(y))=f(x^2)+f(y+f(y))$ becomes $f(u+v)=f(u)+f(v)$ $\forall u\ge 0$ and $\forall v$
So (since odd) : $\boxed{f(u+v)=f(u)+f(v)}$ $\forall u,v$
But $f(x^2)=f(x)^2$ implies $f(v)\ge 0$ $\forall v\ge 0$ and then $f(u+v)=f(u)+f(v)$ implies $f(x)$ non decreasing.

So $f(x)=cx$ (monotonous solution of Cauchy equation) and, plugging in original equation, we get :
$c^2=c$ and $\alpha=2c$ and so $c=1$ and $\alpha =2$

\begin{bolded}Hence the answer \end{bolded}\end{underlined}:
If $\alpha\notin\{0,2\}$ : no solution
If $\alpha =0$ : at least two solutions
If $\boxed{\alpha=2}$ : exactly one solution $f(x)=x$
\end{solution}
*******************************************************************************
-------------------------------------------------------------------------------

\begin{problem}[Posted by \href{https://artofproblemsolving.com/community/user/77226}{wya}]
	Let $f:\mathbb{R}\rightarrow\mathbb{R}$ be such that for all $x,y\in\mathbb{R}$,
$\left|f(x-y)\right|=\left|f(x)-f(y)\right|$.
Can we conclude that 
$f(x+y)=f(x)+f(y)$
for all $x,y\in\mathbb{R}$? Justify your answer.
	\flushright \href{https://artofproblemsolving.com/community/c6h416592}{(Link to AoPS)}
\end{problem}



\begin{solution}[by \href{https://artofproblemsolving.com/community/user/29428}{pco}]
	\begin{tcolorbox}Let $f:\mathbb{R}\rightarrow\mathbb{R}$ be such that for all $x,y\in\mathbb{R}$,
$\left|f(x-y)\right|=\left|f(x)-f(y)\right|$.
Can we conclude that 
$f(x+y)=f(x)+f(y)$
for all $x,y\in\mathbb{R}$? Justify your answer.\end{tcolorbox}
$f(x)=0$ $\forall x$ is a solution of the functional equation and is such that $f(x+y)=f(x)+f(y)$ $\forall x,y$
So, let us look from now only for non allzero solutions.
Let $P(x,y)$ be the assertion $|f(x-y)|=|f(x)-f(y)|$
Let $w$ such that $f(w)\ne 0$

$P(0,0)$ $\implies$ $f(0)=0$
$P(0,x)$ $\implies$ $|f(-x)|=|f(x)|$

Suppose now that $\exists u,v$ such that $f(-u)=-f(u)$ and $f(-v)=f(v)$
$P(-u,-v)$ $\implies$ $|f(-u+v)|=|f(u)+f(v)|$ and so $|f(u-v)|=|f(u)+f(v)|$ and since $|f(u-v)|=|f(u)-f(v)|$ :
either $f(u)=0$ and so $f(-u)=f(u)$ and so both $u,v$ are such that $f(-x)=f(x)$
either $f(v)=0$ and so $f(-v)=-f(v)$ and so both $u,v$ are such that $f(-x)=-f(x)$

So $f(-x)=f(x)$ $\forall x$ or $f(-x)=-f(x)$ $\forall x$

But if $f(-x)=f(x)$ $\forall x$, then : $P(\frac w2,-\frac w2)$ $\implies$ $|f(w)|=|f(\frac w2)-f(-\frac w2)|$ $=|f(\frac w2)-f(\frac w2)|=0$, impossible (definition of $w$)

So $f(-x)=-f(x)$ $\forall x$

Let us call $(x,y)\in\mathbb R^2$ :
"white" if $f(x)=f(y)$ and so $f(x-y)=0$
"green" if $f(x-y)=f(x)-f(y)\ne 0$
"red" if $f(x-y)=f(y)-f(x)\ne 0$
Notice that $f(-x)=-f(x)$ implies that $(x,y)$ and $(y,x)$ have same colours

Let then $(a,b)$ and $(b,c)$ two non white pairs.
If $(a,b)$ and $(c,b)$ dont have the same color, then :
$|f(a)-f(c)|$ $=|f(a-c)|=|f((a-b)-(c-b))|$ $=|f(a-b)-f(c-b)|=|f(a)+f(c)-2f(b)|$ and so :
either $f(a)-f(c)=f(a)+f(c)-2f(b)$ and so $f(c)=f(b)$, impossible since $(c,b)$ is not white
either $f(a)-f(c)=-f(a)-f(c)+2f(b)$ and so $f(a)=f(b)$, impossible since $(a,b)$ is not white
So $(a,b)$ and $(c,b)$ have same color


Let then $(x,y)$ and $(z,t)$ two non white pairs. :
$P(w,-w)$ $\implies$ $|f(2w)|=2|f(w)|\ne 0$ 
So $f(w),f(2w),f(4w)$ are pairwise different
So one of these three numbers (let us call it $f(u)$) is different from $f(y)$ and from $f(z)$ and so $(y,u)$ and $(z,u)$ both are non white.

$(x,y)$ and $(y,u)$ are both non white, so have same colours
$(y,u)$ and $(u,z)$ are both non white, so have same colours
$(z,u)$ and $(z,t)$ are both non white, so have same colours

So $(x,y)$ and $(z,t)$ both have same colours and so :
either all pairs are either white, either green
either all pairs are either white, either red

In the first case, we get $f(x-y)=f(x)-f(y)$ $\forall x,y$ and so $f(x+y)=f(x)+f(y)$ $\forall x,y$
In the second case, we get $f(x-y)=f(y)-f(x)$ $\forall x,y$ and so (choose $x=w$ and $y=0$) contradiction

\begin{bolded}Hence the result \end{bolded}\end{underlined}: $f(x+y)=f(x)+f(y)$ $\forall x,y$
\end{solution}



\begin{solution}[by \href{https://artofproblemsolving.com/community/user/115805}{gallantry}]
	Let's say this is the most brilliant solution I've seen for days.
Personally, I love the technique used here.
\begin{tcolorbox}
Suppose now that $\exists u,v$ such that $f(-u)=-f(u)$ and $f(-v)=f(v)$
$P(-u,-v)$ $\implies$ $|f(-u+v)|=|f(u)+f(v)|$ and so $|f(u-v)|=|f(u)+f(v)|$ and since $|f(u-v)|=|f(u)-f(v)|$ :
either $f(u)=0$ and so $f(-u)=f(u)$ and so both $u,v$ are such that $f(-x)=f(x)$
either $f(v)=0$ and so $f(-v)=-f(v)$ and so both $u,v$ are such that $f(-x)=-f(x)$

So $f(-x)=f(x)$ $\forall x$ or $f(-x)=-f(x)$ $\forall x$
\end{tcolorbox}

And I also think that the way pco coloured things is elegant.
This makes the solution very easy to read.

\begin{tcolorbox}
Let us call $(x,y)\in\mathbb R^2$ :
"white" if $f(x)=f(y)$ and so $f(x-y)=0$
"green" if $f(x-y)=f(x)-f(y)\ne 0$
"red" if $f(x-y)=f(y)-f(x)\ne 0$
Notice that $f(-x)=-f(x)$ implies that $(x,y)$ and $(y,x)$ have same colours
\end{tcolorbox}

Nicely done. Thanks to pco. :)
\end{solution}



\begin{solution}[by \href{https://artofproblemsolving.com/community/user/29428}{pco}]
	Thanks  :)
\end{solution}
*******************************************************************************
-------------------------------------------------------------------------------

\begin{problem}[Posted by \href{https://artofproblemsolving.com/community/user/89818}{gauman}]
	Find all functions $f: \mathbb R \to \mathbb R$ which satisfy for all $x, y \in \mathbb R$,
\[f(x+f(y))=f(x)+ \frac{1}{8}xf(4y)+f(f(y)).\]
	\flushright \href{https://artofproblemsolving.com/community/c6h416628}{(Link to AoPS)}
\end{problem}



\begin{solution}[by \href{https://artofproblemsolving.com/community/user/29428}{pco}]
	\begin{tcolorbox}Find all function $f :R \to R$ such that: $f(x+f(y))=f(x)+ \frac{1}{8}xf(4y)+f(f(y))$\end{tcolorbox}
$f(x)=0$ $\forall x$ is a solution. Let us from now look for non allzero solutions.
Let $P(x,y)$ be the assertion $f(x+f(y))=f(x)+\frac 18xf(4y)+f(f(y))$
Let $t$ such that $f(t)\ne 0$

$P(0,0)$ $\implies$ $f(0)=0$

$P(f(x),f(t))$ $\implies$ $f(f(x)+f(t))=f(f(x))+\frac 18f(x)f(4t)+f(f(t))$
$P(f(t),f(x))$ $\implies$ $f(f(x)+f(t))=f(f(x))+\frac 18f(t)f(4x)+f(f(t))$

So $f(x)f(4t)=f(t)f(4x)$ and so $f(4x)=8af(x)$ for some $a\in\mathbb R$ (remember $f(t)\ne 0$)

$P(x,y)$ implies then new assertion $Q(x,y)$ : $f(x+f(y))=f(x)+axf(y)+f(f(y))$

Choosing $y=t$ and the appropriate $x$ in $Q(x,y)$, we immediately get that any real may be written as $f(u)-f(v)$ for some real $u,v$

$Q(f(u)-f(v),v)$ $\implies$ $f(f(u))=f(f(u)-f(v))+af(u)f(v)-af(v)^2+f(f(v))$
$Q(f(v)-f(u),u)$ $\implies$ $f(f(v))=f(f(v)-f(u))+af(v)f(u)-af(u)^2+f(f(u))$
Adding these two lines, we get $f(f(u)-f(v))+f(f(v)-f(u))=a(f(u)-f(v))^2$

And so $f(x)+f(-x)=ax^2$ $\forall x$
Using then $4x$ instead of $x$ in this equality and remembering that $f(4x)=8af(x)$, we get $a=2$ and so we now have :

$Q(x,y)$ : $f(x+f(y))=f(x)+2xf(y)+f(f(y))$
$f(4x)=16f(x)$
$f(x)+f(-x)=2x^2$

$Q(f(x),x)$ $\implies$ $f(2f(x))=2f(f(x))+2f(x)^2$
$Q(2f(x),x)$ $\implies$ $f(3f(x))=3f(f(x))+6f(x)^2$
$Q(3f(x),x)$ $\implies$ $f(4f(x))=4f(f(x))+12f(x)^2$
And since $f(4f(x))=16f(f(x))$, we get $f(f(x))=f(x)^2$

And so $Q(x,y)$ becomes new assertion $R(x,y)$ : $f(x+f(y))=f(x)+2xf(y)+f(y)^2$

$R(-f(v),v)$ $\implies$ $0=f(-f(v))-2f(v)^2+f(v)^2$ and so $f(-f(v))=f(v)^2$
$R(-f(v),u)$ $\implies$ $f(f(u)-f(v))=f(-f(v))-2f(u)f(v)+f(u)^2$ $=f(u)^2-2f(u)f(v)+f(v)^2=(f(u)-f(v))^2$

And so $f(x)=x^2$ which indeed is a solution.

\begin{bolded}Hence the solutions \end{bolded}\end{underlined}:
$f(x)=0$ $\forall x$
$f(x)=x^2$ $\forall x$
\end{solution}
*******************************************************************************
-------------------------------------------------------------------------------

\begin{problem}[Posted by \href{https://artofproblemsolving.com/community/user/93274}{khaitang}]
	Find all function $f:\mathbb{R}\to\mathbb{R}$ such that
\[f(f(x+y))=f(x+y)+f(x)f(y)-xy\]
holds for all reals $x$ and $y$.
	\flushright \href{https://artofproblemsolving.com/community/c6h416632}{(Link to AoPS)}
\end{problem}



\begin{solution}[by \href{https://artofproblemsolving.com/community/user/29428}{pco}]
	\begin{tcolorbox}Find all function $f:\mathbb{R}\rightarrow \mathbb{R}$ such that:
   $f(f(x+y))=f(x+y)+f(x)f(y)-xy$\end{tcolorbox}
Let $P(x,y)$ be the assertion $f(f(x+y))=f(x+y)+f(x)f(y)-xy$

$P(x+y,0)$ $\implies$ $f(f(x+y))=f(x+y)+f(0)f(x+y)$
Subtracting this from $P(x,y)$, we get new assertion $Q(x,y)$ : $f(0)f(x+y)=f(x)f(y)-xy$

$Q(1,1)$ $\implies$ $f(0)f(2)=f(1)^2-1$
$Q(x,1)$ $\implies$ $f(0)f(x+1)=f(x)f(1)-x$
$Q(x+1,1)$ $\implies$ $f(0)f(x+2)=f(x+1)f(1)-(x+1)$ $\implies$ $f(0)^2f(x+2)=f(x)f(1)^2-xf(1)-f(0)x-f(0)$
$Q(2,x)$ $\implies$ $f(0)f(x+2)=f(2)f(x)-2x$ $\implies$ $f(0)^2f(x+2)=(f(1)^2-1)f(x)-2f(0)x$

And so $f(x)f(1)^2-xf(1)-f(0)x-f(0)=(f(1)^2-1)f(x)-2f(0)x$ which implies $f(x)=x(f(1)-f(0))+f(0)$

So $f(x)=ax+b$ and plugging this in original equation, we get $a=1$ and $b=0$

Hence the solution $\boxed{f(x)=x}$ $\forall x$
\end{solution}
*******************************************************************************
-------------------------------------------------------------------------------

\begin{problem}[Posted by \href{https://artofproblemsolving.com/community/user/10045}{socrates}]
	Determine all pairs of functions $f,g:\mathbb{Q} \to\mathbb{Q}$ satisfying the following equality
\[  f(x+g(y))=g(x)+2y+f(y)\]
for all  $x,y \in \mathbb{Q}.$
	\flushright \href{https://artofproblemsolving.com/community/c6h427313}{(Link to AoPS)}
\end{problem}



\begin{solution}[by \href{https://artofproblemsolving.com/community/user/29428}{pco}]
	\begin{tcolorbox}Determine all pairs of functions $f,g:\mathbb{Q} \rightarrow \mathbb{Q}$ satisfying the following equality

\[ \displaystyle{ f(x+g(y))=g(x)+2y+f(y), }\]

for all  $x,y \in \mathbb{Q}.$\end{tcolorbox}
If $f(x)$ is a solution, then so is $f(x)+c$. So Wlog consider that $f(0)=0$
Let $P(x,y)$ be the assertion $f(x+g(y))=g(x)+2y+f(y)$

$P(-g(0),0)$ $\implies$ $g(-g(0))=0$
$P(-g(0),-g(0))$ $\implies$ $g(0)=0$
$P(x,0)$ $\implies$ $f(x)=g(x)$

So we are looking for $f(x)$ such that $f(0)=0$ and $f(x+f(y))=f(x)+2y+f(y)$
Let $Q(x,y)$ be the assertion $f(x+f(y))=f(x)+2y+f(y)$

$Q(x-f(x),x)$ $\implies$ $f(x-f(x))=-2x$ and so $f(x)$ is surjective

$Q(x,y)$ $\implies$ $f(x+f(y))=f(x)+2y+f(y)$
$Q(0,y)$ $\implies$ $f(f(y))=2y+f(y)$
Subtracting, we get $f(x+f(y))=f(x)+f(f(y))$ and, since surjective : $f(x+y)=f(x)+f(y)$

Since $f(x)$ is from $\mathbb Q\to\mathbb Q$, this immediately gives $f(x)=ax$ and, plugging this in $Q(x,y)$ : $a^2-a-2=0$

Hence the two solutions :
$f(x)=2x+c$ and $g(x)=2x$ $\forall x$ and for any real $c$, which indeed is a solution
$f(x)=-x+c$ and $g(x)=-x$ $\forall x$ and for any real $c$, which indeed is a solution
\end{solution}
*******************************************************************************
-------------------------------------------------------------------------------

\begin{problem}[Posted by \href{https://artofproblemsolving.com/community/user/118989}{Most}]
	Find all continuous functions $f: \mathbb R \to \mathbb R$ which satisfy for all $x, y \in \mathbb R$,
\[f(xy) + f(x+y) = f(xy+x) + f(y).\]
	\flushright \href{https://artofproblemsolving.com/community/c6h429068}{(Link to AoPS)}
\end{problem}



\begin{solution}[by \href{https://artofproblemsolving.com/community/user/29428}{pco}]
	[color=#f00][mod edit: also posted [url=https:\/\/artofproblemsolving.com\/community\/c6h448405p2524160]here[\/url].]
[\/color]
\begin{tcolorbox}Determine all functions f: R-> R continuous on R such that:
$f(xy) + f(x+y) = f(xy+x) + f(y)$ which x,y in R\end{tcolorbox}
$f(x)$ solution implies $f(x)+b$ solution. So Wlog say $f(0)=0$
Let $P(x,y)$ be the assertion $f(xy)+f(x+y)=f(xy+x)+f(y)$

Let $a\ge 0$
Let $x\ge 0$
Let $S=ax+x+a$

Let then the sequence $u_0=x$ and $u_{n+1}=u_n+\frac{S-u_n}{u_n+1}$

$P(\frac{S-u_n}{u_n+1},u_n)$ $\implies$ $f(u_n\frac{S-u_n}{u_n+1})+f(u_n+\frac{S-u_n}{u_n+1})=f(S-u_n)+f(u_n)$

Which may also be written : $f(u_{n+1})+f(S-u_{n+1})=f(u_n)+f(S-u_n)$

It's easy to see that $u_n\in[0,S]$ and is a increasing bounded sequence converging towards $S$

Continuity implies then $f(u_n)+f(S-u_n)=f(S)+f(0)$ and so (set $n=0$) : $f(x)+f(ax+a)=f(x+ax+a)$

So $f(x)+f(y)=f(x+y)$ $\forall x,y\ge 0$

Continuity again implies $f(x)=cx$ $\forall x\ge 0$

Let $x\ge 1$ : $P(x,-1)$ $\implies$ $f(-x)+f(x-1)=f(-1)$ and so $f(-x)=-cx+c+f(-1)$ and so $f(x)=cx+c+f(-1)$ $\forall x\le -1$

Let $y\in(-1,0)$ and $x<-\frac 1{y+1}\le -1-y$ :
$xy>0$ and so $f(xy)=cxy$
$x+y<-1$ and so $f(x+y)=f(-1)+c+cx+cy$
$xy+x<-1$ and so $f(xy+x)=f(-1)+c+cxy+cx$
$P(x,y)$ becomes then $cxy+f(-1)+c+cx+cy=f(-1)+c+cxy+cx+f(y)$ and so $f(y)=cy$ $\forall y\in (-1,0)$

Continuity at $-1$ implies $f(-1)=-c$ and the formula $f(x)=cx+c+f(-1)$ $\forall x\le -1$ becomes $f(x)=cx$ $\forall x\le -1$

Hence the solution : $\boxed{f(x)=cx+b}$ $\forall x$ and for any real $c$ (we added $b$ since the previous solution supposed $f(0)=0$)
\end{solution}
*******************************************************************************
-------------------------------------------------------------------------------

\begin{problem}[Posted by \href{https://artofproblemsolving.com/community/user/121558}{Bigwood}]
	find all the function $f$ from $\mathbb{R}$ to $\mathbb{R}$ such that 
\[ f(x)^3+f(y)^3+f(z)^3\\
=3f(xyz)+\frac{1}{2}(f(x-y)^2+f(y-z)^2+f(z-x)^2)\] 
for all $x,y,z\in\mathbb{R}$.
	\flushright \href{https://artofproblemsolving.com/community/c6h434836}{(Link to AoPS)}
\end{problem}



\begin{solution}[by \href{https://artofproblemsolving.com/community/user/29428}{pco}]
	\begin{tcolorbox}find all the function $f$ from $\mathbb{R}$ to$\mathbb{R}$ such that 
\[ f(x)^3+f(y)^3+f(z)^3\\
=3f(xyz)+\frac{1}{2}(f(x-y)^2+f(y-z)^2+f(z-x)^2)\] 
for all $x,y,z\in\mathbb{R}$.\end{tcolorbox}
Let $P(x,y,z)$ be the assertion $f(x)^3+f(y)^3+f(z)^3=3f(xyz)+\frac 12f(x-y)^2+\frac 12f(y-z)^2+\frac 12f(z-x)^2$

$P(x,0,0)$ $\implies$ $f(x)^3+2f(0)^3=3f(0)+\frac 12f(x)^2+\frac 12f(0)^2+\frac 12f(-x)^2$
$P(-x,0,0)$ $\implies$ $f(-x)^3+2f(0)^3=3f(0)+\frac 12f(-x)^2+\frac 12f(0)^2+\frac 12f(x)^2$
Subtracting, we get $f(x)^3=f(-x)^3$ and so $f(x)=f(-x)$

Then, using the fact that $f(x)$ is an even function :
$P(x,0,0)$ $\implies$ $f(x)^3+2f(0)^3=3f(0)+f(x)^2+\frac 12f(0)^2$
$P(x,x,0)$ $\implies$ $2f(x)^3+f(0)^3=3f(0)+\frac 12f(0)^2+f(x)^2$
Subtracting, we get $f(x)^3=f(0)^3$ and so $f(x)=c$ constant.

Plugging this in original equation, we get $3c^3=3c+\frac 32c^2$ $\iff$ $c(2c^2-c-2)=0$ and so $c\in\{\frac {1-\sqrt{17}}4,0,\frac{1+\sqrt{17}}4\}$

\begin{bolded}Hence the answer \end{bolded}\end{underlined}:
Only three solutions to this functional equation :
$f(x)=\frac{1-\sqrt{17}}4$ $\forall x$

$f(x)=0$ $\forall x$

$f(x)=\frac{1+\sqrt{17}}4$ $\forall x$
\end{solution}



\begin{solution}[by \href{https://artofproblemsolving.com/community/user/68025}{Pirkuliyev Rovsen}]
	Patrick you are a genius 
\end{solution}
*******************************************************************************
-------------------------------------------------------------------------------

\begin{problem}[Posted by \href{https://artofproblemsolving.com/community/user/31915}{Batominovski}]
	Find all functions $f:\mathbb{R}\to\mathbb{R}$ such that \[x^2f(x)+y^2f(y)-(x+y)f(xy)=(x-y)^2f(x+y)\] holds for every pair $(x,y)\in\mathbb{R}^2$.
	\flushright \href{https://artofproblemsolving.com/community/c6h435405}{(Link to AoPS)}
\end{problem}



\begin{solution}[by \href{https://artofproblemsolving.com/community/user/29428}{pco}]
	\begin{tcolorbox}Find all functions $f:\mathbb{R}\to\mathbb{R}$ such that \[x^2f(x)+y^2f(y)-(x+y)f(xy)=(x-y)^2f(x+y)\] holds for every pair $(x,y)\in\mathbb{R}^2$.\end{tcolorbox}
Let $P(x,y)$ be the assertion $x^2f(x)+y^2f(y)-(x+y)f(xy)=(x-y)^2f(x+y)$
Let $a=f(1)$

$P(1,0)$ $\implies$ $f(0)=0$
$P(x,-x)$ $\implies$ $x^2(f(x)+f(-x))=0$ and so $f(-x)=-f(x)$ $\forall x\ne 0$ $\implies$ $f(-x)=-f(x)$ $\forall x$


$P(x,1)$ $\implies$ $x^2f(x)+a-(x+1)f(x)=(x-1)^2f(x+1)$
$P(x+1,-1)$ $\implies$ $(x+1)^2f(x+1)-a+xf(x+1)=(x+2)^2f(x)$
Adding : $xf(x+1)=(x+1)f(x)$ and so $f(x+1)=\frac{x+1}xf(x)$ $\forall x\ne 0$

Plugging this in $P(x,1)$, we get $a=\frac 1xf(x)$ $\forall x\ne 0$ and so $f(x)=ax$ $\forall x\ne 0$ and so $f(x)=ax$ $\forall x$

And it is easy to check back that this indeed is a solution, whatever is $a$

Hence the answer : $\boxed{f(x)=ax}$ $\forall x$ and for any $a\in\mathbb R$
\end{solution}
*******************************************************************************
-------------------------------------------------------------------------------

\begin{problem}[Posted by \href{https://artofproblemsolving.com/community/user/15524}{phuong}]
	Let $f:\mathbb{R}\to\mathbb{R}$ be a continuous function which satisfies
\[f(x+\sqrt 2)\le f(x)\le f(x+1), \quad \forall x\in\mathbb{R}.\]
Prove that $f$ is a constant function.
	\flushright \href{https://artofproblemsolving.com/community/c6h435564}{(Link to AoPS)}
\end{problem}



\begin{solution}[by \href{https://artofproblemsolving.com/community/user/29428}{pco}]
	\begin{tcolorbox}Let $f:\mathbb{R}\to\mathbb{R}$ be a continuous function satified
$f(x+\sqrt 2)\le f(x)\le f(x+1),\forall x\in\mathbb{R}$
Prove that $f$ is a constant function.\end{tcolorbox}
So $f(x+m\sqrt 2)\le f(x)$ $\forall x$ and $\forall m\in\mathbb N_0$
And $f(x)\le f(x+n)$ $\forall x$ and $\forall n\in\mathbb N_0$

So $f(x+m\sqrt 2)\le f(x+n)$ $\forall x$ and $\forall m,n\in\mathbb N_0$

So $f(x+m\sqrt 2-n)\le f(x)$ $\forall x\in\mathbb R$ and $\forall m,n\in\mathbb N_0$

And since we can make $m\sqrt 2-n$ as close as we want of $y-x$ for any real $y$, continuity implies $f(y)\le f(x)$ $\forall x,y\in\mathbb R$

Hence the result.
\end{solution}
*******************************************************************************
-------------------------------------------------------------------------------

\begin{problem}[Posted by \href{https://artofproblemsolving.com/community/user/68025}{Pirkuliyev Rovsen}]
	Find all non-constant functions $f: \mathbb{Z}\to\mathbb{N}$ satisfying all of the following conditions:
a) $f(x-y)+f(y-z)+f(z-x)=3(f(x)+f(y)+f(z))-f(x+y+z)$ for all integers $x,y$, and $z$.
b) $\sum_{k=1}^{15}f(k)\leq1995$.
	\flushright \href{https://artofproblemsolving.com/community/c6h437618}{(Link to AoPS)}
\end{problem}



\begin{solution}[by \href{https://artofproblemsolving.com/community/user/29428}{pco}]
	\begin{tcolorbox}Find all non-constant functions $f: \mathbb{Z}\to\mathbb{N}$ satisfying all of the following conditions:
a)$f(x-y)+f(y-z)+f(z-x)=3(f(x)+f(y)+f(z))-f(x+y+z)$
b)$\sum_{k=1}^{15}f(k)\leq1995$\end{tcolorbox}
Setting $x=y=z=0$ in the equation, we get $f(0)=0\notin\mathbb N$ and so \begin{bolded}no solution\end{bolded}\end{underlined}
Since OP is a brand new user on this forum, I'll consider that he ignored that we use here the notation $\mathbb N$ for positive integers and that he meant $\mathbb N_0$, set of all non negative integers. If so :

Let $P(x,y,z)$ be the assertion $f(x-y)+f(y-z)+f(z-x)=3(f(x)+f(y)+f(z))-f(x+y+z)$

$P(0,0,0)$ $\implies$ $f(0)=0$
$P(x,0,0)$ $\implies$ $f(-x)=f(x)$
$P(x,-x,0)$ $\implies$ $f(2x)=4f(x)$
$P(x+1,-1,-x-1)$ $\implies$ $f(x+2)=2f(x+1)-f(x)+2f(1)$

This recurrence definition (plus $f(0)=0$) is quite classical and has simple general solution $f(x)=ax^2$

$f(x)\in\mathbb N_0$ $\forall x\in\mathbb Z$ $\implies$ $a\ge 0$
$f(x)$ non constant $\implies$ $a>0$
$\sum_{k=1}^{15}f(k)=a\sum_{k=1}^{15}k^2=1240a\le 1995$ $\implies$ $a\le 1$

\begin{bolded}Hence the unique solution of the modified problem\end{bolded}\end{underlined} : $f(x)=x^2$ $\forall x$, which indeed is a solution
\end{solution}
*******************************************************************************
-------------------------------------------------------------------------------

\begin{problem}[Posted by \href{https://artofproblemsolving.com/community/user/68025}{Pirkuliyev Rovsen}]
	Find all functions $f: \mathbb{R}\to\mathbb{R}$ satisfying the equality 
\[f(y)+f(x+f(y))=y+f(f(x)+f(f(y)))\]
for any $x, y \in \mathbb R$.
	\flushright \href{https://artofproblemsolving.com/community/c6h438121}{(Link to AoPS)}
\end{problem}



\begin{solution}[by \href{https://artofproblemsolving.com/community/user/29428}{pco}]
	\begin{tcolorbox}Find all functions $f: \mathbb{R}\to\mathbb{R}$ satisfying the equality 
$f(y)+f(x+f(y))=y+f(f(x)+f(f(y)))$\end{tcolorbox}
Let $P(x,y)$ be the assertion $f(y)+f(x+f(y))=y+f(f(x)+f(f(y)))$

$P(f(x),0)$ $\implies$ $f(0)+f(f(x)+f(0))=f(f(f(x))+f(f(0)))$
$P(f(0),x)$ $\implies$ $f(x)+f(f(x)+f(0))=x+f(f(f(x))+f(f(0)))$

Subtracting, we get $f(x)=x+f(0)$

Plugging back $f(x)=x+a$ in original equation, we get $a=0$ and the unique solution $\boxed{f(x)=x\forall x}$
\end{solution}
*******************************************************************************
-------------------------------------------------------------------------------

\begin{problem}[Posted by \href{https://artofproblemsolving.com/community/user/107185}{mymath7}]
	Find all functions $f:\mathbb{R}\rightarrow\mathbb{R}$ such that for all $x,y\in\mathbb R$,
\[f(f(y+f(x))) = f(x+y) + f(x)+y.\]
	\flushright \href{https://artofproblemsolving.com/community/c6h438139}{(Link to AoPS)}
\end{problem}



\begin{solution}[by \href{https://artofproblemsolving.com/community/user/29428}{pco}]
	\begin{tcolorbox}Find all functions $f:\mathbb{R}\rightarrow\mathbb{R}$ such that for all $x,y\in\mathbb R$,

$f(f(y+f(x))) = f(x+y) + f(x)+y$\end{tcolorbox}
Let $P(x,y)$ be the assertion $f(f(y+f(x)))=f(x+y)+f(x)+y$

$P(x,f(y))$ $\implies$ $f(f(f(x)+f(y)))=f(x+f(y))+f(x)+f(y)$
$P(y,f(x))$ $\implies$ $f(f(f(x)+f(y)))=f(y+f(x))+f(x)+f(y)$
Subtracting, we get $f(x+f(y))=f(y+f(x))$

So $f(f(x+f(y)))=f(f(y+f(x)))$
So (using $P(x,y)$ and $P(y,x)$) : $f(x+y)+f(y)+x=f(x+y)+f(x)+y$

So $f(x)-x=f(y)-y$ and so $f(x)=x+a$, which is never a solution.

\begin{bolded}So no solution for this equation\end{bolded}.
\end{solution}



\begin{solution}[by \href{https://artofproblemsolving.com/community/user/107185}{mymath7}]
	Great solution, and nice to see you again, pco  :) 

Where have you been during the past few months?
\end{solution}



\begin{solution}[by \href{https://artofproblemsolving.com/community/user/29428}{pco}]
	\begin{tcolorbox}Great solution, and nice to see you again, pco  :) 

Where have you been during the past few months?\end{tcolorbox}
Thanks :)
I was very late for my professional work :(
\end{solution}



\begin{solution}[by \href{https://artofproblemsolving.com/community/user/103150}{Djurre}]
	\begin{tcolorbox}
Subtracting, we get $f(x+f(y))=f(y+f(x))$

So $f(f(x+f(y)))=f(f(y+f(x)))$\end{tcolorbox}

\begin{tcolorbox}
So $f(x)-x=f(y)-y$ and so $f(x)=x+a$, which is never a solution.
\end{tcolorbox}

Hello pco,

I don't understand the two steps shown above, can you explain them for me?
\end{solution}



\begin{solution}[by \href{https://artofproblemsolving.com/community/user/60729}{GlassBead}]
	The first step was done by taking $f$ of both sides.

For the second step, we have that $f(x)-x=f(y)-y$ for any $x, y$. If $g(x)=f(x)-x$, this implies that $g(x)=g(y)$ for all $x, y$ thus $g(x)=a \implies f(x)=x+a$.
\end{solution}



\begin{solution}[by \href{https://artofproblemsolving.com/community/user/103150}{Djurre}]
	Thanks, I get it.
\end{solution}



\begin{solution}[by \href{https://artofproblemsolving.com/community/user/29428}{pco}]
	\begin{tcolorbox}I don't understand the two steps shown above, can you explain them for me?\end{tcolorbox}

\begin{tcolorbox}
Subtracting, we get $f(x+f(y))=f(y+f(x))$

So $f(f(x+f(y)))=f(f(y+f(x)))$\end{tcolorbox}
When $a=b$, we can freely conclude $f(a)=f(b)$ (that's true for any function)
So $f(x+f(y))=f(y+f(x))$ implies $f(f(x+f(y)))=f(f(y+f(x)))$

2)
\begin{tcolorbox}
So $f(x)-x=f(y)-y$ and so $f(x)=x+a$, which is never a solution.
\end{tcolorbox}

In $f(x)-x=f(y)-y$, true for any $x,y\in\mathbb R$, just choose $y=0$ and you get $f(x)-x=f(0)-0$ and so $f(x)=x+f(0)$
Then, just define $a=f(0)$ and you get $f(x)=x+a$

Then, just plug this value in original equation and you see that this original equation cant be true for any $x$ when $f(x)=x+a$
\end{solution}
*******************************************************************************
-------------------------------------------------------------------------------

\begin{problem}[Posted by \href{https://artofproblemsolving.com/community/user/54178}{hungnsl}]
	Find all pairs of functions $f,g: \mathbb R \to \mathbb R$ such that $f$ is strictly increasing and for all $x,y \in \mathbb R$ we have $f(xy)=g(y)f(x)+f(y)$.
	\flushright \href{https://artofproblemsolving.com/community/c6h438683}{(Link to AoPS)}
\end{problem}



\begin{solution}[by \href{https://artofproblemsolving.com/community/user/83160}{hungnguyenvn}]
	$f(x)= a - a.x^t$ and $g(x)=x^t$  . ( Cauchy's function).
\end{solution}



\begin{solution}[by \href{https://artofproblemsolving.com/community/user/54178}{hungnsl}]
	Are there some additional conditions for $a$ and $t$ ? Note that $f$ is strictly increasing
\end{solution}



\begin{solution}[by \href{https://artofproblemsolving.com/community/user/29428}{pco}]
	\begin{tcolorbox}Find all pairs of functions $f,g: R \rightarrow R$ such that $f$ is strictly increasing and for all $x,y \in R$ we have $f(xy)=g(y)f(x)+f(y)$\end{tcolorbox}
Let $P(x,y)$ be the assertion $f(xy)=g(y)f(x)+f(y)$

$f(x)$ strictly increasing implies $\exists u$ such that $f(u)\ne 0$

$P(x,u)$ $\implies$ $f(xu)=g(u)f(x)+f(u)$
$P(u,x)$ $\implies$ $f(xu)=g(x)f(u)+f(x)$
Subtracting, we get $g(x)=\frac{g(u)-1}{f(u)}f(x)+1$ and so $g(x)=af(x)+1$ for some real $a$

Plugging this in original equation, we get new assertion $Q(x,y)$ : $f(xy)=af(x)f(y)+f(x)+f(y)$

If $a=0$, we get $f(xy)=f(x)+f(y)$ but then :
$Q(1,1)$ $\implies$ $f(1)=0$
$Q(-1,-1)$ $\implies$ $f(-1)=0$
And so $f(-1)=f(1)$ which is impossible since $f(x)$ is strictly increasing

So $a\ne 0$. Let then $h(x)=af(x)+1$
$h(x)$ is strictly monotonous (increasing if $a>0$ and decreasing if $a<0$) and $Q(x,y)$ becomes $h(xy)=h(x)h(y)$
This is a well known functional equation whose only monotonous solutions are $h(x)=sign(x)|x|^t$ where $t\in\mathbb R^+$
(where $sign(x)=-1$ $\forall x<0$, $sign(0)=0$, $sign(x)=1$ $\forall x>0$)

Then $a>0$ and\begin{bolded} the solutions of original equation are\end{underlined}\end{bolded} : 
Let any $c,t\in\mathbb R^+$
$f(x)=c(sign(x)|x|^t-1)$ $\forall x$
$g(x)=sign(x)|x|^t$ $\forall x$
which inded are solutions

Notice that hungnguyenvn'solution is not well defined for $x<0$ and, if he\/she adds the condition $t\in\mathbb N$ in order to have the function fully defined, then a lot of solutions are missing.
\end{solution}
*******************************************************************************
-------------------------------------------------------------------------------

\begin{problem}[Posted by \href{https://artofproblemsolving.com/community/user/29034}{newsun}]
	Construct a bijection between $\mathbb N \times \mathbb  N \times \mathbb N$ and $\mathbb N$.
Which triple will be in location $45$ according to your bijection?
	\flushright \href{https://artofproblemsolving.com/community/c6h438891}{(Link to AoPS)}
\end{problem}



\begin{solution}[by \href{https://artofproblemsolving.com/community/user/77454}{Trumba}]
	Bijection between $\mathbb{N}^k$ and $\mathbb{N}$ for fixed $k$ can be construct uniformly: points $(x_1,...,x_k)$ of $\mathbb{N}^k$ grouped into classes $K_N$ such that $x_1+...+x_k=N, N \in \mathbb{N}$, and points of set $x_1+...+x_k=N$ sorted by coordinates.
For $k=3$:
$x_1+x_2+x_3=0 : (0;0;0)$
$x_1+x_2+x_3=1 : (1;0;0),(0;1;0),(0;0;1)$
$x_1+x_2+x_3=2 : (2;0;0),(1;1;0),(1;0;1),(0;2;0),(0;1;1),(0;0;2)$
etc...
$|K_N|$ - polynomial degree $k$. Use It.
\end{solution}



\begin{solution}[by \href{https://artofproblemsolving.com/community/user/29428}{pco}]
	\begin{tcolorbox}Construct the bijection between $N \times N \times N$ and $N$?
Which triple will be in location $45$ according to your  bijection?\end{tcolorbox}
I suppose that you mean "construct \begin{bolded}a\end{underlined}\end{bolded} bijection"

Just take a classical bijection from $%Error. "mathnn" is a bad command.
N^2\to\mathbb N$, for example : $f(m,n)=\frac{(m+n-1)(m+n-2)}2+n$

And build $g(m,n,p)=f(f(m,n),p)$

Here you get $g(m,n,p)=\frac{(\frac{(m+n-1)(m+n-2)}2+n+p-1)(\frac{(m+n-1)(m+n-2)}2+n+p-2)}2+p$

In this example, $g(1,1,9)=45$
\end{solution}



\begin{solution}[by \href{https://artofproblemsolving.com/community/user/29034}{newsun}]
	Yes. I know it. But, What is the formula do you use to find the location $45$. I know that for $k=2$, assume that $(a,b)$ ($a,b \in \mathbb{N}$) has location $45$ then we have $1+2+3+...+(a+b)+b+1=45$. What about $k=3$?
\end{solution}



\begin{solution}[by \href{https://artofproblemsolving.com/community/user/29034}{newsun}]
	\begin{tcolorbox}[quote="newsun"]Construct the bijection between $N \times N \times N$ and $N$?
Which triple will be in location $45$ according to your  bijection?\end{tcolorbox}
I suppose that you mean "construct \begin{bolded}a\end{underlined}\end{bolded} bijection"

Just take a classical bijection from $%Error. "mathnn" is a bad command.
N^2\to\mathbb N$, for example : $f(m,n)=\frac{(m+n-1)(m+n-2)}2+n$

And build $g(m,n,p)=f(f(m,n),p)$

Here you get $g(m,n,p)=\frac{(\frac{(m+n-1)(m+n-2)}2+n+p-1)(\frac{(m+n-1)(m+n-2)}2+n+p-2)}2+p$

In this example, $g(1,1,9)=45$\end{tcolorbox}
I think, your function must be $f(m,n)=\frac{(m+n-1)(m+n-2)}2+m$
\end{solution}



\begin{solution}[by \href{https://artofproblemsolving.com/community/user/29428}{pco}]
	\begin{tcolorbox} I think, your function must be $f(m,n)=\frac{(m+n-1)(m+n-2)}2+m$\end{tcolorbox}

Are you joking ?
There are infinitely many bijections from $\mathbb N^2\to \mathbb N$

$f(m,n)=\frac{(m+n-1)(m+n-2)}2+n$ is one of them
$h(m,n)=\frac{(m+n-1)(m+n-2)}2+m$ is another one

Another one  :

$\forall (m,n,p)\ne(1,1,1)$ and $\ne (1,1,9)$ : $g(m,n,p)=\frac{(\frac{(m+n-1)(m+n-2)}2+n+p-1)(\frac{(m+n-1)(m+n-2)}2+n+p-2)}2+p$
$g(1,1,9)=1$
$g(1,1,1)=45$

Hence the simpler answer : $(m,n,p)=(1,1,1)$
\end{solution}



\begin{solution}[by \href{https://artofproblemsolving.com/community/user/99639}{tuan119}]
	Hi!
What about construct a bijection between $\mathbf{Q}$ and $\mathbf{N}$? :(
\end{solution}



\begin{solution}[by \href{https://artofproblemsolving.com/community/user/29034}{newsun}]
	Sorry. I checked it. But, I see in your function, $g(1,1,1)=1$ and  $g(1,1,9)=45$. In addition, It is not easy to show it is a bijection.
\end{solution}



\begin{solution}[by \href{https://artofproblemsolving.com/community/user/29428}{pco}]
	\begin{tcolorbox}Sorry. I checked it. But, I see in your function, $g(1,1,1)=1$ and  $g(1,1,9)=45$. In addition, It is not easy to show it is a bijection.\end{tcolorbox}
It's a bijection since it is the composition of the previous one (the first I gave) with the permutation $(1,1,1)\leftrightarrow(1,1,9)$ :)
\end{solution}



\begin{solution}[by \href{https://artofproblemsolving.com/community/user/77454}{Trumba}]
	\begin{tcolorbox}Hi!
What about construct a bijection between $\mathbf{Q}$ and $\mathbf{N}$? :(\end{tcolorbox}
We already have bijection between $\mathbb{N}^2$ and $\mathbb{N}$. Arbitrary $q \in \mathbb{Q}$ is $q=\frac{m}{n} \leftrightarrow (m,n), m \in \mathbb{Z}, n \in \mathbb{N}, \gcd (m,n)=1$. Do similarly: $|m|+n=0;1;2;...$
\end{solution}



\begin{solution}[by \href{https://artofproblemsolving.com/community/user/29428}{pco}]
	I think that this thread is just a fake one.

Op seems to think that there is ONE bijection and looks for the value of $f^{-1}(45)$ with THE bijection.
This is meaningless.
It exists infinitely many bijections from $\mathbb N^2\to\mathbb N$ and we can find bijections in which $f^{-1}(45)$ can take any positive integer value we want.
\end{solution}



\begin{solution}[by \href{https://artofproblemsolving.com/community/user/99639}{tuan119}]
	\begin{tcolorbox}
We already have bijection between $\mathbb{N}^2$ and $\mathbb{N}$. Arbitrary $q \in \mathbb{Q}$ is $q=\frac{m}{n} \leftrightarrow (m,n), m \in \mathbb{Z}, n \in \mathbb{N}, \gcd (m,n)=1$. Do similarly: $|m|+n=0;1;2;...$\end{tcolorbox}

That's between $\mathbf{Q}^+$ and $\mathbf{N}$! :)
\end{solution}



\begin{solution}[by \href{https://artofproblemsolving.com/community/user/29034}{newsun}]
	I agree that it is not necessary to find $f^{-1}(45)$ but My real question is how to find the formula to find the location of any point in the Enumeration. So, the first reply (written by Trumba) is a direction I want. Please, complete your answer, Trumba!
\end{solution}



\begin{solution}[by \href{https://artofproblemsolving.com/community/user/77454}{Trumba}]
	\begin{tcolorbox}[quote="Trumba"]
We already have bijection between $\mathbb{N}^2$ and $\mathbb{N}$. Arbitrary $q \in \mathbb{Q}$ is $q=\frac{m}{n} \leftrightarrow (m,n), m \in \mathbb{Z}, n \in \mathbb{N}, \gcd (m,n)=1$. Do similarly: $|m|+n=0;1;2;...$\end{tcolorbox}

That's between $\mathbf{Q}^+$ and $\mathbf{N}$! :)\end{tcolorbox}
The meaning is clear to you? It is enough for construct, write exact formula I am too lazy :)
\end{solution}



\begin{solution}[by \href{https://artofproblemsolving.com/community/user/99639}{tuan119}]
	Uhm, "lazy", it's not good! :P
 You can write more clearly, it seems not really to convince people to read.
 :(
\end{solution}



\begin{solution}[by \href{https://artofproblemsolving.com/community/user/29428}{pco}]
	Ahhhhh ! You are looking for the formal expression of $f^{-1}(x)$ when $f(m,n)=\frac{(m+n-1)(m+n-2)}2+n$ (your formula) ?

Answer is $f^{-1}(x)=( \left\lceil\frac{\sqrt{8x+1}+1}2\right\rceil$ $+\frac 12(\left\lceil\frac{\sqrt{8x+1}+1}2\right\rceil-1)$ $(\left\lceil\frac{\sqrt{8x+1}+1}2\right\rceil-2)-x$ $,x-\frac 12(\left\lceil\frac{\sqrt{8x+1}+1}2\right\rceil-1)$ $(\left\lceil\frac{\sqrt{8x+1}+1}2\right\rceil-2)$
\end{solution}



\begin{solution}[by \href{https://artofproblemsolving.com/community/user/29034}{newsun}]
	\begin{tcolorbox}Ahhhhh ! You are looking for the formal expression of $f^{-1}(x)$ when $f(m,n)=\frac{(m+n-1)(m+n-2)}2+n$ (your formula) ?

Answer is $f^{-1}(x)=( \left\lceil\frac{\sqrt{8x+1}+1}2\right\rceil$ $+\frac 12(\left\lceil\frac{\sqrt{8x+1}+1}2\right\rceil-1)$ $(\left\lceil\frac{\sqrt{8x+1}+1}2\right\rceil-2)-x$ $,x-\frac 12(\left\lceil\frac{\sqrt{8x+1}+1}2\right\rceil-1)$ $(\left\lceil\frac{\sqrt{8x+1}+1}2\right\rceil-2)$\end{tcolorbox}
I know this ideal may be used to prove $f(m,n)$ is surjective ( It is a main part in proof $f(m,n)$ is bijective). But, I am confused with $g(m,n,p)$ is a bijection. Since $ A \times B \times C $ and $(A \times B) \times C $ are not the same.
\end{solution}



\begin{solution}[by \href{https://artofproblemsolving.com/community/user/29428}{pco}]
	\begin{tcolorbox}[quote="pco"]Ahhhhh ! You are looking for the formal expression of $f^{-1}(x)$ when $f(m,n)=\frac{(m+n-1)(m+n-2)}2+n$ (your formula) ?

Answer is $f^{-1}(x)=( \left\lceil\frac{\sqrt{8x+1}+1}2\right\rceil$ $+\frac 12(\left\lceil\frac{\sqrt{8x+1}+1}2\right\rceil-1)$ $(\left\lceil\frac{\sqrt{8x+1}+1}2\right\rceil-2)-x$ $,x-\frac 12(\left\lceil\frac{\sqrt{8x+1}+1}2\right\rceil-1)$ $(\left\lceil\frac{\sqrt{8x+1}+1}2\right\rceil-2)$\end{tcolorbox}
I know this ideal may be used to prove $f(m,n)$ is surjective ( It is a main part in proof $f(m,n)$ is bijective). But, I am confused with $g(m,n,p)$ is a bijection. Since $ A \times B \times C $ and $(A \times B) \times C $ are not the same.\end{tcolorbox}
You are welcome, glad to have helped you.

About $ A \times B \times C $ and $(A \times B) \times C $ not being the same : maybe it could be possible to find (very very difficult, but maybe possible) a bijection between these two sets. :?:
\end{solution}



\begin{solution}[by \href{https://artofproblemsolving.com/community/user/29034}{newsun}]
	Lots of replies. Thanks everyone. especially, Pco! I will think in my mind what i am confused. Maybe why $g(m,n,p)$ in Pco's construction is a bijection ?!
\end{solution}



\begin{solution}[by \href{https://artofproblemsolving.com/community/user/29428}{pco}]
	\begin{tcolorbox}Lots of replies. Thanks everyone. especially, Pco! I will think in my mind what i am confused. Maybe why $g(m,n,p)$ in Pco's construction is a bijection ?!\end{tcolorbox}
Maybe, who knows ?

Btw, nice troll, newsun. I understood very late :)
\end{solution}
*******************************************************************************
-------------------------------------------------------------------------------

\begin{problem}[Posted by \href{https://artofproblemsolving.com/community/user/121558}{Bigwood}]
	1. Find all $f:\mathbb N \to \mathbb N$ such that
\[xf(y)+yf(x)=(xf(f(x))+yf(f(y)))f(xy), \quad \forall x,y \in \mathbb N\]
and $f$ is increasing (not necessarily strictly increasing).

2. Find all $f:\mathbb Q^+ \to \mathbb Q^+$ such that
\[xf(y)+yf(x)=(xf(f(x))+yf(f(y)))f(xy), \quad \forall x,y \in \mathbb Q^+\]
and $f$ is increasing (not necessarily strictly increasing).
	\flushright \href{https://artofproblemsolving.com/community/c6h439130}{(Link to AoPS)}
\end{problem}



\begin{solution}[by \href{https://artofproblemsolving.com/community/user/29428}{pco}]
	\begin{tcolorbox}1. Find all $f:\mathbb N \to \mathbb N$ such that
\[xf(y)+yf(x)=(xf(f(x))+yf(f(y)))f(xy), \quad \forall x,y \in \mathbb N\]
and $f$ is increasing (not necessarily strictly increasing).\end{tcolorbox}


Let $P(x,y)$ be the assertion $xf(y)+yf(x)=(xf(f(x))+yf(f(y)))f(xy)$

$P(1,1)$ $\implies$ $f(f(1))=1$ and so $f(1)\le f(f(1))=1$ (since non decreasing) and so $f(1)=1$
$P(x,1)$ $\implies$ $f(f(x))f(x)=1$ and so $f(x)=f(f(x))=1$

Hence the  unique solution : $\boxed{f(x)=1\text{    }\forall x}$
\end{solution}



\begin{solution}[by \href{https://artofproblemsolving.com/community/user/121558}{Bigwood}]
	Sorry,pro. I made an awful typo.m(_ _)m
not $\mathbb{Z}^+$, but $\mathbb{Q}^+$.
Can you construct an example which is not increasing? It is not so easy.
\end{solution}



\begin{solution}[by \href{https://artofproblemsolving.com/community/user/29428}{pco}]
	\begin{tcolorbox}Sorry,pro. I made an awful typo.m(_ _)m
not $\mathbb{Z}^+$, but $\mathbb{Q}^+$.
Can you construct an example which is not increasing? It is not so easy.\end{tcolorbox}

\begin{tcolorbox}
2. Find all $f:\mathbb Q^+ \to \mathbb Q^+$ such that
\[xf(y)+yf(x)=(xf(f(x))+yf(f(y)))f(xy), \quad \forall x,y \in \mathbb Q^+\]
and $f$ is increasing (not necessarily strictly increasing).\end{tcolorbox}


For $f(x)$ : $\mathbb Q^+\to\mathbb Q^+$ :

$P(1,1)$ $\implies$ $f(f(1))=1$
$P(x,1)$ $\implies$ $f(f(x))f(x)=f(1)$

From there we get $f(x)=\frac ax$ $\forall x\in f(\mathbb Q^+)$ and so $f(x)$ cant be non decreasing if $f(\mathbb Q^+)$ contains at least two elements.

So, any non decreasing solution must be constant and so is $f(x)=1$ $\forall x$

And, obviously, an example of non increasing solution is $f(x)=\frac ax$ for any $a\in\mathbb Q^+$
\end{solution}
*******************************************************************************
-------------------------------------------------------------------------------

\begin{problem}[Posted by \href{https://artofproblemsolving.com/community/user/109704}{dien9c}]
	Find all functions $f: \mathbb R \to \mathbb R$ which satisfy for all $x, y \in \mathbb R$,
\[ f(y-f(x))=f(x^{2002}-y)-2001yf(x). \]
	\flushright \href{https://artofproblemsolving.com/community/c6h439927}{(Link to AoPS)}
\end{problem}



\begin{solution}[by \href{https://artofproblemsolving.com/community/user/29428}{pco}]
	\begin{tcolorbox}Find all function $f:R-R$ such that 
$ f(y-f(x))=f(x^{2002}-y)-2001yf(x) $
This problem is hard and I need a nice solution.\end{tcolorbox}
Let $P(x,y)$ be the assertion $f(y-f(x))=f(x^{2002}-y)-2001yf(x)$

$P(x,\frac{f(x)+x^{2002}}2)$ $\implies$ $f(x)(f(x)+x^{2002})=0$ and so $\forall x$ : either $f(x)=0$, either $f(x)=-x^{2002}$

If $\exists a\ne 0$ such that $f(a)=0$ and $b\ne 0$ such that $f(b)=-b^{2002}$, then :

$P(a,b)$ $\implies$ $-b^{2002}=f(a^{2002}-b)$ and so $-b^{2002}=-(a^{2002}-b)^{2002}$ and :
either $-b=a^{2002}-b$, impossible
either $b=\frac 12a^{2002}$

And so, if $\exists a\ne 0$ such that $f(a)=0$, at most one $b\ne 0$ such that $f(b)=-b^{2002}$ may exist. But this is impossible (choose any other $a$).

So :
either $f(x)=0$ $\forall x$ which indeed is a solution
either $f(x)=-x^{2002}$ $\forall x$ which is not a solution.

Hence the answer : $\boxed{f(x)=0}$ $\forall x$
\end{solution}



\begin{solution}[by \href{https://artofproblemsolving.com/community/user/109704}{dien9c}]
	\begin{tcolorbox}
Let $P(x,y)$ be the assertion $f(y-f(x))=f(x^{2002}-y)-2001yf(x)$

$P(x,\frac{f(x)+x^{2002}}2)$ $\implies$ $f(x)(f(x)+x^{2002})=0$ and so $\forall x$ : either $f(x)=0$, either $f(x)=-x^{2002}$

If $\exists a\ne 0$ such that $f(a)=0$ and $b\ne 0$ such that $f(b)=-b^{2002}$, then :

$P(a,b)$ $\implies$ $-b^{2002}=f(a^{2002}-b)$ and so $-b^{2002}=-(a^{2002}-b)^{2002}$ and :
either $-b=a^{2002}-b$, impossible
either $b=\frac 12a^{2002}$

And so, if $\exists a\ne 0$ such that $f(a)=0$, at most one $b\ne 0$ such that $f(b)=-b^{2002}$ may exist. But this is impossible [color=#FF0000](choose any other $a$)[\/color].

So :
[color=#FF0000]either $f(x)=0$ $\forall x$ which indeed is a solution
either $f(x)=-x^{2002}$ $\forall x$ which is not a solution.[\/color]

Hence the answer : $\boxed{f(x)=0}$ $\forall x$\end{tcolorbox}
I don't understand at this
\end{solution}



\begin{solution}[by \href{https://artofproblemsolving.com/community/user/29428}{pco}]
	\begin{tcolorbox}[quote="pco"]
...
And so, if $\exists a\ne 0$ such that $f(a)=0$, at most one $b\ne 0$ such that $f(b)=-b^{2002}$ may exist. But this is impossible [color=#FF0000](choose any other $a$)[\/color].

So :
[color=#FF0000]either $f(x)=0$ $\forall x$ which indeed is a solution
either $f(x)=-x^{2002}$ $\forall x$ which is not a solution.[\/color]

Hence the answer : $\boxed{f(x)=0}$ $\forall x$\end{tcolorbox}
I don't understand at this\end{tcolorbox}
If $\exists a\ne 0$ such that $f(a)=0$, then at most one $b\ne 0$ is such that $f(b)=-b^{2002}$ : this $b$ is $\frac 12a^{2002}$

Choose then any $a_1\notin \{0,-a,a,b=\frac 12a^{2002}\}$ : $f(a_1)=0$ since $a_1\ne \frac 12a^{2002}$ but then $b=\frac 12a_1^{2002}\ne \frac 12a^{2002}$, impossible.


So :
Either $\not\exists a\ne 0$ such that $f(a)=0$ and so $f(x)=-x^{2002}$ $\forall x$ and a simple check shows that this is not a solution.

Either $\exists a\ne 0$ such that $f(a)=0$ but then $\not\exists b\ne 0$ such that $f(b)=-b^{2002}$ and so $f(x)=0$ $\forall x$ and a simple check shows that this indeed is a solution.
\end{solution}
*******************************************************************************
-------------------------------------------------------------------------------

\begin{problem}[Posted by \href{https://artofproblemsolving.com/community/user/73986}{ZetaSelberg}]
	Find all the functions $f:\mathbb{R}\to\mathbb{R}$ with continuous derivative that satisfies the following condition: On any closed interval $[a,b]$ $(a<b)$, there are $m,n\in[a,b]$ such that 

\[f^{\prime}(m)=\min_{x\in[a,b]}f(x)\;\;\;\;\;\mbox{and}\;\;\;\;\;\;\;f^{\prime}(n)=\max_{x\in[a,b]}f(x)\]
	\flushright \href{https://artofproblemsolving.com/community/c6h439940}{(Link to AoPS)}
\end{problem}



\begin{solution}[by \href{https://artofproblemsolving.com/community/user/29428}{pco}]
	\begin{tcolorbox}Find all the functions $f:\mathbb{R}\to\mathbb{R}$ with continuous derivative that satisfies the following condition: On any closed interval $[a,b]$ $(a<b)$, there are $m,n\in[a,b]$ such that 

\[f^{\prime}(m)=\min_{x\in[a,b]}f(x)\;\;\;\;\;\mbox{and}\;\;\;\;\;\;\;f^{\prime}(n)=\max_{x\in[a,b]}f(x)\]\end{tcolorbox}
Let $u\in\mathbb R$
Let $I_k=[u-\frac 1k,u+\frac 1k]$
Let $m_k$ and $n_k$ the $(m,n)$ obtained from the given property applied to $I_k$

$f'(n_k)\ge f(u)\ge f'(m_k)$ and so, setting $k\to+\infty$ and using continuity of derivative, we get $f'(u)=f(u)$


And so $\boxed{f(x)=ce^x}$ $\forall x$ and for any real $c$, which indeed is a solution
\end{solution}
*******************************************************************************
-------------------------------------------------------------------------------

\begin{problem}[Posted by \href{https://artofproblemsolving.com/community/user/90062}{vntbqpqh234}]
	1. Find all functions $f: \mathbb R^+ \to \mathbb R^+$ which satisfy for all $x> y> 0$,
\[f(x+2y)-f(x+y)=3\left(f(y)+\sqrt{f(x) \cdot f(y)}\right).\]

2. Find all functions $f: \mathbb R^+ \to \mathbb R^+$ which satisfy for all $x> y> 0$,
\[f(x+2y)-f(x+y)=3\left(f(y)+2\sqrt{f(x)\cdot f(y)}\right).\]

3. Find all functions $f: \mathbb R^+ \to \mathbb R^+$ which satisfy for all $x> y> 0$,
\[f(x+2y)-f(x-y)=3\left(f(y)+2\sqrt{f(x)\cdot f(y)}\right).\]
	\flushright \href{https://artofproblemsolving.com/community/c6h440093}{(Link to AoPS)}
\end{problem}



\begin{solution}[by \href{https://artofproblemsolving.com/community/user/29428}{pco}]
	\begin{tcolorbox}Find all function $f: R_{+} \to R_{+}$ such that:
$f(x+2y)-f(x+y)=3(f(y)+\sqrt{f(x).f(y)})$ all $x>y>0$\end{tcolorbox}
If $f(x)$ is a solution, so is $af(x)$ for any $a>0$. So Wlog consider that $f(1)=1$

Let $P(x,y)$ be the assertion $f(x+2y)-f(x+y)=3f(y)+3\sqrt{f(x)f(y)}$ $\forall x>y>0$

1) $f(x)$ is strictly increasing over $\mathbb R^+$
===============================
Let $a,b$ such that $\frac 32b>a>b>0$. Then $2b-a>a-b>0$ and $P(2b-a,a-b)$ $\implies$ $f(a)-f(b)>0$
So $f(x)$ is stricly increasing over $(u,\frac 32u)$ $\forall u>0$
And the conclusion easily follows

2) $f(x)$ is continuous
==============
$f(x)>0$ and stricly increasing implies that $f(x)$ has a right limit $\forall x$
Then $\lim_{y\to 0^+}f(x+2y)=\lim_{y\to 0^+}f(x+y)$ and setting $y\to 0^+$ in $P(x,y)$, we get $\lim_{x\to 0+}f(x)=0$

Let $x>0$ and $\frac x2>y>0$ : $P(x-y,y)$ $\implies$ $0<f(x+y)-f(x)=3f(y)+3\sqrt{f(x-y)f(y)}$ $<3f(y)+3\sqrt{f(x)f(y)}$
Setting $y\to 0+$ in this equality, we get $\lim_{y\to 0+}f(x+y)=f(x)$

Let $x>0$ and $\frac x3>y>0$ : $P(x-2y,y)$ $\implies$ $0<f(x)-f(x-y)=3f(y)+3\sqrt{f(x-2y)f(y)}$ $<3f(y)+3\sqrt{f(x)f(y)}$
Setting $y\to 0+$ in this equality, we get $\lim_{y\to 0+}f(x-y)=f(x)$
Q.E.D

3) $f(2)=4.9202614...$
==============
Let $f(2)=a$
Continuity implies that $P(x,y)$ is true for $x=y$ too.
$P(1,1)$ $\implies$ $f(3)=a+6$
$P(2,1)$ $\implies$ $f(4)=a+9+3\sqrt a$
$P(3,1)$ $\implies$ $f(5)=a+12+3\sqrt a+3\sqrt{a+6}$
$P(4,1)$ $\implies$ $f(6)=a+15+3\sqrt a +3\sqrt{a+6}+3\sqrt{a+9+3\sqrt a}$

$P(2,2)$ $\implies$ $f(6)=f(4)+6f(2)=7a+9+3\sqrt a$

So $a$ is a root of equation $x+15+3\sqrt x +3\sqrt{x+6}+3\sqrt{x+9+3\sqrt x}$ $=7x+9+3\sqrt x$
$\iff$ $-2x+2+\sqrt{x+6}+\sqrt{x+9+3\sqrt x}=0$

And this equation has a unique real root $x\sim 4.9202614...$

4) no solution to this equation
=====================
$P(5,1)$ $\implies$ $f(7)=a+18+3\sqrt a +3\sqrt{a+6}+3\sqrt{a+9+3\sqrt a}$ $+3\sqrt{a+12+3\sqrt a+3\sqrt{a+6}}$
$P(3,2)$ $\implies$ $f(7)=4a+12+3\sqrt a+3\sqrt{a+6}+3\sqrt{a^2+6a}$

And so $2+\sqrt{a+9+3\sqrt a}$ $+\sqrt{a+12+3\sqrt a+3\sqrt{a+6}}=a+\sqrt{a^2+6a}$
And it is easy to check that the value found in 3) above does not match this equality
Q.E.D

And so \begin{bolded}no solution to this equation\end{underlined}\end{bolded}

I'm impatient to see your own solution, please, and to know from what olympiad contest does this problem come.
\end{solution}



\begin{solution}[by \href{https://artofproblemsolving.com/community/user/29428}{pco}]
	\begin{tcolorbox}S
Find all function $f: R_{+} \to R_{+}$ such that:
$f(x+2y)-f(x+y)=3(f(y)+2\sqrt{f(x).f(y)})$ all $x>y>0$\end{tcolorbox}
If $f(x)$ is a solution, so is $af(x)$ for any $a>0$. So Wlog consider that $f(1)=1$

Let $P(x,y)$ be the assertion $f(x+2y)-f(x+y)=3f(y)+6\sqrt{f(x)f(y)}$ $\forall x>y>0$

1) $f(x)$ is strictly increasing over $\mathbb R^+$
===============================
Let $a,b$ such that $\frac 32b>a>b>0$. Then $2b-a>a-b>0$ and $P(2b-a,a-b)$ $\implies$ $f(a)-f(b)>0$
So $f(x)$ is stricly increasing over $(u,\frac 32u)$ $\forall u>0$
And the conclusion easily follows

2) $f(x)$ is continuous
==============
$f(x)>0$ and stricly increasing implies that $f(x)$ has a right limit $\forall x$
Then $\lim_{y\to 0^+}f(x+2y)=\lim_{y\to 0^+}f(x+y)$ and setting $y\to 0^+$ in $P(x,y)$, we get $\lim_{x\to 0+}f(x)=0$

Let $x>0$ and $\frac x2>y>0$ : $P(x-y,y)$ $\implies$ $0<f(x+y)-f(x)=3f(y)+6\sqrt{f(x-y)f(y)}$ $<3f(y)+6\sqrt{f(x)f(y)}$
Setting $y\to 0+$ in this equality, we get $\lim_{y\to 0+}f(x+y)=f(x)$

Let $x>0$ and $\frac x3>y>0$ : $P(x-2y,y)$ $\implies$ $0<f(x)-f(x-y)=3f(y)+6\sqrt{f(x-2y)f(y)}$ $<3f(y)+6\sqrt{f(x)f(y)}$
Setting $y\to 0+$ in this equality, we get $\lim_{y\to 0+}f(x-y)=f(x)$
Q.E.D

3) $f(2)=7.3387244...$
===============
Let $f(2)=a$
Continuity implies that $P(x,y)$ is true for $x=y$ too.
$P(1,1)$ $\implies$ $f(3)=a+9$
$P(2,1)$ $\implies$ $f(4)=a+12+6\sqrt a$
$P(3,1)$ $\implies$ $f(5)=a+15+6\sqrt a+6\sqrt{a+9}$
$P(4,1)$ $\implies$ $f(6)=a+18+6\sqrt a +6\sqrt{a+9}+6\sqrt{a+12+6\sqrt a}$

$P(2,2)$ $\implies$ $f(6)=f(4)+9f(2)=10a+12+6\sqrt a$

So $a$ is a root of equation $x+18+6\sqrt x +6\sqrt{x+9}+6\sqrt{x+12+6\sqrt x}$ $=10x+12+6\sqrt x$

$\iff$ $-9x+6+6\sqrt{x+9}+6\sqrt{x+12+6\sqrt x}=0$

And this equation has a unique real root $x\sim 7.3387244...$

4) no solution to this equation
=====================
$P(5,1)$ $\implies$ $f(7)=a+18+6\sqrt a +6\sqrt{a+9}+6\sqrt{a+12+6\sqrt a}$ $+6\sqrt{a+15+6\sqrt a+6\sqrt{a+9}}$
$P(3,2)$ $\implies$ $f(7)=f(5)+3f(2)+6\sqrt{f(3)f(2)}$

And it is easy to check that the value found in 3) above does not match this equality
Q.E.D

And so \begin{bolded}no solution to this equation\end{underlined}\end{bolded}

I'm impatient to see your own solution, please, and to know from what olympiad contest does this problem come.
\end{solution}



\begin{solution}[by \href{https://artofproblemsolving.com/community/user/29428}{pco}]
	\begin{tcolorbox}$f: R_{+} \to R_{+}$ such that:
$f(x+2y)-f(x-y)=3(f(y)+2\sqrt{f(x).f(y)})$ all $x>y>0$\end{tcolorbox}
Let $P(x,y)$ be the assertion $f(x+2y)-f(x-y)=3f(y)+6\sqrt{f(x)f(y)}$ $\forall x>y>0$

1) $f(x)$ is strictly increasing over $\mathbb R^+$
===============================
Let $a>b>0$ : $P(\frac{a+2b}3,\frac{a-b}3)$ $\implies$ $f(a)-f(b)=3f(\frac{a-b}3)+6\sqrt{f(\frac{a+2b}3)f(\frac{a-b}3)}>0$
So $f(a)>f(b)$
Q.E.D.

2) $f(x)$ is continuous
==============
$f(x)>0$ and stricly increasing implies that $f(x)$ has a right limit $\forall x$
$P(x+2y,y)$ $\implies$ $f(x+4y)-f(x+y)=3f(y)+6\sqrt{f(x+2y)f(y)}$

Then $\lim_{y\to 0^+}f(x+4y)=\lim_{y\to 0^+}f(x+y)$ and setting $y\to 0^+$ in $P(x+2y,y)$, we get $\lim_{x\to 0+}f(x)=0$

Let $1>y>0$ : $P(x+y,y)$ $\implies$ $f(x+3y)-f(x)=3f(y)+6\sqrt{f(x+y)f(y)}$ $<3f(y)+6\sqrt{f(x+1)f(y)}$
Setting $y\to 0+$ in this equality, we get $\lim_{y\to 0+}f(x+3y)=f(x)$

Let $\frac x3>y>0$ : $P(x-2y,y)$ $\implies$ $f(x)-f(x-3y)=3f(y)+6\sqrt{f(x-2y)f(y)}$ $<3f(y)+6\sqrt{f(x)f(y)}$
Setting $y\to 0+$ in this equality, we get $\lim_{y\to 0+}f(x-3y)=f(x)$
Q.E.D

3) $f(nx)=n^2f(x)$
==============
Let $x>0$ and $a=f(x)$ and $b=f(2x)$
Setting $y\to x^-$ in $P(x,y)$, we get $f(3x)=9f(x)$

$P(2x,x)$ $\implies$ $f(4x)=4a+6\sqrt{ab}$
$P(3x,x)$ $\implies$ $f(5x)=21a+b$
$P(4x,x)$ $\implies$ $f(6x)=12a+6\sqrt{4a^2+6a\sqrt{ab}}$
$P(5x,x)$ $\implies$ $f(7x)=7a+6\sqrt{ab}+6\sqrt{21a^2+ab}$
$P(3x,2x)$ $\implies$ $f(7x)=a+3b+18\sqrt{ab}$

So $7a+6\sqrt{ab}+6\sqrt{21a^2+ab}=a+3b+18\sqrt{ab}$
and so, setting $z=\frac ba$ : $z+4\sqrt{z}-2-2\sqrt{21+z}=0$
It's easy to see that $LHS$ is increasing and is zero for $z=4$ and so $b=4a$ and $f(2x)=4f(x)$

Using then $f(2x)=4f(x)$, $f(3x)=9f(x)$ and $P(nx,x)$, we get with induction $f(nx)=n^2f(x)$
Q.E.D.

4) $f(x)=f(1)x^2$
===============
From $f(nx)=n^2f(x)$, we get $f(\frac xn)=\frac {f(x)}{n^2}$ and so $f(\frac pqx)=\frac {p^2}{q^2}f(x)$

So $f(x)=f(1)x^2$ $\forall x\in\mathbb Q^+$
And continuity gives the result.

5) Solution
===========
It's easy to check back that $f(x)=ax^2$ is indeed a solution.

Hence the result : $\boxed{f(x)=ax^2}$ $\forall x$ and for any real $a>0$
\end{solution}



\begin{solution}[by \href{https://artofproblemsolving.com/community/user/105288}{trenkialabautroj}]
	Pco's solutions are always clear.
\end{solution}
*******************************************************************************
-------------------------------------------------------------------------------

\begin{problem}[Posted by \href{https://artofproblemsolving.com/community/user/90286}{maschinima}]
	Suppose that $f $ satisfies $f(x+y)=f(x)+f(y) $ and that $f $ is continuous at $0 $ . Prove that $f $ is continuous at $a $ for all $a $.
	\flushright \href{https://artofproblemsolving.com/community/c6h440236}{(Link to AoPS)}
\end{problem}



\begin{solution}[by \href{https://artofproblemsolving.com/community/user/91148}{BigSams}]
	\begin{bolded}Proposition.\end{bolded} $f(x)$ is a function such that $f(x+y)=f(x)+f(y)$ $\forall x,y\in\mathbb{R}$, and $f(x)$ is continuous at $p$. Then $f(x)$ is continuous at $a$ for all $a\in\mathbb{R}$.

\begin{bolded}Proof.\end{bolded}
By the $\delta-\epsilon$ defintion of limits, and subsequently continuity, $\lim_{y\to p}{f(y)}=f(p)\iff \forall\epsilon\in\mathbb{R}^{+},\exists\delta\in\mathbb{R}^{+}:|y-p|<\delta\implies |f(y)-f(p)|<\epsilon$.
Let $a\in\mathbb{R}$ be an arbitrary constant. Let $x=y+a-p$. Note that $x$ ranges over all reals.
Then $|y-p|<\delta\implies |f(y)-f(p)|<\epsilon$ becomes $|x-a|<\delta\implies |f(x-a+p)-f(p)|<\epsilon$.
Note that $f(x+y)=f(x)+f(y)\implies f(x-a)+f(a)=f(x)$.
Then $|x-a|<\delta\implies |f(x-a+p)-f(p)|<\epsilon$ becomes $|x-a|<\delta\implies |f(x)-f(a)|<\epsilon$.
Thus, since $\forall\epsilon\in\mathbb{R}^{+},\exists\delta\in\mathbb{R}^{+}:|x-a|<\delta\implies |f(x)-f(a)|<\epsilon$, we have that $\lim_{x\to a}=f(a)$ $\forall a\in\mathbb{R}$; i.e. $f(x)$ is continuous everywhere.

\begin{bolded}Corollary.\end{bolded} The necessary result is the case where $p=0$.
\end{solution}



\begin{solution}[by \href{https://artofproblemsolving.com/community/user/29428}{pco}]
	\begin{tcolorbox}Suppose that $f $ satisfies $f(x+y)=f(x)+f(y) $ and that $f $ is continuous at $0 $ . Prove that $f $ is continuous at $a $ for all $a $.\end{tcolorbox}
$f(0)=f(0+0)=f(0)+f(0)=2f(0)$ and so $f(0)=0$
$\lim_{h\to 0}f(x+h)=\lim_{h\to 0}(f(x)+f(h))=f(x)+\lim_{h\to 0}f(h)=f(x)+f(0)=f(x)$
Q.E.D.
\end{solution}



\begin{solution}[by \href{https://artofproblemsolving.com/community/user/90286}{maschinima}]
	\begin{tcolorbox}[quote="maschinima"]Suppose that $f $ satisfies $f(x+y)=f(x)+f(y) $ and that $f $ is continuous at $0 $ . Prove that $f $ is continuous at $a $ for all $a $.\end{tcolorbox}
$f(0)=f(0+0)=f(0)+f(0)=2f(0)$ and so $f(0)=0$
$\lim_{h\to 0}f(x+h)=\lim_{h\to 0}(f(x)+f(h))=f(x)+\lim_{h\to 0}f(h)=f(x)+f(0)=f(x)$
Q.E.D.\end{tcolorbox}

I am sorry i do not understand your solution. Are you taking $a=x+h $ where $ a\in\mathbb{R} $ ? And why have you considered limit as $h\rightarrow\ 0 $ ?
\end{solution}



\begin{solution}[by \href{https://artofproblemsolving.com/community/user/73986}{ZetaSelberg}]
	What \begin{bolded}pco \end{bolded}did was to show that, for every $x\in\mathbb{R}$, $\lim_{h\to 0}f(x+h)=f(x)$, it is equivalent to the definition of continuous function. Then if you ask why he take $h\to 0$, I understand that you did not know this definition of continuous function :). This def.  can be deduced from the following "$f$ is continuous on $x_0$ if $\lim_{x\to x_0}f(x)=f(x_0)$".

Hope this clarify :)
\end{solution}
*******************************************************************************
-------------------------------------------------------------------------------

\begin{problem}[Posted by \href{https://artofproblemsolving.com/community/user/125557}{Nathanion92}]
	Let $f:\mathbb R \to \mathbb R$ be a function such that $f(1)=2$ and \[\left|xf(y)-yf(x) \right|\leq 2\] holds for all $x,y \in \mathbb{R}$. Prove that $f(x)=2x$ for all $x\in \mathbb{R}$.
	\flushright \href{https://artofproblemsolving.com/community/c6h440327}{(Link to AoPS)}
\end{problem}



\begin{solution}[by \href{https://artofproblemsolving.com/community/user/29428}{pco}]
	\begin{tcolorbox}$\displaystyle \begin{cases}
\left|xf(y)-yf(x) \right|\leq 2 ,\forall x,y \in \mathbb{R}  \\ 
 f(1)=2  \\ 
  
\end{cases}\Longrightarrow f(x)=2x,x\in \mathbb{R}$\end{tcolorbox}
Let $P(x,y)$ be the assertion $|xf(y)-yf(x)|\le 2$

$P(x,0)$ $\implies$ $|xf(0)|\le 2$ and so $f(0)=0$ (else set $x=\frac 3{f(0)}$)

Let $x>0$ : $P(x,1)$ $\implies$ $|2x-f(x)|\le 2$ $\implies$ $\left|\frac{f(x)}x-2\right|\le\frac 2x$

Setting $x\to+\infty$ in the above line, we get $\lim_{x\to+\infty}\frac{f(x)}x=2$

Let $x>0,y\ne 0$ : $P(x,y)$ $\implies$ $|xf(y)-yf(x)|\le 2$ $\implies$ $\left|\frac{f(x)}x-\frac{f(y)}y\right|\le\frac 2{x|y|}$

Setting $x\to+\infty$ in the above line, we get $\left|2-\frac{f(y)}y\right|\le 0$ and so $f(y)=2y$ $\forall y\ne 0$

And so $f(x)=2x$ $\forall x$
\end{solution}
*******************************************************************************
-------------------------------------------------------------------------------

\begin{problem}[Posted by \href{https://artofproblemsolving.com/community/user/76671}{VHCR}]
	Find all functions $f$ from positive reals to reals such that, for all positive reals $x$ and $y$, 
(i) $f(1)=2008$,
(ii) $|f(x)| \leq x^2+1004^2$, and
(iii) \[f\left(x+y+\frac{1}{x}+\frac{1}{y}\right)=f\left(x+\frac{1}{y})+f(y+\frac{1}{x}\right).\]
	\flushright \href{https://artofproblemsolving.com/community/c6h440604}{(Link to AoPS)}
\end{problem}



\begin{solution}[by \href{https://artofproblemsolving.com/community/user/29428}{pco}]
	\begin{tcolorbox}Find all functions $f$ from positive reals to reals such that, for all positive reals $x$ and $y$, 
$(i)$ $f(1)=2008$
$(ii)$ $|f(x)| \leq x^2+1004^2$
$(iii)$ $f(x+y+\frac{1}{x}+\frac{1}{y})=f(x+\frac{1}{y})+f(y+\frac{1}{x})$\end{tcolorbox}
Let $P(x,y)$ be the assertion $f(x+y+\frac 1x+\frac 1y)=f(x+\frac 1y)+f(y+\frac 1x)$

Let $u,v>0$ : the system $x+\frac 1y=u$ and $y+\frac 1x=v$ has a solution $x,y>0$ $\iff$ $uv\ge 4$ and so :

$f(x+y)=f(x)+f(y)$ $\forall x,y>0$ such that $xy\ge 4$

Using then induction, it's easy to show that $f(nx)=nf(x)$ $\forall x\ge 2,n\in\mathbb N$

Let then $x\in(0,2)$ and $n\in\mathbb N$ : $(nx)(\frac{4n}x)\ge 4$ $\implies$ $f(nx)+f(\frac{4n}x)=f(nx+\frac {4n}x)$

But $\frac 4x\ge 2$ and so $f(n\frac 4x)=nf(\frac 4x)$ and $x+\frac 4x\ge 2$ and so $f(nx+\frac {4n}x)=nf(x+\frac {4}x)$

And so $f(nx)=n(f(x+\frac {4}x)-f(\frac 4x))=nf(x)$

So $f(nx)=nf(x)$ $\forall x>0$

Let then $x,y>0$ : $\exists n\in\mathbb N$ such that $n^2xy\ge 4$ and so $f(nx)+f(ny)=f(nx+ny)$ and so $f(x+y)=f(x)+f(y)$ $\forall x,y$

So we are looking for a solution of Cauchy equation which is locally bounded $(ii)$ and so is $f(x)=ax$

$(i)$ gives $a=2008$ and it is easy to check that this indeed is a solution.

Hence the answer : $\boxed{f(x)=2008x}$
\end{solution}
*******************************************************************************
-------------------------------------------------------------------------------

\begin{problem}[Posted by \href{https://artofproblemsolving.com/community/user/121558}{Bigwood}]
	1. Determine all functions $f: \mathbb N \to \mathbb N$ such that
\[f(f(m)+n)=m+f(n)\]
for all $m,n\in\mathbb{N}$.

2. Determine all functions $f: \mathbb N \to \mathbb N$ such that
\[f(f(x)+y)=x+f(f(y))\]
for all $x,y\in\mathbb{N}$.
	\flushright \href{https://artofproblemsolving.com/community/c6h441662}{(Link to AoPS)}
\end{problem}



\begin{solution}[by \href{https://artofproblemsolving.com/community/user/29428}{pco}]
	\begin{tcolorbox}Determine all the function $N$ to $N$ such that
\[f(f(m)+n)=m+f(n)\]
for all $m,n\in\mathbb{N}$\end{tcolorbox}
Let $P(x,y)$ be the assertion $f(f(x)+y)=x+f(y)$

$P(x,f(1))$ $\implies$ $f(f(x)+f(1))=x+f(f(1))$
$P(1,f(x))$ $\implies$ $f(f(1)+f(x))=1+f(f(x))$
And so $f(f(x))=x+u$ where $u=f(f(1))-1$

$P(f(x),2)$ $\implies$ $f(x+2+u)=f(x)+f(2)$
$P(f(x+1),1)$ $\implies$ $f(x+2+u)=f(x+1)+f(1)$
And so $f(x+1)=f(x)+f(2)-f(1)$

And so $f(x)=ax+b$ for some $a=f(2)-f(1)$ and $b=2f(1)-f(2)$
Plugging this back in original equation, we get $a=1$ and $b=0$

Hence the solution : $\boxed{f(n)=n}$ $\forall n$
\end{solution}



\begin{solution}[by \href{https://artofproblemsolving.com/community/user/29428}{pco}]
	\begin{tcolorbox}Sorry.$f:\mathbb{N}$ to $\mathbb{N}$
\[f(f(x)+y)=x+f(f(y))\]\end{tcolorbox}
Let $P(x,y)$ be the assertion $f(f(x)+y))=x+f(f(y))$
$f(x)$ is injective.

$P(x,f(1))$ $\implies$ $f(f(x)+f(1)))=x+f(f(f(1)))$
$P(1,f(x))$ $\implies$ $f(f(1)+f(x)))=1+f(f(f(x)))$
And so $f(f(f(x)))=x+u$ for some $u=f(f(f(1)))-1$

$P(f(f(x)),2)$ $\implies$ $f(x+u+2))=f(f(x))+f(f(2))$
$P(f(f(x+1)),1)$ $\implies$ $f(x+u+2))=f(f(x+1))+f(f(1))$
And so $f(f(x+1))-f(f(x))=a$ for some $a=f(f(2))-f(f(1))$
And so $f(f(x))=ax+b$ for some $b=2f(f(1))-f(f(2))$ and $P(x,y)$ becomes $f(f(x)+y))=x+ay+b$

$P(x,f(1))$ $\implies$ $f(f(x)+f(1)))=x+af(1)+b$
$P(1,f(x))$ $\implies$ $f(f(1)+f(x)))=1+af(x)+b$
And so $f(x)=cx+d$ for some $c=\frac 1a$ and $d=f(1)-\frac 1a$

Plugging this back in original equation, we get $c=1$

Hence the result : $\boxed{f(x)=x+d}$ $\forall x$ and for any integer $d\ge 0$
\end{solution}



\begin{solution}[by \href{https://artofproblemsolving.com/community/user/121558}{Bigwood}]
	Ooh, your answer always go beyond my predict...
Yes, that is all right and much superior to mine. :blush:
\end{solution}



\begin{solution}[by \href{https://artofproblemsolving.com/community/user/93837}{jjax}]
	\begin{tcolorbox}Determine all the function $N$ to $N$ such that
\[f(f(m)+n)=m+f(n)\]
for all $m,n\in\mathbb{N}$\end{tcolorbox}

$f(f(f(m)+n)+p)=f(m)+n+f(p)$
$f(f(f(m)+n)+p)=f(m+p+f(n))=n+f(m+p)$
Thus, $f(m+p)=f(m)+f(p)$ for all $m,p$, and so $f(m)=am$ and a quick check shows that $a=1$.
\end{solution}
*******************************************************************************
-------------------------------------------------------------------------------

\begin{problem}[Posted by \href{https://artofproblemsolving.com/community/user/125513}{hal9v4ik}]
	Find all functions $f: \mathbb R^+ \to \mathbb R^+$ which satisfy for all $x, y \in \mathbb R^+$,
\[(1+yf(x))(1-yf(x+y))=1.\]
	\flushright \href{https://artofproblemsolving.com/community/c6h441866}{(Link to AoPS)}
\end{problem}



\begin{solution}[by \href{https://artofproblemsolving.com/community/user/29428}{pco}]
	\begin{tcolorbox}f:R+ → R+
(1+yf(x))(1-yf(x+y))=1 for all x,y in positive real set
Find all functions\end{tcolorbox}
Let $P(x,y)$ be the assertion $(1+yf(x))(1-yf(x+y))=1$

Let $x,y>0$ and $x\ne y$. Wlog say $y>x$ : $P(x,y-x)$ $\implies$ $x-\frac 1{f(x)}=y-\frac 1{f(y)}$

And so $x-\frac 1{f(x)}$ is a constant function and so $f(x)=\frac 1{x+a}$ which indeed is a solution whenever $a\ge 0$

Hence the answer : $\boxed{f(x)=\frac 1{x+a}}$ $\forall x$ and for any $a\ge 0$
\end{solution}



\begin{solution}[by \href{https://artofproblemsolving.com/community/user/111941}{Rudin}]
	\begin{tcolorbox}
Let $P(x,y)$ be the assertion $(1+yf(x))(1-yf(x+y))=1$

Let $x,y>0$ and $x\ne y$. Wlog say $y>x$ .

\end{tcolorbox}
I apologise for my ignorance but why can we assume $y>x$?P(x) is not symmetric here, I guess.Or am I missing something?   :oops: 
\begin{tcolorbox} $P(x,y-x)$ $\implies$ $x-\frac 1{f(x)}=y-\frac 1{f(y)}$

And so $x-\frac 1{f(x)}$ is a constant function and so $f(x)=\frac 1{x+a}$ \end{tcolorbox}
Can you please explain that?I am trouble following your argument,especially the part about it being  a constant function.In general, when can we conclude that a particular function is constant?
Thank you. :)
\end{solution}



\begin{solution}[by \href{https://artofproblemsolving.com/community/user/29428}{pco}]
	1) If $x\ne y$, either $x>y$, either $y>x$. 
If $y>x$, follow the proof. If $x>y$, just rename $x$ in $y$ and $y$ in $x$ and follow the proof.

2) Let $g(x)=x-\frac 1{f(x)}$. We proved that $g(x)=g(y)$ $\forall x,y$. This is the definition of a constant function.
\end{solution}
*******************************************************************************
-------------------------------------------------------------------------------

\begin{problem}[Posted by \href{https://artofproblemsolving.com/community/user/51901}{KittyOK}]
	In each of the following problems, find all continuous functions $f,g:\mathbb{R}\to \mathbb{R}$ satisfying the equation.

(a)	$f(x-y)=f(x)f(y)+g(x)g(y)$

(b)	$f(x+y)=f(x)f(y)-g(x)g(y)$

(c)	$g(x-y)=g(x)f(y)-f(x)g(y)$

(d)	$g(x+y)=g(x)f(y)+f(x)g(y)$

Note that these are individual problems, not a single system of equations.
	\flushright \href{https://artofproblemsolving.com/community/c6h442145}{(Link to AoPS)}
\end{problem}



\begin{solution}[by \href{https://artofproblemsolving.com/community/user/29428}{pco}]
	\begin{tcolorbox}In each of the following problems, find all continuous functions $f,g:\mathbb{R}\to \mathbb{R}$ satisfying the equation.
(a)	$f(x-y)=f(x)f(y)+g(x)g(y)$\end{tcolorbox}
Let $P(x,y)$ be the assertion $f(x-y)=f(x)f(y)+g(x)g(y)$

Using $P(x,x)$ and remembering that $f,g$ are continuous, we immediately get :
If $f(x)=c_1$ is constant, then $g(x)=c_2$ constant with $(c_1-\frac 12)^2+c_2^2=\frac 14$ 
If $g(x)=c_2$ is constant, then $f(x)=c_1$ constant with $(c_1-\frac 12)^2+c_2^2=\frac 14$ 
So we'll from now only look for non constant $f,g$


1) $g(0)=0$
=======
If $g(0)\ne 0$ : 
$P(x,0)$ $\implies$ $g(x)=f(x)\frac{1-f(0)}{g(0)}=cf(x)$ for some $c$
Then $P(x,x)$ $\implies$ $f(x)^2=\frac{f(0)}{c^2+1}$ and so $f(x)=d$ constant (remember $f(x)$ is continuous), impossible.
Q.E.D

2) $f(0)=1$
===========
If $f(0)\ne 1$, $P(x,0)$ $\implies$ $f(x)=f(x)f(0)$ $\implies$ $f(x)=0$ $\forall x$ and so is constant, impossible
So $f(0)=1$

3) $f(x)$ is even and $g(x)$ is odd
=====================
$P(0,x)$ $\implies$ $f(-x)=f(x)$ an so $f(x)$ is even

$P(x,x)$ $\implies$ $f(x)^2+g(x)^2=1$
$P(-x,-x)$ $\implies$ $f(x)^2+g(-x)^2=1$
And so $g(-x)^2=g(x)^2$
Suppose now $\exists u,v$ such that $g(-u)=g(u)$ and $g(-v)=-g(v)$
$P(u,v)$ $\implies$ $f(u-v)=f(u)f(v)+g(u)g(v)$
$P(-u,-v)$ $\implies$ $f(u-v)=f(u)f(v)-g(u)g(v)$
And so $g(u)g(v)=0$ and so :
either $g(u)=0$ and so $g(-u)=-g(u)$ and so both $u,v$ are such that $g(-x)=-g(x)$
either $g(v)=0$ and so $g(-v)=g(v)$ and so both $u,v$ are such that $g(-x)=g(x)$
And so either $g(-x)=g(x)$ $\forall x$, either $g(-x)=-g(x)$ $\forall x$

But if $g(x)$ is even : $P(\frac x2,-\frac x2)$ $\implies$ $f(x)=f(\frac x2)^2+g(\frac x2)^2=1$ and $f(x)$ is constant, impossible
So $g(x)$ is odd
Q.E.D.

4) $\exists a>0$ such that $f(a)=0$ and $1>f(x)>0$ $\forall x\in (0,a)$
=============================================
$P(x,-x)$ $\implies$ $f(2x)=f(x)^2-g(x)^2$ and so $f(2x)=2f(x)^2-1$
It's immediate to show that the sequence $u_{n+1}=2u_n^2-1$ with $u_0\in (0,1)$ always has some negative elements.

Since $f(0)=1$ and $f(x)$ is even and non constant, $\exists w>0$ such that $f(w)\ne 1$
$f(0)=1$ and $f(w)\ne 1$ and $f(x)$ continuous implies $\exists u_0>0$ such that $f(u_0)\in(0,1)$ and so $f(2^ku_0)<0$ for some $k$

So $\exists v>0$ such that $f(v)=0$ and so $A=\{x>0$ such that $f(x)=0\}$ is non empty and $\exists a=\inf(A)$
Continuity implies $f(a)=0$ and so $a>0$ and we get $f(x)>0$ $\forall x\in[0,a)$ (definition of $a$)

Suppose then $f(t)=1$ for some $t\in (0,a)$
$P(t,t)$ $\implies$ $g(t)=0$
$P(x+t,t)$ $\implies$ $f(x+t)=f(x)$
$P(\frac t2,\frac t2)$ $\implies$ $1=f(\frac t2)^2-g(\frac t2)^2=2f(\frac t2)^2-1$ and si $f(\frac t2)^2=1$ and so $f(\frac t2)=1$ since $\frac t2\in(0,a)$
So $f(t2^{-n})=1$ $\forall n$ and $f(x+t2^{-n})=f(x)$ 
So $f(x)$ is periodic with periods as small as we want and so, since continuous, is constant, which is impossible
So $f(x)\ne 1$ $\forall x\in(0,a)$
Q.E.D

5) $f(x)=\cos \frac{\pi}{2a}x$ $\forall x\in[0,a]$
===========================
$P(a,a)$ $\implies$ $g(a)=\pm 1$
Notice that $(f,g)$ solution implies $(f,-g)$ solution and so WLOG say $g(a)=1$
$f(x)\in(0,1)$ $\forall x\in(0,a)$ and so $g(x)\in(0,1)$ $\forall x\in(0,a)$

Then $P(x,-x)$ $\implies$ $g(x)=\sqrt{1-f(x)^2}$ $\forall x\in[0,a]$ since $g(x)\ge 0$ $\forall x\in[0,a]$

$P(x,y)$ $\implies$ $f(x+y)=f(x)f(y)-\sqrt{1-f(x)^2}\sqrt{1-f(y)^2}$ $\forall x,y\in[0,a]$
$P(\frac x2,-\frac x2)$ $\implies$ $f(\frac x2)=\sqrt{\frac{1+f(x)}2}$ $\forall x\in[0,2a]$ since $f(x)\ge 0$ $\forall x\in[0,a]$

So knowledge of $f(x),f(y)$ for $x,y\in[0,a]$ gives full knowledge of $f(\frac {x+y}2)$

$f(0)=\cos \frac{\pi}{2a}x$ and $f(a)=\cos \frac{\pi}{2a}a$ implies then $f(\frac a2)=\cos \frac{\pi}{2a}\frac a2$
...
It's very easy then to show with induction that $f(\frac p{2^n}a)=\cos \frac{\pi}{2a}\frac p{2^n}a$ $\forall$ non negative integers $n,p$ such that $\frac p{2^n}\in[0,1]$

And continuity gives the result

6) $f(x)=\cos \frac{\pi}{2a}x$ $\forall x$
========================
$f(x)$ is even and so $f(x)=\cos \frac{\pi}{2a}x$ $\forall x\in[-a,a]$

$P(x+a,a)$ $\implies$ $f(x)=g(x+a)$
$P(-x-a,a)$ $\implies$ $f(x+2a)=-g(x+a)$
Adding, we get $f(x+2a)=-f(x)$

And so knowledge of $f(x)$ over $[-a,+a]$ gives knowledge of $f(x)$ over $\mathbb R$
Q.E.D

Notice that, using $g(a)=+1$ and $P(x,a)$, this gives $g(x)=\sin \frac{\pi}{2a}x$ $\forall x$
And so another solution $g(x)=-\sin \frac{\pi}{2a}x$ $\forall x$

7) Synthesis of solutions
=================

$f(x)=c_1$ $\forall x$ and $g(x)=c_2$ $\forall x$ with $(c_1-\frac 12)^2+c_2^2=\frac 14$ 

$f(x)=\cos \frac{\pi}{2a}x$ $\forall x$ and $g(x)=\sin \frac{\pi}{2a}x$ $\forall x$ where $a>0$

$f(x)=\cos \frac{\pi}{2a}x$ $\forall x$ and $g(x)=-\sin \frac{\pi}{2a}x$ $\forall x$ where $a>0$
\end{solution}



\begin{solution}[by \href{https://artofproblemsolving.com/community/user/29034}{newsun}]
	I like any solutions in detail like this. Very nice, Pco! :)
\end{solution}



\begin{solution}[by \href{https://artofproblemsolving.com/community/user/51901}{KittyOK}]
	Do you have solutions to other equations?
\end{solution}
*******************************************************************************
-------------------------------------------------------------------------------

\begin{problem}[Posted by \href{https://artofproblemsolving.com/community/user/86097}{hurricane}]
	Find all functions $f:\mathbb{Z} \rightarrow \mathbb{Z}$ such that $f(x+y+f(y))=f(x)+2y$ holds for all integers $x$ and $y$.
	\flushright \href{https://artofproblemsolving.com/community/c6h442320}{(Link to AoPS)}
\end{problem}



\begin{solution}[by \href{https://artofproblemsolving.com/community/user/91362}{goldeneagle}]
	consider $f(a)=f(b)$ then:
$f(a+b+f(b))=f(a)+2b$ and $f(b+a+f(a))=f(b)+2a$  so $a=b$ 
$p(a,b)$ means put $x=a , y=b$
put $x=y=0 \Rightarrow f(f(0))=f(0) \Rightarrow f(0)=0$ now  $p(0,x) \Rightarrow f(x+f(x))=2x$
so we have:
$p(x+f(x),y) \Rightarrow f(x+f(x)+y+f(y))=2x+2y=f(x+y+f(x+y)) \Rightarrow f(x+y)=f(x)+f(y)  $ so f$(x)=ax$

replace it we have $a=1$
\end{solution}



\begin{solution}[by \href{https://artofproblemsolving.com/community/user/29428}{pco}]
	\begin{tcolorbox} replace it we have $a=1$\end{tcolorbox}
Or $a=-2$
\end{solution}
*******************************************************************************
-------------------------------------------------------------------------------

\begin{problem}[Posted by \href{https://artofproblemsolving.com/community/user/126219}{georgijgeorge}]
	Find all functions $f: \mathbb R \to \mathbb R$ which satisfy for all $x \in \mathbb R$,
\[f(f(x))=f(-x).\]
	\flushright \href{https://artofproblemsolving.com/community/c6h442519}{(Link to AoPS)}
\end{problem}



\begin{solution}[by \href{https://artofproblemsolving.com/community/user/29428}{pco}]
	\begin{tcolorbox}Find all functions f:R->R such a f(f(x))=f(-x)\end{tcolorbox}
Is it a real olympiad exercise you got in some contest or training session ? Could you give us the source ?

There are infinitely many solutions to this functional equation and I would be surprised if it was possible to find all of them.

For example :
1) $f(x)=c$ constant

2) let $g(x)$ any involutive function from $\mathbb R^+\to\mathbb R^+$ (so that $g(g(x))=x$) and define $f(x)$ as :
$\forall x>0$ : $f(x)=g(x)$
$\forall x\le 0$ : $f(x)=-x$

3) Choose any solution $f(x)$ and any odd bijection $g(x)$ from $\mathbb R\to\mathbb R$
then $g^{[-1]}(f(g(x)))$ is also a solution

and so on ....

So i think you should really change your teacher \/ olympiad trainer \/ friend \/ little sister \/ book ....
\end{solution}



\begin{solution}[by \href{https://artofproblemsolving.com/community/user/64716}{mavropnevma}]
	We get $f(f(f(x))) = f(f(-x)) = f(x)$. Let $f(0) = a$, then $f(a) = f(f(0)) = f(-0) = f(0) = a$, $f(f(a)) = f(a) = a$, $f(-a) = f(f(a)) = a$. So either $a=0$, with $f(0)=0$, or $a\neq 0$, with $f(0)=a$, $f(a) = f(-a) = a$.

Partition now arbitrarily all other real numbers, different from $0,a,-a$, into triplets of opposite numbers $\{x,y,z,-x,-y,-z\}$, and take $f(x) = f(z) = f(-y) = y$, $f(-x) = f(y) = f(-z) =z$. Of course, we may also take 
$\bullet$ $x=y$, and then $z=x$ and $f(x)=f(-x)=x$;
$\bullet$ $y=z$, and then $z=y$ and $f(x)=f(-x)=f(y) = f(-y) = y$;
$\bullet$ $z=x$, and then $f(x)=f(-y)=y$, $f(-x) = f(y) = x$.

And of course, we may make other combinations, taking some elements $w$ with $f(w)$ within one of the already constructed groups, like $f(w) = 0$, or $f(w)=a$, or $f(w)$ within a set $\{x,y,z,-x,-y,-z\}$ ... Probably a tedious exploration of all possibilities may come up with some complete classification of such functions.
\end{solution}



\begin{solution}[by \href{https://artofproblemsolving.com/community/user/29428}{pco}]
	\begin{tcolorbox}We get $f(f(f(x))) = f(f(-x)) = f(x)$. Let $f(0) = a$, then $f(a) = f(f(0)) = f(-0) = f(0) = a$, $f(f(a)) = f(a) = a$, $f(-a) = f(f(a)) = a$. So either $a=0$, with $f(0)=0$, or $a\neq 0$, with $f(0)=a$, $f(a) = f(-a) = a$.

Partition now all other real numbers, different from $0,a,-a$, into triplets of opposite numbers $\{x,y,z,-x,-y,-z\}$, and take $f(x) = f(z) = f(-y) = y$, $f(-x) = f(y) = f(-z) =z$.\end{tcolorbox}
Notice that this is not a general solution
\end{solution}
*******************************************************************************
-------------------------------------------------------------------------------

\begin{problem}[Posted by \href{https://artofproblemsolving.com/community/user/10045}{socrates}]
	Find all functions $f:\mathbb{R}\to \mathbb{R}$ such that 
\[f\left(\frac{x+f(x)}{2}+y+f(2z)\right)=2x-f(x)+f(y)+2f(z), \]
for all $x,y,z \in \mathbb{R}.$
	\flushright \href{https://artofproblemsolving.com/community/c6h442598}{(Link to AoPS)}
\end{problem}



\begin{solution}[by \href{https://artofproblemsolving.com/community/user/91362}{goldeneagle}]
	$P(a,b,c) $ means put $x=a,y=b,z=c$. 
now if $f(a)=f(b) $ then  $P(0,0,\frac {a}{2}), P(0,0, \frac{b}{2}) \Rightarrow f(\frac{a}{2})=f(\frac{b}{2})$ and then $P(a,\frac{b}{2},0),P(b,\frac{a}{2},0) \Rightarrow a=b $
so $f$ is injective. and $P(x,x,0)$ give us that $f$ is surjective.
$P(0,f(2x),y),P(0,f(2y),x) \Rightarrow f(2f(x))+2f(y)=f(2f(y))+2f(x)$ and because f is surjective we have $f(2x)+2y=f(2y)+2x$ or $f(x)-x=c$
replace it. we have $c=0$
\end{solution}



\begin{solution}[by \href{https://artofproblemsolving.com/community/user/29428}{pco}]
	\begin{tcolorbox}$P(0,f(2x),y),P(0,f(2y),x) \Rightarrow f(2f(x))+2f(y)=f(2f(y))+2f(x)$ \end{tcolorbox}
How ? For me $P(0,f(2x),y),P(0,f(2y),x) \Rightarrow f(f(2x))+2f(y)=f(f(2y))+2f(x)$
\end{solution}



\begin{solution}[by \href{https://artofproblemsolving.com/community/user/91362}{goldeneagle}]
	\begin{tcolorbox}[quote="goldeneagle"]$P(0,f(2x),y),P(0,f(2y),x) \Rightarrow f(2f(x))+2f(y)=f(2f(y))+2f(x)$ \end{tcolorbox}
How ? For me $P(0,f(2x),y),P(0,f(2y),x) \Rightarrow f(f(2x))+2f(y)=f(f(2y))+2f(x)$\end{tcolorbox}

yeah! you are right! i made a bad mistake!  :oops: 
but here is my complete solution: 
i have proved $f$ is surjective and injective. define $f(0)=t$ and $f(a)=0$
$P(0,0,a) \Rightarrow a=\frac{t}{2}+f(2a) $
$P(0,a,a) \Rightarrow f( \frac {t}{2} + a+f(2a)) =-t  \Rightarrow f(2a)=-t$ so $t=-2a, f(2a)=2a$
$P(2a,a,a) \Rightarrow f(5a)=2a \Rightarrow 5a=2a \Rightarrow a=0 $
$P(0,0,x) \Rightarrow f(f(2x))=2f(x)$
$P(0,x,y): f(x+f(2y))=f(x)+2f(y)=f(x)+f(f(2y)) \Rightarrow f(x+y)=f(x)+f(y)$ ($f $is surjective!)
now $f(f(2x))=2f(x)=f(2x) \Rightarrow f(x)=x$
\end{solution}
*******************************************************************************
-------------------------------------------------------------------------------

\begin{problem}[Posted by \href{https://artofproblemsolving.com/community/user/110552}{youarebad}]
	Find all functions $f : \mathbb{N} \rightarrow \mathbb{N}$, such that :
\[ f(m^2 + f(n)) = (f(m))^2 + n, \]
for every $m, n \in \mathbb{N}$
	\flushright \href{https://artofproblemsolving.com/community/c6h442803}{(Link to AoPS)}
\end{problem}



\begin{solution}[by \href{https://artofproblemsolving.com/community/user/29428}{pco}]
	\begin{tcolorbox}Find all function $f : \mathbb{N} \rightarrow \mathbb{N}$, such that :

\[ f(m^2 + f(n)) = (f(m))^2 + n, \] for every $m, n \in \mathbb{N}$\end{tcolorbox}
Let $P(x,y)$ be the assertion $f(x^2+f(y))=f(x)^2+y$

$P(f(x),1+f(z))$ $\implies$ $f(f(x)^2+f(1)^2+z)=f(f(x))^2+1+f(z)$
$P(f(1),x^2+f(z))$ $\implies$ $f(f(x)^2+f(1)^2+z)=f(f(1))^2+x^2+f(z)$
And so $f(f(x))^2=x^2+a$ where $a=f(f(1))^2-1$

So $x^2+a$ is a perfect square for any positive integer $x$ and so $a=0$ and $f(f(x))=x$

$P(1,f(x))$ $\implies$ $f(x+1)=f(x)+f(1)^2$ and so $f(x)=f(1)^2x+f(1)-f(1)^2$

Pluging this back in $f(f(x))=x$, we get $f(1)=1$ and $\boxed{f(x)=x}$ $\forall x$, which indeed is a solution
\end{solution}
*******************************************************************************
-------------------------------------------------------------------------------

\begin{problem}[Posted by \href{https://artofproblemsolving.com/community/user/110552}{youarebad}]
	Find all functions $f : \mathbb{Z} \to\mathbb{Z}$, such that
\[ f(x+y) + f(x-y) = 2(f(x) + f(y)), \quad \forall x, y \in \mathbb{Z}.\]
	\flushright \href{https://artofproblemsolving.com/community/c6h442806}{(Link to AoPS)}
\end{problem}



\begin{solution}[by \href{https://artofproblemsolving.com/community/user/29428}{pco}]
	\begin{tcolorbox}Find all function $f : \mathbb{Z} \rightarrow \mathbb{Z}$, such that :

\[ f(x+y) + f(x-y) = 2(f(x) + f(y)), \forall x, y \in \mathbb{Z} \]\end{tcolorbox}
Let $P(x,y)$ be the assertion $f(x+y)+f(x-y)=2(f(x)+f(y))$

$P(0,0)$ $\implies$ $f(0)=0$

$P(x+1,1)$ $\implies$ $f(x+2)=2f(x+1)-f(x)+2f(1)$

This is a classical linear sequence whose solution is $f(x)=f(1)x^2+ax+b$

$f(0)=0$ $\implies$ $b=0$
$f(1)=f(1)$ $\implies$ $a=0$

Hence the solution $\boxed{f(x)=cx^2}$ $\forall x$ and for any integer $c$, which indeed is a solution
\end{solution}
*******************************************************************************
-------------------------------------------------------------------------------

\begin{problem}[Posted by \href{https://artofproblemsolving.com/community/user/110552}{youarebad}]
	Find all functions $f : \mathbb{R} \to\mathbb{R}$, such that
\[ f(xf(x) + f(y)) = (f(x))^2 + y, \quad \forall x, y, \in \mathbb{R}. \]
	\flushright \href{https://artofproblemsolving.com/community/c6h442807}{(Link to AoPS)}
\end{problem}



\begin{solution}[by \href{https://artofproblemsolving.com/community/user/29428}{pco}]
	\begin{tcolorbox}Find all function $f : \mathbb{R} \rightarrow \mathbb{R}$, such that :

\[ f(xf(x) + f(y)) = (f(x))^2 + y, \forall x, y, \in \mathbb{R} \]\end{tcolorbox}
Let $P(x,y)$ be the assertion $f(xf(x)+f(y))=f(x)^2+y$

$P(0,x)$ $\implies$ $f(f(x))=x+f(0)^2$ and $f(x)$ is a bijection

So $\exists a$ such that $f(a)=0$. Then :
$P(a,a)$ $\implies$ $a=f(0)$
$P(0,a)$ $\implies$ $a=f(0)-f(0)^2$
And so $f(0)=0$

$P(0,x)$ $\implies$ $f(f(x))=x$

$P(x,0)$ $\implies$ $f(xf(x))=f(x)^2$
$P(f(x),0)$ $\implies$ $f(xf(x))=x^2$

And so $\forall x$ : either $f(x)=x$, either $f(x)=-x$

Let then $x,y$ such that $f(x)=x$ and $f(y)=-y$
$P(x,y)$ $\implies$ $f(x^2-y)=x^2+y$ but :
either $f(x^2-y)=x^2-y$ and we get $y=0$ and so both $x,y$ are such that $f(x)=x$ and $f(y)=y$
either $f(x^2-y)=-(x^2-y)$ and we get $x=0$ and so both $x,y$ are such that $f(x)=-x$ and $f(y)=-y$

And so :
either $f(x)=x$ $\forall x$ which indeed is a solution
either $f(x)=-x$ $\forall x$ which indeed is a solution

\begin{bolded}And so two solutions\end{underlined}\end{bolded} :
$f(x)=x$ $\forall x$ 
$f(x)=-x$ $\forall x$
\end{solution}
*******************************************************************************
-------------------------------------------------------------------------------

\begin{problem}[Posted by \href{https://artofproblemsolving.com/community/user/110552}{youarebad}]
	Find all functions $f : \mathbb{R} \to\mathbb{R}$, such that
\[ f(x^2 - y^2) = (x - y)(f(x) + f(y)), \quad \forall x, y \in \mathbb{R}.\]
	\flushright \href{https://artofproblemsolving.com/community/c6h442809}{(Link to AoPS)}
\end{problem}



\begin{solution}[by \href{https://artofproblemsolving.com/community/user/29428}{pco}]
	\begin{tcolorbox}Find all function $f : \mathbb{R} \rightarrow \mathbb{R}$, such that :

\[ f(x^2 - y^2) = (x - y)(f(x) + f(y)), \forall x, y \in \mathbb{R} \]\end{tcolorbox}
Let $P(x,y)$ be the assertion $f(x^2-y^2)=(x-y)(f(x)+f(y))$

$P(0,0)$ $\implies$ $f(0)=0$
$P(x,-x)$ $\implies$ $f(-x)=-f(x)$ and $f(x)$ is an odd function

$P(x,1)$ $\implies$ $f(x^2-1)=(x-1)(f(x)+f(1))$ 
$P(x,-1)$ $\implies$ $f(x^2-1)=(x+1)(f(x)-f(1))$ 
So $(x-1)(f(x)+f(1))=(x+1)(f(x)-f(1))$ and so $f(x)=xf(1)$ which indeed is a solution

Hence the answer : $\boxed{f(x)=ax}$ $\forall x$ and for any real $a$.
\end{solution}
*******************************************************************************
-------------------------------------------------------------------------------

\begin{problem}[Posted by \href{https://artofproblemsolving.com/community/user/110552}{youarebad}]
	Find all functions $f : \mathbb{R} \to\mathbb{R}$, such that
\[ xf(x) - yf(y) = (x-y)f(x+y), \quad \forall x, y \in \mathbb{R}. \]
	\flushright \href{https://artofproblemsolving.com/community/c6h442810}{(Link to AoPS)}
\end{problem}



\begin{solution}[by \href{https://artofproblemsolving.com/community/user/29428}{pco}]
	\begin{tcolorbox}Find all function $f : \mathbb{R} \rightarrow \mathbb{R}$, such that :

\[ xf(x) - yf(y) = (x-y)f(x+y), \forall x, y \in \mathbb{R} \]\end{tcolorbox}
Let $P(x,y)$ be the assertion $xf(x)-yf(y)=(x-y)f(x+y)$


$P(\frac{x+1}2,\frac{x-1}2)$ $\implies$ $\frac{x+1}2f(\frac{x+1}2)-\frac{x-1}2f(\frac{x-1}2)=f(x)$

$P(\frac{x-1}2,\frac{1-x}2)$ $\implies$ $\frac{x-1}2f(\frac{x-1}2)-\frac{1-x}2f(\frac{1-x}2)=(x-1)f(0)$

$P(\frac{1-x}2,\frac{x+1}2)$ $\implies$ $\frac{1-x}2f(\frac{1-x}2)-\frac{x+1}2f(\frac{x+1}2)=-xf(1)$

Adding these three lines, we get $f(x)+(x-1)f(0)-xf(1)=0$ and so $f(x)=x(f(1)-f(0))+f(0)$

And so $\boxed{f(x)=ax+b}$ $\forall x$ and for any real $a,b$ which indeed is a solution
\end{solution}



\begin{solution}[by \href{https://artofproblemsolving.com/community/user/109704}{dien9c}]
	Anothor function
Find all function $f : (1;+\infty)-\mathbb{R}$, such that :

\[ xf(x) - yf(y) = (x-y)f(x+y), \forall x, y \in \mathbb{R} \]
\end{solution}



\begin{solution}[by \href{https://artofproblemsolving.com/community/user/29428}{pco}]
	\begin{tcolorbox}Anothor function
... $f : (1;+\infty)$, ...\end{tcolorbox}

What does this mean ?
domain ?
codomain ?
image ?
\end{solution}



\begin{solution}[by \href{https://artofproblemsolving.com/community/user/109704}{dien9c}]
	Can you solved this problem ?
\end{solution}



\begin{solution}[by \href{https://artofproblemsolving.com/community/user/29428}{pco}]
	\begin{tcolorbox}Can you solved this problem ?\end{tcolorbox}
Can you answer my question ?
\end{solution}



\begin{solution}[by \href{https://artofproblemsolving.com/community/user/109704}{dien9c}]
	\begin{tcolorbox}[quote="dien9c"]Can you solved this problem ?\end{tcolorbox}
Can you answer my question ?\end{tcolorbox}
Ok :D
\end{solution}



\begin{solution}[by \href{https://artofproblemsolving.com/community/user/29428}{pco}]
	\begin{tcolorbox}Anothor function
Find all function $f : (1;+\infty)\to\mathbb{R}$, such that :

\[ xf(x) - yf(y) = (x-y)f(x+y), \forall x, y \in \mathbb{R} \]\end{tcolorbox}
Let $P(x,y)$ be the assertion $xf(x)-yf(y)=(x-y)f(x+y)$

For $x,y,z>1$, adding $P(x,y), P(y,z), P(z,x)$ implies $(x-y)f(x+y)+(y-z)f(y+z)+(z-x)f(z+x)=0$

Let $n>2$ and $t\in\left(2+\frac 1n,2n+\frac 1n-2\right)$.

Let $x=\frac 12(t-\frac 1n)$ : $x>1$

Let $y=\frac 12(t+\frac 1n)$. $y>1$

Let $z=n-\frac 12(t-\frac 1n)$. $z>1$

The above formula may then be used and gives $-\frac 1nf(t)+(t-n)f(n+\frac 1n)+(n-t+\frac 1n)f(n)=0$

And so $f(t)=t(nf(n+\frac 1n)-nf(n)) -n^2f(n+\frac 1n)+n^2f(n)+f(n)$

And so $f(x)=a_nx+b_n$ $\forall n>2$, $\forall x\in\left(2+\frac 1n,2n+\frac 1n-2\right)$

And so $f(x)=ax+b$ $\forall x>2$


Plugging this in original equation (since $x+y>2$) : $xf(x)-yf(y)=(x-y)(ax+ay+b)=ax^2+bx-ay^2-by$

And so $x(f(x)-ax-b)=y(f(y)-ay-b)$ and so $x(f(x)-ax-b)=c$

Setting $x>2$ here, we get $c=0$ and so $\boxed{f(x)=ax+b}$ $\forall x>1$
\end{solution}
*******************************************************************************
-------------------------------------------------------------------------------

\begin{problem}[Posted by \href{https://artofproblemsolving.com/community/user/60032}{Stephen}]
	Find all functions $f: \mathbb N \to \mathbb N$ which satisfy for all $n \in \mathbb N$,
\[f(f(n))=f(n+2)-f(n).\]
	\flushright \href{https://artofproblemsolving.com/community/c6h443227}{(Link to AoPS)}
\end{problem}



\begin{solution}[by \href{https://artofproblemsolving.com/community/user/29428}{pco}]
	\begin{tcolorbox}$f : \mathbb{N} \rightarrow \mathbb{N}$, $f(f(n))=f(n+2)-f(n)$.\end{tcolorbox}
1) $f(n)\ge\frac n2$ $\forall n$
==============
From the given equation, we get :
$f(2n+1)=f(1)+\sum_{k=1}^n f(f(2k-1))\ge n+1>\frac{2n+1}2$ and so $f(n)\ge\frac n2$ $\forall$ odd integer $\ge 3$

$f(2n+2)=f(2)+\sum_{k=1}^nf(f(2k))\ge n+1=\frac{2n+2}2$ and so $f(n)\ge\frac n2$ $\forall$ even integer $\ge 4$

And since $f(1)\ge \frac 12$ and $f(2)\ge \frac 22$, we get the required result.
Q.E.D.

2) $f(n)$ and $n$ have opposite parity $\forall n\ge 18$
=================================

$f(f(n))\ge\frac{f(n)}2\ge \frac n4$

$f(2n+1)=f(1)+\sum_{k=1}^n f(f(2k-1))\ge 1+\sum_{k=1}^n \frac{2k-1}4=\frac{n^2}4+1$

$f(2n+2)=f(2)+\sum_{k=1}^nf(f(2k))\ge 1+\sum_{k=1}^n \frac k2=\frac{n^2+n}4+1$

If $f(2p+1)=2n+1$ for some $n,p\ge 1$, the equality $f(2n+1)=f(1)+\sum_{k=1}^n f(f(2k-1))$ becomes :
$f(f(2p+1))=f(1)+\sum_{k=1}^n f(f(2k-1))$ and we need $p\ge n$ else $f(f(2p+1))$ appears both sides.
So $f(2p+1)\le 2p+1$ and so $\frac{p^2}4+1\le 2p+1$ and $p\le 8$
So $\forall p>8$ : either $f(2p+1)=1$, either $f(2p+1)$ is even
But $p>8$ $\implies$ $f(2p+1)=f(f(2p-1))+f(2p-1)>1$
So $f(2p+1)$ is even $\forall p>8$

If $f(2p+2)=2n+2$ for some $n,p\ge 1$, the equality $f(2n+2)=f(2)+\sum_{k=1}^nf(f(2k))$ becomes :
$f(f(2p+2))=f(2)+\sum_{k=1}^nf(f(2k))$ and we need $p\ge n$ else $f(f(2p+2))$ appears both sides.
So $f(2p+2)\le 2p+2$ and so $\frac{p^2+p}4+1\le 2p+2$ and $p\le 7$
So $\forall p>7$ : either $f(2p+2)=2$, either $f(2p+2)$ is odd.
But $p>7$ $\implies$ $f(2p+2)=f(f(2p))+f(2p)=f(f(2p))+f(f(2p-2))+f(2p-2)>2$
So $f(2p+2)$ is odd $\forall p>7$

Q.E.D.

3) No such function exists
=================
$f(37)$ is even and $>18$ (since $f(n)\ge \frac n2$) and so $f(f(37))$ is odd
But both $f(39)$ and $f(37)$ are even and so $f(39)-f(37)$ is even and can not be equal to $f(f(37))$ odd
So contradiction.

Q.E.D.
\end{solution}
*******************************************************************************
-------------------------------------------------------------------------------

\begin{problem}[Posted by \href{https://artofproblemsolving.com/community/user/92753}{WakeUp}]
	Let $f:\mathbb{R}\to\mathbb{R}$ be a function such that
\[f(f(x))=x^2-x+1\]
for all real numbers $x$. Determine $f(0)$.
	\flushright \href{https://artofproblemsolving.com/community/c6h443695}{(Link to AoPS)}
\end{problem}



\begin{solution}[by \href{https://artofproblemsolving.com/community/user/64716}{mavropnevma}]
	We have $f(f(0)) = f(f(1)) = 1$.
Now, $f(x^2-x+1) = f(f(f(x))) = f(x)^2 - f(x) + 1$, so $f(0)^2 - f(0) + 1 = f(1)^2 - f(1) + 1 = f(1)$.
So $f(1)^2 - 2f(1) + 1 = 0$, hence $\boxed{f(1)=1}$. It follows $f(0)(f(0)-1) = 0$.
Assume $f(0) = 0$; then $1=f(f(0)) = f(0) = 0$, absurd. So $\boxed{f(0)=1}$.
\end{solution}



\begin{solution}[by \href{https://artofproblemsolving.com/community/user/89593}{Swistak}]
	It's not end of this problem. It's necessary to show that such function exists.
But I heard that none of teams on Baltic Way showed this function and 10 out of 11 teams got 5\/5 pts.
\end{solution}



\begin{solution}[by \href{https://artofproblemsolving.com/community/user/29428}{pco}]
	\begin{tcolorbox}It's not end of this problem. It's necessary to show that such function exists.
But I heard that none of teams on Baltic Way showed this function and 10 out of 11 teams got 5\/5 pts.\end{tcolorbox}

I dont think showing the existence is demanded (the problem statement supposed existence)

But, if really required,  it's not very difficult to build infinitely many solutions.
\end{solution}



\begin{solution}[by \href{https://artofproblemsolving.com/community/user/109774}{littletush}]
	it suffices to notice the following conclusion:
if $f(x_1)=f(x_2)$,then $x_1=x_2$ or $x_1+x_2=1$.
\end{solution}



\begin{solution}[by \href{https://artofproblemsolving.com/community/user/29428}{pco}]
	\begin{tcolorbox}But, if really required,  it's not very difficult to build infinitely many solutions.\end{tcolorbox}
I was required thru pm to give some examples. Here is one construction :

Let $g(x)=x^2-x+1$. Notice that $g(1-x)=g(x)$, $f(x)\ge x$ $\forall x$ and $g(x)$ is increasing over $[\frac 12, +\infty)$

1) Defining $f(x)$ over $[\frac 12, 1)$
===================
Let $a\in(\frac 12,\frac 34)$
Let the sequence $a_n$ defined as $a_0=\frac 12$, $a_1=a$ and $a_{n+2}=g(a_n)$ $\forall n\ge 0$
$a_n$ is an increasing sequence whose limit is $1$
Let $h_0(x)$ be any bijection from $[a_0,a_1)\to [a_1,a_2)$
It's easy to build a sequence of bijections $h_n(x)$ from $[a_n,a_{n+1})\to[a_{n+1},a_{n+2})$ as $h_{n+1}(x)=g(h_n^{-1}(x))$ $\forall n\ge 0$

We can define $f(x)$ over $[\frac 12,1)$ as :
$\forall x\in [a_n,a_{n+1})$ : $f(x)=h_n(x)$
Obviously $f(f(x))=x^2-x+1$ $\forall x\in [\frac 12, 1)$

2) defining $f(x)$ for $x=1$
=================
Choose $f(1)=1$
Obviously $f(f(x))=x^2-x+1$ $\forall x\in [1,1])$

3) defining $f(x)$ over $(1,+\infty)$
====================
Let $b>1$ and $c\in(b,g(b))$

3.1) Defining $f(x)$ over $[b,+\infty)$
--------------------------------------
Let the sequence $b_n$ defined as $b_0=b$, $b_1=c$ and $b_{n+2}=g(b_n)$ $\forall n\ge 0$
$b_n$ is an increasing sequence whose limit is $+\infty$
Let $k_0(x)$ be any bijection from $[b_0,b_1)\to [b_1,b_2)$
It's easy to build a sequence of bijections $k_n(x)$ from $[b_n,b_{n+1})\to[b_{n+1},a_{b+2})$ as $k_{n+1}(x)=g(k_n^{-1}(x))$ $\forall n\ge 0$

We can define $f(x)$ over $[b,+\infty)$ as :
$\forall x\in [b_n,b_{n+1})$ : $f(x)=k_n(x)$
Obviously $f(f(x))=x^2-x+1$ $\forall x\in [b,+\infty)$

3.2) Defining $f(x)$ over $(1,b)$
--------------------------------
Let $g_{1+}$ be the restriction of $g(x)$ over $(1,+\infty)$. $g_{1+}$ is a bijection from $(1,+\infty)\to(1,+\infty)$
Let the sequence $c_n$ defined as $c_0=c$, $c_1=b$ and $c_{n+2}=g_{1+}^{-1}(c_n)$ $\forall n\ge 0$
$c_n$ is a decreasing sequence whose limit is $1$
It's easy to build a sequence of bijections $l_n(x)$ from $[c_{n+2},c_{n+1})\to[c_{n+1},c_n)$ as :
$l_0(x)=k_0^{-1}(g(x))$ $l_{n+1}(x)=l_n^{-1}(g(x))$ $\forall n\ge 0$

We can define $f(x)$ over $(1,b)$ as :
$\forall x\in [c_{n+2},c_{n+1})$ : $f(x)=l_n(x)$
Obviously $f(f(x))=x^2-x+1$ $\forall x\in (1,b)$

4) Defining $f(x)$ over $(-\infty,\frac 12)$
=========================
Just choose $f(x)=f(1-x)$
\end{solution}
*******************************************************************************
-------------------------------------------------------------------------------

\begin{problem}[Posted by \href{https://artofproblemsolving.com/community/user/90621}{Love_Math1994}]
	1. Find all functions $ f : (0,\infty) \to \mathbb R $ such that  $f(x^5+y^5)=x^4f(x)+y^4f(y)$ for all $x,y>1$.

2. Find all functions- $ f : (1,\infty) \to \mathbb R $ such that  $f(x^5+y^5)=x^4f(x)+y^4f(y)$ for all $x,y>1$.
	\flushright \href{https://artofproblemsolving.com/community/c6h444128}{(Link to AoPS)}
\end{problem}



\begin{solution}[by \href{https://artofproblemsolving.com/community/user/29428}{pco}]
	\begin{tcolorbox}1. Find all functions $ f : (0,\infty) \to \mathbb R $ such that  $f(x^5+y^5)=x^4f(x)+y^4f(y)$ for all $x,y>1$.
\end{tcolorbox}


Let $P(x,y)$ be the assertion $f(x^5+y^5)=x^4f(x)+y^4f(y)$

1) $f(x+y)=f(x)+g(y)$ $\forall x>2,y>0$
==========================
$P(x,x)$ $\implies$ $f(2x^5)=2x^4f(x)$ and so $f(x^5+y^5)=\frac{f(2x^5)+f(2y^5)}2$ and so new assertion $Q(x,y)$ : $f(x+y)=\frac{f(2x)+f(2y)}2$ $\forall x,y>1$

Let $x>2,y>0$ : 
$Q(\frac x2,y+1)$ $\implies$ $2f(\frac x2+y+1)=f(x)+f(2y+2)$
$Q(\frac{x+y}2,\frac y2+1)$ $\implies$ $2f(\frac x2+y+1)=f(x+y)+f(y+2)$

And so new assertion $R(x,y)$ : $f(x+y)=f(x)+g(y)$ $\forall x>2,y>0$ where $g(x)=f(2x+2)-f(x+2)$
Q.E.D.

2) $f(x+y)=f(x)+f(y)+c$ $\forall x>2,y>2$
=========================
Let $x,y>2$
$R(x,y)$ $\implies$ $f(x+y)=f(x)+g(y)$
$R(y,x)$ $\implies$ $f(x+y)=f(y)+g(x)$
And so (subtracting) $g(x)-f(x)=g(y)-f(y)=c$ and $g(x)=f(x)+c$ 

And new assertion $S(x,y)$ : $f(x+y)=f(x)+f(y)+c$ $\forall x>2,y>2$
Q.E.D.

3) $f(x+y)=f(x)+f(y)$ $\forall x>2,y>2$
=======================

So $f(nx)=nf(x)+(n-1)c$ and $f(2n)=nf(2)+(n-1)c$ $\forall n\in\mathbb N$
So $f(64n^5)=32n^5f(2)+(32n^5-1)c$ $\forall n\in\mathbb N$
But $P(2n,2n)$ $\implies$ $f(64n^5)=32n^4f(2n)$ $=32n^4(nf(2)+(n-1)c)=32n^5f(2)+(32n^5-32n^4)c$
So $c=0$ and $S(x,y)$ becomes $f(x+y)=f(x)+f(y)$ $\forall x,y>2$
Q.E.D.

4) $f(x)=ax$ $\forall x>2$
==============
We get then $f(nx)=nf(x)$ $\forall x>2$ and so $f(3n)=nf(3)$ $\forall n\in\mathbb N$

Let $x>2$ and $n\in\mathbb N$ : $P(x+3n,x+3n)$ $\implies$ $f(2(x+3n)^5)=2(x+3n)^4f(x+3n)$

And so $2f(x^5)+30nf(x^4)+180n^2f(x^3)+540n^3f(x^2)+810n^4f(x)+162n^5f(3)$ $=2(x^4+12nx^3+54n^2x^2+108n^3x+81n^4)(f(x)+nf(3))$

This gives a polynomial in $n$ which is zero for any $n\in\mathbb N$ and so which is the zero polynomial.
Looking then at coefficient of $n^4$, we get $810f(x)=216f(3)x+162f(x)$ and so $f(x)=\frac{f(3)}3x$

So $f(x)=ax$ $\forall x>2$
Q.E.D.

5) $f(x)=ax$ $\forall x>1$
===============
Let $x>1$ : $P(x,x)$ $\implies$ $f(2x^5)=2x^4f(x)$ and since $2x^5>2$, we get $2ax^5=2x^4f(x)$ and so $f(x)=ax$ $\forall x>1$
Q.E.D

6) solutions
=======
$f(x)=ax$ $\forall x>1$ is indeed a solution and we get :

$f(x)=$ any value $\forall x\in(0,1]$
$f(x)=ax$ $\forall x>1$ and for any $a\in\mathbb R$
\end{solution}



\begin{solution}[by \href{https://artofproblemsolving.com/community/user/29428}{pco}]
	\begin{tcolorbox}
2. Find all functions- $ f : (1,\infty) \to \mathbb R $ such that  $f(x^5+y^5)=x^4f(x)+y^4f(y)$ for all $x,y>1$.\end{tcolorbox}


Let $P(x,y)$ be the assertion $f(x^5+y^5)=x^4f(x)+y^4f(y)$

1) $f(x+y)=f(x)+g(y)$ $\forall x>2,y>0$
==========================
$P(x,x)$ $\implies$ $f(2x^5)=2x^4f(x)$ and so $f(x^5+y^5)=\frac{f(2x^5)+f(2y^5)}2$ and so new assertion $Q(x,y)$ : $f(x+y)=\frac{f(2x)+f(2y)}2$ $\forall x,y>1$

Let $x>2,y>0$ : 
$Q(\frac x2,y+1)$ $\implies$ $2f(\frac x2+y+1)=f(x)+f(2y+2)$
$Q(\frac{x+y}2,\frac y2+1)$ $\implies$ $2f(\frac x2+y+1)=f(x+y)+f(y+2)$

And so new assertion $R(x,y)$ : $f(x+y)=f(x)+g(y)$ $\forall x>2,y>0$ where $g(x)=f(2x+2)-f(x+2)$
Q.E.D.

2) $f(x+y)=f(x)+f(y)+c$ $\forall x>2,y>2$
=========================
Let $x,y>2$
$R(x,y)$ $\implies$ $f(x+y)=f(x)+g(y)$
$R(y,x)$ $\implies$ $f(x+y)=f(y)+g(x)$
And so (subtracting) $g(x)-f(x)=g(y)-f(y)=c$ and $g(x)=f(x)+c$ 

And new assertion $S(x,y)$ : $f(x+y)=f(x)+f(y)+c$ $\forall x>2,y>2$
Q.E.D.

3) $f(x+y)=f(x)+f(y)$ $\forall x>2,y>2$
=======================

So $f(nx)=nf(x)+(n-1)c$ and $f(2n)=nf(2)+(n-1)c$ $\forall n\in\mathbb N$
So $f(64n^5)=32n^5f(2)+(32n^5-1)c$ $\forall n\in\mathbb N$
But $P(2n,2n)$ $\implies$ $f(64n^5)=32n^4f(2n)$ $=32n^4(nf(2)+(n-1)c)=32n^5f(2)+(32n^5-32n^4)c$
So $c=0$ and $S(x,y)$ becomes $f(x+y)=f(x)+f(y)$ $\forall x,y>2$
Q.E.D.

4) $f(x)=ax$ $\forall x>2$
==============
We get then $f(nx)=nf(x)$ $\forall x>2$ and so $f(3n)=nf(3)$ $\forall n\in\mathbb N$

Let $x>2$ and $n\in\mathbb N$ : $P(x+3n,x+3n)$ $\implies$ $f(2(x+3n)^5)=2(x+3n)^4f(x+3n)$

And so $2f(x^5)+30nf(x^4)+180n^2f(x^3)+540n^3f(x^2)+810n^4f(x)+162n^5f(3)$ $=2(x^4+12nx^3+54n^2x^2+108n^3x+81n^4)(f(x)+nf(3))$

This gives a polynomial in $n$ which is zero for any $n\in\mathbb N$ and so which is the zero polynomial.
Looking then at coefficient of $n^4$, we get $810f(x)=216f(3)x+162f(x)$ and so $f(x)=\frac{f(3)}3x$

So $f(x)=ax$ $\forall x>2$
Q.E.D.

5) $f(x)=ax$ $\forall x>1$
===============
Let $x>1$ : $P(x,x)$ $\implies$ $f(2x^5)=2x^4f(x)$ and since $2x^5>2$, we get $2ax^5=2x^4f(x)$ and so $f(x)=ax$ $\forall x>1$
Q.E.D

6) solutions
=======
$f(x)=ax$ $\forall x>1$ and for any $a\in\mathbb R$, which indeed is a solution
\end{solution}



\begin{solution}[by \href{https://artofproblemsolving.com/community/user/83160}{hungnguyenvn}]
	Sorry, i proved that $f(x^5)=x^4f(x) $ from $f(x+y)=f(x)+f(y) $ for all $x,y$. Used $limit$ $->$ $f(x)=x^{4(k+k^2+...+k^N)}.f(x^{k^N})$ k=1\/5. And exist $limf(x)$ $x->1$, then $f(x)=kx$.  "functionKING Pco, COULD YOU PLEASE TELL ME SOME ADVISE ?THANK YOU
\end{solution}



\begin{solution}[by \href{https://artofproblemsolving.com/community/user/29428}{pco}]
	\begin{tcolorbox}Sorry, i proved that $f(x^5)=x^4f(x) $ from $f(x+y)=f(x)+f(y) $ for all $x,y$. Used $limit$ $->$ $f(x)=x^{4(k+k^2+...+k^N)}.f(x^{k^N})$ k=1\/5. And exist $limf(x)$ $x->1$, then $f(x)=kx$.  "functionKING Pco, COULD YOU PLEASE TELL ME SOME ADVISE ?THANK YOU\end{tcolorbox}

My advice : clarify your proof. I understood nothing. Sorry.

\begin{bolded}* edited \end{underlined}*\end{bolded} : It seems in fact :
1) that you proved $f(x+y)=f(x)+f(y)$ $\forall x,y>1$ : I'm interested in this proof

2) that you proved that $\lim_{x\to 1+}f(x)$ exists : I'm interested in this proof

If these two proofs are OK, I do agree that your final step is OK
\end{solution}



\begin{solution}[by \href{https://artofproblemsolving.com/community/user/83160}{hungnguyenvn}]
	$P(x,y):f(2x^5)=2x^4f(x)$ and $f(2x)=2f(x) $ (from the first of your proof ). Then $f(x^5)=x^4f(x)$, .$k=1\/5$ Determine sequence $a_1=x,a_{n+1}={a_n}^k$ , $f(a_n)={a_{n+1}}^4f(a_{n+1})$.x,because $x>1 -> $ so $a_n$ $>1$,so sequence $f(a_n)$ increasing $<0$ or decreasing $>0$ then exist $limf(a_n)$ . And $lima_n=1$ so exist $limf(x) x->1$. Because $f(a_1)=(a_2.a_3...a_N)^4.f(a_{N+1})$ 21,and  $a_2.a_3...a_N=x^k.x^{k^2}...x^{k^N}$$->k\/{1-k}=1\/4$ . so $f(a_1)=a.x$. $a=limf(a_n)$ A}F3. Sorry ,Sir.
\end{solution}



\begin{solution}[by \href{https://artofproblemsolving.com/community/user/83160}{hungnguyenvn}]
	Yes,Sir,. $2f(x+y)=f(2x)+f(2y)=f(2x+2y)$,$x,y>1$ $-> 2^5f(x^5+y^5)=f(2^5x^5)+f(2^5y^5) -> 2^5(x^4f(x)+y^4f(y))=(2x)^4f(2x)+(2y)^4f(2y) 5 -> f(2x)=2f(x)$ for all $x>1$. And$ f(2x^5)=2x^4f(x)->f(x^5)=x^4f(x) $ for all $x>1$. And, in my proof : exist $limf(a_n)$. ( thank you very much,Sir).7s49
\end{solution}



\begin{solution}[by \href{https://artofproblemsolving.com/community/user/29428}{pco}]
	\begin{tcolorbox}$P(x,y):f(2x^5)=2x^4f(x)$ and $f(2x)=2f(x) $ (from the first of your proof ). \end{tcolorbox}
Not at all. I proved $f(2x)=2f(x)$ only when $x>2$
\end{solution}



\begin{solution}[by \href{https://artofproblemsolving.com/community/user/83160}{hungnguyenvn}]
	sorry. You proved that $f(x+y)=f(x)+f(y),x,y>2$ .yes? And so $f(2x+2y)=f(2y)+f(2x),x,y>1$ . and the other hand $2f(x+y)=f(2x)+f(2y),x,y>1$ then $f(2x+2y)=2f(x+y),x,y>1.  $ $x->2^4x^5,y->2^4y^5  --> f((2x)^5)+f((2y)^5)=2f(2^4x^5+2^4y^5)=...=2^5f(x^5+y^5) -->f((2x)^5)+f((2y)^5=2^5f(x^5+y^5) -> (2x)^4f(2x)+(2y)^4f(2y)=2^5(x^4f(x)=y^4f(y)) $ for all $x,y>1$. Then $x=y -> f(2x)=2f(x) for all x>1 j$ .qp3.  yes ? d
\end{solution}



\begin{solution}[by \href{https://artofproblemsolving.com/community/user/29428}{pco}]
	\begin{tcolorbox}sorry. You proved that $f(x+y)=f(x)+f(y),x,y>2$ .yes? And so $f(2x+2y)=f(2y)+f(2x),x,y>1$ . and the other hand $2f(x+y)=f(2x)+f(2y),x,y>1$ then $f(2x+2y)=2f(x+y),x,y>1.  $ $x->2^4x^5,y->2^4y^5  --> f((2x)^5)+f((2y)^5)=2f(2^4x^5+2^4y^5)=...=2^5f(x^5+y^5) -->f((2x)^5)+f((2y)^5=2^5f(x^5+y^5) -> (2x)^4f(2x)+(2y)^4f(2y)=2^5(x^4f(x)=y^4f(y)) $ for all $x,y>1$. Then $x=y -> f(2x)=2f(x) for all x>1 j$ .qp3.  yes ? d\end{tcolorbox}
I'm sorry, but if you want some help, make some effort :
Check parenthesis (some are missing)
Avoid writing eveything in one line ; dont hesitate to insert some spaces in order we can better understand 
...
Btw, it seems you use $f(x^5)=x^4f(x)$ somewhere in your piece of proof. But you did not prove this first (it's easy if we know that $f(2x)=2f(x)$ but if you use it inside the proof of $f(2x)=2f(x)$, it's an error).

Notice too that after having proved $f(x+y)=f(x)+f(y)$ $\forall x,y>2$, I just got the final result in 5-10 lines. So if you need $30$ in order to prove first $f(2x)=2f(x)$ $\forall x>1$ then $\lim_{x\to 1+}f(x)$ exists, then $f(x)=ax$ ... I'm not sure it's of great interest :)
\end{solution}



\begin{solution}[by \href{https://artofproblemsolving.com/community/user/29428}{pco}]
	\begin{tcolorbox}... then exist $limf(a_n)$ . And $lima_n=1$ so exist $limf(x) x->1$. \end{tcolorbox}
Btw, this seems wrong too : existence of limit of $f(a_n)$ does not imply existence of limit in $1$ : you should also prove that all the sequence $f(a_n)$ have same limit.
\end{solution}
*******************************************************************************
-------------------------------------------------------------------------------

\begin{problem}[Posted by \href{https://artofproblemsolving.com/community/user/24720}{Babak}]
	1. Suppose that $f: P\to P$, where $P$ is the set of non-negative integers such that for all $x,y \in P$,
\[f(x^2+y^2)=f(x)^2 +f(y)^2.\]
Prove that either $f(x)= 0$ for all $x$ or $f$ is the identity map.

2. In the above problem replace $P$ with $\mathbb N$ (the set of positive integers). Prove that $f(10)=10$. Is it true that $f$ is the identity map?
	\flushright \href{https://artofproblemsolving.com/community/c6h444465}{(Link to AoPS)}
\end{problem}



\begin{solution}[by \href{https://artofproblemsolving.com/community/user/29428}{pco}]
	Part 2 :

Let $P(x,y)$ be the assertion $f(x^2+y^2)=f(x)^2+f(y)^2$
Let $a=f(1)$
Let $b=f(3)$

$P(1,1)$ $\implies$ $f(2)=2a^2$
$P(2,1)$ $\implies$ $f(5)=a^2(4a^2+1)$

$P(7,1)$ $\implies$ $f(50)=f(7)^2+a^2$
$P(5,5)$ $\implies$ $f(50)=2a^4(4a^2+1)^2$
And so $f(7)^2=2a^4(4a^2+1)^2-a^2$

$P(2,2)$ $\implies$ $f(8)=8a^4$

$P(1,3)$ $\implies$ $f(10)=a^2+b^2$
$P(2,3)$ $\implies$ $f(13)=4a^4+b^2$

$P(11,2)$ $\implies$ $f(125)=f(11)^2+4a^4$
$P(10,5)$ $\implies$ $f(125)=(a^2+b^2)^2+a^4(4a^2+1)^2$
And so $f(11)^2=(a^2+b^2)^2+a^4(4a^2+1)^2-4a^4$

$P(13,1)$ $\implies$ $f(170)=(4a^4+b^2)^2+a^2$
$P(11,7)$ $\implies$ $f(170)=(a^2+b^2)^2+a^4(4a^2+1)^2-4a^4+2a^4(4a^2+1)^2-a^2$

Hence the equation $(4a^4+b^2)^2+a^2=(a^2+b^2)^2+a^4(4a^2+1)^2-4a^4+2a^4(4a^2+1)^2-a^2$
$\iff$ $16a^8+8a^4b^2+a^2=2a^2b^2+48a^8+24a^6-a^2$
$\iff$ ($a\ne 0$) : $b^2(4a^2-1)=(4a^2-1)(2a^2+1)^2$
And so $b=2a^2+1$

So, we got up to now :
$f(1)=a$
$f(2)=2a^2$
$f(3)=2a^2+1$
$f(5)=4a^4+a^2$
$f(7)^2=32a^8+16a^6+2a^4-a^2$
$f(8)=8a^4$
$f(10)=4a^4+5a^2+1$
$f(11)^2=32a^8+48a^6+30a^4+10a^2+1$
$f(13)=8a^4+4a^2+1$

Then :
$P(11,3)$ $\implies$ $f(130)=f(11)^2+f(3)^2$
$P(9,7)$ $\implies$ $f(130)=f(9)^2+f(7)^2$
And so $f(9)^2=f(11)^2+f(3)^2-f(7)^2$

$P(9,2)$ $\implies$ $f(85)=f(9)^2+f(2)^2$
$P(7,6)$ $\implies$ $f(85)=f(7)^2+f(6)^2$
And so $f(6)^2=f(9)^2+f(2)^2-f(7)^2=f(11)^2+f(3)^2+f(2)^2-2f(7)^2$

So $f(6)^2=-32a^8+16a^6+34a^4+16a^2+2$
And it's easy to see that $RHS<0$ $\forall a\ge 2$ and so $a=1$

And so $f(n)=n$ $\forall n\in\{1,2,3,5,7,8,10,11,13\}$
Previous line gives $f(6)=6$

$P(8,1)$ $\implies$ $f(65)=65$
$P(7,4)$ $\implies$ $f(65)=49+f(4)^2$ and so $f(4)=4$

$P(11,3)$ $\implies$ $f(130)=130$
$P(9,7)$ $\implies$ $f(130)=f(9)^2+49$ and so $f(9)=9$

So $f(n)=n$ $\forall n\in[1,11]$

Let $n>2$ :
$P(2n+1,n-2)$ $\implies$ $f(5n^2+5)=f(2n+1)^2+f(n-2)^2$
$P(2n-1,n+2)$ $\implies$ $f(5n^2+5)=f(2n-1)^2+f(n+2)^2$
And so $f(2n+1)^2=f(2n-1)^2+f(n+2)^2-f(n-2)^2$

Let $n>4$ :
$P(2n+2,n-4)$ $\implies$ $f(5n^2+20)=f(2n+2)^2+f(n-4)^2$
$P(2n-2,n+4)$ $\implies$ $f(5n^2+20)=f(2n-2)^2+f(n+4)^2$
And so $f(2n+2)^2=f(2n-2)^2+f(n+4)^2-f(n-4)^2$

These two formulas allow recurrence definition for $f(n)$ $\forall n\ge 11$ as soon as values $f(1)\to f(9)$ are known

And since we had $f(n)=n$ $\forall n\in[1,11]$, these two formulas give thru induction $\boxed{f(n)=n}$ $\forall n\in\mathbb N$
\end{solution}



\begin{solution}[by \href{https://artofproblemsolving.com/community/user/24720}{Babak}]
	thank you. the induction for n>11 was what I couldn't find.
\end{solution}
*******************************************************************************
-------------------------------------------------------------------------------

\begin{problem}[Posted by \href{https://artofproblemsolving.com/community/user/10045}{socrates}]
	Find all functions $f:\Bbb{R}\to \Bbb{R}$ such that 
\[f(f(x+f(y)))=x+f(y)+f(x+y), \]
for all $x,y \in \Bbb{R}.$
	\flushright \href{https://artofproblemsolving.com/community/c6h444696}{(Link to AoPS)}
\end{problem}



\begin{solution}[by \href{https://artofproblemsolving.com/community/user/29428}{pco}]
	\begin{tcolorbox}Find all functions $f:\Bbb{R}\to \Bbb{R}$ such that 

\[f(f(x+f(y)))=x+f(y)+f(x+y), \]

for all $x,y \in \Bbb{R}.$\end{tcolorbox}
Let $P(x,y)$ be the assertion $f(f(x+f(y)))=x+f(y)+f(x+y)$

If $f(a)=f(b)$ for some $a,b$, then, comparing $P(x-b,a)$ and $P(x-b,b)$, we get $f(x)=f(x+a-b)$ $\forall x$

But then, comparing $P(x,y)$ and $P(x+a-b,y)$, we get $x=x+a-b$ and so $a=b$ and $f(x)$ is injective.

$P(-f(x),x)$ $\implies$ $f(f(0)))=f(x-f(x))$ and so, since injective : $f(0)=x-f(x)$ and $f(x)=x+a$, which is never a solution.

So\begin{bolded} no solution\end{underlined}\end{bolded}
\end{solution}
*******************************************************************************
-------------------------------------------------------------------------------

\begin{problem}[Posted by \href{https://artofproblemsolving.com/community/user/10045}{socrates}]
	Find all functions $f:\Bbb{R}^+\cup\{0\}\to \Bbb{R}^+\cup\{0\}$ such that 
\[f(f(x+f(y)))=2x+f(x+y), \]
for all $x,y \in \Bbb{R}^+\cup\{0\}.$
	\flushright \href{https://artofproblemsolving.com/community/c6h444697}{(Link to AoPS)}
\end{problem}



\begin{solution}[by \href{https://artofproblemsolving.com/community/user/29428}{pco}]
	\begin{tcolorbox}Find all functions $f:\Bbb{R}^+\cup\{0\}\to \Bbb{R}^+\cup\{0\}$ such that 

\[f(f(x+f(y)))=2x+f(x+y), \]

for all $x,y \in \Bbb{R}^+\cup\{0\}.$\end{tcolorbox}
Let $P(x,y)$ be the assertion $f(f(x+f(y)))=2x+f(x+y)$

If $f(a)=f(b)$ for some $a,b$, then, comparing $P(x,a)$ and $P(x,b)$, we get $f(x+a)=f(x+b)$ $\forall x$

But then, comparing $P(x+a,y)$ and $P(x+b,y)$, we get $a=b$ and so $f(x)$ is injective.

$P(0,x)$ $\implies$ $f(f(f(x)))=f(x)$ and, since injective, $f(f(x))=x$

So $P(x,0)$ becomes $x+f(0)=2x+f(x)$ and so $f(x)=f(0)-x$ which is never a solution. (since $<0$ for $x$ great enough)

So \begin{bolded}no solution\end{underlined}\end{bolded}
\end{solution}
*******************************************************************************
-------------------------------------------------------------------------------

\begin{problem}[Posted by \href{https://artofproblemsolving.com/community/user/10045}{socrates}]
	Find all functions $f:\Bbb{R}^+\to\Bbb{R}^+$ such that for all $x,y \in \Bbb{R}^+$ : 
\[f(x+y^n+f(y))=f(x), \]
where $n\in \Bbb{N}_{n\geq 2}.$
	\flushright \href{https://artofproblemsolving.com/community/c6h444698}{(Link to AoPS)}
\end{problem}



\begin{solution}[by \href{https://artofproblemsolving.com/community/user/104682}{momo1729}]
	I haven't thought about the solution, but excuse me to say that the title of the book is probably "101 problems in algebra".
\end{solution}



\begin{solution}[by \href{https://artofproblemsolving.com/community/user/10045}{socrates}]
	I mean [url=http://www.artofproblemsolving.com/Forum/viewtopic.php?t=406530]\begin{bolded}116 Problems in Algebra\end{bolded}[\/url]
\end{solution}



\begin{solution}[by \href{https://artofproblemsolving.com/community/user/82334}{bappa1971}]
	Nice!

Proof:

Let, $\exists w %Error. "nocomma" is a bad command.
, z$ such that, $\frac{f \left( w \right) + w^n}{f
\left( z \right) + z^n} \not\in \mathbb m{Q}$

Denote, $s = f \left( w \right) + w^n %Error. "nocomma" is a bad command.
, t = f \left( z \right) + z^n$

Then, $\forall \in > 0,$ $\exists a, b \in \mathbb m{N}$ such that, $0 <
\left| a s - b t \right| < \in$ ( See [url=http://www.artofproblemsolving.com/Forum/viewtopic.php?f=38&t=432389]here[\/url] )

So, $f \left( x \right) = f \left( x + b t - b t \right) = f \left( x + b t +
\left( a s - b t \right) \right) = f \left( x + b t + \in \right) = f \left( x
+ \in \right)$

Which implies, $\lim_{\in \longrightarrow 0} f \left( x + \in \right) = f
\left( x \right)$, so $f$ in continious

Now, if $f \left( x \right) + x^n = c$ for some constant $c$
then, for large $x$, $f \left( x \right) = c - x^n < 0$

So, take, $u = \liminf_{x \longrightarrow \infty} \left( f(x)+x^n \right)$ and $v = \limsup_{x \longrightarrow \infty} \left(  f(x)+x^n \right)$,
we have $v = \infty$

So, continiuty of $f$ implies $\forall x \geq u$, $\exists j$ such that, $x =
f \left( j \right) + j^n$

Hence, $f \left( x \right) = f \left( x + k \right)$ for all $k > u$

Now, take arbitary $x %Error. "nocomma" is a bad command.
, y$ and then take $z$ such that $z > \max
\left( x, y \right) + u$

Then, $f \left( x \right) = f \left( z \right) = f \left( y \right)$

So, $f$ is constant.



Now let, $\frac{f \left( b \right) + b^n}{f \left( a \right) + a^n} \in
\mathbb m{Q}$ for all $a, b$

Take $r = f \left( 1 \right) + 1 > 1$ and $g : \mathbb m{Q} \longrightarrow
\mathbb m{Q} %Error. "nocomma" is a bad command.
$, $g \left( x \right) = \frac{f \left( x \right) +
x^n}{r}$

Then we have $f \left( x \right) = r g \left( x \right) - x^n$

So, $r g \left( x + r g \left( y \right) \right) - \left( x + r g \left( y
\right) \right)^n = r g \left( x \right) - x^n$

$\Longrightarrow g \left( x + r g \left( y \right) \right) - g \left( x
\right) = \frac{\left( x + r g \left( y \right) \right)^n - x^n}{r} = \sum_{i
= 1}^n c_i r^{i - 1} g \left( y \right)^i x^{n - i} \in \mathbb m{Q}$ for all $x \in \mathbb m{R^+}$

$x = g \left( y \right) \Longrightarrow \sum_{i = 1}^n c_i r^i = \frac{\left( r + 1 \right)^n - 1}{r} \in \mathbb m{Q}$     (1)

$x = r \Longrightarrow r^{n - 1} \in \mathbb m{Q}$    (2)

$y = 1, x = r^2 \Longrightarrow r^{n - 1} \left( \left( r + 1 \right)^n - r^n \right) \in \mathbb m{Q} \Longrightarrow \left( r + 1 \right)^n - r^n = u \in \mathbb m{Q}$   (3)

(1)-(2) and (3) $\Longrightarrow \frac{u - 1}{r} \in \mathbb m{Q}
\Longrightarrow r \in \mathbb m{Q}$

Now, $y = 1$, $x = \pi \Longrightarrow \frac{\left( r + \pi \right)^n - \pi^n}{r} = v \in \mathbb m{Q} \Longrightarrow \left( r + \pi \right)^n - \pi^n - r v = 0$

the polynomial $h \left( x \right) = \left( x + r \right)^n - x^n - r v$ has $\pi$ as a root as well as has all rational co-efficients. An impossibility! (See [url=http://en.wikipedia.org\/wiki\/Pi#Irrationality_and_transcendence]here[\/url])

So, $f \left( x \right) = c$ is the only solution.
\end{solution}



\begin{solution}[by \href{https://artofproblemsolving.com/community/user/31919}{tenniskidperson3}]
	\begin{tcolorbox}Denote, $s = f \left( a \right) + a^n %Error. "nocomma" is a bad command.
, t = f \left( b \right) + b^n$

Then, $\forall \in > 0,$ $\exists u, v \in \mathbb m{N}$ such that, $0 <
\left| a s - b t \right| < \in$ ( See [url=http://www.artofproblemsolving.com/Forum/viewtopic.php?f=38&t=432389]here[\/url] )\end{tcolorbox}

Could you explain what you mean here?  I see no $u$ or $v$ in the problem, and because $s$ depends on $a$ and $t$ depends on $b$, I don't think you can use Kronecker's theorem here.

And by the way, if you're trying to write epsilon, you can use $\epsilon$ instead of $\in$, if you'd like. :)
\end{solution}



\begin{solution}[by \href{https://artofproblemsolving.com/community/user/82334}{bappa1971}]
	\begin{tcolorbox}[quote="bappa1971"]Denote, $s = f \left( a \right) + a^n %Error. "nocomma" is a bad command.
, t = f \left( b \right) + b^n$

Then, $\forall \in > 0,$ $\exists u, v \in \mathbb m{N}$ such that, $0 <
\left| a s - b t \right| < \in$ ( See [url=http://www.artofproblemsolving.com/Forum/viewtopic.php?f=38&t=432389]here[\/url] )\end{tcolorbox}

Could you explain what you mean here?  I see no $u$ or $v$ in the problem, and because $s$ depends on $a$ and $t$ depends on $b$, I don't think you can use Kronecker's theorem here.

And by the way, if you're trying to write epsilon, you can use $\epsilon$ instead of $\in$, if you'd like. :)\end{tcolorbox}

Sorry!  :blush: 
Now edited.
\end{solution}



\begin{solution}[by \href{https://artofproblemsolving.com/community/user/104143}{Markomak1}]
	i will try 
let x=y 

$f( y(1+y^{n-1}) +f(y))=f(y)$


we get that unless $f(x)=c$  
$y^n + f(y)=0$

$f(y)=-y^n$

Getting us the 2 solutions

$f(y)=-y^n$   and  $f(x)=c$ 

Edit: the first solution is not in the definition field, so only the second one holds.

P.s. I am newbie at functional equations so can anyone please check my solution?
\end{solution}



\begin{solution}[by \href{https://artofproblemsolving.com/community/user/29428}{pco}]
	\begin{tcolorbox} 
we get that unless $f(x)=c$  
$y^n + f(y)=0$\end{tcolorbox}
Why ?
\end{solution}



\begin{solution}[by \href{https://artofproblemsolving.com/community/user/104143}{Markomak1}]
	If

$f(a)=f(b)$

does that not mean that either $ a=b$
or $f(a)=f(b)=c$?
\end{solution}



\begin{solution}[by \href{https://artofproblemsolving.com/community/user/29428}{pco}]
	\begin{tcolorbox}If

$f(a)=f(b)$

does that not mean that either $ a=b$
or $f(a)=f(b)=c$?\end{tcolorbox}
Surely not. Example $f(x)=x^2$ is not the constant function and $f(-1)=f(1)$ while $-1\ne 1$
\end{solution}



\begin{solution}[by \href{https://artofproblemsolving.com/community/user/93837}{jjax}]
	An elementary solution.
Let $P(x,y): f(x+y^n+f(y))=f(x)$.
Let $c=1^n+f(1)$. Clearly $c>0$.
$P(x,1): f(x+c)=f(x)$ for all $x$.
$P(x,y+c): f(x+(y+c)^n+f(y+c))=f(x)$.
But $P(x,y): f(x+y^n+f(y))=f(x)$. Comparing these two equations and noting that $f(y)=f(y+c)$,
$f(x+y^n+f(y))=f(x+(y+c)^n+f(y))$ for all $x,y>0$. Denote this proposition by $Q(x,y)$.

Consider the polynomial $g(a)=(a+c)^n-a^n=nca^{n-1}+...$.
It is clearly not the zero polynomial since $n-1 \geq 1$, and it has a positive leading coefficient.
Thus, there exists some real $M>0$ such that for all $d>M$, there exists $k>0$ satisfying $g(k)=d$.

Let us show that for each $d>M$, $d$ is a period of $f$.
Choose $k>0$ satisfying $g(k)=d$.
$Q(x,k): f(x+k^n+f(k))=f(x+k^n+f(k)+g(k))=f(x+k^n+f(k)+d)$.
Thus, $f(x)=f(x+d)$ for all sufficiently large $x$.
Set $x=nc+r$ for some sufficiently large $n$ and for any $r>0$, so $f(nc+r)=f(nc+r+d)$ for all positive $r$.
Since $f(x+c)=f(x)$, we obtain $f(r)=f(r+d)$ for each $r>0$, proving periodicity.

Now, for any given pair of positive reals $x,y$, we may choose $p$ satisfying $p-x>M$ and $p-y>M$.
Then, $f(x)=f(x+(p-x))=f(p)=f(y+(p-y))=f(y)$, so $f$ must be constant.
\end{solution}



\begin{solution}[by \href{https://artofproblemsolving.com/community/user/93837}{jjax}]
	\begin{tcolorbox}
Then, $\forall \in > 0,$ $\exists a, b \in \mathbb m{N}$ such that, $0 <
\left| a s - b t \right| < \in$ ( See [url=http://www.artofproblemsolving.com/Forum/viewtopic.php?f=38&t=432389]here[\/url] )

So, $f \left( x \right) = f \left( x + b t - b t \right) = f \left( x + b t +
\left( a s - b t \right) \right) = f \left( x + b t + \in \right) = f \left( x
+ \in \right)$

Which implies, $\lim_{\in \longrightarrow 0} f \left( x + \in \right) = f
\left( x \right)$, so $f$ in continious
\end{tcolorbox}

This isn't a proof of continuity. You're proving that for any $\epsilon >0$, there exists $r$ such that $0<r< \epsilon$ satisfying $f(x+r)=f(x)$ for all $x$. A discontinuous function can satisfy this property too; consider $f(x)=0$ for irrational $x$ and $f(x)=1$ for rational $x$. Then, any rational $r$ will satisfy $f(x)=f(x+r)$, and of course one may find a rational number in $(0, \epsilon )$
\end{solution}
*******************************************************************************
-------------------------------------------------------------------------------

\begin{problem}[Posted by \href{https://artofproblemsolving.com/community/user/68025}{Pirkuliyev Rovsen}]
	Find all monotonic functions $f: \mathbb{R}\to\mathbb{R}$ such that $f(2^x)=1-f(x)$ for all reals $x$.
	\flushright \href{https://artofproblemsolving.com/community/c6h444927}{(Link to AoPS)}
\end{problem}



\begin{solution}[by \href{https://artofproblemsolving.com/community/user/29428}{pco}]
	\begin{tcolorbox}Find all monotonic  function $f: \mathbb{R}\to\mathbb{R}$ such that $f(2^x)=1-f(x)$\end{tcolorbox}
If $f(x)$ is non increasing, then $LHS$ is non increasing while $RHS$ is non decreasing and so both are constant
If $f(x)$ is non decreasing, then $LHS$ is non decreasing while $RHS$ is non increasing and so both are constant

Hence the unique solution : $\boxed{f(x)=\frac 12}$ $\forall x$
\end{solution}
*******************************************************************************
-------------------------------------------------------------------------------

\begin{problem}[Posted by \href{https://artofproblemsolving.com/community/user/103227}{shohvanilu}]
	Find all functions $f: \mathbb R \to \mathbb R$ such that for all non-zero reals $x$ and $y$,
\[xf(y)-yf(x)=f\left(\frac{y}{x}\right)\]
	\flushright \href{https://artofproblemsolving.com/community/c6h444980}{(Link to AoPS)}
\end{problem}



\begin{solution}[by \href{https://artofproblemsolving.com/community/user/29428}{pco}]
	\begin{tcolorbox}Find all functions $f:R\toR$ and for x,y are is not equal to 0 and real such that
$xf(y)-yf(x)=f(\frac{y}{x})$\end{tcolorbox}
Let $P(x,y)$ be the assertion $xf(y)-yf(x)=f(\frac yx)$

$P(x,1)$ $\implies$ $xf(1)-f(x)=f(\frac 1x)$
$P(1,x)$ $\implies$ $f(x)-xf(1)=f(x)$
And so, adding these two lines) : $f(\frac 1x)=-f(x)$

$P(\frac 1x,2)$ $\implies$ $\frac{f(2)}x+2f(x)=f(2x)$
$P(\frac 12,x)$ $\implies$ $\frac{f(x)}2+xf(2)=f(2x)$
And so (subtracting these two lines) : $f(x)=\frac{2f(2)}3(x-\frac 1x)$

And so $\boxed{f(x)=a\frac {x^2-1}x}$ $\forall x\ne 0$ and for any real $a$, which indeed is a solution.
\end{solution}
*******************************************************************************
-------------------------------------------------------------------------------

\begin{problem}[Posted by \href{https://artofproblemsolving.com/community/user/67037}{smiley4t}]
	Define \[f(x)=\frac{2x^{3}-3}{3(x^{2}-1)} \] for all $x\neq \pm 1$. Prove that there exists a continuous function $g(x)$ over $\mathbb R$ such that $f(g(x))=x$ and $g(x)>x$.
	\flushright \href{https://artofproblemsolving.com/community/c6h445210}{(Link to AoPS)}
\end{problem}



\begin{solution}[by \href{https://artofproblemsolving.com/community/user/29428}{pco}]
	\begin{tcolorbox}Let $f(x)=\frac{2x^{3}-3}{3(x^{2}-1)} $ $\forall x\neq -1$
Prove that there exist continuos function g(x) over R such that f(g(x))=x and g(x)>x.\end{tcolorbox}
It's easy to see that the restriction $h(x)$ of $f(x)$ over $(1,+\infty)$ is a continuous strictly increasing bijection from $(1,+\infty)\to(-\infty,+\infty)$

So $g(x)=h^{-1}(x)$ is a stricly increasing continuous bijection from $\mathbb R\to(1;+\infty)$ such that $f(g(x))=x$ $\forall x\in\mathbb R$

And it just remains to prove that $h(x)<x$ $\forall x\in(1,+\infty)$ which is quite easy since $x-h(x)=\frac{1+(x-1)(x+2)}{3(x^2-1)}$
\end{solution}
*******************************************************************************
-------------------------------------------------------------------------------

\begin{problem}[Posted by \href{https://artofproblemsolving.com/community/user/83160}{hungnguyenvn}]
	Find all continuous functions $f: \mathbb{R}^+ \to \mathbb{R}^+$ such that \[f(x)f(y)=f(xy)+f\left(\frac xy\right)\] for all $x,y>0$.
	\flushright \href{https://artofproblemsolving.com/community/c6h445373}{(Link to AoPS)}
\end{problem}



\begin{solution}[by \href{https://artofproblemsolving.com/community/user/29428}{pco}]
	\begin{tcolorbox}Find all continuous functions $f:$ $\mathbb R^+\to \mathbb R^+$ such that $f(x)f(y)=f(xy)+f(x\/y)$ for all $x,y>0$.\end{tcolorbox}
One quick way is to write $f(x)=2g(\ln x)$ (where $g(x)$ is a continuous function from $\mathbb R\to\mathbb R^+$) and we get $2g(x)g(y)=g(x+y)+g(x-y)$ whose only continuous solutions are $1,\cos ax$ and $\cosh ax$ (classical d'Alembert).

Since $g(x)>0$, it remains $1$ and $\cosh ax$ and so the solutions :

$f(x)=2$ $\forall x>0$
$f(x)=x^a+x^{-a}$ $\forall x>0$ and for any real $a$
\end{solution}



\begin{solution}[by \href{https://artofproblemsolving.com/community/user/83160}{hungnguyenvn}]
	Excuse me,Sir. ,I don't know the classic d'Alembert so I don't undertand
\end{solution}



\begin{solution}[by \href{https://artofproblemsolving.com/community/user/64716}{mavropnevma}]
	Learn to do google (or wiki) searches, so just type "d'Alembert functional equation" as google search keywords ...
\end{solution}



\begin{solution}[by \href{https://artofproblemsolving.com/community/user/29428}{pco}]
	Thanks, mavropnevma.
I give hereunder a direct solution (of D'Alembert :) )
\begin{tcolorbox}Find $f: \mathbb{R}^+ \to \mathbb{R}^+$ continuous, such that $f(x)f(y)=f(xy)+f(x\/y)$ for all $x,y>0$.\end{tcolorbox}
Let $g(x)$ from $\mathbb R\to\mathbb R^+$ defined as $g(x)=\frac 12f(e^x)$
$g(x)$ is continuous and the functional equation becomes $2g(x)g(y)=g(x+y)+g(x-y)$ $\forall x,y\in\mathbb R$

Let $P(x,y)$ be the assertion $2g(x)g(y)=g(x+y)+g(x-y)$
$g(x)=1$ $\forall x$ is a solution. Let us from now look for non all-one solutions.

$P(x,0)$ $\implies$ $g(x)(g(0)-1)=0$ and, since $g(x)>0$ $\forall x$, we get $g(0)=1$
$P(0,x)$ $\implies$ $g(x)=g(-x)$ and $g(x)$ is an even function. 

If $0<g(u)<1$ for some $u$, then let $g(u)=\cos v$ with $v\in(0,\frac{\pi}2)$ and $P(u,u)$ $\implies$ $g(2u)=\cos 2v$ and so $g(2^nu)=\cos 2^nv$
But $\exists n\in\mathbb N$ such that $2^nv\in(\frac{\pi} 2,\pi]$ and so $g(2^nu)<0$, impossible.
So $g(x)\ge 1$ $\forall x$

Let then $u>0$ such that $g(u)>1$ ($u>0$ is always possible since $g(x)$ is an even function)
Let $a>0$ such that $g(u)=\cosh au$

$P(x,x)$ $\implies$ $g(x)^2=\frac{1+g(2x)}2$ and so, since $g(x)>0$ : $g(x)=\sqrt{\frac{1+g(2x)}2}$

From there, it's easy to get with induction (on $n$) that $g(2^{n}u)=\cosh (a2^{n}u)$ $\forall n\in\mathbb Z$ (important part is for $n<0$)

$P(x+y,y)$ $\implies$ $g(x+2y)=2g(x+y)g(y)-g(x)$

From there, it's now easy to get with induction again (on $p$) that $g(p2^nu)=\cosh (pa2^{n}u)$ $\forall n,p\in\mathbb Z$

And since the set $\{pa2^{n}u$ $\forall n,p\in\mathbb Z\}$ (remember $a,u>0$) is dense in $\mathbb R$, continuity gives us $g(x)=\cosh ax$ $\forall x\in\mathbb R$ which indeed is a solution.

So $f(x)=2\cosh (a\ln x)=x^a+x^{-a}$

Hence the two solutions of the original equation :
$f(x)=2$ $\forall x>0$
$f(x)=x^a+x^{-a}$ $\forall x>0$ and for any real $a$ (notice that $a=0$ gives the previous case)
\end{solution}
*******************************************************************************
-------------------------------------------------------------------------------

\begin{problem}[Posted by \href{https://artofproblemsolving.com/community/user/10045}{socrates}]
	A function $f$ satisfies $f(\cos x) = \cos (17x)$ for every real $x$. Show that $f(\sin x) =\sin (17x)$ for every $x \in \mathbb{R}.$
	\flushright \href{https://artofproblemsolving.com/community/c6h445546}{(Link to AoPS)}
\end{problem}



\begin{solution}[by \href{https://artofproblemsolving.com/community/user/29428}{pco}]
	\begin{tcolorbox}A function f satisfies $f(\cos x) = \cos (17x),$ for every real $x.$ 
Show that $f(\sin x) =\sin (17x),$ for every $x \in \Bbb{R}.$\end{tcolorbox}
$f(\sin x)=f(\cos(\frac{\pi}2-x))=\cos(17(\frac{\pi}2-x))$ $=\cos(\frac{\pi}2-17x)$ $=\sin 17x$
Q.E.D.
\end{solution}
*******************************************************************************
-------------------------------------------------------------------------------

\begin{problem}[Posted by \href{https://artofproblemsolving.com/community/user/68025}{Pirkuliyev Rovsen}]
	Find an infinite family of differentiable functions $f$ such that $f(f(x))=\sin x$.
	\flushright \href{https://artofproblemsolving.com/community/c6h445845}{(Link to AoPS)}
\end{problem}



\begin{solution}[by \href{https://artofproblemsolving.com/community/user/29428}{pco}]
	\begin{tcolorbox}Find all differentiable function $f$ such that $f(f(x))=sinx$\end{tcolorbox}

Sorry again, but are you sure about the problem statement ?
It seems to me that there are infinitely many such functions which may in a classical way be built piece per piece.

The only difficulty seems to be showing that $f'(x)$ is continuous at $0$ but I think it is possible.
Is this again a Gazeta Mathematica problem ?
Are such problems claimed as olympiad-level problems ?
Are solutions of these problems published by Gazeta Mathematica ?
\end{solution}



\begin{solution}[by \href{https://artofproblemsolving.com/community/user/68025}{Pirkuliyev Rovsen}]
	This task of the journal Kvant. It is written that is a difficult task.If you can give a complete answer.THANKS YOU
\end{solution}



\begin{solution}[by \href{https://artofproblemsolving.com/community/user/29428}{pco}]
	\begin{tcolorbox}Find all differentiable function $f$ such that $f(f(x))=sinx$
\end{tcolorbox}
Here is a method for building infinitely many solutions. But maybe it exist a lot of other.

1) building continuous solutions to $f(f(x))=\sin x$
==================================
Choose any $a\in(1,\frac{\pi}2)$ 
Let then the sequence of real numbers : $a_0=\frac{\pi}2$ and $a_1=a$ $a_{n+2}=\sin a_n$
This is a strictly decreasing positive sequence whose limit is $0$

Choose a  $C_0$ function $h(x)$ strictly increasing from $[a_1,a_0]\to[a_2,a_1]$
Let then the sequence of functions $f_n(x)$ from $[a_{n+1},a_n]\to[a_{n+2},a_{n+1}]$ defined as :
$f_0(x)=h(x)$
$f_{n+1}(x)=\sin f_n^{-1}(x)$

Each $f_n(x)$ is a continuous bijection (and so the definition of $f_n$ is legal)
Let $u(x)$ from $[0,\frac{\pi}2]\to[0,a]$ defined as :
$u(0)=0$ and $u(x)=f_n(x)$ $\forall x\in [a_{n+1},a_n]$ 
$u(x)$ is a continous increasing function such that $u(u(x))=\sin x$

Let $f(x)$ from $\mathbb R\to[-a,a]$ defined as :
$f(x)=u(x)$ $\forall x\in[0,\frac{\pi}2]$
$f(x)=u(\pi-x)$ $\forall x\in[\frac{\pi}2,\pi]$
$f(x)=-u(-x)$ $\forall x\in[-\frac{\pi}2,0]$
$f(x)=-u(\pi+x)$ $\forall x\in[-\pi,-\frac{\pi}2]$
$f(x+2\pi)=f(x)$ $\forall x$
$f(x)$ is a continuous fonction such that $f(f(x))=\sin x$ $\forall x$

2) building differentiable solutions to $f(f(x))=\sin x$
=================================
Obviously we need to choose $h(x)$ as for example a $C_1$ function and it remains to check that $f(x)$ is differentiable at each $a_n$ and at $0$

Differentiable at $a_0$ is obtained thru $h'(\frac{\pi}2)=0$

Differentiable at $a_1$ is more complex.
We need to obtain that $f_1'(a_1^-)=f_0'(a_1^+)$ where $f_1(x)=\sin h^{-1}(x)$ and $f_0(x)=h(x)$
$f_1'(x)=\frac{\cos(h^{-1}(x))}{h'(h^{-1}(x))}$
And since $h^{-1}(a_1)=\frac{\pi}2$, we get $\frac 00$
So we could choose :
$h(x)$ is $C_2$ around $\frac{\pi}2$ and $h''(\frac{\pi}2)< 0$
$h'(a_1)=-\frac 1{h''(\frac{\pi}2)}$

From there, $f(\sin x)=\sin (f(x))$ implies $f(x)$ differentiable at $a_n$ for any $n>1$

$\sin x<f(x)<x$ on $(0,\frac{\pi}2]$ implies $f'(0)$ exists and is $1$

So $f(x)$ is differentiable over $[0,\frac{\pi}2]$ and so over $\mathbb R$

The only interesting remaining point would be to know if $f'(x)$ is continuous at $0$. I think so but it's not required, so ... .


Notice that it may exist infinitely many other solutions (for example choosing $f(x)<0$ over $[0,\frac{\pi}2]$)
I'm surprised to see this published in a journal as an olympiad-level exercise :?:
\end{solution}



\begin{solution}[by \href{https://artofproblemsolving.com/community/user/68025}{Pirkuliyev Rovsen}]
	This is not a problem Olympiad.That the reports of Young Mathematicians.Another task will be.Thanks for all the 
\end{solution}
*******************************************************************************
-------------------------------------------------------------------------------

\begin{problem}[Posted by \href{https://artofproblemsolving.com/community/user/68025}{Pirkuliyev Rovsen}]
	Find all continuous functions $f: \mathbb{R}^+ \to\mathbb{R}$ such that
\[f(f(x))=\sqrt{f(x)x}\]
holds for all positive reals $x$.
	\flushright \href{https://artofproblemsolving.com/community/c6h446102}{(Link to AoPS)}
\end{problem}



\begin{solution}[by \href{https://artofproblemsolving.com/community/user/29428}{pco}]
	\begin{tcolorbox}Find all continuous functions $f: \mathbb{R_+}\to\mathbb{R}$ such that

$f(f(x))=\sqrt{f(x)x}$\end{tcolorbox}
$f(x)\ne 0$ $\forall x>0$ in order to have $LHS$ defined for any $x>0$
$f(x)\ge 0$ $\forall x>0$ in order to have $RHS$ defined for any $x>0$
So $f(x)>0$ $\forall x$

As a consequence, $f(x)$ is injective and so, since continuous, monotonous.

1) $f(x)=x$ $\forall x$ is the unique increasing solution.
===================================
Suppose $f(x)$ increasing.
If $f(a)>a$ for some $a$, then $f(f(a))=\sqrt{f(a)a}<f(a)$ in contradiction with increasing property
If $f(a)<a$ for some $a$, then $f(f(a))=\sqrt{f(a)a}>f(a)$ in contradiction with increasing property
So $f(x)=x$ $\forall x$ which indeed is a solution.

2) $f(x)=\frac a{\sqrt x}$ $\forall x$ where $a>0$ are the only decreasing solutions
=================================================
The equation $f(x)=x$ has exactly one solution since $f(x)$ is decreasing. 
Let then $f(c)=c$

Writing $f(x^{u_n}f(x)^{v_n})=x^{u_{n+1}}f(x)^{v_{n+1}}$ and writing then $f(LHS)=f(RHS)$, we get $f(x^{u_{n+1}}f(x)^{v_{n+1}})=x^{\frac{u_n+u_{n+1}}2}f(x)^{\frac{v_n+v_{n+1}}2}$

And so $u_n=\frac 13+\frac 23(-2)^{-n}$ and $v_n=\frac 23-\frac 23(-2)^{-n}$

And $f(x^{\frac 13+\frac 23(-2)^{-n}}f(x)^{\frac 23-\frac 23(-2)^{-n}})$ $=x^{\frac 13+\frac 23(-2)^{-n-1}}f(x)^{\frac 23-\frac 23(-2)^{-n-1}}$

Setting $n\to +\infty$ and using continuity, we get ${f(x^{\frac 13}}f(x)^{\frac 23})$ $=x^{\frac 13}f(x)^{\frac 23}$

So ${x^{\frac 13}}f(x)^{\frac 23}=c$ and $f(x)=\frac{\sqrt{c^3}}{\sqrt x}$

Plugging back $f(x)=\frac a{\sqrt x}$ in original equation, we get that all $a>0$ fit.
Q.E.D
\end{solution}
*******************************************************************************
-------------------------------------------------------------------------------

\begin{problem}[Posted by \href{https://artofproblemsolving.com/community/user/68025}{Pirkuliyev Rovsen}]
	Does there exist the function $f(x)$ satisfying the following conditions?
a) $f(x){\notin} \mathbb Z$, for all $x{\in} \mathbb Z$, and
b) $f(f(x)){\in} \mathbb Z$, for all $x{\in} \mathbb Z$.
	\flushright \href{https://artofproblemsolving.com/community/c6h446522}{(Link to AoPS)}
\end{problem}



\begin{solution}[by \href{https://artofproblemsolving.com/community/user/29428}{pco}]
	\begin{tcolorbox}Does there exist the function $f(x)$, satisfying all of the following conditions:

a) $f(x){\notin}Z$, for all $x{\in}Z$
b) $f(f(x)){\in}Z$, for all $x{\in}Z$\end{tcolorbox}
$f(x)=x+\frac 12$
\end{solution}
*******************************************************************************
-------------------------------------------------------------------------------

\begin{problem}[Posted by \href{https://artofproblemsolving.com/community/user/86345}{namdan}]
	Find all non-decreasing $f:\mathbb{R}\rightarrow \mathbb{R}$ functions that satisfy
\[f(x+f(y))=f(x)+y, \quad \forall x, y\in \mathbb{R}.\]
	\flushright \href{https://artofproblemsolving.com/community/c6h446553}{(Link to AoPS)}
\end{problem}



\begin{solution}[by \href{https://artofproblemsolving.com/community/user/29428}{pco}]
	\begin{tcolorbox}Find all function (non-decreasing) $f:\mathbb{R}\rightarrow \mathbb{R}$ satisfy:
$f(x+f(y))=f(x)+y, \forall x, y\in \mathbb{R}$\end{tcolorbox}
Let $P(x,y)$ be the assertion $f(x+f(y))=f(x)+y$

If $f(a)=f(b)$, then, comparing $P(0,a)$ and $P(0,b)$, we get $a=b$ and so $f(x)$ is injective.
$P(0,0)$ $\implies$ $f(f(0))=f(0)$ and so, since injective, $f(0)=0$

$P(0,x)$ $\implies$ $f(f(x))=x$

$P(x,f(y))$ $\implies$ $f(x+y)=f(x)+f(y)$ and so (non decreasing solution of Cauchy's equation) $f(x)=ax$

Plugging this back in original equation, we get \begin{bolded}two solutions\end{underlined}\end{bolded} :
$f(x)=x$ $\forall x$
$f(x)=-x$ $\forall x$
\end{solution}



\begin{solution}[by \href{https://artofproblemsolving.com/community/user/96376}{gilbert}]
	But $ f(x)=-x $ is  decreasing function. So the only solution $ f(x)=x $.
\end{solution}



\begin{solution}[by \href{https://artofproblemsolving.com/community/user/29428}{pco}]
	\begin{tcolorbox}But $ f(x)=-x $ is  decreasing function. So the only solution $ f(x)=x $.\end{tcolorbox}
:oops:

Quite right, indeed !

Thanks for the remark
\end{solution}
*******************************************************************************
-------------------------------------------------------------------------------

\begin{problem}[Posted by \href{https://artofproblemsolving.com/community/user/92794}{lypnol}]
	Find all functions $f:\mathbb{R}-\{0,1\}\to\mathbb{R}-\{0,1\}$ which satisfy for all $x\in\mathbb{R}-\{0,1\}$,
\[f(x)+f\left(\frac{1}{1-x}\right) = \frac{2(1-2x)}{x(1-x)}.\]
	\flushright \href{https://artofproblemsolving.com/community/c6h446757}{(Link to AoPS)}
\end{problem}



\begin{solution}[by \href{https://artofproblemsolving.com/community/user/29428}{pco}]
	\begin{tcolorbox}Find all functions $f:\mathbb{R}-\{0,1\}\to\mathbb{R}-\{0,1\}$ which satisfy:
for all $x\in\mathbb{R}-\{0,1\}$ : $f(x)+f\left(\frac{1}{1-x}\right) = \frac{2(1-2x)}{x(1-x)}$\end{tcolorbox}
Let $P(x)$ be the assertion $f(x)+f(\frac 1{1-x})=\frac{2(1-2x)}{x(1-x)}$

(a) : $P(-1)$ $\implies$ $f(-1)+f(\frac 12)=-3$

(b) : $P(\frac 12)$ $\implies$ $f(\frac 12)+f(2)=0$

(c) : $P(2)$ $\implies$ $f(2)+f(-1)=3$

(a)-(b)+(c) : $f(-1)=0$ , impossible since $f(x)$ is from $\mathbb R\setminus\{0,1\}\to\mathbb R\setminus\{0,1\}$ and so cant be zero.

So \begin{bolded}no solution\end{underlined}\end{bolded}.
\end{solution}
*******************************************************************************
-------------------------------------------------------------------------------

\begin{problem}[Posted by \href{https://artofproblemsolving.com/community/user/68025}{Pirkuliyev Rovsen}]
	Determine the  monotone functions $f: \mathbb{R}\to\mathbb{R}$ such that \[xf(\arctan x)=\arctan f(x)\] for all $x\in{R}$.
	\flushright \href{https://artofproblemsolving.com/community/c6h446794}{(Link to AoPS)}
\end{problem}



\begin{solution}[by \href{https://artofproblemsolving.com/community/user/29428}{pco}]
	\begin{tcolorbox}Determine the  monotone function $f: \mathbb{R}\to\mathbb{R}$ such that $xf(\arctan x)=\arctan f(x)$ for all $x\in{R}$.\end{tcolorbox}
Let $P(x)$ be the assertion $xf(\arctan x)=\arctan f(x)$

$P(0)$ $\implies$ $f(0)=0$

Let $x_n$ be a positive increasing sequence whose limit is $+\infty$. $y_n=\arctan x_n$ is an increasing sequence whose limit is $\frac{\pi}2$

$P(x_n)$ $\implies$ $x_nf(y_n)=\arctan f(x_n)$ and so $f(y_n)\in(-\frac{\pi}{2x_n},\frac{\pi}{2x_n})$ and so, since monotonous, $f(x)=0$ $\forall x\in[0,\frac{\pi}2)$

Using instead a negative decreasing sequence whose limit is $-\infty$, we get that $f(x)=0$ $\forall x\in(-\frac{\pi}2,\frac{\pi}2)$

But then $P(x)$ $\implies$ $LHS=0$ and so $\boxed{f(x)=0}$ $\forall x$ which indeed is a solution.
\end{solution}
*******************************************************************************
-------------------------------------------------------------------------------

\begin{problem}[Posted by \href{https://artofproblemsolving.com/community/user/92794}{lypnol}]
	Find all functions $f:\mathbb{R}\to\mathbb{R}$ which satisfy for all $x,y \in \mathbb{R}$ the following equation: 
\[ f(x^{2})-f(y^{2})=(x+y)(f(x)-f(y)).\]
	\flushright \href{https://artofproblemsolving.com/community/c6h446818}{(Link to AoPS)}
\end{problem}



\begin{solution}[by \href{https://artofproblemsolving.com/community/user/29428}{pco}]
	\begin{tcolorbox}Find all functions $f:\mathbb{R}\to\mathbb{R}$ which satisfy: for all $x,y \in \mathbb{R}$ 
$ f(x^{2})-f(y^{2})=(x+y)(f(x)-f(y)) $\end{tcolorbox}
Let $P(x,y)$ be the assertion $f(x^2)-f(y^2)=(x+y)(f(x)-f(y))$

$P(x,1)$ $\implies$ $f(x^2)-f(1)=(x+1)f(x)-(x+1)f(1)$
$P(x,-1)$ $\implies$ $f(x^2)-f(1)=(x-1)f(x)-(x-1)f(-1)$

Subtracting : $f(x)=x\frac{f(1)-f(-1)}2+\frac{f(1)+f(-1)}2$

Plugging back $f(x)=ax+b$ in original equation, we see that this is indeed a solution whatever are $a,b$

Hence the answer : $\boxed{f(x)=ax+b}$ $\forall x$ and for any real $a,b$
\end{solution}
*******************************************************************************
-------------------------------------------------------------------------------

\begin{problem}[Posted by \href{https://artofproblemsolving.com/community/user/92794}{lypnol}]
	Find all functions $f:\mathbb{R}\to\mathbb{R}$ satisfying for all $x,y\in\mathbb{R}$: \[ f(xf(x)+f(y))=(f(x))^{2}+y. \]
	\flushright \href{https://artofproblemsolving.com/community/c6h446840}{(Link to AoPS)}
\end{problem}



\begin{solution}[by \href{https://artofproblemsolving.com/community/user/29034}{newsun}]
	\begin{tcolorbox}find all functions $f:\mathbb{R}\to\mathbb{R}$ satisfying:
for all $x,y\in\mathbb{R}$: $ f(xf(x)+f(y))=(f(x))^{2}+y $\end{tcolorbox}
Balkan-2000.

The correct answer is 
$\boxed{f(x) =x , f(x)=-x}, x \in \mathbb{R}$.
\end{solution}



\begin{solution}[by \href{https://artofproblemsolving.com/community/user/122637}{Diehard}]
	Note that $f(x)$ is surjective since the RHS can be anything we want it. Then, there exists some $s$ such that $f(s)=0$. Replacing $x$ by $s$ in the functional equation gives $f(f(x))=x$. Plugging in $x=0$ gives $f(0)=0$. Finally, plugging in $y=0$ gives $f(xf(x))=(f(x))^{2}$. By letting $m=xf(x)$ and $n=f(y)$ and using the surjectivity of $f(x)$, we see that $f(m+n)=f(m)+f(n)$. This is just Cauchy's equation with solution $f(x)=cx$. Plugging this into the original equation shows that $c=1,-1$.
\end{solution}



\begin{solution}[by \href{https://artofproblemsolving.com/community/user/116153}{LoveMath4ever}]
	\begin{tcolorbox}Note that $f(x)$ is surjective since the RHS can be anything we want it. Then, there exists some $s$ such that $f(s)=0$. Replacing $x$ by $s$ in the functional equation gives $f(f(x))=x$. Plugging in $x=0$ gives $f(0)=0$. Finally, plugging in $y=0$ gives $f(xf(x))=(f(x))^{2}$. By letting $m=xf(x)$ and $n=f(y)$ and using the surjectivity of $f(x)$, we see that $f(m+n)=f(m)+f(n)$. This is just Cauchy's equation with solution $f(x)=cx$. Plugging this into the original equation shows that $c=1,-1$.\end{tcolorbox}

I also went there but there are other options to find the solution .

\begin{bolded}First\end{bolded}: put $f(1)=a$. easily: $a^2=1$ leads $a=1$ or $a=-1$

\begin{bolded}Second\end{bolded}:

\begin{bolded}I\end{bolded}\/ If $a=1$. we calculate the value $f ((x +1)f(x+1))$ in two ways as follows:

1, $f((x+1)f(x+1))=f(xf(x)+f(x)+ax+a)=f^2(x)+x+f(ax)+1$ (because $f(a)=f(f(1))=1$)  

2, $f((x+1)f(x+1))=f^2(x+1)=f^2(x)+2af(x)+1$

From which we get: $2f(x)=f(x)+x$ leads $f(x)=x$ for all $x$. it's true!

\begin{bolded}II\end{bolded}\/ If $a=-1$, leads $f(-1)=1$. 

Similarly, we have $f((x-1)f(x-1))=f^2(x)+f(-f(x))+f(x)+1=f^2(x)+2f(x)+1$

it gives: $f(x)=f(-f(x))$. But $f$ is surjective so $f(x)=-x$ for all $x$. 

\begin{bolded}Finally\end{bolded}: we have tow solutions: $f(x)=x, f(x)=-x$

----------------Done-------------------
\end{solution}



\begin{solution}[by \href{https://artofproblemsolving.com/community/user/29034}{newsun}]
	$ a=0$ or $a=1$, Why?? :)
\end{solution}



\begin{solution}[by \href{https://artofproblemsolving.com/community/user/116153}{LoveMath4ever}]
	\begin{tcolorbox}$ a=0$ or $a=1$, Why?? :)\end{tcolorbox}

I'm sorry! I have edited and fully resolve it. Thanks!!  :)  :)
\end{solution}



\begin{solution}[by \href{https://artofproblemsolving.com/community/user/29428}{pco}]
	\begin{tcolorbox}Note that $f(x)$ is surjective since the RHS can be anything we want it. Then, there exists some $s$ such that $f(s)=0$. Replacing $x$ by $s$ in the functional equation gives $f(f(x))=x$. Plugging in $x=0$ gives $f(0)=0$. Finally, plugging in $y=0$ gives $f(xf(x))=(f(x))^{2}$. By letting $m=xf(x)$ and $n=f(y)$ and using the surjectivity of $f(x)$, we see that $f(m+n)=f(m)+f(n)$. This is just Cauchy's equation with solution $f(x)=cx$. Plugging this into the original equation shows that $c=1,-1$.\end{tcolorbox}

There is an error in this proof : $f(x)$ surjective does not imply $xf(x)$ surjective (and actually it is not). So $f(m+n)=f(m)+f(n)$ is not true $\forall m,n$ but only $\forall n$ and $\forall m\in $ image of $xf(x)$
\end{solution}



\begin{solution}[by \href{https://artofproblemsolving.com/community/user/29034}{newsun}]
	Ok. Let me fix it.
I agree that Diehard's solution is true until the part $f(x(f(x))=(f(x))^2$  \begin{bolded}(1)\end{bolded}
Let $ x \sim f(x) $ in \begin{bolded}(1)\end{bolded} we obtain 
$ f(f(x)f(f(x)))=(f(f(x)))^2 \Rightarrow f(xf(x))=x^2 $ for all $x \in \mathbb{R} $ \begin{bolded}(2)\end{bolded}

Now combining \begin{bolded}(1)\end{bolded} and \begin{bolded}(2)\end{bolded} then we have
$ (f(x))^2 = x^2 $
Then $f(x)=x$ or $f(x)=-x$.

Assume that there exist $ a , b $ such that $f(a)=-a, f(b)=b $ for $a, b \neq 0$.
Plugging $ x \sim a$ and $ y \sim b$ into original equation gives
$ f(a^2+b)=a^2+b \neq \pm (-a^2+b) $, this is a contradiction.

In conclusion, the correct answer is 
$ \boxed{f(x)=x, f(x)=-x}, x \in \mathbb{R}$.
\end{solution}



\begin{solution}[by \href{https://artofproblemsolving.com/community/user/116153}{LoveMath4ever}]
	\begin{tcolorbox}Ok. Let me fix it.
I agree that Diehard's solution is true until the part $f(x(f(x))=(f(x))^2$  \begin{bolded}(1)\end{bolded}
Let $ x \sim f(x) $ in \begin{bolded}(1)\end{bolded} we obtain 
$ f(f(x)f(f(x)))=(f(f(x)))^2 \Rightarrow f(xf(x))=x^2 $ for all $x \in \mathbb{R} $ \begin{bolded}(2)\end{bolded}

Now combining \begin{bolded}(1)\end{bolded} and \begin{bolded}(2)\end{bolded} then we have
$ (f(x))^2 = x^2 $
Then $f(x)=x$ or $f(x)=-x$.

Assume that there exist $ a , b $ such that $f(a)=-a, f(b)=b $ for $a, b \neq 0$.
Plugging $ x \sim a$ and $ y \sim b$ into original equation gives
$ f(a^2+b)=a^2+b \neq \pm (-a^2+b) $, this is a contradiction.

In conclusion, the correct answer is 
$ \boxed{f(x)=x, f(x)=-x}, x \in \mathbb{R}$.\end{tcolorbox}

Oh, nice solution! thanks! :-D
\end{solution}
*******************************************************************************
-------------------------------------------------------------------------------

\begin{problem}[Posted by \href{https://artofproblemsolving.com/community/user/68025}{Pirkuliyev Rovsen}]
	Find all continuous functions $f: \mathbb{R}\to\mathbb{R}$ such that \[2f(x+y)+f(2x-y)+f(2y-x)=9f(x)+9f(y)\] for all $x,y\in\mathbb{R}$.
	\flushright \href{https://artofproblemsolving.com/community/c6h446907}{(Link to AoPS)}
\end{problem}



\begin{solution}[by \href{https://artofproblemsolving.com/community/user/29428}{pco}]
	\begin{tcolorbox}Find all continuous functions $f: \mathbb{R}\to\mathbb{R}$ such that

$2f(x+y)+f(2x-y)+f(2y-x)=9f(x)+9f(y)$ for all $x,y\in{R}$\end{tcolorbox}
Let $P(x,y)$ be the assertion $2f(x+y)+f(2x-y)+f(2y-x)=9f(x)+9f(y)$

$P(0,0)$ $\implies$ $f(0)=0$

$P(x,x)$ $\implies$ $f(2x)=8f(x)$
$P(x,2x)$ $\implies$ $f(3x)=27f(x)$

$P(x,0)$ $\implies$ $f(-x)=-f(x)$ and $f(x)$ is an odd function.

Let $x\in\mathbb R$ and the sequence $a_n=f(nx)$ for $n\ge 1$ : $a_1=f(x)$ and $a_2=8f(x)$ and $a_3=27f(x)$

For $n\ge 2$ :
$P(nx,nx)$ $\implies$ $a_{2n}=8a_n$
$P(x,(n+1)x)$ $\implies$ $a_{2n+1}=-2a_{n+2}+9a_{n+1}+a_{n-1}+9f(x)$
And so there is a unique such sequence.

And since $a_n=n^3f(x)$ fits, this is the unique one.

So $f(nx)=n^3f(x)$ $\forall n\in\mathbb N$
So $f(nx)=n^3f(x)$ $\forall n\in\mathbb Z$ ($f(0)=0$ and $f(x)$ odd)
So $f(px)=p^3f(x)$ $\forall p\in\mathbb Q$ 
So $f(x)=x^3f(1)$ $\forall x\in\mathbb Q$
So $f(x)=x^3f(1)$ $\forall x\in\mathbb R$ (continuity) which indeed is a solution

Hence the answer : $\boxed{f(x)=ax^3}$ $\forall x$ and for any real $a$
\end{solution}
*******************************************************************************
-------------------------------------------------------------------------------

\begin{problem}[Posted by \href{https://artofproblemsolving.com/community/user/29034}{newsun}]
	Find all functions $ f : \mathbb{R}^+ \to \mathbb{R}^+ $ which satisfy
\[ f \left(\frac{f(x)}{y}\right)=yf(y)f(f(x)) \] for every $ x, y \in \mathbb{R}^+ $ .
	\flushright \href{https://artofproblemsolving.com/community/c6h446915}{(Link to AoPS)}
\end{problem}



\begin{solution}[by \href{https://artofproblemsolving.com/community/user/29428}{pco}]
	\begin{tcolorbox}Find all functions $ f : \mathbb{R}^+ \to \mathbb{R}^+ $ which satisfies
$ f (\frac{f(x)}{y})=yf(y)f(f(x)) $ for every $ x, y \in \mathbb{R}^+ $ .\end{tcolorbox}
Let $P(x,y)$ be the assertion $f\left(\frac {f(x)}y\right)=yf(y)f(f(x))$

$P(1,1)$ $\implies$ $f(1)=1$

$P(x,\sqrt{f(x)})$ $\implies$ $f(f(x))=\frac 1{\sqrt{f(x)}}$

$P(1,x)$ $\implies$ $f\left(\frac 1x\right)=xf(x)$

$P(\frac 1x,f(x))$ $\implies$ $f(f(x))f(f(\frac 1x))=1$ and so $f(x)f(\frac 1x)=1$ and so $xf^2(x)=1$

Hence the unique answer : $\boxed{f(x)=\frac 1{\sqrt x}}$ $\forall x>0$, which indeed is a solution.
\end{solution}



\begin{solution}[by \href{https://artofproblemsolving.com/community/user/72731}{goodar2006}]
	my solution is somehow similar to pco:

$(x,1) \Longrightarrow f(1)=1$

$(1,x) \Longrightarrow f(\frac{1}{x})=xf(x)$

$(x,f(x)) \Longrightarrow f(f(x))=\frac{1}{\sqrt{f(x)}}$

$(x,\frac{1}{x}) \Longrightarrow f(xf(x))=\sqrt{f(x)}$

$f(f(\frac{1}{x}))=\frac{1}{\sqrt{f(\frac{1}{x})}}=f(xf(x))=\sqrt{f(x)} \Longrightarrow f(x)f(\frac{1}{x})=1$ so $f(x)=\frac{1}{\sqrt{x}}$ which is a solution.
\end{solution}
*******************************************************************************
-------------------------------------------------------------------------------

\begin{problem}[Posted by \href{https://artofproblemsolving.com/community/user/68025}{Pirkuliyev Rovsen}]
	Prove that there does not exist a differentiable function $f: \mathbb{R}\to\mathbb{R}$ such that
\[f(x)+f'(x)= \begin{cases} \sin x, & \mbox{if } x \leq 0, \\ \cos x, &\mbox{if } x>0. \end{cases}\]
	\flushright \href{https://artofproblemsolving.com/community/c6h446916}{(Link to AoPS)}
\end{problem}



\begin{solution}[by \href{https://artofproblemsolving.com/community/user/29428}{pco}]
	\begin{tcolorbox}Prove that no exist derivable function $f: \mathbb{R}\to\mathbb{R}$,such that


$f(x)+f'(x)=sinx$, if $x\leq0$.   $cosx$, if $x> 0$\end{tcolorbox}
General solution of this equation is :

$f(x)=\frac{\sin x-\cos x}2+ae^{-x}$ $\forall x\le 0$

$f(x)=\frac{\sin x+\cos x}2+be^{-x}$ $\forall x> 0$

Continuity at $0$ implies $b=a-1$

So $\lim_{x\to 0-}\frac{f(x)-f(0)}x=\frac 12-a$

And $\lim_{x\to 0+}\frac{f(x)-f(0)}x=\frac 32-a$

So $f(x)$ can not be differentiable at $0$
\end{solution}



\begin{solution}[by \href{https://artofproblemsolving.com/community/user/31919}{tenniskidperson3}]
	Could you also say:

If a differentiable function existed, then $\lim_{x\to0}f(x)+f'(x)$ must exist (limits from both sides exist and are equal).  But the limit from the left is $\sin0=0$ and the limit from the right is $\cos0=1$?
\end{solution}



\begin{solution}[by \href{https://artofproblemsolving.com/community/user/29428}{pco}]
	\begin{tcolorbox}Could you also say:

If a differentiable function existed, then $\lim_{x\to0}f(x)+f'(x)$ must exist \end{tcolorbox}

I do not agree. The limits must exist if the derivative is continuous. But a differentiable function may have a derivative non continuous.

See for example $f(x)=x^2\sin\frac 1x$ with $f(0)=0$
$f(x)$ is continuous and has a derivative in each point. But $f'(0)=0$ (from left and from right) while $\lim_{x\to 0}f'(x)$ does not exist.
\end{solution}



\begin{solution}[by \href{https://artofproblemsolving.com/community/user/31919}{tenniskidperson3}]
	Ok, i understand.  I was only thinking of "step" discontinuities, not the oscillating discontinuities as you point out.
\end{solution}
*******************************************************************************
-------------------------------------------------------------------------------

\begin{problem}[Posted by \href{https://artofproblemsolving.com/community/user/10045}{socrates}]
	A function $f : \Bbb{N}_0 \to \Bbb{R}$ satisfies $f(1) = 3$ and \[f(m + n) + f(m - n) - m + n - 1 =\frac{f(2m) + f(2n)}{2},\]
for any non-negative integers $m$ and $n$ with $m \geq n.$ Find all such functions $f$.
	\flushright \href{https://artofproblemsolving.com/community/c6h446945}{(Link to AoPS)}
\end{problem}



\begin{solution}[by \href{https://artofproblemsolving.com/community/user/29428}{pco}]
	\begin{tcolorbox}A function $f : \Bbb{N}_0 \to \Bbb{R}$ satisfies $f(1) = 3$ and \[f(m + n) + f(m - n) - m + n - 1 =\frac{f(2m) + f(2n)}{2},\]
for any nonnegative integers $m, n$ with $m \geq n.$ Find all such functions $f$\end{tcolorbox}
Let $f(x)=x^2+x+1+g(x)$. Then :

$g(1)=0$
Assertion $P(x,y)$ : $g(x+y)+g(x-y)=\frac{g(2x)+g(2y)}2$

$P(0,0)$ $\implies$ $g(0)=0$
$P(x,0)$ $\implies$ $g(2x)=4g(x)$ and $P(x,y)$ becomes $g(x+y)+g(x-y)=2g(x)+2g(y)$
$P(x+1,1)$ $\implies$ $g(x+2)=2g(x+1)-g(x)$ and so, since $g(0)=g(1)=0$ : $g(x)=0$ $\forall x$

Hence the result : $\boxed{f(x)=x^2+x+1}$ $\forall x\in\mathbb Z_{\ge 0}$
\end{solution}
*******************************************************************************
-------------------------------------------------------------------------------

\begin{problem}[Posted by \href{https://artofproblemsolving.com/community/user/10045}{socrates}]
	Find all functions $f:\Bbb{N}\to \Bbb{N}$ such that
\[f(2010f(n)+1389)=1+1389+\cdots+1389^{2010}+n, \quad \forall n \in \Bbb{N}.\]
	\flushright \href{https://artofproblemsolving.com/community/c6h447178}{(Link to AoPS)}
\end{problem}



\begin{solution}[by \href{https://artofproblemsolving.com/community/user/29428}{pco}]
	\begin{tcolorbox}Find all functions $f:\Bbb{N}\to \Bbb{N}$ such that:
\[f(2010f(n)+1389)=1+1389+...+1389^{2010}+n, \ (\forall n \in \Bbb{N}).\]\end{tcolorbox}
Let $g(n)=2010n+1389$ and $b=\sum_{k=0}^{2010}1389^k$ so that the functional equation is assertion $P(x)$ : $f(g(f(x)))=x+b$
$f(x)$ is injective.

Let $x\not\equiv 1389\pmod{2010}$

If $f(x)>b$, then $P(f(x)-b)$ $\implies$ $f(g(f(f(x)-b)))=f(x)$ and so, since injective, $g(f(f(x)-b))=x$ and so $x\equiv 1389\pmod{2010}$, impossible.

So $f(x)\le b$ and we again get a contradiction since we have infinitely many such $x$, but finitely many such $f(x)$ while $f(x)$ is injective.

So \begin{bolded}no solution\end{underlined}\end{bolded} to this functional equation.
\end{solution}



\begin{solution}[by \href{https://artofproblemsolving.com/community/user/210434}{NSamardzic}]
	Can anyone explain thurder the case $f(x)<b$?
\end{solution}



\begin{solution}[by \href{https://artofproblemsolving.com/community/user/29428}{pco}]
	\begin{tcolorbox}Can anyone explain thurder the case $f(x)<b$?\end{tcolorbox}
We have infinitely many such $x\not\equiv 1389\pmod{2010}$ : for example $2010k$ $\forall k\in\mathbb N$

But we have at most $b$ different values for $f(x)$ if $f(x)\le b$

And since $f(x)$ is injective, we cant have infinitely many distinct numbers whose all distinct images (since injective) are in a finite set.
\end{solution}
*******************************************************************************
-------------------------------------------------------------------------------

\begin{problem}[Posted by \href{https://artofproblemsolving.com/community/user/109786}{ansi}]
	Let $n\ge 2$ be an integer. Prove that there is a function $f:\mathbb R \to \mathbb R$ such that \[\sum_{k=1}^nf(kx)=0\] for any real $x$ and also \[f(x) =0\iff x=0.\]
	\flushright \href{https://artofproblemsolving.com/community/c6h447299}{(Link to AoPS)}
\end{problem}



\begin{solution}[by \href{https://artofproblemsolving.com/community/user/29428}{pco}]
	\begin{tcolorbox}Let $n\ge 2\in\mathbb N$. Prove that there is a function $f:\mathbb R \to \mathbb R$ such that $\sum_{k=1}^nf(kx)=0$ $\forall x\in\mathbb R$ and such that $f(x) =0\iff x=0$\end{tcolorbox}
Define $f(x)$ as :

$1)$ : $\forall x\in[1,n)$ : $f(x)=1$

$2_k)$ :  for $k=1\to+\infty$ : $\forall x\in \left[\frac{n^k}{(n-1)^{k-1}},\frac {n^{k+1}}{(n-1)^k}\right)$ : $f(x)=-\sum_{i=1}^{n-1}f(\frac{ix}n)$
In each such step, $\frac{ix}n\in\left[1,\frac{n^k}{(n-1)^{k-1}}\right)$ and this construction is possible and allows definition of $f(x)$ over $[1,+\infty)$

$3_k)$ : for $k=1\to+\infty$ : $\forall x\in\left[2^{-k},2^{1-k}\right)$ : $f(x)=-\sum_{i=2}^{n}f(ix)$
In each such step, $ix\in\left[2^{1-k},n\right)$ and this construction is possible and allows definition of $f(x)$ over $(0,1)$

$4)$ : $f(0)=0$

$5)$ : $\forall x<0$ : $f(x)=f(-x)$

It's easy to see that :

$\sum_{k=1}^{n}f(kx)=0$ $\forall x\in\mathbb R$

$\forall x\ne 0$ : $f(x)\in\mathbb Z$ and $f(x)\equiv 1\pmod n$ and so $f(x)\ne 0$ $\forall x\ne 0$

Q.E.D.
\end{solution}
*******************************************************************************
-------------------------------------------------------------------------------

\begin{problem}[Posted by \href{https://artofproblemsolving.com/community/user/29126}{MellowMelon}]
	Let $f : \mathbb R \to \mathbb R$ be a function, not identically zero, such that $f(x) + xf(y)$ is in the image of $f$ for all reals $x,y$. Prove or disprove: $f$ must be surjective. (i.e. its image is all real numbers)
	\flushright \href{https://artofproblemsolving.com/community/c6h447675}{(Link to AoPS)}
\end{problem}



\begin{solution}[by \href{https://artofproblemsolving.com/community/user/29428}{pco}]
	\begin{tcolorbox}Let $f : \mathbb R \to \mathbb R$ be a function, not identically zero, such that $f(x) + xf(y)$ is in the image of $f$ for all reals $x,y$. Prove or disprove: $f$ must be surjective. (i.e. its image is all real numbers)\end{tcolorbox}
No. $f(x)$ may be non surjective : Choose $f(x)=x-1$ $\forall x\ne 0$ and $f(0)=0$

And the unique family of such non surjective functions is :
$f(x)=a(x-1)$ $\forall x\ne 0$ and $f(0)=b$ with $a\ne 0$ and $a+b\ne 0$
\end{solution}
*******************************************************************************
-------------------------------------------------------------------------------

\begin{problem}[Posted by \href{https://artofproblemsolving.com/community/user/128097}{afmadhf}]
	Find all functions $f:\mathbb{R}\rightarrow \mathbb{R}$ which satisfy for all $x, y \in \mathbb{R}$ the equation
\[f(f(x)-y)=f(x)-f(y)+f(x)f(y)-xy.\]
	\flushright \href{https://artofproblemsolving.com/community/c6h447716}{(Link to AoPS)}
\end{problem}



\begin{solution}[by \href{https://artofproblemsolving.com/community/user/29428}{pco}]
	\begin{tcolorbox}Find all $f:\mathbb{R}\rightarrow \mathbb{R}$,  for all x, y $\in \mathbb{R}$ satisfy:
$f(f(x)-y)=f(x)-f(y)+f(x)f(y)-xy$\end{tcolorbox}
Let $P(x,y)$ be the assertion $f(f(x)-y)=f(x)-f(y)+f(x)f(y)-xy$
Let $a=f(0)$
$f(x)$ is injective

$P(0,0)$ $\implies$ $f(a)=a^2$
$P(0,a)$ $\implies$ $(a-1)a^2=0$
So $a\in\{0,1\}$

If $a=0$ : $P(x,0)$ $\implies$ $f(f(x))=f(x)$ and so, since injective, $f(x)=x$ which indeed is a solution

If $a=1$ : $P(0,0)$ $\implies$ $f(1)=1$ and $P(1,1)$ $\implies$ $1=0$, impossible

Hence the unique solution : $\boxed{f(x)=x}$ $\forall x$
\end{solution}



\begin{solution}[by \href{https://artofproblemsolving.com/community/user/86345}{namdan}]
	$ f(f(x)-y)=f(x)-f(y)+f(x)f(y)-xy $ $(1)$
$f(x)$ is injective
We have exist $a$ satisfy $f(a)=0$
Let $x=a$ and $y=0$ we have:
$f(f(a)-0)=f(a)-f(0)+f(0).f(a)-0
\Leftrightarrow f(0)=-f(0)
\Leftrightarrow f(0)=0$
So $f(0)=0$
Let $y=f(x)$ we have:
$f(0)=f(x)-f(f(x))+f(x)f(f(x))-xf(x)
\Leftrightarrow0=f(x)-f(f(x))+f(x)f(f(x))-xf(x)$
Let $y=0$ in $(1)$ we have:
$f(f(x))=f(x)$
So $0=f(x)-f(x)-f(f(x))+f(x)f(f(x))-xf(x)
\Leftrightarrow  f(x)(f(x)-x)=0$
$\Rightarrow $
$f(x)=0$ and $f(x)=x$
Try again we see $f(x)=x$ satisfy $(1)$
So $f(x)=x$
\end{solution}



\begin{solution}[by \href{https://artofproblemsolving.com/community/user/29428}{pco}]
	\begin{tcolorbox}We have exist $a$ satisfy $f(a)=0$\end{tcolorbox}
Why ?
\end{solution}



\begin{solution}[by \href{https://artofproblemsolving.com/community/user/114281}{opa}]
	\begin{tcolorbox}Find all $f:\mathbb{R}\rightarrow \mathbb{R}$,  for all x, y $\in \mathbb{R}$ satisfy:
$f(f(x)-y)=f(x)-f(y)+f(x)f(y)-xy$\end{tcolorbox}

  
Let $ P(x,y) $ be the assertion $ f(f(x)-y)=f(x)-f(y)+f(x)f(y)-xy $
Let $ f(0)=c $

$ P(x,0) $ $ \implies $ $f(f(x))=f(x)-c+cf(x)$
$ P(x, f(x)) $ $ \implies$ $ c=f(x)-f(f(x))+f^2(x)-xf(x) $
use $ P(x,0) $ we get $c=f(x)-(f(x)-c+cf(x))+f^2(x)-xf(x)$ 

so $f(x)(f(x)-x-c)=0$ $\Leftrightarrow$
Assume that exist $\alpha$ such that $f(\alpha)=0$
$ P(\alpha,0) $ $ \implies$ $c=-c$ $ \implies$ $c=0$ $\implies$ $f(0)=0$
1)$f(x)=0$, but $f(0)=0$ and $f(x)$ is injective
2)$f(x)=x+c$ but just $c=0$ satisfy $ P(x,y) $
Solution is $ \boxed{f(x)=x} $ $ \forall x $
\end{solution}



\begin{solution}[by \href{https://artofproblemsolving.com/community/user/29428}{pco}]
	\begin{tcolorbox}
so $f(x)(f(x)-x-c)=0$ $\Leftrightarrow$
1)$f(x)=0$ is not true
2)$f(x)=x+c$ but just $c=0$ satisfy $ P(x,y) $
Solution is $ \boxed{f(x)=x} $ $ \forall x $\end{tcolorbox}
And what about $f(x)=0$ for some $x$ and $f(x)=x+c$ for the others ?
\end{solution}



\begin{solution}[by \href{https://artofproblemsolving.com/community/user/114281}{opa}]
	\begin{tcolorbox}And what about $f(x)=0$ for some $x$ and $f(x)=x+c$ for the others ?\end{tcolorbox}
Thanks, it is good remark. It is killed when $f(x)$ is injective, but I don't khow how to prove this. =\
\end{solution}



\begin{solution}[by \href{https://artofproblemsolving.com/community/user/29428}{pco}]
	\begin{tcolorbox}[quote]And what about $f(x)=0$ for some $x$ and $f(x)=x+c$ for the others ?\end{tcolorbox}
Thanks, it is good remark. It is killed when $f(x)$ is injective, but I don't khow how to prove this. =\\end{tcolorbox}
Injection is quite simple : if $f(x_1)=f(x_2)$, just compare $P(x_1,1)$ and $P(x_2,1)$
\end{solution}



\begin{solution}[by \href{https://artofproblemsolving.com/community/user/114281}{opa}]
	\begin{tcolorbox}
Injection is quite simple : if $f(x_1)=f(x_2)$, just compare $P(x_1,1)$ and $P(x_2,1)$\end{tcolorbox}
Thank you, I will try to edit my post. :)
\end{solution}



\begin{solution}[by \href{https://artofproblemsolving.com/community/user/86345}{namdan}]
	\begin{tcolorbox}[quote="namdan"]We have exist $a$ satisfy $f(a)=0$\end{tcolorbox}
Why ?\end{tcolorbox}
I think because $f(x)$ is function injective
\end{solution}



\begin{solution}[by \href{https://artofproblemsolving.com/community/user/29428}{pco}]
	\begin{tcolorbox}[quote="pco"]\begin{tcolorbox}We have exist $a$ satisfy $f(a)=0$\end{tcolorbox}
Why ?\end{tcolorbox}
I think because $f(x)$ is function injective\end{tcolorbox}
$f(x)=e^x$ is injective but unfortunately it does not exist any $a$ such that $f(a)=0$

I think you make a confusion between injective and surjective.
\end{solution}



\begin{solution}[by \href{https://artofproblemsolving.com/community/user/86345}{namdan}]
	\begin{bolded}Thanks you pco!\end{bolded}
\end{solution}



\begin{solution}[by \href{https://artofproblemsolving.com/community/user/86345}{namdan}]
	\begin{tcolorbox}[quote="namdan"]We have exist $a$ satisfy $f(a)=0$\end{tcolorbox}
Why ?\end{tcolorbox}
We have:
$f(f(x)-y)=f(x)-f(y)+f(x)f(y)-xy$
$\Leftrightarrow f(f(x)-y)-f(x)-f(x)f(y)=-f(y)-xy$ 
We see right-hand side has degree 1 according to $x$ so has domain valued $\mathbb{R}$. So left-hand side has domain valued $\mathbb{R}$. Mean function has domain valued $\mathbb{R}$. So has $a$ satisfy $f(a)=0$
\end{solution}



\begin{solution}[by \href{https://artofproblemsolving.com/community/user/29428}{pco}]
	\begin{tcolorbox} 
... Mean function has domain valued $\mathbb{R}$. \end{tcolorbox}
Sorry again, but why ?
\end{solution}



\begin{solution}[by \href{https://artofproblemsolving.com/community/user/86345}{namdan}]
	Sorry I think:
We have:
$f(f(x)-y)=f(x)-f(y)+f(x)f(y)-xy$
$\Leftrightarrow f(f(x)-y)-f(x)-f(x)f(y)=-f(y)-xy$ 
We see right-hand side has degree 1 according to $x$ so has domain valued $\mathbb{R}$. So left-hand side has domain valued $\mathbb{R}$. So has $a$ satisfy $f(a)=0$
($f(u)=t \forall t\in \mathbb{R}$)
\end{solution}



\begin{solution}[by \href{https://artofproblemsolving.com/community/user/29428}{pco}]
	\begin{tcolorbox}Sorry I think:
We have:
$f(f(x)-y)=f(x)-f(y)+f(x)f(y)-xy$
$\Leftrightarrow f(f(x)-y)-f(x)-f(x)f(y)=-f(y)-xy$ 
We see right-hand side has degree 1 according to $x$ so has domain valued $\mathbb{R}$. So left-hand side has domain valued $\mathbb{R}$. So has $a$ satisfy $f(a)=0$
($f(u)=t \forall t\in \mathbb{R}$)\end{tcolorbox}

How can you deduce just from "for fixed $y\ne 0$, $f(f(x)-y)-f(x)-f(x)f(y)$ can take any value when $x$ takes any value in $\mathbb R$" that $f(x)$ can take any value in $\mathbb R$" ??????


Take for basic counterexample : $f(x)=x$ $\forall x\ne 0$ and $f(0)=1$


I give up. I think you are just joking (I hope)
\end{solution}



\begin{solution}[by \href{https://artofproblemsolving.com/community/user/121558}{Bigwood}]
	Let me say $f(0)=a$
Plugging $y=0$ we get \[f(f(x))=(a+1)f(x)-a\]
Plugging $f(x)$ into $y$, \[a=f(x)-f(f(x))+f(x)f(f(x))-xf(x)\\
=f(x)-(a+1)f(x)+a+f(x)[(a+1)f(x)-a]-xf(x)\]
\[(1+a)(f(x))^2-(x+2a)f(x)=0\]
(Case1)When $a=-1$, if $x\neq -2a, f(x)=0$ contradiction
(Case 2)Otherwise, $f(x)=\frac{x+2a}{1+a},0$
it is trivial that $f$ is injective. $f(x)=\frac{x+2a}{1+a}$ for all $x$. simple examination shows
\[f(x)=x\]
for all $x$.
\end{solution}
*******************************************************************************
-------------------------------------------------------------------------------

\begin{problem}[Posted by \href{https://artofproblemsolving.com/community/user/128206}{eddy13579}]
	Determine all functions $f: \mathbb Z \to \mathbb R$ such that for all integers $x$,
i) $f(x+2003) \leq f(x)+2003$, and 
ii) $f(x+1987) \geq f(x)+1987$.
	\flushright \href{https://artofproblemsolving.com/community/c6h448115}{(Link to AoPS)}
\end{problem}



\begin{solution}[by \href{https://artofproblemsolving.com/community/user/29428}{pco}]
	\begin{tcolorbox}Determine all functions f:Z->R such that:
i) $f(x+2003)\le f(x)+2003$
ii) $f(x+1987)\ge f(x)+1987$\end{tcolorbox}

i) $\implies$ $f(x+2003n)\le f(x)+2003n$ $\implies$ $f(x+2003\times 1987)\le f(x)+2003\times 1987$
ii) $\implies$ $f(x+1987n)\ge f(x)+1987n$ $\implies$ $f(x+1987\times 2003)\ge f(x)+1987\times 2003$

So $f(x+2003)=f(x)+2003$ and $f(x+1987)=f(x)+1987$ and $f(x+2003p+1987q)=f(x)+2003p+1987q$ and, since $\gcd(2003,1987)=1$ : $f(x+1)=f(x)+1$

So $\boxed{f(x)=x+c}$ $\forall x$ and for any real $c$
\end{solution}



\begin{solution}[by \href{https://artofproblemsolving.com/community/user/75584}{mahdlbelly}]
	how do you add n to the equation . I'm hanged and I can't think.help. no thinking
\end{solution}



\begin{solution}[by \href{https://artofproblemsolving.com/community/user/29428}{pco}]
	\begin{tcolorbox}how do you add n to the equation . I'm hanged and I can't think.help. no thinking\end{tcolorbox}
Just use induction :

$f(x+2003n)=f(x+2003(n-1)+2003)\le f(x+2003(n-1))+2003$
$f(x+2003(n-1))=f(x+2003(n-2)+2003)\le f(x+2003(n-2))+2003$
...
\end{solution}
*******************************************************************************
-------------------------------------------------------------------------------

\begin{problem}[Posted by \href{https://artofproblemsolving.com/community/user/107877}{Quadratrix}]
	Find all functions $f: \mathbb{R} \rightarrow \mathbb{R}$ such that \[f(f(x)+y)=f(x)+f(f(y))\] for all $x,y \in \mathbb{R}$.
	\flushright \href{https://artofproblemsolving.com/community/c6h448123}{(Link to AoPS)}
\end{problem}



\begin{solution}[by \href{https://artofproblemsolving.com/community/user/29428}{pco}]
	\begin{tcolorbox}Find all functions $f: \mathbb{R} \rightarrow \mathbb{R}$ such that $f(f(x)+y)=f(x)+f(f(y))$ , $\forall x,y \in \mathbb{R}$\end{tcolorbox}
I cant believe that this is a real olympiad exercise. Where did you get it ?
Here are some examples of solutions :
$f(x)=0$
$f(x)=x$
$f(x)=\lfloor x+60\{x\}\rfloor$
$f(x)=\lfloor x+17\sin 2\pi x\rfloor$
$f(x)=$ projection of $x$ in $A$ when considering any two supplementrary $\mathbb Q$-vector subspaces $A,B$ of the $\mathbb Q$-vectorspace $\mathbb R$
.....
If your teacher[friend\/book\/little sister] gave you this problem in an olympiad training session, change your teacher [friend\/book\/little sister]!


\begin{bolded}A general solution is\end{underlined}\end{bolded} :
Let $\mathbb A$ any additive subgroup of $\mathbb R$
Let $\sim$ the equivalence relation $x\sim y$ $\iff$ $x-y\in\mathbb A$
Let $r(x)$ any choice function which associates to a real $x$ a representant (unique per class) of its class.
Let $g(x)$ any function from $\mathbb R\to\mathbb A$
Then :

$f(x)=x-r(x)+g(r(x))-g(r(0))+r(0)$

1) [hide="Proof that this indeed is a solution"]
==========================
Notice first that $x-r(x),g(r(x)),g(r(0)),r(0)\in\mathbb A$ and so $f(x)\in\mathbb A$
Notice then that if $x\in\mathbb A$, then $r(x)=r(0)$ and $f(x)=x$

So $r(f(x)+y)=r(y)$ and so $f(f(x)+y)=f(x)+y-r(y)+g(r(y))-g(r(0))+r(0)=f(x)+f(y)$
And since $f(y)\in\mathbb A$, $f(f(y))=f(y)$

And so $f(f(x)+y)=f(x)+f(f(y))$
Q.E.D.
[\/hide]
2) [hide="Proof that any solution may be written in this form (and so it's a general solution)"]
===========================================================
Let $f(x)$ such that $f(f(x)+y)=f(x)+f(f(y))$ $\forall x,y$
Let $P(x,y)$ be the assertion $f(f(x)+y)=f(x)+f(f(y))$
Let $A=f(\mathbb R)$

2.1) $\mathbb A$ is an additive group
-----------------------------------
Let $u,v\in\mathbb A$ and $x,y$ such that $u=f(x)$ and $v=f(y)$ :

$P(0,0)$ $\implies$ $f(0)=0$ and so $0\in \mathbb A$
$P(x,0)$ $\implies$ $f(u)=u$ and $P(x,y)$ may be written $f(f(x)+y)=u+v$ and so $u+v\in\mathbb A$
$P(x,-f(x))$ $\implies$ $-u=f(-u)$ and so $-u\in\mathbb A$
Q.E.D.

2.2) $f(x)$ may be written in the required form $f(x)=x-r(x)+g(r(x))-g(r(0))+r(0)$
---------------------------------------------------------------------------------

Then $\sim$ the relation $x\sim y$ $\iff$ $x-y\in\mathbb A$ : Since $\mathbb A$ is an additive group, this is an equivalence relation.
Let $r(x)$ any choice function which associates to a real $x$ a representant (unique per class) of its class such that $r(0)=0$
Let $g(x)=f(x)$ : $g(x)$ is a function from $\mathbb R\to\mathbb A$

$P(x,0)$ $\implies$ $f(f(x))=f(x)$  and so $f(x)=x$ $\forall x\in\mathbb A$

$x-r(x)\in\mathbb A$ $\implies$ $f(x-r(x))=x-r(x)$
$f(r(x))\in\mathbb A$ $\implies$ $f(f(r(x)))=f(r(x))$

$x-r(x)\in\mathbb A$ $\implies$ $\exists y$ such that $x-r(x)=f(y)$

$P(y,r(x))$ $\implies$ $f((x-r(x))+r(x))=(x-r(x))+f(f(r(x)))=(x-r(x))+f(r(x))$ $\implies$ $f(x)-x=f(r(x))-r(x)$

So $f(x)=x+(f(x)-x)=x+f(r(x))-r(x)$ $=x-r(x)+g(r(x))-g(r(0))+r(0)$
Q.E.D.
[\/hide]
3) [hide="Application : some examples of solution"]
=============================

1) $\mathbb A=\{0\}$
---------------
Then $r(x)=x$ and $g(x)=0$ and $\boxed{f(x)=0}$

2) $\mathbb A=\mathbb R$
---------------
Then we have a unique class and $r(x)=c$ $\forall x$
$f(x)=x-r(x)+g(r(x))-g(r(0))+r(0)=x-c+g(c)-g(c)+c=x$ and the solution $\boxed{f(x)=x}$

3) $\mathbb A=\mathbb Z$
------------------------
We can then choose $r(x)=\{x\}$ and $g(x)=\lfloor h(x)\rfloor$ where $h(x)$ is any function from $\mathbb R\to\mathbb R$

$f(x)=x-r(x)+g(r(x))-g(r(0))+r(0)$ $=x-\{x\}+\lfloor h(\{x\})\rfloor-\lfloor h(0)\rfloor$ $=x-\{x\}+\lfloor h(\{x\})\rfloor-\lfloor h(0)\rfloor$ $=\lfloor x-\{x\}+h(\{x\})- \lfloor h(0)\rfloor\rfloor$
Choosing different $h(x)$ gives a lot of solutions :
$h(x)=0$  gives $\boxed{f(x)=\lfloor x\rfloor}$

$h(x)=(a+1)x$ gives $\boxed{f(x)=\lfloor x+a\{x\}\rfloor}$

But also $\boxed{f(x)=\lfloor x+17\sin 2\pi x\rfloor}$

4) Let $A,B$ two supplementrary $\mathbb Q$-vector subspaces of the $\mathbb Q$-vectorspace $\mathbb R$
--------------------------------------------------------------------------------------
Let $a(x)$ and $b(x)$ the projections of $x$ in $A$ and $B$ (so that $x=a(x)+b(x)$)

Then $\boxed{f(x)=a(x)}$ is a solution
It's easy to check directly that this is a solution

It can be retrieved from the general form using $r(x)=b(x)$ and $g(x)=a(x)$

And a lot of other solutions (using $\mathbb A=\mathbb Q$ or $\mathbb A=\mathbb Z[\sqrt 2]$ for example
[\/hide]
\end{solution}



\begin{solution}[by \href{https://artofproblemsolving.com/community/user/107877}{Quadratrix}]
	Thanks for the answer, now when i see the solution I am not as depressed as i was yesterday when i spent the whole day doing this problem and could not solve it... :D Actually, i got this problem from my friend and i thought it was quite easy, but then i realised it was not xD
\end{solution}
*******************************************************************************
-------------------------------------------------------------------------------

\begin{problem}[Posted by \href{https://artofproblemsolving.com/community/user/35192}{ppt}]
	Find all functions $f:\mathbb{R}\to\mathbb{R}$ such that
a) $f(1)=2011$, and
b) $f(x^2-y)=xf(x)-f(y)$ for all $x,y\in\mathbb{R}$.
	\flushright \href{https://artofproblemsolving.com/community/c6h448262}{(Link to AoPS)}
\end{problem}



\begin{solution}[by \href{https://artofproblemsolving.com/community/user/29428}{pco}]
	\begin{tcolorbox}Find all functions $f:\mathbb{R}\to\mathbb{R}$ such that
a) $f(1)=2011$
b) $f(x^2-y)=xf(x)-f(y)$ for all $x,y\in\mathbb{R}$.
Thank you.\end{tcolorbox}
Let $P(x,y)$ be the assertion $f(x^2-y)=xf(x)-f(y)$

$P(0,0)$ $\implies$ $f(0)=0$
$P(0,x)$ $\implies$ $f(-x)=-f(x)$
$P(x,0)$ $\implies$ $f(x^2)=xf(x)$

So $f(x^2+y)=f(x^2)+f(y)$
So $f(x+y)=f(x)+f(y)$ (use $f(-x)=-f(x)$ if $x<0$

Let us then compute $f((x+1)^2)$ in two ways :
$f((x+1)^2)=(x+1)f(x+1)=(x+1)f(x)+2011(x+1)$
$f((x+1)^2)=f(x^2+2x+1)=f(x^2)+2f(x)+2011=(x+2)f(x)+2011$

So $(x+1)f(x)+2011(x+1)=(x+2)f(x)+2011$

So $\boxed{f(x)=2011x}$ which indeed is a solution.
\end{solution}
*******************************************************************************
-------------------------------------------------------------------------------

\begin{problem}[Posted by \href{https://artofproblemsolving.com/community/user/109399}{Gleek-00}]
	Find all the functions $f:\mathbb{N}\to\mathbb{R}$ that satisfy
\[ f(x+y)=f(x)+f(y) \] for all $x,y\in\mathbb{N}$ satisfying $10^6-\frac{1}{10^6} < \frac{x}{y} < 10^6+\frac{1}{10^6}$.
	\flushright \href{https://artofproblemsolving.com/community/c6h448277}{(Link to AoPS)}
\end{problem}



\begin{solution}[by \href{https://artofproblemsolving.com/community/user/29428}{pco}]
	\begin{tcolorbox}Find all the functions $f:\mathbb{N}\to\mathbb{R}$ that satisfy
\[ f(x+y)=f(x)+f(y) \] for all $x,y\in\mathbb{N}$ satisfying $10^6-\frac{1}{10^6} < \frac{x}{y} < 10^6+\frac{1}{10^6}$.\end{tcolorbox}
Let $x>A=\frac 1210^6(3+\frac 2{10^6+10^{-6}})(10^{12}+10^{-12})$

Then $x(\frac 1{10^6-10^{-6}}-\frac 1{10^6+10^{-6}})>3+\frac 2{10^6+10^{-6}}$

So $\frac x{10^6-10^{-6}}>\frac {x+2}{10^6+10^{-6}}+3$

So $\exists y\in\mathbb N$ such that $\frac x{10^6-10^{-6}}>y>\frac {x+2}{10^6+10^{-6}}+1$

So $10^6-10^{-6}<\frac xy<\frac{x+1}y<\frac{x+1}{y-1}<\frac{x+2}{y-1}<10^6+10^{-6}$

From $10^6-10^{-6}<\frac xy<10^6+10^{-6}$ we get $f(x+y)=f(x)+f(y)$

From $10^6-10^{-6}<\frac{x+1}{y-1}<10^6+10^{-6}$ we get $f(x+y)=f(x+1)+f(y-1)$ and so (see previous line) $f(x+1)-f(x)=f(y)-f(y-1)$

From $10^6-10^{-6}<\frac{x+1}y<10^6+10^{-6}$ we get $f(x+y+1)=f(x+1)+f(y)$

From $10^6-10^{-6}<\frac{x+2}{y-1}<10^6+10^{-6}$ we get $f(x+y+1)=f(x+2)+f(y-1)$ and so (see previous line) $f(x+2)-f(x+1)=f(y)-f(y-1)$

So : $\forall x>A$ : $f(x+2)-f(x+1)=f(x+1)-f(x)$ and so $f(x)=ax+b$ for some $a,b$

So, choosing $y>A$ and $x=10^6y>A$, we get $a(y+10^6y)+b=(a10^6y+b)+(ay+b)$ and so $b=0$ and $f(x)=ax$ $\forall x>A$

Let then $y>A\times 10^{-6}$ so that $x=10^6y>A$ and $x+y=y+10^6y>A$ : $f(x+y)=f(x)+f(y)$ and since $f(x+y)=a(x+y)$ and $f(x)=ax$, ge get $f(y)=ay$

So from $f(x)=ax$ $\forall x>A$, we get $f(x)=ax$ $\forall x>A\times 10^{-6}$

And so $\boxed{f(x)=ax}$ $\forall x\in\mathbb N$ and for any $a\in\mathbb N$, which indeed is a solution
\end{solution}
*******************************************************************************
-------------------------------------------------------------------------------

\begin{problem}[Posted by \href{https://artofproblemsolving.com/community/user/114445}{jejungchv}]
	A real-valued function $ f$ on $ \mathbb{Q}$ satisfies the following conditions for arbitrary $ \alpha, \beta \in \mathbb{Q}:$

(i) $ f(0) = 0,$
(ii) $ f(\alpha) > 0$ if $\alpha \neq 0,$
(iii) $ f(\alpha \cdot \beta) = f(\alpha)f(\beta),$ 
(iv) $ f(\alpha + \beta) \leq f(\alpha) + f(\beta),$ and
(v) $ f(m) \leq 1989$ for all $m \in  \mathbb{Z}.$

Prove that \[ f(\alpha + \beta) = \max\{f(\alpha), f(\beta)\} \text{ if } f(\alpha) \neq f(\beta).\]
	\flushright \href{https://artofproblemsolving.com/community/c6h449061}{(Link to AoPS)}
\end{problem}



\begin{solution}[by \href{https://artofproblemsolving.com/community/user/93044}{nguyenhung}]
	I wonder if I was wrong ?!

From the condition (iii), take the logarithm both sides, we have $g\left( {a + b} \right) = g\left( a \right) + g\left( b \right),\forall x \in \mathbb{Q}\backslash \left\{ 0 \right\}$, where $g\left( x \right) = \ln f\left( x \right),\forall x \in \mathbb{Q}\backslash \left\{ 0 \right\}$
(since (ii): $f(x)>0, \forall x \ne 0$, we can take the logarithm both sides)

We infer
\[g\left( x \right) = \gamma x,\forall x \in \mathbb{Q}\backslash \left\{ 0 \right\}\]
\[ \Rightarrow f\left( x \right) = {\alpha ^x},\forall x \in \mathbb{Q}\backslash \left\{ 0 \right\}\]
But, if $\alpha >1$, then $\exists m \in {\mathbb{Z}^ + }:f\left( m \right) = {\alpha ^m} > 1989$ ($m$ is very very big) - contradiction.
If $\alpha <1$, then $\exists m \in {\mathbb{Z}^ + }:f\left( -m \right) = {\alpha ^{-m}} > 1989$ ($m$ is very very big) - contradiction.
Hence, $\alpha =1$, which means:
\[\left\{ \begin{gathered}
  f\left( x \right) = 1,\forall x \in \mathbb{Q}\backslash \left\{ 0 \right\} \hfill \\
  f\left( 0 \right) = 0 \hfill \\ 
\end{gathered}  \right.\]

However, this result is not true with the statement which your problem require.
\end{solution}



\begin{solution}[by \href{https://artofproblemsolving.com/community/user/29428}{pco}]
	\begin{tcolorbox}I wonder if I was wrong ?!

From the condition (iii), take the logarithm both sides, we have $g\left( {a + b} \right) = g\left( a \right) + g\left( b \right),\forall x \in \mathbb{Q}\backslash \left\{ 0 \right\}$, where $g\left( x \right) = \ln f\left( x \right),\forall x \in \mathbb{Q}\backslash \left\{ 0 \right\}$
(since (ii): $f(x)>0, \forall x \ne 0$, we can take the logarithm both sides)

We infer
\[g\left( x \right) = \gamma x,\forall x \in \mathbb{Q}\backslash \left\{ 0 \right\}\]
\end{tcolorbox}
No you're wrong. You got $g(ab)=g(a)+g(b)$ and not $g(a+b)=g(a)+g(b)$ and you need a new transformation $g(x)=h(\ln x)$ but then the additive equation is no longer true in $\mathbb Q$ and you cant conclude.

In fact there are infinitely many functions $f$ such that $f(ab)=f(a)f(b)$ $\forall a,b\in\mathbb Q$ different from $f(x)=x^t$
\end{solution}



\begin{solution}[by \href{https://artofproblemsolving.com/community/user/29428}{pco}]
	\begin{tcolorbox}A real-valued function $ f$ on $ \mathbb{Q}$ satisfies the following conditions for arbitrary $ \alpha, \beta \in \mathbb{Q}:$

\begin{bolded}(i)\end{bolded} $ f(0) = 0,$
\begin{bolded}(ii)\end{bolded} $ f(\alpha) > 0 \text{ if } \alpha \neq 0,$
\begin{bolded}(iii)\end{bolded} $ f(\alpha \cdot \beta) = f(\alpha)f(\beta),$ 
\begin{bolded}(iv)\end{bolded} $ f(\alpha + \beta) \leq f(\alpha) + f(\beta),$
\begin{bolded}(v)\end{bolded} $ f(m) \leq 1989$ $ \forall m \in  \mathbb{Z}.$

Prove that \[ f(\alpha + \beta) = \max\{f(\alpha), f(\beta)\} \text{ if } f(\alpha) \neq f(\beta).\]\end{tcolorbox}
(ii) and (iii) imply  $f(-1)=f(1)=1$
(iii) implies that $f(x)$ is full defined by knowledge of $f(p_i)$ for all primes $p_i$

If $f(n)>1$ for some integer $n$, then $f(n^k)=(f(n))^k>1989$ for positive integer $k$ great enough and so (v) is wrong. So $f(n)\le 1$ $\forall n\in\mathbb Z$

Let $p,q$ distinct primes and $n$ a positive integer. There exist integers $u_n,v_n$ such that $u_np^n+v_nq^n=1$

(iv) implies $1=f(1)=f(u_np^n+v_nq^n)\le f(u_n)f(p)^n+f(v_n)f(q)^n\le f(p)^n+f(q)^n$

Setting $n\to +\infty$, we get that either $f(p)=1$, either $f(q)=1$ and so $f(x)=1$ for all primes except maybe one.

And so the solutions are :
$S1$ : $f(0)=0$ and $f(x)=1$ $\forall x\ne 0$ 

$S2$ : $f(0)=0$ and $f(x)=t^{v_p(x)}$ $\forall x\ne 0$ where $p$ is some prime and $t\in(0,1)$

And it's easy to check that both fit the requirement $f(\alpha + \beta) = \max\{f(\alpha), f(\beta)\} \text{ if } f(\alpha) \neq f(\beta)$
Hence the result.
\end{solution}
*******************************************************************************
-------------------------------------------------------------------------------

\begin{problem}[Posted by \href{https://artofproblemsolving.com/community/user/104682}{momo1729}]
	Find all functions $f:\mathbb{R} \to\mathbb{R}$ such that for all $x,y \in \mathbb{R}$ with $y\neq 0$,
\[f \left( x - f \left( \frac{x}{y} \right)\right)=xf\left(1-f\left(\frac{1}{y}\right)\right)\] and $f(1-f(1))\neq 0$.
	\flushright \href{https://artofproblemsolving.com/community/c6h449397}{(Link to AoPS)}
\end{problem}



\begin{solution}[by \href{https://artofproblemsolving.com/community/user/29428}{pco}]
	\begin{tcolorbox}Find all functions $f:\mathbb{R} \to\mathbb{R}$ such that 
$\forall x,y \in \mathbb{R}, y\neq 0$
$f(x-f(\frac{x}{y}))=xf(1-f(\frac{1}{y}))$ and $f(1-f(1))\neq 0$\end{tcolorbox}
Let $P(x,y)$ be the assertion $f(x-f(\frac xy))=xf(1-f(\frac 1y))$

$P(x,1)$ $\implies$ $f(x-f(x))=xf(1-f(1))$ and so $f(x)$ is surjective and so let $b$ such that $f(b)=1$

If $f(u)=0$, then $P(u,1)$ $\implies$ $uf(1-f(1))=0$ and so $u=0$ 
$P(0,1)$ $\implies$ $f(-f(0))=0$ and so $f(0)=0$

So $b\ne 0$ and let $a=\frac 1b$ : $P(ax,a)$ $\implies$ $f(ax-f(x))=0$ and so $\boxed{f(x)=ax}$ which indeed is a solution $\forall a\notin\{0,1\}$
\end{solution}
*******************************************************************************
-------------------------------------------------------------------------------

\begin{problem}[Posted by \href{https://artofproblemsolving.com/community/user/114585}{anonymouslonely}]
	Find all the functions  $ f $ defined on the real numbers excepting $ 0 $ and taking values real numbers ($ 0 $ is included) such that $ f(x+y)=f(x)+f(y) $, $ f(f(x))=x $ and the equation $ f(x)+x=0 $ has a finite number of solutions.
	\flushright \href{https://artofproblemsolving.com/community/c6h449881}{(Link to AoPS)}
\end{problem}



\begin{solution}[by \href{https://artofproblemsolving.com/community/user/29428}{pco}]
	\begin{tcolorbox}Find all the functions  $ f $ defined on the real numbers excepting $ 0 $ and taking values real numbers ($ 0 $ is included) such that $ f(x+y)=f(x)+f(y) $, $ f(f(x))=x $ and the equation $ f(x)+x=0 $ has a finite number of solutions.\end{tcolorbox}
Let $g(x)$ from $\mathbb R\to\mathbb R$ defined as $g(0)=0$ and $g(x)=f(x)$ $\forall x\ne 0$
Obviously $g(x)$ matches all the requirements.

Let $A=\{x$ such that $g(x)=x\}$ : $A$ is a $\mathbb Q$-vector space. Notice that $\frac{g(x)+x}2\in A$ $\forall x$
Let $B=\{x$ such that $g(x)=-x\}$ : $B$ is a $\mathbb Q$-vector space. Notice that $\frac{-g(x)+x}2\in B$ $\forall x$

Since $x=\frac{g(x)+x}2+\frac{-g(x)+x}2$, we get that $A,B$ are two supplementary vector subspaces of $\mathbb R$

And since $|B|$ is finite, then $B=\{0\}$ and so $A=\mathbb R$ and $g(x)=x$ $\forall x$

And so $\boxed{f(x)=x}$ $\forall x\ne 0$
\end{solution}
*******************************************************************************
-------------------------------------------------------------------------------

\begin{problem}[Posted by \href{https://artofproblemsolving.com/community/user/109292}{misterayyoub}]
	Find all functions $f: \mathbb{R}^+ \to \mathbb{R}^+ $, such that for all $x,y \in \mathbb{R}^+$  we have $f(xf(y))=f(xy) + x$.
	\flushright \href{https://artofproblemsolving.com/community/c6h450571}{(Link to AoPS)}
\end{problem}



\begin{solution}[by \href{https://artofproblemsolving.com/community/user/29428}{pco}]
	\begin{tcolorbox}Find all functions from R+ ---> R+ , such that for all x,y in R+ we have :
f(xf(y))=f(xy) + x\end{tcolorbox}
Let $P(x,y)$ be the assertion $f(xf(y))=f(xy)+x$

$P(x-f(0),0)$ $\implies$ $f((x-f(0))f(0))=x$ and so $f(x)$ is surjective

$P(1,x)$ $\implies$ $f(f(x))=f(x)+1$ and so, since surjective, $\boxed{f(x)=x+1}$ $\forall x$ which indeed is a soluton.
\end{solution}



\begin{solution}[by \href{https://artofproblemsolving.com/community/user/109292}{misterayyoub}]
	Excuse me , but im not that good in EF , so please how did u deduce f from the last assertion ?
\end{solution}



\begin{solution}[by \href{https://artofproblemsolving.com/community/user/64716}{mavropnevma}]
	\begin{bolded}pco\end{bolded} proved that $f(f(x))=f(x)+1$ for all $x \in \mathbb{R}_+$, and that $f$ is surjective. For any $y \in \mathbb{R}_+$ there thus exists some $x \in \mathbb{R}_+$ such that $y=f(x)$; plug this in the relation above to get $f(y) =y+1$, for all $y \in \mathbb{R}_+$. Hmmm ...
\end{solution}
*******************************************************************************
-------------------------------------------------------------------------------

\begin{problem}[Posted by \href{https://artofproblemsolving.com/community/user/129317}{Dranzer}]
	Find all functions $f:\mathbb{R}\to\mathbb{R}$ such that $x^2f(x)+f(1-x)=2x-x^4$ for all $x\in \mathbb{R}$.
	\flushright \href{https://artofproblemsolving.com/community/c6h450676}{(Link to AoPS)}
\end{problem}



\begin{solution}[by \href{https://artofproblemsolving.com/community/user/29428}{pco}]
	\begin{tcolorbox}Find all functions $f:\mathbb{R}\to\mathbb{R}$ such that $x^2f(x)+f(1-x)=2x-x^4$ for all $x\in \mathbb{R}$.\end{tcolorbox}
Let $P(x)$ be the assertion $x^2f(x)+f(1-x)=2x-x^4$

$P(x)$ $\implies$ $x^2f(x)+f(1-x)=2x-x^4$ and so $x^2(1-x)^2f(x)+(1-x)^2f(1-x)=(2x-x^4)(1-x)^2$ 
$P(1-x)$ $\implies$ $(1-x)^2f(1-x)+f(x)=2(1-x)-(1-x)^4$

Subtracting, we get $(x^2(1-x)^2-1)f(x)=(2x-x^4)(1-x)^2-2(1-x)+(1-x)^4$

and so $(x^2-x+1)(x^2-x-1)(f(x)+x^2-1)=0$ and so $f(x)=1-x^2$ $\forall x\ne\frac{1\pm\sqrt 5}2$

Plugging then any of the two remaining values of $x$ in original equation, we get $\boxed{f(x)=1-x^2}$ $\forall x$ which indeed is a solution.
\end{solution}
*******************************************************************************
-------------------------------------------------------------------------------

\begin{problem}[Posted by \href{https://artofproblemsolving.com/community/user/1098}{Parkdoosung}]
	Find all functions $f:\mathbb{R}\rightarrow\mathbb{R}$ such that
\[f(x^2-y^2)=f(x-y)(f(x)+f(y))\]
for all reals $x,y$.
	\flushright \href{https://artofproblemsolving.com/community/c6h451763}{(Link to AoPS)}
\end{problem}



\begin{solution}[by \href{https://artofproblemsolving.com/community/user/29428}{pco}]
	\begin{tcolorbox}Find all functions $f:\mathbb{R}\rightarrow\mathbb{R}$ such that
\[f(x^2-y^2)=f(x-y)(f(x)+f(y))\]
for all reals $x,y$.\end{tcolorbox}
Let $P(x,y)$ be the assertion $f(x^2-y^2)=f(x-y)(f(x)+f(y))$
Comparing $P(x,y)$ and $P(x,-y)$, we get new assertion $Q(x,y)$ : $f(x-y)(f(x)+f(y))=f(x+y)(f(x)-f(y))$

If $f(0)\ne 0$, then $P(x,x)$ $\implies$ $f(x)=\frac 12$ $\forall x$ which indeed is a solution.
Let us from now consider $f(0)=0$

If $f(u)=0$ for some $u\ne 0$, then $P(\frac12(\frac xu+u),\frac12(\frac xu-u))$ $\implies$ $f(x)=0$ $\forall x$ which indeed is a solution
Let us from now consider $f(x)=0$ $\iff$ $x=0$

Let $x\ne 0$ : $P(x,-x)$ $\implies$ $f(2x)(f(x)+f(-x))=0$ and so $f(-x)=-f(x)$ $\forall x\ne 0$ and so $f(-x)=-f(x)$ $\forall x$

Let $x>0$ : $P(\sqrt x,0)$ $\implies$ $f(x)=2f(\sqrt x)^2>0$ and so $f(x)>0$ $\forall x>0$ and so $f(x)<0$ $\forall x<0$

Let $x>y>0$ : $Q(x,y)$ $\implies$ $f(x)>f(y)$ and it's easy from there to get that $f(x)$ is stricly increasing.

$P(1,0)$ $\implies$ $f(1)=1$
$P(x,1-x)$ $\implies$ $f(2x-1)=f(2x-1)(f(x)+f(1-x))$ and so $f(1-x)=1-f(x)$ $\forall x\ne \frac 12$ or $f(x-1)=f(x)-1$ $\forall x\ne \frac 12$
$P(x,x-1)$ $\implies$ $f(2x-1)=f(x)+f(x-1)$ and so $f(2x)-1=f(x)+f(x)-1$ $\forall x\notin\{\frac 14,\frac 12\}$
So $f(2x)=2f(x)$ $\forall x\notin\{\frac 14,\frac 12\}$

using $Q(nx,x)$ it's immediate to get with induction $f(nx)=nf(x)$ $\forall n\in\mathbb Z$, $\forall x\notin\{\frac 14,\frac 12\}$
Using $x=\frac 18$ and $n=8$ and then $x=\frac 18$ and $n=2$, we get $f(\frac 14)=\frac 14$ and then $f(\frac 12)=\frac 12$

And so we get $f(nx)=nf(x)$ $\forall n\in\mathbb Z$, $\forall x$
So $f(\frac pqx)=\frac pqf(x)$ and so $f(x)=x$ $\forall x\in\mathbb Q$

And since $f(x)$ is increasing, we get $f(x)=x$ $ \forall x$ which indeed is a solution.

\begin{bolded}Hence the three solutions\end{underlined}\end{bolded} :
$f(x)=0$ $\forall x$
$f(x)=\frac 12$ $\forall x$
$f(x)=x$ $\forall x$
\end{solution}



\begin{solution}[by \href{https://artofproblemsolving.com/community/user/91362}{goldeneagle}]
	\begin{tcolorbox}[quote="Parkdoosung"]Find all functions $f:\mathbb{R}\rightarrow\mathbb{R}$ such that
\[f(x^2-y^2)=f(x-y)(f(x)+f(y))\]
for all reals $x,y$.\end{tcolorbox}
Let $P(x,y)$ be the assertion $f(x^2-y^2)=f(x-y)(f(x)+f(y))$
Comparing $P(x,y)$ and $P(x,-y)$, we get new assertion $Q(x,y)$ : $f(x-y)(f(x)+f(y))=f(x+y)(f(x)-f(y))$

If $f(0)\ne 0$, then $P(x,x)$ $\implies$ $f(x)=\frac 12$ $\forall x$ which indeed is a solution.
Let us from now consider $f(0)=0$

If $f(u)=0$ for some $u\ne 0$, then $P(\frac12(\frac xu+u),\frac12(\frac xu-u))$ $\implies$ $f(x)=0$ $\forall x$ which indeed is a solution
Let us from now consider $f(x)=0$ $\iff$ $x=0$

Let $x\ne 0$ : $P(x,-x)$ $\implies$ $f(2x)(f(x)+f(-x))=0$ and so $f(-x)=-f(x)$ $\forall x\ne 0$ and so $f(-x)=-f(x)$ $\forall x$

Let $x>0$ : $P(\sqrt x,0)$ $\implies$ $f(x)=2f(\sqrt x)^2>0$ and so $f(x)>0$ $\forall x>0$ and so $f(x)<0$ $\forall x<0$

Let $x>y>0$ : $Q(x,y)$ $\implies$ $f(x)>f(y)$ and it's easy from there to get that $f(x)$ is stricly increasing.

$P(1,0)$ $\implies$ $f(1)=1$
$P(x,1-x)$ $\implies$ $f(2x-1)=f(2x-1)(f(x)+f(1-x))$ and so $f(1-x)=1-f(x)$ $\forall x\ne \frac 12$ or $f(x-1)=f(x)-1$ $\forall x\ne \frac 12$
$P(x,x-1)$ $\implies$ $f(2x-1)=f(x)+f(x-1)$ and so $f(2x)-1=f(x)+f(x)-1$ $\forall x\notin\{\frac 14,\frac 12\}$
So $f(2x)=2f(x)$ $\forall x\notin\{\frac 14,\frac 12\}$

using $Q(nx,x)$ it's immediate to get with induction $f(nx)=nf(x)$ $\forall n\in\mathbb Z$, $\forall x\notin\{\frac 14,\frac 12\}$
Using $x=\frac 18$ and $n=8$ and then $x=\frac 18$ and $n=2$, we get $f(\frac 14)=\frac 14$ and then $f(\frac 12)=\frac 12$

And so we get $f(nx)=nf(x)$ $\forall n\in\mathbb Z$, $\forall x$
So $f(\frac pqx)=\frac pqf(x)$ and so $f(x)=x$ $\forall x\in\mathbb Q$

And since $f(x)$ is increasing, we get $f(x)=x$ $ \forall x$ which indeed is a solution.

\begin{bolded}Hence the three solutions\end{underlined}\end{bolded} :
$f(x)=0$ $\forall x$
$f(x)=\frac 12$ $\forall x$
$f(x)=x$ $\forall x$\end{tcolorbox}

The idea of my solution is  simillar to you when $f(0)=0$. But when we have $f(0)= \frac 12$ I think your solution has a problem because :
$P(x,y), P(x,-y) \Rightarrow  f(x-y)(f(x)+f(y))=f(x+y)(f(x)+f(-y))$

My solution for this part : If there exists $a \in \mathbb{R}$ such that $f(a)=0$ then $P(x+a,x) \Rightarrow f(a(2x+a))=0$ so because $a \neq 0 $ then $f(x)=0 \forall x \in \mathbb{R}$ contradiction !

$P(0,x), P(0,-x) \Rightarrow f(-x)(f(x)+\frac 12)= f(x)(f(-x)+ \frac 12)$ so $f(-x)=f(x)$ 

$P(x,y) , P(x,-y) \Rightarrow f(x+y)=f(x-y) \Rightarrow f(x)= \frac 12 \forall x \in \mathbb{R}$ 


\end{solution}



\begin{solution}[by \href{https://artofproblemsolving.com/community/user/29428}{pco}]
	\begin{tcolorbox} 
The idea of my solution is  simillar to you when $f(0)=0$. But when we have $f(0)= \frac 12$ I think your solution has a problem because ...\end{tcolorbox}
I understand nothing to your post.
My proof is elementary (see first lines) : if $f(0)\ne 0$, then $P(x,x)$  $\implies$ $f(x)=\frac 12$ $\forall x$

What's wrong with this ?
\end{solution}
*******************************************************************************
-------------------------------------------------------------------------------

\begin{problem}[Posted by \href{https://artofproblemsolving.com/community/user/125018}{horizon}]
	Find all functions $f: \mathbb R \to \mathbb R$ which satisfy for all $x, y \in \mathbb R$,
\[f(x^{2}+f(y))=xf(x)+y.\]
	\flushright \href{https://artofproblemsolving.com/community/c6h453196}{(Link to AoPS)}
\end{problem}



\begin{solution}[by \href{https://artofproblemsolving.com/community/user/29428}{pco}]
	\begin{tcolorbox}find all function:$f$:R to R,such that for any real number $x,y$,we have
$f(x^{2}+f(y))=xf(x)+y$\end{tcolorbox}
Let $P(x,y)$ be the assertion $f(x^2+f(y))=xf(x)+y$

$P(0,x)$ $\implies$ $f(f(x))=x$ and so $f(x)$ is bijective

$P(x,0)$ $\implies$ $f(x^2+f(0))=xf(x)$
$P(f(x),0)$ $\implies$ $f(f(x)^2+f(0))=f(x)f(f(x))=xf(x)$ 
So $f(x^2+f(0))=f(f(x)^2+f(0))$ and, since injective $f(x)^2=x^2$ and so $\forall x$ : either $f(x)=x$, either $f(x)=-x$

Suppose now that $\exists a,b$ both $\ne 0$ such that $f(a)=a$ and $f(b)=-b$
$P(a,b)$ $\implies$ $f(a^2-b)=a^2+b$ but :
either $f(a^2-b)=a^2-b$ and so $b=0$, impossible
either $f(a^2-b)=-a^2+b$ and so $a=0$, impossible

And so :
Either $f(x)=x$ $\forall x$ which indeed is a solution.
Either $f(x)=-x$ $\forall x$ which indeed is a solution.
\end{solution}
*******************************************************************************
-------------------------------------------------------------------------------

\begin{problem}[Posted by \href{https://artofproblemsolving.com/community/user/128206}{eddy13579}]
	Determine all functions $f: \mathbb N \to \mathbb N$ such that:
\[\frac{f(x)+y}{x+f(y)} + \frac{f(x)y}{xf(y)} = \frac{2(x+y)}{f(x+y)} \]
for any $x, y \in \mathbb N$.
	\flushright \href{https://artofproblemsolving.com/community/c6h454392}{(Link to AoPS)}
\end{problem}



\begin{solution}[by \href{https://artofproblemsolving.com/community/user/29428}{pco}]
	\begin{tcolorbox}Determine all functions $f:N->N $ such that:
$ \frac{f(x)+y}{x+f(y)} + \frac{f(x)*y}{x*f(y)} = \frac{2(x+y)}{f(x+y)} $ for any x,y \begin{bolded}natural numbers\end{bolded} different by 0.\end{tcolorbox}
Let $P(x,y)$ be the assertion $\frac{f(x)+y}{x+f(y)}+\frac{yf(x)}{xf(y)}=2\frac{x+y}{f(x+y)}$

$P(x,x)$ $\implies$ $f(2x)=2x$

$P(1,2)$ $\implies$ $f(3)=\frac{9}{2f(1)+1}$ and so $f(1)\in\{1,4\}$

If $f(1)=4$, then $P(1,4)$ $\implies$ $f(5)=\frac{25}{14}\notin\mathbb N$

If $f(1)=1$, then $P(1,2x)$ $\implies$ $f(2x+1)=2x+1$ and so $\boxed{f(x)=x}$ $\forall x$ which indeed is a solution.
\end{solution}
*******************************************************************************
-------------------------------------------------------------------------------

\begin{problem}[Posted by \href{https://artofproblemsolving.com/community/user/107185}{mymath7}]
	Find all functions $f:\mathbb{R}\rightarrow\mathbb{R}$ such that
\[ f(f(x)^2+y) = xf(x)+y \]
for all $x,y\in\mathbb{R}$
	\flushright \href{https://artofproblemsolving.com/community/c6h454479}{(Link to AoPS)}
\end{problem}



\begin{solution}[by \href{https://artofproblemsolving.com/community/user/29428}{pco}]
	\begin{tcolorbox}Find all functions $f:\mathbb{R}\rightarrow\mathbb{R}$ such that
\[ f(f(x)^2+y) = xf(x)+y \]
for all $x,y\in\mathbb{R}$\end{tcolorbox}
Let $P(x,y)$ be the assertion $f(f(x)^2+y)=xf(x)+y$
Let $a=-f(0)^2$

$P(0,x+a)$ $\implies$ $f(x)=x+a$

Plugging this in original equation, we get $a=0$ and so the unique solution $\boxed{f(x)=x}$
\end{solution}
*******************************************************************************
-------------------------------------------------------------------------------

\begin{problem}[Posted by \href{https://artofproblemsolving.com/community/user/128206}{eddy13579}]
	Determine all functions $ f:(0,\infty) \to (0,\infty) $ which satisfy
\[f\left(y \cdot f\left(\frac{x}{y}\right)\right)=\frac{x^4}{f(y)} \]
for any positive $x$ and $y$.
	\flushright \href{https://artofproblemsolving.com/community/c6h454506}{(Link to AoPS)}
\end{problem}



\begin{solution}[by \href{https://artofproblemsolving.com/community/user/29428}{pco}]
	\begin{tcolorbox}\begin{bolded}Determine all functions\end{bolded} $ f:(0,\infty)->(0,\infty) $ which satisfy:
$ f(y*f(\frac{x}{y}))=\frac{x^4}{f(y)} $ for any pozitive x,y.\end{tcolorbox}
Let $P(x,y)$ be the assertion $f(yf(\frac xy))=\frac{x^4}{f(y)}$

$P(x,1)$ $\implies$ $f(f(x))=\frac{x^4}{f(1)}$ and so $f(x)$ is inejctive

$P(1,x)$ $\implies$ $f(xf(\frac 1x))=\frac{1}{f(x)}$

$P(x,f(x))$ $\implies$ $f(f(x)f(\frac x{f(x)}))=\frac{x^4}{f(f(x))}=f(1)$ and so, since injective, $f(x)f(\frac x{f(x)})=1$

So $f(\frac x{f(x)})=\frac 1{f(x)}=f(xf(\frac 1x))$ and so, since injective, $\frac x{f(x)}=xf(\frac 1x)$ and $f(\frac 1x)=\frac 1{f(x)}$

$P(1,\frac 1x)$ $\implies$ $f(\frac{f(x)}x)=\frac{1}{f(\frac 1x)}=f(x)$ and so, since injective, $\frac{f(x)}x=x$ and $\boxed{f(x)=x^2}$ which indeed is a solution
\end{solution}



\begin{solution}[by \href{https://artofproblemsolving.com/community/user/82334}{bappa1971}]
	Let, $P(x,y) \Longrightarrow  f(yf(\frac{x}{y}))=\frac{x^{4}}{f(y)} $
$P(x,1) \Longrightarrow f(f(x))=\frac{x^{4}}{f(1)} $, So, $f$ is surjective.
Let, $f(u)=1$
Then, $P(u x,x) \Longrightarrow f(x)=\frac{x^4 u^4}{f(x)}$
So, $f(x)=u^2 x^2$
Now, $f(u)=u^4=1 \Longrightarrow u=1$
Hence, $f(x)=x^2$
\end{solution}
*******************************************************************************

-------------------------------------------------------------------------------

\begin{problem}[Posted by \href{https://artofproblemsolving.com/community/user/46488}{Raja Oktovin}]
	Find all functions $ f: \mathbb{R} \rightarrow \mathbb{R}$ satisfying \[ f(x^3+y^3)=xf(x^2)+yf(y^2)\] for all real numbers $ x$ and $ y$.
\begin{italicized}Hery Susanto, Malang\end{italicized}
	\flushright \href{https://artofproblemsolving.com/community/c6h311512}{(Link to AoPS)}
\end{problem}



\begin{solution}[by \href{https://artofproblemsolving.com/community/user/60946}{matrix41}]
	Set $ y=0$ then

$ f(x^3)=xf(x^2)$ , so $ f(y^3)=yf(y^2)$ , then

$ f(x^3+y^3)=xf(x^2)+yf(y^2)=f(x^3)+f(y^3)$ $ \rightarrow$ $ f(x)+f(y)=f(x+y)$

simple induction gives : $ f(x_1+x_2+...+x_n)=f(x_1)+f(x_2)+f(x_3)+...+f(x_n)$  so $ \forall \ n\in\mathbb{R}$ 

then $ f(nx)=nf(x)$ and thus by setting $ x=1$ we have $ f(n)=nf(1)$, since $ f(1)$ is a constant then by assuming $ f(1)=a$ , gives $ f(n)=an$ with $ a\in\mathbb{R}$ 

done
\end{solution}



\begin{solution}[by \href{https://artofproblemsolving.com/community/user/34380}{math10}]
	\begin{tcolorbox}Set $ y = 0$ then

$ f(x^3) = xf(x^2)$ , so $ f(y^3) = yf(y^2)$ , then

$ f(x^3 + y^3) = xf(x^2) + yf(y^2) = f(x^3) + f(y^3)$ $ \rightarrow$ $ f(x) + f(y) = f(x + y)$

simple induction gives : $ f(x_1 + x_2 + ... + x_n) = f(x_1) + f(x_2) + f(x_3) + ... + f(x_n)$  so $ \forall \ n\in\mathbb{R}$ then $ f(nx) = nf(x)$ \end{tcolorbox}
Why for all $ n \in R$  :?: 

\begin{tcolorbox}Set $ y = 0$ then

$ f(x^3) = xf(x^2)$ , so $ f(y^3) = yf(y^2)$ , then

$ f(x^3 + y^3) = xf(x^2) + yf(y^2) = f(x^3) + f(y^3)$ $ \rightarrow$ $ f(x) + f(y) = f(x + y)$

simple induction gives : $ f(x_1 + x_2 + ... + x_n) = f(x_1) + f(x_2) + f(x_3) + ... + f(x_n)$  so $ \forall \ n\in\mathbb{R}$ 
then $ f(nx) = nf(x)$ and thus by setting $ x = 1$ we have $ f(n) = nf(1)$, since $ f(1)$ is a constant then by assuming $ f(1) = a$ , gives $ f(n) = an$ with $ a\in\mathbb{R}$ 

done\end{tcolorbox}
you only solve equation on $ N$ not $ R$
We have:$ f(x + y) = f(x) + f(y)$ and $ f(x^3) = xf(x^2)$,$ (x) = - f( - x)$,$ f(kx) = kf(x)$ for all $ k \in N$
We have:
 $ f((x + 1)^3 + (x - 1)^3) = (x + 1)f(x^2 + 2x + 1) + (x - 1)f(x^2 - 2x + 1) = 2xf(x^2) + 2xf(1) + 4f(x)$
and $ f((x + 1)^3 + (x - 1)^3) = f(2x^3 + 6x) = 2xf(x^2) + 6f(x)$
So $ f(x) = xf(1)$ for all $ x \in R$
\end{solution}



\begin{solution}[by \href{https://artofproblemsolving.com/community/user/60946}{matrix41}]
	Sorry there are some mistakes in my solution

Let me write my full solution

First from my first post, we have 

$ f(x_1)+f(x_2)+...+f(x_n)=f(x_1+x_2+x_3+...+x_n)$ with $ n\in\mathbb{N}$ so $ f(nx)=nf(x)$ $ \forall n\in\mathbb{N}$

and then by setting $ x=m+1$ and $ y=m-1$ it gives

$ f((m+1)^3+(m-1)^3)=f((m^3+3m^3+3m+1)+(m^3-3m^3+3m-1))=f(2m^3+6m)=f(2m^3)+f(6m)=2f(m^3)+6f(m)$

and

$ f((m+1)^3+(m-1)^3)=(m+1)f(m^2+2x+1)+(m-1)f(x^2-2x+1)=(m+1)\left(f(m^2)+f(2m)+f(1)\right)+(m-1)\left(f(m^2)+f(-2m)+f(1)\right)=2mf(m^2)+4f(m)+2mf(1)$

since $ 2mf(m^2)=2f(m^3)$ , so $ f((m+1)^3+(m-1)^3)=2mf(m^2)+4f(m)+2mf(1)=2f(m^3)+4f(m)+2mf(1)$

$ 2f(m^3)+6f(m)=f((m+1)^3+(m-1)^3)=2f(m^3)+4f(m)+2mf(1)$

$ 2f(m)=2mf(1)$ $ \rightarrow$ $ f(m)=mf(1)$ $ \forall m\in\mathbb{R}$

please check....

EDITED : Sorry I didn't know that my solution is similar to math10's solution  :oops:
\end{solution}



\begin{solution}[by \href{https://artofproblemsolving.com/community/user/109774}{littletush}]
	not so hard.
let $y=0$,then
$f(x^3)=xf(x^2)$
so $f(x^3+y^3)=f(x^3)+f(y^3)$
hence f satisfies Cauchy's  function,so for rational x,$f(x)=x$
hence $f(1)=1$
by letting $x=x+1$ we get
$f((x+1)^3)=(x+1)f((x+1)^2)$
hence $2f(x^2)=(2x-1)f(x)+x$
let $x=x+1$
we get $2(f(x^2)+2f(x)+1)=(2x+1)(f(x)+1)+x+1$
hence $f(x)=x$.
\end{solution}



\begin{solution}[by \href{https://artofproblemsolving.com/community/user/29428}{pco}]
	\begin{tcolorbox} ... hence f satisfies Cauchy's  function,\end{tcolorbox}Right
\begin{tcolorbox}so for rational x,$f(x)=x$\end{tcolorbox}Wrong : $f(x)=ax$
\end{solution}



\begin{solution}[by \href{https://artofproblemsolving.com/community/user/109774}{littletush}]
	\begin{tcolorbox}[quote="littletush"] ... hence f satisfies Cauchy's  function,\end{tcolorbox}Right
\begin{tcolorbox}so for rational x,$f(x)=x$\end{tcolorbox}Wrong : $f(x)=ax$\end{tcolorbox}
oh gush!such a terrible mistake!
a can be any real number.
\end{solution}



\begin{solution}[by \href{https://artofproblemsolving.com/community/user/215362}{nawaites}]
	So which solution is correct?????
\end{solution}



\begin{solution}[by \href{https://artofproblemsolving.com/community/user/260346}{Takeya.O}]
	It is easy to show that ∃$a\in \mathbb R$ s.t. $f(x)=ax(\forall x\in \mathbb Q)$.

If we show that $f$ is
-continuous at some point
-monotonous
-either upperbounded or lowerbounded on some open interval

$f(x)=ax(\forall x\in \mathbb R)$ :P

Anyone has idea? :?
\end{solution}



\begin{solution}[by \href{https://artofproblemsolving.com/community/user/29428}{pco}]
	\begin{tcolorbox}Find all functions $ f: \mathbb{R} \rightarrow \mathbb{R}$ satisfying \[ f(x^3+y^3)=xf(x^2)+yf(y^2)\] for all real numbers $ x$ and $ y$.
\begin{italicized}Hery Susanto, Malang\end{italicized}\end{tcolorbox}

Let $P(x,y)$ be the assertion $f(x^3+y^3)=xf(x^2)+yf(y^2)$
Let $a=f(1)$

$P(0,0)$ $\implies$ $f(0)=0$
$P(x,0)$ $\implies$ $f(x^3)=xf(x^2)$
And so $f(x^3+y^3)=f(x^3)+f(y^3)$ and so $f(x)$ is additive.

So $f(px)=pf(x)$ $\forall x$ and $\forall p\in\mathbb Q$

Let $x\in\mathbb R$ and $k\in\mathbb Q$
$P(x+k,0)$ $\implies$ $f(x^3+3kx^2+3k^2x+k^3)=(x+k)f(x^2+2kx+k^2)$

Which may be written $k^2(f(x)-ax)+2k(f(x^2)-xf(x))=0$

Considering this as a polynomial in $k$ with infinitely many roots (any rational), we get thet it must be the zero polynomial and so, looking at coefficient of $k^2$ :

$\boxed{f(x)=ax\text{  }\forall x}$ which indeed is a solution, whatever is $a\in\mathbb R$


\end{solution}



\begin{solution}[by \href{https://artofproblemsolving.com/community/user/260515}{anhtaitran}]
	My solution:Plug x=y=0 so f(0)=0.
Plug x=0 so f(x^3)=xf(x^2).(1)
So f(x^3+y^3)=f(x^3)+f(y^3) for every x;y real.
So f is addictive.
Now we will caculate f( (x+1)^3+(x-1)^3) in 2 ways.
f( (x+1)^3+(x-1)^3) = (x+1)f((x+1)^2)+(x-1)f((x-1)^2).
                               =(x+1)[f(x^2)+2f(x)+f(1)]+(x-1)(f(x^2)-2f(x)+f(1)].
                              =2xf(x^2)+2xf(1)+4f(x).(2)
On the other hand,
f( (x+1)^3+(x-1)^3) =f(2x^3+6x)=2f(x^3)+6f(x).(3)
By (1);(2) and (3) we have :
f(x)=cx(c=f(1)).

\end{solution}



\begin{solution}[by \href{https://artofproblemsolving.com/community/user/260346}{Takeya.O}]
	@pco,@anhtaitran
What a nice solution! :w00t: Are you a FE master? 
\end{solution}



\begin{solution}[by \href{https://artofproblemsolving.com/community/user/337737}{Evenprime123}]
	\begin{tcolorbox}
And so $f(x^3+y^3)=f(x^3)+f(y^3)$ and so $f(x)$ is additive.

So $f(px)=pf(x)$ $\forall x$ and $\forall p\in\mathbb Q$
\end{tcolorbox}

How do I conclude this? Induction would work on \(\mathbb N \) but what can I do here?
\end{solution}



\begin{solution}[by \href{https://artofproblemsolving.com/community/user/29428}{pco}]
	\begin{tcolorbox}[quote=pco]
And so $f(x^3+y^3)=f(x^3)+f(y^3)$ and so $f(x)$ is additive.

So $f(px)=pf(x)$ $\forall x$ and $\forall p\in\mathbb Q$
\end{tcolorbox}

How do I conclude this? Induction would work on \(\mathbb N \) but what can I do here?\end{tcolorbox}
You should consider this as a well-known property of additive functions.

If you want to prove it :
$f(2x)=f(x)+f(x)=2f(x)$
$f(3x)=f(2x)+f(x)=3f(x)$
And, with induction : $f(nx)=nf(x)$ $\forall x$, $\forall n\in\mathbb N$

So $f(x)=f(q\frac xq)=qf(\frac xq)$ and so $f(\frac xq)=\frac 1qf(x)$ $\forall x$, $\forall q\in\mathbb N$

So $f(\frac pqx)=pf(\frac xq)=\frac pqf(x)$ and so $f(px)=pf(x)$ $\forall x$, $\forall p\in\mathbb Q^+$

It remains to remember that $f(-x)=-f(x)$ and $f(0)=0$ and so 
$f(px)=pf(x)$ $\forall x$, $\forall p\in\mathbb Q$

\end{solution}



\begin{solution}[by \href{https://artofproblemsolving.com/community/user/180207}{GeronimoStilton}]
	[color=#f00]redacted[\/color]
\end{solution}



\begin{solution}[by \href{https://artofproblemsolving.com/community/user/29428}{pco}]
	\begin{tcolorbox}Some progress:
...
... so I can't finish the problem.\end{tcolorbox}
Sorry, but what is the interest of this post ?
Just publicly claim that you are working on this problem ? And what about your progress about your sister's garden ? and what is the last film you liked ?
Just request for some help ? : read the thread ! there is already a full solution (see post #10)
Just bring some new informations to the community ? but this contribution already has been given at least thrice in the current thread (and btw is the beginning of the full solution in the post #10)



\end{solution}
*******************************************************************************
-------------------------------------------------------------------------------

\begin{problem}[Posted by \href{https://artofproblemsolving.com/community/user/46840}{behdad.math.math}]
	Find all functions $ f: R^+ \longrightarrow R^+$ such that:

$ f$$ (\frac{x+y}{2})$ $ =$ $ \frac{2f(x)f(y)}{f(x)+f(y)}$
	\flushright \href{https://artofproblemsolving.com/community/c6h321965}{(Link to AoPS)}
\end{problem}



\begin{solution}[by \href{https://artofproblemsolving.com/community/user/37328}{KenHungKK}]
	Let $ g(x)=\frac{1}{f(x)}$, $ g(1)=a$ and $ g(2)=a+m$.
Hence we have
\[ g(x)+g(y)=2g(\frac{x+y}{2})\]
Alternatively, we can write this as
\[ g(a+d)-g(a)=g(a+2d)-g(a+d)\]
Now for all $ q$, assume that $ g(1+\frac{1}{q})\ne a+\frac{q}{m}$.
\begin{align*}g(2)-g(1+\frac{q-1}{q})&=g(1+\frac{q-1}{q})-g(1+\frac{q-2}{q})\\&=g(1+\frac{q-2}{q})-g(1+\frac{q-3}{q})\\&=\vdots\\&=g(1+\frac{1}{q})-g(1)\end{align*}
Hence, $ g(2)\ne g(1)+q\frac{m}{q}=a+m$, leading to a contradiction. So we prove that the function is linear for the rational numbers $ q$ within the interval $ [1,2]$. With similar arguments, we can prove that for all rational numbers, the function is linear. Hence, for rational numbers $ q$, we have $ g(q)=mq+c$.
Suppose there is a number $ \alpha$ that $ g(\alpha)\ne m\alpha+c$. There exists a rational number $ q>\alpha$ sufficiently close to $ \alpha$, let $ q-\alpha=\epsilon$. If $ g(q)<g(\alpha)$, $ g(\alpha)-g(\alpha+\epsilon)=g(\alpha+\epsilon)-g(\alpha+2\epsilon)=\cdots=g(\alpha+(n-1)\epsilon)-g(\alpha+n\epsilon)$ for all integer $ n$. Now for sufficiently large $ n$, $ g(\alpha+n\epsilon)=g(\alpha)-n(\alpha-g(\alpha+\epsilon))<0$, leading to a contradiction. Hence, for all rational number within a sufficiently close interval $ q>\alpha$, $ g(q)>g(\alpha)$. By similar argument, for all rational number within a sufficiently close interval $ q<\alpha$, $ g(q)<g(\alpha)$. Hence, by taking sufficiently close $ q$, $ g(\alpha)=m\alpha+c$, leading to a contradiction. And $ g(\alpha)=m\alpha+c$.
Now $ f(x)=\frac{1}{g(x)}=\frac{1}{mx+c}$
\end{solution}



\begin{solution}[by \href{https://artofproblemsolving.com/community/user/152948}{mohammadlari}]
	no more solutions ?
\end{solution}



\begin{solution}[by \href{https://artofproblemsolving.com/community/user/236427}{AmirAlison}]
	We take $g(x)$ and get the equation written above. It means that the function is either convex or concave. So it's a line. Am I right? If not then explain my mistake, please
\end{solution}



\begin{solution}[by \href{https://artofproblemsolving.com/community/user/212018}{Tintarn}]
	\begin{tcolorbox}We take $g(x)$ and get the equation written above. It means that the function is either convex or concave. So it's a line. Am I right? If not then explain my mistake, please\end{tcolorbox}

Technically, this is right. But how do you proceed formally? You would have to prove that the only functions which are both convex and concave are the linear functions.
If $f$ is twice differentiable then this is clear since $f'' \ge 0$ by convexity and $f'' \le 0$ by concavity and hence $f''=0$ which means $f(x)=mx+c$. But how would you go on if $f$ is not differentiable? Probably your formal argument would be somewhat similar to KenHungKK's one...And this is not really a one-liner...

\begin{italicized}Remark\end{italicized}: Actually, even if $f$ is twice differentiable the one-line-argument is somewhat cheated since it requires the (non-trivial) equivalence of the two definitions (or properties, if you want) of twice differentiable convex functions: $\left( \forall x,y: f(x+y)=2f \left( \frac{x+y}{2} \right) \right) \Leftrightarrow \left( \forall x: f''(x) \ge 0 \right) $.

\end{solution}



\begin{solution}[by \href{https://artofproblemsolving.com/community/user/231508}{john10}]
	If we take f(x)=1\/g(x) , it follows that g is Jensen's equation , thus g(x)=ax+b , so f(x)=1\/(ax+b) . done !
\end{solution}



\begin{solution}[by \href{https://artofproblemsolving.com/community/user/231691}{nbasrl}]
	pco or anyone .. coulde you solve this ? i dont fully understand the first poster's solution . thanks ! and sorry for reviving

\end{solution}



\begin{solution}[by \href{https://artofproblemsolving.com/community/user/228888}{yojan_sushi}]
	I believe there is a typo in KenHungKK's solution; the equation $g(\alpha+n\epsilon)=g(\alpha)-n(\alpha-g(\alpha+\epsilon))$ should read \[g(\alpha+n\epsilon)=g(\alpha)-n(g(\alpha)-g(\alpha+\epsilon)).\] He found this equation by adding the $n$ equal differences $g(\alpha)-g(\alpha+\epsilon), g(\alpha+\epsilon)-g(\alpha+2\epsilon), \ldots, g(\alpha+(n-1)\epsilon)-g(\alpha+n\epsilon)$ to yield $n(g(\alpha)-g(\alpha+\epsilon))= g(\alpha) - g(\alpha+n\epsilon),$ from which the above equation follows.
\end{solution}



\begin{solution}[by \href{https://artofproblemsolving.com/community/user/391068}{TuZo}]
	\begin{bolded}Remark:\end{bolded}
The equation  $g(x)+g(y)=2g(\frac{x+y}{2})$ is Jensen-type functional equation, and the solution is the following:
1) Put instead of $x$ the $x+y$, we get: $g\left( \frac{x+2y}{2} \right)=\frac{g(x+y)+g(y)}{2}$
2) If we put $y=0$, we get: $g\left( \frac{x}{2} \right)=\frac{g(x)+g(0)}{2}$
3) Replace $x$ by $x+y$, we get: $g\left( \frac{x+y}{2} \right)=\frac{g(x+y)+g(0)}{2}$
4) On the basis of the proposed equation we get: $g(x+y)+g(0)=g(x)+g(y)$, and this is a Cauchy-type functional equation, which the continuouse solutions $g(x)=ax$.
\end{solution}



\begin{solution}[by \href{https://artofproblemsolving.com/community/user/29428}{pco}]
	\begin{tcolorbox}We take $g(x)$ and get the equation written above. It means that the function is either convex or concave. So it's a line. Am I right? If not then explain my mistake, please\end{tcolorbox}
Not so simple.
Any non linear additive function $g(x)$ matches the equality. So the conclusion needs to add the "lowerbounded" constraint (and so is not so immediate)

\end{solution}



\begin{solution}[by \href{https://artofproblemsolving.com/community/user/29428}{pco}]
	\begin{tcolorbox}\begin{bolded}Remark:\end{bolded}
The equation  $g(x)+g(y)=2g(\frac{x+y}{2})$ is Jensen-type functional equation, and the solution is the following:
1) Put instead of $x$ the $x+y$, we get: $g\left( \frac{x+2y}{2} \right)=\frac{g(x+y)+g(y)}{2}$
2) If we put $y=0$, we get: $g\left( \frac{x}{2} \right)=\frac{g(x)+g(0)}{2}$
3) Replace $x$ by $x+y$, we get: $g\left( \frac{x+y}{2} \right)=\frac{g(x+y)+g(0)}{2}$
4) On the basis of the proposed equation we get: $g(x+y)+g(0)=g(x)+g(y)$, and this is a Cauchy-type functional equation, which the continuouse solutions $g(x)=ax$.\end{tcolorbox}

This is the classical explanation.
But this classical explanation here must be adapted since $0$ does not belong to domain of function.

\end{solution}



\begin{solution}[by \href{https://artofproblemsolving.com/community/user/391068}{TuZo}]
	Oh yeah, I see now!
\end{solution}
*******************************************************************************
-------------------------------------------------------------------------------

\begin{problem}[Posted by \href{https://artofproblemsolving.com/community/user/203965}{wanwan4343}]
	Find all functions $f:\mathbb{Q}\rightarrow\mathbb{R} \setminus \{ 0 \}$ such that 
\[(f(x))^2f(2y)+(f(y))^2f(2x)=2f(x)f(y)f(x+y)\]
for all $x,y\in\mathbb{Q}$
	\flushright \href{https://artofproblemsolving.com/community/c6h1113618}{(Link to AoPS)}
\end{problem}



\begin{solution}[by \href{https://artofproblemsolving.com/community/user/29428}{pco}]
	\begin{tcolorbox}Find all functions $f:\mathbb{Q}\rightarrow\mathbb{R} \setminus \{ 0 \}$ such that 
\[(f(x))^2f(2y)+(f(y))^2f(2x)=2f(x)f(y)f(x+y)\]
for all $x,y\in\mathbb{Q}$\end{tcolorbox}

Let $P(x,y)$ be the assertion $f(x)^2f(2y)+(f(y)^2f(2x)=2f(x)f(y)f(x+y)$

$f(x)$ solution implies $cf(x)$ solution and so WLOG $f(0)=1$ (since $0\notin f(\mathbb Q)$)

$P(x,0)$ $\implies$  $f(2x)=f(x)^2$ and $P(x,y)$ becomes $f(x+y)=f(x)f(y)$

Hence the solution : $\boxed{f(x)=b e^{ax}}$ $\forall x\in\mathbb Q$, which indeed is a solution, whatever is $a\in\mathbb R$, $b\in\mathbb R\setminus\{0\}$


\end{solution}



\begin{solution}[by \href{https://artofproblemsolving.com/community/user/346763}{WolfusA}]
	What about non continuous solutions?
\end{solution}



\begin{solution}[by \href{https://artofproblemsolving.com/community/user/29428}{pco}]
	\begin{tcolorbox}What about non continuous solutions?\end{tcolorbox}

What is your definition of continuous \/ non continuous function when domain is $\mathbb Q$ ???????
\end{solution}



\begin{solution}[by \href{https://artofproblemsolving.com/community/user/346763}{WolfusA}]
	$f:\mathbb Q\to \mathbb R$ is continuous on set $\mathbb Q $ iff for every sequence $(x_n)$ where $\forall n \quad x_n\in \mathbb Q$ and $\lim_{n\to \infty}x_n=y$ holds $\lim_{x_n\to y} f(x_n)=f(y)$
\end{solution}



\begin{solution}[by \href{https://artofproblemsolving.com/community/user/392239}{NorthStarPolaris}]
	\begin{tcolorbox}What about non continuous solutions?\end{tcolorbox}

There's no need to consider it, isn't it? Because the domain is rational number, then we can directly apply Cauchy Functional equation as pco did.
\end{solution}



\begin{solution}[by \href{https://artofproblemsolving.com/community/user/29428}{pco}]
	\begin{tcolorbox}$f:\mathbb Q\to \mathbb R$ is continuous on set $\mathbb Q $ iff for every sequence $(x_n)$ where $\forall n \quad x_n\in \mathbb Q$ and $\lim_{n\to \infty}x_n=y$ holds $\lim_{x_n\to y} f(x_n)=f(y)$\end{tcolorbox}
Huhhh !
And what about $y\notin\mathbb Q$ so that $f(y)$ is undefined ?

\end{solution}



\begin{solution}[by \href{https://artofproblemsolving.com/community/user/346763}{WolfusA}]
	I see your point. So what is the criterium then for using Cauchy's functional equation, because I thought it's that function is continuous.
\end{solution}



\begin{solution}[by \href{https://artofproblemsolving.com/community/user/244394}{InCtrl}]
	Given that the domain is $\mathbb{Q}$, you just apply Cauchy's functional equation the normal way on $\log{f(x)}+\log{f(y)}=\log{f(x+y)}$.
\end{solution}



\begin{solution}[by \href{https://artofproblemsolving.com/community/user/346763}{WolfusA}]
	This solutions also works. $\forall x\in\mathbb Q$ $f(x)=bs^{ax}$ whatever is $a\in\mathbb R$, $s,b\in\mathbb R\setminus\{0\}$,
\end{solution}



\begin{solution}[by \href{https://artofproblemsolving.com/community/user/346763}{WolfusA}]
	\begin{tcolorbox}Given that the domain is $\mathbb{Q}$, you just apply Cauchy's functional equation the normal way on $\log{f(x)}+\log{f(y)}=\log{f(x+y)}$.\end{tcolorbox}

You can't do that, because $f:\mathbb{Q}\rightarrow\mathbb{R} \setminus \{ 0 \}$, and you can't take a logarithm of negative number.
\end{solution}
*******************************************************************************
-------------------------------------------------------------------------------

\begin{problem}[Posted by \href{https://artofproblemsolving.com/community/user/243907}{IstekOlympiadTeam}]
	Find all functions $f:\mathbb{R^+}\to\mathbb{R^+}$ such that $\forall x,y\in\mathbb{R^+}$ \[f(f(x)+x+y)=x\left(1+xf\left(\frac{1}{x+y}\right)\right) \]
	\flushright \href{https://artofproblemsolving.com/community/c6h1157429}{(Link to AoPS)}
\end{problem}



\begin{solution}[by \href{https://artofproblemsolving.com/community/user/29428}{pco}]
	\begin{tcolorbox}Find all functions $f:\mathbb{R^+}\to\mathbb{R^+}$ such that $\forall x,y\in\mathbb{R^+}$ \[f(f(x)+x+y)=x\left(1+xf\left(\frac{1}{x+y}\right)\right) \]\end{tcolorbox}

Let $P(x,y)$ be the assertion $f(f(x)+x+y)=x+x^2f(\frac 1{x+y})$

$P(1,x+y-1)$ $\implies$ $f(f(1)+x+y)=1+f(\frac 1{x+y})$ $\forall x+y>1$
Plugging this in $P(x,y)$, we get 
New assertion $Q(x,y)$ : $f(f(x)+y)=x-x^2+x^2f(y+f(1))$ $\forall y>\max(x,1)$

$Q(x,y+f(z))$ $\implies$ $f(f(x)+f(z)+y)=x-x^2+x^2f(y+f(z)+f(1))$ $\forall y>\max(0,x-f(z),1-f(z))$
$Q(z,y+f(1))$ $\implies$ $f(f(z)+y+f(1))=z-z^2+z^2f(y+2f(1))$ $\forall y>\max(0,z-f(1),1-f(1))$

Combining, we get $f(f(x)+f(z)+y)=x-x^2+x^2(z-z^2+z^2f(y+2f(1)))$ $\forall y>\max(0,x-f(z),1-f(z),z-f(1),1-f(1))$

Swapping $x,z$ and subtracting (for $y$ great enough), this implies $x-x^2+x^2(z-z^2)=z-z^2+z^2(x-x^2)$ $\forall x,z$
Which is wrong.

So $\boxed{\text{no such function}}$

\end{solution}



\begin{solution}[by \href{https://artofproblemsolving.com/community/user/344350}{soryn}]
	Nice, nice, nice....
\end{solution}



\begin{solution}[by \href{https://artofproblemsolving.com/community/user/326412}{javierlc2000}]
	We know that,
$
P(0, x - f(0)) : f(x) = 0 \qquad \forall x \in \mathbb{R}^+
$
which implies that, if there is a solution, it has to be $f(x) = 0$.

As, $f(x) = 0 : x = 0 \quad \forall x \in \mathbb{R}^+$, which is a contradiction, $f(x) = 0$ is NOT a solution

This should be enought to see that there are no solutions, right?
\end{solution}



\begin{solution}[by \href{https://artofproblemsolving.com/community/user/345905}{TLP.39}]
	\begin{tcolorbox}We know that,
$
P(0, x - f(0)) : f(x) = 0 \qquad \forall x \in \mathbb{R}^+
$
which implies that, if there is a solution, it has to be $f(x) = 0$.

As, $f(x) = 0 : x = 0 \quad \forall x \in \mathbb{R}^+$, which is a contradiction, $f(x) = 0$ is NOT a solution

This should be enought to see that there are no solutions, right?\end{tcolorbox}

You cannot plug in $0$.
\end{solution}



\begin{solution}[by \href{https://artofproblemsolving.com/community/user/326412}{javierlc2000}]
	Which $0$ can't I plug in? $P(0, y)$ or $f(x) = 0$?

Edit: Ohh, I see, $0 \not \in \mathbb{R}^+$. Thank you!
\end{solution}



\begin{solution}[by \href{https://artofproblemsolving.com/community/user/29428}{pco}]
	\begin{tcolorbox}Which $0$ can't I plug in? $P(0, y)$ or $f(x) = 0$?\end{tcolorbox}

None : $0\notin\mathbb R^+$
\end{solution}
*******************************************************************************
-------------------------------------------------------------------------------

\begin{problem}[Posted by \href{https://artofproblemsolving.com/community/user/257453}{Abhinandan18}]
	Find all functions $f:\mathbb{R}^+\to\mathbb{R}^+$ satisfying $f(xf(y))=yf(x)$ for all positive reals $x$ and $y$, and $\lim_{x\to\infty} f(x)=0$
	\flushright \href{https://artofproblemsolving.com/community/c6h1344999}{(Link to AoPS)}
\end{problem}



\begin{solution}[by \href{https://artofproblemsolving.com/community/user/29428}{pco}]
	\begin{tcolorbox}Find all functions $f:\mathbb{R}^+\to\mathbb{R}^+$ satisfying $f(xf(y))=yf(x)$ for all positive reals $x$ and $y$, and $\lim_{x\to\infty} f(x)=0$\end{tcolorbox}
$f(x)$ is bijective
$f(xf(1))=f(x)$ and so, since bijective, $f(1)=1$
$f(f(x))=x$
$f(xy)=f(xf(f(y)))=f(x)f(y)$

Setting then $f(x)=e^{g(\ln x)}$, we get $g(x+y)=g(x)+g(y)$ $\forall x,y\in\mathbb R$

$\lim_{x\to+\infty}g(x)=-\infty$ and so $g(x)$ is upperbounded from a given point and so is linear.

Plugging this back in original equation, we get $g(x)=-x$ 

And so $\boxed{f(x)=\frac 1x\text{  }\forall x>0}$
\end{solution}



\begin{solution}[by \href{https://artofproblemsolving.com/community/user/209000}{alexheinis}]
	This is IMO 1983, problem 1. As before one shows $f(1)=1$.Taking $x=y$, we see that $xf(x)$ is a fixed  point
of $f$. If $p$ is any fixed point, then so is $p^2$, hence also $p,p^2,p^4,\cdots$. With $x={1\over p},y=p$ we see
that ${1\over p}$ is also a fixed point. Hence $\cdots,{1\over {p^4}},{1\over {p^2}},{1\over p},1,p,p^2,p^4,\cdots$
are all fixed points. If $p>1$ we find a sequence of fixed points tending to $\infty$, impossible.
Hence $p\le 1$, similarly ${1\over p}\le 1$ and $p=1$. Hence $xf(x)=1$ for all $x$.
\end{solution}



\begin{solution}[by \href{https://artofproblemsolving.com/community/user/29428}{pco}]
	\begin{tcolorbox}...we find a sequence of fixed points tending to $\infty$, impossible...\end{tcolorbox}
Why impossible ?
It exists infinitely many functions with sequences of fixed points tending to $\infty$ ...


\end{solution}



\begin{solution}[by \href{https://artofproblemsolving.com/community/user/226061}{adamov1}]
	\begin{tcolorbox}[quote=alexheinis]...we find a sequence of fixed points tending to $\infty$, impossible...\end{tcolorbox}
Why impossible ?
It exists infinitely many functions with sequences of fixed points tending to $\infty$ ...\end{tcolorbox}

Because a sequence of fixed points tending to infinity would mean that there is a sequence of points $x_1,x_2,...$ tending to infinity with $f(x_i)$ tending to infinity, contradicting $\lim_{x\to\infty} f(x)=0$
\end{solution}



\begin{solution}[by \href{https://artofproblemsolving.com/community/user/29428}{pco}]
	\begin{tcolorbox}Because a sequence of fixed points tending to infinity would mean that there is a sequence of points $x_1,x_2,...$ tending to infinity with $f(x_i)$ tending to infinity, contradicting $\lim_{x\to\infty} f(x)=0$\end{tcolorbox}
Hemmmm, indeed.
Thanks for this answer.
:)



\end{solution}



\begin{solution}[by \href{https://artofproblemsolving.com/community/user/332813}{targo___}]
	\begin{tcolorbox}[quote=Abhinandan18]Find all functions $f:\mathbb{R}^+\to\mathbb{R}^+$ satisfying $f(xf(y))=yf(x)$ for all positive reals $x$ and $y$, and $\lim_{x\to\infty} f(x)=0$\end{tcolorbox}
$f(x)$ is bijective
$f(xf(1))=f(x)$ and so, since bijective, $f(1)=1$
$f(f(x))=x$
$f(xy)=f(xf(f(y)))=f(x)f(y)$

Setting then $f(x)=e^{g(\ln x)}$, we get $g(x+y)=g(x)+g(y)$ $\forall x,y\in\mathbb R$

$\lim_{x\to+\infty}g(x)=-\infty$ and so $g(x)$ is upperbounded from a given point and so is linear.

Plugging this back in original equation, we get $g(x)=-x$ 

And so $\boxed{f(x)=\frac 1x\text{  }\forall x>0}$\end{tcolorbox}

Can you explain how you got $f(f(x)) =x$ ?and which statement implying $f$ is bijective ..
\end{solution}



\begin{solution}[by \href{https://artofproblemsolving.com/community/user/29428}{pco}]
	\begin{tcolorbox}Can you explain how you got $f(f(x)) =x$ ?and which statement implying $f$ is bijective ..\end{tcolorbox}
For bijectivity :
If $f(a)=f(b)$ then $bf(x)=f(xf(b))=f(xf(a))=af(x)$ and so $a=b$ and $f(x)$ is injective
$f(xf(\frac y {f(x)}))=y$ and so $f(x)$ is surjective

For $f(f(x))=x$ : just set $x=1$ in equation and since we previously proved $f(1)=1$, we get $f(f(y))=yf(1)=1$


\end{solution}
*******************************************************************************
-------------------------------------------------------------------------------

\begin{problem}[Posted by \href{https://artofproblemsolving.com/community/user/347831}{umaru}]
	Given real number $a$ different $-1;0;1$.find all $f:R->R$ satisfie 
$f(f(x)+ay)=(a^2+a)x+f(f(y)-x)$
	\flushright \href{https://artofproblemsolving.com/community/c6h1522731}{(Link to AoPS)}
\end{problem}



\begin{solution}[by \href{https://artofproblemsolving.com/community/user/29428}{pco}]
	\begin{tcolorbox}Given real number $a$ different $-1;0;1$.find all $f:R->R$ satisfie 
$f(f(x)+ay)=(a^2+a)x+f(f(y)-x)$\end{tcolorbox}
Let $P(x,y)$ be the assertion $f(f(x)+ay)=(a^2+a)x+f(f(y)-x)$
Let $b=f(0)$

Since $a\ne 0$ : $P(x,-\frac{f(x)}a)$ $\implies$ $f(0)-(a^2+a)x=f(f(-\frac{f(x)}a)-x)$
Since $a^2+a\ne 0$, LHS can take any value we want and so $f(x)$ is surjective.

If $f(u)=f(v)$, subtracting $P(x,u)$ from $P(x,v)$ implies $f(f(x)+au)=f(f(x)+av)$ and so, since surjective :
$f(x+au)=f(x+av)$
And so $f(x)$ is periodic with period $T=a(u-v)$
Subtracting then $P(x,y)$ from $P(x+T,y)$, we get $T=0$ (since $a^2+a\ne 0$) and so $f(x)$ is injective.

$P(0,x)$ $\implies$ $f(f(x))=f(ax+b)$ and so, since injective :
$\boxed{f(x)=ax+b\quad\forall x}$ which indeed is a solution, whatever is $b\in\mathbb R$



\end{solution}



\begin{solution}[by \href{https://artofproblemsolving.com/community/user/393921}{Shaddoll1234}]
	\begin{tcolorbox} $f(f(x)+au)=f(f(x)+av)$ and so, since surjective :$f(x+au)=f(x+av)$\end{tcolorbox}
Could you explain this line please  :D 


\end{solution}



\begin{solution}[by \href{https://artofproblemsolving.com/community/user/29428}{pco}]
	\begin{tcolorbox}[quote=pco] $f(f(x)+au)=f(f(x)+av)$ and so, since surjective :$f(x+au)=f(x+av)$\end{tcolorbox}
Could you explain this line please  :D\end{tcolorbox}
Since $f(x)$ is surjective (proved some lines above), the $f(x)$ in expression $f(f(x)+au)=f(f(x)+av)$ can take any real value we want and so this equality is indeed equivalent to $f(x+au)=f(x+av)$


\end{solution}
*******************************************************************************
-------------------------------------------------------------------------------

\begin{problem}[Posted by \href{https://artofproblemsolving.com/community/user/212515}{adityaguharoy}]
	Is there any function $f: \mathbb{R}^{+} \to \mathbb{R}$ such that $\forall x > 0 ,y >0$ the following is true : (?)
$$f(xf(y)+yf(x))=x^2+y^2$$
	\flushright \href{https://artofproblemsolving.com/community/c6h1557983}{(Link to AoPS)}
\end{problem}



\begin{solution}[by \href{https://artofproblemsolving.com/community/user/331781}{muraza}]
	Put (0;0):  f(0) = 0
Put (1;0):  1 = f (f(0) + 0 * f(1)) = f(0)
0 = f(0) = 1 contradiction
\end{solution}



\begin{solution}[by \href{https://artofproblemsolving.com/community/user/337928}{Mr.Techworm}]
	\begin{tcolorbox}Put (0;0):  f(0) = 0
Put (1;0):  1 = f (f(0) + 0 * f(1)) = f(0)
0 = f(0) = 1 contradiction\end{tcolorbox}

Your proof is wrong, It said that ${x},{y}>0$
\end{solution}



\begin{solution}[by \href{https://artofproblemsolving.com/community/user/29428}{pco}]
	\begin{tcolorbox}Is there any function $f: \mathbb{R}^{+} \to \mathbb{R}$ such that $\forall x > 0 ,y >0$ the following is true : (?)
$$f(xf(y)+yf(x))=x^2+y^2$$\end{tcolorbox}
Let $P(x,y)$ be the assertion $f(xf(y)+yf(x))=x^2+y^2$
Let $c=f(1)$

$(0,+\infty)\subseteq f(\mathbb R^+)$
$P(x,x)$ implies $f(2xf(x))=2x^2$ and so $xf(x)\in\mathbb R^+$ (else LHS is undefined)
And so $f(x)$ is a surjection from $\mathbb R^+\to\mathbb R^+$

1) $f(x)=\frac 12\quad\iff\quad x=\frac 12$
Since surjective, $\exists u>0$ such that $f(u)=\frac 12$ and then :
$P(u,u)$ $\implies$ $u=\frac 12$
Q.E.D.


2) $f(x)<1$ $\forall x\in(0,\frac 12]$
Let $x\le\frac 12$ :
$P(x,\sqrt{\frac 12-x^2})$ implies $f(xf(\sqrt{\frac 12-x^2})+\sqrt{\frac 12-x^2}f(x))=\frac 12$
And so $xf(\sqrt{\frac 12-x^2})+\sqrt{\frac 12-x^2}f(x)=\frac 12$
And so $f(x)<\frac 1{2\sqrt{\frac 12-x^2}}\le 1$
Q.E.D.

3) $f(x)>\frac 12-cx$ $\forall x>0$
$P(x,1)$ $\implies$ $f(cx+f(x))=x^2+1>1$ and so (using 2) above) :
$cx+f(x)>\frac 12$
Q.E.D.

4) No such function
Using 3), we have $x^2+y^2>\frac 12-c(xf(y)+yf(x))$ $\forall x,y>0$
Which is $f(x)>\frac{1-2y^2}{2cy}-x\frac{f(y)}y-\frac{x^2}{cy}$ $\forall x,y>0$
And this shows that we can choose $x$ as small as we want and $y$ such that$f(x)$ is as great as we want
In contradiction with $2)$ above.
Q.E.D.


\end{solution}
*******************************************************************************
-------------------------------------------------------------------------------

\begin{problem}[Posted by \href{https://artofproblemsolving.com/community/user/368111}{Anis2017}]
	find all functions $f: \mathbb{R} \rightarrow \mathbb{R}$ such that $a$ and $b$ are from $R$ and 
  $f(x)f(y)=x^af(y\/2)+y^bf(x\/2)$
	\flushright \href{https://artofproblemsolving.com/community/c6h1567808}{(Link to AoPS)}
\end{problem}



\begin{solution}[by \href{https://artofproblemsolving.com/community/user/391068}{TuZo}]
	The functon f is continuous?
\end{solution}



\begin{solution}[by \href{https://artofproblemsolving.com/community/user/29428}{pco}]
	\begin{tcolorbox}find all functions $f: \mathbb{R} \rightarrow \mathbb{R}$ such that $a$ and $b$ are from $R$ and 
  $f(x)f(y)=x^af(y\/2)+y^bf(x\/2)$\end{tcolorbox}
Let $P(x,y)$ be the assertion $f(x)f(y)=x^af(\frac y2)+y^bf(\frac x2)$
Note that in order $RHS$ be defined, we need $a,b\in\mathbb Z_{>0}$

Subtracting $P(x,y)$ from $P(y,x)$, we get $f(\frac x2)(y^a-y^b)=f(\frac y2)(x^a-x^b)$

1) If $a\ne b$
Previous equation implies $f(x)=c((2x)^a-(2x)^b)$
Plugging this back in original equation, we get $c=0$ and solution 
$\boxed{\text{S1 : }f(x)=0\quad\forall x}$, which indeed is a solution, whatever are $a,b\in\mathbb Z_{>0}$

2) if $a=b$
$P(x,x)$ is $f(\frac x2)=\frac 1{2x^a}f(x)^2$ $\forall x\ne 0$
Plugging this in original equation, we get $f(x)f(y)=\frac {x^a}{2y^a}f(y)^2+\frac {y^a}{2x^a}f(x)^2$

Which is $\left(\frac {x^{\frac a2}}{\sqrt 2y^{\frac a2}}f(y)-\frac {y^{\frac a2}}{\sqrt 2x^{\frac a2}}f(x)\right)^2=0$

And so $f(x)=\alpha x^a$ for some $\alpha$ and $\forall x\ne 0$

Plugging this back in orginal equation, we find
The solution $S1$ already found above.

The solution $\boxed{\text{S2 : }f(x)=2\left(\frac xa\right)^a\quad\forall x}$
Which indeed is a solution, whatever are $a=b\in\mathbb Z_{>0}$



\end{solution}



\begin{solution}[by \href{https://artofproblemsolving.com/community/user/391068}{TuZo}]
	Remark to the \begin{bolded}pco\end{bolded} solution: 
If $a=b$, we can denote $\frac{f(x)}{x^{a}}=g(x)$ , so we have $g(x)g(y)=g(x)+g(y)$ and here put the $h(x)=g(e^{x})$ we got $h(x+y)=h(x)+h(y)$Cauchy equation with the solution $h(x)=kx.$
\end{solution}
*******************************************************************************
-------------------------------------------------------------------------------

\begin{problem}[Posted by \href{https://artofproblemsolving.com/community/user/392130}{JANMATH111}]
	Find all functions $f$ from real to real numers such that for every real nubers $x$ any $y$ this equation holds:

$$(x+f(x)^2)f(y)=f(yf(x))+xyf(x)$$
	\flushright \href{https://artofproblemsolving.com/community/c6h1568991}{(Link to AoPS)}
\end{problem}



\begin{solution}[by \href{https://artofproblemsolving.com/community/user/29428}{pco}]
	\begin{tcolorbox}Find all functions f from real to real numers such that for every real nubers x any y this equation holds:

(x+f(x)^2)f(y)=f(yf(x))+xyf(x)\end{tcolorbox}
Let $P(x,y)$ be the assertion $(x+f(x)^2)f(y)=f(yf(x))+xyf(x)$
Let $a=f(0)$
Let $c=f(1)$

$P(2,0)$ $\implies$ $a(1+f(2)^2)=0$ and so $a=0$

If $f(u)=0$ for some $u\ne 0$ : $P(u,x)$ $\implies$ 
$\boxed{\text{S1 : }f(x)=0\quad\forall x}$ which indeed is a solution

So let us from now consider $f(x)=0$ $\iff$ $x=0$ (note then that $c\ne 0$)
$P(1,1)$ $\implies$ $f(c)=c^3$

$P(-c^2,\frac c{f(-c^2)})$ $\implies$ $f(-c^2)^2=c^2$

$P(-c^2,\frac x{f(-c^2)})$ $\implies$ $f(x)=c^2x$ and so, setting there $x=1$, $c=1$

And so $\boxed{\text{S2 : }f(x)=x\quad\forall x}$ which indeed is a solution


\end{solution}



\begin{solution}[by \href{https://artofproblemsolving.com/community/user/392130}{JANMATH111}]
	Nice solution :D
\end{solution}
*******************************************************************************
-------------------------------------------------------------------------------

\begin{problem}[Posted by \href{https://artofproblemsolving.com/community/user/377794}{Mathuzb}]
	Find all continuous function $f : \mathbb{R}\rightarrow\mathbb{R}$ such that 
$$f(xy)+f(x+y)=f(xy+x)+f(y)$$ for all real numbers $x,y$.
	\flushright \href{https://artofproblemsolving.com/community/c6h1569197}{(Link to AoPS)}
\end{problem}



\begin{solution}[by \href{https://artofproblemsolving.com/community/user/29428}{pco}]
	\begin{tcolorbox}Find all continuous function $f : \mathbb{R}\rightarrow\mathbb{R}$ such that 
$$f(xy)+f(x+y)=f(xy+x)+f(y)$$ for all real numbers $x,y$.\end{tcolorbox}
Let $P(x,y)$ be the assertion $f(xy)+f(x+y)=f(xy+x)+f(y)$
Let $c=f(1)$ and $d=f(-1)$
Note that $f(x)$ solution implies $f(x)+c$ solution. So WLOG $f(0)=0$

Let $a,b>0$ and sequences :
$x_1=a$ and $y_1=b$
$x_{n+1}=\frac{x_n}{y_n+1}$ and $y_{n+1}=\frac{(x_n+y_n+1)y_n}{y_n+1}$
It is easy to show that $\lim_{n\to+\infty}=0$ and $\lim_{n\to+\infty}y_n=a+b$

Subtracting $P(\frac{x_n}{y_n+1},y_n)$ from $P(y_n,\frac{x_n}{y_n+1})$, we get $f(x_n)+f(y_n)=f(x_{n+1})+f(y_{n+1})$
And so, setting $n\to+\infty$ and using continuity : $f(a)+f(b)=f(a+b)$

And so, since continuous, $f(x)=cx$ $\forall x\ge 0$

Let then $x\le -1$ : $P(-x,-1)$ $\implies$ $f(x)=cx+c+d$ $\forall x\le -1$

Let then $x\in(-1,0)$
Choose $y>\max(-x,-\frac 1x)$ so that $y(x+1)>0$, $x+y>0$ and  $xy<-1$
$P(y,x)$ $\implies$ $f(x)=cx+c+d$ $\forall x\in(-1,0)$

Continuity at $0$ implies $c+d=0$
And so $f(x)=cx$ which indeed is a solution
Hence the general solution $\boxed{f(x)=ax+b\quad\forall x}$ which indeed is solution, whatever are $a;b\in\mathbb R$








\end{solution}



\begin{solution}[by \href{https://artofproblemsolving.com/community/user/344350}{soryn}]
	Very nice! I like this solution....
\end{solution}
*******************************************************************************
-------------------------------------------------------------------------------

\begin{problem}[Posted by \href{https://artofproblemsolving.com/community/user/377794}{Mathuzb}]
	Find all functions $ f: \mathbb{R^{+}}\to \mathbb{R^{+}}$ such that $f(x+f(y))=f(x)-x+f(x+y)$ for all positive real numbers $x,y$.
	\flushright \href{https://artofproblemsolving.com/community/c6h1569200}{(Link to AoPS)}
\end{problem}



\begin{solution}[by \href{https://artofproblemsolving.com/community/user/29428}{pco}]
	\begin{tcolorbox}Find all functions $ f: \mathbb{R^{+}}\to \mathbb{R^{+}}$ such that $f(x+f(y))=f(x)-x+f(x+y)$ for all positive real numbers $x,y$.\end{tcolorbox}
Let $P(x,y)$ be the assertion $f(x+f(y))=f(x)-x+f(x+y)$

$P(x+f(z),y)$ $\implies$ $f(x+f(y)+f(z))=f(x+f(z))-x-f(z)+f(x+y+f(z))$
$P(x,z)$ $\implies$ $f(x+f(z))=f(x)-x+f(x+z)$
$P(x+y,z)$ $\implies$ $f(x+y+f(z))=f(x+y)-x-y+f(x+y+z)$
Adding, we get $f(x+f(y)+f(z))=f(x)+f(x+z)+f(x+y)-3x-y-f(z)+f(x+y+z)$
Swapping there $y,z$ : $f(x+f(y)+f(z))=f(x)+f(x+y)+f(x+z)-3x-z-f(y)+f(x+y+z)$
Subtracting : $f(z)-z=f(y)-y$

And so $\boxed{f(x)=x+a\quad\forall x>0}$ which indeed is a solution, whatever is $a\ge 0$


\end{solution}
*******************************************************************************
-------------------------------------------------------------------------------

\begin{problem}[Posted by \href{https://artofproblemsolving.com/community/user/392130}{JANMATH111}]
	If function is surjective and f(f(x))=2018f(x), does that mean f(x)=2018x? Why
	\flushright \href{https://artofproblemsolving.com/community/c6h1569295}{(Link to AoPS)}
\end{problem}



\begin{solution}[by \href{https://artofproblemsolving.com/community/user/335559}{Duarti}]
	This is true only on some cases. Because $f(x)$ can take any value on its codomain, and if we take any arbitrary $y$ on the function's codomain, such that $y=f(c)$ and put $x=c$ on the original equation, we get $f(y)=2018y$. In fact this is a solution only if $2018y$ is included on the codomain, for example this works on the set of real numbers, but if you have a function, whose domain and codomain are $\{ 1,2,3,4\}$, for example, then no.
\end{solution}



\begin{solution}[by \href{https://artofproblemsolving.com/community/user/392130}{JANMATH111}]
	(y) ........
\end{solution}



\begin{solution}[by \href{https://artofproblemsolving.com/community/user/236517}{Fever}]
	Duarti said correct, so its true for $y$ for which $\exists x | f(x) = y$, but you have \begin{bolded}surjective\end{bolded} function, so that means that $\forall y \exists x|f(x) = y$
\end{solution}



\begin{solution}[by \href{https://artofproblemsolving.com/community/user/29428}{pco}]
	\begin{tcolorbox}Duarti said correct, so its true for $y$ for which $\exists x | f(x) = y$, but you have \begin{bolded}surjective\end{bolded} function, so that means that $\forall y \exists x|f(x) = y$\end{tcolorbox}

Read carefully what Duarti said : "so that means that $\forall y \exists x|f(x) = y$" is trivially false. The good sentence would be "so that means that $\forall y$ "\begin{bolded}belonging to codomain\end{underlined}\end{bolded}, $\exists x|f(x) = y$" ...
And since codomain is not given, we can not conclude anything.
\end{solution}



\begin{solution}[by \href{https://artofproblemsolving.com/community/user/392130}{JANMATH111}]
	Domain and codomain are real numbers btw but thanks everyone - the symbol (y) was meant to be like but didn't work
\end{solution}
*******************************************************************************
-------------------------------------------------------------------------------

\begin{problem}[Posted by \href{https://artofproblemsolving.com/community/user/333236}{Xurshid.Turgunboyev}]
	Prove that there exists infinitely many functions  $f $ from $N $ such that for every $n,k \in N $ holds
$f (nf (k)+kf (n))=f (n^2+k^2)f (n+k-1) $.
	\flushright \href{https://artofproblemsolving.com/community/c6h1569654}{(Link to AoPS)}
\end{problem}



\begin{solution}[by \href{https://artofproblemsolving.com/community/user/29428}{pco}]
	\begin{tcolorbox}Prove that there exists infinitely many functions  $f $ from $N $ such that for every $n,k \in N $ holds
$f (nf (k)+kf (n))=f (n^2+k^2)f (n+k-1) $.\end{tcolorbox}
Choose for example $f(n)=a(1+(-1)^n)+\frac{1-(-1)^n}2$, which is solution, whatever is $a\in\mathbb N$


\end{solution}
*******************************************************************************
-------------------------------------------------------------------------------

\begin{problem}[Posted by \href{https://artofproblemsolving.com/community/user/333236}{Xurshid.Turgunboyev}]
	Find all functions  $f:R\longrightarrow R $ which satisfy
(I) $f (x+y)=f (x)+f (y) $ for real $x,y $
(Ii)$f (p (x))=p (f (x)) $ some polynomial  $p (x) $ of degree  $\ge2$
	\flushright \href{https://artofproblemsolving.com/community/c6h1570155}{(Link to AoPS)}
\end{problem}



\begin{solution}[by \href{https://artofproblemsolving.com/community/user/29428}{pco}]
	\begin{tcolorbox}Find all functions  $f:R\longrightarrow R $ which satisfy
(I) $f (x+y)=f (x)+f (y) $ for real $x,y $
(Ii)$f (p (x))=p (f (x)) $ some polynomial  $p (x) $ of degree  $\ge2$\end{tcolorbox}
Let $p(x+y)=\sum a_{ij}x^iy^j$

1) $\exists$ even positive $i$ such that $a_{ij}\ne 0$
If $a_{ij}=0$ for all even positive $i$, then 
$p(x+y)=h(x,y)+q(y)$ with $h(-x,y)=-h(x,y)$
So $p(x+y)+p(-x+y)=2q(y)$
So $p(x)+p(-x)=2q(0)$ and $p(-x+y)=2q(0)-p(x-y)$

And $p(x+y)+p(-x+y)=2q(y)$ becomes $p(x+y)-p(x-y)=2q(y)-2q(0)$
And so for example $p(x+1)-p(x)$ constant, which implies degree of $p(x)\le 1$
And contradiction
A.E.D.

2) $f(x)$ is linear
Let $c=f(1)$
Let $k\in\mathbb Q$

$f(p(x+k))=\sum a_{ij}f(x^i)k^j$
$p(f(x+k))=\sum a_{ij}f(x)^ic^jk^j$

Subtracting, we get $\sum a_{ij}(f(x^i)-f(x)^ic^j)k^=0$

Considering this as a polynomial in $k$ with infinitely many roots (all rational numbers), we conclude that this is the allzero polynomial.

And so $a_{ij}(f(x^i)-f(x)^ic^j)=0$ $\forall i,j$

Choosing one even positive $i$ and a $j$ such that $a_{ij}\ne 0$ (using 1) above), we get $f(x^i)=f(x)^ic^j$ and so $f(x)$ has a constant sign over $\mathbb R^+$
So $f(x)$ is either overbounded, either lowerbounded over an non empty open interval, and so is linear.

3) Solutions
So $f(x)=cx$ and $cp(x)=p(cx)$
Looking at highest degree summand in this last equality, we get $c\in\{-1,0,1\}$

$c=0$ implies solution, $\boxed{\text{S1 : }f(x)=0\quad\forall x}$ which indeed is a solution if $p(0)=0$
$c=1$ implies solution, $\boxed{\text{S2 : }f(x)=x\quad\forall x}$ which indeed is a solution
$c=-1$ implies solution, $\boxed{\text{S3 : }f(x)=-x\quad\forall x}$ which indeed is a solution if $p(x)$ is an odd polynomial.




\end{solution}
*******************************************************************************
-------------------------------------------------------------------------------

\begin{problem}[Posted by \href{https://artofproblemsolving.com/community/user/55393}{makar}]
	$f:\mathbb{R}\to\mathbb{R}$ such that 
$\frac {f(x+y)}{f(x-y)}=\frac{f(x)+f(y)}{f(x)-f(y)}\forall x\ne y$. Find all such $f$
	\flushright \href{https://artofproblemsolving.com/community/c6h1570396}{(Link to AoPS)}
\end{problem}



\begin{solution}[by \href{https://artofproblemsolving.com/community/user/29428}{pco}]
	\begin{tcolorbox}$f:\mathbb{R}\to\mathbb{R}$ such that 
$\frac {f(x+y)}{f(x-y)}=\frac{f(x)+f(y)}{f(x)-f(y)}\forall x\ne y$. Find all such $f$\end{tcolorbox}
Certainly not a real olympiad exercise.

At least all injective additive functions are solution (so infinitely many including infinitely many non continuous) and it is very difficult to give a general form for injective additive functions.



\end{solution}



\begin{solution}[by \href{https://artofproblemsolving.com/community/user/212018}{Tintarn}]
	Okay, but it seems to be an interesting question to prove that these are all the solutions i.e. that the given functional equation is equivalent to $f(x+y)=f(x)+f(y)$ and $f$ injective. I'm rather sure that it should be true and provable, but I haven't worked out all the details (in my approach, computations get rather messy).
\end{solution}



\begin{solution}[by \href{https://artofproblemsolving.com/community/user/55393}{makar}]
	Yes I got $f(x+y) =f(x)+f(y)$ after that as it is additive function has infinite solution among them one continuous is $f(x)=ax$ There are other untamed solutions 
\end{solution}



\begin{solution}[by \href{https://artofproblemsolving.com/community/user/212018}{Tintarn}]
	\begin{tcolorbox}Yes I got $f(x+y) =f(x)+f(y)$\end{tcolorbox} Would you like to show us your proof?


\end{solution}



\begin{solution}[by \href{https://artofproblemsolving.com/community/user/29428}{pco}]
	\begin{tcolorbox}[quote=makar]Yes I got $f(x+y) =f(x)+f(y)$\end{tcolorbox} Would you like to show us your proof?\end{tcolorbox}

I'm interested too  :D 
\end{solution}



\begin{solution}[by \href{https://artofproblemsolving.com/community/user/391068}{TuZo}]
	\begin{bolded}Here is my work:\end{bolded}
If $x=y$, we got $f(0)=0.$ If I put $y\rightarrow-y$ , we got $\frac{f(x)-f(y)}{f(x)+f(y)}=\frac{f(x-y)}{f(x+y)}$=$\frac{f(x)+f(-y)}{f(x)-f(-y)}and$ from here we got: $f(x)(f(y)+f(-y))=0$, but $f\equiv0$ not fit to equation, so $f(-y)=-f(y)$, so $\frac{f(x+y)}{f(x)+f(y)}=\frac{f(x-y)}{f(x)-f(y)}=\frac{f(x-y)}{f(x)+f(-y)}$ so if we denote $g(x+y)=\frac{f(x+y)}{f(x)+f(y)},$we have $g(x+y)=g(x-y)$ and if $x\rightarrow x+1$and $y=x,$we got $g(2x+1)=g(1)=a,$so $g(x)=a,$ thus $f(x+y)=a(f(x)+f(y))$.  But $g(1)=1,$so we got $f(x+y)=f(x)+f(y).$ There are an infinite many non continuouse function f.
\end{solution}



\begin{solution}[by \href{https://artofproblemsolving.com/community/user/212018}{Tintarn}]
	\begin{tcolorbox}so if we denote $g(x)=\frac{f(x+y)}{f(x)+f(y)},$\end{tcolorbox}
Wait, this makes no sense unless you fix some number $y$ (and then you have to remember that the definition of $g$ depends on $y$). But in any case you cannot treat $y$ as a free variable thereafter. This is not so easy.


\end{solution}



\begin{solution}[by \href{https://artofproblemsolving.com/community/user/391068}{TuZo}]
	Yes, you have right! I corrected the mistake!
\end{solution}



\begin{solution}[by \href{https://artofproblemsolving.com/community/user/390591}{awe2}]
	\begin{tcolorbox}Yes, you have right! I corrected the mistake!\end{tcolorbox}
Writing $g(x + y)$ is still wrong, the function you define is in 2 independent variables $x$ and $y$. Also how can we plug in $x = y$? Maybe there are ways to justify $f(0) = 0$, but i don't see them.
\end{solution}



\begin{solution}[by \href{https://artofproblemsolving.com/community/user/391068}{TuZo}]
	For $f(0)=0$ just put $y=0$ in the original equation! The function g is not defined in 2 Independent varialbes, just I marked with $g(x+y)$. :roll: OK, for $g$ we must put the condition x$\neq y$.
\end{solution}



\begin{solution}[by \href{https://artofproblemsolving.com/community/user/212018}{Tintarn}]
	\begin{tcolorbox}The function g is not defined in 2 Independent varialbes, just I marked with $g(x+y)$. :roll:\end{tcolorbox}
Not sure what you mean. If you want to write $g(x+y)=\frac{f(x+y)}{f(x)+f(y)}$, then we first need to make sure that the RHS only depends on $x+y$ and not on individual values of $x,y$.
For instance, $x=0, y=\sqrt{2}$ and $x=1, y=\sqrt{2}-1$ both yield $g(\sqrt{2})$ on the LHS but on the RHS we have $\frac{f(\sqrt{2})}{f(1)+f(\sqrt{2}-1)}$ and $\frac{f(\sqrt{2}}{f(0)+f(\sqrt{2})}$ and it is not at all clear that these are the same (in fact, this would be essentially equivalent to proving $f(x+y)=f(x)+f(y)$).


\end{solution}



\begin{solution}[by \href{https://artofproblemsolving.com/community/user/391068}{TuZo}]
	\begin{bolded}I do not know what I might believe!\end{bolded} :roll: 
\end{solution}



\begin{solution}[by \href{https://artofproblemsolving.com/community/user/55393}{makar}]
	Let $P(x,y): \frac{f(x+y)}{f(x-y)}=\frac{f(x)+f(y)}{f(x)-f(y)}$

$P(x,0): 1  =\frac{f(x)+f(0)}{f(x)-f(0)}\implies f(0)=0$

$P(0,x): \frac{f(x)}{f(-x)}=\frac{f(x)}{-f(x)}\implies f(-x)=-f(x)$ 

Let us rewrite the given condition as $\frac{f(x+y)}{f(x)+f(y)}=\frac{f(x-y)}{f(x)+f(-y)}$

Let $h(x+y)=f(x)+f(y)\implies \frac{f(x+y)}{h(x+y)}=\frac{f(x-y)}{h(x-y)}$

Let $g(x+y)=\frac{f(x+y)}{h(x+y)}\implies g(x+y)=g(x-y): Q(x,y)$ 

$Q\left(\frac x2+\frac 12,\frac x2 -\frac 12\right):g(x)=g(1)=c$ Say

$\implies f(x+y)=c\left(f(x)+f(y)\right)$

Put $y=0$ in above we get $c=1$

$\implies f(x+y)=f(x)+f(y)$
\end{solution}



\begin{solution}[by \href{https://artofproblemsolving.com/community/user/212018}{Tintarn}]
	\begin{tcolorbox}
Let $g(x+y)=\frac{f(x+y)}{f(x)+f(y)}$ \end{tcolorbox}
Huh? Did you even read my comment above? Writing this just makes no sense at all.

\end{solution}



\begin{solution}[by \href{https://artofproblemsolving.com/community/user/55393}{makar}]
	Please see I edited
\end{solution}



\begin{solution}[by \href{https://artofproblemsolving.com/community/user/391068}{TuZo}]
	Yes, in this way You avoided the trap, I am also committed, See #7
\end{solution}



\begin{solution}[by \href{https://artofproblemsolving.com/community/user/212018}{Tintarn}]
	\begin{tcolorbox}Please see I edited\end{tcolorbox}
Oh come on. You just shifted the nonsense to one line earlier:
\begin{tcolorbox}Let $h(x+y)=f(x)+f(y)$\end{tcolorbox}



\end{solution}



\begin{solution}[by \href{https://artofproblemsolving.com/community/user/391068}{TuZo}]
	\begin{bolded}WizardMath\end{bolded} have a good idea: If we have $\frac{f(x+y)}{x+y} = \frac{f(x-y)}{x-y}$, we can put $x=a+b$ and $y=a-b$, sowe got: $\frac{f(2a)}{2a}$=$\frac{f(2b)}{2b}$ i.e. $\frac{f(u)}{u}=\frac{f(v)}{v}$ so we can denote $g(x)=\frac{f(x)}{x}$ and we got $g(x)=constant!$

\end{solution}



\begin{solution}[by \href{https://artofproblemsolving.com/community/user/309179}{whiwho}]
	\begin{tcolorbox}\begin{bolded}WizardMath\end{bolded} have a good idea: If we have $\frac{f(x+y)}{x+y} = \frac{f(x-y)}{x-y}$, we can put $x=a+b$ and $y=a-b$, sowe got: $\frac{f(2a)}{2a}$=$\frac{f(2b)}{2b}$ i.e. $\frac{f(u)}{u}=\frac{f(v)}{v}$ so we can denote $g(x)=\frac{f(x)}{x}$ and we got $g(x)=constant!$\end{tcolorbox}

And?

\end{solution}



\begin{solution}[by \href{https://artofproblemsolving.com/community/user/391068}{TuZo}]
	Read the post #7, so we got $f(x+y)=f(x)+f(y)$, and this is the Cauchy functional equation, so the problem is solved!  :coolspeak: 
\end{solution}



\begin{solution}[by \href{https://artofproblemsolving.com/community/user/212018}{Tintarn}]
	\begin{tcolorbox}\begin{bolded}WizardMath\end{bolded} have a good idea: If we have $\frac{f(x+y)}{x+y} = \frac{f(x-y)}{x-y}$...\end{tcolorbox}
Yes, of course, if we have that, then we are done immediately. But we don't have it right?


\end{solution}
*******************************************************************************
-------------------------------------------------------------------------------

\begin{problem}[Posted by \href{https://artofproblemsolving.com/community/user/376204}{GRCMIRACLES}]
	FIND ALL FUNCTIONS f:c->c[compex numbers] such that
f[x]f[x+1]=f[x^2+x+1]
	\flushright \href{https://artofproblemsolving.com/community/c6h1570579}{(Link to AoPS)}
\end{problem}



\begin{solution}[by \href{https://artofproblemsolving.com/community/user/376204}{GRCMIRACLES}]
	Any solutions??
\end{solution}



\begin{solution}[by \href{https://artofproblemsolving.com/community/user/376204}{GRCMIRACLES}]
	Shall I give the solution.
\end{solution}



\begin{solution}[by \href{https://artofproblemsolving.com/community/user/29428}{pco}]
	\begin{tcolorbox}FIND ALL FUNCTIONS f:c->c[compex numbers] such that
f[x]f[x+1]=f[x^2+x+1]\end{tcolorbox}
Easy to show that there are infinitely many such functions.
I would be very surprised if a general form for all of them exists.

\begin{tcolorbox}Shall I give the solution.\end{tcolorbox}
I would be very pleased to see this ...  :D 


\end{solution}



\begin{solution}[by \href{https://artofproblemsolving.com/community/user/130506}{Mosquitall}]
	Use Taylor series for $f$.
\end{solution}



\begin{solution}[by \href{https://artofproblemsolving.com/community/user/29428}{pco}]
	\begin{tcolorbox}Use Taylor series for $f$.\end{tcolorbox}
Nonsense

There exists infinitely many non continuous solutions.
So "Taylor series" is meaningless.

Let us wait for the OP solution.


\end{solution}



\begin{solution}[by \href{https://artofproblemsolving.com/community/user/376204}{GRCMIRACLES}]
	\begin{tcolorbox}[quote=GRCMIRACLES]FIND ALL FUNCTIONS f:c->c[compex numbers] such that
f[x]f[x+1]=f[x^2+x+1]\end{tcolorbox}
Easy to show that there are infinitely many such functions.
I would be very surprised if a general form for all of them exists.

\begin{tcolorbox}Shall I give the solution.\end{tcolorbox}
I would be very pleased to see this ...  :D\end{tcolorbox}

i think there are only finite functions,what is the generalized  form you got ?
\end{solution}



\begin{solution}[by \href{https://artofproblemsolving.com/community/user/29428}{pco}]
	\begin{tcolorbox}i think there are only finite functions,what is the generalized  form you got ?\end{tcolorbox}
Since you proposed first to publish your solution, dont hesitate to post it.

I'll then post infinitely many continuous solutions as well as infinitely many non continuous solutions.



\end{solution}



\begin{solution}[by \href{https://artofproblemsolving.com/community/user/29428}{pco}]
	Since GRCMIRACLES does not want to give us his solution, here are what I promised (infinitely many examples of continuous solutions and infinitely many examples of non continuous solutions)) :

1)infinitely many continuous solutions :
Choose $f(x)=(x^2+1)^c$ whatever is $c\in\mathbb R^+$

2) infinitely many non continuous solutions :
Let $A=\{$ algebraic numbers over $\mathbb Q\}$
Choose $f(x)=1_A(x)$
This is a non continuous solution.
And multiplying it by any of the infinitely many continuous solutions, we get infinitely many non continuous solutions

(without using such multiplication, we can also choose for example $A$ as the set of algebratic numbers on some field different from $\mathbb Q$)

\end{solution}



\begin{solution}[by \href{https://artofproblemsolving.com/community/user/376204}{GRCMIRACLES}]
	can you give the solution please,my solution is only on Z {integers},but not on C
\end{solution}



\begin{solution}[by \href{https://artofproblemsolving.com/community/user/29428}{pco}]
	\begin{tcolorbox}can you give the solution please,my solution is only on Z {integers},but not on C\end{tcolorbox}
You are welcome.
Glad to have helped you.

Btw, I never claim to have the general solution (unlike you in post #3)
I just claimed (see post #4)  that it was easy to show that there are infinitely many such functions (I did it in previous post) and that I would be very surprised if a general form for all of them exists (which clearly indicates that I dont have such a solution).

And waiting 10 posts in the thread to explain that you have a general solution in $\mathbb Z$ and not in $\mathbb C$ is quite ... funny (especially after the different messages about continuity  :D 

\end{solution}
*******************************************************************************
-------------------------------------------------------------------------------

\begin{problem}[Posted by \href{https://artofproblemsolving.com/community/user/281272}{san1201}]
	Find all functions $f:R^+\rightarrow R^+$ satisfying 
$f(1+xf(y))=y.f(x+y)$ for all $x,y\in\mathbb{R^+} $.
	\flushright \href{https://artofproblemsolving.com/community/c6h1571116}{(Link to AoPS)}
\end{problem}



\begin{solution}[by \href{https://artofproblemsolving.com/community/user/29428}{pco}]
	\begin{tcolorbox}Find all functions $f:R^+\rightarrow R^+$ satisfying 
$f(1+xf(y))=y.f(x+y)$ for all $x,y\in\mathbb{R^+} $.\end{tcolorbox}
Posted (and solved) many many times (2010, 2012, 2013, 2015, april 2017 ...).

Dont hesitate to use the search function (see [url=http://www.artofproblemsolving.com\/community\/c6h1330604p7173058] here [\/url]).
Set (for example copy\/paste) in the "search term" field the exact following string : 

+"f(1+xf(y))" +"yf(x+y)"

You'll get in the \begin{bolded}ten first results\end{underlined}\end{bolded} (excluded your own post and this post itself) all the help you are requesting for.

[hide=(Some excuses)][size=70]I'm sorry not providing you the direct link to this result but I encountered users who never tried the search function, thinking quite easier to have other users make the search for them. So now I prefer to point to the search function and to give the appropriate search term (I checked that it indeed will give you the expected result) instead of the link itself [\/size]([url=https:\/\/en.wiktionary.org\/wiki\/give_a_man_a_fish_and_you_feed_him_for_a_day;_teach_a_man_to_fish_and_you_feed_him_for_a_lifetime]wink[\/url])[\/hide]




\end{solution}
*******************************************************************************
-------------------------------------------------------------------------------

\begin{problem}[Posted by \href{https://artofproblemsolving.com/community/user/285870}{user01}]
	Find all functions $ \,f: {\mathbb{R^+}}\rightarrow{\mathbb{R^+}}\,$ such that
$f(x)f(yf(x))=f(x+y)$ $ \text{for all}\,x,y\in\mathbb{R^+}.$
	\flushright \href{https://artofproblemsolving.com/community/c6h1571168}{(Link to AoPS)}
\end{problem}



\begin{solution}[by \href{https://artofproblemsolving.com/community/user/29428}{pco}]
	\begin{tcolorbox}Find all functions $ \,f: {\mathbb{R^+}}\rightarrow{\mathbb{R^+}}\,$ such that
$f(x)f(yf(x))=f(x+y)$ $ \text{for all}\,x,y\in\mathbb{R^+}.$\end{tcolorbox}
Posted (and solved) many many times (2011, 2012, 2013, 2014, 2015, 2017 (july, august, september), ...).

Dont hesitate to use the search function (see [url=http://www.artofproblemsolving.com\/community\/c6h1330604p7173058] here [\/url]).
Set (for example copy\/paste) in the "search term" field the exact following string : 

+"f(x)f(yf(x))" +"f(x+y)"

You'll get in the \begin{bolded}ten first results\end{underlined}\end{bolded} (excluded your own post and this post itself) all the help you are requesting for.

[hide=(Some excuses)][size=70]I'm sorry not providing you the direct link to this result but I encountered users who never tried the search function, thinking quite easier to have other users make the search for them. So now I prefer to point to the search function and to give the appropriate search term (I checked that it indeed will give you the expected result) instead of the link itself [\/size]([url=https:\/\/en.wiktionary.org\/wiki\/give_a_man_a_fish_and_you_feed_him_for_a_day;_teach_a_man_to_fish_and_you_feed_him_for_a_lifetime]wink[\/url])[\/hide]


\end{solution}
*******************************************************************************
-------------------------------------------------------------------------------

\begin{problem}[Posted by \href{https://artofproblemsolving.com/community/user/354682}{Alex27}]
	Find all functions$ f:N->R$ such that $f(n) =f(n^2 +n+1)$,for any $ n \in N$
	\flushright \href{https://artofproblemsolving.com/community/c6h1571179}{(Link to AoPS)}
\end{problem}



\begin{solution}[by \href{https://artofproblemsolving.com/community/user/29428}{pco}]
	\begin{tcolorbox}Find all functions$ f:N->R$ such that $f(n) =f(n^2 +n+1)$,for any $ n \in N$\end{tcolorbox}
Let $g(n)=n^2+n+1$ injection from $\mathbb N\to\mathbb N$

Let $\sim$ the equivalence relation defined over $\mathbb N$ as :
$x\sim y$ $\iff$ $\exists k\in\mathbb Z_{\ge 0}$ such that $\max(x,y)=g^{[k]}(\min(x,y))$ (composition of function)
(note that the fact this is an equivalence relation is not immediate concerning transitivity)

Let $r(x)$ from $\mathbb N\to\mathbb N$ any choice function associating to a natural number a representant (unique per class) of its equivalence class.

Let $h(x$ any function from $\mathbb N\to\mathbb N$
Then $\boxed{f(x)=h(r(x))}$

This is clearly the general solution and this is a trivial expression. One certainly would be interested (and I think your teacher is waiting for that)  in a clever general representation of equivalence classes.
But I did not find up to now.



\end{solution}



\begin{solution}[by \href{https://artofproblemsolving.com/community/user/354682}{Alex27}]
	Thank you for your oustanding solution,Mr pco !
\end{solution}
*******************************************************************************
-------------------------------------------------------------------------------

\begin{problem}[Posted by \href{https://artofproblemsolving.com/community/user/393921}{Shaddoll1234}]
	Find all functions  $f: \mathbb R \to \mathbb R$ such that 
$\[ f(xf(y)-yf(x))=f(xy)-xy \]$  $,$ $\forall\, x,y\in\mathbb{R}.$
	\flushright \href{https://artofproblemsolving.com/community/c6h1572299}{(Link to AoPS)}
\end{problem}



\begin{solution}[by \href{https://artofproblemsolving.com/community/user/29428}{pco}]
	\begin{tcolorbox}Find all functions  $f: \mathbb R \to \mathbb R$ such that 
$\[ f(xf(y)-yf(x))=f(xy)-xy \]$  $,$ $\forall\, x,y\in\mathbb{R}.$\end{tcolorbox}
Posted (and solved) many times (2014 (2 times), 2015, 2016, 2017 (3 times)...).

Dont hesitate to use the search function (see [url=http://www.artofproblemsolving.com\/community\/c6h1330604p7173058] here [\/url]).
Set (for example copy\/paste) in the "search term" field the exact following string : 

+"f(xf(y)-yf(x))" +"f(xy)"

You'll get in the \begin{bolded}ten first results\end{underlined}\end{bolded} (excluded your own post and this post itself) all the help you are requesting for.

[hide=(Some excuses)][size=70]I'm sorry not providing you the direct link to this result but I encountered users who never tried the search function, thinking quite easier to have other users make the search for them. So now I prefer to point to the search function and to give the appropriate search term (I checked that it indeed will give you the expected result) instead of the link itself [\/size]([url=https:\/\/en.wiktionary.org\/wiki\/give_a_man_a_fish_and_you_feed_him_for_a_day;_teach_a_man_to_fish_and_you_feed_him_for_a_lifetime]wink[\/url])[\/hide]



\end{solution}



\begin{solution}[by \href{https://artofproblemsolving.com/community/user/393921}{Shaddoll1234}]
	Actually I have searched before but didn't find anything
Thanks
\end{solution}



\begin{solution}[by \href{https://artofproblemsolving.com/community/user/335401}{G.K-1.618}]
	See here 
https:\/\/artofproblemsolving.com\/community\/q1h1369915p7541261
\end{solution}
*******************************************************************************
-------------------------------------------------------------------------------

\begin{problem}[Posted by \href{https://artofproblemsolving.com/community/user/68025}{Pirkuliyev Rovsen}]
	Find all continuous functions $f: \mathbb{R}\to\mathbb{R}$ such that $f(x)=f(x+1)=f(x+\sqrt{2})$

	\flushright \href{https://artofproblemsolving.com/community/c6h1572880}{(Link to AoPS)}
\end{problem}



\begin{solution}[by \href{https://artofproblemsolving.com/community/user/29428}{pco}]
	\begin{tcolorbox}Find all continuous functions $f: \mathbb{R}\to\mathbb{R}$ such that $f(x)=f(x+1)=f(x+\sqrt{2})$\end{tcolorbox}

So $f(x)=f(x+m+n\sqrt 2)$ $\forall x$, $\forall m,n\in\mathbb Z$

And since $\{m+n\sqrt 2\}_{m;n\in\mathbb Z}$ is dense in $\mathbb R$, continuity allows to conclude 
 $\boxed{f(x)=c\text{  constant  }\forall x}$ which indeed is a solution, whatever is $c\in\mathbb R$

\end{solution}



\begin{solution}[by \href{https://artofproblemsolving.com/community/user/309179}{whiwho}]
	\begin{tcolorbox}[quote=Pirkuliyev Rovsen]Find all continuous functions $f: \mathbb{R}\to\mathbb{R}$ such that $f(x)=f(x+1)=f(x+\sqrt{2})$\end{tcolorbox}

 since $\{m+n\sqrt 2\}_{m;n\in\mathbb Z}$ is dense in $\mathbb R$$\end{tcolorbox}


Could you explain why please? 

\end{solution}



\begin{solution}[by \href{https://artofproblemsolving.com/community/user/68025}{Pirkuliyev Rovsen}]
	this follows from Kronecker's theorem
\end{solution}



\begin{solution}[by \href{https://artofproblemsolving.com/community/user/374509}{abbosjon2002}]
	\begin{tcolorbox}this follows from Kronecker's theorem\end{tcolorbox}

Can you write this theorem
\end{solution}



\begin{solution}[by \href{https://artofproblemsolving.com/community/user/391068}{TuZo}]
	About the Kronecker density theoreme you can read here:
[url]http://mathworld.wolfram.com\/KroneckersApproximationTheorem.html[\/url]
\end{solution}



\begin{solution}[by \href{https://artofproblemsolving.com/community/user/309179}{whiwho}]
	\begin{tcolorbox}About the Kronecker density theoreme you can read here:
[url]http://mathworld.wolfram.com\/KroneckersApproximationTheorem.html[\/url]\end{tcolorbox}

Thanks you a lot
\end{solution}



\begin{solution}[by \href{https://artofproblemsolving.com/community/user/179739}{Hello_Kitty}]
	to show density:
\end{solution}
*******************************************************************************
-------------------------------------------------------------------------------

\begin{problem}[Posted by \href{https://artofproblemsolving.com/community/user/345324}{virnoy}]
	Determine all functions $f,g:\mathbb{R}\rightarrow\mathbb{R}$ such that $f(g(x))=x^3$ and $g(f(x))=x^2$. 
	\flushright \href{https://artofproblemsolving.com/community/c6h1573398}{(Link to AoPS)}
\end{problem}



\begin{solution}[by \href{https://artofproblemsolving.com/community/user/345324}{virnoy}]
	Bump. pmuB.
\end{solution}



\begin{solution}[by \href{https://artofproblemsolving.com/community/user/122611}{oty}]
	Already posted , try the search function , sorry I am not good at using it 
\end{solution}



\begin{solution}[by \href{https://artofproblemsolving.com/community/user/345324}{virnoy}]
	Couldn't find it. 
\end{solution}



\begin{solution}[by \href{https://artofproblemsolving.com/community/user/260515}{anhtaitran}]
	From f(g(x))=x^3 we have:g(x) is injective or else a^3=f(g(a))=f(g(b))=b^3(contradiction).
From g(f(x))=x^2 ,using the fact that g(x) is injective,we could show that f(a)=f(b) if and only if a=-b.
Hence,by replace x with cube root of f(x) in the equality:f(g(x))=x^3 and using the above fact:
g(cube root of f(x))+x=0.(1)
By replace x by f(x) in f(g(x))=x^3 we have:f(x^2)=f(x)^3 (2)
By replacing x by x^2 in (1) and using (2) we have:g(f(x))+x^2=0 for every x.
              Or 2x^2 =0 for every x(contradition).
Hence,there does not exist those f;g.
\end{solution}



\begin{solution}[by \href{https://artofproblemsolving.com/community/user/29428}{pco}]
	\begin{tcolorbox}Couldn't find it.\end{tcolorbox}
First result using search string :

 +"f(g(x))=x^" +"g(f(x))=x^"


\end{solution}



\begin{solution}[by \href{https://artofproblemsolving.com/community/user/345324}{virnoy}]
	Which IMO shortlist is this from?
\end{solution}
*******************************************************************************
-------------------------------------------------------------------------------

\begin{problem}[Posted by \href{https://artofproblemsolving.com/community/user/364791}{CinarArslan}]
	$f$  is a function on real numbers: $$f(x)^2=f(2x)+2f(x)-2$$ and $$f(1)=3$$ What is the value of $f(6)$ ?

	\flushright \href{https://artofproblemsolving.com/community/c6h1573785}{(Link to AoPS)}
\end{problem}



\begin{solution}[by \href{https://artofproblemsolving.com/community/user/29428}{pco}]
	\begin{tcolorbox}$f$  is a function on real numbers: $$f(x)^2=f(2x)+2f(x)-2$$ and $$f(1)=3$$ What is the value of $f(6)$ ?\end{tcolorbox}
$f(6)$ can take any value we want in $[1,+\infty)$


\end{solution}



\begin{solution}[by \href{https://artofproblemsolving.com/community/user/364791}{CinarArslan}]
	\begin{tcolorbox}[quote=CinarArslan]$f$  is a function on real numbers: $$f(x)^2=f(2x)+2f(x)-2$$ and $$f(1)=3$$ What is the value of $f(6)$ ?\end{tcolorbox}
$f(6)$ can take any value we want in $[1,+\infty)$\end{tcolorbox}

I think so but answer says $65$. $f(x)=2^x+1$
\end{solution}



\begin{solution}[by \href{https://artofproblemsolving.com/community/user/392094}{Duy_Thai2002}]
	So f is a function on natural number
\end{solution}



\begin{solution}[by \href{https://artofproblemsolving.com/community/user/29428}{pco}]
	\begin{tcolorbox}I think so but answer says $65$. $f(x)=2^x+1$\end{tcolorbox}
And what about $f(0)=1$ and $f(x)=1+2^{x\left\lfloor 1+100\{\log_2|x|\}\right\rfloor}$ $\forall x\ne 0$ ?



\end{solution}



\begin{solution}[by \href{https://artofproblemsolving.com/community/user/344350}{soryn}]
	Sir pco, how obtained this form for f(x)? Thank You in advance. 
\end{solution}



\begin{solution}[by \href{https://artofproblemsolving.com/community/user/29428}{pco}]
	\begin{tcolorbox}Sir pco, how obtained this form for f(x)? Thank You in advance.\end{tcolorbox}

This is just one of the infinitely many solutions : $f(x)=1+e^{xh(\{\log_2|x|\})}$

\end{solution}



\begin{solution}[by \href{https://artofproblemsolving.com/community/user/344350}{soryn}]
	h îs an arbitrary function? I fi not understand....
\end{solution}



\begin{solution}[by \href{https://artofproblemsolving.com/community/user/29428}{pco}]
	\begin{tcolorbox}h îs an arbitrary function? I fi not understand....\end{tcolorbox}

Have you just checked ?

Yes, any $h(x)$ gives a suitable solution (you need just the constraint on $h(0)$ if you want $f(1)=3$)
\end{solution}



\begin{solution}[by \href{https://artofproblemsolving.com/community/user/344350}{soryn}]
	Clearly, this check! How rezultă this very interesting formula for f(x)?This is Nice functional equation....
\end{solution}



\begin{solution}[by \href{https://artofproblemsolving.com/community/user/333350}{AnArtist}]
	\begin{tcolorbox}[quote=soryn]Sir pco, how obtained this form for f(x)? Thank You in advance.\end{tcolorbox}

This is just one of the infinitely many solutions : $f(x)=1+e^{xh(\{\log_2|x|\})}$\end{tcolorbox}

Wow how did you get that?
\end{solution}



\begin{solution}[by \href{https://artofproblemsolving.com/community/user/29428}{pco}]
	Setting $g(x)=f(x)-1$, equation is $g(2x)=g(x)^2$

Considering only cases where $g(x)>0$ (we are just looking for infinitely many large families of solution, not for all the solutions), we can write $g(x)=e^{h(x)}$ and equation becomes $h(2x)=2h(x)$

So $h(0)=0$ and, setting for $x\ne 0$ $h(x)=xk(x)$  : $k(2x)=k(x)$

Considering there $x>0$ and setting $k(x)=l(\log_2 x)$, equation is $l(x+1)=l(x)$ with general solution $m(\{x\})$

And so on ...

And, once again, I just creared some\end{underlined} solutions, not ALL of them.


\end{solution}



\begin{solution}[by \href{https://artofproblemsolving.com/community/user/333350}{AnArtist}]
	Ohhh nice.
\end{solution}



\begin{solution}[by \href{https://artofproblemsolving.com/community/user/344350}{soryn}]
	C'est ravissant, monsieur pco!!!! Merci beaucoup...
\end{solution}
*******************************************************************************
-------------------------------------------------------------------------------

\begin{problem}[Posted by \href{https://artofproblemsolving.com/community/user/363632}{mathisreal}]
	1) Find all function(s), such that $f(0) = 0$ and:
$f(x) = 1 + 7f(|\frac{x}{2}|) - 6f(|\frac{x}{4}|)$ for all $x$ positive real number.
Note: $|x|$ is the greatest integer is not greater than $x$.

2)Let $p(x)$ is a polynomial with integer coefficients. We have a sequence such that $a_0 = 0$ and $a_{n + 1} = p(a_n)$, suppose that there is a positive integer $m$ where $a_m = 0$.
Prove that $a_1 = 0$ or $a_2 = 0$
	\flushright \href{https://artofproblemsolving.com/community/c6h1574033}{(Link to AoPS)}
\end{problem}



\begin{solution}[by \href{https://artofproblemsolving.com/community/user/29428}{pco}]
	\begin{tcolorbox}1) Find all function(s), such that $f(0) = 0$ and:
$f(x) = 1 + 7f(|\frac{x}{2}|) - 6f(|\frac{x}{4}|)$ for all $x$ positive real number.
Note: $|x|$ is the greatest integer is not greater than $x$.\end{tcolorbox}
Setting $x=1$, we get $f(1)=1$
Setting $x=2$, we get $f(2)=8$

Let $a_n=f(2^n)$ and we get 
$a_0=1$, $a_1=8$ and $a_{n+2}=1+7a_{n+1}-6a_n$

Which is easily solved as $a_n=\frac{6^{n+2}-5n-11}{25}$

From there, it is immediate to get 
$\forall x\in(0,2)$ : $f(x)=1$
$\forall x\in[2^n,2^{n+1})$ ($n\in\mathbb N$) : $f(x)=f(2^n)=\frac{6^{n+2}-5n-11}{25}$

And so $\boxed{f(x)=\frac{6^{\max(0,\lfloor\log_2 x\rfloor)+2}-5\max(0,\lfloor\log_2 x\rfloor)-11}{25}\quad\forall x>0}$



\end{solution}
*******************************************************************************
-------------------------------------------------------------------------------

\begin{problem}[Posted by \href{https://artofproblemsolving.com/community/user/335975}{Taha1381}]
	Find all functions $f: \mathbb{R} \to \mathbb{R}$ that for any $x,y \in \mathbb{R}$ satisfy:

$f(x)+f(x+f(y))=y+f(f(x)+f(f(y)))$
	\flushright \href{https://artofproblemsolving.com/community/c6h1574379}{(Link to AoPS)}
\end{problem}



\begin{solution}[by \href{https://artofproblemsolving.com/community/user/394709}{amansdreamwillneverdie}]
	first of all i can prove that exist a real number a such that f(a)=0 anyone can continue?

\end{solution}



\begin{solution}[by \href{https://artofproblemsolving.com/community/user/394709}{amansdreamwillneverdie}]
	anyone can point out a possible function satisfy above problem?
\end{solution}



\begin{solution}[by \href{https://artofproblemsolving.com/community/user/29428}{pco}]
	\begin{tcolorbox}Find all functions $f: \mathbb{R} \to \mathbb{R}$ that for any $x,y \in \mathbb{R}$ satisfy:

$f(x)+f(x+f(y))=y+f(f(x)+f(f(y)))$\end{tcolorbox}
Let $P(x,y)$ be the assertion $f(x)+f(x+f(y))=y+f(f(x)+f(f(y)))$
Let $a=f(0)$ and $b=f(a)$

Subtracting $P(f(x),0)$ from $P(a,x)$, we get  $f(f(x))=b-x$
Note that this implies that $f(x)$ is bijective

Then $P(f(x),y)$ $\implies$ $f(f(x)+f(y))=x+y-b+f(2b-x-y)$
And $P(a,x+y)$ $\implies$ $f(f(0)+f(x+y))=x+y-b+f(2b-x-y)$
And so, since injective, $f(x+y)+a=f(x)+f(y)$ and $f(x)-a$ is additive.

And using this additivity back in original equation, it is easy to show that
$\boxed{\text{No such function}}$


\end{solution}



\begin{solution}[by \href{https://artofproblemsolving.com/community/user/335975}{Taha1381}]
	\begin{tcolorbox}[quote=Taha1381]Find all functions $f: \mathbb{R} \to \mathbb{R}$ that for any $x,y \in \mathbb{R}$ satisfy:

$f(x)+f(x+f(y))=y+f(f(x)+f(f(y)))$\end{tcolorbox}
Let $P(x,y)$ be the assertion $f(x)+f(x+f(y))=y+f(f(x)+f(f(y)))$
Let $a=f(0)$ and $b=f(a)$

Subtracting $P(f(x),0)$ from $P(a,x)$, we get  $f(f(x))=b-x$
Note that this implies that $f(x)$ is bijective

Then $P(f(x),y)$ $\implies$ $f(f(x)+f(y))=x+y-b+f(2b-x-y)$
And $P(a,x+y)$ $\implies$ $f(f(0)+f(x+y))=x+y-b+f(2b-x-y)$
And so, since injective, $f(x+y)+a=f(x)+f(y)$ and $f(x)-a$ is additive.

And using this additivity back in original equation, it is easy to show that
$\boxed{\text{No such function}}$\end{tcolorbox}

Thanks for your Nice answer.Do you know any source for learning functional equations and polynomials for a begginer?
\end{solution}



\begin{solution}[by \href{https://artofproblemsolving.com/community/user/29428}{pco}]
	\begin{tcolorbox}Do you know any source for learning functional equations and polynomials for a begginer?\end{tcolorbox}
I'm sorry but I'm neither a student, neither involved in education world (I just "play" maths since many many years). So I have no reference or link to give you.

Dont hesitate to browse AOPS forum. You'll find there tons of problems, solutions and comments.



\end{solution}
*******************************************************************************
-------------------------------------------------------------------------------

\begin{problem}[Posted by \href{https://artofproblemsolving.com/community/user/372289}{nangtoando}]
	Find $ k$: for continuous function: $ f: R\rightarrow  R$ satisfying:
$ f\left ( f\left ( x \right ) \right )= kx^{9}$
	\flushright \href{https://artofproblemsolving.com/community/c6h1574621}{(Link to AoPS)}
\end{problem}



\begin{solution}[by \href{https://artofproblemsolving.com/community/user/333350}{AnArtist}]
	I don't understand the question will you please explain? (I m new to F.E.)
\end{solution}



\begin{solution}[by \href{https://artofproblemsolving.com/community/user/391068}{TuZo}]
	\begin{tcolorbox}Find $ k$: for continuous function: $ f: R\rightarrow  R$ satisfying:
$ f\left ( f\left ( k \right ) \right )= kx^{9}$\end{tcolorbox}

I think that the correct problem is $ f\left ( f\left ( x \right ) \right )= kx^{9}$. It is true?


\end{solution}



\begin{solution}[by \href{https://artofproblemsolving.com/community/user/300322}{gausskarl}]
	That is "Find $k\in\mathbb{R}$ such that there exists a continuous function $f:\mathbb{R} \rightarrow\mathbb{R}$ satisfying $f(f(x))=kx^9$ for all $x\in\mathbb{R}$."
\end{solution}



\begin{solution}[by \href{https://artofproblemsolving.com/community/user/29428}{pco}]
	\begin{tcolorbox}That is "Find $k\in\mathbb{R}$ such that there exists a continuous function $f:\mathbb{R} \rightarrow\mathbb{R}$ satisfying $f(f(x))=kx^9$ for all $x\in\mathbb{R}$."\end{tcolorbox}
$k\ge 0$ is always possible (choose for example $f(x)=\sqrt[4]kx^3$)
$k<0$ is never possible since $k\ne 0$ implies $f(x)$ injective and so, since continuous, monotonous, and so $f(f(x))$ strictly increasing.
\end{solution}
*******************************************************************************
-------------------------------------------------------------------------------

\begin{problem}[Posted by \href{https://artofproblemsolving.com/community/user/335975}{Taha1381}]
	Find all functions $f:\mathbb{R} \to \mathbb{R}$ that satisfy:

$[f(x^4-5x^2+2015)^3]+[f(x^4-5x^2+2015)]=[x^4+x^2+1]$

Where $[x]$ denotes the floor function.

**Proposed by mohammad Jafari**
	\flushright \href{https://artofproblemsolving.com/community/c6h1574656}{(Link to AoPS)}
\end{problem}



\begin{solution}[by \href{https://artofproblemsolving.com/community/user/29428}{pco}]
	\begin{tcolorbox}Find all functions $f:\mathbb{R} \to \mathbb{R}$ that satisfy:

$[f(x^4-5x^2+2015)^3]+[f(x^4-5x^2+2015)]=[x^4+x^2+1]$

Where $[x]$ denotes the floor function.

**Proposed by mohammad Jafari**\end{tcolorbox}
$x=1$ implies $\lfloor f(2011)^3\rfloor+\lfloor f(2011)\rfloor=3$
$x=2$ implies $\lfloor f(2011)^3\rfloor+\lfloor f(2011)\rfloor=21$
And so contradiction

And $\boxed{\text{No such function}}$


\end{solution}



\begin{solution}[by \href{https://artofproblemsolving.com/community/user/391068}{TuZo}]
	Remark: Instead of $2011$ we got $2015$ if we use the roots of the $x^4-5x^2=0$, i.e. $x=0, $ $x=\pm\sqrt{5}$ and we got a contradiction.
\end{solution}
*******************************************************************************
-------------------------------------------------------------------------------

\begin{problem}[Posted by \href{https://artofproblemsolving.com/community/user/335975}{Taha1381}]
	Find all functions $f:\mathbb{Q} \to \mathbb{Q}$ that for any $m,n \in \mathbb{Q}$ satisfy:

$f(m+n)=\frac{1}{f(\frac{1}{m})+f(\frac{1}{n})}$

I think we should take $m,n \neq 0$.While the question doesn't have such conditions.
	\flushright \href{https://artofproblemsolving.com/community/c6h1574685}{(Link to AoPS)}
\end{problem}



\begin{solution}[by \href{https://artofproblemsolving.com/community/user/29428}{pco}]
	\begin{tcolorbox}Find all functions $f:\mathbb{Q} \to \mathbb{Q}$ that for any $m,n \in \mathbb{Q}$ satisfy:

$f(m+n)=\frac{1}{f(\frac{1}{m})+f(\frac{1}{n})}$

I think we should take $m,n \neq 0$.While the question doesn't have such conditions.\end{tcolorbox}
Let $P(x,y)$ be the assertion $f(x+y)=\frac 1{f(\frac 1x)+f(\frac 1y)}$, true $\forall x,y\ne 0$

Subtracting $P(x+y,z)$ from $P(x,y+z)$, we get
$f(\frac 1{x+y})+f(\frac 1z)=f(\frac 1x)+f(\frac 1{y+z})$ $\forall x,y,z,x+y,y+z\ne 0$

Setting $g(x)$ from $\mathbb Q\setminus\{0\}\to\mathbb Q$ as $g(x)=f(\frac 1x)$, we get
$g(x+y)+g(z)=g(x)+g(y+z)$ $\forall x,y,z,x+y,y+z\ne 0$

It is immediate from thee to get $g(x+y)=g(x)+h(y)$ for some function $h(x)$ from $\mathbb Q\setminus\{0\}\to\mathbb Q$
swapping $x,y$ and subtracting, we get $h(y)=g(y)-c$ for some contant $c$ and equation becomes
$g(x+y)=g(x)+g(y)-c$ $\forall x,y,x+y\ne 0$

Concluding $g(x)=ax+c$ (remember we deal with rational numbers) is easy.

And then $f(x)=\frac ax+c$
Plugging this in original equation, we need $c=0$ (else $c\notin\mathbb Q$) and $a\in\{-1,+1\}$

And so :
Either $\boxed{f(x)=\frac 1x\quad\forall x\in\mathbb Q\setminus\{0\}}$ 
Either $\boxed{f(x)=-\frac 1x\quad\forall x\in\mathbb Q\setminus\{0\}}$

And unfortunately none of these two solutions fits the requirements (just set $m=1$ and $n=-1$ in original equation.

The above solutions could fit if the problem was to find all functions from $\mathbb Q\setminus\{0\}\to\mathbb Q$

But, as often, bad indication of domain of function, codomain of function and domain of funtional equation ...


\end{solution}



\begin{solution}[by \href{https://artofproblemsolving.com/community/user/209049}{math90}]
	Maybe this problem:
Find all functions $f:\mathbb{Q^+} \to \mathbb{Q^+}$ that for any $m,n \in \mathbb{Q^+}$ satisfy:

$f(m+n)=\frac{1}{f(\frac{1}{m})+f(\frac{1}{n})}$

Your solution still works.
\end{solution}
*******************************************************************************
-------------------------------------------------------------------------------

\begin{problem}[Posted by \href{https://artofproblemsolving.com/community/user/152589}{rightways}]
	Find all functions $f:R->R$ for which
$$f(x-f(y)) =f(x) - [y] $$
for all reals $x, y\in R$

[y] is an integer part
	\flushright \href{https://artofproblemsolving.com/community/c6h1574804}{(Link to AoPS)}
\end{problem}



\begin{solution}[by \href{https://artofproblemsolving.com/community/user/29428}{pco}]
	\begin{tcolorbox}Find all functions $f:R->R$ for which
$$f(x-f(y)) =f(x) - [y] $$
for all reals $x, y\in R$

[y] is an integer part\end{tcolorbox}
Let $P(x,y)$ be the assertion $f(x-f(y))=f(x)-\lfloor y\rfloor$

If $f(u)=f(v)$, then $P(x,u)$ versus $P(x,v)$ implies $\lfloor u\rfloor=\lfloor v\rfloor$

$y\in[0,1)$ $\implies$ $f(x-f(y))=f(x)$ and so $\lfloor x-f(y)\rfloor=\lfloor x\rfloor$ $\forall x$
And so $f(x)=0$ $\forall x\in[0,1)$

$P(f(x),x)$ $\implies$ $f(f(x))=\lfloor x\rfloor$
$P(x,f(y))$ $\implies$ $f(x-\lfloor y\rfloor)=f(x)-\lfloor f(y)\rfloor$
And so $0=f(x-\lfloor x\rfloor)=f(x)-\lfloor f(x)\rfloor$ and so $f(x)\in\mathbb Z$ $\forall x$

Then repeated application of $P(x,y)$ implies $f(x-kf(y))=f(x)-k\lfloor y\rfloor$
Setting $k=f(z)$, this is $f(x-f(z)f(y))=f(x)-f(z)\lfloor y\rfloor$
Swapping $y,z$ : $f(x-f(y)f(z))=f(x)-f(y)\lfloor z\rfloor$

And so $f(z)\lfloor y\rfloor=f(y)\lfloor z\rfloor$ $\forall y,z$
And so $f(x)=k\lfloor x\rfloor$ $\forall x$ and for some $k\in\mathbb Z$

Plugging this back in original equation, we get $k^2=1$ and so :

$\boxed{\text{S1 : }f(x)=\lfloor x\rfloor\quad\forall x}$ and $\boxed{\text{S2 : }f(x)=-\lfloor x\rfloor\quad\forall x}$




\end{solution}
*******************************************************************************
-------------------------------------------------------------------------------

\begin{problem}[Posted by \href{https://artofproblemsolving.com/community/user/361694}{Muradjl}]
	Find all functions $f :\mathbb{N} \rightarrow \mathbb{R}$ such that :

 $i)$ $f$ is multiplicative
 $ii)$ $f(n+2017)=f(n)$ for all positive integers $n$
	\flushright \href{https://artofproblemsolving.com/community/c6h1574931}{(Link to AoPS)}
\end{problem}



\begin{solution}[by \href{https://artofproblemsolving.com/community/user/333350}{AnArtist}]
	\begin{tcolorbox}Find all functions $f :\mathbb{N} \rightarrow \mathbb{R}$ such that :

 $i)$ $f$ is multiplicative
 $ii)$ $f(n+2017)=f(n)$ for all positive integers $n$\end{tcolorbox}

Weird F.E. I got three functions.

1)$f(n)=0$

2)$f(n)=1$ if n is not divisible by 2017 and $f(n)=f(2017)$ if n is divisible by 2017 and $f(2017)$ is arbitrary.

3)$f(1)=1$ , $f$ is arbitrary for powers of prime except 2017 , $f(2017^k)=0$.

Edit- These are the solutions when the multiplicative condition is interpretated as $f(mn)=f(m)f(n)$ when $gcd(m,n)=1$
\end{solution}



\begin{solution}[by \href{https://artofproblemsolving.com/community/user/29428}{pco}]
	\begin{tcolorbox}Find all functions $f :\mathbb{N} \rightarrow \mathbb{R}$ such that :

 $i)$ $f$ is multiplicative
 $ii)$ $f(n+2017)=f(n)$ for all positive integers $n$\end{tcolorbox}
For easier writing, let $p=2017$ (prime number).
From ii, we get that $f(x)$ can only take a finite number of values (at most $2017$)
From i, we get that if $a\in f(\mathbb N)$, then $a^2,a^3,...,a^k\in f(\mathbb N)$ too

So, combining these two assertions, $f(n)\in\{-1,0,1\}$ $\forall n\in\mathbb N$

If $f(1)\ne 1$, then $f(n)=f(n)f(1)$ gives solution :
$\boxed{\text{S1 : }f(n)=0\quad\forall n\in\mathbb N}$ which indeed is a solution

If $f(p)\ne 0$, then $f(p)=f(np)$ (using ii) $=f(n)f(p)$ (using i) and so 
$\boxed{\text{S2 : }f(n)=1\quad\forall n\in\mathbb N}$ which indeed is a solution

If $f(1)=1$ and $f(p)=0$
If $f(x)=0$ for some $x\not\equiv 0\pmod p$ :
Let $y\in\mathbb N$ such that $xy\equiv 1\pmod p$ : $f(xy)=f(1)=1$ (using ii) and $f(xy)=f(x)f(y)=0$ (using i)
So contradiction  and $f(x)\in\{-1,+1\}$ $\forall x\not\equiv 0\pmod p$

It remains to see that $f(n)=1$ for any quadratic residue mod $2017$ and that :
Either $f(n)=1$ for any non quadratic residue mod $p$
Either $f(n)=-1$ for any non quadratic residue mod $p$

And so the two last solutions :
$\boxed{\text{S3 : }f(n)=0\quad\forall n\equiv 0\pmod{2017}\text{  and  }f(n)=1\quad\forall n\not\equiv 0\pmod{2017}}$ which indeed is a solution

$\boxed{\text{S4 : }f(n)=\left(\frac n{2017}\right)\forall n\text{ (Legendre symbol)}}$ which indeed is a solution



\end{solution}



\begin{solution}[by \href{https://artofproblemsolving.com/community/user/333350}{AnArtist}]
	@pco I don't understand why if $a \in f(n)$ then $a^k \in f(n)$.
I thought multiplicative means $f(mn)=f(m)f(n)$ if $gcd(m,n)=1$
\end{solution}



\begin{solution}[by \href{https://artofproblemsolving.com/community/user/29428}{pco}]
	\begin{tcolorbox}@pco I don't understand why if $a \in f(n)$ then $a^k \in f(n)$.
I thought multiplicative means $f(mn)=f(m)f(n)$ if $gcd(m,n)=1$\end{tcolorbox}
Multiplicative means $f(mn)=f(m)f(n)$
Nothing more.
Nothing less.




\end{solution}



\begin{solution}[by \href{https://artofproblemsolving.com/community/user/333350}{AnArtist}]
	Ok then my solution is wrong.
\end{solution}



\begin{solution}[by \href{https://artofproblemsolving.com/community/user/342536}{xilias}]
	\begin{tcolorbox}[quote=AnArtist]@pco I don't understand why if $a \in f(n)$ then $a^k \in f(n)$.
I thought multiplicative means $f(mn)=f(m)f(n)$ if $gcd(m,n)=1$\end{tcolorbox}
Multiplicative means $f(mn)=f(m)f(n)$
Nothing more.
Nothing less.\end{tcolorbox}

I think it actually depends on whether the function is defined on reals or on integers
[hide=<Wikipedia][img]https:\/\/i.imgur.com\/9KyiS02.png[\/img][\/hide]
but nonetheless we have $f(a)=f(b)$ whenever $a \equiv b [2017]$,and we have infinitely many coprime numbers $n_1,n_2...$ such that $n_i \equiv a [2017]$,so $f(1)=f(a^2016)=f(n_{1}...n_{2016})=f(a)^{2016}$ and then your argument works.

\end{solution}



\begin{solution}[by \href{https://artofproblemsolving.com/community/user/342536}{xilias}]
	Also if we interpret "multiplicative" as $f(ab)=f(a)f(b)$ when $gcd(a,b)=1$ then we can have another set of solutions wich is $f(2017n)=c$ for all $n$ where $c$ is a real constant and $f(n)=1$ when $2017$ divides $n$
\end{solution}



\begin{solution}[by \href{https://artofproblemsolving.com/community/user/333350}{AnArtist}]
	Please see post no. 2 to see all the solutions for that interpretation.
\end{solution}



\begin{solution}[by \href{https://artofproblemsolving.com/community/user/342538}{conker23}]
	multiplicative is definitely on coprime numbers. Otherwise, we say strongly multiplicative.
\end{solution}
*******************************************************************************
-------------------------------------------------------------------------------

\begin{problem}[Posted by \href{https://artofproblemsolving.com/community/user/90995}{hakarimian}]
	Find all continuous functions $f:R\rightarrow R$ such that:
$f(x-f(y))=f(x)-y$ for all real numbers $x,y$.
	\flushright \href{https://artofproblemsolving.com/community/c6h1575796}{(Link to AoPS)}
\end{problem}



\begin{solution}[by \href{https://artofproblemsolving.com/community/user/390573}{Idea_lover}]
	\begin{tcolorbox}Find all functions $f:R\rightarrow R$ such that:
$f(x-f(y))=f(x)-y$ for all real numbers $x,y$.\end{tcolorbox}

I managed to prove $f$ is bijective ...
\end{solution}



\begin{solution}[by \href{https://artofproblemsolving.com/community/user/333350}{AnArtist}]
	Me too......
\end{solution}



\begin{solution}[by \href{https://artofproblemsolving.com/community/user/390573}{Idea_lover}]
	Seems like the only solutions are $f(x)=x$ and $f(x)=-x$.
\end{solution}



\begin{solution}[by \href{https://artofproblemsolving.com/community/user/390573}{Idea_lover}]
	What was wrong with The Dark Prince's proof
\end{solution}



\begin{solution}[by \href{https://artofproblemsolving.com/community/user/333350}{AnArtist}]
	It never led to solution
\end{solution}



\begin{solution}[by \href{https://artofproblemsolving.com/community/user/390573}{Idea_lover}]
	\begin{tcolorbox}Hmmm writing mistake.\end{tcolorbox}

You managed the problem?
\end{solution}



\begin{solution}[by \href{https://artofproblemsolving.com/community/user/333350}{AnArtist}]
	Nope.........
\end{solution}



\begin{solution}[by \href{https://artofproblemsolving.com/community/user/297527}{TheDarkPrince}]
	Some progress :(:

Let $P(x,y)$ be the assertion of the problem statement.

Claim 1: $f$ is bijective.
Proof: Let $f(a)=f(b)$. $P(x,a)$ and $P(x,b)$ gives $-a=-b$. So, $f$ is injective. $P(0,y)$ gives $f$ is surjective.

Claim 2: $f(0)=0$.
Proof: Let $f(a)=0$. $P(a,a)$ gives $0=f(a)=-a$. So, $f(0)=0$.

Claim 3: $f$ is additive
Proof: $P(0,x)$ gives $f(-f(x))=-x$. $P(x,-f(y))$ gives $f(x+y)=f(x)+f(y)$.

$P(f(x),x)$ gives $x=f(f(x))$.
\end{solution}



\begin{solution}[by \href{https://artofproblemsolving.com/community/user/390573}{Idea_lover}]
	\begin{tcolorbox}Some progress :(:

Let $P(x,y)$ be the assertion of the problem statement.

Claim 1: $f$ is bijective.
Proof: Let $f(a)=f(b)$. $P(x,a)$ and $P(x,b)$ gives $-a=-b$. So, $f$ is injective. $P(0,y)$ gives $f$ is surjective.

Claim 2: $f(0)=0$.
Proof: Let $f(a)=0$. $P(a,a)$ gives $0=f(a)=-a$. So, $f(0)=0$.

Claim 3: $f$ is additive
Proof: $P(0,x)$ gives $f(-f(x))=-x$. $P(x,-f(y))$ gives $f(x+y)=f(x)+f(y)$.

$P(f(x),x)$ gives $x=f(f(x))$.

Claim 4: $f$ is odd
Proof: $P(x,-y)$ gives $f(x-f(-y))=f(x)+y = f(x)+f(f(y))=f(x+f(y))$. So, $-f(-y)=f(y)$.\end{tcolorbox}

Everybody has made this progress, thanks for typing the whole thing  :) 
\end{solution}



\begin{solution}[by \href{https://artofproblemsolving.com/community/user/333350}{AnArtist}]
	Same here can't move further. :(
\end{solution}



\begin{solution}[by \href{https://artofproblemsolving.com/community/user/362567}{ayan.nmath}]
	I am pretty sure that this has numerous non-continous solutions.(pathological solutions)
\end{solution}



\begin{solution}[by \href{https://artofproblemsolving.com/community/user/333350}{AnArtist}]
	I think I got the solution.
\end{solution}



\begin{solution}[by \href{https://artofproblemsolving.com/community/user/390573}{Idea_lover}]
	\begin{tcolorbox}I think I got the solution.\end{tcolorbox}

Post it then.
\end{solution}



\begin{solution}[by \href{https://artofproblemsolving.com/community/user/333350}{AnArtist}]
	I m not sure if this is correct.

We claim that $f$ is monotonically increasing or decreasing.

Clearly $f$ is not constant. 

If there exists $p,q$ such that $p>q$ and $f(p)>f(q)$

$P(p,f(q))$ gives $f(p-q) = f(p)-f(q)>0$

So $P(x, q-p)$ gives $f$ is monotonically increasing.

Similarly we can deal with the other case.

Since $f$ is additive and monotonic $f(x)=kx$ (by Cauchy equation.)

Edit- incorrect solution.
\end{solution}



\begin{solution}[by \href{https://artofproblemsolving.com/community/user/333350}{AnArtist}]
	Oops sorry I got a typo I'll edit it.
\end{solution}



\begin{solution}[by \href{https://artofproblemsolving.com/community/user/390573}{Idea_lover}]
	............
\end{solution}



\begin{solution}[by \href{https://artofproblemsolving.com/community/user/390573}{Idea_lover}]
	\begin{tcolorbox}I m not sure if this is correct.

We claim that $f$ is monotonically increasing or decreasing.

Clearly $f$ is not constant. 

If there exists $p,q$ such that $p>q$ and $f(p)>f(q)$

$P(p,f(q))$ gives $f(p-q) = f(p)-f(q)$

So $P(x, p-q)$ gives $f$ is monotonically increasing.

Similarly we can deal with the other case.

Since $f$ is additive and monotonic $f(x)=kx$ (by Cauchy equation.)\end{tcolorbox}

So this solves the cauchy's F.E without any assumptions (boundedness,monotonicity) etc... because you used no other fact other than the cauchy's F.E..... I don't think it's correct... though I can't find any mistake in it. :maybe:\end{tcolorbox}


\end{solution}



\begin{solution}[by \href{https://artofproblemsolving.com/community/user/333350}{AnArtist}]
	@ayan see the last lines of https:\/\/artofproblemsolving.com\/wiki\/index.php?title=Cauchy_Functional_Equation

Monotonicity can be used for Cauchy.
\end{solution}



\begin{solution}[by \href{https://artofproblemsolving.com/community/user/333350}{AnArtist}]
	\begin{tcolorbox}[quote=AnArtist]I m not sure if this is correct.

We claim that $f$ is monotonically increasing or decreasing.

Clearly $f$ is not constant. 

If there exists $p,q$ such that $p>q$ and $f(p)>f(q)$

$P(p,f(q))$ gives $f(p-q) = f(p)-f(q)$

So $P(x, p-q)$ gives $f$ is monotonically increasing.

Similarly we can deal with the other case.

Since $f$ is additive and monotonic $f(x)=kx$ (by Cauchy equation.)\end{tcolorbox}

So this solves the cauchy's F.E without any assumptions (boundedness,monotonicity) etc... because you used no other fact other than the cauchy's F.E..... I don't think it's correct... though I can't find any mistake in it. :maybe:\end{tcolorbox}\end{tcolorbox}


Kindly note that we had two conditions on our F.E.

1)$f(f(x))=x$
2)$f$ is additive.

So we didn't solve Cauchy equation without any assumption. We assumed the 1st condition.
\end{solution}



\begin{solution}[by \href{https://artofproblemsolving.com/community/user/333350}{AnArtist}]
	Sorry my solution is incorrect.
\end{solution}



\begin{solution}[by \href{https://artofproblemsolving.com/community/user/367931}{Vrangr}]
	It's easy to get that $f(f(x))=x$. @TheDarkPrince proved it too.

\begin{tcolorbox}Some progress :(:

Let $P(x,y)$ be the assertion of the problem statement.

Claim 1: $f$ is bijective.
Proof: Let $f(a)=f(b)$. $P(x,a)$ and $P(x,b)$ gives $-a=-b$. So, $f$ is injective. $P(0,y)$ gives $f$ is surjective.

Claim 2: $f(0)=0$.
Proof: Let $f(a)=0$. $P(a,a)$ gives $0=f(a)=-a$. So, $f(0)=0$.

Claim 3: $f$ is additive
Proof: $P(0,x)$ gives $f(-f(x))=-x$. $P(x,-f(y))$ gives $f(x+y)=f(x)+f(y)$.

$P(f(x),x)$ gives $x=f(f(x))$.\end{tcolorbox}
However, @AnArtist's solutiion is valid if and only if there exist atleast one pair of $p,q$ such $p>q$ and $f(p)>f(q)$ but this is not always true. Eg-$f(x)=-x$
\end{solution}



\begin{solution}[by \href{https://artofproblemsolving.com/community/user/390573}{Idea_lover}]
	\begin{tcolorbox}It's easy to get that $f(f(x))=x$. @TheDarkPrince proved it too.

\begin{tcolorbox}Some progress :(:

Let $P(x,y)$ be the assertion of the problem statement.

Claim 1: $f$ is bijective.
Proof: Let $f(a)=f(b)$. $P(x,a)$ and $P(x,b)$ gives $-a=-b$. So, $f$ is injective. $P(0,y)$ gives $f$ is surjective.

Claim 2: $f(0)=0$.
Proof: Let $f(a)=0$. $P(a,a)$ gives $0=f(a)=-a$. So, $f(0)=0$.

Claim 3: $f$ is additive
Proof: $P(0,x)$ gives $f(-f(x))=-x$. $P(x,-f(y))$ gives $f(x+y)=f(x)+f(y)$.

$P(f(x),x)$ gives $x=f(f(x))$.\end{tcolorbox}
However, @AnArtist's solutiion is valid if and only if there exist atleast one pair of $p,q$ such $p>q$ and $f(p)>f(q)$ but this is not always true. Eg-$f(x)=-x$\end{tcolorbox}

There's an even big error in his sol.
\end{solution}



\begin{solution}[by \href{https://artofproblemsolving.com/community/user/333350}{AnArtist}]
	\begin{tcolorbox}It's easy to get that $f(f(x))=x$. @TheDarkPrince proved it too.

\begin{tcolorbox}Some progress :(:

Let $P(x,y)$ be the assertion of the problem statement.

Claim 1: $f$ is bijective.
Proof: Let $f(a)=f(b)$. $P(x,a)$ and $P(x,b)$ gives $-a=-b$. So, $f$ is injective. $P(0,y)$ gives $f$ is surjective.

Claim 2: $f(0)=0$.
Proof: Let $f(a)=0$. $P(a,a)$ gives $0=f(a)=-a$. So, $f(0)=0$.

Claim 3: $f$ is additive
Proof: $P(0,x)$ gives $f(-f(x))=-x$. $P(x,-f(y))$ gives $f(x+y)=f(x)+f(y)$.

$P(f(x),x)$ gives $x=f(f(x))$.\end{tcolorbox}
However, @AnArtist's solutiion is valid if and only if there exist atleast one pair of $p,q$ such $p>q$ and $f(p)>f(q)$ but this is not always true. Eg-$f(x)=-x$\end{tcolorbox}

Nope we have proved if there exists such $p,q$ then function is increasing . if there do not exist such p,q then function is decreasing.
\end{solution}



\begin{solution}[by \href{https://artofproblemsolving.com/community/user/390573}{Idea_lover}]
	Man you are awesome !, changed the question now !!!
\end{solution}



\begin{solution}[by \href{https://artofproblemsolving.com/community/user/90995}{hakarimian}]
	\begin{tcolorbox}Man you are awesome !, changed the question now !!!\end{tcolorbox}

Only the continuity was added.
\end{solution}



\begin{solution}[by \href{https://artofproblemsolving.com/community/user/390573}{Idea_lover}]
	That matters a lot !
\end{solution}



\begin{solution}[by \href{https://artofproblemsolving.com/community/user/350483}{adhikariprajitraj}]
	Elegant solution!!
\end{solution}



\begin{solution}[by \href{https://artofproblemsolving.com/community/user/333350}{AnArtist}]
	\begin{tcolorbox}[quote=Idea_lover]Man you are awesome !, changed the question now !!!\end{tcolorbox}

Only the continuity was added.\end{tcolorbox}

Are you kidding me?  That trivializes the problem with cauchys equation. 

\end{solution}



\begin{solution}[by \href{https://artofproblemsolving.com/community/user/90995}{hakarimian}]
	\begin{tcolorbox}[quote=hakarimian][quote=Idea_lover]Man you are awesome !, changed the question now !!!\end{tcolorbox}

Only the continuity was added.\end{tcolorbox}

Are you kidding me?  That trivializes the problem with cauchys equation.\end{tcolorbox}

Otherwise it is not solvable. Unless you find a solution.
\end{solution}



\begin{solution}[by \href{https://artofproblemsolving.com/community/user/29428}{pco}]
	\begin{tcolorbox}Find all [recently added\end{underlined} "continuous"] functions $f:R\rightarrow R$ such that:
$f(x-f(y))=f(x)-y$ for all real numbers $x,y$.\end{tcolorbox}
Let $P(x,y)$ be the assertion $f(x-f(y))=f(x)-y$

$P(f(x),x)$ $\implies$ $f(f(x))=x+f(0)$ and so $f(x)$ is bijective
Let then $u$ such that $f(u)=0$ : $P(x,u)$ $\implies$ $u=0$ and so $f(0)=0$

And so $f(f(x))=x$
$P(x+y,f(y))$ $\implies$ $f(x+y)=f(x)+f(y)$

Easy to conclude that problem is equivalent to \begin{bolded}"find all involutive additive functions"\end{underlined}\end{bolded}.

Now, with recently added continuity condition, additivity implies linearity and involutivity implies two solutions :
$\boxed{\text{S1 }f(x)=x\quad\forall x}$ and $\boxed{\text{S2 }f(x)=-x\quad\forall x}$

Without recently added continuity, problem remains quite classical :
Let $A,B$ any two supplementary subvectorspaces of the $\mathbb Q$-vectorspace $\mathbb R$
Let $a(x)$ from $\mathbb R\to A$ and $b(x)$ from $\mathbb R\to B$ the two projections of a real $x$ in $(A,B)$
Then $\boxed{f(x)=a(x)-b(x)}$

Note that $(A,B)=(\mathbb R,\{0\})$ gives the trivial $f(x)=x$
And that $(A,B)=(\{0\},\mathbb R)$ gives the trivial $f(x)=-x$
And a lot of other solutons ...


\end{solution}
*******************************************************************************
-------------------------------------------------------------------------------

\begin{problem}[Posted by \href{https://artofproblemsolving.com/community/user/350436}{ShantoBro}]
	For any rational numbers $x, y$ function $f(x)$ is a real number and $f(x+y)=f(x)f(y)-f(xy)+1$. Again $f(2017)\neq f(2018)$. $f(\frac{2017}{2018})=\frac{a}{b}$.Where $a,b$ are co-prime $a+b=?$

Long Solution need which will help me understanding solving this kind of problems!  Thanks in Advance :)
	\flushright \href{https://artofproblemsolving.com/community/c6h1576392}{(Link to AoPS)}
\end{problem}



\begin{solution}[by \href{https://artofproblemsolving.com/community/user/29428}{pco}]
	\begin{tcolorbox}For any rational numbers $x, y$ function $f(x)$ is a real number and $f(x+y)=f(x)f(y)-f(xy)+1$. Again $f(2017)\neq f(2018)$. $f(\frac{2017}{2018})=\frac{a}{b}$.Where $a,b$ are co-prime $a+b=?$\end{tcolorbox}
Let $P(x,y)$ be the assertion $f(x+y)=f(x)f(y)-f(xy)+1$
$P(0,0)$ $\implies$ $f(0)=1$
$P(x,1)$ $\implies$ $f(x+1)=cf(x)+1$ where $c=f(1)-1$
This implies easily $f(2)=c^2+c+1$ and $f(4)=c^4+c^3+c^2+c+1$

Then $P(2,2)$ $\implies$ $c^4=c^2$ and so $c\in\{-1,0,1\}$

$c=0$ implies $f(x+1)=f(x)$ and so $f(2017)=f(2018)$, impossible

$c=-1$ implies $f(1)=0$ and $f(2)=1$ and $f(x+1)=1-f(x)$ and so $f(x+2)=f(x)$
Then $P(x,2)$ $\implies$ $f(2x)=1$ and so $f(x)=1$ $\forall x$, in contradiction with $f(2017)\ne f(2018)$

So $c=1$ and $f(1)=2$ and $f(x+1)=f(x)+1$ and so $f(x+n)=f(x)+n$
Then $P(x,n)$ $\implies$ $f(nx)-1=n(f(x)-1)$
This immediately implies $f(x)=x+1$ $\forall x\in\mathbb Q$

So $f(\frac{2017}{2018})=\frac{2017}{2018}+1=\frac{4035}{2018}$ and so required result is $\boxed{6053}$



\end{solution}



\begin{solution}[by \href{https://artofproblemsolving.com/community/user/333350}{AnArtist}]
	\begin{tcolorbox}
Then $P(x,n)$ $\implies$ $f(nx)-1=n(f(x)-1)$
This immediately implies $f(x)=x+1$ $\forall x\in\mathbb Q$

\end{tcolorbox}

Why ?


\end{solution}



\begin{solution}[by \href{https://artofproblemsolving.com/community/user/29428}{pco}]
	\begin{tcolorbox}[quote=pco]
Then $P(x,n)$ $\implies$ $f(nx)-1=n(f(x)-1)$
This immediately implies $f(x)=x+1$ $\forall x\in\mathbb Q$

\end{tcolorbox}

Why ?\end{tcolorbox}
Let $g(x)=f(x)-1$ : we have $g(nx)=ng(x)$ and so $g(\frac pqx)=\frac pqg(x)$ and so $g(\frac pq)=\frac pqg(1)$

And so $g(x)=xg(1)$ $\forall x\in\mathbb Q$

And so $f(x)=x(f(1)-1)+1$ $\forall x\in\mathbb Q$

And so $f(x)=x+1$ $\forall x\in\mathbb Q$



\end{solution}



\begin{solution}[by \href{https://artofproblemsolving.com/community/user/333350}{AnArtist}]
	@above you assumed $n$ is an integer.
\end{solution}



\begin{solution}[by \href{https://artofproblemsolving.com/community/user/29428}{pco}]
	\begin{tcolorbox}@above you assumed $n$ is an integer.\end{tcolorbox}

Obviously, since I used the previously proved $f(x+n)=f(x)+n$ and $f(n)=n+1$ ........
\end{solution}



\begin{solution}[by \href{https://artofproblemsolving.com/community/user/333350}{AnArtist}]
	\begin{tcolorbox}
Let $g(x)=f(x)-1$ : we have $g(nx)=ng(x)$ and so $g(\frac pqx)=\frac pqg(x)$ and so $g(\frac pq)=\frac pqg(1)$

\end{tcolorbox}

You assumed $n$ is an integer. But then how did you substitute $\frac{p}{q}$ instead of $n$.
\end{solution}



\begin{solution}[by \href{https://artofproblemsolving.com/community/user/29428}{pco}]
	\begin{tcolorbox}You assumed $n$ is an integer. But then how did you substitute $\frac{p}{q}$ instead of $n$.\end{tcolorbox}
Muhhhh! Make a litlle effort !
$g(px)=pg(x)$ $\implies$ $g(\frac xq)=\frac 1qg(x)$ and so $g(\frac pqx)=\frac pqg(x)$

This is quite quite basic and classical (this is for example the basic method for solving Cauchy equaton over $\mathbb Q$)



\end{solution}



\begin{solution}[by \href{https://artofproblemsolving.com/community/user/333350}{AnArtist}]
	Sorry don't mind. I have just started studying F.E's
\end{solution}
*******************************************************************************
-------------------------------------------------------------------------------

\begin{problem}[Posted by \href{https://artofproblemsolving.com/community/user/335975}{Taha1381}]
	The function $f: \mathbb{R}^+ \to \mathbb{R} ^+$ satisfy the following property:

$f(x+f(x)+2y+f(z))=x+f(x)+y+f(y)+2f(z)$

Prove that $f$ is injective.
	\flushright \href{https://artofproblemsolving.com/community/c6h1576404}{(Link to AoPS)}
\end{problem}



\begin{solution}[by \href{https://artofproblemsolving.com/community/user/29428}{pco}]
	\begin{tcolorbox}The function $f: \mathbb{R}^+ \to \mathbb{R} ^+$ satisfy the following property:

$f(x+f(x)+2y+f(z))=x+f(x)+y+f(y)+2f(z)$

Prove that $f$ is injective.\end{tcolorbox}

Fake problem.
Easy to prove that such function does not exist.
So we can prove anything we want :
- it is injective
- it is surjective
- it is periodic (and so non injective)
- is has uniquely value in prime numbers (and so non surjective)
....
\end{solution}



\begin{solution}[by \href{https://artofproblemsolving.com/community/user/335975}{Taha1381}]
	\begin{tcolorbox}[quote=Taha1381]The function $f: \mathbb{R}^+ \to \mathbb{R} ^+$ satisfy the following property:

$f(x+f(x)+2y+f(z))=x+f(x)+y+f(y)+2f(z)$

Prove that $f$ is injective.\end{tcolorbox}

Fake problem.
Easy to prove that such function does not exist.
So we can prove anything we want :
- it is injective
- it is surjective
- it is periodic (and so non injective)
- is has uniquely value in prime numbers (and so non surjective)
....\end{tcolorbox}

In fact the problem asked for all such functions.Since it was in the chapter"using injectivity" and by using the injectivity we can prove there are no such functions I only asked for proving the Injectivity.Anyway how do you prove there are no such functions?
\end{solution}



\begin{solution}[by \href{https://artofproblemsolving.com/community/user/333350}{AnArtist}]
	\begin{tcolorbox}[quote=Taha1381]The function $f: \mathbb{R}^+ \to \mathbb{R} ^+$ satisfy the following property:

$f(x+f(x)+2y+f(z))=x+f(x)+y+f(y)+2f(z)$

Prove that $f$ is injective.\end{tcolorbox}

Fake problem.
Easy to prove that such function does not exist.
So we can prove anything we want :
- it is injective
- it is surjective
- it is periodic (and so non injective)
- is has uniquely value in prime numbers (and so non surjective)
....\end{tcolorbox}

Wow that's cool.
\end{solution}



\begin{solution}[by \href{https://artofproblemsolving.com/community/user/29428}{pco}]
	\begin{tcolorbox}Anyway how do you prove there are no such functions?\end{tcolorbox}
Let $P(x,y,z)$ be the assertion $f(x+f(x)+2y+f(z))=x+f(x)+y+f(y)+2f(z)$
If the function is injective, then simple swapping of $x,y$ implies $f(x)=x+c$, which is never a solution.
So $f(x)$ is not injective and $\exists$ $a>b$ such that $f(a)=f(b)$
Let $t=\frac{a-b}2>0$

Then comparing $P(a,x,1)$ with $P(b,x+t,1)$, we get $f(x+t)=f(x)+t$ $\forall x$

Comparing then $P(1,1,1)$ with $P(1,1,t+1)$ gives final contradiction.
\end{solution}
*******************************************************************************
-------------------------------------------------------------------------------

\begin{problem}[Posted by \href{https://artofproblemsolving.com/community/user/292982}{dmowziz}]
	$[f (x+y) = yf(x) + xf(y)]$

I guess this problem is too easy ( it's exercise 1 from a book) but I can't do it :(

Please help a novice out

for all real x and y except 0
	\flushright \href{https://artofproblemsolving.com/community/c6h1576541}{(Link to AoPS)}
\end{problem}



\begin{solution}[by \href{https://artofproblemsolving.com/community/user/391068}{TuZo}]
	Put $x=lna, y=lnb$, and denote $g(t)=f(lnt)\/t$, and you got the Cauchy functional equation $g(a+b)=g(a)+g(b)$.
You can read about this here:
[url]https:\/\/en.wikipedia.org\/wiki\/Cauchy%27s_functional_equation[\/url] or here:
[url]https:\/\/ipfs.io\/ipfs\/QmXoypizjW3WknFiJnKLwHCnL72vedxjQkDDP1mXWo6uco\/wiki\/Cauchy%27s_functional_equation.html[\/url]
\end{solution}



\begin{solution}[by \href{https://artofproblemsolving.com/community/user/279587}{RagvaloD}]
	$f(x)=f(x-1+1)=f(x-1)+(x-1)f(1)=-f(x)+xf(-1)+(x-1)f(1)$
$2f(x)=x(f(1)+f(-1))-f(1)$ so $f(x)=ax+b$
$ax+b=a(x-1)+b+(x-1)(a+b)$
$(x-1)(a+b)-a=0 \to a=0,b=0$
So $f(x) \equiv 0$
\end{solution}



\begin{solution}[by \href{https://artofproblemsolving.com/community/user/29428}{pco}]
	\begin{tcolorbox}$[f (x+y) = yf(x) + xf(y)]$

I guess this problem is too easy ( it's exercise 1 from a book) but I can't do it :(

Please help a novice out

for all real x and y except 0\end{tcolorbox}
Let $P(x,y)$ be the assertion $f(x+y)=xf(y)+yf(x)$ true $\forall x,y\ne 0$
I suppose that missing domain and codomain of $f(x)$ both are $\mathbb R$ (while domain of functional equation is $(\mathbb R\setminus\{0\})^2$)

$P(1,1)$ $\implies$ $f(2)=2f(1)$
$P(x,2)$ $\implies$ $f(x+2)=2xf(1)+2f(x)$ $\forall x\ne 0$

$P(x,1)$ $\implies$ $f(x+1)=xf(1)+f(x)$ $\forall x\ne 0$
$P(1,x+1)$ $\implies$ $f(x+2)=f(x)+(2x+1)f(1)$ $\forall x\ne -1$

Equating the two expressions we got for $f(x+2)$, we get $f(x)=f(1)$ $\forall x\notin\{-1,0\}$

Plugging this back in original equation, we get $f(x)=0$ $\forall x\notin\{-1,0\}$
Then $P(2,-2)$ $\implies$ $f(0)=0$
And $P(2,-3)$ $\implies$ $f(-1)=0$

And so $\boxed{f(x)=0\quad\forall x}$


\end{solution}



\begin{solution}[by \href{https://artofproblemsolving.com/community/user/355697}{LoveMaths26102003}]
	Continuity or increasingness is missing then only Cauchy can be applied
\end{solution}



\begin{solution}[by \href{https://artofproblemsolving.com/community/user/29428}{pco}]
	\begin{tcolorbox}Continuity or increasingness is missing then only Cauchy can be applied\end{tcolorbox}

Nonsense
This problem has quite no relation with Cauchy equation.

\end{solution}



\begin{solution}[by \href{https://artofproblemsolving.com/community/user/355697}{LoveMaths26102003}]
	Ya thats what i meant to say. I meant that for @TuZo.
\end{solution}
*******************************************************************************
-------------------------------------------------------------------------------

\begin{problem}[Posted by \href{https://artofproblemsolving.com/community/user/247598}{MATH1945}]
	find all continous functions from real to itself such that for any $x,y,z$ real, $$f(x+y+z)+f(x)+f(y)+f(z)=f(x+y)+f(y+z)+f(z+x)$$
	\flushright \href{https://artofproblemsolving.com/community/c6h1577506}{(Link to AoPS)}
\end{problem}



\begin{solution}[by \href{https://artofproblemsolving.com/community/user/29428}{pco}]
	\begin{tcolorbox}find all functions from real to itself such that for any $x,y,z$ real, $$f(x+y+z)+f(x)+f(y)+f(z)=f(x+y)+f(y+z)+f(z+x)$$\end{tcolorbox}

Are you sure you did not forget the constraint "continuous" ?
With this constraint, it has been posted many times, with solutions $ax^2+bx$ (use search function)

Without this continuity constraint,  lot of weird solutions will appear.

\end{solution}



\begin{solution}[by \href{https://artofproblemsolving.com/community/user/247598}{MATH1945}]
	@pco I've searched and got nothing.
\end{solution}



\begin{solution}[by \href{https://artofproblemsolving.com/community/user/29428}{pco}]
	\begin{tcolorbox}@pco I've searched and got nothing.\end{tcolorbox}

What search string did you use ?

\end{solution}



\begin{solution}[by \href{https://artofproblemsolving.com/community/user/247598}{MATH1945}]
	nevermind i forget to use " lol
\end{solution}
*******************************************************************************
-------------------------------------------------------------------------------

\begin{problem}[Posted by \href{https://artofproblemsolving.com/community/user/376479}{davidmath7}]
	$f:Z  -> Z $

$f(f(n)) - f(n) - 1 = 0$

\begin{bolded}\begin{italicized}Find f\end{italicized}\end{bolded}
	\flushright \href{https://artofproblemsolving.com/community/c6h1579145}{(Link to AoPS)}
\end{problem}



\begin{solution}[by \href{https://artofproblemsolving.com/community/user/376213}{Wizard_32}]
	We claim that the only solution is $f(n)=n+1$ for all integers $n$. It is easy to see that this works. We now show that this is the only possible solution.

\begin{bolded}Claim:\end{bolded} $f$ is surjective
\begin{italicized}Proof:\end{italicized} Let there exist an integer $m$ such that the equation $f(n)=m$ has no integer value of $n$ satisfying it.
Let $r$ be an integer such that $f(r)=m_{o}$, $m_{o}$ being the smallest integer less than $m$ which has an inverse; $m=m_{o}+k$.

Thus, $f(m_{o})=f(f(r))=f(r)+1=m_{o}+1$.

Again, $f(m_{o}+1)=f(f(m_{o}))=f(m_{o})+1=m_{o}+2$

Inductively proceeding we find that $f(m_{o}+(k-1))=m_{o}+k=m$, a contradiction.
Hence the claim has been proved.

Now, set $f(n)=s$ to get $f(s)=s+1$ for all naturals $s$, proving the uniqueness.
\end{solution}



\begin{solution}[by \href{https://artofproblemsolving.com/community/user/376213}{Wizard_32}]
	\begin{tcolorbox}$f:Z  -> Z $

$f(f(n)) - f(n) - 1 = 0$

\begin{bolded}\begin{italicized}Find f\end{italicized}\end{bolded}\end{tcolorbox}
I assumed that the equation was true for all integers $n$.
\end{solution}



\begin{solution}[by \href{https://artofproblemsolving.com/community/user/345905}{TLP.39}]
	\begin{tcolorbox}We claim that the only solution is $f(n)=n+1$ for all integers $n$. It is easy to see that this works. We now show that this is the only possible solution.

\begin{bolded}Claim:\end{bolded} $f$ is surjective
\begin{italicized}Proof:\end{italicized} Let there exist an integer $m$ such that the equation $f(n)=m$ has no integer value of $n$ satisfying it.
Let $r$ be an integer such that $f(r)=m_{o}$, $m_{o}$ being the smallest integer less than $m$ which has an inverse; $m=m_{o}+k$.

Thus, $f(m_{o})=f(f(r))=f(r)+1=m_{o}+1$.

Again, $f(m_{o}+1)=f(f(m_{o}))=f(m_{o})+1=m_{o}+2$

Inductively proceeding we find that $f(m_{o}+(k-1))=m_{o}+k=m$, a contradiction.
Hence the claim has been proved.

Now, set $f(n)=s$ to get $f(s)=s+1$ for all naturals $s$, proving the uniqueness.\end{tcolorbox}

How can you sure that $m_o$ exists? What if $1,2,...,m\not\in f(\mathbb{R})$?
\end{solution}



\begin{solution}[by \href{https://artofproblemsolving.com/community/user/271171}{xdiegolazarox}]
	If we choose  an fixed integer $a$. The function $f(a+t)=a+t+1$ $\forall t  \geq 0$ and $f(a-t)=c_{a-t}$ $\forall t  \geq 1$, where $c_{a-t}$ could be any integer greater than $a-1$, we get that this function works.
\end{solution}



\begin{solution}[by \href{https://artofproblemsolving.com/community/user/29428}{pco}]
	\begin{tcolorbox}$f:Z  -> Z $

$f(f(n)) - f(n) - 1 = 0$

\begin{bolded}\begin{italicized}Find f\end{italicized}\end{bolded}\end{tcolorbox}
Let $A=f(\mathbb Z)$
Note that $f(f(n))=f(n)+1$ implies that $a\in A$ $\implies$ $a+1\in A$

And so :
1) Either $A=\mathbb Z$
Then $f(x)$ is surjective and $f(f(n))=f(n)+1$ implies 
$\boxed{\text{S1 : }f(x)=x+1\quad\forall x\in\mathbb Z}$, which indeed is a solution.

2) Either $A=[a,+\infty)\cap \mathbb Z$ for some $a\in\mathbb Z$
Then $f(f(n))=f(n)+1$ implies $f(n)=n+1$ $\forall n\ge a$
And $f(n)\in A$ $\forall n<a$
And these mandatoty conditions trivially are sufficient and so
$\boxed{\text{S2 : }f(n)=n+1\quad\forall n\ge a\text{ and }f(n)=\max(g(n),a)\quad\forall n<a}$
Which indeed is a solution, whatever are $a\in\mathbb Z$ and $g(x)$ from $\mathbb Z\to\mathbb Z$

\begin{italicized}Note that this is exactly the solution proposed by xdiegolazarox in the previous post.\end{italicized}


\end{solution}
*******************************************************************************
-------------------------------------------------------------------------------

\begin{problem}[Posted by \href{https://artofproblemsolving.com/community/user/247598}{MATH1945}]
	Prove that there aren't any functions $f : R^{+} ->R^{+}$ such that

$$f(x+y) \geq f(x)+yf(x)^2$$

for all positive reals $x,y$ 
	\flushright \href{https://artofproblemsolving.com/community/c6h1579607}{(Link to AoPS)}
\end{problem}



\begin{solution}[by \href{https://artofproblemsolving.com/community/user/29428}{pco}]
	\begin{tcolorbox}Prove that there aren't any functions $f : R^{+} ->R^{+}$ such that

$$f(x+y) \geq f(x)+yf(x)^2$$

for all positive reals $x,y$\end{tcolorbox}
$f(x+y)\ge f(x)+yf(x)>f(x)$ and so $f(x)$ is increasing.

Setting $y=\frac 1{f(x)}$, we get $f(x+\frac 1{f(x)})\ge 2f(x)$

Suppose now we have $f(x+\frac a{f(x)})\ge bf(x)$ $\forall x$ and for some $a,b>0$

We get $f(x+\frac 1{f(x)}+\frac a{f(x+\frac 1{f(x)})})\ge bf(x+\frac 1{f(x)})\ge 2bf(x)$

We also have $\frac a{f(x+\frac 1{f(x)})}\le \frac a{2f(x)}$ and so, since increasing :

$f(x+\frac 1{f(x)}+\frac a{2f(x)})\ge 2bf(x)$

And so $f(x+(\frac a2+1)\frac 1{f(x)})\ge 2bf(x)$

And so $(a,b)\to(1+\frac a2,2b)$

And so, starting from $(1,2)$, we get $(1+\frac 12,4)$, $(1+\frac 12+\frac 14,2^3)$, ... and so 

$f(x+\sum_{k=0}^n2^{-k}\frac 1{f(x)})\ge 2^{n+1}f(x)$

And since $x+\sum_{k=0}^n2^{-k}\frac 1{f(x)}<x+\frac 2{f(x)}$ and $f(x)$ increasing, we get 

$f(x+\frac 2{f(x)})\ge 2^{n+1} f(x)$ $\forall x>0$ and $\forall n\in\mathbb N$
Which is trivially impossible since $f(x)>0$.

And so $\boxed{\text{No such function}}$
Q.E.D.


\end{solution}



\begin{solution}[by \href{https://artofproblemsolving.com/community/user/376213}{Wizard_32}]
	\begin{tcolorbox}
$f(x+y)\ge f(x)+yf(x)>f(x)$ and so $f(x)$ is increasing.
\end{tcolorbox}
There's a small typo; it should be $f(x+y)\ge f(x)+yf(x)^{2}$

\end{solution}



\begin{solution}[by \href{https://artofproblemsolving.com/community/user/376213}{Wizard_32}]
	\begin{tcolorbox}
Suppose now we have $f(x+\frac a{f(x)})\ge bf(x)$ $\forall x$ and for some $a,b>0$
\end{tcolorbox}
What was the motivation for making this move!

\end{solution}
*******************************************************************************
-------------------------------------------------------------------------------

\begin{problem}[Posted by \href{https://artofproblemsolving.com/community/user/343113}{gmail.com}]
	Find all functions $f:\mathbb R\to\mathbb R$ such that for all real numbers $x$ and $y$,

1) $f$ is continuous
2) $$f(1-x) = f(x)-2x+1,             
 \forall x,y \in \mathbb R$$
	\flushright \href{https://artofproblemsolving.com/community/c6h1579631}{(Link to AoPS)}
\end{problem}



\begin{solution}[by \href{https://artofproblemsolving.com/community/user/346843}{jrc1729}]
	$$\boxed{f(x)=x}\_\_\_\_\&\_\_\_\_\boxed{f(x)=x^2}$$ :) :)
\end{solution}



\begin{solution}[by \href{https://artofproblemsolving.com/community/user/279587}{RagvaloD}]
	$f(x)=x+g(x(1-x))$
\end{solution}



\begin{solution}[by \href{https://artofproblemsolving.com/community/user/29428}{pco}]
	\begin{tcolorbox}Find all functions $f:\mathbb R\to\mathbb R$ such that for all real numbers $x$ and $y$,

1) $f$ is continuous
2) $$f(1-x) = f(x)-2x+1,             
 \forall x,y \in \mathbb R$$\end{tcolorbox}
$f(x)=x+h(|x-\frac 12|)$, whatever if continuous $h(x)$.

\end{solution}
*******************************************************************************
-------------------------------------------------------------------------------

\begin{problem}[Posted by \href{https://artofproblemsolving.com/community/user/377794}{Mathuzb}]
	Prove that if $f:\mathbb{R}\to \mathbb{R}$ is an additive function and bounded on some interval, then $f(x)=cx$ for some constant $c$
	\flushright \href{https://artofproblemsolving.com/community/c6h1583638}{(Link to AoPS)}
\end{problem}



\begin{solution}[by \href{https://artofproblemsolving.com/community/user/391068}{TuZo}]
	I know this problem, I have the proof in Hungarian I post here, I hope than you can understand the mathematics prove.
The 4-th problem we prove, it $f$ is bounded, than the problem reduce to the first problem.
The 1-th problem prove that $f$ is continouse in only one value, than is continouse in every value.
\end{solution}



\begin{solution}[by \href{https://artofproblemsolving.com/community/user/243405}{ThE-dArK-lOrD}]
	Note that if $f:\mathbb{R} \rightarrow \mathbb{R}$ is an additive function and non-linear then the graph of $f$ is dense in $\mathbb{R}^2$, done.
\end{solution}



\begin{solution}[by \href{https://artofproblemsolving.com/community/user/377794}{Mathuzb}]
	\begin{tcolorbox}Note that if $f:\mathbb{R} \rightarrow \mathbb{R}$ is an additive function and non-linear then the graph of $f$ is dense in $\mathbb{R}^2$, done.\end{tcolorbox}

Do you have a full solution???
\end{solution}



\begin{solution}[by \href{https://artofproblemsolving.com/community/user/29428}{pco}]
	\begin{tcolorbox}[quote=ThE-dArK-lOrD]Note that if $f:\mathbb{R} \rightarrow \mathbb{R}$ is an additive function and non-linear then the graph of $f$ is dense in $\mathbb{R}^2$, done.\end{tcolorbox}

Do you have a full solution???\end{tcolorbox}

http://artofproblemsolving.com\/community\/c6h1143607p7341161

http://artofproblemsolving.com\/community\/c6h487789p8926128




\end{solution}



\begin{solution}[by \href{https://artofproblemsolving.com/community/user/342947}{ghoshsaga}]
	here is your full solution  :)  
\end{solution}



\begin{solution}[by \href{https://artofproblemsolving.com/community/user/377794}{Mathuzb}]
	thank everyone
\end{solution}
*******************************************************************************
-------------------------------------------------------------------------------

\begin{problem}[Posted by \href{https://artofproblemsolving.com/community/user/60772}{bazili}]
	Find functions  f, g with g surjective such that f(x+3g(y))-g(5f(x)+y)=0, for all x, y real numbers.
	\flushright \href{https://artofproblemsolving.com/community/c6h1584184}{(Link to AoPS)}
\end{problem}



\begin{solution}[by \href{https://artofproblemsolving.com/community/user/29428}{pco}]
	\begin{tcolorbox}Find functions  f, g with g surjective such that f(x+3g(y))-g(5f(x)+y)=0, for all x, y real numbers.\end{tcolorbox}
Let $P(x,y)$ be the assertion $f(x+3g(y))-g(5f(x)+y)=0$
Let $a=f(0)$
Since $g(x)$ surjective, let $u$ such that $g(u)=0$
Let $v=3g(u-5a)$

$P(0,u-5a)$ $\implies$ $f(v)=0$
$P(v,x)$ $\implies$ $f(3g(x)+v)=g(x)$ and so, since $g(x)$ surjective :
$f(x)=\frac{x-v}3$ (which may be also written $f(x)=\frac x3+a$)

Plugging this back in $P(3(x-a),5y)$, we get $g(5x+5y)=x+g(5y)$
Swapping there $x$ and $y$, we get $g(5y+5x)=y+g(5x)$
Subtracting, this becomes $g(5x)-x=g(5y)-y$ and so $g(x)=\frac x5+b$

And so $\boxed{f(x)=\frac x3+a\text{  and  }g(x)=\frac x5+b\quad\forall x}$ which indeed is a solution, whatever are $a,b\in\mathbb R$
\end{solution}



\begin{solution}[by \href{https://artofproblemsolving.com/community/user/60772}{bazili}]
	Thank you.
\end{solution}



\begin{solution}[by \href{https://artofproblemsolving.com/community/user/60772}{bazili}]
	Can you explain the step: since $g(x)$ surjective :
$f(x)=\frac{x-v}3$ 
Thank you.
\end{solution}



\begin{solution}[by \href{https://artofproblemsolving.com/community/user/29428}{pco}]
	\begin{tcolorbox}Can you explain the step: since $g(x)$ surjective :
$f(x)=\frac{x-v}3$ 
Thank you.\end{tcolorbox}

$f(3g(x)+v)=g(x)=\frac{(3g(x)+v)-v}3$ and since $3g(x)+v$ can take any real value we want ...
\end{solution}
*******************************************************************************
-------------------------------------------------------------------------------

\begin{problem}[Posted by \href{https://artofproblemsolving.com/community/user/393430}{georgeado17}]
	Find all functions \(f:\mathbb{R}\rightarrow\mathbb{R}\) such that 
\(f(x+y)f(x-y)=(f(x)+f(y))^2-4x^2f(y)\)
\(x,y\in\mathbb{R}\)
	\flushright \href{https://artofproblemsolving.com/community/c6h1585525}{(Link to AoPS)}
\end{problem}



\begin{solution}[by \href{https://artofproblemsolving.com/community/user/29428}{pco}]
	\begin{tcolorbox}Find all functions \(f:\mathbb{R}\rightarrow\mathbb{R}\) such that 
\(f(x+y)f(x-y)=(f(x)+f(y))^2-4x^2f(y)\)
\(x,y\in\mathbb{R}\)\end{tcolorbox}
Let $P(x,y)$ be the assertion $f(x+y)f(x-y)=(f(x)+f(y))^2-4x^2f(y)$

$P(0,0)$ $\implies$ $f(0)=0$
$P(x,x)$ $\implies$ $f(x)(f(x)-x^2)=0$ and so :
$\forall x$, either $f(x)=0$, either $f(x)=x^2$

Supposte now that $\exists u,v\ne 0$ such that $f(u)=u^2$ and $f(v)=0$
$P(u,v)$ $\implies$ $f(u+v)f(u-v)=u^4$

But LHS is either $0$, either $(u^2-v^2)^2$, and so we get 
Either $u^4=0$, impossible
Either $u^4=(u^2-v^2)^2$, and so $v^2=2u^2$
But in this last case, we had also $f(u-v)=(u-v)^2$ and so we can use the same method using $u-v,v$ instead of $u,v$ (and since $u\ne v$, xhich gives us $v^2=2(u-v)^2$
And clearly $u\ne v$,$u\ne 0$, $v\ne 0$, $v^2=2u^2$ and $v^2=2(u-v)^2$ is impossible.

So no such $u,v$ and only two possibilities :

Either $\boxed{\text{S1 : }f(x)=0\quad\forall x}$ which indeed is a solution

Either $\boxed{\text{S2 : }f(x)=x^2\quad\forall x}$ which indeed is a solution



\end{solution}



\begin{solution}[by \href{https://artofproblemsolving.com/community/user/277552}{WizardMath}]
	$P(x,0)$ gives $0=2f(x)f(0) + f(0)^2 - 4x^2f(0)$. If $f(0) \ne 0$, then $f(x) = 2x^2 - f(0)\/2$, which gives a contradiction. So $f(0) = 0$. 
$P(0,y)$ gives $f(y)f(-y) = f(y)^2$, so for $f(x) \ne 0$, we have $f(x) = f(-x)$. 
$P(x,x)$ gives that for every $x \in \mathbb{R}$, we have $f(x) = 0$ or $x^2$. This with the previous sentence means that $f(x) = f(-x)$ always. 
Suppose for non-zero $a,b$ we have $f(a)=0, f(b)=b^2$. Then we have by $P(b,a)$ that $(a^2-b^2)^2 = b^4$ or $0=b^4$, the second of which is a contradiction and the first one leads to $a^2=2b^2$. Now $P(a,b)$ gives a contradiction as the LHS is $\ge 0$ and the RHS is $-b^4$.

So we have $f(x) \equiv 0 \quad \forall x \in \mathbb{R}$, or $f(x) \equiv x \quad \forall x \in \mathbb{R}$.
\end{solution}
*******************************************************************************
-------------------------------------------------------------------------------

\begin{problem}[Posted by \href{https://artofproblemsolving.com/community/user/344350}{soryn}]
	
Let $F$ the set of all functions$f$ :$\mathbb{N}$$\rightarrow\mathbb{N}$
which satisfy : $f\left(f\left(x\right)\right)-2f\left(x\right)+x=0$,
for all $x\in\mathbb{N}$.

Determine the set $A=\left\{ f\left(2018\right)\mid f\in F\right\}
	\flushright \href{https://artofproblemsolving.com/community/c6h1585807}{(Link to AoPS)}
\end{problem}



\begin{solution}[by \href{https://artofproblemsolving.com/community/user/29428}{pco}]
	\begin{tcolorbox}Let $F$ the set of all functions$f$ :$\mathbb{N}$$\rightarrow\mathbb{N}$
which satisfy : $f\left(f\left(x\right)\right)-2f\left(x\right)+x=0$,
for all $x\in\mathbb{N}$.

Determine the set $A=\left\{ f\left(2018\right)\mid f\in F\right\}$\end{tcolorbox}
$f(x)>a_0x$ $\forall x$ with $a_0=0\in[0,2)$
So $f(f(x))=2f(x)-x>a_0f(x)$ which is $f(x)> a_1x$ $\forall x$ with $a_1=\frac 1{2-a_0}\in[0,2)$
And since the sequence $a_0=0$ and $a_{n+1}=\frac 1{2-a_n}$ is always in $[0,2)$ and has limit $1$, we get :
$f(x)\ge x$ $\forall x\in\mathbb N$

And since $f(x)=x+a$ is a solution, whatever is $a\in\mathbb Z_{\ge 0}$ :

$\boxed{A=\mathbb N\cap[2018,+\infty)}$


\end{solution}



\begin{solution}[by \href{https://artofproblemsolving.com/community/user/391068}{TuZo}]
	How you conclude this?
\begin{tcolorbox}
...
$f(x)\ge x$ $\forall x\in\mathbb N$... RESULT? $f(x)=x+a$ 
\end{tcolorbox}
\end{solution}



\begin{solution}[by \href{https://artofproblemsolving.com/community/user/243405}{ThE-dArK-lOrD}]
	You shouldn't quote someone and change context in the quote. Obviously, @\begin{bolded}pco\end{bolded} didn't say $f(x)\geq x \forall x\in\mathbb N$ implies $f(x)=x+a$.
\end{solution}



\begin{solution}[by \href{https://artofproblemsolving.com/community/user/391068}{TuZo}]
	Ok, I understan you, but hi are concluded that $f(x)=x+a$, and I don't see how? We must find the $f(x)$, or not?
\end{solution}



\begin{solution}[by \href{https://artofproblemsolving.com/community/user/29428}{pco}]
	\begin{tcolorbox}Ok, I understan you, but hi are concluded that $f(x)=x+a$, and I don't see how? We must find the $f(x)$, or not?\end{tcolorbox}
No we dont have to  find all $f(x)$
We must just find $A$

And I proved that $f(x)\ge x$ for all solutions so no integer less than $2018$ may be in $A$

Then I proved that $f(x)=x+a$ is a solution (just check) and so that any element greater or equat to $2018$ is in $A$
(maybe some other solutions may exist but it does not matter)

Hence the conclusion

\end{solution}



\begin{solution}[by \href{https://artofproblemsolving.com/community/user/391068}{TuZo}]
	Ok, thank you!
\end{solution}
*******************************************************************************
-------------------------------------------------------------------------------

\begin{problem}[Posted by \href{https://artofproblemsolving.com/community/user/334862}{ZzIsaacNewtonZz}]
	Find all function $ f:\mathbb{R}\rightarrow \mathbb{R} $ satisfy: \[ f(x^4+f(y))=f(x)^4+y\], for all $x,y\in\mathbb R.$
	\flushright \href{https://artofproblemsolving.com/community/c6h1586095}{(Link to AoPS)}
\end{problem}



\begin{solution}[by \href{https://artofproblemsolving.com/community/user/29428}{pco}]
	\begin{tcolorbox}Find all function $ f:\mathbb{R}\rightarrow \mathbb{R} $ satisfy: \[ f(x^4+f(y))=f(x)^4+y\], for all $x,y\in\mathbb R.$\end{tcolorbox}
Let $P(x,y)$ be the assertion $f(x^4+f(y))=f(x)^4+y$
We immediately get from equation that $f(x)$ is bijective.

Comparing $P(x,y)$ with $P(-x,y)$ and using injectivity, we get $f(-x)=-f(x)$ $\forall x\ne 0$
From there we get that $f(x)\ne 0$ $\forall x\ne 0$ (else $f(-x)=f(x)=0$ and no injectivity)
From there we get that $f(0)=0$ (else no surjectivity) and so $f(x)$ is odd.

$P(x,0)$ $\implies$ $f(x^4)=f(x)^4$ (and so $f(x)$ lowerbounded over $\mathbb R^+$)
$P(0,y)$ $\implies$ $f(f(y))=y$

And so $P(x,f(y))$ is $f(x^4+y)=f(x^4)+f(y)$
And so $f(x+y)=f(x)+f(y)$ $\forall x\ge 0$, $\forall y$
And so $f(x+y)=f(x)+f(y)$ $\forall x,y$ (since odd)

And so, additive and lowerbounded, $f(x)$ is linear.
Plugging $f(x)=cx$ in original equation, we get $c=1$

And so $\boxed{f(x)=x\quad\forall x}$


\end{solution}



\begin{solution}[by \href{https://artofproblemsolving.com/community/user/277552}{WizardMath}]
	We see that $f$ is surjective by varying $y$ and injective by the LHS.

Suppose $f(t)=0, f(0) = u$. 

$P(0,0) \implies f(u) = u^4$. $P(0,t) \implies u=u^4+t$. $P(t,0) \implies u=t-t^4$. So $u=t=0$. 
Now $P(0,y) \implies f(f(y))=y,$ so $P(x,f(y))$ gives $f(x^4+y) = f(x^4)+f(y)$. 
$P(x,y), P(-x,y)$ gives us $f$ is odd. So we have $f(x+y)=f(x)+f(y) \forall \ x \in \mathbb{R}$. 
$P(x,0)$ gives us the $f(\text{positive}) = \text{positive}$, so $f$ is bounded below on an interval of $\mathbb{R}$. Thus we get $f$ is linear, which gives $f(x)=x$.

\end{solution}
*******************************************************************************
-------------------------------------------------------------------------------

\begin{problem}[Posted by \href{https://artofproblemsolving.com/community/user/335975}{Taha1381}]
	Find all injective  function $f : \mathbb{N} \rightarrow \mathbb{N}$, that satisfy :

$f(f(n)) \le \frac{n + f(n)}{2}$, $\forall n \in \mathbb{N}$
	\flushright \href{https://artofproblemsolving.com/community/c6h1586753}{(Link to AoPS)}
\end{problem}



\begin{solution}[by \href{https://artofproblemsolving.com/community/user/246291}{dipdas}]
	\begin{tcolorbox}Find all function $f : \mathbb{N} \rightarrow \mathbb{N}$, that satisfy :

$f(f(n)) \le \frac{n + f(n)}{2}$, $\forall n \in \mathbb{N}$\end{tcolorbox}

$$f(x) = x$$ is a solution
\end{solution}



\begin{solution}[by \href{https://artofproblemsolving.com/community/user/29428}{pco}]
	\begin{tcolorbox}Find all function $f : \mathbb{N} \rightarrow \mathbb{N}$, that satisfy :

$f(f(n)) \le \frac{n + f(n)}{2}$, $\forall n \in \mathbb{N}$\end{tcolorbox}
There are infinitely many such functions.
For example, any function such that $f(n)\le n$ $\forall n$ fits.


\end{solution}



\begin{solution}[by \href{https://artofproblemsolving.com/community/user/335975}{Taha1381}]
	\begin{tcolorbox}[quote=Taha1381]Find all function $f : \mathbb{N} \rightarrow \mathbb{N}$, that satisfy :

$f(f(n)) \le \frac{n + f(n)}{2}$, $\forall n \in \mathbb{N}$\end{tcolorbox}
There are infinitely many such functions.
For example, any function such that $f(n)\le n$ $\forall n$ fits.\end{tcolorbox}

Sorry I missed the injective condition.In fact our teacher gave this as an exercise I serched the forum and found this:https:\/\/artofproblemsolving.com\/community\/c6h504548p2834125

It had'nt the injective condition and there wasn't any solutions so I reposted it but I didn't notice the injective condition was less.Anyway sorry for that.
\end{solution}



\begin{solution}[by \href{https://artofproblemsolving.com/community/user/77832}{abhinavzandubalm}]
	First lets see that something has to map to $1$.
Take the series
\[ f(1), f(f(1)), f(f(f(1))), \cdots \]
This is strictly decreasing (as none of these are $1$) and will eventually reach either $0$ or $1$.
Which is a contradiction.
Hence we have that $f(1)=1$.

Similarly we can prove that the series 
\[ f(n), f(f(n)), f(f(f(n))), \cdots\]
is strictly decreasing, assuming that we have proved $f(i)=i~\forall~i<n$.
So we will reach some point where we go below $n$.
Which will again give us a contradiction for the injection condition.

Hence the only solution is $f(n)=n~\forall n\in\mathbb{N}$

\end{solution}



\begin{solution}[by \href{https://artofproblemsolving.com/community/user/335975}{Taha1381}]
	\begin{tcolorbox}First lets see that something has to map to $1$.
Take the series
\[ f(1), f(f(1)), f(f(f(1))), \cdots \]
This is strictly decreasing (as none of these are $1$) and will eventually reach either $0$ or $1$.
Which is a contradiction.
Hence we have that $f(1)=1$.

Similarly we can prove that the series 
\[ f(n), f(f(n)), f(f(f(n))), \cdots\]
is strictly decreasing, assuming that we have proved $f(i)=i~\forall~i<n$.
So we will reach some point where we go below $n$.
Which will again give us a contradiction for the injection condition.

Hence the only solution is $f(n)=n~\forall n\in\mathbb{N}$\end{tcolorbox}

You proved sth has to map to 1.Not 1 has to map to 1.
\end{solution}



\begin{solution}[by \href{https://artofproblemsolving.com/community/user/335975}{Taha1381}]
	Nobody?????????
\end{solution}



\begin{solution}[by \href{https://artofproblemsolving.com/community/user/213821}{Kirilbangachev}]
	I have seen it. It is from a Romanian competition in the years 2000-2007. I will try to find it.

But the main idea was to make it $k$ times, $f^k(n)\le a_kn+b_kf(n)$ where  $a_k$ and $b_k$ are positive and have sum $1$. Now it is easy to show that for each $n,$ if we apply $f$ $t$ times (this might be a different number for different $n$'s) it gets back to $n$, since $a_kn + b_kf(n)\le n+f(n)$. $f^t(n)=n$.
Now it is sufficient to consider the maximum of $\{f^0(n),f^1(n),\ldots f^t(n)\}$ to obtain that $f(n)=n$.
\end{solution}



\begin{solution}[by \href{https://artofproblemsolving.com/community/user/77832}{abhinavzandubalm}]
	\begin{tcolorbox}[quote=abhinavzandubalm]First lets see that something has to map to $1$.
Take the series
\[ f(1), f(f(1)), f(f(f(1))), \cdots \]
This is strictly decreasing (as none of these are $1$) and will eventually reach either $0$ or $1$.
Which is a contradiction.
Hence we have that $f(1)=1$.

Similarly we can prove that the series 
\[ f(n), f(f(n)), f(f(f(n))), \cdots\]
is strictly decreasing, assuming that we have proved $f(i)=i~\forall~i<n$.
So we will reach some point where we go below $n$.
Which will again give us a contradiction for the injection condition.

Hence the only solution is $f(n)=n~\forall n\in\mathbb{N}$\end{tcolorbox}

You proved sth has to map to 1.Not 1 has to map to 1.\end{tcolorbox}

No, I proved that $f(1) = 1$, because you only decrease if you start with $n=1$.
\end{solution}



\begin{solution}[by \href{https://artofproblemsolving.com/community/user/281272}{san1201}]
	https:\/\/artofproblemsolving.com\/community\/c6h5449p17637
\end{solution}
*******************************************************************************
-------------------------------------------------------------------------------

\begin{problem}[Posted by \href{https://artofproblemsolving.com/community/user/347734}{LittleKesha}]
	Knowing that $f$ is a function from the real numbers to the real numbers such that
$f(x)=f(x+1)+1$.
Show that $f(x+n)=f(x)-n$ for all natural numbers $n$.
	\flushright \href{https://artofproblemsolving.com/community/c6h1587332}{(Link to AoPS)}
\end{problem}



\begin{solution}[by \href{https://artofproblemsolving.com/community/user/333350}{AnArtist}]
	:rotfl: induction on $n$.
\end{solution}



\begin{solution}[by \href{https://artofproblemsolving.com/community/user/347734}{LittleKesha}]
	Thanks AnArtist, but how exactly? :(
\end{solution}



\begin{solution}[by \href{https://artofproblemsolving.com/community/user/29428}{pco}]
	\begin{tcolorbox}Thanks AnArtist, but how exactly? :(\end{tcolorbox}

$f(x+1)=f(x)-1$
$f(x+2)=f(x+1)-1=(f(x)-1)-1=f(x)-2$
$f(x+3)=f(x+2)-1=(f(x)-2)-1=f(x)-3$
 Are you able to proceed alone from here ?
\end{solution}



\begin{solution}[by \href{https://artofproblemsolving.com/community/user/391068}{TuZo}]
	\begin{tcolorbox}Thanks AnArtist, but how exactly? :(\end{tcolorbox}

$f(x+1)=f(x)-1,f(x+2)=f(x+1)-1=f(x)-2,$...so supposte that $f(x+n)=f(x)-n$, we prove that $f(x+n+1)=f(x+n)-n-1,$ this is true, because $f(x+n+1)=f(x+n)-1=(x-n)-1=x-n-1$ q.e.d. 

\end{solution}



\begin{solution}[by \href{https://artofproblemsolving.com/community/user/347734}{LittleKesha}]
	mmm I'm not sure but I think that TuZo's solution contein some mistakes...
Someone can show me how exactly use induction to solve the problem?? :(
Thanks :)
\end{solution}



\begin{solution}[by \href{https://artofproblemsolving.com/community/user/346843}{jrc1729}]
	\begin{bolded}Proof by Mathematical Induction :\end{bolded}\end{underlined}

The base case is given, i.e. $f(x+1)=f(x)-1$
Suppose, $f(x+n)=f(x)-n$ is true.
So,
$f(x+n+1)$
$=f\big((x+n)+1\big)$
$=f(x+n)-1$
$=f(x)-n-1$
$=f(x)-(n+1)$

So, by the principle of \begin{bolded}\begin{italicized}Mathematical Induction\end{italicized}\end{bolded}, we conclude that $$\boxed{f(x+n)=f(x)-n}$$
Proved, I think. :)
\end{solution}



\begin{solution}[by \href{https://artofproblemsolving.com/community/user/347734}{LittleKesha}]
	Now is clear, thanks jrc1729 very gentle! :)
\end{solution}



\begin{solution}[by \href{https://artofproblemsolving.com/community/user/391068}{TuZo}]
	\begin{tcolorbox}Now is clear, thanks jrc1729 very gentle! :)\end{tcolorbox}

\begin{tcolorbox}mmm I'm not sure but I think that TuZo's solution contein some mistakes...
Someone can show me how exactly use induction to solve the problem?? :(
Thanks :)\end{tcolorbox}

Dear LittleKesha! My solution is identical with jrc1729's solution, so shall contain some mistake? :o
So difficult the induction?  :huh:
\end{solution}



\begin{solution}[by \href{https://artofproblemsolving.com/community/user/330150}{L3435}]
	\begin{tcolorbox}[quote=LittleKesha]Thanks AnArtist, but how exactly? :(\end{tcolorbox}

$f(x+1)=f(x)-1,f(x+2)=f(x+1)-1=f(x)-2,$...so supposte that $f(x+n)=f(x)-n$, we prove that $f(x+n+1)=f(x+n)-n-1,$ this is true, because $f(x+n+1)=f(x+n)-1=(x-n)-1=x-n-1$ q.e.d.\end{tcolorbox}

\begin{tcolorbox}[quote=LittleKesha]Now is clear, thanks jrc1729 very gentle! :)\end{tcolorbox}

\begin{tcolorbox}mmm I'm not sure but I think that TuZo's solution contein some mistakes...
Someone can show me how exactly use induction to solve the problem?? :(
Thanks :)\end{tcolorbox}

Dear LittleKesha! My solution is identical with jrc1729's solution, so shall contain some mistake? :o
So difficult the induction?  :huh:\end{tcolorbox}

You wrote $x$ instead of $f(x)$ at the end, everything else is right.
\end{solution}
*******************************************************************************
-------------------------------------------------------------------------------

\begin{problem}[Posted by \href{https://artofproblemsolving.com/community/user/355882}{TomMarvoloRiddle}]
	Find all continuous and bijective function $f:[0,1]\rightarrow [0,1]$ such that :-

$f(2x-f(x))=x$ for all $x\in [0,1]$
	\flushright \href{https://artofproblemsolving.com/community/c6h1587442}{(Link to AoPS)}
\end{problem}



\begin{solution}[by \href{https://artofproblemsolving.com/community/user/345008}{Kayak}]
	You probably forgot to add the condition that there exists a fixed point of $f$, which makes this exercise 6.36 of BJV (see page 212), which was in turn taken from Functional Equations and How to Solve Them, by Christopher G Small.

EDIT: Sorry, in the problem I mentioned the function is from $\mathbb{R}$ to itself. But you can proceed in this problem similarly as to that problem.
\end{solution}



\begin{solution}[by \href{https://artofproblemsolving.com/community/user/355882}{TomMarvoloRiddle}]
	\begin{tcolorbox}You probably forgot to add the condition that there exists a fixed point of $f$, which makes this exercise 6.36 of BJV (see page 212), which was in turn taken from Functional Equations and How to Solve Them, by Christopher G Small.\end{tcolorbox}

Currently, I don't have that book.

But why to mention about fixed point?

It is well-known that, continuous function has a unique fixed point!
\end{solution}



\begin{solution}[by \href{https://artofproblemsolving.com/community/user/333350}{AnArtist}]
	..............
\end{solution}



\begin{solution}[by \href{https://artofproblemsolving.com/community/user/355882}{TomMarvoloRiddle}]
	I wanted to say that ,

If $f:[a,b]\rightarrow [a,b]$ is continuous then it has a fixed point.It is well-known.
\end{solution}



\begin{solution}[by \href{https://artofproblemsolving.com/community/user/333350}{AnArtist}]
	@below seems right.
\end{solution}



\begin{solution}[by \href{https://artofproblemsolving.com/community/user/345008}{Kayak}]
	\begin{tcolorbox} If $f:[a,b]\rightarrow [a,b]$ is continuous then it has an unique fixed point.It is well-known.\end{tcolorbox}

This is clearly false: take $f(x) = x$ in your favorite interval. What's the unique fixed point ?


\end{solution}



\begin{solution}[by \href{https://artofproblemsolving.com/community/user/29428}{pco}]
	\begin{tcolorbox}I wanted to say that ,

If $f:[a,b]\rightarrow [a,b]$ is continuous then it has an unique fixed point.It is well-known.\end{tcolorbox}

Mmmmmmmm ?
What about $(a,b,f)=(0,100\pi,x+\sin x)$ which is continuous and has $101$ fixed points over $[0,100\pi]$ ?

\end{solution}



\begin{solution}[by \href{https://artofproblemsolving.com/community/user/355882}{TomMarvoloRiddle}]
	\begin{tcolorbox}[quote=TomMarvoloRiddle]I wanted to say that ,

If $f:[a,b]\rightarrow [a,b]$ is continuous then it has an unique fixed point.It is well-known.\end{tcolorbox}

Mmmmmmmm ?
What about $(a,b,f)=(0,100\pi,x+\sin x)$ which is continuous and has $101$ fixed points over $[0,100\pi]$ ?\end{tcolorbox}

Ok.Fine.So there should not be the term 'unique'.
\end{solution}



\begin{solution}[by \href{https://artofproblemsolving.com/community/user/355882}{TomMarvoloRiddle}]
	\begin{tcolorbox}I wanted to say that ,

If $f:[a,b]\rightarrow [a,b]$ is continuous then it has a fixed point.It is well-known.\end{tcolorbox}


\end{solution}



\begin{solution}[by \href{https://artofproblemsolving.com/community/user/29428}{pco}]
	\begin{tcolorbox}Ok.Fine.So there should not be the term 'unique'.\end{tcolorbox}
Sure, and it is a pity that with the word "unique", il was supposed \/ claimed "well known"



\end{solution}



\begin{solution}[by \href{https://artofproblemsolving.com/community/user/355882}{TomMarvoloRiddle}]
	Is anyone interested to solve the problem? :)
\end{solution}



\begin{solution}[by \href{https://artofproblemsolving.com/community/user/29428}{pco}]
	\begin{tcolorbox}Find all continuous and bijective function $f:[0,1]\rightarrow [0,1]$ such that :-

$f(2x-f(x))=x$ for all $x\in [0,1]$\end{tcolorbox}
No need for continuity or bijectivity.

We trivially get $0\le 2x-f(x)\le 1$ and so $2x-1\le f(x)\le 2x$

If $a_nx-b_n\le f(x)\le a_nx$, then $a_n(2x-f(x))-b_n\le x\le a_n(2x-f(x))$

And so $(2-\frac 1{a_n})x-\frac {b_n}{a_n}\le f(x)\le (2-\frac 1{a_n})x$

And since the sequence starting with $(2,1)$ and such that $(a_n,b_n)\to (2-\frac 1{a_n},\frac {b_n}{a_n})$ converges towards $(1,0)$, we immediately get 
$\boxed{f(x)=x\quad\forall x\in[0,1]}$ which indeed fits.
\end{solution}
*******************************************************************************
-------------------------------------------------------------------------------

\begin{problem}[Posted by \href{https://artofproblemsolving.com/community/user/309179}{whiwho}]
	\begin{tcolorbox}
\begin{bolded}Problem\end{bolded}
Let $f$ be a a function such that $f(f(x))=x+1$. Find all the functional equations that satisfy each condition.

\begin{bolded}a)\end{bolded} Let $f : \mathbb{N} \to \mathbb{N} $
\begin{bolded}b)\end{bolded} Let $f : \mathbb{R} \to \mathbb{R}$ be continuous.
\begin{bolded}c)\end{bolded} Let $f : \mathbb{R} \to \mathbb{R}$ be continuous and derivable.
\begin{bolded}d)\end{bolded} Let $f : \mathbb{R} \to \mathbb{R}$ be continuous and such that it has second derivative.

\end{tcolorbox}

[hide=My try with the option b]

Let $P(x) = f(f(x))=x+1$
Is clear that the function $f$ is biyective. Since the function is biyective and continuous, we conclude that it is monotic and increasing. By induction we can prove that $f(x+n) = f(x)+n$ for $n\in \mathbb{Z}$. [hide = Claim 1]
$P(f(x)) = f(x)+1 = f(x+1) \implies  f(x)+n = f(x+n) $
$P(f(x-1)) = f(x-1) = f(x)-1 \implies f(x)-n = f(x-n)$
[\/hide]

 We claim that $f(x) \in (x,x+1)$. [hide=Claim 2]
If $f(x)\leq x$ for some $x$,  then either $f(x)<x \forall x$ or there exists a point where $f(x)=x$, a fixed point, since the function is continous and Bolzano's Theorem.

\begin{bolded}Case $f(x) < x \forall x$: \end{bolded} $f(f(x)) < f(x) < x < x+1$, which contradicts $f(f(x))=x+1$
\begin{bolded}Case of fixed point:\end{bolded} Let a be a number such that  $f(a)=a$, then $f(f(a))=a \not = a+1$, contradicting the initial statement again.

From which we conclude the lowerbound on $x$

If $f(x)\geq x+1$, then: let $a$ be a number such that $f(a)\geq x+1$, then $f(f(a)) \geq f(a+1) > a+1$, contradicting the statement. And so, implying the upperbound.
[\/hide]

From this two, I conclude that all the solution need to be in a form of $f(x) = x+g(x)$, where $g : \mathbb{R} \to (0,1)$ is a continuous periodic function. Which fullfills $g(x)+g(x+g(x))=1$. [hide=Claim 3]
$f(f(x)) = f(x)+g(f(x)) = x+g(x)+g(x+g(x)) = x+1 \implies g(x)+g(x+g(x))=1.$
[\/hide]
Altough I cannot solve this last functional equation.
[\/hide]

Any hints or solutions would be really aprecciated, and please, hide your solutions. :D 
	\flushright \href{https://artofproblemsolving.com/community/c6h1587995}{(Link to AoPS)}
\end{problem}



\begin{solution}[by \href{https://artofproblemsolving.com/community/user/29428}{pco}]
	\begin{tcolorbox}
\begin{bolded}Problem\end{bolded}
Let $f$ be a a function such that $f(f(x))=x+1$. Find all the functional equations that satisfy each condition.

\begin{bolded}a)\end{bolded} Let $f : \mathbb{N} \to \mathbb{N} $
\begin{bolded}b)\end{bolded} Let $f : \mathbb{R} \to \mathbb{R}$ be continuous.
\begin{bolded}c)\end{bolded} Let $f : \mathbb{R} \to \mathbb{R}$ be continuous and derivable.
\begin{bolded}d)\end{bolded} Let $f : \mathbb{R} \to \mathbb{R}$ be continuous and such that it has second derivative.

\end{tcolorbox}

Certainly not a real olympiad problem :

a) trivially no solution (classical classical : $f(f(x))=x+n$ has solutions only if $n$ is even)

b)c)d) infinitely many solutions in each case which can be built piece per piece without any difficulty.
\end{solution}



\begin{solution}[by \href{https://artofproblemsolving.com/community/user/309179}{whiwho}]
	\begin{tcolorbox}[quote=whiwho]
\begin{bolded}Problem\end{bolded}
Let $f$ be a a function such that $f(f(x))=x+1$. Find all the functional equations that satisfy each condition.

\begin{bolded}a)\end{bolded} Let $f : \mathbb{N} \to \mathbb{N} $
\begin{bolded}b)\end{bolded} Let $f : \mathbb{R} \to \mathbb{R}$ be continuous.
\begin{bolded}c)\end{bolded} Let $f : \mathbb{R} \to \mathbb{R}$ be continuous and derivable.
\begin{bolded}d)\end{bolded} Let $f : \mathbb{R} \to \mathbb{R}$ be continuous and such that it has second derivative.

\end{tcolorbox}

Certainly not a real olympiad problem :

a) trivially no solution (classical classical : $f(f(x))=x+n$ has solutions only if $n$ is even)

b)c)d) infinitely many solutions in each case which can be built piece per piece without any difficulty.\end{tcolorbox}

Could you please give me any hints or solution for part b, c and d? :) 
\end{solution}



\begin{solution}[by \href{https://artofproblemsolving.com/community/user/29428}{pco}]
	\begin{tcolorbox}Could you please give me any hints or solution for part b, c and d? :)\end{tcolorbox}
Let $a\in(0,1)$

Define $f(x)$ over $[0,a]$ as any increasing continuous bijection $g(x)$ from $[0,a]\to[a,1]$

Define $f(x)$ over $[a,1]$ as $f(x)=g^{-1}(x)+1$

Define $f(x)$ over any interval $[n,n+1]$ (where $n\in\mathbb Z\setminus\{0\}$) as $f(x)=f(x-n)+n$

This define infinitely many continuous increasing functions (maybe not all of them) solution of $f(f(x))=x+1$, as required in b)

If you want to add constraints like "derivable" or "twice derivable", just add some contraints on $g(x)$ :
For c) : $g(x)$ is "derivable" over $[0,a]$ and $g'(0)g'(a)=1$
For d) : add similar constraints for $g''(x)$
\end{solution}



\begin{solution}[by \href{https://artofproblemsolving.com/community/user/391067}{rikabcd123456}]
	option a is clear. no function exists

\end{solution}



\begin{solution}[by \href{https://artofproblemsolving.com/community/user/289554}{programjames1}]
	[hide=Proof for A]
Let $f(0) = x$ for some $x$. If $x = 0$, then $f(f(0)) = 0 \ne 1$, which is impossible, so $f(0) = x\ne 0$. Then, $f(x) = 1$, so $f(1) = x + 1$, $f(x + 1) = 2$ and so forth. This gives us a function
$$f(n)=\begin{cases}n+1-x&\text{if }n\ge x\\n+x&\text{if }n\ge0\end{cases}$$
for some $x$. Then for $n \ge x$, we have $n+x=n+1-x\implies x=\frac12$.
So, $f(n)=n+\frac12$
[\/hide]
\end{solution}



\begin{solution}[by \href{https://artofproblemsolving.com/community/user/29428}{pco}]
	\begin{tcolorbox}This define infinitely many continuous increasing functions (maybe not all of them) solution of $f(f(x))=x+1$, as required in b)\end{tcolorbox}
In fact, it is easy to show that continuity implies $x<f(x)<x+1$ $\forall x$ and so these are indeed all the solutions.



\end{solution}
*******************************************************************************
-------------------------------------------------------------------------------

\begin{problem}[Posted by \href{https://artofproblemsolving.com/community/user/368751}{Dattier}]
	Describe all function continuous $f$ in the real with : $\exists a>0, \forall x \in \mathbb R, f(x+1)=a \times f(x)$ 
	\flushright \href{https://artofproblemsolving.com/community/c6h1588012}{(Link to AoPS)}
\end{problem}



\begin{solution}[by \href{https://artofproblemsolving.com/community/user/391068}{TuZo}]
	Remark: 
For this kind of function, it there are true the following: 1) if $0<a<1$, $\underset{x\rightarrow\infty}{lim}f(x)=0,$ 2) if $a>1$, then $\underset{x\rightarrow-\infty}{lim}f(x)=0,$ 3) if $a=1$, the functionis periodic, the main period is $1$

\end{solution}



\begin{solution}[by \href{https://artofproblemsolving.com/community/user/29428}{pco}]
	\begin{tcolorbox}Describe all function continuous $f$ in the real with : $\exists a>0, \forall x \in \mathbb R, f(x+1)=a \times f(x)$\end{tcolorbox}

General solution : $\boxed{f(x)=a^{\lfloor x\rfloor}(g(\{x\})+(ag(0)-g(1))\{x\})\quad\forall x}$ which indeed is a solution, whatever is $g(x)$ continuous from $[0,1]\to\mathbb R$.

\end{solution}



\begin{solution}[by \href{https://artofproblemsolving.com/community/user/368751}{Dattier}]
	@pco : bravo
\end{solution}



\begin{solution}[by \href{https://artofproblemsolving.com/community/user/29428}{pco}]
	\begin{tcolorbox}@pco : you have forgot a condition on $g$\end{tcolorbox}
Dont hesitate to prove your claim. So everyone could understand you.
\end{solution}



\begin{solution}[by \href{https://artofproblemsolving.com/community/user/368751}{Dattier}]
	2\/ what about the case when $f\in C^{\infty}$, is the only solution of the form $c\times \exp(d\times x)$ ?
\end{solution}



\begin{solution}[by \href{https://artofproblemsolving.com/community/user/29428}{pco}]
	\begin{tcolorbox}2\/ what about the case when $f\in C^{\infty}$, is the only solution of the form $c\times \exp(d\times x)$ ?\end{tcolorbox}

Certainly not.
For example $f(x)=a^{\lfloor x\rfloor}e^{-\frac 1{\{x\}^2(1-\{x\})^2}}$  (with extension to $0$ over $\mathbb Z$ is such a $C^{\infty}$ solution.
\end{solution}



\begin{solution}[by \href{https://artofproblemsolving.com/community/user/368751}{Dattier}]
	@pco : Bravo
\end{solution}



\begin{solution}[by \href{https://artofproblemsolving.com/community/user/368751}{Dattier}]
	3\/What about the case when $f$ is analytic ?

PS : here I don't know the answer
\end{solution}



\begin{solution}[by \href{https://artofproblemsolving.com/community/user/391068}{TuZo}]
	\begin{tcolorbox}[quote=Dattier]Describe all function continuous $f$ in the real with : $\exists a>0, \forall x \in \mathbb R, f(x+1)=a \times f(x)$\end{tcolorbox}

General solution : $\boxed{f(x)=a^{\lfloor x\rfloor}(g(\{x\})+(ag(0)-g(1))\{x\})\quad\forall x}$ which indeed is a solution, whatever is $g(x)$ continuous from $[0,1]\to\mathbb R$.\end{tcolorbox}

Please tell me your solution! Thank you!
\end{solution}



\begin{solution}[by \href{https://artofproblemsolving.com/community/user/29428}{pco}]
	\begin{tcolorbox}3\/What about the case when $f$ is analytic ?

PS : here I don't know the answer\end{tcolorbox}
I dont know the answer too :)


\end{solution}



\begin{solution}[by \href{https://artofproblemsolving.com/community/user/29428}{pco}]
	\begin{tcolorbox}[quote=pco][quote=Dattier]Describe all function continuous $f$ in the real with : $\exists a>0, \forall x \in \mathbb R, f(x+1)=a \times f(x)$\end{tcolorbox}

General solution : $\boxed{f(x)=a^{\lfloor x\rfloor}(g(\{x\})+(ag(0)-g(1))\{x\})\quad\forall x}$ which indeed is a solution, whatever is $g(x)$ continuous from $[0,1]\to\mathbb R$.\end{tcolorbox}

Please tell me your solution! Thank you!\end{tcolorbox}
I gave it !

In order to find it, just build the solution piece per piece starting with its definition $h(x)$ over $[0,1)$ 
And add the continuity constraint :
1) $h(x)$ continuous over $[0,1]$
2) $h(1)=ah(0)$ 
Which is $h(x)=g(x)+(ag(0)-g(1))x$ with $g(x)$ continuous without other constraints.



\end{solution}



\begin{solution}[by \href{https://artofproblemsolving.com/community/user/29428}{pco}]
	\begin{tcolorbox}3\/What about the case when $f$ is analytic ?\end{tcolorbox}
Yes, there exist other analytic solutions than $a e^{bx}$ :

$f(x)=g(x)e^{bx}$ where $g(x)$ is any analytic function with period $1$ (infinitely many such functions do exist)



\end{solution}



\begin{solution}[by \href{https://artofproblemsolving.com/community/user/368751}{Dattier}]
	@pco : Excellent
\end{solution}
*******************************************************************************
-------------------------------------------------------------------------------

\begin{problem}[Posted by \href{https://artofproblemsolving.com/community/user/366114}{AlexTheMathemagician}]
	Let be $m,n$ $\in$ $\mathbb{Z}$. Determine the functions $f,g$: $\mathbb{R}$ $\longrightarrow$ $\mathbb{R}$  with g surjective, such that $f(x+mg(y))=g(nf(x)+y)$, $\forall$ $x,y$ $\in$ $\mathbb{R}$
	\flushright \href{https://artofproblemsolving.com/community/c6h1588021}{(Link to AoPS)}
\end{problem}



\begin{solution}[by \href{https://artofproblemsolving.com/community/user/29428}{pco}]
	\begin{tcolorbox}Let be $m,n$ $\in$ $\mathbb{z}$. Determine the functions $f,g$: $\mathbb{R}$ $\longrightarrow$ $\mathbb{R}$  with g surjective, such that $f(x+mg(y))=g(nf(x)+y)$, $\forall$ $x,y$ $\in$ $\mathbb{R}$\end{tcolorbox}

http://artofproblemsolving.com\/community\/c6h1584184p9792768 (just four days ago)
\end{solution}
*******************************************************************************
-------------------------------------------------------------------------------

\begin{problem}[Posted by \href{https://artofproblemsolving.com/community/user/349572}{automatryr}]
	$f: R-> R$
Show that if
$f(xy+y+x)=f(xy)+f(x)+f(y)$
Then 
$f(x+y)=f(x)+f(y)$
	\flushright \href{https://artofproblemsolving.com/community/c6h1588141}{(Link to AoPS)}
\end{problem}



\begin{solution}[by \href{https://artofproblemsolving.com/community/user/29428}{pco}]
	\begin{tcolorbox}$f: R-> R$
Show that if
$f(xy+y+x)=f(xy)+f(x)+f(y)$
Then 
$f(x+y)=f(x)+f(y)$\end{tcolorbox}
Posted (and solved) many times (2007, 2010, 2013, 2014, ...).

Dont hesitate to use the search function (see [url=http://www.artofproblemsolving.com\/community\/c6h1330604p7173058] here [\/url]).
Set (for example copy\/paste) in the "search term" field the exact following string : 

+"f(xy+x+y)" +"f(xy)+f(x)+f(y)" +"f(x+y)"

You'll get in the \begin{bolded}four first results\end{underlined}\end{bolded} (excluded your own post and this post itself) all the help you are requesting for.

[hide=(Some excuses)][size=70]I'm sorry not providing you the direct link to this result but I encountered users who never tried the search function, thinking quite easier to have other users make the search for them. So now I prefer to point to the search function and to give the appropriate search term (I checked that it indeed will give you the expected result) instead of the link itself [\/size]([url=https:\/\/en.wiktionary.org\/wiki\/give_a_man_a_fish_and_you_feed_him_for_a_day;_teach_a_man_to_fish_and_you_feed_him_for_a_lifetime]wink[\/url])[\/hide]



\end{solution}



\begin{solution}[by \href{https://artofproblemsolving.com/community/user/391068}{TuZo}]
	Using this search for "f(xy+x+y)" +"f(xy)+f(x)+f(y)" +"f(x+y)" I got  hundred of other type of equation  :mad: , but I don't see where are these post exactly! Please tell me only one link, where can I found the solution! Thank you! (you can see only one exact result, this post)
\end{solution}



\begin{solution}[by \href{https://artofproblemsolving.com/community/user/29428}{pco}]
	\begin{tcolorbox}Using this search for "f(xy+x+y)" +"f(xy)+f(x)+f(y)" +"f(x+y)" I got  hundred of other type of equation  :mad: , but I don't see where are these post exactly! Please tell me only one link, where can I found the solution! Thank you! (you can see only one exact result, this post)\end{tcolorbox}
Using this search, I found exactly four.
Sorry, we are certainly not on the same site.
So any link will be useless for you.



\end{solution}



\begin{solution}[by \href{https://artofproblemsolving.com/community/user/391068}{TuZo}]
	You got four after how many other post? 50, 100,..
I used this :
\end{solution}



\begin{solution}[by \href{https://artofproblemsolving.com/community/user/277552}{WizardMath}]
	Strangely, I got it [url=https:\/\/artofproblemsolving.com\/community\/q1h379566p2099316]here[\/url] at the fourth post after this post. 
\end{solution}



\begin{solution}[by \href{https://artofproblemsolving.com/community/user/391068}{TuZo}]
	\begin{bolded}Thank you WizardMath, for me it is inexplicable and irritating, but nobody can't help me with practical advice, only theoretical advice! :-(\end{bolded}
\end{solution}



\begin{solution}[by \href{https://artofproblemsolving.com/community/user/346289}{etrnlMathS}]
	Instead of doing Advanced Community Search ,I did simple community search using the string...and found it on the fourth post..I think advance community search searches all possible results using the key terms
\end{solution}



\begin{solution}[by \href{https://artofproblemsolving.com/community/user/391068}{TuZo}]
	Thank you, here is the community search, and I got more wrong hit, that the previouse time!  :mad: 
\begin{bolded}I feel like I'm on a different planet.\end{bolded} :o
\end{solution}



\begin{solution}[by \href{https://artofproblemsolving.com/community/user/212321}{SHARKYKESA}]
	\begin{tcolorbox}Thank you, here is the community search, and I got more wrong hit, that the previouse time!  :mad: 
\begin{bolded}I feel like I'm on a different planet.\end{bolded} :o\end{tcolorbox}

Because you didn't have the '+' symbol in front of "f(xy)+f(x)+f(y)" when you searched. :|
\end{solution}
*******************************************************************************
-------------------------------------------------------------------------------

\begin{problem}[Posted by \href{https://artofproblemsolving.com/community/user/335975}{Taha1381}]
	Find all functions $f: \mathbb{R}^+ \cup \{0\} \to \mathbb{R}^+ \cup \{0\}$ that for any non-negative $x,y,z$ satisfy:

$f(\frac{x+f(x)}{2}+y+f(2z))=2x-f(x)+f(f(y))+2f(z)$

"Proposed by mohammad jafari"

Hint:Show that $f$ is injective.
	\flushright \href{https://artofproblemsolving.com/community/c6h1588727}{(Link to AoPS)}
\end{problem}



\begin{solution}[by \href{https://artofproblemsolving.com/community/user/335975}{Taha1381}]
	Nobody??
\end{solution}



\begin{solution}[by \href{https://artofproblemsolving.com/community/user/277552}{WizardMath}]
	It was already posted [url=http://artofproblemsolving.com\/community\/c6h444651p2561119]here[\/url], so maybe that's why no one's solving it.
\end{solution}



\begin{solution}[by \href{https://artofproblemsolving.com/community/user/335975}{Taha1381}]
	\begin{tcolorbox}It was already posted [url=http://artofproblemsolving.com\/community\/c6h444651p2561119]here[\/url], so maybe that's why no one's solving it.\end{tcolorbox}

Yes but that isn't the solution Mr.jafari mentioned(Our teacher).After proving the injectivity the problem becomes almost trivial.But proving the injectivity is hard.
\end{solution}



\begin{solution}[by \href{https://artofproblemsolving.com/community/user/335975}{Taha1381}]
	\begin{tcolorbox}[quote=WizardMath]It was already posted [url=http://artofproblemsolving.com\/community\/c6h444651p2561119]here[\/url], so maybe that's why no one's solving it.\end{tcolorbox}

Yes but that isn't the solution Mr.jafari mentioned(Our teacher).After proving the injectivity the problem becomes almost trivial.But proving the injectivity is hard.\end{tcolorbox}

Can anyone prove the injectivity?
\end{solution}



\begin{solution}[by \href{https://artofproblemsolving.com/community/user/29428}{pco}]
	\begin{tcolorbox}Can anyone prove the injectivity?\end{tcolorbox}
Use link given in post #3 and you have a method to prove that $f(x)\equiv x$
Add a line to this proof claiming that $f(x)\equiv x$ is injective.
And you get the proof you are looking for.


\end{solution}



\begin{solution}[by \href{https://artofproblemsolving.com/community/user/335975}{Taha1381}]
	I got a solution.Let $P(x,y,z)$ denote the assertion $f(\frac{x+f(x)}{2}+y+f(2z))=2x-f(x)+f(f(y))+2f(z)$.Let $f(y_1)=f(y_2)$

$P(x,y_1,z),P(x,y_2,z) \Rightarrow f(\frac{x+f(x)}{2}+y_1+f(2z))=f(\frac{x+f(x)}{2}+y_2+f(2z))$

Let $A=f(\frac{x+f(x)}{2}+y_1+f(2z))$ and $B=f(\frac{x+f(x)}{2}+y_2+f(2z))$.

$P(A,\frac{y_2}{2},z),P(B, \frac{y_1}{2},z) \Rightarrow A=B \Rightarrow y_1=y_2$

So $f$ is injective.

$P(0,0,0) \Rightarrow f(\frac{3}{2} a)=a+f(a)$

$P(\frac{3}{2}a,0,0) \Rightarrow f(\frac{9}{4}a +\frac{f(a)}{2})=4a$

$P(a,0,0) \Rightarrow f(\frac{3a+f(a)}{2})=4a$

So injectivity implies:

$a=0 \Rightarrow f(0)=0$

$P(0,y,0) \Rightarrow f(y)=f(f(y)) \Rightarrow f(y)=y  \forall y  \in \mathbb{R}^{+} \cup \{0 \}$
\end{solution}
*******************************************************************************
-------------------------------------------------------------------------------

\begin{problem}[Posted by \href{https://artofproblemsolving.com/community/user/68025}{Pirkuliyev Rovsen}]
	Determine all functions $f: \mathbb{Z}\to\mathbb{Z}$ for which $f^3(x)+f^3(y)+f^3(z)=3f^2(x)f(y)+3f^2(y)f(z)+3f^2(z)f(x)$ for all $x,y,z{\in}Z$ such that $x+y+z=0$

	\flushright \href{https://artofproblemsolving.com/community/c6h1589380}{(Link to AoPS)}
\end{problem}



\begin{solution}[by \href{https://artofproblemsolving.com/community/user/29428}{pco}]
	\begin{tcolorbox}Determine all functions $f: \mathbb{Z}\to\mathbb{Z}$ for which $f^3(x)+f^3(y)+f^3(z)=3f^2(x)f(y)+3f^2(y)f(z)+3f^2(z)f(x)$ for all $x,y,z{\in}Z$ such that $x+y+z=0$\end{tcolorbox}
Let $P(x,y,z)$ be the assertion $f(x)^3+fy)^3+f(z)^3$ $=3f(x)^2f(y)+3f(y)^2f(z)+3f(z)^2f(x)$

$P(0,0,0)$ $\implies$ $f(0)=0$

$P(x,-x,0)$ $\implies$ $f(x)^3+f(-x)^3=3f(x)^2f(-x)$
$P(-x,x,0)$ $\implies$ $f(x)^3+f(-x)^3=3f(-x)^2f(x)$
Subtracting, we get $f(x)f(-x)(f(x)-f(-x))=0$
And so :
Either $f(x)=0$
Either $f(-x)=0$ and so $f(x)^3+f(-x)^3=3f(x)^2f(-x)$ implies $f(x)=0$
Either $f(x)=f(-x)$ and so $f(x)^3+f(-x)^3=3f(x)^2f(-x)$ again implies $f(x)=0$

And so $\boxed{f(x)=0\quad\forall x\in\mathbb Z}$ which indeed is a solution.



\end{solution}
*******************************************************************************
-------------------------------------------------------------------------------

\begin{problem}[Posted by \href{https://artofproblemsolving.com/community/user/206573}{pandyhu2001}]
	If a real-valued function of one real variable $f: R \to R$ satisfies $f(f(x))=x+1$, show that for every integer $n$, $f(n)$ is not an integer.
	\flushright \href{https://artofproblemsolving.com/community/c6h1589730}{(Link to AoPS)}
\end{problem}



\begin{solution}[by \href{https://artofproblemsolving.com/community/user/371903}{little-fermat}]
	$f(f(f(x)))=f(x+1)=f(x)+1\implies f(x+n)=f(x)+n$
so if $f(n)$ is an integer for some $n\in\mathbb{Z}$ then $f(n)$ is an integer for all $n\in\mathbb{Z}$
if $f(0)=c$ then $f(n)=c+n$ HENCE :
$f(f(n))=f(c+n)=2c+n=n+1\implies c=\frac{1}{2}$
a contradiction
\end{solution}



\begin{solution}[by \href{https://artofproblemsolving.com/community/user/309179}{whiwho}]
	\begin{tcolorbox}If a real-valued function of one real variable $f: R \to R$ satisfies $f(f(x))=x+1$, show that for every integer $n$, $f(n)$ is not an integer.\end{tcolorbox}

Since $f(x) \in (x,x+1)$
\end{solution}



\begin{solution}[by \href{https://artofproblemsolving.com/community/user/29428}{pco}]
	\begin{tcolorbox}[quote=pandyhu2001]If a real-valued function of one real variable $f: R \to R$ satisfies $f(f(x))=x+1$, show that for every integer $n$, $f(n)$ is not an integer.\end{tcolorbox}

Since $f(x) \in (x,x+1)$\end{tcolorbox}
I think we can easily prove that $f(x)\in(x,x+1)$ using continuity of $f(x)$ when this property is given.
But here, without this property, I dont see how you can easily conclude that $f(x)\in(x,x+1)$

(the required property claimed in post #1 is still true without continuity : look at the correct proof given by little-fermat in post #2)


\end{solution}



\begin{solution}[by \href{https://artofproblemsolving.com/community/user/249633}{Gryphos}]
	I think there is an example where $f(x) \in (x,x+1)$ does not always hold:
Let $f(x) = x-1\/2$ if $\lfloor x \rfloor \in [0,1\/2)$ and $f(x) = x+3\/2$ else. Then for all $x$, $f(f(x)) = x-1\/2+3\/2 = x+1$.
\end{solution}
*******************************************************************************
-------------------------------------------------------------------------------

\begin{problem}[Posted by \href{https://artofproblemsolving.com/community/user/335975}{Taha1381}]
	Find all functions $f: \mathbb{R} \to \mathbb{R}$ that for any real $x,y,z$ satisfy:

$f(f(x)+f(y)+f(z))=f(f(x)-f(y))+f(2xy+f(z))+2f(xz-yz)$
	\flushright \href{https://artofproblemsolving.com/community/c6h1589758}{(Link to AoPS)}
\end{problem}



\begin{solution}[by \href{https://artofproblemsolving.com/community/user/29428}{pco}]
	\begin{tcolorbox}Find all functions $f: \mathbb{R} \to \mathbb{R}$ that for any real $x,y,z$ satisfy:

$f(f(x)+f(y)+f(z))=f(f(x)-f(y))+f(2xy+f(z))+2f(xz-yz)$\end{tcolorbox}
Posted (and solved) many many times (2010(2X),2011,2013,2015, 2017 (march)).

Dont hesitate to use the search function (see [url=http://www.artofproblemsolving.com\/community\/c6h1330604p7173058] here [\/url]).
Set (for example copy\/paste) in the "search term" field the exact following string : 

+"f(f(x)+f(y)+f(z))" +"f(2xy+f(z))"

You'll get in the \begin{bolded}ten first results\end{underlined}\end{bolded} (excluded your own post and this post itself) all the help you are requesting for.

[hide=(Some excuses)][size=70]I'm sorry not providing you the direct link to this result but I encountered users who never tried the search function, thinking quite easier to have other users make the search for them. So now I prefer to point to the search function and to give the appropriate search term (I checked that it indeed will give you the expected result) instead of the link itself [\/size]([url=https:\/\/en.wiktionary.org\/wiki\/give_a_man_a_fish_and_you_feed_him_for_a_day;_teach_a_man_to_fish_and_you_feed_him_for_a_lifetime]wink[\/url])[\/hide]\end{tcolorbox}


\end{solution}
*******************************************************************************
-------------------------------------------------------------------------------

\begin{problem}[Posted by \href{https://artofproblemsolving.com/community/user/216306}{khlongez}]
	Find all function $f:\mathbb{R}^+\rightarrow \mathbb{R}^+$ satisfying
 $f(xf(y))=yf(x)$ for all $x,y>0$.

I get that $f(1)=1$, $f(f(x))=x$ for all $x>0$ and $f$ is bijection.
Furthermore, $f(xy)=f(x)f(y)$ for all $x,y>0$. It follows that $f(x^p)=(f(x))^p$ for all rational number $p$.
It seems all solution to the problem are $f(x)=x$ and $f(x)=\frac{1}{x}$. How to prove it or are there other solutions?

	\flushright \href{https://artofproblemsolving.com/community/c6h1590240}{(Link to AoPS)}
\end{problem}



\begin{solution}[by \href{https://artofproblemsolving.com/community/user/29428}{pco}]
	\begin{tcolorbox}Find all function $f:\mathbb{R}^+\rightarrow \mathbb{R}^+$ satisfying
 $f(xf(y))=yf(x)$ for all $x,y>0$.\end{tcolorbox}
Let $P(x,y)$ be the assertion $f(xf(y))=yf(x)$

$P(1,x)$ $\implies$ $f(f(x))=f(1)x$ and so $f(x)$ is bijective.
$P(x,1)$ $\implies$ $f(f(1)x)=f(x)$ and so, since injective, $f(1)=1$ and $f(f(x))=x$
$P(x,f(y))$ $\implies$ $f(xy)=f(x)f(y)$ and $f(x)$ is multiplicative.

Setting there $g(x)=\ln f(e^x)$ from $\mathbb R\to\mathbb R$, we get 
$g(x+y)=g(x)+g(y)$ with $g(g(x))=x$ and $g(0)=0$

This is a quite classical function with general solution :
Let $A,B$ any two supplementary subvectorspaces of the $\mathbb Q$-vectorspace $\mathbb R$
Let $a(x)$ from $\mathbb R\to A$ and $b(x)$ from $\mathbb R\to B$ the two projections of a real $x$ in $(A,B)$
Then $g(x)=a(x)-b(x)$

And so $f(x)=e^{a(\ln x)-b(\ln x)}$ $\forall x>0$

Note that $(A,B)=(\mathbb R,\{0\})$ gives solution $f(x)=x$

And that $(A,B)=(\{0\},\mathbb R)$ gives solution $f(x)=\frac 1x$

But infinitely many other solutions may exist (assuming axiom of choice).
\end{solution}
*******************************************************************************
-------------------------------------------------------------------------------

\begin{problem}[Posted by \href{https://artofproblemsolving.com/community/user/337289}{Supercali}]
	Find all functions $f:\mathbb{R}^+ \rightarrow \mathbb{R}^+$ satisfying $$f(xy) \geq f(x+y)$$ for all $x,y>0$.
	\flushright \href{https://artofproblemsolving.com/community/c6h1590247}{(Link to AoPS)}
\end{problem}



\begin{solution}[by \href{https://artofproblemsolving.com/community/user/29428}{pco}]
	\begin{tcolorbox}Find all functions $f:\mathbb{R}^+ \rightarrow \mathbb{R}^+$ satisfying $$f(xy) \geq f(x+y)$$ for all $x,y>0$.\end{tcolorbox}
Here is a partial result which shows :
 - some constraints in 1),2), 3)
 - an infinite family of solutions in 4)
 
I think that the infinite family in 4) is likely the whole set of solutions (no other one) but till know I failed proving it. Hope somebody could finish this paart.

Let $c=f(5)$

1) $f(x)=c$ constant $\forall x\ge 5$
=======================
Let $x\ge 5$ and $y\in[x,x+1]$

$x\ge 5$ $\implies$ $x^2\ge 4(x+1)\ge 4y$ and so $\exists a,b>0$ such that :
$a+b=x$ and $ab=y$ 
And so $f(y)\ge f(x)=f(\sqrt x\sqrt x)$ $\ge f(\sqrt x+\sqrt x)f(2\sqrt x)$

$y\ge x\ge 5$ implies that $y^2\ge x^2 >8\sqrt x$ and so $\exists c,d>0$ such that :
$c+d=y$ and $cd=2\sqrt x$
And so $f(2\sqrt x)=f(cd)\ge f(c+d)=f(y)$

So $f(y)\ge f(x)\ge f(2\sqrt x)\ge f(y)$

And so $f(x)=f(y)$ $\forall x\ge 5$ and $y\in[x,x+1]$
It is immediate from there to show that $f(x)=f(y)$ $\forall x,y\ge 5$
Q.E.D.

2) $f(x)\ge c$ $\forall x>0$
================
This is aleardy true $\forall x\ge 5$
Let then $x<5$
We have $5^2\ge 4x$ and so $\exists a,b>0$ such that :
$a+b=5$ and $ab=x$
And so $f(x)=f(ab)\ge f(a+b)=c$
Q.E.D.

3) $f(x)=c$ $\forall x>4$
================
$f(\frac x2\frac x2)\ge f(\frac x2+\frac x2)$ and so $f(x)\le f(\frac{x^2}4)$ $\forall x>0$

Easy induction gives $f(x)\le f(\frac{x^{2^n}}{2^{2^{n+1}-2}})$ $\forall x>0$, $\forall n\in\mathbb Z_{\ge 0}$
If $x>4$, setting $n\to+\infty$ implies $\frac{x^{2^n}}{2^{2^{n+1}-2}}$ becomes greater than $5$

And so $f(x)\le c$ $\forall x>4$ and so, using 2) above, $f(x)=c$ $\forall x>4$
Q.E.D.

4) Any non increasing function constant over $(4,+\infty)$ fits
=============================================
Let $f(x)$ such that $f(x)\ge f(y)$ $\forall x>y$ and $f(x)=c$ $\forall x\ge 4$
Let $x,y>0$
If $xy>4$, then $x+y\ge 2\sqrt{xy}>4$ and so $f(xy)=f(x+y)=c$
If $xy\le 4$, then $x+y\ge 2\sqrt{xy}\ge xy$ and so $f(x+y)\le f(xy)$
Q.E.D.



\end{solution}
*******************************************************************************
-------------------------------------------------------------------------------

\begin{problem}[Posted by \href{https://artofproblemsolving.com/community/user/335975}{Taha1381}]
	Does there exist an additive perioudic function?
	\flushright \href{https://artofproblemsolving.com/community/c6h1590479}{(Link to AoPS)}
\end{problem}



\begin{solution}[by \href{https://artofproblemsolving.com/community/user/238386}{ythomashu}]
	yes because f(x)=0
\end{solution}



\begin{solution}[by \href{https://artofproblemsolving.com/community/user/29428}{pco}]
	\begin{tcolorbox}Does there exist an additive perioudic function?\end{tcolorbox}
The only linear additive periodic function is, as given in previous post, $f(x)\equiv 0$

The general form of all additive periodic functions is :
Let $A,B$, $B\ne\{0\}$ any two supplementary subvectorspaces of the $\mathbb Q$-vectorspace $\mathbb R$
Let $a(x)$ from $\mathbb R\to A$ and $b(x)$ from $\mathbb R\to B$ the two projections of a real $x$ in$(A,B)$

Then $f(x)=a(x)$
(and any element of $B$ is a period).

Note that choosing $(A,B)=(\{0\},\mathbb R)$, we get the linear solution $f(x)=0$ $\quad\forall x$
And infinitely many non trivial solutions assuming axiom of choice)


\end{solution}
*******************************************************************************
-------------------------------------------------------------------------------

\begin{problem}[Posted by \href{https://artofproblemsolving.com/community/user/344350}{soryn}]
	Give an example of $n$, $n\geq2$ of periodic functions

$f_{1},f_{2},....,f_{n}:\mathbb{R}$$\rightarrow\mathbb{R}$,each
with main period $T$,so 

that their sum has the mean period $\frac{T}{n}.$

	\flushright \href{https://artofproblemsolving.com/community/c6h1590489}{(Link to AoPS)}
\end{problem}



\begin{solution}[by \href{https://artofproblemsolving.com/community/user/29428}{pco}]
	\begin{tcolorbox}Give an example of $n$, $n\geq2$ of periodic functions

$f_{1},f_{2},....,f_{n}:\mathbb{R}$$\rightarrow\mathbb{R}$,each
with main period $T$,so 

that their sum has the mean period $\frac{T}{n}.$\end{tcolorbox}

$n=2$, $T=2\pi$, $f_1(x)=\max(\sin x,0)$ and $f_2(x)=\max(-\sin x,0)$

\end{solution}



\begin{solution}[by \href{https://artofproblemsolving.com/community/user/344350}{soryn}]
	For n>2??
\end{solution}



\begin{solution}[by \href{https://artofproblemsolving.com/community/user/354215}{pab}]
	$T=n,f_k(x)=x-\lfloor x\rfloor$ if $x\equiv k[n],$ $0$ else
The sum of the $f_k$ is $f:x\mapsto x-\lfloor x\rfloor,$ which has main period $1.$
\end{solution}



\begin{solution}[by \href{https://artofproblemsolving.com/community/user/29428}{pco}]
	\begin{tcolorbox}For n>2??\end{tcolorbox}
You are welcome.
Glad to have helped you

\end{solution}



\begin{solution}[by \href{https://artofproblemsolving.com/community/user/29428}{pco}]
	\begin{tcolorbox}... if $x\equiv k[n],$ ...\end{tcolorbox}
What is the meaning of "$x\equiv k[n],$" when $x=\pi$ for example ?


\end{solution}



\begin{solution}[by \href{https://artofproblemsolving.com/community/user/29428}{pco}]
	\begin{tcolorbox}Give an example of $n$, $n\geq2$ of periodic functions

$f_{1},f_{2},....,f_{n}:\mathbb{R}$$\rightarrow\mathbb{R}$,each
with main period $T$,so 

that their sum has the mean period $\frac{T}{n}.$\end{tcolorbox}

$f_k(x)=\max(0,(x-T\left\lfloor\frac xT\right\rfloor-(k-1)\frac Tn)(k\frac Tn-x+T\left\lfloor\frac xT\right\rfloor))$
\end{solution}



\begin{solution}[by \href{https://artofproblemsolving.com/community/user/344350}{soryn}]
	Oooooooooo, very interesting! Sure is correct?
\end{solution}



\begin{solution}[by \href{https://artofproblemsolving.com/community/user/29428}{pco}]
	\begin{tcolorbox}Oooooooooo, very interesting! Sure is correct?\end{tcolorbox}
Sure

\end{solution}



\begin{solution}[by \href{https://artofproblemsolving.com/community/user/29428}{pco}]
	\begin{tcolorbox}How to prove that the sum of the above functions has the mean period T\/n?\end{tcolorbox}

Just check
\end{solution}



\begin{solution}[by \href{https://artofproblemsolving.com/community/user/344350}{soryn}]
	For each function, clearly,wrn managed.But for the sum? Why do not your tale just the right expression, and took max(0,.....)? Please, explain me! Thanks!
\end{solution}



\begin{solution}[by \href{https://artofproblemsolving.com/community/user/29428}{pco}]
	Trivially, with $f(x+T)=f(x)$ $\forall x$ :
$g(x)=\sum_{k=1}^nf(x+\frac {kT}n)$ is such that $g(x+\frac Tn)=g(x)$ $\forall x$
Just apply

\end{solution}



\begin{solution}[by \href{https://artofproblemsolving.com/community/user/344350}{soryn}]
	We brackets, in the expression of f, îs *, or missing a comma? ( f=max(0,...., .....) pt f= max(0,....))
\end{solution}



\begin{solution}[by \href{https://artofproblemsolving.com/community/user/29428}{pco}]
	\begin{tcolorbox}We brackets, in the expression of f, îs *, or missing a comma? ( f=max(0,...., .....) pt f= max(0,....))\end{tcolorbox}

I am sorry but I dont see any mismatched parenthesis nor $\lfloor...\rfloor$

And these are just little parts of parabola above x-axis, nothing else.
I choosed this in order to have all $f_k$ continuous.

But the real key is trivial and is the post #14 above. All the rest is just "fun".

\end{solution}



\begin{solution}[by \href{https://artofproblemsolving.com/community/user/29428}{pco}]
	In order to be quite quite clear : With $T=n=5$ and my example (given just for fun), see uploaded pdf.

Clearer now ? :


\end{solution}



\begin{solution}[by \href{https://artofproblemsolving.com/community/user/344350}{soryn}]
	Thank you!
\end{solution}
*******************************************************************************
-------------------------------------------------------------------------------

\begin{problem}[Posted by \href{https://artofproblemsolving.com/community/user/377375}{Probowldan}]
	Find all $f: \mathbb{R}^+ \to \mathbb{R}^+$ such that
\[f(f(x+y)) - f(y) - f(x-y) = y\]
	\flushright \href{https://artofproblemsolving.com/community/c6h1590678}{(Link to AoPS)}
\end{problem}



\begin{solution}[by \href{https://artofproblemsolving.com/community/user/212321}{SHARKYKESA}]
	\begin{tcolorbox}Find all $f: \mathbb{R}^+ \to \mathbb{R}^+$ such that
\[f(f(x+y)) - f(y) - f(x-y) = y\]\end{tcolorbox}

You're missing the condition $x>y$.
\end{solution}



\begin{solution}[by \href{https://artofproblemsolving.com/community/user/29428}{pco}]
	\begin{tcolorbox}Find all $f: \mathbb{R}^+ \to \mathbb{R}^+$ such that
\[f(f(x+y)) - f(y) - f(x-y) = y\]\end{tcolorbox}
Assuming, as SHARKYKESA noticed, that domain of functional equation is in fact $\forall x>y>0$ :

Let $P(x,y)$ be the assertion $f(f(x+y))-f(y)-f(x-y)=y$
Let $c=1+f(1)-f(2)$
Let $g(x)=f(x+1)-f(1)$

$P(y+1,1)$ $\implies$ (a) :$f(f(y+2))-f(1)-f(y)=1$
$P(\frac y2+2,\frac y2)$ $\implies$ (b) : $f(f(y+2))-f(\frac y2)-f(2)=\frac y2$
$P(x+\frac y2,\frac y2)$ $\implies$ (c) :$f(f(x+y))-f(\frac y2)-f(x)=\frac y2$
(a)-(b)+(c) : new assertion $Q(x,y)$ : $f(f(x+y))=f(x)+f(y)+c$

$Q(x+y,1)$ $\implies$ $f(f(x+y+1))=f(x+y)+f(1)+c$
$Q(x,y+1)$ $\implies$ $f(f(x+y+1))=f(x)+f(y+1)+c$
Subtracting, we get new assertion $R(x,y)$ : $f(x+y)=f(x)+g(y)$

$R(y,1)$ $\implies$ (a) : $f(y+1)=f(y)+g(1)$
$R(1,y)$ $\implies$ (b) : $f(y+1)=f(1)+g(y)$
$R(x,y)$ $\implies$ (c) : $f(x+y)=f(x)+g(y)$
(a)-(b)+(c)  : $f(x+y)=f(x)+f(y)+d$ where $d=g(1)-f(1)$

So $f(x)+d$ is additive and lowerbounded, so linear.
Plugging $f(x)=ux+v$ in original equation, we get $u=1$ and $v\ge 0$

And so the solutions $\boxed{f(x)=x+a\quad\forall x>0}$ which indeed are solutions, whatever is $a\ge 0$




\end{solution}
*******************************************************************************
-------------------------------------------------------------------------------

\begin{problem}[Posted by \href{https://artofproblemsolving.com/community/user/348607}{ahmedAbd}]
	Find all continuous functions $f:[0, +\infty) \to [0, +\infty)$ satisfying 
$$\sqrt{f(a)}+\sqrt{f(b)}=\sqrt{f(a+b)}\;\; \forall a,b$$
	\flushright \href{https://artofproblemsolving.com/community/c6h1591050}{(Link to AoPS)}
\end{problem}



\begin{solution}[by \href{https://artofproblemsolving.com/community/user/289554}{programjames1}]
	Let $P$ be this assertion.
$$P(0,0)\to 2\sqrt{f(0)}=\sqrt{f(0)}\to \sqrt{f(0)}=0$$
$$P(a,-a)\to \sqrt{f(a)}+\sqrt{f(-a)}=\sqrt{f(0)}=0\implies f(a)=0\ \forall\ a$$
\end{solution}



\begin{solution}[by \href{https://artofproblemsolving.com/community/user/348607}{ahmedAbd}]
	Sorry fixed.
\end{solution}



\begin{solution}[by \href{https://artofproblemsolving.com/community/user/330150}{L3435}]
	I don't think the condition that $f$ is continuous is needed:

Let $g(a)=\sqrt{f(a)}$. The equation becomes $g(a)+g(b)=g(a+b)$, so $g$ is additive. Since $g$ is defined on nonnegative reals $g(a)=c\cdot a$ for every $a$ and some constant $c$. So $f(a)=da^2$ for some $d\geq 0$. It's trivial to check that these functions solve the equation.
\end{solution}



\begin{solution}[by \href{https://artofproblemsolving.com/community/user/354682}{Alex27}]
	\begin{tcolorbox}I don't think the condition that $f$ is continuous is needed:

Let $g(a)=\sqrt{f(a)}$. The equation becomes $g(a)+g(b)=g(a+b)$, so $g$ is additive. Since $g$ is defined on nonnegative reals $g(a)=c\cdot a$ for every $a$ and some constant $c$. So $f(a)=da^2$ for some $d\geq 0$. It's trivial to check that these functions solve the equation.\end{tcolorbox}
The condition that $f$ is countinuous is needed because otherwise $g$ is not necessarily continuous and you cannot conclude that $g(a)=ca$ from $g(a)+g(b)=g(a+b)$

\end{solution}



\begin{solution}[by \href{https://artofproblemsolving.com/community/user/344076}{fastlikearabbit}]
	\begin{tcolorbox}[quote=L3435]I don't think the condition that $f$ is continuous is needed:

Let $g(a)=\sqrt{f(a)}$. The equation becomes $g(a)+g(b)=g(a+b)$, so $g$ is additive. Since $g$ is defined on nonnegative reals $g(a)=c\cdot a$ for every $a$ and some constant $c$. So $f(a)=da^2$ for some $d\geq 0$. It's trivial to check that these functions solve the equation.\end{tcolorbox}
The condition that $f$ is countinuous is needed because otherwise $g$ is not necessarily continuous and you cannot conclude that $g(a)=ca$ from $g(a)+g(b)=g(a+b)$\end{tcolorbox}

Wrong! To obtain that $g(a)=ca$ it’s enough to have that g is additive and $g\geq0$.
\end{solution}



\begin{solution}[by \href{https://artofproblemsolving.com/community/user/391068}{TuZo}]
	\begin{tcolorbox}
Wrong! To obtain that $g(a)=ca$ it’s enough to have that g is additive and $g\geq0$.\end{tcolorbox}

\begin{bolded}No, it is not wrong!\end{bolded}
We can got from the additivity that $g(a)=ca$, in the following situations:
1) g is continuouse everyvhere
2) g is continouse in inly one point
3) g is monotonous 
4) g is non negative on an interval


\end{solution}



\begin{solution}[by \href{https://artofproblemsolving.com/community/user/345905}{TLP.39}]
	\begin{tcolorbox}[quote=fastlikearabbit]
Wrong! To obtain that $g(a)=ca$ it’s enough to have that g is additive and $g\geq0$.\end{tcolorbox}

\begin{bolded}No, it is not wrong!\end{bolded}
We can got from the additivity that $g(a)=ca$, in the following situations:
1) g is continuouse everyvhere
2) g is continouse in inly one point
3) g is monotonous 
4) g is non negative on an interval\end{tcolorbox}He has the condition 4).
\end{solution}



\begin{solution}[by \href{https://artofproblemsolving.com/community/user/391068}{TuZo}]
	And the 1) and 2) too! ;) 
\end{solution}



\begin{solution}[by \href{https://artofproblemsolving.com/community/user/29428}{pco}]
	\begin{tcolorbox}We can got from the additivity that $g(a)=ca$, in the following situations:
1) g is continuouse everyvhere
2) g is continouse in inly one point
3) g is monotonous 
4) g is non negative on an interval\end{tcolorbox}
In fact (out of this specific problem), the weaker sufficient condition to conclude linearity is likely "graph of $g(x)$ is not dense in $\mathbb R^2$"

This condition is enough and implies all the four you gave
So for example we could also conclude linearity with condition as $|g(x)|\ge 1$ $\forall x\in[1,2)$



\end{solution}



\begin{solution}[by \href{https://artofproblemsolving.com/community/user/348607}{ahmedAbd}]
	\begin{tcolorbox}[quote=TuZo]We can got from the additivity that $g(a)=ca$, in the following situations:
1) g is continuouse everyvhere
2) g is continouse in inly one point
3) g is monotonous 
4) g is non negative on an interval\end{tcolorbox}
In fact (out of this specific problem), the weaker sufficient condition to conclude linearity is likely "graph of $g(x)$ is not dense in $\mathbb R^2$"

This condition is enough and implies all the four you gave
So for example we could also conclude linearity with condition as $|g(x)|\ge 1$ $\forall x\in[1,2)$\end{tcolorbox}

So, is the continuity necessary or not?
\end{solution}



\begin{solution}[by \href{https://artofproblemsolving.com/community/user/330150}{L3435}]
	\begin{tcolorbox}[quote=pco][quote=TuZo]We can got from the additivity that $g(a)=ca$, in the following situations:
1) g is continuouse everyvhere
2) g is continouse in inly one point
3) g is monotonous 
4) g is non negative on an interval\end{tcolorbox}
In fact (out of this specific problem), the weaker sufficient condition to conclude linearity is likely "graph of $g(x)$ is not dense in $\mathbb R^2$"

This condition is enough and implies all the four you gave
So for example we could also conclude linearity with condition as $|g(x)|\ge 1$ $\forall x\in[1,2)$\end{tcolorbox}

So, is the continuity necessary or not?\end{tcolorbox}

No, it is not
\end{solution}



\begin{solution}[by \href{https://artofproblemsolving.com/community/user/29428}{pco}]
	\begin{tcolorbox}So, is the continuity necessary or not?\end{tcolorbox}
It is difficult to answer to this question :

A lot of weaker conditions are enough to imply that an additive function is linear.
But then, since linear, it is continuous.

So continuity condition \begin{bolded}is quite equivalent\end{underlined}\end{bolded} in this case with a lot of other [apparently weaker] conditions.

The important thing is that apparently weaker conditions are generally easier to prove than continuity (which can always be proved with just one line more)



\end{solution}



\begin{solution}[by \href{https://artofproblemsolving.com/community/user/341079}{Sapi123}]
	So in this problem continuity of $f$ is not needed, right?
\end{solution}



\begin{solution}[by \href{https://artofproblemsolving.com/community/user/29428}{pco}]
	\begin{tcolorbox}So in this problem continuity of $f$ is not needed, right?\end{tcolorbox}

I dont understand "is not needed"

Obviously, presence of continuity constraint in problem statement is not required to achieve the solution.
But obviously also, all solutions are continuous (there does not exist non continuous solutions for this problem)
\end{solution}
*******************************************************************************
-------------------------------------------------------------------------------

\begin{problem}[Posted by \href{https://artofproblemsolving.com/community/user/392546}{onlygeo}]
	Find all functions $f:\mathbb{R}\to \mathbb{R}$ such that $$f(x)f(y)=|x-y|f(\frac{xy+1}{x-y})$$ holds for all $x\not=y\in\mathbb{R}$.
	\flushright \href{https://artofproblemsolving.com/community/c6h1594490}{(Link to AoPS)}
\end{problem}



\begin{solution}[by \href{https://artofproblemsolving.com/community/user/29428}{pco}]
	\begin{tcolorbox}Find all functions $f:\mathbb{R}\to \mathbb{R}$ such that $$f(x)f(y)=|x-y|f(\frac{xy+1}{x-y})$$ holds for all $x,y\in\mathbb{R}$.\end{tcolorbox}

None.
Just set $x=y=0$ (since domain of functional equation is "$\forall x,y\in\mathbb R$$) and RHS is not defined and so equation wrong.

\end{solution}



\begin{solution}[by \href{https://artofproblemsolving.com/community/user/392546}{onlygeo}]
	Of course,i forgot to write $x\not=y$.
\end{solution}



\begin{solution}[by \href{https://artofproblemsolving.com/community/user/29428}{pco}]
	\begin{tcolorbox}Find all functions $f:\mathbb{R}\to \mathbb{R}$ such that $$f(x)f(y)=|x-y|f(\frac{xy+1}{x-y})$$ holds for all $x\not=y\in\mathbb{R}$.\end{tcolorbox}
Let $P(x,y)$ be the assertion $f(x)f(y)=|x-y|f(\frac{xy+1}{x-y})$ true $\forall x\ne y$

1) If $f(u)=0$ for some $u\ne 0$ 
let $x\ne u$ : Obviously $\frac{ux+1}{x-u}\ne u$ and so 
$P(\frac{ux+1}{x-u},u)$ $\implies$ $f(x)=0$ $\forall x\ne u$

And so $\boxed{\text{S1 : }f(x)=0\quad\forall x}$ which indeed is a solution

2) If $f(x)\ne 0$ $\forall x$
$P(x+\sqrt{x^2+1},x)$ $\implies$ $\boxed{\text{S2 : }f(x)=\sqrt{x^2+1}\quad\forall x}$ 
Which indeed is a solution.
\end{solution}
*******************************************************************************
-------------------------------------------------------------------------------

\begin{problem}[Posted by \href{https://artofproblemsolving.com/community/user/380742}{Yimself}]
	Given this function, how can we find the real part of the complex roots aproximatively?$$f(x)=x^3-x^2-x-1$$
The book gives those possible intervals as answers: a)(-1\/2;0) b)(-1;-1\/2) c)(-2;-1). Now using wolfram alpha i found out to be a), but I have no ideea how to proceed. Any help please?
	\flushright \href{https://artofproblemsolving.com/community/c6h1596784}{(Link to AoPS)}
\end{problem}



\begin{solution}[by \href{https://artofproblemsolving.com/community/user/29428}{pco}]
	\begin{tcolorbox}Given this function, how can we find the real part of the complex roots aproximatively?$$f(x)=x^3-x^2-x-1$$
The book gives those possible intervals as answers: a)(-1\/2;0) b)(-1;-1\/2) c)(-2;-1). Now using wolfram alpha i found out to be a), but I have no ideea how to proceed. Any help please?\end{tcolorbox}
We are looking real numbers $(u,v)$ with $v\ne 0$ and such that 
$(u+iv)^3-(u+iv)^2-(u+iv)-1=0$

Looking at imaginary part, we get $3u^2v-v^3-2uv-v=0$ and since $v\ne 0$ : 
$v^2=3u^2-2u-1$

Looking at real part, we get $u^3-u^2-u-1-v^2(3u-1)=0$
Plugging there $v^2=3u^2-2u-1$, this becomes
$4u^3-4u^2+1=0$

And it is immediate to see that this cubic has a unique real root $\in(-\frac 12,0)$


\end{solution}
*******************************************************************************
-------------------------------------------------------------------------------

\begin{problem}[Posted by \href{https://artofproblemsolving.com/community/user/393430}{georgeado17}]
	$f :\mathbb{R} \to \mathbb{R}$ find all functions such that
$(x+y)(f(x)-f(y))=(x-y)f(x+y)$
$x$ and $y$ are real numbers

	\flushright \href{https://artofproblemsolving.com/community/c6h1598114}{(Link to AoPS)}
\end{problem}



\begin{solution}[by \href{https://artofproblemsolving.com/community/user/29428}{pco}]
	\begin{tcolorbox}$f :\mathbb{R} \to \mathbb{R}$ find all functions such that
$(x+y)(f(x)-f(y))=(x-y)f(x+y)$
$x$ and $y$ are real numbers\end{tcolorbox}
Let $P(x,y)$ be the assertion $(x+y)(f(x)-f(y))=(x-y)f(x+y)$

$P(1,0)$ $\implies$ $f(0)=0$
Let $x\ne 0$ :
$P(\frac{x+1}2,\frac{x-1}2)$ $\implies$ $f(\frac{x+1}2)-f(\frac{x-1}2)=\frac{f(x)}x$

$P(\frac{x-1}2,\frac{3-x}2)$ $\implies$ $f(\frac{x-1}2)-f(\frac{3-x}2)=(x-2)f(1)$

$P(\frac{3-x}2,\frac{x+1}2)$ $\implies$ $f(\frac{3-x}2)-f(\frac{x+1}2)=\frac{1-x}2f(2)$

Adding these three lines, we get 
$\frac{f(x)}x+(x-2)f(1)+\frac{1-x}2f(2)=0$ $\forall x\ne 0$ 

And so $f(x)=x^2(\frac{f(2)}2-f(1))+x(2f(1)-\frac{f(2)}2)$ $\forall x\ne 0$, still true when $x=0$

And so $\boxed{f(x)=ax^2+bx\quad\forall x}$ which indeed is a solution, whatever are $a,b\in\mathbb R$
\end{solution}



\begin{solution}[by \href{https://artofproblemsolving.com/community/user/393430}{georgeado17}]
	\begin{tcolorbox}[quote=georgeado17]$f :\mathbb{R} \to \mathbb{R}$ find all functions such that
$(x+y)(f(x)-f(y))=(x-y)f(x+y)$
$x$ and $y$ are real numbers\end{tcolorbox}
Let $P(x,y)$ be the assertion $(x+y)(f(x)-f(y))=(x-y)f(x+y)$

$P(1,0)$ $\implies$ $f(0)=0$
Let $x\ne 0$ :
$P(\frac{x+1}2,\frac{x-1}2)$ $\implies$ $f(\frac{x+1}2)-f(\frac{x-1}2)=\frac{f(x)}x$

$P(\frac{x-1}2,\frac{3-x}2)$ $\implies$ $f(\frac{x-1}2)-f(\frac{3-x}2)=(x-2)f(1)$

$P(\frac{3-x}2,\frac{x+1}2)$ $\implies$ $f(\frac{3-x}2)-f(\frac{x+1}2)=\frac{1-x}2f(2)$

Adding these three lines, we get 
$\frac{f(x)}x+(x-2)f(1)+\frac{1-x}2f(2)=0$ $\forall x\ne 0$ 

And so $f(x)=x^2(\frac{f(2)}2-f(1))+x(2f(1)-\frac{f(2)}2)$ $\forall x\ne 0$, still true when $x=0$

And so $\boxed{f(x)=ax^2+bx\quad\forall x}$ which indeed is a solution, whatever are $a,b\in\mathbb R$\end{tcolorbox}

Thx.  ;) 
\end{solution}
*******************************************************************************
-------------------------------------------------------------------------------

\begin{problem}[Posted by \href{https://artofproblemsolving.com/community/user/398931}{cycliccircle}]
	Find all $f: \mathbb{R}\to\mathbb{R}$ such that $$f\left(1-\frac{4}{x}\right) + f(x-1) = 1$$ for all non-zero real $x$.
	\flushright \href{https://artofproblemsolving.com/community/c6h1598214}{(Link to AoPS)}
\end{problem}



\begin{solution}[by \href{https://artofproblemsolving.com/community/user/29428}{pco}]
	\begin{tcolorbox}Find all $f: \mathbb{R}\to\mathbb{R}$ such that $$f\left(1-\frac{4}{x}\right) + f(x-1) = 1$$ $\forall x \in \mathbb{R}$.\end{tcolorbox}
None.
Set $x=0$ (allowed since you gave domain of functional equation $\forall x\in\mathbb R$) and LHS is not defined, and so functional equation false.


\end{solution}



\begin{solution}[by \href{https://artofproblemsolving.com/community/user/398931}{cycliccircle}]
	Congrats! Now try this little variation.

Find all $f: \mathbb{R}\to\mathbb{R}$ such that $$f\left(1-\frac{4}{x}\right) + f(x-1) = 1$$ for all non zero $x \in \mathbb{R}$.
\end{solution}



\begin{solution}[by \href{https://artofproblemsolving.com/community/user/398931}{cycliccircle}]
	bump.\/.\/.\/
\end{solution}



\begin{solution}[by \href{https://artofproblemsolving.com/community/user/350483}{adhikariprajitraj}]
	\begin{tcolorbox}Congrats! Now try this little variation.

Find all $f: \mathbb{R}\to\mathbb{R}$ such that $$f\left(1-\frac{4}{x}\right) + f(x-1) = 1$$ for all non zero $x \in \mathbb{R}$.\end{tcolorbox}

Set $1-\frac{4}{x}=x-1=p$,
then we get $f(p)=\frac{1}{2}$ where $x=1 \pm \sqrt{5}$,
Also, we see when $x=2$
$$f(-1)+f(1)=1$$ 
And $f(-x)\not=-f(x)$
So, $\boxed{f(x)= \frac{1}{2}}$. :)
\end{solution}



\begin{solution}[by \href{https://artofproblemsolving.com/community/user/384646}{TheMathsBoy}]
	Will this work:\end{underlined}
Putting $(x = 2)$, we get $f(1) +f(-1) = 1$.
I only found a trivial solution i.e. $f(x) = \frac{1}{2}$
\end{solution}



\begin{solution}[by \href{https://artofproblemsolving.com/community/user/350483}{adhikariprajitraj}]
	\begin{tcolorbox}Will this work:\end{underlined}
Putting $(x = 2)$, we get $f(1) +f(-1) = 1$.
I only found a trivial solution i.e. $f(x) = \frac{1}{2}$\end{tcolorbox}

That can work for $f(x)= \frac{x^2}{2}$, also for $f(x)=\frac{|x|}{2}$. So, probably it is not the solution. :D
Either what we can observe from your solution is that 
$$f(-x)\not= -f(x)$$
Which supports my solution!
Which gives inspiration that at least we should try the problems! Thanks!
\end{solution}



\begin{solution}[by \href{https://artofproblemsolving.com/community/user/384646}{TheMathsBoy}]
	No, I meant that only solution is $f(x) = \frac{1}{2}$ for all $\mathbb{R}$ except $0$.
\end{solution}



\begin{solution}[by \href{https://artofproblemsolving.com/community/user/350483}{adhikariprajitraj}]
	\begin{tcolorbox}No, I meant that only solution is $f(x) = \frac{1}{2}$ for all $\mathbb{R}$ except $0$.\end{tcolorbox}

Seems legit though!
\end{solution}



\begin{solution}[by \href{https://artofproblemsolving.com/community/user/333350}{AnArtist}]
	\begin{tcolorbox}Congrats! Now try this little variation.

Find all $f: \mathbb{R}\to\mathbb{R}$ such that $$f\left(1-\frac{4}{x}\right) + f(x-1) = 1$$ for all non zero $x \in \mathbb{R}$.\end{tcolorbox}

Let $t \in \mathbb {R}$, $t \neq \pm 1$

Putting $x=t+1$ we get,

$$f(\frac{t-3}{t+1}) + f(t) =1$$

Putting $x= \frac{2t-2}{t+1}$ gives,

$$f( \frac{-t-3}{t-1} ) + f( \frac{t-3}{t+1}) =1$$

Putting $x= \frac{-4}{t-1}$ gives,

$$f(t) + f( \frac{-t-3}{t-1} ) =1$$

Subtract equation 2 from equation 1 and add in equation 3 to get,

$$f(t)= \frac{1}{2}$$.

Now put $x=2$ to get,

$$f(1) + f(-1) =1$$

Now the given equation shows that $f(1),f(-1) $ can be arbitrary with their sum equal to $1$.
\end{solution}



\begin{solution}[by \href{https://artofproblemsolving.com/community/user/333350}{AnArtist}]
	\begin{tcolorbox}[quote=cycliccircle]Congrats! Now try this little variation.

Find all $f: \mathbb{R}\to\mathbb{R}$ such that $$f\left(1-\frac{4}{x}\right) + f(x-1) = 1$$ for all non zero $x \in \mathbb{R}$.\end{tcolorbox}

Set $1-\frac{4}{x}=x-1=p$,
then we get $f(p)=\frac{1}{2}$ where $x=1 \pm \sqrt{5}$,
Also, we see when $x=2$
$$f(-1)+f(1)=1$$ 
And $f(-x)\not=-f(x)$
So, $\boxed{f(x)= \frac{1}{2}}$. :)\end{tcolorbox}

wrong solution. $f(1)$ need not be $\frac{1}{2}$.

More importantly how does this even lead to a solution.
\end{solution}



\begin{solution}[by \href{https://artofproblemsolving.com/community/user/350483}{adhikariprajitraj}]
	\begin{tcolorbox}[quote=adhikariprajitraj][quote=cycliccircle]Congrats! Now try this little variation.

Find all $f: \mathbb{R}\to\mathbb{R}$ such that $$f\left(1-\frac{4}{x}\right) + f(x-1) = 1$$ for all non zero $x \in \mathbb{R}$.\end{tcolorbox}

Set $1-\frac{4}{x}=x-1=p$,
then we get $f(p)=\frac{1}{2}$ where $x=1 \pm \sqrt{5}$,
Also, we see when $x=2$
$$f(-1)+f(1)=1$$ 
And $f(-x)\not=-f(x)$
So, $\boxed{f(x)= \frac{1}{2}}$. :)\end{tcolorbox}

wrong solution. $f(1)$ need not be $\frac{1}{2}$.

More importantly how does this even lead to a solution.\end{tcolorbox}

Hehe.. sorry I messed up! :D
\end{solution}



\begin{solution}[by \href{https://artofproblemsolving.com/community/user/398931}{cycliccircle}]
	Nice. Now try this.

Find all $f: \mathbb{R}\to\mathbb{R}$ such that

$$f \left( 1-2x \right) + f \left( \frac{1}{x} + 3 \right) = 1$$

for all non-zero real $x$.
\end{solution}



\begin{solution}[by \href{https://artofproblemsolving.com/community/user/303223}{Gluncho}]
	@AnArtist

What was the motivation for this substitutions?
\end{solution}



\begin{solution}[by \href{https://artofproblemsolving.com/community/user/333350}{AnArtist}]
	\begin{tcolorbox}@AnArtist

What was the motivation for this substitutions?\end{tcolorbox}

Since we have only one variable. We are too restricted. So we tried to make equal terms so we can cancel unwanted ones.
\end{solution}



\begin{solution}[by \href{https://artofproblemsolving.com/community/user/333350}{AnArtist}]
	\begin{tcolorbox}Nice. Now try this.

Find all $f: \mathbb{R}\to\mathbb{R}$ such that

$$f \left( 1-2x \right) + f \left( \frac{1}{x} + 3 \right) = 1$$

for all non-zero real $x$.\end{tcolorbox}

Are you making your own problems? 
\end{solution}



\begin{solution}[by \href{https://artofproblemsolving.com/community/user/398931}{cycliccircle}]
	Yeah I made them
\end{solution}



\begin{solution}[by \href{https://artofproblemsolving.com/community/user/313451}{pro_4_ever}]
	\begin{tcolorbox}Yeah I made them\end{tcolorbox}

LOL, you are not supposed to do that, unless you yourself have a proper proof!
\end{solution}



\begin{solution}[by \href{https://artofproblemsolving.com/community/user/398931}{cycliccircle}]
	It's not some random FE. I've made it thoughtfully. I have a (lengthy) proof.
\end{solution}



\begin{solution}[by \href{https://artofproblemsolving.com/community/user/398931}{cycliccircle}]
	\begin{tcolorbox}Nice. Now try this.

Find all $f: \mathbb{R}\to\mathbb{R}$ such that

$$f \left( 1-2x \right) + f \left( \frac{1}{x} + 3 \right) = 1$$

for all non-zero real $x$.\end{tcolorbox}

Anyone?
\end{solution}



\begin{solution}[by \href{https://artofproblemsolving.com/community/user/350483}{adhikariprajitraj}]
	Let me try !
\end{solution}



\begin{solution}[by \href{https://artofproblemsolving.com/community/user/333350}{AnArtist}]
	@2above it can be solved similarly as the first one (though it gets too lengthy).
\end{solution}



\begin{solution}[by \href{https://artofproblemsolving.com/community/user/350483}{adhikariprajitraj}]
	\begin{tcolorbox}@2above it can be solved similarly as the first one (though it gets too lengthy).\end{tcolorbox}

Yeah I am trying but it is getting so lengthy. :D
\end{solution}
*******************************************************************************
-------------------------------------------------------------------------------

\begin{problem}[Posted by \href{https://artofproblemsolving.com/community/user/344350}{soryn}]
	Let $f$ :$\left(0,\infty\right)\rightarrow\left(0,\infty\right)$a
non-constant function,satisfying the relationship:

$f\left(x\right)\cdotp f\left(yf\left(x\right)\right)\cdotp f\left(zf\left(x+y\right)\right)=f\left(x+y+z\right)$,whatever
$x,y,z\in\left(0,\infty\right)$.

prove that $f$ is injective and all this functions.
	\flushright \href{https://artofproblemsolving.com/community/c6h1600592}{(Link to AoPS)}
\end{problem}



\begin{solution}[by \href{https://artofproblemsolving.com/community/user/344350}{soryn}]
	The solution is f(x)=1\/(x+k)?
\end{solution}



\begin{solution}[by \href{https://artofproblemsolving.com/community/user/29428}{pco}]
	\begin{tcolorbox}Let $f$ :$\left(0,\infty\right)\rightarrow\left(0,\infty\right)$a
non-constant function,satisfying the relationship:

$f\left(x\right)\cdotp f\left(yf\left(x\right)\right)\cdotp f\left(zf\left(x+y\right)\right)=f\left(x+y+z\right)$,whatever
$x,y,z\in\left(0,\infty\right)$.

prove that $f$ is injective and all this functions.\end{tcolorbox}
Already posted (and solved) (04\/2017 and 10\/2013)

Dont hesitate to use the search function (see [url=http://www.artofproblemsolving.com\/community\/c6h1330604p7173058] here [\/url]).
Set (for example copy\/paste) in the "search term" field the exact following string : 

+"f(x)f(yf(x))f(zf(x+y))" +"f(x+y+z)"

You'll get in the \begin{bolded}three first results\end{underlined}\end{bolded} (excluded your own post and this post itself) all the help you are requesting for.

[hide=(Some excuses)][size=70]I'm sorry not providing you the direct link to this result but I encountered users who never tried the search function, thinking quite easier to have other users make the search for them. So now I prefer to point to the search function and to give the appropriate search term (I checked that it indeed will give you the expected result) instead of the link itself [\/size]([url=https:\/\/en.wiktionary.org\/wiki\/give_a_man_a_fish_and_you_feed_him_for_a_day;_teach_a_man_to_fish_and_you_feed_him_for_a_lifetime]wink[\/url])[\/hide]




\end{solution}



\begin{solution}[by \href{https://artofproblemsolving.com/community/user/344350}{soryn}]
	Sir pco,your ingenios solutions,always I was delighted! Please, explain me your solution! Thanks!
\end{solution}



\begin{solution}[by \href{https://artofproblemsolving.com/community/user/29428}{pco}]
	\begin{tcolorbox}Sir pco,your ingenios solutions,always I was delighted! Please, explain me your solution! Thanks!\end{tcolorbox}
I dont understand your question.
The solution given thru the search function (as indicated in my post #2) is not produced by me but by user socrates.

So, if you dont understand his solution, go in the appropriate thread and ask him for explanation of the first line you dont understand.

It does not seem very difficult.


\end{solution}



\begin{solution}[by \href{https://artofproblemsolving.com/community/user/344350}{soryn}]
	Ooh,sorry,I have not seen it!
\end{solution}
*******************************************************************************
-------------------------------------------------------------------------------

\begin{problem}[Posted by \href{https://artofproblemsolving.com/community/user/211457}{rod16}]
	Determine all functions f defined from the set of positive real numbers to itself such that
$f( g(x)) = \frac{x}{xf(x) - 2}$ and  $g( f(x)) = \frac{x}{xg(x) - 2} $
for all positive real numbers x.
	\flushright \href{https://artofproblemsolving.com/community/c6h1601278}{(Link to AoPS)}
\end{problem}



\begin{solution}[by \href{https://artofproblemsolving.com/community/user/29428}{pco}]
	\begin{tcolorbox}Determine all functions f defined from the set of positive real numbers to itself such that
$f( g(x)) = \frac{x}{xf(x) - 2}$ and  $g( f(x)) = \frac{x}{xg(x) - 2} $
for all positive real numbers x.\end{tcolorbox}

The two equations imply $f(x)>\frac 2x$ and $g(x)>\frac 2x$ $\forall x>0$

This implies $\frac x{xf(x)-2}>\frac 2{g(x)}$ and so $xg(x)>2xf(x)-4$
Same, we get $xf(x)>2xg(x)-4$
These two inequalities imply $xf(x)<4$ and $xg(x)<4$

Let us consider now $a>xf(x),xg(x)>b$ $\forall x>0$
Applying the above process, we get $\frac {2b}{b-1}>xf(x),xg(x)>\frac {2a}{a-1}$

And since sequence $(a_0,b_0)=(4,2)$ and $(a_{n+1},b_{n+1}=(\frac {2b_n}{b_n-1},\frac{2a_n}{a_n-1})$ is convergent towards $(3,3)$

We get the only possibility $\boxed{f(x)=g(x)=\frac 3x\quad\forall x>0}$ which indeed is a solution

\end{solution}



\begin{solution}[by \href{https://artofproblemsolving.com/community/user/350483}{adhikariprajitraj}]
	\begin{tcolorbox}[quote=rod16]Determine all functions f defined from the set of positive real numbers to itself such that
$f( g(x)) = \frac{x}{xf(x) - 2}$ and  $g( f(x)) = \frac{x}{xg(x) - 2} $
for all positive real numbers x.\end{tcolorbox}

The two equations imply $f(x)>\frac 2x$ and $g(x)>\frac 2x$ $\forall x>0$

This implies $\frac x{xf(x)-2}>\frac 2{g(x)}$ and so $xg(x)>2xf(x)-4$
Same, we get $xf(x)>2xg(x)-4$
These two inequalities imply $xf(x)<4$ and $xg(x)<4$
-_-

Let us consider now $a>xf(x),xg(x)>b$ $\forall x>0$
Applying the above process, we get $\frac {2b}{b-1}>xf(x),xg(x)>\frac {2a}{a-1}$

And since sequence $(a_0,b_0)=(4,2)$ and $(a_{n+1},b_{n+1}=(\frac {2b_n}{b_n-1},\frac{2a_n}{a_n-1})$ is convergent towards $(3,3)$

We get the only possibility $\boxed{f(x)=g(x)=\frac 3x\quad\forall x>0}$ which indeed is a solution\end{tcolorbox}

Will you please elaborate? -_-
\end{solution}



\begin{solution}[by \href{https://artofproblemsolving.com/community/user/29428}{pco}]
	\begin{tcolorbox}Will you please elaborate? -_-\end{tcolorbox}
Dont hesitate to indicate the first line you dont understand.


\end{solution}



\begin{solution}[by \href{https://artofproblemsolving.com/community/user/350483}{adhikariprajitraj}]
	This one: [hide=......]Thank you![\/hide]
The two equations imply $f(x)>\frac 2x$ and $g(x)>\frac 2x$ $\forall x>0$
\end{solution}



\begin{solution}[by \href{https://artofproblemsolving.com/community/user/29428}{pco}]
	\begin{tcolorbox}This one: [hide=......]Thank you![\/hide]
The two equations imply $f(x)>\frac 2x$ and $g(x)>\frac 2x$ $\forall x>0$\end{tcolorbox}

$f(g(x))>0$ and $x>0$ imply $xf(x)-2>0$ and so ...
\end{solution}



\begin{solution}[by \href{https://artofproblemsolving.com/community/user/350483}{adhikariprajitraj}]
	Thank you!
\end{solution}
*******************************************************************************
-------------------------------------------------------------------------------

\begin{problem}[Posted by \href{https://artofproblemsolving.com/community/user/212515}{adityaguharoy}]
	Find all functions $F: [0,1] \to [0,1]$ such that $F$ is continuous over $[0,1]$, $F(0)=0,F(1)=1$, $F$ is non decreasing function, and also obeys $F(1-x)=1-F(x)$ , $\forall x \in [0,1]$

[hide=A sketch that uses probability]
Consider the problem stated below :
Let there be a stick $1$ metre long. We choose $3$ points uniformly, randomly and independently along the length of the stick. Then we break the stick at these three points. Find the distribution of the sum of length of the two leftmost parts.

We solve it, in two closely related ways.
Let us start with the first way
[color=#00f]\begin{bolded}First way of solving it \end{bolded}[\/color]
Let the length of the left to right parts be $a,b,c,d$ respectively, then as the three break points were chosen uniformly, randomly and independently along the length of the stick, so $a+b$ and $c+d$ will have the same distribution. 
And note that if we let $G$ be the CDF of the sum of the lengths of the leftmost two parts, then $$G(x)=\mathbb{P}(a+b \le x)$$ and, $$=\mathbb{P}(c+d \le x) =\mathbb{P}(a+b > 1-x ) =1- \mathbb{P}(a+b \le 1-x)$$ $$=1-G(1-x)$$
This thus gives us that $$G(1-x)=1-G(x)$$ $\forall x \in [0,1]$ and now, we also ensure that $G$ is continuous on $[0,1]$, $G$ is non decreasing as $G$ is a cumulative distribution function, and also we ensure that $G(0)=0$ and $G(1)=1$ from the way $G$ is defined. 
So, $F=G$ is a solution to the functional equation that is given.

Now, let us see the second method to solve it.

[color=#00f]\begin{bolded}Second method\end{bolded}[\/color]
Since, the sum of the lengths of the two leftmost parts and the two rightmost parts are same so, it is equivalent to breaking the stick at one point and finding distribution of any of the two pieces we get. 
This distribution is named the Uniform $(0,1)$ distribution, and has the CDF 
$$G(x)=x \textbf{if} x \in [0,1]$$
$$G(x)=0 \textbf{  if  } x < 0$$
$$G(x)=1 \textbf{  if  } x > 0$$

Thus, the function $F$ (as $F$ behaves identically as $G$) must be $F(x)=x \forall x \in (0,1)$. 

So, this indicates that $\boxed{F(x)=x \forall x \in [0,1]}$ is the only solution to the functional equation. 

[\/hide]

[hide=...............]
I further apologise for posting it in two forums simultaneously (something I always discourage since last year), but I think that this problem has well connections with both the forums, and so posting it in two separate forums may help to gather more audience attention to the problem. 
[\/hide]
	\flushright \href{https://artofproblemsolving.com/community/c6h1601617}{(Link to AoPS)}
\end{problem}



\begin{solution}[by \href{https://artofproblemsolving.com/community/user/249633}{Gryphos}]
	Can't we just take any continuous increasing function $F$ on $[0, 1\/2]$ with $F(0)=0$, $F(1\/2)=1\/2$ and extend it to all of $[0,1]$ via $F(1-x)=1-F(x)$?
\end{solution}



\begin{solution}[by \href{https://artofproblemsolving.com/community/user/212515}{adityaguharoy}]
	No, I think because of  the given properties of $F$. 
Can you give any other example ?
\end{solution}



\begin{solution}[by \href{https://artofproblemsolving.com/community/user/249633}{Gryphos}]
	I don't know if I am missing something, but which of the given properties would be violated by such a function?
\end{solution}



\begin{solution}[by \href{https://artofproblemsolving.com/community/user/212515}{adityaguharoy}]
	Can you give me an example ?
\end{solution}



\begin{solution}[by \href{https://artofproblemsolving.com/community/user/29428}{pco}]
	\begin{tcolorbox}Can you give me an example ?\end{tcolorbox}
$f(x)=2x^2$ $\forall x\in[0,\frac 12]$

$f(x)=1-2(1-x)^2$ $\forall x\in[\frac 12,1]$



\end{solution}



\begin{solution}[by \href{https://artofproblemsolving.com/community/user/394502}{JohnHankock}]
	Subbing in 0.5 for x gives f(0.5) = 1 - f(0.5) -> f(0.5) = 0.5. By moving around the first and last terms, we see that any two terms the same distance from 0.5 must average out to 0.5, with the latter greater than the former--ie symmetric about (0.5,0.5). Technically though, @Gryphos has an idea. If we take any continuous increasing function according to his constraints, and pivot it 180 degrees about (0.5,0.5), we have the solution.
\end{solution}
*******************************************************************************
-------------------------------------------------------------------------------

\begin{problem}[Posted by \href{https://artofproblemsolving.com/community/user/403452}{Saeed.Taghavi}]
	$f: \mathbb{R^+} \to \mathbb{R^+}$
$f(xf(y))=f(xy)+x$

	\flushright \href{https://artofproblemsolving.com/community/c6h1602151}{(Link to AoPS)}
\end{problem}



\begin{solution}[by \href{https://artofproblemsolving.com/community/user/29428}{pco}]
	\begin{tcolorbox}$f: \mathbb{R^+} \to \mathbb{R^+}$
$f(xf(y))=f(xy)+x$\end{tcolorbox}
Let $P(x,y)$ be the assertion $f(xf(y))=f(xy)+x$
Let $c=f(1)$

Note that $f(xy)+x\in f(\mathbb R^+)$ $\forall x,y>0$
Setting there $y=\frac 1x$, we get $(c,+\infty)\subseteq f(\mathbb R)$

$P(1,x)$ $\implies$ $f(f(x))=f(x)+1$ and so $f(x)=x+1$ $\forall x\in f(\mathbb R^+)$ and so 
$f(x)=x+1$ $\forall x>c$

Let then $x>0$ and $y>\max(\frac cx,\frac c{f(x)})$
$P(y,x)$ $\implies$ $f(yf(x))=f(xy)+y$ and so 
$yf(x)+1=xy+1+y$
And so $\boxed{f(x)=x+1\quad\forall x>0}$ which indeed is a solution.


\end{solution}



\begin{solution}[by \href{https://artofproblemsolving.com/community/user/390186}{chiyukiRIP}]
	i tried with roots of the function:
$f(xf(y))=f(xy)+x$
$f(a)=0$
$f(1f(a))=f(a)+1$
$f(0)=1$
and $f(xf(0))=f(0)+x$
$f(x)=x+1$
\end{solution}



\begin{solution}[by \href{https://artofproblemsolving.com/community/user/293568}{igli.2001}]
	Is somebody going to tell @above that $0$ is not a positive real number?
\end{solution}



\begin{solution}[by \href{https://artofproblemsolving.com/community/user/335559}{Duarti}]
	\begin{tcolorbox}i tried with roots of the function:
$f(xf(y))=f(xy)+x$
$f(a)=0$
$f(1f(a))=f(a)+1$
$f(0)=1$
and $f(xf(0))=f(0)+x$
$f(x)=x+1$\end{tcolorbox}

$0$ is not a positive real number.


\end{solution}



\begin{solution}[by \href{https://artofproblemsolving.com/community/user/209049}{math90}]
	$P(f(x),y)\implies f(f(x)f(y))=f(f(x)y)+f(x)=f(xy)+y+f(x)$
$\implies f(x)+y=f(y)+x$
$\implies f(x)=x+c$ for some constant $c$.
$P(1,1)\implies f(f(1))=f(1)+1$, so $c=1$.
Hence $\boxed{f(x)=x+1}$, which indeed is a solution.

\end{solution}



\begin{solution}[by \href{https://artofproblemsolving.com/community/user/303223}{Gluncho}]
	Fixing $xy$ gives surjectivity.
$x=1$ gives $f(f(y))=f(y)+1$. 
So $f(f(f(y)))=f(f(y)+1)=f(f(y)+1=f(y)+2$. So $f(f(y)+1)=f(y)+2$, hence $f(x)=x+1 \forall x \in \mathbb{R^+}$
\end{solution}



\begin{solution}[by \href{https://artofproblemsolving.com/community/user/29428}{pco}]
	\begin{tcolorbox}Fixing $xy$ gives surjectivity.\end{tcolorbox}
No, it does not (remember $x>0$)



\end{solution}



\begin{solution}[by \href{https://artofproblemsolving.com/community/user/361807}{RezaRajabzadeh}]
	\begin{tcolorbox}Is somebody going to tell @above that $0$ is not a positive real number?\end{tcolorbox}

Ha Ha Ha Ha Ha :rotfl:  :rotfl: 
\end{solution}
*******************************************************************************
-------------------------------------------------------------------------------

\begin{problem}[Posted by \href{https://artofproblemsolving.com/community/user/386303}{longnhat2002}]
	Find all function \[f:{Q^ + } \to {Q^ + }\] satisfying the two conditions:
\[\begin{array}{l}
(i)\,f(x + 1) = f(x) + 1\,(for\,all\,x \in {Q^ + })\\
(ii)\,f({x^2}) = {f^2}(x)\,(for\,all\,x \in {Q^ + })
\end{array}\]
	\flushright \href{https://artofproblemsolving.com/community/c6h1602721}{(Link to AoPS)}
\end{problem}



\begin{solution}[by \href{https://artofproblemsolving.com/community/user/29428}{pco}]
	\begin{tcolorbox}Find all function \[f:{Q^ + } \to {Q^ + }\] satisfying the two conditions:
\[\begin{array}{l}
(i)\,f(x + 1) = f(x) + 1\,(for\,all\,x \in {Q^ + })\\
(ii)\,f({x^2}) = {f^2}(x)\,(for\,all\,x \in {Q^ + })
\end{array}\]\end{tcolorbox}
(i) implies $f(x+n)=f(x)+n$ $\forall x\in\mathbb Q^+$, $\forall n\in\mathbb Z_{\ge 0}$

Let then $p,q\in\mathbb N$ :
$f((\frac pq+q)^2)=f(\frac pq+q)^2=(f(\frac pq)+q)^2$ $=f(\frac pq)^2+2qf(\frac pq)+q^2$
But also :
$f((\frac pq+q)^2)=f(\frac{p^2}{q^2}+2p+q^2)=f(\frac{p^2}{q^2})+2p+q^2$ $=f(\frac pq)^2+2p+q^2$

An so, subtracting, $f(\frac pq)=\frac pq$
And so $\boxed{f(x)=x\quad\forall x\in\mathbb Q^+}$ which indeed is a solution


\end{solution}
*******************************************************************************
-------------------------------------------------------------------------------

\begin{problem}[Posted by \href{https://artofproblemsolving.com/community/user/393430}{georgeado17}]
	$f:\mathbb{R} \to \mathbb{R}$ $x,y\in R$
$f^2(x+y)-f^2(x-y)=4f(x)f(y)$
$f(x)=?$
	\flushright \href{https://artofproblemsolving.com/community/c6h1602747}{(Link to AoPS)}
\end{problem}



\begin{solution}[by \href{https://artofproblemsolving.com/community/user/398931}{cycliccircle}]
	$f^2(x) = [f(x)]^2$?
\end{solution}



\begin{solution}[by \href{https://artofproblemsolving.com/community/user/367931}{Vrangr}]
	@above, usually \[f^n(x)= \underbrace{f(f(f(\dots f}_{n\text{ times}}(x)\dots)))\]
and 
\[f(x)^n=(f(x))^n\]
\end{solution}



\begin{solution}[by \href{https://artofproblemsolving.com/community/user/335705}{Arthur.}]
	[hide=Assuming cycliccircle's interpretation, this works ... ?]The only solutions are $f \equiv 0$ and $f \equiv x$. It's easy to verify these work.

$x=y=0$ implies $f(0)=0$.

$x=y$ implies $f(2x)=2f(x)$.

Assuming that $f$ isn't a monster (pathological solution), we can conclude $f(x)=kx$ (right?  :maybe: ), but only $k=1$ or $k=0$ work once we plug this in, so we're done.

[\/hide]
\end{solution}



\begin{solution}[by \href{https://artofproblemsolving.com/community/user/29428}{pco}]
	\begin{tcolorbox}$x=y$ implies $f(2x)=2f(x)$
Assuming that $f$ isn't a monster (pathological solution), we can conclude $f(x)=kx$ (right?  :maybe: ),\end{tcolorbox}
What is your definition of "pathological solution" ?)

For example $f(0)=0$ and $f(x)=x\{\log_2 |x|\}$ is one of the infinitely many solutions for FE $f(2x)=2f(x)$
Is it a "pathologocal solution" ?
\end{solution}



\begin{solution}[by \href{https://artofproblemsolving.com/community/user/335705}{Arthur.}]
	\begin{tcolorbox}[quote=Arthur.]$x=y$ implies $f(2x)=2f(x)$
Assuming that $f$ isn't a monster (pathological solution), we can conclude $f(x)=kx$ (right?  :maybe: ),\end{tcolorbox}
What is your definition of "pathological solution" ?)

For example $f(0)=0$ and $f(x)=x\{\log_2 |x|\}$ is one of the infinitely many solutions for FE $f(2x)=2f(x)$
Is it a "pathologocal solution" ?\end{tcolorbox}

$f(2x)=2f(x)$ and $f$ \begin{bolded}continuous\end{bolded} implies $f \equiv kx$ ?

[hide]If the OP is verbatim copying from a contest, fine, I concede. But I expect they are not, and I interpreted the question such that $f$ is 'nice', as is usually the case in MOs[\/hide]
\end{solution}



\begin{solution}[by \href{https://artofproblemsolving.com/community/user/29428}{pco}]
	\begin{tcolorbox}$f(2x)=2f(x)$ and $f$ \begin{bolded}continuous\end{bolded} implies $f \equiv kx$ ?\end{tcolorbox}
Certainly not.

Look for example at $f(0)=0$ and $f(x)=x\{\log_2|x|\}(1-\{\log_2|x|\})$ $\forall x\ne 0$



\end{solution}



\begin{solution}[by \href{https://artofproblemsolving.com/community/user/309179}{whiwho}]
	\begin{tcolorbox}$f:\mathbb{R} \to \mathbb{R}$ $x,y\in R$
$f^2(x+y)-f^2(x-y)=4f(x)f(y)$
$f(x)=?$\end{tcolorbox}
\begin{bolded}Note:\end{bolded} I'm assuming that $f^2(x) = f(f(x))$
[hide = My attempt]


[hide = Some initial results]
$P(0,0) = f^2(0)-f^2(0) = 0 = f(0)^2 \implies f(0)=0$
$P(x,x) = f^2(2x) = 4f(x)^2$
$P(0,y) = f^2(y) - f^2(-y) = 0 \implies f^2(y) = f^2(-y)$
$P(x,-x) = -f^2(-2x) = 4f(x)f(-x) = -f^2(2x)$
[\/hide]

\begin{bolded}Lemma 1.1\end{bolded} $f(-x) = -f(x) \forall x \in \mathbb{R}$
[hide= Proof]
$P(x,x)+P(x,-x) = f^2(2x)-f^2(2x) = 0 = 4f(x)(f(x)+f(-x)) \implies f(x)=0$ or $f(-x)= -f(x)$ for each $x\in \mathbb{R}$
[\/hide]

Lemma 1.2- The only value that points to $0$ is $0$
[hide=Proof]

Let $u$ be a real number such that $f(u)=0$ and $u\not = 0$, then
$P(x-u,u) \implies f^2(x) = f^2(x-2u)$ which implies that either $u=0$ is the only number such that $f(u)=0$ or the function is peridic (at least every 2u) or constant.

Asume that the minimum period is $T$.

$f^2(2(x+\frac{T}{2})) = f^2(2x+T) = f^2(2x)= 4f(x)^2 = 4f(x+\frac{T}{2})^2 \forall x \in \mathbb{R}$ which contradicts the minimalty of $T$ implying the new and less period  $\frac{T}{2}$. So $T = 0$. And so the only $u$ such that $f(u)=0$ is $0$. We also can now say that $f(-x)=-f(x)$.
[\/hide]

Result-$f(x) = 0 \forall x \in \mathbb{R}$
[Hide= Proof]
We also have that:
$P(0,y) = f^2(y) - f^2(-y) = 0 \implies f^2(y) = f^2(-y) \implies f(f(y))= f(f(-y)) = f(-f(y)) = -f(f(y)) \implies f(f(y)) = 0 \forall y \in \mathbb{R}$

From this the assertion changes to $P(x,y) = f(x)f(y)= 0$. From which is easy to conclude that $f(x) = 0 \forall x\in \mathbb{R}$
[\/hide]
 
[\/hide]
\end{solution}



\begin{solution}[by \href{https://artofproblemsolving.com/community/user/403767}{mkhayech}]
	The solutions to this equation are exactly the solutions to the Cauchy equation (i.e. additive functions).
Let $P(x,y) :  f^2(x+y)-f^2(x-y)=4f(x)f(y)$
$$P(0,0): f(0)=0$$
$$P(x,x): f^2(2x)=f^2(x) \text{                  }  \textbf{(1)}$$
$$P(x+y,2y): f^2(x+3y)-f^2(x-y)=4f(x+y)f(2y) \text{                  } [\textbf{(2)}$$
$$P(x+2y,y): f^2(x+3y)-f^2(x+y)=4f(x+2y)f(y)  \text{                  }  \textbf{(3)}$$
by $\textbf{(2)}$ and $\textbf{(3)}$  $$ f^2(x+3y)-f^2(x-y)= (f^2(x+3y)-f^2(x+y)) + (f^2(x+y)-f^2(x-y)=4f(x+2y)f(y) + 4f(x)f(y) \text{      } 
 \textbf{(4)}$$
by $\textbf{(2)}$ and $\textbf{(4)}$  $$f(x+y)f(2y)=(f(x+2y)+f(x))f(y) \text{                  }  \textbf{(5)}$$
$f(x)=0$ is a solution. Now assume $f$ is not a constant function. 
Let y such that $f(y)\neq 0$ then by $\textbf{(1)}$ $f(2y)\neq0$. 
For $x=y$ in $\textbf{(5)}$ $f(3y)=3f(y)$. For $x=3y$  $f(4y)=2f(2y)$ so $f(2y)=2f(y)$ if $f(y)\neq0$, but that is also true if $f(y)=0$ by (1) hence $f(2y)=2f(y)$for all y.
If $f(y)\neq0$ then 
$$f(x)+f(x+2y)=2f(x+y)  \textbf{(6)}$$
$$f(2x)+f(2x+2y)=2f(2x+y) f(x+y)+f(x)=f(2x+y)$$
if $f(x)\neq0$ then by $ \textbf{(6)}$
$f(y)+f(y+2x)=2f(x+y)$ hence $f(x+y)=f(x)+f(y) \text{                  } \textbf{(7)}$ for all x,y st $f(x)f(y)\neq 0$
Assume $f(a)=0$, then by $(1)$    $f(a\/ 2)=0$
then $$f^2(x+a\/2)=f^2(x-a\/2)$$ 
$$x=z+a\/2 y=z-a\/2$$
$$f^2(2z)=4f(z-a\/2)f(z+a\/2) \geq 0$$ hence for all y $f(y+a\/2)=f(y+a\/2)$ so f is a-periodic , hence $f(x)+f(a)=f(x+a)$ which means $\textbf{(7)} $holds for all x,y hence f is additive.
QED
\end{solution}



\begin{solution}[by \href{https://artofproblemsolving.com/community/user/277552}{WizardMath}]
	\begin{tcolorbox}[quote=georgeado17]$f:\mathbb{R} \to \mathbb{R}$ $x,y\in R$
$f^2(x+y)-f^2(x-y)=4f(x)f(y)$
$f(x)=?$\end{tcolorbox}
\begin{bolded}Note:\end{bolded} I'm assuming that $f^2(x) = f(f(x))$
[hide = My attempt]


[hide = Some initial results]
$P(0,0) = f^2(0)-f^2(0) = 0 = f(0)^2 \implies f(0)=0$
$P(x,x) = f^2(2x) = 4f(x)^2$
$P(0,y) = f^2(y) - f^2(-y) = 0 \implies f^2(y) = f^2(-y)$
$P(x,-x) = -f^2(-2x) = 4f(x)f(-x) = -f^2(2x)$
[\/hide]

\begin{bolded}Lemma 1.1\end{bolded} $f(-x) = -f(x) \forall x \in \mathbb{R}$
[hide= Proof]
$P(x,x)+P(x,-x) = f^2(2x)-f^2(2x) = 0 = 4f(x)(f(x)+f(-x)) \implies f(x)=0$ or $f(-x)= -f(x)$ for each $x\in \mathbb{R}$
[\/hide]

Lemma 1.2- The only value that points to $0$ is $0$
[hide=Proof]

Let $u$ be a real number such that $f(u)=0$ and $u\not = 0$, then
$P(x-u,u) \implies f^2(x) = f^2(x-2u)$ which implies that either $u=0$ is the only number such that $f(u)=0$ or the function is peridic (at least every 2u) or constant.

Asume that the minimum period is $T$.

$f^2(2(x+\frac{T}{2})) = f^2(2x+T) = f^2(2x)= 4f(x)^2 = 4f(x+\frac{T}{2})^2 \forall x \in \mathbb{R}$ which contradicts the minimalty of $T$ implying the new and less period  $\frac{T}{2}$. So $T = 0$. And so the only $u$ such that $f(u)=0$ is $0$. We also can now say that $f(-x)=-f(x)$.
[\/hide]

Result-$f(x) = 0 \forall x \in \mathbb{R}$
[Hide= Proof]
We also have that:
$P(0,y) = f^2(y) - f^2(-y) = 0 \implies f^2(y) = f^2(-y) \implies f(f(y))= f(f(-y)) = f(-f(y)) = -f(f(y)) \implies f(f(y)) = 0 \forall y \in \mathbb{R}$

From this the assertion changes to $P(x,y) = f(x)f(y)= 0$. From which is easy to conclude that $f(x) = 0 \forall x\in \mathbb{R}$
[\/hide]
 
[\/hide]\end{tcolorbox}

A minimum period need not exist. To use a minimum period you need to show that one exists. 
\end{solution}



\begin{solution}[by \href{https://artofproblemsolving.com/community/user/309179}{whiwho}]
	\begin{tcolorbox}[quote=whiwho][quote=georgeado17]$f:\mathbb{R} \to \mathbb{R}$ $x,y\in R$
$f^2(x+y)-f^2(x-y)=4f(x)f(y)$
$f(x)=?$\end{tcolorbox}
\begin{bolded}Note:\end{bolded} I'm assuming that $f^2(x) = f(f(x))$
[hide = My attempt]


[hide = Some initial results]
$P(0,0) = f^2(0)-f^2(0) = 0 = f(0)^2 \implies f(0)=0$
$P(x,x) = f^2(2x) = 4f(x)^2$
$P(0,y) = f^2(y) - f^2(-y) = 0 \implies f^2(y) = f^2(-y)$
$P(x,-x) = -f^2(-2x) = 4f(x)f(-x) = -f^2(2x)$
[\/hide]

\begin{bolded}Lemma 1.1\end{bolded} $f(-x) = -f(x) \forall x \in \mathbb{R}$
[hide= Proof]
$P(x,x)+P(x,-x) = f^2(2x)-f^2(2x) = 0 = 4f(x)(f(x)+f(-x)) \implies f(x)=0$ or $f(-x)= -f(x)$ for each $x\in \mathbb{R}$
[\/hide]

Lemma 1.2- The only value that points to $0$ is $0$
[hide=Proof]

Let $u$ be a real number such that $f(u)=0$ and $u\not = 0$, then
$P(x-u,u) \implies f^2(x) = f^2(x-2u)$ which implies that either $u=0$ is the only number such that $f(u)=0$ or the function is peridic (at least every 2u) or constant.

Asume that the minimum period is $T$.

$f^2(2(x+\frac{T}{2})) = f^2(2x+T) = f^2(2x)= 4f(x)^2 = 4f(x+\frac{T}{2})^2 \forall x \in \mathbb{R}$ which contradicts the minimalty of $T$ implying the new and less period  $\frac{T}{2}$. So $T = 0$. And so the only $u$ such that $f(u)=0$ is $0$. We also can now say that $f(-x)=-f(x)$.
[\/hide]

Result-$f(x) = 0 \forall x \in \mathbb{R}$
[Hide= Proof]
We also have that:
$P(0,y) = f^2(y) - f^2(-y) = 0 \implies f^2(y) = f^2(-y) \implies f(f(y))= f(f(-y)) = f(-f(y)) = -f(f(y)) \implies f(f(y)) = 0 \forall y \in \mathbb{R}$

From this the assertion changes to $P(x,y) = f(x)f(y)= 0$. From which is easy to conclude that $f(x) = 0 \forall x\in \mathbb{R}$
[\/hide]
 
[\/hide]\end{tcolorbox}

A minimum period need not exist. To use a minimum period you need to show that one exists.\end{tcolorbox}

$f^2(x)=f^2(x+2u)$ right?
\end{solution}



\begin{solution}[by \href{https://artofproblemsolving.com/community/user/277552}{WizardMath}]
	@above, what I meant was that you need to show that a minimum period exists. Not an ordinary period.  
\end{solution}



\begin{solution}[by \href{https://artofproblemsolving.com/community/user/313451}{pro_4_ever}]
	\begin{tcolorbox}@above, what I meant was that you need to show that a minimum period exists. Not an ordinary period.\end{tcolorbox}

Isn't period supposed to be the Minimum? 
I do understand the contextual meaning, just clarifying a doubt.
Thanks
\end{solution}



\begin{solution}[by \href{https://artofproblemsolving.com/community/user/403767}{mkhayech}]
	\begin{tcolorbox}[quote=WizardMath]@above, what I meant was that you need to show that a minimum period exists. Not an ordinary period.\end{tcolorbox}

Isn't period supposed to be the Minimum? 
I do understand the contextual meaning, just clarifying a doubt.
Thanks\end{tcolorbox}

Not all functions have a minimum period.
\end{solution}



\begin{solution}[by \href{https://artofproblemsolving.com/community/user/309179}{whiwho}]
	\begin{tcolorbox}[quote=whiwho][quote=georgeado17]$f:\mathbb{R} \to \mathbb{R}$ $x,y\in R$
$f^2(x+y)-f^2(x-y)=4f(x)f(y)$
$f(x)=?$\end{tcolorbox}
\begin{bolded}Note:\end{bolded} I'm assuming that $f^2(x) = f(f(x))$
[hide = My attempt]


[hide = Some initial results]
$P(0,0) = f^2(0)-f^2(0) = 0 = f(0)^2 \implies f(0)=0$
$P(x,x) = f^2(2x) = 4f(x)^2$
$P(0,y) = f^2(y) - f^2(-y) = 0 \implies f^2(y) = f^2(-y)$
$P(x,-x) = -f^2(-2x) = 4f(x)f(-x) = -f^2(2x)$
[\/hide]

\begin{bolded}Lemma 1.1\end{bolded} $f(-x) = -f(x) \forall x \in \mathbb{R}$
[hide= Proof]
$P(x,x)+P(x,-x) = f^2(2x)-f^2(2x) = 0 = 4f(x)(f(x)+f(-x)) \implies f(x)=0$ or $f(-x)= -f(x)$ for each $x\in \mathbb{R}$
[\/hide]

Lemma 1.2- The only value that points to $0$ is $0$
[hide=Proof]

Let $u$ be a real number such that $f(u)=0$ and $u\not = 0$, then
$P(x-u,u) \implies f^2(x) = f^2(x-2u)$ which implies that either $u=0$ is the only number such that $f(u)=0$ or the function is peridic (at least every 2u) or constant.

Asume that the minimum period is $T$.

$f^2(2(x+\frac{T}{2})) = f^2(2x+T) = f^2(2x)= 4f(x)^2 = 4f(x+\frac{T}{2})^2 \forall x \in \mathbb{R}$ which contradicts the minimalty of $T$ implying the new and less period  $\frac{T}{2}$. So $T = 0$. And so the only $u$ such that $f(u)=0$ is $0$. We also can now say that $f(-x)=-f(x)$.
[\/hide]

Result-$f(x) = 0 \forall x \in \mathbb{R}$
[Hide= Proof]
We also have that:
$P(0,y) = f^2(y) - f^2(-y) = 0 \implies f^2(y) = f^2(-y) \implies f(f(y))= f(f(-y)) = f(-f(y)) = -f(f(y)) \implies f(f(y)) = 0 \forall y \in \mathbb{R}$

From this the assertion changes to $P(x,y) = f(x)f(y)= 0$. From which is easy to conclude that $f(x) = 0 \forall x\in \mathbb{R}$
[\/hide]
 
[\/hide]\end{tcolorbox}

A minimum period need not exist. To use a minimum period you need to show that one exists.\end{tcolorbox}

But I said "Asume that a minimum period exists" and then I prove that it can't exist.
\end{solution}



\begin{solution}[by \href{https://artofproblemsolving.com/community/user/403767}{mkhayech}]
	Can anyone check if my solution is correct?
\end{solution}



\begin{solution}[by \href{https://artofproblemsolving.com/community/user/398616}{MNJ2357}]
	\begin{tcolorbox}Asume that a minimum period exists\end{tcolorbox}
You can not assume that a minimum period exists. For example, $$f(x)= \begin{cases} 0 & \text{if } x \notin \mathbb{Q} \\ 1 &\text{if } x \in \mathbb{Q} \end{cases} $$
has a period, but not a minimum one. Also, if $T$is a period, $\frac{T}{2}$ is also a period, while 0 is not a period.
\end{solution}
*******************************************************************************
-------------------------------------------------------------------------------

\begin{problem}[Posted by \href{https://artofproblemsolving.com/community/user/398306}{s2hasafshl}]
	find all the funtion $ f(x) \in R $ such that $ f(x^2 + f(xy))= x. f(x+y) $ with all real numbers x,y
	\flushright \href{https://artofproblemsolving.com/community/c6h1602769}{(Link to AoPS)}
\end{problem}



\begin{solution}[by \href{https://artofproblemsolving.com/community/user/29428}{pco}]
	\begin{tcolorbox}find all the funtion $ f(x) \in R $ such that $ f(x^2 + f(xy))= x. f(x+y) $ with all real numbers x,y\end{tcolorbox}
Let $P(x,y)$ be the assertion $f(x^2+f(xy))=xf(x+y)$

Subtracting $P(1,-1)$ from $P(-1,1)$, we get $f(0)=0$
Subtracting $P(x,0)$ from $P(-x,0)$, we get $f(-x)=-f(x)$ $\forall x\ne 0$, still true when $x=0$
So $f(x)$ is an odd function

Suppose then $\exists u\ne 0$ such that $f(u)=0$
Let $x\ne 0$
Subtracting $P(-x^2,0)$ from $P(-x^2,-\frac u{x^2})$, we get $f(-x^2-\frac u{x^2})=f(-x^2)$
subtracting $P(x,-x)$ from $P(x,-x-\frac u{x^3})$, we get $f(-\frac u{x^3})=0$

And so $\boxed{\text{S1 : }f(x)=0\quad\forall x}$ which indeed is a solution

So let us from now consider that $f(u)=0$ implies $u=0$
$P(x,-x)$ $\implies$ $f(x^2-f(x^2))=0$ and so $f(x^2)=x^2$
So $f(x)=x$ $\forall x\ge 0$ and since $f(x)$ is odd :

$\boxed{\text{S2 : }f(x)=x\quad\forall x}$ which indeed is a solution

\end{solution}
*******************************************************************************
-------------------------------------------------------------------------------

\begin{problem}[Posted by \href{https://artofproblemsolving.com/community/user/377375}{Probowldan}]
	Find all functions $f: \mathbb{R} \to \mathbb{R}$ such that
\[f(xf(x) + f(y)) = \left ( f(x) \right )^2 + y\]
	\flushright \href{https://artofproblemsolving.com/community/c6h1603339}{(Link to AoPS)}
\end{problem}



\begin{solution}[by \href{https://artofproblemsolving.com/community/user/29428}{pco}]
	\begin{tcolorbox}Find all functions $f: \mathbb{R} \to \mathbb{R}$ such that
\[f(xf(x) + f(y)) = \left ( f(x) \right )^2 + y\]\end{tcolorbox}
Posted (and solved) many many many times (2005, 2007, 2008, 2010, 2016, 2017, 2018 - less than two monthes ago -).

Dont hesitate to use the search function (see [url=http://www.artofproblemsolving.com\/community\/c6h1330604p7173058] here [\/url]).
Set (for example copy\/paste) in the "search term" field the exact following string : 

+"f(xf(x)+f(y))"+"f(x)^2+y"

You'll get in the \begin{bolded}ten first results\end{underlined}\end{bolded} (excluded your own post and this post itself) all the help you are requesting for.

[hide=(Some excuses)][size=70]I'm sorry not providing you the direct link to this result but I encountered users who never tried the search function, thinking quite easier to have other users make the search for them. So now I prefer to point to the search function and to give the appropriate search term (I checked that it indeed will give you the expected result) instead of the link itself [\/size]([url=https:\/\/en.wiktionary.org\/wiki\/give_a_man_a_fish_and_you_feed_him_for_a_day;_teach_a_man_to_fish_and_you_feed_him_for_a_lifetime]wink[\/url])[\/hide]





\end{solution}
*******************************************************************************
-------------------------------------------------------------------------------

\begin{problem}[Posted by \href{https://artofproblemsolving.com/community/user/404071}{thubup}]
	f(n+f(n))=f(n) and f(x0)=a
	\flushright \href{https://artofproblemsolving.com/community/c6h1604376}{(Link to AoPS)}
\end{problem}



\begin{solution}[by \href{https://artofproblemsolving.com/community/user/337737}{Evenprime123}]
	\begin{tcolorbox}f(x0)=a\end{tcolorbox}
What do you mean by this?
And also, is it in \(\mathbb{N}, \mathbb{R}, ...\)?
\end{solution}



\begin{solution}[by \href{https://artofproblemsolving.com/community/user/29428}{pco}]
	\begin{tcolorbox}f(n+f(n))=f(n) and f(x0)=a\end{tcolorbox}
Let us add the precisions :
i) $f(x)$ is a function from $A\to A$ with $0\in A$ and $A$ stable by addition and subtraction
ii) $x_0,a\in A$

Then \begin{bolded}general solution is\end{underlined}\end{bolded} :
Let $U$ any additive subgroup of $A$ such that $a\in U$
Let $\sim$ the equivalence relation over $A$ defined as $x\sim y\iff x-y\in U$
Let $r(x)$ any function from $A\to A$ associating to an element of $A$ a representant (unique per class) of its equivalence class.
Let $g(x)$ any function from $A\to U$

Then $f(x)=g(r(x))+a-g(r(x_0))$
\end{solution}
*******************************************************************************
-------------------------------------------------------------------------------

\begin{problem}[Posted by \href{https://artofproblemsolving.com/community/user/315979}{LGKOm1}]
	Find every continuous function such that :
\[f :R^{+}\rightarrow R\]
\[f(x+\frac{1}{y})+f(y+\frac{1}{x})=f(x+\frac{1}{x})+f(y+\frac{1}{y})\]
 
	\flushright \href{https://artofproblemsolving.com/community/c6h1605378}{(Link to AoPS)}
\end{problem}



\begin{solution}[by \href{https://artofproblemsolving.com/community/user/315979}{LGKOm1}]
	can anyone help please?

\end{solution}



\begin{solution}[by \href{https://artofproblemsolving.com/community/user/29428}{pco}]
	\begin{tcolorbox}Find every continuous function such that :
\[f :R^{+}\rightarrow R\]
\[f(x+\frac{1}{y})+f(y+\frac{1}{x})=f(x+\frac{1}{x})+f(y+\frac{1}{y})\]\end{tcolorbox}
Already posted (and solved) at least once in 2013.

Dont hesitate to use the search function (see [url=http://www.artofproblemsolving.com\/community\/c6h1330604p7173058] here [\/url]).
Set (for example copy\/paste) in the "search term" field the exact following string : 

+"f(x+\frac 1x)" +"f(y+\frac 1y)"


You'll get in the \begin{bolded}second \end{underlined}\end{bolded}result (excluded your own post and this post itself) the exact help you are requesting for.

[hide=(Some excuses)][size=70]I'm sorry not providing you the direct link to this result but I encountered users who never tried the search function, thinking quite easier to have other users make the search for them. So now I prefer to point to the search function and to give the appropriate search term (I checked that it indeed will give you the expected result) instead of the link itself [\/size]([url=https:\/\/en.wiktionary.org\/wiki\/give_a_man_a_fish_and_you_feed_him_for_a_day;_teach_a_man_to_fish_and_you_feed_him_for_a_lifetime]wink[\/url])[\/hide]







\end{solution}



\begin{solution}[by \href{https://artofproblemsolving.com/community/user/391068}{TuZo}]
	\begin{tcolorbox}Find every continuous function such that :
\[f :R^{+}\rightarrow R\]
\[f(x+\frac{1}{y})+f(y+\frac{1}{x})=f(x+\frac{1}{x})+f(y+\frac{1}{y})\]\end{tcolorbox}

\begin{bolded}The mysterious post from 2013 is here:\end{bolded}
[url]https:\/\/artofproblemsolving.com\/community\/c6h566767[\/url]
\begin{bolded}This is a better help!\end{bolded}
\end{solution}
*******************************************************************************
-------------------------------------------------------------------------------

\begin{problem}[Posted by \href{https://artofproblemsolving.com/community/user/330031}{knm2608}]
	Prove that there exists no function $f: \mathbb{R} \to \mathbb{R}$ satisfying
$$f(x+y)=e^xf(y)+e^yf(x)+xy \;\; \forall x,y\in  \mathbb{R}$$
	\flushright \href{https://artofproblemsolving.com/community/c6h1605422}{(Link to AoPS)}
\end{problem}



\begin{solution}[by \href{https://artofproblemsolving.com/community/user/350483}{adhikariprajitraj}]
	\begin{tcolorbox}Prove that the exists no function $f: \mathbb{R} \to \mathbb{R}$ satisfying
$$f(x+y)=e^xf(y)+e^yf(x)+xy \;\; \forall x,y\in  \mathbb{R}$$\end{tcolorbox}

Expecting @pco,@tuzo(sir),etc,etc to answer this!:-D :D
\end{solution}



\begin{solution}[by \href{https://artofproblemsolving.com/community/user/391068}{TuZo}]
	Hint: denote $g(x)=\frac{f(x)}{e^{x}},$we have $g(x+y)=g(x)+g(y)+\frac{xy}{e^{x+y}}$
\end{solution}



\begin{solution}[by \href{https://artofproblemsolving.com/community/user/350483}{adhikariprajitraj}]
	@below has solution. :D
\end{solution}



\begin{solution}[by \href{https://artofproblemsolving.com/community/user/297382}{Filipjack}]
	\begin{tcolorbox}Prove that the exists no function $f: \mathbb{R} \to \mathbb{R}$ satisfying
$$f(x+y)=e^xf(y)+e^yf(x)+xy \;\; \forall x,y\in  \mathbb{R}$$\end{tcolorbox}

Let $g(x)=\frac{f(x)}{e^x}$ and the relation becomes $$P(x,y):~~~g(x+y)=g(x)+g(y)+ \frac{xy}{e^{x+y}} ~\forall~ x,y \in \mathbb{R}.$$
From $P(1,3),~P(2,2),~P(1,1),~P(1,2)$ we get $$g(4)=g(1)+g(3)+ \frac{3}{e^4}~~~(1)$$ $$g(4)=2g(2)+ \frac{4}{e^4}~~~(2)$$ $$g(2)=2g(1)+ \frac{1}{e^2}~~~(3)$$ $$g(3)=g(1)+g(2)+ \frac{2}{e^3}~~~(4)$$
From $(2)$ and $(3)$ we get $g(4)=4g(1)+ \frac{2}{e^2}+ \frac{4}{e^4}.~~~(5)$
From $(3)$ and $(4)$ we get $g(3)=3g(1)+ \frac{1}{e^2}+ \frac{2}{e^3}.~~~(6)$
From $(1)$ and $(6)$ we get $g(4)=4g(1)+ \frac{1}{e^2}+ \frac{2}{e^3}+ \frac{3}{e^4}.~~~(7)$
From $(5)$ and $(7)$ we get $\frac{1}{e^2}+ \frac{1}{e^4}= \frac{2}{e^3} \Leftrightarrow \left( \frac{1}{e}- \frac{1}{e^2} \right)^2=0,$ which is false. 
In conclusion there are no functions $g$ and so there are no functions $f$ with the given property.
\end{solution}



\begin{solution}[by \href{https://artofproblemsolving.com/community/user/253232}{Pure_IQ}]
	Denote $g(x)=\frac{f(x)}{e^x}$. We get : $g(x+y)=g(x)+g(y)+\frac{xy}{e^{x+y}}, \forall x,y \in R$.

$P(0,0)$ : $g(0)=0$;

$P(x,-x)$ : $g(x)+g(-x)=x^2$;

$P(-x,-y)$ : $e^{x+y}+\frac{1}{e^{x+y}}=2 \Leftrightarrow e^{x+y}=1, \forall x,y \in R^*$, which is clearly impossible.

\end{solution}



\begin{solution}[by \href{https://artofproblemsolving.com/community/user/350483}{adhikariprajitraj}]
	@above what if $x+y=0$.
\end{solution}



\begin{solution}[by \href{https://artofproblemsolving.com/community/user/376213}{Wizard_32}]
	\begin{tcolorbox}@above what if $x+y=0$.\end{tcolorbox}
$x,y \in R^+$ so...

\end{solution}



\begin{solution}[by \href{https://artofproblemsolving.com/community/user/391068}{TuZo}]
	\begin{tcolorbox}Denote $g(x)=\frac{f(x)}{e^x}$. We get : $g(x+y)=g(x)+g(y)+\frac{xy}{e^{x+y}}, \forall x,y \in R$.

$P(0,0)$ : $g(0)=0$;

$P(x,-x)$ : $g(x)+g(-x)=x^2$;

$P(-x,-y)$ : $e^{x+y}+\frac{1}{e^{x+y}}=2 \Leftrightarrow e^{x+y}=1, \forall x,y \in R^*$, which is clearly impossible.\end{tcolorbox}

Splendid solution! Congratulation!
\end{solution}



\begin{solution}[by \href{https://artofproblemsolving.com/community/user/391068}{TuZo}]
	\begin{tcolorbox}@above what if $x+y=0$.\end{tcolorbox}

This is for ALL $x,y$ (for example $x=y=1$  ;) 
\end{solution}



\begin{solution}[by \href{https://artofproblemsolving.com/community/user/29428}{pco}]
	\begin{tcolorbox}Prove that there exists no function $f: \mathbb{R} \to \mathbb{R}$ satisfying
$$f(x+y)=e^xf(y)+e^yf(x)+xy \;\; \forall x,y\in  \mathbb{R}$$\end{tcolorbox}
$f(x+y+z)=e^x(e^yf(z)+e^zf(y)+yz)+e^{y+z}f(x)+x(y+z)$
$=e^{x+y}f(z)+e^{x+z}f(y)+e^{y+z}f(x)+xy+xz+e^xyz$
Swapping $x,y$ and subtracting, we  get : $yz+e^yxz=xz+e^xyz$ $\forall x,y,z$
Clearly wrong


\end{solution}



\begin{solution}[by \href{https://artofproblemsolving.com/community/user/350483}{adhikariprajitraj}]
	\begin{tcolorbox}[quote=knm2608]Prove that there exists no function $f: \mathbb{R} \to \mathbb{R}$ satisfying
$$f(x+y)=e^xf(y)+e^yf(x)+xy \;\; \forall x,y\in  \mathbb{R}$$\end{tcolorbox}
$f(x+y+z)=e^x(e^yf(z)+e^zf(y)+yz)+e^{y+z}f(x)+x(y+z)$
$=e^{x+y}f(z)+e^{x+z}f(y)+e^{y+z}f(x)+xy+xz+e^xyz$
Swapping $x,y$ and subtracting, we  get : $yz+e^yxz=xz+e^xyz$ $\forall x,y,z$
Clearly wrong\end{tcolorbox}

Sir, this can be falsified if you consider $x=y=z$. :D
\end{solution}



\begin{solution}[by \href{https://artofproblemsolving.com/community/user/29428}{pco}]
	\begin{tcolorbox}Sir, this can be falsified if you consider $x=y=z$. :D\end{tcolorbox}
Ok. Sorry.
In my old knowledge (many years ago), the sentence $\forall x,y,z$ meant "for all values of $x,y,z$

I understand thru your post that today meaning is different.
I'm very sorry to have confused you.

Would you be kind enough to explain an old man what is the current method (symbol) for writing "for any values of variables $x,y,z$" ?

Thanks in advance.


\end{solution}



\begin{solution}[by \href{https://artofproblemsolving.com/community/user/350483}{adhikariprajitraj}]
	\begin{tcolorbox}[quote=adhikariprajitraj]Sir, this can be falsified if you consider $x=y=z$. :D\end{tcolorbox}
Ok. Sorry.
In my old knowledge (many years ago), the sentence $\forall x,y,z$ meant "for all values of $x,y,z$

I understand thru your post that today meaning is different.
I'm very sorry to have confused you.

Would you be kind enough to explain an old man what is the current method (symbol) for writing "for any values of variables $x,y,z$" ?

Thanks in advance.\end{tcolorbox}

Is that supposed to be a satire or thanksgiving statement, I am confused?
\end{solution}



\begin{solution}[by \href{https://artofproblemsolving.com/community/user/29428}{pco}]
	\begin{tcolorbox}Is that supposed to be a satire or thanksgiving statement, I am confused?\end{tcolorbox}
If expression "$\forall x,y,z$", written in my original post, currently still means "for all x,y,z", this is clearly a satire
If expression "$\forall x,y,z$", written in my original post, no longer means "for all x,y,z", this is not a satire, but apologies and humble question.




\end{solution}



\begin{solution}[by \href{https://artofproblemsolving.com/community/user/350483}{adhikariprajitraj}]
	\begin{tcolorbox}[quote=adhikariprajitraj]Is that supposed to be a satire or thanksgiving statement, I am confused?\end{tcolorbox}
If expression "$\forall x,y,z$", written in my original post, currently still means "for all x,y,z", this is clearly a satire
If expression "$\forall x,y,z$", written in my original post, no longer means "for all x,y,z", this is not a satire, but apologies and humble question.\end{tcolorbox}

Savage LOL! Yeah that still means same! Sorry, for the inconvenience!

\end{solution}
*******************************************************************************
-------------------------------------------------------------------------------

\begin{problem}[Posted by \href{https://artofproblemsolving.com/community/user/376542}{falantrng}]
	Find all functions $f:(0,+\infty)\to \mathbb{R}$
$f(y)>(y-x)f^2(x)$ where $y>x>0.$
	\flushright \href{https://artofproblemsolving.com/community/c6h1606063}{(Link to AoPS)}
\end{problem}



\begin{solution}[by \href{https://artofproblemsolving.com/community/user/238386}{ythomashu}]
	that's not a functional equation
\end{solution}



\begin{solution}[by \href{https://artofproblemsolving.com/community/user/29428}{pco}]
	\begin{tcolorbox}Find all functions $f:(0,+\infty)\to \mathbb{R}$
$f(y)>(y-x)f^2(x)$ where $y>x>0.$\end{tcolorbox}

Could you precise your writing ? Is $f^2(x)$ :
$f(f(x))$ or $(f(x))^2$ ?

\end{solution}



\begin{solution}[by \href{https://artofproblemsolving.com/community/user/391068}{TuZo}]
	\begin{tcolorbox}that's not a functional equation\end{tcolorbox}

This is a functional inequation!
\end{solution}



\begin{solution}[by \href{https://artofproblemsolving.com/community/user/376542}{falantrng}]
	\begin{tcolorbox}[quote=falantrng]Find all functions $f:(0,+\infty)\to \mathbb{R}$
$f(y)>(y-x)f^2(x)$ where $y>x>0.$\end{tcolorbox}

Could you precise your writing ? Is $f^2(x)$ :
$f(f(x))$ or $(f(x))^2$ ?\end{tcolorbox}

$f^2(x)=(f(x))^2.$
\end{solution}



\begin{solution}[by \href{https://artofproblemsolving.com/community/user/29428}{pco}]
	\begin{tcolorbox}Find all functions $f:(0,+\infty)\to \mathbb{R}$
$f(y)>(y-x)f^2(x)$ where $y>x>0.$\end{tcolorbox}
\begin{tcolorbox}$f^2(x)=(f(x))^2.$\end{tcolorbox}

If so, see http://artofproblemsolving.com\/community\/c6h350187p2195376



\end{solution}
*******************************************************************************
-------------------------------------------------------------------------------

\begin{problem}[Posted by \href{https://artofproblemsolving.com/community/user/313022}{FEcreater}]
	Find all functions $ f: \mathbb{R} \to \mathbb{R} $ such that $$ f\left(f\left(x+y\right)-x\right)f\left(f\left(x+y\right)-y\right) = xy $$ holds for all $ x,y \in \mathbb{R} $
	\flushright \href{https://artofproblemsolving.com/community/c6h1606397}{(Link to AoPS)}
\end{problem}



\begin{solution}[by \href{https://artofproblemsolving.com/community/user/29428}{pco}]
	\begin{tcolorbox}Find all functions $ f: \mathbb{R} \to \mathbb{R} $ such that $$ f\left(f\left(x+y\right)-x\right)f\left(f\left(x+y\right)-y\right) = xy $$ holds for all $ x,y \in \mathbb{R} $\end{tcolorbox}
Let $P(x,y)$ be the assertion $f(f(x+y)-x)f(f(x+y)-y)=xy$

If $\exists u$ such that $f(u)=0$ :
$P(-u,2u)$ $\implies$ $u=0$

$P(x,0)$ $\implies$ $f(f(x)-x)f(f(x))=0$ and so either $f(x)=x$, either $x=0$ and $f(0)=0$

And so $\boxed{f(x)=x\quad\forall x}$ which indeed is a solution.


\end{solution}
*******************************************************************************
-------------------------------------------------------------------------------

\begin{problem}[Posted by \href{https://artofproblemsolving.com/community/user/313022}{FEcreater}]
	Find all functions $ f: \mathbb{R} \to \mathbb{R} $ satisfying $$ f\left(f\left(x+f\left(y\right)\right)-y-f\left(x\right)\right) = xf\left(y\right)-yf\left(x\right) $$ for all real numbers $ x,y $
	\flushright \href{https://artofproblemsolving.com/community/c6h1606398}{(Link to AoPS)}
\end{problem}



\begin{solution}[by \href{https://artofproblemsolving.com/community/user/274926}{rmtf1111}]
	[hide=wrong]
Let $P(x,y)$ be the given assertion. Suppose that $f$ is non-constant. From $P(0,0)$ we have that there exists $u$ such that $f(u)=0$. 
$$P(x,u)-P(0,u) \implies f(-u)=-uf(x)=-uf(0) \implies \boxed{f(u)=0 \Longleftrightarrow u=0}$$
$$P(0,x) \implies f(f(f(x))-x)=0 \implies f(f(x))=x$$
$$P(x,x) \implies f(x+f(x))=x+f(x)$$ 
$$P(x+f(x),y+f(y)) \implies f(x+y+f(x)+f(y))=0 \implies f(f(x)+f(-x))=0 \implies f(-x)=-f(x) \ \ \ (1)$$
$$P(f(-x),x) \implies f(x)f(-x)+x^2=0 \implies f(f(x))f(-f(x))+f(x)^2=0 \implies xf(-f(x))+f(x)^2=0 $$
$$P(x,-f(x)) \implies f(f(x+f(-x)))=xf(-f(x))+f(x)^2=0 \implies x+f(-f(x))=0 \implies f(x)^2=x^2 \implies f(x)=\pm x$$
Now suppose there exist non-zero reals $x_1$ and $x_2$ such that$f(x_1)=x_1$ and $f(x_2)=-x_2$, plug in $x=x_1+x_2$ and $y=x_2$ and now by breaking down it into two cases $f(x_1+x_2)=x_1+x_2$ and $f(x_1+x_2)=-(x_1+x_2)$, we obtain that either one of $x_1$ or $x_2$ is equal to $0$ or $x_1+x_2=\pm 1$, both cases leading to a contradiction. The only solutions are: $f(x)=x$, $f(x)=-x$ or $f(x)=0$ for any real $x$.
[\/hide]
@below: Thanks, will try to fix it
\end{solution}



\begin{solution}[by \href{https://artofproblemsolving.com/community/user/29428}{pco}]
	\begin{tcolorbox}$P(x+f(x),y+f(y)) \implies f(x+y+f(x)+f(y))=0$.\end{tcolorbox}
No.
$P(x+f(x),y+f(y)) \implies f(x+y+f(x)+f(y))=x+y+f(x)+f(y)$

\end{solution}



\begin{solution}[by \href{https://artofproblemsolving.com/community/user/390214}{R8450932}]
	[hide=Hmm easy one]
Let $P (x,y) $ be the given fe.
Firstly $P (x,x) $ yields that there exist $a\in R$ s.t $f (a)=0$.Then $P (x,a) $ gives that either $a=0$ or $f $ is constant.
Let's do easy part first
1)f is constant $\implies $ $f\equiv 0$.
2) $a=0$:
So $f (0)=0$ and plug $P (0,x) $ yields that $f(f (x))=x $.
$P (f (x),-x) $ $\implies $ $f (x)+f (-x)=f (x)f (-x)+x^2$        $(1) $
$P (x,f (-x)) $ gives $f (-f (x)-f (-x))=-f (x)-f (-x) $
The key pluggings are coming:
$P (-x,x+f (x))$ $\implies $ $(x+f (x))(-x-f (-x))=-(f (x)+f (-x)) $ .By (1) we easily get $f $ is odd.
$P (x+f (x),-x)$ gives $f (x)\in {x,-x} $.
Let's assume there exist $a,b \neq 0$ s.t $f (a)=a $ and $f (b)=-b $.Use this $P (a,f (b)) $ to get contradiction so there are 3 solutions 
1)$f\equiv 0$
2)$f(x)=x $                     All of them work.So done.
3) $f (x)=-x $ 

\end{solution}



\begin{solution}[by \href{https://artofproblemsolving.com/community/user/29428}{pco}]
	\begin{tcolorbox}... .By (1) we easily get $f $ is odd...\end{tcolorbox}
How ?

\end{solution}



\begin{solution}[by \href{https://artofproblemsolving.com/community/user/390214}{R8450932}]
	We got $(x+f (x))(-x-f (-x))=-(f (x)+f (-x)) $ but we also have  $f (x)+f (-x)=f (x)f (-x)+x^2$ which easily implies $f (x)=-f (-x) $
\end{solution}



\begin{solution}[by \href{https://artofproblemsolving.com/community/user/29428}{pco}]
	\begin{tcolorbox}We got $(x+f (x))(-x-f (-x))=-(f (x)+f (-x)) $ but we also have  $f (x)+f (-x)=f (x)f (-x)+x^2$ which easily implies $f (x)=-f (-x) $\end{tcolorbox}

Yep, clear,
Thanks

\end{solution}
*******************************************************************************
-------------------------------------------------------------------------------

\begin{problem}[Posted by \href{https://artofproblemsolving.com/community/user/346854}{ShR}]
	Find all continuous functions $ f (f (x+y))=f (x)+f (y)$
if $f:R =>R$
	\flushright \href{https://artofproblemsolving.com/community/c6h1606481}{(Link to AoPS)}
\end{problem}



\begin{solution}[by \href{https://artofproblemsolving.com/community/user/29428}{pco}]
	\begin{tcolorbox}Find all continuous functions $ f (f (x+y))=f (x)+f (y)$
if $f:R =>R$\end{tcolorbox}
Let $P(x,y)$ be the assertion $f(f(x+y))=f(x)+f(y)$
Let $a=f(0)$

Subtracting $P(x,y)$ from $P(x+y,0)$, we get $f(x+y)-a=f(x)-a+f(y)-a$
And so $f(x)-a$ is additive and so, since continuous, linear.

Plugging $f(x)=cx+a$ in original equation, we get :
$\boxed{\text{S1 : }f(x)=0\quad\forall x}$

$\boxed{\text{S2 : }f(x)=x+a}$ whatever is $a\in\mathbb R$


\end{solution}



\begin{solution}[by \href{https://artofproblemsolving.com/community/user/346854}{ShR}]
	

Subtracting $P(x,y)$ from $P(x+y,0)$, we get $f(x+y)-a=f(x)-a+f(y)-a$  please explain it please 
\end{solution}



\begin{solution}[by \href{https://artofproblemsolving.com/community/user/401024}{furioushatter}]
	\begin{tcolorbox}Subtracting $P(x,y)$ from $P(x+y,0)$, we get $f(x+y)-a=f(x)-a+f(y)-a$  please explain it please\end{tcolorbox}

$P(x+y, 0)$ gives $f(f(x+y))=f(x+y)+a$. $P(x, y)$ gives $f(f(x+y))=f(x)+f(y)$. This means $f(x+y)+a=f(x)+f(y)$, then subtract $2a$ from both sides.
\end{solution}



\begin{solution}[by \href{https://artofproblemsolving.com/community/user/346854}{ShR}]
	\begin{tcolorbox}[quote=ShR]Subtracting $P(x,y)$ from $P(x+y,0)$, we get $f(x+y)-a=f(x)-a+f(y)-a$  please explain it please\end{tcolorbox}

$P(x+y, 0)$ gives $f(f(x+y))=f(x+y)+a$. $P(x, y)$ gives $f(f(x+y))=f(x)+f(y)$. This means $f(x+y)+a=f(x)+f(y)$, then subtract $2a$ from both sides.\end{tcolorbox}

thanks
\end{solution}
*******************************************************************************
-------------------------------------------------------------------------------

\begin{problem}[Posted by \href{https://artofproblemsolving.com/community/user/361694}{Muradjl}]
	let $k$ be a fixed positive integer
find all functions $f : \mathbb{N} \rightarrow \mathbb{N}$ : such that :
 $$f(m+f(n))+f(n+f(m))  | 2 (m+n)^k$$ 
hold for all integers $m,n \geq 1 $
	\flushright \href{https://artofproblemsolving.com/community/c6h1606664}{(Link to AoPS)}
\end{problem}



\begin{solution}[by \href{https://artofproblemsolving.com/community/user/403767}{mkhayech}]
	asking for source cuz it looks made-up in which case no thanks I will not waste hours on a potentially unsolvable problem.
\end{solution}



\begin{solution}[by \href{https://artofproblemsolving.com/community/user/29428}{pco}]
	\begin{tcolorbox}let $k$ be a fixed positive integer
find all functions $f : \mathbb{N} \rightarrow \mathbb{N}$ : such that :
 $$f(m+f(n))+f(n+f(m))  | 2 (m+n)^k$$ 
hold for all integers $m,n \geq 1 $\end{tcolorbox}
Indeed, we are interested in the fact that this is a real olympiad-level exercise (and not just a crazy invented problem). This is the reason for which giving source is advised (it allows also to better retrieve the poblem later).

Here for example, we have a lot of solutions and there may exist serious doubts about the existence of a general form giving all of them.

For example :
$f(n)=1$ $\forall n$
$f(n)=n$ $\forall n$
$f(n)=2^u(1+(-1)^n)+\frac{1-(-1)^n}2$ $\forall n$, whatever is $u\in\{0,1,2,...,k-1\}$
$f(n)=2^u(1-(-1)^n)+\frac{1+(-1)^n}2$ $\forall n$, whatever is $u\in\{0,1,2,...,k-1\}$
...



\end{solution}
*******************************************************************************
-------------------------------------------------------------------------------

\begin{problem}[Posted by \href{https://artofproblemsolving.com/community/user/346854}{ShR}]
	Find all $f:R=>R$ such that for all $x,y$ are real numbers,
       $f (x)+f (y)= f(x+y)$   and  $f(x^{2018})=f (x)^{2018}  $
	\flushright \href{https://artofproblemsolving.com/community/c6h1606939}{(Link to AoPS)}
\end{problem}



\begin{solution}[by \href{https://artofproblemsolving.com/community/user/190895}{ltf0501}]
	f bounded in positive reals. Finish
\end{solution}



\begin{solution}[by \href{https://artofproblemsolving.com/community/user/346854}{ShR}]
	\begin{tcolorbox}f bounded in positive reals. Finish\end{tcolorbox}

please explain it
\end{solution}



\begin{solution}[by \href{https://artofproblemsolving.com/community/user/29428}{pco}]
	You have an additive function bounded over $\mathbb R^+$ and so linear.
Hence finished, as wrote ltf0501 : just plug $f(x)=ax$ in original equation and you get $a=a^{2018}$ and so $a\in\{0,1\}$

\end{solution}



\begin{solution}[by \href{https://artofproblemsolving.com/community/user/366424}{futurestar}]
	We can also do this by proving that f is an increasing function!
$f(x^{2018})=f(x)^{2018}$ means that for all positive $x $, $f(x) $ is positive.
Now take $y>x $ and plug $(x,y-x) $ in first relation. $f(y-x) $ is positive,therefore $f(y)>f(x) $.
\end{solution}



\begin{solution}[by \href{https://artofproblemsolving.com/community/user/345056}{Lamp909}]
	It is also true if we replace $2018$ with an arbitrary positive integer $n \geq 2$.
\end{solution}



\begin{solution}[by \href{https://artofproblemsolving.com/community/user/29428}{pco}]
	\begin{tcolorbox}It is also true if we replace $2018$ with an arbitrary positive integer $n \geq 2$.\end{tcolorbox}

Yes, and it has been proved a lot of times here. But the general proof including odd positive integers is a little bit more complex than the specific elementary case given here (even positive integer)
\end{solution}
*******************************************************************************
-------------------------------------------------------------------------------

\begin{problem}[Posted by \href{https://artofproblemsolving.com/community/user/348984}{JokerZ}]
	Find all $f:\mathbb{Q}^+\rightarrow \mathbb{Q}^+$ such that:
1. $f(x+1)=f(x)$
2. $f(\frac{1}{x})=x^2f(x)$
for all positive rational number $x$.
	\flushright \href{https://artofproblemsolving.com/community/c6h1608160}{(Link to AoPS)}
\end{problem}



\begin{solution}[by \href{https://artofproblemsolving.com/community/user/29428}{pco}]
	\begin{tcolorbox}Find all $f:\mathbb{Q}^+\rightarrow \mathbb{Q}^+$ such that:
1. $f(x+1)=f(x)$
2. $f(\frac{1}{x})=x^2f(x)$
for all positive rational number $x$.\end{tcolorbox}
$f(x)$ solution implies $tf(x)$ solution whatever is $t\in\mathbb Q^+$. So WLOG $f(1)=1$

Using continued fractions, it is easy to prove that if a solution exists, it must be unique.
And since $f(\frac pq)=\left(\frac q{\gcd(p,q)}\right)^2$ is a rether trivial solution, it is the unique one (with $f(1)=1$)

Hence the answer : $\boxed{f(\frac pq)=k\left(\frac q{\gcd(p,q)}\right)^2}$ which indeed is a solution, whatever is $k\in\mathbb Q^+$


\end{solution}
*******************************************************************************
-------------------------------------------------------------------------------

\begin{problem}[Posted by \href{https://artofproblemsolving.com/community/user/390214}{R8450932}]
	Find all functions $f:\mathbb {R^{+}}\to\mathbb {R^{+}} $ such that for all $x,y\in \mathbb {R^{+}} $
$f (3f (xy)^2+(xy)^2)=(xf (y)+yf (x))^2$

[hide=Note] $f (a)^2=(f (a))^2$ not $f (f (a)) $
	\flushright \href{https://artofproblemsolving.com/community/c6h1610002}{(Link to AoPS)}
\end{problem}



\begin{solution}[by \href{https://artofproblemsolving.com/community/user/260515}{anhtaitran}]
	If xy=c=const,then:
x*f(c\/x)+c\/x*f(x)=f(x;c)=const.
For every c1+c2=c3(ci>0),
f(x;c1)+f(x;c2)-f(x;c3)=x[f(c1\/x)+f(c2\/x)-f(c3\/x)]=const.
Or in other words,
f(c1*x)+f(c2*x)-f(c3*x)=const*x for every real x and c1+c2=c3.
Now,replace (c1;c2;c3) by (1;2;3);(2;2;4) and (1;3;4),we could possibly calculate:
f(x)\=tx for every x real.
Now plug such f(x) into the initial relation we could find t=1.
Hence,f(x)=x.
\end{solution}



\begin{solution}[by \href{https://artofproblemsolving.com/community/user/361694}{Muradjl}]
	Solution:
taking $y=\frac{1}{x}$ we get  $f(3f(1)^2+1)=(xf(\frac{1}{x})+\frac{f(x)}{x})^2$
taking $x=y=1$ we get $f(3f(1)^2+1)=4f(1)$ hence we get $xf(\frac{1}{x})+\frac{f(x)}{x}=2\sqrt{f(1)}$
let $x=1$ we get $f(1)+f(1)=2\sqrt{f(1)}$ thus $f(1)=1$
now let $g(x)=\frac{f(x)}{x}$
 hence we get $g(x)+g(\frac{1}{x})=2$ 
notice that this implies $g(x)g(\frac{1}{x}) \leq 1$
this implies that one of $g(x),g(\frac{1}{x})$ is less than $1$ while the other is greater than $1$
hence by the condition this happens if and only if they are equal thus $g(x)=1$  hence $f(x)=x$ 
easy to verify that this function verify the given statement.
\end{solution}



\begin{solution}[by \href{https://artofproblemsolving.com/community/user/390214}{R8450932}]
	Firstly #3 is wrong and #2 is hard to understand as there is no latex.
\end{solution}



\begin{solution}[by \href{https://artofproblemsolving.com/community/user/390214}{R8450932}]
	Any idea?
Pls help me.
\end{solution}



\begin{solution}[by \href{https://artofproblemsolving.com/community/user/29428}{pco}]
	\begin{tcolorbox}Find all functions $f:\mathbb {R^{+}}\to\mathbb {R^{+}} $ such that for all $x,y\in \mathbb {R^{+}} $
$f (3f (xy)^2+(xy)^2)=(xf (y)+yf (x))^2$

[hide=Note] $f (a)^2=(f (a))^2$ not $f (f (a)) $\end{tcolorbox}
Let $P(x,y)$ be the assertion $f(3f(xy)^2+x^2y^2)=(xf(y)+yf(x))^2$
Let $c=f(1)$

Subtracting $P(xy,1)$ from $P(x,y)$ and remembering $f(x)>0$ $\forall x$, we get 
$xf(y)+yf(x)=cxy+f(xy)$

Setting there $h(x)=e^{-x}f(e^x)-c$ from $\mathbb R\to(-c,+\infty)$, we get $h(x+y)=h(x)+h(y)$ 
And so $h\equiv 0$ (lowerbounded over $\mathbb R$)

So $f(x)=cx$ $\forall x>0$ and plugging this back in original equation, we get $c\in\{\frac 13,1\}$ and so :

$\boxed{\text{S1 : }f(x)=\frac x3\quad\forall x>0}$

$\boxed{\text{S2 : }f(x)=x\quad\forall x>0}$



\end{solution}
*******************************************************************************
-------------------------------------------------------------------------------

\begin{problem}[Posted by \href{https://artofproblemsolving.com/community/user/350533}{Sarbajit10598}]
	The function $f$ is defined on non negative integers by:
  
  $f(0)=0$ and $f(2n+1)=2f(n)$,for $n\geq 0$ and $f(2n)=2f(n)+1$ for $n{\geq} 1.$   
Now show that If $g(n)=f(f(n))$ then $g(n-g(n))=0$ $\forall$ $n{\geq}0.$
	\flushright \href{https://artofproblemsolving.com/community/c6h1611182}{(Link to AoPS)}
\end{problem}



\begin{solution}[by \href{https://artofproblemsolving.com/community/user/29428}{pco}]
	\begin{tcolorbox}The function $f$ is defined on non negative integers by:
  
  $f(0)=0$ and $f(2n+1)=2f(n)$,for $n\geq 0$ and $f(2n)=2f(n)+1$ for $n{\geq} 1.$   
Now show that If $g(n)=f(f(n))$ then $g(n-g(n))=0$ $\forall$ $n{\geq}0.$\end{tcolorbox}

Just look at binary form :
$f(n)$ is just swapping $0$ and $1$ in binary representation of $n$ (from rightmost digit up to and including the leftmost binary $1$)

So $g(n)$ is just erasing the first leftmost sequence of binary $1$ and then the leftmost sequence of binary $0$

So $n-g(n)$ is just this first leftmost sequence of binary $1$ and then the leftmost sequence of binary $0$.

So $g(n-g(n))$ is trivially $0$

\end{solution}



\begin{solution}[by \href{https://artofproblemsolving.com/community/user/387033}{maXplanK}]
	Can you(@pco) give an example of this ? Not getting it properly  :(
\end{solution}



\begin{solution}[by \href{https://artofproblemsolving.com/community/user/29428}{pco}]
	\begin{tcolorbox}Can you(@pco) give an example of this ? Not getting it properly  :(\end{tcolorbox}
Let $n=\overline{11000101011}$
$f(n)=f(\overline{11000101011})=\overline{111010100}$
$g(n)=f(\overline{111010100})=\overline{101011}$
$n-g(n)=\overline{11000000000}$
$f(n-g(n))=f(\overline{11000000000})=\overline{111111111}$
$g(n-g(n))=f(\overline{111111111})=0$

\end{solution}
*******************************************************************************
-------------------------------------------------------------------------------

\begin{problem}[Posted by \href{https://artofproblemsolving.com/community/user/335975}{Taha1381}]
	Find all functions $f:\mathbb{R} \to \mathbb{R}$ that satisfy:

$\lfloor{f(x^3+3x+1)} \rfloor =\lfloor{f(x)^2} \rfloor +1397$  

$\forall x \in \mathbb{R}$

**Proposed by Mohammad Jafari**
	\flushright \href{https://artofproblemsolving.com/community/c6h1611674}{(Link to AoPS)}
\end{problem}



\begin{solution}[by \href{https://artofproblemsolving.com/community/user/391068}{TuZo}]
	\begin{tcolorbox}Find all functions $f:\mathbb{R} \to \mathbb{R}$ that satisfy:

$\lfloor{f(x^3+3x+1)} \rfloor =\lfloor{f(x)^2} \rfloor +1397$  

$\forall x \in \mathbb{R}^{+}$

**Proposed by Mohammad Jafari**\end{tcolorbox}

What is the true: $f:\mathbb{R} \to \mathbb{R}$, or $\forall x \in \mathbb{R}^{+}$?


\end{solution}



\begin{solution}[by \href{https://artofproblemsolving.com/community/user/335975}{Taha1381}]
	That was my question too.I just copied what was on the pdf file.You can Ask Mr.Jafari about that.
\end{solution}



\begin{solution}[by \href{https://artofproblemsolving.com/community/user/391068}{TuZo}]
	Because if $x$ is only real number (not positive), there not exist such function!
\end{solution}



\begin{solution}[by \href{https://artofproblemsolving.com/community/user/29428}{pco}]
	\begin{tcolorbox}Find all functions $f:\mathbb{R} \to \mathbb{R}$ that satisfy:

$\lfloor{f(x^3+3x+1)} \rfloor =\lfloor{f(x)^2} \rfloor +1397$  

$\forall x \in \mathbb{R}^{+}$

**Proposed by Mohammad Jafari**\end{tcolorbox}
Uhh ?
There are obviously infinitely many solutions
Let $g(x)=x^3+3x+1$, increasing bijection from $[0,+\infty)\to[1,+\infty)$
Let the increasing sequence $a_0=0$ and $a_{n+1}=g(a_n)$

Define freely $f(x)$ over $[a_0,a_1)$
Then you can define $f(x)$ over $[a_1,a_2)$ easily using $\lfloor f(x)\rfloor=1397+\lfloor f(g^{-1}(x))^2\rfloor$
(which allows you to define infinitely many $f(x)$ over this interval
And continue this on every interval $[a_n,a_{n+1})$

\end{solution}



\begin{solution}[by \href{https://artofproblemsolving.com/community/user/62273}{momed66}]
	it is true : $\forall x \in \mathbb{R}$
you can see this change here :
https:\/\/artofproblemsolving.com\/community\/c32h1605033_220_problem_proposed_by_mohammad_jafari
\end{solution}
*******************************************************************************
-------------------------------------------------------------------------------

\begin{problem}[Posted by \href{https://artofproblemsolving.com/community/user/335975}{Taha1381}]
	Let $g(x)$ be a polynomial with real coefficients.Find all functions $f: \mathbb{R} \to \mathbb{R}$ that satisfy:

$\lfloor{f(xg(x^2))}\rfloor=\lfloor{f(g(x))^2}\rfloor+1397$

$\forall x \in \mathbb{R}$

**Proposed by Mohammad Jafari**
	\flushright \href{https://artofproblemsolving.com/community/c6h1611677}{(Link to AoPS)}
\end{problem}



\begin{solution}[by \href{https://artofproblemsolving.com/community/user/29428}{pco}]
	\begin{tcolorbox}Let $g(x)$ be a polynomial with real coefficients.Find all functions $f: \mathbb{R} \to \mathbb{R}$ that satisfy:

$\lfloor{f(xg(x^2))}\rfloor=\lfloor{f(g(x))^2}\rfloor+1397$

$\forall x \in \mathbb{R}$
\end{tcolorbox}
Let $a=f(g(1))$
Setting $x=1$ in equation, we get $\lfloor{a}\rfloor=\lfloor{a^2}\rfloor+1397$

So $(a^2-1)+1397< RHS=LHS\le a$
Which is $a^2-a+1396<0$, impossible.

Hence $\boxed{\text{no such function}}$


\end{solution}
*******************************************************************************
-------------------------------------------------------------------------------

\begin{problem}[Posted by \href{https://artofproblemsolving.com/community/user/335975}{Taha1381}]
	Find all functions $f:\mathbb{R}^{+} \to \mathbb{R}^{+}$ that satisfy:

$f(x+f(x)+y)=x+f(x)+3y-2f(y)(\forall x,y \in \mathbb{R}^{+})$

**Proposed by Mohammad Jafari**
	\flushright \href{https://artofproblemsolving.com/community/c6h1611695}{(Link to AoPS)}
\end{problem}



\begin{solution}[by \href{https://artofproblemsolving.com/community/user/29428}{pco}]
	\begin{tcolorbox}Find all functions $f:\mathbb{R}^{+} \to \mathbb{R}^{+}$ that satisfy:

$f(x+f(x)+y)=x+f(x)+3y-2f(y)(\forall x,y \in \mathbb{R}^{+})$\end{tcolorbox}
Let $P(x,y)$ be the assertion $f(x+f(x)+y)=x+f(x)+3y-2f(y)$
Let $u=1+f(1)$

1) $\frac 34x-\frac u2<f(x)<\frac 32x+\frac u2$ $\forall x$
$P(1,x)$ $\implies$ $f(x+u)=3x-2f(x)+u>0$ and so $f(x)<\frac 32x+\frac u2$ $\forall x$
So $f(x+u)<\frac 32(x+u)+\frac u2$ which is $3x-2f(x)+u<\frac 32(x+u)+\frac u2$
And so $f(x)>\frac 34x-\frac u2$
Q.E.D.

2) $f(x)=x$
If $a_nx-b_nu<f(x)<c_nx+d_nu$ $\forall x>0$, then :
$a_n(x+u)x-b_nu<f(x+u)<c_n(x+u)+d_nu$
$a_nx+(a_n-b_n)u<3x-2f(x)+u<c_nx+(c_n+d_n)u$
$\frac{3-c_n}2x-\frac{c_n+d_n-1}2u<f(x)<\frac{3-a_n}2x+\frac{-a_n+b_n+1}2u$

And it is easy to check that the sequence :
$(a_1,b_1,c_1,d_1)=(\frac 34,\frac 12,\frac 32,\frac 12)$
$(a_{n+1},b_{n+1},c_{n+1},d_{n+1})=(\frac{3-c_n}2,\frac{c_n+d_n-1}2,\frac{3-a_n}2,\frac{-a_n+b_n+1}2)$
is convergent towards $(1,0,1,0)$

And so $\boxed{f(x)=x\quad\forall x>0}$ which indeed is a solution.



\end{solution}
*******************************************************************************
-------------------------------------------------------------------------------

\begin{problem}[Posted by \href{https://artofproblemsolving.com/community/user/335975}{Taha1381}]
	Find all functions $f,g \mathbb{R}^{+} \to \mathbb{R}^{+}$ that satisfy:

$f(x+y)=g(x)g(y)^2+g(y)g(x)^2 (\forall x,y \in \mathbb{R}^{+})$

$g(x+y)=f(x)f(y)^2+f(y)f(x)^2 (\forall x,y \in \mathbb{R}^{+})$
	\flushright \href{https://artofproblemsolving.com/community/c6h1611702}{(Link to AoPS)}
\end{problem}



\begin{solution}[by \href{https://artofproblemsolving.com/community/user/29428}{pco}]
	\begin{tcolorbox}Find all functions $f,g \mathbb{R}^{+} \to \mathbb{R}^{+}$ that satisfy:

$f(x+y)=g(x)g(y)^2+g(y)g(x)^2 (\forall x,y \in \mathbb{R}^{+})$

$g(x+y)=f(x)f(y)^2+f(y)f(x)^2 (\forall x,y \in \mathbb{R}^{+})$\end{tcolorbox}
Let $P(x,y)$ be the assertion $f(x+y)=g(x)g(y)^2+g(y)g(x)^2$
Let $Q(x,y)$ be the assertion $g(x+y)=f(x)f(y)^2+f(y)f(x)^2$

Let $x,y,z,t>0$ and, for easier writing, $a=f(x)$, $b=f(y)$, $c=f(z)$ and $d=f(t)$

$Q(x,y)$ $\implies$ $g(x+y)=ab^2+a^2b$
$Q(z,t)$ $\implies$ $g(z+t)=cd^2+c^2d$
$P(x+y,z+t)$ $\implies$ 
$f(x+y+z+t)=(ab^2+a^2b)(cd^2+c^2d)^2+(ab^2+a^2b)^2(cd^2+c^2d)$

Swapping $y,z$ (and so $b,c$) we get
$f(x+y+z+t)=(ac^2+a^2c)(bd^2+b^2d)^2+(ac^2+a^2c)^2(bd^2+b^2d)$

And so $(ab^2+a^2b)(cd^2+c^2d)^2+(ab^2+a^2b)^2(cd^2+c^2d)=$ $(ac^2+a^2c)(bd^2+b^2d)^2+(ac^2+a^2c)^2(bd^2+b^2d)$

Which is $abcd(a-d)(c-b)(a^2d+2abc+2abd+2acd+ad^2+b^2c+bc^2+2bcd)=0$
And so $a=d$ or $b=c$ (remember $a,b,c,d>0$)

And so $\forall x,y,z,t>0$, either $f(x)=f(t)$, either $f(y)=f(z)$
And so $f(x)$ is constant, and so $g(x)$ is constant too.

Plugging $f\equiv u$ and $g\equiv v$ in original equations, we get $u=v=\frac 1{\sqrt 2}$

Hence the solution $\boxed{f(x)=g(x)=\frac 1{\sqrt 2}\quad\forall x>0}$


\end{solution}
*******************************************************************************
-------------------------------------------------------------------------------

\begin{problem}[Posted by \href{https://artofproblemsolving.com/community/user/335975}{Taha1381}]
	Find all functions $f: \mathbb{R}^{+} \to \mathbb{R}^{+}$ that satisfy:

$f(x+2f(x)+3y)=f(3x)+y+2f(y) (\forall x,y \in \mathbb{R}^{+})$

**Proposed by Mohammad Jafari**


	\flushright \href{https://artofproblemsolving.com/community/c6h1611707}{(Link to AoPS)}
\end{problem}



\begin{solution}[by \href{https://artofproblemsolving.com/community/user/29428}{pco}]
	\begin{tcolorbox}Find all functions $f: \mathbb{R}^{+} \to \mathbb{R}^{+}$ that satisfy:

$f(x+2f(x)+3y)=f(3x)+y+2f(y) (\forall x,y \in \mathbb{R}^{+})$\end{tcolorbox}
Let $P(x,y)$ be the assertion $f(x+2f(x)+3y)=f(3x)+y+2f(y)$
Let $A=\{f(x)-f(y)\quad\forall x,y\in\mathbb R\}$

If $f(x+a)=f(x)+b$ $\forall x>c$ for some real $a,b,c$ ($c\ge 0$), then :
Choosing $y>c$ and subtracting $P(x,y)$ from $P(x,y+a)$, we get $a=b$

Let $t>0$
$P(t,x)$ $\implies$ $x+2f(x)=f(3x+u)-v$ $\forall x>0$ where $u=t+2f(t)>0$ and $v=f(3t)>0$

$P(x,y)$ may then be written $f(f(3x+u)+3y-v)=f(3x)+f(3y+u)-v$
This may also be written $f(3y+u +(f(3x+u)-u-v))=f(3y+u)+(f(3x)-v)$ and so (see first property established) :
$f(3x+u)-u-v=f(3x)-v$ and so $f(3x+u)=f(3x)+u$

So $P(x,y)$ becomes $f(f(3x)+3y-v)=f(3x)+f(3y)-v$ whence LHS is defined
And so : $f(f(x)+y-v)=f(x)+f(y)-v$ $\forall x>0$, $\forall y>v-f(x)$
Back to definition of $v$, this is :
$f(x+f(y)-f(z))=f(x)+f(y)-f(z)$ $\forall x,y,z>0$ such that $x+f(y)-f(z)>0$
Which is $f(x+a)=f(x)+a$ $\forall a\in A$, $\forall x>\max(0,-a)$

From previously got "$x+2f(x)=f(3x+u)-v$" and "$f(3x+u)=f(3x)+u$", we get $x+2f(x)=f(3x)+u-v$
And so $x+2f(x)-f(3x)=y+2f(y)-f(3y)$ $\forall x,y>0$
Which is $x-y=(f(y)-f(x))+(f(y)-f(x))+(f(3x)-f(3y))$ 
And so any real may be written as sum of three elements of $A$

Let then $y>0$ and $a\ge b\ge c$ three elements of $A$ such that $y=a+b+c$
We obviously have $a,a+b,a+b+c>0$ and so :
$f(x+a)=f(x)+a$
$f((x+a)+b)=f(x+a)+b=f(x)+a+b$
$f((x+a+b)+c)=f(x+a+b)+c=f(x)+a+b+c$
And so $f(x+y)=f(x)+y$ $\forall x,y>0$
Writting $f(x)+y=f(y)+x$, we get 
$\boxed{f(x)=x+c\quad\forall x>0}$ which indeed is a solution, whatever is $c\ge 0$



\end{solution}
*******************************************************************************
-------------------------------------------------------------------------------

\begin{problem}[Posted by \href{https://artofproblemsolving.com/community/user/335975}{Taha1381}]
	Find all functions $f,g:\mathbb{R}^{+} \to \mathbb{R}^{+}$ such that $g(x)$ is injective and they satisfy:

$f(g(x)+g(2y))=g(x)+3f(y)-y(\forall x,y \in \mathbb{R}^{+})$

$g(2f(x)+2g(y))=3f(x)-x+g(2y) (\forall x,y \in \mathbb{R}^{+})$

**Proposed by Mohammad Jafari**
	\flushright \href{https://artofproblemsolving.com/community/c6h1611718}{(Link to AoPS)}
\end{problem}



\begin{solution}[by \href{https://artofproblemsolving.com/community/user/29428}{pco}]
	\begin{tcolorbox}Find all functions $f,g:\mathbb{R}^{+} \to \mathbb{R}^{+}$ such that $g(x)$ is injective and they satisfy:

$f(g(x)+g(2y))=g(x)+3f(y)-y(\forall x,y \in \mathbb{R}^{+})$

$g(2f(x)+2g(y))=3f(x)-x+g(2y) (\forall x,y \in \mathbb{R}^{+})$

**Proposed by Mohammad Jafari**\end{tcolorbox}
Let $P(x,y)$ be the assertion $f(g(x)+g(2y))=g(x)+3f(y)-y$
Let $Q(x,y)$ be the assertion $g(2f(x)+2g(y))=3f(x)-x+g(2y)$

Subtracting $P(2y,x)$ from $Q(x,y)$, we get $\implies$ $f(g(2y)+g(2x))=g(2f(x)+2g(y))$
Swapping ther $x,y$ and using injectivity of $g(x)$, we get $g(x)=f(x)+c$ for some constant $c$

Plugging this in $P(x,y)$ and $Q(x,y)$, we get new assertions :
Let $P1(x,y)$ be the assertion $f(f(x)+f(2y)+2c)=f(x)+3f(y)-y+c$
Let $Q1(x,y)$ be the assertion $f(2f(x)+2f(y)+2c)=3f(x)-x+f(2y)$

Subtracting $Q1(y,x)$ from $Q1(x,y)$, we get $f(2x)=3f(x)-x+b$ for some constant $b$

Subtracting $P1(2,1)$ from $Q1(1,1)$, we get $f(u)=f(v)+c$ where $u=2f(2)+2$ and $v=4f(1)+2c$
$Q1(x,u)$ $\implies$ $f(2f(x)+2f(v)+4c)=3f(x)-x+f(2u)$

But $f(2f(x)+2f(v)+4c)=3f(f(x)+f(v)+2c)-(f(x)+f(v)+2c)+b$
And so $f(2f(x)+2f(v)+4c)=3(f(x)+3f(v)-v+c)-(f(x)+f(v)+2c)+b$
And so $3f(x)-x+f(2u)=3(f(x)+3f(v)-v+c)-(f(x)+f(v)+2c)+b$

Which is $f(x)=x+d$ for some constant $d$

Plugging this back in original equations, we get 
$\boxed{f(x)=g(x)=x\quad\forall x>0}$


\end{solution}
*******************************************************************************
-------------------------------------------------------------------------------

\begin{problem}[Posted by \href{https://artofproblemsolving.com/community/user/335975}{Taha1381}]
	Find all functions $f:\mathbb{R}^{+} \to \mathbb{R}^{+}$ that satisfy:

$f(f(x)+2y)=f(2x+y)+2y (\forall x,y \in \mathbb{R}^{+})$

**Proposed by Mohammad Jafari**
	\flushright \href{https://artofproblemsolving.com/community/c6h1611727}{(Link to AoPS)}
\end{problem}



\begin{solution}[by \href{https://artofproblemsolving.com/community/user/29428}{pco}]
	\begin{tcolorbox}Find all functions $f:\mathbb{R}^{+} \to \mathbb{R}^{+}$ that satisfy:

$f(f(x)+2y)=f(2x+y)+2y (\forall x,y \in \mathbb{R}^{+})$\end{tcolorbox}
Let $P(x,y)$ be the assertion $f(f(x)+2y)=f(2x+y)+2y$

Suppose $f(x+a)=f(x)+b$ $\forall x>c$ for some $a,b\in\mathbb R$ and $c\in\mathbb R^+$
Choosing $y>c$ and subtracting $P(x,y)$ from $P(x,y+a)$, we get $b=2a$

Subtracting $P(z,\frac{f(y)}2+x)$ from $P(y,\frac{f(z)}2+x)$, we get :
$f(x+2z+\frac{f(y)}2)=f(x+2y+\frac{f(z)}2)+f(z)-f(y)$
This is $f(x+u)=f(x+v)+w$ with $u=2z+\frac{f(y)}2$, $v=2y+\frac{f(z)}2$ and $w=f(z)-f(y)$
which is $f(x+u-v)=f(x)+w$ $\forall x>v$
And so (see 6 lines above) $w=2(u-v)$ 
Which is $f(z)-f(y)=2((2z+\frac{f(y)}2)-(2y+\frac{f(z)}2))$

Which is $f(z)-2z=f(y)-2y$
And so $f(x)=2x+a$ for some $a\in\mathbb R$.
Plugging this back in original equation, we get $a=0$

And so $\boxed{f(x)=2x\quad\forall x>0}$



\end{solution}
*******************************************************************************
-------------------------------------------------------------------------------

\begin{problem}[Posted by \href{https://artofproblemsolving.com/community/user/335975}{Taha1381}]
	Find all functions $f,g:\mathbb{R}^{+} \to \mathbb{R}^{+}$ that satisfy:

$f(x+g(x)+y)=g(x)+f(x)+f(y) (\forall x,y \in \mathbb{R}^{+})$

$g(x+f(x)+y)=f(x)+g(x)+g(y) (\forall x,y \in \mathbb{R}^{+})$

**Proposed by Mohammad Jafari**
	\flushright \href{https://artofproblemsolving.com/community/c6h1611731}{(Link to AoPS)}
\end{problem}



\begin{solution}[by \href{https://artofproblemsolving.com/community/user/29428}{pco}]
	\begin{tcolorbox}Find all functions $f,g:\mathbb{R}^{+} \to \mathbb{R}^{+}$ that satisfy:

$f(x+g(x)+y)=g(x)+f(x)+f(y) (\forall x,y \in \mathbb{R}^{+})$

$g(x+f(x)+y)=f(x)+g(x)+g(y) (\forall x,y \in \mathbb{R}^{+})$

**Proposed by Mohammad Jafari**\end{tcolorbox}
Let $P(x,y)$ be the assertion $f(x+g(x)+y)=g(x)+f(x)+f(y)$
Let $Q(x,y)$ be the assertion $g(x+f(x)+y)=f(x)+g(x)+g(y)$

1) If $f(x+a)=f(x)+b$ $\forall x>0$ for some $a,b>0$, then $a=b$
Subtracting $Q(x+a+b,x)$ from $Q(x+b,x+a+b)$, we get $g(x+b)=g(x)+b$
Subtracting $P(x+2b,x)$ from $P(x+b,x+2b)$, we get $f(x+b)=f(x)+b$
So $f(x+a)=f(x+b)$ and so, setting $T=|a-b|$ : $f(x+T)=f(x)$ for ny $x$ great enough
If $T>0$, $P(x,y)$ with $x$ great enough, may be written 
$f(x+g(x)+y)=g(x)+f(x+kT)+f(y)$
Choosing then $k$ great enough and $y=kT-g(x)>0$, this becomes $g(x)+f(y)=0$, impossible
So $T=0$ and $a=b$
Q.E.D.

2) $f(x)=x$ $\forall x>0$
$P(x,y)$ is $f(y+a)=f(x)+b$ with $a=x+g(x)$ and $b=f(x)+g(x)$
So $a=b$, which is $f(x)=x$
Q.E.D.

3) $g(x)=x$ $\forall x>0$
$Q(x,y)$ becomes new assertion $R(x,y)$
$g(2x+y)=x+g(x)+g(y)$
Subtracting $R(x,2y)$ from $R(y,2x)$, we get $g(2x)=x+g(x)+c$ for some real $c$
And so $R(x,y)$ may be written $g(2x+y)=g(2x)+g(y)-c$
And so $g(x)-c$ is additive, and since lowerbounded over $\mathbb R^+$, linear
Plugging $g(x)=ax+c$ in $R(x,y)$, we get $a=1$ and $c=0$
Q.E.D.

And so $\boxed{f(x)=g(x)=x\quad\forall x>0}$, which indeed is a solution


\end{solution}
*******************************************************************************
-------------------------------------------------------------------------------

\begin{problem}[Posted by \href{https://artofproblemsolving.com/community/user/335975}{Taha1381}]
	For all functions $f,g:\mathbb{R}^{+} \to \mathbb{R}^{+}$ that satisfy:

$f(g(x)+y)=f(x)+g(y) (\forall x,y \in \mathbb{R}^{+})$

$g(f(x)+y)=g(x)+f(y) (\forall x,y \in \mathbb{R}^{+})$

$f(1)=g(1)$  

Prove that $f(x)=g(x) \forall x \in \mathbb{R}^{+}$.

**Proposed by Mohammad Jafari**

	\flushright \href{https://artofproblemsolving.com/community/c6h1611756}{(Link to AoPS)}
\end{problem}



\begin{solution}[by \href{https://artofproblemsolving.com/community/user/29428}{pco}]
	\begin{tcolorbox}For all functions $f,g:\mathbb{R}^{+} \to \mathbb{R}^{+}$ that satisfy:

$f(g(x)+y)=f(x)+g(y) (\forall x,y \in \mathbb{R}^{+})$

$g(f(x)+y)=g(x)+f(y) (\forall x,y \in \mathbb{R}^{+})$

$f(1)=g(1)$  

Prove that $f(x)=g(x) \forall x \in \mathbb{R}^{+}$.

**Proposed by Mohammad Jafari**\end{tcolorbox}
Let $P(x,y)$ be the assertion $f(g(x)+y)=f(x)+g(y)$
Let $Q(x;y)$ be the assertion $g(f(x)+y)=g(x)+f(y)$

1) $f(x)\ge x$ and $g(x)\ge x$ $\forall x>0$
If $f(x)<x$ for some $x$ : $Q(x,x-f(x))$ implies contradiction. 
So $f(x)\ge x$ $\forall x>0$
If $g(x)<x$ for some $x$ : $P(x,x-g(x))$ implies contradiction. 
So $g(x)\ge x$ $\forall x>0$
Q.E.D.

2) If $f(x+a)=f(x)+b$ $\forall x>0$ then $a=b$
Suppose $\exists$ $a>0$ and $b$ such that $f(x+a)=f(x)+b$ $\forall x>0$
Subtracting $P(y,x)$ from $P(y,x+a)$, we get $g(x+a)=g(x)+b$ $\forall x>0$
Subtracting $P(1,x-g(1))$ from $P(1+a,x-g(1))$, we get $f(x+b)=f(x)+b=f(x+a)$ $\forall x>g(1)$
Let $T=|a-b|$ this implies $f(x+T)=f(x)$ for all $x$ great enough

But this implies $f(x)=f(x+nT)\ge x+nT$, impossible (just set $n\to +\infty$) if $T\ne 0$
Q.E.D.

3) $g(x)-f(x)$ is constant
Adding $Q(y,x)$ and $P(z,f(y)+x)$, we get $f(x+f(y)+g(z))=f(x)+g(y)+f(z)$
And so, using 2) : $f(y)+g(z)=g(y)+f(z)$ which is $g(y)-f(y)=g(z)-f(z)$ constant
Q.E.D.

And since $f(1)=g(1)$, we got the required conclusion



\end{solution}
*******************************************************************************
-------------------------------------------------------------------------------

\begin{problem}[Posted by \href{https://artofproblemsolving.com/community/user/364684}{kakasmino}]
	 Find $f:\mathbb{R}\to \mathbb{R}$
 $(x-2).f(y)+f(y+2.f(x))=f(x+y.f(x))$
	\flushright \href{https://artofproblemsolving.com/community/c6h1612717}{(Link to AoPS)}
\end{problem}



\begin{solution}[by \href{https://artofproblemsolving.com/community/user/29428}{pco}]
	\begin{tcolorbox}Find $f:\mathbb{R}\to \mathbb{R}$
 $(x-2).f(y)+f(y+2.f(x))=f(x+y.f(x))$\end{tcolorbox}
Let $P(x,y)$ be the assertion $(x-2)f(y)+f(y+2f(x))=f(x+yf(x))$

If $f(0)=0$, then $P(0,x)$ $\implies$ $\boxed{\text{S1 : }f(x)=0\quad\forall x}$
Which indeed is a solution

If $f(0)\ne 0$ :
$P(x,0)$ $\implies$ $(x-2)f(0)+f(2f(x))=f(x)$ and so $f(x)$ is injective
$P(2,3)$ $\implies$ $f(3+2f(2))=f(2+3f(2))$ and so, since injective :  $3+2f(2)=2+3f(2)$
Which is $f(2)=1$ and so $f(3)\ne 1$ (since injective) 

$P(3,\frac 3{1-f(3)})$ $\implies$ $f(u)=0$ for some $u=\frac 3{1-f(3)}+2f(3)$
$P(u,2)$ $\implies$ $u=1$
$P(x,1)$ $\implies$ $f(1+2f(x))=f(x+f(x))$ and so, since injective $1+2f(x)=x+f(x)$

And so $\boxed{\text{S2 : }f(x)=x-1\quad\forall x}$ which indeed is a solution.


\end{solution}



\begin{solution}[by \href{https://artofproblemsolving.com/community/user/350483}{adhikariprajitraj}]
	How do you figure out that $f(x)$ is injective?
Thank you @below. I am such a troll!
\end{solution}



\begin{solution}[by \href{https://artofproblemsolving.com/community/user/347186}{p_square}]
	$f(x) = f(y) \implies (x-2)f(0) = f(x) - f(2f(x)) = f(y) - f(2f(y)) = (y-2)f(0) \implies x = y$

\end{solution}
*******************************************************************************
-------------------------------------------------------------------------------

\begin{problem}[Posted by \href{https://artofproblemsolving.com/community/user/392546}{onlygeo}]
	Find all $f:\mathbb{R}\to\mathbb{R}$ functions such that 
$$f(x+yf(x))=f(x)+xf(y)$$ for all real numbers $x,y.$
	\flushright \href{https://artofproblemsolving.com/community/c6h1614049}{(Link to AoPS)}
\end{problem}



\begin{solution}[by \href{https://artofproblemsolving.com/community/user/350483}{adhikariprajitraj}]
	The functions satisfying the condition are:
$f(x)=0$ and $f(x)=x$.
I used search function but can't find the link. Please help someone. I know this problem has been posted more than 10 times.Thank you!
EDIT: I found it [url=http://artofproblemsolving.com\/community\/c6h1252810p6463939]here.[\/url]
\end{solution}



\begin{solution}[by \href{https://artofproblemsolving.com/community/user/29428}{pco}]
	\begin{tcolorbox}I used search function but can't find the link. Please help someone. \end{tcolorbox}
Use elementary search string : +"f(x+yf(x))" +"xf(y)"


\end{solution}



\begin{solution}[by \href{https://artofproblemsolving.com/community/user/350483}{adhikariprajitraj}]
	Thank you @pco! :)
\end{solution}
*******************************************************************************
-------------------------------------------------------------------------------

\begin{problem}[Posted by \href{https://artofproblemsolving.com/community/user/361694}{Muradjl}]
	Find all functions $f,g : \Bbb{Q}_{>0}\to \Bbb{Z}_{>0}$ such that $$f(xy)\cdot \gcd\left( f(x)f(\frac{1}{y}), f(\frac{1}{x})f(y)\right)
= xy.g(xy).lcm(f(x),f(y)),$$ for all $x, y \in  \Bbb{Q}_{>0}$
	\flushright \href{https://artofproblemsolving.com/community/c6h1614944}{(Link to AoPS)}
\end{problem}



\begin{solution}[by \href{https://artofproblemsolving.com/community/user/361694}{Muradjl}]
	anyone here ?
\end{solution}



\begin{solution}[by \href{https://artofproblemsolving.com/community/user/29428}{pco}]
	\begin{tcolorbox}Find all functions $f,g : \Bbb{Q}_{>0}\to \Bbb{Z}_{>0}$ such that $$f(xy)\cdot \gcd\left( f(x)f(\frac{1}{y}), f(\frac{1}{x})f(y)\right)
= xy.g(xy).lcm(f(x),f(y)),$$ for all $x, y \in  \Bbb{Q}_{>0}$\end{tcolorbox}
Let $P(x,y)$ be the assertion $f(xy)\gcd(f(x)f(\frac 1y),f(\frac 1x)f(y))=xyg(xy)\text{lcm}(f(x),f(y))$
Let $c=f(1)$

$P(1,1)$ $\implies$ $g(1)=c^2$

$P(x,\frac 1x)$ $\implies$ $cf(x)f(\frac 1x)=\gcd(f(x),f(\frac 1x))^3$
Writing there $f(x)=a(x)b(x)$ and $f(\frac 1x)=a(x)c(x)$ with $a(x)=\gcd(f(x),f(\frac 1x))$, the previous line implies 
$a(x)=cb(x)c(x)$
This implies $c|f(x)$ $\forall x$

$P(x^2,1)$ $\implies$ $c\gcd(f(x^2),f(\frac 1{x^2}))=x^2g(x^2)$
$P(x,x)$ $\implies$ $f(x^2)f(\frac 1x)=x^2g(x^2)$
And so $\frac{f(x^2)}c\frac{f(\frac 1x)}c=\gcd(\frac{f(x^2)}c,\frac{f(\frac 1{x^2})}c)$
But this implies $\frac{f(x^2)}c=\gcd(\frac{f(x^2)}c,\frac{f(\frac 1{x^2})}c)$ since (each divides the other)
And so $f(\frac 1x)=c$
And so $f(x)=c$ $\forall x\in\mathbb Q_{>0}$ which unfortunately is never a solution.

Hence $\boxed{\text{No such functions}}$


\end{solution}



\begin{solution}[by \href{https://artofproblemsolving.com/community/user/345905}{TLP.39}]
	Another Solution:
This is equivalent to:
\begin{tcolorbox}Find all functions $f,h : \Bbb{Q}_{>0}\to \Bbb{Z}_{>0}$ such that $$\frac{\gcd\left( f(x)f(\frac{x}{a}), f(\frac{1}{x})f(\frac{a}{x})\right)}
{lcm(f(x),f(\frac{a}{x}))}=h(a)$$ for all $x, a \in  \Bbb{Q}_{>0}$\end{tcolorbox}
Let $P(x,a)$ be the assertion.
$P(k,k^2)\implies f(\frac{1}{k})=h(k^2)$.
Hence $P(x,k^2)\implies f(x)f(\frac{1}{k})|f(x)f(\frac{x}{k^2})\implies f(\frac{1}{k})|f(\frac{x}{k^2})\,\forall x,k\in\mathbb{Q}_{>0}\implies f(a)|f(b)\,\forall a,b\in\mathbb{Q}_{>0}$.
Hence $f(x)$ must be a constant function,but the constant functions are not solution to the assertion.
Hence this assertion has no solution.
\end{solution}
*******************************************************************************
-------------------------------------------------------------------------------

\begin{problem}[Posted by \href{https://artofproblemsolving.com/community/user/247598}{MATH1945}]
	Does there exists a surjective function $f$ at reals so that $f(x+1)-f(x)=c$ for each $x$ real and$f(x)$ periodic? ($c$ is a fixed number)
	\flushright \href{https://artofproblemsolving.com/community/c6h1617939}{(Link to AoPS)}
\end{problem}



\begin{solution}[by \href{https://artofproblemsolving.com/community/user/29428}{pco}]
	\begin{tcolorbox}Does there exists a surjective function $f$ at reals so that $f(x+1)-f(x)=c$ for each $x$ real and$f(x)$ periodic? ($c$ is a fixed number)\end{tcolorbox}
Here is a proof of existence using continuum hypothesis and so certainly not suitable for olympiad answer.
Since this is an olympiad real problem (in what olympiad did you get it ?) there surely exists a simpler solution.
I hope somebody will find it.

Let $A=\{m+n\pi\quad\forall m,n\in\mathbb Z\}$ 
$A$ is an additive subgroup and so we can define an equivalence relation $x\sim y\iff\quad(x-y)\in A$

Let $r(x)$ from $\mathbb R\to\mathbb R$ any choice function associating to any real $x$ a representant (unique per class) of its equivalence class.

Let $B=r(\mathbb R)$
Since any real $x$ may be written in a unique manner as $x=r(x)+m(x)+n(x)\pi$ where $m(x)n(x)\in\mathbb Z$ we can conclude that $B$ is uncountable, else $\{r(x)+m(x)+n(x)\pi\}$ would be countable (as it is $\sim\mathbb N^3$)

So (continuum hypothesis), $\exists g(x)$ bijection from $B\to\mathbb R$.

Let us then define $f(x)=g(r(x))+cm(x)$ :
1) $m(x+1)=m(x)+1$ and $r(x+1)=r(x)$ and so $f(x+1)=f(x)+c$
2) $m(x+\pi)=m(x)$ and $r(x+\pi)=r(x)$ and so $f(x+\pi)=f(x)$ periodic
3) $f(r(x))=g(r(x))$ and $g(B)=\mathbb R$ and so $f(x)$ is surjective

Q.E.D.
\end{solution}
*******************************************************************************
-------------------------------------------------------------------------------

\begin{problem}[Posted by \href{https://artofproblemsolving.com/community/user/68025}{Pirkuliyev Rovsen}]
	Find all continuous functions $f: \mathbb{R}\to\mathbb{R}$ such that $f(x^3)-f(y^3)=(x^2+xy+y^2)(f(x)-f(y))$
	\flushright \href{https://artofproblemsolving.com/community/c6h1617962}{(Link to AoPS)}
\end{problem}



\begin{solution}[by \href{https://artofproblemsolving.com/community/user/29428}{pco}]
	\begin{tcolorbox}Find all continuous functions $f: \mathbb{R}\to\mathbb{R}$ such that $f(x^3)-f(y^3)=(x^2+xy+y^2)(f(x)-f(y))$\end{tcolorbox}
Let $P(x,y)$ be the assertion $f(x^3)-f(y^3)=(x^2+xy+y^2)(f(x)-f(y))$

$f(x)$ solution implies $f(x)+c$ solution and so WLOG $f(0)=0$
$P(x,0)$ $\implies$ $f(x^3)=x^2f(x)$ and so $P(x,y)$ may be written :
$x^2f(x)-y^2f(y)=(x^2+xy+y^2)(f(x)-f(y))$

Which is $y(x+y)f(x)=x(x+y)f(y)$

And so $\frac{f(x)}x=\frac{f(y)}y$ $\forall x,y,x+y\ne 0$
And so trivially $f(x)=ax$

And so $\boxed{f(x)=ax+b\quad\forall x}$ which indeed is a solution, whatever are $a,b\in\mathbb R$

And no need for continuity.


\end{solution}



\begin{solution}[by \href{https://artofproblemsolving.com/community/user/375767}{matis}]
	I don't understand why if f(x) is a solution it implies that $f(x)+c$ is a solution?
And why you say that wlog we can assume $f(0)=0$?
\end{solution}



\begin{solution}[by \href{https://artofproblemsolving.com/community/user/277552}{WizardMath}]
	Dividing both sides by $f(x)-f(y)$ and taking the limit as $y\to x$, we have $f$ differentiable and $f'(x^3)=f'(x)$. Now differentiate both sides partially wrt $x$ and we have $3x^2 f'(x) = (2x+y)f(x) + (x^2+xy+y^2)f'(x) - (2x+y)f(y)$. So we have $(2x+y)(x-y)f'(x) = (f(x)-f(y))(2x+y)$. So for all $y$ except $y=x,-2x$, we have that $f(x)$ is linear, and by continuity we have $f$ is linear everywhere. The condition $f'(x^3)= f'(x)$ implies that $f'$ is constant and this means $f$ is linear. It is easy to see that all such functions work.
\end{solution}



\begin{solution}[by \href{https://artofproblemsolving.com/community/user/29428}{pco}]
	\begin{tcolorbox}I don't understand why if f(x) is a solution it implies that $f(x)+c$ is a solution?
And why you say that wlog we can assume $f(0)=0$?\end{tcolorbox}
Just replace $f(x)$ by $f(x)+c$ and you get the same equation. So if equation is true for $f(x)$ it is trivially still true for $f(x)+c$

So, replacing $f(x)$ by $f(x)-f(0)$ we can suppose $f(0)=0$ (not forgetting to add a free constant to the solutions we'll then find).

\end{solution}



\begin{solution}[by \href{https://artofproblemsolving.com/community/user/375767}{matis}]
	Now very clear pco, thanks! :)
\end{solution}



\begin{solution}[by \href{https://artofproblemsolving.com/community/user/375767}{matis}]
	Ah sorry only a doubt, why at the end when you obtein $\frac{f(x)}{x} =\frac{f(y)}{y}$ you say that it implies that $f(x)=ax$?
\end{solution}



\begin{solution}[by \href{https://artofproblemsolving.com/community/user/29428}{pco}]
	\begin{tcolorbox}Ah sorry only a doubt, why at the end when you obtein $\frac{f(x)}{x} =\frac{f(y)}{y}$ you say that it implies that $f(x)=ax$?\end{tcolorbox}
Setting $y=1$, this becomes $f(x)=xf(1)$ $\forall x\notin\{0,-1\}$
Setting $x=-1$ and $y=2$, this becomes $-f(-1)=\frac{f(2)}2=f(1)$ and so $f(-1)=-f(1)$
And since we already now $f(0)=0$

We get $f(x)=xf(1)$ $\forall x$


\end{solution}



\begin{solution}[by \href{https://artofproblemsolving.com/community/user/375767}{matis}]
	Thanks! :)
\end{solution}



\begin{solution}[by \href{https://artofproblemsolving.com/community/user/360096}{Kelly2007}]
	Why if $f(0)=0$ and $f(-1)=-f(1)$ it implies that $f(x)=ax$? :(
\end{solution}



\begin{solution}[by \href{https://artofproblemsolving.com/community/user/29428}{pco}]
	\begin{tcolorbox}Why if $f(0)=0$ and $f(-1)=-f(1)$ it implies that $f(x)=xf(1)$? :(\end{tcolorbox}

We proved that $f(x)=xf(1)$ $\forall x\notin\{0,-1\}$
We proved that $f(-1)=-f(1)$ and so $f(x)=xf(1)$ for $x=-1$
We proved that $f(0)=0$ and so $f(x)=xf(1)$ for $x=0$

And so $f(x)=xf(1)$ $\forall x$
Where is the problem ?

\end{solution}



\begin{solution}[by \href{https://artofproblemsolving.com/community/user/360096}{Kelly2007}]
	Ahh maybe I understand

but why when you substitute $y=1$ and obtein $f(x)=xf(1)$ you say that it is true $\forall x\notin\{0,-1\}$ the $0$ because before we had $\frac{f(x)}{x})$ and $x$ can't be $0$ and it implies that we can't say at first that $f(x)=xf(1)$
is true also for $x=0$ is correct this?
But the $-1$? why do you exclude it at first?
\end{solution}



\begin{solution}[by \href{https://artofproblemsolving.com/community/user/29428}{pco}]
	\begin{tcolorbox}But the $-1$? why do you exclude it at first?\end{tcolorbox}
Back to my proof (far far above) :
\begin{tcolorbox}...Which is $y(x+y)f(x)=x(x+y)f(y)$

And so $\frac{f(x)}x=\frac{f(y)}y$ $\forall x,y,x+y\ne 0$...\end{tcolorbox}
I got $\frac{f(x)}x=\frac{f(y)}y$ true only for $x,y,x+y\ne 0$ (because we divided by $x+y$ in order to get this equality).
So, setting $y=1$, we can not choose $x=-1$



\end{solution}



\begin{solution}[by \href{https://artofproblemsolving.com/community/user/360096}{Kelly2007}]
	Ahhh, I'm stupid... lol
Sorry pco, now it is clear, thanks for the explanation! :)
\end{solution}
*******************************************************************************
-------------------------------------------------------------------------------

\begin{problem}[Posted by \href{https://artofproblemsolving.com/community/user/361694}{Muradjl}]
	let $S$ be the set of nonzero  functions $f : \mathbb{R} \rightarrow \mathbb{R}$ satisfying:
$$fx^2+yf(z))=xf(x)+zf(y)$$ for all reals $x,y,z$.
  then  find $\max_{f \in S} (2017^{2018^{2019 }})$
	\flushright \href{https://artofproblemsolving.com/community/c6h1618752}{(Link to AoPS)}
\end{problem}



\begin{solution}[by \href{https://artofproblemsolving.com/community/user/29428}{pco}]
	\begin{tcolorbox}let $S$ be the set of nonzero  functions $f : \mathbb{R} \rightarrow \mathbb{R}$ satisfying:
$$fx^2+yf(z))=xf(x)+zf(y)$$ for all reals $x,y,z$.
  then  find $\max_{f \in S} (2017^{2018^{2019 }})$\end{tcolorbox}
Let $P(x,y,z)$ be the assertion $f(x^2+yf(z))=xf(x)+zf(y)$

$P(x,0,z)$ $\implies$ $f(0)=0$
If $f(1)\ne 1$, then $P(x,\frac{x^2}{1-f(1)},1)$ implies $xf(x)=0$ and so $f\equiv 0$, impossible
So $f1)=1$

$P(x,y,1)$ $\implies$ New assertion $Q(x,y)$ : $f(x^2+y)=xf(x)+f(y)$

Comparing $P(x,0)$ with $P(-x,0)$, we get $f(-x)=-f(x)$ and $f(x)$ is odd
Subtracting $P(x,0)$ from $P(x,y)$ we get $f(x+y)=f(x)+f(y)$ $\forall x\ge 0,\forall y$
Since odd, this implies $f(x)$ additive
And $f((x+1)^2)=(x+1)f(x+1)$ implies then $\boxed{f(x)=x\quad\forall x}$
Which indeed is a solution (and so the unique one)

And finishing the problem is then immediate :)





\end{solution}
*******************************************************************************
-------------------------------------------------------------------------------

\begin{problem}[Posted by \href{https://artofproblemsolving.com/community/user/361694}{Muradjl}]
	Does there exist a function $f:[0,2017] \rightarrow \mathbb{R}$ such that :
$$f(x+y^2) \geq y+f(x)$$
for all $x,y$ with $x \in [0,2017]$ and $x+y^2 \in [0,2017]$ ?
	\flushright \href{https://artofproblemsolving.com/community/c6h1618755}{(Link to AoPS)}
\end{problem}



\begin{solution}[by \href{https://artofproblemsolving.com/community/user/29428}{pco}]
	\begin{tcolorbox}Does there exist a function $f:[0,2017] \rightarrow \mathbb{R}$ such that :
$$f(x+y^2) \geq y+f(x)$$
for all $x,y$ with $x \in [0,2017]$ and $x+y^2 \in [0,2017]$ ?\end{tcolorbox}
Easy induction gives $f(x+\sum y_i^2)\ge f(x)+\sum y_i$ $\forall x,x+\sum y_i^2\in[0,2017]$

Choose then $x=0$ and $n$ $y_i=\frac 1{\sqrt n}$ and this becomes $f(1)\ge f(0)+\sqrt n$ whatever is $n$
Which is impossible


\end{solution}
*******************************************************************************
-------------------------------------------------------------------------------

\begin{problem}[Posted by \href{https://artofproblemsolving.com/community/user/361694}{Muradjl}]
	find all functions $f: \mathbb{Q} \rightarrow \mathbb{R}$ such that :
$$f(x+y)+f(x-y)=2max(f(x),f(y))$$ for all rationals $x,y$
	\flushright \href{https://artofproblemsolving.com/community/c6h1618764}{(Link to AoPS)}
\end{problem}



\begin{solution}[by \href{https://artofproblemsolving.com/community/user/29428}{pco}]
	\begin{tcolorbox}find all functions $f: \mathbb{Q} \rightarrow \mathbb{R}$ such that :
$$f(x+y)+f(x-y)=2maxf(f(x),f(y))$$ for all rationals $x,y$\end{tcolorbox}
What could be the meaning of $maxf(f(x),f(y))$ ??? 

\end{solution}



\begin{solution}[by \href{https://artofproblemsolving.com/community/user/361694}{Muradjl}]
	sorry corrected
\end{solution}



\begin{solution}[by \href{https://artofproblemsolving.com/community/user/29428}{pco}]
	\begin{tcolorbox}find all functions $f: \mathbb{Q} \rightarrow \mathbb{R}$ such that :
$$f(x+y)+f(x-y)=2max(f(x),f(y))$$ for all rationals $x,y$\end{tcolorbox}
Let $P(x,y)$ be the assertion $f(x+y)+f(x-y)=2\max(f(x),f(y))$
Let $b=f(0)$

Comparing $P(x,y)$ with $P(y,x)$ we get that $f(x)$ is even

$P(x,0)$ implies $f(x)=\max(f(x),b)$ and so $f(x)\ge b$ $\forall x$
$P(x,x)$ $\implies$ $f(2x)+b=2f(x)$
$P(2x,x)$ $\implies$ $f(3x)+f(x)=2\max(2f(x)-b,f(x))=2(2f(x)-b)$ and so $f(3x)=3f(x)-2b$
$P(3x,x)$ $\implies$ $f(4x)+2f(x)-b=2\max(3f(x)-2b,f(x))=2(3f(x)-2b)$ and so $f(4x)=4f(x)-3b$

And easy induction gives $f(nx)=nf(x)-(n-1)b$

And so $\boxed{f(x)=a|x|+b\quad\forall x\in\mathbb Q}$ 
which indeed is a solution, whatever are $a\in\mathbb Q_{\ge 0},b\in\mathbb Q$


\end{solution}
*******************************************************************************
-------------------------------------------------------------------------------

\begin{problem}[Posted by \href{https://artofproblemsolving.com/community/user/361694}{Muradjl}]
	does there exist a function $f : \mathbb{N} \rightarrow \mathbb{N}$ satisying :
$$f(f(n))=n^{2018}$$ for all positive integers $n$.
	\flushright \href{https://artofproblemsolving.com/community/c6h1618771}{(Link to AoPS)}
\end{problem}



\begin{solution}[by \href{https://artofproblemsolving.com/community/user/29428}{pco}]
	\begin{tcolorbox}does there exist a function $f : \mathbb{N} \rightarrow \mathbb{N}$ satisying :
$$f(f(n))=n^{2018}$$ for all positive integers $n$.\end{tcolorbox}
Yes. General solution :
Let $A=\{n^{2018^m}\quad\forall m,n\in\mathbb Z^+, n\ne 1\}$
Let $B=\mathbb Z^+\setminus\{1\}\setminus A$
$|B|=+\infty$ and so we can split $B$ in two equinumerous infinite subsets $B_1,B_2$
Let $g(x)$ any bijection from $B_1\to B_2$

Let $u(n)$ from $A\to B$ and $v(n)$ from $A\to\mathbb Z^+$ defines thru
$n=u(n)^{2018^{v(n)}}$

Define $f(n)$ as :
$f(1)=1$
$\forall n\in B_1$ : $f(n)=g(n)$
$\forall n\in B_2$ : $f(n)=(g^{-1}(n))^{2018}$
$\forall n\in A$ : $f(n)=f(u(n))^{2018^{v(n)}}$

\end{solution}



\begin{solution}[by \href{https://artofproblemsolving.com/community/user/361694}{Muradjl}]
	well $2018$ is replaced by any positive inetegr $k \geq 2$
\end{solution}



\begin{solution}[by \href{https://artofproblemsolving.com/community/user/29428}{pco}]
	\begin{tcolorbox}what about the general case ,when $2018$ is replaced by any positive inetegr $k \geq 2$\end{tcolorbox}
You are welcome. Glad to have helped you.

Since you have understood my previous post, just replace $2018$ with $k$ and it works exactly in the same way, giving the general solution for any such $k\ge 2$


\end{solution}
*******************************************************************************
-------------------------------------------------------------------------------

\begin{problem}[Posted by \href{https://artofproblemsolving.com/community/user/335975}{Taha1381}]
	Find all functions $f,g:\mathbb{R}^{+} \to \mathbb{R}^{+}$ that satisfy:

$f(x+y)+g(x-y)=f(4y) (\forall x,y \in \mathbb{R}^{+},x>y$)
	\flushright \href{https://artofproblemsolving.com/community/c6h1618792}{(Link to AoPS)}
\end{problem}



\begin{solution}[by \href{https://artofproblemsolving.com/community/user/29428}{pco}]
	\begin{tcolorbox}Find all functions $f,g:\mathbb{R}^{+} \to \mathbb{R}^{+}$ that satisfy:

$f(x+y)+g(x-y)=f(4y) (\forall x,y \in \mathbb{R}^{+},x>y$)\end{tcolorbox}
Let $P(x,y)$ be the assertion $f(x+y)+g(x-y)=f(4y)$

$P(3,1)$ $\implies$ $g(2)=0$, impossible
And so $\boxed{\text{No such function}}$



\end{solution}
*******************************************************************************
-------------------------------------------------------------------------------

\begin{problem}[Posted by \href{https://artofproblemsolving.com/community/user/361694}{Muradjl}]
	find all functions $f:]0,+\infty[ \rightarrow \mathbb{R}$ satisfying the following conditions:
 i)$$f(x)+f(y) \leq \frac{f(x+y)}{4}$$
 ii)$$ \frac{f(x)}{y}+\frac{f(y)}{x} \geq (\frac{1}{x}+\frac{1}{y}) .\frac{f(x+y)}{8}$$
for alll positive reals $x,y$.
	\flushright \href{https://artofproblemsolving.com/community/c6h1619414}{(Link to AoPS)}
\end{problem}



\begin{solution}[by \href{https://artofproblemsolving.com/community/user/29428}{pco}]
	\begin{tcolorbox}find all functions $f:]0,+\infty[ \rightarrow \mathbb{R}$ satisfying the following conditions:
 i)$$f(x)+f(y) \leq \frac{f(x+y)}{4}$$
 ii)$$ \frac{f(x)}{y}+\frac{f(y)}{x} \geq (\frac{1}{x}+\frac{1}{y}) .\frac{f(x+y)}{8}$$
for alll positive reals $x,y$.\end{tcolorbox}
Both inequality may be written as :
$P(x,y)$ :  $8\frac{xf(x)+yf(y)}{x+y}\ge f(x+y)\ge 4f(x)+4f(y)$

So $8\frac{xf(x)+yf(y)}{x+y}\ge 4f(x)+4f(y)$
Which is $(x-y)(f(x)-f(y))\ge 0$ and so $f(x)\ge f(y)$ $\forall x>y$

$P(x,x)$ $\implies$ $f(2x)=8f(x)$ and since $f(2x)\ge f(x)$ (previous line), we get $f(x)\ge 0$ $\forall x$

So $f(x+y)\ge 4f(x)+4f(y)\ge 4f(x)$
From there, easy induction gives $f(x+ny)\ge 4^nf(x)$
Setting there $y=\frac 1n$, we get $f(x+1)\ge 4^nf(x)$ $\forall n\in\mathbb N$ and so, since $f(x)\ge 0$ :

$\boxed{f(x)=0\quad\forall x}$ which indeed is a solution.



\end{solution}
*******************************************************************************
-------------------------------------------------------------------------------

\begin{problem}[Posted by \href{https://artofproblemsolving.com/community/user/213306}{saturzo}]
	Does there exist a function $f: \mathbb{R} \to \mathbb{R}\setminus\{0\}$ such that $(x-y)^2 \geq 4f(x)f(y), \forall x \neq y \in \mathbb{R}$?
	\flushright \href{https://artofproblemsolving.com/community/c6h1619679}{(Link to AoPS)}
\end{problem}



\begin{solution}[by \href{https://artofproblemsolving.com/community/user/29428}{pco}]
	\begin{tcolorbox}Does there exist a function $f: \mathbb{R} \to \mathbb{R}\setminus\{0\}$ such that $(x-y)^2 \geq 4f(x)f(y), \forall x \neq y \in \mathbb{R}$?\end{tcolorbox}
Could you kindly confirml us that this is the exact real olympiad exercise you got in your olympiad session \/ training session  (what session, btw ?).

It is possible to give example of such a function if the problem is over $\mathbb Q$ and not over $\mathbb R$. Could you confirm us that the olympiad containing this problem was indeed with $\mathbb R$ ?


\end{solution}
*******************************************************************************
-------------------------------------------------------------------------------

\begin{problem}[Posted by \href{https://artofproblemsolving.com/community/user/335975}{Taha1381}]
	For all functions $f:\mathbb{R}^{+} \to \mathbb{R}^{+}$ that satisfy:

$f(x+2f(x)+f(3y))=3f(x)+y+2f(y) (\forall x,y \in \mathbb{R}^{+})$

Prove that $f(x)$ is injective.
	\flushright \href{https://artofproblemsolving.com/community/c6h1620555}{(Link to AoPS)}
\end{problem}



\begin{solution}[by \href{https://artofproblemsolving.com/community/user/29428}{pco}]
	\begin{tcolorbox}For all functions $f:\mathbb{R}^{+} \to \mathbb{R}^{+}$ that satisfy:

$f(x+2f(x)+f(3y))=3f(x)+y+2f(y) (\forall x,y \in \mathbb{R}^{+})$

Prove that $f(x)$ is injective.\end{tcolorbox}
Let $P(x,y)$ be the assertion $f(x+2f(x)+f(3y))=3f(x)+y+2f(y)$

If $f(3a)=f(3b)$ :
Comparing $P(1,a)$ with $P(1,b)$, we get $a+2f(a)=b+2f(b)$
Comparing $P(a,1)$ with $P(b,1)$, we get $f(a)=f(b)$
And so $a=b$
And so $3a=3b$
Q.E.D.


\end{solution}
*******************************************************************************
-------------------------------------------------------------------------------

\begin{problem}[Posted by \href{https://artofproblemsolving.com/community/user/335975}{Taha1381}]
	For all functions $f:\mathbb{R}^{+} \to \mathbb{R}^{+}$ that satisfy:

$f(3x+f(y))=f(2x)+x+2f(y)-y (\forall x,y \in \mathbb{R}^{+})$

Prove that $f(x)$ is injective.
	\flushright \href{https://artofproblemsolving.com/community/c6h1620556}{(Link to AoPS)}
\end{problem}



\begin{solution}[by \href{https://artofproblemsolving.com/community/user/29428}{pco}]
	\begin{tcolorbox}For all functions $f:\mathbb{R}^{+} \to \mathbb{R}^{+}$ that satisfy:

$f(3x+f(y))=f(2x)+x+2f(y)-y (\forall x,y \in \mathbb{R}^{+})$

Prove that $f(x)$ is injective.\end{tcolorbox}
Let $P(x,y)$ be the assertion $f(3x+f(y))=f(2x)+x+2f(y)-y$

If $f(a)=f(b)$, subtracting $P(1,a)$ from $P(1,b)$, we get $a=b$
Q.E.D.


\end{solution}
*******************************************************************************
-------------------------------------------------------------------------------

\begin{problem}[Posted by \href{https://artofproblemsolving.com/community/user/335975}{Taha1381}]
	For all functions $f:\mathbb{R}^{+} \to \mathbb{R}^{+}$ that satisfy:

$f(3xy)=2f(xy)+xf(y) (\forall x,y \in \mathbb{R}^{+})$

Prove that $f(x)$ is injective.
	\flushright \href{https://artofproblemsolving.com/community/c6h1620557}{(Link to AoPS)}
\end{problem}



\begin{solution}[by \href{https://artofproblemsolving.com/community/user/29428}{pco}]
	\begin{tcolorbox}For all functions $f:\mathbb{R}^{+} \to \mathbb{R}^{+}$ that satisfy:

$f(3xy)=2f(xy)+xf(y) (\forall x,y \in \mathbb{R}^{+})$

Prove that $f(x)$ is injective.\end{tcolorbox}
Let $P(x,y)$ be the assertion $f(3xy)=2f(xy)+xf(y)$

Subtracting $P(1,1)$ from $P(\frac 1x,x)$, we get $f(x)=xf(1)$
Which is injective.



\end{solution}
*******************************************************************************
-------------------------------------------------------------------------------

\begin{problem}[Posted by \href{https://artofproblemsolving.com/community/user/335975}{Taha1381}]
	For all functions $f:\mathbb{R}^{+} \to \mathbb{R}^{+}$ that satisfy:

$f(f(3x)+3f(y))=x+2f(x)+f(3y) (\forall x,y \in \mathbb{R}^{+})$

Prove that $f(x)$ is injective.
	\flushright \href{https://artofproblemsolving.com/community/c6h1620558}{(Link to AoPS)}
\end{problem}



\begin{solution}[by \href{https://artofproblemsolving.com/community/user/29428}{pco}]
	\begin{tcolorbox}For all functions $f:\mathbb{R}^{+} \to \mathbb{R}^{+}$ that satisfy:

$f(f(3x)+3f(y))=x+2f(x)+f(3y) (\forall x,y \in \mathbb{R}^{+})$

Prove that $f(x)$ is injective.\end{tcolorbox}
Let $P(x,y)$ be the assertion $f(f(3x)+3f(y))=x+2f(x)+f(3y)$

If $f(a)=f(b)$, then :
Comparing $P(1,a)$ with $P(1,b)$, we get $f(3a)=f(3b)$
Comparing $P(a,1)$ with $P(b,1)$, we get $a+2f(a)=b+2f(b)$
And so $a=b$

Q.E.D.


\end{solution}
*******************************************************************************
-------------------------------------------------------------------------------

\begin{problem}[Posted by \href{https://artofproblemsolving.com/community/user/335975}{Taha1381}]
	For all functions $f:\mathbb{R}^{+} \to \mathbb{R}^{+}$ that satisfy:

$f(f(x)+f(y))=f(2x)-f(x)+f(2y)-y(\forall x,y \in \mathbb{R}^{+})$

Prove that $f(x)$ is injective.
	\flushright \href{https://artofproblemsolving.com/community/c6h1620559}{(Link to AoPS)}
\end{problem}



\begin{solution}[by \href{https://artofproblemsolving.com/community/user/29428}{pco}]
	\begin{tcolorbox}For all functions $f:\mathbb{R}^{+} \to \mathbb{R}^{+}$ that satisfy:

$f(f(x)+f(y))=f(2x)-f(x)+f(2y)-y(\forall x,y \in \mathbb{R}^{+})$

Prove that $f(x)$ is injective.\end{tcolorbox}
Let $P(x,y)$ be the assertion $f(f(x)+f(y))=f(2x)-f(x)+f(2y)-y$

Subtracting $P(x,1)$ from $P(1,x)$, we get $f(x)=x+f(1)-1$
Which is injective.


\end{solution}



\begin{solution}[by \href{https://artofproblemsolving.com/community/user/335975}{Taha1381}]
	\begin{tcolorbox}[quote=Taha1381]For all functions $f:\mathbb{R}^{+} \to \mathbb{R}^{+}$ that satisfy:

$f(f(x)+f(y))=f(2x)-f(x)+f(2y)-y(\forall x,y \in \mathbb{R}^{+})$

Prove that $f(x)$ is injective.\end{tcolorbox}
Let $P(x,y)$ be the assertion $f(f(x)+f(y))=f(2x)-f(x)+f(2y)-y$

Subtracting $P(x,1)$ from $P(1,x)$, we get $f(x)=x+f(1)-1$
Which is injective.\end{tcolorbox}

Wow You found the function too.Really nice.
\end{solution}



\begin{solution}[by \href{https://artofproblemsolving.com/community/user/335975}{Taha1381}]
	Let $P(x,y)$ denote the assertion $f(f(x)+f(y))=f(2x)-f(x)+f(2y)-y$ and let $f(x_1)=f(x_2)$

$P(x_1,y),P(x_2,y) \Rightarrow f(2x_1)=f(2x_2)$

$P(x,x_1),P(x,x_2) \Rightarrow x_1=x_2$

So $f$ is injective.
\end{solution}
*******************************************************************************
-------------------------------------------------------------------------------

\begin{problem}[Posted by \href{https://artofproblemsolving.com/community/user/369782}{mathenthusiastic}]
	find all continuous functions $f:\mathbb{R}\to\mathbb{R}$ such that for every $x_1,x_2,x_3,y_1,y_2,y_3\in\mathbb{R}$ that $x_1+x_2+x_3=y_1+y_2+y_3$ we have :
$f(x_1)+f(x_2)+f(x_3)=f(y_1)+f(y_2)+f(y_3)$
	\flushright \href{https://artofproblemsolving.com/community/c6h1621220}{(Link to AoPS)}
\end{problem}



\begin{solution}[by \href{https://artofproblemsolving.com/community/user/29428}{pco}]
	\begin{tcolorbox}find all continuous functions $f:\mathbb{R}\to\mathbb{R}$ such that for every $x_1,x_2,x_3,y_1,y_2,y_3\in\mathbb{R}$ we have :
$f(x_1)+f(x_2)+f(x_3)=f(y_1)+f(y_2)+f(y_3)$\end{tcolorbox}

Uhh?
This immediately gives $f(x)=c$ constant.

\end{solution}



\begin{solution}[by \href{https://artofproblemsolving.com/community/user/369782}{mathenthusiastic}]
	\begin{tcolorbox}[quote=mathenthusiastic]find all continuous functions $f:\mathbb{R}\to\mathbb{R}$ such that for every $x_1,x_2,x_3,y_1,y_2,y_3\in\mathbb{R}$ we have :
$f(x_1)+f(x_2)+f(x_3)=f(y_1)+f(y_2)+f(y_3)$\end{tcolorbox}

Uhh?
This immediately gives $f(x)=c$ constant.\end{tcolorbox}
 
sorry, how? I think it can be also $f(x)=ax$
\end{solution}



\begin{solution}[by \href{https://artofproblemsolving.com/community/user/29428}{pco}]
	\begin{tcolorbox}sorry, how? I think it can be also $f(x)=ax$\end{tcolorbox}
$f(x)=ax$ trivially does not fit (if $a\ne 0$)

Choose as counter example $(x_1,x_2,x_3,y_1,y_2,y_3)=(1,0,0,0,0,0)$



\end{solution}



\begin{solution}[by \href{https://artofproblemsolving.com/community/user/297527}{TheDarkPrince}]
	\begin{tcolorbox}find all continuous functions $f:\mathbb{R}\to\mathbb{R}$ such that for every $x_1,x_2,x_3,y_1,y_2,y_3\in\mathbb{R}$ we have :
$f(x_1)+f(x_2)+f(x_3)=f(y_1)+f(y_2)+f(y_3)$\end{tcolorbox}

Compare $P(x_1,x_2,x_3,y_1,y_2,z_1)$ and $P(x_1,x_2,x_3,y_1,y_2,z_2)$ where $P(x_1,x_2,x_3,y_1,y_2,y_3)$ is the assertion of the problem statement.
\end{solution}



\begin{solution}[by \href{https://artofproblemsolving.com/community/user/369782}{mathenthusiastic}]
	\begin{tcolorbox}[quote=mathenthusiastic]sorry, how? I think it can be also $f(x)=ax$\end{tcolorbox}
$f(x)=ax$ trivially does not fit (if $a\ne 0$)

Choose as counter example $(x_1,x_2,x_3,y_1,y_2,y_3)=(1,0,0,0,0,0)$\end{tcolorbox}

sorry I have edited my question 


\end{solution}



\begin{solution}[by \href{https://artofproblemsolving.com/community/user/29428}{pco}]
	\begin{tcolorbox}find all continuous functions $f:\mathbb{R}\to\mathbb{R}$ such that for every $x_1,x_2,x_3,y_1,y_2,y_3\in\mathbb{R}$ that $x_1+x_2+x_3=y_1+y_2+y_3$ we have :
$f(x_1)+f(x_2)+f(x_3)=f(y_1)+f(y_2)+f(y_3)$\end{tcolorbox}
Then write $f(x)+f(y)+f(0)=f(x+y)+f(0)+f(0)$
And so $f(x)-f(0)$ is additive and so, since continuous, linear.
So $\boxed{f(x)=ax+b\quad\forall x}$ which indeed is a solution, whatever are $a,b\in\mathbb R$


\end{solution}



\begin{solution}[by \href{https://artofproblemsolving.com/community/user/369782}{mathenthusiastic}]
	\begin{tcolorbox}[quote=mathenthusiastic]find all continuous functions $f:\mathbb{R}\to\mathbb{R}$ such that for every $x_1,x_2,x_3,y_1,y_2,y_3\in\mathbb{R}$ that $x_1+x_2+x_3=y_1+y_2+y_3$ we have :
$f(x_1)+f(x_2)+f(x_3)=f(y_1)+f(y_2)+f(y_3)$\end{tcolorbox}
Then write $f(x)+f(y)+f(0)=f(x+y)+f(0)+f(0)$
And so $f(x)-f(0)$ is additive and so, since continuous, linear.
So $\boxed{f(x)=ax+b\quad\forall x}$ which indeed is a solution, whatever are $a,b\in\mathbb R$\end{tcolorbox}

thanks dear pco
\end{solution}
*******************************************************************************
-------------------------------------------------------------------------------

\begin{problem}[Posted by \href{https://artofproblemsolving.com/community/user/282353}{Samsuman}]
	Consider the functional equation $af(z)+bf(w^2z)=g(z)$ ;where a and b are fixed complex numbers and g(z) is known complex function. Prove that complex function f(z) can be uniquely determined if $a^3+b^2 \neq 0$
Note: $w$ is complex cube root of unity
	\flushright \href{https://artofproblemsolving.com/community/c6h1621785}{(Link to AoPS)}
\end{problem}



\begin{solution}[by \href{https://artofproblemsolving.com/community/user/29428}{pco}]
	\begin{tcolorbox}Consider the functional equation $af(z)+bf(w^2z)=g(z)$ ;where a and b are fixed complex numbers and g(z) is known complex function. Prove that complex function f(z) can be uniquely determined if $a^3+b^2 \neq 0$
Note: $w$ is complex cube root of unity\end{tcolorbox}
I suppose you mean $a^3+b^3\ne 0$ (and not $a^3+b^2\ne 0$). If so :

$af(z)+bf(w^2z)=g(z)$
$af(wz)+bf(z)=g(wz)$
$af(w^2z)+bf(wz)=g(w^2z)$

If $a=0$ and $b\ne 0$ : first equation uniquely defines $f(z)$
If $a\ne 0$ and $b=0$ : first equation uniquely defines $f(z)$
If $ab\ne 0$, it is easy to cancel $f(wz)$ and $f(w^2z)$ amongst the three equations and we get :
$(a^3+b^3)f(z)=$some expression not depending on $f$

Q.E.D.



\end{solution}



\begin{solution}[by \href{https://artofproblemsolving.com/community/user/282353}{Samsuman}]
	Thanks PCO
\end{solution}
*******************************************************************************
-------------------------------------------------------------------------------

\begin{problem}[Posted by \href{https://artofproblemsolving.com/community/user/237462}{2009_hy_26}]
	\begin{bolded}PROBLEM\end{bolded}  Let $f \left( x \right) = \begin{cases} z& {Re(z) \ge 0}\\ -z& {Re(z) < 0} \end{cases}$ be a function defined on $\mathbb{C}$. A sequence $\left \{ z_{n} \right \}$ is defined as $z_{1} = u$ and $z_{n+1} = f \left( z_{n}^2+z_{n}+1 \right)$. Given $\left \{ z_{n} \right \}$ is periodic, find all possible values of $u$.

I wonder if there exist any elegant solutions to this problem. Moreover, I'm also interested in the source of this problem, since I might have seen it sometime before.
	\flushright \href{https://artofproblemsolving.com/community/c6h1621856}{(Link to AoPS)}
\end{problem}



\begin{solution}[by \href{https://artofproblemsolving.com/community/user/29428}{pco}]
	\begin{tcolorbox}\begin{bolded}PROBLEM\end{bolded}  Let $f \left( x \right) = \begin{cases} z& {Re(z) \ge 0}\\ -z& {Re(z) < 0} \end{cases}$ be a function defined on $\mathbb{C}$. A sequence $\left \{ z_{n} \right \}$ is defined as $z_{1} = u$ and $z_{n+1} = f \left( z_{n}^2+z_{n}+1 \right)$. Given $\left \{ z_{n} \right \}$ is periodic, find all possible values of $u$.\end{tcolorbox}
Note that $Re(f(x))\ge 0$ whatever is $x$
And so $Re(z_n)\ge 0$ $\forall n\ge 2$

Let $z_n=a+ib$ for some $n\ge 2$ (so that $a\ge 0$
$z_n^2+z_n+1=(a^2-b^2+a+1)+i(2ab+b)$
$|z_n^2+z_n+1|^2=(a^2-b^2+a+1)^2+(2ab+b)^2$
$=a^2+b^2+(a^2-b^2+1)^2+4a^2b^2+2a(a^2+b^2+1)$ $\ge a^2+b^2$
And since $|f(x)|=|x|$ $\forall x$, we get $|z_{n+1}|\ge |z_n|$ $\forall n\ge 2$

So periodicity implies $|z_{n+1}|= |z_n|$ and so $a=0$ and $b=\pm 1$
So $z_n=\pm i$ $\forall n\ge 2$

I consider that "periodic" means "periodic from the beginning" and so we get 
$\boxed{u\in\{-i,+i\}}$




\end{solution}
*******************************************************************************
-------------------------------------------------------------------------------

\begin{problem}[Posted by \href{https://artofproblemsolving.com/community/user/369782}{mathenthusiastic}]
	find all continuous function  $f: \mathbb{R}\to\mathbb{R}$ such that :
$\forall x,y\in\mathbb{R}:f(x+y)f(x-y)=(f(x)f(y))^2$
	\flushright \href{https://artofproblemsolving.com/community/c6h1621877}{(Link to AoPS)}
\end{problem}



\begin{solution}[by \href{https://artofproblemsolving.com/community/user/29428}{pco}]
	\begin{tcolorbox}find all continuous function  $f: \mathbb{R}\to\mathbb{R}$ such that :
$\forall x,y\in\mathbb{R}:f(x+y)f(x-y)=(f(x).f(y))^2$\end{tcolorbox}
Let $P(x,y)$ be the assertion $f(x+y)f(x-y)=f(x)^2f(y)^2$

$P(0,0)$ $\implies$ $f(0)\in\{-1,0,1\}$

1) If $f(0)=0$
$P(x,0)$ $\implies$ $\boxed{\text{S1 : }f(x)=0\quad\forall x}$
Which indeed is a solution

2) If $f(0)\ne 0$
$f(x)$ solution implies $-f(x)$ solution too. So WLOG $f(0)=1$
If $f(a)=0$ for some $a$, then 
$P(\frac a2,\frac a2)$ $\implies$ $f(\frac a2)0$ and so $f(\frac a{2^n})=0$ and continuity would implies $f(0)=0$
So no such $a$ and $f(x)>0$ $\forall x$
Let then $g(x)=\ln f(x)$ so that functional equation becomes 
Assertion $Q(x,y)$ : $g(x+y)+g(x-y)=2g(x)+2g(y)$ with $g(x)$ continuous and $g(0)=0$

$Q(0,x)$ $\implies$ $g(-x)=g(x)$
$Q(x,x)$ $\implies$ $g(2x)=4g(x)$
$Q(2x,x)$ $\implies$ $g(3x)=9g(x)$
Easy induction $Q(nx,x)$ gives $g(nx)=n^2g(x)$ $\forall x$ and $\forall n\in\mathbb N$

From there we easily get $g(x)=x^2g(1)$ $\forall x\in\mathbb Q$

And continuity allows to conclude $g(x)=cx^2$

And so $\boxed{\text{S2 : }f(x)=e^{cx^2}\quad\forall x}$
Which indeed is a solution, whatever is $c\in\mathbb R$

And since we used WLOG $f(0)=1$, we must add :
$\boxed{\text{S3 : }f(x)=-e^{cx^2}\quad\forall x}$
Which indeed is a solution, whatever is $c\in\mathbb R$





\end{solution}



\begin{solution}[by \href{https://artofproblemsolving.com/community/user/369782}{mathenthusiastic}]
	thanks for your great solution but unfortunately I haven't learned logarithm till now is there a solution without using logarithm?
\end{solution}



\begin{solution}[by \href{https://artofproblemsolving.com/community/user/29428}{pco}]
	\begin{tcolorbox}thanks for your great solution but unfortunately I haven't learned logarithm till now is there a solution without using logarithm?\end{tcolorbox}
Prove $f(-x)=f(x)$
Prove $f(2x)=f(x)^4$, then $f(3x)=f(x)^9$ and then $f(nx)=f(x)^{n^2}$ (induction)
Conclude $f(\frac pq)=f(1)^{\frac{p^2}{q^2}}$
And with continuity $f(x)=a^{x^2}$ for some $a>0$ (what I wrote as $e^{cx^2}$)

And dont forget that $f$ solution implies $-f$ solution and so $-a^{x^2}$ too.


\end{solution}
*******************************************************************************
-------------------------------------------------------------------------------

\begin{problem}[Posted by \href{https://artofproblemsolving.com/community/user/366424}{futurestar}]
	Find all functions $f : \mathbb R \to \mathbb R $ such that:
$f(x+y)=f(x)+f(y)+f(xy) $ $\forall x,y \in \mathbb R $
	\flushright \href{https://artofproblemsolving.com/community/c6h1621929}{(Link to AoPS)}
\end{problem}



\begin{solution}[by \href{https://artofproblemsolving.com/community/user/297527}{TheDarkPrince}]
	(partial)
Let $P(x,y)$ be the assertion of the problem statement.

$P(0,0)$ gives $f(0)=0$. Suppose $f(1)=c$, $f(x+1)=2f(x)+f(1)$. So $f(n)=(2^n-1)c$ for all integers $n$. $P(m,n)$ gives $c(-2^{m+n}+1+2^m-1+2^n-1+2^{mn}-1)=0$ for all integers $m,n$. If $c\neq 0$, then for $m=n=2$ we get a contradiction. So $c=0\implies f(n)=0$ for all integers $n$. 

$\Rightarrow f(x+1)=2f(x)$ or $f(x+n)=2^nf(x)$ for all integers $n$. $P(x,n)$ gives $f(x+n)=f(x)+f(nx)$, so $f(x)(2^n-1)=f(nx)$.
\end{solution}



\begin{solution}[by \href{https://artofproblemsolving.com/community/user/29428}{pco}]
	\begin{tcolorbox}Find all functions $f : \mathbb R \to \mathbb R $ such that:
$f(x+y)=f(x)+f(y)+f(xy) $ $\forall x,y \in \mathbb R $\end{tcolorbox}
Let $P(x,y)$ be the assertion $f(x+y)=f(x)+f(y)+f(xy)$

$P(0,0)$ $\implies$ $f(0)=0$

$P(y,z)$ $\implies$ $f(y+z)=f(y)+f(z)+f(yz)$
$P(xy,xz)$ $\implies$ $f(xy+xz)=f(xy)+f(xz)+f(x^2yz)$
$P(x,y+z)$ $\implies$ $f(x+y+z)=f(x)+f(y+z)+f(xy+xz)$ and so :
$f(x+y+z)=f(x)+f(y)+f(z)+f(yz)+f(xy)+f(xz)+f(x^2yz)$

Swapping there $x$ and $y$ and subtracting, we get $f(x^2yz)=f(xy^2z)$ 
Let $u,v\ne 0$ : setting $x=1$, $y=\frac vu$ and $z=\frac{u^2}v$, this is 
$f(u)=f(v)=c$ constant $\forall u,v\ne 0$

Then $P(u,-u)$ where $u\ne 0$ implies $c=0$

And so $\boxed{f(x)=0\quad\forall x}$ which indeed is a solution


\end{solution}



\begin{solution}[by \href{https://artofproblemsolving.com/community/user/366424}{futurestar}]
	nice @above,same as my solution :P
\end{solution}



\begin{solution}[by \href{https://artofproblemsolving.com/community/user/29428}{pco}]
	\begin{tcolorbox}Find all functions $f : \mathbb R \to \mathbb R $ such that:
$f(x+y)=f(x)+f(y)+f(xy) $ $\forall x,y \in \mathbb R $\end{tcolorbox}
A little bit simpler :
Let $P(x,y)$ be the assertion $f(x+y)=f(x)+f(y)+f(xy)$
Let $a=f(1)$

$P(0,0)$ $\implies$ $f(0)=0$
$P(x,1)$ $\implies$ $f(x+1)=2f(x)+a$

$P(x,y+1)$ $\implies$ $2f(x+y)+a=f(x)+2f(y)+a+f(xy+x)$
And so $2(f(x)+f(y)+f(xy))=f(x)+2f(y)+f(xy+x)$
Which is $f(xy+x)=f(x)+2f(xy)$

Setting there $y\to \frac yx$, this is $f(x+y)=f(x)+2f(y)$ $\forall y$, $\forall x\ne 0$
Swapping $x,y$ and subtracting, we get $f(x)=f(y)=c$ constant $\forall x,y\ne 0$

Then $P(1,-1)$ $\implies$ $c=0$

And so $\boxed{f(x)=0\quad\forall x}$ which indeed is a solution


\end{solution}
*******************************************************************************
-------------------------------------------------------------------------------

\begin{problem}[Posted by \href{https://artofproblemsolving.com/community/user/388743}{FieldBird}]
	Find all $ f : R -> R$, such, that $ f(xf(y))=(1-y)f(xy)+x^2y^2f(y)$
	\flushright \href{https://artofproblemsolving.com/community/c6h1621947}{(Link to AoPS)}
\end{problem}



\begin{solution}[by \href{https://artofproblemsolving.com/community/user/29428}{pco}]
	\begin{tcolorbox}Find all $ f : R -> R$, such, that $ f(x(f(y))=(1-y)f(xy)+x^2y^2f(y)$\end{tcolorbox}

Parenthesis mismatch in LHS
\end{solution}



\begin{solution}[by \href{https://artofproblemsolving.com/community/user/388743}{FieldBird}]
	Just corrected:
Find all $ f : R -> R$, such, that $ f(xf(y))=(1-y)f(xy)+x^2y^2f(y)$
\end{solution}



\begin{solution}[by \href{https://artofproblemsolving.com/community/user/29428}{pco}]
	\begin{tcolorbox}Find all $ f : R -> R$, such, that $ f(xf(y))=(1-y)f(xy)+x^2y^2f(y)$\end{tcolorbox}
$\boxed{\text{S1 : }f(x)=0\quad\forall x}$ is a solution. So let us from now look only for non allzero solutions.
Let $P(x,y)$ be the assertion $f(xf(y))=(1-y)f(xy)+x^2y^2f(y)$
Let $u$ such that $f(u)=v\ne 0$ 

$P(0,1)$ $\implies$ $f(0)=0$ and so $u\ne 0$
If $u=1$, $P(\frac xv,u)$ $\implies$ $f(x)=\frac{x^2}v$ which unfortunately is never a solution.
So $u\ne 1$

$P(f(x),f(y))$ $\implies$ $f(f(x)f(y))=(1-y)f(yf(x))+f(x)^2y^2f(y)$
And so $f(f(x)f(y))=(1-y)((1-x)f(xy)+x^2y^2f(x))+f(x)^2y^2f(y)$

So $f(f(x)f(y))=(1-x)(1-y)f(xy)+x^2y^2(1-y)f(x)+y^2f(x)^2f(y)$
Swapping $x,y$ and subtracting, we get :
$x^2y^2(1-y)f(x)+y^2f(x)^2f(y)=x^2y^2(1-x)f(y)+x^2f(x)f(y)^2$

Considering $xy\ne 0$, this may be written :
$(\frac{f(x)^2}{x^2}+x-1)f(y)=(\frac{f(y)^2}{y^2}+y-1)f(x)$

Setting there $y=u$, we get $\frac{f(x)^2}{x^2}+x-1=kf(x)$ where $k=\frac 1v(\frac{v^2}{u^2}+u-1)$


Which is $f(x)^2-kx^2f(x)+x^3-x^2=0$

$P(x,u)$ $\implies$ $f(vx)=(1-u)f(ux)+u^2vx^2$
Squaring this gives $f(vx)^2=(1-u)^2f(ux)^2+2u^2v(1-u)x^2f(ux)+u^4v^2x^4$
And so $kv^2x^2f(vx)+v^2x^2-v^3x^3=(1-u)^2(ku^2x^2f(ux)+u^2x^2-u^3x^3)+2u^2v(1-u)x^2f(ux)+u^4v^2x^4$

And so $kv^2x^2((1-u)f(ux)+u^2vx^2)+v^2x^2-v^3x^3$ $=(1-u)^2(ku^2x^2f(ux)+u^2x^2-u^3x^3)+2u^2v(1-u)x^2f(ux)+u^4v^2x^4$

Simplifying and considering $x\ne 0$, this is :

$(1-u)f(ux)(kv^2-k(1-u)u^2-2u^2v)=u^2v^2x^2(u^2-kv)+x(v^3-u^3(1-u)^2)+u^2(1-u)^2-v^2$
Note that RHS can only be the allzero polynomial if $u=v=k=2$ but then 
equation $f(x)^2-kx^2f(x)+x^3-x^2=0$ is not verified when $x=u$

So RHS is not the allzero polynomial and so xoefficient of $f(ux)$ in LHS can not be zero
And we have $f(x)=\alpha x^2+\beta x+\gamma$ $\forall x\ne 0$

It remains to plug this back in original equation and we get :
$\boxed{\text{S2 : }f(x)=-x^2+x\quad\forall x}$



\end{solution}



\begin{solution}[by \href{https://artofproblemsolving.com/community/user/29428}{pco}]
	\begin{tcolorbox}Find all $ f : R -> R$, such, that $ f(xf(y))=(1-y)f(xy)+x^2y^2f(y)$\end{tcolorbox}
Simpler :
Let $P(x,y)$ be the assertion $f(xf(y))=(1-y)f(xy)+x^2y^2f(y)$

$P(0,1)$ $\implies$ $f(0)=0$

If $f(1)\ne 0$ : $P(\frac x{f(1)},1)$ $\implies$ $f(x)=\frac{x^2}{f(1)}$ $\forall x$, which is never a solution
So $f(1)=0$

If $f(u)=0$ for some $u\notin\{0,1\}$, then :
$P(\frac xu,u)$ $\implies$ $\boxed{\text{S1 : }f(x)=0\quad\forall x}$ which indeed is a solution

So let us consider from now that $f(x)=0$ $\iff$ $x\in\{0,1\}$

$P(1,x)$ $\implies$ $f(f(x))=(x^2-x+1)f(x)$
This implies that $f(a)=f(b)$ for $a,b\notin\{0,1\}$ implies $a=b$ or $a+b=1$

Let $x\notin\{0,1\}$ :
$P(\frac 1x,x)$ $\implies$ $f(\frac{f(x)}x)=f(x)$ and so :
Either $\frac{f(x)}x=x$ and so $f(x)=x^2$
Either $\frac{f(x)}x+x=1$ and so $f(x)=x-x^2$

Let then $x\notin\{0,\frac 12,1\}$ such that $f(x)=x^2$
$P(1,x)$ $\implies$ $f(x^2)=(x^2-x+1)x^2$ and so :
Either $f(x^2)=x^4$ and so $x^4=(x^2-x+1)x^2$, impossible
Either $f(x^2)=x^2-x^4$ and so $x^2-x^4=(x^2-x+1)x^2$, impossible

So $\boxed{\text{S2 : }f(x)=x-x^2\quad\forall x}$ which indeed is a solution.
\end{solution}
*******************************************************************************
-------------------------------------------------------------------------------

\begin{problem}[Posted by \href{https://artofproblemsolving.com/community/user/369782}{mathenthusiastic}]
	find all injective function $f : \mathbb R \to \mathbb R $ such that:

 $\forall x\ne y\in\mathbb R :f(\frac{x+y}{x-y})=\frac{f(x)+f(y)}{f(x)-f(y)}$
	\flushright \href{https://artofproblemsolving.com/community/c6h1622408}{(Link to AoPS)}
\end{problem}



\begin{solution}[by \href{https://artofproblemsolving.com/community/user/222649}{mira74}]
	Plugging in $y=0$ immediately gives $f(1)=\frac{f(x)+f(0)}{f(x)-f(0)}$, which we can solve to get $f(x)=someconstant$ for all $x \neq 0$. However, $f$ must be injective which is a contradiction.
Edit: oops this is wrong
\end{solution}



\begin{solution}[by \href{https://artofproblemsolving.com/community/user/382168}{NahTan123xyz}]
	\begin{tcolorbox}Plugging in $y=0$ immediately gives $f(1)=\frac{f(x)+f(0)}{f(x)-f(0)}$, which we can solve to get $f(x)=someconstant$ for all $x \neq 0$. However, $f$ must be injective which is a contradiction.\end{tcolorbox}

What will happen if $f(1)=1, f(0)=0$ ? There'll be no contradiction! :)
\end{solution}



\begin{solution}[by \href{https://artofproblemsolving.com/community/user/335975}{Taha1381}]
	The main idea is that $\frac{x+y}{x-y}$ is hemogenized so $R.H.S$ won't change if you multiply $x,y$ by $k$.I mean if $P(x,y)$ denotes the assertion $f(\frac{x+y}{x-y})=\frac{f(x)+f(y)}{f(x)-f(y)}$ try to compare $P(x,y)$ and $P(kx,ky)$.The unique solution is $f(x)=x \forall x \in \mathbb{R}$.
\end{solution}



\begin{solution}[by \href{https://artofproblemsolving.com/community/user/29428}{pco}]
	\begin{tcolorbox}find all injective function $f : \mathbb R \to \mathbb R $ such that:

 $\forall x,y \in \mathbb R :f(\frac{x+y}{x-y})=\frac{f(x)+f(y)}{f(x)-f(y)}$\end{tcolorbox}

Let $x=y=1$ and functional equation (whose domain is explicitly stated "$\forall x,y\in\mathbb R$") is no longer true (both LHS, RHS are undefined).

So $\boxed{\text{No such function}}$

\end{solution}



\begin{solution}[by \href{https://artofproblemsolving.com/community/user/369782}{mathenthusiastic}]
	as you said $\frac{x+y}{x-y}$ should be a real number so we are not allowed to choose $x,y$ such that it becomes undefined. am I wrong?
\end{solution}



\begin{solution}[by \href{https://artofproblemsolving.com/community/user/29428}{pco}]
	\begin{tcolorbox}as you said $\frac{x+y}{x-y}$ should be a real number so we are not allowed to choose $x,y$ such that it becomes undefined. am I wrong?\end{tcolorbox}
You are allowed to use it since domain is $\forall x,y\in\mathbb R$
The classical real problem is "$\forall x\ne y\in\mathbb R$".
Here this constraint is not given and this is certainly a trap. hence my answer.

As I often repeat : the condition "where all the elements of expression are defined" is never considered as a default condition.

For example if I give you the problem :
$f(x)$ is from $\mathbb R^+\to\mathbb R^+$ and :
$f(x-f(x))=$ some expression $\forall x,y>0$
You generally immediately conclude $x-f(x)>0$
And this proves that the "where all the elements of expression are defined" is not considered as a default condition
(else $f(x)=x+x^2+9$ would always be a solution in my last example ... .


\end{solution}



\begin{solution}[by \href{https://artofproblemsolving.com/community/user/369782}{mathenthusiastic}]
	\begin{tcolorbox}[quote=mathenthusiastic]as you said $\frac{x+y}{x-y}$ should be a real number so we are not allowed to choose $x,y$ such that it becomes undefined. am I wrong?\end{tcolorbox}
You are allowed to use it since domain is $\forall x,y\in\mathbb R$
The classical real problem is "$\forall x\ne y\in\mathbb R$".
Here this constraint is not given and this is certainly a trap. hence my answer.

As I often repeat : the condition "where all the elements of expression are defined" is never considered as a default condition.

For example if I give you the problem :
$f(x)$ is from $\mathbb R^+\to\mathbb R^+$ and :
$f(x-f(x))=$ some expression $\forall x,y>0$
You generally immediately conclude $x-f(x)>0$
And this proves that the "where all the elements of expression are defined" is not considered as a default condition
(else $f(x)=x+x^2+9$ would always be a solution in my last example ... .\end{tcolorbox}

sorry, I have edited my question. can you help now? 
\end{solution}



\begin{solution}[by \href{https://artofproblemsolving.com/community/user/29428}{pco}]
	\begin{tcolorbox}find all injective function $f : \mathbb R \to \mathbb R $ such that:

 $\forall x\ne y\in\mathbb R :f(\frac{x+y}{x-y})=\frac{f(x)+f(y)}{f(x)-f(y)}$\end{tcolorbox}
Posted (and solved) many many times (2011, 2013, 2014, 2016, ..).

Dont hesitate to use the search function (see [url=http://www.artofproblemsolving.com\/community\/c6h1330604p7173058] here [\/url]).
Set (for example copy\/paste) in the "search term" field the exact following string : 

+"\frac{x+y}{x-y}" +"\frac{f(x)+f(y)}{f(x)-f(y)}"

You'll get in the \begin{bolded}ten first results\end{underlined}\end{bolded} (excluded your own post and this post itself) all the help you are requesting for.

[hide=(Some excuses)][size=70]I'm sorry not providing you the direct link to this result but I encountered users who never tried the search function, thinking quite easier to have other users make the search for them. So now I prefer to point to the search function and to give the appropriate search term (I checked that it indeed will give you the expected result) instead of the link itself [\/size]([url=https:\/\/en.wiktionary.org\/wiki\/give_a_man_a_fish_and_you_feed_him_for_a_day;_teach_a_man_to_fish_and_you_feed_him_for_a_lifetime]wink[\/url])[\/hide]


\end{solution}
*******************************************************************************
-------------------------------------------------------------------------------

\begin{problem}[Posted by \href{https://artofproblemsolving.com/community/user/339661}{mathematics4u}]
	\begin{bolded}\begin{italicized}If f is function defining from R to R satisfying f(3x+2) +f(3x+29) =0 then find all such functions.\end{italicized}\end{bolded}
	\flushright \href{https://artofproblemsolving.com/community/c6h1623365}{(Link to AoPS)}
\end{problem}



\begin{solution}[by \href{https://artofproblemsolving.com/community/user/29428}{pco}]
	\begin{tcolorbox}\begin{bolded}\begin{italicized}If f is function defining from R to R satisfying f(3x+2) +f(3x+29) =0 then find all such functions.\end{italicized}\end{bolded}\end{tcolorbox}

Trivial general solution : $f(x)=(-1)^{\left\lfloor\frac x{27}\right\rfloor}g(\left\{\frac x{27}\right\})$
Whatever is $g(x)$ from $[0,1)\to\mathbb R$

\end{solution}



\begin{solution}[by \href{https://artofproblemsolving.com/community/user/339661}{mathematics4u}]
	Any solution please

Is it a periodic function,

If it is so, then what is its period

\end{solution}



\begin{solution}[by \href{https://artofproblemsolving.com/community/user/29428}{pco}]
	\begin{tcolorbox}Any solution please

Is it a periodic function,

If it is so, then what is its period\end{tcolorbox}
I gave you\begin{bolded} all the solutions\end{underlined}\end{bolded}. What more are you asking for ?

And this function obviously at least has a period 54.

\end{solution}



\begin{solution}[by \href{https://artofproblemsolving.com/community/user/339661}{mathematics4u}]
	Solution means way of solving, 

Ok any how thank you
\end{solution}



\begin{solution}[by \href{https://artofproblemsolving.com/community/user/29428}{pco}]
	\begin{tcolorbox}Solution means way of solving, \end{tcolorbox}
I dont understand, what are you missing ?

It is trivial to check that my form is a solution
And just two lines to get that it is a general one.

What is the step you dont see ?


\end{solution}



\begin{solution}[by \href{https://artofproblemsolving.com/community/user/391068}{TuZo}]
	He, and me alsoo we can understand the following:
You claimed this: "Trivial general solution..." and you have a formula.
How we can PROVE that your solutions are ALL the solutions?
So, we want to see HOW YOU GET THIS SOLUTION?
\end{solution}



\begin{solution}[by \href{https://artofproblemsolving.com/community/user/375117}{achen29}]
	How did you get there? (I think)
\end{solution}



\begin{solution}[by \href{https://artofproblemsolving.com/community/user/29428}{pco}]
	OK, for all beginners :
The problem is $f(x+27)=-f(x)$
(which btw implies $f(x+54)=-f(x+27)=f(x)$)

Claim : a general solution is 
$f(x)=(-1)^{\left\lfloor\frac x{27}\right\rfloor}g(\left\{\frac x{27}\right\})$

1)  The claimed form indeed is a solution.
Moving $x\to x+27$, the $g(.)$ part is unchanged while the $(-1)^.$ part is multiplied by $-1$
Q.E.D.

2) The claimed form indeed is a general solution.
If $f(x)$ is a solution then :
$f(x+27n)=(-1)^nf(x)$ and so, writing $x=27\left\{\frac x{27}\right\}+27\left\lfloor\frac x{27}\right\rfloor$ :
$f(x)=(-1)^{\left\lfloor\frac x{27}\right\rfloor}f(27\left\{\frac x{27}\right\})$
Q.E.D.
\end{solution}



\begin{solution}[by \href{https://artofproblemsolving.com/community/user/29428}{pco}]
	\begin{tcolorbox}He, and me alsoo we can understand the following:
You claimed this: "Trivial general solution..." and you have a formula.
How we can PROVE that your solutions are ALL the solutions?
So, we want to see HOW YOU GET THIS SOLUTION?\end{tcolorbox}
Sorry, I really thought it was immediate.

Is my previous last post an answer to your questions ?


\end{solution}



\begin{solution}[by \href{https://artofproblemsolving.com/community/user/391068}{TuZo}]
	Yes, thank you!
\end{solution}
*******************************************************************************
-------------------------------------------------------------------------------

\begin{problem}[Posted by \href{https://artofproblemsolving.com/community/user/386303}{longnhat2002}]
	\[\begin{array}{l}
Find\,all\,funtion\,f:R \to R\,such\,that:\\
f(f(x + y)) = f(x + y) + f(x)f(y) - xy
\end{array}\]
	\flushright \href{https://artofproblemsolving.com/community/c6h1623893}{(Link to AoPS)}
\end{problem}



\begin{solution}[by \href{https://artofproblemsolving.com/community/user/29428}{pco}]
	\begin{tcolorbox}\[\begin{array}{l}
Find\,all\,funtion\,f:R \to R\,such\,that:\\
f(f(x + y)) = f(x + y) + f(x)f(y) - xy
\end{array}\]\end{tcolorbox}
Let $P(x,y)$ be the assertion $f(f(x+y))=f(x+y)+f(x)f(y)-xy$
Let $f(0)=a$

Subtracting $P(x+y,0)$ from $P(x,y)$, we get $af(x+y)=f(x)f(y)-xy$

If $a=0$, this easily gives $f(x)\equiv x$ or $f(x)\equiv -x$ an only first fits :
$\boxed{f(x)=x\quad\forall x}$

If $a\ne 0$, we have $f(x+y)=\frac 1af(x)f(y)-\frac 1axy$
And so $f(x+y+z)=\frac 1af(x+y)f(z)-\frac 1a(x+y)z$ and so
$f(x+y+z)=\frac 1{a^2}f(x)f(y)f(z)-\frac 1{a^2}xyf(z)-\frac 1axz-\frac 1ayz$
Swapping there $y,z$ and subtracting, we get :
$xyf(z)+axz=xzf(y)+axy$
Which is $\frac{f(z)-a}z=\frac{f(y)-a}y$ $\forall y,z\ne 0$
And so $f(x)=cx+a$ $\forall x\ne 0$, still true when $x=0$

Plugging this back in original equation, we get no solution with $a\ne 0$



\end{solution}
*******************************************************************************
-------------------------------------------------------------------------------

\begin{problem}[Posted by \href{https://artofproblemsolving.com/community/user/400713}{RnstTrjyn}]
	f : R --> R
f(a + 2f(a)f(b)) = f(a) + 2af(b)

	\flushright \href{https://artofproblemsolving.com/community/c6h1624578}{(Link to AoPS)}
\end{problem}



\begin{solution}[by \href{https://artofproblemsolving.com/community/user/365498}{Math-Ninja}]
	\begin{tcolorbox}$f : R\to R$
$f(a + 2f(a)f(b)) = f(a) + 2af(b)$\end{tcolorbox}


\end{solution}



\begin{solution}[by \href{https://artofproblemsolving.com/community/user/400713}{RnstTrjyn}]
	any ideas?
\end{solution}



\begin{solution}[by \href{https://artofproblemsolving.com/community/user/391407}{Hamel}]
	Wrong....
\end{solution}



\begin{solution}[by \href{https://artofproblemsolving.com/community/user/400713}{RnstTrjyn}]
	what if $f(-1\/2)=1\/2$
\end{solution}



\begin{solution}[by \href{https://artofproblemsolving.com/community/user/391407}{Hamel}]
	Oh yeah. You get the third $f(x)=-x$
\end{solution}



\begin{solution}[by \href{https://artofproblemsolving.com/community/user/400713}{RnstTrjyn}]
	but how?
\end{solution}



\begin{solution}[by \href{https://artofproblemsolving.com/community/user/400713}{RnstTrjyn}]
	how we get $f(x)=-x$ if $f(-1\/2)=1\/2$     ?
\end{solution}



\begin{solution}[by \href{https://artofproblemsolving.com/community/user/29428}{pco}]
	\begin{tcolorbox}f : R --> R
f(a + 2f(a)f(b)) = f(a) + 2af(b)\end{tcolorbox}
$\boxed{\text{S1 : }f(x)=0\quad\forall x}$ is  solution. So let us from now look only for non allzero solutions.
Let $P(x,y)$ be the assertion $f(x+2f(x)f(y))=f(x)+2xf(y)$
Let $u,v$ such that $f(u)=v\ne 0$
Let $A=f(\mathbb R)$

If $f(t)=0$ for some $t$; then $P(t,u)$ $\implies$ $t=0$ and so $f(0)=0$ (if such $t$ exists)
$P(-\frac 12,-\frac 12)$ $\implies$ $f(-\frac 12+2f(-\frac 12)^2)=0$ and so $f(-\frac 12)^2=\frac 14$
And since $f$ solution implies $-f$ solution, WLOG $f(-\frac 12)=-\frac 12$ and $f(0)=0$

$P(-\frac 12,x)$ $\implies$ $f(-\frac 12-f(x))=-\frac 12-f(x)$
$P(-\frac 12, -\frac 12-f(x))$ $\implies$ $f(f(x))=f(x)$
$P(f(x),-\frac 12-f(y))$ $\implies$ $f(-2f(x)f(y))=-2f(x)f(y)$
So $a,b\in A$ implies $-2ab\in A$
So $a,b,c,d\in A$ implies $-8abcd\in A$

Let $a\ne 0\in A$
$P(\frac 1{4a},-2f(x)a)$ $\implies$ $f(\frac 1{4a}-4f(\frac 1{4a})^2a)=0$
And so $f(\frac 1{4a})^2=\frac 1{16a^2}$ and so $\frac{\epsilon(a)}{4a}\in A$ for some $\epsilon(a)\in\{-1,+1\}$

Choosing then $a,b,c,d$ in the three-lines-above property as $\frac{\epsilon(a)}{4a},\frac{\epsilon(a)}{4a},a,b$ we get :
$-\frac{b}{2a}\in A$ $\forall a\ne 0\in A$ and $\forall b\in A$

$P(x,-2f(x)f(u))$ $\implies$ $f(x-4f(x)^2v)=f(x)(1-4xv)$ an so
$\frac{f(x-4f(x)^2v)}{f(x)}=1-4xv$ $\forall x\ne 0$
And so $\frac ba$ with $a,b\in A$, $a\ne 0$ can take any value we want

And so, combining with the fact that $-\frac{b}{2a}\in A$, we get that $f(x)$ is surjective

And the previously got $f(f(x))=f(x)$ implies then, using surjectivity
$\boxed{\text{S2 : }f(x)=x\quad\forall x}$ which indeed is a solution.

And since $f$ solution implies $-f$ solution, we get also :
$\boxed{\text{S3 : }f(x)=-x\quad\forall x}$ which indeed is a solution.




\end{solution}



\begin{solution}[by \href{https://artofproblemsolving.com/community/user/400713}{RnstTrjyn}]
	thanks a lot
\end{solution}
*******************************************************************************
-------------------------------------------------------------------------------

\begin{problem}[Posted by \href{https://artofproblemsolving.com/community/user/310950}{lmht}]
	Find $f(x): R \to R$ such that:

$f^2(1+2x)=x-f^3(1-x)$
	\flushright \href{https://artofproblemsolving.com/community/c6h1625901}{(Link to AoPS)}
\end{problem}



\begin{solution}[by \href{https://artofproblemsolving.com/community/user/392239}{NorthStarPolaris}]
	\begin{tcolorbox}Find $f(x): R \to R$ such that:

$f^2(1+2x)=x-f^3(1-x)$\end{tcolorbox}

What does $f^2(x)$ suppose to mean? $f(f(x))$ or $(f(x))^2$
\end{solution}



\begin{solution}[by \href{https://artofproblemsolving.com/community/user/310950}{lmht}]
	\begin{tcolorbox}[quote=lmht]Find $f(x): R \to R$ such that:

$f^2(1+2x)=x-f^3(1-x)$\end{tcolorbox}

What does $f^2(x)$ suppose to mean? $f(f(x))$ or $(f(x))^2$\end{tcolorbox}

(f(x))^2 :D
\end{solution}



\begin{solution}[by \href{https://artofproblemsolving.com/community/user/29428}{pco}]
	\begin{tcolorbox}Find $f(x): R \to R$ such that:

$f^2(1+2x)=x-f^3(1-x)$\end{tcolorbox}
Once again a fake problem, just crazy invented one.
Infinitely many solutions and certainly no general form for all of them.

Here is a method for building infinitely many (but certainly not all of them) :

Writing $g(x)=f(x+1)$ functional equation is $g(-2x)^2=-x-g(x)^3$

Let $h(x)$ any function any function from $[2,8)\to\mathbb R^-$ such that $h(x)\le\sqrt[3]{-x}$ $\forall x\in[2,8)$

Let then $a\in[2,8)$ and the following sequences $a_n,b_n)$ :
$a_0=h(a)$ and $a_{n+1}=\sqrt[3]{-\frac a{(-2)^{n+1}}-a_{n}^2}$
$b_0=h(a)$ and $b_{n+1}=-\sqrt{-(-2)^na-b_n^3}$

The first important thing is to show that $b_n$ is always defined and so that the quantity inside the $\sqrt .$ sign is $\ge 0$
$b_0$ is defined and $<\sqrt[3]{-a}$
So $b-1=-\sqrt{-a-b_0^3}$ is defined
Suppose then $b_{2n-1}$ is defined and $<0$ for some $n\in\mathbb Z_{>0}$ :
$-(-2)^{2n-1}a>0$ and $-b_{2n-1}^3>$ and so $b_{2n}=-\sqrt{-(-2)^{2n-1}a-b_{2n-1}^3}$is indeed defined
Note that above line implies $-b_{2n}\ge \sqrt{2^{2n-1}a}$
So $-b_{2n}^3\ge 2^{\frac{6n-3}2}a^{\frac 32}$
And it is easy to show that $a\ge 2$ implies $2^{\frac{6n-3}2}a^{\frac 32}\ge 2^{2n}a$
And so $-(-2)^{2n}a-b_{2n}^3\ge 0$ and so $b_{2n+1}=-\sqrt{-(-2)^{2n}a-b_{2n}^3}$ is defined
So $b_n$ is well defined $\forall n\in\mathbb Z_{\ge 0}$

From there, we easy get that defining $g(\frac a{(-2)^n})=a_n$ and $g((-2)^na)=b_n$
gives consistant values for $g(x)$ for all $x=a(-2)^n$ where $n\in\mathbb Z$

Applying this method $\forall a\in[2,8)$ gives full consistant values for $g(x)$ all over $\mathbb R\setminus\{0\}$

And choosing any $g(0)\in\{0,-1\}$ ends the full consistant definition of $g(x)$ all over $\mathbb R$


\end{solution}



\begin{solution}[by \href{https://artofproblemsolving.com/community/user/310950}{lmht}]
	\begin{tcolorbox}[quote=lmht]Find $f(x): R \to R$ such that:

$f^2(1+2x)=x-f^3(1-x)$\end{tcolorbox}
Once again a fake problem, just crazy invented one.
Infinitely many solutions and certainly no general form for all of them.

Here is a method for building infinitely many (but certainly not all of them) :

Writing $g(x)=f(x+1)$ functional equation is $g(-2x)^2=-x-g(x)^3$

Let $h(x)$ any function any function from $[2,8)\to\mathbb R^-$ such that $h(x)\le\sqrt[3]{-x}$ $\forall x\in[2,8)$

Let then $a\in[2,8)$ and the following sequences $a_n,b_n)$ :
$a_0=h(a)$ and $a_{n+1}=\sqrt[3]{-\frac a{(-2)^{n+1}}-a_{n}^2}$
$b_0=h(a)$ and $b_{n+1}=-\sqrt{-(-2)^na-b_n^3}$

The first important thing is to show that $b_n$ is always defined and so that the quantity inside the $\sqrt .$ sign is $\ge 0$
$b_0$ is defined and $<\sqrt[3]{-a}$
So $b-1=-\sqrt{-a-b_0^3}$ is defined
Suppose then $b_{2n-1}$ is defined and $<0$ for some $n\in\mathbb Z_{>0}$ :
$-(-2)^{2n-1}a>0$ and $-b_{2n-1}^3>$ and so $b_{2n}=-\sqrt{-(-2)^{2n-1}a-b_{2n-1}^3}$is indeed defined
Note that above line implies $-b_{2n}\ge \sqrt{2^{2n-1}a}$
So $-b_{2n}^3\ge 2^{\frac{6n-3}2}a^{\frac 32}$
And it is easy to show that $a\ge 2$ implies $2^{\frac{6n-3}2}a^{\frac 32}\ge 2^{2n}a$
And so $-(-2)^{2n}a-b_{2n}^3\ge 0$ and so $b_{2n+1}=-\sqrt{-(-2)^{2n}a-b_{2n}^3}$ is defined
So $b_n$ is well defined $\forall n\in\mathbb Z_{\ge 0}$

From there, we easy get that defining $g(\frac a{(-2)^n})=a_n$ and $g((-2)^na)=b_n$
gives consistant values for $g(x)$ for all $x=a(-2)^n$ where $n\in\mathbb Z$

Applying this method $\forall a\in[2,8)$ gives full consistant values for $g(x)$ all over $\mathbb R\setminus\{0\}$

And choosing any $g(0)\in\{0,-1\}$ ends the full consistant definition of $g(x)$ all over $\mathbb R$\end{tcolorbox}

Sorry, I don't know how to say, I mean the simple of f(x), for example $f(x)=(kx+h)^a, f(x)=a^kx+h, f(x)=ln(x), f(x)=tanx+cosx$.. I wonder if it has a function like that, sorry beacuse I don't know how to write  :( 
\end{solution}



\begin{solution}[by \href{https://artofproblemsolving.com/community/user/29428}{pco}]
	\begin{tcolorbox}Sorry, I don't know how to say, I mean the simple of f(x), for example $f(x)=(kx+h)^a, f(x)=a^kx+h, f(x)=ln(x), f(x)=tanx+cosx$.. I wonder if it has a function like that, sorry beacuse I don't know how to write  :(\end{tcolorbox}
The best method : just copy the exact words you got in your olympiad exam \/ ytaining session ... .



\end{solution}
*******************************************************************************
-------------------------------------------------------------------------------

\begin{problem}[Posted by \href{https://artofproblemsolving.com/community/user/403801}{cassia}]
	Find out all the functions f:N to N,"N" is a non negative integer set,meet at the same time,(1)f(1)>0;(2)For any m,n belong to N,such that f(m^2+n^2)=f^2(m)+f^2(n).
	\flushright \href{https://artofproblemsolving.com/community/c6h1625918}{(Link to AoPS)}
\end{problem}



\begin{solution}[by \href{https://artofproblemsolving.com/community/user/403801}{cassia}]
	No one can solve?
\end{solution}



\begin{solution}[by \href{https://artofproblemsolving.com/community/user/29428}{pco}]
	\begin{tcolorbox}Find out all the functions f:N to N,"N" is a non negative integer set,meet at the same time,(1)f(1)>0;(2)For any m,n belong to N,such that f(m^2+n^2)=f^2(m)+f^2(n).\end{tcolorbox}
Let $P(m,n)$ be the assertion $f(m^2+n^2)=f(m)^2+f(n)^2$
Let $a=f(3)$

Subtracting $P(2n-1,n+2)$ from $P(2n+1,n-2)$, we get :
Assertion $O(n)$ : $f(2n+1)^2=f(2n-1)^2+f(n+2)^2-f(n-2)^2$ $\forall n\ge 2$

Subtracting $P(2n-2,n+4)$ from $P(2n+2,n-4)$, we get :
Assertion $E(n)$ : $f(2n+2)^2=f(2n-2)^2+f(n+4)^2-f(n-4)^2$ $\forall n\ge 4$

$P(0,0)$ $\implies$ $f(0)=0$
$P(1,0)$ $\implies$ $f(1)=1$
$P(1,1)$ $\implies$ $f(2)=2$
$P(2,0)$ $\implies$ $f(4)=4$
$P(2,2)$ $\implies$ $f(8)=8$
$O(2)$ $\implies$ $f(5)^2=a^2+16$ and so $(f(5),a)\in\{(4,0),(5,3)\}$
$O(3)$ $\implies$ $f(7)^2=2f(5)^2-1$ and so $f(5)=4$ does not fit and so :
$f(3)=3$
$f(5)=5$
$f(7)=7$
$P(3,0)$ $\implies$ $f(9)=9$
$O(4)$ $\implies$ $f(6)=6$

And so we got $f(n)=n$ $\forall n\in\{0,1,2,3,4,5,6,7,8,9\}$
From there, the pair $O(n),E(n)$ easily allows induction for proving 
$\boxed{f(n)=n\quad\forall n\in\mathbb Z_{\ge 0}}$



\end{solution}



\begin{solution}[by \href{https://artofproblemsolving.com/community/user/400713}{RnstTrjyn}]
	We can show that the only such function is the identity function f(k) = k. To prove this, first note that setting m = 1 and n = 0 in equation (1) and re-arranging terms gives the condition
$f(0)^2=f(1)-f(1)^2=f(1)[1-f(1)]$              \begin{bolded}(1)\end{bolded}
Since the integer f(1) is positive, and the square integer f(0)2 must be non-negative, it follows that f(1) must equal 1, and hence f(0) = 0. We then immediately have
$f(m^2)=f(m)^2$                    \begin{bolded}(2)\end{bolded}
for all positive integers m. Also, setting m = 1 and n = 1 in equation (1) gives
$f(2)=f(1^2+1^2)=f(1)^2+f(1)^2=2$
Using this result along with equation (2) gives f(4) = 4, and we can then determine f(5) from the condition
$f(5)=f(1^2+2^2)=f(1)^2+f(2)^2=5$
We can now make use of the smallest Pythaogrean triple to give
$f(25)=f(3^2+4^2)=f(3)^2+f(4)^2=f(5)^2$
Since we know f(4) = 4 and f(5) = 5, this equation gives f(3)2 = 9, and hence f(3) = 3. Using these results we also have
$f(8)=f(2^2+2^2)=f(2)^2+f(2)^2=8$
$f(9)=f(3^2)=f(3)^2=9$
$f(10)=f(3^2+1^2)=f(3)^2+f(1)^2=10$
$f(64)=f(8^2)=f(8)^2=64$
$f(8)=f(10^2)=f(10)^2=100$
Doubling the terms of the smallest Pythagorean triple, we can now compute
$f(10^2)=f(6^2+8^2)=f(6)^2+f(8)^2$
Using the values of f(8) and f(10), we get f(6)2 = 36 and hence f(6) = 6. In summary, we have determined that f(k) = k for all k less than 7.
Now, to generalize this procedure, suppose that f(k) = k for all k less than m, and suppose for the integer m there exist three smaller integers n,r,s such that
$m^2+n^2=r^2+s^2$                  \begin{bolded}(3)\end{bolded}
Applying the function f to both sides and making use of equation (1), this would imply
$f(m)^2+f(n)^2=f(r)^2+f(s)^2$
Furthermore, since by assumption we have f(k) = k for all k less than m, and since by assumption the integers n,r,s are each less than m, it follows that
$f(m)^2+n^2=r^2+s^2$
and therefore f(m) = m. Thus we need only prove that, for every integer m greater than 6, there exist smaller integers n,r,s such that equation (3) is satisfied. To prove this, we first re-write equation (3) in the form
$m^2-r^2=s^2-n^2$
If we choose r with the same parity as m, then we have the following integer factorizations of both sides
$(2(m-r))((m+r)\/2)=(s-n)(s+n)$
Identifying the factors on the two sides of this equation, we get two conditions
$2(m-r)=(s-n)$ and $m+r=2(s+n)$
Solving these for s and n, we get the integers
$s=(5m-3r)\/4$ and $m=(5r-3m)\/4$
To make these as small as possible, while still having r smaller than m and with the same parity, we set r = m-2. Substituting this into the expressions for s and n, and inserting these back into equation (3), we arrive at the identity
$m^2+((m-5)\/2)^2=(m-2)^2+((m+3)\/2)^2$
Note that for every odd integer m greater than 6, the three other terms in this equation are strictly less than m. For even values of m, we can multiply through the above equation by 22, and then replace 2m with m. This gives the identity
$m^2+(m\/2-5)^2=(m-4)^2+((m\/2+3)^2$
For every even integer m greater than 6, the three other terms in this equation are strictly less than m. Therefore, using these two equations, along with equation (1), we can complete the induction step, so we have proven that f(k) = k for all integers k.

 

We might wonder if a similar proof can be made to work if we allow f(k) to take on real values. The value of f(0) is forced by the relation
$f(0)=f(0)^2+f(0)^2$
to be either 0 or 1\/2. If we let f(0) = 1\/2 then we have a quadratic in f(1)
$f(1)=1\/4+f(1)^2$
which implies that f(1) also equals 1\/2. Then we have the equation
f(2)=2*(f(1)^2)=1\/2
From here it's clear that setting f(k) = 1\/2 for all k gives a consistent solution, and this is the only other solution besides f(k) = k (which, as shown above, is the only solution for integer functions).

\end{solution}
*******************************************************************************
-------------------------------------------------------------------------------

\begin{problem}[Posted by \href{https://artofproblemsolving.com/community/user/390773}{Georgemoukas}]
	Find $f$: $\mathbb{R} \to \mathbb{R},$ if $f(x) f(xf(y)) = x^2 f(y)$
	\flushright \href{https://artofproblemsolving.com/community/c6h1626253}{(Link to AoPS)}
\end{problem}



\begin{solution}[by \href{https://artofproblemsolving.com/community/user/29428}{pco}]
	\begin{tcolorbox}Find $f$: $\mathbb{R} \to \mathbb{R},$ if $f(x) f(xf(y)) = x^2 f(y)$\end{tcolorbox}
$\boxed{\text{S1 : }f(x)=0\quad\forall x}$ is a solution. So let us romnow look only for non allzero solutions.
Let $P(x,y)$ be the assertion $f(x)f(xf(y))=x^2f(y)$
Let $u,v$ such that $f(u)=v\ne 0$
Let $a=f(1)$

$P(0,0)$ $\implies$ $f(0)=0$
If $f(t)=0$ for some $t\ne 0$, then $P(t,u)$ $\implies$ $v=0$, impossible
So $f(x)=0\iff x=0$ (and so $a\ne 0$)

Let $x$ such that $vx>0$
$P(\sqrt{\frac xv},u)$ $\implies$ $f(\sqrt{\frac xv})f(v\sqrt{\frac xv}))=x$
So any nonzero real with same sign than $v$ may be written as $x=f(w)f(t)$ for some real $w,t\ne 0$

Let then such nonzero $x$ with same sign than $v$
$\exists w,t\ne 0$ such that $x=f(w)f(t)$
$P(1,w)$ $\implies$ $f(f(w))=\frac{f(w)}a$
$P(f(w),t)$ $\implies$ $f(f(w)f(t))=af(w)f(t)$ $\forall w,t\ne 0$
And so $f(x)=ax$ for any nonzero $x$ with same sign than $v$

Obviously, if $\exists$ $u'\ne 0$ such that $f(u')=v'$ with opposit sign to $v$, same path leads to
$f(x)=ax$ for any nonzero $x$ with same sign than $v'$
And so $f(x)=ax$ $\forall x$
Plugging this back in original equation, we get $a^2=1$ and so :
$\boxed{\text{S2 : }f(x)=x\quad\forall x}$ and $\boxed{\text{S3 : }f(x)=-x\quad\forall x}$

On the other hand, if $f(x)$ has a constant sign, we got
$f(x)=ax$ for all $x$ with this common sign, which then must be $>0$ (since $f(x)$ and $a$ both have same sign.
So $f(x)>0$ $\forall x\ne 0$ and $f(x)=ax$ $\forall x\ge 0$
From $f(f(x))=\frac 1af(x)$, we then get $a=1$
Then $P(x,1)$ $\implies$ $f(x)^2=x^2$
And so, since $f(x)>0$ $\forall x\ne 0$ :
$\boxed{\text{S4 : }f(x)=|x|\quad\forall x}$ which indeed is a solution



\end{solution}



\begin{solution}[by \href{https://artofproblemsolving.com/community/user/405366}{Smita}]
	\begin{tcolorbox}[quote=Georgemoukas]Find $f$: $\mathbb{R} \to \mathbb{R},$ if $f(x) f(xf(y)) = x^2 f(y)$\end{tcolorbox}
$\boxed{\text{S1 : }f(x)=0\quad\forall x}$ is a solution. So let us romnow look only for non allzero solutions.
Let $P(x,y)$ be the assertion $f(x)f(xf(y))=x^2f(y)$
Let $u,v$ such that $f(u)=v\ne 0$
Let $a=f(1)$

$P(0,0)$ $\implies$ $f(0)=0$
If $f(t)=0$ for some $t\ne 0$, then $P(t,u)$ $\implies$ $v=0$, impossible
So $f(x)=0\iff x=0$ (and so $a\ne 0$)

Let $x$ such that $vx>0$
$P(\sqrt{\frac xv},u)$ $\implies$ $f(\sqrt{\frac xv})f(v\sqrt{\frac xv}))=x$
So any nonzero real with same sign than $v$ may be written as $x=f(w)f(t)$ for some real $w,t\ne 0$

Let then such nonzero $x$ with same sign than $v$
$\exists w,t\ne 0$ such that $x=f(w)f(t)$
$P(1,w)$ $\implies$ $f(f(w))=\frac{f(w)}a$
$P(f(w),t)$ $\implies$ $f(f(w)f(t))=af(w)f(t)$ $\forall w,t\ne 0$
And so $f(x)=ax$ for any nonzero $x$ with same sign than $v$

Obviously, if $\exists$ $u'\ne 0$ such that $f(u')=v'$ with opposit sign to $v$, same path leads to
$f(x)=ax$ for any nonzero $x$ with same sign than $v'$
And so $f(x)=ax$ $\forall x$
Plugging this back in original equation, we get $a^2=1$ and so :
$\boxed{\text{S2 : }f(x)=x\quad\forall x}$ and $\boxed{\text{S3 : }f(x)=-x\quad\forall x}$

On the other hand, if $f(x)$ has a constant sign, we got
$f(x)=ax$ for all $x$ with this common sign, which then must be $>0$ (since $f(x)$ and $a$ both have same sign.
So $f(x)>0$ $\forall x\ne 0$ and $f(x)=ax$ $\forall x\ge 0$
From $f(f(x))=\frac 1af(x)$, we then get $a=1$
Then $P(x,1)$ $\implies$ $f(x)^2=x^2$
And so, since $f(x)>0$ $\forall x\ne 0$ :
$\boxed{\text{S4 : }f(x)=|x|\quad\forall x}$ which indeed is a solution\end{tcolorbox}
 
GENIUS SOLUTION


\end{solution}



\begin{solution}[by \href{https://artofproblemsolving.com/community/user/402732}{Arkmmq}]
	If f(1)=0 then set x=0 we get f(x)=0

If f(1)$\neq$0 then set x=y=1
 
$\Rightarrow$ff(1)=1.

set y=f(1) 

$\Rightarrow$$f(x)^2$=$x^2$ .

and now complete as PCO. 


\end{solution}
*******************************************************************************
-------------------------------------------------------------------------------

\begin{problem}[Posted by \href{https://artofproblemsolving.com/community/user/333236}{Xurshid.Turgunboyev}]
	Find all the real numbers  $a $ and the functions  $f (x)$ defined for all real  $x $ and take real values that satisfy the following two conditions:

$(i) $ the following inequality holds  $af(x)-x\le af (f (y))-y $ for all real numbers  $x $ and $y$.

$(ii)$ there is  a real number  $x_{0} $ is such that  $f (x_{0})=x_{0}$.
	\flushright \href{https://artofproblemsolving.com/community/c6h1626623}{(Link to AoPS)}
\end{problem}



\begin{solution}[by \href{https://artofproblemsolving.com/community/user/29428}{pco}]
	\begin{tcolorbox}Find all the real numbers  $a $ and the functions  $f (x)$ defined for all real  $x $ and take real values that satisfy the following two conditions:

$(i) $ the following inequality holds  $af(x)-x\le af (f (y))-y $ for all real numbers  $x $ and $y$.

$(ii)$ there is  a real number  $x_{0} $ is such that  $f (x_{0})=x_{0}$.\end{tcolorbox}
Let $P(x,y)$ be the assertion $af(x)-x\le af(f(y))-y$

$P(f(x),x)$ $\implies$ $f(x)\ge x$ $\forall x$

If $a<0$ : $f(f(y))\ge f(y)$ implies $af(f(y))\le af(y)$ and so :
$af(x)-x\le af(f(y))-y\le af(y)-y$
And so $af(x)-x=c$ constant but then $c=af(x)-x\le ax-x$ impossible if $a<0$

So $a\ge 0$ and so $(a-1)x\le af(x)-x\le af(f(y))-y$ and so $a=1$
$P(x,x_0)$ becomes then $f(x)-x\le f(f(x_0))-x_0=0$

And so $\boxed{a=1\text{  and  }f(x)=x\quad\forall x}$ which indeed is a solution.


\end{solution}
*******************************************************************************
-------------------------------------------------------------------------------

\begin{problem}[Posted by \href{https://artofproblemsolving.com/community/user/400713}{RnstTrjyn}]
	$f: (0,+\infty)\rightarrow (0,+\infty)$
$f(x)f(y)=2f(x+yf(x))$
	\flushright \href{https://artofproblemsolving.com/community/c6h1626670}{(Link to AoPS)}
\end{problem}



\begin{solution}[by \href{https://artofproblemsolving.com/community/user/353869}{FuadAnzurov2003}]
	First answer: f(x)=0
Second answer: f(x)=2
\end{solution}



\begin{solution}[by \href{https://artofproblemsolving.com/community/user/375117}{achen29}]
	\begin{tcolorbox}First answer: f(x)=0
Second answer: f(x)=2\end{tcolorbox}

note that 0 is not part of the range of f!
\end{solution}



\begin{solution}[by \href{https://artofproblemsolving.com/community/user/345056}{Lamp909}]
	Already posted.
\end{solution}



\begin{solution}[by \href{https://artofproblemsolving.com/community/user/400713}{RnstTrjyn}]
	\begin{tcolorbox}Already posted.\end{tcolorbox}

Can u send me link?
\end{solution}



\begin{solution}[by \href{https://artofproblemsolving.com/community/user/29428}{pco}]
	\begin{tcolorbox}$f: (0,+\infty)\rightarrow (0,+\infty)$
$f(x)f(y)=2f(x+yf(x))$\end{tcolorbox}
Let $P(x,y)$ be the assertion $f(x)f(y)=2f(x+yf(x))$

1) $f(x)>1$ $\forall x>0$
If $f(x)<1$ for some $x>0$, then $P(x,\frac x{1-f(x)})$ implies $f(x)=2$ contradiction
If $f(x)=1$ for some $x>0$, then $P(x,x)$ implies $f(2x)=\frac 12<1$ impossible
Q.E.D.

2) there are no injective solutions.
If $f(x)$ is injective, then comparing $P(x,y)$ with $P(y,x)$ and using injectivity, we get 
$x+yf(x)=y+xf(y)$, which is $\frac{f(x)-1}x=\frac{f(y)-1}y$
And so $f(x)=cx+1$ with $c\ne 0$ (since injective), which unfortunately is never a solution
Q.E.D.

3) the only non injective solution is $\boxed{f(x)=2\quad\forall x>0}$
Since non injective, let $a,b,T>0$ such that $f(a+T)=f(a)=b$

Subtracting $P(a,\frac{x-a}c)$ from $P(a+T,\frac{x-a}c)$, we get $f(x+T)=f(x)$ $\forall x>a$

Let then $x>0$. We know that $f(x)>1$ and let then $n\in\mathbb Z_{>0}$ such that $y=\frac {nT-x}{f(x)-1}>a$
We have $f(x)f(y+nT)=f(x)f(y)=2f(x+yf(x))$
And since $y+nT=x+yf(x)$, this becomes $f(x)=2$
Q.E.D.
\end{solution}
*******************************************************************************
-------------------------------------------------------------------------------

\begin{problem}[Posted by \href{https://artofproblemsolving.com/community/user/368111}{Anis2017}]
	\begin{tcolorbox}Let $\mathbb{R}$ be the set of real numbers. Determine all functions $f: \mathbb{R} \rightarrow \mathbb{R}$ such that, for any real numbers $x$ and $y$, \[ f(f(x)f(y))  =f(xy)+ f(x+y)
	\flushright \href{https://artofproblemsolving.com/community/c6h1627026}{(Link to AoPS)}
\end{problem}



\begin{solution}[by \href{https://artofproblemsolving.com/community/user/363884}{matinyousefi}]
	any result?
\end{solution}



\begin{solution}[by \href{https://artofproblemsolving.com/community/user/375117}{achen29}]
	Letting $f(0)= c$; and through $P(0,y)$; we get $\boxed{f(y)=o}$ for all y; which is a solution.
\end{solution}



\begin{solution}[by \href{https://artofproblemsolving.com/community/user/363884}{matinyousefi}]
	$$f(f(x)f(y)) =f(xy)+ f(x+y):P(x,y)$$
$$P(c,y):f(f(c)f(y)) =f(cy)+ f(c+y)$$
and then?
\end{solution}



\begin{solution}[by \href{https://artofproblemsolving.com/community/user/375117}{achen29}]
	Sorry I meant P(0,y)
\end{solution}



\begin{solution}[by \href{https://artofproblemsolving.com/community/user/398616}{MNJ2357}]
	$f(x)=x+1,-x-1$ are solutions as well.

\end{solution}



\begin{solution}[by \href{https://artofproblemsolving.com/community/user/375117}{achen29}]
	We have the clear conclusion that for f satisfying the given, so does -f satisfy the given condition. 
 
\end{solution}



\begin{solution}[by \href{https://artofproblemsolving.com/community/user/29428}{pco}]
	\begin{tcolorbox}Let $\mathbb{R}$ be the set of real numbers. Determine all functions $f: \mathbb{R} \rightarrow \mathbb{R}$ such that, for any real numbers $x$ and $y$, \[ f(f(x)f(y))  =f(xy)+ f(x+y)\end{tcolorbox}
http://artofproblemsolving.com\/community\/c6h1482251p8682062


\end{solution}
*******************************************************************************
-------------------------------------------------------------------------------

\begin{problem}[Posted by \href{https://artofproblemsolving.com/community/user/295501}{m4ths}]
	Find all functions \[f:\mathbb{R}\rightarrow \mathbb{R}\]
such that \[f(x^{2}+f(y))=y+f^{2}(x) \ for \ each \ x,y \ \epsilon \ \mathbb{R}\]

	\flushright \href{https://artofproblemsolving.com/community/c6h1627136}{(Link to AoPS)}
\end{problem}



\begin{solution}[by \href{https://artofproblemsolving.com/community/user/29428}{pco}]
	\begin{tcolorbox}Find all functions \[f:\mathbb{R}\rightarrow \mathbb{R}\]
such that \[f(x^{2}+f(y))=y+f^{2}(x) \ for \ each \ x,y \ \epsilon \ \mathbb{R}\]\end{tcolorbox}
Posted (and solved) many times (2008, 2012, twice in 2017).

Dont hesitate to use the search function (see [url=http://www.artofproblemsolving.com\/community\/c6h1330604p7173058] here [\/url]).
Set (for example copy\/paste) in the "search term" field the exact following string : 

+"f(x^2+f(y))" +"f(x)^2"

You'll get in the \begin{bolded}ten first results\end{underlined}\end{bolded} (excluded your own post and this post itself) all the help you are requesting for.

[hide=(Some excuses)][size=70]I'm sorry not providing you the direct link to this result but I encountered users who never tried the search function, thinking quite easier to have other users make the search for them. So now I prefer to point to the search function and to give the appropriate search term (I checked that it indeed will give you the expected result) instead of the link itself [\/size]([url=https:\/\/en.wiktionary.org\/wiki\/give_a_man_a_fish_and_you_feed_him_for_a_day;_teach_a_man_to_fish_and_you_feed_him_for_a_lifetime]wink[\/url])[\/hide]





\end{solution}
*******************************************************************************
-------------------------------------------------------------------------------

\begin{problem}[Posted by \href{https://artofproblemsolving.com/community/user/224483}{Dadgarnia}]
	Find all functions $f:\mathbb{R}\rightarrow \mathbb{R}$ that satisfy the following conditions:
a. $x+f(y+f(x))=y+f(x+f(y)) \quad \forall x,y \in \mathbb{R}$
b. The set $I=\left\{\frac{f(x)-f(y)}{x-y}\mid x,y\in \mathbb{R},x\neq y \right\}$ is an interval.

\begin{italicized}Proposed by Navid Safaei\end{italicized}
	\flushright \href{https://artofproblemsolving.com/community/c6h1627524}{(Link to AoPS)}
\end{problem}



\begin{solution}[by \href{https://artofproblemsolving.com/community/user/29428}{pco}]
	\begin{tcolorbox}Find all functions $f:\mathbb{R}\rightarrow \mathbb{R}$ that satisfy the following conditions:
a. $x+f(y+f(x))=y+f(x+f(y)) \quad \forall x,y \in \mathbb{R}$
b. The set $I=\left\{\frac{f(x)-f(y)}{x-y}\mid x,y\in \mathbb{R},x\neq y \right\}$ is an interval.

\begin{italicized}Proposed by Navid Safaei\end{italicized}\end{tcolorbox}

When you write "an interval", do you mean a "bounded interval" (which implies continuity for example)? or is $[0,+\infty)$ for example considered as an "interval" in your problem statement ?
\end{solution}



\begin{solution}[by \href{https://artofproblemsolving.com/community/user/224483}{Dadgarnia}]
	\begin{tcolorbox}[quote=Dadgarnia]Find all functions $f:\mathbb{R}\rightarrow \mathbb{R}$ that satisfy the following conditions:
a. $x+f(y+f(x))=y+f(x+f(y)) \quad \forall x,y \in \mathbb{R}$
b. The set $I=\left\{\frac{f(x)-f(y)}{x-y}\mid x,y\in \mathbb{R},x\neq y \right\}$ is an interval.

\begin{italicized}Proposed by Navid Safaei\end{italicized}\end{tcolorbox}

When you write "an interval", do you mean a "bounded interval" (which implies continuity for example)? or is $[0,+\infty)$ for example considered as an "interval" in your problem statement ?\end{tcolorbox}
Yes this implies continuity but this is not necessarily a bounded interval. 

\end{solution}



\begin{solution}[by \href{https://artofproblemsolving.com/community/user/29428}{pco}]
	\begin{tcolorbox}Find all functions $f:\mathbb{R}\rightarrow \mathbb{R}$ that satisfy the following conditions:
a. $x+f(y+f(x))=y+f(x+f(y)) \quad \forall x,y \in \mathbb{R}$
b. The set $I=\left\{\frac{f(x)-f(y)}{x-y}\mid x,y\in \mathbb{R},x\neq y \right\}$ is an interval.\end{tcolorbox}
Let $P(x,y)$ be the assertion $x+f(y+f(x))=y+f(x+f(y))$

If $1\in I$, then $\exists x\ne y$ such that $f(x)-f(y)=x-y$
And so $y+f(x)=x+f(y)$ and then $P(x,y)$ implies $x=y$, contradiction.
So $1\notin I$

It $t\ne 1\in I$, then let $x\ne y$ such that $\frac{f(x)-f(y)}{x-y}=t$
We have $x+f(y)\ne y+f(x)$ and 
$A=\frac{f(x+f(y))-f(y+f(x))}{(x+f(y))-(y+f(x))}=\frac{x-y}{(x-y)-(f(x)-f(y)}=\frac 1{1-t}$

So $t\in I$ implies $\frac 1{1-t}\in I$ which unfortunately is impossible :
If $t>1$, then $\frac 1{1-t}<0$ and we can not have both in I (else $1\in I$)

If $t<1$, then we need $\frac 1{1-t}<1$ which means $t<0$ but then $\frac 1{1-t}>0$ and replacing $t$ by $\frac 1{1-t}$ gives contradiction.

So no such I.
So $\boxed{\text{no such function}}$


\end{solution}



\begin{solution}[by \href{https://artofproblemsolving.com/community/user/288210}{tenplusten}]
	Can anyone explain me what does "being interval" mean i.e if some set is interval what do we know about that set?
Thanks!
\end{solution}



\begin{solution}[by \href{https://artofproblemsolving.com/community/user/29428}{pco}]
	\begin{tcolorbox}Can anyone explain me what does "being interval" mean i.e if some set is interval what do we know about that set?
Thanks!\end{tcolorbox}
The only property of "interval" I used in my previous post is : $\forall x<y\in I$, then $[x,y]\subseteq I$


\end{solution}



\begin{solution}[by \href{https://artofproblemsolving.com/community/user/288210}{tenplusten}]
	\begin{tcolorbox}
 $\forall x<y\in I$\end{tcolorbox}
Does $x $ also belong to $I $ ,or just $y $?

\begin{tcolorbox} $[x,y]\subseteq I$\end{tcolorbox}
Does it mean any number in the interval $[x,y] $ belongs to $I $.

Sorry if above questions are stupid.
Thanks!
\end{solution}



\begin{solution}[by \href{https://artofproblemsolving.com/community/user/29428}{pco}]
	What I wrote : $x,y\in I$ and $x<y$ $\implies$ $z\in I$ $\forall z$ such that $x\le z\le y$

\end{solution}



\begin{solution}[by \href{https://artofproblemsolving.com/community/user/288210}{tenplusten}]
	Lol.I now see that first question is quite stupid.
Thanks for help.
\end{solution}
*******************************************************************************
-------------------------------------------------------------------------------

\begin{problem}[Posted by \href{https://artofproblemsolving.com/community/user/295501}{m4ths}]
	Find all functions 
\[f:\mathbb{R}\rightarrow \mathbb{R} \ ; \ f(x+y)=f(x)\cdot f(y)\cdot f(xy)\]

	\flushright \href{https://artofproblemsolving.com/community/c6h1627540}{(Link to AoPS)}
\end{problem}



\begin{solution}[by \href{https://artofproblemsolving.com/community/user/312364}{Rmo}]
	This is an old Indian national Olympiad problem
My solution
First take y=x\/x-1 then y=1\/x-1.replace x with x+1 and claim that f(x)=f(1\/x) and now use f(x)f(x-1\/x)=1 to prove that f(x-1\/x)f(1\/1-x)=1 again use this to show that f(1\/1-x)f(x)=1 now claim that f(x)=+-1 and the rest is easy
\end{solution}



\begin{solution}[by \href{https://artofproblemsolving.com/community/user/29428}{pco}]
	\begin{tcolorbox}Find all functions 
\[f:\mathbb{R}\rightarrow \mathbb{R} \ ; \ f(x+y)=f(x)\cdot f(y)\cdot f(xy)\]\end{tcolorbox}
I suppose that missing functional equation domain is $\forall x,y\in\mathbb R$. If so :
Let $P(x,y)$ be the assertion $f(x+y)=f(x)f(y)f(xy)$

If $f(t)=0$ for some $t$, then $P(x-t,t)$ $\implies$ 
$\boxed{\text{S1 : }f(x)=0\quad\forall x}$ which indeed is a solution.

So let us from now consider $f(x)\ne 0$ $\forall x$
$P(x,y)$ $\implies$ $f(x+y)=f(x)f(y)f(xy)$
$P(xz,yz)$ $\implies$ $f(xz+yz)=f(xz)f(yz)f(xyz^2)$
$P(x+y,z)$ $\implies$ $f(x+y+z)=f(x+y)f(z)f(xz+yz)$

And so $f(x+y+z)=f(x)f(y)f(z)f(xy)f(xz)f(yz)f(xyz^2)$
Swapping there $y,z$ and subtracting, we get $f(xyz^2)=f(xy^2z)$

This implies $f(x)=c$ constant $\forall x\ne 0$
Plugging this back in original equation, we get $c\in\{-1,1\}$
And then $P(1,-1)$ implies that $f(0)$ is the same constant.

And so $\boxed{\text{S2 : }f(x)=1\quad\forall x}$ which indeed is a solution.

And $\boxed{\text{S3 : }f(x)=-1\quad\forall x}$ which indeed is a solution.
\end{solution}
*******************************************************************************
-------------------------------------------------------------------------------

\begin{problem}[Posted by \href{https://artofproblemsolving.com/community/user/374509}{abbosjon2002}]
	Find all functions from reals to reals such that $f(x+y)=f(x)f(y)f(xy)$ for all x,y
	\flushright \href{https://artofproblemsolving.com/community/c6h1627924}{(Link to AoPS)}
\end{problem}



\begin{solution}[by \href{https://artofproblemsolving.com/community/user/375117}{achen29}]
	Guess the search option is pretty useful; this was posted like yesterday lol!
\end{solution}



\begin{solution}[by \href{https://artofproblemsolving.com/community/user/382168}{NahTan123xyz}]
	\begin{tcolorbox}Find all functions from reals to reals such that $f(x+y)=f(x)f(y)f(xy)$ for all x,y\end{tcolorbox}

See [url=https:\/\/artofproblemsolving.com\/community\/c6h1424585p8021252] here [\/url]
\end{solution}



\begin{solution}[by \href{https://artofproblemsolving.com/community/user/29428}{pco}]
	Or here : http://artofproblemsolving.com\/community\/c6h1627540p10206787

\end{solution}
*******************************************************************************
-------------------------------------------------------------------------------

\begin{problem}[Posted by \href{https://artofproblemsolving.com/community/user/333236}{Xurshid.Turgunboyev}]
	Find all  functions  $f:Z\longrightarrow Z $ such that $f (x+f(x+2y))=f (2x)+f (2y) $ and $f(0)=2$ for all integer number $x,y $
	\flushright \href{https://artofproblemsolving.com/community/c6h1628333}{(Link to AoPS)}
\end{problem}



\begin{solution}[by \href{https://artofproblemsolving.com/community/user/29428}{pco}]
	\begin{tcolorbox}Find all  functions  $f:Z\longrightarrow Z $ such that $f (x+f(x+2y))=f (2x)+f (2y) $ and $f(0)=2$ for all integer number $x,y $\end{tcolorbox}
Already posted (and solved) at least in 2010 and 2012.

Dont hesitate to use the search function (see [url=http://www.artofproblemsolving.com\/community\/c6h1330604p7173058] here [\/url]).
Set (for example copy\/paste) in the "search term" field the exact following string : 

+"f(x+f(x+2y))" +"f(2x)" +"f(2y)"

You'll get in the \begin{bolded}ten first results\end{underlined}\end{bolded} (excluded your own post and this post itself) all the help you are requesting for.

[hide=(Some excuses)][size=70]I'm sorry not providing you the direct link to this result but I encountered users who never tried the search function, thinking quite easier to have other users make the search for them. So now I prefer to point to the search function and to give the appropriate search term (I checked that it indeed will give you the expected result) instead of the link itself [\/size]([url=https:\/\/en.wiktionary.org\/wiki\/give_a_man_a_fish_and_you_feed_him_for_a_day;_teach_a_man_to_fish_and_you_feed_him_for_a_lifetime]wink[\/url])[\/hide]



\end{solution}



\begin{solution}[by \href{https://artofproblemsolving.com/community/user/377794}{Mathuzb}]
	\begin{tcolorbox}Find all  functions  $f:Z\longrightarrow Z $ such that $f (x+f(x+2y))=f (2x)+f (2y) $ and $f(0)=2$ for all integer number $x,y $\end{tcolorbox}

It is easy to find $f(2n)= 2n+2$ for all integers $n$.
Now, we know $f(x+f(x))=2x+4$ for all $x$.  If $f(a)= 0$ some $a$, then $f(a+f(a))=2a+4=0$ $\implies$ $a=-2$.
So, we have $f (x+f(x+2y))=2x+2y+4 $.Let $y=-x-2$, then
$f(x+f(-x-4))=0$ $\implies$ $f(-x-4)-2-x$ for all $x$.Thus
we have $f(x)=x+2$ for all integer $x$.
\end{solution}
*******************************************************************************
-------------------------------------------------------------------------------

\begin{problem}[Posted by \href{https://artofproblemsolving.com/community/user/134102}{jonny}]
	A derivable function $f:\mathbb{R^{+}}\rightarrow R$ satisfies the condition $f(x)-f(y)\geq \ln(x\/y)+x-y\forall x,y\in\mathbb{R^{+}}$. Then $f(x)$ is
	\flushright \href{https://artofproblemsolving.com/community/c6h1628547}{(Link to AoPS)}
\end{problem}



\begin{solution}[by \href{https://artofproblemsolving.com/community/user/346843}{jrc1729}]
	Are you sure that there is no mod sign in the problem ?? :maybe:
\end{solution}



\begin{solution}[by \href{https://artofproblemsolving.com/community/user/29428}{pco}]
	\begin{tcolorbox}A derivable function $f:\mathbb{R^{+}}\rightarrow R$ satisfies the condition $f(x)-f(y)\geq \ln(x\/y)+x-y\forall x,y\in\mathbb{R^{+}}$. Then $f(x)$ is\end{tcolorbox}
Equation is $f(x)-\ln x-x\ge f(y)-\ln y-y$ $\forall x,y\in\mathbb R^+$
Swapping $x,y$ and comparing, we get

$\boxed{f(x)=x+\ln x+a\quad\forall x>0}$ which indeed is a solution, whatever is $a\in\mathbb R$
And no need for continuity of "derivable" ...
\end{solution}
*******************************************************************************
-------------------------------------------------------------------------------

\begin{problem}[Posted by \href{https://artofproblemsolving.com/community/user/400713}{RnstTrjyn}]
	$f:\mathbb{R} \rightarrow \mathbb{R}$
$f(x+y)=f(x)+f(y)$ and $f(x^3)=x^2f(x)$
	\flushright \href{https://artofproblemsolving.com/community/c6h1628605}{(Link to AoPS)}
\end{problem}



\begin{solution}[by \href{https://artofproblemsolving.com/community/user/29428}{pco}]
	\begin{tcolorbox}$f:\mathbb{R} \rightarrow \mathbb{R}$
$f(x+y)=f(x)+f(y)$ and $f(x^3)=x^2f(x)$\end{tcolorbox}
I suppose rthat missing domain of functional equation is "$\forall x,y\in\mathbb R$". If so :

Let $k\in\mathbb Q$

Write $f((x+k)^3)=(x+k)^2f(x+k)$
Which is $f(x^3)+3kf(x^2)+3k^2f(x)+k^3f(1)=(x^2+2kx+k^2)(f(x)+kf(1))$

Which is $2k^2(f(x)-xf(1))+k(3f(x^2)-2xf(x)-x^2f(1))=0$

This is a polynomial (in $k$) with infinitely many roots (any rational number) and so this is the allzero polynomial and all its coefficients are zero.
Looking at coefficient of $k^2$, this gives 

$\boxed{f(x)=ax\quad\forall x}$ which indeed is a solution, whatever is $a=f(1)\in\mathbb R$
\end{solution}



\begin{solution}[by \href{https://artofproblemsolving.com/community/user/377794}{Mathuzb}]
	\begin{tcolorbox}$f:\mathbb{R} \rightarrow \mathbb{R}$
$f(x+y)=f(x)+f(y)$ and $f(x^3)=x^2f(x)$\end{tcolorbox}

$$f(x^3+3x^2+3x+1)=(x^2+2x+1)f(x+1)=f(x^3)+3f(x^2)+3f(x)+f(1)=x^2f(x)+2xf(x)+f(x)+x^2f(1)+2f(1)x+f(1)$$ $\implies$ $3f(x^2)+2f(x)=2xf(x)+x^2f(1)+2xf(1)$ and we know that $f$ is odd function.Thus we have $3f(x^2)-2f(x)=2xf(x)+x^2f(1)-2xf(1)$.Then we find that 
$f(x)=f(1)x$ for all real number $x$
\end{solution}
*******************************************************************************
-------------------------------------------------------------------------------

\begin{problem}[Posted by \href{https://artofproblemsolving.com/community/user/363884}{matinyousefi}]
	for any function $f:Q+\longrightarrow Q $ satisfying $$f(xy)=f(x)+f(y)$$ prove that : it can't be injective and there exist surjective such a function.
	\flushright \href{https://artofproblemsolving.com/community/c6h1628763}{(Link to AoPS)}
\end{problem}



\begin{solution}[by \href{https://artofproblemsolving.com/community/user/391068}{TuZo}]
	Where is defined the function?
\end{solution}



\begin{solution}[by \href{https://artofproblemsolving.com/community/user/363884}{matinyousefi}]
	$f:Q^+\longrightarrow Q $
\end{solution}



\begin{solution}[by \href{https://artofproblemsolving.com/community/user/246560}{Catalin}]
	Remark: this is Romania National Olympiad 2017, Grade 10.
\end{solution}



\begin{solution}[by \href{https://artofproblemsolving.com/community/user/409532}{Mozart_Concerto_No.4}]
	[hide=Partial Solution...]We know that $f$ cannot be injective because $x, y = 0 \rightarrow f(0) = 0$ and $x, y = 1 \rightarrow f(1) = 0$. So, both values of $0$ and $1$ in the domain of $f$ map to a single value $0$ in the codomain of $f$, which is not allowed by the definition of injectivity. 

As for finding a surjective $f$, I am having trouble finding such an $f:Q^+\longrightarrow Q $. If you meant to say $f:R^+\longrightarrow R $, then a good example would be $f(x) = ln(x)$. We know this is surjective because all values in the domain of $f$ map to a single value in the codomain of $f$.

Can anyone find such surjective $f:R^+\longrightarrow R $?
[\/hide]


Edit: Apparently this solution is incorrect as well. Please just ignore this entire post.
\end{solution}



\begin{solution}[by \href{https://artofproblemsolving.com/community/user/331394}{Kaskade}]
	\begin{tcolorbox}[hide=Solution]We know that $f$ cannot be injective because $x, y = 0 \rightarrow f(0) = 0$ and $x, y = 1 \rightarrow f(1) = 0$. So, both values of $0$ and $1$ in the domain of $f$ map to a single value $0$ in the codomain of $f$, which is not allowed by the definition of injectivity. As for finding a surjective $f$, a good example is $f(x) = ln(x)$. We know this is surjective because all values in the domain of $f$ map to a single value in the codomain of $f$.[\/hide]\end{tcolorbox}

I don't think $0$ is in the domain
\end{solution}



\begin{solution}[by \href{https://artofproblemsolving.com/community/user/409532}{Mozart_Concerto_No.4}]
	\begin{tcolorbox}
I don't think $0$ is in the domain\end{tcolorbox}

Oh, you're right. My bad :oops_sign:. 

So how does one prove that $f$ cannot be injective then?
\end{solution}



\begin{solution}[by \href{https://artofproblemsolving.com/community/user/409532}{Mozart_Concerto_No.4}]
	\begin{tcolorbox}Remark: this is Romania National Olympiad 2017, Grade 10.\end{tcolorbox}

Can you please include a link? 
\end{solution}



\begin{solution}[by \href{https://artofproblemsolving.com/community/user/409532}{Mozart_Concerto_No.4}]
	Sorry to triple post and bump this thread up yet another time, but:

1. Hold up. You haven't even told us for what $x$ and $y$ that the constraint holds...
2. Is or isn't this related to Cauchy's functional equation?
3. Can someone please post a solution? I'm dying to see one.

Thanks
\end{solution}



\begin{solution}[by \href{https://artofproblemsolving.com/community/user/29428}{pco}]
	\begin{tcolorbox}for any function $f:Q+\longrightarrow Q $ satisfying $$f(xy)=f(x)+f(y)$$ prove that : it can't be injective ...\end{tcolorbox}
If $f(2)=0$, $f(2^2)=0$ and $f(x)$ is not injective
If $f(3)=0$, $f(3^2)=0$ and $f(x)$ is not injective

If $f(2)=\frac ab$ and $f(3)=\frac cd$ for some $a,b,c,d\in\mathbb Z_{\ne 0}$

Then $f(2^{bc})=f(3^{ad})=ac$ and $f(x)$ is not injective.


\end{solution}



\begin{solution}[by \href{https://artofproblemsolving.com/community/user/328030}{Loppukilpailija}]
	$f$ isn't injective

Proof: we will prove there are some $p, q, a, b \in Z$, $p$ and $q$ primes, $p \neq q$, such that $f(\frac{p^a}{q^b}) = 0$. As $f(1) = 0$ (plug in $y = 1$) this proves the claim.
Now, we see that $f(\frac{p^a}{q^b}) = af(p) - bf(q)$. Let $f(p) = \frac{e}{c}$ and $f(q) = \frac{g}{d}$.

Choose $a = c \cdot \frac{lcm(e, g)}{e}$ and $b = d \cdot \frac{lcm(e, g)}{g}$. Now we have $af(p) - bf(q) = 0$


$f$ can be surjective

Proof: the cardinality of rational numbers and the prime numbers are the same. Let $p_1, p_2, \ldots$ be all the primes, and define $f(p_1), f(p_2), \ldots$ in such a way that they go through all the rational numbers. This is possible due to the cardinality. Now, for the rest of the numbers, just define

$f(x) = f(q_1^{a_1} \cdot q_2^{a_2} \cdot \ldots \cdot q_n^{a_n}) = a_1f(q_1) + \ldots + a_n  f(q_n)$, where $q_1^{a_1}  \ldots q_n^{a_n}$ is the prime factorization of $x$. It's easy to see that this satisfies the given equation. $f$ is surjective, as for all rational numbers $q$ there is some prime $p$ so that $f(p) = q$.
\end{solution}



\begin{solution}[by \href{https://artofproblemsolving.com/community/user/246560}{Catalin}]
	\begin{tcolorbox}[quote=Catalin]Remark: this is Romania National Olympiad 2017, Grade 10.\end{tcolorbox}

Can you please include a link?\end{tcolorbox}

http://www.tmmate.ro\/onm\/docs\/subiecte10.pdf
\end{solution}



\begin{solution}[by \href{https://artofproblemsolving.com/community/user/406621}{sharifymatholympiad}]
	 it has second part.
\end{solution}
*******************************************************************************
-------------------------------------------------------------------------------

\begin{problem}[Posted by \href{https://artofproblemsolving.com/community/user/307148}{j___d}]
	Find all real numbers $c$ for which there exists a function $f:\mathbb R\rightarrow \mathbb R$ such that for each $x, y\in\mathbb R$ it's true that
$$f(f(x)+f(y))+cxy=f(x+y).$$
	\flushright \href{https://artofproblemsolving.com/community/c6h1629425}{(Link to AoPS)}
\end{problem}



\begin{solution}[by \href{https://artofproblemsolving.com/community/user/29428}{pco}]
	\begin{tcolorbox}Find all real numbers $c$ for which there exists a function $f:\mathbb R\rightarrow \mathbb R$ such that for each $x, y\in\mathbb R$ it's true that
$$f(f(x)+f(y))+cxy=f(x+y).$$\end{tcolorbox}
Let $P(x,y)$ be the assertion $f(f(x)+f(y))+cxy=f(x+y)$
Let $A=f(\mathbb R)$
Let $a=f(0)$

1) If $c=0$ then such function exists
Just choose for example $f(x)=x$ $\forall x$
Q.E.D.

2) If $c>0$, no such function
$P(x,-x)$ $\implies$ $f(f(x)+f(-x))=cx^2+a$ and so $[a,+\infty)\subseteq A$
$P(x,0)$ $\implies$ $f(f(x)+a)=f(x)$
And so $f(x)=x-a$ $\forall x\ge 2a$

Let then $x,y$ such that : $x,y\ge 2a$ and $x+y\ge\max(2a,4a)$
$P(x,y)$ becomes $x+y-3a+cxy=x+y-a$ which obviously is impossible
Q.E.D.

3) If $c<0$, no such function :
$P(x,-x)$ $\implies$ $f(f(x)+f(-x))=cx^2+a$ and so $(-\infty,a]\subseteq A$
$P(x,0)$ $\implies$ $f(f(x)+a)=f(x)$
And so $f(x)=x-a$ $\forall x\le 2a$

Let then $x,y$ such that : $x,y\le 2a$ and $x+y\le\min(2a,4a)$
$P(x,y)$ becomes $x+y-3a+cxy=x+y-a$ which obviously is impossible
Q.E.D.

Hence the answer : $\boxed{c=0}$


\end{solution}



\begin{solution}[by \href{https://artofproblemsolving.com/community/user/313022}{FEcreater}]
	Here is another approach for the case $ c \neq 0 $:

Suppose that there are $ a \neq b \in \mathbb{R} $ such that $ f\left(a\right) = f\left(b\right) $.  Compare $ P\left(x,a\right), P\left(x,b\right) $: $$ f\left(x+a\right)-cax = f\left(x+b\right)-cbx , \forall x \in \mathbb{R} $$ Rewrite the identity as $$ f\left(x+T\right) = f\left(x\right)+cT\left(x-b\right), \forall x \in \mathbb{R} $$ where $ T = a-b $. Now, compare $ f\left(x,2b-x\right), f\left(x+T,2b-x+T\right) $, $$ 2cTb+cT^2 = cTb \rightarrow T = -b \rightarrow a =0 $$ Notice we have $ f\left(f\left(x\right)+f\left(0\right)\right) = f\left(x\right) $ ...
\end{solution}



\begin{solution}[by \href{https://artofproblemsolving.com/community/user/377794}{Mathuzb}]
	\begin{tcolorbox}[quote=j___d]Find all real numbers $c$ for which there exists a function $f:\mathbb R\rightarrow \mathbb R$ such that for each $x, y\in\mathbb R$ it's true that
$$f(f(x)+f(y))+cxy=f(x+y).$$\end{tcolorbox}
Let $P(x,y)$ be the assertion $f(f(x)+f(y))+cxy=f(x+y)$
Let $A=f(\mathbb R)$
Let $a=f(0)$

1) If $c=0$ then such function exists
Just choose for example $f(x)=x$ $\forall x$
Q.E.D.

2) If $c>0$, no such function
$P(x,-x)$ $\implies$ $f(f(x)+f(-x))=cx^2+a$ and so $[a,+\infty)\subseteq A$
$P(x,0)$ $\implies$ $f(f(x)+a)=f(x)$
And so $f(x)=x-a$ $\forall x\ge 2a$

Let then $x,y$ such that : $x,y\ge 2a$ and $x+y\ge\max(2a,4a)$
$P(x,y)$ becomes $x+y-3a+cxy=x+y-a$ which obviously is impossible
Q.E.D.

3) If $c<0$, no such function :
$P(x,-x)$ $\implies$ $f(f(x)+f(-x))=cx^2+a$ and so $(-\infty,a]\subseteq A$
$P(x,0)$ $\implies$ $f(f(x)+a)=f(x)$
And so $f(x)=x-a$ $\forall x\le 2a$

Let then $x,y$ such that : $x,y\le 2a$ and $x+y\le\min(2a,4a)$
$P(x,y)$ becomes $x+y-3a+cxy=x+y-a$ which obviously is impossible
Q.E.D.

Hence the answer : $\boxed{c=0}$\end{tcolorbox}

My solution is similar to pco's solution.
\end{solution}
*******************************************************************************
-------------------------------------------------------------------------------

\begin{problem}[Posted by \href{https://artofproblemsolving.com/community/user/283252}{Masac9}]
	let $A=\{1,2,3,...,n\}$.How many functions $f:A\to A$ can be defined such that $f(1)<f(2)<f(3)?$ 
	\flushright \href{https://artofproblemsolving.com/community/c6h1629950}{(Link to AoPS)}
\end{problem}



\begin{solution}[by \href{https://artofproblemsolving.com/community/user/29428}{pco}]
	\begin{tcolorbox}let $A=\{1,2,3,...,n\}$.How many functions $f:A\to A$ can be defined such that $f(1)<f(2)<f(3)?$\end{tcolorbox}
The number of possibiliies for $(f(1),f(2),f(3))$ is 
$\sum_{i=1}^{n-2}\sum_{j=i+1}^{n-1}\sum_{k=j+1}^n1$
$=\sum_{i=1}^{n-2}\sum_{j=i+1}^{n-1}(n-j)$
$=\sum_{i=1}^{n-2}\frac{(n-i)(n-i-1)}2$
$=\sum_{i=1}^{n-2}\frac{i^2+i}2$
$=\frac{n(n-1)(n-2)}6$

hence the answer $\boxed{\frac{n^{n-2}(n-1)(n-2)}6}$


\end{solution}



\begin{solution}[by \href{https://artofproblemsolving.com/community/user/187786}{scrabbler94}]
	[Quote=pco]
The number of possibiliies for $(f(1),f(2),f(3))$ is 
$\sum_{i=1}^{n-2}\sum_{j=i+1}^{n-1}\sum_{k=j+1}^n1$
\end{tcolorbox}

@above no need for complicated triple sums.

There are $\binom{n}{3}$ ways to choose values for $f(1)$, $f(2)$, $f(3)$ and $n^{n-3}$ ways to choose values for $f(4), \ldots, f(n)$. Multiplying gives $\frac{n^{n-2}(n-1)(n-2)}{6}$.
\end{solution}
*******************************************************************************
-------------------------------------------------------------------------------

\begin{problem}[Posted by \href{https://artofproblemsolving.com/community/user/404977}{GeometryIsMyWeakness}]
	Let $f$ be a function from the set of real numbers to the set of real numbers on the interval $[-1,1]$ such that

$$f(x+13\/42)+f(x)=f(x+1\/6)+f(x+1\/7)$$

for all real numbers x.

Prove $f(x)$ is a periodic function.
	\flushright \href{https://artofproblemsolving.com/community/c6h1630265}{(Link to AoPS)}
\end{problem}



\begin{solution}[by \href{https://artofproblemsolving.com/community/user/29428}{pco}]
	\begin{tcolorbox}Let $f$ be a function from the set of real numbers to the set of real numbers on the interval $[-1,1]$ such that

$$f(x+13\/42)+f(x)=f(x+1\/6)+f(x+1\/7)$$

for all real numbers x.

Prove $f(x)$ is a periodic function.\end{tcolorbox}
Let $a=\frac 16$ and $b=\frac 17$

Equation is $f(x+a+b)+f(x)=f(x+a)+f(x+b)$
Then $f(x+2a+b)+f(x+a)=f(x+2a)+f(x+a+b)$ $=f(x+2a)+f(x+a)+f(x+b)-f(x)$
Which is $f(x+2a+b)+f(x)=f(x+2a)+f(x+b)$
And easy induction gives $f(x+na+b)+f(x)=f(x+na)+f(x+b)$ $\forall n\in\mathbb Z_{>0}$

Same method gives easily $f(x+na+mb)+f(x)=f(x+na)+f(x+mb)$ $\forall m,n\in\mathbb Z_{>0}$

Setting $n=6$ and $m=7$, this becomes $f(x+2)=2f(x+1)-f(x)$

If $f(x+1)\ne f(x)$ this implies that $f(x+n)$ is unbounded, in contradiction with $f(x)\in[-1;+1]$ $\forall x$

Hence $\boxed{f(x+1)=f(x)\quad\forall x}$
Q.E.D.


\end{solution}



\begin{solution}[by \href{https://artofproblemsolving.com/community/user/3236}{test20}]
	This is problem A7 from the IMO 1996 shortlist:
https:\/\/artofproblemsolving.com\/community\/c3947_1996_imo_shortlist
https:\/\/artofproblemsolving.com\/community\/c6h219655p1218567

\end{solution}



\begin{solution}[by \href{https://artofproblemsolving.com/community/user/404977}{GeometryIsMyWeakness}]
	\begin{tcolorbox}This is problem A7 from the IMO 1996 shortlist:
https:\/\/artofproblemsolving.com\/community\/c3947_1996_imo_shortlist
https:\/\/artofproblemsolving.com\/community\/c6h219655p1218567\end{tcolorbox}

It was used in my country's TST and I wasn't able to find the original anywhere. Thanks!

\begin{tcolorbox}[quote=GeometryIsMyWeakness]Let $f$ be a function from the set of real numbers to the set of real numbers on the interval $[-1,1]$ such that

$$f(x+13\/42)+f(x)=f(x+1\/6)+f(x+1\/7)$$

for all real numbers x.

Prove $f(x)$ is a periodic function.\end{tcolorbox}
Let $a=\frac 16$ and $b=\frac 17$

Equation is $f(x+a+b)+f(x)=f(x+a)+f(x+b)$
Then $f(x+2a+b)+f(x+a)=f(x+2a)+f(x+a+b)$ $=f(x+2a)+f(x+a)+f(x+b)-f(x)$
Which is $f(x+2a+b)+f(x)=f(x+2a)+f(x+b)$
And easy induction gives $f(x+na+b)+f(x)=f(x+na)+f(x+b)$ $\forall n\in\mathbb Z_{>0}$

Same method gives easily $f(x+na+mb)+f(x)=f(x+na)+f(x+mb)$ $\forall m,n\in\mathbb Z_{>0}$

Setting $n=6$ and $m=7$, this becomes $f(x+2)=2f(x+1)-f(x)$

If $f(x+1)\ne f(x)$ this implies that $f(x+n)$ is unbounded, in contradiction with $f(x)\in[-1;+1]$ $\forall x$

Hence $\boxed{f(x+1)=f(x)\quad\forall x}$
Q.E.D.\end{tcolorbox}

Thank you! Very clear and elegant!
\end{solution}



\begin{solution}[by \href{https://artofproblemsolving.com/community/user/3236}{test20}]
	\begin{tcolorbox}It was used in my country's TST and I wasn't able to find the original anywhere. Thanks!
\end{tcolorbox}You should state (a) the country and (b) TST in the problem description.


\end{solution}
*******************************************************************************
-------------------------------------------------------------------------------

\begin{problem}[Posted by \href{https://artofproblemsolving.com/community/user/226233}{steppewolf}]
	Determine all functions $f: \mathbb{R} \rightarrow \mathbb{R}$ such that:$$f(\max \left\{ x, y \right\} + \min \left\{ f(x), f(y) \right\}) = x+y $$ for all real $x,y \in \mathbb{R}$
	\flushright \href{https://artofproblemsolving.com/community/c6h1631265}{(Link to AoPS)}
\end{problem}



\begin{solution}[by \href{https://artofproblemsolving.com/community/user/361694}{Muradjl}]
	i think  you are kidding ,just set $x=y$
\end{solution}



\begin{solution}[by \href{https://artofproblemsolving.com/community/user/339220}{omarius}]
	\begin{tcolorbox}i think  you are kidding ,just set $x=y$\end{tcolorbox}

He's right, a small discussion proves injectivity and the result follows.
\end{solution}



\begin{solution}[by \href{https://artofproblemsolving.com/community/user/377794}{Mathuzb}]
	\begin{tcolorbox}Determine all functions $f: \mathbf{R} \rightarrow \mathbf{R}$ such that:$$f(\max \left\{ x, y \right\} + \min \left\{ f(x), f(y) \right\}) = x+y $$ for all real $x,y \in \mathbf{R}$\end{tcolorbox}

Let $P(x,y)$ be the assertion  $$f(\max \left\{ x, y \right\} + \min \left\{ f(x), f(y) \right\}) = x+y $$ 
$P(x,x)$ $\implies$ $f(x+f(x))=2x$ for all real $x$.
$P(x,y)$ with $x>y$ $\implies$ $$f(x+ \min \left\{ f(x), f(y) \right\}) = x+y $$. If $f(y)\geq f(x)$ then $$f(x+f(x))=2x=x+y$$ $\implies$ $x=y$, but  $x>y$.
So if  $x>y$ then $f(x)>f(y)$.And we find that if $x>y$
Then $f(x+f(y))=x+y$  $\implies$ $f(x)=x$  for all real number $x$.
\end{solution}



\begin{solution}[by \href{https://artofproblemsolving.com/community/user/29428}{pco}]
	\begin{tcolorbox}Then $f(x+f(y))=x+y$  $\implies$ $f(x)=x$  for all real number $x$.\end{tcolorbox}
Why this implication ?


\end{solution}



\begin{solution}[by \href{https://artofproblemsolving.com/community/user/391068}{TuZo}]
	If I am correct, this is why:
If $f(y)=a$, put $y=0$, you get $f(x+a)=x$, so $f(x)=x-a$, putting $x=0$, we get $a=0-a$, so $a=0$, thus $f(x)=x$

\end{solution}



\begin{solution}[by \href{https://artofproblemsolving.com/community/user/29428}{pco}]
	\begin{tcolorbox}Determine all functions $f: \mathbf{R} \rightarrow \mathbf{R}$ such that:$$f(\max \left\{ x, y \right\} + \min \left\{ f(x), f(y) \right\}) = x+y $$ for all real $x,y \in \mathbf{R}$\end{tcolorbox}
Let $P(x,y)$ be the assertion $f(\max(x,y)+\min(f(x),f(y)))=x+y$
Let $a=f(0)$

$P(x,x)$ $\implies$ $f(x+f(x))=2x$
Let $x>y$. If $f(x)\le f(y)$, then $P(x,y)$ $\implies$ $f(x+f(x))=x+y=2x$ and so $x=y$, impossible.
So $x>y$ $\implies$ $f(x)>f(y)$ and $P(x,y$ $\implies$ :
$f(x+f(y))=x+y$ $\forall x>y$ (I)

So $f(x)=x+y-f(y)$ $\forall x>f(y)$
Setting there $y=0$, we get 
$f(x)=x-a$ $\forall x>a$ (II)

Let then $y\in\mathbb R$ and $x>\max(y,a-f(y))$ :
$x>y$ implies $f(x+f(y))=x+y$ (see I above)
$x+f(y)>a$ implies $f(x+f(y))=x+f(y)-a$ (see II above)
And so $f(y)=y+a$ $\forall y$

Plugging this back in original equation, we get $a=0$ and so
$\boxed{f(x)=x\quad\forall x}$


\end{solution}



\begin{solution}[by \href{https://artofproblemsolving.com/community/user/377794}{Mathuzb}]
	\begin{tcolorbox}[quote=Mathuzb]Then $f(x+f(y))=x+y$  $\implies$ $f(x)=x$  for all real number $x$.\end{tcolorbox}
Why this implication ?\end{tcolorbox}

We have for all $x>y$ $\implies$ $f(x+f(y))=x+y$.
Then $x>0$ $\implies$ $f(x+f(0))=x$ and $f(0)=0$.
So  $f(x)=x$ for positive real. Similarly we find $f(x)=x$ for all negative real.
\end{solution}



\begin{solution}[by \href{https://artofproblemsolving.com/community/user/243405}{ThE-dArK-lOrD}]
	Let me answer @pco question to @Mathuzb sol:

Bc that implies $f$ is linear (we'll show $f(y)-y=f(z)-z$ for all $y$ with fixed $z$, consider $t>\max \{ y+f(y),z+f(z)\}$ and substitute $x$ by $t-f(y)$ and $t-f(z)$.)
Then let $f(x)=x+c$ for all real $x$ for a constant $c$, and so $$x+y=f(\max \left\{ x, y \right\} + \min \left\{ f(x), f(y) \right\}) =f(\max \left\{ x, y \right\} + \min \left\{  x, y \right\} +c)=f(x+y+c)=x+y+2c\implies c=0.$$
\end{solution}



\begin{solution}[by \href{https://artofproblemsolving.com/community/user/226233}{steppewolf}]
	This was my proposal :)

\end{solution}



\begin{solution}[by \href{https://artofproblemsolving.com/community/user/209049}{math90}]
	Let $P(x,y)$ be the given assertion.
Clearly $f$ is surjective. So there exists a real number $u$ such that $f(u)=0$.
$P(u,u)\implies u=0$, so $f(x)=0\iff x=0$.
$P(x,x)\implies f(x+f(x))=2x$
Suppose there exists $a>b$ such that $f(a)\leq f(b)$.
$P(a,b)\implies 2a=f(a+f(a))=a+b$, contradiction. Hence $f$ is strictly increasing.
Suppose $x>0$. Then
$P(x,0)\implies f(x)=x$
$P(x,-x)\implies f(x+f(-x))=0\implies f(-x)=-x$
Thus $f(x)=x$ for all $x\in\mathbb R$, which indeed is a solution.
\end{solution}
*******************************************************************************
-------------------------------------------------------------------------------

\begin{problem}[Posted by \href{https://artofproblemsolving.com/community/user/394604}{Mr.Bogus19}]
	Let $f: \mathbb{N} \longrightarrow \mathbb{N}$ is a strictly increasing  function $(f(n+1)>f(n))$ for which 
$$f(f(n))=3n$$
Evaluate $f(2017)$.
	\flushright \href{https://artofproblemsolving.com/community/c6h1631338}{(Link to AoPS)}
\end{problem}



\begin{solution}[by \href{https://artofproblemsolving.com/community/user/395825}{JLW}]
	This looks like a good start so far which can be used to solve the problem.
as $f: \mathbb{N} \longrightarrow \mathbb{N}$ we know that $f(1) \geq 1$ so by applying $f$ to both sides $3\geq f(1)$
Now we know $1\leq f(1) \leq 3$ if $f(1)=1$ then by considering that $f(f(1))=3$ we deduce $f(1)=3$ which contradicts $f(1)=1$. Now suppose $f(1)=3$ then $f(3)=3$, but $f$ is not the indentity function, thus the only possibility left is $f(1)=2$. This implies $f(2)=3$. Therefore $f(2)=3$ and $f(3)=6$ and $f(6)=9$ but what is $f(4)$ and $f(5)?$. Using the fact the function is increasing, $f(3)<f(4)<f(5)<f(6)$ and the only solutions are $f(4)=7$ and $f(5)=8$.
\end{solution}



\begin{solution}[by \href{https://artofproblemsolving.com/community/user/29428}{pco}]
	\begin{tcolorbox}Let $f: \mathbb{N} \longrightarrow \mathbb{N}$ is a strictly increasing  function $(f(n+1)>f(n))$ for which 
$$f(f(n))=3n$$
Evaluate $f(2017)$.\end{tcolorbox}
Posted (and solved) many many times (2006, 2008, 2010, 2011, 2016, ...).

Dont hesitate to use the search function (see [url=http://www.artofproblemsolving.com\/community\/c6h1330604p7173058] here [\/url]).
Set (for example copy\/paste) in the "search term" field the exact following string : 

+"f(f(n))=3n"

You'll get in the \begin{bolded}ten first results\end{underlined}\end{bolded} (excluded your own post and this post itself) all the help you are requesting for.

[hide=(Some excuses)][size=70]I'm sorry not providing you the direct link to this result but I encountered users who never tried the search function, thinking quite easier to have other users make the search for them. So now I prefer to point to the search function and to give the appropriate search term (I checked that it indeed will give you the expected result) instead of the link itself [\/size]([url=https:\/\/en.wiktionary.org\/wiki\/give_a_man_a_fish_and_you_feed_him_for_a_day;_teach_a_man_to_fish_and_you_feed_him_for_a_lifetime]wink[\/url])[\/hide]



\end{solution}
*******************************************************************************
-------------------------------------------------------------------------------

\begin{problem}[Posted by \href{https://artofproblemsolving.com/community/user/377794}{Mathuzb}]
	Find all monotone functions $f: \mathbb{R} \longrightarrow \mathbb{R}$ such that $f(2018x)=f(x)+2017x$ for all real number $x$.
	\flushright \href{https://artofproblemsolving.com/community/c6h1631378}{(Link to AoPS)}
\end{problem}



\begin{solution}[by \href{https://artofproblemsolving.com/community/user/1088}{hendrata01}]
	I don't understand, seems like there are infinitely many such functions. For example any function $f(x) = x+c$ satisfies this equation. But more over, take any monotonically increasing curve in the interval of [1,2018) that maps to $[c+1, 2018+c)$ (meaning that $\lim_{x \to 2018} f(x) = 2018+c$)

Then $f(x)$ is fully defined on $[2018,2018^2)$, which then defines it completely for $[2018^2, 2018^3)$ and so on... and likewise it's defined for $[1\/2018,1)$, then fully defined on $[1\/2018^2, 1\/2018)$ and so on (by just "copying" that curve over and over again, with the appropriate stretching and scaling etc)

And for negative values of $x$ just repeat the process again...

Point is, there are infinitely many such functions and they're not even part of the same "class" of functions (you can have piece-wise quadratic, piece-wise exponential etc) as long as they're defined on interval [1,2018) then it's defined elsewhere.
\end{solution}



\begin{solution}[by \href{https://artofproblemsolving.com/community/user/29428}{pco}]
	\begin{tcolorbox}Find all monotone functions $f: \mathbb{R} \longrightarrow \mathbb{R}$ such that $f(2018x)=f(x)+2017x$ for all real number $x$.\end{tcolorbox}
Let $a=\lim_{x\to 0+}f(x)$ (which exists since monotonous)
Let $b=\lim_{x\to 0-}f(x)$ (which exists since monotonous)

Easy induction implies $f(2018^nx)=f(x)+(2018^n-1)x$ $\forall x\in\mathbb R$ and $\forall n\in\mathbb Z$
Let $x>0$. Setting above $n\to-\infty$, we get $f(x)=x+a$ $\forall x>0$
Let $x<0$. Setting above $n\to-\infty$, we get $f(x)=x+b$ $\forall x<0$

Hence the solution :
$\boxed{f(x)=x+a\quad\forall x<0\text{  and  }f(0)=c\text{  and  }f(x)=x+b\quad\forall x>0}$
Which indeed is a solution $\forall a\le c\le b$



\end{solution}
*******************************************************************************
-------------------------------------------------------------------------------

\begin{problem}[Posted by \href{https://artofproblemsolving.com/community/user/406831}{GorgonMathDota}]
	Determine all such functions $f: \mathbb{N} \to \mathbb{N} $ such that
 \[ 
LCM(f(a),b) = LCM(a,f(b))
\]
For every natural numbers $a$ and $b$.
	\flushright \href{https://artofproblemsolving.com/community/c6h1631610}{(Link to AoPS)}
\end{problem}



\begin{solution}[by \href{https://artofproblemsolving.com/community/user/405366}{Smita}]
	I think so all the functions which satisfies f(0)=0 works 
\end{solution}



\begin{solution}[by \href{https://artofproblemsolving.com/community/user/410090}{Synthetic_Potato}]
	\begin{tcolorbox}I think so all the functions which satisfies f(0)=0 works\end{tcolorbox}

$0$ is not in the domian and range of $f$.

[hide=solution]
Setting $a$ = $1$, we see that $f(a)=[a,f(1)]$. Note that this is indeed a solution.

$\blacksquare$.
\end{solution}



\begin{solution}[by \href{https://artofproblemsolving.com/community/user/406831}{GorgonMathDota}]
	@above So you're saying that there are no such function exists?
\end{solution}



\begin{solution}[by \href{https://artofproblemsolving.com/community/user/29428}{pco}]
	\begin{tcolorbox}Determine all such functions $f: \mathbb{N} \to \mathbb{N} $ such that
 \[ 
LCM(f(a),b) = LCM(a,f(b))
\]
For every natural numbers $a$ and $b$.\end{tcolorbox}
Let $P(x,y)$ be the assertion $\text{lcm}(f(x),y)=\text{lcm}(x,f(y))$

$P(x,1)$ implies $f(x)=\text{lcm}(x,f(1))$

And so $\boxed{f(x)=\text{lcm} (x,u)\quad\forall x\in\mathbb Z_{>0}}$ which indeed is a solution, whatever is $u\in\mathbb Z_{>0}$ since :
$\text{lcm}(\text{lcm}(x,u),y)=\text{lcm}(x,\text{lcm}(y,u))=\text{lcm}(x,y,u)$


\end{solution}
*******************************************************************************
-------------------------------------------------------------------------------

\begin{problem}[Posted by \href{https://artofproblemsolving.com/community/user/380239}{khan.academy}]
	\begin{bolded}Problem Section #2

c). Denote by $\mathbb{Q^+}$ the set of all positive rational numbers. Determine all functions $f:\mathbb{Q^+}\to\mathbb{Q^+}$ which satisfy the following equation for all 
$x,y \in \mathbb{Q^+} : f(f(x)^2.y)=x^3.f(xy)$.
	\flushright \href{https://artofproblemsolving.com/community/c6h1631690}{(Link to AoPS)}
\end{problem}



\begin{solution}[by \href{https://artofproblemsolving.com/community/user/29428}{pco}]
	\begin{tcolorbox}\begin{bolded}Problem Section #2

c). Denote by $\mathbb{Q^+}$ the set of all positive rational numbers. Determine all functions $f:\mathbb{Q^+}\to\mathbb{Q^+}$ which satisfy the following equation for all 
$x,y \in \mathbb{Q^+} : f(f(x)^2.y)=x^3.f(xy)$.\end{tcolorbox}
Let $P(x,y)$ be the assertion $f(f(x)^2y)=x^3f(xy)$

$P(x,1)$ $\implies$ $f(f(x)^2)=x^3f(x)$ and so $f(x)$ is injective.
$P(1,1)$ $\implies$ $f(f(1)^2)=f(1)$ and so, since injective, $f(1)=1$

1) $f(xy)=f(x)f(y)$ $\forall x,y\in\mathbb Q^+$
$P(x,y)$ $\implies$ $f(f(x)^2y)=x^3f(xy)$ and so $y^3f(f(x)^2y)=x^3y^3f(xy)$
$P(xy,1)$ $\implies$ $f(f(xy)^2)=x^3y^3f(xy)$ and so :
$y^3f(f(x)^2y)=f(f(xy)^2)$ 

$P(y,f(x)^2)$ $\implies$ $f(f(y)^2f(x)^2)=y^3f(f(x)^2y)$
And so $f(f(y)^2f(x)^2)=f(f(xy)^2)$

And so, using injectivity : $f(xy)=f(x)f(y)$
Q.E.D.

2) $\boxed{f(x)=\frac 1x\quad\forall x\in\mathbb Q^+}$
Using previous property, $P(x,y)$ becomes $f(f(x))^2=x^3f(x)$
And so $xf(x)$ is a perfect square and it exists a multiplicative function $g(x)$ such that $f(x)=\frac{g(x)^2}x$

Then $f(f(x))^2=x^3f(x)$ becomes $g(g(x))^4=g(x)^5$
This means that $g(x)=h(x)^4$ for some multiplicative $h(x)$ and so 
$h(h(x))^{16}=h(x)^5$ and so $h(x)=k(x)^4$ for some multiplicative $h(x)$ and so ....

And it is easy to conclude that the only possibility is $g(x)=1$ $\forall x\in\mathbb Q^+$
Q.E.D.



\end{solution}



\begin{solution}[by \href{https://artofproblemsolving.com/community/user/209049}{math90}]
	They simply copied ISL 2010 A5, wow.
\end{solution}
*******************************************************************************
-------------------------------------------------------------------------------

\begin{problem}[Posted by \href{https://artofproblemsolving.com/community/user/331781}{muraza}]
	f:R --> R
f(x + f(x)f(y)) = f(x) + xf(y)
	\flushright \href{https://artofproblemsolving.com/community/c6h1632340}{(Link to AoPS)}
\end{problem}



\begin{solution}[by \href{https://artofproblemsolving.com/community/user/360297}{Mr.Chem-Mathy}]
	\begin{tcolorbox}f:R --> R
f(x + f(x)f(y)) = f(x) + xf(y)\end{tcolorbox} 

Determine all $f:R\rightarrow R$ which satisfy the following equation
$f(x+f(x)f(y))=f(x)+xf(y)$
\end{solution}



\begin{solution}[by \href{https://artofproblemsolving.com/community/user/396004}{arc_ankon}]
	Let us bound it. let $p$ be a polynomial which satisfies this. If $p$ is greater than or equal to degree 2, then the equality in both side is impossible. as an example consider it $f(x)=x^2$ (the least) then 
$f(x+f(x)f(y))$= $f(x)$ + $xf(y)$ will become 
$x^2+2x^3y^2+x^4y^4$= $x^2$ $+$ $xy^2$ which is impossible. so the left are $p=x+c$ or $p=c$. setting c=0 satisfies our condition. 

So such that function $f$ which satisfies the functional equation are $f(x)=x$ and $f(x)$=0. and we also proved that without these nothing can be possible outcome
\end{solution}



\begin{solution}[by \href{https://artofproblemsolving.com/community/user/303223}{Gluncho}]
	\begin{tcolorbox}Let us bound it. let $p$ be a polynomial which satisfies this. If $p$ is greater than or equal to degree 2, then the equality in both side is impossible. as an example consider it $f(x)=x^2$ (the least) then 
$f(x+f(x)f(y))$= $f(x)$ + $xf(y)$ will become 
$x^2+2x^3y^2+x^4y^4$= $x^2$ $+$ $xy^2$ which is impossible. so the left are $p=x+c$ or $p=c$. setting c=0 satisfies our condition. 

So such that function $f$ which satisfies the functional equation are $f(x)=x$ and $f(x)$=0. and we also proved that without these nothing can be possible outcome\end{tcolorbox}
f is not a polynomial, f is a function ;)

\end{solution}



\begin{solution}[by \href{https://artofproblemsolving.com/community/user/396004}{arc_ankon}]
	\begin{tcolorbox}[quote=arc_ankon]Let us bound it. let $p$ be a polynomial which satisfies this. If $p$ is greater than or equal to degree 2, then the equality in both side is impossible. as an example consider it $f(x)=x^2$ (the least) then 
$f(x+f(x)f(y))$= $f(x)$ + $xf(y)$ will become 
$x^2+2x^3y^2+x^4y^4$= $x^2$ $+$ $xy^2$ which is impossible. so the left are $p=x+c$ or $p=c$. setting c=0 satisfies our condition. 

So such that function $f$ which satisfies the functional equation are $f(x)=x$ and $f(x)$=0. and we also proved that without these nothing can be possible outcome\end{tcolorbox}
f is not a polynomial, f is a function ;)\end{tcolorbox}

yeah.... I assumed it as a polynomial and solved it.Does any other $f$ exist??
I actually assumed $f(x)=p$. I don't know that it has any effect in solution... has it??
\end{solution}



\begin{solution}[by \href{https://artofproblemsolving.com/community/user/29428}{pco}]
	\begin{tcolorbox}yeah.... I assumed it as a polynomial and solved it.Does any other $f$ exist??
I actually assumed $f(x)=p$. I don't know that it has any effect in solution... has it??\end{tcolorbox}
1) you cant assume that $f$ is a polynomial.
2) If you assume that $f$ is a polynomial, then you missed the solution $f(x)=-x$ $\forall x$



\end{solution}



\begin{solution}[by \href{https://artofproblemsolving.com/community/user/396004}{arc_ankon}]
	$p=ax+c$ so here $a$ can be -1. I just missed it.
\end{solution}



\begin{solution}[by \href{https://artofproblemsolving.com/community/user/281710}{seoneo}]
	This problem has been posted many times.
It's origin seems like the american mathematical monthly. See [url]https:\/\/artofproblemsolving.com\/community\/c6h292835[\/url] and attached file therein.
See also [url]https:\/\/artofproblemsolving.com\/community\/c6h88456p526193[\/url], [url]https:\/\/artofproblemsolving.com\/community\/c6h132331p749029[\/url], 
[url]https:\/\/artofproblemsolving.com\/community\/c6h494421p2775110[\/url],
[url]https:\/\/artofproblemsolving.com\/community\/c6h1259896p6534316[\/url],
[url]https:\/\/artofproblemsolving.com\/community\/c6h593334p3518429[\/url],
[url]https:\/\/artofproblemsolving.com\/community\/c6h1162764p5543949[\/url],
[url]https:\/\/artofproblemsolving.com\/community\/c6h579656p3421342[\/url]

Some easier variation is
[url]https:\/\/artofproblemsolving.com\/community\/c6h41679p2772681[\/url]


\end{solution}
*******************************************************************************
-------------------------------------------------------------------------------

\begin{problem}[Posted by \href{https://artofproblemsolving.com/community/user/401393}{mruczek}]
	Decide whether exists function $f:  \mathbb{N} \rightarrow  \mathbb{N}$, such that for each $n \in  \mathbb{N}$ is $f(f(n) )= 2n$.
	\flushright \href{https://artofproblemsolving.com/community/c6h1632408}{(Link to AoPS)}
\end{problem}



\begin{solution}[by \href{https://artofproblemsolving.com/community/user/238386}{ythomashu}]
	Infinite functions exist.
Take pairs of distinct odd integers. for example $(1,3),(5,7),(9,11),\dots,(4n+1,4n+3),\dots$
and then $f(4k+1)=4k+3$
\end{solution}



\begin{solution}[by \href{https://artofproblemsolving.com/community/user/363884}{matinyousefi}]
	\begin{tcolorbox}Infinite functions exist.
Take pairs of distinct odd integers. for example $(1,3),(5,7),(9,11)\dots,(4n+1,4n+3),\dots$
and then $f(4k+1)=4k+3$\end{tcolorbox}

then what do you define $f(2)$
\end{solution}



\begin{solution}[by \href{https://artofproblemsolving.com/community/user/238386}{ythomashu}]
	isn't it self explanatory? $f(2)=6$
or is my solution wrong
\end{solution}



\begin{solution}[by \href{https://artofproblemsolving.com/community/user/375117}{achen29}]
	\begin{tcolorbox}isn't it self explanatory? $f(2)=6$
or is my solution wrong\end{tcolorbox}

might be dumb lol but how does your function satisfy the given!
\end{solution}



\begin{solution}[by \href{https://artofproblemsolving.com/community/user/238386}{ythomashu}]
	this works right?
$1\to3\to2\to6\to4\to12\to8\to\cdots$
actually i might be dumb
edit: no i'm pretty sure this works
\end{solution}



\begin{solution}[by \href{https://artofproblemsolving.com/community/user/187786}{scrabbler94}]
	\begin{tcolorbox}this works right?
$1\to3\to2\to6\to4\to12\to8\to\cdots$
actually i might be dumb
edit: no i'm pretty sure this works\end{tcolorbox}

Is $f$ defined as follows?

$1 \to 3 \to 2 \to 6 \to 4 \to 12 \to 8 \to \ldots$
$5 \to 7 \to 10 \to 14 \to 20 \to 28 \to 40 \to \ldots$
$9 \to 11 \to 18 \to 22 \to 36 \to 44 \to 72 \to \ldots$

In other words, to compute $f(n)$, write $n = 2^a p$ where $p$ is odd. If $p \equiv 1 \pmod{4}$, then $f(p) = p+2$ and $f(n) = 2^a(p+2)$. Otherwise, if $p \equiv 3 \pmod{4}$, then $f(p) = 2(p-2)$ and $f(n) = 2^{a+1}(p-2)$.

This seems to work.
\end{solution}



\begin{solution}[by \href{https://artofproblemsolving.com/community/user/29428}{pco}]
	\begin{tcolorbox}Decide whether exists function $f:  \mathbb{N} \rightarrow  \mathbb{N}$, such that for each $n \in  \mathbb{N}$ is $f(f(n) )= 2n$.\end{tcolorbox}
This has been posted tons of times.
General solution of equation $f(f(x)=2x$ over $\mathbb Z_{>0}$  :
Let $(A,B)$ any split of the set of odd positive integers in two infinite (and so equinumerous) subsets.
Let $g(x)$ any bijection from $A\to B$
Then :

Let $n=2^km$ where $k\in\mathbb Z_{\ge 0}$ and $m$ is an odd positive integer :
If $m\in A$ : $f(n)=2^kg(m)$
If $m\in B$ : $f(n)=2^{k+1}g^{-1}(m)$
\end{solution}
*******************************************************************************
-------------------------------------------------------------------------------

\begin{problem}[Posted by \href{https://artofproblemsolving.com/community/user/307525}{Haismit}]
	For function $y=\frac{x+3}{x-1}$with graph is C. Point M on  line y = 1-2x such that from M has two tangents with C and two contact terminals A and B respectively. Know that the AB line passes through K (0.2). find length OM.
	\flushright \href{https://artofproblemsolving.com/community/c6h1632713}{(Link to AoPS)}
\end{problem}



\begin{solution}[by \href{https://artofproblemsolving.com/community/user/29428}{pco}]
	\begin{tcolorbox}For function $y=\frac{x+3}{x-1}$with graph is C. Point M on  line y = 1-2x such that from M has two tangents with C and two contact terminals A and B respectively. Know that the AB line passes through K (0.2). find length OM.\end{tcolorbox}
Let $f(x)=\frac{x+3}{x-1}$
Let $a\ne b$ and $a,b\ne 1$ and points $A(a,f(a))$ and $B(b,f(b))$
Rather easy to check that $K(0,2)\in AB$ is equivalent to $5(a+b)-5-ab=0$
[hide=Simple proof]Line $AB$ is $x\frac{f(b)-f(a)}{b-a}+\frac{bf(a)-af(b)}{b-a}$
$K(0,2)\in AB$ means $2=\frac{bf(a)-af(b)}{b-a}$

Which is $2b-2a=bf(a)-af(b)=\frac{ab+3b}{a-1}-\frac{av+3a}{b-1}$ $=\frac{ab(b-a)+3(b-a)(a+b)-3(b-a)}{ab-a-b+1}$

And so $2=\frac{ab+3(a+b)-3}{ab-(a+b)+1}$

And so $2ab-2(a+b)+2=ab+3(a+b)-3$

And so $5(a+b)-ab-5=0$[\/hide]

Tangent at point $t\ne 1$ at $f(x)=\frac{x+3}{x-1}$ is 
$y=-\frac{4(x-t)}{(t-1)^2}+\frac{t+3}{t-1}$

If $M(u,1-2u)$ belongs to tangent, this means 
$1-2u=-\frac{4(u-t)}{(t-1)^2}+\frac{t+3}{t-1}$

This is $t\ne 1$ and $ut^2-2t(u-2)-(u+2)=0$
In order this equation has two solutions different from $1$, we just need $u\notin\{0,1\}$
And then $A(a,f(a))$ and $B(b,f(b))$ are such that 
$a+b=\frac{2(u-2)}u$ and $ab=-\frac{u+2}u$

And equation $5(a+b)-5-ab=0$ (see first lines) is then just $u=3$
So $M$ is $(3,-5)$ and $\boxed{OM=\sqrt{34}}$
\end{solution}
*******************************************************************************
-------------------------------------------------------------------------------

\begin{problem}[Posted by \href{https://artofproblemsolving.com/community/user/274173}{Anar24}]
	Define $f$ an increasing function from positive integers to positive real numbers.Is it possible for $f$ to be upper-bounded?
	\flushright \href{https://artofproblemsolving.com/community/c6h1632764}{(Link to AoPS)}
\end{problem}



\begin{solution}[by \href{https://artofproblemsolving.com/community/user/29428}{pco}]
	\begin{tcolorbox}Define $f$ a function from positive integers to positive real numbers.Is it possible for $f$ to be upper-bounded?\end{tcolorbox}

Yes. Choose for example $f(x)=1$ $\forall x\in\mathbb Z_{>0}$

\end{solution}



\begin{solution}[by \href{https://artofproblemsolving.com/community/user/274173}{Anar24}]
	\begin{tcolorbox}[quote=Anar24]Define $f$ a function from positive integers to positive real numbers.Is it possible for $f$ to be upper-bounded?\end{tcolorbox}

Yes. Choose for example $f(x)=1$ $\forall x\in\mathbb Z_{>0}$\end{tcolorbox}

Sorry,i forgot to mention crucial point!
\end{solution}



\begin{solution}[by \href{https://artofproblemsolving.com/community/user/29428}{pco}]
	\begin{tcolorbox}Define $f$ an increasing function from positive integers to positive real numbers.Is it possible for $f$ to be upper-bounded?\end{tcolorbox}
Yes.
Choose for example $f(x)=1-\frac 1{2^x}$ $\forall x\in\mathbb Z_{>0}$


\end{solution}



\begin{solution}[by \href{https://artofproblemsolving.com/community/user/379434}{hansu}]
	sorry, wrong solution.
\end{solution}



\begin{solution}[by \href{https://artofproblemsolving.com/community/user/29428}{pco}]
	\begin{tcolorbox}Or just $f(n)=1-n$\end{tcolorbox}
No. Codomain is supposed to be $\mathbb R_{>0}$ which is not the case in your example.

\end{solution}
*******************************************************************************
-------------------------------------------------------------------------------

\begin{problem}[Posted by \href{https://artofproblemsolving.com/community/user/346843}{jrc1729}]
	The domain of a function $f$ is $\mathbb{N}$ (The set of natural numbers). The function is defined as follows : $$f(n)=n+\lfloor\sqrt{n}\rfloor$$ where $\lfloor k\rfloor$ denotes the nearest integer smaller than or equal to $k$. Prove that, for every natural number $m$, the following sequence contains at least one perfect square $$m,~f(m),~f^2(m),~f^3(m),\cdots$$ The notation $f^k$ denotes the function obtained by composing $f$ with itself $k$ times.
	\flushright \href{https://artofproblemsolving.com/community/c6h1633923}{(Link to AoPS)}
\end{problem}



\begin{solution}[by \href{https://artofproblemsolving.com/community/user/29428}{pco}]
	\begin{tcolorbox}The domain of a function $f$ is $\mathbb{N}$ (The set of natural numbers). The function is defined as follows : $$f(n)=n+\lfloor\sqrt{n}\rfloor$$ where $\lfloor k\rfloor$ denotes the nearest integer smaller than or equal to $k$. Prove that, for every natural number $m$, the following sequence contains at least one perfect square $$m,~f(m),~f^2(m),~f^3(m),\cdots$$ The notation $f^k$ denotes the function obtained by composing $f$ with itself $k$ times.\end{tcolorbox}
Already posted (and solved) at least in 2013 and 2017.

Dont hesitate to use the search function (see [url=http://www.artofproblemsolving.com\/community\/c6h1330604p7173058] here [\/url]).
Set (for example copy\/paste) in the "search term" field the exact following string : 

+"\sqrt n" +"sequence" +"perfect square"

You'll get in the \begin{bolded}first \end{underlined}\end{bolded}result (excluded your own post and this post itself) the exact help you are requesting for.

[hide=(Some excuses)][size=70]I'm sorry not providing you the direct link to this result but I encountered users who never tried the search function, thinking quite easier to have other users make the search for them. So now I prefer to point to the search function and to give the appropriate search term (I checked that it indeed will give you the expected result) instead of the link itself [\/size]([url=https:\/\/en.wiktionary.org\/wiki\/give_a_man_a_fish_and_you_feed_him_for_a_day;_teach_a_man_to_fish_and_you_feed_him_for_a_lifetime]wink[\/url])[\/hide]


\end{solution}



\begin{solution}[by \href{https://artofproblemsolving.com/community/user/346843}{jrc1729}]
	<<Post removed>>
\end{solution}



\begin{solution}[by \href{https://artofproblemsolving.com/community/user/346843}{jrc1729}]
	This was previous posted and it is actually from \begin{bolded}Putnam 1983\end{bolded}. :)

Links to the solutions :
[url=https:\/\/artofproblemsolving.com\/community\/c6h1462017p8445654]Solution by @pco[\/url]
[url=https:\/\/artofproblemsolving.com\/community\/c6h554419p3222587] Solution by @djb86[\/url]
\end{solution}
*******************************************************************************
-------------------------------------------------------------------------------

\begin{problem}[Posted by \href{https://artofproblemsolving.com/community/user/333072}{Dayal83603}]
	Find all functions $f:N\rightarrow N$ such that
$f(m^2+f(n))=f(m)^2+n$, for all $m, n\in N$.
[I am a beginner in functional equations, please help.]
	\flushright \href{https://artofproblemsolving.com/community/c6h1634550}{(Link to AoPS)}
\end{problem}



\begin{solution}[by \href{https://artofproblemsolving.com/community/user/29428}{pco}]
	\begin{tcolorbox}Find all functions $f:N\rightarrow N$ such that
$f(m^2+f(n))=f(m)^2+n$, for all $m, n\in N$.
[I am a beginner in functional equations, please help.]\end{tcolorbox}
Let $P(x,y)$ be the assertion $f(x^2+f(y))=f(x)^2+y$

$P(z,x^2+f(y))$ $\implies$ $f(y+z^2+f(x)^2)=f(y)+x^2+f(z)^2$
Simple induction implies then $f(y+k(z^2+f(x)^2))=f(y)+k(x^2+f(z)^2)$
Setting there $k=u^2+f(v)^2$, we get : $f(y+(u^2+f(v)^2)(z^2+f(x)^2))=f(y)+(u^2+f(v)^2)(x^2+f(z)^2)$

Swapping $(u,z)$ and $(v,x)$ and comparing, we get :
$(u^2+f(v)^2)(x^2+f(z)^2)=(z^2+f(x)^2)(v^2+f(u)^2)$
Setting $u=v$, this implies $f(x)^2=x^2+a$ for some $a\in\mathbb Z$
In order LHS always be a perfect square, we need $a=0$

And so $\boxed{f(x)=x\quad\forall x\in\mathbb Z_{>0}}$ which indeed is a solution.



\end{solution}



\begin{solution}[by \href{https://artofproblemsolving.com/community/user/3236}{test20}]
	This problem is closely related to problem 2 of IMO 1992:
https:\/\/artofproblemsolving.com\/community\/c3819_1992_imo
https:\/\/artofproblemsolving.com\/community\/c6h60715p366399

\end{solution}
*******************************************************************************
-------------------------------------------------------------------------------

\begin{problem}[Posted by \href{https://artofproblemsolving.com/community/user/60529}{silvergrasshopper}]
	Is it possible to find all functions $f$ such that $\frac{f(x)}{3x-1}=f\left(\frac{x}{3x-1}\right)$? Thanks.
	\flushright \href{https://artofproblemsolving.com/community/c6h1634995}{(Link to AoPS)}
\end{problem}



\begin{solution}[by \href{https://artofproblemsolving.com/community/user/367931}{Vrangr}]
	What is the domain, what is the codomain? Is this true for all $x$ or only for some $x$? Is it true even for $x = \frac{1}{3}$?
\end{solution}



\begin{solution}[by \href{https://artofproblemsolving.com/community/user/60529}{silvergrasshopper}]
	For all $x$ in, say, $\left[\frac{1}{2},1\right]$.
\end{solution}



\begin{solution}[by \href{https://artofproblemsolving.com/community/user/367931}{Vrangr}]
	I'm assuming $f:\mathbb{R}\to\mathbb{R}$.
\end{solution}



\begin{solution}[by \href{https://artofproblemsolving.com/community/user/60529}{silvergrasshopper}]
	That's also fine. 
\end{solution}



\begin{solution}[by \href{https://artofproblemsolving.com/community/user/405366}{Smita}]
	f(x) =cx where c is a constant.if i am correct pls tell me I will post my solution 
\end{solution}



\begin{solution}[by \href{https://artofproblemsolving.com/community/user/29428}{pco}]
	\begin{tcolorbox}Is it possible to find all functions $f$ such that $\frac{f(x)}{3x-1}=f\left(\frac{x}{3x-1}\right)$? Thanks.\end{tcolorbox}

\begin{tcolorbox}For all $x$ in, say, $\left[\frac{1}{2},1\right]$.\end{tcolorbox}

General solution : $f(x)=xg\left(\left|\frac 1x-\frac 32\right|\right)$ whatever is $g(x)$ defined over $[0,\frac 12]$
\end{solution}



\begin{solution}[by \href{https://artofproblemsolving.com/community/user/60529}{silvergrasshopper}]
	\begin{tcolorbox}[quote=silvergrasshopper]Is it possible to find all functions $f$ such that $\frac{f(x)}{3x-1}=f\left(\frac{x}{3x-1}\right)$? Thanks.\end{tcolorbox}

\begin{tcolorbox}For all $x$ in, say, $\left[\frac{1}{2},1\right]$.\end{tcolorbox}

General solution : $f(x)=xg\left(\left[\frac 1x-\frac 32\right|\right)$ whatever is $g(x)$ defined over $[0,\frac 12]$\end{tcolorbox}

Thanks, but can you explain how did you obtain this? And what does $[...|$ mean?
\end{solution}



\begin{solution}[by \href{https://artofproblemsolving.com/community/user/60529}{silvergrasshopper}]
	And what about $f\left(\frac{x}{x-1}\right)=\frac{f(x)-x}{1-x}$? Thanks.
\end{solution}



\begin{solution}[by \href{https://artofproblemsolving.com/community/user/29428}{pco}]
	\begin{tcolorbox}.... And what does $[...|$ mean?\end{tcolorbox}
Typo (edited) : this is $|..|$ (absolute value)



\end{solution}



\begin{solution}[by \href{https://artofproblemsolving.com/community/user/29428}{pco}]
	\begin{tcolorbox}Thanks, but can you explain how did you obtain this? \end{tcolorbox}
Just write $f(x)=xg(\frac 1x-\frac 32)$ and equation becomes $g(\frac 1x-\frac 32)=g(\frac 32-\frac 1x)$
Hence the proposed general form for solutions.


\end{solution}



\begin{solution}[by \href{https://artofproblemsolving.com/community/user/29428}{pco}]
	\begin{tcolorbox}And what about $f\left(\frac{x}{x-1}\right)=\frac{f(x)-x}{1-x}$? Thanks.\end{tcolorbox}
Once again (as too often), WHAT IS THE DOMAIN OF FUNCTIONAL EQUATION, pleaaaaaase.
It could no longer be $[\frac 12,1]$ since LHS, RHS are undefined when $x=1$

Please copy exactly \begin{bolded}all \end{underlined}\end{bolded}the words you got in your exam, even if you think they are useless (they generally are not).



\end{solution}



\begin{solution}[by \href{https://artofproblemsolving.com/community/user/60529}{silvergrasshopper}]
	Sorry. Take $\left[-\frac{1}{2},\frac{1}{2}\right]$ as the domain. Thanks
\end{solution}



\begin{solution}[by \href{https://artofproblemsolving.com/community/user/29428}{pco}]
	\begin{tcolorbox}And what about $f\left(\frac{x}{x-1}\right)=\frac{f(x)-x}{1-x}$? Thanks.\end{tcolorbox}

\begin{tcolorbox}Sorry. Take $\left[-\frac{1}{2},\frac{1}{2}\right]$ as the domain. Thanks\end{tcolorbox}

[hide=Step by step approach]Equation is $f(\frac x{x-1})=\frac{f(x)-x}{1-x}$ $\forall x\in[-\frac 12,\frac 12]$
Note that we have no information on domain of $f(x)$ and so domain of functional equation implies that $f(x)$ must be defined over $[-1,\frac 12]$

Note that $x=0$ implies that $f(0)$ can be any value we want.
So let us consider now only $x\ne 0$

We can define $g(x)=f(\frac 1x)$, defined over $(-\infty,-1]\cup[2,+\infty)$ 
Equation becomes $g(1-\frac 1x)=\frac{g(\frac 1x)-x}{1-x}$ $\forall x\in[-\frac 12,0)\cup(0,\frac 12]$

And so $g(1-x)=\frac{g(x)-\frac 1x}{1-\frac 1x}$ $\forall x\in(-\infty,-2)\cup(2,+\infty)$

Which is $(xg(x)-\frac 12)+((1-x)g(1-x)-\frac 12)=0$ $\forall x\in(-\infty,-2)\cup(2,+\infty)$

Let then $h(x)=xg(x)-\frac 12$, defined over $(-\infty,-1]\cup[2,+\infty)$ 
Equation becomes $h(x)+h(1-x)=0$ $\forall x\in(-\infty,-2)\cup(2,+\infty)$

Let then $k(x)=h(x+\frac 12)$ defined over $(-\infty,-\frac 32]\cup[\frac 32,+\infty)$ 
Equation is $k(x-\frac 12)+k(\frac 12-x)$ $\forall x\in(-\infty,-2)\cup(2,+\infty)$
Which is $k(-x)=-k(x)$ $\forall x\in(-\infty,-\frac 52)\cup(\frac 32,+\infty)$
[\/hide]
Hence the solution :
Let $k(x)$ any function defined over $(-\infty,-\frac 32]\cup[\frac 32,+\infty)$ and such that :
$k(-x)=-k(x)$ $\forall x\in(-\infty,-\frac 52)\cup(\frac 32,+\infty)$
(you can choose any odd function over $\mathbb R$ but some non-odd functions also fit
Let $a\in\mathbb R$

$f(0)=a$
$f(x)=xk(\frac 1x-\frac 12)+\frac x2$ $\forall x\in[-1,0)\cup(0,\frac 12]$


\end{solution}
*******************************************************************************
-------------------------------------------------------------------------------

\begin{problem}[Posted by \href{https://artofproblemsolving.com/community/user/410895}{Idk2018}]
	$f\ :\ R^{+}\ \rightarrow\ R^{+}$
for every $x > 0$
find all $f(x)$ such that 
$f(2x)\geq x + f(f(x))$
	\flushright \href{https://artofproblemsolving.com/community/c6h1635326}{(Link to AoPS)}
\end{problem}



\begin{solution}[by \href{https://artofproblemsolving.com/community/user/29428}{pco}]
	\begin{tcolorbox}$f\ :\ R^{+}\ \rightarrow\ R^{+}$
for every $x > 0$
$f(2x)\geq x + f(f(x))$\end{tcolorbox}
Is the question you really got in your exam "find all $f(x)$ such that ..." ???
If so, change your problem source (best friend, little sister, national gazetta, ...).

It is elementary to prove that $f(x)\ge x$ $\forall x>0$
But, besides the trivial $f(x)=x$, there are obviously infinitely many functions matching the given inequality and I have serious doubts about the existence of a general form for all of them.

For example : $f(x)=2^{\left\lceil\log_2 x\right\rceil}$



\end{solution}



\begin{solution}[by \href{https://artofproblemsolving.com/community/user/410984}{gr8est}]
	I know that only solution is x
\end{solution}



\begin{solution}[by \href{https://artofproblemsolving.com/community/user/29428}{pco}]
	\begin{tcolorbox}I know that only solution is x\end{tcolorbox}
A lot of people know wrong things .... that's life.

And if you are right, then you proved that $2^{\left\lceil\log_2 x\right\rceil}=x$ $\forall x>0$ !!! (since $f(x)=2^{\left\lceil\log_2 x\right\rceil}$ is easily checked as one of the infinitely many solutions) and so $\log_2 x\in\mathbb Z$ $\forall x>0$
Which is Greeeeaaaaaat   :D 

Dont hesitate to post your proof.

\end{solution}
*******************************************************************************
-------------------------------------------------------------------------------

\begin{problem}[Posted by \href{https://artofproblemsolving.com/community/user/288210}{tenplusten}]
	Find all functions $f:\mathbb{R^+}\to\mathbb {R^+} $ such that for all $x,y\in R^+$ the followings hold:
$i) $ $f (x+y)\ge f (x)+y $
$ii) $ $f (f (x))\le x $
	\flushright \href{https://artofproblemsolving.com/community/c6h1635355}{(Link to AoPS)}
\end{problem}



\begin{solution}[by \href{https://artofproblemsolving.com/community/user/29428}{pco}]
	\begin{tcolorbox}Find all functions $f:\mathbb{R^+}\to\mathbb {R^+} $ such that for all $x,y\in R^+$ the followings hold:
$i) $ $f (x+y)\ge f (x)+y $
$ii) $ $f (f (x))\le x $\end{tcolorbox}
i) implies $f(x)$ is strictly increasing
ii) implies then $f(x)\le x$
i) implies $f(x)\ge f(x-y)+y>y$ $\forall y\in(0,x)$
Setting there $y\to x^-$, we get $f(x)\ge x$ $\forall x$

And so $\boxed{f(x)=x\quad\forall x>0}$ which indeed is a solution.


\end{solution}



\begin{solution}[by \href{https://artofproblemsolving.com/community/user/288210}{tenplusten}]
	I got the first 3 steps ,can you please eloborate what you do in your final step?
\end{solution}



\begin{solution}[by \href{https://artofproblemsolving.com/community/user/29428}{pco}]
	\begin{tcolorbox}I got the first 3 steps ,can you please eloborate what you do in your final step?\end{tcolorbox}

We have $f(x)>y$ $\forall y\in(0,x)$ 
Set there $y$ as near as you want of $x$ (but below) and this gives $f(x)\ge x$

\end{solution}
*******************************************************************************
-------------------------------------------------------------------------------

\begin{problem}[Posted by \href{https://artofproblemsolving.com/community/user/391407}{Hamel}]
	Find all functions $f:\mathbb{R} \rightarrow \mathbb{R}$ such that $f(yf(x+y)+x)=f(y)^2+f((x-1)f(y))$
(I have no full proof of whether they are countable.)
	\flushright \href{https://artofproblemsolving.com/community/c6h1635535}{(Link to AoPS)}
\end{problem}



\begin{solution}[by \href{https://artofproblemsolving.com/community/user/29428}{pco}]
	\begin{tcolorbox}Find all functions $f:\mathbb{R} \rightarrow \mathbb{R}$ such that $f(yf(x+y)+x)=f(y)^2+f((x-1)f(y))$
(I have no full proof of whether they are countable.)\end{tcolorbox}
Easy to show that :
1) The only constant solution is $f(x)=0\quad\forall x$
2) The only injective solution is $f(x)=x+1\quad\forall x$
3) The only surjective solution is $f(x)=x+1\quad\forall x$
4) there are infinitely many nonconstant, noninjective, nonsurjective solutions.

For example :
Let $\mathbb Q,A$ two supplementary subvectorspaces of the $\mathbb Q$-vectorspace $\mathbb R$
Let $q(x)$ from $\mathbb R\to\mathbb Q$ and $a(x)$ from $\mathbb R\to A$ the two projections of a real $x$ in $(\mathbb Q,A)$
Then choose $f(x)=q(x)+1$

Note that this is far from being the whole set of solutions.


\end{solution}
*******************************************************************************
-------------------------------------------------------------------------------

\begin{problem}[Posted by \href{https://artofproblemsolving.com/community/user/209049}{math90}]
	Find all strictly increasing functions $f:\mathbb R^+\to\mathbb R^+$ such that
$f(2x)\geq x+f(f(x))$
for all $x\in\mathbb R^+$.

Inspired from:
https:\/\/artofproblemsolving.com\/community\/c6t300f6h1635326_function
	\flushright \href{https://artofproblemsolving.com/community/c6h1635852}{(Link to AoPS)}
\end{problem}



\begin{solution}[by \href{https://artofproblemsolving.com/community/user/29428}{pco}]
	\begin{tcolorbox}Find all strictly increasing functions $f:\mathbb R^+\to\mathbb R^+$ such that
$f(2x)\geq x+f(f(x))$
for all $x\in\mathbb R^+$.

Inspired from:
https:\/\/artofproblemsolving.com\/community\/c6t300f6h1635326_function\end{tcolorbox}
I dont understand.
The pointed problem is clearly a fake problem.

So I supposed that this modified \/ adapted problem was the good one (and so that adding the "strictly increasing" constraint allowed to solve it and find likely $f(x)\equiv x$ as unique solution).

But, once again, this problem has infinitely many solutions, likely without general form.
Are you just kidding, trying to create problems without having their solution ?
Or have you all a real problem really submitted in some olympiad exam \/ training session ? (and noone succeeded in posting the exact copy of the official statement) ?

Example of infinite family of solutions :

Let $a_0\in(0,1)$ and $a_{n+1}=\frac{a_n^2+1}2$
Define $f(x)$ as :

$\forall x\in(0,1]$ : $f(x)=x$
$\forall x\in\left(2^n,2^{n+1}\right]$ : $f(x)=2^{n+1}+a_n(x-2^{n+1})$ whatever is $n\in\mathbb Z_{\ge 0}$

I let you, as a rather simple exercise, check that these indeed are strictly increasing functionsmatching the property $f(2x)\ge x+f(f(x))$



\end{solution}



\begin{solution}[by \href{https://artofproblemsolving.com/community/user/209049}{math90}]
	I indeed tried to find an additional constraint to make $f(x)=x$ be the unique solution. I see now it didn’t succeed so much.
\end{solution}
*******************************************************************************
-------------------------------------------------------------------------------

\begin{problem}[Posted by \href{https://artofproblemsolving.com/community/user/401393}{mruczek}]
	Determine, whether exists function $f$, which assigns each integer $k$, nonnegative integer $f(k)$ and meets the conditions:
$f(0) > 0$,
for each integer $k$ minimal number of the form $f(k - l) + f(l)$, where $l \in \mathbb{Z}$, equals $f(k)$.
	\flushright \href{https://artofproblemsolving.com/community/c6h1636068}{(Link to AoPS)}
\end{problem}



\begin{solution}[by \href{https://artofproblemsolving.com/community/user/29428}{pco}]
	\begin{tcolorbox}Determine, whether exists function $f$, which assigns each integer $k$, nonnegative integer $f(k)$ and meets the conditions:
$f(0) > 0$,
for each integer $k$ minimal number of the form $f(k - l) + f(l)$, where $l \in \mathbb{Z}$, equals $f(k)$.\end{tcolorbox}
Note that $\min_{y\in Z}(f(x-y)+f(y))=f(x)$ is equivalent to :
c1) $f(x+y)\le f(x)+f(y)$ $\forall x,y\in\mathbb Z$
c2) $\forall x$, $\exists y$ such that $f(x)=f(x-y)+f(y)$

Let $m=\min_{x\in Z}f(x)$ ($m\ge 0$)
Let $u$ such that $f(u)=m$

Using c2, $\exists v$ such that $m=f(u)=f(u-v)+f(v)\ge 2m$ and so $m=0$
Let $A=\{x\in\mathbb Z$ such that $f(x)=0\}$. Note that $0\notin A$
c1) implies then : c3) $a,b\in A$ implies $a+b\in A$

If all elements of $A$ are $>0$ : c2 is wrong for $x=\min (A)$
If all elements of $A$ are $<0$ : c2 is wrong for $x=\max (A)$
So $\exists a=\min(A\cap\mathbb Z_{>0})$ and $b=-\max(A\cap\mathbb Z_{<0})$
Let $u=\gcd(a,b)$
$\exists$ nonnegative integers $m,n,p,q$ such that $ma+n(-b)=u$ and $pa+q(-b)=-u$
Then , using c3, ge get that $u,-u\in A$ and so $u+(-u)=0\in A$, impossible

And so $\boxed{\text{No such function}}$




\end{solution}
*******************************************************************************
-------------------------------------------------------------------------------

\begin{problem}[Posted by \href{https://artofproblemsolving.com/community/user/407889}{DurdonTyler}]
	Let's say a $f$ function: non-decreasing in $[0,1]$ interval. For $x\in[0,1]$: $f(x) + f(1-x)=1$ and$f(x)=2f(\frac{x}{3})$. Find $ f(\frac {5} {8}) $
	\flushright \href{https://artofproblemsolving.com/community/c6h1636426}{(Link to AoPS)}
\end{problem}



\begin{solution}[by \href{https://artofproblemsolving.com/community/user/29428}{pco}]
	\begin{tcolorbox}Let's say a $f$ function: non-decreasing in $[0,1]$ interval. For $x\in[0,1]$: $f(x) + f(1-x)=1$ and$f(x)=2f(\frac{x}{3})$. Find $ f(\frac {5} {8}) $\end{tcolorbox}

It is just CANTOR function (see https:\/\/en.wikipedia.org\/wiki\/Cantor_function)

And so $f(\frac 58)=f(\overline{0.1.....}_3)=\overline{0.1}_2=\frac 12$

\end{solution}



\begin{solution}[by \href{https://artofproblemsolving.com/community/user/407889}{DurdonTyler}]
	\begin{tcolorbox}[quote=DurdonTyler]Let's say a $f$ function: non-decreasing in $[0,1]$ interval. For $x\in[0,1]$: $f(x) + f(1-x)=1$ and$f(x)=2f(\frac{x}{3})$. Find $ f(\frac {5} {8}) $\end{tcolorbox}

It is just CANTOR function (see https:\/\/en.wikipedia.org\/wiki\/Cantor_function)

And so $f(\frac 58)=f(\overline{0.1.....}_3)=\overline{0.1}_2=\frac 12$\end{tcolorbox}
Thank you and can you show us the proof?
\end{solution}



\begin{solution}[by \href{https://artofproblemsolving.com/community/user/29428}{pco}]
	\begin{tcolorbox}Thank you and can you show us the proof?\end{tcolorbox}
This is a course question. Look for "Cantor function" on AOPS, or on Google, or anywhere in the world



\end{solution}
*******************************************************************************
-------------------------------------------------------------------------------

\begin{problem}[Posted by \href{https://artofproblemsolving.com/community/user/374509}{abbosjon2002}]
	Find all functions from reals to reals such that $(a-b)f(a+b)+(b-c)f(b+c)+(c-a)f(c+a)=0$ for all $a,b,c$
	\flushright \href{https://artofproblemsolving.com/community/c6h1636428}{(Link to AoPS)}
\end{problem}



\begin{solution}[by \href{https://artofproblemsolving.com/community/user/29428}{pco}]
	\begin{tcolorbox}Find all functions from reals to reals such that $(a-b)f(a+b)+(b-c)f(b+c)+(c-a)f(c+a)=0$ for all $a,b,c$\end{tcolorbox}
Let $P(x,y,z)$ be the assertion $(x-y)f(x+y)+(y-z)f(y+z)+(z-x)f(z+x)=0$

$P(\frac{x+1}2,\frac{x-1}2,\frac{1-x}2)$ $\implies$ $f(x)=x(f(1)-f(0))+f(0)$

And so $\boxed{f(x)=ax+b\quad\forall x}$
which indeed is a solution, whatever are $a,b\in\mathbb R$


\end{solution}



\begin{solution}[by \href{https://artofproblemsolving.com/community/user/209049}{math90}]
	This problem is old and I saw it many times. I also think that pco posted a solution for it more than once.
\end{solution}
*******************************************************************************
-------------------------------------------------------------------------------

\begin{problem}[Posted by \href{https://artofproblemsolving.com/community/user/353273}{Jason99}]
	Given a function $f : \mathbb{R} \to \mathbb{R}$ so that
$$f(z f(x) f(y)) + f( f(z) (f(x) + f(y) ) ) = f(xyz + xz + yz) \quad \forall x,y,z \in \mathbb{R}$$

Find the sum of all possible value of $f(2018)$.
	\flushright \href{https://artofproblemsolving.com/community/c6h1636434}{(Link to AoPS)}
\end{problem}



\begin{solution}[by \href{https://artofproblemsolving.com/community/user/406831}{GorgonMathDota}]
	0 exactly
\end{solution}



\begin{solution}[by \href{https://artofproblemsolving.com/community/user/406831}{GorgonMathDota}]
	Only three such functions exist, indeed $f(x) = x$, $f(x) = -x$ and $f(x) = 0$.
\end{solution}



\begin{solution}[by \href{https://artofproblemsolving.com/community/user/363884}{matinyousefi}]
	\begin{tcolorbox}Only three such functions exist, indeed $f(x) = x$, $f(x) = -x$ and $f(x) = 0$.\end{tcolorbox}

could you post your solution.
\end{solution}



\begin{solution}[by \href{https://artofproblemsolving.com/community/user/29428}{pco}]
	No need to find all the solutions.

Besides $f\equiv 0$, we get that $f(x)$ solution implies $-f(x)$ solution and so any $f(2018)$ is cancelled by the $-f(2018)$
Hence the result.

[hide=formal doubt]
In fact one could claim "what if infinitely many such $f(x)$ existed ?
Then just claiming thar reordering the sum gives zerp is not enough : it would be mandatory to proove that the sum indeed converges ... [\/hide]
\end{solution}



\begin{solution}[by \href{https://artofproblemsolving.com/community/user/377323}{IndoMathXdZ}]
	I always wonder, how to solve this type of complex functional equation.
\end{solution}



\begin{solution}[by \href{https://artofproblemsolving.com/community/user/376213}{Wizard_32}]
	Probably take help from the user just above you ;) 
\end{solution}



\begin{solution}[by \href{https://artofproblemsolving.com/community/user/376542}{falantrng}]
	Find all function $f : \mathbb{R} \to \mathbb{R}$ so that
$$f(z f(x) f(y)) + f( f(z) (f(x) + f(y) ) ) = f(xyz + xz + yz) \quad \forall x,y,z \in \mathbb{R}$$
 Let's do this.



\end{solution}



\begin{solution}[by \href{https://artofproblemsolving.com/community/user/29428}{pco}]
	\begin{tcolorbox}Find all function $f : \mathbb{R} \to \mathbb{R}$ so that
$$f(z f(x) f(y)) + f( f(z) (f(x) + f(y) ) ) = f(xyz + xz + yz) \quad \forall x,y,z \in \mathbb{R}$$
 Let's do this.\end{tcolorbox}
$\boxed{\text{S1 : }f(x)=0\quad\forall x}$ is the only constant solution. So let us from now look only for nonconstant solutions.
Let $P(x,y,z)$ be the assertion $f(zf(x)f(y))+f(f(z)(f(x)+f(y)))=f(xyz+xz+yz)$

1) $f(x)=0$ $\iff$ $x=0$
Let $a=f(0)$
$P(0,0,0)$ $\implies$ $f(2a^2)=0$
$P(2a^2,2a^2,0)$ $\implies$ $2a=a$ and so $a=0$
If $f(u)=0$ for some $u\ne 0$, then $P(x,0,u)$ $\implies$ $f(ux)=0$, impossible since $f(x)$ is not a constant function.
Q.E.D.

2) The only injective solutions are $f\equiv x$ and $f\equiv -x$
$P(x,0,1)$ $\implies$ $f(f(1)f(x))=f(x)$ and so, since injective, $f(x)=f(1)x$
Plugging this in $P(x,y,z)$ we get $f(1)\in\{-1,+1\}$ and so :
$\boxed{\text{ S2 : }f(x)=x\quad\forall x}$ and $\boxed{\text{ S3 : }f(x)=-x\quad\forall x}$ 
Q.E.D.

3) No nonconstant non injective solutions
$f(a)=f(b)=0$ implies $a=b=0$ (see 1) above)
If $f(a)=f(b)\ne 0$ for some $a\ne b$, then :
Subtracting $P(x,0,a)$ from $P(x,0,b)$, ge get $f(ax)=f(bx)$ $\forall x$
And so $f(ux)=f(x)$ $\forall x$ and for some real $u=\frac ab\notin\{0,1\}$

Comparing then $P(x,y,z)$ with $P(ux,y,z)$ where $x,y,z\in\mathbb R$ we get :
$f(uxyz+uxz+yz)=f(xyz+xz+yz)$
And, choosing $s\ne t$, it is easy to find $x,y,z$ such that :
$xyz+xz+yz=s$
$uxyz+uxz+yz=t$
And so $f(s)=f(t)$ $\forall s\ne t$, impossible since $f(x)$ is non constant
Q.E.D.



\end{solution}
*******************************************************************************
-------------------------------------------------------------------------------

\begin{problem}[Posted by \href{https://artofproblemsolving.com/community/user/363884}{matinyousefi}]
	Find all functions $f :\mathbb R \to \mathbb R$ such that for all real $x, y$
\[f(x^2 - y) = xf(x) - f(y).\]
	\flushright \href{https://artofproblemsolving.com/community/c6h1636471}{(Link to AoPS)}
\end{problem}



\begin{solution}[by \href{https://artofproblemsolving.com/community/user/363884}{matinyousefi}]
	actually we can easily get $f(x+y)=f(x)+f(y)$ and $f(x^2)=xf(x)$ but i don't know what to do after that

\end{solution}



\begin{solution}[by \href{https://artofproblemsolving.com/community/user/353869}{FuadAnzurov2003}]
	\begin{tcolorbox}actually we can easily get $f(x+y)=f(x)+f(y)$ and $f(x^2)=xf(x)$ but i don't know what to do after that\end{tcolorbox}

I think $f(x+y)=f(x)+f(y)$ is first Cauchy's function. $f(x)=cx$. Then we check and it is true.
\end{solution}



\begin{solution}[by \href{https://artofproblemsolving.com/community/user/392094}{Duy_Thai2002}]
	But f is not Continuous function so that we can't thus  $f(x)=cx$.
\end{solution}



\begin{solution}[by \href{https://artofproblemsolving.com/community/user/330150}{L3435}]
	\begin{tcolorbox}actually we can easily get $f(x+y)=f(x)+f(y)$ and $f(x^2)=xf(x)$ but i don't know what to do after that\end{tcolorbox}

$$(x+y)f(x+y)=(x+y)(f(x)+f(y))$$
$$f(x^2+2xy+y^2)=f(x^2)+f(y^2)+yf(x)+xf(y)$$
$$f(2xy)=yf(x)+xf(y)$$
$y\rightarrow 1$
$$f(x)=xf(1)$$
$$\boxed{f(x)=cx}$$
\end{solution}



\begin{solution}[by \href{https://artofproblemsolving.com/community/user/363884}{matinyousefi}]
	\begin{tcolorbox}[quote=matinyousefi]actually we can easily get $f(x+y)=f(x)+f(y)$ and $f(x^2)=xf(x)$ but i don't know what to do after that\end{tcolorbox}

$$(x+y)f(x+y)=(x+y)(f(x)+f(y))$$
$$f(x^2+2xy+y^2)=f(x^2)+f(y^2)+yf(x)+xf(y)$$
$$f(2xy)=yf(x)+xf(y)$$
$y\rightarrow 1$
$$f(x)=xf(1)$$
$$\boxed{f(x)=cx}$$\end{tcolorbox}

Thanks for your help  :-D 
\end{solution}



\begin{solution}[by \href{https://artofproblemsolving.com/community/user/363884}{matinyousefi}]
	what if we have $$f(x+y)=f(x)+f(y),f(x^3)=x^2f(x)$$ 
\end{solution}



\begin{solution}[by \href{https://artofproblemsolving.com/community/user/398616}{MNJ2357}]
	Maybe [url=https:\/\/artofproblemsolving.com\/community\/c6h620959p3710932]this[\/url] would help.
\end{solution}



\begin{solution}[by \href{https://artofproblemsolving.com/community/user/29428}{pco}]
	\begin{tcolorbox}what if we have $$f(x+y)=f(x)+f(y),f(x^3)=x^2f(x)$$\end{tcolorbox}
Let $k\in\mathbb Q$.
Consider the expression $f((x+k)^3)=(x+k)^2f(x+k)$ as a polynomial in $k$ with infinitely many rooots (all rational numbers) and so as the zero polynomial, so with all coefficients being zero.

Look then at the coefficient of $k^2$ : $2(f(x)-xf(1))$ and you get $f(x)=f(1)x$ $\forall x$



\end{solution}
*******************************************************************************
-------------------------------------------------------------------------------

\begin{problem}[Posted by \href{https://artofproblemsolving.com/community/user/307525}{Haismit}]
	for function f(x) such that $\int_{0}^{1}xf(x)dx=0$ and on [0;1] then max|f(x)|=1 prove that $\frac{-5}{4}<\int_{0}^{1}e^xf(x)dx<\frac{3}{2}$
	\flushright \href{https://artofproblemsolving.com/community/c7h1633308}{(Link to AoPS)}
\end{problem}



\begin{solution}[by \href{https://artofproblemsolving.com/community/user/307525}{Haismit}]
	Where pco ? Can you help me
\end{solution}



\begin{solution}[by \href{https://artofproblemsolving.com/community/user/29428}{pco}]
	\begin{tcolorbox}for function f(x) such that $\int_{0}^{1}xf(x)dx=0$ and on [0;1] then max|f(x)|=1 prove that $\frac{-5}{4}<\int_{0}^{1}e^xf(x)dx<\frac{3}{2}$\end{tcolorbox}
There exists quite better bounds :

$A=\int_0^1e^xf(x)dx=\int_0^1(e^x-ex)f(x)dx$ 
And since $e^x-ex\ge 0$ $\forall x\in[0,1]$ and $f(x)\in[-1,+1]$ $\forall x\in[0,1]$ :

$-\int_0^1(e^x-ex)dx\le A\le \int_0^1(e^x-ex)dx=\frac e2-1<\frac 32-1=\frac 12$

And so $\boxed{-\frac 12<\int_0^1e^xf(x)dx<\frac 12}$



\end{solution}
*******************************************************************************

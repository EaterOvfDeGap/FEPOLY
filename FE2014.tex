-------------------------------------------------------------------------------

\begin{problem}[Posted by \href{https://artofproblemsolving.com/community/user/78444}{Babai}]
	Find all functions f:R(+)-R(+) such that f(f(x))=6x-f(x) for all real x.
	\flushright \href{https://artofproblemsolving.com/community/c6h336063}{(Link to AoPS)}
\end{problem}



\begin{solution}[by \href{https://artofproblemsolving.com/community/user/29428}{pco}]
	\begin{tcolorbox}Find all functions f:R(+)-R(+) such that f(f(x))=6x-f(x) for all real x.\end{tcolorbox}

Let $ a > 0$ and $ b = f(a)$ and let $ x_n$ be the sequence defined as :

$ x_0 = a$, $ x_1 = b$ and $ x_{n + 2} = - x_{n + 1} + 6x_n$ such that we get $ x_{n + 1} = f(x_n)$

Direct resolution of the sequence gives $ x_n = \frac {2a - b}5( - 3)^n + \frac {3a + b}52^n$

If $ 2a - b > 0$ then  $ x_{2n + 1} =f(x_{2n})  < 0$ for $ n$ great enough and this is impossible since $ f(x)$ is from $ \mathbb R^ + \to\mathbb R^ +$
If $ 2a - b < 0$ then  $ x_{2n}=f(x_{2n-1}) < 0$ for $ n$ great enough and this is impossible since $ f(x)$ is from $ \mathbb R^ + \to\mathbb R^ +$

So $ b = 2a$ and so $ f(x) = 2x$, which indeed is a solution.

Hence the answer : $ \boxed{f(x) = 2x}$ $ \forall x$
\end{solution}



\begin{solution}[by \href{https://artofproblemsolving.com/community/user/206331}{TripteshBiswas}]
	Ignore, Wrong proof.
\end{solution}



\begin{solution}[by \href{https://artofproblemsolving.com/community/user/29428}{pco}]
	Yes, there is something quite wrong in your solution : "Let f(x)=mx+n" ... You can't make such an assumtion without any reason !
\end{solution}



\begin{solution}[by \href{https://artofproblemsolving.com/community/user/206331}{TripteshBiswas}]
	A lot of thanks to PCO i thought that $f$ is injective and so $f(x)$ must be in the form $mx+n.$
\end{solution}



\begin{solution}[by \href{https://artofproblemsolving.com/community/user/206331}{TripteshBiswas}]
	PCO can u please help me to solve the problem. I solved the problem similar method above.
Find all continuous function f satisfying $3f(2x+1)=f(x)+5x$
\end{solution}



\begin{solution}[by \href{https://artofproblemsolving.com/community/user/29428}{pco}]
	\begin{tcolorbox}PCO can u please help me to solve the problem. I solved the problem similar method above.
Find all continuous function f satisfying 3f(2x+1)=f(x)+5x\end{tcolorbox}
Let $f(x)=g(x+1)+x-\frac 32$ and equation becomes $3g(2(x+1))=g(x+1)$ and so $3g(2x)=g(x)$

Continuity implies then $g(x)=0$ $\forall x$ and so $\boxed{f(x)=x-\frac 32}$
\end{solution}



\begin{solution}[by \href{https://artofproblemsolving.com/community/user/206331}{TripteshBiswas}]
	Sorry I can not $LATEXIFIED$ it. (as it needs calculus)
\end{solution}



\begin{solution}[by \href{https://artofproblemsolving.com/community/user/29428}{pco}]
	I'm sorry but your post is quite impossible to read and understand.
I suggest you to read some simple tutorial on $\LaTeX$ and \/ or to click on formulas of existing posts and then to repost something understandable.
\end{solution}
*******************************************************************************
-------------------------------------------------------------------------------

\begin{problem}[Posted by \href{https://artofproblemsolving.com/community/user/83662}{ilikemaths}]
	Let's start a marathon on functional equations:
When you solve a problem, you should post a new one.

Here's problem 1:
Find all functions $f: \mathbb{Q}_{>0} \rightarrow \mathbb{Q}_{>0}$ that satisfy:
$f(x+1)=f(x)+1$ $\forall x \in \mathbb{Q}_{>0}$ and
$f(x^2) = f(x)^2$ $\forall x \in \mathbb{Q}_{>0}$.

[color=#FF0000][moderator edit: stickied in Pre-Olympiad forum.][\/color]
	\flushright \href{https://artofproblemsolving.com/community/c6h350187}{(Link to AoPS)}
\end{problem}



\begin{solution}[by \href{https://artofproblemsolving.com/community/user/61082}{Pain rinnegan}]
	\begin{tcolorbox}Let's start a marathon on functional equations:
When you solve a problem, you should post a new one.

\begin{bolded}Problem 1:\end{bolded} Find all functions $f: \mathbb{Q}_{+} \rightarrow \mathbb{Q}_{+}$ that satisfy:

(1).$f(x+1)=f(x)+1$ $\ ,\ \forall x \in \mathbb{Q}_{+}$ and

(2).$f(x^2) = f(x)^2$ $\ ,\ \forall x \in \mathbb{Q}_{+}$.\end{tcolorbox}

I like this idea :)

[hide="Problem 1"]From (1), we can easily find by induction that $f(x+n)=f(x)+n\ ,\ (\forall)x\in \mathbb{Q}_+\ ,\ (\forall)n\in \mathbb{N}$. Therefore by (2) we have:

\[f((x+n)^2)=f^2(x+n)\Leftrightarrow f(x^2+2nx+n^2)=(f(x)+n)^2\Leftrightarrow\]

\[ f(x^2+2nx)+n^2=f^2(x)+2nf(x)+n^2\Leftrightarrow f(x^2+2nx)=f^2(x)+2nf(x)\]

Now let's put $x=\displaystyle\frac{p}{q}\ ,\ p,q\in \mathbb{N}^*$ and let $n\rightarrow q$

\[\Rightarrow f\left(\frac{p^2}{q^2}+2p\right)=f^2\left(\frac{p}{q}\right)+2qf\left(\frac{p}{q}\right)\Leftrightarrow \]

\[f\left(\frac{p^2}{q^2}\right)+2p=f\left(\frac{p^2}{q^2}\right)+2qf\left(\frac{p}{q}\right)\]

So $f\left(\displaystyle\frac{p}{q}\right)=\displaystyle\frac{p}{q}$. So $f(x)=x\ ,\ (\forall)x\in \mathbb{Q}_+$ which verifies the initial equation.[\/hide]

\begin{bolded}Problem 2:\end{bolded} Determine all the functions $f:\mathbb{R}\rightarrow \mathbb{R}$ such that:

\[f(x^3)-f(y^3)=(x^2+xy+y^2)(f(x)-f(y))\ ,\ (\forall)x,y\in \mathbb{R}\]
\end{solution}



\begin{solution}[by \href{https://artofproblemsolving.com/community/user/64868}{mahanmath}]
	\begin{tcolorbox}\begin{bolded}Problem 2:\end{bolded} Determine all the functions $f:\mathbb{R}\rightarrow \mathbb{R}$ such that:

\[f(x^3)-f(y^3)=(x^2+xy+y^2)(f(x)-f(y))\ ,\ (\forall)x,y\in \mathbb{R}\]\end{tcolorbox}

 [hide="Problem 2"]WLOG assume $f(0)=0$ (otherwise let $F(x)= f(x)-f(0)$ then you can easily see that it works in equation!) .

Now put $y=0$ , we get $f(x^3) =(x^2)f(x)$ . Substitute in the main equation we get $f(x)=xf(1)$.

So the answer is $f(x)=xf(1) + f(0)$[\/hide]

\begin{bolded}Problem 3\end{bolded} : Find all the continuous functions $ f :\mathbb{R}\mapsto\mathbb{R} $ such that $ \forall x,y\in\mathbb{R} $ :

$ (1+f(x)f(y))f(x+y)=f(x)+f(y) $
\end{solution}



\begin{solution}[by \href{https://artofproblemsolving.com/community/user/61082}{Pain rinnegan}]
	\begin{tcolorbox}

\begin{bolded}Problem 3\end{bolded} : Find all the continuous functions $ f :\mathbb{R}\mapsto\mathbb{R} $ such that $ \forall x,y\in\mathbb{R} $ :

$ (1+f(x)f(y))f(x+y)=f(x)+f(y) $\end{tcolorbox}

It's a too recent question: http://www.artofproblemsolving.com/Forum/viewtopic.php?f=36&t=350104&start=0&hilit=continuous .
\end{solution}



\begin{solution}[by \href{https://artofproblemsolving.com/community/user/83662}{ilikemaths}]
	Then pose a new problem.
\end{solution}



\begin{solution}[by \href{https://artofproblemsolving.com/community/user/61082}{Pain rinnegan}]
	\begin{bolded}Problem 4:\end{bolded} Determine all the functions $f:\mathbb{R}\rightarrow \mathbb{R}$ such that:

\[f(x^3+y^3)=xf(x^2)+yf(y^2)\ ,\ (\forall)x,y\in \mathbb{R}\]
\end{solution}



\begin{solution}[by \href{https://artofproblemsolving.com/community/user/83662}{ilikemaths}]
	$x=y=0$ yields $f(0)=0$.
$x=0$ yields $f(y^3)=yf(y^2)$, so the given functional equation reduces to:
$f(x^3+y^3)=f(x^3)+f(y^3)$.
Setting $a=x^3$, $b=y^3$ gives:
$f(a+b)=f(a)+f(b)$, which is a Cauchy-equation with solutions:
$f(x)=0$ and $f(x)=cx$ for some $c \in \mathbb{R}$.
So we have two possible functions:
$f(x)=0$ and $f(x)=cx$ for some $c \in \mathbb{R}$ and a quick check tells us that both functions satisfy.

New problem:
find all functions $f: \mathbb{R}_{>0} \rightarrow \mathbb{R}$ satisfying:
$f(x+y)-f(y) = \frac{x}{y(x+y)}$
\end{solution}



\begin{solution}[by \href{https://artofproblemsolving.com/community/user/64868}{mahanmath}]
	[hide="Problem 5"]Same as problem 2 ! WLOG assume $f(1) = -1$ , We claim that $f(x)= \frac{-1}{x}$ .To see this fact just put $y=1$ . Hence the answer is $f(x) = \frac{-1}{x} + c$ for some real $c$.[\/hide]

\begin{bolded}Problem 6\end{bolded}.Determine all the functions $ f:\mathbb{R}\rightarrow\mathbb{R} $ such that:

[img]http://latex.codecogs.com\/gif.latex?f%5Cbig%28x+yf%28x%29%5Cbig%29+f%5Cbig%28xf%28y%29-y%5Cbig%29=f%28x%29-f%28y%29+2xy[\/img]
\end{solution}



\begin{solution}[by \href{https://artofproblemsolving.com/community/user/64868}{mahanmath}]
	\begin{tcolorbox}
$f(a+b)=f(a)+f(b)$, which is a Cauchy-equation with solutions:
$f(x)=0$ and $f(x)=cx$ for some $c \in \mathbb{R}$.
So we have two possible functions:
$f(x)=0$ and $f(x)=cx$ for some $c \in \mathbb{R}$ and a quick check tells us that both functions satisfy.
$\end{tcolorbox}

I`m not sure but I think Cauchy-equation just solve continuous functions. Am I right ?
\end{solution}



\begin{solution}[by \href{https://artofproblemsolving.com/community/user/67223}{Amir Hossein}]
	\begin{tcolorbox}WLOG assume $f(1) = -1$\end{tcolorbox}

Can we do this ?
How  Without Loss Of Generality ?
\end{solution}



\begin{solution}[by \href{https://artofproblemsolving.com/community/user/52090}{Dumel}]
	\begin{tcolorbox}
\begin{bolded}Problem 6\end{bolded}.Determine all the functions $ f:\mathbb{R}\rightarrow\mathbb{R} $ such that:
[img]http://latex.codecogs.com\/gif.latex?f%5Cbig%28x+yf%28x%29%5Cbig%29+f%5Cbig%28xf%28y%29-y%5Cbig%29=f%28x%29-f%28y%29+2xy[\/img]\end{tcolorbox}[hide="solution"]$P(0,0)$ gives $f(0)=0$. now $P(0,x) \to  \ f(-x)=-f(x)$
Finally adding $P(x,y)$ and $P(-y,x)$ we get $f(x)=0$[\/hide]

Post a bit harder problems, please! :-)
\begin{bolded}Problem 7\end{bolded}
Find the least possible value of $f(1998)$ where $f: \mathbb{N} \to \mathbb{N} $ satisfies 
$f(n^2f(m)) = m(f(n))^2$
\end{solution}



\begin{solution}[by \href{https://artofproblemsolving.com/community/user/83662}{ilikemaths}]
	Dumel, your solution isn't correct:
$f(x)=x$ also satisfies!
\end{solution}



\begin{solution}[by \href{https://artofproblemsolving.com/community/user/61082}{Pain rinnegan}]
	\begin{tcolorbox} we get $f(x)=0$\end{tcolorbox}

$f(x)=0$ is not a solution . :|
\end{solution}



\begin{solution}[by \href{https://artofproblemsolving.com/community/user/52090}{Dumel}]
	oh what a terrible mistake :blush:
At this moment I don't know how to solve this problem.
\end{solution}



\begin{solution}[by \href{https://artofproblemsolving.com/community/user/64868}{mahanmath}]
	\begin{tcolorbox}[quote="mahanmath"]WLOG assume $f(1) = -1$\end{tcolorbox}

Can we do this ?
How  Without Loss Of Generality ?\end{tcolorbox}
As I said ~~ SAME AS PROBLEM \begin{bolded}2\end{bolded} ~~
\end{solution}



\begin{solution}[by \href{https://artofproblemsolving.com/community/user/335829}{vjdjmathaddict}]
	\begin{tcolorbox}[quote="shmm"]Yes , it was Olympiad problem\end{tcolorbox}
Dont hesitate to post here the official general form for the infinitely many solutions when you'll get it :)\end{tcolorbox}

Let's consider the functions

$ p_n(x) = (\sin x + \cos x)^n - \sin^n x - \cos^n x$.

Now, $ f(x) = \sum a_k x^k$ is a solution iff $ \sum a_k p_k(x) = 0$.

Lemma: Every symmetric polynomial $ P(a, b)$ can be written as a polynomial in $ ab, a + b$.

Now, since $ p_n(x)$ is a symmetric polynomial in $ \sin x, \cos x$ and since $ \sin x \cos x = \frac {(\sin x + \cos x)^2 - 1}{2}$, it follows that $ p_n(x)$ is a polynomial of degree $ n$ in $ u = \sin x + \cos x$ (except when $ n = 1$, in which case $ p_1(x) = 0$.)

Note that $ u$ takes on every value in $ [ - \sqrt {2}, \sqrt {2} ]$. It follows that if $ \sum_{k = 1}^{n} a_k p_k(x) = 0$ then the corresponding polynomial in $ u$ must be identically zero, but since $ p_n(x)$ is the only term of degree $ n$ this is not possible unless $ n = 1$.

Hence the only solution is $ f(x) = \boxed{ax}$.
\end{solution}



\begin{solution}[by \href{https://artofproblemsolving.com/community/user/29428}{pco}]
	Quite wrong.
For example, any additive non linear function is trivially a solution.
You seemed in your answer to consider that "function" means "polynomial" ...
... which unfortunately is wrong.

\end{solution}



\begin{solution}[by \href{https://artofproblemsolving.com/community/user/335829}{vjdjmathaddict}]
	I am very sorry mr pco I have indeed assumed f to be a polynomial.I didn't read that question properly and as far as that goes this was the problem which he meant to give in olympiad.
\end{solution}



\begin{solution}[by \href{https://artofproblemsolving.com/community/user/328399}{mqcase1004}]
	Hello everyone, I'm new. ^^ I hope everyone will help me.
Problem 439: Find all functions $f:(0,\infty )\rightarrow (0,\infty )$ such that $f(x+1)=f(\frac{x+2}{x+3}),\forall x>0 $

\end{solution}



\begin{solution}[by \href{https://artofproblemsolving.com/community/user/29428}{pco}]
	\begin{tcolorbox}Hello everyone, I'm new. ^^ I hope everyone will help me.
Problem 439: Find all functions $f:(0,\infty )\rightarrow (0,\infty )$ such that $f(x+1)=f(\frac{x+2}{x+3}),\forall x>0 $\end{tcolorbox}

General solution :

Let $g(x)$ any function from $(0,1]\to (0,+\infty)$

Just define $f(x)$ as :
$\forall x\le 1$ : $f(x)=g(x)$
$\forall x>1$ : $f(x)=g(\frac{x+1}{x+2})$


\end{solution}



\begin{solution}[by \href{https://artofproblemsolving.com/community/user/328399}{mqcase1004}]
	Can you solve it in detail? Thanks :v
\end{solution}



\begin{solution}[by \href{https://artofproblemsolving.com/community/user/29428}{pco}]
	\begin{tcolorbox}Can you solve it in detail? Thanks :v\end{tcolorbox}

What detail more do you want ?
This is just your definition : knowledge of $f(x)$ over $(0,1]$ gives full knowledge over $\mathbb R^+$

\end{solution}



\begin{solution}[by \href{https://artofproblemsolving.com/community/user/328399}{mqcase1004}]
	If the problem is changed into "Find the constant function", what will the solution be? Thanks.
\end{solution}



\begin{solution}[by \href{https://artofproblemsolving.com/community/user/29428}{pco}]
	\begin{tcolorbox}If the problem is changed into "Find the constant function", what will the solution be? Thanks.\end{tcolorbox}
I dont understand : trivially any positive constant function is a solution to this problem.


\end{solution}



\begin{solution}[by \href{https://artofproblemsolving.com/community/user/328399}{mqcase1004}]
	\begin{tcolorbox}[quote=mqcase1004]If the problem is changed into "Find the constant function", what will the solution be? Thanks.\end{tcolorbox}
I dont understand : trivially any positive constant function is a solution to this problem.\end{tcolorbox}

Oh Sorry for my English. I'm mean the continuous function not the constant one. So sorry
\end{solution}



\begin{solution}[by \href{https://artofproblemsolving.com/community/user/29428}{pco}]
	\begin{tcolorbox} I'm mean the continuous function not the constant one. So sorry\end{tcolorbox}
Just add to my previous solution the two constraints :
1) $g(x)$ is continuous over $(0,1]$ (in order to have continuity of $f(x)$ over $(0,1)$ and $(1,+\infty)$

2) $g(1)=g(\frac 23)$ in order to have continuity of $f(x)$ at $1$)



\end{solution}



\begin{solution}[by \href{https://artofproblemsolving.com/community/user/248946}{techguy2}]
	Is there an ongoing problem currently?
\end{solution}



\begin{solution}[by \href{https://artofproblemsolving.com/community/user/328608}{Thinking666}]
	@pco If I am not mistaken you are wrong on the solution of problem $257$. In the second part you say that $f(1) = 2$ and $f(2) = 1$ works but this is wrong because then $f(f(1)) + f(2) = 2 \ne 3$. 

Here is a new problem : 
Let $f:\mathbf{R} \rightarrow \mathbf{R}$. Find all functions $f$ such that 
$$f(f(x)+2y) = 10x + f(f(y)-3x)$$
\end{solution}



\begin{solution}[by \href{https://artofproblemsolving.com/community/user/345905}{TLP.39}]
	\begin{tcolorbox}
Here is a new problem : 
Let $f:\mathbf{R} \rightarrow \mathbf{R}$. Find all functions $f$ such that 
$$f(f(x)+2y) = 10x + f(f(y)-3x)$$\end{tcolorbox}
Mmmmm........
[hide=solution]Let $P(x,y)$ be the assertion.
Clearly,$f(x)+2y=f(y)-3x\implies x=0--------------------(1)$
Now,$P\left(x,\frac{a-f(x)}{2}\right)\implies f(a)-10x=f\left(f\left(\frac{a-f(x)}{2}\right)-3x\right)$ which implies that $f(x)$ is surjective.
Let call the above equation $Q(a,x)$
If $f(a)=f(b)\exists a\ne b$
Then $Q(2a+f(x),x),Q(2b+f(x),x)\implies f(2a+f(x))=f(2b+f(x))$,but since $f(x)$ is surjective,this implies that $f(x)$ is periodic.
Let $f(x)=f(x+c)\forall x$ for some constant $c\ne 0$
Then $(1),f\left(\frac{2c}{5}\right)+2\left(-\frac{3c}{5}\right)=f\left(-\frac{3c}{5}\right)-3\left(\frac{2c}{5}\right)\implies \frac{2c}{5}=0\implies c=0$,contradiction.Hence $f(x)$ is bijective.
Finally,$P(0,x)\implies f(0)+2x=f(x)\implies\boxed{f(x)=2x+c}\text{ for some constant } c$ which clearly satisfied the equation.[\/hide]
I don't have problem,so just review the old unsolved one :) .
\begin{tcolorbox}[quote=Amir Hossein]I'm posting next problem.

\begin{bolded}Problem  43\end{bolded} :
Let $ f$ be a real function defined on the positive half-axis for which $ f(xy)=xf(y)+yf(x)$ and $ f(x+1) \leq f(x)$ hold for every positive $ x$ and $ y$. Show that if $ f(1\/2)=1\/2$, then \[ f(x)+f(1-x) \geq -x \log_2 x -(1-x) \log_2 (1-x)\] for every $ x\in (0,1)$.\end{tcolorbox}

the problem not solved\end{tcolorbox}



\end{solution}



\begin{solution}[by \href{https://artofproblemsolving.com/community/user/285709}{don2001}]
	Any solution?
\end{solution}
*******************************************************************************
-------------------------------------------------------------------------------

\begin{problem}[Posted by \href{https://artofproblemsolving.com/community/user/10045}{socrates}]
	:)

3. $\forall x, y \in \mathbb{R} \ , \  \  f(x+xf(y))=f(x)+xf(y)$

http://www.artofproblemsolving.com/Forum/viewtopic.php?f=36&t=461479&
	\flushright \href{https://artofproblemsolving.com/community/c6h477809}{(Link to AoPS)}
\end{problem}



\begin{solution}[by \href{https://artofproblemsolving.com/community/user/122611}{oty}]
	Edit: I had wrong  :P
\end{solution}



\begin{solution}[by \href{https://artofproblemsolving.com/community/user/106080}{a123}]
	\begin{tcolorbox}$P(1,x)$ imply : $f(1+f(x))=f(x)+xf(1)$(1),  .\end{tcolorbox}
Not true.It will be
$f(1+f(x))=f(1)+f(x)$
\end{solution}



\begin{solution}[by \href{https://artofproblemsolving.com/community/user/122611}{oty}]
	\begin{tcolorbox}[quote="oty"]$P(1,x)$ imply : $f(1+f(x))=f(x)+xf(1)$(1),  .\end{tcolorbox}
Not true.It will be
$f(1+f(x))=f(1)+f(x)$\end{tcolorbox}
yes you are right , thank you .
\end{solution}



\begin{solution}[by \href{https://artofproblemsolving.com/community/user/10045}{socrates}]
	Any solution for this problem? If you have a solution assuming continuity, monotonicity etc don't hesitate to post it! :) 

Can we also solve the similar problem?
$ \forall x, y \in \mathbb{R}^+ \ , \  \  f(x+xf(y))=f(x)+xf(y)$
\end{solution}



\begin{solution}[by \href{https://artofproblemsolving.com/community/user/29428}{pco}]
	\begin{tcolorbox}3. $\forall x, y \in \mathbb{R} \ , \  \  f(x+xf(y))=f(x)+xf(y)$\end{tcolorbox}
\begin{tcolorbox}Any solution for this problem? If you have a solution assuming continuity, monotonicity etc don't hesitate to post it! :) \end{tcolorbox}
With continuity constraint, it's quite easy to establish $f(x)=0$ $\forall x$ or $f(x)=x+a$ $\forall x$

But since this is a real olympiad exercise you got in a real exam (isn't it ?) and since this real exercise did not contain continuity constraint, there is surely an olympiad level solution not demanding continuity.
\end{solution}



\begin{solution}[by \href{https://artofproblemsolving.com/community/user/10045}{socrates}]
	In fact, I took the problem from http://www.artofproblemsolving.com/Forum/viewtopic.php?f=36&t=461479& 
I haven't seen the problem anywhere else... I asked for a solution using continuity, monotonicity just in case there were missing...

So, could you please post your solution assuming continuity? :)
\end{solution}



\begin{solution}[by \href{https://artofproblemsolving.com/community/user/29428}{pco}]
	\begin{tcolorbox}3. $\forall x, y \in \mathbb{R} \ , \  \  f(x+xf(y))=f(x)+xf(y)$
[color=#FF0000]assuming continuity[\/color]

\end{tcolorbox}
Let $g(x)=f(x)+1-x$ so that equation becomes $g(x(g(y)+y))=g(x)$

If $g(y)+y=1$ $\forall y$, we get $\boxed{f(x)=0}$ $\forall x$ which indeed is a solution
If $g(y)+y=0$ $\forall y$, we get $f(x)=-1$ $\forall x$ which is not a solution
If $g(y)+y=-1$ $\forall y$, we get $f(x)=-2$ $\forall x$ which is not a solution
Else, continuity implies $\exists u,v$ such that $g(u)+u=v\notin\{-1,0,1\}$

Then we get $g(vx)=g(x)$ and so $g(v^nx)=g(x)$ $\forall x,\forall n\in\mathbb Z$
Setting then $n\to+\infty$ or $n\to-\infty$ and using again continuity, we get $g(x)=g(0)$ and so $\boxed{f(x)=x+a}$ $\forall x$, which indeed is a solution, whatever is $a\in\mathbb R$
\end{solution}
*******************************************************************************
-------------------------------------------------------------------------------

\begin{problem}[Posted by \href{https://artofproblemsolving.com/community/user/109704}{dien9c}]
	Find continuous function $f:\mathbb{R} \to \mathbb{R}$ such that
\[a f(f(x)) = b f(x) + cx\]
	\flushright \href{https://artofproblemsolving.com/community/c6h485367}{(Link to AoPS)}
\end{problem}



\begin{solution}[by \href{https://artofproblemsolving.com/community/user/29428}{pco}]
	\begin{tcolorbox}Find continuous function $f:\mathbb{R} \to \mathbb{R}$ such that
\[a f(f(x)) = b f(x) + cx\]\end{tcolorbox}
Is this the exact exercice you got in your olympiad contest or training session ? (it's a little bit different that the one you asked me in pm).

Without any restriction on $a,b,c$ this implies huge discussion depending on values of $a,b,c$ and some strange solutions.

For, exemple, with $(a,b,c)=(1,1,0)$ we get a continuous solution $f(x)=sign(x)\min(|x|,\sin\frac{\pi |x|}2)$ and infinitely many other functions ...

So I'm surprised that such a general problem occured in a real olympiad contest or training session.
\end{solution}



\begin{solution}[by \href{https://artofproblemsolving.com/community/user/109704}{dien9c}]
	\begin{tcolorbox}[quote="dien9c"]Find continuous function $f:\mathbb{R} \to \mathbb{R}$ such that
\[a f(f(x)) = b f(x) + cx\]\end{tcolorbox}
Is this the exact exercice you got in your olympiad contest or training session ? (it's a little bit different that the one you asked me in pm).

Without any restriction on $a,b,c$ this implies huge discussion depending on values of $a,b,c$ and some strange solutions.

For, exemple, with $(a,b,c)=(1,1,0)$ we get a continuous solution $f(x)=sign(x)\min(|x|,\sin\frac{\pi |x|}2)$ and infinitely many other functions ...

So I'm surprised that such a general problem occured in a real olympiad contest or training session.\end{tcolorbox}
I saw the two private function
1. \[f(f(x))=f(x)+x\] (math contest)
2. \[f(f(x))=f(x)+2x\] (Belarusia 1998)
and I asked the general above
\end{solution}



\begin{solution}[by \href{https://artofproblemsolving.com/community/user/29428}{pco}]
	\begin{tcolorbox}Find continuous function $f:\mathbb{R} \to \mathbb{R}$ such that
\[a f(f(x)) = b f(x) + cx\]\end{tcolorbox}
Here is a rather quick answer to your general question.

(I hope that if you encounter a problem "solve in $\mathbb Z$ equation $x+17=3$" and another problem "solve in $\mathbb R$ equation $x^2+2x+1=0$", you will not ask in the forum, claiming that it is a real olympiad exercise : "solve in $\mathbb C$ equation $\sum_{k=0}^na_kx^k$ where $a_i\in\mathbb C$" just in order to get a general solution to all your polynomial equation exercises :( )


\begin{bolded}Some personal comments\end{underlined}\end{bolded} :
I did not check all the typos (too tiring) and I hope you'll excuse them
I dont claim that all my proofs are the shortest.
Maybe some subcases could have been merged.
Subcase 7.2 is not finished. It seems to me that it is one of the most complex and it has infinitely many solutions.
Your two examples in previous posts both are in subcase 7.1.1
Some subcases was full independant questions in this forum (for example $f(f(x))=x$ is subcase 5.3 and $f(x)+f^{-1}(x)=2x$ is subcase 7.3.2)

I hope you'll read this post line per line.



\begin{bolded}Case 1\end{underlined}\end{bolded} : If $a=b=c=0$ : any continuous function is solution

\begin{bolded}Case 2\end{underlined}\end{bolded} : If $a=b=0$ and $c\ne 0$ : no solution

\begin{bolded}Case 3\end{underlined} \end{bolded}: If $a=0$ and $b\ne 0$ : unique solution is $f(x)=-\frac cbx$ $\forall x$

\begin{bolded}Case 4\end{underlined}\end{bolded} : If $a\ne 0$ and $b=c=0$, equation is $f(f(x))=0$ and so $f(x)=0$ $\forall x\in f(\mathbb R)$ and, since continuous, $f(\mathbb R)$ is an interval and so : 

\begin{bolded}Subcase 4.1\end{bolded} : $f(\mathbb R)=\mathbb R$ and the unique solution $f(x)=0$ $\forall x$

\begin{bolded}Subcase 4.2\end{bolded} : $f(\mathbb R)=[a,+\infty)$. with $a\le 0$ (since obviously $0\inf(\mathbb R)$
Then we get infinitely many solutions :
Let $a\le 0$
Let $h(x)$ any continuous surjection from $(-\infty,a]to[a,+\infty)$ such that $h(a)=0$
Then $f(x)=h(x)$ $\forall x<a$ and $f(x)=0$ $\forall x\ge a$

\begin{bolded}Subcase 4.3\end{bolded} : $f(\mathbb R)=(a,+\infty)$. with $a< 0$ (since obviously $0\inf(\mathbb R)$
Then we get infinitely many solutions :
Let $a<0$
Let $h(x)$ any continuous surjection from $(-\infty,a]to(a,+\infty)$ such that $h(a)=0$
Then $f(x)=h(x)$ $\forall x<a$ and $f(x)=0$ $\forall x\ge a$

\begin{bolded}Subcase 4.4 \end{bolded}: $f(\mathbb R)=(-\infty,a]$. with $a\ge 0$ (since obviously $0\inf(\mathbb R)$
Then we get infinitely many solutions :
Let $a\ge 0$
Let $h(x)$ any continuous surjection from $[a,+\infty)to(-\infty,a]$ such that $h(a)=0$
Then $f(x)=h(x)$ $\forall x>a$ and $f(x)=0$ $\forall x\le a$

\begin{bolded}Subcase 4.5\end{bolded} : $f(\mathbb R)=(-\infty,a)$. with $a>0$ (since obviously $0\inf(\mathbb R)$
Then we get infinitely many solutions :
Let $a>0$
Let $h(x)$ any continuous surjection from $[a,+\infty)to(-\infty,a)$ such that $h(a)=0$
Then $f(x)=h(x)$ $\forall x>a$ and $f(x)=0$ $\forall x\le a$

\begin{bolded}Subcase 4.6\end{bolded} : $f(\mathbb R)=[a,b]$ with $a\le 0\le b$ (since obviously $0\inf(\mathbb R)$
Then we get infinitely many solutions :
Let $a\le 0\le b$
Let $h(x)$ any continuous function from $\mathbb R\to\mathbb R$ such that $h(\mathbb R\setminus(a,b))=[a,b]$ and $h(a)=h(b)=0$.
Then $f(x)=h(x)$ $\forall x\notin[a,b]$ and $f(x)=0$ $\forall x\in[a,b]$

\begin{bolded}Subcase 4.7\end{bolded} : $f(\mathbb R)=[a,b)$ with $a\le 0< b$ (since obviously $0\inf(\mathbb R)$
Then we get infinitely many solutions :
Let $a\le 0< b$
Let $h(x)$ any continuous function from $\mathbb R\to\mathbb R$ such that $hf(\mathbb R\setminus(a,b))=[a,b)$ and $h(a)=h(b)=0$.
Then $f(x)=h(x)$ $\forall x\notin[a,b]$ and $f(x)=0$ $\forall x\in[a,b]$

\begin{bolded}Subcase 4.8\end{bolded} : $f(\mathbb R)=(a,b]$ with $a< 0\le b$ (since obviously $0\inf(\mathbb R)$
Then we get infinitely many solutions :
Let $a< 0\le b$
Let $h(x)$ any continuous function from $\mathbb R\to\mathbb R$ such that $h(\mathbb R\setminus(a,b))=(a,b]$ and $h(a)=h(b)=0$.
Then $f(x)=h(x)$ $\forall x\notin[a,b]$ and $f(x)=0$ $\forall x\in[a,b]$

\begin{bolded}Subcase 4.9\end{bolded} : $f(\mathbb R)=(a,b)$ with $a< 0\< b$ (since obviously $0\inf(\mathbb R)$
Then we get infinitely many solutions :
Let $a< 0< b$
Let $h(x)$ any continuous function from $\mathbb R\to\mathbb R$ such that $h(\mathbb R\setminus(a,b))=(a,b)$ and $h(a)=h(b)=0$.
Then $f(x)=h(x)$ $\forall x\notin[a,b]$ and $f(x)=0$ $\forall x\in[a,b]$

\begin{bolded}Case 5\end{underlined}\end{bolded} : If $a\ne 0$ and $b=0$ and $c\ne 0$, equation is $f(f(x))=tx$ with $t=\frac ca\ne 0$
So $f(x)$ is bijective and so, since continuous, monotonous.
So $f(f(x))$ is an increasing function.

\begin{bolded}Subcase 5.1\end{bolded} : If $t<0$ ($\iff ac<0$) then  no solution since $LHS$ is increasing while RHS is decreasing

\begin{bolded}Subcase 5.2\end{bolded} : if $1>t>0$ ($\iff$ $a>c>0$ or $a<c<0$), this is a classical equation with infinitely many solutions which may be built piece par piece.

5.2.1) increasing solutions\end{underlined} :
$\forall x>0$ :
Let $a\in(t,1)$
Let $h_1(x)$ any continuous increasing bijection from $[a,1]\to[t,a]$
Define $f(x)$ as :
$\forall x\in(a,1]$ : $f(x)=h_1(x)$
$\forall x\in(t,a]$ : $f(x)=th_1^{[-1]}(x)$
$\forall x\in(0,t]\cup(1,+\infty)$ : $f(x)=t^{\lfloor\log_t x\rfloor}f(xt^{-\lfloor\log_t x\rfloor})$

$f(0)=0$

$\forall x<0$
Let $b\in(-1,-t)$
Let $h_2(x)$ any continuous increasing bijection from $[-1,b]\to[b,-t]$
Define $f(x)$ as :
$\forall x\in[-1,b)$ : $f(x)=h_2(x)$
$\forall x\in[b,-t)$ : $f(x)=th_2^{[-1]}(x)$
$\forall x\in(-\infty,-1)\cup[-t,0)$ : $f(x)=t^{\lfloor\log_t -x\rfloor}f(xt^{-\lfloor\log_t -x\rfloor})$

5.2.2) decreasing solutions\end{underlined} :
Let $a<0$
Let $h(x)$ be any continuous decreasing bijection from $[t,1]\to[a,ta]$
$\forall x\ge 0$ : $f(x)=t^{\lfloor\log_t x\rfloor}h(xt^{-\lfloor\log_t x\rfloor})$
$\forall x<0$ : $f(x)=th^{[-1]}(x)$

\begin{bolded}Subcase 5.3\end{bolded} : If $t=1$ ($\iff a=c$), then equation is $f(f(x))=x$ and is very classical :

5.3.1) increasing solutions\end{underlined}
It's well known that equation $f(f(x))=x$ has a unique continuous increasing solution $f(x)=x$ $\forall x$

5.3.2) decreasing solutions\end{underlined}
A general form is :
Let $h(x)$ be any continuous decreasing bijection from $[0,+\infty)\to(-\infty,0]$
$\forall x\ge 0$ : $f(x)=h(x)$
$\forall x<0$ : $f(x)=h^{[-1]}(x)$

\begin{bolded}Subcase 5.4\end{bolded} : if $t>1$ ($\iff$ $c>a>0$ or $c<a<0$), this is a classical equation with infinitely many solutions which may be built piece par piece.

5.4.1) increasing solutions\end{underlined} :
$\forall x>0$ :
Let $a\in(1,t)$
Let $h_1(x)$ any continuous increasing bijection from $[1,a]\to[a,t]$
Define $f(x)$ as :
$\forall x\in[1,a)$ : $f(x)=h_1(x)$
$\forall x\in[a,t)$ : $f(x)=th_1^{[-1]}(x)$
$\forall x\in(0,1)\cup[t,+\infty)$ : $f(x)=t^{\lfloor\log_t x\rfloor}f(xt^{-\lfloor\log_t x\rfloor})$

$f(0)=0$

$\forall x<0$
Let $b\in(-t,-1)$
Let $h_2(x)$ any continuous increasing bijection from $[b,-1]\to[-t,b]$
Define $f(x)$ as :
$\forall x\in(b,-1]$ : $f(x)=h_2(x)$
$\forall x\in(-t,b]$ : $f(x)=th_2^{[-1]}(x)$
$\forall x\in(-\infty,-t]\cup(-1,0)$ : $f(x)=t^{\lfloor\log_t -x\rfloor}f(xt^{-\lfloor\log_t -x\rfloor})$

5.4.2) decreasing solutions\end{underlined} :
Let $a<0$
Let $h(x)$ be any continuous decreasing bijection from $[1,t]\to[ta,a]$
$\forall x\ge 0$ : $f(x)=t^{\lfloor\log_t x\rfloor}h(xt^{-\lfloor\log_t x\rfloor})$
$\forall x<0$ : $f(x)=th^{[-1]}(x)$

\begin{bolded}Case 6\end{underlined}\end{bolded} : If $a\ne 0$ and $b=\ne 0$ and $c=0$, equation is $f(f(x))=tf(x)$ with $t=\frac ba\ne 0$
So $f(x)=tx$ $\forall x\in f(\mathbb R)$ 
Since continuous, $f(\mathbb R)$ is an interval and so :

\begin{bolded}Subcase 6.1\end{bolded} : if $f(\mathbb R)=\mathbb R$, we get $f(x)=tx$ $\forall x$ which indeed is a solution

\begin{bolded}Subcase 6.2\end{bolded} : if $f(\mathbb R)=[a,+\infty)$ 
This means $ta\ge a$ (since $f(a)=ta\in f(\mathbb R)$) and the solutions :

\begin{bolded}Subcase 6.2.1\end{bolded} : if $t>1$, then $a\ge 0$ and :
Let $a\ge 0$
Let $h(x)$ be any continuous function from $(-\infty,a]\to[a,+\infty)$ such that $h(a)=ta$ and $[a,ta]\subseteq h((-\infty,a])$ then 
$f(x)=h(x)$ $\forall x\le a$
$f(x)=tx$ $\forall x>a$

\begin{bolded}Subcase 6.2.2\end{bolded} : if $t=1$, then no constraint on $a$ and :
Let $a \in\mathbb R$
Let $h(x)$ be any continuous function from $(-\infty,a]\to[a,+\infty)$ such that $h(a)=a$. Then :
$f(x)=h(x)$ $\forall x\le a$
$f(x)=x$ $\forall x>a$

\begin{bolded}Subcase 6.2.3\end{bolded} : if $1>t>0$, then $a\le 0$ and :
Let $a\le 0$
Let $h(x)$ be any continuous function from $(-\infty,a]\to[a,+\infty)$ such that $h(a)=ta$ and $[a,ta]\subseteq h((-\infty,a])$ then 
$f(x)=h(x)$ $\forall x\le a$
$f(x)=tx$ $\forall x>a$

\begin{bolded}Subcase 6.2.4\end{bolded} : if $t<0$
Choosing some $f(x)$ positive and great enough, we get that $tf(x)<a$ and so can not be in $f(\mathbb R)$
So no solution

\begin{bolded}Subcase 6.3\end{bolded} : if $f(mathbb R)=(a,+\infty)$
We get $f(x)=tx$ $\forall x>a$ and so (continuity) $f(a)=ta$ and so $ta>a$ and the solutions :

\begin{bolded}Subcase 6.3.1\end{bolded} : if $t>1$, then $a>0$ and :
Let $a>0$
Let $h(x)$ be any continuous function from $(-\infty,a]\to(a,+\infty)$ such that $h(a)=ta$ and $(a,ta]\subseteq h((-\infty,a])$ then 
$f(x)=h(x)$ $\forall x\le a$
$f(x)=tx$ $\forall x>a$

\begin{bolded}Subcase 6.3.2\end{bolded} : if $t=1$, then $ta>a$ is impossible and so no solution

\begin{bolded}Subcase 6.3.3\end{bolded} : if $1>t>0$, then $a< 0$ and :
Let $a<0$
Let $h(x)$ be any continuous function from $(-\infty,a]\to(a,+\infty)$ such that $h(a)=ta$ and $(a,ta]\subseteq h((-\infty,a])$ then 
$f(x)=h(x)$ $\forall x\le a$
$f(x)=tx$ $\forall x>a$

\begin{bolded}Subcase 6.3.4\end{bolded} : if $t<0$
Choosing some $f(x)$ positive and great enough, we get that $tf(x)\le a$ and so can not be in $f(\mathbb R)$
So no solution

\begin{bolded}Subcase 6.4\end{bolded} : if $f(\mathbb R)=(-\infty,a]$
This means $ta\le a$ (since $f(a)=ta\in f(\mathbb R)$) and the solutions :

\begin{bolded}Subcase 6.4.1\end{bolded}: if $t>1$, then $a\le 0$ and :
Let $a\le 0$ 
Let $h(x)$ any continuous function from $[a,+\infty)\to(-\infty,a]$ such that $h(a)=ta$ and $[ta,a]\subseteq h([a,+\infty))$. Then :
$f(x)=tx$ $\forall x\le a$
$f(x)=h(x)$ $\forall x>a$

\begin{bolded}Subcase 6.4.2\end{bolded}: if $t=1$, then no constraint on $a$ and :
Let $a\in\mathbb R$ 
Let $h(x)$ any continuous function from $[a,+\infty)\to(-\infty,a]$ such that $h(a)=a$. Then :
$f(x)=x$ $\forall x\le a$
$f(x)=h(x)$ $\forall x>a$

\begin{bolded}Subcase 6.4.3\end{bolded} : if $1>t>0$, then $a\ge 0$ and :
Let $a\ge 0$ 
Let $h(x)$ any continuous function from $[a,+\infty)\to(-\infty,a]$ such that $h(a)=ta$ and $[ta,a]\subseteq h([a,+\infty))$. Then :
$f(x)=tx$ $\forall x\le a$
$f(x)=h(x)$ $\forall x>a$

\begin{bolded}Subcase 6.4.4\end{bolded} : if $t<0$
Choosing some $f(x)$ negative and small enough, we get that $tf(x)> a$ and so can not be in $f(\mathbb R)$
So no solution

\begin{bolded}Subcase 6.5\end{bolded} : if $f(\mathbb R)=(-\infty,a)$
We get $f(x)=tx$ $\forall x<a$ and so (continuity) $f(a)=ta$ and so $ta<a$ and the solutions :

\begin{bolded}Subcase 6.5.1\end{bolded}: if $t>1$, then $a< 0$ and :
Let $a<0$ 
Let $h(x)$ any continuous function from $[a,+\infty)\to(-\infty,a)$ such that $h(a)=ta$ and $[ta,a)\subseteq h([a,+\infty))$. Then :
$f(x)=tx$ $\forall x\le a$
$f(x)=h(x)$ $\forall x>a$

\begin{bolded}Subcase 6.5.2\end{bolded}: if $t=1$, then $ta<a$ is impossible and so no solution

\begin{bolded}Subcase 6.5.3\end{bolded} : if $1>t>0$, then $a>0$ and :
Let $a>0$ 
Let $h(x)$ any continuous function from $[a,+\infty)\to(-\infty,a)$ such that $h(a)=ta$ and $[ta,a)\subseteq h([a,+\infty))$. Then :
$f(x)=tx$ $\forall x\le a$
$f(x)=h(x)$ $\forall x>a$

\begin{bolded}Subcase 6.5.4\end{bolded} : if $t<0$
Choosing some $f(x)$ negative and small enough, we get that $tf(x)\ge a$ and so can not be in $f(\mathbb R)$
So no solution

\begin{bolded}Subcase 6.6 \end{bolded}: if $f(\mathbb R)=[a,b]$
Then $a\le ta\le b$ and $a\le tb\le b$

\begin{bolded}Subcase 6.6.1\end{bolded} : if $t>1$, then $a\le ta\le b$ and $a\le tb\le b$ imply $a=b=0$ and so $f(x)=0$ $\forall x$, which indeed is a solution

\begin{bolded}Subcase 6.6.2\end{bolded} : if $t=1$, we get the solutions :
Let $a\le b$
Let $h_1(x)$ any continuous function from $(-\infty,a]\to[a,b]$ such that $h_1(a)=a$
Let $h_2(x)$ any continuous function from $[b,+\infty]\to[a,b]$ such that $h_2(b)=b$
Then :
$\forall x<a$ : $f(x)=h_1(x)$
$\forall x\in[a,b]$ : $f(x)=x$
$\forall x>b$ : $f(x)=h_2(x)$

\begin{bolded}Subcase 6.6.3\end{bolded} : if $1>t>0$, then $a\le ta\le b$ and $a\le tb\le b$ imply $a\le 0\le b$ and :
Let $a\le 0\le b$
Let $h(x)$ be any continuous function from $\mathbb R\to[a,b]$ such that :
$h(a)=ta$
$h(b)=tb$
$[a,ta]\cup[tb,b]\subseteq h((-\infty,a]\cup[b,+\infty))$
Then :
$\forall x\in[a,b]$ : $f(x)=tx$
$\forall x\notin[a,b]$ : $f(x)=h(x)$

\begin{bolded}Subcase 6.6.4\end{bolded} : if $-1\le t<0$, then $a\le ta\le b$ and $a\le tb\le b$ imply $a\le 0$ and $\frac at\ge b\ge ta$ and :
Let $a\le 0$ and $b\in[ta,\frac at]$
Let $h(x)$ be any continuous function from $\mathbb R\to[a,b]$ such that :
$h(a)=ta$
$h(b)=tb$
$[a,tb]\cup[ta,b]\subseteq h((-\infty,a]\cup[b,+\infty))$
Then :
$\forall x\in[a,b]$ : $f(x)=tx$
$\forall x\notin[a,b]$ : $f(x)=h(x)$

\begin{bolded}Subcase 6.6.5 \end{bolded}: if $t<-1$, then $a\le ta\le b$ and $a\le tb\le b$ imply $a=b=0$ and the solution $f(x)=0$ $\forall x$ which indeed is a solution.

\begin{bolded}Subcase 6.7\end{bolded} : if $f(\mathbb R)=[a,b)$
Then $f(x)=tx$ $\forall x\in [a,b)$ and so $tx\in[a,b)$ $\forall x\in[a,b)$
And so continuity implies $a\le ta< b$ and $a\le tb< b$

\begin{bolded}Subcase 6.7.1\end{bolded} : if $t\ge 1$, then $a\le ta< b$ and $a\le tb< b$ imply no solution

\begin{bolded}Subcase 6.7.2\end{bolded} : if $1>t>0$, then $a\le ta< b$ and $a\le tb< b$ imply $a\le 0< b$ and :
Let $a\le 0< b$
Let $h(x)$ be any continuous function from $\mathbb R\to[a,b)$ such that :
$h(a)=ta$
$h(b)=tb$
$[a,ta]\cup[tb,b)\subseteq h((-\infty,a]\cup[b,+\infty))$
Then :
$\forall x\in[a,b]$ : $f(x)=tx$
$\forall x\notin[a,b]$ : $f(x)=h(x)$

\begin{bolded}Subcase 6.7.3\end{bolded} : if $-1< t<0$, then $a\le ta< b$ and $a\le tb< b$ imply $a<0$ and $\frac at\ge b> ta$ and :
Let $a<0$ and $b\in(ta,\frac at]$
Let $h(x)$ be any continuous function from $\mathbb R\to[a,b)$ such that :
$h(a)=ta$
$h(b)=tb$
$[a,tb]\cup[ta,b)\subseteq h((-\infty,a]\cup[b,+\infty))$
Then :
$\forall x\in[a,b]$ : $f(x)=tx$
$\forall x\notin[a,b]$ : $f(x)=h(x)$

\begin{bolded}Subcase 6.7.4\end{bolded} : if $t\le -1$, then $a\le ta< b$ and $a\le tb< b$ imply no solution

\begin{bolded}Subcase 6.8\end{bolded} : if $f(\mathbb R)=(a,b]$
Then $f(x)=tx$ $\forall x\in (a,b]$ and so $tx\in(a,b]$ $\forall x\in(a,b]$
And so continuity implies $a< ta\le b$ and $a< tb\le b$

\begin{bolded}Subcase 6.8.1\end{bolded} : if $t\ge 1$, then $a< ta\le b$ and $a< tb\le b$ imply no solution

\begin{bolded}Subcase 6.8.2\end{bolded} : if $1>t>0$, then $a< ta\le b$ and $a< tb\le b$ imply $a<0\le b$ and :
Let $a< 0\le b$
Let $h(x)$ be any continuous function from $\mathbb R\to[a,b]$ such that :
$h(a)=ta$
$h(b)=tb$
$(a,ta]\cup[tb,b]\subseteq h((-\infty,a]\cup[b,+\infty))$
Then :
$\forall x\in[a,b]$ : $f(x)=tx$
$\forall x\notin[a,b]$ : $f(x)=h(x)$

\begin{bolded}Subcase 6.8.3\end{bolded} : if $-1<t<0$, then $a< ta\le b$ and $a< tb\le b$ imply $a< 0$ and $\frac at> b\ge ta$ and :
Let $a< 0$ and $b\in[ta,\frac at)$
Let $h(x)$ be any continuous function from $\mathbb R\to(a,b]$ such that :
$h(a)=ta$
$h(b)=tb$
$(a,tb]\cup[ta,b]\subseteq h((-\infty,a]\cup[b,+\infty))$
Then :
$\forall x\in[a,b]$ : $f(x)=tx$
$\forall x\notin[a,b]$ : $f(x)=h(x)$

\begin{bolded}Subcase 6.8.4\end{bolded} : if $t\le -1$, then $a< ta\le b$ and $a< tb\le b$ imply no solution

\begin{bolded}Subcase 6.9\end{bolded} : if $f(\mathbb R)=(a,b)$
Then $f(x)=tx$ $\forall x\in (a,b)$ and so $tx\in(a,b)$ $\forall x\in(a,b)$
And so continuity implies $a< ta< b$ and $a< tb< b$

\begin{bolded}Subcase 6.9.1\end{bolded} : if $t\ge 1$, then $a< ta<b$ and $a< tb< b$ imply no solution

\begin{bolded}Subcase 6.9.2\end{bolded} : if $1>t>0$, then $a< ta<b$ and $a< tb< b$ imply $a< 0< b$ and :
Let $a< 0< b$
Let $h(x)$ be any continuous function from $\mathbb R\to(a,b)$ such that :
$h(a)=ta$
$h(b)=tb$
$(a,ta]\cup[tb,b)\subseteq h((-\infty,a]\cup[b,+\infty))$
Then :
$\forall x\in[a,b]$ : $f(x)=tx$
$\forall x\notin[a,b]$ : $f(x)=h(x)$

\begin{bolded}Subcase 6.9.3\end{bolded} : if $-1< t<0$, then $a< ta<b$ and $a< tb< b$ imply $a< 0$ and $\frac at> b> ta$ and :
Let $a<0$ and $b\in(ta,\frac at)$
Let $h(x)$ be any continuous function from $\mathbb R\to(a,b)$ such that :
$h(a)=ta$
$h(b)=tb$
$(a,tb]\cup[ta,b)\subseteq h((-\infty,a]\cup[b,+\infty))$
Then :
$\forall x\in[a,b]$ : $f(x)=tx$
$\forall x\notin[a,b]$ : $f(x)=h(x)$

\begin{bolded}Subcase 6.9.4\end{bolded} : if $t\le -1$, then $a< ta<b$ and $a< tb< b$ imply no solution


\begin{bolded}Case  7\end{underlined}\end{bolded} : $a,b,c\ne 0$ and the equation is $f(f(x))=uf(x)+vx$ with $u,v\ne 0$

$f(x)$ is injective and so, since continuous, monotonous.
If $\lim_{x\to+\infty}f(x)=L$, then setting $x\to+\infty$ in functional equation and using continuity gives contradiction
So $\lim_{x\to+\infty}f(x)=\pm\infty$
If $\lim_{x\to-\infty}f(x)=L$, then setting $x\to-\infty$ in functional equation and using continuity gives contradiction
So $\lim_{x\to-\infty}f(x)=\pm\infty$

And since $f(x)$ is monotonous, we get that $f(\mathbb R)=\mathbb R$ and so $f(x)$ is a bijection.

\begin{bolded}Subcase 7.1\end{bolded} : $v>0$
let $x\in\mathbb R$ and the sequence $a_n$ defined as :
$a_0=x$
$a_1=f(x)$
$a_{n+2}=ua_{n+1}+va_n$ $\forall n\ge 0$
Notice than $a_n=f^{[n]}(x)$
Since $v>0$, the characteristic equation $x^2-ux-v$ has two distinct real roots $r_1>0>r_2$

And so $f^{[n]}(x)=\frac{(f(x)-r_2x)r_1^n-(f(x)-r_1x)r_2^n}{r_1-r_2}$

Since $f(x)$ is a bijection, then $f^{-1}(x)$ exists and it is easy to show that the above expression is true $\forall n\in\mathbb Z$

\begin{bolded}Subcase 7.1.1\end{bolded} : $u+v\ne 1$ 
$u+v\ne 1$ $\implies$ $r_1\ne 1$
$u\ne 0$ $\implies$ $r_2\ne -r_1$

If the equation $f(x)=x$ has real root $r$, then functional equation implies $r=ur+vr$ and so $r=0$
So, if $x\ne 0$, $f^{k+1}(x)\ne f^k(x)$ $\forall k$

For $x\ne 0$, we can then define $\Delta_n(x)=\frac{f^{n+2}(x)-f^{n+1}(x)}{f^{n+1}(x)-f^{n}(x)}$
$\Delta_n(x)\ne 0$
Since $f(x)$ is monotonous, $\Delta_n(x)$ has a constant sign, for any values of $n\in\mathbb Z$ and $x\in\mathbb R^*$

$\Delta_n(x)=$ $\frac{(f(x)-r_2x)(r_1-1)r_1^{n+1}-(f(x)-r_1x)(r_2-1)r_2^{n+1}}{(f(x)-r_2x)(r_1-1)r_1^{n}-(f(x)-r_1x)(r_2-1)r_2^{n}}$

Let $x\ne 0$ such that $f(x)\ne r_1x$ and $f(x)\ne r_2x$
We know that $|r_1|\ne |r_2|$ and so :
If $|r_1|>|r_2|$ : $\lim_{n\to +\infty}\Delta_n(x)=r_1$ and $\lim_{n\to -\infty}\Delta_n(x)=r_2$
If $|r_1|<|r_2|$ : $\lim_{n\to +\infty}\Delta_n(x)=r_2$ and $\lim_{n\to -\infty}\Delta_n(x)=r_1$
In both cases $\Delta_n(x)$ does not have a constant sign.

So $\forall x\ne 0$, either $f(x)=r_1x$, either $f(x)=r_2x$ and continuity + monotonicity imply :
either $f(x)=r_1x$ $\forall x$ which indeed is a solution
either $f(x)=r_2x$ $\forall x$ which indeed is a solution

Hence two solutons in this subcase

\begin{bolded}Subcase 7.1.2\end{bolded} : $u+v=1$
So $r_1=1$ and $r_2=u-1=-v<0$
$f(x)=x$ is a solution.
Let us from now in this subcase look for other solutions (different from $f(x)=x$ $\forall x$)

Let then $x$ such that $f(x)\ne x$ : $\frac{f(f(x))-f(x)}{f(x)-x}=-v<0$ and so $f(x)$ must be decreasing.

Since $f(x)$ is continuous and decreasing, then equation $f(x)=x$ has a unique root $r$.
Let then $g(x)=f(x+r)-r$ : it's easy to check that $g(x)$ is such that $g(g(x))=ug(x)+vx$ and $g(0)=0$

So WLOG consider from now that $f(0)=0$ and so $f(x)\ne 0$ $\forall x\ne 0$
Since $f(x)$ is decreasing, $f^{n}(x)$ is increasing for even $n$ and decreasing for odd $n$
And since $f^{n}(0)=0$, we can conclude

$\forall x\ne 0$ : $\frac{f^{n}(x)}x$ is nonzero and has same sign as $(-1)^n$

But $\frac{f^{n}(x)}x$ $=\frac{f(x)-r_2x-(f(x)-x)r_2^n}{(1-r_2)x}$

Since $r_2\ne \pm 1$, this quantity has limit  $\frac{f(x)-r_2x}{(1-r_2)x}$ when $n$ is set to $+\infty$ if $|r_2|<1$ or $-\infty$ if $|r_2|>1$
And so $f(x)=r_2x$, else this quantity can no longer has same sign as $(-1)^n$ when $n$ is set to the appropriate $\infty$
So $f(x)=-vx$ $\forall x$, which indeed is a solution

And so the solutions in this subcase :
$f(x)=x$ $\forall x$
$f(x)=c-vx$ $\forall x$

\begin{bolded}Subcase 7.2\end{bolded} : $v<0$ and $u^2+4v>0$
\begin{bolded}[color=red]This case need to be developped a bit more[\/color].\end{bolded}
We again obviously have the two solutions $f(x)=r_1x$ and $f(x)=r_2x$, both $r_1,r_2$ having the same sign
But there are in some cases a lot of other solutions.
Look for example at $f(f(x))=5f(x)-6x$ : it's possible to buid piece per piece infinitely many solutions such that for example $f(x)\in(2x,3x)$ $\forall x>0$


\begin{bolded}Subcase 7.3 \end{bolded}: $v<0$ and $u^2+4v=0$
\begin{bolded}Subcase 7.3.1\end{bolded} : $u\ne 2$
let $x\in\mathbb R$ and the sequence $a_n$ defined as :
$a_0=x$
$a_1=f(x)$
$a_{n+2}=ua_{n+1}+va_n$ $\forall n\ge 0$
Notice than $a_n=f^{[n]}(x)$
Since $u^2+4v=0$, the characteristic equation $x^2-ux-v$ has one double real root $r=\frac u2\notin\{0,1\}$

And so $f^{[n]}(x)=r^{n-1}(nf(x)-r(n-1)x)$

Since $f(x)$ is a bijection, then $f^{-1}(x)$ exists and it is easy to show that the above expression is true $\forall n\in\mathbb Z$

Note that $u^2+4v=0$ and $u\ne 2$ implu $u+v\ne 1$
If the equation $f(x)=x$ has real root $z$, then functional equation implies $z=uz+vz$ and so $z=0$
So, if $x\ne 0$, $f^{k+1}(x)\ne f^k(x)$ $\forall k$

For $x\ne 0$, we can then define $\Delta_n(x)=\frac{f^{n+2}(x)-f^{n+1}(x)}{f^{n+1}(x)-f^{n}(x)}$
$\Delta_n(x)\ne 0$
Since $f(x)$ is monotonous, $\Delta_n(x)$ has a constant sign, for any values of $n\in\mathbb Z$ and $x\in\mathbb R^*$

$\Delta_n(x)=$ $r\frac{(n+1)(r-1)(f(x)-rx)+r(f(x)-x)}{n(r-1)(f(x)-rx)+r(f(x)-x)}$

If $f(x)\ne rx$ for some $x\ne 0$, and since $r\ne 1$, we get $(r-1)(f(x)-rx)\ne 0$ and so $\exists n\in\mathbb Z$ such that the two parts of the fraction have opposite signs.
And since, for $n$ great enough, these two parts have same sign, we get that $\Delta_n(x)$ can not have a constant sign.
So $f(x)=rx$ $\forall x$, whch indeed is a solution.

\begin{bolded}Subcase 7.3.2\end{bolded} : $u=2$ and $v=-1$
Equation is $f(f(x))=2f(x)-x$

Writing $g(x)=f(x)-x$, the equation is $g(x+g(x))=g(x)$

$g(x)=0$ $\forall x$ is a solution and let us from now look for non all zero solutions.
If $g(x)$ is solution, then $-g(-x)$ is solution too and so Wlog say $g(p)=q>0$ for some $p$

Let $A=\{x\ge p$ such that $g(x)=g(p)=q\}$

From $g(x+g(x))=g(x)$, we get $g(x+ng(x))=g(x)$ and so $p+nq\in A$ $\forall n\in\mathbb N\cup\{0\}$

If $A$ is not dense in $[p,+\infty)$, let then $a,b\in A$ such that $p\le a<b$ and $(a,b)\cap A=\emptyset$. (existence of $a,b$ needs continuity of $g(x)$)

Let then $y\in(a,b)$. So $g(y)\ne q$ 
Consider then $y-a+n(g(y)-q)$ for $n\in\mathbb N$
Since $g(y)\ne q$, this quantity, for $n$ great enough is out of $[-q,+q]$ and so let $m>0$ such that $y-a+m(g(y)-q)\notin[-q,+q]$ and so such that $y+mg(y)\notin[a+(m-1)q,a+(m+1)q]$

Looking at the continuous function $h(x)=x+mg(x)$, we get :
$h(a)=a+mq\in(a+(m-1)q,a+(m+1)q)$
$h(y)=y+mg(y)\notin[a+(m-1)q,a+(m+1)q]$

So (using continuity of $h(x)$), $\exists z\in(a,y)$ such that $h(z)=a+(m-1)q$ or $h(z)=a+(m+1)q$
But then $g(h(z))=q$ and so $g(z+mg(z))=g(z)=q$, impossible since $z\in(a,b)$ and $(a,b)\cap A=\emptyset$.

So $A$ is dense in $[p,+\infty)$

Then continuity of $g(x)$ implies $g(x)=q$ $\forall x\ge p$.
Let then any $w<p$ : If $g(w)>0$, then $\exists n\in\mathbb N$ such that $w+ng(w)>p$ and so $g(w)=q$. So $\forall x<p$ : either $g(x)=q$, either $g(x)\le 0$ and continuity gives the conclusion $g(x)=q$ $\forall x$

So $g(x)=c$ and $f(x)=x+c$ which indeed is a solution.

\begin{bolded}Subcase 7.4 \end{bolded}: $v<0$ and $u^2+4v<0$
let $x\in\mathbb R$ and the sequence $a_n$ defined as :
$a_0=x$
$a_1=f(x)$
$a_{n+2}=ua_{n+1}+va_n$ $\forall n\ge 0$
Notice than $a_n=f^{[n]}(x)$
Since $u^2+4v<0$, the characteristic equation $x^2-ux-v$ has two distinct complex roots $re^{it}$ and $re^{-it}$ with $r>0$ and $t\in(0,\pi)$

And so $f^{[n]}(x)=r^{n-1}\frac{f(x)\sin nt-rx\sin(n-1)t}{\sin t}$

Since $f(x)$ is a bijection, then $f^{-1}(x)$ exists and it is easy to show that the above expression is true $\forall n\in\mathbb Z$

Note that $u^2+4v<0$ implies $u+v<1$
If the equation $f(x)=x$ has real root $r$, then functional equation implies $r=ur+vr$ and so $r=0$
So, if $x\ne 0$, $f^{k+1}(x)\ne f^k(x)$ $\forall k$

For $x\ne 0$, we can then define $\Delta_n(x)=\frac{f^{n+2}(x)-f^{n+1}(x)}{f^{n+1}(x)-f^{n}(x)}$
$\Delta_n(x)\ne 0$
Since $f(x)$ is monotonous, $\Delta_n(x)$ has a constant sign, for any values of $n\in\mathbb Z$ and $x\in\mathbb R^*$

$f^{k+1}(x)-f^{k}(x)$ $=r^{k}\frac{f(x)\sin (k+1)t-rx\sin kt}{\sin t}$ $-r^{k-1}\frac{f(x)\sin kt-rx\sin(k-1)t}{\sin t}$
$=\frac{r^{k-1}}{\sin t}$ $(rf(x)\sin (k+1)t-r^2x\sin kt-f(x)\sin kt+rx\sin(k-1)t)$

$=cr^k\sin(kt+d)$ for some $c\ne 0,d$ depending on $x$

So $\Delta_n(x)=$ $r\frac{\sin (n+1)t+d}{\sin nt+d}$

But this quantity can not be of constant sign $\forall n$ (since $t\in(0,\pi)$)

Hence no solution
\end{solution}



\begin{solution}[by \href{https://artofproblemsolving.com/community/user/109704}{dien9c}]
	Thank you very much, Mr. Patrick
\end{solution}



\begin{solution}[by \href{https://artofproblemsolving.com/community/user/10045}{socrates}]
	\begin{tcolorbox}[quote="dien9c"]Find continuous function $f:\mathbb{R} \to \mathbb{R}$ such that
\[a f(f(x)) = b f(x) + cx\]\end{tcolorbox}
Here is a rather quick answer to your general question.

(I hope that if you encounter a problem "solve in $\mathbb Z$ equation $x+17=3$" and another problem "solve in $\mathbb R$ equation $x^2+2x+1=0$", you will not ask in the forum, claiming that it is a real olympiad exercise : "solve in $\mathbb C$ equation $\sum_{k=0}^na_kx^k$ where $a_i\in\mathbb C$" just in order to get a general solution to all your polynomial equation exercises :( )


\begin{bolded}Some personal comments\end{underlined}\end{bolded} :
I did not check all the typos (too tiring) and I hope you'll excuse them
I dont claim that all my proofs are the shortest.
Maybe some subcases could have been merged.
Subcase 7.2 is not finished. It seems to me that it is one of the most complex and it has infinitely many solutions.
Your two examples in previous posts both are in subcase 7.1.1
Some subcases was full independant questions in this forum (for example $f(f(x))=x$ is subcase 5.3 and $f(x)+f^{-1}(x)=2x$ is subcase 7.3.2)

I hope you'll read this post line per line.



\begin{bolded}Case 1\end{underlined}\end{bolded} : If $a=b=c=0$ : any continuous function is solution

\begin{bolded}Case 2\end{underlined}\end{bolded} : If $a=b=0$ and $c\ne 0$ : no solution

\begin{bolded}Case 3\end{underlined} \end{bolded}: If $a=0$ and $b\ne 0$ : unique solution is $f(x)=-\frac cbx$ $\forall x$

\begin{bolded}Case 4\end{underlined}\end{bolded} : If $a\ne 0$ and $b=c=0$, equation is $f(f(x))=0$ and so $f(x)=0$ $\forall x\in f(\mathbb R)$ and, since continuous, $f(\mathbb R)$ is an interval and so : 

\begin{bolded}Subcase 4.1\end{bolded} : $f(\mathbb R)=\mathbb R$ and the unique solution $f(x)=0$ $\forall x$

\begin{bolded}Subcase 4.2\end{bolded} : $f(\mathbb R)=[a,+\infty)$. with $a\le 0$ (since obviously $0\in f(\mathbb R)$
Then we get infinitely many solutions :
Let $a\le 0$
Let $h(x)$ any continuous surjection from $(-\infty,a]\to [a,+\infty)$ such that $h(a)=0$
Then $f(x)=h(x)$ $\forall x<a$ and $f(x)=0$ $\forall x\ge a$

\begin{bolded}Subcase 4.3\end{bolded} : $f(\mathbb R)=(a,+\infty)$. with $a< 0$ (since obviously $0\in f(\mathbb R)$
Then we get infinitely many solutions :
Let $a<0$
Let $h(x)$ any continuous surjection from $(-\infty,a] \to (a,+\infty)$ such that $h(a)=0$
Then $f(x)=h(x)$ $\forall x<a$ and $f(x)=0$ $\forall x\ge a$

\begin{bolded}Subcase 4.4 \end{bolded}: $f(\mathbb R)=(-\infty,a]$. with $a\ge 0$ (since obviously $0\inf(\mathbb R)$
Then we get infinitely many solutions :
Let $a\ge 0$
Let $h(x)$ any continuous surjection from $[a,+\infty)to(-\infty,a]$ such that $h(a)=0$
Then $f(x)=h(x)$ $\forall x>a$ and $f(x)=0$ $\forall x\le a$

\begin{bolded}Subcase 4.5\end{bolded} : $f(\mathbb R)=(-\infty,a)$. with $a>0$ (since obviously $0\in f(\mathbb R)$
Then we get infinitely many solutions :
Let $a>0$
Let $h(x)$ any continuous surjection from $[a,+\infty)to(-\infty,a)$ such that $h(a)=0$
Then $f(x)=h(x)$ $\forall x>a$ and $f(x)=0$ $\forall x\le a$

\begin{bolded}Subcase 4.6\end{bolded} : $f(\mathbb R)=[a,b]$ with $a\le 0\le b$ (since obviously $0\in f(\mathbb R)$
Then we get infinitely many solutions :
Let $a\le 0\le b$
Let $h(x)$ any continuous function from $\mathbb R\to\mathbb R$ such that $h(\mathbb R\setminus(a,b))=[a,b]$ and $h(a)=h(b)=0$.
Then $f(x)=h(x)$ $\forall x\notin[a,b]$ and $f(x)=0$ $\forall x\in[a,b]$

\begin{bolded}Subcase 4.7\end{bolded} : $f(\mathbb R)=[a,b)$ with $a\le 0< b$ (since obviously $0\in f(\mathbb R)$
Then we get infinitely many solutions :
Let $a\le 0< b$
Let $h(x)$ any continuous function from $\mathbb R\to\mathbb R$ such that $hf(\mathbb R\setminus(a,b))=[a,b)$ and $h(a)=h(b)=0$.
Then $f(x)=h(x)$ $\forall x\notin[a,b]$ and $f(x)=0$ $\forall x\in[a,b]$

\begin{bolded}Subcase 4.8\end{bolded} : $f(\mathbb R)=(a,b]$ with $a< 0\le b$ (since obviously $0\in f(\mathbb R))$
Then we get infinitely many solutions :
Let $a< 0\le b$
Let $h(x)$ any continuous function from $\mathbb R\to\mathbb R$ such that $h(\mathbb R\setminus(a,b))=(a,b]$ and $h(a)=h(b)=0$.
Then $f(x)=h(x)$ $\forall x\notin[a,b]$ and $f(x)=0$ $\forall x\in[a,b]$

\begin{bolded}Subcase 4.9\end{bolded} : $f(\mathbb R)=(a,b)$ with $a< 0< b$ (since obviously $0\in f(\mathbb R))$
Then we get infinitely many solutions :
Let $a< 0< b$
Let $h(x)$ any continuous function from $\mathbb R\to\mathbb R$ such that $h(\mathbb R\setminus(a,b))=(a,b)$ and $h(a)=h(b)=0$.
Then $f(x)=h(x)$ $\forall x\notin[a,b]$ and $f(x)=0$ $\forall x\in[a,b]$

\begin{bolded}Case 5\end{underlined}\end{bolded} : If $a\ne 0$ and $b=0$ and $c\ne 0$, equation is $f(f(x))=tx$ with $t=\frac ca\ne 0$
So $f(x)$ is bijective and so, since continuous, monotonous.
So $f(f(x))$ is an increasing function.

\begin{bolded}Subcase 5.1\end{bolded} : If $t<0$ ($\iff ac<0$) then  no solution since $LHS$ is increasing while RHS is decreasing

\begin{bolded}Subcase 5.2\end{bolded} : if $1>t>0$ ($\iff$ $a>c>0$ or $a<c<0$), this is a classical equation with infinitely many solutions which may be built piece par piece.

5.2.1) increasing solutions\end{underlined} :
$\forall x>0$ :
Let $a\in(t,1)$
Let $h_1(x)$ any continuous increasing bijection from $[a,1]\to[t,a]$
Define $f(x)$ as :
$\forall x\in(a,1]$ : $f(x)=h_1(x)$
$\forall x\in(t,a]$ : $f(x)=th_1^{[-1]}(x)$
$\forall x\in(0,t]\cup(1,+\infty)$ : $f(x)=t^{\lfloor\log_t x\rfloor}f(xt^{-\lfloor\log_t x\rfloor})$

$f(0)=0$

$\forall x<0$
Let $b\in(-1,-t)$
Let $h_2(x)$ any continuous increasing bijection from $[-1,b]\to[b,-t]$
Define $f(x)$ as :
$\forall x\in[-1,b)$ : $f(x)=h_2(x)$
$\forall x\in[b,-t)$ : $f(x)=th_2^{[-1]}(x)$
$\forall x\in(-\infty,-1)\cup[-t,0)$ : $f(x)=t^{\lfloor\log_t -x\rfloor}f(xt^{-\lfloor\log_t -x\rfloor})$

5.2.2) decreasing solutions\end{underlined} :
Let $a<0$
Let $h(x)$ be any continuous decreasing bijection from $[t,1]\to[a,ta]$
$\forall x\ge 0$ : $f(x)=t^{\lfloor\log_t x\rfloor}h(xt^{-\lfloor\log_t x\rfloor})$
$\forall x<0$ : $f(x)=th^{[-1]}(x)$

\begin{bolded}Subcase 5.3\end{bolded} : If $t=1$ ($\iff a=c$), then equation is $f(f(x))=x$ and is very classical :

5.3.1) increasing solutions\end{underlined}
It's well known that equation $f(f(x))=x$ has a unique continuous increasing solution $f(x)=x$ $\forall x$

5.3.2) decreasing solutions\end{underlined}
A general form is :
Let $h(x)$ be any continuous decreasing bijection from $[0,+\infty)\to(-\infty,0]$
$\forall x\ge 0$ : $f(x)=h(x)$
$\forall x<0$ : $f(x)=h^{[-1]}(x)$

\begin{bolded}Subcase 5.4\end{bolded} : if $t>1$ ($\iff$ $c>a>0$ or $c<a<0$), this is a classical equation with infinitely many solutions which may be built piece par piece.

5.4.1) increasing solutions\end{underlined} :
$\forall x>0$ :
Let $a\in(1,t)$
Let $h_1(x)$ any continuous increasing bijection from $[1,a]\to[a,t]$
Define $f(x)$ as :
$\forall x\in[1,a)$ : $f(x)=h_1(x)$
$\forall x\in[a,t)$ : $f(x)=th_1^{[-1]}(x)$
$\forall x\in(0,1)\cup[t,+\infty)$ : $f(x)=t^{\lfloor\log_t x\rfloor}f(xt^{-\lfloor\log_t x\rfloor})$

$f(0)=0$

$\forall x<0$
Let $b\in(-t,-1)$
Let $h_2(x)$ any continuous increasing bijection from $[b,-1]\to[-t,b]$
Define $f(x)$ as :
$\forall x\in(b,-1]$ : $f(x)=h_2(x)$
$\forall x\in(-t,b]$ : $f(x)=th_2^{[-1]}(x)$
$\forall x\in(-\infty,-t]\cup(-1,0)$ : $f(x)=t^{\lfloor\log_t -x\rfloor}f(xt^{-\lfloor\log_t -x\rfloor})$

5.4.2) decreasing solutions\end{underlined} :
Let $a<0$
Let $h(x)$ be any continuous decreasing bijection from $[1,t]\to[ta,a]$
$\forall x\ge 0$ : $f(x)=t^{\lfloor\log_t x\rfloor}h(xt^{-\lfloor\log_t x\rfloor})$
$\forall x<0$ : $f(x)=th^{[-1]}(x)$
\end{tcolorbox}


\end{solution}



\begin{solution}[by \href{https://artofproblemsolving.com/community/user/10045}{socrates}]
	\begin{tcolorbox}\begin{bolded}Case 6\end{underlined}\end{bolded} : If $a\ne 0$ and $b=\ne 0$ and $c=0$, equation is $f(f(x))=tf(x)$ with $t=\frac ba\ne 0$
So $f(x)=tx$ $\forall x\in f(\mathbb R)$ 
Since continuous, $f(\mathbb R)$ is an interval and so :

\begin{bolded}Subcase 6.1\end{bolded} : if $f(\mathbb R)=\mathbb R$, we get $f(x)=tx$ $\forall x$ which indeed is a solution

\begin{bolded}Subcase 6.2\end{bolded} : if $f(\mathbb R)=[a,+\infty)$ 
This means $ta\ge a$ (since $f(a)=ta\in f(\mathbb R)$) and the solutions :

\begin{bolded}Subcase 6.2.1\end{bolded} : if $t>1$, then $a\ge 0$ and :
Let $a\ge 0$
Let $h(x)$ be any continuous function from $(-\infty,a]\to[a,+\infty)$ such that $h(a)=ta$ and $[a,ta]\subseteq h((-\infty,a])$ then 
$f(x)=h(x)$ $\forall x\le a$
$f(x)=tx$ $\forall x>a$

\begin{bolded}Subcase 6.2.2\end{bolded} : if $t=1$, then no constraint on $a$ and :
Let $a \in\mathbb R$
Let $h(x)$ be any continuous function from $(-\infty,a]\to[a,+\infty)$ such that $h(a)=a$. Then :
$f(x)=h(x)$ $\forall x\le a$
$f(x)=x$ $\forall x>a$

\begin{bolded}Subcase 6.2.3\end{bolded} : if $1>t>0$, then $a\le 0$ and :
Let $a\le 0$
Let $h(x)$ be any continuous function from $(-\infty,a]\to[a,+\infty)$ such that $h(a)=ta$ and $[a,ta]\subseteq h((-\infty,a])$ then 
$f(x)=h(x)$ $\forall x\le a$
$f(x)=tx$ $\forall x>a$

\begin{bolded}Subcase 6.2.4\end{bolded} : if $t<0$
Choosing some $f(x)$ positive and great enough, we get that $tf(x)<a$ and so can not be in $f(\mathbb R)$
So no solution

\begin{bolded}Subcase 6.3\end{bolded} : if $f(\mathbb R)=(a,+\infty)$
We get $f(x)=tx$ $\forall x>a$ and so (continuity) $f(a)=ta$ and so $ta>a$ and the solutions :

\begin{bolded}Subcase 6.3.1\end{bolded} : if $t>1$, then $a>0$ and :
Let $a>0$
Let $h(x)$ be any continuous function from $(-\infty,a]\to(a,+\infty)$ such that $h(a)=ta$ and $(a,ta]\subseteq h((-\infty,a])$ then 
$f(x)=h(x)$ $\forall x\le a$
$f(x)=tx$ $\forall x>a$

\begin{bolded}Subcase 6.3.2\end{bolded} : if $t=1$, then $ta>a$ is impossible and so no solution

\begin{bolded}Subcase 6.3.3\end{bolded} : if $1>t>0$, then $a< 0$ and :
Let $a<0$
Let $h(x)$ be any continuous function from $(-\infty,a]\to(a,+\infty)$ such that $h(a)=ta$ and $(a,ta]\subseteq h((-\infty,a])$ then 
$f(x)=h(x)$ $\forall x\le a$
$f(x)=tx$ $\forall x>a$

\begin{bolded}Subcase 6.3.4\end{bolded} : if $t<0$
Choosing some $f(x)$ positive and great enough, we get that $tf(x)\le a$ and so can not be in $f(\mathbb R)$
So no solution

\begin{bolded}Subcase 6.4\end{bolded} : if $f(\mathbb R)=(-\infty,a]$
This means $ta\le a$ (since $f(a)=ta\in f(\mathbb R)$) and the solutions :

\begin{bolded}Subcase 6.4.1\end{bolded}: if $t>1$, then $a\le 0$ and :
Let $a\le 0$ 
Let $h(x)$ any continuous function from $[a,+\infty)\to(-\infty,a]$ such that $h(a)=ta$ and $[ta,a]\subseteq h([a,+\infty))$. Then :
$f(x)=tx$ $\forall x\le a$
$f(x)=h(x)$ $\forall x>a$

\begin{bolded}Subcase 6.4.2\end{bolded}: if $t=1$, then no constraint on $a$ and :
Let $a\in\mathbb R$ 
Let $h(x)$ any continuous function from $[a,+\infty)\to(-\infty,a]$ such that $h(a)=a$. Then :
$f(x)=x$ $\forall x\le a$
$f(x)=h(x)$ $\forall x>a$

\begin{bolded}Subcase 6.4.3\end{bolded} : if $1>t>0$, then $a\ge 0$ and :
Let $a\ge 0$ 
Let $h(x)$ any continuous function from $[a,+\infty)\to(-\infty,a]$ such that $h(a)=ta$ and $[ta,a]\subseteq h([a,+\infty))$. Then :
$f(x)=tx$ $\forall x\le a$
$f(x)=h(x)$ $\forall x>a$

\begin{bolded}Subcase 6.4.4\end{bolded} : if $t<0$
Choosing some $f(x)$ negative and small enough, we get that $tf(x)> a$ and so can not be in $f(\mathbb R)$
So no solution

\begin{bolded}Subcase 6.5\end{bolded} : if $f(\mathbb R)=(-\infty,a)$
We get $f(x)=tx$ $\forall x<a$ and so (continuity) $f(a)=ta$ and so $ta<a$ and the solutions :

\begin{bolded}Subcase 6.5.1\end{bolded}: if $t>1$, then $a< 0$ and :
Let $a<0$ 
Let $h(x)$ any continuous function from $[a,+\infty)\to(-\infty,a)$ such that $h(a)=ta$ and $[ta,a)\subseteq h([a,+\infty))$. Then :
$f(x)=tx$ $\forall x\le a$
$f(x)=h(x)$ $\forall x>a$

\begin{bolded}Subcase 6.5.2\end{bolded}: if $t=1$, then $ta<a$ is impossible and so no solution

\begin{bolded}Subcase 6.5.3\end{bolded} : if $1>t>0$, then $a>0$ and :
Let $a>0$ 
Let $h(x)$ any continuous function from $[a,+\infty)\to(-\infty,a)$ such that $h(a)=ta$ and $[ta,a)\subseteq h([a,+\infty))$. Then :
$f(x)=tx$ $\forall x\le a$
$f(x)=h(x)$ $\forall x>a$

\begin{bolded}Subcase 6.5.4\end{bolded} : if $t<0$
Choosing some $f(x)$ negative and small enough, we get that $tf(x)\ge a$ and so can not be in $f(\mathbb R)$
So no solution

\begin{bolded}Subcase 6.6 \end{bolded}: if $f(\mathbb R)=[a,b]$
Then $a\le ta\le b$ and $a\le tb\le b$

\begin{bolded}Subcase 6.6.1\end{bolded} : if $t>1$, then $a\le ta\le b$ and $a\le tb\le b$ imply $a=b=0$ and so $f(x)=0$ $\forall x$, which indeed is a solution

\begin{bolded}Subcase 6.6.2\end{bolded} : if $t=1$, we get the solutions :
Let $a\le b$
Let $h_1(x)$ any continuous function from $(-\infty,a]\to[a,b]$ such that $h_1(a)=a$
Let $h_2(x)$ any continuous function from $[b,+\infty]\to[a,b]$ such that $h_2(b)=b$
Then :
$\forall x<a$ : $f(x)=h_1(x)$
$\forall x\in[a,b]$ : $f(x)=x$
$\forall x>b$ : $f(x)=h_2(x)$

\begin{bolded}Subcase 6.6.3\end{bolded} : if $1>t>0$, then $a\le ta\le b$ and $a\le tb\le b$ imply $a\le 0\le b$ and :
Let $a\le 0\le b$
Let $h(x)$ be any continuous function from $\mathbb R\to[a,b]$ such that :
$h(a)=ta$
$h(b)=tb$
$[a,ta]\cup[tb,b]\subseteq h((-\infty,a]\cup[b,+\infty))$
Then :
$\forall x\in[a,b]$ : $f(x)=tx$
$\forall x\notin[a,b]$ : $f(x)=h(x)$

\begin{bolded}Subcase 6.6.4\end{bolded} : if $-1\le t<0$, then $a\le ta\le b$ and $a\le tb\le b$ imply $a\le 0$ and $\frac at\ge b\ge ta$ and :
Let $a\le 0$ and $b\in[ta,\frac at]$
Let $h(x)$ be any continuous function from $\mathbb R\to[a,b]$ such that :
$h(a)=ta$
$h(b)=tb$
$[a,tb]\cup[ta,b]\subseteq h((-\infty,a]\cup[b,+\infty))$
Then :
$\forall x\in[a,b]$ : $f(x)=tx$
$\forall x\notin[a,b]$ : $f(x)=h(x)$

\begin{bolded}Subcase 6.6.5 \end{bolded}: if $t<-1$, then $a\le ta\le b$ and $a\le tb\le b$ imply $a=b=0$ and the solution $f(x)=0$ $\forall x$ which indeed is a solution.

\begin{bolded}Subcase 6.7\end{bolded} : if $f(\mathbb R)=[a,b)$
Then $f(x)=tx$ $\forall x\in [a,b)$ and so $tx\in[a,b)$ $\forall x\in[a,b)$
And so continuity implies $a\le ta< b$ and $a\le tb< b$

\begin{bolded}Subcase 6.7.1\end{bolded} : if $t\ge 1$, then $a\le ta< b$ and $a\le tb< b$ imply no solution

\begin{bolded}Subcase 6.7.2\end{bolded} : if $1>t>0$, then $a\le ta< b$ and $a\le tb< b$ imply $a\le 0< b$ and :
Let $a\le 0< b$
Let $h(x)$ be any continuous function from $\mathbb R\to[a,b)$ such that :
$h(a)=ta$
$h(b)=tb$
$[a,ta]\cup[tb,b)\subseteq h((-\infty,a]\cup[b,+\infty))$
Then :
$\forall x\in[a,b]$ : $f(x)=tx$
$\forall x\notin[a,b]$ : $f(x)=h(x)$

\begin{bolded}Subcase 6.7.3\end{bolded} : if $-1< t<0$, then $a\le ta< b$ and $a\le tb< b$ imply $a<0$ and $\frac at\ge b> ta$ and :
Let $a<0$ and $b\in(ta,\frac at]$
Let $h(x)$ be any continuous function from $\mathbb R\to[a,b)$ such that :
$h(a)=ta$
$h(b)=tb$
$[a,tb]\cup[ta,b)\subseteq h((-\infty,a]\cup[b,+\infty))$
Then :
$\forall x\in[a,b]$ : $f(x)=tx$
$\forall x\notin[a,b]$ : $f(x)=h(x)$

\begin{bolded}Subcase 6.7.4\end{bolded} : if $t\le -1$, then $a\le ta< b$ and $a\le tb< b$ imply no solution

\begin{bolded}Subcase 6.8\end{bolded} : if $f(\mathbb R)=(a,b]$
Then $f(x)=tx$ $\forall x\in (a,b]$ and so $tx\in(a,b]$ $\forall x\in(a,b]$
And so continuity implies $a< ta\le b$ and $a< tb\le b$

\begin{bolded}Subcase 6.8.1\end{bolded} : if $t\ge 1$, then $a< ta\le b$ and $a< tb\le b$ imply no solution

\begin{bolded}Subcase 6.8.2\end{bolded} : if $1>t>0$, then $a< ta\le b$ and $a< tb\le b$ imply $a<0\le b$ and :
Let $a< 0\le b$
Let $h(x)$ be any continuous function from $\mathbb R\to[a,b]$ such that :
$h(a)=ta$
$h(b)=tb$
$(a,ta]\cup[tb,b]\subseteq h((-\infty,a]\cup[b,+\infty))$
Then :
$\forall x\in[a,b]$ : $f(x)=tx$
$\forall x\notin[a,b]$ : $f(x)=h(x)$

\begin{bolded}Subcase 6.8.3\end{bolded} : if $-1<t<0$, then $a< ta\le b$ and $a< tb\le b$ imply $a< 0$ and $\frac at> b\ge ta$ and :
Let $a< 0$ and $b\in[ta,\frac at)$
Let $h(x)$ be any continuous function from $\mathbb R\to(a,b]$ such that :
$h(a)=ta$
$h(b)=tb$
$(a,tb]\cup[ta,b]\subseteq h((-\infty,a]\cup[b,+\infty))$
Then :
$\forall x\in[a,b]$ : $f(x)=tx$
$\forall x\notin[a,b]$ : $f(x)=h(x)$

\begin{bolded}Subcase 6.8.4\end{bolded} : if $t\le -1$, then $a< ta\le b$ and $a< tb\le b$ imply no solution

\begin{bolded}Subcase 6.9\end{bolded} : if $f(\mathbb R)=(a,b)$
Then $f(x)=tx$ $\forall x\in (a,b)$ and so $tx\in(a,b)$ $\forall x\in(a,b)$
And so continuity implies $a< ta< b$ and $a< tb< b$

\begin{bolded}Subcase 6.9.1\end{bolded} : if $t\ge 1$, then $a< ta<b$ and $a< tb< b$ imply no solution

\begin{bolded}Subcase 6.9.2\end{bolded} : if $1>t>0$, then $a< ta<b$ and $a< tb< b$ imply $a< 0< b$ and :
Let $a< 0< b$
Let $h(x)$ be any continuous function from $\mathbb R\to(a,b)$ such that :
$h(a)=ta$
$h(b)=tb$
$(a,ta]\cup[tb,b)\subseteq h((-\infty,a]\cup[b,+\infty))$
Then :
$\forall x\in[a,b]$ : $f(x)=tx$
$\forall x\notin[a,b]$ : $f(x)=h(x)$

\begin{bolded}Subcase 6.9.3\end{bolded} : if $-1< t<0$, then $a< ta<b$ and $a< tb< b$ imply $a< 0$ and $\frac at> b> ta$ and :
Let $a<0$ and $b\in(ta,\frac at)$
Let $h(x)$ be any continuous function from $\mathbb R\to(a,b)$ such that :
$h(a)=ta$
$h(b)=tb$
$(a,tb]\cup[ta,b)\subseteq h((-\infty,a]\cup[b,+\infty))$
Then :
$\forall x\in[a,b]$ : $f(x)=tx$
$\forall x\notin[a,b]$ : $f(x)=h(x)$

\begin{bolded}Subcase 6.9.4\end{bolded} : if $t\le -1$, then $a< ta<b$ and $a< tb< b$ imply no solution


\begin{bolded}Case  7\end{underlined}\end{bolded} : $a,b,c\ne 0$ and the equation is $f(f(x))=uf(x)+vx$ with $u,v\ne 0$

$f(x)$ is injective and so, since continuous, monotonous.
If $\lim_{x\to+\infty}f(x)=L$, then setting $x\to+\infty$ in functional equation and using continuity gives contradiction
So $\lim_{x\to+\infty}f(x)=\pm\infty$
If $\lim_{x\to-\infty}f(x)=L$, then setting $x\to-\infty$ in functional equation and using continuity gives contradiction
So $\lim_{x\to-\infty}f(x)=\pm\infty$

And since $f(x)$ is monotonous, we get that $f(\mathbb R)=\mathbb R$ and so $f(x)$ is a bijection.

\begin{bolded}Subcase 7.1\end{bolded} : $v>0$
let $x\in\mathbb R$ and the sequence $a_n$ defined as :
$a_0=x$
$a_1=f(x)$
$a_{n+2}=ua_{n+1}+va_n$ $\forall n\ge 0$
Notice than $a_n=f^{[n]}(x)$
Since $v>0$, the characteristic equation $x^2-ux-v$ has two distinct real roots $r_1>0>r_2$

And so $f^{[n]}(x)=\frac{(f(x)-r_2x)r_1^n-(f(x)-r_1x)r_2^n}{r_1-r_2}$

Since $f(x)$ is a bijection, then $f^{-1}(x)$ exists and it is easy to show that the above expression is true $\forall n\in\mathbb Z$

\begin{bolded}Subcase 7.1.1\end{bolded} : $u+v\ne 1$ 
$u+v\ne 1$ $\implies$ $r_1\ne 1$
$u\ne 0$ $\implies$ $r_2\ne -r_1$

If the equation $f(x)=x$ has real root $r$, then functional equation implies $r=ur+vr$ and so $r=0$
So, if $x\ne 0$, $f^{k+1}(x)\ne f^k(x)$ $\forall k$

For $x\ne 0$, we can then define $\Delta_n(x)=\frac{f^{n+2}(x)-f^{n+1}(x)}{f^{n+1}(x)-f^{n}(x)}$
$\Delta_n(x)\ne 0$
Since $f(x)$ is monotonous, $\Delta_n(x)$ has a constant sign, for any values of $n\in\mathbb Z$ and $x\in\mathbb R^*$

$\Delta_n(x)=$ $\frac{(f(x)-r_2x)(r_1-1)r_1^{n+1}-(f(x)-r_1x)(r_2-1)r_2^{n+1}}{(f(x)-r_2x)(r_1-1)r_1^{n}-(f(x)-r_1x)(r_2-1)r_2^{n}}$

Let $x\ne 0$ such that $f(x)\ne r_1x$ and $f(x)\ne r_2x$
We know that $|r_1|\ne |r_2|$ and so :
If $|r_1|>|r_2|$ : $\lim_{n\to +\infty}\Delta_n(x)=r_1$ and $\lim_{n\to -\infty}\Delta_n(x)=r_2$
If $|r_1|<|r_2|$ : $\lim_{n\to +\infty}\Delta_n(x)=r_2$ and $\lim_{n\to -\infty}\Delta_n(x)=r_1$
In both cases $\Delta_n(x)$ does not have a constant sign.

So $\forall x\ne 0$, either $f(x)=r_1x$, either $f(x)=r_2x$ and continuity + monotonicity imply :
either $f(x)=r_1x$ $\forall x$ which indeed is a solution
either $f(x)=r_2x$ $\forall x$ which indeed is a solution

Hence two solutons in this subcase

\begin{bolded}Subcase 7.1.2\end{bolded} : $u+v=1$
So $r_1=1$ and $r_2=u-1=-v<0$
$f(x)=x$ is a solution.
Let us from now in this subcase look for other solutions (different from $f(x)=x$ $\forall x$)

Let then $x$ such that $f(x)\ne x$ : $\frac{f(f(x))-f(x)}{f(x)-x}=-v<0$ and so $f(x)$ must be decreasing.

Since $f(x)$ is continuous and decreasing, then equation $f(x)=x$ has a unique root $r$.
Let then $g(x)=f(x+r)-r$ : it's easy to check that $g(x)$ is such that $g(g(x))=ug(x)+vx$ and $g(0)=0$

So WLOG consider from now that $f(0)=0$ and so $f(x)\ne 0$ $\forall x\ne 0$
Since $f(x)$ is decreasing, $f^{n}(x)$ is increasing for even $n$ and decreasing for odd $n$
And since $f^{n}(0)=0$, we can conclude

$\forall x\ne 0$ : $\frac{f^{n}(x)}x$ is nonzero and has same sign as $(-1)^n$

But $\frac{f^{n}(x)}x$ $=\frac{f(x)-r_2x-(f(x)-x)r_2^n}{(1-r_2)x}$

Since $r_2\ne \pm 1$, this quantity has limit  $\frac{f(x)-r_2x}{(1-r_2)x}$ when $n$ is set to $+\infty$ if $|r_2|<1$ or $-\infty$ if $|r_2|>1$
And so $f(x)=r_2x$, else this quantity can no longer has same sign as $(-1)^n$ when $n$ is set to the appropriate $\infty$
So $f(x)=-vx$ $\forall x$, which indeed is a solution

And so the solutions in this subcase :
$f(x)=x$ $\forall x$
$f(x)=c-vx$ $\forall x$

\begin{bolded}Subcase 7.2\end{bolded} : $v<0$ and $u^2+4v>0$
\begin{bolded}[color=red]This case need to be developped a bit more[\/color].\end{bolded}
We again obviously have the two solutions $f(x)=r_1x$ and $f(x)=r_2x$, both $r_1,r_2$ having the same sign
But there are in some cases a lot of other solutions.
Look for example at $f(f(x))=5f(x)-6x$ : it's possible to buid piece per piece infinitely many solutions such that for example $f(x)\in(2x,3x)$ $\forall x>0$


\begin{bolded}Subcase 7.3 \end{bolded}: $v<0$ and $u^2+4v=0$
\begin{bolded}Subcase 7.3.1\end{bolded} : $u\ne 2$
let $x\in\mathbb R$ and the sequence $a_n$ defined as :
$a_0=x$
$a_1=f(x)$
$a_{n+2}=ua_{n+1}+va_n$ $\forall n\ge 0$
Notice than $a_n=f^{[n]}(x)$
Since $u^2+4v=0$, the characteristic equation $x^2-ux-v$ has one double real root $r=\frac u2\notin\{0,1\}$

And so $f^{[n]}(x)=r^{n-1}(nf(x)-r(n-1)x)$

Since $f(x)$ is a bijection, then $f^{-1}(x)$ exists and it is easy to show that the above expression is true $\forall n\in\mathbb Z$

Note that $u^2+4v=0$ and $u\ne 2$ implu $u+v\ne 1$
If the equation $f(x)=x$ has real root $z$, then functional equation implies $z=uz+vz$ and so $z=0$
So, if $x\ne 0$, $f^{k+1}(x)\ne f^k(x)$ $\forall k$

For $x\ne 0$, we can then define $\Delta_n(x)=\frac{f^{n+2}(x)-f^{n+1}(x)}{f^{n+1}(x)-f^{n}(x)}$
$\Delta_n(x)\ne 0$
Since $f(x)$ is monotonous, $\Delta_n(x)$ has a constant sign, for any values of $n\in\mathbb Z$ and $x\in\mathbb R^*$

$\Delta_n(x)=$ $r\frac{(n+1)(r-1)(f(x)-rx)+r(f(x)-x)}{n(r-1)(f(x)-rx)+r(f(x)-x)}$

If $f(x)\ne rx$ for some $x\ne 0$, and since $r\ne 1$, we get $(r-1)(f(x)-rx)\ne 0$ and so $\exists n\in\mathbb Z$ such that the two parts of the fraction have opposite signs.
And since, for $n$ great enough, these two parts have same sign, we get that $\Delta_n(x)$ can not have a constant sign.
So $f(x)=rx$ $\forall x$, whch indeed is a solution.

\begin{bolded}Subcase 7.3.2\end{bolded} : $u=2$ and $v=-1$
Equation is $f(f(x))=2f(x)-x$

Writing $g(x)=f(x)-x$, the equation is $g(x+g(x))=g(x)$

$g(x)=0$ $\forall x$ is a solution and let us from now look for non all zero solutions.
If $g(x)$ is solution, then $-g(-x)$ is solution too and so Wlog say $g(p)=q>0$ for some $p$

Let $A=\{x\ge p$ such that $g(x)=g(p)=q\}$

From $g(x+g(x))=g(x)$, we get $g(x+ng(x))=g(x)$ and so $p+nq\in A$ $\forall n\in\mathbb N\cup\{0\}$

If $A$ is not dense in $[p,+\infty)$, let then $a,b\in A$ such that $p\le a<b$ and $(a,b)\cap A=\emptyset$. (existence of $a,b$ needs continuity of $g(x)$)

Let then $y\in(a,b)$. So $g(y)\ne q$ 
Consider then $y-a+n(g(y)-q)$ for $n\in\mathbb N$
Since $g(y)\ne q$, this quantity, for $n$ great enough is out of $[-q,+q]$ and so let $m>0$ such that $y-a+m(g(y)-q)\notin[-q,+q]$ and so such that $y+mg(y)\notin[a+(m-1)q,a+(m+1)q]$

Looking at the continuous function $h(x)=x+mg(x)$, we get :
$h(a)=a+mq\in(a+(m-1)q,a+(m+1)q)$
$h(y)=y+mg(y)\notin[a+(m-1)q,a+(m+1)q]$

So (using continuity of $h(x)$), $\exists z\in(a,y)$ such that $h(z)=a+(m-1)q$ or $h(z)=a+(m+1)q$
But then $g(h(z))=q$ and so $g(z+mg(z))=g(z)=q$, impossible since $z\in(a,b)$ and $(a,b)\cap A=\emptyset$.

So $A$ is dense in $[p,+\infty)$

Then continuity of $g(x)$ implies $g(x)=q$ $\forall x\ge p$.
Let then any $w<p$ : If $g(w)>0$, then $\exists n\in\mathbb N$ such that $w+ng(w)>p$ and so $g(w)=q$. So $\forall x<p$ : either $g(x)=q$, either $g(x)\le 0$ and continuity gives the conclusion $g(x)=q$ $\forall x$

So $g(x)=c$ and $f(x)=x+c$ which indeed is a solution.

\begin{bolded}Subcase 7.4 \end{bolded}: $v<0$ and $u^2+4v<0$
let $x\in\mathbb R$ and the sequence $a_n$ defined as :
$a_0=x$
$a_1=f(x)$
$a_{n+2}=ua_{n+1}+va_n$ $\forall n\ge 0$
Notice than $a_n=f^{[n]}(x)$
Since $u^2+4v<0$, the characteristic equation $x^2-ux-v$ has two distinct complex roots $re^{it}$ and $re^{-it}$ with $r>0$ and $t\in(0,\pi)$

And so $f^{[n]}(x)=r^{n-1}\frac{f(x)\sin nt-rx\sin(n-1)t}{\sin t}$

Since $f(x)$ is a bijection, then $f^{-1}(x)$ exists and it is easy to show that the above expression is true $\forall n\in\mathbb Z$

Note that $u^2+4v<0$ implies $u+v<1$
If the equation $f(x)=x$ has real root $r$, then functional equation implies $r=ur+vr$ and so $r=0$
So, if $x\ne 0$, $f^{k+1}(x)\ne f^k(x)$ $\forall k$

For $x\ne 0$, we can then define $\Delta_n(x)=\frac{f^{n+2}(x)-f^{n+1}(x)}{f^{n+1}(x)-f^{n}(x)}$
$\Delta_n(x)\ne 0$
Since $f(x)$ is monotonous, $\Delta_n(x)$ has a constant sign, for any values of $n\in\mathbb Z$ and $x\in\mathbb R^*$

$f^{k+1}(x)-f^{k}(x)$ $=r^{k}\frac{f(x)\sin (k+1)t-rx\sin kt}{\sin t}$ $-r^{k-1}\frac{f(x)\sin kt-rx\sin(k-1)t}{\sin t}$
$=\frac{r^{k-1}}{\sin t}$ $(rf(x)\sin (k+1)t-r^2x\sin kt-f(x)\sin kt+rx\sin(k-1)t)$

$=cr^k\sin(kt+d)$ for some $c\ne 0,d$ depending on $x$

So $\Delta_n(x)=$ $r\frac{\sin (n+1)t+d}{\sin nt+d}$

But this quantity can not be of constant sign $\forall n$ (since $t\in(0,\pi)$)

Hence no solution\end{tcolorbox}


\end{solution}



\begin{solution}[by \href{https://artofproblemsolving.com/community/user/29428}{pco}]
	Thanks Socrates, but I already reposted these posts here :
http://artofproblemsolving.com\/community\/c6h1344617p7316443

(the above cut post was cut when the migration of the forum occured)
\end{solution}
*******************************************************************************
-------------------------------------------------------------------------------

\begin{problem}[Posted by \href{https://artofproblemsolving.com/community/user/10045}{socrates}]
	Determine all surjective functions $f : \mathbb{R}^+ \to \mathbb{R}^+$ such that \[f(x+f(y))+f(y+f(x))=f(f(2x)+2y) ,\]
       for all $x,y \in \mathbb{R}^+$
	\flushright \href{https://artofproblemsolving.com/community/c6h557118}{(Link to AoPS)}
\end{problem}



\begin{solution}[by \href{https://artofproblemsolving.com/community/user/29428}{pco}]
	\begin{tcolorbox}Determine all surjective functions $f : \mathbb{R}^+ \to \mathbb{R}^+$ such that \[f(x+f(y))+f(y+f(x))=f(f(2x)+2y) ,\]
       for all $x,y \in \mathbb{R}^+$\end{tcolorbox}
Let $P(x,y)$ be the assertion $f(x+f(y))+f(y+f(x))=f(f(2x)+2y)$, true $\forall x,y>0$

Comparing $P(\frac x2,\frac y2)$ with $P(\frac y2,\frac x2)$, we get new assertion $Q(x,y)$ : $f(x+f(y))=f(y+f(x))$ $\forall x,y>0$

If $f(a)=f(b)$ for some $a>b$, subtracting $Q(a,y)$ from $Q(b,y)$ implies $f(a+f(y))=f(b+f(y))$
And surjectivity implies then $f(x+(a-b))=f(x)$ $\forall x>b$

Let then $n\in\mathbb N$ great enough to have $f(1)-f(2)+n(a-b)>0$ and $2f(1)-f(2)+n(a-b)>b$
$P(1,f(1)-f(2)+n(a-b))$ $\implies$ $f(1+f(f(1)-f(2)+n(a-b)))$ $+f(2f(1)-f(2)+n(a-b))$ $=f(2f(1)-f(2)+2n(a-b))$ and so $f(1+f(f(1)-f(2)+n(a-b)))=0$, impossible

So $f(x)$ is injective and $Q(x,y)$ immediately gives $f(x)=x+c$ for some $c$.
Plugging this in original equation, we get $c=0$ and the unique surjective solution $\boxed{f(x)=x}$ $\forall x>0$
\end{solution}
*******************************************************************************
-------------------------------------------------------------------------------

\begin{problem}[Posted by \href{https://artofproblemsolving.com/community/user/10045}{socrates}]
	Find all functions $f,g:\Bbb{R}^+ \to \Bbb{R}^+$ such that: 
\[ \begin{cases} 
f(2g(x)+f(y))=g(2x)+f(y) & (\forall x,y\in \Bbb{R}^+) \\ 
g(g(x)+f(y))=2x-g(x)+2y-f(y) \ \ \ \ \   & (\forall x,y\in \Bbb{R}^+)
 \end{cases} \]
	\flushright \href{https://artofproblemsolving.com/community/c6h558087}{(Link to AoPS)}
\end{problem}



\begin{solution}[by \href{https://artofproblemsolving.com/community/user/29428}{pco}]
	\begin{tcolorbox}Find all functions $f,g:\Bbb{R}^+ \to \Bbb{R}^+$ such that: 
\[ \begin{cases} 
f(2g(x)+f(y))=g(2x)+f(y) & (\forall x,y\in \Bbb{R}^+) \\ 
g(g(x)+f(y))=2x-g(x)+2y-f(y) \ \ \ \ \   & (\forall x,y\in \Bbb{R}^+)
 \end{cases} \]\end{tcolorbox}
We can solve this problem using only second equation :
Let $P(x,y)$ be the assertion $g(g(x)+f(y))=2x-g(x)+2y-f(y)$

Setting $x,y\to 0^+$ in $P(x,y)$ RHS, we get $\lim_{x\to 0^+}f(x)=0$ and $\lim_{x\to 0^+}g(x)=0$ and so $f(x)\le 2x$ and $g(x)\le 2x$ $\forall x>0$

As a consequence, $P(x,y)$ $\implies$ $2x-g(x)+2y-f(y)\le 2g(x)+2f(y)$ and so $\frac 23x\le f(x)\le 2x$ and $\frac 23x\le g(x)\le 2x$

If $a_nx\le f(x)\le b_nx$ and $a_nx\le g(x)\le b_nx$, then :
$P(x,y)$ $\implies$ $a_ng(x)+a_nf(y)\le 2x-g(x)+2y-f(y)\le b_ng(x)+b_nf(y)$

And so $\frac 2{b_n+1}x+\le g(x)\le \frac 2{a_n+1}x$ and $\frac 2{b_n+1}x+\le f(x)\le \frac 2{a_n+1}x$

And a simple sequence study implies $\boxed{f(x)=g(x)=x}$ $\forall x$, which indeed is a solution for both equations.
\end{solution}
*******************************************************************************
-------------------------------------------------------------------------------

\begin{problem}[Posted by \href{https://artofproblemsolving.com/community/user/59765}{hungkg}]
	Find all Functions $f:R \to R$ such that $f\left( {{f^2}\left( x \right) + f\left( y \right)} \right) = xf\left( x \right) + y,\forall x,y \in R.$
	\flushright \href{https://artofproblemsolving.com/community/c6h565030}{(Link to AoPS)}
\end{problem}



\begin{solution}[by \href{https://artofproblemsolving.com/community/user/29428}{pco}]
	\begin{tcolorbox}Find all Functions $f:\mathbb R \to \mathbb R$ such that $f\left( {{f^2}\left( x \right) + f\left( y \right)} \right) = xf\left( x \right) + y,\forall x,y \in \mathbb R.$\end{tcolorbox}
Let $P(x,y)$ be the assertion $f(f(x)^2+f(y))=xf(x)+y$
$f(x)$ is bijective.
Let then $u$ such that $f(u)=0$

$P(u,x)$ $\implies$ $f(f(x))=x$

$P(x,u)$ $\implies$ $f(f(x)^2)=xf(x)+u$
$P(f(x),u)$ $\implies$ $f(x^2)=xf(x)+u$
So $f(f(x)^2)=f(x^2)$ and so $\forall x$ either $f(x)=x$, either $f(x)=-x$

Suppose now that $\exists u,v\ne 0$ such that $f(u)=u$ and $f(v)=-v$
$P(u,v)$ $\implies$ $f(u^2-v)=u^2+v$ and so :
either $f(u^2-v)=u^2-v$ and so $v=0$, impossible
either $f(u^2-v)=-(u^2-v)$ and so $u=0$, impossible
So no such $u,v$ and :

Either $\boxed{f(x)=x}$ $\forall x$, which indeed is a solution

Either $\boxed{f(x)=-x}$ $\forall x$, which indeed is a solution
\end{solution}



\begin{solution}[by \href{https://artofproblemsolving.com/community/user/179088}{Panoz93}]
	Another approach ...
In the initial equation we set $x=y=0$ and we get $f(f(0)^{2}+f(0))=0$ 
Also , we set $x=y=f(0)^{2}+f(0)$ and we immidiately conclude that $f(0)^2=0\Leftrightarrow f(0)=0$

So , now we set $x=0$ to get $f(f(y))=y$        $(1)$
Using $(1)$ we can substitute $x=f(x), y=0$ to get $f(x^2)=xf(x)$
Therefore $f$ is odd 

The given equation reduces to 
$f(x^{2}+y)=f(x^{2})+f(y)\Leftrightarrow f(a+b)=f(a)+f(b) $ for a non-negative 
Since odd , though , we conclude that $f$ satisfies cauchy for any real numbers $a,b$

Combining $f((x+1)^2)=(x+1)f(x+1)$ with cauchy we get f(x)=xf(1)
and thus , substitution to the initial gives us two possible solutions :

$f(x)=x$
$f(x)=-x$
\end{solution}



\begin{solution}[by \href{https://artofproblemsolving.com/community/user/204311}{Onlygodcanjudgeme}]
	we know that function is bijective.
$  f(f(x^2) + f(y)) = f(x) \cdot x +y  $
$  (x,y) = (0,0) $ then  $ f(f(o)^2 + f(0)) =0 $
if $ f(k)= 0 $ , and $ f(k) = f(f(0)^2 + f(0)) $ we know that function is injective, so  $ k= f(0)^2 + f(0)  $
Put $ (x,y) = (k,k) $ then we have  $ f(0) = k $ 
 $ k= f(0)^2 + f(0)  $ and  $ f(0) =k $
$ f(0)= 0 $.So k is equal to 0 .
    Put $ (x,y) = (x,0) $ then we have 
$ f(f(x)^2)= f(x) \cdot x $  
    Put    $ (x,y) =(f(x) ,0) $ so we have $ f(x^2)= f(x) \cdot x $
in there $ f(f(x)^2) = f(x^2) $  then   $ f(x)^2 = x^2  $ 
$ f(x) = x $ or $ f(x) =-x $
\end{solution}



\begin{solution}[by \href{https://artofproblemsolving.com/community/user/29428}{pco}]
	\begin{tcolorbox}...
in there $ f(f(x)^2) = f(x^2) $  then   $ f(x)^2 = x^2  $ 
$ f(x) = x $ or $ f(x) =-x $\end{tcolorbox}
No, $f(x)^2=x^2$ does not imply : "either $f(x)=x$ $\forall x$, either $f(x)=-x$ $\forall x$"

It implies : "$\forall x$, either $f(x)=x$, either $f(x)=-x$"
Which is quite different. For example $f(x)=|x|$ is such that $f(x)^2=x^2$

So you miss some work again before conclusion.
\end{solution}
*******************************************************************************
-------------------------------------------------------------------------------

\begin{problem}[Posted by \href{https://artofproblemsolving.com/community/user/125553}{lehungvietbao}]
	Determine all real numbers $a$ such that the function $f(x)=\{ax+\sin x\}$ is periodic. Here $\{y\}$ denotes the fractional part of $y$.
	\flushright \href{https://artofproblemsolving.com/community/c6h566090}{(Link to AoPS)}
\end{problem}



\begin{solution}[by \href{https://artofproblemsolving.com/community/user/29428}{pco}]
	\begin{tcolorbox}Determine all real numbers $a$ such that the function $f(x)=\{ax+\sin x\}$ is periodic. Here $\{y\}$ denotes the fractional part of $y$.\end{tcolorbox}
Let $t$ one period and we get $a(x+t)+\sin(x+t)-ax-\sin x\in\mathbb Z$ $\forall x$

So, since continuous, $\sin(x+t)-\sin x=c$ constant and so $t=2k\pi$ and so $\boxed{a=\frac{\alpha}{\pi}}$, whatever is $\alpha\in\mathbb Q$
\end{solution}
*******************************************************************************
-------------------------------------------------------------------------------

\begin{problem}[Posted by \href{https://artofproblemsolving.com/community/user/68025}{Pirkuliyev Rovsen}]
	Show that there is no continuous function $f: \mathbb{R}\to\mathbb{R}$ satisfyng $f(x-f(x))=\frac{x}{2}$ for all real numbers $x$.
	\flushright \href{https://artofproblemsolving.com/community/c6h569283}{(Link to AoPS)}
\end{problem}



\begin{solution}[by \href{https://artofproblemsolving.com/community/user/29428}{pco}]
	\begin{tcolorbox}Show that there is no continuous function $f: \mathbb{R}\to\mathbb{R}$ satisfyng $f(x-f(x))=\frac{x}{2}$ for all real numbers $x$.\end{tcolorbox}
Setting $g(x)=x-f(x)$, we get the equation $g(g(x))=g(x)-\frac x2$

$g(x)$ is injective and so, since continuous, monotonous. So $g(g(x))$ is increasing and so $g(x)=g(g(x))+\frac x2$ is increasing too.
So $g(x)$ is an increasing continuous injection from $\mathbb R\to\mathbb R$

Let $x\in\mathbb R^*$ and the sequence $a_n$ defined as :
$a_0=x$
$a_1=g(x)$
$a_{n+2}=a_{n+1}-\frac 12a_n$
And so $a_n=\alpha 2^{-\frac n2}\cos(n\frac{\pi}4+\varphi)$ for some $\alpha,\varphi$ depending on $x$ and $f(x)$.

So $A_n=\frac{a_{n+2}-a_{n+1}}{a_{n+1}-a_n}$ $=\frac 1{\sqrt 2}\frac{\sin(\frac{(n+2)\pi}4+\varphi)}{\sin(\frac{(n+1)\pi}4+\varphi)}$

But $A_n=\frac{g(g^{n+1}(x))-g(g^n(x))}{g^{n+1}(x)-g^n(x)}\ge 0$ (since $g(x)$ is increasing) while $\frac{\sin(\frac{(n+2)\pi}4+\varphi)}{\sin(\frac{(n+1)\pi}4+\varphi)}$ does not have a constant sign.
Hence contradiction.

And so no such function.
\end{solution}
*******************************************************************************
-------------------------------------------------------------------------------

\begin{problem}[Posted by \href{https://artofproblemsolving.com/community/user/146467}{changpotato}]
	Find all $ f: R \to R $ such that $ f(f(x)-y) = 2x - f(x+y)$    $ \forall x,y$
	\flushright \href{https://artofproblemsolving.com/community/c6h569288}{(Link to AoPS)}
\end{problem}



\begin{solution}[by \href{https://artofproblemsolving.com/community/user/125513}{hal9v4ik}]
	put $y=-x$ and $y=f(x)$ and get $f(x)=x$
\end{solution}



\begin{solution}[by \href{https://artofproblemsolving.com/community/user/146467}{changpotato}]
	\begin{tcolorbox}put $y=-x$ and $y=f(x)$ and get $f(x)=x$\end{tcolorbox}
but $f(x)=-2x$ also works
\end{solution}



\begin{solution}[by \href{https://artofproblemsolving.com/community/user/29428}{pco}]
	In fact, without some constraints more (continuity, monotonic, ...), and assuming axiom of choice, there are infinitely many solutions.
It's strange that such a problem was submitted in a real olympiad contest :?:
\end{solution}



\begin{solution}[by \href{https://artofproblemsolving.com/community/user/29428}{pco}]
	\begin{tcolorbox}Find all $ f: R \to R $ such that $ f(f(x)-y) = 2x - f(x+y)$    $ \forall x,y$\end{tcolorbox}
1) Claim about general solution :
=====================
Let $A,B$ two supplementary subvectorspaces of the $\mathbb Q$-vectorspace $\mathbb R$
Let $t\in B$
Let $a(x)$ from $\mathbb R\to A$ and $b(x)$ from $\mathbb R\to B$ the two projections of $x$ in $A,B$ (so that $x=a(x)+b(x)$)

Then $\boxed{f(x)=a(x)-2b(x)+t}$

2) Proof that any function in the form of 1) is indeed a solution
==========================================
Just a trivial check :
$f(x)-y=(a(x)-a(y))+(t-2b(x)-b(y))$
$f(f(x)-y)=(a(x)-a(y))-2(t-2b(x)-b(y))+t$ $=a(x)-a(y)+4b(x)+2b(y)-t$
$f(x+y)=(a(x)+a(y))+(t-2b(x)-2b(y))$
$2x-f(x+y)=2a(x)+2b(x)-(a(x)+a(y))-(t-2b(x)-2b(y))$ $=a(x)-a(y)+4b(x)+2b(y)-t$

And so $f(f(x)-y)=2x-f(x+y)$
Q.E.D.

3) Proof that any solution may be written in the form of 1), so that this indeed is a general solution
====================================================================
Let $f(x)$ such that $f(f(x)-y)=2x-f(x+y)$
Let $t=f(0)$
Let $g(x)=f(x)-t$
Let $P(x,y)$ be the assertion $f(f(x)-y)=2x-f(x+y)$

$P(0,0)$ $\implies$ $f(t)=-t$ and so $g(t)=-2t$

$P(x,0)$ $\implies$  $f(f(x))+f(x)=2x$ and so $f(x)+x$ is surjective

Subtracting $P(x,-x)$ from $P(x,y-x)$ gives $f(f(x)+x-y)=t+f(f(x)+x)-f(y)$ and so, since $f(x)+x$ is surjective :
New assertion $Q(x,y)$ : $g(x-y)=g(x)-g(y)$ 
Subtracting $Q(0,y)$ from $Q(x,-y)$, we get
New assertion $R(x,y)$ : $g(x+y)=g(x)+g(y)$

$P(x,y)$ becomes then $g(g(x))=2x-g(x)$

Let $A=\{x$ such that $g(x)=x\}$ : $A$ is a $\mathbb Q$-vectorspace.
Let $B=\{x$ such that $g(x)=-2x\}$ : $B$ is a $\mathbb Q$-vectorspace. Notice that $t\in B$
$A\cap B=\{0\}$
Let $a(x)=\frac {2x+g(x)}3$ and $b(x)=\frac{x-g(x)}3$

$a(x)$ is linear and $g(a(x))=g(\frac {2x+g(x)}3)=\frac{2g(x)+g(g(x))}3$ $=\frac{2x+g(x)}3=a(x)$ and so $a(x)\in A$

$b(x)$ is linear and $g(b(x))=g(\frac{x-g(x)}3)=\frac{g(x)-g(g(x))}3$ $=-2\frac{x-g(x)}3=-2b(x)$ and so $b(x)\in B$

And $x=a(x)+b(x)$ so that $A,B$ are indeed supplementary vectorspaces.

And $g(x)=a(x)-2b(x)$
And so $f(x)=a(x)-2b(x)+t$

Q.E.D.

4) Trivial solutions
============
4.1) $A=\mathbb R$ and $B=\{0\}$
----------------------------------
So $a(x)=x$ and $b(x)=0$ and $t=0$ and solution $\boxed{f(x)=x}$

4.2) $A=\{0\}$ and $B=\mathbb R$
-----------------------------------
So $a(x)=0$ and $b(x)=x$ and $t$ is any real we want and so $\boxed{f(x)=t-2x}$
\end{solution}
*******************************************************************************
-------------------------------------------------------------------------------

\begin{problem}[Posted by \href{https://artofproblemsolving.com/community/user/68025}{Pirkuliyev Rovsen}]
	Function ${{f: \mathbb[0;1]}\to\mathbb[0;1]}$ is continuous and satisfies the condition $f(x){\in}{\sin{f(x)},f(\sin{x})}$.Prove that the function $f$ constant.(where $x{\in}[0;1]$)
	\flushright \href{https://artofproblemsolving.com/community/c6h569375}{(Link to AoPS)}
\end{problem}



\begin{solution}[by \href{https://artofproblemsolving.com/community/user/29428}{pco}]
	\begin{tcolorbox}Function ${{f: \mathbb[0;1]}\to\mathbb[0;1]}$ is continuous and satisfies the condition $f(x){\in}\{\sin{f(x)},f(\sin{x})\}$.Prove that the function $f$ constant.(where $x{\in}[0;1]$)\end{tcolorbox}
So $\forall x$, either $f(x)=0$, either $f(x)=f(\sin x)$

So $\forall x$, either $f(x)=0$, either $f(x)=f(\sin^{[n]}x)$ (composition of functions)

Setting $n\to+\infty$ and using continuity, we get :
$\forall x$, either $f(x)=0$, either $f(x)=f(0)$

And continuity again implies : 
Either $f(x)=0$ $\forall x$, either $f(x)=f(0)$ $\forall x$

Q.E.D.
\end{solution}
*******************************************************************************
-------------------------------------------------------------------------------

\begin{problem}[Posted by \href{https://artofproblemsolving.com/community/user/125553}{lehungvietbao}]
	Find all functions $f: \mathbb{Z}\to\mathbb{Z}$ such that \[f(m+f(n))=n+f(m+b)\quad \forall m,n\in\mathbb Z\]
Where $b$ is a given real number.
	\flushright \href{https://artofproblemsolving.com/community/c6h569379}{(Link to AoPS)}
\end{problem}



\begin{solution}[by \href{https://artofproblemsolving.com/community/user/29428}{pco}]
	\begin{tcolorbox}Find all functions $f: \mathbb{Z}\to\mathbb{Z}$ such that \[f(m+f(n))=n+f(m+b)\quad \forall m,n\in\mathbb Z\]
Where $b$ is a given real number.\end{tcolorbox}
Obviously, if $b\in\mathbb R\setminus\mathbb Z$ : no solution.
So we'll consider from now that $b\in\mathbb Z$

$f(x)$ is a surjection and so let $u$ such that $f(u)=0$ and let $v$ such that $f(v)=1$ 

Subtracting $P(x,u)$ from $P(x,v)$, we get $f(x+1)=f(x)+v-u$ and so $f(x)=(v-u)x+f(0)$

Plugging this in equation, we get two solutions : 
$\boxed{f(x)=x+b}$ $\forall x\in\mathbb Z$ and $\boxed{f(x)=-x+b}$ $\forall x\in\mathbb Z$
\end{solution}
*******************************************************************************
-------------------------------------------------------------------------------

\begin{problem}[Posted by \href{https://artofproblemsolving.com/community/user/125553}{lehungvietbao}]
	Find all functions $f: \mathbb{Q}^+\to\mathbb{Q}^+$ satisfy all the following the conditions
a) $f(x+1)=f(x)+1$
b) $(f(x))^2=f(x^2)$
for all $x,y\in\mathbb{Q}^+$
	\flushright \href{https://artofproblemsolving.com/community/c6h569381}{(Link to AoPS)}
\end{problem}



\begin{solution}[by \href{https://artofproblemsolving.com/community/user/29428}{pco}]
	\begin{tcolorbox}Find all functions $f: \mathbb{Q}^+\to\mathbb{Q}^+$ satisfy all the following the conditions
a) $f(x+1)=f(x)+1$
b) $(f(x))^2=f(x^2)$
for all $x,y\in\mathbb{Q}^+$\end{tcolorbox}
Setting $x=1$ in $b)$, we get $f(1)=1$ and so, using $a)$ $f(n)=n$ $\forall n\in \mathbb N$ and $f(x+n)=f(x)+n$

Then $f((\frac pq+q)^2)=f((\frac pq)^2+2p+q^2)$ $=f((\frac pq)^2)+2p+q^2=f(\frac pq)^2+2p+q^2$

But $f((\frac pq+q)^2)=(f(\frac pq+q))^2$ $=(f(\frac pq)+q)^2=f(\frac pq)^2+2qf(\frac pq)+q^2$

And so $f(\frac pq)=\frac pq$ and so $\boxed{f(x)=x}$ $\forall x\in\mathbb Q^+$, which indeed is a solution.
\end{solution}
*******************************************************************************
-------------------------------------------------------------------------------

\begin{problem}[Posted by \href{https://artofproblemsolving.com/community/user/125553}{lehungvietbao}]
	Find a function $f: \mathbb{N}^*\to\mathbb{N}^*$ satisfy all the following the conditions
a) $f(mf(n))=n^{2}f(mn)\quad \forall m,n\in\mathbb{N}^*$
b) If $p$ is a prime then $f(p)$ is a composite number but  it isn't a perfect square.

Where $\mathbb{N}^*$ is set of positive integers.
	\flushright \href{https://artofproblemsolving.com/community/c6h569383}{(Link to AoPS)}
\end{problem}



\begin{solution}[by \href{https://artofproblemsolving.com/community/user/29428}{pco}]
	\begin{tcolorbox}Find a function $f: \mathbb{N}^*\to\mathbb{N}^*$ satisfy all the following the conditions
a) $f(mf(n))=n^{2}f(mn)\quad \forall m,n\in\mathbb{N}^*$
b) If $p$ is a prime then $f(p)$ is a composite number but  it isn't a perfect square.

Where $\mathbb{N}^*$ is set of positive integers.\end{tcolorbox}
Let $p_n$ be the $n^{\text{th}}$ prime number (with $p_1=2$).

One such function (amongst infinitely many) could be :

$f(1)=1$

$\forall n\in\mathbb N$ :  $f(p_{3n-2})=p_{3n-1}p_{3n}$ and $f(p_{3n-1})=p_{3n-2}p_{3n}$ and $f(p_{3n})=p_{3n-2}p_{3n-1}$

$f(\prod p_i^{n_i})=\prod f(p_i)^{n_i}$ (multiplicative function)
\end{solution}
*******************************************************************************
-------------------------------------------------------------------------------

\begin{problem}[Posted by \href{https://artofproblemsolving.com/community/user/125553}{lehungvietbao}]
	$f: \mathbb{N}^*\to\mathbb{N}^*$ satisfy all the following the conditions:
a) $f$ is strictly increasing
b) $f(f(n))=3n \quad \forall n\in\mathbb{N}^*$
Calculate $f(2004)$

Where $\mathbb{N}^*$ is set of positive integer.
	\flushright \href{https://artofproblemsolving.com/community/c6h569384}{(Link to AoPS)}
\end{problem}



\begin{solution}[by \href{https://artofproblemsolving.com/community/user/29428}{pco}]
	\begin{tcolorbox}$f: \mathbb{N}^*\to\mathbb{N}^*$ satisfy all the following the conditions:
a) $f$ is strictly increasing
b) $f(f(n))=3n \quad \forall n\in\mathbb{N}^*$
Calculate $f(2004)$

Where $\mathbb{N}^*$ is set of positive integer.\end{tcolorbox}
$f(3n)=3f(n)$

If $f(1)>3$, then $3=f(f(1))\ge f(f(1)-3)+3$ and so $f(f(1)-3)\le 0$, impossible. So $f(1)\le 3$
If $f(1)=1$, then $3=f(f(1))=1$, impossible.
If $f(1)=3$, then $f(1)=3=f(f(1))$ and so, since injective, $f(1)=1$, impossible.
So $f(1)=2$ and $f(3^n)=2.3^n$
So $f(2)=3$ and $f(2.3^n)=3^{n+1}$

Since we have exacty $3^n-1$ values in $(2.3^n,3^{n+1})$ for exactly $3^n-1$ args (in $(3^n,2.3^n)$), we get :
$f(3^n+u)=2.3^n+u$ $\forall u\in[0,3^n]$

So, using base $3$ representation : $f(\overline{1\text{--anything--}}_3=\overline{2\text{--same thing--}}_3$

And then $f(2.3^n+u)=3(3^n+u)$ and $f(\overline{2\text{--anything--}}_3=\overline{1\text{--same thing--}0}_3$

And so $f(2004)=f(\overline{2202020}_3)=\overline{12020200}_3=\boxed{3825}$
\end{solution}
*******************************************************************************
-------------------------------------------------------------------------------

\begin{problem}[Posted by \href{https://artofproblemsolving.com/community/user/125553}{lehungvietbao}]
	Let $a\in\mathbb R$. Find all functions  $f(x,y)$ defined on $\mathbb R$ such that \[\begin{cases}f(x+2,y)=f(x,y+2)=f(x,y)\\af(x,y)+f(x,y+1)+f(x+1,y)=x+y\end{cases}\quad \forall x,y\in\mathbb R\]
	\flushright \href{https://artofproblemsolving.com/community/c6h569389}{(Link to AoPS)}
\end{problem}



\begin{solution}[by \href{https://artofproblemsolving.com/community/user/29428}{pco}]
	\begin{tcolorbox}Let $a\in\mathbb R$. Find all functions  $f(x,y)$ defined on $\mathbb R$ such that \[\begin{cases}f(x+2,y)=f(x,y+2)=f(x,y)\\af(x,y)+f(x,y+1)+f(x+1,y)=x+y\end{cases}\quad \forall x,y\in\mathbb R\]\end{tcolorbox}
Let $P(x,y)$ be the assertion $af(x,y)+f(x,y+1)+f(x+1,y)=x+y$

1) If $a\ne 0$, no solution.

$P(x,y+1)$ $\implies$ $af(x,y+1)+f(x,y)+f(x+1,y+1)=x+y+1$
$P(x+1,y)$ $\implies$ $af(x+1,y)+f(x+1,y+1)+f(x,y)=x+y+1$
So $f(x,y+1)=f(x+1,y)$ and so $f(x,y)=f(x,y+2)=f(x+1,y+1)$

$P(x,y+1)$ $\implies$ $af(x,y+1)+2f(x,y)=x+y+1$
$P(x,y+2)$ $\implies$ $af(x,y)+2f(x,y+1)=x+y+2$

So (eliminating $f(x,y+1)$) : $(a^2-4)f(x,y)=(a-2)(x+y)+2a-2$
If $a^2=4$ : no solution
If $a^2\ne 4$ : $f(x,y)=\frac{x+y}{a+2}+\frac{2a-2}{a^2-4}$ which is never a solution
Q.E.D.


2) If $a=0$, no solution
$P(x,y+1)$ $\implies$ $f(x,y)+f(x+1,y+1)=x+y+1$
$P(x+1,y+2)$ $\implies$ $f(x+1,y+1)+f(x,y)=x+y+3$
Impossible (subtracting leads to contradiction)
So no solution.
\end{solution}
*******************************************************************************
-------------------------------------------------------------------------------

\begin{problem}[Posted by \href{https://artofproblemsolving.com/community/user/68025}{Pirkuliyev Rovsen}]
	Determine all function $f: \mathbb{R}\to\mathbb{R}$ such that $f(x-\frac{b}{2a})+\frac{b}{2}{\leq}ax+b{\leq}f(x+\frac{b}{2a})-\frac{b}{2}$ for all $x{\in}R$,$a,b{{\in}R, a{\neq}0,b{\neq}0}$.
	\flushright \href{https://artofproblemsolving.com/community/c6h569438}{(Link to AoPS)}
\end{problem}



\begin{solution}[by \href{https://artofproblemsolving.com/community/user/29428}{pco}]
	\begin{tcolorbox}Determine all function $f: \mathbb{R}\to\mathbb{R}$ such that $f(x-\frac{b}{2a})+\frac{b}{2}{\leq}ax+b{\leq}f(x+\frac{b}{2a})-\frac{b}{2}$ for all $x{\in}R$,$a,b{{\in}R, a{\neq}0,b{\neq}0}$.\end{tcolorbox}
$f(x-\frac b{2a})+\frac b2\le ax+b$ $\iff$ $f(x)+\frac b2\le a(x+\frac b{2a})+b$ $=ax+3\frac b2$ $\iff$ $f(x)\le ax+b$

$ax+b\le f(x+\frac b{2a})-\frac b2$ $\iff$ $ax+\frac b2=a(x-\frac b{2a})+b\le f(x)-\frac b2$ $\iff$ $ax+b\le f(x)$

And so $\boxed{f(x)=ax+b}$ $\forall x$
\end{solution}
*******************************************************************************
-------------------------------------------------------------------------------

\begin{problem}[Posted by \href{https://artofproblemsolving.com/community/user/125553}{lehungvietbao}]
	Let $a\in\mathbb R$. Find all functions $f$ defined on $\mathbb R$ such that \[\begin{cases}f(x+5)=f(x)\\af(x)+f(x+3)+f(x+4)=x^2+x+1\end{cases}\quad \forall x\in\mathbb R\]
	\flushright \href{https://artofproblemsolving.com/community/c6h569514}{(Link to AoPS)}
\end{problem}



\begin{solution}[by \href{https://artofproblemsolving.com/community/user/89198}{chaotic_iak}]
	Let $P(x) : af(x) + f(x+3) + f(x+4) = x^2 + x + 1$.

$P(9) + P(7) + P(5) + P(3) + P(1)$
$\implies a(f(9) + f(7) + f(5) + f(3) + f(1)) + (f(13)+f(12)+\ldots+f(4)) = (9^2+7^2+5^2+3^2+1^2) + (9+7+5+3+1) + 5$
$\implies a(f(4) + f(2) + f(0) + f(3) + f(1)) + 2(f(0)+f(1)+f(2)+f(3)+f(4)) = (9^2+7^2+5^2+3^2+1^2) + (9+7+5+3+1) + 5$
$\implies (a+2) (f(0)+f(1)+f(2)+f(3)+f(4)) = (9^2+7^2+5^2+3^2+1^2) + (9+7+5+3+1) + 5$

$P(8) + P(6) + P(4) + P(2) + P(0)$
$\implies a(f(8) + f(6) + f(4) + f(2) + f(0)) + (f(12)+f(11)+\ldots+f(3)) = (8^2+6^2+4^2+2^2+0^2) + (8+6+4+2+0) + 5$
$\implies a(f(3) + f(1) + f(4) + f(2) + f(0)) + 2(f(0)+f(1)+f(2)+f(3)+f(4)) = (8^2+6^2+4^2+2^2+0^2) + (8+6+4+2+0) + 5$
$\implies (a+2) (f(0)+f(1)+f(2)+f(3)+f(4)) = (8^2+6^2+4^2+2^2+0^2) + (8+6+4+2+0) + 5$
$\implies (9^2+7^2+5^2+3^2+1^2) + (9+7+5+3+1) + 5 = (8^2+6^2+4^2+2^2+0^2) + (8+6+4+2+0) + 5$

Contradiction, so no $f$ exists.

(Edit: The LaTeX is derping. Too long apparently.)
\end{solution}



\begin{solution}[by \href{https://artofproblemsolving.com/community/user/29428}{pco}]
	\begin{tcolorbox}Let $a\in\mathbb R$. Find all functions $f$ defined on $\mathbb R$ such that \[\begin{cases}f(x+5)=f(x)\\af(x)+f(x+3)+f(x+4)=x^2+x+1\end{cases}\quad \forall x\in\mathbb R\]\end{tcolorbox}
Set $x\to x+5$ in second equation and you get $x^2+x+1=(x+5)^2+(x+5)+1$ $\forall x$, which is wrong.
So no solution for this functional equation.
\end{solution}
*******************************************************************************
-------------------------------------------------------------------------------

\begin{problem}[Posted by \href{https://artofproblemsolving.com/community/user/125553}{lehungvietbao}]
	Let $h(x)$ be a given function defined  on  $\mathbb R$. Find all functions $f$ defined on $\mathbb R$ such that \[\begin{cases}f(8x)=f(27x)\\f(4x)+f(9x)-2f(6x)=h(x)\end{cases}\quad \forall x\in\mathbb R\]
	\flushright \href{https://artofproblemsolving.com/community/c6h569515}{(Link to AoPS)}
\end{problem}



\begin{solution}[by \href{https://artofproblemsolving.com/community/user/29428}{pco}]
	\begin{tcolorbox}Let $h(x)$ be a given function defined  on  $\mathbb R$. Find all functions $f$ defined on $\mathbb R$ such that \[\begin{cases}f(8x)=f(27x)\\f(4x)+f(9x)-2f(6x)=h(x)\end{cases}\quad \forall x\in\mathbb R\]\end{tcolorbox}
Let $a=\frac 32$ and $g(x)=h(\frac x4)$ so that system is :
$f(a^3x)=f(x)$
$f(x)-2f(ax)+f(a^2x)=g(x)$

Replacing $x$ with $ax$ and then $a^2x$ in the second equation, we get :
$f(x)-2f(ax)+f(a^2x)=g(x)$
$f(ax)-2f(a^2x)+f(x)=g(ax)$
$f(a^2x)-2f(x)+f(ax)=g(a^2x)$

If $g(x)+g(ax)+g(a^2x)\ne 0$ for some $x$, then no solution.
If $g(x)+g(ax)+g(a^2x)=0$ $\forall x$, then we immediately get $f(x)=\frac{g(x)+g(ax)}3+u(x)$ where $u(ax)=u(x)$ $\forall x$

And this last equation is quite trivial :

$u(0)=0$
$u(x)=v(\left\{\frac{\ln x}{\ln a}\right\})$ $\forall x>0$ where $v(x)$ is any function from $[0,1)\to\mathbb R$
$u(x)=w(\left\{\frac{\ln -x}{\ln a}\right\})$ $\forall x<0$ where $w(x)$ is any function from $[0,1)\to\mathbb R$
\end{solution}
*******************************************************************************
-------------------------------------------------------------------------------

\begin{problem}[Posted by \href{https://artofproblemsolving.com/community/user/125553}{lehungvietbao}]
	Let $a,\alpha \in\mathbb R;a\neq 0$. Find all functions $f:\mathbb R\to \mathbb R$ such that \[\begin{cases}f(x+3a)=f(x)f(x+2a)=\alpha f(x+a)+(1-\alpha)f(x)\end{cases}\quad \forall x\in\mathbb R\]
	\flushright \href{https://artofproblemsolving.com/community/c6h569516}{(Link to AoPS)}
\end{problem}



\begin{solution}[by \href{https://artofproblemsolving.com/community/user/29428}{pco}]
	\begin{tcolorbox}Let $a,\alpha \in\mathbb R;a\neq 0$. Find all functions $f:\mathbb R\to \mathbb R$ such that \[\begin{cases}f(x+3a)=f(x)\\f(x+2a)=\alpha f(x+a)+(1-\alpha)f(x)\end{cases}\quad \forall x\in\mathbb R\]\end{tcolorbox}
A lot of the problems for which you  require help are quite similar. You should learn from the first help we gave you and be able to solve yourself the following problems. Are you simply trying ???? :?:

Replacing $x$ by $x+a$ and then $x+2a$ in second equation, we get :

$(1-\alpha)f(x)+\alpha f(x+a)-f(x+2a)=0$
$(1-\alpha)f(x+a)+\alpha f(x+2a)-f(x)=0$
$(1-\alpha)f(x+2a)+\alpha f(x)-f(x+a)=0$

And so $f(x+a)=f(x)$ and general solution $\boxed{f(x)=u(\left\{\frac xa\right\})}$ $\forall x$ where $u(x)$ is any function from $[0,1)\to\mathbb R$
\end{solution}
*******************************************************************************
-------------------------------------------------------------------------------

\begin{problem}[Posted by \href{https://artofproblemsolving.com/community/user/68025}{Pirkuliyev Rovsen}]
	Prove there is no a bijection $f: \mathbb{N}\to\mathbb{N}$ such that $f(mn)=f(m)+f(n)+3f(m)f(n)$.
	\flushright \href{https://artofproblemsolving.com/community/c6h569544}{(Link to AoPS)}
\end{problem}



\begin{solution}[by \href{https://artofproblemsolving.com/community/user/29428}{pco}]
	\begin{tcolorbox}Prove there is no a bijection $f: \mathbb{N}\to\mathbb{N}$ such that $f(mn)=f(m)+f(n)+3f(m)f(n)$.\end{tcolorbox}
Set $m=n=1$ and you get $f(1)=0\notin \mathbb N$ or $f(1)=-\frac 13\notin\mathbb N$
Hence the result.
\end{solution}
*******************************************************************************
-------------------------------------------------------------------------------

\begin{problem}[Posted by \href{https://artofproblemsolving.com/community/user/68025}{Pirkuliyev Rovsen}]
	Determine all function $f: \mathbb{R}\to\mathbb{R}$ such that $f^2(x)+f(x)f(y)+f^2(y)=\frac{a^x-a^y}{f(x)-f(y)}$ ,where $x,y{\in}R$,$a{\in}R_+{\setminus}{1}$, $x{\neq}y$, $f(0)=1$
	\flushright \href{https://artofproblemsolving.com/community/c6h569607}{(Link to AoPS)}
\end{problem}



\begin{solution}[by \href{https://artofproblemsolving.com/community/user/29428}{pco}]
	\begin{tcolorbox}Determine all function $f: \mathbb{R}\to\mathbb{R}$ such that $f^2(x)+f(x)f(y)+f^2(y)=\frac{a^x-a^y}{f(x)-f(y)}$ ,where $x,y{\in}R$,$a{\in}R_+{\setminus}{1}$, $x{\neq}y$, $f(0)=1$\end{tcolorbox}
So $f(x)^3-f(0)^3=a^x-1$ $\forall x\ne 0$

So $f(x)=a^{\frac x3}$ $\forall x\ne 0$

So $\boxed{f(x)=a^{\frac x3}}$ $\forall x$, which indeed is a solution.
\end{solution}
*******************************************************************************
-------------------------------------------------------------------------------

\begin{problem}[Posted by \href{https://artofproblemsolving.com/community/user/171969}{mehrdad1st}]
	Find all functions N-->N such that
\[n\geq2 : f(n)=f(f(n-1))+f(f(n+1))\]
	\flushright \href{https://artofproblemsolving.com/community/c6h569748}{(Link to AoPS)}
\end{problem}



\begin{solution}[by \href{https://artofproblemsolving.com/community/user/29428}{pco}]
	\begin{tcolorbox}Find all functions N-->N such that
\[n\geq2 : f(n)=f(f(n-1))+f(f(n+1))\]\end{tcolorbox}
Without any precision, I suppose that, as usual on this site, $\mathbb N$ is the set of positive integers ($0\notin\mathbb N$)

Let $P(x)$ be the assertion $f(x)=f(f(x-1))+f(f(x+1))$
Note that $f(x)=c$ constant is not a solution. So $|f(\mathbb N)|\ge 2$

Let $a_1=\min(f(\mathbb N))$ and $a_2=\min(f(\mathbb N)\setminus\{a_1\})$
Let $A_1=f^{-1}(\{a_1\})$

Let $u\in A_1$. If $u\ge 2$, $P(u)$ $\implies$ $f(f(u-1))<a_1$, impossible. So $u=1$ and $A_1=\{1\}$

Let $u\in A_2$. Since $u\ne 1$, $P(u)$ $\implies$ $a_2=f(f(u-1))+f(f(u+1))$ and so $f(f(u-1))<a_2$ and $f(f(u+1))<a_2$
$\implies$ $f(f(u-1))=f(f(u+1))=a_1$
$\implies$ $f(u-1)=f(u+1)=1$
$\implies$ $a_1=1$ and $u-1=u+1=1$, impossible.

So \begin{bolded}no solution\end{underlined}\end{bolded}
\end{solution}



\begin{solution}[by \href{https://artofproblemsolving.com/community/user/171969}{mehrdad1st}]
	But f(n)=n\/2 is one solution
\end{solution}



\begin{solution}[by \href{https://artofproblemsolving.com/community/user/29428}{pco}]
	\begin{tcolorbox}But f(n)=n\/2 is one solution\end{tcolorbox}
Uhhh ? $f(1)=\frac 12\notin\mathbb N$
\end{solution}



\begin{solution}[by \href{https://artofproblemsolving.com/community/user/171969}{mehrdad1st}]
	Sorry for this silly mistake
\end{solution}



\begin{solution}[by \href{https://artofproblemsolving.com/community/user/171969}{mehrdad1st}]
	How about f:N-->R
\end{solution}



\begin{solution}[by \href{https://artofproblemsolving.com/community/user/29428}{pco}]
	\begin{tcolorbox}How about f:N-->R\end{tcolorbox}
Then $f(f(n-1))$ is undefined if $f(n-1)\notin\mathbb N$. So no solution.

What exactly is the problem you got in your olympiad contest ? (and that you claimed to have solved) ?
\end{solution}



\begin{solution}[by \href{https://artofproblemsolving.com/community/user/171969}{mehrdad1st}]
	f:R-->R
this is the last one..
\end{solution}



\begin{solution}[by \href{https://artofproblemsolving.com/community/user/29428}{pco}]
	\begin{tcolorbox}f:R-->R
this is the last one..\end{tcolorbox}
From what olympiad ?
And do you confirm you already solved it and you'll post the solution when required ?

Is the constraint $n\ge 2$ still present ?

Is the function supposed continuous ?
\end{solution}



\begin{solution}[by \href{https://artofproblemsolving.com/community/user/29428}{pco}]
	\begin{tcolorbox}f:R-->R
this is the last one..\end{tcolorbox}
If the problem is "find all functions $f(x)$ from $\mathbb R\to\mathbb R$ such that $f(x)=f(f(x-1))+f(f(x+1))$ $\forall x\in\mathbb R$", then you have infinitely many solutions.

Surely, $f(x)=\frac x2$ is one of them.

But for example, here is a nice continuous periodic solution : $f(x)=\min(\frac{\{x\}}2,\frac{|\sin \pi x|}6)$

And tons of other solutions (and I dont think we can easily find a general form for all of them).
\end{solution}
*******************************************************************************
-------------------------------------------------------------------------------

\begin{problem}[Posted by \href{https://artofproblemsolving.com/community/user/115131}{BakyX}]
	Find all strictly increasing functions $f: \mathbb{N} \to \mathbb{N}$ such that for all $m,n \in \mathbb{N}$ is $f(mn)=f(m)f(n)$ and $f(2)=4$.
	\flushright \href{https://artofproblemsolving.com/community/c6h569869}{(Link to AoPS)}
\end{problem}



\begin{solution}[by \href{https://artofproblemsolving.com/community/user/29428}{pco}]
	\begin{tcolorbox}Find all strictly increasing functions $f: \mathbb{N} \to \mathbb{N}$ such that for all $m,n \in \mathbb{N}$ is $f(mn)=f(m)f(n)$ and $f(2)=4$.\end{tcolorbox}
$f(1)=1$ and so, since stricly increasing, $f(n)>1$ $\forall n>1$
Let $x,y>1$ and $p,q\ge 1$ (so that $f(x),f(y)>1$) :
$\frac{\ln x}{\ln y}>\frac pq$ $\implies$ $x^q>y^p$ $\implies$ $f(x^q)>f(y^p)$ $\implies$ $f(x)^q>f(y)^p$ $\implies$ $\frac{\ln f(x)}{\ln f(y)}>\frac pq$
And so $\frac{\ln f(x)}{\ln f(y)}\ge\frac{\ln x}{\ln y}$

$\frac{\ln x}{\ln y}<\frac pq$ $\implies$ $x^q<y^p$ $\implies$ $f(x^q)<f(y^p)$ $\implies$ $f(x)^q<f(y)^p$ $\implies$ $\frac{\ln f(x)}{\ln f(y)}<\frac pq$
And so $\frac{\ln f(x)}{\ln f(y)}\le\frac{\ln x}{\ln y}$

So $\frac{\ln f(x)}{\ln f(y)}=\frac{\ln x}{\ln y}$ $\forall x,y>1$

Taking then $y=2$ in this equality, we get $\boxed{f(x)=x^2}$ $\forall x\ne 1$, still true when $x=1$ and which indeed is a solution.
\end{solution}
*******************************************************************************
-------------------------------------------------------------------------------

\begin{problem}[Posted by \href{https://artofproblemsolving.com/community/user/68025}{Pirkuliyev Rovsen}]
	$f$ is a function defined on all reals in the interval $[0;1]$ and satisfies $f(0)=0$, $f(\frac{x}{3})=\frac{f(x)}{2}$, $f(1-x)=1-f(x)$.Find $f(\frac{18}{1991})$.
	\flushright \href{https://artofproblemsolving.com/community/c6h569915}{(Link to AoPS)}
\end{problem}



\begin{solution}[by \href{https://artofproblemsolving.com/community/user/29428}{pco}]
	\begin{tcolorbox}$f$ is a function defined on all reals in the interval $[0;1]$ and satisfies $f(0)=0$, $f(\frac{x}{3})=\frac{f(x)}{2}$, $f(1-x)=1-f(x)$.Find $f(\frac{18}{1991})$.\end{tcolorbox}
It seems to me that there are not enough information to uniquely define $f(\frac pq)$ when $q\equiv 5\pmod 6$

Are you sure of the problem statement ?
\end{solution}



\begin{solution}[by \href{https://artofproblemsolving.com/community/user/68025}{Pirkuliyev Rovsen}]
	Conditions of spelled correctly.Maybe error in the source.
\end{solution}



\begin{solution}[by \href{https://artofproblemsolving.com/community/user/16261}{Rust}]
	Because $1991\not =3^k$ we need motonically all continiosly for $f(x)$.
\end{solution}



\begin{solution}[by \href{https://artofproblemsolving.com/community/user/29428}{pco}]
	\begin{tcolorbox}Because $1991\not =3^k$ we need motonically all continiosly for $f(x)$.\end{tcolorbox}
Wrong.
1) No monotonous solution exists

2) $4\ne 3^k$ buf $f(\frac 14)$ may be uniquely defined (conditions are enough) at $f(\frac 14)=\frac 13$
\end{solution}



\begin{solution}[by \href{https://artofproblemsolving.com/community/user/16261}{Rust}]
	\begin{tcolorbox}[quote="Rust"]Because $1991\not =3^k$ we need motonically all continiosly for $f(x)$.\end{tcolorbox}
Wrong. 
1) No monotonous solution exists \end{tcolorbox}
Why?
Let $x=\sum_{k=1}^{\infty} a_k*3^{-k}$. 
Define  
 $f(x)=2^{-k}$, if $3^{-k}\le x\le 3^{1-k}-3^{-k}$ and $f(x)=2^{-k}+2^{-k-1}$ if $3^{1-k}-3^{-k}<x<3^{-k}$
and  $f(1-x)=1-f(x)$.
This function monotonous and sftisfy conditions.

\begin{tcolorbox}
2) $4\ne 3^k$ buf $f(\frac 14)$ may be uniquely defined (conditions are enough) at $f(\frac 14)=\frac 13$\end{tcolorbox}
Are you shure?
\end{solution}



\begin{solution}[by \href{https://artofproblemsolving.com/community/user/29428}{pco}]
	About monotonicity : :oops:, I'm not so sure. I'll check your example.

About $f(\frac 14)$, I'm sure : $1=f(\frac 34)+f(\frac 14)=f(\frac 34)+\frac{f(\frac 34)}2$ and so $f(\frac 14)=\frac 13$ and $f(\frac 34)=\frac 23$
\end{solution}



\begin{solution}[by \href{https://artofproblemsolving.com/community/user/29428}{pco}]
	About monoticity, I'm sorry, you're right. 

See http://mathworld.wolfram.com\/CantorFunction.html (your example)

And there is surely a missing "increasing" in the OP statement.
\end{solution}



\begin{solution}[by \href{https://artofproblemsolving.com/community/user/16261}{Rust}]
	All solutions.
Let $\phi(x):(\frac 29,\frac 23]\to R$ any function with condition $\phi(\frac 13)=\phi(\frac 23)=\frac 12,$
and for $x<\frac 13$, suth that $3^kx=1-x$ or $x=\frac{1}{1+3^k}, k\ge 1$ must be $\phi(x)=\frac{1}{1+2^k}$. Only one of them in interval $(\frac 29, \frac 14)$ (case $k=1$). 
But must be $\phi(\frac{3^{k-1}}{1+3^k})=\frac{2^{k-1}}{1+2^k}$, because $\frac{3^{k-1}}{1+3^k}\in(\frac 29, \frac 13).$

Define $f(x):[0,1]\to R$ as \[f(0)=0, f(x)=2^{1-k}\phi(3^{k-1}x) \ if \ x\in (\frac{2}{3^{k+1}},\frac{2}{3^k}], \ f(x)=1-f(x), if \ x>\frac 23.\]
This definition give all solutions.
If $\phi(x)$ monotic in $(\frac 29,\frac 13],  \phi(\frac 13)=\frac 12 =\phi(x), x\in [\frac 13,\frac 23]$, then $f(x)$ monotoniosly.
There are infinetely many monotoniosly solutions.
\end{solution}



\begin{solution}[by \href{https://artofproblemsolving.com/community/user/29428}{pco}]
	\begin{tcolorbox}...
There are infinetely many monotoniosly solutions.\end{tcolorbox}
My previous link (mathworld) says that Cantor devil staircase is the unique solution of this problem when adding "increasing" property. So ... it seems that there are not infinitely many monotonous solutions .
\end{solution}



\begin{solution}[by \href{https://artofproblemsolving.com/community/user/16261}{Rust}]
	We find unique value for suth rational $x=x_0=x_l$, were $x_i=1-\frac{x_{i-1}}{3^{k_i}}$.
For suth $x$ $f(x_i)=1-\frac{1}{2^{k_i}}$.
If suth $x$ density in interval $(\frac 29,\frac 13]$ then monotonically solution is unique.
I think, that you are right. suth rational numbers are density.
If $\frac{18}{1991}$ is suth x solution is unique visout monotoniosly condition.

\[\frac{18}{1991}=\frac{A}{3^{45}-1}, A=\frac{18*(3^{45}-1)}{1991}\in N.\]
$A=3^{k_1+...k_l}-3^{k_1+...k_{l-1}}+...-3^2$.
It give $f(\frac{18}{1991})=\frac{2^{k_1+...+k_l}-2^{k_1+...+k_{l-1}}+...-2^2}{2^{45}-1}$.
\end{solution}
*******************************************************************************
-------------------------------------------------------------------------------

\begin{problem}[Posted by \href{https://artofproblemsolving.com/community/user/171969}{mehrdad1st}]
	Find all functions N--->N-{1} such that 
\[f(n) +f(n+1)=f(n+2)f(n+3)-168\]
	\flushright \href{https://artofproblemsolving.com/community/c6h570003}{(Link to AoPS)}
\end{problem}



\begin{solution}[by \href{https://artofproblemsolving.com/community/user/29428}{pco}]
	\begin{tcolorbox}Find all functions N--->N-{1} such that 
\[f(n) +f(n+1)=f(n+2)f(n+3)-168\]\end{tcolorbox}
$f(n)+f(n+1)=f(n+2)f(n+3)-168$
$f(n+1)+f(n+2)=f(n+3)f(n+4)-168$
Subtracting the first from the second, we get $f(n+2)-f(n)=f(n+3)((f(n+4)-f(n+2))$ $\forall n$

And so, since $f(n)\ne 1$, $f(n+2)=f(n)$ $\forall n$ and $f(\mathbb N)$ is $a,b,a,b,...$ with $a+b=ab-168$
$\iff$ $(a-1)(b-1)=169$ and so $(a,b)\in\{(2,170),(14,14),(170,2)\}$

\begin{bolded}Hence the three solutions\end{underlined}\end{bolded} :

S1 : $f(n)=14$ $\forall n$
S2 : $f(2n)=2$ and $f(2n+1)=170$
S3 : $f(2n)=170$ and $f(2n+1)=2$
\end{solution}



\begin{solution}[by \href{https://artofproblemsolving.com/community/user/156523}{dizzy}]
	\begin{tcolorbox}[quote="mehrdad1st"]Find all functions N--->N-{1} such that 
\[f(n) +f(n+1)=f(n+2)f(n+3)-168\]\end{tcolorbox}


And so, since $f(n)\ne 1$, $f(n+2)=f(n)$ $\forall n$ \end{tcolorbox}

\begin{bolded}pco\end{bolded}, Could you please explain how did you get this?
\end{solution}



\begin{solution}[by \href{https://artofproblemsolving.com/community/user/29428}{pco}]
	\begin{tcolorbox}[quote="pco"]And so, since $f(n)\ne 1$, $f(n+2)=f(n)$ $\forall n$ \end{tcolorbox}

\begin{bolded}pco\end{bolded}, Could you please explain how did you get this?\end{tcolorbox}
Because we got $|f(n+2)-f(n)|\ge 2|f(n+4)-f(n+2)|$ and so $|f(n+2)-f(n)|\ge 2^k|f(n+2k)-f(n+2k-2)|$

So $|f(n+2k)-f(n+2k-2)|\le 2^{-k}|f(n+2)-f(n)|$ $\forall k$ and so $|f(n+2k)-f(n+2k-2)|=0$ and so $f(n+2)-f(n)=0$

(in fact, consider that if none of $f(n+2k)-f(n+2k-2)$ is zero, $|f(n+2)-f(n)|$ is infinite, while if one is zero, then $|f(n+2)-f(n)|$ is zero).
\end{solution}
*******************************************************************************
-------------------------------------------------------------------------------

\begin{problem}[Posted by \href{https://artofproblemsolving.com/community/user/201392}{nima-amini}]
	find all continuouse function \[f:\mathbb{R}\rightarrow \mathbb{R}\]
\[f(1-x)=1-f(f(x))\]
	\flushright \href{https://artofproblemsolving.com/community/c6h570139}{(Link to AoPS)}
\end{problem}



\begin{solution}[by \href{https://artofproblemsolving.com/community/user/29428}{pco}]
	\begin{tcolorbox}find all function \[f:\mathbb{R}\rightarrow \mathbb{R}\]
\[f(1-x)=1-f(f(x))\]\end{tcolorbox}
What is "amir" ?. Is it a real olympiad exercise for which you are sure that there exists olympiad level solutions ?

At first glance, there exists infinitely many solutions and I dont think that there exists a general form covering all of them.

Some examples :

1) $f(x)=\frac 12$ $\forall x$

2) $f(x)=x$ $\forall x$

3) Let $A\subseteq\mathbb R$ and $B=A\cup\{\pi\}$
$f(\frac 12)=\frac 12$
$\forall x>\frac 12$, $f(x)=1-\pi+(2\pi-1)1_B(x)$
$\forall x<\frac 12$, $f(x)=\pi-(2\pi-1)1_B(1-x)$

And tons of other strange examples.
\end{solution}



\begin{solution}[by \href{https://artofproblemsolving.com/community/user/201392}{nima-amini}]
	\begin{tcolorbox}[quote="nima-amini"]find all function \[f:\mathbb{R}\rightarrow \mathbb{R}\]
\[f(1-x)=1-f(f(x))\]\end{tcolorbox}
What is "amir" ?. Is it a real olympiad exercise for which you are sure that there exists olympiad level solutions ?

At first glance, there exists infinitely many solutions and I dont think that there exists a general form covering all of them.

Some examples :

1) $f(x)=\frac 12$ $\forall x$

2) $f(x)=x$ $\forall x$

3) Let $A\subseteq\mathbb R$ and $B=A\cup\{\pi\}$
$f(\frac 12)=\frac 12$
$\forall x>\frac 12$, $f(x)=1-\pi+(2\pi-1)1_B(x)$
$\forall x<\frac 12$, $f(x)=\pi-(2\pi-1)1_B(1-x)$

And tons of other strange examples.\end{tcolorbox}
amir=Amir Hossein Parvardi Functional Equations Problems


http://www.google.com\/url?sa=t&rct=j&q=Functional+Equations+Problems+amir&source=web&cd=2&cad=rja&ved=0CDQQFjAB&url=http%3A%2F%2Fohkawa.cc.it-hiroshima.ac.jp%2FAoPS.pdf%2F100%2520Functional%2520Equations%2520Problems.pdf&ei=lATMUrDfL8umrQfUgoH4AQ&usg=AFQjCNG9AGTx0NbR7GfXYFdv7U6BFuQ3fQ&sig2=kyDtRasQCt_1oIp-Fvte-Q
\end{solution}



\begin{solution}[by \href{https://artofproblemsolving.com/community/user/201392}{nima-amini}]
	http://www.artofproblemsolving.com/Forum/viewtopic.php?t=321967
\end{solution}



\begin{solution}[by \href{https://artofproblemsolving.com/community/user/29428}{pco}]
	\begin{tcolorbox}http://www.artofproblemsolving.com/Forum/viewtopic.php?t=321967\end{tcolorbox}
You are quite dishonest :

You silently edited :( your problem and added the "continuous" constraint which changes strongly the problem.

Fortunately :
1) One can see that you edited after my answer
2) I quoted your initial problem in my answer and it's easy to see that "continuous" was not in the first problem ...
\end{solution}



\begin{solution}[by \href{https://artofproblemsolving.com/community/user/201392}{nima-amini}]
	yes i edit it 
in amir pdf dont write continuouse and i think it is not necessary
i am sorrry.
\end{solution}
*******************************************************************************
-------------------------------------------------------------------------------

\begin{problem}[Posted by \href{https://artofproblemsolving.com/community/user/201392}{nima-amini}]
	find all function 
\[f:\mathbb{R}\rightarrow \mathbb{R}\]
\[xf(y)-yf(x)=f(y\/x)\]
	\flushright \href{https://artofproblemsolving.com/community/c6h570142}{(Link to AoPS)}
\end{problem}



\begin{solution}[by \href{https://artofproblemsolving.com/community/user/29428}{pco}]
	\begin{tcolorbox}find all function 
\[f:\mathbb{R}\rightarrow \mathbb{R}\]
\[xf(y)-yf(x)=f(y\/x)\]\end{tcolorbox}
Without any precision, I suppose that domain of functional equation is "$\forall x\ne 0$, $\forall y$". If so :

Let $P(x,y)$ be the assertion $xf(y)-yf(x)=f(\frac yx)$, true $\forall x\ne 0$, $\forall y$

$P(1,1)$ $\implies$ $f(1)=0$
$P(2,0)$ $\implies$ $f(0)=0$

Let $x\ne 0$ : $P(x,1)$ $\implies$ $-f(x)=f(\frac 1x)$

Let $x\ne 0$ :

$P(\frac 1x,2)$ $\implies$ $\frac 1xf(2)+2f(x)=f(2x)$

$P(\frac 12,x)$ $\implies$ $\frac 12f(x)+xf(2)=f(2x)$

Subtracting, we get $f(x)=\frac{2f(2)}3(x-\frac 1x)$

And so $\boxed{f(x)=a\frac{x^2-1}x\text{    }\forall x\ne 0\text{   and   }f(0)=0}$ which indeed is a solution.
\end{solution}
*******************************************************************************
-------------------------------------------------------------------------------

\begin{problem}[Posted by \href{https://artofproblemsolving.com/community/user/201392}{nima-amini}]
	find all function
\[f:\mathbb{R}\rightarrow \mathbb{R}\]
\[\forall x\in \mathbb{R}: f(f(x))=x^{2}-2\]
	\flushright \href{https://artofproblemsolving.com/community/c6h570148}{(Link to AoPS)}
\end{problem}



\begin{solution}[by \href{https://artofproblemsolving.com/community/user/201392}{nima-amini}]
	any solution???
\end{solution}



\begin{solution}[by \href{https://artofproblemsolving.com/community/user/29428}{pco}]
	\begin{tcolorbox}find all function
\[f:\mathbb{R}\rightarrow \mathbb{R}\]
\[\forall x\in \mathbb{R}: f(f(x))=x^{2}-2\]\end{tcolorbox}
So $f(f(f(f(x))))=(x^2-2)^2-2=x^4-4x^2+2$ $=x+(x+1)(x-2)(x^2+x-1)$ and so equation $f^{[4]}(x)=x$ has four roots $-1,2,a=\frac{-1-\sqrt 5}2$, $b=\frac{-1+\sqrt 5}2$

But $f^{[4]}(x)=x$ implies $f^{[4]}(f(x))=f^{[5]}(x)=f(f^{[4]}(x))=f(x)$ and so $f(\{-1,2,a,b\})\subseteq \{-1,2,a,b\}$

And since $f(x)=f(y)$ implies $x^2=y^2$, we get $f(\{-1,2,a,b\})=\{-1,2,a,b\}$

Note then that $f(f(-1))=-1$ and $f(f(2))=2$ and $f(f(a))=b$ and $f(f(b))=a$

If $f(a)=-1$, then $f(-1)=f(f(a))=b$ and $f(b)=f(f(-1))=-1$ and so $f(a)=f(b)$, impossible
If $f(a)=2$, then $f(2)=f(f(a))=b$ and $f(b)=f(f(2))=2$ and so $f(a)=f(b)$, impossible
If $f(a)=a$, then $f(f(a))=a$, impossible.
If $f(a)=b$, then $f(b)=f(f(a))=b$ and so $f(a)=f(b)$, impossible.

So \begin{bolded}no solution for this functional equation\end{underlined}\end{bolded}.
\end{solution}
*******************************************************************************
-------------------------------------------------------------------------------

\begin{problem}[Posted by \href{https://artofproblemsolving.com/community/user/201392}{nima-amini}]
	is there a function 
\[f:\mathbb{Q}^{+}\rightarrow \mathbb{Q}^{+}\]
\[f(xf(y))=\frac{f(x)}{y}\]
	\flushright \href{https://artofproblemsolving.com/community/c6h570151}{(Link to AoPS)}
\end{problem}



\begin{solution}[by \href{https://artofproblemsolving.com/community/user/29428}{pco}]
	\begin{tcolorbox}is there a function 
\[f:\mathbb{Q}^{+}\rightarrow \mathbb{Q}^{+}\]
\[f(xf(y))=\frac{f(x)}{y}\]\end{tcolorbox}
Yes. For example :

Let $p_n$ the ordered sequence of primes with $p_1=2$

Define $f(x)$ as :

$f(1)=1$
$f(p_{2k})=\frac 1{p_{2k-1}}$
$f(p_{2k-1})=p_{2k}$

$f(\prod p_i^{n_i})=\prod f(p_i)^{n_i}$ (where $n_i\in\mathbb Z\setminus\{0\}$)
\end{solution}
*******************************************************************************
-------------------------------------------------------------------------------

\begin{problem}[Posted by \href{https://artofproblemsolving.com/community/user/201392}{nima-amini}]
	find all function 
\[f:\mathbb{R}-\left \{ 0 \right \}\rightarrow \mathbb{R}\]
\[\forall x\in \mathbb{R}-\left \{ 0 \right \}\]                \[f(x)=-f(\frac{1}{x})\]
	\flushright \href{https://artofproblemsolving.com/community/c6h570153}{(Link to AoPS)}
\end{problem}



\begin{solution}[by \href{https://artofproblemsolving.com/community/user/29428}{pco}]
	\begin{tcolorbox}find all function 
\[f:\mathbb{R}-\left \{ 0 \right \}\rightarrow \mathbb{R}\]
\[\forall x\in \mathbb{R}-\left \{ 0 \right \}\]                \[f(x)=-f(\frac{1}{x})\]\end{tcolorbox}
General solution : Let $h(x)$ any function from $(-1,1)\to\mathbb R$.

$\forall x\in(-1,0)\cup(0,1)$ : $f(x)=h(x)$
$f(-1)=f(1)=0$
$\forall x\in(-\infty,-1)\cup(1,+\infty)$ : $f(x)=-h(\frac 1x)$
\end{solution}



\begin{solution}[by \href{https://artofproblemsolving.com/community/user/201392}{nima-amini}]
	\begin{tcolorbox}[quote="nima-amini"]find all function 
\[f:\mathbb{R}-\left \{ 0 \right \}\rightarrow \mathbb{R}\]
\[\forall x\in \mathbb{R}-\left \{ 0 \right \}\]                \[f(x)=-f(\frac{1}{x})\]\end{tcolorbox}
General solution : Let $h(x)$ any function from $(-1,1)\to\mathbb R$.

$\forall x\in(-1,0)\cup(0,1)$ : $f(x)=h(x)$
$f(-1)=f(1)=0$
$\forall x\in(-\infty,-1)\cup(1,+\infty)$ : $f(x)=-h(\frac 1x)$\end{tcolorbox}
how you can find them?
can you prove?
\end{solution}



\begin{solution}[by \href{https://artofproblemsolving.com/community/user/29428}{pco}]
	\begin{tcolorbox}how you can find them?
can you prove?\end{tcolorbox}

Are you joking ?

1) Proof that any function in the form I gave is indeed a solution.
===========================================
$\forall x\in(-1,0)\cup(0,1)$ : 
$f(x)=h(x)$
$\frac 1x\in(-\infty,-1)\cup(1,+\infty)$ and so $f(\frac 1x)=-h(x)$ $=-f(x)$

$\forall x\in(-\infty,-1)\cup(1,+\infty)$ : 
$f(x)=-h(\frac 1x)$
$\frac 1x\in(-1,0)\cup(0,1)$ and so $f(\frac 1x)=h(\frac 1x)$ $=-f(x)$

If $x=-1$ : $f(x)=0$ and $f(\frac 1x)=f(-1)=0=-f(x)$

If $x=1$ : $f(x)=0$ and $f(\frac 1x)=f(1)=0=-f(x)$
Q.E.D.

2) Proof that any solution may be written in the form I gave, and so that it is indeed a general solution 
=======================================================================
Setting $x=1$, we get $f(1)=-f(1)$ and so $f(1)=0$
Setting $x=-1$, we get $f(-1)=-f(-1)$ and so $f(-1)=0$

Let $h(x)=f(x)$

$\forall x\in(-1,0)\cup(0,1)$ : $f(x)=h(x)$

$\forall x\in(-\infty,-1)\cup(1,+\infty)$ : $\frac 1x\in(-1,0)\cup(0,1)$ and  $f(x)=-f(\frac 1x)$ $=-h(\frac 1x)$

Q.E.D.
\end{solution}



\begin{solution}[by \href{https://artofproblemsolving.com/community/user/29428}{pco}]
	\begin{tcolorbox}find all function 
\[f:\mathbb{R}-\left \{ 0 \right \}\rightarrow \mathbb{R}\]
\[\forall x\in \mathbb{R}-\left \{ 0 \right \}\]                \[f(x)=-f(\frac{1}{x})\]\end{tcolorbox}
Another form for general solution (nicer than the previous one, in my opinion, although full equivalent) :

Let $h(x)$ any function from $\mathbb R\to\mathbb R$ : $f(x)=h(x)-h(\frac 1x)$ $\forall x\ne 0$
\end{solution}
*******************************************************************************
-------------------------------------------------------------------------------

\begin{problem}[Posted by \href{https://artofproblemsolving.com/community/user/125553}{lehungvietbao}]
	\begin{bolded}Problem 1)\end{bolded}
 Suppose $\mathbb{R}^{+}$ is the set of positive numbers. 
Find all functions $f:\mathbb{R}^{+}\to\mathbb{R}^{+}$ such that
\[f\left(\frac{(f(y))^{11}}{x^3y}\right).\left(f(\frac{x}{(f(y))^{2010}})\right)^3=1\quad \forall x,y\in\mathbb{R}^+\]

\begin{bolded}Problem 2)\end{bolded} 
Find all functions $f:\mathbb{R}^*\to\mathbb{R}^*$ such that

\[f\left(\frac{(f(y))^{11}}{x^3y}\right).\left\lfloor f(\frac{x}{(f(y))^{2010}})\right\rfloor^3=1\quad \forall x,y\in\mathbb{R}^*\]
where $\mathbb{R}^*=\mathbb{R}- \{0\}$
	\flushright \href{https://artofproblemsolving.com/community/c6h570280}{(Link to AoPS)}
\end{problem}



\begin{solution}[by \href{https://artofproblemsolving.com/community/user/29428}{pco}]
	\begin{tcolorbox}\begin{bolded}Problem 1)\end{bolded}
 Suppose $\mathbb{R}^{+}$ is the set of positive numbers. 
Find all functions $f:\mathbb{R}^{+}\to\mathbb{R}^{+}$ such that
\[f\left(\frac{(f(y))^{11}}{x^3y}\right).\left(f(\frac{x}{(f(y))^{2010}})\right)^3=1\quad \forall x,y\in\mathbb{R}^+\]\end{tcolorbox}
Let $P(x,y)$ be the assertion $f\left(\frac{(f(y))^{11}}{x^3y}\right).\left(f(\frac{x}{(f(y))^{2010}})\right)^3=1$

Let $g(x)=-6019\ln f(e^x)$ from $\mathbb R\to\mathbb R$ so that $f(x)=e^{-\frac{g(\ln x)}{6019}}$ 

$P(e^{x-\frac{2010}{6019}g(y)},e^y)$ becomes new assertion $Q(x,y)$ : $g(g(y)-3x-y)+3g(x)=0$

$Q(\frac{g(x)-x}4,x)$ $\implies$ $g(\frac{g(x)-x}4)=0$ $\forall x$

Let then $A=g^{-1}(\{0\})\ne\emptyset$
Let $u\in A$ : $Q(x,u)$ $\implies$ $g(-3x-u)+3g(x)=0$ and so, subtracting from $Q(x,y)$ $g(g(y)-3x-y)=g(-3x-u)$

And so $Q(x,y)$ is equivalent to the two assertions :
$R(x,u)$ : $g(-3x-u)+3g(x)=0$ true $\forall x\in\mathbb R$, $\forall u\in A$
$S(x,y,u)$ : $g(x+g(y)-y+u)=g(x)$ true $\forall x,y\in\mathbb R$, $\forall u\in A$

From here, we find the trivial solutions $g(x)=0$ and $g(x)=x$ (which give $f(x)=1$ and $f(x)=x^{-\frac 1{6019}}$ but infinitely many other solutions :
For example, choose any $g(x)$ such that :
$g(x)=0$ $\forall x \in\mathbb Q$
$g(x+q)=g(x)$ $\forall x\in\mathbb Q$
$g(x)-x\in\mathbb Q$ $\forall x$
$g(-3x)=-3g(x)$ $\forall x$

But a lot of other possibilities exist and I dont think we can find a general form for all of them.

Once again, could you kindly confirm us that :
1) this is a real olympiad exercise you got in a real contest \/ exam (what contest ?)
2) you are quite sure that there is an olympiad level solution.

My own opinion is that this is a crazy invented problem and that your source is a quite bad source.
\end{solution}



\begin{solution}[by \href{https://artofproblemsolving.com/community/user/29428}{pco}]
	\begin{tcolorbox}[\begin{bolded}Problem 2)\end{bolded} 
Find all functions $f:\mathbb{R}^*\to\mathbb{R}^*$ such that

\[f\left(\frac{(f(y))^{11}}{x^3y}\right).\left\lfloor f(\frac{x}{(f(y))^{2010}})\right\rfloor^3=1\quad \forall x,y\in\mathbb{R}^*\]
where $\mathbb{R}^*=\mathbb{R}- \{0\}$\end{tcolorbox}
Let $P(x,y)$ be the assertion $f\left(\frac{(f(y))^{11}}{x^3y}\right).\left\lfloor f(\frac{x}{(f(y))^{2010}})\right\rfloor^3=1$

$P(\sqrt[3]{\frac{f(1)^{11}}x},1)$ $\implies$ $f(x)=\frac 1{n(x)}$ for some $n(x)\in\mathbb Z$

But then $\lfloor f(x)\rfloor\in\{-1,0,1\}$ $\forall x$ and so $\lfloor f(x)\rfloor\in\{-1,1\}$ $\forall x$ So $n(x)\in\{-1,1\}$

So $f(x)\in\{-1,1\}$ $\forall x$ and $P(x,y)$ becomes new assertion $Q(x,y)$ : $f(\frac{f(y)}{x^3y})=f(x)$

Obviously, we get the two constant solutions $\boxed{f(x)=1}$ $\forall x$ and $\boxed{f(x)=-1}$ $\forall x$

Let us from now consider non constant solutions

Let then $x\ne 0$
Let $u\ne 0$ such that $f(u)=-f(x)$
$Q(u,x)$ $\implies$ $f(\frac{f(x)}{u^3x})=f(u)$

$Q(u,\frac{f(x)}{u^3x})$ $\implies$ $f(\frac{f(u)x}{f(x)})=f(u)$ and so $f(-x)=f(u)=-f(x)$ and $f(x)$ is odd.

So $Q(x,y)$ may be written as new assertion $R(x,y)$ : $f(\frac{1}{x^3y})=f(x)f(y)$
Let $a=f(1)\in\{-1,1\}$

$R(1,x)$ $\implies$ $f(\frac 1x)=af(x)$ and so $R(x,y)$ becpmes $f(x^3y)=af(x)f(y)$

$R(x,1)$ $\implies$ then $f(x^3)=f(x)$ and $R(x,y)$ becomes $f(xy)=af(x)f(y)$

But then $R(x,\frac 1x)$ implies $a=1$ and $f(xy)=f(x)f(y)$ and so $f(x)=1$ $\forall x>0$ and so $f(x)=-1$ $\forall x<0$

And so $\boxed{f(x)=\text{sign}(x)}$ $\forall x$, which indeed is a solution.
\end{solution}



\begin{solution}[by \href{https://artofproblemsolving.com/community/user/125553}{lehungvietbao}]
	\begin{tcolorbox}
Once again, could you kindly confirm us that :
1) this is a real olympiad exercise you got in a real contest \/ exam (what contest ?)
2) you are quite sure that there is an olympiad level solution.

My own opinion is that this is a crazy invented problem and that your source is a quite bad source.\end{tcolorbox}
Dear Mr.Patrick
1) It's a local contest in my province ! :)
2) I'm sure that It's an olympiad level solution

And ''Maths is thinking'' or '' Mathematics is gymnastics''  so it's always good for our brain 
\end{solution}
*******************************************************************************
-------------------------------------------------------------------------------

\begin{problem}[Posted by \href{https://artofproblemsolving.com/community/user/171969}{mehrdad1st}]
	Find all functions $f:\mathbb R\to \mathbb R$ such that 
\[f(xf(y))+f(yf(x))=2xy \quad \forall x,y\in \mathbb R\]
	\flushright \href{https://artofproblemsolving.com/community/c6h570625}{(Link to AoPS)}
\end{problem}



\begin{solution}[by \href{https://artofproblemsolving.com/community/user/190093}{KamalDoni}]
	setting x=y=1 we find that f((1)) = 1 let's take f(1)=k then 
f(kf(y))+f(y)=2ky (it's obvious that f(0)=0)  then we can say that function is injective  but
by setting x=1 and y=k we find that f($\ k^2$)=k then by injection k^2=1 . So we proved that k=1 or k=-1 , in the first we take f(x)=x the second f(x)=-x
\end{solution}



\begin{solution}[by \href{https://artofproblemsolving.com/community/user/29428}{pco}]
	\begin{tcolorbox}..., in the first we take f(x)=x the second f(x)=-x\end{tcolorbox}
Uhhh ? How ?
Thanks for any explanation more.
\end{solution}
*******************************************************************************
-------------------------------------------------------------------------------

\begin{problem}[Posted by \href{https://artofproblemsolving.com/community/user/125553}{lehungvietbao}]
	$\varepsilon>0 $ is a given real number. The function $ f: \mathbb R \to \mathbb R $ satisfying
\[\left | f (x + y)-f (xy)-2f (y) \right | \leq \varepsilon, \forall x, y \in \mathbb R \]
Prove that $ \exists  g: \mathbb R \to \mathbb R$ is additive such that $ \left | f (x)-g (x) \right | \leq \varepsilon $
	\flushright \href{https://artofproblemsolving.com/community/c6h570635}{(Link to AoPS)}
\end{problem}



\begin{solution}[by \href{https://artofproblemsolving.com/community/user/29428}{pco}]
	\begin{tcolorbox}$\varepsilon>0 $ is a given real number. The function $ f: \mathbb R \to \mathbb R $ satisfying
\[\left | f (x + y)-f (xy)-2f (y) \right | \leq \varepsilon, \forall x, y \in \mathbb R \]
Prove that $ \exists  g: \mathbb R \to \mathbb R$ is additive such that $ \left | f (x)-g (x) \right | \leq \varepsilon $\end{tcolorbox}
Let $P(x,y)$ be the assertion $|f(x+y)-f(xy)-2f(y)|\le \varepsilon$

Let $x\ne 1$ : $P(\frac x{x-1},x)$ $\implies$ $|f(x)|\le\frac{\varepsilon}2$ $\forall x\ne 1$

$P(1,1)$ $\implies$  $|f(2)-3f(1)|\le \varepsilon$ and so $-\frac{3\varepsilon}2$ $\le f(2)-\varepsilon\le 3f(1)$ $\le f(2)+\varepsilon$ $\le \frac{3\varepsilon}2$

And so $|f(1)|\le\frac{\varepsilon}2$ and so $|f(x)|\le\frac{\varepsilon}2$ $\forall x$

Then, it remains to choose $g(x)=0$ $\forall x$ to end the proof.
\end{solution}
*******************************************************************************
-------------------------------------------------------------------------------

\begin{problem}[Posted by \href{https://artofproblemsolving.com/community/user/68025}{Pirkuliyev Rovsen}]
	Find all functions${{f: \mathbb(0;+\infty)}\to\mathbb(0;+\infty)}$ satisfying the conditions: a) $f(xf(y))=y^2f(\frac{x}{y})$ for all $x,y{\in}(0;+\infty)$, b) $\lim_{x\to+\infty}f(x)=0$.
	\flushright \href{https://artofproblemsolving.com/community/c6h570644}{(Link to AoPS)}
\end{problem}



\begin{solution}[by \href{https://artofproblemsolving.com/community/user/16000}{tchebytchev}]
	Very similar to an old one 
-Let $a$ with $f(a)=1$ for $y=a$, $f(ax)=a^2f(\frac{x}{a})$ and by induction $f(a^nx)=a^{2n}f(x)$ and $f(\frac{x}{a^n})=\frac{f(x)}{a^{2n}}$ so from the condition b it is not possible that $a>1$ or $a<1$ so if $a$ exists then $a=1$
-Let $f(1)=a$ for  $x=y=1$, $f(a)=a$, $y=1,x=a$ gives $f(a^2)=f(a)=a$ so by induction again $f(a^n)=a$ and for $y=a,a^2,...,x=1$ we get $f(\frac{1}{a}),...,f(\frac{1}{a^n})$ so by condition b again we have $a=1$.
-for $x=f(\frac{1}{y})$ we have $f(f(\frac{1}{y})f(y))=y^2f(\frac{f(\frac{1}{y})}{y})$ and for $x=y$ we have $f(xf(x))=x^2f(1)=x^2$ so $f(f(\frac{1}{y})f(y))=1$ and by the first item $f(\frac{1}{y})f(y)=1$ 
-for $x=\frac{1}{f(y)}$ we have  $1=f(1)=y^2f(\frac{1}{yf(y)})$ so $f(yf(y))=y^2$ so $f(y^2)=f(f(yf(y)))=y^2f^2(y)\frac{1}{f(yf(y))}=f^2(y)$ and by induction ${{f(y^{2^n}})=f(y)^{2^n}}$
- Let $x^2f(x)=a$ then if we replace $x$ by $x^2$ and $y$ by $x$ then $f(a)=a$ and by the last remarks and the condition b we have $a=1$ and then the only function solution is $\frac{1}{t^2}$ which satisfy the conditions of the problem.
\end{solution}



\begin{solution}[by \href{https://artofproblemsolving.com/community/user/29428}{pco}]
	\begin{tcolorbox}Find all functions$f: \mathbb(0;+\infty)\to\mathbb(0;+\infty)$ satisfying the conditions: a) $f(xf(y))=y^2f(\frac{x}{y})$ for all $x,y{\in}(0;+\infty)$, b) $\lim_{x\to+\infty}f(x)=0$.\end{tcolorbox}
Another solution :
Let $g(x)=\ln(f(e^x))$ from $\mathbb R\to \mathbb R$. 
Functional equation becomes assertion $P(x,y)$ : $g(x+g(y))=2y+g(x-y)$
Condition b) becomes $\lim_{x\to+\infty}g(x)=-\infty$

$P(x,0)$ $\implies$ $g(x+g(0))=g(x)$ and so $g(x+ng(0))=g(x)$ $\forall n\in\mathbb Z$ and so (using condition b) $g(0)=0$ 

If $g(a)=g(b)=c$ :
$P(a+b,b)$ $\implies$ $g(a+b+c)=2b+c$
$P(a+b,a)$ $\implies$ $g(a+b+c)=2a+c$
And so $a=b$ and $g(x)$ is injective.

$P(0,-x)$ $\implies$ $g(g(-x))=-2x+g(x)$
$P(x-g(x),x)$ $\implies$ $g(-g(x))=-2x+g(x)$
So $g(g(-x))=g(-g(x))$ and, since injective, $g(-x)=-g(x)$ and $g(x)$ is odd.

$P(0,x)$ $\implies$ $g(g(x))+g(x)=2x$ and so $g(x)+x$ is surjective.

$P(x+y,y)$ $\implies$ $g(x+y+g(y))=2y+g(x)$
$P(y,y)$ $\implies$ $g(y+g(y))=2y$
Subtracting, we get $g(x+y+g(y))=g(x)+g(y+g(y))$ and so, since $x+g(x)$ is surjective, $g(x+y)=g(x)+g(y)$

We know that non continuous solutions are unbounded everywhere and so condition b) implies continuity and $g(x)=ax$

Plugging in original equation, we get $a\in\{1,-2\}$ and condition b implies $a=-2$ and $g(x)=-2x$

And so $\boxed{f(x)=\frac 1{x^2}}$ $\forall x>0$, which indeed is a solution.
\end{solution}
*******************************************************************************
-------------------------------------------------------------------------------

\begin{problem}[Posted by \href{https://artofproblemsolving.com/community/user/199065}{MarkJohn}]
	Find all continous functions $f: R \to R$ such that the following identity is satisfied for all $x,y$ 
   $f(x+y)+f(x-y)+2=2f(x)+2f(y)$
	\flushright \href{https://artofproblemsolving.com/community/c6h570759}{(Link to AoPS)}
\end{problem}



\begin{solution}[by \href{https://artofproblemsolving.com/community/user/29428}{pco}]
	\begin{tcolorbox}Find all continous functions $f: R \to R$ such that the following identity is satisfied for all $x,y$ 
   $f(x+y)+f(x-y)+2=2f(x)+2f(y)$\end{tcolorbox}
Let $g(x)=f(x)-(f(1)-1)x^2-1$ so that $g(1)=0$ and functional equation is :
Assertion $P(x,y)$ : $g(x+y)+g(x-y)=2g(x)+2g(y)$

$P(0,0)$ $\implies$ $g(0)=0$

$P((n+1)x,x)$ $\implies$  $g((n+2)x)=2g((n+1)x)-g(nx)+2g(x)$ and simple induction implies $g(nx)=n^2g(x)$

So $g(qx)=q^2g(x)$ $\forall x\in\mathbb R,\forall q\in\mathbb Q$

So $g(x)=0$ $\forall x\in\mathbb Q$ and so, since continuous, $g(x)=0$ $\forall x$

Hence the solution $\boxed{f(x)=ax^2+1}$ $\forall x$, which indeed is a solution, whatecver is $a\in\mathbb R$
\end{solution}
*******************************************************************************
-------------------------------------------------------------------------------

\begin{problem}[Posted by \href{https://artofproblemsolving.com/community/user/152100}{Data-SM}]
	Find all continuous functions $ [ 0, +\infty ) \to [ 0, +\infty ) $ such that for all $x\ge 0$ :

$2f(2x)= f^2(x)+1$

[hide="Hint"]f is bijective.[\/hide]
	\flushright \href{https://artofproblemsolving.com/community/c6h570767}{(Link to AoPS)}
\end{problem}



\begin{solution}[by \href{https://artofproblemsolving.com/community/user/29428}{pco}]
	\begin{tcolorbox}... f is bijective....\end{tcolorbox}
Wrong : we immediately get that $f(x)=\frac 12+\frac{f(\frac x2)^2}2\ge \frac 12$ and so $f(x)$ is not surjective, so not bijective.
\end{solution}



\begin{solution}[by \href{https://artofproblemsolving.com/community/user/29428}{pco}]
	\begin{tcolorbox}Find all continuous functions $ [ 0, +\infty ) \to [ 0, +\infty ) $ such that for all $x\ge 0$ :

$2f(2x)= f^2(x)+1$
\end{tcolorbox}
There are infinitely many solutions, including neither injective, neither surjective one.

Choose any continuous function $h(x)$ from $[1,2]\to[1,+\infty)$ such that $h(2)=\frac{h(1)^2+1}2$

From there :
1) build the sequence of functions $h_n(x)$ from $[2^n,2^{n+1})\to[1,+\infty)$ as :
$h_0(x)=h(x)$
$h_{n+1}(x)=\frac{h_n(\frac x2)+1}2$

2) build the sequence of functions $k_n(x)$ from $(2^{-n},2^{-n+1}]\to[1,+\infty)$ as :
$k_0(x)=h(x)$
$k_{n+1}(x)=\sqrt{2k_n(2x)-1}$

Then define $f(x)$ as :
$f(0)=1$
$\forall x\in(2^{-n},2^{-n+1}]$ : $f(x)=k_n(x)$ $\forall n\in\mathbb N$
$\forall x\in(1,2)$ : $f(x)=h(x)$
$\forall x\in[2^n,2^{n+1})$ : $f(x)=h_n(x)$ $\forall n\in\mathbb N$

It's easy to check that this function fits all the requirements (the less obvious part is continuity at zero).
\end{solution}



\begin{solution}[by \href{https://artofproblemsolving.com/community/user/152100}{Data-SM}]
	But, using AM-GM we can get : $ f(2x) \ge f(x) $ for all $ x\ge 0 $ . Doesn't it mean that f is increasing on $[0,+\infty[$ ?
\end{solution}



\begin{solution}[by \href{https://artofproblemsolving.com/community/user/29428}{pco}]
	\begin{tcolorbox}But, using AM-GM we can get : $ f(2x) \ge f(x) $ for all $ x\ge 0 $ . Doesn't it mean that f is increasing on $[0,+\infty[$ ?\end{tcolorbox}
Certainly not. Maybe $f(x)$ is decreasing then increasing beween $x$ and $2x$
\end{solution}



\begin{solution}[by \href{https://artofproblemsolving.com/community/user/152100}{Data-SM}]
	Suppose it's decreasing on an interval [a,b]. Putting x=a we get that f is increasing on [a,2a]. Putting x=2a we get : f is increasing on [2a,4a]. 
And hence we get that f is increansing on [a,4a] and so on ... because $ f(2x) \ge f(x) $ FOR ALL $ x\ge0$ . Thus, it is increansing on [a,b]
Where do I err, please ?
\end{solution}



\begin{solution}[by \href{https://artofproblemsolving.com/community/user/64716}{mavropnevma}]
	You say "putting $x=a$ we get that $f$ is increasing on $[a,2a]$". Re-read what \begin{bolded}pco\end{bolded} had said. All you know is $f(2a) \geq f(a)$, and $f(2(a+\varepsilon)) \geq f(a+\varepsilon)$, but you have no known inequality between $f(a)$ and $f(a+\varepsilon)$. Mull it over.
\end{solution}



\begin{solution}[by \href{https://artofproblemsolving.com/community/user/152100}{Data-SM}]
	Yes, thank you, I see now. Can you give me how, in the case of functional equations, we can prove that the solution is monotonous ?
\end{solution}



\begin{solution}[by \href{https://artofproblemsolving.com/community/user/29428}{pco}]
	\begin{tcolorbox}Yes, thank you, I see now. Can you give me how, in the case of functional equations, we can prove that the solution is monotonous ?\end{tcolorbox}
Prove for example that $f(x+y)\ge f(x)$ $\forall y\ge 0$
\end{solution}
*******************************************************************************
-------------------------------------------------------------------------------

\begin{problem}[Posted by \href{https://artofproblemsolving.com/community/user/197087}{math4evernever}]
	Find all functions $ f: \mathbb{Z}\to \mathbb{Z} $ such that:

$ f(x+f(y))=y+f(x+c) $  $ \forall x,y\in\mathbb{Z} $
	\flushright \href{https://artofproblemsolving.com/community/c6h570904}{(Link to AoPS)}
\end{problem}



\begin{solution}[by \href{https://artofproblemsolving.com/community/user/29428}{pco}]
	\begin{tcolorbox}Find all functions $ f: \mathbb{Z}\to \mathbb{Z} $ such that:

$ f(x+f(y))=y+f(x+c) $  $ \forall x,y\in\mathbb{Z} $\end{tcolorbox}
I suppose $c\in\mathbb Z$

$f(x)$ is surjective. Let then $u$ such that $f(u)=1$ and $v$ such that $f(v)=0$

$f(x+1)=u+f(x+c)$
$f(x)=v+f(x+c)$
Subtracting, we get $f(x+1)-f(x)=u-v$ and so $f(x)=(u-v)x+f(0)$
 
Plugging this back in original equation, we find two solutions $\boxed{f(x)=x+c}$ and $\boxed{f(x)=c-x}$
\end{solution}



\begin{solution}[by \href{https://artofproblemsolving.com/community/user/148207}{Particle}]
	Suppose $z\in \mathbb Z$.
$f(z+f(x+f(y)))=f(z+y+f(x+c))$

$\implies x+f(y)+f(z+c)=x+c+f(z+y+c)$

$\implies f(y)+f(z+c)=f(y+z+c)+c$

Define $g(x)=f(x)-c$. Therefore $g(y)+g(z+c)=g(y+z+c)$. Since both of $y$ and $z+c$ range through all the integers, $g$ satisfies cauchy equation for all integers. So $g(x)=ax$. Hence $f(x)=ax+c$. By checking, we get $a=1,-1$. So all the functions are $f(x)=x+c$ and $f(x)=c-x$.
\end{solution}
*******************************************************************************
-------------------------------------------------------------------------------

\begin{problem}[Posted by \href{https://artofproblemsolving.com/community/user/201392}{nima-amini}]
	is there a function \[f:\mathbb{N}\rightarrow \mathbb{N}\]
 such that
\[\forall n\in \mathbb{N},f(f(n))=f(n)+n\]
	\flushright \href{https://artofproblemsolving.com/community/c6h570932}{(Link to AoPS)}
\end{problem}



\begin{solution}[by \href{https://artofproblemsolving.com/community/user/29428}{pco}]
	\begin{tcolorbox}is there a function f such that
\[f(f(n))=f(n)+n\]\end{tcolorbox}
What is domain of function ?
What is codomain of function ?
What is domain of functional equation ?

Thanks for any precision (dont hesitate to copy \begin{bolded}all \end{underlined}\end{bolded}the words of the problem you got from your teacher; even if you dont think so, all may be useful)
\end{solution}



\begin{solution}[by \href{https://artofproblemsolving.com/community/user/201392}{nima-amini}]
	\begin{tcolorbox}[quote="nima-amini"]is there a function f such that
\[f(f(n))=f(n)+n\]\end{tcolorbox}
What is domain of function ?
What is codomain of function ?
What is domain of functional equation ?

Thanks for any precision (dont hesitate to copy \begin{bolded}all \end{underlined}\end{bolded}the words of the problem you got from your teacher; even if you dont think so, all may be useful)\end{tcolorbox}
im sorry
i forgot wrote them.
\end{solution}



\begin{solution}[by \href{https://artofproblemsolving.com/community/user/29428}{pco}]
	\begin{tcolorbox}is there a function \[f:\mathbb{N}\rightarrow \mathbb{N}\]
 such that
\[\forall n\in \mathbb{N},f(f(n))=f(n)+n\]\end{tcolorbox}

$f(n)=\left\lfloor\frac{1+\sqrt 5}2(n+1)\right\rfloor-1$
\end{solution}



\begin{solution}[by \href{https://artofproblemsolving.com/community/user/201392}{nima-amini}]
	\begin{tcolorbox}[quote="nima-amini"]is there a function \[f:\mathbb{N}\rightarrow \mathbb{N}\]
 such that
\[\forall n\in \mathbb{N},f(f(n))=f(n)+n\]\end{tcolorbox}

$f(n)=\left\lfloor\frac{1+\sqrt 5}2(n+1)\right\rfloor-1$\end{tcolorbox}
thank you.
please explain how you find the function.
\end{solution}



\begin{solution}[by \href{https://artofproblemsolving.com/community/user/29428}{pco}]
	\begin{tcolorbox}please explain how you find the function.\end{tcolorbox}
That's a rather classical function.
Easy to check it works.
I think there exists many threads in AOPS dealing with this function.
\end{solution}



\begin{solution}[by \href{https://artofproblemsolving.com/community/user/187896}{Ashutoshmaths}]
	\begin{tcolorbox}[quote="nima-amini"]is there a function f such that
\[f(f(n))=f(n)+n\]\end{tcolorbox}
What is domain of function ?
What is codomain of function ?
What is domain of functional equation ?

Thanks for any precision (dont hesitate to copy \begin{bolded}all \end{underlined}\end{bolded}the words of the problem you got from your teacher; even if you dont think so, all may be useful)\end{tcolorbox}
What do you mean by the domain of the functional equation?
\end{solution}



\begin{solution}[by \href{https://artofproblemsolving.com/community/user/29428}{pco}]
	\begin{tcolorbox}What do you mean by the domain of the functional equation?\end{tcolorbox}
The set of values for which the functional equation is claimed to be true.

For example :
"Find all the functions $f(x)$ from$\mathbb R\to\mathbb R^+$ such that $f(x+y)=\frac 1{f(x)}+\frac 1{f(y)}$ $\forall x,y\in(0,1)$

Domain of $f(x)$ is $\mathbb R$
Codomain of $f(x)$ is $\mathbb R^+$
Domain of functional equation is $(0,1)\times(0,1)$
\end{solution}
*******************************************************************************
-------------------------------------------------------------------------------

\begin{problem}[Posted by \href{https://artofproblemsolving.com/community/user/187896}{Ashutoshmaths}]
	Find all functions $f:R-->R$
such that $f(x^2+yf(z))=xf(x)+zf(y)$
when $x,y,z\in R^$
I am new to functional equations so please explain each step. :)
	\flushright \href{https://artofproblemsolving.com/community/c6h571033}{(Link to AoPS)}
\end{problem}



\begin{solution}[by \href{https://artofproblemsolving.com/community/user/29428}{pco}]
	\begin{tcolorbox}Find all functions $f:R-->R$
such that $f(x^2+yf(z))=xf(x)+zf(y)$
when $x,y,z\in R^$
I am new to functional equations so please explain each step. :)\end{tcolorbox}
Let $P(x,y,z)$ be the assertion $f(x^2+yf(z))=xf(x)+zf(y)$

$P(0,0,0)$ $\implies$ $f(0)=0$

$P(x,0,0)$ $\implies$ $f(x^2)=xf(x)$

$P(0,1,x^2)$ $\implies$ $f(xf(x))=f(1)x^2$
$P(0,x,x)$ $\implies$ $f(xf(x))=xf(x)$

Subtracting the two last lines, we get $f(x)=xf(1)$ $\forall x\ne 0$, still true when $x=0$

Plugging $f(x)=ax$ back in original equation, we get $a\in\{0,1\}$ and so the two solutions :
$\boxed{f(x)=0}$ $\forall x$

$\boxed{f(x)=x}$ $\forall x$
\end{solution}



\begin{solution}[by \href{https://artofproblemsolving.com/community/user/187896}{Ashutoshmaths}]
	\begin{tcolorbox}
Let $P(x,y,z)$ be the assertion $f(x^2+yf(z))=xf(x)+zf(y)$

\end{tcolorbox}
I think by this you mean the value that we plug in our assertion is also plugged to equate the given relation? :maybe:
\end{solution}



\begin{solution}[by \href{https://artofproblemsolving.com/community/user/29428}{pco}]
	\begin{tcolorbox}[quote="pco"]
Let $P(x,y,z)$ be the assertion $f(x^2+yf(z))=xf(x)+zf(y)$

\end{tcolorbox}
I think by this you mean the value that we plug in our assertion is also plugged to equate the given relation? :maybe:\end{tcolorbox}
Yes, indeed.
\end{solution}



\begin{solution}[by \href{https://artofproblemsolving.com/community/user/187896}{Ashutoshmaths}]
	I understand till $f(x)=xf(1)$
then you plug $f(x)=ax$ in which equation?
\end{solution}



\begin{solution}[by \href{https://artofproblemsolving.com/community/user/29428}{pco}]
	\begin{tcolorbox}I understand till $f(x)=xf(1)$
then you plug $f(x)=ax$ in which equation?\end{tcolorbox}
I know that if $f(x)$ exists, it must be in the form $f(x)=ax$ for some $a\in\mathbb R$

So I just write the original functional equation to see if indeed some values of $a$ give a real solution. And replacing $f(x)$ by $ax$ in this original equation, I see that equation can be true only when $a=0$ or $a=1$.
\end{solution}



\begin{solution}[by \href{https://artofproblemsolving.com/community/user/187896}{Ashutoshmaths}]
	is the function injective?
How to prove it then?
\end{solution}



\begin{solution}[by \href{https://artofproblemsolving.com/community/user/187896}{Ashutoshmaths}]
	from $P(0,x,x)\implies f(f(x)x)=x.f(x)$
as $x,f(x)\in R$
can't we say $f(x)=x$
\end{solution}



\begin{solution}[by \href{https://artofproblemsolving.com/community/user/29428}{pco}]
	\begin{tcolorbox}is the function injective?
How to prove it then?\end{tcolorbox}
Since $f(x)=0$ $\forall x$ is a non injective solution, it we be impossible to deduce from functional equation that all solutions must be injective.
So no.

\begin{tcolorbox}from $P(0,x,x)\implies f([f(x)x])=x.f(x)$
as $x,f(x)\in r$
can't we say $f(x)=x$\end{tcolorbox}

Since $f(x)=0$ $\forall x$ is a solution, your conclusion is wrong.

The reason for which it is wrong is that $f(xf(x))=xf(x)$ implies that $f(u)=u$ not for all $u$ but only for those $u$ which may be written in the form $u=xf(x)$ for some $x$.
\end{solution}
*******************************************************************************
-------------------------------------------------------------------------------

\begin{problem}[Posted by \href{https://artofproblemsolving.com/community/user/169452}{TheBernuli}]
	Let $f:(0,+\infty)\rightarrow\mathbb{R}$ function such that:
1) $f$ is strictly increasing.
2) $f(x)>-\dfrac{1}{x}$ for all $x>0$.
3) $f(x)f(f(x)+\dfrac{1}{x})=1$ for all $x>0$.
Find $f(1)$.
	\flushright \href{https://artofproblemsolving.com/community/c6h571356}{(Link to AoPS)}
\end{problem}



\begin{solution}[by \href{https://artofproblemsolving.com/community/user/29428}{pco}]
	\begin{tcolorbox}Let $f:(0,+\infty)\rightarrow\mathbb{R}$ function such that:
1) $f$ is strictly increasing.
2) $f(x)>-\dfrac{1}{x}$ for all $x>0$.
3) $f(x)f(f(x)+\dfrac{1}{x})=1$ for all $x>0$.
Find $f(1)$.\end{tcolorbox}
From $f(x)f(f(x)+\frac 1x)$ and $f(f(x)+\frac 1x)f(f(f(x)+\frac 1x)+\frac 1{f(x)+\frac 1x})=1$, we get $f(x)=f(f(f(x)+\frac 1x)+\frac 1{f(x)+\frac 1x})$

So, since injective : $f(f(x)+\frac 1x)+\frac 1{f(x)+\frac 1x}=x$

So $\frac 1{f(x)}+\frac 1{f(x)+\frac 1x}=x$

So $xf(x)^2-f(x)-\frac 1x=0$ and so $f(x)=\frac{1-\sqrt 5}{2x}$ $\forall x$, which indeed is a solution (the other possibilities, where $f(x)=\frac{1+\sqrt 5}{2x}$ for some $x$, may easily be rejected because $f(x)$ is strictly increasing).

Hence the answer : $\boxed{f(1)=\frac{1-\sqrt 5}2}$
\end{solution}
*******************************************************************************
-------------------------------------------------------------------------------

\begin{problem}[Posted by \href{https://artofproblemsolving.com/community/user/169829}{jimk}]
	Find all functions f : R → R such that  f(x)=f(x^2)
	\flushright \href{https://artofproblemsolving.com/community/c6h571482}{(Link to AoPS)}
\end{problem}



\begin{solution}[by \href{https://artofproblemsolving.com/community/user/29428}{pco}]
	\begin{tcolorbox}Find all functions f : R → R such that  f(x)=f(x^2)\end{tcolorbox}
It's a shame (according to me) to get such a question in an olympiad contest. It's a classical course question (interesting as a course question about piece per piece construction, but with very few interest in a contest).
Solution :

Let $u(x),v(x)$ any functions from $[0,1)\to\mathbb R$
Let $a,b$ any real numbers

One general form for the solutions is :

$\forall x>1$ : $f(x)=u\left(\left\{\frac{\ln |\ln x|}{\ln 2}\right\}\right)$
For $x=1$ : $f(1)=a$
$\forall x\in (0,1)$ :  $f(x)=v\left(\left\{\frac{\ln |\ln x|}{\ln 2}\right\}\right)$
For $x=0$ : $f(0)=b$
$\forall x<0$ : $f(x)=f(-x)$
\end{solution}
*******************************************************************************
-------------------------------------------------------------------------------

\begin{problem}[Posted by \href{https://artofproblemsolving.com/community/user/125553}{lehungvietbao}]
	A function $ f(x)$ is called \begin{bolded}antiperiodic\end{bolded} (additive)  function on $M$ if there exists $ a>0$ such that
          \[\begin{cases}\forall x\in M \implies x\pm a\in M \\f(x+a)= -f(x) , \forall x\in M \end{cases}\]
 $ a$ is a period  of the  function $ f(x)$
 $ a_0$ is called the basic period of the function $ f(x)$ if $ a_0=\min{a}$
See [url=http://mathworld.wolfram.com\/AntiperiodicFunction.html]antiperiodic[\/url]

1) Let $h(x)$ be a given antiperiodic function on $\mathbb R$ ( $a>0$ is period ). Find $f(x)$ such that \[f(x+a)=h(x)f(x) \quad \forall x\in\mathbb R\]

2) Let $b\neq 1$ and let $h(x)$ be a  given antiperiodic function on $\mathbb R$ ( $a>0$ is period ). Find all functions $f(x)$ such that  \[f(x+a)+bf(x)=h(x) \quad \forall x\in\mathbb R \]
	\flushright \href{https://artofproblemsolving.com/community/c6h571573}{(Link to AoPS)}
\end{problem}



\begin{solution}[by \href{https://artofproblemsolving.com/community/user/29428}{pco}]
	\begin{tcolorbox}A function $ f(x)$ is called \begin{bolded}antiperiodic\end{bolded} (additive)  function on $M$ if there exists $ a>0$ such that
          \[\begin{cases}\forall x\in M \implies x\pm a\in M \\f(x+a)= -f(x) , \forall x\in M \end{cases}\]
 $ a$ is a period  of the  function $ f(x)$
 $ a_0$ is called the basic period of the function $ f(x)$ if $ a_0=\min{a}$
See [url=http://mathworld.wolfram.com\/AntiperiodicFunction.html]antiperiodic[\/url]

1) Let $h(x)$ be a given antiperiodic function on $\mathbb R$ ( $a>0$ is period ). Find $f(x)$ such that \[f(x+a)=h(x)f(x) \quad \forall x\in\mathbb R\]
\end{tcolorbox}
$f(x)=0$ $\forall x$
\end{solution}



\begin{solution}[by \href{https://artofproblemsolving.com/community/user/29428}{pco}]
	\begin{tcolorbox}2) Let $b\neq 1$ and let $h(x)$ be a  given antiperiodic function on $\mathbb R$ ( $a>0$ is period ). Find all functions $f(x)$ such that  \[f(x+a)+bf(x)=h(x) \quad \forall x\in\mathbb R \]\end{tcolorbox}
If $b=0$, we get $f(x)=-h(x)$

If $b\ne 0$, induction gives $f(x+na)=(-b)^nf(x)-(-1)^n\frac{b^n-1}{b-1}h(x)$ $\forall x$ and $\forall n\in\mathbb Z$

Hence the result : 
$\boxed{f(x)=(-b)^{\left\lfloor\frac xa\right\rfloor}u\left(a\left\{\frac xa\right\}\right)-(-1)^{\left\lfloor\frac xa\right\rfloor}\frac{b{\left\lfloor\frac xa\right\rfloor}-1}{b-1}h\left(a\left\{\frac xa\right\}\right)}$
Whatever is the function $u(x)$ from $[0,a)\to\mathbb R$
\end{solution}
*******************************************************************************
-------------------------------------------------------------------------------

\begin{problem}[Posted by \href{https://artofproblemsolving.com/community/user/187896}{Ashutoshmaths}]
	Hi Mathlinkers,
PROBLEM:
Find all functions $f:\mathbb{R}\rightarrow \mathbb{R}$ 
Such that $f(x^2-y^2)=(x-y)(f(x)+f(y))$
[hide="P.S."]I am a beginner at functional equations, So please explain each step :)[\/hide]
[hide="What I've done so far:"]Well, I got 
$f(x)=-f(-x),f(x^2)=x.f(x),f(0)=0$ But I cannot proceed from here. :([\/hide]
	\flushright \href{https://artofproblemsolving.com/community/c6h571907}{(Link to AoPS)}
\end{problem}



\begin{solution}[by \href{https://artofproblemsolving.com/community/user/177724}{vanu1996}]
	sorry,for post.
\end{solution}



\begin{solution}[by \href{https://artofproblemsolving.com/community/user/187896}{Ashutoshmaths}]
	Thanks pco,
I had a misconception(Reply to the following post).
\end{solution}



\begin{solution}[by \href{https://artofproblemsolving.com/community/user/29428}{pco}]
	\begin{tcolorbox}change $y$ by $x-1$ th you get a Cauchy type function.\end{tcolorbox}
Huhhhh ? A Cauchy function with one unique variable ? Never saw.
\end{solution}



\begin{solution}[by \href{https://artofproblemsolving.com/community/user/174401}{kautokid}]
	If $f(x^{2}-y^{2})=(x-y)(f(x)+f(y))$
This is Bangladesh MO -2012 haha 
If $x=y$ i have $f(0)=0$ , if $y=0$ i have $f(x^{2})=xf(x)$ 
If $x=-y$ i have $0=f(0)=2x[f(x)+f(-x)]$ so that $f(x)=-f(-x)$ 
$f(x^{2}-y^{2})=f(x^{2}-(-y)^{2})=(x+y)[f(x)+f(-y)]=(x+y)[f(x)-f(y)]$
So i have $(x-y)[f(x)+f(y)]=(x+y)[f(x)-f(y)]<=>\frac{f(x)}{x}=\frac{f(y)}{y}<=>f(x)=cx$
\end{solution}



\begin{solution}[by \href{https://artofproblemsolving.com/community/user/118092}{IDMasterz}]
	You might be interested in a somewhat similar problem:

Find all functions $f : R \to R$ which satisfy $f(x^3) + f(y^3) = (x + y)(f(x^2) + f(y^2) - f(xy))$ for all real numbers $x$ and $y$.

[hide="A solution"]I claim the only solution if $f(x) = cx$ for some constant $c \in \mathbb{R}$.

Let us denote $P(x, y)$ to be the assertion. Notice that $P(x, -x) \implies f(x^3) = -f(-x^3)$ and letting $x^3 = z \implies f(z) = -f(-z)$, hence the function is an odd function. Now, $P(x, 0) \implies f(x^3) = xf(x^2)$. Using this result we obtain a new functional equation:

$f(x^3) + f(y^3) = xf(x^2) + yf(y^2) = (x+y)(f(x^2) + f(y^2) + f(xy)) \implies \boxed{yf(x^2) + xf(y^2) = (x+y)f(xy)...(1)}$

Let $T(x, y)$ denote $(1)$. Notice that $T(x, 1) \implies xf(1) = (x+1)f(x) - f(x^2) \implies f(1) = \dfrac{(x+1)f(x) - f(x^2)}{x}$.
Further, $T(-x, 1) \implies -f(1) = \dfrac{(1-x)f(-x) - f(x^2)}{x}$. Letting $f(1) = c$ for some constant, and subtracting $T(-x, 1) - T(x, 1) \implies \boxed{2cx = (x+1)f(x) + (-f(-x)(1-x))...(2)}$.

However, since $f(x) = -f(-x)$ (odd function), $(2)$ can be simplified as $2f(x) = 2cx \implies f(x) = cx$ for $c \in \mathbb{R}$.[\/hide]
\end{solution}



\begin{solution}[by \href{https://artofproblemsolving.com/community/user/174401}{kautokid}]
	 haha if $x=y=0$ i have $f(0)=0$ 
If $y=0$ i have $f(x^{3})=xf(x^{2})$
So i have $(x+y)(f(x^{2})+f(y^{2})-f(xy))=xf(x^{2})+yf(x^{2})+xf(y^{2})+yf(y^{2})-(x+y)f(xy)=f(x^{3})+f(y^{3})$ 
So that $xf(y^{2})+yf(x^{2})=xy(x+y)$ 
If $x=y$ i have $2xf(x^{2})=x^{2}.2x$ so i have $f(a)=a$ for $a>0$
But $f(x^{3})=xf(x^{2})=x^{3}$ so i have $f(x)=x$ for all $x$
\end{solution}



\begin{solution}[by \href{https://artofproblemsolving.com/community/user/185327}{shadow10}]
	\begin{tcolorbox} So that $xf(y^{2})+yf(x^{2})=xy(x+y)$ \end{tcolorbox}
How did you conclude this?This is apparently wrong I guess.Because you already assume $f(xy)=xy$ 
\end{solution}



\begin{solution}[by \href{https://artofproblemsolving.com/community/user/174401}{kautokid}]
	OH sorry hihi  i will check
\end{solution}



\begin{solution}[by \href{https://artofproblemsolving.com/community/user/174401}{kautokid}]
	:P I have $xf(y^{2})+yf(x^{2})=(x+y)f(xy)$ , if $y=1$ and i set $f(1)=a$ so $xf(1)+f(x^{2})=(x+1)f(x)$
And if $y=-1$ i have $xf(1)-f(x^{2})=(x-1)f(-x)$
Because $f(x^{3})=xf(x^{2})$ so i have $f(x)=-f(-x)$ => $2xf(1)=(x+1+1-x)f(x)$ so i have $f(x)=f(1).x$ hihi check i have $f(1)=1$
\end{solution}



\begin{solution}[by \href{https://artofproblemsolving.com/community/user/185327}{shadow10}]
	see there is no variation between your proof and that of IDMasterz.both are quite same.
\end{solution}



\begin{solution}[by \href{https://artofproblemsolving.com/community/user/118092}{IDMasterz}]
	\begin{tcolorbox}[quote="IDMasterz"]You might be interested in a somewhat similar problem:

Find all functions $f : R \to R$ which satisfy $f(x^3) + f(y^3) = (x + y)(f(x^2) + f(y^2) - f(xy))$ for all real numbers $x$ and $y$.

[hide="A solution"]I claim the only solution if $f(x) = cx$ for some constant $c \in \mathbb{R}$.

Let us denote $P(x, y)$ to be the assertion. Notice that $P(x, -x) \implies f(x^3) = -f(-x^3)$ and letting $x^3 = z \implies f(z) = -f(-z)$, hence the function is an odd function. Now, $P(x, 0) \implies f(x^3) = xf(x^2)$. Using this result we obtain a new functional equation:

$f(x^3) + f(y^3) = xf(x^2) + yf(y^2) = (x+y)(f(x^2) + f(y^2) + f(xy)) \implies \boxed{yf(x^2) + xf(y^2) = (x+y)f(xy)...(1)}$

Let $T(x, y)$ denote $(1)$. Notice that $T(x, 1) \implies xf(1) = (x+1)f(x) - f(x^2) \implies f(1) = \dfrac{(x+1)f(x) - f(x^2)}{x}$.
Further, $T(-x, 1) \implies -f(1) = \dfrac{(1-x)f(-x) - f(x^2)}{x}$. Letting $f(1) = c$ for some constant, and subtracting $T(-x, 1) - T(x, 1) \implies \boxed{2cx = (x+1)f(x) + (-f(-x)(1-x))...(2)}$.

However, since $f(x) = -f(-x)$ (odd function), $(2)$ can be simplified as $2f(x) = 2cx \implies f(x) = cx$ for $c \in \mathbb{R}$.[\/hide]\end{tcolorbox}
This is from a past BrMO2: http://www.bmoc.maths.org\/home\/bmo2-2009.pdf

My solution is very similar to yours, but here is pco's solution:

\begin{tcolorbox}Let $P(x,y)$ be the assertion $f(x^3)+f(y^3)=(x+y)(f(x^2)+f(y^2)-f(xy))$
Let $a=f(1)$

$P(0,0)$ $\implies$ $f(0)=0$
$P(x,0)$ $\implies$ $f(x^3)=xf(x^2)$ and so $f(-x)=-f(x)$

$P(x,1)$ $\implies$ $f(x^3)+a=(x+1)(f(x^2)+a-f(x))$
$P(x,0)$ $\implies$ $f(x^3)=xf(x^2)$
Subtracting, we get $0=f(x^2)+ax-(x+1)f(x)$
Setting $x\to -x$ in this last line, we get $0=f(x^2)-ax+(-x+1)f(x)$
Subtracting then these two last equalities, we get $\boxed{f(x)=ax}$ $\forall x$ and whatever is $a\in\mathbb R$, which indeed is a solution\end{tcolorbox}\end{tcolorbox}

Im actually privileged to have a solution basically like Pco's :D Congrats to both of us hehe
\end{solution}
*******************************************************************************
-------------------------------------------------------------------------------

\begin{problem}[Posted by \href{https://artofproblemsolving.com/community/user/125553}{lehungvietbao}]
	1) Given $\{a_n\}$ is a series of non-negative numbers and $f(n)=a_1+a_2+...+a_n-n\sqrt[n]{a_1a_2\cdot...\cdot a_n}$.
Prove that $f$ is an increasing function.

2)  For all reals $\alpha_1$, $\alpha_2$,..., ${\alpha_n}$ prove that:
\[\sum_{i=1}^n\sum_{j=1}^n\cos\left(\alpha_i-\alpha_j\right)\geq0\]
	\flushright \href{https://artofproblemsolving.com/community/c6h572025}{(Link to AoPS)}
\end{problem}



\begin{solution}[by \href{https://artofproblemsolving.com/community/user/29428}{pco}]
	\begin{tcolorbox}1) Given $\{a_n\}$ is a series of non-negative numbers and $f(n)=a_1+a_2+...+a_n-n\sqrt[n]{a_1a_2\cdot...\cdot a_n}$.
Prove that $f$ is an increasing function.\end{tcolorbox}
Let $t_n=(\prod_{k=1}^na_k)^{\frac 1{n(n+1)}}$
Let $g_n(x)=x^{n+1}-(n+1)t_n^nx+nt_n^{n+1}$

$g_n'(x)=(n+1)(x^n-t_n^n)$ so $g_n(x)$ is decreasing over $[0,t_n]$ and increasing over $[t_n,+\infty)$

And since $g_n(t_n)=0$, we get $g_n(x)\ge 0$ $\forall x\ge 0$

Then it remains to write $f(n+1)-f(n)=g_n(\sqrt[n+1]{a_{n+1}})\ge 0$

Q.E.D.
\end{solution}



\begin{solution}[by \href{https://artofproblemsolving.com/community/user/29428}{pco}]
	\begin{tcolorbox}2)  For all reals $\alpha_1$, $\alpha_2$,..., ${\alpha_n}$ prove that:
\[\sum_{i=1}^n\sum_{j=1}^n\cos\left(\alpha_i-\alpha_j\right)\geq0\]\end{tcolorbox}
The requested quantity is $\text{RE}\left(\sum_{k=1}^n\sum_{j=1}^ne^{i(\alpha_k-\alpha_j)}\right)$

$=\text{RE}\left(\left(\sum_{k=1}^ne^{i\alpha_k}\right)\left(\sum_{j=1}^ne^{-i\alpha_j}\right)\right)$

$=\text{RE}\left(\left(\sum_{k=1}^n\cos \alpha_k +i\sum_{k=1}^n\sin \alpha_k\right)\left(\sum_{k=1}^n\cos \alpha_k -i\sum_{k=1}^n\sin \alpha_k\right)\right)$

$=\left(\sum_{k=1}^n\cos \alpha_k\right)^2+\left(\sum_{k=1}^n\sin \alpha_k\right)^2$ $\ge 0$

Q.E.D.
\end{solution}



\begin{solution}[by \href{https://artofproblemsolving.com/community/user/164292}{babylon5}]
	1)Consider $f(n+1)$ as a function of the term $a_{n+1}$ so $f(n+1)=g(a_{n+1})$ and let $a=a_1a_2 \dots a_n$.

Now $g'(a_{n+1})=1-a^{\frac{1}{n+1}}\left(a_{n+1}\right)^{-\frac{n}{n+1}}$ with root $a_{n+1}=a^{\frac{1}{n}}$.The function $g$ takes minimum at this value of $a_{n+1}$ thus $g(a_{n+1})\geq g(a^{\frac{1}{n}})$ or $f(n+1) \geq f(n)$.

2)For a similar way as @pco did look at [url]http://www.artofproblemsolving.com/Forum/viewtopic.php?f=52&t=539697[\/url]
\end{solution}
*******************************************************************************
-------------------------------------------------------------------------------

\begin{problem}[Posted by \href{https://artofproblemsolving.com/community/user/68025}{Pirkuliyev Rovsen}]
	Given a function  $f(x)=\frac{a^x}{a^x+\sqrt{a}}$. Find $f(\frac{1}{2012})+f(\frac{2}{2012})+...+f(\frac{2011}{2012})$.
	\flushright \href{https://artofproblemsolving.com/community/c6h572064}{(Link to AoPS)}
\end{problem}



\begin{solution}[by \href{https://artofproblemsolving.com/community/user/29428}{pco}]
	\begin{tcolorbox}Given a function  $f(x)=\frac{a^x}{a^x+\sqrt{a}}$. Find $f(\frac{1}{2012})+f(\frac{2}{2012})+...+f(\frac{2011}{2012})$.\end{tcolorbox}
$f(x)+f(1-x)=1$. So requested sum is $1005+f(\frac 12)$ $=\boxed{\frac{2011}2}$
\end{solution}



\begin{solution}[by \href{https://artofproblemsolving.com/community/user/192463}{arkanm}]
	Let the sum be $S$. Just note that $f(x)+f(1-x)=\frac{a^x}{a^x+a^{\frac{1}{2}}}+\frac{a^{1-x}}{a^{1-x}+a^{\frac{1}{2}}}=1$, so $f(\frac{a}{2012})+f(\frac{2012-a}{2012})=1$. Hence we can pair the summands whose argument has numerators of: $(1,2011), (2,2010), ..., (1005, 1007)$ to get $S=1005+f(\frac{1006}{2012})=1005+f(\frac{1}{2})$. Now since $f(x)+f(1-x)=1$, we have $f(\frac{1}{2})+f(1-\frac{1}{2})=2f(\frac{1}{2})=1$, or $f(\frac{1}{2})=\frac{1}{2}$ (i.e. the function has a half-turn symmetry about $(\frac{1}{2}, \frac{1}{2})$). Thus, $S=1005+\frac{1}{2}=\boxed{\frac{2011}{2}}$.
\end{solution}



\begin{solution}[by \href{https://artofproblemsolving.com/community/user/192463}{arkanm}]
	Oops, beaten by none other than pco.

gg
\end{solution}
*******************************************************************************
-------------------------------------------------------------------------------

\begin{problem}[Posted by \href{https://artofproblemsolving.com/community/user/202360}{ANONYMOUS039}]
	1.Find all functions $f:\mathbb{R} \rightarrow \mathbb{R} $ that 
$f(f(x-y))=f(x)-f(y)$.
2.Find all functions $f:\mathbb{R} \rightarrow \mathbb{R} $ that 
$f(x-f(y))=f(f(y))+xf(y)+f(x)-1$ for all $x,y real$ numbers.
	\flushright \href{https://artofproblemsolving.com/community/c6h572121}{(Link to AoPS)}
\end{problem}



\begin{solution}[by \href{https://artofproblemsolving.com/community/user/190093}{KamalDoni}]
	1)nice but  easy )) , it's not difficult to find that f(f(x))-f(x)=constant so then f(x+y)+c =f(x)+f(y) then by Causchy's equation we came to 
f(x)=kx+c the checking answer is the end of solution
\end{solution}



\begin{solution}[by \href{https://artofproblemsolving.com/community/user/202360}{ANONYMOUS039}]
	im not getting why f(f(x))-f(x)=const.
\end{solution}



\begin{solution}[by \href{https://artofproblemsolving.com/community/user/190093}{KamalDoni}]
	hmm , we know that f(f(x-y)) = f(x)-f(y) and f(f(y))=f(x)-f(x-y) then by subtracting we got f(f(y))-f(y)=f(f(x-y))-f(x-y)
\end{solution}



\begin{solution}[by \href{https://artofproblemsolving.com/community/user/29428}{pco}]
	\begin{tcolorbox}1.Find all functions $f:\mathbb{R} \rightarrow \mathbb{R} $ that 
$f(f(x-y))=f(x)-f(y)$.\end{tcolorbox}
1) Claim about general solution
====================

Let $A,B$ any two supplementary vectorspaces of the $\mathbb Q$-vectorspace $\mathbb R$
Let $a(x)$ from $\mathbb R\to A$ an $b(x)$ from $\mathbb R\to B$ the projections of $x$ in $A,B$ so that $x=a(x)+b(x)$ in a unique manner.

Then $f(x)=a(x)$

2) Proof that any function in the form of 1) indeed is a solution
========================================

$f(x-y)=a(x)-a(y)$
$f(f(x-y))=a(x)-a(y)=f(x)-f(y)$
Q.E.D

3) Proof that any solution may be written in the form of 1), so that this indeed is a general solution
===================================================================
Let $f(x)$ from $\mathbb R\to\mathbb R$ such that assertion $P(x,y)$ : $f(f(x-y))=f(x)-f(y)$ is true $\forall x,y$

2.1) $f(x)$ is additive
---------------------
Let $a=f(0)$
Let $g(x)=f(x)-a$

$P(x+y,y)$ $\implies$ $f(f(x))=f(x+y)-f(y)$
$P(x,0)$ $\implies$ $f(f(x))=f(x)-a$
Subtracting, we get $f(x+y)=f(x)+f(y)-a$ and so $g(x+y)=g(x)+g(y)$ and $g(x)$ is additive.

$P(0,0)$ $\implies$ $g(a)=-a$
$P(a,0)$ $\implies$ $a=-a$ and so $a=0$
Q.E.D.

2.2) $f(x)$ may be written in the form of 1),
------------------------------------
Let $A=\{x$ such that $f(x)=x\}$. $A$ is a vectorspace.
Let $B=\{x$ such that $f(x)=0\}$. $B$ is a vectorspace.
$A\cap B=\{0\}$

Let $a(x)=f(x)$. $P(x,0)$ $\implies$ $f(f(x))=f(x)$ and so $f(x)\in A$ and $a(x)$ is a linear function from $\mathbb R\to A$
Let $b(x)=x-f(x)$. $f(b(x))=f(x)-f(f(x))=0$ and so $b(x)$ is a linear function from $\mathbb R\to B$

$a(x)+b(x)=x$  and so $A,B$ are two supplementary vectorspaces.
And $f(x)=a(x)$
Q.E.D.

3) some examples
============
3.1) trivial case $A=\{0\}$ and $B=\mathbb R$
--------------------------------------------
Then $a(x)=0$ and we got the trivial solution $f(x)=0$ $\forall x$

3.2) trivial case $A=\mathbb R$ and $B=\{0\}$ 
--------------------------------------------
Then $a(x)=x$ and we got the trivial solution $f(x)=x$ $\forall x$

3.3) ...
--------
And obviously infinitely many other solutions
\end{solution}



\begin{solution}[by \href{https://artofproblemsolving.com/community/user/202360}{ANONYMOUS039}]
	Thank you @pco and @KamalDoni  :)
\end{solution}



\begin{solution}[by \href{https://artofproblemsolving.com/community/user/29428}{pco}]
	\begin{tcolorbox}2.Find all functions $f:\mathbb{R} \rightarrow \mathbb{R} $ that 
$f(x-f(y))=f(f(y))+xf(y)+f(x)-1$ for all $x,y real$ numbers.\end{tcolorbox}
Let $P(x,y)$ be the assertion $f(x-f(y))=f(f(y))+xf(y)+f(x)-1$
Let $a=f(0)$
$f(x)=0$ $\forall x$ is not a solution. So let $t$ such that $f(t)\ne 0$

$P(\frac{x+1-f(f(t))}{f(t)},t)$ $\implies$ $f\left(\frac{x+1-f(f(t))}{f(t)}-f(t)\right)-f\left(\frac{x+1-f(f(t))}{f(t)}\right)=x$
So any real $x$ may be written as $f(u)-f(v)$ for some real $u,v$

$P(f(u),u)$ $\implies$ $f(f(u))=\frac{a+1-f(u)^2}2$
$P(f(v),v)$ $\implies$ $f(f(v))=\frac{a+1-f(v)^2}2$

And so $P(f(u),v)$ $\implies$ $f(f(u)-f(v))=a-\frac{(f(u)-f(v))^2}2$

And so, since any real $x$ may be written as $f(u)-f(v)$ for some $u,v$ : $f(x)=a-\frac{x^2}2$ $\forall x$

Plugging this back in original equation, we get $a=1$ and so the unique solution $\boxed{f(x)=1-\frac{x^2}2}$ $\forall x$
\end{solution}
*******************************************************************************
-------------------------------------------------------------------------------

\begin{problem}[Posted by \href{https://artofproblemsolving.com/community/user/180537}{icp}]
	Given a function $f(x)=\sqrt[3]{\frac{x^3-3x+(x^2-1)\sqrt{x^2-4}}{2}}+\sqrt[3]{\frac{x^3-3x-(x^2-1)\sqrt{x^2-4}}{2}}$.Find$f(\sqrt[4]{2014})$
	\flushright \href{https://artofproblemsolving.com/community/c6h572387}{(Link to AoPS)}
\end{problem}



\begin{solution}[by \href{https://artofproblemsolving.com/community/user/29428}{pco}]
	\begin{tcolorbox}Given a function $f(x)=\sqrt[3]{\frac{x^3-3x+(x^2-1)\sqrt{x^2-4}}{2}}+\sqrt[3]{\frac{x^3-3x-(x^2-1)\sqrt{x^2-4}}{2}}$.

Find$f(\sqrt[4]{2014})$\end{tcolorbox}
Note that $f(x)^3=x^3-3x+3f(x)$ and so $f(x)^3-3f(x)=x^3-3x$ and since $x^3-3x$ is injective over $(2,+\infty)$ : $f(x)=x$ $\forall x>2$

Hence the result : $\boxed{\sqrt[4]{2014}}$
\end{solution}
*******************************************************************************
-------------------------------------------------------------------------------

\begin{problem}[Posted by \href{https://artofproblemsolving.com/community/user/65976}{mudok}]
	Find all functions $f: R\to R$ such that \[xf(y)+yf(x)=(x+y)f(x)f(y)\] for all $x,y\in R$
	\flushright \href{https://artofproblemsolving.com/community/c6h572541}{(Link to AoPS)}
\end{problem}



\begin{solution}[by \href{https://artofproblemsolving.com/community/user/187896}{Ashutoshmaths}]
	Taking $y=x$
gives
$f(x)=0\text {or} 1\forall x\in R$
Am i right? :maybe:
\end{solution}



\begin{solution}[by \href{https://artofproblemsolving.com/community/user/29428}{pco}]
	\begin{tcolorbox}Taking $y=x$
gives
$f(x)=0\text {or} 1\forall x\in R$
Am i right? :maybe:\end{tcolorbox}
No. Taking $y=x$ implies $xf(x)(f(x)-1)=0$ and so $\forall x\ne 0$, either $f(x)=0$, either $f(x)=1$

And job is not finished.
\end{solution}



\begin{solution}[by \href{https://artofproblemsolving.com/community/user/204311}{Onlygodcanjudgeme}]
	if y=0 we take f(0) = 0, and f(x) = 1. 
Firstly let we take f(0) =0 then we now that function is even.
let (x,y) = (1,1) then f(1) =0 or f(1) = 1, 
if f(1) =0 then we now that f(x) =0 .
if f(1) =1 then we now that all of the function is equal to 1
\end{solution}
*******************************************************************************
-------------------------------------------------------------------------------

\begin{problem}[Posted by \href{https://artofproblemsolving.com/community/user/155470}{fandogh}]
	find all functions $f:R^+ \rightarrow R^+$ such taht for all $x,y>0$:
$xf(x)+yf(y)=f(x+y)$
	\flushright \href{https://artofproblemsolving.com/community/c6h572549}{(Link to AoPS)}
\end{problem}



\begin{solution}[by \href{https://artofproblemsolving.com/community/user/29428}{pco}]
	\begin{tcolorbox}find all functions $f:R^+ \rightarrow R^+$ such taht for all $x,y>0$:
$xf(x)+yf(y)=f(x+y)$\end{tcolorbox}
Let $P(x,y)$ be the assertion $xf(x)+yf(y)=f(x+y)$

$P(x,x)$ $\implies$ $2xf(x)=f(2x)$
So $P(2x,2y)$ $\implies$ $2xf(2x)+2yf(2y)=f(2x+2y)$ $\implies$ $4x^2f(x)+4y^2f(y)=2(x+y)f(x+y)$
$P(x,y)$ $\implies$ $xf(x)+yf(y)=f(x+y)$ $\implies$ $2x(x+y)f(x)+2y(x+y)f(y)=2(x+y)f(x+y)$

Subtracting, we get $x(x-y)f(x)+y(y-x)f(y)=0$ and so $xf(x)=yf(y)$ $\forall x\ne y$

Setting $y=1$, we get $f(x)=\frac{f(1)}x$ $\forall x\ne 1$, which unfortunately is never a solution.

So \begin{bolded}no solution for this functional equation\end{underlined}\end{bolded}.
\end{solution}
*******************************************************************************
-------------------------------------------------------------------------------

\begin{problem}[Posted by \href{https://artofproblemsolving.com/community/user/155470}{fandogh}]
	find all functions $f:R^+ \rightarrow R^+$ such taht for all $x,y>0$:
$xf(x)+yf(y)=f(x^2+y^2)$
	\flushright \href{https://artofproblemsolving.com/community/c6h572567}{(Link to AoPS)}
\end{problem}



\begin{solution}[by \href{https://artofproblemsolving.com/community/user/29428}{pco}]
	\begin{tcolorbox}find all functions $f:R^+ \rightarrow R^+$ such taht for all $x,y>0$:
$xf(x)+yf(y)=f(x^2+y^2)$\end{tcolorbox}
Let $P(x,y)$ be the assertion $xf(x)+yf(y)=f(x^2+y^2)$

$P(x,x)$ $\implies$ $2xf(x)=f(2x^2)$ and so $P(x,y)$ may be  written $f(x^2+y^2)=\frac{f(2x^2)+f(2y^2)}2$

And so $f(\frac{x+y}2)=\frac{f(x)+f(y)}2$ $\forall x,y>0$

Let positive real numbers $a\ne b$. Let $E_{a,b}=\{x>0$ such that $\exists m,n\in\mathbb Z,p\in\mathbb N$ such that $x=\frac{ma+nb}{2^p}\}$

Easy induction gives $f(x)=\alpha_{a,b}x+\beta_{a,b}$ $\forall x\in E_{a,b}$ for some real numbers $\alpha_{a,b},\beta_{a,b}$ depending on $a,b$

If $f(x)$ is not continuous at some point $u$, it's possible to find $v_k\to u$ such that the above line defined  from $v_k, u$ is negative for some $x>0$, which is impossible.

So $fx)$ is continuous and $f(x)=\alpha x+\beta$. 

Plugging this back in original equation, we get $\beta=0$ and $\alpha >0$.

Hence the answer : $\boxed{f(x)=\alpha x}$ $\forall x$, and whatever is $\alpha >0$
\end{solution}
*******************************************************************************
-------------------------------------------------------------------------------

\begin{problem}[Posted by \href{https://artofproblemsolving.com/community/user/201392}{nima-amini}]
	find all Strictly increasing functions $f:\mathbb{R}^{+}\rightarrow \mathbb{R}$ 

$i) f(x)> \frac{-1}{x} (\forall x> 0)$ 
$ii) f(x)f(f(x)+\frac{1}{x}) =1 (\forall x> 0)$
	\flushright \href{https://artofproblemsolving.com/community/c6h572661}{(Link to AoPS)}
\end{problem}



\begin{solution}[by \href{https://artofproblemsolving.com/community/user/29428}{pco}]
	\begin{tcolorbox}find all Strictly increasing functions $f:\mathbb{R}^{+}\rightarrow \mathbb{R}$ 

$i) f(x)> \frac{-1}{x} (\forall x> 0)$ 
$ii) f(x)f(f(x)+\frac{1}{x}) =1 (\forall x> 0)$\end{tcolorbox}
See http://www.artofproblemsolving.com/Forum/viewtopic.php?f=37&t=571356
\end{solution}
*******************************************************************************
-------------------------------------------------------------------------------

\begin{problem}[Posted by \href{https://artofproblemsolving.com/community/user/186084}{acupofmath}]
	$ find $  $ all $  $ real $  $ functions $  $ f $ $ and $ $ g $  $ such $ $ that $ $ f(x)-f(y)=(x-y).g(x+y) $  :read:

[hide=" Hint!"]

$ f(x)-f(0) =(x-0).g(x+0) \Rightarrow f(x)-f(y)=(x-y).g(x+y) =x.g(x)-y.g(y) $

$ x.f(x) - y.g(y) = (x-y).g(x+y) $ :)

[\/hide]
	\flushright \href{https://artofproblemsolving.com/community/c6h572695}{(Link to AoPS)}
\end{problem}



\begin{solution}[by \href{https://artofproblemsolving.com/community/user/29428}{pco}]
	\begin{tcolorbox}$ find $  $ all $  $ real $  $ functions $  $ f $ $ and $ $ g $  $ such $ $ that $ $ f(x)-f(y)=(x-y).g(x+y) $\end{tcolorbox}
Setting $y=0$, we get $f(x)=f(0)+xg(x)$ and equation becomes assertion $P(x,y)$ : $xg(x)-yg(y)=(x-y)g(x+y)$

$P(\frac{x+1}2,\frac{x-1}2)$ $\implies$ $\frac{x+1}2g(\frac{x+1}2)-\frac{x-1}2g(\frac{x-1}2)=g(x)$

$P(\frac{x-1}2,\frac{1-x}2)$ $\implies$ $\frac{x-1}2g(\frac{x-1}2)-\frac{1-x}2g(\frac{1-x}2)=(x-1)g(0)$

$P(\frac{1-x}2,\frac{x+1}2)$ $\implies$ $\frac{1-x}2g(\frac{1-x}2)-\frac{x+1}2g(\frac{x+1}2)=-xg(1)$

Adding these three lines, we get $g(x)=x(g(1)-g(0))+g(0)$ and so $g(x)=ax+b$ which indeed is a solution, whatever are $a,b$

Hence the solution $\boxed{(f(x),g(x))=(ax^2+bx+c,ax+b)}$
\end{solution}



\begin{solution}[by \href{https://artofproblemsolving.com/community/user/186084}{acupofmath}]
	wow! very fast! :)
\end{solution}
*******************************************************************************
-------------------------------------------------------------------------------

\begin{problem}[Posted by \href{https://artofproblemsolving.com/community/user/174113}{victory1204}]
	Find all f:R->R such that for all x,y,z $\in$ R:
f(xf(y))+f(yf(z))+f(zf(x))=xy+yz+zx
	\flushright \href{https://artofproblemsolving.com/community/c6h572999}{(Link to AoPS)}
\end{problem}



\begin{solution}[by \href{https://artofproblemsolving.com/community/user/29428}{pco}]
	\begin{tcolorbox}Find all f:R->R such that for all x,y,z $\in$ R:
f(xf(y))+f(yf(z))+f(zf(x))=xy+yz+zx\end{tcolorbox}
$f(x)=0$ $\forall x$ is not a solution. So let $u$ such that $f(u)\ne 0$
Let $P(x,y,z)$ be the assertion $f(xf(y))+f(yf(z))+f(zf(x))=xy+yz+zx$

$P(0,0,0)$ $\implies$ $f(0)=0$

$P(\frac x{f(u)},u,0)$ $\implies$ $f(x)=\frac u{f(u)}x$.

Plugging $f(x)=ax$ back in original equation, we get $a^2=1$ and so two solutions :

$\boxed{f(x)=x}$ $\forall x$ and $\boxed{f(x)=-x}$ $\forall x$
\end{solution}
*******************************************************************************
-------------------------------------------------------------------------------

\begin{problem}[Posted by \href{https://artofproblemsolving.com/community/user/68025}{Pirkuliyev Rovsen}]
	Find all pairs of functions $(f,h)$, $f: \mathbb{R}\to\mathbb{R}$ and $h: \mathbb{R}\to\mathbb{R}$ such that $f(x^2+yh(x))=xh(x)+f(xy)$.
	\flushright \href{https://artofproblemsolving.com/community/c6h573000}{(Link to AoPS)}
\end{problem}



\begin{solution}[by \href{https://artofproblemsolving.com/community/user/29428}{pco}]
	\begin{tcolorbox}Find all pairs of functions $(f,h)$, $f: \mathbb{R}\to\mathbb{R}$ and $h: \mathbb{R}\to\mathbb{R}$ such that $f(x^2+yh(x))=xh(x)+f(xy)$.\end{tcolorbox}
Let $P(x,y)$ be the assertion $f(x^2+yh(x))=xh(x)+f(xy)$

If $h(u)=0$ for some $u\ne 0$, then $P(u,\frac xu)$ $\implies$ $f(x)=f(u^2)$ $\forall x$ and the solution :

$\boxed{\text{S1: }f(x)=c\text{  }\forall x\text{ and }h(x)=0\text{  }\forall x\ne 0\text{ and }h(0)\text{ is any value}}$

So let us from now consider that $h(x)\ne 0$ $\forall x\ne 0$

Let $x\ne 0$. If $h(x)\ne x$, then $P(x,\frac{x^2}{x-h(x)})$ $\implies$ $h(x)=0$, impossible and so $h(x)=x$ $\forall x\ne 0$

Functional equation is then new assertion $Q(x,y)$ : $f(x^2+xy)=x^2+f(xy)$ $\forall x\ne 0$, $\forall y$, still true when $x=0$

$Q(x,0)$ $\implies$ $f(x^2)=x^2+f(0)$
$Q(x,-x)$ $\implies$ $f(-x^2)=f(0)-x^2$
And so $f(x)=x+a$ which indeed is a solution, whatever is $a\in\mathbb R$ and so the solution ($h(0)=0$ is easy to get):

$\boxed{\text{S2: }f(x)=x+a\text{  }\forall x\text{ and }h(x)=x\text{  }\forall x}$
\end{solution}
*******************************************************************************
-------------------------------------------------------------------------------

\begin{problem}[Posted by \href{https://artofproblemsolving.com/community/user/68025}{Pirkuliyev Rovsen}]
	Find all functions  $f: \mathbb{Z}\to\mathbb{Z}$ such that $f(x+f(f(y)))=-f(f(x+1))-y$ for all $x,y{\in}Z$.
	\flushright \href{https://artofproblemsolving.com/community/c6h573152}{(Link to AoPS)}
\end{problem}



\begin{solution}[by \href{https://artofproblemsolving.com/community/user/29428}{pco}]
	\begin{tcolorbox}Find all functions  $f: \mathbb{Z}\to\mathbb{Z}$ such that $f(x+f(f(y)))=-f(f(x+1))-y$ for all $x,y{\in}Z$.\end{tcolorbox}
Let $P(x,y)$ be the assertion $f(x+f(f(y)))=-f(f(x+1))-y$

$f(x)$ is surjective. So $f(f(x))$ is surjective. Let then $a$ such that $f(f(a))=1$

$P(x-1,a)$ $\implies$ $f(f(x))=a-f(x)$ and so, since surjective, $f(x)=a-x$ $\forall x$

Plugging this back in original equation, we get $\boxed{f(x)=-1-x}$ $\forall x$
\end{solution}
*******************************************************************************
-------------------------------------------------------------------------------

\begin{problem}[Posted by \href{https://artofproblemsolving.com/community/user/185787}{gobathegreat}]
	Find all such functions $f$ which map set of real numbers into itself for which holds $ f(f(x-y)) = f(x)f(y)-f(x)+f(y)-xy $ for every real $x$ and $y$.
	\flushright \href{https://artofproblemsolving.com/community/c6h573331}{(Link to AoPS)}
\end{problem}



\begin{solution}[by \href{https://artofproblemsolving.com/community/user/29428}{pco}]
	\begin{tcolorbox}Find all such functions $f$ which map set of real numbers into itself for which holds $ f(f(x-y)) = f(x)f(y)-f(x)+f(y)-xy $ for every real $x$ and $y$.\end{tcolorbox}
Let $P(x,y)$ be the assertion $f(f(x-y))=f(x)f(y)-f(x)+f(y)-xy$
Let $a=f(0)$

Note that $f(x)$ is not bounded. So $\exists u$ such that $f(x)\notin\{0,a\}$

Comparing $P(x,x)$ with $P(-x,-x)$, we get $f(x)^2=f(-x)^2$ and so $f(-x)=\pm f(x)$

$P(u,0)$ $\implies$ $f(f(u))=(a-1)f(u)+a$
$P(0,u)$ $\implies$ $f(f(-u))=(a+1)f(u)-a$ and so :
Either $f(f(-u))=f(f(u))$ and so $(a+1)f(u)-a=(a-1)f(u)+a$ and so $f(u)=a$, impossible
Either $f(f(-u))=-f(f(u))$ and so $(a+1)f(u)-a=-(a-1)f(u)-a$ and so $af(u)=0$ and so $a=0$

$P(x,x)$ $\implies$ $f(x)^2=x^2$ and so :
$\forall x$, either $f(x)=x$, either $f(x)=-x$

If $\exists x,y\notin\{-1,0\}$ such that $f(x)=x$ and $f(y)=-y$, then :
$P(x,y)$ $\implies$ $f(f(x-y))=-2xy-x-y$ and so :
either $f(f(x-y))=x-y$ and so $x-y=-2xy-x-y$ and so $y=-1$, impossible
either $f(f(x-y))=-x+y$ and so $-x+y=-2xy-x-y$ and so $x=-1$, impossible

And so :
Either $f(x)=x$ $\forall x\ne -1$ and then $P(2,1)$ implies contradiction
Either $f(x)=-x$ $\forall x\ne -1$ and then $P(-1,1)$ $\implies$ $f(-1)=1$

And so $\boxed{f(x)=-x}$ $\forall x$ which indeed is a solution.
\end{solution}



\begin{solution}[by \href{https://artofproblemsolving.com/community/user/197870}{amirahul}]
	I think i solve it.Let me explain.

Let x & y both be 0.then f(f(0))=f(0)f(0).
Again let x=y.then f(f(0))=f(x)f(x)-x^2.
f(0)f(0)=(f(x)+x)(f(x)-x).=>f(x)=x+f(0) or f(x)=-x+f(0).
Let f(0)=k (constant) then f(x)=x+k or f(x)=-x+k.
1st case........
Let f(x)=x+k then the condition becomes
x-y+2k=(x+k)(y+k)-(x+k)+(y+k)-xy => k^2+(x+y)k-2(x-y)=0 this is a quadratic eqn of k which can be solved easily.But solving the eqn we cant get k as a constant.So contradiction.
2nd case,
Now let f(x)=-x+k.doing as 1st case we get the quadratic eqn as follows
k^2-(x+y)k=0.
from this we get k=0 or k=(x+y).But k is a constant.So k=0.It concluds that f(x)=-x
Please tell if I am wrong.
\end{solution}



\begin{solution}[by \href{https://artofproblemsolving.com/community/user/202863}{Mathdoctor41}]
	f(x+y)=f(x)+f(y)+xy(x+y) Would you assist me?
\end{solution}



\begin{solution}[by \href{https://artofproblemsolving.com/community/user/29428}{pco}]
	\begin{tcolorbox}I think i solve it.Let me explain.

Let x & y both be 0.then f(f(0))=f(0)f(0).
Again let x=y.then f(f(0))=f(x)f(x)-x^2.
f(0)f(0)=(f(x)+x)(f(x)-x).=>f(x)=x+f(0) or f(x)=-x+f(0).
....\end{tcolorbox}
No, this implies either $f(x)=\sqrt{x^2+f(0)^2}$, either $f(x)=-\sqrt{x^2+f(0)^2}$
\end{solution}
*******************************************************************************
-------------------------------------------------------------------------------

\begin{problem}[Posted by \href{https://artofproblemsolving.com/community/user/68025}{Pirkuliyev Rovsen}]
	Find all functions ${f: \mathbb[0;+\infty)}\to\mathbb{R}$ such that $f(x)+\sqrt{f^2([ x ])+f^2(\{ x\})}=x$ for all $x{\in}[0;+\infty)$.
	\flushright \href{https://artofproblemsolving.com/community/c6h573455}{(Link to AoPS)}
\end{problem}



\begin{solution}[by \href{https://artofproblemsolving.com/community/user/29428}{pco}]
	\begin{tcolorbox}Find all functions ${f: \mathbb[0;+\infty)}\to\mathbb{R}$ such that $f(x)+\sqrt{f^2([ x ])+f^2(\{ x\})}=x$ for all $x{\in}[0;+\infty)$.\end{tcolorbox}
I suppose $f^2(.)$ means $(f(.))^2$ and not $f(f(.))$

If so, let $P(x)$ be the assertion $f(x)+\sqrt{f(\lfloor x\rfloor)^2+f(\{x\})^2}=x$

$P(0)$ $\implies$ $f(0)+\sqrt 2|f(0)|=0$ and so $f(0)=0$

Let $x\in(0,1)$ : $P(x)$ $\implies$ $f(x)+|f(x)|=x$ and so $f(x)=\frac x2$ : 

Let $x\in\mathbb Z_{\ge 0}$ $P(x)$ $\implies$ $f(x)+|f(x)|=x$ and so $f(x)=\frac x2$

And so $\boxed{f(x)=x-\frac 12\sqrt{\lfloor x\rfloor^2+\{x\}^2}}$
\end{solution}
*******************************************************************************
-------------------------------------------------------------------------------

\begin{problem}[Posted by \href{https://artofproblemsolving.com/community/user/185787}{gobathegreat}]
	Find all functions which map set of real numbers to itself and $ f(x+y)=f(x)+f(y)+f(xy) $ for all real x and y.
	\flushright \href{https://artofproblemsolving.com/community/c6h573472}{(Link to AoPS)}
\end{problem}



\begin{solution}[by \href{https://artofproblemsolving.com/community/user/29428}{pco}]
	\begin{tcolorbox}Find all functions which map set of real numbers to itself and $ f(x+y)=f(x)+f(y)+f(xy) $ for all real x and y.\end{tcolorbox}
Let $P(x,y)$ be the assertion $f(x+y)=f(x)+f(y)+f(xy)$
Let $a=f(1)$

$P(0,0)$ $\implies$ $f(0)=0$
$P(x,1)$ $\implies$ $f(x+1)=2f(x)+a$ and so $f(x+n)=2^nf(x)+(2^n-1)a$ $\forall x$ and $\forall n\in\mathbb Z$
So $f(n)=(2^n-1)a$ $\forall n\in\mathbb Z$

Then $P(x,n)$ $\implies$ new assertion $Q(x,n)$ ; $f(nx)=(2^n-1)f(x)$

$Q(x,2)$ $\implies$ $f(2x)=3f(x)$
$Q(2x,2)$ $\implies$ $f(4x)=9f(x)$
$Q(x,4)$ $\implies$ $f(4x)=15f(x)$

And so $\boxed{f(x)=0}$ $\forall x$, which indeed is a solution.
\end{solution}



\begin{solution}[by \href{https://artofproblemsolving.com/community/user/184652}{CanVQ}]
	\begin{tcolorbox}Find all functions which map set of real numbers to itself and $ f(x+y)=f(x)+f(y)+f(xy) $ for all real x and y.\end{tcolorbox}
Let $x=y=0,$ we get $f(0)=0.$ Now, for any $x,\,y,\,z \in \mathbb R,$ we have \[\begin{aligned} f(x+y+z)&=f(x)+f(y+z)+f\big(x(y+z)\big)\\ &=f(x)+f(y)+f(z)+f(yz)+f(xy)+f(xz)+f(x^2yz).\quad (1) \end{aligned}\] Changing the position of $x$ and $y$ in $(1),$ we get \[f(x^2yz)=f(y^2zx),\quad \forall x,\,y,\,z \in \mathbb R. \quad (2)\] Now, let $x,\,y \ne  0$ and take $z=\frac{1}{xy}$ in $(2),$ we have \[f(x)=f(y),\quad \forall x,\,y \in \mathbb R ^*.\quad (3)\] So $f(x)=c,\, \forall x \in \mathbb R^*.$ Plugging this result into the original equation, we get $c=0.$ So we have $f(x)=0,\, \forall x \in \mathbb R.$
\end{solution}
*******************************************************************************
-------------------------------------------------------------------------------

\begin{problem}[Posted by \href{https://artofproblemsolving.com/community/user/203659}{the_boss}]
	$1.$ find all real valued function $f$ such that $f(\sqrt{x^2+y^2})=f(x)f(y)$

$2.$ find all real valued function $f$ and $g$ such that $f(x+y)+f(x-y)=f(x)g(y)$
	\flushright \href{https://artofproblemsolving.com/community/c6h573497}{(Link to AoPS)}
\end{problem}



\begin{solution}[by \href{https://artofproblemsolving.com/community/user/29428}{pco}]
	\begin{tcolorbox}$1.$ find all real valued function $f$ such that $f(\sqrt{x^2+y^2})=f(x)f(y)$\end{tcolorbox}
Let $P(x,y)$ be the assertion $f(\sqrt{x^2+y^2})=f(x)f(y)$

If $f(0)=0$, $P(x,0)$ $\implies$ $f(|x|)=0$ $\forall x$ and then $P(x,x)$ $\implies$ $\boxed{S_1\text{  :  }f(x)=0\text{  }\forall x}$, which indeed is a solution.

If $f(0)\ne 0$ :
$P(0,0)$ $\implies$ $f(0)=1$
$P(x,0)$ $\implies$ $f(|x|)=f(x)$ and so $f(x)$ is even.

If $f(u)=0$ for some $u$, WLOG $u>0$, then :
$P(u,x)$ $\implies$ $f(x)=0$ $\forall x\ge u$
$P(u2^{-\frac 12},u2^{-\frac 12})$ $\implies$ $f(u2^{-\frac 12})=0$ and so $f(u2^{-\frac n2})=0$ and so (using previous line) $f(x)=0$ $\forall x>0$
Hence a solution $\boxed{S_2\text{  :  }f(0)=1\text{  and  }f(x)=0\text{  }\forall x\ne 0}$ which indeed is a solution

If $f(x)\ne 0$ $\forall x$, then $P(x2^{-\frac 12},x2^{-\frac 12})$ $\implies$ $f(x)>0$ $\forall x$
Let then $f(x)=e^{g(x^2)}$ and we get $g(x+y)=g(x)+g(y)$

Hence a solution $\boxed{f(x)=e^{a(x^2)}\text{  }\forall x}$ which indeed is a solution, whatever is additive $a(x)$ (solution of Cauchy additive equation).
\end{solution}



\begin{solution}[by \href{https://artofproblemsolving.com/community/user/16261}{Rust}]
	1. $f(-x)=f(x)$. If $f(o)=0$, then $f(x)\equiv 0$ else $f(0)=1$.
Let $g(x)=f(\sqrt x)$. Then $g(x+y)=f(\sqrt{x+y})=f(\sqrt x)f(\sqrt y)=g(x)g(y)$. $g(x)=g(\frac x2)^2>0$
Let $\phi(x)=\ln(g(x)$. Then $\phi(x+y)=\phi(x)+\phi(y)$. If consider only continiusly functions, solution is $f(x)=exp(ax^2)$.

2. $f(x)\equiv 0, g(y)- any$ solution.
Else $g(0)=2, g(-y)=g(y)$.
If $g(y)\equiv 2$, then $f(x)=ax+b$.
Else solution $f(x)=Asin(ax),g(y)=2cos(ay)$.
\end{solution}



\begin{solution}[by \href{https://artofproblemsolving.com/community/user/1430}{JBL}]
	Nearly 5000 posts and you don't know to put a "\" before the letters "sin", "cos" or "exp"?  :P
\end{solution}



\begin{solution}[by \href{https://artofproblemsolving.com/community/user/203659}{the_boss}]
	\begin{tcolorbox}Else solution $f(x)=Asin(ax),g(y)=2cos(ay)$.\end{tcolorbox}

how do you find this solution?
\end{solution}



\begin{solution}[by \href{https://artofproblemsolving.com/community/user/16261}{Rust}]
	Let $f(x)\not =0$.
1)$f(x+y_1+y_2)+f(x-y_1-y_2)=f(x)g(y_1+y_2)$,
2)$f(x+y_1-y_2)+f(x-y_1+y_2)=f(x)g(y_1-y_2)$,
3)$f(x+y_2+y_1)+f(x+y_2-y_1)=f(x+y_2)g(y_1)$,
4)$f(x-y_2+y_1)+f(x-y_2-y_1)=f(x-y_2)g(y_1)$.
From 3),4) 
$A=f(x+y_1+y_2)+f(x+y_2-y_1)+f(x-y_2+y_1)+f(x-y_2-y_1)=g(y_1)(f(x+y_2)+f(x-y_2))=g(y_1)g(y_2)f(x_1).$
From 1),2) 
$A=(g(y_1+y_2)+g(y_1-y_2))f(x)$.
Because $f(x)\not \equiv 0$
5)$g(y_1+y_2)+g(y_1-y_2)=g(y_1)g(y_2)$.
We know, that $g(0)=2,g(-y)=g(y)$.
Let $g(y)$ is continiosly and for small $y$ $1<g(y)<2$. We can find $a=a(y)$, suth that $g(y)=2\cos(ay)$.
Then we cheked, that $g(2y)=2\cos(a2y), g(3y)=g(y+2y)=2\cos(a3y),...$
For $y_1=qy_2, q\in Q, q\not =0$ must be $a(y_1)=a(y_2)$. By continiously $g(y)=2\cos(ay)$ for all y and $f(x)=A\sin(ax+b).
 If $g(y_1)>2$, then $g(y)=2\ch(ay)$. In this case $f(x)=A\sh(ax+b).$
\end{solution}



\begin{solution}[by \href{https://artofproblemsolving.com/community/user/29428}{pco}]
	Quite wrong.

This is one of the very classical d'Alembert functional equations and you need continuity to get these conclusions.

And this (likely not real) problem does not indicate continuity.
\end{solution}



\begin{solution}[by \href{https://artofproblemsolving.com/community/user/16261}{Rust}]
	\begin{tcolorbox}Let $f(x)\not =0$.
Let $g(y)$ is continiosly and for small $y$ $1<g(y)<2$. We can find $a=a(y)$, suth that $g(y)=2\cos(ay)$.
Then we cheked, that $g(2y)=2\cos(a2y), g(3y)=g(y+2y)=2\cos(a3y),...$
For $y_1=qy_2, q\in Q, q\not =0$ must be $a(y_1)=a(y_2)$. By continiously $g(y)=2\cos(ay)$ for all y and $f(x)=A\sin(ax+b)$.
 If $g(y_1)>2$, then $g(y)=2\ch(ay)$. In this case $f(x)=A\sh(ax+b).$\end{tcolorbox}
I wrote, that I consider only continiously functions.
\end{solution}



\begin{solution}[by \href{https://artofproblemsolving.com/community/user/29428}{pco}]
	\begin{tcolorbox}...
I wrote, that I consider only continiously functions.\end{tcolorbox}
So you did not solve the required problem.
\end{solution}



\begin{solution}[by \href{https://artofproblemsolving.com/community/user/16261}{Rust}]
	there are infinetely many solutions without it as for problem 1.
\end{solution}
*******************************************************************************
-------------------------------------------------------------------------------

\begin{problem}[Posted by \href{https://artofproblemsolving.com/community/user/177353}{Parnpaniti}]
	find all function f : $\Re^+ \rightarrow \Re$ , f(1)=2000 , f($x^2$)+f($y^2$)=2f(xy)
	\flushright \href{https://artofproblemsolving.com/community/c6h573927}{(Link to AoPS)}
\end{problem}



\begin{solution}[by \href{https://artofproblemsolving.com/community/user/29428}{pco}]
	\begin{tcolorbox}find all function f : $\Re^+ \rightarrow \Re$ , f(1)=2000 , f($x^2$)+f($y^2$)=2f(xy)\end{tcolorbox}
Let $g(x)=f(e^{2x})$ from $\mathbb R\to\mathbb R$ and functional equation becomes $g(\frac{x+y}2)=\frac{g(x)+g(y)}2$ $\forall x,y\in\mathbb R$

Solution of this functional equation is very classical  : $g(x)=h(x)+b$ where $h(x)$ is any solution of additive Cauchy equation.

Hence the answer : $\boxed{f(x)=a(\ln x)+2000}$ $\forall x$, which indeed is a solution, whatever is $a(x)$ solution of additive Cauchy equation.
\end{solution}



\begin{solution}[by \href{https://artofproblemsolving.com/community/user/177353}{Parnpaniti}]
	I can't solve  $g(\frac{x+y}2)=\frac{g(x)+g(y)}2$ .Can you solve it?
\end{solution}



\begin{solution}[by \href{https://artofproblemsolving.com/community/user/109988}{abdelkrim-amine}]
	\begin{tcolorbox}I can't solve  $g(\frac{x+y}2)=\frac{g(x)+g(y)}2$ .Can you solve it?\end{tcolorbox}

let $P(x,y) \Rightarrow g(\frac{x+y}2)=\frac{g(x)+g(y)}2$

$P(x,0) \Rightarrow g(\frac{x}2)=\frac{g(x)+g(0)}2$

so $g(\frac{x}2+\frac{y}2)=\frac{g(x)+g(y)}2=g(\frac{x}2)+g(\frac{y}2)-g(0)$

than $g(X+Y)-g(0)=g(X)-g(0)+g(Y)-g(0)$

pose that $h(x)=g(x)-g(0)$ we have $h(x+y)=h(x)+h(y)$ Cauchy's functional equation
\end{solution}
*******************************************************************************
-------------------------------------------------------------------------------

\begin{problem}[Posted by \href{https://artofproblemsolving.com/community/user/174113}{victory1204}]
	Find all f:Z->Z such that
f(m+n)+f(mn-1)=f(m)f(n)+a ;a is not equal to 0
	\flushright \href{https://artofproblemsolving.com/community/c6h573996}{(Link to AoPS)}
\end{problem}



\begin{solution}[by \href{https://artofproblemsolving.com/community/user/192463}{arkanm}]
	Let $P(x,y)$ be the assertion $f(x+y)+f(xy-1)=f(x)f(y)+a$

$P(x,0)\implies f(x)+f(-1)=f(x)f(0)+a\implies f(x)(1-f(0))=a-f(-1)$
There are two cases :

\begin{bolded}Case 1.\end{bolded} $f(0)=1$, and so $f(-1)=a$
For this case, I'm getting these weird polynomials :
$f(0)=1$
$f(\pm 1)=a$
$f(\pm 2)=a^2+a-1$
$f(\pm 3)=a^3+a^2-a$
$f(\pm 4)=a^4+a^3-2a^2+1$
$f(\pm 5)=a^5+a^4-3a^3-a^2+3a$
$f(\pm 6)=a^6+a^5-4a^4-2a^3+5a^2+a-1$
and etc.

The second case is easy, just let $f(x)=c\in \bf Z$ and then solve.
\end{solution}



\begin{solution}[by \href{https://artofproblemsolving.com/community/user/29428}{pco}]
	\begin{tcolorbox}Find all f:Z->Z such that
f(m+n)+f(mn-1)=f(m)f(n)+a ;a is not equal to 0\end{tcolorbox}
Let $P(x,y)$ be the assertion $f(x+y)+f(xy-1)=f(x)f(y)+a$

1) Constant solutions
==============
If $x^2-2x+a=0$ has integer solutions (in fact if $a=1-n^2$), then we have the solutions :
$\boxed{S_1 : a=1-n^2\text{ gives }f(x)=1+n\text{  }\forall n}$

$\boxed{S_2 : a=1-n^2\text{ gives }f(x)=1-n\text{  }\forall n}$

2)Non constant solutions
=================
Let $b=f(1)$

2.1) $f(0)=1$ and $f(-1)=a$ and $b=a$ and $a\in\{-2,-1,1,2\}$
--------------------------------------------------------------------------
If $f(0)\ne 1$ : $P(x,0)$ $\implies$ $f(-1)-a=f(x)(f(0)-1)$ and so $f(x)$ is contant, impossible. So $f(0)=1$ and $f(-1)=a$

$P(-1,1)$ $\implies$  $f(-2)=ab+a-1$
$P(-1,-1)$ $\implies$  $f(-2)=a^2+a-1$
Subtracting, we get $b=a$ (remember $a\ne 0$)

$f(0)=1$
$f(1)=a$
$P(1,1)$ $\implies$ $f(2)=a^2+a-1$
$P(2,1)$ $\implies$ $f(3)=a^3+a^2-a$
$P(3,1)$ $\implies$ $f(4)=a^4+a^3-2a^2+1$
$P(4,1)$ $\implies$ $f(5)=a^5+a^4-3a^3-a^2+3a$
$P(3,2)$ $\implies$ $(a-2)(a-1)(a+1)(a+2)=0$
Q.E.D.

2.2) Case where $a=b=-2$ 
----------------------------
$f(0)=1$
$f(1)=-2$
$P(x+1,1)$ $\implies$ $f(x+2)=-2f(x+1)-f(x)-2$
Simple induction gives :

$\boxed{S_3 : a=-2\text{ gives }f(2n)=1\text{  and }f(2n+1)=-2\text{  }\forall n}$

which indeed is a solution

2.3) Case where $a=b=-1$ 
---------------------------
$f(0)=1$
$f(1)=-1$
$P(x+1,1)$ $\implies$ $f(x+2)=-f(x+1)-f(x)-1$
Simple induction gives :

$\boxed{S_4 : a=-1\text{ gives }f(3n)=1\text{  and }f(3n+1)=-1\text{  and }f(3n+2)=-1\text{  }\forall n}$

which indeed is a solution

2.4) Case where $a=b=1$ 
------------------------
$f(0)=1$
$f(1)=1$
$P(x+1,1)$ $\implies$ $f(x+2)=-f(x+1)-f(x)-1$
Simple induction gives $f(n)=1$, which indeed is a solution, already noted in the paragraph "constant solutions"

2.5) Case where $a=b=2$ 
-------------------------------
$f(0)=1$
$f(1)=2$
$P(x+1,1)$ $\implies$ $f(x+2)=2f(x+1)-f(x)+2$
Simple induction gives :

$\boxed{S_5 : a=2\text{ gives }f(n)=n^2+1\text{  }\forall n}$

which indeed is a solution
\end{solution}



\begin{solution}[by \href{https://artofproblemsolving.com/community/user/174113}{victory1204}]
	How about a=0? Is there a solution?
\end{solution}



\begin{solution}[by \href{https://artofproblemsolving.com/community/user/29428}{pco}]
	\begin{tcolorbox}How about a=0? Is there a solution?\end{tcolorbox}
Uhhh ?, this is exactly the same exercise. You should have found alone from the previous one , according to me.

Let $P(x,y)$ be the assertion $f(x+y)+f(xy-1)=f(x)f(y)$

1) Constant solutions
==============
We immediately get two constant solutions :
$\boxed{S_1 : f(x)=0\text{   }\forall x}$

$\boxed{S_2 : f(x)=2\text{   }\forall x}$

2)Non constant solutions
=================
2.1) $f(0)=1$ and $f(-1)=0$ and $f(1)\in \{-1,0,2\}$
-------------------------------------------------
Let $b=f(1)$
If $f(0)\ne 1$ : $P(x,0)$ $\implies$ $f(-1)=f(x)(f(0)-1)$ and so $f(x)$ is contant, impossible. So $f(0)=1$ and $f(-1)=0$

$P(x+1,1)$ $\implies$ $f(x+2)=bf(x+1)-f(x)$ and so :
$f(-1)=0$
$f(0)=1$
$f(1)=b$
$f(2)=b^2-1$
$f(3)=b^3-2b$
$f(4)=b^4-3b^2+1$

$P(2,2)$ $\implies$ $b(b+1)(b-2)=0$ and so $b\in\{-1,0,2\}$
Q.E.D.

2.2) Case where $f(1)=-1$ 
--------------------------
$f(0)=1$
$f(1)=-1$
$P(x+1,1)$ $\implies$ $f(x+2)=-f(x+1)-f(x)$
Simple induction gives :

$\boxed{S_3 : f(3n)=1\text{  and }f(3n+1)=-1\text{  and }f(3n+2)=0\text{  }\forall n}$

which indeed is a solution

2.3) Case where $f(1)=0$ 
--------------------------
$f(0)=1$
$f(1)=0$
$P(x+1,1)$ $\implies$ $f(x+2)=-f(x)$
Simple induction gives :

$\boxed{S_4 : f(2n)=(-1)^n\text{  and }f(2n+1)=0\text{  }\forall n}$

which indeed is a solution

2.4) Case where $f(1)=2$ 
--------------------------
$f(0)=1$
$f(1)=2$
$P(x+1,1)$ $\implies$ $f(x+2)=2f(x+1)-f(x)$
Simple induction gives :

$\boxed{S_5 : f(n)=n+1\text{  }\forall n}$

which indeed is a solution
\end{solution}
*******************************************************************************
-------------------------------------------------------------------------------

\begin{problem}[Posted by \href{https://artofproblemsolving.com/community/user/143628}{MANMAID}]
	Let $g$ be a continuous function with $g(1)=1$ such that

$g(x+y)=5g(x)g(y)$ for all $x,y$.

Find $g(x)$, using the fact that, If $f$ is a continuous function that satisfies $f(x + y) =f(x) + f(y)$ for all $x, y$, then $f(x) = xf(1)$.
	\flushright \href{https://artofproblemsolving.com/community/c6h574392}{(Link to AoPS)}
\end{problem}



\begin{solution}[by \href{https://artofproblemsolving.com/community/user/127783}{Sayan}]
	Suppose $g(x)=0$ for some $x$. Then $1=g(1)=g(x+1-x)=5g(x)g(1-x)=0$ which is a contradiction. Therefore $g(x) \neq 0$ for all $x$. Note that $g(x)=5g\left(\frac{x}2\right)^2 > 0$. Therefore there is no harm in taking the logarithms. Hence
\[\log g(x+y) =\log 5+\log g(x)+ \log g(y)\]
Now, let $\log g(x)+\log 5 = f(x)$ and use the fact to reach the conclusion.
\end{solution}



\begin{solution}[by \href{https://artofproblemsolving.com/community/user/177508}{mathuz}]
	Let $h(x)=5g(x)$ and we have $h(x+y)=h(x)h(y)$ - cauchy exponential equation.
\end{solution}



\begin{solution}[by \href{https://artofproblemsolving.com/community/user/143628}{MANMAID}]
	Please solve in full version.
\end{solution}



\begin{solution}[by \href{https://artofproblemsolving.com/community/user/29428}{pco}]
	\begin{tcolorbox}Please solve in full version.\end{tcolorbox}
What exactly are you waiting for ?
All elements above are correct. And Sayan's post is quite precise.
\end{solution}



\begin{solution}[by \href{https://artofproblemsolving.com/community/user/143628}{MANMAID}]
	Sorry, but I had done this all things before I posted it. The solution did not satisfy $g(1)$
\end{solution}



\begin{solution}[by \href{https://artofproblemsolving.com/community/user/29428}{pco}]
	\begin{tcolorbox}Sorry, but I had done this all things before I posted it. The solution did not satisfy $g(1)$\end{tcolorbox}
Maybe you could do a small effort :(

Sayan ends its quite OK post with $\ln g(x)+\ln 5 =f(x)$ and suggests you can finish alone ....

Problem statement said $f(x)=cx$ and $g(1)=1$ implies $c=\ln 5$ and so $\boxed{g(x)=5^{x-1}}$

Ending Sayan's solution was not so difficult.
\end{solution}



\begin{solution}[by \href{https://artofproblemsolving.com/community/user/16261}{Rust}]
	Let $f(x)=5g(x)$ (or $g(x)=\frac 15 f(x)$). Then
$f(x+y)=f(x)f(y)$. Let $\phi(x)=\ln(x+y)$. Then
$\phi(x+y)=\phi(x)+\phi(y)$. It has infinetely many solutions.
If consuder only continiusly solutions, then
$\phi(x)=ax$. It give $g(x)=\frac 15 \exp(ax).$
\end{solution}



\begin{solution}[by \href{https://artofproblemsolving.com/community/user/29428}{pco}]
	\begin{tcolorbox}Let $f(x)=5g(x)$ (or $g(x)=\frac 15 f(x)$). Then
$f(x+y)=f(x)f(y)$. Let $\phi(x)=\ln(x+y)$. Then
$\phi(x+y)=\phi(x)+\phi(y)$. It has infinetely many solutions.
If consuder only continiusly solutions, then
$\phi(x)=ax$. It give $g(x)=\frac 15 \exp(ax).$\end{tcolorbox}

1) continuity is given in OP's first post

2) $g(1)=1$ and so $e^a=5$  and $g(x)=5^{x-1}$, as previously posted.
\end{solution}
*******************************************************************************
-------------------------------------------------------------------------------

\begin{problem}[Posted by \href{https://artofproblemsolving.com/community/user/190967}{Opel98}]
	Help...

Suppose a function $ f(x) $ has real value and satisfy

$ x^5+x^4-x^3f(1+x)-2x^2-xf(x)-f(1-x^2)-1=0 $

What is the value of $ f(2014) - f(2013)+f(2012)-f(2011)+...+f(2)-f(1)+f(0) $ ?

thanks bfore :D
	\flushright \href{https://artofproblemsolving.com/community/c6h574425}{(Link to AoPS)}
\end{problem}



\begin{solution}[by \href{https://artofproblemsolving.com/community/user/29428}{pco}]
	\begin{tcolorbox}Suppose a function $ f(x) $ has real value and satisfy

$ x^5+x^4-x^3f(1+x)-2x^2-xf(x)-f(1-x^2)-1=0 $

What is the value of $ f(2014) - f(2013)+f(2012)-f(2011)+...+f(2)-f(1)+f(0) $ ?\end{tcolorbox}
Fake problem. Infinitely many such $f(x)$ exist and this quantity may take any value we want.

Here is a family of solutions :

Let $a\in\mathbb R$. Define $f(x)$ as :
$f(0)=a$
$f(x)=x^2-2x+a\frac{(-1)^{x-1}}{((x-1)!)^2}$ $\forall x\in\mathbb Z_{\ge 2}$
$f(x)=x^2-2x$ everywhere else.

Varying $a$, the required quantity may take any value we want.
\end{solution}



\begin{solution}[by \href{https://artofproblemsolving.com/community/user/16261}{Rust}]
	\begin{tcolorbox}[quote="Opel98"]
Let $a\in\mathbb R$. Define $f(x)$ as :
$f(0)=a$
$f(x)=x^2-2x+a\frac{(-1)^{x-1}}{((x-1)!)^2}$ $\forall x\in\mathbb Z_{\ge 2}$
$f(x)=x^2-2x$ everywhere else.

Varying $a$, the required quantity may take any value we want.\end{tcolorbox}\end{tcolorbox}
pco? I think your solution for $f(x)$ is not correct.
I found, that $f(x)=x^2-2x+f_1(x)$, were
$x^3f_1(1+x)+xf_1(x)+f_1(1-x^2)=0$.
$f(0)=a$ can be any. But $f_1(y)\not \equiv 0$ for $y<0$ as in your solution.
There are infinetely many another solutions.
\end{solution}



\begin{solution}[by \href{https://artofproblemsolving.com/community/user/29428}{pco}]
	\begin{tcolorbox}...
pco? I think your solution for $f(x)$ is not correct.
...\end{tcolorbox}
I'm sorry but I dont understand why my solution is not correct :?:

Could you kindly give me one $x$ for which the functional equation is not verified by my family of solutions ?
Thanks a lot.
\end{solution}



\begin{solution}[by \href{https://artofproblemsolving.com/community/user/16261}{Rust}]
	Your solutution correct, but not full.
There are infinetely many suth $f$.
For example, I can define $f_1(x)\equiv e 0$ if $x<-N=1-k^2$ and $f_1(-N)=a_k$.
for $-N<x<-k$ $f_1(x)=\frac{-f_1(x-1)}{(x-1)^2}$, for $-k\le x<-1$ by $f_1(x)=\frac{-(x-1)f_1(x-1)-f_1(2x-x^2)}{(x-1)^3}$.
Then I can define in suth way other values of $f(x)$.
\end{solution}



\begin{solution}[by \href{https://artofproblemsolving.com/community/user/29428}{pco}]
	\begin{tcolorbox}Your solutution correct, but not full....\end{tcolorbox}
Ahh ! That's why I wrote "Here is\begin{bolded} a family \end{underlined}\end{bolded}of solutions"

I never claimed that this family was unique. But it was enough to prove that the required quantity could take any value we want.
\end{solution}



\begin{solution}[by \href{https://artofproblemsolving.com/community/user/16261}{Rust}]
	\begin{tcolorbox}[quote="Rust"] But it was enough to prove that the required quantity could take any value we want.\end{tcolorbox}\end{tcolorbox}
Not enough. We need $f(0)$ and $f(1-k^2), k=45,...,2013$ and $f(-2013)$ or $f(-2014)$. This parametres can be any.
\end{solution}



\begin{solution}[by \href{https://artofproblemsolving.com/community/user/29428}{pco}]
	Read more carefully my example. In my [correct] family of solutions, $f(x)=0$ $\forall x<0$. So no need for $f(1-k^2)$.
\end{solution}
*******************************************************************************
-------------------------------------------------------------------------------

\begin{problem}[Posted by \href{https://artofproblemsolving.com/community/user/68025}{Pirkuliyev Rovsen}]
	Find all functions $f: \mathbb{R}\to\mathbb{R}$ that satisfy $f(xy+f(x))=xf(y)+f(x)$ for all $x,y$.
	\flushright \href{https://artofproblemsolving.com/community/c6h574572}{(Link to AoPS)}
\end{problem}



\begin{solution}[by \href{https://artofproblemsolving.com/community/user/124745}{Uzbekistan}]
	Let $P(x,y)$ be the assertion $f(xy+f(x))=xf(y)+f(x)$

$f(x)=0$ $\forall x$ is a solution and we'll consider from now that $\exists a$ such that $f(a)\ne 0$

1) $f(0)=0$ and $f(f(x))=f(x)$

Suppose $f(0)\ne 0$. Then $P(x,0)$ $\implies$ $f(f(x))=xf(0)+f(x)$ and so $f(x_1)=f(x_2)$ $\implies$ $x_1=x_2$ and $f(x)$ is injective.
Then $P(0,0)$ $\implies$ $f(f(0))=f(0)$ and, since $f(x)$ is injective, $f(0)=0$, so contradiction.
So $f(0)=0$ and $P(x,0)$ $\implies$ $f(f(x))=f(x)$
Q.E.D.

2) $f(-1)=-1$

$P(f(a),-1)$ $\implies$ $0=f(a)(f(-1)+1)$ and so $f(-1)=-1$
Q.E.D.

3) $f(x)=x$ $\forall x$

Let $g(x)=f(x)-x$

Suppose now $\exists b$ such that $f(b)\ne b$

$P(\frac x{f(b)-b},b)$ $\implies$ $f(b\frac x{f(b)-b}+f(\frac x{f(b)-b}))=\frac x{f(b)-b}f(b)+f(\frac x{f(b)-b})$

and so $f(b\frac x{f(b)-b}+f(\frac x{f(b)-b}))-(b\frac x{f(b)-b}+f(\frac x{f(b)-b}))=x$

and so $g(b\frac x{f(b)-b}+f(\frac x{f(b)-b}))=x$ and $g(\mathbb R)=\mathbb R$

but $P(x,-1)$ $\implies$ $f(f(x)-x)=f(x)-x$ and so $f(x)=x$ $\forall x\in g(\mathbb R)$ 
And it's immediate to see that this indeed is a solution
Q.E.D.

4) Synthesis of results 

We got two solutions :
$f(x)=0$ $\forall x$
$f(x)=x$ $\forall x$
\end{solution}



\begin{solution}[by \href{https://artofproblemsolving.com/community/user/29428}{pco}]
	Quite nice :) :)

And, what strange! it's \begin{bolded}exactly \end{underlined}\end{bolded}the same( letter per letter ) than my solution to problem 10 here : 
http://www.artofproblemsolving.com/Forum/viewtopic.php?p=1890738#p1890738

It seems we have exactly the same way of thinking.
I'm quite proud.
\end{solution}
*******************************************************************************
-------------------------------------------------------------------------------

\begin{problem}[Posted by \href{https://artofproblemsolving.com/community/user/68025}{Pirkuliyev Rovsen}]
	Find all functions $f: \mathbb{N}\to\mathbb{N}$ such that $(2^m+1)f(n)f(2^{m}n)=2^mf^2(n)+f^2(2^{m}n)+(2^m-1)^2n$ for all $m,n{\in}N$.
	\flushright \href{https://artofproblemsolving.com/community/c6h574574}{(Link to AoPS)}
\end{problem}



\begin{solution}[by \href{https://artofproblemsolving.com/community/user/29428}{pco}]
	\begin{tcolorbox}Find all functions $f: \mathbb{N}\to\mathbb{N}$ such that $(2^m+1)f(n)f(2^{m}n)=2^mf^2(n)+f^2(2^{m}n)+(2^m-1)^2n$ for all $m,n{\in}N$.\end{tcolorbox}
What is the meaning of $f^2(x)$ here ? : $f(f(x))$ or $(f(x))^2$ ?
\end{solution}



\begin{solution}[by \href{https://artofproblemsolving.com/community/user/68025}{Pirkuliyev Rovsen}]
	Where  $(f(x))^2$
\end{solution}



\begin{solution}[by \href{https://artofproblemsolving.com/community/user/29428}{pco}]
	\begin{tcolorbox}Find all functions $f: \mathbb{N}\to\mathbb{N}$ such that $(2^m+1)f(n)f(2^{m}n)=2^mf^2(n)+f^2(2^{m}n)+(2^m-1)^2n$ for all $m,n{\in}N$.\end{tcolorbox}
It's rather easy to build infinitely many such functions.

May be you forgot some conditions. Adding for example the constraint "increasing", we get the unique solution $f(n)=n+1$

...
\end{solution}
*******************************************************************************
-------------------------------------------------------------------------------

\begin{problem}[Posted by \href{https://artofproblemsolving.com/community/user/172778}{Shayanhas}]
	find all $f:R\longrightarrow{R}$ such that
1)there exist real number $m$ such that $f(x)\leq{m}$ 
2)$f(xf(y))+yf(x)=xf(y)+f(xy)$
	\flushright \href{https://artofproblemsolving.com/community/c6h574907}{(Link to AoPS)}
\end{problem}



\begin{solution}[by \href{https://artofproblemsolving.com/community/user/141363}{alibez}]
	[hide="hint"]

it is easy to see that $m=0$ !!!

prove that : $f(1)=f(-1)=0$ so $f(x)=-f(-x)$ !!!!

[\/hide]
\end{solution}



\begin{solution}[by \href{https://artofproblemsolving.com/community/user/190093}{KamalDoni}]
	I couldn't solve it but found some interesting facts
1)put x=1 then we find that 
f(f(y))+yf(1)=2(y) then if f(1) is not equal to 0 ,we have f injective so 
2)put x=y we find that $f(xf(x))=f(x^2)$ then by injectivity f(x)=x and we will have contradiction because f is upper bounded
then we see that f(1)=0 and f(f(x))=2f(x) 
3)put x=y=0 we find that f(0)=0
if the max value of f is positive then take f(x)=k is max => f(f(x))=2f(x)>f(x)=k which is a contradiction 
hence f is not positive then max value is 0 (f(1)=f(0)=0)
The rest I didn't do but I know that f(x)=0 is the answer:)
please solve this problem I want to see the solution
\end{solution}



\begin{solution}[by \href{https://artofproblemsolving.com/community/user/190093}{KamalDoni}]
	ohh I see that f(-1)=0 and f(x)=-f(-x) then if there is x such that f(x) negative then f(-x) is positive which is a contradiction . So the only answer is f(x)=0
\end{solution}



\begin{solution}[by \href{https://artofproblemsolving.com/community/user/172778}{Shayanhas}]
	$f(x)=0$  $\forall{x\geq{0}}$and$f(x)=2x$  $\forall{x\leq{0}}$ is answer
\end{solution}



\begin{solution}[by \href{https://artofproblemsolving.com/community/user/29428}{pco}]
	\begin{tcolorbox}find all $f:R\longrightarrow{R}$ such that
1)there exist real number $m$ such that $f(x)\leq{m}$ 
2)$f(xf(y))+yf(x)=xf(y)+f(xy)$\end{tcolorbox}
$f(x)=0$ $\forall x$ is a solution. So let us from now look only for non allzero solutions.
Let $P(x,y)$ be the assertion $f(xf(y))+yf(x)=xf(y)+f(xy)$
Let $t$ such that $f(t)\ne 0$

If $f(0)\ne 0$, $P(\frac{x}{f(0)},0)$ $\implies$ $f(x)=x+f(0)$ which is not a solution, since not upper bounded. So $f(0)=0$

If $f(1)\ne 0$, $P(\frac x{f(1)},1)$ $\implies$ $f(x)=x$ which is not a solution, since not upper bounded. So $f(1)=0$
If $f(x)>0$ for some $x$, $P(1,x)$ $\implies$ $f(f(x))=2f(x)$ and, thru repeated appplications, $f(x)$ is not upper bounded. So $f(x)\le 0$ $\forall x$

Let $y\ne 0$ : adding $P(y,\frac 1y)$ and $P(\frac 1y,y)$, we get $f(yf(\frac 1y))+f(\frac 1yf(y))=0$ and so $f(yf(\frac 1y))=f(\frac 1yf(y))=0$

Let then $x,y\ne 0$ : adding $P(x,\frac {f(y)}{xy})$ and $P(\frac {f(y)}{xy},x)$, we get $f(xf(\frac {f(y)}{xy}))+f(\frac {f(y)}{xy}f(x))=0$
And so $f(xf(\frac {f(y)}{xy}))=f(\frac {f(y)}{xy}f(x))=0$
Then, $P(x,\frac {f(y)}{xy})$ $\implies$ new assertion $Q(x,y)$ : $\frac {f(x)f(y)}{xy}=xf(\frac {f(y)}{xy})$ $\forall x,y\ne 0$

If $f(x)\ne 0$, then $x\ne 0$ and :
$P(1,x)$ $\implies$ $f(f(x))=2f(x)$
$Q(f(x),x)$ $\implies$ $f(\frac 1x)=\frac 2x$ 
So $f(\frac 1x)\ne 0$ and then $Q(f(\frac 1x),\frac 1x)$ $\implies$ $f(x)=2x$ 

And so $\forall x$, either $f(x)=0$, either $f(x)=2x$

Note that this implies $f(x)=0$ $\forall x>0$ and $t<0$ and $f(t)=2t$

Let then $x<0$ : $P(x,t)$ $\implies$ $f(2xt)+tf(x)=2xt+f(xt)$ and since $xt>0$, this implies $f(x)=2x$ $\forall x<0$, which indeed is a solution.

Hence the two solutions  :

$\boxed{S_1\text{  :  }f(x)=0\text{  }\forall x}$

$\boxed{S_2\text{  :  }f(x)=x-|x|\text{  }\forall x}$
\end{solution}
*******************************************************************************
-------------------------------------------------------------------------------

\begin{problem}[Posted by \href{https://artofproblemsolving.com/community/user/68025}{Pirkuliyev Rovsen}]
	Find all functions $f: \mathbb{Z}\to\mathbb{Z}$ such that $f(-1)=f(1)$ and $f(x)+f(y)=f(x+2xy)+f(y-2xy)$ for all $x,y{\in}Z$.
	\flushright \href{https://artofproblemsolving.com/community/c6h575002}{(Link to AoPS)}
\end{problem}



\begin{solution}[by \href{https://artofproblemsolving.com/community/user/29428}{pco}]
	\begin{tcolorbox}Find all functions $f: \mathbb{Z}\to\mathbb{Z}$ such that $f(-1)=f(1)$ and $f(x)+f(y)=f(x+2xy)+f(y-2xy)$ for all $x,y{\in}Z$.\end{tcolorbox}
Let $f(1)=f(-1)=a$
Let $P(x,y)$ be the assertion $f(x)+f(y)=f(x+2xy)+f(y-2xy)$

$P(1,x)$ $\implies$ $a+f(x)=f(2x+1)+f(-x)$
$P(x,-1)$ $\implies$ $f(x)+a=f(-x)+f(2x-1)$
Subtracting, we get $f(2x+1)=f(2x-1)$ and so $f(x)=a$ $\forall$ odd $x$.

Then :
$P(1,x)$ $\implies$ $f(x)=f(-x)$
$P(x,2y+1)$ $\implies$ $f(x)=f(x(4y+3))$
$P(2y+1,x)$ $\implies$ $f(x)=f(-x(4y+1))=f(x(4y+1))$
And so $f(x)=f(x(2y+1))$ $\forall x,y$

Hence the answer $\boxed{f(0)=h(0)\text{ and }f(x)=h(2^{v_2(|x|)})\text{  }\forall x\ne 0}$ which indeed is a solution, whatever is $h(x)$ from $\mathbb Z\to\mathbb Z$
\end{solution}
*******************************************************************************
-------------------------------------------------------------------------------

\begin{problem}[Posted by \href{https://artofproblemsolving.com/community/user/143628}{MANMAID}]
	Let $f:(-1,1)\rightarrow R$ be a continuous function such that $f(x)=f(x^4)$. We have $f(1\/2)=a$. Then prove that $f(x)=a$ for all $x\in R$
	\flushright \href{https://artofproblemsolving.com/community/c6h575168}{(Link to AoPS)}
\end{problem}



\begin{solution}[by \href{https://artofproblemsolving.com/community/user/29428}{pco}]
	\begin{tcolorbox}Let $f:(-1,1)\rightarrow R$ be a function such that $f(x)=f(x^4)$. We have $f(1\/2)=a$. Then prove that $f(x)=a$ for all $x\in R$\end{tcolorbox}
Quite wrong.

Counter-example : $f(0)=a+1$ and $f(x)=a$ $\forall x\in(-1,0)\cup(0,1)$
\end{solution}



\begin{solution}[by \href{https://artofproblemsolving.com/community/user/167245}{TheChainheartMachine}]
	Perhaps you require some continuity. Somewhere. Continuity at $0$ is plenty.

(And of course, you can't ask for continuity on $\mathbb{R}$ when $f$ is really only defined on $(-1, 1)$.)
\end{solution}



\begin{solution}[by \href{https://artofproblemsolving.com/community/user/29428}{pco}]
	\begin{tcolorbox}Let $f:(-1,1)\rightarrow R$ be a continuous function such that $f(x)=f(x^4)$. We have $f(1\/2)=a$. Then prove that $f(x)=a$ for all $x\in R$\end{tcolorbox}
You are welcome. Glad to have helped you.
I see that you have silently edited your problem, adding the "continuous" constraint. Then :

$f(x)=f\left(|x|^{4^n}\right)$ $\forall x\in(-1,1)$ and $\forall n\in\mathbb Z$

Setting there $n\to+\infty$ and using continuity at $0$, we get $f(x)=f(0)$ $\forall x\in (-1,1)$

Hence the result.
\end{solution}
*******************************************************************************
-------------------------------------------------------------------------------

\begin{problem}[Posted by \href{https://artofproblemsolving.com/community/user/167935}{nguyenthihuonggiang}]
	Find all Functions $f:R \to R$ such that $f\left( x^2 - {{f^2}\left( y \right)  \right) = xf\left( x \right) - y^2,\forall x,y \in R.}$
	\flushright \href{https://artofproblemsolving.com/community/c6h575180}{(Link to AoPS)}
\end{problem}



\begin{solution}[by \href{https://artofproblemsolving.com/community/user/29428}{pco}]
	\begin{tcolorbox}Find all Functions $f:R \to R$ such that $f\left( x^2 - {{f^2}\left( y \right)  \right) = xf\left( x \right) - y^2,\forall x,y \in R.}$\end{tcolorbox}
Let $P(x,y)$ be the assertion $f(x^2-f(y)^2)=xf(x)-y^2$
Let $a=f(0)$

If $f(u)=f(v)$, then comparaison of $P(0,u)$ and $P(0,v)$ implies $u=\pm v$

$P(1,0)$ $\implies$ $f(1-a^2)=f(1)$ and so $1-a^2=\pm 1$ and so $a\in\{-\sqrt 2,0,\sqrt 2\}$

$P(0,0)$ $\implies$ $f(-a^2)=0$
$P(0,-a^2)$ $\implies$  $a=-a^4$ and so $a\in\{-1,0\}$

So $a=0$

$P(x,0)$ $\implies$ $f(x^2)=xf(x)$ and so $f(-x)=-f(x)$ $\forall x$
$P(0,y)$ $\implies$ $f(-f(y)^2)=-y^2$ and so $\mathbb R_{\le 0}\subseteq f(\mathbb R)$ and so $f(x)$ is surjective (since odd).

So $P(x,y)$ may be written $f(x^2-f(y)^2)=f(x^2)-f(f(y)^2)$ and so $f(u-v)=f(u)-f(v)$ $\forall u,v\ge 0$
So, using odd property, $f(x+y)=f(x)+f(y)$ $\forall x,y)$

And $f(-f(y)^2)=-y^2$ implies $f(x)\le 0$ $\forall x\le 0$ and so $f(x)$ is a monotonous solution of Cauchy additive equation, so is $f(x)=cx$

Plugging back in original equation, we get $c=1$ and so $\boxed{f(x)=x}$ $\forall x$
\end{solution}
*******************************************************************************
-------------------------------------------------------------------------------

\begin{problem}[Posted by \href{https://artofproblemsolving.com/community/user/205002}{Patcharapong}]
	[size=150]Problem2 find all f:N-->N such that f(m+f(n)) = n+f(m+K) for all m,n E N [\/size]
	\flushright \href{https://artofproblemsolving.com/community/c6h575183}{(Link to AoPS)}
\end{problem}



\begin{solution}[by \href{https://artofproblemsolving.com/community/user/29428}{pco}]
	\begin{tcolorbox}Problem2 find all f:N-->N such that f(m+f(n)) = n+f(m+K) for all m,n E N \end{tcolorbox}
I suppose $k\in\mathbb N\cup\{0\}$
Let $P(x,y)$ be the assertion $f(x+f(y))=y+f(x+k)$
$f(x)$ is injective

Let $c=f(f(k+1))-1$. Note that $c\ge 0$
Subtracting $P(x,f(k+1))$ from $P(1,f(x+k))$, we get $f(x+f(f(k+1)))=f(1+f(f(x+k)))$ and so, since injective, $f(f(x+k))=x+c$

$P(x,f(k+2))$ $\implies$ $f(x+k+c+2)=f(x+k)+f(k+2)$
$P(x+1,f(k+1))$ $\implies$ $f(x+k+c+2)=f(x+k+1)+f(k+1)$
Subtracting, we get $f(x+k+1)=f(x+k)+(f(k+2)-f(k+1))$ and so $f(x)=ax+b$ $\forall x>k$ with $a>0$ (since injective)

Plugging this in $P(x,y)$ for $x,y$ great enough, we get $a=1$ and $b=k$ and so $f(x)=x+k$ $\forall x>k$

Then $P(k,x)$ $\implies$ $\boxed{f(x)=x+k}$ $\forall x\ge 1$, which indeed is a solution.
\end{solution}



\begin{solution}[by \href{https://artofproblemsolving.com/community/user/205002}{Patcharapong}]
	nice! pco . I will share many problem soon .
\end{solution}
*******************************************************************************
-------------------------------------------------------------------------------

\begin{problem}[Posted by \href{https://artofproblemsolving.com/community/user/205002}{Patcharapong}]
	[size=150]problem1 find all f:N U {0} ---> N U {0}  such that 1) f(m^2 + n^2) = {f(m)}^2 + {f(n)}^2  and 2) f(1)>0[\/size]
	\flushright \href{https://artofproblemsolving.com/community/c6h575184}{(Link to AoPS)}
\end{problem}



\begin{solution}[by \href{https://artofproblemsolving.com/community/user/29428}{pco}]
	\begin{tcolorbox} Find all functions $f:\mathbb{N}\cup \{0\}\to\mathbb{N}\cup  \{0\}$  such that 
$ f(m^2 + n^2) = {f(m)}^2 + {f(n)}^2$  and  $f(1)>0$\end{tcolorbox}
Let $P(x,y)$ be the assertion $f(x^2+y^2)=f(x)^2+f(y)^2$

Let $x\ge 2$ : subtracting $P(x+2,2x-1)$ from $P(x-2,2x+1)$ we get :
New assertion $Q(x)$ : $f(2x+1)^2=f(2x-1)^2+f(x+2)^2-f(x-2)^2$

Let $x\ge 5$ : subtracting $P(x+3,2x-4)$ from $P(x-5,2x)$ we get :
New assertion $R(x)$ : $f(2x)^2=f(2x-4)^2+f(x+3)^2-f(x-5)^2$

$P(0,0)$ $\implies$ $f(0)=0$ (since $\in\mathbb N\cup\{0\}$)
$P(1,0)$ $\implies$ $f(1)=1$ (since $>0$)
$P(1,1)$ $\implies$ $f(2)=2$
$P(2,0)$ $\implies$ $f(4)=4$
$P(2,1)$ $\implies$ $f(5)=5$
$Q(2)$ $\implies$ $f(3)=3$
$Q(3)$ $\implies$ $f(7)=7$
$P(2,2)$ $\implies$ $f(8)=8$
$P(3,0)$ $\implies$ $f(9)=9$
$Q(4)$ $\implies$ $f(6)=6$

And then, using direct induction with formulas $Q(x)$ and $R(x)$ ge get $\boxed{f(x)=x}$ $\forall x\in\mathbb N\cup\{0\}$, which indeed is a solution.
\end{solution}
*******************************************************************************
-------------------------------------------------------------------------------

\begin{problem}[Posted by \href{https://artofproblemsolving.com/community/user/68025}{Pirkuliyev Rovsen}]
	Find all functions $f: \mathbb{Z}\to\mathbb{Z}$ such that for all integers $m,n$ ,  $f(m-n+f(n))=f(m)+f(n)$.
	\flushright \href{https://artofproblemsolving.com/community/c6h575948}{(Link to AoPS)}
\end{problem}



\begin{solution}[by \href{https://artofproblemsolving.com/community/user/141363}{alibez}]
	let :
$f(m)=m+g(m)\Rightarrow m+g(n)+g(m+g(n))=g(m)+n+m+g(n)\Rightarrow g(m+g(n))=g(m)+n$

$f$ is surjective and injective and we know $f(f(n))=n$ so $f$ is additive ......

it is France Team Selection Test 2007 :  http://www.artofproblemsolving.com/Forum\/resources.php?c=60&cid=82&year=2007&sid=aae2562172fec01d2b35403d4298eab4
\end{solution}



\begin{solution}[by \href{https://artofproblemsolving.com/community/user/29428}{pco}]
	\begin{tcolorbox}Find all functions $f: \mathbb{Z}\to\mathbb{Z}$ such that for all integers $m,n$ ,  $f(m-n+f(n))=f(m)+f(n)$.\end{tcolorbox}
Let $P(x,y)$ be the assertion $f(x-y+f(y))=f(x)+f(y)$

$P(-f(0),0)$ $\implies$  $f(-f(0))=0$ and so $A=f^{-1}(\{0\})\ne \emptyset$

If $\exists u\ne 0\in A$, then $P(x+u,u)$ $\implies$ $f(x+u)=f(x)$ and so $|f(\mathbb Z)|$ is finite
$P(x,x)$ $\implies$ $f(f(x))=2f(x)$ and so either $f(\mathbb Z)=\{0\}$, either $|f(\mathbb Z)|=+\infty$


So :
1) if $|f(\mathbb Z)|$ is finite, we need $\boxed{f(x)=0}$ $\forall x$ which indeed is a solution

2) if $|f(\mathbb Z)|=+\infty$, then $A=\{0\}$ and $P(2x-f(x),x)$ $\implies$ $f(2x-f(x))=0$ and so $\boxed{f(x)=2x}$ $\forall x$ which indeed is a solution
\end{solution}



\begin{solution}[by \href{https://artofproblemsolving.com/community/user/199091}{maryam_m}]
	hi...
$m\rightarrow -f(n)$  gives $f(-n)=f(-f(n))+f(n)$ if we set $n=0$ well have $f(-f(0))=0$

$n\rightarrow -f(0)$   gives $f(m+f(0))=f(m)+f(-f(0))=f(m) $

if we set $m=-f(0)$ in the last equation well have $f(0)=f(-f(0))=0$ 

$\Rightarrow  f(0)=0$

set $m=0$  so $f(f(n)-n)=f(n)$ let $n \rightarrow m$ well have $f(f(m)-m)=f(m)$

set $n=f(n)-n$ in the main equation so $f(m-f(n)+n+f(f(n)-n))=f(m)+f(f(n)-n)$

$\Rightarrow f(m+n)=f(m)+f(n)$  $\mathbb{Z} : f(m)=mf(1) \Rightarrow ... f(m)=0$ and $f(m)=2m$ are the solutions.
\end{solution}



\begin{solution}[by \href{https://artofproblemsolving.com/community/user/201392}{nima-amini}]
	http://www.artofproblemsolving.com/Forum/viewtopic.php?p=521359&sid=a98f2f9870243edbe98ae75a59acc315#p521359
\end{solution}



\begin{solution}[by \href{https://artofproblemsolving.com/community/user/201392}{nima-amini}]
	\begin{tcolorbox}let :
$f(m)=m+g(m)\Rightarrow m+g(n)+g(m+g(n))=g(m)+n+m+g(n)\Rightarrow g(m+g(n))=g(m)+n$

$f$ is surjective and injective and we know $f(f(n))=n$ so $f$ is additive ......

it is France Team Selection Test 2007 :  http://www.artofproblemsolving.com/Forum\/resources.php?c=60&cid=82&year=2007&sid=aae2562172fec01d2b35403d4298eab4\end{tcolorbox}
it is italy tst too
http://www.artofproblemsolving.com/Forum/viewtopic.php?p=521359&sid=a98f2f9870243edbe98ae75a59acc315#p521359
\end{solution}
*******************************************************************************
-------------------------------------------------------------------------------

\begin{problem}[Posted by \href{https://artofproblemsolving.com/community/user/127581}{sahadian}]
	Find all functions $f: (0,\infty) \to (0,\infty)$ such that $ \dfrac {f (y)}{x}+f \left (\dfrac {x^2-y}{f (x)}\right )=f (x)$ for all $x,y>0$.

[color=#FF0000][mod: the OP has added the restriction $y\in (0,x^2)$][\/color]
	\flushright \href{https://artofproblemsolving.com/community/c6h576480}{(Link to AoPS)}
\end{problem}



\begin{solution}[by \href{https://artofproblemsolving.com/community/user/172163}{joybangla}]
	Umm what if $y=x^2$ then we get $\frac{f(y)}{x}+f(0)=f(x)$ but $f(0)$ is not defined.This may be silly but its not clear to me please clear it out.Thanks in advance.
\end{solution}



\begin{solution}[by \href{https://artofproblemsolving.com/community/user/127581}{sahadian}]
	Excuse me
$y$ should be in the interval $(0,x^2)$
\end{solution}



\begin{solution}[by \href{https://artofproblemsolving.com/community/user/29428}{pco}]
	\begin{tcolorbox}Find all functions $f: (0,\infty) \to (0,\infty)$ such that $ \dfrac {f (y)}{x}+f \left (\dfrac {x^2-y}{f (x)}\right )=f (x)$ for all $x,y>0$ and  $y\in (0,x^2)$\end{tcolorbox}
Let $P(x,y)$ be the assertion $\frac{f(y)}x+f\left(\frac{x^2-y}{f(x)}\right)=f(x)$

If $f(x)<x$ for some $x$, then $P(x,x^2-xf(x))$ $\implies$ $f(x^2-xf(x))=0$, impossible, and so $f(x)\ge x$ $\forall x$

As a consequence, $f(x)\ge\frac yx+\frac{x^2-y}{f(x)}$ $=\frac{x^2}{f(x)}+y\frac{f(x)-x}{xf(x)}$

If $f(x)>x$ for some $x$, then setting $y\to+\infty$ in above inequality implies impossibility. And so $f(x)\le x$ $\forall x$

So $\boxed{f(x)=x}$ $\forall x$, which indeed is a solution.
\end{solution}



\begin{solution}[by \href{https://artofproblemsolving.com/community/user/184652}{CanVQ}]
	\begin{tcolorbox}If $f(x)>x$ for some $x$, then setting $y\to+\infty$ in above inequality implies impossibility. And so $f(x)\le x$ $\forall x$\end{tcolorbox}
Sorry, I think we cannot take $y \to +\infty$ here because we have the condition $0<y<x^2.$
\end{solution}



\begin{solution}[by \href{https://artofproblemsolving.com/community/user/29428}{pco}]
	Yes, indeed :oops:
Thanks
\end{solution}
*******************************************************************************
-------------------------------------------------------------------------------

\begin{problem}[Posted by \href{https://artofproblemsolving.com/community/user/125553}{lehungvietbao}]
	Let $f(x)$ be a continuous function on $\mathbb{R}$, and $x_1, \: x_2$ satisfing  $f(x_1)f(x_2)<0$. Prove that :  $\exists a,\: b,\: c$ such that $a<b<c,\: a+c=2b$ and 
\[15f(a)+2f(b)+2014f( c)=0\]
	\flushright \href{https://artofproblemsolving.com/community/c6h576484}{(Link to AoPS)}
\end{problem}



\begin{solution}[by \href{https://artofproblemsolving.com/community/user/29428}{pco}]
	\begin{tcolorbox}Let $f(x)$ be a continuous function on $\mathbb{R}$, and $x_1, \: x_2$ satisfing  $f(x_1)f(x_2)<0$. Prove that :  $\exists a,\: b,\: c$ such that $a<b<c,\: a+c=2b$ and 
\[15f(a)+2f(b)+2014f( c)=0\]\end{tcolorbox}
Since $f(x_1)f(x_2)<0$ and continuous, $\exists$ non empty open intervals $I,J$ such that $f(x)>0$ $\forall x\in I$ and $f(x)<0$ $\forall x\in J$

Let then $\Delta>0$ and $i,j$ such that $i-\Delta,i,i+\Delta\in I$ and $j-\Delta,j,j+\Delta\in J$
The function $g(x)=15f(x-\Delta)+2f(x)+2014f(x+\Delta)$ is continuous and $g(i)>0$ and $g(j)<0$ and so $\exists t$ such that $g(t)=0$

Choosing then $(a,b,c)=(t-\Delta,t,t+\Delta)$ ends the proof.
\end{solution}
*******************************************************************************
-------------------------------------------------------------------------------

\begin{problem}[Posted by \href{https://artofproblemsolving.com/community/user/157407}{Ikeronalio}]
	Find all functions $ f:\mathbb{R}\to\mathbb{R} $ such that
(a) $ f(-x) = f(x) $ for all $ x $.
(b) $ f(0) = 1, f(1) = 0, f(2) = -3 $.
(c) $ f(x)f(y) + f(xy+1) = f(x+y) $ for all $ x, y $.
	\flushright \href{https://artofproblemsolving.com/community/c6h576630}{(Link to AoPS)}
\end{problem}



\begin{solution}[by \href{https://artofproblemsolving.com/community/user/29428}{pco}]
	\begin{tcolorbox}Find all functions $ f:\mathbb{R}\to\mathbb{R} $ such that
(a) $ f(-x) = f(x) $ for all $ x $.
(b) $ f(0) = 1, f(1) = 0, f(2) = -3 $.
(c) $ f(x)f(y) + f(xy+1) = f(x+y) $ for all $ x, y $.\end{tcolorbox}
Could you confirm us that there is no missing information (like continuity or monotonicity) ?
It's rather classical and easy to show that $f(x)=1-x^2$ $\forall x\in\mathbb Q$ but extending to all $\mathbb R$ does not seem very obvious to me :oops:
\end{solution}



\begin{solution}[by \href{https://artofproblemsolving.com/community/user/157407}{Ikeronalio}]
	I found that the condition (c) was also in
[url=http://www.artofproblemsolving.com/Forum/viewtopic.php?p=3160554&sid=91ee5783cd92fbe221369d2590a83517#p3160554]ISL 2012 A5[\/url]
\end{solution}



\begin{solution}[by \href{https://artofproblemsolving.com/community/user/125513}{hal9v4ik}]
	But ISL 2012's problem is completely different!
There we have $f(-1)$ not equal to $0$
\end{solution}
*******************************************************************************
-------------------------------------------------------------------------------

\begin{problem}[Posted by \href{https://artofproblemsolving.com/community/user/68025}{Pirkuliyev Rovsen}]
	Find all function ${{f: \mathbb(0;+\infty)}\to\mathbb(0;+\infty)}$ such that $f(x+\sqrt{f(x)}){\leq}f(x)+\sqrt{f(x)}$.
	\flushright \href{https://artofproblemsolving.com/community/c6h577127}{(Link to AoPS)}
\end{problem}



\begin{solution}[by \href{https://artofproblemsolving.com/community/user/29428}{pco}]
	\begin{tcolorbox}Find all function ${{f: \mathbb(0;+\infty)}\to\mathbb(0;+\infty)}$ such that $f(x+\sqrt{f(x)}){\leq}f(x)+\sqrt{f(x)}$.\end{tcolorbox}
Is it a real olympiad exercise ?
From where ?

There are obviously infinitely many solutions. For example any bounded function such that $f(x)\in[a,b]$ where $0<a\le b\le a+\sqrt a$

But there exists also unbounded functions. For example any increasing concave differentiable function such that $f'(x)\le 1$ $\forall x$
\end{solution}
*******************************************************************************
-------------------------------------------------------------------------------

\begin{problem}[Posted by \href{https://artofproblemsolving.com/community/user/125553}{lehungvietbao}]
	Find all functions $f,g:\mathbb{Z}\to\mathbb{Z}$ satisfy all the following conditions
a) $f(g(x)+y)=g(f(y)+x) \quad \forall x,y\in\mathbb{Z}$
b) $g$ is injective.
	\flushright \href{https://artofproblemsolving.com/community/c6h577769}{(Link to AoPS)}
\end{problem}



\begin{solution}[by \href{https://artofproblemsolving.com/community/user/29428}{pco}]
	\begin{tcolorbox}Find all functions $f,g:\mathbb{Z}\to\mathbb{Z}$ satisfy all the following conditions
a) $f(g(x)+y)=g(f(y)+x) \quad \forall x,y\in\mathbb{Z}$
b) $g$ is injective.\end{tcolorbox}
Let $P(x,y)$ be the assertion $f(g(x)+y)=g(f(y)+x)$

$P(x,y+g(0)-g(x))$ $\implies$ $f(y+g(0))=g(f(y+g(0)-g(x))+x)$
$P(0,y)$ $\implies$ $f(y+g(0))=g(f(y))$
So, since $g(x)$ is injective : $f(y+g(0)-g(x))=f(y)-x$
This implies $f(x)$ surjective, and so $g(x)$ surjective
Choosing then $x=t$ such that $g(0)-g(t)=1$, this last equation becomes $f(y+1)=f(y)-t$ and so $f(x)=a-tx$ and since surjective, $f(x)=x+a$ or $f(x)=a-x$ 

If $f(x)=x+a$, $P(0,x-a)$ $\implies$ $g(x)=x+g(0)$ and so $\boxed{f(x)=x+a\text{ and }g(x)=x+b\text{  }\forall x}$ which indeed is a solution, whatever are $a,b\in\mathbb Z$

If $f(x)=a-x$, $P(0,a-x)$ $\implies$ $g(x)=x-g(0)$ which is never a solution
\end{solution}
*******************************************************************************
-------------------------------------------------------------------------------

\begin{problem}[Posted by \href{https://artofproblemsolving.com/community/user/125553}{lehungvietbao}]
	Given $0<a<1$, let $f$ be continuous function on $[0;1]$ such that $f(0)=0,f(1)=1$ and $f\left(\frac{2xy}{x+y}\right)=(1-a)f(x)+af(y)\quad \forall x,y\in[0;1], \ x\leq y$
Calculate $f\left(\frac{1}{7}\right)$
	\flushright \href{https://artofproblemsolving.com/community/c6h578346}{(Link to AoPS)}
\end{problem}



\begin{solution}[by \href{https://artofproblemsolving.com/community/user/29428}{pco}]
	\begin{tcolorbox}Given $0<a<1$, let $f$ be continuous function on $[0;1]$ such that $f(0)=0,f(1)=1$ and $f\left(\frac{2xy}{x+y}\right)=(1-a)f(x)+af(y)\quad \forall x,y\in[0;1], \ x\leq y$
Calculate $f\left(\frac{1}{7}\right)$\end{tcolorbox}
No such function : LHS of functional equation is not defined when $x=y=0$
\end{solution}
*******************************************************************************
-------------------------------------------------------------------------------

\begin{problem}[Posted by \href{https://artofproblemsolving.com/community/user/125553}{lehungvietbao}]
	Find all functions $f:\mathbb{R}\setminus \{0\}\to\mathbb{R}$ which satisfy all the following conditions
a) $f(1)=1$
b) $f(\left(\frac{1}{x+y}\right)=f\left(\frac{1}{x}\right)+f\left(\frac{1}{y}\right)$
c) $(x+y)f(x+y)=xyf(x)f(y)\quad \forall xy(x+y)\neq 0$
	\flushright \href{https://artofproblemsolving.com/community/c6h578347}{(Link to AoPS)}
\end{problem}



\begin{solution}[by \href{https://artofproblemsolving.com/community/user/179121}{kurowassann}]
	$x=y=\frac{1}{2z}$ in b) $\Rightarrow$ $2f(2z)=f(z)$
$y=x$ in c) $\Rightarrow$ $2xf(2x)=x^{2}f(x)^{2}$ (*)
If there is $t\neq 0$ such that $f(t)=0$ (Of course $t\neq 1$), then substituting $t$ for $y$ in c) leads to $(x+t)f(x+t)=0$.
Let $x=1-t(\neq 0)$, contradiction.
So there isn't such $t$. So...
(*) $\Rightarrow$ $xf(x)=(xf(x))^2$ $\Rightarrow$ $1=xf(x)$
And the only answer is $f(x)=\frac{1}{x}$ (It's satisfying a), b), c))
\end{solution}



\begin{solution}[by \href{https://artofproblemsolving.com/community/user/29428}{pco}]
	\begin{tcolorbox}Find all functions $f:\mathbb{R}\setminus \{0\}\to\mathbb{R}$ which satisfy all the following conditions
a) $f(1)=1$
b) $f(\left(\frac{1}{x+y}\right)=f\left(\frac{1}{x}\right)+f\left(\frac{1}{y}\right)$
c) $(x+y)f(x+y)=xyf(x)f(y)\quad \forall xy(x+y)\neq 0$\end{tcolorbox}
No such function : condition b) is never respected for $(x,y)=(+1,-1)$
\end{solution}



\begin{solution}[by \href{https://artofproblemsolving.com/community/user/179121}{kurowassann}]
	But $(x,y)=(+1, -1)$ $\Rightarrow$ $xy(x+y)=0$, right?
\end{solution}



\begin{solution}[by \href{https://artofproblemsolving.com/community/user/29428}{pco}]
	\begin{tcolorbox}But $(x,y)=(+1, -1)$ $\Rightarrow$ $xy(x+y)=0$, right?\end{tcolorbox}
Yes, but OP tells that this rextriction is only for condition $c)$
No restriction is given for condition $b)$ and so it's supposed true $\forall x,y$ belonging to domain of $f(x)$
\end{solution}
*******************************************************************************
-------------------------------------------------------------------------------

\begin{problem}[Posted by \href{https://artofproblemsolving.com/community/user/145904}{Mulpin}]
	Find all functions $f : \mathbb{N} \to \mathbb{N}$ such that

$f(a^2 + f(b) + c) = af(a) + f(b+c) \forall a,b,c \in \mathbb{N}$
	\flushright \href{https://artofproblemsolving.com/community/c6h578359}{(Link to AoPS)}
\end{problem}



\begin{solution}[by \href{https://artofproblemsolving.com/community/user/29428}{pco}]
	\begin{tcolorbox}Find all functions $f : \mathbb{N} \to \mathbb{N}$ such that

$f(a^2 + f(b) + c) = af(a) + f(b+c) \forall a,b,c \in \mathbb{N}$\end{tcolorbox}
I consider that, as usual in this forum without precision, $0\notin\mathbb N$
Let $P(x,y,z)$ be the assertion $f(x^2+f(y)+z)=xf(x)+f(y+z)$

If $f(a)=f(b)$ for some $a>b$, then comparaison of $P(1,a,x-b)$ and $P(1,b,x-b)$ implies $f(x+(a-b))=f(x)$ $\forall x>b$
As a consequence $f(x)$ is bounded, in contradiction with functional equation whose RHS is not upper bounded.
So $f(x)$ is injective.

Comparaison of $P(1,x,f(1))$ and $P(1,1,f(x))$ implies then $f(x+f(1))=f(f(x)+1)$ and, since injective, $f(x)=x+f(1)-1$

Plugging this in functional equation, we get $f(1)=1$ and so $\boxed{f(x)=x}$ $\forall x$
\end{solution}
*******************************************************************************
-------------------------------------------------------------------------------

\begin{problem}[Posted by \href{https://artofproblemsolving.com/community/user/125553}{lehungvietbao}]
	Find all functions $f,g:\mathbb{R}\to\mathbb{R}$ such that $f$ is strictly increasing and \[f(xy)=f(x)g(y)+f(y)\quad \forall x,y\in \mathbb{R}\]
	\flushright \href{https://artofproblemsolving.com/community/c6h578361}{(Link to AoPS)}
\end{problem}



\begin{solution}[by \href{https://artofproblemsolving.com/community/user/29428}{pco}]
	\begin{tcolorbox}Find all functions $f,g:\mathbb{R}\to\mathbb{R}$ such that $f$ is strictly increasing and \[f(xy)=f(x)g(y)+f(y)\quad \forall x,y\in \mathbb{R}\]\end{tcolorbox}
Let $P(x,y)$ be the assertion $f(xy)=f(x)g(y)+f(y)$

If $f(0)=0$, then $P(0,x)$ $\implies$ $f(x)=0$ $\forall x$, impossible since $f(x)$ is strictly increasing. So $f(0)=\ne 0$
Let $a=\frac 1{f(0)}$

$P(0,x)$ $\implies$ $g(x)=1-af(x)$ and functional equation becomes :
$g(x)$ strictly monotonous and $g(0)=0$ and new assertion $Q(x,y)$ : $g(xy)=g(x)g(y)$
And so very classicaly $g(x)=\text{sign}(x)|x|^v$ $\forall x$ and for any $v>0$

And so $\boxed{(f(x),g(x))=\left(u(\text{sign}(x)|x|^v-1),\text{sign}(x)|x|^v\right)}$ $\forall x$ which indeed is a solution, whatever are $u,v>0$
\end{solution}
*******************************************************************************
-------------------------------------------------------------------------------

\begin{problem}[Posted by \href{https://artofproblemsolving.com/community/user/125553}{lehungvietbao}]
	Find all continuous functions $f:\mathbb{R}^+\to\mathbb{R}^+$ such that 
\[\begin{cases}f(2x)=2f(x)\quad \forall x\in \mathbb{R}^+ \\ f((f(x))^2)=xf(x)\quad \forall x\in \mathbb{R}^+ \\ f(x)\in\mathbb{N}^*\quad \forall x\in \mathbb{N}^*\end{cases}\]

Where $ \mathbb{N}^*$ is set of natural numbers (i.e :  $\mathbb{N}^*=\{0,1,2,..\}$)
	\flushright \href{https://artofproblemsolving.com/community/c6h578362}{(Link to AoPS)}
\end{problem}



\begin{solution}[by \href{https://artofproblemsolving.com/community/user/29428}{pco}]
	\begin{tcolorbox}Find all continuous functions $f:\mathbb{R}^+\to\mathbb{R}^+$ such that 
\[\begin{cases}f(2x)=2f(x)\quad \forall x\in \mathbb{R}^+ \\ f((f(x))^2)=xf(x)\quad \forall x\in \mathbb{R}^+ \\ f(x)\in\mathbb{N}^*\quad \forall x\in \mathbb{N}^*\end{cases}\]

Where $ \mathbb{N}^*$ is set of natural numbers (i.e :  $\mathbb{N}^*=\{0,1,2,..\}$)\end{tcolorbox}
No such function : third condition can not be verified for $x=0$ since $0\notin\mathbb R^+$ and so LHS is not defined.
\end{solution}
*******************************************************************************
-------------------------------------------------------------------------------

\begin{problem}[Posted by \href{https://artofproblemsolving.com/community/user/195015}{Jul}]
	Find functions $f,g:\mathbb{R}\rightarrow \mathbb{R}$ and satisfy :
\[f(g(x))=x^2,g(f(x))=x^3,\;\forall x\in \mathbb{R}\]
	\flushright \href{https://artofproblemsolving.com/community/c6h578366}{(Link to AoPS)}
\end{problem}



\begin{solution}[by \href{https://artofproblemsolving.com/community/user/29428}{pco}]
	\begin{tcolorbox}Find functions $f,g:\mathbb{R}\rightarrow \mathbb{R}$ and satisfy :
\[f(g(x))=x^2,g(f(x))=x^3,\;\forall x\in \mathbb{R}\]\end{tcolorbox}
$f(g(x))=x^2$ $\implies$ $g(f(g(x)))=g(x^2)$
$g(f(x))=x^3$ $\implies$ $g(f(g(x)))=g(x)^3$
And so $g(x^2)=g(x)^3$

A a consequence $g(-x)=g(x)$

If $g(u)=g(v)$, we get $u^2=f(g(u))=f(g(v))=v^2$

From $g(f(x))=x^3$, we get that $g(x)$ is surjective.
Let $a$ such that $g(a)=-1$ and so $g(a^2)=g(a)^3=-1=g(a)$ and so $a^4=a^2$
Let $b$ such that $g(b)=0$ and so $g(b^2)=g(b)^3=0=g(b)$ and so $b^4=b^2$
Let $c$ such that $g(c)=1$ and so $g(c^2)=g(c)^3=1=g(c)$ and so $c^4=c^2$

So $a,b,c$ are three distinct real roots of equation $x^4=x^2$ and so $\{a,b,c\}=\{-1,0,1\}$
And since $g(-1)=g(1)$, we get a contradiction since $g(\{a,b,c\})=\{g(a),g(b),g(c)\}=\{-1,0,1\}$ while $|g(\{-1,0,1\})|\le 2$

So no such functions.
\end{solution}



\begin{solution}[by \href{https://artofproblemsolving.com/community/user/201392}{nima-amini}]
	it is shortlist ....
\end{solution}
*******************************************************************************
-------------------------------------------------------------------------------

\begin{problem}[Posted by \href{https://artofproblemsolving.com/community/user/122611}{oty}]
	Find all increasing functions $f: \mathbb{R} \to \mathbb{R}$ such that : 
\[ f(x+f(y))=f(f(x))+f(y)  , \forall x,y \in \mathbb{R} \] .
	\flushright \href{https://artofproblemsolving.com/community/c6h578442}{(Link to AoPS)}
\end{problem}



\begin{solution}[by \href{https://artofproblemsolving.com/community/user/29428}{pco}]
	\begin{tcolorbox}Find all increasing functions $f: \mathbb{R} \to \mathbb{R}$ such that : 
\[ f(x+f(y))=f(f(x))+f(y)  , \forall x,y \in \mathbb{R} \] .\end{tcolorbox}
Let $P(x,y)$ be the assertion $f(x+f(y))=f(f(x))+f(y)$

$P(x-f(x),x)$ $\implies$ $f(f(x-f(x)))=0$ and so, since injective, $f(x)=x+a$ for some real $a$

Plugging this back in original equation, we get $a=0$ and so $\boxed{f(x)=x}$ $\forall x$
\end{solution}



\begin{solution}[by \href{https://artofproblemsolving.com/community/user/122611}{oty}]
	there is two others non obivious solution dear pco , here increasing don't eventually mean injective , like 
${f(x)=a+a\lfloor{\frac{x-b}{a}}\rfloor} , a\geq b>0 $ ,
\end{solution}



\begin{solution}[by \href{https://artofproblemsolving.com/community/user/29428}{pco}]
	Do you mean that $f(x)=0$ $\forall x$ is considered as an increasing function ?
\end{solution}



\begin{solution}[by \href{https://artofproblemsolving.com/community/user/122611}{oty}]
	yes Dear pco
\end{solution}



\begin{solution}[by \href{https://artofproblemsolving.com/community/user/89198}{chaotic_iak}]
	You should have said "nondecreasing" instead. "Increasing" is to be taken as "strictly increasing"; that is, if $x < y$, then $f(x) < f(y)$. (With nondecreasing, if $x < y$, then $f(x) \le f(y)$; equality might occur.)
\end{solution}



\begin{solution}[by \href{https://artofproblemsolving.com/community/user/29428}{pco}]
	\begin{tcolorbox}Find all increasing functions $f: \mathbb{R} \to \mathbb{R}$ such that : 
\[ f(x+f(y))=f(f(x))+f(y)  , \forall x,y \in \mathbb{R} \] .\end{tcolorbox}
Let $P(x,y)$ be the assertion $f(x+f(y))=f(f(x))+f(y)$
Let $A=f(\mathbb R)$
Let $B=\{x$ such that $f(x)=0\}$
Let $C=\{x$ such that $f(x)=x\}$. Note that $C\subseteq A$
Let $z=f(0)$

$P(-z,0)$ $\implies$ $f(f(-z))=0$ and so $0\in A$
$P(x,f(-z))$ $\implies$ $f(f(x))=f(x)$ $\forall x$ 
This last property implies $A=C$ and $z=0$

$P(x,y)$ becomes then $f(x+f(y))=f(x)+f(y)$ and from there it's easy to get that $A$ is an additive subgroup of $\mathbb R$

From there :

If $A$ is dense in $\mathbb R$, we easily get (using increasing property) $a=\mathbb R$ and $\boxed{\text{S1 : }f(x)=x}$ $\forall x$, which indeed is a solution

If $A$ is not dense in $\mathbb R$, then :
either $A=\{0\}$ and we get then $\boxed{\text{S2 : }f(x)=0}$ $\forall x$, which indeed is a solution
either $A=a\mathbb Z$ for some $a\in(0,+\infty)$ and so :
$f(na)=na$
$\forall x\in(na,(n+1)a)$ : either $f(x)=na$, either $f(x)=(n+1)a$ (we got a rising staircase function).
Using then $f(x+a)=f(x)+a$, it's easy to get the two families :

$\boxed{\text{S3 : }f(x)=a\left\lfloor\frac{x+b}a\right\rfloor}$ $\forall x$ for any $a>b\ge 0$

$\boxed{\text{S4 : }f(x)=a\left\lceil\frac{x-b}a\right\rceil}$ $\forall x$ for any $a>b\ge 0$


\begin{tcolorbox}...${f(x)=a+\lfloor{\frac{x-b}{a}}\rfloor} , a\geq b>0 $ ,\end{tcolorbox}
If this was an example of solution, it is wrong.
\end{solution}



\begin{solution}[by \href{https://artofproblemsolving.com/community/user/122611}{oty}]
	Wonderfull Dear pco, congratulation   :cool:
\end{solution}
*******************************************************************************
-------------------------------------------------------------------------------

\begin{problem}[Posted by \href{https://artofproblemsolving.com/community/user/145173}{Aiscrim}]
	Prove that there is an infinite number of functions $f:\mathbb{Z}\rightarrow \mathbb{Z}$ so that $f\circ f (x)=-x$
	\flushright \href{https://artofproblemsolving.com/community/c6h578546}{(Link to AoPS)}
\end{problem}



\begin{solution}[by \href{https://artofproblemsolving.com/community/user/29428}{pco}]
	\begin{tcolorbox}Prove that there is an infinite number of functions $f:\mathbb{Z}\rightarrow \mathbb{Z}$ so that $f\circ f (x)=-x$\end{tcolorbox}
Split the set of all positive integers in two equinumerous subsets $A,B$ and let $g(x)$ any bijection from $A\to B$
Define $f(x)$ as :

$f(0)=0$
If $x\in A$ : $f(x)=g(x)$
If $x\in B$ : $f(x)=-g^{-1}(x)$
If $-x\in A$ : $f(x)=-g(-x)$
If $-x\in B$ : $f(x)=g^{-1}(-x)$

And you have infinitely many such $A,B,g(x)$
\end{solution}
*******************************************************************************
-------------------------------------------------------------------------------

\begin{problem}[Posted by \href{https://artofproblemsolving.com/community/user/173116}{Sardor}]
	Find all continuouos functions $ f: \mathbb{R}\to\mathbb{R} $ such that for all reals $ x $ and $ y $ 
$ f(x+f(y))=y+f(x+1) $ .
	\flushright \href{https://artofproblemsolving.com/community/c6h578688}{(Link to AoPS)}
\end{problem}



\begin{solution}[by \href{https://artofproblemsolving.com/community/user/29428}{pco}]
	\begin{tcolorbox}Find all continuouos functions $ f: \mathbb{R}\to\mathbb{R} $ such that for all reals $ x $ and $ y $ 
$ f(x+f(y))=y+f(x+1) $ .\end{tcolorbox}
Let $P(x,y)$ be the assertion $f(x+f(y))=y+f(x+1)$

$P(0,y)$ $\implies$ $f(f(y))=y+f(1)$ and so $f(x)$ is bijective
$P(x,0)$ $\implies$ $f(x+f(0))=f(x+1)$ and so, since injective, $f(0)=1$

So $P(x,y)$ may be written $f(x+f(y))=f(f(y))-f(1)+f(x+1)$ and since surjective, $f(x+y)=f(x+1)+f(y)-f(1)$

Writing $g(x)=f(x+1)-f(1)$, this becomes $g(x+y)=g(x)+g(y)$ and, since continuous, $g(x)=ax$ and $f(x)=ax+1$

Plugging this back in original equation, we get $a^2=1$ and so two solutions :

$\boxed{\text{S1 : }f(x)=x+1}$ $\forall x$

$\boxed{\text{S2 : }f(x)=1-x}$ $\forall x$
\end{solution}



\begin{solution}[by \href{https://artofproblemsolving.com/community/user/173116}{Sardor}]
	Thanks ! Very nice solution.
\end{solution}



\begin{solution}[by \href{https://artofproblemsolving.com/community/user/173116}{Sardor}]
	Another Solution: Taking $ y = -f (x + 1) $, we see that there is a value a such that $ f (a) = 0 $. We
consider two cases.
Let first  $ a≠1 $. Taking $ y = x + 1 $, we get $ f (x + f (x + 1)) = x + 1 + f (x + 1) $. Let $
g(x) = x + f (x + 1) $, then $ f (g(x)) = 1 + g(x) $ for all x. Since $ f $ is continuous, so is $ g $.
Taking $ y = a $ in the initial relation, we get $ f (x) = a+ f (x +1) $, and so $ g(x -1)- g(x) =
a - 1 $ for all $ x $. Since $ a ≠ 1 $, $ g $ is unbounded and by continuity, takes all real values, so $ 
f (z) = 1 + z $ for all $ z $.
Let now $ a = 1 $, i.e, $ f (1) = 0 $. Then $ x = 0 $ yields $ f ( f (y)) = y  $ for all reals $ y $. Taking now $
y = f (1 - x) $ in the initial relation, we get $ f (x + f ( f (1 - x))) = f (1 - x) + f (x + 1) $,
or $ 0 = f (1 - x) + f (x + 1) $. Finally, taking $ y = 1 - x $ yields $ f (x + f (1 - x)) = 1 - x + f (x + 1) $, so $ f (x + f (1 - x)) = 1 - x - f (1 - x) $. Let $ h(x) = x + f (1 - x) $, then $
f (h(x)) = 1 - h(x) $ holds for all x. Replacing $ x $ with $ -x $ and taking $ y = 1 $ in the initial
relation, we get $ f (-x) = 1 + f (1 - x) $, so $ h(x + 1) - h(x) = 2 $. Again, $ h $ is continuous
and must take all real values, so $ f (z) = 1- z $ for all $ z $.
It is straightforward to verify that both solutions indeed satisfy the initial relation.
\end{solution}
*******************************************************************************
-------------------------------------------------------------------------------

\begin{problem}[Posted by \href{https://artofproblemsolving.com/community/user/119826}{seby97}]
	Find all f,g:R->R such that $f(x)-f(y)=(x-y)g(x+y)$,for all x,y real numbers
	\flushright \href{https://artofproblemsolving.com/community/c6h578806}{(Link to AoPS)}
\end{problem}



\begin{solution}[by \href{https://artofproblemsolving.com/community/user/29428}{pco}]
	\begin{tcolorbox}Find all f,g:R->R such that $f(x)-f(y)=(x-y)g(x+y)$,for all x,y real numbers\end{tcolorbox}
Setting $y=0$ in functional equation, we get $f(x)=xg(x)+f(0)$
Plugging this back in original equation, we get assertion $P(x,y)$ : $xg(x)-yg(y)=(x-y)g(x+y)$

$P(\frac{x+1}2,\frac{x-1}2)$ $\implies$ $\frac{x+1}2g(\frac{x+1}2)-\frac{x-1}2g(\frac{x-1}2)=g(x)$

$P(\frac{x-1}2,\frac{3-x}2)$ $\implies$ $\frac{x-1}2g(\frac{x-1}2)-\frac{3-x}2g(\frac{3-x}2)=(x-2)g(1)$

$P(\frac{3-x}2,\frac{x+1}2)$ $\implies$ $\frac{3-x}2g(\frac{3-x}2)-\frac{x+1}2g(\frac{x+1}2)=(1-x)g(2)$

Adding these three lines, we get $g(x)=x(g(2)-g(1))+2g(1)-g(2)$ and so $g(x)=ax+b$, which indeed is a solution, whatever are $a,b\in\mathbb R$

Hence the answer : $\boxed{(f,g)=(ax^2+bx+c,ax+b)}$ which indeed is a solution, whatever are $a,b,c\in\mathbb R$
\end{solution}
*******************************************************************************
-------------------------------------------------------------------------------

\begin{problem}[Posted by \href{https://artofproblemsolving.com/community/user/172778}{Shayanhas}]
	Find all functions $ f :\mathbb{R}^+\mapsto\mathbb{R}^+ $ such that $y\in{(0,x^{2})}$ and
\[f(x)=f(\frac{x^{2}-y}{f(x)})+f(\frac{f(y)}{x})\]
	\flushright \href{https://artofproblemsolving.com/community/c6h578815}{(Link to AoPS)}
\end{problem}



\begin{solution}[by \href{https://artofproblemsolving.com/community/user/29428}{pco}]
	\begin{tcolorbox}Find all functions $ f :\mathbb{R}^+\mapsto\mathbb{R}^+ $ such that $y\in{(0,x^{2})}$ and
\[f(x)=f(\frac{x^{2}-y}{f(x)})+f(\frac{f(y)}{x})\]\end{tcolorbox}
Let $P(x,y)$ the assertion $f(x)=f\left(\frac{x^2-y}{f(x)}\right)+f\left(\frac{f(y)}x\right)$, true $\forall x>0$, $\forall y\in(0,x^2)$

If $f(x)>x$, then $x\in(0,f(x))$ and so $P(\sqrt{f(x)},x)$ $\implies$ $f\left(\frac{f(x)-x}{f(\sqrt{f(x)})}\right)=0$, impossible. So $f(x)\le x$ $\forall x$

If $f(x)<x$, then $x^2-xf(x)\in(0,x^2)$ and so $P(x,x^2-xf(x))$ $\implies$ $f\left(\frac{f(x^2-xf(x))}x\right)=0$, impossible. So $f(x)\ge x$ $\forall x$

And so $\boxed{f(x)=x}$ $\forall x$, which indeed is a solution.
\end{solution}
*******************************************************************************
-------------------------------------------------------------------------------

\begin{problem}[Posted by \href{https://artofproblemsolving.com/community/user/172778}{Shayanhas}]
	Find all functions $ f :\mathbb{R}^+\mapsto\mathbb{R}^+ $ such that $ y\in{(0,x^{2})} $ and
\[ f(x)=f(\frac{x^{2}-y}{f(x)})+\frac{f(y)}{x} \]
	\flushright \href{https://artofproblemsolving.com/community/c6h578844}{(Link to AoPS)}
\end{problem}



\begin{solution}[by \href{https://artofproblemsolving.com/community/user/141363}{alibez}]
	it is easy to see that :$f(x)\geq x$ because if $f(x)< x\rightarrow P(x,x^{2}-xf(x))\Rightarrow \frac{f(x^{2}-xf(x))}{x}=0$
so we have : $f(x)\geq x$

now let $g(x)=f(x)-x$ so we have $g:\mathbb{R^{+}}\rightarrow  [ 0 , +\infty )$ and for any $x$ and $y\in (0,x^{2})$ we have : 

$g(x)y+g(x)g(y)+xg(y)=2x^{2}g(x)+xg(x)^{2}$ and now it is easy to see that : $g(x)=-x+\sqrt{x^{4}+xc}\:\: \: \:  \: \forall x> 1$ so ...........
\end{solution}



\begin{solution}[by \href{https://artofproblemsolving.com/community/user/29428}{pco}]
	\begin{tcolorbox}... we have : 

$g(x)y+g(x)g(y)+xg(y)=2x^{2}g(x)+xg(x)^{2}$ ...\end{tcolorbox}
How ?
\end{solution}



\begin{solution}[by \href{https://artofproblemsolving.com/community/user/141363}{alibez}]
	\begin{tcolorbox}[quote="alibez"]... we have : 

$g(x)y+g(x)g(y)+xg(y)=2x^{2}g(x)+xg(x)^{2}$ ...\end{tcolorbox}
How ?\end{tcolorbox}

sorry PCO for my bad mistake . :blush:
\end{solution}
*******************************************************************************
-------------------------------------------------------------------------------

\begin{problem}[Posted by \href{https://artofproblemsolving.com/community/user/125513}{hal9v4ik}]
	Is it true that for  polynomials $P,Q,R$:
a) if $P$ and $Q$ are polynomials(with leading coefficint 1) of degree $n$ with n real roots each,then for $P+Q=2*R$, $R$ have $n$ real roots.
b)if $P$ and $Q$ don't have real roots then $R$ also don't have real root.
	\flushright \href{https://artofproblemsolving.com/community/c6h578944}{(Link to AoPS)}
\end{problem}



\begin{solution}[by \href{https://artofproblemsolving.com/community/user/29428}{pco}]
	\begin{tcolorbox}Is it true that for  polynomials $P,Q,R$:
a) if $P$ and $Q$ are polynomials(with leading coefficint 1) of degree $n$ with n real roots each,then for $P+Q=2*R$, $R$ have $n$ real roots.\end{tcolorbox}
Wrong. Choose as counter-example :
$P(x)=x^2-x$
$Q(x)=x^2-5x+6$
$R(x)=x^2-3x+3$
\end{solution}



\begin{solution}[by \href{https://artofproblemsolving.com/community/user/29428}{pco}]
	\begin{tcolorbox}Is it true that for  polynomials $P,Q,R$:
b)if $P$ and $Q$ don't have real roots then $R$ also don't have real root.\end{tcolorbox}
Trivially true if monic property is true also for question b) : $P(x)>0$ $\forall x$ and $Q(x)>0$ $\forall x$ and so $R(x)>0$ $\forall x$
Trivially false if monic property is not true for question b) : $P(x)=x^2+1$ and $Q(x)=-x^2-1$
\end{solution}
*******************************************************************************
-------------------------------------------------------------------------------

\begin{problem}[Posted by \href{https://artofproblemsolving.com/community/user/68025}{Pirkuliyev Rovsen}]
	Find all of the function $f: \mathbb{R}\to\mathbb{R}$ such that $f(\frac{1+x}{1-x})=af(x)$, where $a$ is constant real number.
	\flushright \href{https://artofproblemsolving.com/community/c6h578955}{(Link to AoPS)}
\end{problem}



\begin{solution}[by \href{https://artofproblemsolving.com/community/user/29428}{pco}]
	\begin{tcolorbox}Find all of the function $f: \mathbb{R}\to\mathbb{R}$ such that $f(\frac{1+x}{1-x})=af(x)$, where $a$ is constant real number.\end{tcolorbox}
What is domain of functional equation ? $\mathbb R$ ?
\end{solution}



\begin{solution}[by \href{https://artofproblemsolving.com/community/user/68025}{Pirkuliyev Rovsen}]
	Maybe be so $x{\neq}1$. Condition is correctly written.
\end{solution}



\begin{solution}[by \href{https://artofproblemsolving.com/community/user/64716}{mavropnevma}]
	First, clearly the functional equation cannot apply for $x=1$. Then, for $x=0$ we have $f(1)=af(0)$, and for $x=-1$ we have $f(0)=af(-1)$. Thus, for $a\neq 0$ we can arbitrarily take $f(1) = b$, and we get $f(0)=b\/a$ and $f(-1) = b\/a^2$, while for $a=0$ we get $f(1) = f(0)=0$ and we can arbitrarily take $f(-1) = b$. Except for these particular three values, proceed as follows.

Define $\varphi : \mathbb{R}\setminus \{-1,0,1\} \to \mathbb{R}\setminus \{-1,0,1\} $ given by $\varphi(x)=\dfrac {1+x} {1-x}$. Then $\varphi^2(x) = -\dfrac {1} {x}$, $\varphi^3(x) = -\dfrac {1-x} {1+x}$, $\varphi^4(x) = x$.

Your equation is $f(\varphi(x)) = af(x)$, so $f(\varphi^2(x)) = af(\varphi(x))$, $f(\varphi^3(x)) = af(\varphi^2(x))$, $f(x) = f(\varphi^4(x)) = af(\varphi^3(x)) = \cdots = a^4f(x)$, so $(a^4-1)f(x) = 0$. Thus for $a\neq \pm 1$ we have $f(x) = 0$ for all $x\in \mathbb{R}\setminus \{-1,0,1\}$.

For $a=\pm 1$, for any $x \in \mathbb{R}\setminus  \{-1,0,1\} \{-1,0,1\}$ we have the $4$-orbit $\{x,\varphi(x), \varphi^2(x), \varphi^3(x)\}$, and on it we can arbitrarily take $f(x) = b_x$, with the other three values uniquely determined.
\end{solution}
*******************************************************************************
-------------------------------------------------------------------------------

\begin{problem}[Posted by \href{https://artofproblemsolving.com/community/user/119826}{seby97}]
	Fie all monotonic function f:R->R such that $f(f(f(x)))-3f(f(x))+6f(x)=4x+3$,for all real numbers x
	\flushright \href{https://artofproblemsolving.com/community/c6h578967}{(Link to AoPS)}
\end{problem}



\begin{solution}[by \href{https://artofproblemsolving.com/community/user/29428}{pco}]
	\begin{tcolorbox}Fie all monotonic function f:R->R such that $f(f(x)))-3f(f(x))+6f(x)=4x+3$,for all real numbers x\end{tcolorbox}
Mismatch parenthesis in LHS. And I doubt you really wanted to write $f(f(x))-3f(f(x))$ ...
Please check.
\end{solution}



\begin{solution}[by \href{https://artofproblemsolving.com/community/user/119826}{seby97}]
	\begin{tcolorbox}[quote="seby97"]Fie all monotonic function f:R->R such that $f(f(x)))-3f(f(x))+6f(x)=4x+3$,for all real numbers x\end{tcolorbox}
Mismatch parenthesis in LHS. And I doubt you really wanted to write $f(f(x))-3f(f(x))$ ...
Please check.\end{tcolorbox}
I modified,thank you
\end{solution}



\begin{solution}[by \href{https://artofproblemsolving.com/community/user/29428}{pco}]
	\begin{tcolorbox}Fie all monotonic function f:R->R such that $f(f(f(x)))-3f(f(x))+6f(x)=4x+3$,for all real numbers x\end{tcolorbox}
Let $x\in\mathbb R$ and the sequence $a_n=f^{[n]}(x)$ and we get :
$a_0=x$
$a_1=f(x)$
$a_2=f(f(x))$
$a_{n+3}=3a_{n+2}-6a_{n+1}+4a_n+3$

So $a_n=n+c+r2^n\cos(n\frac{\pi}3+\varphi)$ for some real $c,r,\varphi$ depending on $x,f(x), f(f(x))$

If $r\ne 0$, then $\frac{a_{n+4}-a_{n+1}}{a_{n+3}-a_n}$ $=\frac{1-3r2^{n+1}\cos((n+1)\frac{\pi}3+\varphi)}{1-3r2^{n}\cos(n\frac{\pi}3+\varphi)}$ may take, when $n\to +\infty$ positive and negative values, which is impossible since $f(x)$ is monotonous.

So $r=0$ and $a_n=n+c$. So $c=a_0=x$ and $a_1=\boxed{f(x)=x+1}$ $\forall x$ which indeed is a solution
\end{solution}



\begin{solution}[by \href{https://artofproblemsolving.com/community/user/19538}{dominicleejun}]
	roots of $x^3-3x^2+6x-4=0$ are 1 and $1\pm \sqrt{3}i$
\end{solution}
*******************************************************************************
-------------------------------------------------------------------------------

\begin{problem}[Posted by \href{https://artofproblemsolving.com/community/user/78770}{thuanspdn}]
	Find all continous functions $f:\mathbb{R}\to\mathbb{R}$ such that the following identity is satisfied for all $x,y,z$ 
$f(x-y).f(y-z).f(z-x)+8=0$
	\flushright \href{https://artofproblemsolving.com/community/c6h579116}{(Link to AoPS)}
\end{problem}



\begin{solution}[by \href{https://artofproblemsolving.com/community/user/29428}{pco}]
	\begin{tcolorbox}Find all continous functions $f:\mathbb{R}\to\mathbb{R}$ such that the following identity is satisfied for all $x,y,z$ 
$f(x-y).f(y-z).f(z-x)+8=0$\end{tcolorbox}
Let $P(x,y,z)$ be the assertion $f(x-y)f(y-z)f(z-x)+8=0$

$P(0,0,0)$ $\implies$ $f(0)=-2$
If $\exists u$ such that $f(u)=0$, then $P(u,0,0)$ is wrong. So, since continuous, $f(x)$ has a constant sign, and so $f(x)<0$ $\forall x$

Let $f(x)=-2g(x)$ and $P(x,y,z)$ becomes  $g(x-y)g(y-z)g(z-x)=1$ $\forall x,y,z$ with $g(x)>0$ $\forall x$

$P(x+y,0,x+y)$ $\implies$ $g(x+y)g(-x-y)=1$
$P(x+y,y,0)$ $\implies$ $g(x)g(y)g(-x-y)=1$
So, since $>0$, $g(x+y)=g(x)g(y)$ $\forall x,y$ and so, since continuous, $g(x)=e^{ax}$

And so $\boxed{f(x)=-2e^{ax}}$ $\forall x$ which indeed is a solution, whatever is $a\in\mathbb R$
\end{solution}
*******************************************************************************
-------------------------------------------------------------------------------

\begin{problem}[Posted by \href{https://artofproblemsolving.com/community/user/150353}{shezi1995}]
	f:R-->R
f(xy + f(x)) = xf(y) + f(x) for all real x,y

My attempt:
1) I put y=0 and got ff(x)=xf(0)+f(x)
2) Then I put y=f(y) and after some manipulations and application of above result got f(0)=0
3) This gets me to ff(x)=f(x)
4) I also got f(f(x)^2+f(x))=f(x)^2+f(x)

I can't get anywhere further. I would highly appreciate if you could guide me to another step or give me a small push in the right direction. Please don't put the whole solution. Hide the solution if you can.
	\flushright \href{https://artofproblemsolving.com/community/c6h579493}{(Link to AoPS)}
\end{problem}



\begin{solution}[by \href{https://artofproblemsolving.com/community/user/29428}{pco}]
	\begin{tcolorbox}f:R-->R
I can't get anywhere further. I would highly appreciate if you could guide me to another step or give me a small push in the right direction. Please don't put the whole solution. Hide the solution if you can.\end{tcolorbox}
Hint : 
What conclusion if $f(a)=f(b)$ for some $a,b$ ?
And so ... ?
\end{solution}



\begin{solution}[by \href{https://artofproblemsolving.com/community/user/192241}{mad}]
	\begin{tcolorbox}f:R-->R
f(xy + f(x)) = xf(y) + f(x) for all real x,y
.\end{tcolorbox}
Replacing $y$ by $\frac{x-f(x)}{x}$ we get
$f(\frac{x-f(x)}{x})=0$ for all $x\neq 0$.
But what to do next?
\end{solution}



\begin{solution}[by \href{https://artofproblemsolving.com/community/user/150353}{shezi1995}]
	\begin{tcolorbox}[quote="shezi1995"]f:R-->R
f(xy + f(x)) = xf(y) + f(x) for all real x,y
.\end{tcolorbox}
Replacing $y$ by $\frac{x-f(x)}{x}$ we get
$f(\frac{x-f(x)}{x})=0$ for all $x\neq 0$.
But what to do next?\end{tcolorbox}
Substitute it in place of x and y=1 and I get f(x)=x (for nonzero x and also true for x=0) which is a solution or f(1)=0. What to do with f(1)?
\end{solution}



\begin{solution}[by \href{https://artofproblemsolving.com/community/user/192241}{mad}]
	I think $f(1)=0$ gives the only constant solution $f(x)=0$.
\end{solution}



\begin{solution}[by \href{https://artofproblemsolving.com/community/user/150353}{shezi1995}]
	\begin{tcolorbox}[quote="shezi1995"]f:R-->R
I can't get anywhere further. I would highly appreciate if you could guide me to another step or give me a small push in the right direction. Please don't put the whole solution. Hide the solution if you can.\end{tcolorbox}
Hint : 
What conclusion if $f(a)=f(b)$ for some $a,b$ ?
And so ... ?\end{tcolorbox}
I tried proving injectivity but to no avail. I got af(1\/a)=bf(1\/b).
\end{solution}



\begin{solution}[by \href{https://artofproblemsolving.com/community/user/29428}{pco}]
	\begin{tcolorbox}...I tried proving injectivity but to no avail. I got af(1\/a)=bf(1\/b).\end{tcolorbox}
If $f(a)=f(b)=c$ then $(a+1)c=af(b)+f(a)$ $=f(ab+f(a))=f(ba+f(b))$ $=bf(a)+f(b)=(b+1)c$

And so $f(a)=f(b)$ implies either $f(a)=f(b)=0$, either $a=b$

Now, from $f(f(x))=f(x)$, we get ?
\end{solution}



\begin{solution}[by \href{https://artofproblemsolving.com/community/user/150353}{shezi1995}]
	\begin{tcolorbox}[quote="shezi1995"]...I tried proving injectivity but to no avail. I got af(1\/a)=bf(1\/b).\end{tcolorbox}
If $f(a)=f(b)=c$ then $(a+1)c=af(b)+f(a)$ $=f(ab+f(a))=f(ba+f(b))$ $=bf(a)+f(b)=(b+1)c$

And so $f(a)=f(b)$ implies either $f(a)=f(b)=0$, either $a=b$

Now, from $f(f(x))=f(x)$, we get ?\end{tcolorbox}
Oh yes! Either f(x)=0 or f(x)=x both of which satisfy the given functional equation.
\end{solution}



\begin{solution}[by \href{https://artofproblemsolving.com/community/user/29428}{pco}]
	\begin{tcolorbox}.. Oh yes! Either f(x)=0 or f(x)=x both of which satisfy the given functional equation.\end{tcolorbox}
Not exactly.
You get : $\forall x$, either $f(x)=0$, either $f(x)=x$
Which is different from : Either $f(x)=x$ $\forall x$, either $f(x)=0$ $\forall x$. 
You need some lines more.
\end{solution}
*******************************************************************************
-------------------------------------------------------------------------------

\begin{problem}[Posted by \href{https://artofproblemsolving.com/community/user/197930}{ThisIsART}]
	Find all function $f: \mathbb{N} \rightarrow \mathbb{N}$ so for every positive integer number $a, b, c$ apply

\[ f(ab+f( c ))=a+f(b)c \]
	\flushright \href{https://artofproblemsolving.com/community/c6h579662}{(Link to AoPS)}
\end{problem}



\begin{solution}[by \href{https://artofproblemsolving.com/community/user/29428}{pco}]
	\begin{tcolorbox}Find all function $f: \mathbb{N} \rightarrow \mathbb{N}$ so for every positive integer number $a, b, c$ apply

\[ f(ab+f( c ))=a+f(b)c \]\end{tcolorbox}
Let $P(x,y,z)$ be the assertion $f(xy+f(z))=x+f(y)z$

$P(x,1,1)$ $\implies$ $f(x+f(1))=x+f(1)$
$P(1,x,1)$ $\implies$ $f(x+f(1))=1+f(x)$
Subtracting, we get $f(x)=x+(f(1)-1)$ which unfortunately is never a solution, whatever is the constant $f(1)-1$

So no solution.
\end{solution}



\begin{solution}[by \href{https://artofproblemsolving.com/community/user/197930}{ThisIsART}]
	Any other solution?
\end{solution}



\begin{solution}[by \href{https://artofproblemsolving.com/community/user/29428}{pco}]
	\begin{tcolorbox}Any other solution?\end{tcolorbox}
You are welcome. Glad to have helped you.

Btw, is there any problem with my solution ?
\end{solution}



\begin{solution}[by \href{https://artofproblemsolving.com/community/user/197930}{ThisIsART}]
	Oh i see now your answer is correct i am sorry to find another solution
\end{solution}
*******************************************************************************
-------------------------------------------------------------------------------

\begin{problem}[Posted by \href{https://artofproblemsolving.com/community/user/196603}{nbh}]
	problem: find all continuous functions $f:(0,+\infty)\to(0,+\infty)$ such that
\[f(x)f(y)f(z)=f(xyz)+\frac{f(x)}{f(y)} +\frac{f(y)}{f(z)} +\frac{f(z)}{f(x)},\forall x,y,z\in (0,+\infty)\]
	\flushright \href{https://artofproblemsolving.com/community/c6h579704}{(Link to AoPS)}
\end{problem}



\begin{solution}[by \href{https://artofproblemsolving.com/community/user/29428}{pco}]
	\begin{tcolorbox}problem: find all continuous functions $f:(0,+\infty)\to(0,+\infty)$ such that
\[f(x)f(y)f(z)=f(xyz)+\frac{f(x)}{f(y)} +\frac{f(y)}{f(z)} +\frac{f(z)}{f(x)},\forall x,y,z\in (0,+\infty)\]\end{tcolorbox}
Let $a=f(1)$.
Setting $x=y=z=1$ in functional equation, we get $a^3-a-3=0$
And so a unique positive value for $a$ easily got thru Cardano's formulas.

Setting $y=z=1$ in functional equation and using the above equation for $a$, we get $(f(x)-a)(2f(x)+a)=0$ and so :
$\boxed{f(x)=a}$ $\forall x$, which indeed is a solution.
\end{solution}
*******************************************************************************
-------------------------------------------------------------------------------

\begin{problem}[Posted by \href{https://artofproblemsolving.com/community/user/194363}{PhamKhacLinh}]
	Find continued function f(x): R -> R  so that :
f(f(x)) = x
	\flushright \href{https://artofproblemsolving.com/community/c6h579786}{(Link to AoPS)}
\end{problem}



\begin{solution}[by \href{https://artofproblemsolving.com/community/user/29428}{pco}]
	\begin{tcolorbox}Find continued function $f(x): R -> R$  so that :
\[f(f(x)) = x\]\end{tcolorbox}
$f(f(x))=x$ implies $f(x)$ bijective and, so, since continuous, strictly monotonous.

If $f(x)$ is strictly increasing, then it's immediate to get the unique solution $f(x)=x$ $\forall x$

If $f(x)$ is strictly decreasing, then $\exists a$ such that $f(a)=a$.
Let then $g(x)=f(x+a)-a$ and we get $g(x)$ continuous strictly decreasing and $g(g(x))=x$ and $g(0)=0$

The general solution for $g(x)$ is then :
Let $h(x)$ any continuous decreasing bijective function from $[0,+\infty)\to[0,-\infty)$
$g(x)$ is defined as :
$\forall x\ge 0$ : $g(x)=h(x)$
$\forall x<0$ : $g(x)=h^{-1}(x)$

Hence the general solution for original equation.
\end{solution}



\begin{solution}[by \href{https://artofproblemsolving.com/community/user/115063}{PhantomR}]
	\begin{tcolorbox}
If $f(x)$ is strictly decreasing, then $\exists a$ such that $f(a)=a$.
\end{tcolorbox}

Greetings sir,

If I may ask, where does this follow from? It looks like a nice result.
\end{solution}



\begin{solution}[by \href{https://artofproblemsolving.com/community/user/29428}{pco}]
	\begin{tcolorbox}[quote="pco"]
If $f(x)$ is strictly decreasing, then $\exists a$ such that $f(a)=a$.
\end{tcolorbox}

Greetings sir,

If I may ask, where does this follow from? It looks like a nice result.\end{tcolorbox}
$f(f(x))=x$ implies $f(x)$ bijective.

$f(x)$ is bijective and decreasing and so $\lim_{x\to-\infty}f(x)=+\infty$
$f(x)$ is bijective and decreasing and so $\lim_{x\to+\infty}f(x)=-\infty$
So $f(x)-x$ is continuous and positive when $x\to-\infty$ and negative when $x\to+\infty$ and so has a zero.
Hence the result.
\end{solution}
*******************************************************************************
-------------------------------------------------------------------------------

\begin{problem}[Posted by \href{https://artofproblemsolving.com/community/user/188482}{Gzego}]
	Determine all functions $f:\mathbb{R}\rightarrow\mathbb{R}$ such that for all $x, y,$

$f(x+2)=f(x)$ and $f(x^2)=(f(x))^2$.
	\flushright \href{https://artofproblemsolving.com/community/c6h579917}{(Link to AoPS)}
\end{problem}



\begin{solution}[by \href{https://artofproblemsolving.com/community/user/29428}{pco}]
	\begin{tcolorbox}Determine all functions $f:\mathbb{R}\rightarrow\mathbb{R}$ such that for all $x, y,$

$f(x+2)=f(x)$ and $f(x^2)=(f(x))^2$.\end{tcolorbox}
Sorry, but could you kindly confirm us that this is a real exercise got in a real olympiad or training session ?

There are obviously infinitely many solutions :
$f(x)=0$ $\forall x$
$f(x)=1$ $\forall x$
$f(x)=1_{\text{Algebraic numbers}}(x)$

But also some nice unbounded solutions ...

And I'm rather surprised that a general form is requested for all these solutions .... .
\end{solution}
*******************************************************************************
-------------------------------------------------------------------------------

\begin{problem}[Posted by \href{https://artofproblemsolving.com/community/user/208238}{phamngocsonyb}]
	Determine all functions $f:Z - Z$ where $Z$ is the set of integers, such that 

$f(m+f(f(n)))=-f(f(m+1))-n$ for all integers $m$ and $n$
	\flushright \href{https://artofproblemsolving.com/community/c6h580160}{(Link to AoPS)}
\end{problem}



\begin{solution}[by \href{https://artofproblemsolving.com/community/user/29428}{pco}]
	\begin{tcolorbox}Determine all functions $f:Z - Z$ where $Z$ is the set of integers, such that 

$f(m+f(f(n)))=-f(f(m+1))-n$ for all integers $m$ and $n$\end{tcolorbox}
Let $P(x,y)$ be the assertion $f(x+f(f(y)))=-f(f(x+1))-y$
$f(x)$ is bijective

$P(0,y)$ $\implies$ $f(f(f(y)))=a-y$ for some $a=-f(f(1))$
Using surjectivity, let $u$ such that $f(f(u+1))=0$ : $P(u,f(y))$ $\implies$  $f(u+a-y)=-f(y)$

So $P(x,y)$ may be written $f(x+f(f(y)))=f(u+a-f(x+1))-y$ and so $P(x,0)$ becomes $f(x+f(f(0)))=f(u+a-f(x+1))$

Then, injectivity implies $f(x)=c-x$ for some $c$ and, plugging this back in original equation, we get $c=-1$

Hence the unique solution $\boxed{f(x)=-x-1}$
\end{solution}
*******************************************************************************
-------------------------------------------------------------------------------

\begin{problem}[Posted by \href{https://artofproblemsolving.com/community/user/179643}{_Mark_01}]
	Find all functions f :$ R \to R$ such that for all $x,y \in R $
$f \left( \sqrt{x^2+y^2} \right) = f(x)f(y)$
	\flushright \href{https://artofproblemsolving.com/community/c6h580500}{(Link to AoPS)}
\end{problem}



\begin{solution}[by \href{https://artofproblemsolving.com/community/user/29428}{pco}]
	\begin{tcolorbox}Find all functions f :$ R \to R$ such that for all $x,y \in R $
$f \left( \sqrt{x^2+y^2} \right) = f(x)f(y)$\end{tcolorbox}
$\boxed{\text{S1: }f(x)=0}$ $\forall x$ is a solution. So let us from now look only for non allzero solutions.
Let $P(x,y)$ be the assertion $f(\sqrt{x^2+y^2})=f(x)f(y)$
Let $u$ such that $f(u)\ne 0$

Comparing $P(x,u)$ with $P(-x,u)$, we get $f(x)=f(-x)$ and so $f(x)$ is an even function and WLOG $u\ge 0$

$P(0,0)$ $\implies$ $f(0)\in\{0,1\}$. But if $f(0)=0$, then $P(u,0)$ implies $f(u)=0$, impossible. So $f(0)=1$

If $f(a)=0$ for some $a>0$, then :
$P(x,a)$ $\implies$ $f(\sqrt{x^2+a^2})=0$ and so $f(x)=0$ $\forall x\ge a$
$P(\frac a{\sqrt 2},\frac a{\sqrt 2})$ $\implies$ $f(\frac a{\sqrt 2})=0$ and so $\boxed{\text{S2 : }f(0)=1\text{ and }f(x)=0\text{   }\forall x\ne 0}$ which indeed is a solution.

So let us from now consider $f(x)\ne 0$ $\forall x$.

Let $x>0$ : $P(\frac x{\sqrt 2},\frac x{\sqrt 2})$ $\implies$ $f(x)>0$ and, since even, $f(x)>0$ $\forall x$

Let then $g(x)=\ln f(\sqrt{|x|})$ so that $f(x)=e^{g(x^2)}$ and functional equation becomes $g(x+y)=g(x)+g(y)$ $\forall x,y\ge 0$

And so $\boxed{\text{S3 : }f(x)=e^{a(x^2)}}$ where $a(x)$ is any additive function (solution of additive Cauchy equation)
\end{solution}



\begin{solution}[by \href{https://artofproblemsolving.com/community/user/179643}{_Mark_01}]
	Hi, Pco. I did as follows:
We have $\left[ f\left(\sqrt{x^2+y^2}\right) \right]'_x = f'_x(x)f(y) $

$\rightarrow f\left(\sqrt{x^2+y^2}\right) \cdot \frac{x}{\sqrt{x^2+y^2}} = f'_x(x)f(y) $                 $(1)$

and $\left[ f\left(\sqrt{x^2+y^2}\right) \right]'_y = f(x)f'_y(y) $

so $f\left(\sqrt{x^2+y^2}\right) \cdot \frac{y}{\sqrt{x^2+y^2}} = f(x)f'_y(y) $                              $(2)$

$(1)(2) \rightarrow \frac{f'(x)}{xf(x)}=\frac{f'(y)}{yf(y)}$

so $\frac{f'(x)}{xf(x)}=const$
then $f(x)=e^{cx^2}$

It is correct?
\end{solution}



\begin{solution}[by \href{https://artofproblemsolving.com/community/user/29428}{pco}]
	\begin{tcolorbox}Hi, Pco. I did as follows:
We have $\left[ f\left(\sqrt{x^2+y^2}\right) \right]'_x = f'_x(x)f(y) $

...

It is correct?\end{tcolorbox}
No, it is not : There is no reason to suppose that $f(x)$ is differentiable (and even simply continuous). Note that in my previous post, I found many non continuous solution.
\end{solution}
*******************************************************************************
-------------------------------------------------------------------------------

\begin{problem}[Posted by \href{https://artofproblemsolving.com/community/user/150054}{wer}]
	Determine convex functions f: [0,∞) - [0,∞) with the following properties: 
a)f (0) = 0.
b)2∙f((x+y)\/2)≥|(xf(x)-yf(y))\/(x-y)|, x,y ∈(0,∞),x≠y.
	\flushright \href{https://artofproblemsolving.com/community/c6h580511}{(Link to AoPS)}
\end{problem}



\begin{solution}[by \href{https://artofproblemsolving.com/community/user/29428}{pco}]
	\begin{tcolorbox}Determine convex functions f: [0,∞) - [0,∞) with the following properties: 
a)f (0) = 0.
b)2∙f((x+y)\/2)≥|(xf(x)-yf(y))\/(x-y)|, x,y ∈(0,∞),x≠y.\end{tcolorbox}
Conditions imply continuity. Setting then $y\to 0$, we get $\frac{f(\frac x2)}{\frac x2}\ge \frac{f(x)}x$

But $f(0)=0$ and convexity imply that $\frac{f(x)}x$ is increasing.

So $\frac{f(x)}x$ is constant and we get the solution $\boxed{f(x)=ax}$ $\forall x$ which indeed is a solution, whatever is $a\ge 0$
\end{solution}
*******************************************************************************
-------------------------------------------------------------------------------

\begin{problem}[Posted by \href{https://artofproblemsolving.com/community/user/184652}{CanVQ}]
	Let $a$ be a fixed positive real number. Find all strictly monotone function $f: \mathbb R^+ \to \mathbb R^+$ such that \[(x+a)\cdot f(x+y)=a\cdot f\big(y\cdot f(x)\big),\quad \forall x,\,y >0.\]
	\flushright \href{https://artofproblemsolving.com/community/c6h580575}{(Link to AoPS)}
\end{problem}



\begin{solution}[by \href{https://artofproblemsolving.com/community/user/178701}{Mikesar}]
	My solution was wrong, thanks pco for noticing.
\end{solution}



\begin{solution}[by \href{https://artofproblemsolving.com/community/user/29428}{pco}]
	\begin{tcolorbox}...Hence there are no functions which satisfy the initial relation.\end{tcolorbox}

And what about $f(x)=\frac a{x+a}$ ?
\end{solution}
*******************************************************************************
-------------------------------------------------------------------------------

\begin{problem}[Posted by \href{https://artofproblemsolving.com/community/user/68025}{Pirkuliyev Rovsen}]
	Find all continuous function $f: \mathbb{R}\to\mathbb{R}$ such that $f(f(x))+f(x)=2x$.
	\flushright \href{https://artofproblemsolving.com/community/c6h580618}{(Link to AoPS)}
\end{problem}



\begin{solution}[by \href{https://artofproblemsolving.com/community/user/148231}{sqing}]
	\begin{tcolorbox}Find all continuous function $f: \mathbb{R}\to\mathbb{R}$ such that $f(f(x))+f(x)=2x$.\end{tcolorbox}
$f(x)=x$ is a solution.
\end{solution}



\begin{solution}[by \href{https://artofproblemsolving.com/community/user/29428}{pco}]
	\begin{tcolorbox}Find all continuous function $f: \mathbb{R}\to\mathbb{R}$ such that $f(f(x))+f(x)=2x$.\end{tcolorbox}
Two solutions :
$f(x)=x$
$f(x)=c-2x$

See http://www.artofproblemsolving.com/Forum/viewtopic.php?p=2721097#p2721097 subcase 7.1.2
\end{solution}



\begin{solution}[by \href{https://artofproblemsolving.com/community/user/148231}{sqing}]
	1.Find all continuous function $f: \mathbb{R}\to\mathbb{R}$ such that $f(x+y)=f(x)+f(y)$.
2.Find all continuous function $f: \mathbb{R}\to\mathbb{R}$ such that $f(\sqrt{x^2+y^2})=f(x)f(y)$.
\end{solution}



\begin{solution}[by \href{https://artofproblemsolving.com/community/user/29428}{pco}]
	\begin{tcolorbox}1.Find all continuous function $f: \mathbb{R}\to\mathbb{R}$ such that $f(x+y)=f(x)+f(y)$.
2.Find all continuous function $f: \mathbb{R}\to\mathbb{R}$ such that $f(\sqrt{x^2+y^2})=f(x)f(y)$.\end{tcolorbox}
Q1 : Cauchy. Basic course question. $f(x)=ax$

Q2 : see http://www.artofproblemsolving.com/Forum/viewtopic.php?f=38&t=580500 and add continuity to suppress S2 and limit the choice of $a(x)$ in S3 to $a(x)=cx$
\end{solution}
*******************************************************************************
-------------------------------------------------------------------------------

\begin{problem}[Posted by \href{https://artofproblemsolving.com/community/user/201543}{Txttn_128}]
	Find all functions  $ f:\mathbb{R}\to\mathbb{R} $  that satisfy the conditions
$ f(0)=\frac{1}{2} $ and  there exist $ a\in\mathbb{R} $ such that $ f(x+y) = f(x)f(a-y) + f(y)f(a-x),
 \quad\forall x,y\in\mathbb{R} $
	\flushright \href{https://artofproblemsolving.com/community/c6h580813}{(Link to AoPS)}
\end{problem}



\begin{solution}[by \href{https://artofproblemsolving.com/community/user/29428}{pco}]
	\begin{tcolorbox}Find all functions  $ f:\mathbb{R}\to\mathbb{R} $  that satisfy the conditions
$ f(0)=\frac{1}{2} $ and  there exist $ a\in\mathbb{R} $ such that $ f(x+y) = f(x)f(a-y) + f(y)f(a-x),
 \quad\forall x,y\in\mathbb{R} $\end{tcolorbox}
Let $P(x,y)$ be the assertion $f(x+y)=f(x)f(a-y)+f(y)f(a-x)$

$P(0,0)$ $\implies$ $f(a)=\frac 12$
$P(x,0)$ $\implies$ $f(a-x)=f(x)$ and so $P(x,y)$ implies new assertion $Q(x,y)$ : $f(x+y)=2f(x)f(y)$

If $f(u)=0$ for some $u$, $Q(u,-u)$ implies contradiction and so $f(x)\ne 0$ $\forall x$

$Q(\frac x2,\frac x2)$ $\implies$ $f(x)>0$ $\forall x$

Let then $g(x)=\ln (2f(x))$ and $Q(x,y)$ implies $g(x+y)=g(x)+g(y)$ and so $g(x)$ is an additive function.

Setting then $f(x)=\frac 12 e^{g(x)}$ where $g(x)$ is an additive function, we get $g(a-x)=g(x)$ and so $g(x)=\frac 12g(a)$ and so $g(x)=0$ $\forall x$

Hence the unique solution $\boxed{f(x)=\frac 12}$ $\forall x$
\end{solution}
*******************************************************************************
-------------------------------------------------------------------------------

\begin{problem}[Posted by \href{https://artofproblemsolving.com/community/user/125553}{lehungvietbao}]
	Find all functions $f:[-1;1] \to [-1;1]$ such that \[f(f(x))f(f(f(x))) = 6f(x)f(f(x)) - 8xf(x) \quad \forall x \in[-1;1]\]
	\flushright \href{https://artofproblemsolving.com/community/c6h581254}{(Link to AoPS)}
\end{problem}



\begin{solution}[by \href{https://artofproblemsolving.com/community/user/29428}{pco}]
	\begin{tcolorbox}Find all functions $f:[-1;1] \to [-1;1]$ such that \[f(f(x))f(f(f(x))) = 6f(x)f(f(x)) - 8xf(x) \quad \forall x \in[-1;1]\]\end{tcolorbox}
Let $x\in[-1,1]$. Let $a_n=f^{[n]}(x)f^{[n+1]}(x)$

Then :
$a_0=xf(x)$
$a_1=f(x)f(f(x))$
$a_2=f(f(x))f(f(f(x)))$
$a_{n+2}=6a_{n+1}-8a_n$

And so $a_n=\alpha 2^n+\beta 4^n$ for some $\alpha,\beta$ depending on $x$

$|a_n|\le 1$ $\forall n$ implies $\alpha=\beta=0$ and so $a_0=xf(x)=0$ $\forall x\in[-1,1]$ and so :

$\boxed{f(x)=0\text{  }\forall x\in[-1,1]\setminus\{0\}\text{  and  }f(0)=a}$ which indeed is a solution, whatever is $a\in[-1,1]$
\end{solution}
*******************************************************************************
-------------------------------------------------------------------------------

\begin{problem}[Posted by \href{https://artofproblemsolving.com/community/user/125553}{lehungvietbao}]
	Find  all functions $f: [m,+\infty ) \to  [m,+\infty )$  such that $f(x^{2}+f(y) ) =y+f^{2}(x)\quad \forall x, y \leq m$ and $x,y$ are integer numbers.
( Here $m$  is a given real number ) .
	\flushright \href{https://artofproblemsolving.com/community/c6h581257}{(Link to AoPS)}
\end{problem}



\begin{solution}[by \href{https://artofproblemsolving.com/community/user/29428}{pco}]
	\begin{tcolorbox}Find  all functions $f: [m,+\infty ) \to  [m,+\infty )$  such that $f(x^{2}+f(y) ) =y+f^{2}(x)\quad \forall x, y \leq m$ and $x,y$ are integer numbers.
( Here $m$  is a given real number ) .\end{tcolorbox}
$x,y\le m$ and domain  $=[m,+\infty)$ implies $x=y=m$

If $m$ is not an integer, any function is solution
If $m$ is an integer, then the only constraint is $f(m^2+f(m))=m+f(m)^2$
\end{solution}
*******************************************************************************
-------------------------------------------------------------------------------

\begin{problem}[Posted by \href{https://artofproblemsolving.com/community/user/68025}{Pirkuliyev Rovsen}]
	Find all function $f: \mathbb{R}\to\mathbb{R}$ such that $xf(y)+yf(x)=(x+y)f(x)f(y)$.
	\flushright \href{https://artofproblemsolving.com/community/c6h581275}{(Link to AoPS)}
\end{problem}



\begin{solution}[by \href{https://artofproblemsolving.com/community/user/29428}{pco}]
	\begin{tcolorbox}Find all function $f: \mathbb{R}\to\mathbb{R}$ such that $xf(y)+yf(x)=(x+y)f(x)f(y)$.\end{tcolorbox}
Let $P(x,y)$ be the assertion $xf(y)+yf(x)=(x+y)f(x)f(y)$

If $f(u)=0$ for some $u\ne 0$, then $P(x,u)$ $\implies$ $f(x)=0$ $\forall x$ and so :
$\boxed{\text{S1 :}f(x)=0\text{  }\forall x}$

If $f(x)\ne 0$ $\forall x\ne 0$, then $P(x,x)$ $\implies$ $f(x)=1$ $\forall x\ne 0$ and so :
$\boxed{\text{S2 :}f(x)=1\text{  }\forall x\ne 0\text{ and }f(0)=a}$ which indeed is a solution, whatever is $a\in\mathbb R$
\end{solution}
*******************************************************************************
-------------------------------------------------------------------------------

\begin{problem}[Posted by \href{https://artofproblemsolving.com/community/user/125553}{lehungvietbao}]
	For which $a,b,c,p,q,r $ is there a continuous function $f:\mathbb R\to\mathbb R$ satisfying \[f(ax+by+c)=pf(x)+qf(y)+r\] and what is the general form of the solution ?
	\flushright \href{https://artofproblemsolving.com/community/c6h581440}{(Link to AoPS)}
\end{problem}



\begin{solution}[by \href{https://artofproblemsolving.com/community/user/29428}{pco}]
	\begin{tcolorbox}For which $a,b,c,p,q,r $ is there a continuous function $f:\mathbb R\to\mathbb R$ satisfying \[f(ax+by+c)=pf(x)+qf(y)+r\] and what is the general form of the solution ?\end{tcolorbox}
What a strange quite uninteresting olympiad exercise !
This looks like a school course your teacher does not want to build himself :(

There are many many many different subcases (depending on situations where $a=0$, or $b=0$ with or without $p=0$ or $q=0$, plus the cases where $a+b=1$ with or without $p+q=1$, and so on ...)
Each case is quite elementary and classical but the sum of all cases makes it a boring long long long post.

I hope you are not trying to get a "general solution" for three specific exercises you got :(
\end{solution}



\begin{solution}[by \href{https://artofproblemsolving.com/community/user/125553}{lehungvietbao}]
	I hope you can help me  like you could http://www.artofproblemsolving.com/Forum/viewtopic.php?f=36&t=485367&p=2719214 . Thank for efforts .
\end{solution}



\begin{solution}[by \href{https://artofproblemsolving.com/community/user/29428}{pco}]
	\begin{tcolorbox}I hope you can help me  like you could http://www.artofproblemsolving.com/Forum/viewtopic.php?f=36&t=485367&p=2719214 . Thank for efforts .\end{tcolorbox}
I wont.
That's not a serious problem and you can solve each individual case alone obviously.
\end{solution}
*******************************************************************************
-------------------------------------------------------------------------------

\begin{problem}[Posted by \href{https://artofproblemsolving.com/community/user/179088}{Panoz93}]
	Find all  functions $f$,$ g:\mathbb{R}\rightarrow\mathbb{R} $  that satisfy
 
                      $g(f(x+y))=f(x)+(2x+y)g(y)$
	\flushright \href{https://artofproblemsolving.com/community/c6h581665}{(Link to AoPS)}
\end{problem}



\begin{solution}[by \href{https://artofproblemsolving.com/community/user/29428}{pco}]
	\begin{tcolorbox}Find all  functions $f$,$ g:\mathbb{R}\rightarrow\mathbb{R} $  that satisfy
 
                      $g(f(x+y))=f(x)+(2x+y)g(y)$\end{tcolorbox}
Let $P(x,y)$ be the assertion $g(f(x+y))=f(x)+(2x+y)g(y)$
Let $a=g(1)-g(0)$
Let $b=g(0)$

e1 : $P(x,1)$ $\implies$ $g(f(x+1))=f(x)+(2x+1)(a+b)$
e2 : $P(x,0)$ $\implies$ $g(f(x))=f(x)+2bx$
e3 : $P(0,x)$ $\implies$ $g(f(x))=f(0)+xg(x)$
e4 : $P(1,x)$ $\implies$ $g(f(1+x))=f(1)+(x+2)g(x)$
e5 : $P(1,0)$ $\implies$ $g(f(1))=f(1)+2b$
e6 : $P(0,1)$ $\implies$ $g(f(1))=f(0)+a+b$

e1-e2+e3-e4+e5-e6 :  $g(x)=ax+b$
Then $P(x,-x)$ implies $f(x)=ax^2-bx+af(0)+b$

Plugging this back in original equation, we get :

$\boxed{\text{S1 : }f(x)=g(x)=0\text{   }\forall x}$

$\boxed{\text{S2 : }f(x)=x^2+a\text{  and  }g(x)=x\text{   }\forall x}$
\end{solution}
*******************************************************************************
-------------------------------------------------------------------------------

\begin{problem}[Posted by \href{https://artofproblemsolving.com/community/user/186076}{John532}]
	Find all $f:\mathbb{Z} \rightarrow \mathbb{Z} $ such that

\[f(m+f(f(n)))=-f(f(m+1))-n\]
	\flushright \href{https://artofproblemsolving.com/community/c6h581716}{(Link to AoPS)}
\end{problem}



\begin{solution}[by \href{https://artofproblemsolving.com/community/user/187896}{Ashutoshmaths}]
	\begin{italicized}Solution\end{italicized}:
Let $P(m,n)$ be the assertion.
$f(m+f(f(n)))=-f(f(m+1))-n\forall m,n\in\mathbb{Z}$
$f(m+f(f(n')))=-f(f(m+1))-n'\forall m,n'\in\mathbb{Z}$
If $f(n)=f(n')\implies n=n'$
Hence the function is injective.
$P(0,n)\implies f(f(f(n)))=-f(f(1))-n$
$\implies f(f(f(n)))+n=-f(f(1))\cdots\cdots\star$
$P(m,f(n))\implies f(m+f(f(f(n))))=-f(f(m+1))-f(n)$
Using $\star$,we see that
$f(m-n-f(f(1)))=-f(f(m+1))-f(n)$
$P(f(f(1)),n)\implies f(-n)=-f(f(f(f(1))+1))-f(n)$
Denote $-f(f(f(f(1))+1))$ by $c$.
Hence $f(-n)+f(n)=c$
The original equation was
$f(m+f(f(n)))=-f(f(m+1))-n\forall m,n\in\mathbb{Z}$
It can be written as
$f(m+f(f(n)))=f(-f(m+1))-c-n\forall m,n\in\mathbb{Z}\cdots\cdots\star\star$
$P(m,-c)$(in $\star\star$)$\implies f(m+f(f(-c)))=f(-f(m+1))$
Due to injectivity,
$m+f(f(-c))=-f(m+1)$
$\implies f(m+1)=-m-f(f(-c))$
$\implies f(m)=-m+(1-f(f(-c)))$
Let $(1-f(f(-c)))$ be denoted by $k$.
Hence $f(m)=-m+k$.
Plugging this is the original equation,we get $k$ to be $-1$
hence $f(m)=-(m+1)\forall m\in\mathbb{Z}$.$\blacksquare$
[hide="Remark"]My solution may be wrong as the given function is from integers to integers but I have used no property of integers in my solution.
If my solution is right then the function should be real valued.[\/hide]
pco,please check my solution.
\end{solution}



\begin{solution}[by \href{https://artofproblemsolving.com/community/user/29428}{pco}]
	\begin{tcolorbox}Find all $f:\mathbb{Z} \rightarrow \mathbb{Z} $ such that

\[f(m+f(f(n)))=-f(f(m+1))-n\]\end{tcolorbox}
See http://www.artofproblemsolving.com/Forum/viewtopic.php?f=37&t=580160
\end{solution}



\begin{solution}[by \href{https://artofproblemsolving.com/community/user/187896}{Ashutoshmaths}]
	But shouldn't the function be real valued?
Why is the integers to integers condition given?
\end{solution}



\begin{solution}[by \href{https://artofproblemsolving.com/community/user/29428}{pco}]
	\begin{tcolorbox}But shouldn't the function be real valued?
Why is the integers to integers condition given?\end{tcolorbox}
You're right. Tthe proof  doesn not use the integer characteristic. Problem (and solution) would be the same with statement telling "from $\mathbb R \to\mathbb R$"
\end{solution}



\begin{solution}[by \href{https://artofproblemsolving.com/community/user/47092}{utsab001}]
	\begin{tcolorbox}But shouldn't the function be real valued?
Why is the integers to integers condition given?\end{tcolorbox}
When you are using the fact that $P(m,n)$ be the assertion...you're using the countable nature of the integers.On reals it wont hold true..
\end{solution}



\begin{solution}[by \href{https://artofproblemsolving.com/community/user/29428}{pco}]
	\begin{tcolorbox}[quote="Ashutoshmaths"]But shouldn't the function be real valued?
Why is the integers to integers condition given?\end{tcolorbox}
When you are using the fact that $P(m,n)$ be the assertion...you're using the countable nature of the integers.On reals it wont hold true..\end{tcolorbox}
Huhhhh !
That's complete nonsense for me! I'm very sorry.

Could you explain a bit your argument ?
(if you are right, about 800 of my proofs in this forum will become suddenly false ...)
\end{solution}
*******************************************************************************
-------------------------------------------------------------------------------

\begin{problem}[Posted by \href{https://artofproblemsolving.com/community/user/196603}{nbh}]
	find all functions $f:\mathbb{R^+}\to\mathbb{R^+}$ such that \[x^2(f(x)+f(y))=(x+y)f(yf(x))\forall x,y\in\mathbb{R^+}\]
	\flushright \href{https://artofproblemsolving.com/community/c6h581727}{(Link to AoPS)}
\end{problem}



\begin{solution}[by \href{https://artofproblemsolving.com/community/user/29428}{pco}]
	\begin{tcolorbox}find all functions $f:\mathbb{R^+}\to\mathbb{R^+}$ such that \[x^2(f(x)+f(y))=(x+y)f(yf(x))\forall x,y\in\mathbb{R^+}\]\end{tcolorbox}
Let $P(x,y)$ be the assertion $x^2(f(x)+f(y))=(x+y)f(yf(x))$

Let $A=\{x>0$ such that $f(x)=x\}$
If $u,v\in A$, then $P(u,v)$ $\implies$ $f(uv)=u^2$ and $P(v,u)$ $\implies$ $f(uv)=v^2$ and so $u=v$ and $|A|\in\{0,1\}$

$P(x,x)$ $\implies$ $f(xf(x))=xf(x)$ and so $xf(x)\in A$ and so $|A|=1$ and $A=\{a\}$ for some $a>0$ and $xf(x)=a$ $\forall x$

Plugging back $f(x)=\frac ax$ in original equation, we get $a=1$ and so the unique solution $\boxed{f(x)=\frac 1x}$ $\forall x$
\end{solution}



\begin{solution}[by \href{https://artofproblemsolving.com/community/user/187896}{Ashutoshmaths}]
	A nice problem.
Let $P(x,y)$ be the assertion 
$P(x,x)\implies f(xf(x))=xf(x)$ hence $f(f(1))=f(1)$
Hence we observe that $f(1)$ has to be used.
$P(xf(x),f(1))\implies x^2f(x)^2=f(xf(x)f(1))$
$P(f(1),xf(x)) \implies f(1)^2=f(xf(x)f(1))$
Hence$f(x)=\frac{f(1)}x$
Plugging this in original equation, we get $f(1)=1$
Therefore, $f(x)=\frac {1}{x}\forall x\in\mathbb{R}^+$
\end{solution}
*******************************************************************************
-------------------------------------------------------------------------------

\begin{problem}[Posted by \href{https://artofproblemsolving.com/community/user/125553}{lehungvietbao}]
	Find all functions $f:{{\mathbb{Z}}^{+}}\to {{\mathbb{Z}}^{+}}$ such that
\[{{f}^{2}}(n)<nf(n+1)\le 2{{n}^{2}}f\left( \left[ \frac{n+1}{2} \right] \right)\] for all $n\in {{\mathbb{Z}}^{+}}$.
	\flushright \href{https://artofproblemsolving.com/community/c6h581741}{(Link to AoPS)}
\end{problem}



\begin{solution}[by \href{https://artofproblemsolving.com/community/user/29428}{pco}]
	\begin{tcolorbox}Find all functions $f:{{\mathbb{Z}}^{+}}\to {{\mathbb{Z}}^{+}}$ such that
\[{{f}^{2}}(n)<nf(n+1)\le 2{{n}^{2}}f\left( \left[ \frac{n+1}{2} \right] \right)\] for all $n\in {{\mathbb{Z}}^{+}}$.\end{tcolorbox}
Setting $n=1$, we get $f^2(1)<f(2)\le 2f(1)$ and so $f(1)=1$ and $f(2)=2$

Using these values and the inequality $f^2(n)<nf(n+1)$, simple induction implies $f(n)\ge n$ $\forall n$
Using these values and the inequality $f^2(n)<2n^2f\left(\left\lfloor\frac{n+1}2\right\rfloor\right)$, induction implies $f(n)\le 2n^2$

Let $f(n)=n+a_n$ where $a_n$ is a non negative integer:
$f(n)^2<nf(n+1)$ implies $a_{n+1}\ge 2a_n$ and so $a_{n+m}\ge 2^ma_n$ and so $f(n+m)\ge 2^m(f(n)-n)+n+m$

So $2(n+m)^2\ge 2^m(f(n)-n)+n+m$ and, setting $m\to+\infty$ : $f(n)-n=0$ and $\boxed{f(n)=n}$ $\forall n$, which indeed is a solution.
\end{solution}
*******************************************************************************
-------------------------------------------------------------------------------

\begin{problem}[Posted by \href{https://artofproblemsolving.com/community/user/187760}{Aguero}]
	Find $f:R\rightarrow R$ such that
$f(x+y^2+z)=f(f(x))+yf(x)+f(z)$ for $x,y,z \in  R$
	\flushright \href{https://artofproblemsolving.com/community/c6h581758}{(Link to AoPS)}
\end{problem}



\begin{solution}[by \href{https://artofproblemsolving.com/community/user/29428}{pco}]
	\begin{tcolorbox}Find $f:R\rightarrow R$ such that
$f(x+y^2+z)=f(f(x))+yf(x)+f(z)$ for $x,y,z \in  R$\end{tcolorbox}
$\boxed{f(x)=0}$ $\forall x$ is a solution. So let us from now look only for non allzero solutions.

Let $P(x,y,z)$ be the assertion $f(x+y^2+z)=f(f(x))+yf(x)+f(z)$
Let $u$ such that $f(u)\ne 0$
$P(u,y,0)$, varying $y$ over $\mathbb R$ shows that $f(x)$ is surjective.

$P(x,0,1)$ $\implies$ $f(x+1)=f(f(x))+f(1)$
$P(1,0,x)$ $\implies$ $f(1+x)=f(f(1))+f(x)$
And so, subtracting, $f(f(x))=f(x)+f(f(1))-f(1)$ and, since surjective, $f(x)=x+a$ for some $a$, which, unfortunately, is never a solution, whatever is $a\in\mathbb R$

So no other solution for this functional equation.
\end{solution}



\begin{solution}[by \href{https://artofproblemsolving.com/community/user/187760}{Aguero}]
	Please show that f be subjective.
I'm stupid boy 555.
\end{solution}



\begin{solution}[by \href{https://artofproblemsolving.com/community/user/29428}{pco}]
	\begin{tcolorbox}Please show that f be subjective.\end{tcolorbox}
$P(u,\frac{x-f(0)-f(f(u))}{f(u)},0)$ $\implies$ $f\left(u+\left(\frac{x-f(0)-f(f(u))}{f(u)}\right)^2\right)=x$
\end{solution}
*******************************************************************************
-------------------------------------------------------------------------------

\begin{problem}[Posted by \href{https://artofproblemsolving.com/community/user/187760}{Aguero}]
	Find all $f:R \rightarrow R$ such that
$f(f(m)+f(n))=m+n$
	\flushright \href{https://artofproblemsolving.com/community/c6h581790}{(Link to AoPS)}
\end{problem}



\begin{solution}[by \href{https://artofproblemsolving.com/community/user/29428}{pco}]
	\begin{tcolorbox}Find all $f:\mathbb R \to \mathbb R$ such that
\[f(f(m)+f(n))=m+n\]\end{tcolorbox}
Let $P(x,y)$ be the assertion $f(f(x)+f(y))=x+y$
$f(x)$ is bijective

Let $u$ such that $f(u)=0$ : $P(x,u)$ $\implies$ $f(f(x))=x+u$

Let $g(x)=f(x-2u)$ : $P(f(x-2u),f(y-2u))$ $\implies$ $g(x+y)=g(x)+g(y)$
So $f(x)=g(x)+2g(u)$ and so $g(u)=0$ and so $f(0)=0$ and (using $f(f(x))=x+u$) $u=0$

So $f(x)$ is any additive involutive function 

A general form for such functions is :

Let $A,B$ two supplementary sub-vectorspaces of the $\mathbb Q-$vectorspace $\mathbb R$
Let $a(x)$ from $\mathbb R\to A$ and $b(x)$ from $\mathbb R\to B$ the two projections of $x$ in $A,B$ (so that $x=a(x)+b(x)$)

Then $f(x)=a(x)-b(x)$

(Note that the only two continuous such functions are $f(x)=x$ and $f(x)=-x$)
\end{solution}
*******************************************************************************
-------------------------------------------------------------------------------

\begin{problem}[Posted by \href{https://artofproblemsolving.com/community/user/125553}{lehungvietbao}]
	Find all functions $f: \mathbb R^+ \to\mathbb R^+$ such that  for all $x > 0$ and for all  $ 0 < z < 1$ then \[ (1-z)f(x) = f\left( \dfrac{(1-z)f(xz)}{z}\right)\]
	\flushright \href{https://artofproblemsolving.com/community/c6h581862}{(Link to AoPS)}
\end{problem}



\begin{solution}[by \href{https://artofproblemsolving.com/community/user/29428}{pco}]
	\begin{tcolorbox}Find all functions $f: \mathbb R^+ \to\mathbb R^+$ such that  for all $x > 0$ and for all  $ 0 < z < 1$ then \[ (1-z)f(x) = f\left( \dfrac{(1-z)f(xz)}{z}\right)\]\end{tcolorbox}
Let $P(x,z)$ be the assertion $(1-z)f(x)=f\left(\frac{(1-z)f(xz)}{z}\right)$
Let $g(x)=\frac{f(x)}x$ from $\mathbb R^+\to\mathbb R^+$

Let $x,y>0$. $P(x+y,\frac y{x+y})$ $\implies$ new assertion $Q(x,y)$ : $g(x+y)=g(y)g(xg(y))$

If $g(u)>1$ for some $u>0$, then $Q(\frac u{g(u)-1},u)$ $\implies$ $g(u)=1$, impossible. And so $g(x)\le 1$ $\forall x$
As a consequence $Q(x,y)$ implies $g(x+y)\le g(y)$ $\forall x,y$ and so $g(x)$ is decreasing (maybe not strictly)
If $g(u)=1$ for some $u$, $Q(u,u)$ $\implies$ $g(2u)=1$ and so $g(2^n u)=1$ $\forall n\in\mathbb N\cup\{0\}$ and so, since decreasing, $g(x)=1$ $\forall x$
Hence a first solution $f(x)=x$ $\forall x$

Let us from now consider $g(x)<1$ $\forall x$ and so $g(x+y)<g(y)$ and so $g(x)$ is strictly decreasing and so injective.
$Q(\frac x{g(1)},1)$ $\implies$ $g(\frac x{g(1)}+1)=g(1)g(x)$
$Q(\frac 1{g(x)},x)$ $\implies$ $g(\frac 1{g(x)}+x)=g(x)g(1)$
And so $g(\frac x{g(1)}+1)=g(\frac 1{g(x)}+x)$ and, since injective, $\frac x{g(1)}+1=\frac 1{g(x)}+x$ and so $g(x)=\frac 1{ax+1}$ where $a=\frac {1-g(1)}{g(1)}>0$

And so $\boxed{f(x)=\frac x{ax+1}}$ $\forall x$, which indeed is a solution, whatever is $a\ge 0$ (the case $a=0$ being the first solution we got previously).
\end{solution}
*******************************************************************************
-------------------------------------------------------------------------------

\begin{problem}[Posted by \href{https://artofproblemsolving.com/community/user/187760}{Aguero}]
	Find all $f:R \rightarrow R$ such that
$f(f(x)+y)=2x+f(f(y)-x)$
	\flushright \href{https://artofproblemsolving.com/community/c6h581869}{(Link to AoPS)}
\end{problem}



\begin{solution}[by \href{https://artofproblemsolving.com/community/user/209597}{Element118}]
	Clearly, $f(x)=x+c$ works.
Clearly, f is surjective.

We aim to show f is injective, then setting $x=0$:
$f(f(0)+y)=f(f(y))$
$f(0)+y=f(y)$
$f(y)=y+c$

Suppose not, and $f(z)=f(w)=c$, where $z-w$ is non-zero.

Note $f(f(z)+y)=2z+f(f(y)-z)=2w+f(f(y)-w)=f(f(w)+y)$
If $f(y)=z+w$ (since f surjective)
$2w+f(z)=2z+f(w)$,
$2w=2z$
Contradiction.

Thus we are done.
\end{solution}



\begin{solution}[by \href{https://artofproblemsolving.com/community/user/31919}{tenniskidperson3}]
	"Clearly, $f$ is surjective" is not true.  It's not clear.  Clearly $f(u)-f(v)$ is surjective, but not clearly $f$ itself.
\end{solution}



\begin{solution}[by \href{https://artofproblemsolving.com/community/user/141363}{alibez}]
	\begin{tcolorbox}"Clearly, $f$ is surjective" is not true.  It's not clear.  Clearly $f(u)-f(v)$ is surjective, but not clearly $f$ itself.\end{tcolorbox}

$P(\frac{-x+f(0)}{2},-f(\frac{-x+f(0)}{2}))\Rightarrow x=f(f(-f(\frac{-x+f(0)}{2}))-\frac{x-f(0)}{2})$
\end{solution}



\begin{solution}[by \href{https://artofproblemsolving.com/community/user/29428}{pco}]
	\begin{tcolorbox}[quote="tenniskidperson3"]"Clearly, $f$ is surjective" is not true.  It's not clear.  Clearly $f(u)-f(v)$ is surjective, but not clearly $f$ itself.\end{tcolorbox}

$P(\frac{-x-f(0)}{2},-f(x))\Rightarrow x=f(f(-f(\frac{x+f(0)}{2}))-\frac{f(x)+f(0)}{2})$\end{tcolorbox}
Hum, according to me 

$P(\frac{-x-f(0)}2,-f(x))$ $\implies$ $f(f(\frac{-x-f(0)}2)-f(x))=-x-f(0)+f(f(-f(x))+\frac{x+f(0)}2)$

Which does not prove surjectivity ...
\end{solution}



\begin{solution}[by \href{https://artofproblemsolving.com/community/user/141363}{alibez}]
	\begin{tcolorbox}[quote="alibez"][quote="tenniskidperson3"]"Clearly, $f$ is surjective" is not true.  It's not clear.  Clearly $f(u)-f(v)$ is surjective, but not clearly $f$ itself.\end{tcolorbox}

$P(\frac{-x-f(0)}{2},-f(x))\Rightarrow x=f(f(-f(\frac{x+f(0)}{2}))-\frac{f(x)+f(0)}{2})$\end{tcolorbox}
Hum, according to me 

$P(\frac{-x-f(0)}2,-f(x))$ $\implies$ $f(f(\frac{-x-f(0)}2)-f(x))=-x-f(0)+f(f(-f(x))+\frac{x+f(0)}2)$

Which does not prove surjectivity ...\end{tcolorbox}


i edit it . is it true ? 
\end{solution}



\begin{solution}[by \href{https://artofproblemsolving.com/community/user/29428}{pco}]
	\begin{tcolorbox}i edit it . is it true ? \end{tcolorbox}

Ahhh yes, indeed!
\end{solution}
*******************************************************************************
-------------------------------------------------------------------------------

\begin{problem}[Posted by \href{https://artofproblemsolving.com/community/user/169208}{zhumazhenis}]
	$\mathbb{Q}$ is set of all rational numbers. Find all functions $f:\mathbb{Q}\times\mathbb{Q}\rightarrow\mathbb{Q}$ such that for all $x$, $y$, $z$ $\in\mathbb{Q}$  satisfy 
$f(x,y)+f(y,z)+f(z,x)=f(0,x+y+z)$
	\flushright \href{https://artofproblemsolving.com/community/c6h581896}{(Link to AoPS)}
\end{problem}



\begin{solution}[by \href{https://artofproblemsolving.com/community/user/29428}{pco}]
	\begin{tcolorbox}$\mathbb{Q}$ is set of all rational numbers. Find all functions $f:\mathbb{Q}\times\mathbb{Q}\rightarrow\mathbb{Q}$ such that for all $x$, $y$, $z$ $\in\mathbb{Q}$  satisfy 
$f(x,y)+f(y,z)+f(z,x)=f(0,x+y+z)$\end{tcolorbox}
Let $P(x,y,z)$ be the assertion $f(x,y)+f(y,z)+f(z,x)=f(0,x+y+z)$
Let $g(x)=f(0,x)$

$P(0,0)$ $\implies$ $g(0)=f(0,0)=0$
$P(x,0,0)$ $\implies$ $f(x,0)=0$
$P(x,y,0)$ $\implies$ $f(x,y)=g(x+y)-g(x)$

So $P(x,y,z)$ may be written as new assertion $Q(x,y,z)$ : $g(x+y+z)+g(x)+g(y)+g(z)=g(x+y)+g(y+z)+g(z+x)$

$Q((n+1)x,x,-x)$ $\implies$ $g((n+2)x)=2g((n+1)x)-g(nx)+g(x)+g(-x)$ and so :

$g(nx)=\frac{g(x)+g(-x)}2n^2+\frac{g(x)-g(-x)}2n$

Let then $h_1(x)=g(x)+g(-x)$. Previous equation implies $h_1(nx)=n^2h_1(x)$ and so $h_1(x)=h_1(1)x^2$ $\forall x\in\mathbb Q$
Let then $h_2(x)=g(x)-g(-x)$. Previois equation implies $h_2(nx)=nh_2(x)$ and so $h_2(x)=h_2(1)x$ $\forall x\in\mathbb Q$

And so $g(x)=ax^2+bx$ $\forall x\in\mathbb Q$ which indeed is a solution, whatever are $a,b\in\mathbb Q$

And so $\boxed{f(x,y)=ay^2+2axy+by}$ $\forall x,y$
\end{solution}
*******************************************************************************
-------------------------------------------------------------------------------

\begin{problem}[Posted by \href{https://artofproblemsolving.com/community/user/186076}{John532}]
	Find all $f:\mathbf{N} \rightarrow \mathbf{N} $ such that
$f(xy+x+y)=f(x)+f(y)+f(xy).$
	\flushright \href{https://artofproblemsolving.com/community/c6h581914}{(Link to AoPS)}
\end{problem}



\begin{solution}[by \href{https://artofproblemsolving.com/community/user/29428}{pco}]
	\begin{tcolorbox}Find all $f:\mathbf{N} \rightarrow \mathbf{N} $ such that
$f(xy+x+y)=f(x)+f(y)+f(xy).$\end{tcolorbox}
I suppose that $0\notin\mathbb N$, as usual in this forum when no precision is given.

Let $P(x,y)$ be the assertion $f(xy+x+y)=f(x)+f(y)+f(xy)$
Let $a=f(1)$

$P(x,1)$ $\implies$ $f(2x+1)=2f(x)+a$ and so $f(4x+3)=2f(2x+1)+a=4f(x)+3a$
$P(1,1)$ $\implies$ $f(3)=3a$
$P(x,3)$ $\implies$ $f(3x)=3f(x)$

$f(6x+3)=f(3(2x+1))=3f(2x+1)=6f(x)+3a$
$f(6x+3)=f(2(3x+1)+1)=2f(3x+1)+a$
And so $f(3x+1)=3f(x)+a$

$f(6x+1)=f(3(2x)+1)=3f(2x)+a$
$f(6x+1)=f(2(3x)+1)=2f(3x)+a=6f(x)+a$
And so $f(2x)=2f(x)$

From $f(2x+1)=2f(x)+a$ and $f(2x)=2f(x)$, we easily get $\boxed{f(x)=ax}$ $\forall x\in \mathbb N$, which indeed is a solution.
\end{solution}
*******************************************************************************
-------------------------------------------------------------------------------

\begin{problem}[Posted by \href{https://artofproblemsolving.com/community/user/173448}{BruceLee}]
	We call a function $ f:N->N $ almost identically if there is a $ g:N->N $ 
so that $ f(f(n))+g(f(n))=n $ for any n natural
a) if f is almost identically, prove that $ g(n)=n $ for all n natural
b) prove that the only function almost identically and monotonous is the identically function
	\flushright \href{https://artofproblemsolving.com/community/c6h582162}{(Link to AoPS)}
\end{problem}



\begin{solution}[by \href{https://artofproblemsolving.com/community/user/29428}{pco}]
	\begin{tcolorbox}We call a function $ f:N->N $ almost identically if there is a $ g:N->N $ 
so that $ f(f(n))+g(f(n))=n $ for any n natural
a) if f is almost identically, prove that $ g(n)=n $ for all n natural
b) prove that the only function almost identically and monotonous is the identically function\end{tcolorbox}
Set $n=1$ and $ f(f(1))+g(f(1))=1 $  is impossible since $LHS\ge 2$
So no such functions.
\end{solution}



\begin{solution}[by \href{https://artofproblemsolving.com/community/user/172163}{joybangla}]
	Hmmm [url=http://www.artofproblemsolving.com/Forum/viewtopic.php?p=3411865#p3411865]This[\/url] says that if such a set of function exists then $g\equiv 0$ and then $f$ is an involution.So whatever the OP has meant seems to be wrong.
\end{solution}
*******************************************************************************
-------------------------------------------------------------------------------

\begin{problem}[Posted by \href{https://artofproblemsolving.com/community/user/184652}{CanVQ}]
	Find all functions $f:\mathbb Z \to \mathbb Z$ such that \[f\big(2m+f(m)+f(m)\cdot f(n)\big)=n\cdot f(m)+m \quad \forall m,\,n \in \mathbb Z.\]
	\flushright \href{https://artofproblemsolving.com/community/c6h582347}{(Link to AoPS)}
\end{problem}



\begin{solution}[by \href{https://artofproblemsolving.com/community/user/29428}{pco}]
	\begin{tcolorbox}Find all functions $f:\mathbb Z \to \mathbb Z$ such that \[f\big(2m+f(m)+f(m)\cdot f(n)\big)=n\cdot f(m)+m,\quad\forall m,\,n \in \mathbb Z.\]\end{tcolorbox}
Let $P(x,y)$ be the assertion $f(2x+f(x)+f(x)f(y))=yf(x)+x$
Considering that $f(x)$ can not be identically zero, we immediately get that $f(x)$ is bijective.

If $f(0)=-1$, then $P(f(1),0)$ $\implies$ $f(2f(1))=f(1)$ and, since injective, $2f(1)=1$, impossible. So $f(0)\ne -1$

Let $a=f(0)$ ($a\ne -1$)
Let $u$ such that $f(u)=0$
Let $v$ such that $f(v)=-1$ ($v\ne 0$)
Let $w$ such that $f(w)=1$ ($w\ne u$)

$P(w,u-w)$ $\implies$ $f(2w+1+f(u-w))=u$
$P(u,0)$ $\implies$ $f(2u)=u$
And so, since injective : $2w+1+f(u-w)=2u$ and so $f(u-w)+1=2(u-w)$

$P(x,u-w)$ $\implies$ $f(2(x+(u-w)f(x)))=x+(u-w)f(x)$

$P(x+(u-w)f(x),v)$ $\implies$ $x+(u-w)f(x)=u$ and so $f(x)=\frac{u-x}{u-w}$

Plugging $f(x)=bx+a$ in original equation, we get $\boxed{f(x)=x-2}$
\end{solution}



\begin{solution}[by \href{https://artofproblemsolving.com/community/user/187896}{Ashutoshmaths}]
	\begin{tcolorbox}Find all functions $f:\mathbb Z \to \mathbb Z$ thỏa mãn \[f\big(2m+f(m)+f(m)\cdot f(n)\big)=n\cdot f(m)+m,\quad\forall m,\,n \in \mathbb Z.\]\end{tcolorbox}
My solution is completely different.
This problem is very hard(at least for me) :P
My solution:
Let $P(m,n)$ be the assertion.
First we will prove that there exist no constant functions.
If there exists such a function,
Let $f(n)=c\forall n\in \mathbb{Z}$
$\implies c=nc+m$ which is not possible as $m,n$ vary through integers.
Hence no constant functions exist.
$P(m,n')\implies f\big(2m+f(m)+f(m)\cdot f(n')\big)=n'\cdot f(m)+m$
Suppose $f(n)=f(n')$
$\implies nf(m)+m=n'f(m)+m$
As $f$ is non-constant, $\exists m$ such that $f(m)\neq 0$
$\implies f$ is injective.
$P(m,0)\implies f(2m+f(m)+f(0))=m$
Hence the function is surjective, hence bijective.
Hence, $\exists c$ such that $f(c)=-1$
$P(m,c)\implies f(2m)=cf(m)+m$
Taking $m=0$
$f(0)=cf(0)\implies f(0)(c-1)=0$
If $f(0)=0$
===============================================================
If $c-1=0\implies c=1\implies f(1)=-1$
$P(m,1)\implies f(2m)=f(m)+m$
Taking $m=1$ we get $f(2)=f(1)+1=-1+1=0$,hence $f(2)=0$
We had already proved
$P(1,n)\implies f(1-f(n))=1-n\cdots(\bigstar)$
$P(2,n)\implies f(4)=2$
In $(\bigstar)$, we substitute $n=4\implies f(1-f(4))=-3\implies f(-1)=-3$
Now it is easy to see that $f(n)=n-2\forall n\in\mathbb{Z}$
We can easily prove this claim using induction.
==============================================================
If $f(0)=0$
By surjectivity,
$\exists k$ such that $f(k)=-1$
$P\left(m,\frac{-m}{f(m)}\right)\implies f\left(2m+f(m)+f(m)f\left(\frac{-m}{f(m)}\right)\right)=0$
As $n$ is an integer, we are taking $n=\frac{-m}{f(m)}$, hence $f(m)$ must divide $m$ so I took $f(m)=-1$ as by surjectivity, there exists such $m$
$m=k\implies f(2k-1-f(k))=0\implies f(k)=2k-1\implies 2k-1=-1$
$\implies k=0$
Not possible
Only solutions are $f(x)=x-2\forall x\in\mathbb{Z}$
\end{solution}



\begin{solution}[by \href{https://artofproblemsolving.com/community/user/29428}{pco}]
	\begin{tcolorbox}...
Now it is easy to see that $f(n)=n-2\forall n\in\mathbb{Z}$
We can easily prove this claim using induction.
...
\end{tcolorbox}
Could you show us this "easy" induction ?

\begin{tcolorbox}.==============================================================
...
$P\left(m,\frac{-m}{f(m)}\right)\implies f\left(2m+f(m)+f(m)f\left(\frac{-m}{f(m)}\right)\right)=0$
...
\end{tcolorbox}
You can only write this for those $m$ such that $f(m)|m$
And not say that $f(m)|m$ since you wrote that !!!!
\end{solution}



\begin{solution}[by \href{https://artofproblemsolving.com/community/user/187896}{Ashutoshmaths}]
	Ok pco, I will complete it:
$ P\left(m,\frac{-m}{f(m)}\right)\implies f\left(2m+f(m)+f(m)f\left(\frac{-m}{f(m)}\right)\right)=0 $
Only for those $m$ for which $f(m)$ divides $m$.
Due to surjectivity, $\exists k$ such that $f(k)=-1$
And $\frac{k}{f(k)}\in\mathbb{Z}$ as $-1$ divides everything.
I got $k=0$This means $f(k)=-1\implies f(0)=-1$ but $f(0)=0$
But bijectivity implies one to one correspondence,hence there cannot exist an element in the codomain for which
there are two images in the domain.
---------------------------------------------------------------------
And I did not mean induction,(I typed that by a mistake)
I meant an algorithm:
We have $f(1)=-1$,
and $ f\big(2m+f(m)+f(m)\cdot f(n)\big)=n\cdot f(m)+m,\quad\forall m,\,n\in\mathbb{Z}$
I had found some values which suggested that $f(n)$ is possibly $n-2$.
I had thought that by finding out some values, one can guess an algorithm.
Wait, I will find it and post it..
Everything else is okay ,no?
\end{solution}



\begin{solution}[by \href{https://artofproblemsolving.com/community/user/187896}{Ashutoshmaths}]
	Please check the full solution:(Induction included here)
\begin{tcolorbox}[quote="CanVQ"]Find all functions $f:\mathbb Z \to \mathbb Z$ thỏa mãn \[f\big(2m+f(m)+f(m)\cdot f(n)\big)=n\cdot f(m)+m,\quad\forall m,\,n \in \mathbb Z.\]\end{tcolorbox}
Let $P(m,n)$ be the assertion.
First we will prove that there exist no constant functions.
If there exists such a function,
Let $f(n)=c\forall n\in \mathbb{Z}$
$\implies c=nc+m$ which is not possible as $m,n$ vary through integers.
Hence no constant functions exist.
$P(m,n')\implies f\big(2m+f(m)+f(m)\cdot f(n')\big)=n'\cdot f(m)+m$
Suppose $f(n)=f(n')$
$\implies nf(m)+m=n'f(m)+m$
As $f$ is non-constant, $\exists m$ such that $f(m)\neq 0$
$\implies f$ is injective.
$P(m,0)\implies f(2m+f(m)+f(0))=m$
Hence the function is surjective, hence bijective.
Hence, $\exists c$ such that $f(c)=-1$
$P(m,c)\implies f(2m)=cf(m)+m$
Taking $m=0$
$f(0)=cf(0)\implies f(0)(c-1)=0$
If $f(0)=0$
===============================================================
If $c-1=0\implies c=1\implies f(1)=-1$
$P(m,1)\implies f(2m)=f(m)+m$
Taking $m=1$ we get $f(2)=f(1)+1=-1+1=0$,hence $f(2)=0$
We had already proved
$P(1,n)\implies f(1-f(n))=1-n\cdots(\bigstar)$
$P(2,n)\implies f(4)=2$
In $(\bigstar)$, we substitute $n=4\implies f(1-f(4))=-3\implies f(-1)=-3$
Now it is easy to see that $f(n)=n-2\forall n\in\mathbb{Z}$
Here is the induction:
$f\big(2m+f(m)+f(m)\cdot f(n)\big)=n\cdot f(m)+m,\quad\forall m,\,n\in\mathbb{Z}$
We have $f(1)=-1$
Aim:To prove $f(n)=n-2$.
$P(m,1)\implies f(2m)=f(m)+m$
Hence $f(2(1+f(n)))=f(1+f(n))+(1+f(n))$
Say $f(n)=n-2$ for some $n$.
$P(1,n)\implies f(2-1-n+2)=-n+1\implies f(3-n)=-n+1$
$\implies f(3-(n+3))=-n-3+1\implies f(-n)=-n-2$
For the positive integers part, we do this:
$P(3,n)\implies f(6+1+1\cdot (n-2))=n+3$
$\implies f(n+5)=n+3$
This covers the positive integers part.$f\big(2m+f(m)+f(m)\cdot f(n)\big)=n\cdot f(m)+m,\quad\forall m,\,n\in\mathbb{Z}$
We have $f(1)=-1$
Aim:To prove $f(n)=n-2$.
$P(m,1)\implies f(2m)=f(m)+m$
Hence $f(2(1+f(n)))=f(1+f(n))+(1+f(n))$
Say $f(n)=n-2$ for some particular $n$
$P(1,n)\implies f(2-1-n+2)=-n+1\implies f(3-n)=-n+1$
$\implies f(3-(n+3))=-n-3+1\implies f(-n)=-n-2$
For the positive integers part, we do this:
$P(3,n)\implies f(6+1+1\cdot (n-2))=n+3$
$\implies f(n+5)=n+3$
This means if we know the value of $f(n)$,we can find $f(-n)$
As we have $\implies f(n+5)=n+3$
for any $n$ which has $f(n)=n-2$,
To complete the induction, we have to find $5$ continuous values...
Which we have already found out. $f(-1),f(0),f(1),f(2),f(3)$
$P(0,0)\implies f(f(0)^2+f(0))=0$ and we have already proved $f(2)=0$
Hence by injectivity, $f(0)^2+f(0)=2$ either $f(0)=-2$ or $f(0)=1$
$f(0)= 1$ can be easily disapproved giving $f(2)=0$,
With $f(0)$, it is easy to find $f(3)$
And this is how we know the 5 continuous values.
Hence it is proved for positive integers that $f(n)=n-2$,
And we have already proved that if we know $f(n)$, we can easily find $f(-n)$
Thus, we have proved $f(n)=n-2$ for all integers.
Induction complete
==============================================================
If $f(0)=0$
By surjectivity,
$\exists k$ such that $f(k)=-1$
$P\left(m,\frac{-m}{f(m)}\right)\implies f\left(2m+f(m)+f(m)f\left(\frac{-m}{f(m)}\right)\right)=0$
Only for those $m$ for which $f(m)$ divides $m$.
Due to surjectivity, $\exists k$ such that$ f(k)=-1$
And $\frac{k}{f(k)}\in\mathbb{Z}$ as $-1$ divides everything.
$m=k\implies f(2k-1-f(k))=0\implies f(k)=2k-1\implies 2k-1=-1$
$\implies k=0$
This means $f(k)=-1\implies f(0)=-1$ but $f(0)=0$
But bijectivity implies one to one correspondence,hence there cannot exist an element in the codomain for which
there are two images in the domain.
Only solutions are $f(x)=x-2\forall x\in\mathbb{Z}\Box$[\/hide]
\end{solution}
*******************************************************************************
-------------------------------------------------------------------------------

\begin{problem}[Posted by \href{https://artofproblemsolving.com/community/user/180735}{mmaht}]
	Find all functions $f:\mathbb{N}\rightarrow \mathbb{R}$ which satisfy the following conditions
(a) $f(x+22)=f(x)$ for all $x\in \mathbb{N}$;
(b) $f(x^2y)=f(f(x))^2 f(y)$ for all $x,y \in \mathbb{N}$.
	\flushright \href{https://artofproblemsolving.com/community/c6h582503}{(Link to AoPS)}
\end{problem}



\begin{solution}[by \href{https://artofproblemsolving.com/community/user/29428}{pco}]
	\begin{tcolorbox}Find all functions $f:\mathbb{N}\rightarrow \mathbb{R}$ which satisfy the following conditions
(a) $f(x+22)=f(x)$ for all $x\in \mathbb{N}$;
(b) $f(x^2y)=f(f(x))^2 f(y)$ for all $x,y \in \mathbb{N}$.\end{tcolorbox}
I considered, as usual in this forum, that without any precision $0\notin\mathbb N$

In order $f(f(x))$ be defined in b), we need $f(x)$ be from $\mathbb N\to\mathbb N$
If $f(f(x))>1$ for some $x$, then $b)$ implies $f(x)$ not upperbounded, impossible since periodic (and so $|f(\mathbb N)|\le 22$). 
So $f(f(x))=1$ $\forall x$ and $f(x^2y)=f(y)$ $\forall x,y$

So (using a) $f(22)=f(22^2x)$ and (using b) $f(22^2x)=f(x)$ and so $f(x)$ is constant and so, since $f(f(x))=1$ :

$\boxed{f(x)=1\text{   }\forall x}$
\end{solution}
*******************************************************************************
-------------------------------------------------------------------------------

\begin{problem}[Posted by \href{https://artofproblemsolving.com/community/user/186076}{John532}]
	Find all $f:\mathbf{Z} \rightarrow \mathbf{Z}$ such that
 $ f(m+n-mn)=f(m)+f(n)-f(mn).$
	\flushright \href{https://artofproblemsolving.com/community/c6h582533}{(Link to AoPS)}
\end{problem}



\begin{solution}[by \href{https://artofproblemsolving.com/community/user/29428}{pco}]
	\begin{tcolorbox}Find all $f:\mathbf{Z} \rightarrow \mathbf{Z}$ such that
 $ f(m+n-mn)=f(m)+f(n)-f(mn).$\end{tcolorbox}
We'll first consider the case where $f(x)$ is from $\mathbb Z\to\mathbb Q$ allowing the set of solutions to be a $\mathbb Q-$vectorspace.
We'll show then that this vectorspace has a dimension $\le 4$ and we'll exhibit four independant solutions, and so a basis.
From there, general solution will be immediate and coming back to the $\mathbb Z$-casis very easy.

Let $P(x,y)$ be the assertion $f(x+y-xy)=f(x)+f(y)-f(xy)$

1) The space of solutions is a $\mathbb Q-$vectorspace.
===================================
Trivial

2) knowledge of $f(-1),f(0),f(1),f(2),f(3)$ is enough to define $f(x)$ and so dimension is $\le 5$
==============================================================
Subtracting $P(2x-2,-1)$ from $P(2x,2)$, we get $f(4x)=f(4x-5)+f(2x)-f(2x-2)+f(2)-f(-1)$
Subtracting $P(2x-1,-1)$ from $P(2x+1,2)$, we get $f(4x+2)=f(4x-3)+f(2x+1)-f(2x-1)+f(2)-f(-1)$

$P(2x+1,-1)$ $\implies$ $f(4x+1)=f(2x+1)+f(-1)-f(-2x-1)$
$P(-2x-1,-1)$ $\implies$ $f(-4x-3)=f(-2x-1)+f(-1)-f(2x+1)$
$P(-x,-3)$ $\implies$ $f(-4x-3)=f(-x)+f(-3)-f(3x)$
$P(x,-1)$ $\implies$ $f(2x-1)=f(x)+f(-1)-f(-x)$
$P(-1,-1)$ $\implies$ $f(-3)=2f(-1)-f(1)$
Adding the two first lines and subtracting the three last, we get : $f(4x+1)=f(3x)+f(2x-1)-f(x)+f(1)-f(-1)$

$P(2x+2,-1)$ $\implies$ $f(4x+3)=f(2x+2)+f(-1)-f(-2x-2)$
$P(x+1,-1)$ $\implies$ $f(2x+1)=f(x+1)+f(-1)-f(-x-1)$
$P(-x-1,2)$ $\implies$ $f(x+3)=f(-x-1)+f(2)-f(-2x-2)$
Subtracting the two last lines from the first, we get : $f(4x+3)=f(2x+2)+f(2x+1)+f(x+3)-f(x+1)-f(2)$

And so we got four equations :
$Q_0(x)$ : $f(4x)=f(4x-5)+f(2x)-f(2x-2)+f(2)-f(-1)$
$Q_1(x)$ : $f(4x+1)=f(3x)+f(2x-1)-f(x)+f(1)-f(-1)$
$Q_2(x)$ : $f(4x+2)=f(4x-3)+f(2x+1)-f(2x-1)+f(2)-f(-1)$
$Q_3(x)$ : $f(4x+3)=f(2x+2)+f(2x+1)+f(x+3)-f(x+1)-f(2)$
Which allow, varying $x$ from $1\to+\infty$ to determine $f(x)$ $\forall x\ge 4$ wth knowledge of $f(-1),f(0),f(1),f(2),f(3)$

Then $P(x,-1)$ $\implies$ $f(-x)=-f(2x-1)+f(x)+f(-1)$ and so we get also knowledge of $f(x)$ $\forall x\le -2$
Q.E.D.

3) Dimension is $\le 4$
==============
(a) $P(-2,3)$ $\implies$ $f(7)=f(-2)+f(3)-f(-6)$
(b) $P(2,-1)$ $\implies$ $f(3)=f(2)+f(-1)-f(-2)$
(c) $P(6,-1)$ $\implies$ $f(11)=f(6)+f(-1)-f(-6)$
(d) $Q_3(2)$ $\implies$ $f(11)=f(6)+2f(5)-f(3)-f(2)$
(e) $Q_3(1)$ $\implies$ $f(7)=2f(4)+f(3)-2f(2)$
(f) $Q_1(1)$ $\implies$ $f(5)=f(3)+f(1)-f(-1)$
(g) $Q_0(1)$ $\implies$ $f(4)=2f(2)-f(0)$

a+b-c+d-e+2f-2g gives then  $f(-1)=-f(2)+f(1)+f(0)$ and so knowledge of $f(0),f(1),f(2),f(3)$ is enough to fully determine $f(x)$
Q.E.D.

4) Dimension is $4$
=============
We can (rather) easily exhibit four solutions :

$f_1(x)=x$ $\forall x$ is a solution 

$f_2(x)=\left\lceil\frac x2\right\rceil$ $\forall x$ is a solution

$f_3(x)=\left\lceil\frac x3\right\rceil$ $\forall x$ is a solution

$f_4(x)=1$ $\forall x$ is a solution

And it's immediate to check that these four solutions are independant
Q.E.D

5) General solution
=============
Considering the previous basis, we got a general solution of the $\mathbb Q$-problem : 
$f(x)=ax$ $+b\left\lceil\frac x2\right\rceil$ $+c\left\lceil\frac x3\right\rceil$ $+d$ where $a,b,c,d\in\mathbb Q$

And it's easy (but not automatic, it does not work with all basis) to get that $f(x)\in\mathbb Z$ $\forall x$ $\iff$ $a,b,c,d\in\mathbb Z$

Hence the result :

$\boxed{f(x)=ax+b\left\lceil\frac x2\right\rceil+c\left\lceil\frac x3\right\rceil+d}$ $\forall x$, whatever are $a,b,c,d\in\mathbb Z$
\end{solution}
*******************************************************************************
-------------------------------------------------------------------------------

\begin{problem}[Posted by \href{https://artofproblemsolving.com/community/user/187760}{Aguero}]
	Find all $f:R\rightarrow R$ such that
$f(xf(x+y)) = f(yf(x)) + x^2$ For all real $x,y.$
	\flushright \href{https://artofproblemsolving.com/community/c6h583851}{(Link to AoPS)}
\end{problem}



\begin{solution}[by \href{https://artofproblemsolving.com/community/user/29428}{pco}]
	\begin{tcolorbox}Find all $f:R\rightarrow R$ such that
$f(xf(x+y)) = f(yf(x)) + x^2$ For all real $x,y.$\end{tcolorbox}
Let $P(x,y)$ be the assertion $f(xf(x+y))=f(yf(x))+x^2$

If $f(0)\ne 0$, then $P(0,\frac{x}{f(0)})$ $\implies$ $f(x)=f(0)$ $\forall x$, which is not a solution. So $f(0)=0$

1) $f(x)$ is injective
============
If $f(x)=0$ for some $x$, then $P(x,0)$ $\implies$ $x=0$
If $f(u)=f(v)=0$ for some $u,v$, then previous line implies $u=v=0$
If $f(u)=f(v)\ne 0$, then :
$P(u,0)$ $\implies$ $f(uf(u))=u^2$
$P(u,v-u)$ $\implies$ $f(uf(v))=f((v-u)f(u))+u^2$ and so $f((v-u)f(u))=0$ and so $u=v$
Q.E.D.

2) $f(2x)=2f(x)$
==========
$P(x\sqrt 2,0)$ $\implies$ $f(x\sqrt 2f(x\sqrt 2))=2x^2$
$P(x,0)$ $\implies$ $f(xf(x))=x^2$
$P(x,x)$ $\implies$ $f(xf(2x))=f(xf(x))+x^2=2x^2$ $=f(x\sqrt 2f(x\sqrt 2))$ and so, since injective, $xf(2x)=x\sqrt 2f(x\sqrt 2)$
And so $f(x\sqrt 2)=\sqrt 2f(x)$ $\forall x\ne 0$, still true when $x=0$
Q.E.D.

3) $f(x)=f(1)x$ $\forall x$ 
================
a: $P(x,1-x)$ $\implies$ $f(xf(1))=f((1-x)f(x))+x^2$
b: $P(2x,1-x)$ $\implies$ $f(xf(x+1))=f((1-x)f(x))+2x^2$
c: $P(x,1)$ $\implies$ $f(xf(x+1))=f(f(x))+x^2$
a-b+c : $f(xf(1))=f(f(x))$

And so, since injective, $f(x)=xf(1)$ $\forall x$
Q.E.D.

4) Solutions
========
Plugging $f(x)=xf(1)$ back in original equation, we get $f(1)^2=1$ and so the two solutions :
$\boxed{\text{S1: }f(x)=x\text{  }\forall x}$

$\boxed{\text{S2: }f(x)=-x\text{  }\forall x}$
\end{solution}
*******************************************************************************
-------------------------------------------------------------------------------

\begin{problem}[Posted by \href{https://artofproblemsolving.com/community/user/186076}{John532}]
	Find all $f:R \rightarrow R$ such that

$f(x)f(yf(x)-1)=x^2f(y)-f(x)$ for all $x,y \in R$.
	\flushright \href{https://artofproblemsolving.com/community/c6h584016}{(Link to AoPS)}
\end{problem}



\begin{solution}[by \href{https://artofproblemsolving.com/community/user/29428}{pco}]
	\begin{tcolorbox}Find all $f:R \rightarrow R$ such that

$f(x)f(yf(x)-1)=x^2f(y)-f(x)$ for all $x,y \in R$.\end{tcolorbox}
Let $P(x,y)$ be the assertion $f(x)f(yf(x)-1)=x^2f(y)-f(x)$

$P(1,1)$ $\implies$ $f(1)f(f(1)-1)=0$ and so $\exists u$ such that $f(u)=0$

If $u\ne 0$, then $P(u,x)$ $\implies$ $\boxed{S1: f(x)=0\text{  }\forall x}$ which indeed is a solution

So let us from now consider $f(x)=0$ $\iff$ $x=0$
$P(1,1)$ $\implies$ $f(1)=1$
$P(-1,-1)$ $\implies$ $f(-1)=-1$
$P(1,x)$ $\implies$ $f(x-1)=f(x)-1$ and so $P(x,y)$ may be written as new assertion $Q(x,y)$ : $f(x)f(yf(x))=x^2f(y)$
$P(-1,-x)$ $\implies$ $f(-x)=-f(x)$ and the function is odd.

Let $x\ne 0$ : $Q(x,x)$ $\implies$ $f(xf(x))=x^2$ still true if $x=0$ and so, since odd, $f(x)$ is surjective 
$Q(x,1)$ $\implies$ $f(x)f(f(x))=x^2$ and so, since odd, $f(x)$ is injective, and so bijective.

$Q(x,1)$ $\implies$ $x^2=f(x)f(f(x))$ and qo $Q(x,y)$ implies $f(x)f(yf(x))=x^2f(y)=f(x)f(y)f(f(x))$ and so $f(yf(x))=f(y)f(f(x))$ $\forall x\ne 0$, still true when $x=0$

And  so, since surjective : $f(xy)=f(x)f(y)$ and $f(x)$ is multiplicative.

So $f(x)$ is an odd multiplicative bijection such that $f(x+1)=f(x)+1$
It's immediate then to get that $f(x)$ is additive
And then (classical result), since bijective, additive and multiplicative, we get $\boxed{S2: f(x)=x\text{  }\forall x}$ which indeed is a solution
\end{solution}
*******************************************************************************
-------------------------------------------------------------------------------

\begin{problem}[Posted by \href{https://artofproblemsolving.com/community/user/186076}{John532}]
	Find all $f:R \rightarrow R $ such that
$f(f(x)-y)+f(x+y)=2x.$
	\flushright \href{https://artofproblemsolving.com/community/c6h584184}{(Link to AoPS)}
\end{problem}



\begin{solution}[by \href{https://artofproblemsolving.com/community/user/29428}{pco}]
	\begin{tcolorbox}Find all $f:R \rightarrow R $ such that
$f(f(x)-y)+f(x+y)=2x.$\end{tcolorbox}
Let $P(x,y)$ be the assertion $f(f(x)-y)+f(x+y)=2x$
Let $g(x)=f(x)+x$
$P(x,y-x)$ becomes then new assertion $Q(x,y)$ : $g(g(x)-y)=g(x)-g(y)+2x$

$Q(x,0)$ $\implies$ $g(g(x))=g(x)-g(0)+2x$ and so $g(x)$ is injective.
$Q(x,x)$ $\implies$ $g(g(x)-x)=2x$ and so $g(x)$ is surjective

Let then $u$ such that $g(u)=0$

a: $Q(x,-g(y))$ $\implies$ $g(g(x)+g(y))=g(x)-g(-g(y))+2x$
b: $Q(x,0)$ $\implies$ $g(g(x))=g(x)-g(0)+2x$
c: $Q(u,g(y))$ $\implies$ $g(-g(y))=-g(g(y))+2u$
a-b-c : $g(g(x)+g(y))=g(g(x))+g(g(y))-2u-g(0)$

And so, since surjective : $g(x+y)=g(x)+g(y)-2u-g(0)$
Setting then $x=y=0$, we get $u=0$ and so $g(0)=0$ and $g(x+y)=g(x)+g(y)$

So problem is equivalent (look at $Q(x,y)$) to 
$g(x)$ is additive and such that $g(g(x))=g(x)+2x$

From there, it's easy to get the general solution :
Let $A,B$ two supplementary sub-vectorspaces of the $\mathbb Q$-vectordpace $\mathbb R$
Let $a(x)$ from $\mathbb R\to A$ and $b(x)$ from $\mathbb R\to B$ the two projections of $x$ so that $x=a(x)+b(x)$ in a unique way $\forall x$

Then $g(x)=2a(x)-b(x)$ and so $f(x)=a(x)-2b(x)$

Note this gives infinitely many solutions.
The only two continuous solutions are the trivial $f(x)=x$ and $f(x)=-2x$
\end{solution}



\begin{solution}[by \href{https://artofproblemsolving.com/community/user/207556}{elmrini}]
	\begin{tcolorbox}Find all $f:R \rightarrow R $ such that
$f(f(x)-y)+f(x+y)=2x.$\end{tcolorbox}
I don't have the full solution (i am a beginner in ef) but I find some results 
$P\left ( x,\frac{f(x)-x}{2} \right )\Rightarrow f\left ( \frac{x+f(x)}{2} \right )=x$ then $f$ surjective.
and we have $P(x,0) \Rightarrow f(f(x))+f(x)=2x$ then $f(x)=f(y) \Rightarrow f(f(x))+f(x)=f(f(y))+f(y)  \Rightarrow x=y  \Rightarrow f$ injective.
therfore $f$ is a bijective function , so there is only one real $c$ such that $f(c)=0$ 
we have $P(0,f(0)+c)\Rightarrow f(f(0)+c)=0\Rightarrow f(0)+c=c\Rightarrow f(0)=0 \Rightarrow c=0$
$P(x,-x)\Rightarrow f(x+f(x))=2x$ then $P(x+f(x),0) \Rightarrow f(2x)+2x=2x+2f(x)\Rightarrow f(2x)=2f(x) $
and $P(0,x) \Rightarrow f(-x)=-f(x) \Rightarrow f$ is an odd function. (maybe can help  :maybe: )
\end{solution}



\begin{solution}[by \href{https://artofproblemsolving.com/community/user/144840}{kakkokari}]
	Useless post, please delete it.
\end{solution}



\begin{solution}[by \href{https://artofproblemsolving.com/community/user/144840}{kakkokari}]
	\begin{tcolorbox}
$P\left ( x,\frac{x-f(x)}{2} \right )\Rightarrow f\left ( \frac{3x-f(x)}{2} \right )=x$ then $f$ surjective.
\end{tcolorbox}

This is not correct
It should be $P\left ( x,\frac{x-f(x)}{2} \right )\Rightarrow  f\left ( \frac{3f(x)-x}{2} \right )+f\left ( \frac{3x-f(x)}{2} \right )=2x$
\end{solution}



\begin{solution}[by \href{https://artofproblemsolving.com/community/user/207556}{elmrini}]
	\begin{tcolorbox}[quote="elmrini"]
$P\left ( x,\frac{x-f(x)}{2} \right )\Rightarrow f\left ( \frac{3x-f(x)}{2} \right )=x$ then $f$ surjective.
\end{tcolorbox}

This is not correct
It should be $P\left ( x,\frac{x-f(x)}{2} \right )\Rightarrow  f\left ( \frac{3f(x)-x}{2} \right )+f\left ( \frac{3x-f(x)}{2} \right )=2x$\end{tcolorbox}
sorry I want to say $P\left ( x,\frac{f(x)-x}{2} \right )\Rightarrow f\left ( \frac{x+f(x)}{2} \right )=x$
\end{solution}



\begin{solution}[by \href{https://artofproblemsolving.com/community/user/141363}{alibez}]
	\begin{tcolorbox}[quote="John532"]Find all $f:R \rightarrow R $ such that
$f(f(x)-y)+f(x+y)=2x.$\end{tcolorbox}
Let $P(x,y)$ be the assertion $f(f(x)-y)+f(x+y)=2x$
Let $g(x)=f(x)+x$
$P(x,y-x)$ becomes then new assertion $Q(x,y)$ : $g(g(x)-y)=g(x)-g(y)+2x$

$Q(x,0)$ $\implies$ $g(g(x))=g(x)-g(0)+2x$ and so $g(x)$ is injective.
$Q(x,x)$ $\implies$ $g(g(x)-x)=2x$ and so $g(x)$ is surjective

Let then $u$ such that $g(u)=0$

a: $Q(x,-g(y))$ $\implies$ $g(g(x)+g(y))=g(x)-g(-g(y))+2x$
b: $Q(x,0)$ $\implies$ $g(g(x))=g(x)-g(0)+2x$
c: $Q(u,g(y))$ $\implies$ $g(-g(y))=-g(g(y))+2u$
a-b-c : $g(g(x)+g(y))=g(g(x))+g(g(y))-2u-g(0)$

And so, since surjective : $g(x+y)=g(x)+g(y)-2u-g(0)$
Setting then $x=y=0$, we get $u=0$ and so $g(0)=0$ and $g(x+y)=g(x)+g(y)$

So problem is equivalent (look at $Q(x,y)$) to 
$g(x)$ is additive and such that $g(g(x))=g(x)+2x$

From there, it's easy to get the general solution :
Let $A,B$ two supplementary sub-vectorspaces of the $\mathbb Q$-vectordpace $\mathbb R$
Let $a(x)$ from $\mathbb R\to A$ and $b(x)$ from $\mathbb R\to B$ the two projections of $x$ so that $x=a(x)+b(x)$ in a unique way $\forall x$

Then $g(x)=2a(x)-b(x)$ and so $f(x)=a(x)-2b(x)$

Note this gives infinitely many solutions.
The only two continuous solutions are the trivial $f(x)=x$ and $f(x)=-2x$\end{tcolorbox}


you have a sulotion in this topic :   

http://www.artofproblemsolving.com/Forum/viewtopic.php?p=3341017#p3341017
\end{solution}



\begin{solution}[by \href{https://artofproblemsolving.com/community/user/64716}{mavropnevma}]
	Last time I checked, it was Fe26 in Mendeleev's Periodic Table. :)
\end{solution}
*******************************************************************************
-------------------------------------------------------------------------------

\begin{problem}[Posted by \href{https://artofproblemsolving.com/community/user/208238}{phamngocsonyb}]
	Find all solutions of the functional system equation

$i) f(2x-1)+g(1-x)=x+1$
$ii) f(\frac{1}{x+1})+2g(\frac{1}{2x+2})=3$
	\flushright \href{https://artofproblemsolving.com/community/c6h584557}{(Link to AoPS)}
\end{problem}



\begin{solution}[by \href{https://artofproblemsolving.com/community/user/29428}{pco}]
	\begin{tcolorbox}Find all solutions of the functional system equation

$i) f(2x-1)+g(1-x)=x+1$
$ii) f(\frac{1}{x+1})+2g(\frac{1}{2x+2})=3$\end{tcolorbox}
I suppose second equation is true only for $x\ne -1$. If not : no solution.
If yes : first equaton implies $g(x)=2-x-f(1-2x)$

Plugging this in second equation, we get $f(\frac 1{x+1})-2f(1-\frac 1{x+1})=\frac 1{x+1}-1$

And so $f(x)-2f(1-x)=x-1$ $\forall x\ne 0$
And so, moving $x\to 1-x$ : $f(1-x)-2f(x)=-x$ $\forall x\ne 1$

Solving this system, we get $f(x)=\frac{x+1}3$ $\forall x\notin\{0,1\}$

Pluging $x=1$ in $f(x)-2f(1-x)=x-1$ $\forall x\ne 0$, we get $f(1)=2f(0)$

Hence the answer :

$f(0)=a$, $f(1)=2a$, $f(x)=\frac{x+1}3$ $\forall x\notin\{0,1\}$

$g(0)=2-2a$, $g(\frac 12)=\frac 32-a$, $g(x)=\frac{4-x}3$ $\forall x\notin\{0,\frac 12\}$
\end{solution}
*******************************************************************************
-------------------------------------------------------------------------------

\begin{problem}[Posted by \href{https://artofproblemsolving.com/community/user/186084}{acupofmath}]
	find all functions $ (f,g) $ such that

$ \forall x,y \in \mathbb{R} : f(x+g(y))=f(x)+y , g(x+f(y))=g(x)-y $

is this solution true?!

let $ x=y=a $ $ f(a+g(a))=f(a)+a , g(a+f(a))=g(a)-a $

so $ g(a)+a=g(a+f(a))+2a \Rightarrow f(g(a)+a)=f(g(a+f(a))+2a) $

$ \Rightarrow f(a)+a=f(g(a)+a)=f(g(a+f(a))+2a) $

$ \Rightarrow a+f(a)=f(2a+g(a+f(a))) $

$ \Rightarrow a+f(a)=f(2a)+(a+f(a)) \Rightarrow f(2a)=0 \forall a \in \mathbb{R} $

let $ x=2a \Rightarrow \forall x \in \mathbb{R} f(x)=0 $

hence $ f \equiv 0 $

let $ y=k \neq 0 $

$ \Rightarrow f(x+g(k))=f(x)+k \Rightarrow 0=f(x)+k \Rightarrow k=0 $ and it is impossible because we suppose that $ k \neq 0 $

so there is no functions $ f,g $
	\flushright \href{https://artofproblemsolving.com/community/c6h585300}{(Link to AoPS)}
\end{problem}



\begin{solution}[by \href{https://artofproblemsolving.com/community/user/29428}{pco}]
	Seems quite OK for me
\end{solution}
*******************************************************************************
-------------------------------------------------------------------------------

\begin{problem}[Posted by \href{https://artofproblemsolving.com/community/user/207996}{HQN}]
	Let $a >2$.Find all function $ f:\mathbb{R}^+\to\mathbb{R}^+ $ such that
a) $0<x<y \rightarrow  f(x) > f(y)$
b)$f(2x)=\frac{f(x)}{a^{x}}$
	\flushright \href{https://artofproblemsolving.com/community/c6h585650}{(Link to AoPS)}
\end{problem}



\begin{solution}[by \href{https://artofproblemsolving.com/community/user/29428}{pco}]
	\begin{tcolorbox}Let $a >2$.Find all function $ f:\mathbb{R}^+\to\mathbb{R}^+ $ such that
a) $0<x<y \rightarrow  f(x) > f(y)$
b)$f(2x)=\frac{f(x)}{a^{x}}$\end{tcolorbox}
Let $g(x)=\log_a f(x)$ : $g(x)$ is a strictly decreasing function from $\mathbb R^+\to \mathbb R$ and $g(2x)=g(x)-x$

So $g(2x)+2x=g(x)+x$ and so $g(x)+x=g(\frac x{2^n})+\frac x{2^n}$

This implies (using the fact that $g(x)$ is strictly decreasing) that $\lim_{n\to+\infty}g(\frac x{2^n})$ exists and so $g(x)=c-x$ for some $c\in\mathbb R$

Hence the solution : $\boxed{f(x)=\lambda a^{-x}}$ $\forall x$, which indeed is a soluton, whatever is $\lambda\in\mathbb R^+$
\end{solution}
*******************************************************************************
-------------------------------------------------------------------------------

\begin{problem}[Posted by \href{https://artofproblemsolving.com/community/user/176847}{Atlas_ha78}]
	Find all functions $ f,g: R \longrightarrow R$ satisfying:
$g(f(x)-y)=f(g(y))+x$.
	\flushright \href{https://artofproblemsolving.com/community/c6h586280}{(Link to AoPS)}
\end{problem}



\begin{solution}[by \href{https://artofproblemsolving.com/community/user/29428}{pco}]
	\begin{tcolorbox}Find all functions $ f,g: R \longrightarrow R$ satisfying:
$g(f(x)-y)=f(g(y))+x$.\end{tcolorbox}
Let $P(x,y)$ be the assertion $g(f(x)-y)=f(g(y))+x$

$P(x,0)$ $\implies$ $g(f(x))=x+f(g(0))$ and so $g(x)$ is surjective and $f(x)$ is injective
$P(0,f(x))$ $\implies$ $f(g(f(x)))=g(0)-x$ and so $f(x)$ is surjective and so bijective.
$P(f^{-1}(x),x)$ $\implies$ $f(g(x))=g(0)-f^{-1}(x)$ and so $g(x)$ is injective, and so bijective.

Let $u$ such that $f(g(u))=0$ 
$P(x,u)$ $\implies$ $f(x)=g^{-1}(x)+u$
Plugging this in $P(g(x),x)$, we get $g(x)=g(u)-u-x$ and so $f(x)=g(u)-x$

Setting $x=u$ in first expression, we get $u=0$ and so $\boxed{f(x)=g(x)=a-x\text{   }\forall x}$ which indeed is a solution, whatever is $a\in\mathbb R$
\end{solution}



\begin{solution}[by \href{https://artofproblemsolving.com/community/user/213306}{saturzo}]
	[hide="Bijectivity of $f$ and $g$ isn't required to be proved in my solution!"]$P(x, 0) : g(f(x)) = f(g(0)) +x, \forall x \in \mathbb{R}$. Using this, we get,
$P(0, f(x-f(g(0)))) : g(0) = f(g(f(x-f(g(0)))))+x-f(g(0)) \implies f(x)=f(g(0))+g(0)-x. \therefore f(x) = \alpha -x, \forall x \in \mathbb{R}$ where $\alpha=f(g(0))+g(0)$ is constant.
Using this, the original equation becomes $g(\alpha-x-y)=\alpha -g(y)+x$
$P(-x, \alpha): g(x)=\alpha - g(\alpha) -x, \forall x \in \mathbb{R}$

Putting these $(g(x)=\alpha - g(\alpha) -x$ and $f(x) = \alpha -x, \forall x \in \mathbb{R})$ into the original equation, we get $g(\alpha)=0$.

Finally, $f(x)=g(x)=\alpha-x, \forall x \in \mathbb{R}$ where $\alpha$ is any real constant -- which is clearly a solution!   [\/hide]
\end{solution}
*******************************************************************************
-------------------------------------------------------------------------------

\begin{problem}[Posted by \href{https://artofproblemsolving.com/community/user/185787}{gobathegreat}]
	Let $f$ be a function defined on real numbers. Find all functions $f$ such that for all real $x$,$y$ holds $f(x+y)+y\leq f(f(f(x)))$
	\flushright \href{https://artofproblemsolving.com/community/c6h586425}{(Link to AoPS)}
\end{problem}



\begin{solution}[by \href{https://artofproblemsolving.com/community/user/29428}{pco}]
	\begin{tcolorbox}Let $f$ be a function defined on real numbers. Find all functions $f$ such that for all real $x$,$y$ holds $f(x+y)+y\leq f(f(f(x)))$\end{tcolorbox}
Let $P(x,y)$ be the assertion $f(x+y)+y\le f(f(f(x)))$
Let $a=f(0)$

$P(f(x),f(f(f(x)))-f(x))$ $\implies$ $f(f(f(x)))\le f(x)$ and so $P(x,y)$ $\implies$ new assertion $Q(x,y)$ : $f(x+y)+y\le f(x)$

$Q(x,-x)$ $\implies$ $a-x\le f(x)$
$Q(0,x)$ $\implies$ $f(x)\le a-x$

And so $\boxed{f(x)=a-x\text{   }\forall x}$ which indeed is a solution, whatever is $a\in\mathbb R$
\end{solution}
*******************************************************************************
-------------------------------------------------------------------------------

\begin{problem}[Posted by \href{https://artofproblemsolving.com/community/user/211924}{Nemo-Mnemo}]
	Let $h$ be a real number. The function $f$ is defined such that:

$f(0,0)=1$
$f(m,0)=f(0,m)=0$  for every positive integer $m$
$f(m,n)=hf(m,n-1)+(1-h)f(m-1,n-1)$, if $m$ and $n$ are positive integers

Find all values of $h$ such that $|f(m,n)| \le 1989$ for all positive integers $m$ and $n$.
	\flushright \href{https://artofproblemsolving.com/community/c6h586428}{(Link to AoPS)}
\end{problem}



\begin{solution}[by \href{https://artofproblemsolving.com/community/user/29428}{pco}]
	\begin{tcolorbox}Let $h$ be a real number. The function $f$ is defined such that:

$f(0,0)=1$
$f(m,0)=f(0,m)=0$  for every positive integer $m$
$f(m,n)=hf(m,n-1)+(1-h)f(m-1,n-1)$, if $m$ and $n$ are positive integers

Find all values of $h$ such that $|f(m,n)| \le 1989$ for all positive integers $m$ and $n$.\end{tcolorbox}
If $h\in\{0,1\}$, we get $f(m,n)\in\{0,1\}$ $\forall m,n\in\mathbb Z^+$ and so both values fit.

If $h\notin\{0,1\}$, easy induction gives $f(m,n)=(1-h)^mh^{n-m}\binom{n-1}{m-1}$ $\forall m,n\in\mathbb Z^+$ and then :

If $h>1$, $\lim_{n\to+\infty} f(1,n)=+\infty$ and so $h$ does not fit.

If $h\in(0,1)$, then $f(m,n)>0$ $\forall m,n\in\mathbb Z^+$ and $\sum_{m=1}^nf(m,n)=1-h$ and so $f(m,n)\in[0,1)$ and $h$ fits

If $h<0$, then $\lim_{n\to+\infty}f(n,n)=+\infty$ and $h$ does not fit

Hence the answer : $\boxed{h\in[0,1]}$
\end{solution}



\begin{solution}[by \href{https://artofproblemsolving.com/community/user/211924}{Nemo-Mnemo}]
	Can you explain me how did you get to $ f(m,n)=(1-h)^mh^{n-m}\binom{n-1}{m-1} $, please?  :blush:
\end{solution}



\begin{solution}[by \href{https://artofproblemsolving.com/community/user/29428}{pco}]
	\begin{tcolorbox}Can you explain me how did you get to $ f(m,n)=(1-h)^mh^{n-m}\binom{n-1}{m-1} $, please?  :blush:\end{tcolorbox}
You can check it with induction.

Or you can write $f(m,n)=(1-h)^mh^{n-m}g(m,n)$ and equation becomes $g(m,n)=g(m,n-1)+g(m-1,n-1)$ and so ...
\end{solution}
*******************************************************************************
-------------------------------------------------------------------------------

\begin{problem}[Posted by \href{https://artofproblemsolving.com/community/user/125553}{lehungvietbao}]
	Find all possible values of function $f(x)$
\[f(x)=\left \lfloor  x\right \rfloor+\left \lfloor 2x \right \rfloor +\left \lfloor \frac{5x}{3} \right \rfloor +\left \lfloor 3x \right \rfloor +\left \lfloor 4x \right \rfloor \quad  \forall x\in [0;100]\]
	\flushright \href{https://artofproblemsolving.com/community/c6h586706}{(Link to AoPS)}
\end{problem}



\begin{solution}[by \href{https://artofproblemsolving.com/community/user/29428}{pco}]
	\begin{tcolorbox}Find all possible values of function $f(x)$
\[f(x)=\left \lfloor  x\right \rfloor+\left \lfloor 2x \right \rfloor +\left \lfloor \frac{5x}{3} \right \rfloor +\left \lfloor 3x \right \rfloor +\left \lfloor 4x \right \rfloor \quad  \forall x\in [0;100]\]\end{tcolorbox}
Clearly $f(x+3)=f(x)+35$ and breaks can occur only for rational numbers $x=\frac n{60}$

So we just have to compute the $180$ values $f(\frac n{60})$ for $n\in\{0,1,...,179\}$ and we get $\{0,1,2,4,5,6,7,$ $11,12,13,14,16,17,18,$ $19,23,24,25,26,28,29,30\}$


Hence the answer : 
all integers equal to $\{0,1,2,4,5,6,7,$ $11,12,13,14,16,17,18,$ $19,23,24,25,26,28,29,30\}$ $\pmod{35}$ in $[0,f(100)=1166]$
\end{solution}
*******************************************************************************
-------------------------------------------------------------------------------

\begin{problem}[Posted by \href{https://artofproblemsolving.com/community/user/125553}{lehungvietbao}]
	Find all the functions $f:\mathbb{R}\to\mathbb{R}$ such that 
\[f(x+f(y))=(f(y))^4+4x^3(f(y))+6x^2(f(y))^2+4x(f(y))^3+f(-x) \quad \forall x,y \in\mathbb R\]
	\flushright \href{https://artofproblemsolving.com/community/c6h586709}{(Link to AoPS)}
\end{problem}



\begin{solution}[by \href{https://artofproblemsolving.com/community/user/29428}{pco}]
	\begin{tcolorbox}Find all the functions $f:\mathbb{R}\to\mathbb{R}$ such that 
\[f(x+f(y))=(f(y))^4+4x^3(f(y))+6x^2(f(y))^2+4x(f(y))^3+f(-x) \quad \forall x,y \in\mathbb R\]\end{tcolorbox}
$\boxed{\text{S1 : }f(x)=0\text{  }\forall x}$ is a solution. So let us from now look only fo non allzero solutions.

Let $P(x,y)$ be the assertion $f(x+f(y))=$ $f(y)^4+4x^3f(y)+6x^2f(y)^2+4xf(y)^3+f(-x)$ $=(x+f(y))^4+f(-x)-x^4$
Let $a=f(0)$
Let $b$ such that $c=f(b)\ne 0$

The equation $4cx^3+6c^2x^2+4c^3x+c^4=t$ is a cubic (since $c\ne 0$) and so has at least a real solution $w$ whatever is $t$
Then $P(w,b)$ $\implies$ $f(w+c)-f(-w)=t$ and so any real $t$ may be written as $f(u)-f(v)$ for some $u,v$

$P(0,v)$ $\implies$ $f(f(v))=f(v)^4+a$
$P(f(v)-f(u),u)$ $\implies$ $f(f(v))=f(v)^4+f(f(u)-f(v))-(f(u)-f(v))^4$

Subtracting, we get $f(f(u)-f(v))=(f(u)-f(v))^4+a$ and so $\boxed{\text{S2 : }f(x)=x^4+a\text{   }\forall x}$, which indeed is a solution, whatever is $a\in\mathbb R$
\end{solution}
*******************************************************************************
-------------------------------------------------------------------------------

\begin{problem}[Posted by \href{https://artofproblemsolving.com/community/user/186076}{John532}]
	Find all $f:R \rightarrow R$  continuous such that
$f(x+f(y))=f(x+y).$
	\flushright \href{https://artofproblemsolving.com/community/c6h586978}{(Link to AoPS)}
\end{problem}



\begin{solution}[by \href{https://artofproblemsolving.com/community/user/29428}{pco}]
	\begin{tcolorbox}Find all $f:R \rightarrow R$  continuous such that
$f(x+f(y))=f(x+y).$\end{tcolorbox}
Let $f(x)=g(x)+x$ and problem becomes assertion $P(x,y)$ : $g(x+g(y))=g(x)-g(y)$
We immediately get with induction $g(x+ng(y))=g(x)-ng(y)$ $\forall x,y\in\mathbb R$, $\forall n\in\mathbb Z$

If $g(x)=0$ $\forall x$, we got a first solution $\boxed{\text{S1 : }f(x)=x\text{   }\forall x}$

If $\exists b$ such that $g(b)=c\ne 0$, the equation $g(x+nc)=g(x)-nc$ shows that $g(x)$ is neither upper-bounded, neither lower-bounded and so, since continuous, is surjective.
Then $P(0,x)$ $\implies$ $g(g(x))=g(0)-g(x)$ and, since surjective, $g(x)=a-x$ $\forall x$ which indeed is a solution, whatever is $a\in\mathbb R$

And so $\boxed{\text{S2 : }f(x)=a\text{   }\forall x}$
\end{solution}



\begin{solution}[by \href{https://artofproblemsolving.com/community/user/184652}{CanVQ}]
	\begin{tcolorbox}Find all $f:R \rightarrow R$  continuous such that
$f(x+f(y))=f(x+y).$\end{tcolorbox}
My solution is similar to pco's one but I still want to post it.

From the given equation, we can easily prove by induction that \[f\Big(x+n\cdot \big(f(y)-y\big)\Big)=f(x),\quad \forall x,\, y \in \mathbb R,\, n \in \mathbb Z.\quad (1)\] If $f(y) -y$ is surjective, then for each $t \in \mathbb R,$ there exists $y_0$ such that $t=f(y_0)-y_0.$ Plugging this into $(1)$ and taking $n=1,$ we get \[f(x+t)=f(x),\quad \forall x,\,t \in \mathbb R.\quad (2)\] So $f(x)\equiv c,\, \forall x \in \mathbb R$ and this is clearly a solution.

If $f(y)-y$ is not surjective, then its image should have one of the following forms: $[a,\,b],$ $(-\infty, \, a]$ or $[a,\, +\infty),$ due to the continuity.
[list][*] If $f(y)-y\in [a,\,b],\, \forall y \in \mathbb R,$ we have \[a+y \le f(y)\le b+y,\quad \forall y \in \mathbb R,\] from which it follows that $f$ is surjective due to the continuity. Since $f\big(f(y)\big)=f(y)$ (taking $x=0$ in the orginal equation), it follows that $f(x)=x,\, \forall x \in \mathbb R.$ This is a solution.
[*] If $f(y)-y \in (-\infty,\, a],\, \forall y \in \mathbb R,$ we have \[f(y)\le y+a,\quad \forall y \in \mathbb R.\] Using this result in $(1),$ we get \[n \cdot \big( f(y)-y\big)+a \ge f(x)-x,\quad \forall x,\, y \in \mathbb R,\, n \in \mathbb Z,\] which occurs only when $f(x)=x,\,\forall x \in \mathbb R$ and $a=0.$
[*] The last case $f(y)-y \in [a,\, +\infty),\, \forall y \in \mathbb R,$ we can use the same argument as above to obtain $f(x)=x,\, \forall x \in \mathbb R.$[\/list][\/list] Finally, we get $f(x)\equiv c$ and $f(x)=x$ are the solutions.
\end{solution}
*******************************************************************************
-------------------------------------------------------------------------------

\begin{problem}[Posted by \href{https://artofproblemsolving.com/community/user/153511}{Konigsberg}]
	Let $\mathbb{R}*$ denote the set of nonzero real numbers. Find all functions $f:\mathbb{R}* \rightarrow \mathbb{R}*$ such that $f(x^2+y)=f(f(x))+\frac{f(xy)}{f(x)}$ for every pair of nonzero real numbers $x$ and $y$ with $x^2+y \neq 0$.
	\flushright \href{https://artofproblemsolving.com/community/c6h587225}{(Link to AoPS)}
\end{problem}



\begin{solution}[by \href{https://artofproblemsolving.com/community/user/29428}{pco}]
	\begin{tcolorbox}Let $\mathbb{R}*$ denote the set of nonzero real numbers. Find all functions $f:\mathbb{R}* \rightarrow \mathbb{R}*$ such that $f(x^2+y)=f(f(x))+\frac{f(xy)}{f(x)}$ for every pair of nonzero real numbers $x$ and $y$ with $x^2+y \neq 0$.\end{tcolorbox}
Let $P(x,y)$ be the assertion $f(x^2+y)=f(f(x))+\frac{f(xy)}{f(x)}$ true whatever are $x,y$ such that $x,y,x^2+y\ne 0$

If $f(x)\ne x^2$ for some $x\ne 0$, then $P(x,f(x)-x^2)$ $\implies$  $f(x(f(x)-x^2))=0$, impossible.

So $f(x)=x^2$ $\forall x\ne 0$ which unfortunately is not a solution.

So no solution for this functional equation.
\end{solution}
*******************************************************************************
-------------------------------------------------------------------------------

\begin{problem}[Posted by \href{https://artofproblemsolving.com/community/user/125553}{lehungvietbao}]
	Find all functions $f:\mathbb{R}\rightarrow \mathbb{R}$ such that
\[f(x+f(y)))=f(x)+\frac{1}{8}xf(4y)+f(f(y))\quad \forall x,y\in\mathbb{R}\]
	\flushright \href{https://artofproblemsolving.com/community/c6h587306}{(Link to AoPS)}
\end{problem}



\begin{solution}[by \href{https://artofproblemsolving.com/community/user/29428}{pco}]
	\begin{tcolorbox}Find all functions $f:\mathbb{R}\rightarrow \mathbb{R}$ such that
\[f(x+f(y)))=f(x)+\frac{1}{8}xf(4y)+f(f(y))\quad \forall x,y\in\mathbb{R}\]\end{tcolorbox}
$\boxed{\text{S1 : }f(x)=0\text{  }\forall x}$ is a solution. So let us from now look only for non allzero solutions.

Let $P(x,y)$ be the assertion $f(x+f(y))=f(x)+\frac 18xf(4y)+f(f(y))$
Let $a$ such that $f(a)=b\ne 0$
Let $c=\frac{f(4a)}{8b}$


$P(0,0)$ $\implies$ $f(0)=0$
$P(f(x),a)$ $\implies$ $f(f(x)+b)=f(f(x))+\frac 18f(x)f(4a)+f(b)$
$P(b,x)$ $\implies$ $f(b+f(x))=f(b)+\frac 18bf(4x)+f(f(x))$
Subtracting, we get $f(4x)=8cf(x)$ and so $c\ne 0$ in this part of the proof.

And so $P(x,y)$ becomes new assertion $Q(x,y)$  : $f(x+f(y))=f(x)+cxf(y)+f(f(y))$

$Q(f(x),x)$ $\implies$ $f(2f(x))=2f(f(x))+cf(x)^2$
$Q(2f(x),x)$ $\implies$ $f(3f(x))=3f(f(x))+3cf(x)^2$
$Q(3f(x),x)$ $\implies$ $f(4f(x))=4f(f(x))+6cf(x)^2$
And since we know that $f(4f(x))=8cf(f(x))$, last equation implies $(4c-2)f(f(x))=3cf(x)^2$
If $c=\frac 12$, this implies $f(x)=0$ $\forall x$, impossible in this part of the proof and so $c\ne \frac 12$ and $f(f(x))=\frac{3c}{2(2c-1)}f(x)^2$

Let then $d=\frac{3c}{2(2c-1)}$ and we get $f(f(x))=df(x)^2$ and so 
$Q(x,y)$ becomes new assertion $R(x,y)$ : $f(x+f(y))=f(x)+cxf(y)+df(y)^2$

$R(-f(x),x)$ $\implies$ $f(-f(x))=(c-d)f(x)^2$
$R(-2f(x),x)$ $\implies$ $f(-2f(x))=(3c-2d)f(x)^2$
$R(-3f(x),x)$ $\implies$ $f(-3f(x))=(6c-3d)f(x)^2$
$R(-4f(x),x)$ $\implies$ $f(-4f(x))=(10c-4d)f(x)^2$
And since we know that $f(-4f(x))=8cf(-f(x))=8c(c-d)f(x)^2$, we get $4c(c-d)=5c-2d$ and, since $d=\frac{3c}{2(2c-1)}$, this gives equation in $c$ :
$c(c-2)(2c-1)=0$ and so $c=2$ (we already wrote that $c\ne 0$ and $c\ne \frac 12$) and so $d=1$ 

And so $R(x,y)$ becomes new assertion $S(x,y)$ : $f(x+f(y))=f(x)+2xf(y)+f(y)^2$
$R(\frac{x-b^2}{2b},a)$ $\implies$ $f(\frac{x-b^2}{2b}+b)-f(\frac{x-b^2}{2b})=x$ and so any real $x$ may be written as $x=f(u)-f(v)$ for some $u,v\in\mathbb R$

$R(-f(v),v)$ $\implies$ $f(-f(v))=f(v)^2$
$R(-f(v),u)$ $\implies$ $f(f(u)-f(v))=f(u)^2-2f(u)f(v)+f(v)^2$ $=(f(u)-f(v))^2$

And so $\boxed{\text{S2 : }f(x)=x^2\text{   }\forall x}$ which indeed is a solution.
\end{solution}
*******************************************************************************
-------------------------------------------------------------------------------

\begin{problem}[Posted by \href{https://artofproblemsolving.com/community/user/146934}{Sowmitra}]
	Given that, a function $f(n)$, defined on the natural numbers, satisfies the following conditions: (i)$f(n)=n-12$ if $n>2000$; (ii)$f(n)=f(f(n+16))$ if $n \leq 2000$.
(a) Find $f(n)$.
(b) Find all solutions to $f(n)=n$.
	\flushright \href{https://artofproblemsolving.com/community/c6h587326}{(Link to AoPS)}
\end{problem}



\begin{solution}[by \href{https://artofproblemsolving.com/community/user/29428}{pco}]
	\begin{tcolorbox}Given that, a function $f(n)$ defined on the natural numbres satisfies the following conditions: (i)$f(n)=n-12$ if $n>2000$; (ii)$f(n)=f(f(n+16))$ if $n \leq 2000$.
(a) Find $f(n)$.
(b) Find all solutions to $f(n)=n$.\end{tcolorbox}
If $n>2000$, $f(n)=n-12$
If $2000\ge n>1984$, then $n+16>2000$ and $f(n)=f(f(n+16))=f(n+4)$ and so $f(n)$ depends only on $n\pmod 4$ :
-- If $n\equiv 0\pmod 4$, then $f(n)=1992$
-- If $n\equiv 1\pmod 4$, then $f(n)=1989$
-- If $n\equiv 2\pmod 4$, then $f(n)=1990$
-- If $n\equiv 3\pmod 4$, then $f(n)=1991$
If $1984\ge n>1968$, then $2000\ge n+16>1984$ and so $f(n+16)\in\{1989,1990,1991,1992\}\subset (1984,2000]$ and so $f(n)$ depends only on $n\pmod 4$ :
-- If $n\equiv 0\pmod 4$, then $f(n)=f(f(n+16))=f(1992)=1992$
-- If $n\equiv 1\pmod 4$, then $f(n)=f(f(n+16))=f(1989)=1989$
-- If $n\equiv 2\pmod 4$, then $f(n)=f(f(n+16))=f(1990)=1990$
-- If $n\equiv 3\pmod 4$, then $f(n)=f(f(n+16))=f(1991)=1991$
...

And it's then easy to get the final result :
If $n>2000$, $f(n)=n-12$
If $2000\ge n$, $f(n)=1989+\mod(n-1,4)$

And solutions to equation $f(n)=n$ are $\{1989,1990,1991,1992\}$
\end{solution}



\begin{solution}[by \href{https://artofproblemsolving.com/community/user/332813}{targo___}]
	So for all $n$ we have that property right ?
as, when $n\in (1952,1968]$ then also $n+16\in (1968,1984]$ so, $f(n)=f(n+4)$ . and also same for $n\in (1946,1952]$ .I am not getting the behavior of $f(n)$ ..Please correct me if i am wrong .
\end{solution}
*******************************************************************************
-------------------------------------------------------------------------------

\begin{problem}[Posted by \href{https://artofproblemsolving.com/community/user/194411}{EdsonBR}]
	Does exist functions f and g on real numbers such that f(g(x))=x² and g(f(x))=x³ ?
	\flushright \href{https://artofproblemsolving.com/community/c6h587761}{(Link to AoPS)}
\end{problem}



\begin{solution}[by \href{https://artofproblemsolving.com/community/user/29428}{pco}]
	\begin{tcolorbox}Does exist functions f and g on real numbers such that $f(g(x))=x^2$ and $g(f(x))=x^3$ ?\end{tcolorbox}
$g(f(x))=x^3$ implies $f(x)$ injective.

$f(x^3)=f(g(f(x)))=f(x)^2$  and so :
$f(-1)=f(-1)^2$ and so $f(-1)\in\{0,1\}$
$f(0)=f(0)^2$ and so $f(0)\in\{0,1\}$
$f(1)=f(1)^2$ and so $f(1)\in\{0,1\}$

So contradiction and no such functions.
\end{solution}
*******************************************************************************
-------------------------------------------------------------------------------

\begin{problem}[Posted by \href{https://artofproblemsolving.com/community/user/194411}{EdsonBR}]
	Prove that the only function \[f:Z\rightarrow Z\]  such that \[f(f(n))=n , f(f(n+2)+2)=n,\] and, \[f(0)=1\] is \[f(n)=1-n\]
	\flushright \href{https://artofproblemsolving.com/community/c6h587763}{(Link to AoPS)}
\end{problem}



\begin{solution}[by \href{https://artofproblemsolving.com/community/user/29428}{pco}]
	\begin{tcolorbox}Prove that the only function \[f:Z\rightarrow Z\]  such that \[f(f(n))=n , f(f(n+2)+2)=n,\] and, \[f(0)=1\] is \[f(n)=1-n\]\end{tcolorbox}
$0=f(f(0))=f(1)$
$f(f(n))=n$ implies$f(n)$ injective.
Then $f(f(n+2)+2)=n=f(f(n))$ implies $f(n+2)=f(n)-2$

So $f(2n)=f(0)-2n=1-2n$ and $f(2n+1)=f(1)-2n=-2n=1-(2n+1)$

And so $f(n)=1-n$ $\forall n$, which indeed is a solution (easy check)
\end{solution}
*******************************************************************************
-------------------------------------------------------------------------------

\begin{problem}[Posted by \href{https://artofproblemsolving.com/community/user/194411}{EdsonBR}]
	Give all functions f such that  $ f(x+y)+f(y+z)+f(z+x))\geq 3.f(x+2y+3z) $  for all x,y and z real .
	\flushright \href{https://artofproblemsolving.com/community/c6h587770}{(Link to AoPS)}
\end{problem}



\begin{solution}[by \href{https://artofproblemsolving.com/community/user/29428}{pco}]
	\begin{tcolorbox}Give all functions f such that  $ f(x+y)+f(y+z)+f(z+x))\geq 3.f(x+2y+3z) $  for all x,y and z real .\end{tcolorbox}
Let $P(x,y,z)$ be the assertion $f(x+y)+f(y+z)+f(z+x)\ge 3f(x+2y+3z)$

$P(x,0,0)$ $\implies$ $f(0)\ge f(x)$

$P(\frac x2,\frac x2,-\frac x2)$ $\implies$ $f(x)\ge f(0)$

And so $\boxed{f(x)=a}$ $\forall x$, which indeed is a solution, whatever is $a\in\mathbb R$
\end{solution}
*******************************************************************************
-------------------------------------------------------------------------------

\begin{problem}[Posted by \href{https://artofproblemsolving.com/community/user/194411}{EdsonBR}]
	Give all functions$ f:R\rightarrow R $ such that all real $ x $ and $ y $  , is valid the relation $ f(x+y+f(y)=4x-f(x)+f(3y) $
	\flushright \href{https://artofproblemsolving.com/community/c6h587775}{(Link to AoPS)}
\end{problem}



\begin{solution}[by \href{https://artofproblemsolving.com/community/user/29428}{pco}]
	\begin{tcolorbox}Give all functions$ f:R\rightarrow R $ such that all real $ x $ and $ y $  , is valid the relation $ f(x+y+f(y)=4x-f(x)+f(3y) $\end{tcolorbox}
Missing parenthesis in LHS. May be understood as $f(x+y+f(y))$ or $f(x+y)+f(y)$ or $f(x)+y+f(y)$.
\end{solution}



\begin{solution}[by \href{https://artofproblemsolving.com/community/user/194411}{EdsonBR}]
	$f(x+y +f(y)) $ is The right one
\end{solution}



\begin{solution}[by \href{https://artofproblemsolving.com/community/user/29428}{pco}]
	\begin{tcolorbox}Give all functions$ f:R\rightarrow R $ such that all real $ x $ and $ y $  , is valid the relation $ f(x+y+f(y))=4x-f(x)+f(3y) $\end{tcolorbox}
Let $P(x,y)$ be the assertion $f(x+y+f(y))=4x-f(x)+f(3y)$
Let $a=f(0)$

$P(0,0)$ $\implies$ $f(a)=0$
$P(a,0)$ $\implies$ $f(2a)=5a$
$P(0,a)$ $\implies$ $f(3a)=a$
$P(2a,0)$ $\implies$ $f(3a)=4a$
So $a=4a$ and so $a=0$

$P(x,0)$ $\implies$ $\boxed{f(x)=2x}$ $\forall x$, which indeed is a solution.
\end{solution}
*******************************************************************************
-------------------------------------------------------------------------------

\begin{problem}[Posted by \href{https://artofproblemsolving.com/community/user/125553}{lehungvietbao}]
	Find all the functions $f:\mathbb{R}^+\to\mathbb{R}^+$ such that $f(x)f(y)=\beta f(x+yf(x)) \quad \forall x,y\in\mathbb{R}^+$ for a given $\beta \in\mathbb R ,\beta>1$
	\flushright \href{https://artofproblemsolving.com/community/c6h588834}{(Link to AoPS)}
\end{problem}



\begin{solution}[by \href{https://artofproblemsolving.com/community/user/29428}{pco}]
	\begin{tcolorbox}Find all the functions $f:\mathbb{R}^+\to\mathbb{R}^+$ such that $f(x)f(y)=\beta f(x+yf(x)) \quad \forall x,y\in\mathbb{R}^+$ for a given $\beta \in\mathbb R ,\beta>1$\end{tcolorbox}
Where is $\beta$ in the functional equation ?
\end{solution}



\begin{solution}[by \href{https://artofproblemsolving.com/community/user/106845}{simba_master}]
	\begin{tcolorbox}Find all the functions $f:\mathbb{R}^+\to\mathbb{R}^+$ such that $f(x)f(y)=\beta f(x+yf(x)) \quad \forall x,y\in\mathbb{R}^+$ for a given $\beta \in\mathbb R ,\beta>1$\end{tcolorbox}

Correct is: $f(x)f(y)=\beta\cdot f(x+yf(x))$
\end{solution}



\begin{solution}[by \href{https://artofproblemsolving.com/community/user/29428}{pco}]
	\begin{tcolorbox}Find all the functions $f:\mathbb{R}^+\to\mathbb{R}^+$ such that $f(x)f(y)=\beta f(x+yf(x)) \quad \forall x,y\in\mathbb{R}^+$ for a given $\beta \in\mathbb R ,\beta>1$\end{tcolorbox}
Let $P(x,y)$ be the assertion $f(x)f(y)=\beta f(x+yf(x))$

1) $f(x)\ge \beta$ $\forall x$
=============
If $f(x)<1$ for some $x$, then $P(x,\frac x{1-f(x)})$ $\implies$ $f(x)=\beta>1$, impossible. So $f(x)\ge 1$ $\forall x$

If $f(x)\ge \beta^t$ $\forall x$ and for some $t\ge 0$; then $P(x,x)$ $\implies$ $f(x)^2\ge \beta^{1+t}$ and so $f(x)\ge \beta^{\frac{1+t}2}$ $\forall x$

Setting then $a_0=0$ and $a_{n+1}=\frac{1+a_n}2$, we get $f(x)\ge \beta^{a_n}$ $\forall x$, $\forall n$ and so $f(x)\ge \beta$ $\forall x$
Q.E.D.

2) If $f(u)=\beta$ for some $u>0$, then $f(x)=\beta$ $\forall x$
=====================================
If $f(u)=\beta$ for some $u>0$, then, for $x<u$ : $P(x,\frac{u-x}{f(x)})$ $\implies$ $f(x)f(\frac{u-x}{f(x)})=\beta^2$ and so $f(x)=\beta$ $\forall x\le u$

But $P(u,u)$ $\implies$ $f(u(1+\beta))=\beta$ and so $f(u(1+\beta)^n)=\beta$ and so, using previous line, $f(x)=\beta$ $\forall x$
Q.E.D.

3) $f(x)$ is not injective
================
If $f(x)$ is injective, comparaison of $P(x,1)$ with $P(1,x)$ implies $f(x+f(x))=f(1+xf(1))$ and so $x+f(x)=1+xf(1)$ and so $f(x)=1+x(f(1)-1)$
Plugging this back in original equation, we get $\beta=1$, impossible.
Q.E.D.

4) $f(x)=\beta$ $\forall x$
===============
Since non injective, let $a>b$ such that $f(a)=f(b)$. Then let $u=\frac{b-a}{f(a)}$ : $P(a,u)$ $\implies$ $f(u)=\beta$ 

Hence the result (using 2) above).
\end{solution}
*******************************************************************************
-------------------------------------------------------------------------------

\begin{problem}[Posted by \href{https://artofproblemsolving.com/community/user/105700}{ssilwa}]
	Find all functions $f: \mathbb{R} \rightarrow \mathbb{R}$ such that $f(x)f \left( \frac{1}x \right) = 1$.
	\flushright \href{https://artofproblemsolving.com/community/c6h589432}{(Link to AoPS)}
\end{problem}



\begin{solution}[by \href{https://artofproblemsolving.com/community/user/31919}{tenniskidperson3}]
	Assuming the domain doesn't include $0$... Just define $f$ however you want on $(-1, 1)-\{0\}$, define $f(1)$ and $f(-1)$ as $\pm 1$, and then define $f(x)=\frac{1}{f\left(\frac{1}{x}\right)}$ for $|x|>1$.
\end{solution}



\begin{solution}[by \href{https://artofproblemsolving.com/community/user/29428}{pco}]
	\begin{tcolorbox}Find all functions $f: \mathbb{R} \rightarrow \mathbb{R}$ such that $f(x)f \left( \frac{1}x \right) = 1$.\end{tcolorbox}
As usual for such functional equations, many equivalent general forms for solutions exist. Here is another one : 

$\boxed{f(x)=s(x+\frac 1x)e^{h(x)-h(\frac 1x)}}$ where $s(x)$ is any function from $\mathbb R\setminus\{0\}\to\{-1,+1\}$ and $h(x)$ is any function from $\mathbb R\setminus\{0\}\to\mathbb R$
\end{solution}
*******************************************************************************
-------------------------------------------------------------------------------

\begin{problem}[Posted by \href{https://artofproblemsolving.com/community/user/105169}{Nikpour}]
	$f{(x+y)^2} = {(f(x) + f(y))^2}$
	\flushright \href{https://artofproblemsolving.com/community/c6h589543}{(Link to AoPS)}
\end{problem}



\begin{solution}[by \href{https://artofproblemsolving.com/community/user/172163}{joybangla}]
	\begin{tcolorbox}$f{(x,y)^2} = {(f(x) + f(y))^2}$\end{tcolorbox}
This is meaningless. Left side seems to show this function is of two variables. But right side does not. What is the domain and range?? $f:A\to B$ ? where $A$ is the set of finite binary strings and $B$ is the set of students in my school?? Please post questions in a meaningful manner. This post is just extreme case of spamming.
\end{solution}



\begin{solution}[by \href{https://artofproblemsolving.com/community/user/105169}{Nikpour}]
	Thanks. Now it has edited.
\end{solution}



\begin{solution}[by \href{https://artofproblemsolving.com/community/user/172163}{joybangla}]
	Did you read the full post of mine?? Specify the domain. This is pretty annoying dude.
\end{solution}



\begin{solution}[by \href{https://artofproblemsolving.com/community/user/105169}{Nikpour}]
	Find \begin{bolded}[color=#FF0000]all[\/color]\end{bolded} functions $f:\mathbb{R}\to \mathbb{R}$ satisfying the $f{(x+y)^2} = {(f(x) + f(y))^2}$ for all $x,y\in \mathbb{R}$.
$f{{(x+y)}^{2}}$ means $f({{(x+y)}^{2}})$.
\end{solution}



\begin{solution}[by \href{https://artofproblemsolving.com/community/user/29428}{pco}]
	\begin{tcolorbox}Find \begin{bolded}[color=#FF0000]all[\/color]\end{bolded} functions $f:\mathbb{R}\to \mathbb{R}$ satisfying the $f((x+y)^2) = {(f(x) + f(y))^2}$ for all $x,y\in \mathbb{R}$\end{tcolorbox}
Let $P(x,y)$ be the assertion $f((x+y)^2)=(f(x)+f(y))^2$
Let $a=f(0)$

Obviously $f(x)\ge 0$ $\forall x\ge 0$

Subtracting $P(x+y,0)$ from $P(x,y)$, we get $(f(x+y)+a)^2=(f(x)+f(y))^2$ and so $f(x+y)+a=f(x)+f(y)$ $\forall x,y\ge 0$

So $f(x+y)-a=(f(x)-a)+(f(y)-a)$ $\forall x,y\ge 0$ and $f(x)-a$ is an additive lowerbounded function over $R^+$ and so is $cx$ for some $c\ge 0$

So $f(x)=cx+a$ $\forall x\ge 0$
Plugging this in original equation, we get $(c,a)\in\{(0,0),(0,\frac 14),(1,0)\}$

1) $(c,a)=(0,0)$ so that $f(x)=0$ $\forall x\ge 0$
===============================
$P(x,x)$ implies then $\boxed{\text{S1 : }f(x)=0\text{   }\forall x}$

2) $(c,a)=(0,\frac 14)$ so that $f(x)=\frac 14$ $\forall x\ge 0$
================================
$P(x,x)$ implies then $f(x)=\pm\frac 14$ $\forall x$
But if $f(u)=-\frac 14$ for some $u<0$, then $P(u,-u)$ implies contradiction  and so $\boxed{\text{S2 : }f(x)=\frac 14\text{   }\forall x}$

3) $(c,a)=(1,0)$ so that $f(x)=x$ $\forall x\ge 0$
===============================
Let $x<0$ and any $y\ge -x$. $P(x,y)$ $\implies$ $(f(x)-x)(f(x)+x+2y)=0$ $\forall y\ge -x$ and so $\boxed{\text{S3 : }f(x)=x\text{   }\forall x}$
\end{solution}



\begin{solution}[by \href{https://artofproblemsolving.com/community/user/199494}{IMI-Mathboy}]
	We know$\boxed{f(0)=0}$ is a solution.Let $P(x,y)$ be assertion of the function: $f$. Then it easy to see $f(x)$ is nonnegative for nonnegative $x$.$(*)$ Putting $P(0,0)$ we get $f(0)=\{0,\frac{1}{4}\}$  $1.CASE$ :   $f(0)=\frac{1}{4}$ then $P(x,-x)$ we get $ f(x)+f(-x)=\{-\frac{1}{2},\frac{1}{2}\}$ If $ f(u)+f(-u)=-\frac{1}{2}$ for some $u$ $u\in\mathbb{R}$ then comparing $P(u,u)$,$P(-u,-u)$ we get contradiction to $(*)$. $\Rightarrow$ $ f(x)+f(-x)=\frac{1}{2}$ then comparing $P(x,0)$ and $P(-x,0)$ we get $\boxed{f(x)=\frac{1}{4}}$ for all $x\in\mathbb{R}$. $ 2.CASE$ $f(0)=0$ then $P(x,-x)$ we have $f(x)=-f(-x)$ $(**)$ then $P(x,0)$ $\Rightarrow$  $f(x^2)=f(x)^2.$ then we have $f(x)+f(y)=f(x+y)$ $(***)$ for $x,y\in\mathbb{R^+}$ then by $(**)$ it is true for all  $x,y\in\mathbb{R}$. Then it is easy  to find $f(r)=r$ for  all $r\in\mathbb{Q}$ and $f$ is increasing from $(***)$ and $(*).$  . If $f(u)>u$ for some u. Then there exists $r\in\mathbb{Q}$ such that $f(u)>r>u$ then $f(u)<r$ contradiction then assuming $f(u)<u$ we get another contradiction.So $\boxed{f(x)=x}$ for all $x\in\mathbb{R}$
\end{solution}
*******************************************************************************
-------------------------------------------------------------------------------

\begin{problem}[Posted by \href{https://artofproblemsolving.com/community/user/211457}{rod16}]
	Find all functions $f:\mathbb{R}^{+}\rightarrow \mathbb{R}^{+} $ which  $f(a^{b}) = f(a) ^{f(b)}$.
	\flushright \href{https://artofproblemsolving.com/community/c6h589660}{(Link to AoPS)}
\end{problem}



\begin{solution}[by \href{https://artofproblemsolving.com/community/user/29428}{pco}]
	\begin{tcolorbox}Find all functions $f:\mathbb{R}^{+}\rightarrow \mathbb{R}^{+} $ which  $f(a^{b}) = f(a) ^{f(b)}$.\end{tcolorbox}
$\boxed{\text{S1 : }f(x)=1\text{  }\forall x>0}$  is a solution
Let us from now look only for non allOne solutions.

Let $P(x,y)$ be the assertion $f(x^y)=f(x)^{f(y)}$
Let $u>0$ such that $f(u)\ne 1$

Comparing $P(u^y,x)$ with $P(u,xy)$, we get $f(xy)=f(x)f(y)$ $\forall x,y$
Then $f(u)^{f(x)+f(y)}=f(u)^{f(x)}f(u)^{f(y)}$ $=f(u^x)f(u^y)$ $=f(u^{x+y})=f(u)^{f(x+y)}$ and so $f(x+y)=f(x)+f(y)$

So $f(x)$ is positive, additive and multiplicative and it's very classical to get :

$\boxed{\text{S2 : }f(x)=x\text{  }\forall x>0}$  which indeed is a solution
\end{solution}



\begin{solution}[by \href{https://artofproblemsolving.com/community/user/173116}{Sardor}]
	I think this problem from ARMO ( before some year)
\end{solution}
*******************************************************************************
-------------------------------------------------------------------------------

\begin{problem}[Posted by \href{https://artofproblemsolving.com/community/user/153511}{Konigsberg}]
	Consider three functions: $f_1(x)=x+\frac{1}{x}$, $f_2(x)=x^2$, $f_3(x)=(x-1)^2$. You may add, subtract and multiply these functions (in particular, you can square and raise to higher powers, and so on), multiply by an arbitrary number, add an arbitrary number to your result and perform the above described operations with the expressions obtained. Obtain $\frac{1}{x}$ in this way. Prove that if one of the functions $f_1$, $f_2$ or $f_3$ is taken out, then it is impossible to obtain $\frac{1}{x}$ in the way described. 

Proposed by M. Evdokimov.
	\flushright \href{https://artofproblemsolving.com/community/c6h589697}{(Link to AoPS)}
\end{problem}



\begin{solution}[by \href{https://artofproblemsolving.com/community/user/29428}{pco}]
	\begin{tcolorbox}Consider three functions: $f_1(x)=x+1$, $f_2(x)=x^2$, $f_3(x)=(x-1)^2$. You may add, subtract and multiply these functions (in particular, you can square and raise to higher powers, and so on), multiply by an arbitrary number, add an arbitrary number to your result and perform the above described operations with the expressions obtained. Obtain $\frac{1}{x}$ in this way. Prove that if one of the functions $f_1$, $f_2$ or $f_3$ is taken out, then it is impossible to obtain $x$ in the way described. 

Proposed by M. Evdokimov.\end{tcolorbox}
$\frac 1x=\frac 1{2-f_1+f_2-f_3}$

$x=\frac{f_1^2-f_2-1}2$, without using $f_3$
\end{solution}



\begin{solution}[by \href{https://artofproblemsolving.com/community/user/213572}{olimovich}]
	\begin{tcolorbox}[quote="Konigsberg"]Consider three functions: $f_1(x)=x+1$, $f_2(x)=x^2$, $f_3(x)=(x-1)^2$. You may add, subtract and multiply these functions (in particular, you can square and raise to higher powers, and so on), multiply by an arbitrary number, add an arbitrary number to your result and perform the above described operations with the expressions obtained. Obtain $\frac{1}{x}$ in this way. Prove that if one of the functions $f_1$, $f_2$ or $f_3$ is taken out, then it is impossible to obtain $x$ in the way described. 

Proposed by M. Evdokimov.\end{tcolorbox}
$\frac 1x=\frac 1{2-f_1+f_2-f_3}$

$x=\frac{f_1^2-f_2-1}2$, without using $f_3$\end{tcolorbox}
Konigsberg, please make sure that the condition of your problem is correct. If the source you are mentioning is correct, then \begin{bolded}pco\end{bolded}'s solution confirms that you must have done a typo somewhere. 
I have no doubt for that as it already happened here http://www.artofproblemsolving.com/Forum/viewtopic.php?f=36&t=589698&p=3492226#p3492226
\end{solution}



\begin{solution}[by \href{https://artofproblemsolving.com/community/user/153511}{Konigsberg}]
	Sorry, edited. $f_1(x)=x+\frac{1}{x}$, not $x+1$.
\end{solution}



\begin{solution}[by \href{https://artofproblemsolving.com/community/user/29428}{pco}]
	\begin{tcolorbox}Consider three functions: $f_1(x)=x+\frac{1}{x}$, $f_2(x)=x^2$, $f_3(x)=(x-1)^2$. You may add, subtract and multiply these functions (in particular, you can square and raise to higher powers, and so on), multiply by an arbitrary number, add an arbitrary number to your result and perform the above described operations with the expressions obtained. Obtain $\frac{1}{x}$ in this way. Prove that if one of the functions $f_1$, $f_2$ or $f_3$ is taken out, then it is impossible to obtain $x$ in the way described. 

Proposed by M. Evdokimov.\end{tcolorbox}
$\frac 1x=\frac{2f_1-f_2+f_3-1}2$

$x=\frac{f_2-f_3+1}2$, \begin{bolded}without usage of $f_1$.\end{bolded}
\end{solution}



\begin{solution}[by \href{https://artofproblemsolving.com/community/user/153511}{Konigsberg}]
	Oh my... next time I should always proofread... I edited it again... But seriously... are you using a program or something; how do you find those so fast?
\end{solution}
*******************************************************************************
-------------------------------------------------------------------------------

\begin{problem}[Posted by \href{https://artofproblemsolving.com/community/user/211391}{Aranya}]
	Find all positive integers $n$ such that $n^2-1$ divides $3^n+5^n$
	\flushright \href{https://artofproblemsolving.com/community/c6h589845}{(Link to AoPS)}
\end{problem}



\begin{solution}[by \href{https://artofproblemsolving.com/community/user/211391}{Aranya}]
	I realized two things: 1) $n=3k$ for some positive integer $k$, and 2) $n$ is congruent to 3,5,7 modulo 10.

I am facing a doubt in trying to understand if $(3^{n-1}-5.3^{n-2}+...+5^{n-1})$ is reducible into factors or not. Till now, my idea says there can exist only one possible solution i.e. $n=3$ if this is not factorizable. And, I believe it is not factorizable :D

But suppose I write $n=3k$ then we get $3^{3k}+5^{3k}$ is reducible into THREE factors, one of which is 8. But, is there any possibility that the two factors other than 8 are also reducible into factors?
\end{solution}



\begin{solution}[by \href{https://artofproblemsolving.com/community/user/179088}{Panoz93}]
	~Deleted due to lack of validity !!
Thnx mavropnevma for noticing :)
\end{solution}



\begin{solution}[by \href{https://artofproblemsolving.com/community/user/64716}{mavropnevma}]
	\begin{tcolorbox}... Thus $p-1\mid 2n(a-b) \Rightarrow p-1\mid 2(a-b) $ ...\end{tcolorbox}
You seem to think that $p\mid n^2 -1$ implies $\gcd(p-1,n)=1$ (so you can infer the quoted result). Unfortunately, this is not the case; for example for $n=12$ and $p=13$.
\end{solution}



\begin{solution}[by \href{https://artofproblemsolving.com/community/user/167924}{utkarshgupta}]
	I think I have seen that before (the expression and not the question).
@aranya try to use algebra and roots to prove that it is not factorizable.

I will post whenever I have time.
\end{solution}



\begin{solution}[by \href{https://artofproblemsolving.com/community/user/215754}{nomik}]
	n is odd 
proof
if $n$ is even we have all prime divisors of $5^n+3^n$ of the form 
$4k+1$ but $n^2-1$ has prime divisor of the form $4k+3$ so contradiction
\end{solution}



\begin{solution}[by \href{https://artofproblemsolving.com/community/user/215754}{nomik}]
	http://www.artofproblemsolving.com/Forum/viewtopic.php?t=275139& 
here is full solution of the problem the nice and so beatiful
\end{solution}



\begin{solution}[by \href{https://artofproblemsolving.com/community/user/18418}{\u0391\u03c1\u03c7\u03b9\u03bc\u03ae\u03b4\u03b7\u03c2 6}]
	\begin{tcolorbox}Find all positive integers $n$ such that $n^2-1$ divides $3^n+5^n$\end{tcolorbox}

[hide]For $n$ even no solution.

For $n$ odd $n^2-1|3^{{n}^{2}}+5^{{n}^{2}}$   The rest is easy.[\/hide]
\end{solution}



\begin{solution}[by \href{https://artofproblemsolving.com/community/user/151851}{Mathematicalx}]
	Probably im missing obvious thing. Could you give the rest ?
\end{solution}



\begin{solution}[by \href{https://artofproblemsolving.com/community/user/187084}{ngv}]
	\begin{tcolorbox}
For $n$ odd $n^2-1|3^{{n}^{2}}+5^{{n}^{2}}$   The rest is easy.\end{tcolorbox}
The rest would be easy but you are already wrong. Let, $n=3k\pm1$(so that it is odd). Then $3|n^2-1$. But then according to your claim $3|3^{n^2}+5^{n^2}$ which is clearly impossible. :)
Edit: misunderstood the intent
\end{solution}



\begin{solution}[by \href{https://artofproblemsolving.com/community/user/151851}{Mathematicalx}]
	Clearly 3 above post is true since for odd $n$ we have $n^2-1|3^n+5^n|3^{n^2}+5^{n^2}$ but i still
cant solve the problem . Why the rest is easy?
\end{solution}
*******************************************************************************
-------------------------------------------------------------------------------

\begin{problem}[Posted by \href{https://artofproblemsolving.com/community/user/100395}{syk0526}]
	Find all functions $ f : \mathbb{R} \rightarrow \mathbb{R} $ such that
\[ f( xf(x) + 2y) = f(x^2)+f(y)+x+y-1 \]
holds for all $ x, y \in \mathbb{R}$.
	\flushright \href{https://artofproblemsolving.com/community/c6h589872}{(Link to AoPS)}
\end{problem}



\begin{solution}[by \href{https://artofproblemsolving.com/community/user/29428}{pco}]
	\begin{tcolorbox}Find all functions $ f : \mathbb{R} \rightarrow \mathbb{R} $ such that
\[ f( xf(x) + 2y) = f(x^2)+f(y)+x+y-1 \]
holds for all $ x, y \in \mathbb{R}$.\end{tcolorbox}
Let $P(x,y)$ be the assertion $f(xf(x)+2y)=f(x^2)+f(y)+x+y-1$ 

$P(0,0)$ $\implies$ $f(0)=1$

$P(x,-xf(x))$ $\implies$ $f(x^2)=xf(x)-x+1$
$P(0,x)$ $\implies$ $f(2x)=f(x)+x$

From there we'll compute two expressions for $f(4x^2)$ :
$f(2x)=f(x)+x$ $\implies$ $f(4x)=f(x)+3x$ $\implies$ $f(4x^2)=f(x^2)+3x^2$ $=xf(x)+3x^2-x+1$
$f(x^2)=xf(x)-x+1$ $\implies$ $f(4x^2)=2xf(2x)-2x+1$ $=2xf(x)+2x^2-2x+1$

So $xf(x)+3x^2-x+1=2xf(x)+2x^2-2x+1$ and so $x(f(x)-x-1)=0$ and so $f(x)=x+1$ $\forall x\ne 0$, still true when $x=0$

And so $\boxed{f(x)=x+1\text{   }\forall x}$ which indeed is a solution.
\end{solution}
*******************************************************************************
-------------------------------------------------------------------------------

\begin{problem}[Posted by \href{https://artofproblemsolving.com/community/user/154333}{BlackSelena}]
	Find all increased function $f: \mathbb{R+} \rightarrow \mathbb{R}$ such that:
$f(x+1) = f(x) + 2^{-x}$
	\flushright \href{https://artofproblemsolving.com/community/c6h590142}{(Link to AoPS)}
\end{problem}



\begin{solution}[by \href{https://artofproblemsolving.com/community/user/29428}{pco}]
	\begin{tcolorbox}Find all increased function $f: \mathbb{R+} \rightarrow \mathbb{R}$ such that:
$f(x+1) = f(x) + 2^{-x}$\end{tcolorbox}
Setting $g(x)=f(x)+2^{1-x}$, we get $g(x+1)=g(x)$ and so $f(x)=g(x)-2^{1-x}$ where $g(x)$ is a periodic function whose $1$ is a period.

It's then easy to deduce from the fact that $f(x)$ is increasing that $g(x)=c$ is constant (look behaviour when $x\to+\infty$).

Hence the answer : $\boxed{f(x)=a-2^{1-x}\text{  }\forall x>0}$ which indeed is a solution, whatever is $a\in\mathbb R$
\end{solution}
*******************************************************************************
-------------------------------------------------------------------------------

\begin{problem}[Posted by \href{https://artofproblemsolving.com/community/user/198687}{Legend-crush}]
	Determine all continuous functions such that, forall real number x and y
$f(x-y).f(x+y)=(f(x).f(y))^2$
	\flushright \href{https://artofproblemsolving.com/community/c6h590213}{(Link to AoPS)}
\end{problem}



\begin{solution}[by \href{https://artofproblemsolving.com/community/user/29428}{pco}]
	\begin{tcolorbox}Determine all continuous functions such that, forall real number x and y
$f(x-y).f(x+y)=(f(x).f(y))^2$\end{tcolorbox}
$\boxed{\text{S1 : }f(x)=0\text{   }\forall x}$ is a solution. So let us from now look only for non allzero solutions.

Let $P(x,y)$ be the assertion $f(x-y)f(x+y)=f(x)^2f(y)^2$
Let $u$ such that $f(u)\ne 0$

$P(u,0)$ $\implies$ $f(0)^2=1$. Obviously $f(x)$ solution implies $-f(x)$ solution. So WLOG $f(0)=1$

If $f(x)=0$ for some $x$, then $P(\frac x2,\frac x2)$ $\implies$ $f(x)=f(\frac x2)^4$ and so $f(\frac x2)=0$ and so $f(\frac x{2^n})=0$ which is impossible since $f(0)=1$ and $f(x)$ is continuous.
So $f(x)>0$ $\forall x$ (since continuous) and we can define $g(x)=\ln f(x)$ and functional equation becomes :

New assertion $Q(x,y)$ : $g(x+y)+g(x-y)=2g(x)+2g(y)$ where $g(x)$ is continuous and $g(0)=0$

$Q((n+1)x,x)$ $\implies$ $g((n+2)x)=2g((n+1)x)-g(nx)+2g(x)$. From there, simple induction gives $g(nx)=n^2g(x)$ and so $g(x)=x^2g(1)$ $\forall x\in\mathbb Q$

Continuity implies then $g(x)=ax^2$ $\forall x$, which indeed is a solution, whatever is $a\in\mathbb R$

Hence the two last solutions :
$\boxed{\text{S2 : }f(x)=e^{ax^2}\text{   }\forall x}$ whatever is $a\in\mathbb R$

$\boxed{\text{S3 : }f(x)=-e^{ax^2}\text{   }\forall x}$ whatever is $a\in\mathbb R$
\end{solution}
*******************************************************************************
-------------------------------------------------------------------------------

\begin{problem}[Posted by \href{https://artofproblemsolving.com/community/user/78770}{thuanspdn}]
	Find all functions $ f: R \longrightarrow R$ satisfy

$f\left ( x^{2015}+2014y \right )=5f\left ( x+y \right )+f\left ( 2015x+2013y \right ) -2015$         ,$\forall x,y\in \mathbb{R}$
	\flushright \href{https://artofproblemsolving.com/community/c6h590377}{(Link to AoPS)}
\end{problem}



\begin{solution}[by \href{https://artofproblemsolving.com/community/user/29428}{pco}]
	\begin{tcolorbox}Find all functions $ f: R \longrightarrow R$ satisfy

$f\left ( x^{2015}+2014y \right )=5f\left ( x+y \right )+f\left ( 2015x+2013y \right ) -2015$         ,$\forall x,y\in \mathbb{R}$\end{tcolorbox}
Let $P(x,y)$ be the assertion $f(x^{2015}+2014y)=5f(x+y)+f(2015x+2013y)-2015$

$P(x,2015x-x^{2015})$ $\implies$ $f(2016x-x^{2015})=403$ and since $g(x)=2016x-x^{2015}$ is surjective, we get :

$\boxed{f(x)=403}$ $\forall x$, which indeed is a solution
\end{solution}
*******************************************************************************
-------------------------------------------------------------------------------

\begin{problem}[Posted by \href{https://artofproblemsolving.com/community/user/68025}{Pirkuliyev Rovsen}]
	In the quadratic trinomial $f$ and $g$ coefficients of $x^2$ nonzero.It is known that $[f(x)]=[g(x)]$ for all real $x$.Prove that $f(x)=g(x)$ for all real $x$.
	\flushright \href{https://artofproblemsolving.com/community/c6h590406}{(Link to AoPS)}
\end{problem}



\begin{solution}[by \href{https://artofproblemsolving.com/community/user/29428}{pco}]
	\begin{tcolorbox}In the quadratic trinomial $f$ and $g$ coefficients of $x^2$ nonzero.It is known that $[f(x)]=[g(x)]$ for all real $x$.Prove that $f(x)=g(x)$ for all real $x$.\end{tcolorbox}
Let $f(x)=ax^2+bx+c$ and $g(x)=ux^2+vx+w$. We have $f(x)-g(x)\in(-1,+1)$ and so $a=u$ and $b=v$

If $c\ne w$, WLOG $c>w$. Consider $n\in\mathbb Z\cap f(\mathbb R)$ (non empty since $a\ne 0$) and let $t$ such that $f(t)=n$

Then $g(t)=f(t)-(c-w)<f(t)=n$ and so $\lfloor g(t)\rfloor<n=\lfloor f(t)\rfloor$ and so contradiction.

So $c=w$ and $f(x)=g(x)$ $\forall x$
Q.E.D.
\end{solution}
*******************************************************************************
-------------------------------------------------------------------------------

\begin{problem}[Posted by \href{https://artofproblemsolving.com/community/user/167924}{utkarshgupta}]
	Find all functions from real to real numbers such that
a) $f(1)=1$
b)$f( xy + f(x))=xf(y)+f(x)$
	\flushright \href{https://artofproblemsolving.com/community/c6h590447}{(Link to AoPS)}
\end{problem}



\begin{solution}[by \href{https://artofproblemsolving.com/community/user/29428}{pco}]
	\begin{tcolorbox}Find all functions from real to real numbers such that
a) $f(1)=1$
b)$f( xy + f(x))=xf(y)+f(x)$\end{tcolorbox}
Let $P(x,y)$ be the assertion $f(xy+f(x))=xf(y)+f(x)$

$P(1,0)$ $\implies$ $f(0)=0$

If $f(a)=f(b)=c\ne 0$, then, subtracting $P(b,a)$ from $P(a,b)$, we get $a=b$

If $\exists u\ne 0$ such that $f(u)=0$, then $P(1,u)$ $\implies$ $f(u+1)=f(1)=1$ and so $u=0$ and so contradiction.
So $f(x)=0$ $\iff$ $x=0$

Let $x\ne 0$ so that $f(x)\ne 0$. Then $P(x,0)$ $\implies$ $f(f(x))=f(x)$ and so, using previous line, $f(x)=x$, still true when $x=0$

Hence the unique solution : $\boxed{f(x)=x}$ $\forall x$, which indeed is a solution.
\end{solution}
*******************************************************************************
-------------------------------------------------------------------------------

\begin{problem}[Posted by \href{https://artofproblemsolving.com/community/user/167924}{utkarshgupta}]
	Suppose that a real valued function $f(x)$ of real numbers satisfies
$f(x+2xy)=f(x)+2f(xy)$
for all reals $x,y$, and that if $f(1991)=a$, where $a$ is a real number. Compute $f(1992)$.
	\flushright \href{https://artofproblemsolving.com/community/c6h590448}{(Link to AoPS)}
\end{problem}



\begin{solution}[by \href{https://artofproblemsolving.com/community/user/29428}{pco}]
	\begin{tcolorbox}Suppose that a real valued function $f(x)$ of real numbers satisfies
$f(x+2xy)=f(x)+2f(xy)$
for all reals $x,y$, and that if $f(1991)=a$, where $a$ is a real number. Compute $f(1992)$.\end{tcolorbox}
Let $P(x,y)$ be the assertion $f(x+2xy)=f(x)+2f(xy)$

$P(0,0)$ $\implies$ $f(0)=0$

$P(1,-\frac 12)$ $\implies$ $2f(-\frac 12)=-f(1)$

Let $x\ne 0$ : $P(x,-\frac 1{2x})$ becomes $f(x)=f(x-1)+f(1)$

So $a=f(1991)=1991f(1)$ and $\boxed{f(1992)=\frac{1992}{1991}a}$
\end{solution}
*******************************************************************************
-------------------------------------------------------------------------------

\begin{problem}[Posted by \href{https://artofproblemsolving.com/community/user/180130}{nima1376}]
	Find all functions $f:\mathbb{R}^{+}\rightarrow \mathbb{R}^{+}$ such that 
$x,y\in \mathbb{R}^{+},$ \[ f\left(\frac{y}{f(x+1)}\right)+f\left(\frac{x+1}{xf(y)}\right)=f(y) \]
	\flushright \href{https://artofproblemsolving.com/community/c6h590564}{(Link to AoPS)}
\end{problem}



\begin{solution}[by \href{https://artofproblemsolving.com/community/user/180130}{nima1376}]
	let $p(x,y)= f(\frac{y}{f(x+1)})+f(\frac{x+1}{xf(y)})=f(y)$
if we have $y$ such that $yf(y)> 1$,$x=\frac{1}{yf(y)-1}\Rightarrow f(\frac{y}{f(x+1)})=0$ which is not true.
so for all $x$ we have $f(x)\leq \frac{1}{x}$ .
$ f(y)=\frac{y}{f(x+1)})+f(\frac{x+1}{xf(y)})\leq \frac{x}{x+1}f(y)+f(\frac{y}{f(x+1)})\Rightarrow \frac{f(y)}{x+1}\leq f(\frac{y}{f(x+1)})\leq f(x+1).\frac{1}{y}\Rightarrow yf(y)\leq (x+1)f(x+1)\Rightarrow $ for all $1<x$ $f(x)=\frac{a}{x}$
$\rightarrow yf(y)\leq a\leq 1$ 
$p(x,y+1)\Rightarrow a=1$ 
 $p(x,1)\Rightarrow f(1)=1$ 
$p(x,\frac{1}{x+1}),(\frac{x+1}{xf(\frac{1}{x+1})}> 1)\Rightarrow f(\frac{1}{x+1})=x+1\Rightarrow f(x)=\frac{1}{x}$ 
so we are done
\end{solution}



\begin{solution}[by \href{https://artofproblemsolving.com/community/user/141363}{alibez}]
	i think it is not good problem for TST .  :maybe: 

similar idea : http://www.artofproblemsolving.com/Forum/viewtopic.php?f=36&t=578815&p=3415155#p3415155

and this : http://www.artofproblemsolving.com/Forum/viewtopic.php?f=38&t=576480&p=3397645#p3397645
\end{solution}



\begin{solution}[by \href{https://artofproblemsolving.com/community/user/29428}{pco}]
	\begin{tcolorbox}...$\Rightarrow f(\frac{1}{x+1})=x+1\Rightarrow f(x)=\frac{1}{x}$ 
so we are done\end{tcolorbox}I've not read the entire proof but the line above is wrong.

The good implication would be $f(\frac{1}{x+1})=x+1\Rightarrow f(x)=\frac{1}{x}$ $\boxed{\forall x>1}$ 

And so it remains to find the values for $x\in(0,1]$
\end{solution}



\begin{solution}[by \href{https://artofproblemsolving.com/community/user/180130}{nima1376}]
	i proved $ f(x)=\frac{1}{x}  $ for $x>1$
 then i proved $\Rightarrow f(\frac{1}{x+1})=x+1$ and then i prove $f(x)=\frac{1}{x}$
\end{solution}



\begin{solution}[by \href{https://artofproblemsolving.com/community/user/29428}{pco}]
	\begin{tcolorbox}...and then i prove $f(x)=\frac{1}{x}$\end{tcolorbox}
I did not see where.

If you say it, it's certainlay true ...
\end{solution}



\begin{solution}[by \href{https://artofproblemsolving.com/community/user/180130}{nima1376}]
	\begin{tcolorbox} for all $1<x$ $f(x)=\frac{a}{x}$
$p(x,y+1)\Rightarrow a=1$ 
 \end{tcolorbox}
\end{solution}



\begin{solution}[by \href{https://artofproblemsolving.com/community/user/222968}{rkm0959}]
	\begin{tcolorbox}
$p(x,\frac{1}{x+1}),(\frac{x+1}{xf(\frac{1}{x+1})}> 1)\Rightarrow f(\frac{1}{x+1})=x+1\Rightarrow f(x)=\frac{1}{x}$\end{tcolorbox}
How is $\frac{x+1}{xf(\frac{1}{x+1})} > 1$? We would require $f(\frac{1}{x+1}) < 1+\frac{1}{x}$ but that's obviously false.

Here's a solution. Denote $P(x,y)$ what it usually means.

Step 1. 
Assume that there exists an $y$ such that $yf(y)>1$. 
Now do a $P(\frac{1}{yf(y)-1},y)$. This gives a contradiction, since this forces $f(\frac{y}{f(\frac{1}{yf(y)-1}+1)}) = 0$.
Therefore, $yf(y) \le 1$ for all $y$.
This gives us $f(\frac{x+1}{xf(y)}) \le \frac{xf(y)}{x+1}$, so $\frac{f(x+1)}{y} \ge f(\frac{y}{f(x+1)}) \ge \frac{f(y)}{x+1}$.
Therefore, $(x+1)f(x+1) \ge yf(y)$, which gives us that $xf(x)=C \le 1$ for all $x>1$.

Step 2. 
We prove $C=1$. $P(x,y+1)$ gives you $f(\frac{y+1}{f(x+1)}) + f(\frac{x+1}{xf(y+1)}) = f(y+1)$.
This gives us $f(\frac{(x+1)(y+1)}{C}) + f(\frac{(x+1)(y+1)}{Cx}) = \frac{C}{y+1}$
Now since $C \le 1$, we have $\frac{(x+1)(y+1)}{C} \ge 1$ and $\frac{(x+1)(y+1)}{Cx} \ge 1$.
This gives us $\frac{C^2}{(x+1)(y+1)} + \frac{C^2x}{(x+1)(y+1)} = \frac{C}{y+1}$, which gives us $C^2=C$, or $C=1$.
Therefore, for $x>1$, we have $f(x)=\frac{1}{x}$.
$P(x,1)$ gives you $f(\frac{1}{f(x+1)})+f(\frac{x+1}{xf(1)})=f(1)$.
Since $f(1) \le 1$, we have $\frac{x+1}{xf(1)} \ge 1$.
This gives us $\frac{1}{x+1} + \frac{xf(1)}{x+1} = f(1)$, so $f(1)=1$. This gives us $f(x)=\frac{1}{x}$ for $x \ge 1$.

Step 3. 
$P(x,\frac{x}{x+1})$ gives $f(\frac{x}{(x+1)f(x+1)}) + f(\frac{x+1}{xf(\frac{x}{x+1})}) = f(\frac{x}{x+1})$.
Now note that $f(\frac{x}{x+1}) \le \frac{x+1}{x}$, so $\frac{x+1}{xf(\frac{x}{x+1})} \ge 1$. 
This gives us $f(x)+\frac{x}{x+1}f(\frac{x}{x+1}) = f(\frac{x}{x+1})$, so $(x+1)f(x)=f(\frac{x}{x+1})$.
Let $x=\frac{k}{1-k}$, and this function of $0 \le k <1$ can express all reals.
This gives us $f(k)=\frac{1}{1-k}f(\frac{k}{1-k})$. 

Step 4. 
We prove that $f(x)=\frac{1}{x}$ holds for all $x<1$. 
To do this, we break the interval $(0,1)$ into intervals $[\frac{1}{n+1},\frac{1}{n})$ for all naturals $n$.

Base Case: $n=1$. Note that for $k \in [\frac{1}{2},1)$, $\frac{k}{1-k} > 1$.
This gives us $f(k)=\frac{1}{1-k}f(\frac{k}{1-k}) = \frac{1}{1-k} \cdot \frac{1-k}{k} = \frac{1}{k}$.

Inductive Step: Assume the statement for $n=l-1$. 
Note that for $k \in [\frac{1}{l+1},\frac{1}{l})$, we have $\frac{k}{1-k} \in [\frac{1}{l},\frac{1}{l-1})$.
This gives us, with the inductive hypothesis, that $f(k)=\frac{1}{1-k}f(\frac{k}{1-k}) = \frac{1}{1-k} \cdot \frac{1-k}{k} = \frac{1}{k}$.
This completes the proof of the statement for $n=l$.

Given any real $\epsilon < 1$, it is enclosed in some interval in a form of $[\frac{1}{n+1}, \frac{1}{n})$.
Therefore, for all $x<1$, $f(x)=\frac{1}{x}$. Therefore, we conclude that $f(x)=\frac{1}{x}$. Verification is easy. GG. $\blacksquare$
\end{solution}



\begin{solution}[by \href{https://artofproblemsolving.com/community/user/243741}{anantmudgal09}]
	\begin{tcolorbox}Find all functions $f:\mathbb{R}^{+}\rightarrow \mathbb{R}^{+}$ such that 
$x,y\in \mathbb{R}^{+},$ \[ f\left(\frac{y}{f(x+1)}\right)+f\left(\frac{x+1}{xf(y)}\right)=f(y) \]\end{tcolorbox}

[hide=Solution]

[hide=Answer] We claim that the only such function is $f(x)=\frac{1}{x}$ for all $x>0$. Clearly, it satisfies the given condition, so we only need to show that no other function does. [\/hide]

[hide=Notation] Let $P(x,y)$ denote the assertion $$f\left(\frac{y}{f(x+1)}\right)+f\left(\frac{x+1}{xf(y)}\right)=f(y)$$ for all $x,y>0$.
[\/hide]

[hide=Prelim 1] \begin{bolded}Lemma 1:\end{bolded} $xf(x) \le 1$ for all $x>0$.

\begin{italicized}Proof:\end{italicized} Fix $y>0$ such that $yf(y)>1$. From $P\left(\frac{1}{yf(y)-1},y\right)$ we conclude that $$f\left(\frac{y}{f\left(1+\frac{1}{yf(y)-1}\right)}\right)=0$$ contradicting that $0$ is not in the co-domain of $f$. $\square$

[\/hide]

[hide=Prelim 2] \begin{bolded}Lemma 2:\end{bolded} $f(a)=\frac{1}{a}$ for all $a>1$.

\begin{italicized}Proof:\end{italicized} Fix reals $x,z>0$ and $y=1+z>1$. From the fact that $tf(t) \le 1$ for all $t$, $P(x,y)$ yields $$\frac{f(x+1)}{y}+\frac{xf(y)}{x+1} \ge f\left(\frac{y}{f(x+1)}\right)+f\left(\frac{x+1}{xf(y)}\right)=f(y) \Longrightarrow (x+1)f(x+1) \ge (1+z)f(1+z).$$ The same argument with $P(y,x)$ yields that $(1+z)f(1+z) \ge (1+x)f(1+x)$, whence, $af(a)=c \le 1$ for all $a>1$ where $c$ is a constant. Since the equality holds in the last equation, we have $$f\left(\frac{y}{f(x+1)}\right)=\frac{f(x+1)}{y} \Longrightarrow \frac{c^2}{y(1+x)}=\frac{c}{(1+x)y} \Longrightarrow c=1. \, \square$$

[\/hide]

[hide=Prelim 3] \begin{bolded}Lemma 3:\end{bolded} $f$ is strictly decreasing.

\begin{italicized}Proof: \end{italicized} Note that by \begin{bolded}Lemma 2\end{bolded} the assertion $P(x,y)$ can be rephrased as $$f(y(1+x))+f\left(\frac{x+1}{xf(y)}\right)=f(y).$$ For any $z>y$ set $x=\frac{z-y}{y}>0$ and we have $f(z)<f(y)$ as desired. $\square$
[\/hide]

[hide=Prelim 4] \begin{bolded}Lemma 4:\end{bolded} $f(a)=\frac{1}{a}$ for all $a \ge \frac{1}{2}$.

\begin{italicized}Proof:\end{italicized} Note that for any $\epsilon>0$ we have $\frac{1}{1+\epsilon}=f(1+\epsilon)<f(1) \le 1$ and by taking $\epsilon \rightarrow 0^+$ we get that $f(1)=1$. Pick $\frac{1}{2}<y<1$ and note that $f(y)>f(1)=1$. Note that $$y>1-y \Longrightarrow f(y) \le \frac{1}{y}<\frac{1}{1-y} \Longrightarrow \left(1-\frac{1}{y}\right)<\frac{1}{f(y)-1}$$ and so we can choose a positive real $x$ such that $$\left(1-\frac{1}{y}\right)<x<\frac{1}{f(y)-1} \Longrightarrow \frac{x+1}{xf(y)}>1, \qquad y(1+x)>1 \Longrightarrow f(y)=\frac{1}{y}$$ for all $y>\frac{1}{2}$ wherein the last implication holds by \begin{bolded}Lemma 2.\end{bolded} Finally, we have $$\frac{2}{1+2\epsilon}=f\left(\frac{1}{2}+\epsilon \right)<f\left(\frac{1}{2}\right) \le 2$$ and letting $\epsilon \rightarrow 0^+$ we get that $f\left(\frac{1}{2}\right)=2$. $\square$

[\/hide]

[hide=Endgame] \begin{bolded}Conclusion:\end{bolded} Iterating the argument applied in \begin{bolded}Lemma 4,\end{bolded} one may by induction on $n \ge 1$ that for all $y \in \left[\frac{1}{2^n}, \infty \right)$, $f(y)=\frac{1}{y}$. From this we conclude that $f(x)=\frac{1}{x}$ for all $x>0$. $\square$


[\/hide]

[\/hide]
\end{solution}
*******************************************************************************
-------------------------------------------------------------------------------

\begin{problem}[Posted by \href{https://artofproblemsolving.com/community/user/153511}{Konigsberg}]
	Find all functions $f: \mathbb{N} \rightarrow \mathbb{N}$ such that

(a) $f(1)=1$
(b) $f(n+2)+(n^2+4n+3)f(n)=(2n+5)f(n+1)$ for all $n \in \mathbb{N}$. 
(c) $f(n)$ divides $f(m)$ if $m>n$.
	\flushright \href{https://artofproblemsolving.com/community/c6h590835}{(Link to AoPS)}
\end{problem}



\begin{solution}[by \href{https://artofproblemsolving.com/community/user/29428}{pco}]
	\begin{tcolorbox}Find all functions $f: \mathbb{N} \rightarrow \mathbb{N}$ such that

(a) $f(1)=1$
(b) $f(n+2)+(n^2+4n+3)f(n)=(2n+5)f(n+1)$ for all $n \in \mathbb{N}$. 
(c) $f(n)$ divides $f(m)$ if $m>n$.\end{tcolorbox}
$f(3)=7f(2)-8$ and so $f(2)>1$ and $f(2)|8$ and so $f(2)\in\{2,4,8\}$

If $f(2)=2$, simple induction implies $\boxed{f(n)=n!}$ which indeed is a solution

If $f(2)=4$, simple induction implies $\boxed{f(n)=\frac{(n+2)!}6}$ which indeed is a solution

If $f(2)=8$, we get $f(3)=48$ and $f(4)=312$ which is not a solution since $f(3)\not|f(4)$
\end{solution}
*******************************************************************************
-------------------------------------------------------------------------------

\begin{problem}[Posted by \href{https://artofproblemsolving.com/community/user/153511}{Konigsberg}]
	Deos there exist a function $f: \mathbb{R} \rightarrow \mathbb{R}$ such that for all $x$, $y \in \mathbb{R}$, 

$f(x^2y+f(x+y^2))=x^3+y^3+f(xy)$
	\flushright \href{https://artofproblemsolving.com/community/c6h590837}{(Link to AoPS)}
\end{problem}



\begin{solution}[by \href{https://artofproblemsolving.com/community/user/29428}{pco}]
	\begin{tcolorbox}Deos there exist a function $f: \mathbb{R} \rightarrow \mathbb{R}$ such that for all $x$, $y \in \mathbb{R}$, 

$f(x^2y+f(x+y^2))=x^3+y^3+f(xy)$\end{tcolorbox}
Let $P(x,y)$ be the assertion $f(x^2y+f(x+y^2))=x^3+y^3+f(xy)$

The polynomial $(x^2+1)^2x+(x^2+1)x+f(-1)=0$ has degree $5$ and so always has at least one real root. Let then $u$ such a root.

$P(-1-u^2,u)$ $\implies$ $(-1-u^2)^3+u^3=0$ and so $u^2+1=u$ which is impossible.

So no solution for this functional equation.
\end{solution}



\begin{solution}[by \href{https://artofproblemsolving.com/community/user/29428}{pco}]
	\begin{tcolorbox}Deos there exist a function $f: \mathbb{R} \rightarrow \mathbb{R}$ such that for all $x$, $y \in \mathbb{R}$, 

$f(x^2y+f(x+y^2))=x^3+y^3+f(xy)$\end{tcolorbox}
Another simpler way :

Let $P(x,y)$ be the assertion $f(x^2y+f(x+y^2))=x^3+y^3+f(xy)$

$P(4,0)$ $\implies$ $f(f(4))=64+f(0)$
$P(0,2)$ $\implies$ $f(f(4))=8+f(0)$

So contradiction and no such function.
\end{solution}
*******************************************************************************
-------------------------------------------------------------------------------

\begin{problem}[Posted by \href{https://artofproblemsolving.com/community/user/167924}{utkarshgupta}]
	Let $m \ge 1$ and $f : [m;\infty ) \to [1; \infty)$, $f(x) = x^2 - 2mx + m^2 + 1$.
(a) Prove that f is bijective.
(b) Solve the equation $f(x) = f^{-1}(x)$.
(c) Solve the equation $x^2 -2mx + m^2 + 1 = m + \sqrt{x-1}$
	\flushright \href{https://artofproblemsolving.com/community/c6h590869}{(Link to AoPS)}
\end{problem}



\begin{solution}[by \href{https://artofproblemsolving.com/community/user/29428}{pco}]
	\begin{tcolorbox}Let $m \ge 1$ and $f : [m;\infty ) \to [1; \infty)$, $f(x) = x^2 - 2mx + m^2 + 1$.
(a) Prove that f is bijective.
(b) Solve the equation $f(x) = f^{-1}(x)$.
(c) Solve the equation $x^2 -2mx + m^2 + 1 = m + \sqrt{x-1}$\end{tcolorbox}
(a) $\forall x\ge 1$, $f(m+\sqrt{x-1})=x$ and so $f(x)$ is bijective

(b) $f(x)$ is increasing convex while $f^{-1}(x)$ is increasing concave and so at most two solutions for equation $f(x)=f^{-1}(x)$
Equation $f(x)=x$ always has at least a solution $x_m=\frac{2m+1+\sqrt{4m-3}}2>m$
If $m>1$, it's easy to show that $x_m$ is the unique root of $f(x)=f^{-1}(x)$
If $m=1$, it's easy to show that we have a second toot $x=1$

(c) equation is equivalent to $x^2-2mx+m^2-m+1=\sqrt{x-1}$

$\iff$ $x^2-2mx+m^2-m+1\ge 0$ and $(x^2-2mx+m^2-m+1)^2=x-1$

$\iff$ $(x-m)^2\ge m-1$ and $x^4-4mx^3+(6m^2-2m+2)x^2-(4m^3-4m^2+4m+1)x+(m^2-m+1)^2+1=0$

We know that solutions of $f(x)-x=0$ are solutions, and so quartic is divisble by $x^2-(2m+1)x+m^2+1$ and factorization of quartic becomes then easy.

So $(x-m)^2\ge m-1$  and $(x^2-(2m+1)x+m^2+1)(x^2-(2m-1)x +m^2-2m+2)=0$

So $(x-m)^2\ge m-1$  and $x\in\left\{\frac{2m+1\pm\sqrt{4m-3}}2,\frac{2m-1\pm\sqrt{4m-7}}2\right\}$ whence these values are defined.

Hence the solutions :
If $m=1$, then two solutions : $x\in\{1,2\}$

If $m\in(1,2)$, then one solution $x=\frac{2m+1+\sqrt{4m-3}}2$

If $m\in[2,+\infty)$, then two solutions $x\in\left\{\frac{2m+1+\sqrt{4m-3}}2,\frac{2m-1-\sqrt{4m-7}}2\right\}$
\end{solution}
*******************************************************************************
-------------------------------------------------------------------------------

\begin{problem}[Posted by \href{https://artofproblemsolving.com/community/user/211321}{malgnaig}]
	1,Find all continuous $R$->$R$ such that:
$f(x)+f(y)+2= 2f(\frac{x+y}{2})+2f(\frac{x-y}{2})$
	\flushright \href{https://artofproblemsolving.com/community/c6h590981}{(Link to AoPS)}
\end{problem}



\begin{solution}[by \href{https://artofproblemsolving.com/community/user/29428}{pco}]
	\begin{tcolorbox}1,Find all continuous $R$->$R$ such that:
$f(x)+f(y)+2= 2f(\frac{x+y}{2})+2f(\frac{x-y}{2})$\end{tcolorbox}
Let $f(x)=g(x)+1$ and functional equation may be written (moving $x\to2x$ and $y\to 2y$) :
Assertion $P(x,y)$ : $g(x+y)+g(x-y)=\frac{g(2x)+g(2y)}2$

$P(0,0)$ $\implies$ $g(0)=0$
$P(x,0)$ $\implies$ $g(2x)=4g(x)$ and $P(x,y)$ becomes new assertion $Q(x,y)$ : $g(x+y)+g(x-y)=2g(x)+2g(y)$

So $Q((n+1)x,x)$ $\implies$ $g((n+2)x)=2g((n+1)x)-g(nx)+2g(x)$ which is an easy to solve recurrence giving $g(nx)=n^2g(x)$

So $g(x)=g(1)x^2$ $\forall x\in\mathbb Q$ and continuity gives $g(x)=ax^2$ $\forall x$ \begin{bolded}* edited a typo : $ax\to ax^2$ *\end{bolded}

Hence the result :$\boxed{f(x)=ax^2+1\text{  }\forall x}$ which indeed is a solution, whatever is $a\in\mathbb R$
\end{solution}



\begin{solution}[by \href{https://artofproblemsolving.com/community/user/167258}{shaiephraim}]
	\begin{tcolorbox}[quote="malgnaig"]1,Find all continuous $R$->$R$ such that:
$f(x)+f(y)+2= 2f(\frac{x+y}{2})+2f(\frac{x-y}{2})$\end{tcolorbox}
Let $f(x)=g(x)+1$ and functional equation may be written (moving $x\to2x$ and $y\to 2y$) :
Assertion $P(x,y)$ : $g(x+y)+g(x-y)=\frac{g(2x)+g(2y)}2$

$P(0,0)$ $\implies$ $g(0)=0$
$P(x,0)$ $\implies$ $g(2x)=4g(x)$ and $P(x,y)$ becomes new assertion $Q(x,y)$ : $g(x+y)+g(x-y)=2g(x)+2g(y)$

So $Q((n+1)x,x)$ $\implies$ $g((n+2)x)=2g((n+1)x)-g(nx)+2g(x)$ which is an easy to solve recurrence giving $g(nx)=n^2g(x)$

So $g(x)=g(1)x^2$ $\forall x\in\mathbb Q$ and continuity gives $g(x)=ax$ $\forall x$

Hence the result :$\boxed{f(x)=ax^2+1\text{  }\forall x}$ which indeed is a solution, whatever is $a\in\mathbb R$\end{tcolorbox}
when can we assert a function is a polynomial?
\end{solution}



\begin{solution}[by \href{https://artofproblemsolving.com/community/user/29428}{pco}]
	\begin{tcolorbox}when can we assert a function is a polynomial?\end{tcolorbox}
I dont understand the question.
In my proof, I never asert that the function is a polynomial. I just proved it.

In a general way, we can never assert that a function is a polynomial if this is not explicitely stated in the problem sentences.
\end{solution}
*******************************************************************************
-------------------------------------------------------------------------------

\begin{problem}[Posted by \href{https://artofproblemsolving.com/community/user/96532}{dgrozev}]
	Find all functions $f: \mathbb{Q}^+ \to \mathbb{R}^+ $ with the property:
\[f(xy)=f(x+y)(f(x)+f(y)) \,,\, \forall x,y \in \mathbb{Q}^+\]

\begin{italicized}Proposed by Nikolay Nikolov\end{italicized}
	\flushright \href{https://artofproblemsolving.com/community/c6h591459}{(Link to AoPS)}
\end{problem}



\begin{solution}[by \href{https://artofproblemsolving.com/community/user/29428}{pco}]
	\begin{tcolorbox}Find all functions $f: \mathbb{Q}^+ \to \mathbb{R}^+ $ with the property:
\[f(xy)=f(x+y)(f(x)+f(y)) \,,\, \forall x,y \in \mathbb{Q}^+\]\end{tcolorbox}
Let $P(x,y)$ be the assertion $f(xy)=f(x+y)(f(x)+f(y))$
Let $a=f(1)$

$P(1,1)$ $\implies$ $f(2)=\frac 12$
$P(2,1)$ $\implies$ $f(3)=\frac 1{2a+1}$
$P(3,1)$ $\implies$ $f(4)=\frac 1{2a^2+a+1}$
$P(4,1)$ $\implies$ $f(5)=\frac 1{2a^3+a^2+a+1}$

$P(2,3)$ $\implies$ $f(6)=f(5)(f(2)+f(3))$
$P(5,1)$ $\implies$ $f(5)=f(6)(f(1)+f(5)$ and so $(f(2)+f(3))(f(1)+f(5))=1$

Using values got previously in this equation and simplifying, we get $(a-1)(a+1)(2a-1)(2a^2+a+1)=0$ and so $a\in\{\frac 12,1\}$

1) If $a=1$
=========
$P(n,1)$ $\implies$ $\frac 1{f(n+1)}=\frac 1{f(n)}+1$ and a simple induction gives $f(n)=\frac 1n$

$P(x+n,1)$ $\implies$ $\frac 1{f(x+n+1}=1+\frac 1{f(x+n)}$ and a simple induction gives then $f(x+n)=\frac{f(x)}{nf(x)+1}$

$P(x,n)$ $\implies$ $f(nx)=\frac{f(x)}n$ and so $f(\frac pq)=\frac qp$ and so :

$\boxed{\text{S1 : }f(x)=\frac 1x\text{  }\forall x\in\mathbb Q^+}$ which indeed is a solution.


2) If $a=\frac 12$
==============
$P(x,1)$ $\implies$ $f(x+1)=\frac{2f(x)}{2f(x)+1}$ and a simple induction implies $f(n)=\frac 12$

This also implies $f(x+2)=\frac {4f(x)}{6f(x)+1}$ and $f(x+4)=\frac{16f(x)}{30f(x)+1}$

Then $P(x,2)$ $\implies$ $f(2x)=\frac{2f(x)(2f(x)+1)}{6f(x)+1}$ and $P(2x,2)$ $\implies$ $f(4x)=\frac{4f(x)(2f(x)+1)(8f(x)^2+10f(x)+1)}{(6f(x)+1)(24f(x)^2+18f(x)+1)}$

But $P(x,4)$ $\implies$ $f(4x)=\frac{8f(x)(2f(x)+1)}{30f(x)+1}$

Equating these two expressions of $f(4x)$, we get $(2f(x)-1)^2(12f(x)+1)=0$ and so :

$\boxed{\text{S2 : }f(x)=\frac 12\text{  }\forall x\in\mathbb Q^+}$ which indeed is a solution.
\end{solution}



\begin{solution}[by \href{https://artofproblemsolving.com/community/user/285}{harazi}]
	Now let's make this more challenging (and this was the original problem of Nikolai Nikolov): same question with rational numbers replaced with positive real numbers.
\end{solution}



\begin{solution}[by \href{https://artofproblemsolving.com/community/user/96532}{dgrozev}]
	Yes, the author of the problem is Nikolay Nikolov. It was given at the competition in the form as it have been posted here. To the extent that I have seen, the more complicated version was: 
Find all functions $ f: \mathbb{R}^+ \to \mathbb{R}$ with the same property.
\end{solution}



\begin{solution}[by \href{https://artofproblemsolving.com/community/user/285}{harazi}]
	Let's find the functions $f:(0,\infty)\to \mathbf{R}$ such that 
  \[f(xy)=f(x+y)(f(x)+f(y)).\] This is the original version of the problem, apparently submitted to IMO last year and rejected since considered too hard. My solution is pretty long.
  
   Step 1: Let $Z$ be the set of zeros of $f$ and suppose that $S$ is nonempty and 
   $f$ is not the zero map. Observe that:
   
   1) If $x,y\in S$, then $xy\in S$. 
   
   2) If $x\in S$, then $u\in S$ for all $u\in (0, x^2\/4]$. Indeed, for any such $u$ we can find 
   $0<y<x$ such that $ u=y(x-y)$ and then $f(u)=f(x)(f(y)+f(x-y))=0$. 
   
     Hence, if $x\in S$ and $n\geq 1$, then $(0, x^{2n}\/4]\subset S$. If $x>1$, then by choosing 
     $n$ very large we obtain that $(0, A]\subset S$ for all $A>0$, so $f=0$, contradiction. Hence 
     $S\subset (0, 1]$. Observation 2) above shows that we can find $y_1<y_2$ in $S$.
     Set $x=y_2\/y_1>1$, so that $f(y_2)=f(xy_1)=f(x+y_1)f(x)\ne 0$ (because $x>1$ and $x+y_1>1$), a contradiction. 
     
     This shows that if $f$ vanishes at some point, then $f$ is the zero map. From now on I will suppose that 
     $f$ is not the zero map. Since $f(1)=2f(2) f(1)$, we have $f(2)=1\/2$.
     
    Step 2. I claim that $f(1)\in \{1, -1, 1\/2\}$. Let $\alpha=f(1)\ne 0$ and suppose that 
    $\alpha\ne 1, 1\/2$. Set $x_n=1\/f(n)$. Taking $x=n, y=1$, we obtain 
    $f(n)=f(n+1)(f(n)+\alpha)$, hence $x_{n+1}=1+\alpha x_n$.  This gives us explicit values for 
    $x_n$ in terms of $\alpha$. In particular, we find $x_3=1+2\alpha$,
    $x_5=1+\alpha+\alpha^2+2\alpha^3$ and $x_6=1+\alpha+\alpha^2+\alpha^3+2\alpha^4$.
    Now, we have \[f(6)=f(2\cdot 3)=f(5)( 1\/2+f(3)).\]
    Writing this in terms of $x_3, x_5, x_6$ and using the above formulae we arrive at the nasty equation 
    \[ 3\alpha^3+\alpha^2+\alpha= 4\alpha^5+1.\]
     This factors as \[ (\alpha^2-1)(2\alpha-1)(2\alpha^2+\alpha+1)=0\]
     and finishes the proof of step $2$. 
     
     Step 3. We consider the case $f(1)=1$. Then $f(x+1)=\frac{f(x)}{f(x)+1}$ (take $y=1$),
     so $f(x+2)=\frac{f(x)}{2f(x)+1}$. Taking $y=2$ we obtain 
     \[f(2x)=f(x+2)(f(x)+1\/2)=\frac{f(x)}{2}.\]
     Taking $y=x$ and using the previous relation we also obtain $f(x^2)=f(x)^2$. In particular 
     $f(x)>0$ for all $x$. Next, by induction we have $f(x+n)=\frac{f(x)}{nf(x)+1}<\frac{1}{n}$ and $f(n)=1\/n$.
     which shows that $f(x)\leq 1\/n$ for $x\geq n$. Hence for all $x\geq 1$ we have $f(x)\leq 1\/[x]$. 
     But then for all $x>1$ we have \[f(x)^{2^n}=f(x^{2^n})\leq 1\/[x^{2^n}].\]
       Taking the $2^n$th root and passing to the limit we finally obtain $f(x)\leq 1\/x$ for 
       $x\geq 1$. But since $f(2^n x)= f(x)\/2^n$, we get $f(x)\leq 1\/x$ for all $x\geq 2^{-n}$ and this for all 
       $n$, hence $f(x)\leq 1\/x$ for all $x$. Combined with the relation 
       \[ 1=f(1)= f(x+1\/x)(f(x)+f(1\/x))\]
       this immediately implies $f(x)=1\/x$ for all $x$. So we get another solution, the map 
       $x\to 1\/x$, unique solution for which $f(1)=1$.
       
       Step 4.  We consider the case $f(1)=1\/2$. This time 
       $f(x+1)= \frac{f(x)}{f(x)+1\/2}$ and by induction we get 
       \[f(x+n)=\frac{2^n f(x)}{2 (2^n-1) f(x)+1},\]
       in particular $f(n)=1\/2$ for positive integers $n$ and 
       $\lim_{n\to\infty} f(x+n)=1\/2$. Next, we have        
       $f(nx)= f(x+n)(f(x)+1\/2)$, and this tends to $ 1\/2 (f(x)+1\/2)$ when 
       $n\to\infty$. Hence for all positive integers $m$ we have 
       \[ 1\/2 ( f(mx)+1\/2)=\lim_{n\to\infty} f(mnx)=\lim_{n\to\infty} f(nx)=
       1\/2(f(x)+1\/2),\]
        hence $f(mx)=f(x)$ for all $x$ and all positive integers $m$. Making again 
        $m\to\infty$ we obtain $f(x)=\lim_{m} f(mx)= 1\/2 (f(x)+1\/2)$, hence 
        $f(x)=1\/2$ for all $x$. This gives another solution to the problem.
        
       Step 5. I consider the case $f(1)=-1$ and I claim that there is no solution in this case. 
       Now, we have $f(x+1)= \frac{f(x)}{f(x)-1}$ and replacing $x$ by $x+1$ we obtain
       $f(x)=f(x+2)$. This gives in particular $f(4)=f(2)=1\/2$ and also 
       \[ f(2x)= f(x+2)(f(x)+1\/2)=f(x) (f(x)+1\/2),  f(4x)=f(x+4)(f(x)+1\/2)=f(x) (f(x)+1\/2).\]
       Hence $f(2x)=f(4x)$ for all $x$ and so $f(x)=f(2x)=f(x)(f(x)+1\/2)$ for all $x$, which gives 
       $f(x)+1\/2=1$ for all $x$, a contradiction.
\end{solution}



\begin{solution}[by \href{https://artofproblemsolving.com/community/user/20399}{navid}]
	Yes Harazi you are right , it proposed to the last year IMO , and I don't know why it not selected , As I see this try to define function  \begin{bolded}g(x)= 1\/f(x)\end{bolded} ....My approach is same as you except for finding f(1) , which I used Geometric series ..... . 

After this problem this problem was proposed to Brazilian math Olympiads 
http://www.artofproblemsolving.com/Forum/viewtopic.php?f=38&t=559598&p=3256076&hilit=Brazil+2013#p3256076
and Also this problem  proposed to ELMO-2012 (Apparently before the propositions to the IMO).
http://www.artofproblemsolving.com/Forum/viewtopic.php?p=2728445&sid=e8a07fecc683143cc542e3537478bec5#p2728445
\end{solution}
*******************************************************************************
-------------------------------------------------------------------------------

\begin{problem}[Posted by \href{https://artofproblemsolving.com/community/user/78770}{thuanspdn}]
	Deternine all functions $f : \mathbb{R}\rightarrow \mathbb{R}$ that satisfy the two functional equations 
$ f\left (  x+y\right )=f\left ( x \right )+f\left ( y \right )$ and  $f\left ( x.y \right )=f\left ( x \right ).f\left ( y \right )$
for all $x,y \in \mathbb{R}$
	\flushright \href{https://artofproblemsolving.com/community/c6h591642}{(Link to AoPS)}
\end{problem}



\begin{solution}[by \href{https://artofproblemsolving.com/community/user/167924}{utkarshgupta}]
	I have reached somewhere at least.
For f(x) be an identity function,
Clearly $f(x)=0$ is a solution.


Let $f(x)$ not be a constant function and let for some $s \neq 0$ $f(s)=0$.
Then $f(sy)=f(s)f(y)=0$ for all $y$ implying $f(x)$ is identically $0$.
A contradiction!

Thus only $f(0)=0$
Also $f(0)=f(x)+f(-x)$ i.e. $f(x)=f(-x)$.
$f(1)=(f(1))^2$ Thus $f(1)=1$
An easy induction shows that $f(nx)=nf(x)$ where $n \in N$ (from left function)
That is $f(n)=n$ for $n \in N$
Now for $n \in N$ $f(nx)=f(n)f(x)=nf(x)$ where $x \in Q^{+}$ and $nx \in N$
That is $f(x)=\frac{f(nx)}{n}=\frac{nx}{n}=x$.

Thus we have shown $f(x)=x$ for all $x \in Q$ (since also we have $f(x)=-f(x)$
Again an easy induction on the right function shows that $f(x^n)=(f(x))^n$ for $n \in  N$.
Now for $x$ and $n$ where $x^n$ is an integer we have $f(x)=f(x^n)^{\frac{1}{n}}={x^n}^{\frac{1}{n}}=x$
I think this shows the solution.

AM I RIGHT?
What about transcendental numbers?
\end{solution}



\begin{solution}[by \href{https://artofproblemsolving.com/community/user/167924}{utkarshgupta}]
	Can anyone help?
\end{solution}



\begin{solution}[by \href{https://artofproblemsolving.com/community/user/31919}{tenniskidperson3}]
	A problem is that $f(\sqrt{2})^2=f(2)$, but is $f(\sqrt{2})=\sqrt{2}$ or $-\sqrt{2}$?  The point is that it can't be $-\sqrt{2}$ because $f(\sqrt[4]{2})^2=f(\sqrt{2})$, so $f(\sqrt{2})\geq 0$.  This is how you prove it for transcendental numbers, too: for any $x\geq 0$, say $y^2=x$, then $0\leq f(y)^2=f(y^2)=f(x)$.  And then $f(x+z)=f(x)+f(z)\geq f(z)$ if $x\geq 0$, so $f$ is an increasing function.

Now you use the fact that every real is a supremum and infimum of sets of rationals, and pick a sequence of increasing rationals and a sequence of decreasing rational numbers both converging to some real $x$.  You know that applying $f(x)$ must be trapped between the numbers obtained when applying $f$ to each of these sequences, and the only such real number is $x$.  Thus $f(x)=x$ for any arbitrary real number.
\end{solution}



\begin{solution}[by \href{https://artofproblemsolving.com/community/user/29428}{pco}]
	\begin{tcolorbox}Deternine all functions $f : \mathbb{R}\rightarrow \mathbb{R}$ that satisfy the two functional equations 
$ f\left (  x+y\right )=f\left ( x \right )+f\left ( y \right )$ and  $f\left ( x.y \right )=f\left ( x \right ).f\left ( y \right )$
for all $x,y \in \mathbb{R}$\end{tcolorbox}
Second equation means $f(x)\ge 0$ $\forall x\ge 0$ and so we have an additive locally lower bounded function, so continuous.

hence the only two solutions $f(x)=0$ $\forall x$ and $f(x)=x$ $\forall x$
\end{solution}



\begin{solution}[by \href{https://artofproblemsolving.com/community/user/167924}{utkarshgupta}]
	I saw the problem somewhere else too.
I think that book cited sandwich theorem.
Something involving simple contradiction.
I will post the full solution as soon as I have ample time.
\end{solution}
*******************************************************************************
-------------------------------------------------------------------------------

\begin{problem}[Posted by \href{https://artofproblemsolving.com/community/user/78770}{thuanspdn}]
	Find all functions $f,g : \mathbb{R}\rightarrow \mathbb{R}$ that satisfying the functional equation
$ f\left (  x+y\right )=f\left ( x \right )g\left ( y \right ) +  f\left (y \right )$ for all $x,y \in \mathbb{R}$.
	\flushright \href{https://artofproblemsolving.com/community/c6h591643}{(Link to AoPS)}
\end{problem}



\begin{solution}[by \href{https://artofproblemsolving.com/community/user/29428}{pco}]
	\begin{tcolorbox}Find all functions $f,g : \mathbb{R}\rightarrow \mathbb{R}$ that satisfying the functional equation
$ f\left (  x+y\right )=f\left ( x \right )g\left ( y \right ) +  f\left (y \right )$ for all $x,y \in \mathbb{R}$.\end{tcolorbox}
$\boxed{\text{S1 : }f(x)=0\text{   }\forall x\text{ and any }g(x)}$ is always a solution.
So let us from now look only for non allzero functions $f(x)$

Let $P(x,y)$ be the assertion $f(x+y)=f(x)g(y)+f(y)$
Let $u$ such that $f(u)\ne 0$

Subtracting $P(u,x)$ from $P(x,u)$, we get $g(x)=af(x)+1$ for some $a=\frac{g(u)-1}{f(u)}$

If $a=0$, we get $g(x)=1$ $\forall x$ and $f(x+y)=f(x)+f(y)$ and the solution :
$\boxed{\text{S2 : }f(x)=\text{ any additive function and }g(x)=1\text{   }\forall x}$

If $a\ne 0$, we get $g(x+y)=g(x)g(y)$ whose solutions are very classical  : $g(x)=0$ $\forall x$ and $g(x)=e^{u(x)}$ where $u(x)$ is any additive function.

Hence the solutions :
$\boxed{\text{S3 : }f(x)=c\text{   }\forall x\text{ and }g(x)=0\text{   }\forall x}$ which indeed is a solution, whatever is $c \in\mathbb R$

$\boxed{\text{S4 : }f(x)=c(e^{u(x)}-1)\text{   }\forall x\text{ and }g(x)=e^{u(x)}\text{   }\forall x}$ which indeed is a solution, whatever is $u(x)$, additive function.
\end{solution}
*******************************************************************************
-------------------------------------------------------------------------------

\begin{problem}[Posted by \href{https://artofproblemsolving.com/community/user/78770}{thuanspdn}]
	Deternine all functions $f : \mathbb{R}\rightarrow \mathbb{R}$ that satisfy the functional equation
$ f (  x+y + \lambda xy)=f ( x  ). f ( y )$  , $\forall x,y\in \mathbb{R}$
where $\lambda$ is a real constant
	\flushright \href{https://artofproblemsolving.com/community/c6h591644}{(Link to AoPS)}
\end{problem}



\begin{solution}[by \href{https://artofproblemsolving.com/community/user/29428}{pco}]
	\begin{tcolorbox}Deternine all functions $f : \mathbb{R}\rightarrow \mathbb{R}$ that satisfy the functional equation
$ f (  x+y + \lambda xy)=f ( x  ). f ( y )$  , $\forall x,y\in \mathbb{R}$
where $\lambda$ is a real constant\end{tcolorbox}
If $\lambda=0$, we get the trivial $\boxed{\text{S1 :}f(x)=\text{ any additive function}}$

If $\lambda\ne 0$, writing $f(x)=g(\lambda x+1)$, we get $g(xy)=g(x)g(y)$ which is very classical with four solutions :
$g(x)=0$ $\forall x$ and so $\boxed{\text{S2 : }f(x)=0\text{   }\forall x}$

$g(x)=1$ $\forall x$ and so $\boxed{\text{S3 : }f(x)=1\text{   }\forall x}$

$g(0)=0$ and $g(x)=|x|^a$ $\forall x\ne 0$ and so :
$\boxed{\text{S3 : }f(-\frac 1{\lambda})=0\text{ and }f(x)=|\lambda x+1|^a\text{   }\forall x\ne -\frac 1{\lambda}}$ which indeed is a solution, whatever is $a\in\mathbb R$

$g(0)=0$ and $g(x)=\text{sign}(x)|x|^a$ $\forall x\ne 0$ and so :
$\boxed{\text{S3 : }f(-\frac 1{\lambda})=0\text{ and }f(x)=\text{sign}(\lambda x + 1)|\lambda x+1|^a\text{   }\forall x\ne -\frac 1{\lambda}}$ which indeed is a solution, whatever is $a\in\mathbb R$
\end{solution}
*******************************************************************************
-------------------------------------------------------------------------------

\begin{problem}[Posted by \href{https://artofproblemsolving.com/community/user/78770}{thuanspdn}]
	Find all functions $f : \mathbb{R}\rightarrow \mathbb{R}$ that satisfy the functional equation
$ f ( \sqrt{x^{2}+y^{2}+1})=f ( x  )+ f ( y )$  
$\forall x,y\in \mathbb{R}$
	\flushright \href{https://artofproblemsolving.com/community/c6h591645}{(Link to AoPS)}
\end{problem}



\begin{solution}[by \href{https://artofproblemsolving.com/community/user/29428}{pco}]
	\begin{tcolorbox}Find all functions $f : \mathbb{R}\rightarrow \mathbb{R}$ that satisfy the functional equation
$ f ( \sqrt{x^{2}+y^{2}+1})=f ( x  )+ f ( y )$  
$\forall x,y\in \mathbb{R}$\end{tcolorbox}
$f(x)$ is even. Let then $g(x)$ from $[1,+\infty)\to \mathbb R$ as $g(x)=f(\sqrt{x-1})=f(-\sqrt{x-1})$ and we get $g(x+y)=g(x)+g(y)$

Hence the solution : $\boxed{f(x)=g(x^2+1)}$ where $g(x)$ is any additive function over $[1,+\infty)$
\end{solution}
*******************************************************************************
-------------------------------------------------------------------------------

\begin{problem}[Posted by \href{https://artofproblemsolving.com/community/user/78770}{thuanspdn}]
	Find  all functions $f : \mathbb{R}\rightarrow \mathbb{R}$ that satisfy the  functional equation 
$ f\left (  x+xy\right )=f\left ( x \right )+ f\left ( x\right ).f\left ( y\right)$  
for all $x,y \in \mathbb{R}$
	\flushright \href{https://artofproblemsolving.com/community/c6h591646}{(Link to AoPS)}
\end{problem}



\begin{solution}[by \href{https://artofproblemsolving.com/community/user/29428}{pco}]
	\begin{tcolorbox}Find  all functions $f : \mathbb{R}\rightarrow \mathbb{R}$ that satisfy the  functional equation 
$ f\left (  x+xy\right )=f\left ( x \right )+ f\left ( x\right ).f\left ( y\right)$  
for all $x,y \in \mathbb{R}$\end{tcolorbox}
Let $P(x,y)$ be the assertion $f(x+xy)=f(x)+f(x)f(y)$

$P(0,0)$ $\implies$ $f(0)=0$

If $f(-1)\ne -1$, $P(0,-1)$ $\implies$ $\boxed{\text{S1 : }f(x)=0\text{   }\forall x}$ which indeed is a solution

If $f(1)=0$, $P(1,x-1)$ $\implies$ $f(x)=0$, already found

If $f(-1)=-1$ and $f(1)\ne 0$, Let $f(x)=f(1)g(x)$. then :
$P(1,y-1)$ $\implies$ $g(y)=1+f(1)g(y-1)$
$P(x,y-1)$ $\implies$ $g(xy)=g(x)(1+f(1)g(y-1))$ and so $g(xy)=g(x)g(y)$ with four classical solutions :

$g(x)=0$ $\forall x$ which unfortunately is not a solution in this part of the proof.

$g(x)=1$ $\forall x$ which unfortunately is not a solution in this part of the proof.

$g(0)=0$ and $g(x)=|x|^a$ $\forall x\ne 0$ which unfortunately is not a solution in this part of the proof.

$g(0)=0$ and $g(x)=\text{sign}(x)|x|^a$ $\forall x\ne 0$ which is a solution only when $a=1$ and gives $\boxed{\text{S2 : }f(x)=x\text{   }\forall x}$ which indeed is a solution
\end{solution}
*******************************************************************************
-------------------------------------------------------------------------------

\begin{problem}[Posted by \href{https://artofproblemsolving.com/community/user/78770}{thuanspdn}]
	Find  all continuous functions $f : \mathbb{R}\rightarrow \mathbb{R}$ that satisfy the  functional equation 
$ f\left (  xy\right )=y.f\left ( x \right )+x. f\left ( y\right )$   for all $ x,y\in \mathbb{R}\setminus \left \{ 0 \right \}$
	\flushright \href{https://artofproblemsolving.com/community/c6h591647}{(Link to AoPS)}
\end{problem}



\begin{solution}[by \href{https://artofproblemsolving.com/community/user/29428}{pco}]
	\begin{tcolorbox}Find  all continuous functions $f : \mathbb{R}\rightarrow \mathbb{R}$ that satisfy the  functional equation 
$ f\left (  xy\right )=y.f\left ( x \right )+x. f\left ( y\right )$   for all $ x,y\in \mathbb{R}\setminus \left \{ 0 \right \}$\end{tcolorbox}
Let $g(x)$ from $\mathbb R\setminus\{0\}\to\mathbb R$, continous on both $\mathbb R^+$ and $\mathbb R^-$ defined as $g(x)=\frac{f(x)}x$

Equation becomes $g(xy)=g(x)+g(y)$ whose only continuous solutions on $\mathbb R^*$ are : $g(x)=a\ln |x|$ $\forall x\ne 0$

Hence the solution $\boxed{f(0)=0\text{ and }f(x)=ax\ln|x|\text{   }\forall x\ne 0}$ which indeed is a solution.
\end{solution}
*******************************************************************************
-------------------------------------------------------------------------------

\begin{problem}[Posted by \href{https://artofproblemsolving.com/community/user/127783}{Sayan}]
	Find all functions $f:\mathbb{R}\to \mathbb{R}$ satisfying $f(x-y)=f(x)+xy+f(y)$ for every $x \in \mathbb{R}$ and every $y \in \{f(x) \mid x\in \mathbb{R}\}$, where $\mathbb{R}$ is the set of real numbers.
	\flushright \href{https://artofproblemsolving.com/community/c6h591839}{(Link to AoPS)}
\end{problem}



\begin{solution}[by \href{https://artofproblemsolving.com/community/user/141363}{alibez}]
	\begin{tcolorbox}Find all functions $f:\mathbb{R}\to \mathbb{R}$ satisfying $f(x-y)=f(x)+xy+f(y)$ for every $x \in \mathbb{R}$ and every $y \in \{f(x) \mid x\in \mathbb{R}\}$, where $\mathbb{R}$ is the set of real numbers.\end{tcolorbox}

[hide="hint"]
porve that $f(x)-f(y)$ is surjective .

[\/hide]
\end{solution}



\begin{solution}[by \href{https://artofproblemsolving.com/community/user/29428}{pco}]
	\begin{tcolorbox}[quote="Sayan"]Find all functions $f:\mathbb{R}\to \mathbb{R}$ satisfying $f(x-y)=f(x)+xy+f(y)$ for every $x \in \mathbb{R}$ and every $y \in \{f(x) \mid x\in \mathbb{R}\}$, where $\mathbb{R}$ is the set of real numbers.\end{tcolorbox}porve that $f(x)-f(y)$ is surjective .\end{tcolorbox}
Quite wrong, since $f(x)=0$ $\forall x$ is a trivial solution to the problem.
\end{solution}



\begin{solution}[by \href{https://artofproblemsolving.com/community/user/29428}{pco}]
	\begin{tcolorbox}Find all functions $f:\mathbb{R}\to \mathbb{R}$ satisfying $f(x-y)=f(x)+xy+f(y)$ for every $x \in \mathbb{R}$ and every $y \in \{f(x) \mid x\in \mathbb{R}\}$, where $\mathbb{R}$ is the set of real numbers.\end{tcolorbox}
$\boxed{\text{S1 : }f(x)=0\text{   }\forall x}$ is a solution. So let us from now look only for non allzero solutions.

Let $P(x,y)$ be the assertion $f(x-f(y))=f(x)+xf(y)+f(f(y))$, true $\forall x,y\in\mathbb R$
Let $c=f(0$
Let $u$ such that $f(u)\ne 0$

$P(\frac{x-f(f(u))}{f(u)},u)$ $\implies$ $x=f(\frac{x-f(f(u))}{f(u)}-f(u))-f(\frac{x-f(f(u))}{f(u)})$ and so any $x$ may be written as $x=f(a)-f(b)$ for some $a,b$

(1) : $P(f(a),a)$ $\implies$ $\frac c2=f(f(a))+\frac 12f(a)^2$
(2) : $P(f(b),b)$ $\implies$ $\frac c2=f(f(b))+\frac 12f(b)^2$
(3) : $P(f(a),b)$ $\implies$ $f(f(a)-f(b))=f(f(a))+f(a)f(b)+f(f(b))$
(3)-(1)-(2) : $f(f(a)-f(b))=c-\frac 12(f(a)-f(b))^2$ and so $f(x)=c-\frac 12x^2$

Plugging this back in original equation, we get $c=0$ and the solution $\boxed{\text{S2 : }f(x)=-\frac{x^2}2\text{   }\forall x}$
\end{solution}



\begin{solution}[by \href{https://artofproblemsolving.com/community/user/141363}{alibez}]
	\begin{tcolorbox} and so any $x$ may be written as $x=f(a)-f(b)$ for some $a,b$
\end{tcolorbox}

what is it ? :D  i just post a hint without details !  
\end{solution}



\begin{solution}[by \href{https://artofproblemsolving.com/community/user/29428}{pco}]
	\begin{tcolorbox}[quote="pco"] and so any $x$ may be written as $x=f(a)-f(b)$ for some $a,b$
\end{tcolorbox}

what is it ? :D  i just post a hint without details !  \end{tcolorbox}
Your hint was to prove that $f(x)-f(y)$ was surjective, which is not always true.

So which is wrong.

So bad, or at least dishonest hint.

In my personal opinion.
\end{solution}



\begin{solution}[by \href{https://artofproblemsolving.com/community/user/213306}{saturzo}]
	Let $P(x, y)$ be the assertion $f(x-f(y))=f(x)+xf(y)+f(f(y)), \forall x, y \in \mathbb{R}$ [the problem statement is actually this].
One solution is clearly $f(x)=0, \forall x \in \mathbb{R}$.
Let's, now, assume that $\exists u \in \mathbb{R} : f(u) \neq 0$.
$P(\frac{x-f(f(u))}{f(u)}, u) : f(\frac{x-f(f(u))}{f(u)}-f(u)) - f(\frac{x-f(f(u))}{f(u)}) = x, \forall x \in \mathbb{R}$
$\therefore \left\{f(x)-f(y):x,y \in \mathbb{R}\right\} = \mathbb{R}$
Again $P(f(x), y) : f(f(x)-f(y)) = f((x)) + f(x)f(y) + f(f(y)) = f(f(y)-f(x)) [$ interchanging $x$ and $y]$
$\therefore f(x)=f(-x), \forall x \in \mathbb{R}$
Using this, $P(0, y) : f(-f(y))=f(0)+f(f(y))=f(f(y)) \implies f(0)=0$
Now, $P(f(x), x) : f(f(x))=-\frac{f(x)^2}{2}, \forall x \in \mathbb{R}$.
Using all these, we get:
$P(f(x), y): f(f(x)-f(y))=-\frac{f(x)^2}{2}+f(x)f(y)-\frac{f(y)^2}{2} = -\frac{(f(x)-f(y))^2}{2}$
$\therefore f(x) = -\frac{x^2}{2}, \forall x \in \mathbb{R}$ -- which is clearly another solution!


$NB:$ I got the idea to prove $\left\{f(x)-f(y):x,y \in \mathbb{R}\right\} = \mathbb{R}$ from the IMO-6 of 1999 :D
\end{solution}
*******************************************************************************
-------------------------------------------------------------------------------

\begin{problem}[Posted by \href{https://artofproblemsolving.com/community/user/199885}{lfetahu}]
	If f: N -> Z is such that:
a) f(2) = 2
b) f(mn) = f(m)f(n) for all m, n naturals
c) f(m) > f(n), then find f(1983).
	\flushright \href{https://artofproblemsolving.com/community/c6h592327}{(Link to AoPS)}
\end{problem}



\begin{solution}[by \href{https://artofproblemsolving.com/community/user/199885}{lfetahu}]
	I got f(1983) = 1983 through a boring solution. Any nice ideas?
\end{solution}



\begin{solution}[by \href{https://artofproblemsolving.com/community/user/213278}{shmm}]
	I don't understand part c)
\end{solution}



\begin{solution}[by \href{https://artofproblemsolving.com/community/user/213278}{shmm}]
	Must be for all natural m,n where $ m \geq n $  , $ f(m)\geq f(n)$
\end{solution}



\begin{solution}[by \href{https://artofproblemsolving.com/community/user/172163}{joybangla}]
	A well known technique. From which year is this one? I hope $1983$ because otherwise it is sad. 
$f(1\cdot 1)=f(1)^2\implies f(1)=0,1$. See if $f(1)=0$ then $f\equiv 0$ and thus rejected. Suppose $f(1)=1$. Anyways we prove for powers of $2$. And fill the gaps afterwards. Trivially from multiplicativity we have $f(2^k)=2^k$ for all $k\ge 0$. Now suppose $f(m)=\ell$ for some $m>1$. Now $f(m)>1$ so $\ell$ is positive. Using multiplicativity we get $f(m^n)=\ell^n$ for all $n$. Let for some $n$, we have $2^k<m^n<2^{k+1}$. Now we have $2^k<\ell^n<2^{k+1}$. Thus we have $\frac{1}{2}<\left(\frac{m}{\ell}\right)^n<2$. Clearly this is our contradiction. If $m>\ell$ take $n>\frac{\ell}{m-\ell}$ and apply Bernoulli's inequality to get :
\[ \left(\frac{m}{\ell}\right)^n=\left(1+\frac{m-\ell}{\ell}\right)^n>1+n\left(\frac{m-\ell}{\ell}\right)>2 \]
And if $\ell>m$ take $n>\frac{m}{\ell-m}$ and apply Bernoulli's inequality again :
\[ \left(\frac{\ell}{m}\right)^n=\left(1+\frac{\ell-m}{m}\right)^n>1+n\left(\frac{\ell-m}{m}\right)>2 \]
Both lead to a contradiction. Thus the only solution is $f\equiv \text{id}_{\mathbb{N}}$. Thus $f(1983)=1983$
So only possible value of $f(1983)$ is $1983$. I suppose this is not too boring for you.....
FYI [url=http://www.artofproblemsolving.com\/Wiki\/index.php\/LaTeX:About]LaTeX[\/url] actually makes your posts look readable. Try them sometime.
\end{solution}



\begin{solution}[by \href{https://artofproblemsolving.com/community/user/186353}{randomusername}]
	OK so here's a proposition:
Let $f:\mathbb{N}\to\mathbb{N}$ be a function that satisfies $f(2)=2$, $f(m)<f(m+1)$ $\forall m$ and also $f(m)f(n)=f(mn)$ whenever $m$ and $n$ are relatively prime. Find $f$.
\end{solution}



\begin{solution}[by \href{https://artofproblemsolving.com/community/user/218364}{GoJensenOrGoHome}]
	Or just that $ f(mn)=f(m)f(n) $ is one one the four famous Cauchy functional equations, that admit only the solutions $ f(x)=x^c $ , but since $ f(2)=2 $ you get $ c=1 $, thus $ f(1983)=1983 $.
You can find proofs of Cauchy equations everywhere on the net, best regard 
GJOGH
\end{solution}



\begin{solution}[by \href{https://artofproblemsolving.com/community/user/29428}{pco}]
	\begin{tcolorbox}Or just that $ f(mn)=f(m)f(n) $ is one one the four famous Cauchy functional equations, that admit only the solutions $ f(x)=x^c $ \end{tcolorbox}
Quite wrong.
From $\mathbb N\to\mathbb N$, infinitely many other solutions for  $ f(mn)=f(m)f(n) $ exist.
\end{solution}
*******************************************************************************
-------------------------------------------------------------------------------

\begin{problem}[Posted by \href{https://artofproblemsolving.com/community/user/195015}{Jul}]
	Find all $f:\mathbb{R}\rightarrow \mathbb{R}$ satisfy :
\[f(f(x)-2y)=2x-3y+f(f(y)-x),\;\forall x,y\in \mathbb{R}\]
	\flushright \href{https://artofproblemsolving.com/community/c6h593333}{(Link to AoPS)}
\end{problem}



\begin{solution}[by \href{https://artofproblemsolving.com/community/user/29428}{pco}]
	\begin{tcolorbox}Find all $f:\mathbb{R}\rightarrow \mathbb{R}$ satisfy :
\[f(f(x)-2y)=2x-3y+f(f(y)-x),\;\forall x,y\in \mathbb{R}\]\end{tcolorbox}
Let $P(x,y)$ be the assertion $f(f(x)-2y)=2x-3y+f(f(y)-x)$
Let $a=f(0)$

$P(f(x),x)$ $\implies$  $f(f(f(x))-2x)=2f(x)-3x+a$

$P(x,0)$ $\implies$ $f(a-x)=f(f(x))-2x$ $\implies$ $f(f(a-x))=f(f(f(x))-2x)$ $\implies$ $f(f(a-x))=2f(x)-3x+a$

$P(a-x,0)$ $\implies$ $f(f(a-x))=2a-2x+f(x)$

And so $2f(x)-3x+a=2a-2x+f(x)$ and so $\boxed{f(x)=x+a}$ $\forall x$, which indeed is a solution, whatever is $a\in\mathbb R$
\end{solution}



\begin{solution}[by \href{https://artofproblemsolving.com/community/user/213306}{saturzo}]
	Let $f(0)=\alpha$.
$P(x,0): f(f(x)) = 2x + f(\alpha -x)$
Now, $P(f(y), y): 2f(y)-3y+\alpha = f(f(f(y))-2y) = f(f(\alpha-y))=f(y) + 2\alpha - 2y$.
$\therefore f(y) = \alpha +y, \forall y \in \mathbb{R}$ >> which is clearly a solution where $\alpha \in \mathbb{R}$ is any constant.
\end{solution}
*******************************************************************************
-------------------------------------------------------------------------------

\begin{problem}[Posted by \href{https://artofproblemsolving.com/community/user/215205}{BEHZOD_UZ}]
	Determine all functions $ f : R\rightarrow R$ which satisfy  $\forall x\in R $:
$f(x^3 + x)\le x\le f^3(x) + f(x)$.
	\flushright \href{https://artofproblemsolving.com/community/c6h593695}{(Link to AoPS)}
\end{problem}



\begin{solution}[by \href{https://artofproblemsolving.com/community/user/29428}{pco}]
	\begin{tcolorbox}Determine all functions $ f : R\rightarrow R$ which satisfy  $\forall x\in R $:
$f(x^3 + x)\le x\le f^3(x) + f(x)$.\end{tcolorbox}
Let $g(x)=x^3+x$ : $g(x)$ is an increasing bijection from $\mathbb R\to\mathbb R$ and so $\exists$ $g^{-1}(x)$ increasing bijection too.

$f(g(x))\le x$ $\implies$ $f(x)\le g^{-1}(x)$

$g(g^{-1}(x))=x\le g(f(x))$ and $g(x)$ increasing $\implies$ $g^{-1}(x)\le f(x)$

And so $\boxed{f(x)=g^{-1}(x)}$ $\forall x$, which indeed is a solution.

And you can easily use Cardan's method (for example) to get a closed form if you are interested in.
\end{solution}



\begin{solution}[by \href{https://artofproblemsolving.com/community/user/125553}{lehungvietbao}]
	Dear Mr.\begin{bolded}Patrick\end{bolded}
Sorry for my curiosity but  Could you show us more about ''Cardan's method'' ?
\end{solution}



\begin{solution}[by \href{https://artofproblemsolving.com/community/user/213278}{shmm}]
	I think Cardan's method for cubic equations?!
\end{solution}



\begin{solution}[by \href{https://artofproblemsolving.com/community/user/29428}{pco}]
	\begin{tcolorbox}Dear Mr.\begin{bolded}Patrick\end{bolded}
Sorry for my curiosity but  Could you show us more about ''Cardan's method'' ?\end{tcolorbox}
http://en.wikipedia.org\/wiki\/Cubic_function#Cardano.27s_method
\end{solution}
*******************************************************************************
-------------------------------------------------------------------------------

\begin{problem}[Posted by \href{https://artofproblemsolving.com/community/user/172163}{joybangla}]
	Find all functions $f:\mathbb{N}^{\ast}\rightarrow\mathbb{N}^{\ast}$ with
the properties:
[list=a]
[*]$ f(m+n)  -1 \mid f(m)+f(n),\quad  \forall m,n\in\mathbb{N}^{\ast} $
[*]$ n^{2}-f(n)\text{ is a square } \;\forall n\in\mathbb{N}^{\ast} $[\/list]
	\flushright \href{https://artofproblemsolving.com/community/c6h593728}{(Link to AoPS)}
\end{problem}



\begin{solution}[by \href{https://artofproblemsolving.com/community/user/29428}{pco}]
	I suppose you mean $\mathbb N^*$ is the set of positive integers and $\mathbb N$ is the set of nonnegative integers (unusual rule on this forum ...)

\begin{tcolorbox}Find all functions $f:\mathbb{N}^{\ast}\rightarrow\mathbb{N}$ with
the properties:
[list=a]
[*]$ f(m+n)  -1 \mid f(m)+f(n),\quad  \forall m,n\in\mathbb{N}^{\ast} $
[*]$ n^{2}-f(n)\text{ is a square } \forall n\in\mathbb{N}^{\ast} $[\/list]\end{tcolorbox}
Setting $n=1$ in b., we get $f(1)\in\{0,1\}$

Writing $f(n)=n^2-g(n)^2$ for some $g(n)\in[0,n]$, first equation implies

$f(n)+f(1)\ge f(n+1)-1$ $\forall n\ge 1$ and so $g(n+1)^2\ge g(n)^2+2n-f(1)>g(n)^2$ $\forall n\ge 1$

$2\ge g(2)>g(1)$ and so $g(2)\in\{1,2\}$

If $g(2)=2$, we get $g(n)=n$ $\forall n\ge 2$ and so $f(n)=0$ $\forall n\ge 2$ and the two solutions ;
$\boxed{\text{S1 : }f(n)=0\text{   }\forall n>0}$

$\boxed{\text{S2 : }f(1)=1\text{ and }f(n)=0\text{   }\forall n\ge 2}$

If $g(2)=1$, we get $g(1)=0$ and :
Either $g(n)=n-1$ $\forall n\ge 2$ and the solution $\boxed{\text{S3 : }f(n)=2n-1\text{   }\forall n>0}$

Either $g(n)=n-1$ $\forall n\in[2,p]$ and $g(n)=n$ $\forall n>p$ for some $p$ and the solution 
$\boxed{\text{S4 : }f(n)=2n-1\text{   }\forall n\in[1,p]\text{ and }f(n)=0\text{   }\forall n>p}$ which indeed is a solution, whatever is $p\ge 1$
(Note that S2 is a part of S4, choosing $p=1$)
\end{solution}



\begin{solution}[by \href{https://artofproblemsolving.com/community/user/64716}{mavropnevma}]
	\begin{bolded}Fortunately\end{bolded}, the problem, as actually submitted, had the co-domain also equal to $\mathbb{N}^*$. 
Thus only $\boxed{\text{S3 : }f(n)=2n-1\text{   }\forall n>0}$ from the masterly \begin{bolded}pco\end{bolded}'s solution survives :)
\end{solution}



\begin{solution}[by \href{https://artofproblemsolving.com/community/user/172163}{joybangla}]
	Anyways the solution is awesome but I made a typo, the function is supposed to be $f:\mathbb{N}^{\ast}\to \mathbb{N}^{\ast}$. Sorry!!
EDIT : whoops caught before I could say it!!
\end{solution}



\begin{solution}[by \href{https://artofproblemsolving.com/community/user/241428}{Garfield}]
	[hide=Beautiful problem]
$P(1)$ in condition b. so we get $1-f(1)=x^{2}$ so $f(1)=1$
$P(2)$ in b. and we easly get $f(2)=3$.
$P(3)$ in b. and we get $f(3)=5$ so let's try to prove $f(x)=2x-1$.
$P(m,1)$ in a. $f(m+1)-1|f(m)+1$ so let $f(m)+1=k(f(m+1)-1$
Now if $k>1$ we  get $f(m+1)<f(m)$ 
So Let's induct on $n$ to prove $f(n+1)=f(n)+2$
\begin{bolded}1. step\end{bolded}$f(2)=3=f(1)+2$
\begin{bolded}2.nd step\end{bolded}For every $n$ we have $f(n+1)=f(n)+2$ so we easy by induction get $f(n+1)=2(n+1)-1=2n+1$
\begin{bolded}3.rd step\end{bolded}If $k=1$ we're done so assume opposite, $k>1$:
now $f(n+1)>f(n+2)$ so $P(n+2)$ in b. so:
$(n+2)^{2}-f(n+2)>(n+2)^{2}-f(n+1)=n^{2}+4+4n-2n-1=n^{2}+2n+3>(n+1)^{2}$ so it's between $(n+1)^2$ and $(n+2)^{2}$ so $k=1$ so $f(n+2)=f(n+1)+2$.
So now cheking $f(n)=2n-1$ we get solution is:
$f(n)=2n-1$ , $\forall n \in N$.
[\/hide]
\end{solution}
*******************************************************************************
-------------------------------------------------------------------------------

\begin{problem}[Posted by \href{https://artofproblemsolving.com/community/user/172163}{joybangla}]
	Find all functions $f:\mathbb{Q}\to \mathbb{Q}$ such that 
\[ f(x+3f(y))=f(x)+f(y)+2y \quad \forall x,y\in \mathbb{Q}\]
	\flushright \href{https://artofproblemsolving.com/community/c6h593734}{(Link to AoPS)}
\end{problem}



\begin{solution}[by \href{https://artofproblemsolving.com/community/user/29428}{pco}]
	\begin{tcolorbox}Find all functions $f:\mathbb{Q}\to \mathbb{Q}$ such that 
\[ f(x+3f(y))=f(x)+f(y)+2y \quad \forall x,y\in \mathbb{Q}\]\end{tcolorbox}
Let $P(x,y)$ be the assertion $f(x+3f(y))=f(x)+f(y)+2y$
$f(x)$ is injective
$P(x-3f(x),x)$ $\implies$ $f(x-3f(x))=-2x$ and $f(x)$ is surjective.

Let $u=f^{-1}(0)$. $P(u,u)$ $\implies$ $u=0$

$P(0,x)$ $\implies$ $f(3f(x))=f(x)+2x$
$P(-3f(x),x)$ $\implies$ $-f(-3f(x))=f(x)+2x$

So, since surjective, $f(-x)=-f(x)$ and $f(x)$ is an odd funcion


$P(x,-\frac{f(x)}2)$ $\implies$ $f(x+3f(-\frac{f(x)}2))=f(-\frac{f(x)}2)$ and so, since injective,  $x+3f(-\frac{f(x)}2)=-\frac{f(x)}2$
And so $6f(\frac{f(x)}2)=f(x)+2x$

So $f(3f(x))=6f(\frac{f(x)}2)$ and so, since surjective,  $f(6x)=6f(x)$ 

$P(3f(x),x)$ $\implies$ $6f(f(x))=f(3f(x))+f(x)+2x=2f(3f(x))$ and so, since surjective, $f(3x)=3f(x)$ and sio $f(2x)=2f(x)$

From there :
$P(nf(x),x)$ $\implies$ $f((n+3)f(x))=f(nf(x))+f(f(x))+2x$
$P((n+1)f(x),x)$ $\implies$ $f((n+4)f(x))=f((n+1)f(x))+f(f(x))+2x$
Subtracting, we get $f((n+4)f(x))-f((n+3)f(x))=f((n+1)f(x))-f(nf(x))$ and, since surjective, $f((n+4)x)-f((n+3)x)=f((n+1)x)-f(nx)$

And since we previously got $f(3x)-f(2x)=f(2x)-f(x)=f(x)-f(0)$, we get with induction $f(nx)=nf(x)$ $\forall x\in\mathbb Q$, $\forall n\in\mathbb Z$

So $f(x)=ax$, and plugging back in original equation, $a\in\{1,-\frac 23\}$

Hence the solutions $\boxed{\text{S1 : }f(x)=x\text{   }\forall x\in\mathbb Q}$ and $\boxed{\text{S2 : }f(x)=-\frac{2x}3\text{   }\forall x\in\mathbb Q}$
\end{solution}



\begin{solution}[by \href{https://artofproblemsolving.com/community/user/86115}{Mahi}]
	Or, after we have $f(x)$ surjective, and $f(0) = 0$,
$P(0,y) \Rightarrow f(3f(y)) = f(y)+2y$
and thus, we have $f(x+3f(y)) = f(x)+f(3f(y))$
As we know $f(x)$ to be surjective, replacing $3f(y)$ with $y$, $f(x+y) = f(x)+f(y)$, and thus $f(x)=ax \forall x \in \mathbb Q$, and setting it in the original equation we get $a = 1$ or $-\frac 23$.
\end{solution}



\begin{solution}[by \href{https://artofproblemsolving.com/community/user/142879}{ionbursuc}]
	hence
http://www.artofproblemsolving.com/Forum/viewtopic.php?f=37&t=579717
\end{solution}
*******************************************************************************
-------------------------------------------------------------------------------

\begin{problem}[Posted by \href{https://artofproblemsolving.com/community/user/211457}{rod16}]
	Let $ f$ be a function from $ \mathbb{Z}$ to $ \mathbb{R}$ which is bounded from above and satisfies
$ f(n)\leq \frac {1}{2} \Big(f(n - 1) + f(n + 1)\Big)$ for all $ n$ . Show that $ f$ is constant.
	\flushright \href{https://artofproblemsolving.com/community/c6h593771}{(Link to AoPS)}
\end{problem}



\begin{solution}[by \href{https://artofproblemsolving.com/community/user/215548}{h-bar}]
	Suppose $f$ is not constant, then there is $n^\ast\in\mathbb{Z}$ such that $f(n^\ast)\ne f(n^\ast-1)$. Without a loss of generality we can assume that $f(n^\ast) > f(n^\ast-1)$. Since $f(n)-f(n-1)\leq f(n+1)-f(n)$ for all $n\in\mathbb{Z}$, $f(n^\ast+k)\geq f(n^\ast) + k\cdot(f(n^\ast)-f(n^\ast-1))$ for all $k\in\mathbb{Z}^+$ therefore $f$ is not bounded from above. Hence, the supposition is false, and $f$ is constant.
\end{solution}



\begin{solution}[by \href{https://artofproblemsolving.com/community/user/29428}{pco}]
	\begin{tcolorbox}Without a loss of generality we can assume that $f(n^\ast) > f(n^\ast-1)$\end{tcolorbox}
Why ?, You cant change WLOG $f(x)$ to $-f(x)$ since $f(x)$ upperbounded does not imply $-f(x)$ upperbounded.
\end{solution}



\begin{solution}[by \href{https://artofproblemsolving.com/community/user/215548}{h-bar}]
	\begin{tcolorbox}[quote="h-bar"]Without a loss of generality we can assume that $f(n^\ast) > f(n^\ast-1)$\end{tcolorbox}
Why ?, You cant change WLOG $f(x)$ to $-f(x)$ since $f(x)$ upperbounded does not imply $-f(x)$ upperbounded.\end{tcolorbox}
I was referring to the symmetry $\tilde{f}(n)=f((2n^\ast-1)-n)$ ($n\in\mathbb{Z}$).
Maybe it would have been easier to brake it down to cases. Thanks for your comment, I'm here to improve my skills of presenting proofs.

Case $f(n^\ast-1)>f(n^\ast)$: Since $f(n-1)-f(n) \geq f(n)-f(n+1)$ for all $n\in \mathbb{Z}$, $f(n^\ast-1-k)\geq f(n^\ast-1)+k(f(n^\ast-1)-f(n^\ast))$.
\end{solution}
*******************************************************************************
-------------------------------------------------------------------------------

\begin{problem}[Posted by \href{https://artofproblemsolving.com/community/user/125553}{lehungvietbao}]
	Let $\alpha\neq 0$ be a real number, find all  continuous functions $f:\mathbb{R}\to\mathbb{R}$ so that \[f(2x-\frac{f(x)}{\alpha})=\alpha \quad \forall x\in\mathbb{R}\]
	\flushright \href{https://artofproblemsolving.com/community/c6h594151}{(Link to AoPS)}
\end{problem}



\begin{solution}[by \href{https://artofproblemsolving.com/community/user/29428}{pco}]
	\begin{tcolorbox}Let $\alpha\neq 0$ be a real number, find all  continuous functions $f:\mathbb{R}\to\mathbb{R}$ so that \[f(2x-\frac{f(x)}{\alpha})=\alpha \quad \forall x\in\mathbb{R}\]\end{tcolorbox}
Let $g(x)=2x-\frac{f(x)}{\alpha}$ and equation becomes $g(g(x))=2g(x)-1$ $\forall x$ and so $g(x)=2x-1$ $\forall x\in g(\mathbb R)$

If $\exists a\in g(\mathbb R)$ such that $a>1$, then $[a,+\infty)\subseteq g(\mathbb R)$
If $\exists a\in g(\mathbb R)$ such that $a<1$, then $(-\infty,a]\subseteq g(\mathbb R)$

and so four cases :

1) either $g(\mathbb R)=\mathbb R$ and we get the solution $g(x)=2x-1$ and so $\boxed{f(x)=\alpha}$ $\forall x$, which indeed is a solution.

2) either $g(\mathbb R)=\{1\}$ and we get the solution $g(x)=1$ and so $\boxed{f(x)=\alpha(2x-1)}$ $\forall x$, which indeed is a solution.

3) either $g(\mathbb R)=[a,+\infty)$ for some $a\ge 1$ and we get the solution :
$g(x)=2x-1$ $\forall x\ge a$ and $g(x)$ is any value $\ge a$ $\forall x<a$ and so :

Let $a\ge 1$
Let $h(x)$ any continuous function from $(-\infty,a]\to [a,+\infty)$ such that $h(a)=2a-1$
$f(x)=\alpha(2x-h(x))$ $\forall x\le a$
$f(x)=\alpha$ $\forall x\ge a$

4) either $g(\mathbb R)=(-\infty,a]$ for some $a\le 1$ and we get the solution :
$g(x)=2x-1$ $\forall x\le a$ and $g(x)$ is any value $\le a$ $\forall x>a$ and so :

Let $a\le 1$
Let $h(x)$ any continuous function from $[a,+\infty)\to (-\infty,a]$ such that $h(a)=2a-1$
$f(x)=\alpha(2x-h(x))$ $\forall x\ge a$
$f(x)=\alpha$ $\forall x\le a$
\end{solution}
*******************************************************************************
-------------------------------------------------------------------------------

\begin{problem}[Posted by \href{https://artofproblemsolving.com/community/user/125553}{lehungvietbao}]
	Find all  continuous functions $f:\mathbb{R}\to\mathbb{R}$ so that \[\{f(x+y)\}=\{f(x)+f(y)\},\forall x,y\in\mathbb{R}\]
Where  $\{x\}=x-\left \lfloor x \right \rfloor $
	\flushright \href{https://artofproblemsolving.com/community/c6h594152}{(Link to AoPS)}
\end{problem}



\begin{solution}[by \href{https://artofproblemsolving.com/community/user/29428}{pco}]
	\begin{tcolorbox}Find all  continuous functions $f:\mathbb{R}\to\mathbb{R}$ so that \[\{f(x+y)\}=\{f(x)+f(y)\},\forall x,y\in\mathbb{R}\]
Where  $\{x\}=x-\left \lfloor x \right \rfloor $\end{tcolorbox}
So $f(x+y)+m(x,y)=f(x)+f(y)+n(x,y)$ for some $m,n$ functions from $\mathbb R\to\mathbb Z[X,Y]$

So, since continuous, $f(x+y)=f(x)+f(y)+k$ for some $k\in\mathbb Z$

So, since continuous, $\boxed{f(x)=ax+b}$ $\forall x$, which indeed is a solution, whatever are $a\in\mathbb R$ and $b\in\mathbb Z$
\end{solution}
*******************************************************************************
-------------------------------------------------------------------------------

\begin{problem}[Posted by \href{https://artofproblemsolving.com/community/user/163554}{ablbabybb}]
	Problem : find all strictly monotone functions $f:(0,+\infty)\to(0,+\infty)$ such that
\[\left(x+\frac{1}{2}\right)f\left(\dfrac{y}{f(x)}\right)=f(x+y),\forall x,y>0\]
	\flushright \href{https://artofproblemsolving.com/community/c6h594587}{(Link to AoPS)}
\end{problem}



\begin{solution}[by \href{https://artofproblemsolving.com/community/user/29428}{pco}]
	\begin{tcolorbox}Problem : find all strictly monotone functions $f:(0,+\infty)\to(0,+\infty)$ such that
\[\left(x+\frac{1}{2}\right)f\left(\dfrac{y}{f(x)}\right)=f(x+y),\forall x,y>0\]\end{tcolorbox}
Setting $x=\frac 12$ and using injectivity, we get $y(\frac 1{f(\frac 12)}-1)=\frac 12$ $\forall y>0$, which is impossible.

So no solution.
\end{solution}



\begin{solution}[by \href{https://artofproblemsolving.com/community/user/163554}{ablbabybb}]
	thanks ,Pco. Could you solve other problem : find all strictly monotone functions $f:(0,+\infty)\to(0,+\infty)$ such that
\[\left(2x+1\right)f\left(\dfrac{y}{f(x)}\right)=f(x+y),\forall x,y>0\]
\end{solution}



\begin{solution}[by \href{https://artofproblemsolving.com/community/user/141363}{alibez}]
	\begin{tcolorbox}thanks ,Pco. Could you solve other problem : find all strictly monotone functions $f:(0,+\infty)\to(0,+\infty)$ such that
\[(2x+1)f(\dfrac{y}{f(x)})=f(x+y),\forall x,y>0\]\end{tcolorbox}

if $\exists \: x_{0}: f(x_{0})< 1\Rightarrow P(x_{0},\frac{x_{0}f(x_{0})}{1-f(x_{0})})$ .....


so we have : $f(x)\geq 1\: \: \forall x\in \mathbb{R}^{+}$ 

$P(x,y):f(x+y)>2x+1$ we know : $f$ is strictly monotone functions :

 $1)$ if $f$ is non increasing we have $f(y)>f(x+y)>2x+1$ it is impossible .

$2)$ if $f$ is increasing : we know : $P(x,y):f(x+y)>2x+1$ so we have : $f(x)\geq 2x+1$  we have : $(2x+1)f(\dfrac{y}{f(x)}))\geq (2x+1)+2y$ so $f(\dfrac{y}{f(x)}))\geq 1+\dfrac{2y}{2x+1}$ so $2x+1\geq f(x)$ ...
\end{solution}
*******************************************************************************
-------------------------------------------------------------------------------

\begin{problem}[Posted by \href{https://artofproblemsolving.com/community/user/215607}{hoangthailelqd}]
	find all f: R -> R :
\[ f(x+y)+f(x-y)-2f(x).f(1+y)=2xy(3y-x^{2}),(x,y\epsilon R) \]
	\flushright \href{https://artofproblemsolving.com/community/c6h594605}{(Link to AoPS)}
\end{problem}



\begin{solution}[by \href{https://artofproblemsolving.com/community/user/29428}{pco}]
	\begin{tcolorbox}find all f: R -> R :
\[ f(x+y)+f(x-y)-2f(x).f(1+y)=2xy(3y-x^{2}),(x,y\epsilon R) \]\end{tcolorbox}
Let $P(x,y)$ be the assertion $f(x+y)+f(x-y)-2f(x)f(1+y)=2xy(3y-x^2)$

$f(x)=0$ $\forall x$ is not a solution. So let $u$ such that $f(u)\ne 0$

$P(u,0)$ $\implies$ $f(1)=1$

$P(1,1)$ $\implies$  $f(0)-f(2)=4$
$P(1,-1)$ $\implies$ $f(2)-f(0)=8$
Adding these two lines gives contradiction.

So no such function.
\end{solution}
*******************************************************************************
-------------------------------------------------------------------------------

\begin{problem}[Posted by \href{https://artofproblemsolving.com/community/user/183149}{JuanOrtiz}]
	Let $f: \mathbb{R} \mapsto \mathbb{R}$. Suppose that $f(x+y)=f(x)+f(y)$ for all $x,y \in \mathbb{R}$. Also, for all $x \in \mathbb{R}$ we have $f(f(x))=f(x)$. Suppose $f$ isn't the zero function. 

Decide whether these conditions are enough to prove $f(x)=x$ for all $x \in \mathbb{R}$
	\flushright \href{https://artofproblemsolving.com/community/c6h595142}{(Link to AoPS)}
\end{problem}



\begin{solution}[by \href{https://artofproblemsolving.com/community/user/183149}{JuanOrtiz}]
	[hide]Hint: no. Construct (using Hamel basis) a function such that the function decides whether it goes to 0 or to itself.[\/hide]
\end{solution}



\begin{solution}[by \href{https://artofproblemsolving.com/community/user/29428}{pco}]
	\begin{tcolorbox}Let $f: \mathbb{R} \mapsto \mathbb{R}$. Suppose that $f(x+y)=f(x)+f(y)$ for all $x,y \in \mathbb{R}$. Also, for all $x \in \mathbb{R}$ we have $f(f(x))=f(x)$. Suppose $f$ isn't the zero function. 

Decide whether these conditions are enough to prove $f(x)=x$ for all $x \in \mathbb{R}$\end{tcolorbox}
General solution :

Let $A,B$ two supplementary subvectorspaces of the $\mathbb Q$-vectorspace $\mathbb R$
Let $a(x)$ from $\mathbb R\to A$ and $b(x)$ from $\mathbb R\to B$ the two projections of $x$ (so that $x=a(x)+b(x)$ in a unique manner).

Then $f(x)=a(x)$

Examples :
$(A,B)=(\mathbb R,\{0\})$ gives solution $f(x)=x$
$(A,B)=(\{0\},\mathbb R)$ gives solution $f(x)=0$

And any other pair (whose existence is easy to prove with axiom of choice) gives a non trivial solution.
\end{solution}
*******************************************************************************
-------------------------------------------------------------------------------

\begin{problem}[Posted by \href{https://artofproblemsolving.com/community/user/167483}{Math-lover123}]
	Find an example of a bijection between sets $\mathbb{R}$ and $\mathbb{C}$.
	\flushright \href{https://artofproblemsolving.com/community/c6h595988}{(Link to AoPS)}
\end{problem}



\begin{solution}[by \href{https://artofproblemsolving.com/community/user/29428}{pco}]
	\begin{tcolorbox}Find an example of a bijection between sets $\mathbb{R}$ and $\mathbb{C}$.\end{tcolorbox}
It's very classical to build a bijection from $[0,1)X[0,1)\to[0,1)$ :

Take the normalized binary representations of $x,y$ (not ending with infinite sequence of 1s) and build the binary representation of $z=f(x,y)$ as following :

Copy all binary digits of $x$ to $z$ up and including the first $0$
Copy then all binary digits of $y$ to $z$ up and including the first $0$
Back to $x$, copy all binary digits of $x$ to $z$ from the first uncopied digit up and including the first $0$
Back to $y$, copy all binary digits of $x$ to $z$ from the first uncopied digit up and including the first $0$
and so on.

And since there are simple bijections from $[0,1)\to\mathbb R$ and from $\mathbb R^2\to\mathbb C$, end of problem.
\end{solution}
*******************************************************************************
-------------------------------------------------------------------------------

\begin{problem}[Posted by \href{https://artofproblemsolving.com/community/user/167924}{utkarshgupta}]
	The function f is defined on the positive integers as follows;
$f(1) = 1$;
$f(2n) = f(n)$ if $n$ is even;
$f(2n) = 2f(n)$ if $n$ is odd;
$f(2n + 1) = 2f(n) + 1$ if $n$ is even;
$f(2n + 1) = f(n)$ if $n $ is odd.
Find the number of positive integers $n$ which are less than $2011$ and
have the property that $f(n) = f(2011)$.
	\flushright \href{https://artofproblemsolving.com/community/c6h596638}{(Link to AoPS)}
\end{problem}



\begin{solution}[by \href{https://artofproblemsolving.com/community/user/89198}{chaotic_iak}]
	Express $n$ in binary and compact any string of identical digits into only one such digit. For example, $2011_{10} = 11111011011_2$, so $f(2011_{10}) = 10101_2 = 21_{10}$. This can be verified easily using the given four recurrences.

Thus since $f(2011) = 10101_2$, we need to find the number of positive integers less than $2011$ that has the pattern $1\ldots 10\ldots 01\ldots 10\ldots 01\ldots 1_2$.

This can be done as follows. Since $2011$ has $11$ binary digits, we can think of a string in the form $0\ldots 01\ldots 10\ldots 01\ldots 10\ldots 01\ldots 1_2$ with $12$ digits and evaluate its value. Thus letting the lengths of consecutive identical digits be $a,b,c,d,e,f$ in order, we need to find the number of solutions for $a+b+c+d+e+f = 12$ where $a,b,c,d,e,f$ are positive integers. There are $\binom{12-1}{6-1} = 462$ such numbers. However, some of these numbers exceed $2011$; we need to manually compute the number of integers greater than or equal to $2011$ that has the same pattern, and remove the count. If I count correctly, there are $7$ such numbers, so there are $462 - 7 = \boxed{455}$ numbers satisfying the condition.
\end{solution}



\begin{solution}[by \href{https://artofproblemsolving.com/community/user/29428}{pco}]
	\begin{tcolorbox}The function f is defined on the positive integers as follows;
$f(1) = 1$;
$f(2n) = f(n)$ if $n$ is even;
$f(2n) = 2f(n)$ if $n$ is odd;
$f(2n + 1) = 2f(n) + 1$ if $n$ is even;
$f(2n + 1) = f(n)$ if $n $ is odd.
Find the number of positive integers $n$ which are less than $2011$ and
have the property that $f(n) = f(2011)$.\end{tcolorbox}
It is easy to see that $f(n)$ is $n$ wher all consecutive ones are replaced by a unique one and all consecutive zeroes are replaced by a single zero.

For example : $f(\overline{110000111001}_2=\overline{10101}_2)$ 

So $f(2011)=f(\overline{11111011011}_2=\overline{10101}_2=21$

So required number is the number of $11$ bits numbers with exactly 5 changes in the sequence of binary digits ($\binom{11}5=462$) minus the numbers greater than or equal to 2011 :
$\overline{11111011011}_2$
$\overline{11111011101}_2$
$\overline{11111100101}_2$
$\overline{11111101001}_2$
$\overline{11111101011}_2$
$\overline{11111101101}_2$
$\overline{11111110101}_2$

Hence the result :$\boxed{455}$
\end{solution}
*******************************************************************************
-------------------------------------------------------------------------------

\begin{problem}[Posted by \href{https://artofproblemsolving.com/community/user/213278}{shmm}]
	Find all functions $ f: \mathbb{R}^{+}\to \mathbb{R}^{+}$  such that $x^2(f(x)+f(y))=(x+y)f(yf(x))$ for all $x,y \in \mathbb{R}^{+}$.
	\flushright \href{https://artofproblemsolving.com/community/c6h596685}{(Link to AoPS)}
\end{problem}



\begin{solution}[by \href{https://artofproblemsolving.com/community/user/29428}{pco}]
	\begin{tcolorbox}Find all functions $ f: \mathbb{R}^{+}\to \mathbb{R}^{+}$  such that $x^2(f(x)+f(y))=(x+y)f(yf(x))$ for all $x,y \in \mathbb{R}^{+}$.\end{tcolorbox}
Let $P(x,y)$ be the assertion $x^2(f(x)+f(y))=(x+y)f(yf(x))$
Let $a=f(1)$

$P(1,1)$ $\implies$ $f(a)=a$
$P(a,a)$ $\implies$ $f(a^2)=a^2$
$P(a^2,a)$ $\implies$ $f(a^3)=a^4$
$P(a,a^2)$ $\implies$ $f(a^3)=a^2$
So $a=1$

$P(1,x)$ $\implies$ $\boxed{f(x)=\frac 1x}$ $\forall x$, which indeed is a solution.
\end{solution}



\begin{solution}[by \href{https://artofproblemsolving.com/community/user/176456}{aopsaops}]
	Another approach
$P(x,1)$ we obtain $\displaystyle \frac{x+1}{x^2}=\frac{f(x)+f(1)}{f(f(x))}$.
If $f(a)=f(b)$ we obtain that $\displaystyle \frac{a+1}{a^2}=\frac{b+1}{b^2}$, so $a=b$ and $f$ is injective.

$P(1,1)$ we obtain $f(1)=f(f(1)) \Rightarrow f(1)=1$

$P(1,y)$ we obtain $1+f(y)=(1+y)f(y)$ and so $\displaystyle f(y)=\frac{1}{y}$ for all $y>0$.
\end{solution}
*******************************************************************************
-------------------------------------------------------------------------------

\begin{problem}[Posted by \href{https://artofproblemsolving.com/community/user/218396}{vi1lat}]
	Find all functions $f:\mathbb{N}\to \mathbb{N}$ such that $f(f(n))+f(n+1)=n+3$ for all natural $n$.
	\flushright \href{https://artofproblemsolving.com/community/c6h598760}{(Link to AoPS)}
\end{problem}



\begin{solution}[by \href{https://artofproblemsolving.com/community/user/29428}{pco}]
	\begin{tcolorbox}Find all functions $f:\mathbb{N}\to \mathbb{N}$ such that $f(f(n))+f(n+1)=n+3$ for all natural $n$.\end{tcolorbox}
I suppose that, as usual in this forum, $\mathbb N$ is the set of positive integers.
Let $P(n)$ be the assertion $f(f(n))+f(n+1)=n+3$

Case 1 : If $f(2)=1$
==============
$P(2)$ $\implies$ $f(1)+f(3)=5$ and so $f(1)\in\{1,2,3,4\}$

Subcase 1.1 : $f(1)=1$ and $f(2)=1$ and $f(3)=4$
$P(1)$ is then wrong

Subcase 1.2 : $f(1)=2$ and $f(2)=1$ and $f(3)=3$
$P(1)$ is then wrong

Subcase 1.3 : $f(1)=3$ and $f(2)=1$ and $f(3)=2$
$P(1)$ is then wrong

Subcase 1.4 : $f(1)=4$ and $f(2)=1$ and $f(3)=1$
$P(1)$ $\implies$ $f(4)=3$
$P(3)$ $\implies$ $f(4)=2$
And so contradiction


Case 2 : If $f(2)=2$
==============
$P(1)$ $\implies$ $f(f(1))=2$
$P(2)$ $\implies$ $f(3)=3$
$P(3)$ $\implies$ $f(4)=3$
$P(4)$ $\implies$ $f(5)=4$
From there, it's easy to show that $f(n)>2$ $\forall n>2$ and so that $f(1)=2$ and $f(n)$ is uniquely determined.
It's then rather easy to check that $\boxed{f(n)=2+\left\lfloor\frac{\sqrt 5-1}2(n-1)\right\rfloor}$ is the unique solution in this case

Case 3 : If $f(2)=3$
==============
$P(2)$ $\implies$ $2f(3)=5$, impossible


Case 4 : If $f(2)\ge 4$
================
$P(1)$ $\implies$ $f(f(1))\le 0$, impossible



And so a unique solution.
\end{solution}
*******************************************************************************
-------------------------------------------------------------------------------

\begin{problem}[Posted by \href{https://artofproblemsolving.com/community/user/218396}{vi1lat}]
	Find all functions $f:\mathbb{R}\to \mathbb{R}$ such that $f(f(x)+yz)=x^3+f(y)f(z)$  for all reals $x,y,z$
	\flushright \href{https://artofproblemsolving.com/community/c6h599003}{(Link to AoPS)}
\end{problem}



\begin{solution}[by \href{https://artofproblemsolving.com/community/user/29428}{pco}]
	\begin{tcolorbox}Find all functions $f:\mathbb{R}\to \mathbb{R}$ such that $f(f(x)+yz)=x^3+f(y)f(z)$  for all reals $x,y,z$\end{tcolorbox}
Let $P(x,y,z)$ be the assertion $f(f(x)+yz)=x^3+f(y)f(z)$

If $f(0)\ne 0$, $P(0,0,x)$ $\implies$ $f(x)=c$ constant, which is never a solution. So $f(0)=0$

$P(0,x,y)$ $\implies$ $f(xy)=f(x)f(y)$ and $f(x)$ is multiplicative.

$P(\sqrt[3]x,0)$ $\implies$ $f(f(\sqrt[3]x))=x$

(1) : $P(f(\sqrt[3]x),y,1)$ $\implies$ $f(x+y)=f(\sqrt[3]x)^3+f(y)f(1)$
(2) : $P(f(\sqrt[3]x),0,1)$ $\implies$ $f(x)=f(\sqrt[3]x)^3$
(3) : $P(0,y,1)$ $\implies$ $f(y)=f(y)f(1)$
(1)-(2)-(3) $\implies$ $f(x+y)=f(x)+f(y)$ and so $f(x)$ is additive.

And so, since non constant, additive and multiplicative, we classicaly get $\boxed{f(x)=x}$ $\forall x$ which indeed is a solution.
\end{solution}



\begin{solution}[by \href{https://artofproblemsolving.com/community/user/185787}{gobathegreat}]
	\begin{tcolorbox}
And so, since non constant, additive and multiplicative, we classicaly get $\boxed{f(x)=x}$ $\forall x$ which indeed is a solution.
\end{tcolorbox}
I do not think $f(x)=x$ is a solution.

Another very similar solution
Let $P(x,y,z)$ be the assertion $f(f(x)+yz)=x^3+f(y)f(z)$
Obviously $f$ is bijective so there exist real number $a$ such that $f(a)=0$
$P(a,a,1)$ yields $f(0)=0$
$P(x,0,z)$ yields $f(f(x))=x^3$
$P(0,y,z)$ yields $f(y)f(z)=f(yz)$ so $f$ is multiplicative
Now $f(f(x)+yz)=x^3+f(y)f(z)=f(f(x))+f(y)f(z)=f(f(x))+f(yz)$ and since $f$ is surjective we get $f(x+y)=f(x)+f(y)$ so $f$ is additive
And now since function is additive, multiplicative and bijective yields $f(x)=x$ which unfortunately is not a solution. So no solution exist.
\end{solution}



\begin{solution}[by \href{https://artofproblemsolving.com/community/user/29428}{pco}]
	\begin{tcolorbox}I do not think $f(x)=x$ is a solution.\end{tcolorbox}
Shame on me !
You're right, indeed.   :blush:
\end{solution}
*******************************************************************************
-------------------------------------------------------------------------------

\begin{problem}[Posted by \href{https://artofproblemsolving.com/community/user/215607}{hoangthailelqd}]
	find all functions continous $ f:R\rightarrow R $ such that:
$ \{\begin{matrix}
f(\sqrt{2}x)=2f(x) &  & \\ 
f(x+1)=f(x)+2x+1 &  & 
\end{matrix} $
	\flushright \href{https://artofproblemsolving.com/community/c6h602829}{(Link to AoPS)}
\end{problem}



\begin{solution}[by \href{https://artofproblemsolving.com/community/user/29428}{pco}]
	\begin{tcolorbox}find all functions continous $ f:R\rightarrow R $ such that:
$ \{\begin{matrix}
f(\sqrt{2}x)=2f(x) &  & \\ 
f(x+1)=f(x)+2x+1 &  & 
\end{matrix} $\end{tcolorbox}
From the first we get $f(0)=0$ and the second implies $f(n)=n^2$ $\forall n\in\mathbb Z$
From the first, we get then $f(n\sqrt 2)=(n\sqrt 2)^2$ and the second implies $f(m+n\sqrt 2)=(m+n\sqrt 2)^2$ $\forall m,n\in\mathbb Z$

And since $f(x)$ is continuous and $\{m+n\sqrt 2$ where $m,n\in\mathbb Z\}$ is dense in $\mathbb R$, we get :

$\boxed{f(x)=x^2}$ $\forall x\in\mathbb R$, which indeed is a solution.
\end{solution}



\begin{solution}[by \href{https://artofproblemsolving.com/community/user/230501}{kanybekasanbekov97}]
	Is this solution true?
\end{solution}



\begin{solution}[by \href{https://artofproblemsolving.com/community/user/212679}{shanks_ahmad}]
	Yes, It is. You can prove that if some set $A$ is dense in $\mathbb{R}$, $\forall x\in \mathbb{R}$ there exists a sequence $x_n\in A \rightarrow x$ as $n\rightarrow \infty$. Using this property, you can justify the last logical transition in pco's proof.
\end{solution}
*******************************************************************************
-------------------------------------------------------------------------------

\begin{problem}[Posted by \href{https://artofproblemsolving.com/community/user/215607}{hoangthailelqd}]
	find all functions f:R->R such that:
$ f(x^{2}+2f(y))=\frac{y}{2}+2(f(x))^{2} $
	\flushright \href{https://artofproblemsolving.com/community/c6h602952}{(Link to AoPS)}
\end{problem}



\begin{solution}[by \href{https://artofproblemsolving.com/community/user/195149}{BSJL}]
	\begin{tcolorbox}find all functions f:R->R such that:
$ f(x^{2}+2f(y))=\frac{y}{2}+2(f(x))^{2} $\end{tcolorbox}

At first, we consider the function $ g: \mathbb{R} \rightarrow \mathbb{R} $ such that $ g(x)=2f(x) $

Then the condition becomes $ g(x^2+g(y))=y+g(x)^2 $

Let $ P(x,y) $ be the assertion of $ g(x^2+g(y))=y+g(x)^2 $

$ P(x,0),P(0,y) \Rightarrow g(x^2+y)=g(y)+g(x^2)-g(0) $

Now, let $ h(x)=g(x)-g(0) $ then we get $ h(x+y)=h(x)+h(y) $ $ \forall x,y \in \mathbb{R} $  immediately.

$ P(x,0) \Rightarrow h(x) \ge -g(0) $ $ \forall x \ge f(0) $, which means $ h(x) $ is a solution of Cauchy equation with a lower bound on some non empty open interval.

Therefore, $ h(x)=cx $ where $ c $ is a constant $ \rightarrow g(x)=cx+d $

Finally, plugging this result back in original functional equation, we get 

$ c=1,d=0 \rightarrow \boxed{f(x)=\frac{x}{2}} $
\end{solution}



\begin{solution}[by \href{https://artofproblemsolving.com/community/user/215607}{hoangthailelqd}]
	thank you BSJL, since $ P(x,0)\Rightarrow h(x)\ge-g(0) $ , can you tell me why h(x) is a solution of Cauchy equation with a lower bound on some non empty open interval?
----------------------
sory for my bad English
\end{solution}



\begin{solution}[by \href{https://artofproblemsolving.com/community/user/29428}{pco}]
	\begin{tcolorbox}thank you BSJL, since $ P(x,0)\Rightarrow h(x)\ge-g(0) $ , can you tell me why h(x) is a solution of Cauchy equation with a lower bound on some non empty open interval?
----------------------
sory for my bad English\end{tcolorbox}
Uhhhh ?

$h(x)$ has a lower bound ($-g(0)$) over a non empty interval ($\mathbb R$). What is the unclear thing ?
\end{solution}



\begin{solution}[by \href{https://artofproblemsolving.com/community/user/223507}{abyss1616}]
	\begin{tcolorbox}[quote="hoangthailelqd"]find all functions f:R->R such that:
$ f(x^{2}+2f(y))=\frac{y}{2}+2(f(x))^{2} $\end{tcolorbox}

At first, we consider the function $ g: \mathbb{R} \rightarrow \mathbb{R} $ such that $ g(x)=2f(x) $

Then the condition becomes $ g(x^2+g(y))=y+g(x)^2 $

Let $ P(x,y) $ be the assertion of $ g(x^2+g(y))=y+g(x)^2 $

$ P(x,0),P(0,y) \Rightarrow g(x^2+y)=g(y)+g(x^2)-g(0) $

Now, let $ h(x)=g(x)-g(0) $ then we get $ h(x+y)=h(x)+h(y) $ $ \forall x,y \in \mathbb{R} $  immediately.

$ P(x,0) \Rightarrow h(x) \ge -g(0) $ $ \forall x \ge f(0) $, which means $ h(x) $ is a solution of Cauchy equation with a lower bound on some non empty open interval.

Therefore, $ h(x)=cx $ where $ c $ is a constant $ \rightarrow g(x)=cx+d $

Finally, plugging this result back in original functional equation, we get 

$ c=1,d=0 \rightarrow \boxed{f(x)=\frac{x}{2}} $\end{tcolorbox}

sorry 

in your proof,how did you derive $ g(x^2+y)=g(y)+g(x^2)-g(0) $ ?

plz explain in detail
\end{solution}



\begin{solution}[by \href{https://artofproblemsolving.com/community/user/212018}{Tintarn}]
	As you did the substitution, $g(x)=2f(x)$, from there on, it's just IMO 1992 Problem 2:
http://www.artofproblemsolving.com/Forum/viewtopic.php?p=366399&sid=5c65e59119fd6f30402d1e2d024e841c#p366399
\end{solution}
*******************************************************************************
-------------------------------------------------------------------------------

\begin{problem}[Posted by \href{https://artofproblemsolving.com/community/user/215607}{hoangthailelqd}]
	find all functions continous f,g: R->R such that:
$ f(x)-f(y)=(x^{2}-y^{2})g(x-y) $
	\flushright \href{https://artofproblemsolving.com/community/c6h603097}{(Link to AoPS)}
\end{problem}



\begin{solution}[by \href{https://artofproblemsolving.com/community/user/29428}{pco}]
	\begin{tcolorbox}find all functions continous f,g: R->R such that:
$ f(x)-f(y)=(x^{2}-y^{2})g(x-y) $\end{tcolorbox}
Let $P(x,y)$ be the assertion $f(x)-f(y)=(x^2-y^2)g(x-y)$
Let $a=f(0)$
Let $h(x)=xg(x)$ so that $h(0)=0$

$P(x,0)$ $\implies$ $f(x)=a+xh(x)$ and so new assertion $Q(x,y)$ : $xh(x)-yh(y)=(x+y)h(x-y)$

$Q(2x,x)$ $\implies$ $h(2x)=2h(x)$ (using the fact that $h(0)=0$)
$Q(3x,x)$ $\implies$ $h(3x)=3h(x)$ (using the fact that $h(0)=0$)
And easy indution implies $h(nx)=nh(x)$ $\forall n\in\mathbb Z$
So $h(x)=xh(1)$$\forall x\in\mathbb Q$ and continuity implies $h(x)=bx$ $\forall x\in\mathbb R$

Hence the solution $\boxed{f(x)=a+bx^2,g(x)=b}$ $\forall x$, which indeed is a solution, whatever are $a,b\in\mathbb R$
\end{solution}



\begin{solution}[by \href{https://artofproblemsolving.com/community/user/157383}{jowramos}]
	Alternatively, we know that $\frac{f(x+h)-f(x)}{h} = (2x+h)g(h)$. Taking the limit when $h \rightarrow 0 $ on the right, we see that we obtain $2x\cdot g(0)$. So, $f$ must be differentiable (as we just proved that the limit of the incremental quotients exists), with $f'(x)=2cx$, for some $c \in \mathbb{R}$. It follows that $f(x) = cx^2 + c'$ and, thus, that $g(x)\equiv c$.
\end{solution}
*******************************************************************************
-------------------------------------------------------------------------------

\begin{problem}[Posted by \href{https://artofproblemsolving.com/community/user/220006}{Dimasep}]
	find all F:R+ ----> R+ which satisfy $f(x)f(y) = f(y)*f(x*f(y)) +\frac{1}{xy}$ for all real positive numbers x and y
	\flushright \href{https://artofproblemsolving.com/community/c6h604094}{(Link to AoPS)}
\end{problem}



\begin{solution}[by \href{https://artofproblemsolving.com/community/user/29428}{pco}]
	\begin{tcolorbox}find all F:R+ ----> R+ which satisfy $f(x)f(y) = f(y)*f(x*f(y)) +\frac{1}{xy}$ for all real positive numbers x and y\end{tcolorbox}
Let $P(x,y)$ be the assertion $f(x)f(y)=f(y)f(xf(y))+\frac 1{xy}$
Let $a=f(1)$

$P(1,x)$ $\implies$ $f(f(x))=a-\frac 1{xf(x)}$

$P(f(x),1)$ $\implies$ $f(af(x))=a-\frac 1{xf(x)}-\frac 1{af(x)}$
$P(a,x)$ $\implies$ $f(af(x))=a-\frac 1{a}-\frac 1{axf(x)}$

Subtracting, we get $f(x)=1+\frac{a-1}x$ and, plugging this back in original equation, we get $a=2$

And so the solution $\boxed{f(x)=1+\frac 1x}$ $\forall x$
\end{solution}
*******************************************************************************
-------------------------------------------------------------------------------

\begin{problem}[Posted by \href{https://artofproblemsolving.com/community/user/215205}{BEHZOD_UZ}]
	Find all functions $f:\mathbb{R} \to \mathbb{R}$ such that
$ f(x+f(y))+f(2+f(y)-x)=y(x-1)f(y), \forall x,y \in \mathbb{R} $
	\flushright \href{https://artofproblemsolving.com/community/c6h604132}{(Link to AoPS)}
\end{problem}



\begin{solution}[by \href{https://artofproblemsolving.com/community/user/215607}{hoangthailelqd}]
	easy to prove that f is Injective and Surjective
then exist a such that f(a)=0 
for x=y=a, P(a,a)=> f(1)=0 
next P(1,y) we have f(1+f(y))=0=f(1) hence f(x)=0 ( because f is Injective )
\end{solution}



\begin{solution}[by \href{https://artofproblemsolving.com/community/user/29428}{pco}]
	\begin{tcolorbox}easy to prove that f is Injective and Surjective
...
 hence f(x)=0\end{tcolorbox}
And you are not surprised that your result is neither surjective, neither injective ?
\end{solution}



\begin{solution}[by \href{https://artofproblemsolving.com/community/user/217254}{loverofpuremath}]
	Put x=1 and we have f(1+f(y))=0 then we denote  g(x) such that g(x)=f(x)+1 and we have g(g(x))=1 then g(x) is const  and we have g(x)=1 so f(x)=0
\end{solution}



\begin{solution}[by \href{https://artofproblemsolving.com/community/user/212018}{Tintarn}]
	\begin{tcolorbox}...and we have g(g(x))=1 then g(x) is const  and we have g(x)=1 so f(x)=0\end{tcolorbox}
$g(g(x))=1$ for every $x$ does NOT imply that $g(x)=1$ for every $x$.
Counterexample:
$g(x)=1$ where $x=0$ or $x=1$ and $g(x)=0$ everywhere else...
\end{solution}



\begin{solution}[by \href{https://artofproblemsolving.com/community/user/157383}{jowramos}]
	Plugging in $(x,y) = (2,y)$, we have $f(f(y))+f(2+f(y)) = yf(y)$. Plugging $(0,y)$, we have $f(f(y))+f(2+f(y)) = -yf(y)$, which implies $f(y) = 0$ for all $y \ne 0$. For zero, let $(x,y) = (2,1)$, and we have $f(0)=0$, concluding the problem.
\end{solution}



\begin{solution}[by \href{https://artofproblemsolving.com/community/user/212043}{Mikasa}]
	Lets us denote the given statement by $P(x,y)$. Then,
$P(1,y)\Rightarrow 2f(f(y)+1)=0\Rightarrow f(f(y)+1)=0$.
$P(x,f(y)+1)\Rightarrow f(x)+f(2-x)=0\Rightarrow f(2-x)=-f(x)$.
So we rewrite $P(x,y)$ as $f(x+f(y))-f(x-f(y))=y(x-1)f(y)$ since $f(2+f(y)-x)=f(2-(x-f(y)))=-f(x-f(y))$.
Also, $f(1)=f(2-1)=-f(1)\Rightarrow f(1)=0$. Now,
$P(x,2-y)\Rightarrow f(x+f(2-y))+f(2+f(2-y)-x)=(2-y)(x-1)f(2-y)$
$\Rightarrow f(x-f(y))+f(2-x-f(y))=-(2-y)(x-1)f(y)$
$\Rightarrow f(x-f(y))-f(x+f(y))=(y-2)(x-1)f(y)$. 
Now adding the respective expressions for $P(x,y)$ and $P(x,2-y)$ we get,
$2(y-1)(x-1)f(y)=0$. Take $x\neq 1$ then $(y-1)f(y)=0$ so $f(y)=0\forall y\neq 1$. We also know that $f(1)=0$. So the solution is $f(x)=0\forall x\in \mathbb{R}$.
\end{solution}



\begin{solution}[by \href{https://artofproblemsolving.com/community/user/215607}{hoangthailelqd}]
	Hi pco, it is :)
$ f(x+f(y))+f(2+f(y)-x)=y(x-1)f(y),\forall x,y\in\mathbb{R}(1) $
+If $ f(y_{1})=f(y_{2}) $ , from (1) we have $ y_{1}=y_{2} $ => f is injective
+For y=1 , because LHS is Linear of x so f is Surjective
\end{solution}



\begin{solution}[by \href{https://artofproblemsolving.com/community/user/29428}{pco}]
	\begin{tcolorbox}Hi pco, it is :)
$ f(x+f(y))+f(2+f(y)-x)=y(x-1)f(y),\forall x,y\in\mathbb{R}(1) $
+If $ f(y_{1})=f(y_{2}) $ , from (1) we have $ y_{1}=y_{2} $ => f is injective
+For y=1 , because LHS is Linear of x so f is Surjective\end{tcolorbox}
Are you joking ?
$f(x)=0$ $\forall x$ is a solution.
And is neither injective, neither surjective.

So ...
\end{solution}



\begin{solution}[by \href{https://artofproblemsolving.com/community/user/307628}{lebathanh}]
	 :rotfl:  :rotfl:  :wacko: 
\end{solution}
*******************************************************************************
-------------------------------------------------------------------------------

\begin{problem}[Posted by \href{https://artofproblemsolving.com/community/user/214878}{rocklee213}]
	$ f:(1;+\infty) \to \mathbb{R} $
$f(x)-f(y)=(y-x)f(xy)$
Find f
	\flushright \href{https://artofproblemsolving.com/community/c6h605874}{(Link to AoPS)}
\end{problem}



\begin{solution}[by \href{https://artofproblemsolving.com/community/user/29428}{pco}]
	\begin{tcolorbox}$ f:(1;+\infty) \to \mathbb{R} $
$f(x)-f(y)=(y-x)f(xy)$
Find f\end{tcolorbox}
Without any precision, I suppose that domain of functional equation is also $(1,+\infty)$
Let $P(x,y)$ be the assertion $f(x)-f(y)=(y-x)f(xy)$

Let $1<a<b$
Let $x\in(\frac ba,ab)$

$P(\frac{\sqrt{abx}}a,\frac{\sqrt{abx}}b)$ $\implies$ $f(\frac{\sqrt{abx}}a)-f(\frac{\sqrt{abx}}b)=(\frac{\sqrt{abx}}b-\frac{\sqrt{abx}}a)f(x)$

$P(\frac{\sqrt{abx}}b,\frac{\sqrt{abx}}x)$ $\implies$ $f(\frac{\sqrt{abx}}b)-f(\frac{\sqrt{abx}}x)=(\frac{\sqrt{abx}}x-\frac{\sqrt{abx}}b)f(a)$

$P(\frac{\sqrt{abx}}x,\frac{\sqrt{abx}}a)$ $\implies$ $f(\frac{\sqrt{abx}}x)-f(\frac{\sqrt{abx}}a)=(\frac{\sqrt{abx}}a-\frac{\sqrt{abx}}x)f(b)$

Adding these three lines, we get $f(x)=\frac{af(a)-bf(b)}{a-b}+\frac{ab(f(b)-f(a))}{(a-b)x}$ 

Varying $a,b$ so that we cover $(1,+\infty)$ with infinitely many overlaping intervals $(\frac ba,ab)$, we get ${f(x)=u+\frac vx}$ for some $u,v$ and plugging this back in original equation, we get $u=0$

Hence the answer : $\boxed{f(x)=\frac vx}$ $\forall x>1$ and whatever is $v\in\mathbb R$
\end{solution}



\begin{solution}[by \href{https://artofproblemsolving.com/community/user/162790}{mathjmk33}]
	\begin{tcolorbox}Adding these three lines, we get $f(x)=\frac{af(a)-bf(b)}{a-b}+\frac{ab(f(b)-f(a))}{(a-b)x}$ 

Varying $a,b$ so that we cover $(1,+\infty)$ with infinitely many overlaping intervals $(\frac ba,ab)$, we get ${f(x)=u+\frac vx}$ for some $u,v$ and plugging this back in original equation, we get $u=0$

Hence the answer : $\boxed{f(x)=\frac vx}$ $\forall x>1$ and whatever is $v\in\mathbb R$\end{tcolorbox}

If $a,b$ change, the value $\frac{af(a)-bf(b)}{a-b}$ and $\frac{ab(f(b)-f(a))}{a-b}$ also change; can we say $u,v$ are constant?
\end{solution}



\begin{solution}[by \href{https://artofproblemsolving.com/community/user/29428}{pco}]
	If $u_1+\frac{v_1}x=u_2+\frac{v_2}x$ $\forall x\in$ "some non empty open interval", then trivially $u_1=u_2$ and $v_1=v_2$
\end{solution}
*******************************************************************************
-------------------------------------------------------------------------------

\begin{problem}[Posted by \href{https://artofproblemsolving.com/community/user/188879}{chomk}]
	Find all functions $f:\mathbb{R}\backslash\{0\}\to\mathbb{R}$ such that for all nonzero numbers $x,y$ with $x\neq y$  we have
\[f(y)-f(x)=f(y)f\left (\frac{x}{x-y}\right)\]
	\flushright \href{https://artofproblemsolving.com/community/c6h606161}{(Link to AoPS)}
\end{problem}



\begin{solution}[by \href{https://artofproblemsolving.com/community/user/29428}{pco}]
	\begin{tcolorbox}Find all functions $f:\mathbb{R}\backslash\{0\}\to\mathbb{R}$ such that for all nonzero numbers $x,y$ with $x\neq y$  we have
\[f(y)-f(x)=f(y)f\left (\frac{x}{x-y}\right)\]\end{tcolorbox}
$\boxed{\text{S1: }f(x)=0\text{ }\forall x\ne 0}$ is a solution. So let us from now look for non allzero solutions.

Let $u\ne 0$ such that $f(u)\ne 0$
Let $a=f(1)$
Let $P(x,y)$ be the assertion $f(y)-f(x)=f(y)f(\frac x{x-y})$

If $\exists x\notin\{0,u\}$ such that $f(x)=0$, then $P(u,x)$ $\implies$ $f(u)=0$, impossible. So $f(x)\ne 0$ $\forall x\ne 0$
Let then :
 $g(x)$ from $\mathbb R\setminus\{0\}\to\mathbb R\setminus\{0\}$ defined as $g(x)=\frac{a}{f(x)}$
 $h(x)$ from $\mathbb R\setminus\{0,1\}\to\mathbb R$ defined as $h(x)=1-f(\frac x{x-1})$

Let $x,y\notin\{0,1\}$ :
$P(y,1)$ $\implies$ $f(y)=f(1)h(y)$
$P(xy,x)$ $\implies$ $f(xy)=f(x)h(y)$
And so $f(xy)=\frac 1{a}f(x)f(y)$, still true when $x=1$ or $y=1$

So $g(xy)=g(x)g(y)$ $\forall x,y\ne 0$ and $g(1)=1$ and so $g(x)$ is a nonzero multiplicative function

Using this, $P(x,y)$ may be written as new assertion $Q(x,y)$ : $g(x-y)=\frac 1ag(x)-\frac 1ag(y)$ $\forall x\ne y\notin\{0\}$

Let $x\ne 0$ and $y\notin\{0,-x\}$
$Q(x+y,y)$ $\implies$ $g(x)=\frac 1ag(x+y)-\frac 1ag(y)$
$Q(y,x+y)$ $\implies$ $g(-x)=\frac 1ag(y)-\frac 1ag(x+y)$
Adding these two lines, we get $g(-x)=-g(x)$ $\forall x\ne 0$

Let then $x,y$ such that $x,y,x+y\ne 0$ and $z\notin\{0,-x,-y,-x-y\}$
$Q(x+y+z,y+z)$ $\implies$ $g(x)=\frac 1ag(x+y+z)-\frac 1ag(y+z)$
$Q(y+z,z)$ $\implies$ $g(y)=\frac 1ag(y+z)-\frac 1ag(z)$
$Q(x+y+z,z)$ $\implies$ $g(x+y)=\frac 1ag(x+y+z)-\frac 1ag(z)$
Adding the two first lines and subtracting the third, we get $g(x+y)=g(x)+g(y)$ $\forall x,y$ such that $x,y,x+y\notin\{0\}$
As a consequence, $f(1)=1$

It's then immediate to extend $g(x)$ to $\mathbb R$ with $g(0)=0$ and we get a non allzero additive and multiplicative function with classical result $g(x)=x$

And so $\boxed{\text{S2: }f(x)=\frac 1x\text{ }\forall x\ne 0}$
\end{solution}
*******************************************************************************
-------------------------------------------------------------------------------

\begin{problem}[Posted by \href{https://artofproblemsolving.com/community/user/76150}{-SIL-}]
	Find all $f:R\rightarrow R$ and $f$ is continuous on $R$ and satisfying the equation
\[f(f(f(x)))-3f(x)+2x=0\]
Please show me the solution or give me some hints.

[hide="Something that I found & hope"]

1. $f$ is injective
2. $f$ is strictly monotone
3. I hope that the equation has to form of $f$ that is $f(x)=x+c_1,-2x+c_2$  for all $c_1,c_2\in R$
[\/hide]
	\flushright \href{https://artofproblemsolving.com/community/c6h606182}{(Link to AoPS)}
\end{problem}



\begin{solution}[by \href{https://artofproblemsolving.com/community/user/29428}{pco}]
	\begin{tcolorbox}Find all $f:R\rightarrow R$ and $f$ is continuous on $R$ and satisfying the equation
\[f(f(f(x)))-3f(x)+2x=0\]
Please show me the solution or give me some hints.
\end{tcolorbox}
It's immediate to get that $f(x)$ is injective and so, since continuous, strictly monotonous.

Continuity and functional equation imply also that $f(x)$ can be neither upper bounded, neither lower bounded and so is surjective, so bijective.

Easy induction implies then $a_n(x)=f^{n+1}(x)-f^{n}(x)=\frac{f(f(x))+f(x)-2x}3+(-2)^n\frac{-f(f(x))+2f(x)-x}3$ $\forall n\in\mathbb Z$ (using bijectivity for $n<0$)

1) if $f(x)$ is increasing
Let then $x$ such that $f(x)\ne x$ so that $a_n(x)\ne 0$ $\forall n\ge 0$. Then $\frac{a_{n+1}(x)}{a_n(x)}>0$ and so $f(f(x))=2f(x)-x$ (else setting $n\to +\infty$ leads to $\frac{a_{n+1}(x)}{a_n(x)}\to -2<0$)

So we got $f(f(x))=2f(x)-x$ $\forall x$ (even $x$ such that $f(x)=x$) which is quite classical (see for example http://www.artofproblemsolving.com/Forum/viewtopic.php?p=2721097#p2721097 subcase 7.3.2) with solution $\boxed{f(x)=x+a}$ which indeed is a solution whatever is $a\in\mathbb R$

2) If $f(x)$ is decreasing
Same mechanism as above but using $n\to -\infty$ instead of $n\to +\infty$ implies $f(f(x))=-f(x)+2x$ $\forall x$ which is also quite classical (see again for example http://www.artofproblemsolving.com/Forum/viewtopic.php?p=2721097#p2721097 subcase 7.1.2) with solution $\boxed{f(x)=a-2x}$ which indeed is a solution whatever is $a\in\mathbb R$
\end{solution}
*******************************************************************************
-------------------------------------------------------------------------------

\begin{problem}[Posted by \href{https://artofproblemsolving.com/community/user/221848}{yassino}]
	Determine all functions  such that 
1) $ f(2x) =f(x-y)f(y-x) + f(x-y)f(-x-y) $
2) $ f(x) \ge 0 $
	\flushright \href{https://artofproblemsolving.com/community/c6h606184}{(Link to AoPS)}
\end{problem}



\begin{solution}[by \href{https://artofproblemsolving.com/community/user/29428}{pco}]
	\begin{tcolorbox}Determine all functions  such that 
1) $ f(2x) =f(x-y)f(y-x) + f(x-y)f(-x-y) $
2) $ f(x) \ge 0 $\end{tcolorbox}
Let $P(x,y)$ be the assertion $f(2x)=f(x-y)f(y-x)+f(x-y)f(-x-y)$

Subtracting $P(0,-x)$ from $P(0,x)$, we get  $f(x)^2=f(-x)^2$ and so, since $\ge 0$, $f(-x)=f(x)$ $\forall x$

Then $P(0,-x)$ $\implies$ $f(x)=\sqrt{\frac{f(0)}2}$ $\forall x$

Plugging back in original equation we get the two solutions

$\boxed{\text{S1 :}f(x)=0\text{  }\forall x}$

$\boxed{\text{S2 :}f(x)=\frac 12\text{  }\forall x}$
\end{solution}
*******************************************************************************
-------------------------------------------------------------------------------

\begin{problem}[Posted by \href{https://artofproblemsolving.com/community/user/223670}{renatrino}]
	Find a non\end{underlined}-piecewise-defined function $f: \mathbb{N}_{\ne 0} \rightarrow \mathbb{N}_{\ne 0}$ s.t.
\[f(x) = 
  \begin{cases}
    x   & \text{if } 1 \le x \le 20\\
    20 & \text{if }  x > 20
  \end{cases}
\]

Please don't give complete solutions if you have, just hints :)

Thanks in advance
	\flushright \href{https://artofproblemsolving.com/community/c6h606203}{(Link to AoPS)}
\end{problem}



\begin{solution}[by \href{https://artofproblemsolving.com/community/user/29428}{pco}]
	\begin{tcolorbox}Find a non\end{underlined}-piecewise-defined function $f: \mathbb{N}_{\ne 0} \rightarrow \mathbb{N}_{\ne 0}$ s.t.
\[f(x) = 
  \begin{cases}
    x   & \text{if } 1 \le x \le 20\\
    20 & \text{if }  x > 20
  \end{cases}
\]

Please don't give complete solutions if you have, just hints :)

Thanks in advance\end{tcolorbox}
I dont know what is a "non-piecewise defined function" ...

[hide="Full solution 1"]$f(x)=\min(x,20)$[\/hide]

[hide="Full solution 2"]$f(x)=\frac{x+20-|x-20|}2$[\/hide]

[hide="Full solution 3"]$f(x)=\frac{x+20-\sqrt{x^2-40x+400}}2$[\/hide]

[hide="Full solution 4"]$f(x)=\frac{40x}{x+20+\sqrt{x^2-40x+400}}$[\/hide]
\end{solution}
*******************************************************************************
-------------------------------------------------------------------------------

\begin{problem}[Posted by \href{https://artofproblemsolving.com/community/user/162032}{MrRTI}]
	Find all function $f$ which satisfy $f : \mathbb{Q} \rightarrow \mathbb{Q}$ and
\[f(x+y) + f(x-y) = 2f(x) + 2f(y)      \forall x,y \in \mathbb{Q} \]
	\flushright \href{https://artofproblemsolving.com/community/c6h606744}{(Link to AoPS)}
\end{problem}



\begin{solution}[by \href{https://artofproblemsolving.com/community/user/29428}{pco}]
	\begin{tcolorbox}Find all function $f$ which satisfy $f : \mathbb{Q} \rightarrow \mathbb{Q}$ and
\[f(x+y) + f(x-y) = 2f(x) + 2f(y)      \forall x,y \in \mathbb{Q} \]\end{tcolorbox}
Setting $x=y=0$, we get $f(0)=0$

From $f((n+2)x)+f(nx)=2f((n+1)x)+2f(x)$, easy induction gives $f(nx)=n^2f(x)$

And so $f(x)=cx^2$ $\forall x\in\mathbb Q$ which indeed is a solution, whatever is $c\in\mathbb Q$
\end{solution}



\begin{solution}[by \href{https://artofproblemsolving.com/community/user/20689}{BOGTRO}]
	Let $P(x,y)$ be the given assertion. Then $P(0,0)$ tells us that $f(0)=0$. Define the function $g(x)$ to be such that $g(0)=0$ and $g(x)=\frac{f(x)}{x^2}$ for all rational $x \neq 0$. Then 
\[f(x)=x^2g(x)\]
\[\implies f(x+y)+f(x-y)=(x+y)^2g(x+y)+(x-y)^2g(x-y)\]
\[=2f(x)+2f(y)=2x^2g(x)+2y^2g(y).\] 

Then $P(x,x)$ tells us $4x^2g(2x)=4x^2g(x)$, thus $g(2x)=g(x)$ for all rational $x$ (including, obviously, $x=0$). As such $g$ is a constant. Checking $f(x)=cx^2$ confirms that it is indeed a solution. Thus our solution set is $f(x)=cx^2$ for all rational constants $c$.
\end{solution}



\begin{solution}[by \href{https://artofproblemsolving.com/community/user/29428}{pco}]
	\begin{tcolorbox}L...thus $g(2x)=g(x)$ for all rational $x$ (including, obviously, $x=0$). As such $g$ is a constant.....\end{tcolorbox}
There are a lot of non constant functions from $\mathbb Q\to\mathbb Q$ such that $g(2x)=g(x)$ ...
\end{solution}



\begin{solution}[by \href{https://artofproblemsolving.com/community/user/162032}{MrRTI}]
	Could you show me the detail of your induction?
\end{solution}



\begin{solution}[by \href{https://artofproblemsolving.com/community/user/29428}{pco}]
	\begin{tcolorbox}Could you show me the detail of your induction?\end{tcolorbox}
Let $a_n=f(nx)$
We got $a_{n+2}=2a_{n+1}-a_n+2a_1$

Which classically (look at characteristic equation) gives $a_n=a_1n^2-a_0n+a_0$ and since $a_0=0$, we get $a_n=a_1n^2$
\end{solution}



\begin{solution}[by \href{https://artofproblemsolving.com/community/user/213278}{shmm}]
	Problem was in the Nordic 1998 contest.
\end{solution}
*******************************************************************************
-------------------------------------------------------------------------------

\begin{problem}[Posted by \href{https://artofproblemsolving.com/community/user/195015}{Jul}]
	Find all $f:\left ( 1,+\infty \right )\rightarrow \left ( 1,+\infty \right )$ and sastify :
\[f(x^y)=f(x)^{f(y)},\;\forall x,y\in \left ( 1,+\infty \right )\]
	\flushright \href{https://artofproblemsolving.com/community/c6h606853}{(Link to AoPS)}
\end{problem}



\begin{solution}[by \href{https://artofproblemsolving.com/community/user/29428}{pco}]
	\begin{tcolorbox}Find all $f:\left ( 1,+\infty \right )\rightarrow \left ( 1,+\infty \right )$ and sastify :
\[f(x^y)=f(x)^{f(y)},\;\forall x,y\in \left ( 1,+\infty \right )\]\end{tcolorbox}
$f(2^{xy})=f(2)^{f(xy)}$
$f(2^{xy})=f((2^x)^y)=f(2^x)^{f(y)}=f(2)^{f(x)f(y)}$

So $f(xy)=f(x)f(y)$ 
So $f(x)$ is strictly increasing
So, multiplicative, $>1$ and increasing, $f(x)=x^c$ and, plugging back in original equation, we get $c=1$ and so $\boxed{f(x)=x}$ $\forall x>1$
\end{solution}
*******************************************************************************
-------------------------------------------------------------------------------

\begin{problem}[Posted by \href{https://artofproblemsolving.com/community/user/225003}{Jeje}]
	Find all functions $f : \mathbb{R} \to \mathbb{R}$ that satisfy $f(f(x + y)) = f(x + y) + f(x)f(y) - xy$ for all real numbers $x, y$.
	\flushright \href{https://artofproblemsolving.com/community/c6h607396}{(Link to AoPS)}
\end{problem}



\begin{solution}[by \href{https://artofproblemsolving.com/community/user/29428}{pco}]
	\begin{tcolorbox}Find all functions $f : \mathbb{R} \to \mathbb{R}$ that satisfy $f(f(x + y)) = f(x + y) + f(x)f(y) - xy$ for all real numbers $x, y$.\end{tcolorbox}
Let $P(x,y)$ be the assertion $f(f(x+y))=f(x+y)+f(x)f(y)-xy$
$f(x)=0$ $\forall x$ is not a solution. So let $u$ such that $f(u)\ne 0$
Let $a=f(0)$

$P(0,0)$ $\implies$ $f(a)=a^2+a$
$P(a,-a)$ $\implies$ $f(a)=a^2+a+f(a)f(-a)$
Subtracting, we get $f(a)f(-a)=0$ and so $\exists v$ such that $f(v)=0$

$P(v,0)$ $\implies$ $f(0)=0$

$P(x+u,0)$ $\implies$ $f(f(x+u))=f(x+u)$
$P(x,u)$ $\implies$ $f(f(x+u))=f(x+u)+f(x)f(u)-ux$
Subtracting, and since $f(u)\ne 0$, we get $f(x)=cx$ $\forall x$, where $c=\frac u{f(u)}$

Plugging this back in original equation, we get $c=1$ and the unique solution $\boxed{f(x)=x}$ $\forall x$
\end{solution}
*******************************************************************************
-------------------------------------------------------------------------------

\begin{problem}[Posted by \href{https://artofproblemsolving.com/community/user/195015}{Jul}]
	Find all $f:\mathbb{R}\rightarrow \mathbb{R},g:\mathbb{R}\rightarrow \mathbb{R}$ with $g$ is injective and sastify :
\[f(xg(y))+g(yf(x))=2f(x)+2xy-2x,\;\forall x,y\in \mathbb{R}\]

I have a solution for this problem, but I don't use injectivity of $g$. I wonder if the injectivity of $g$ is important ?
	\flushright \href{https://artofproblemsolving.com/community/c6h607439}{(Link to AoPS)}
\end{problem}



\begin{solution}[by \href{https://artofproblemsolving.com/community/user/29428}{pco}]
	\begin{tcolorbox}Find all $f:\mathbb{R}\rightarrow \mathbb{R},g:\mathbb{R}\rightarrow \mathbb{R}$ with $g$ is injective and sastify :
\[f(xg(y))+g(yf(x))=2f(x)+2xy-2x,\;\forall x,y\in \mathbb{R}\]

I have a solution for this problem, but I don't use injectivity of $g$. I wonder if the injectivity of $g$ is important ?\end{tcolorbox}
Let $P(x,y)$ be the assertion $f(xg(y))+g(yf(x))=2f(x)+2xy-2x$
$g(x)=c$ constant is never a solution and so $g(x)$ is not constant.
Let $a=f(0)$

$P(0,0)$ $\implies$ $g(0)=a$
$P(0,x)$ $\implies$ $g(ax)=a$ and so $a=0$ since $g(x)$ is not constant.

$P(x,0)$ $\implies$ $f(x)=x$ $\forall x$

Then $P(1,x)$ $\implies$ $g(x)=x$

And so $\boxed{(f,g)=(Id,Id)}$ which indeed is a solution.

And I did not use the injectivity since I first proved that $g(x)$ is not constant (I could have skipped this step using injectivity).
\end{solution}
*******************************************************************************
-------------------------------------------------------------------------------

\begin{problem}[Posted by \href{https://artofproblemsolving.com/community/user/215205}{BEHZOD_UZ}]
	Find all functions$f:R^+\to R$  that satisfy:
$  f'(x)=f(\frac{1}{x}) $ 
 for all $x\in R^+$.
	\flushright \href{https://artofproblemsolving.com/community/c6h607446}{(Link to AoPS)}
\end{problem}



\begin{solution}[by \href{https://artofproblemsolving.com/community/user/29428}{pco}]
	\begin{tcolorbox}Find all functions$f:R^+\to R$  that satisfy:
$  f'(x)=f(\frac{1}{x}) $ 
 for all $x\in R^+$.\end{tcolorbox}
sketch of result :

Let $f(x)=g(\ln x)$ and equation implies $g(x)-g'(x)+g"(x)=0$

Two roots of $x^2-x+1=0$ are $\frac 12\pm i\frac{\sqrt 3}2$ and so  $g(x)=\alpha e^{\frac x2}\cos\frac{\sqrt 3}2x$ $ + \beta e^{\frac x2}\sin\frac{\sqrt 3}2x$

Plugging back in equation, we get $\alpha=\beta\sqrt 3$

And so $\boxed{f(x)=\beta\left( \sqrt{3x}\cos\left(\frac{\sqrt 3}2\ln x\right) +  \sqrt x\sin\left(\frac{\sqrt 3}2\ln x\right)\right)}$
\end{solution}
*******************************************************************************
-------------------------------------------------------------------------------

\begin{problem}[Posted by \href{https://artofproblemsolving.com/community/user/10045}{socrates}]
	Determine all functions $f : \mathbb{R} \to \mathbb{R}$ such that \[f(xy+f(x))=f(x)f(y)+x , \ \ \ \forall x,y \in  \mathbb{R}.\]
	\flushright \href{https://artofproblemsolving.com/community/c6h607692}{(Link to AoPS)}
\end{problem}



\begin{solution}[by \href{https://artofproblemsolving.com/community/user/203512}{SamISI1}]
	This problem is wrong or no solution put $ y=1; x=0 $ and $ x=y=0 $
\end{solution}



\begin{solution}[by \href{https://artofproblemsolving.com/community/user/10045}{socrates}]
	How about $f(x)=x?$
\end{solution}



\begin{solution}[by \href{https://artofproblemsolving.com/community/user/29428}{pco}]
	\begin{tcolorbox}Determine all functions $f : \mathbb{R} \to \mathbb{R}$ such that \[f(xy+f(x))=f(x)f(y)+x , \ \ \ \forall x,y \in  \mathbb{R}.\]\end{tcolorbox}
Let $P(x,y)$ be the assertion $f(xy+f(x))=f(x)f(y)+x$

If $f(0)\ne 0$, $P(0,x)$ $\implies$ $f(x)=\frac{f(f(0)}{f(0)}$ constant, which is never a solution. So $f(0)=0$
$P(x,0)$ $\implies$ $f(f(x))=x$ and so $f(x)$ is bijective and involutive.

$P(1,1)$ $\implies$ $f(f(1)+1)=f(1)^2+1$
$P(f(1),1)$ $\implies$ $f(f(1)+1)=2f(1)$
Subtracting, we get $f(1)=1$

$P(1,x)$ $\implies$ $f(x+1)=f(x)+1$

Let $x\ne 0$ :
$P(x,\frac{y-f(x)}x)$ $\implies$ $f(y)=f(x)f(\frac{y-f(x)}x)+x$
$P(x,1+\frac{y-f(x)}x)$ $\implies$ $f(x+y)=f(x)f(\frac{y-f(x)}x)+x+f(x)$
Subtracting, we get $f(x+y)=f(x)+f(y)$ $\forall x\ne 0$, $\forall y$, still true when $x=0$

So $f(xy+f(x))=f(xy)+f(f(x))=f(xy)+x$
Comparing with $P(x,y)$, we get $f(xy)=f(x)f(y)$

So $f(x)$ is bijective, additive and multiplicative and it's immediate to get $\boxed{f(x)=x}$ $\forall x$, which indeed is a solution.
\end{solution}



\begin{solution}[by \href{https://artofproblemsolving.com/community/user/173116}{Sardor}]
	Nice and easy function,thank you $ socrates $.
Here my solution: It's easy to see that $ f $ isn't constant.Let $ P(x,y) $ be the assertion of the function and $ f(0)=c, f(1)=a $.
i) $ P(0,y) \implies f(c)=cf(y) $ and $ P(0,0) \implies f(c)=c^2 $.Hence $ f(0)=c=0 $ ( since $ f \neq const $.
ii) $ P(x,0) \implies f(f(x))=x (*) \implies f $ is bijective.
iii) $ P(x,1) \implies f(x+f(x))=f(x)a+x $
iv)  $ P(f(x),1) \implies f(f(x)+x)=xa+f(x) $ ( i use here $ (*) $ ).So that from iii and iv $ (f(x)-x)(a-1) $.If $ f(x)=x $ ,then we are done.Let $ a=1 $.Thus 
$ P(x,1) \implies f(x+f(x))=f(x)+x $ and since $ f $ bijective, we get $ f(x)=x $ is only solution.DONE !
\end{solution}



\begin{solution}[by \href{https://artofproblemsolving.com/community/user/29428}{pco}]
	\begin{tcolorbox}$f(x+f(x))=f(x)+x $ and since $ f $ bijective, we get $ f(x)=x $ is only solution.DONE !\end{tcolorbox}

I dont understand why bijectivity of $f(x)$ implies $f(x)=x$ from $f(x+f(x))=f(x)+x $ ???????

Surjectivity of $f(x)+x$ could help, but you did not prove it.

What about, for example, $f(x)=x$ $\forall x\in\mathbb Q$ and $f(x)=3-x$ $\forall x\notin\mathbb Q$ ??
\end{solution}
*******************************************************************************
-------------------------------------------------------------------------------

\begin{problem}[Posted by \href{https://artofproblemsolving.com/community/user/10045}{socrates}]
	Determine all functions $f : \mathbb{R} \to \mathbb{R}$ such that \[ f^2(x+y)=(y+f(x))(x+f(y)) , \ \ \ \forall x,y \in  \mathbb{R}.  \]
	\flushright \href{https://artofproblemsolving.com/community/c6h607697}{(Link to AoPS)}
\end{problem}



\begin{solution}[by \href{https://artofproblemsolving.com/community/user/29428}{pco}]
	\begin{tcolorbox}Determine all functions $f : \mathbb{R} \to \mathbb{R}$ such that \[ f^2(x+y)=(y+f(x))(x+f(y)) , \ \ \ \forall x,y \in  \mathbb{R}.  \]\end{tcolorbox}
Let $P(x,y)$ be the assertion $f(x+y)^2=(y+f(x))(x+f(y))$
Let $a=f(0)$

$P(-a,0)$ $\implies$ $f(-a)=0$

$P(x,0)$ $\implies$ $\forall x$, either $f(x)=0$; either $f(x)=x+a$

If $f(x)=0$ for some $x$, then :
Let $u\notin\{x,0\}$
$P(u,x-u)$ $\implies$ $(x-u+f(u))(u+f(x-u))=0$ and so four possibilities :
If $f(u)=0$ and $f(x-u)=0$, we get $(x-u)u=0$, impossible
If $f(u)=0$ and $f(x-u)=x-u+a$, we get $(x-u)(x+a)$, and so $x=-a$
If $f(u)=u+a$ and $f(x-u)=0$, we get $(x+a)u=0$, and so $x=-a$
If $f(u)=u+a$ and $f(x-u)=x-u+a$, we get $(x+a)^2=0$, and so $x=-a$

So $f(x)=0$ $\iff$ $x=-a$

And so $\boxed{f(x)=x+a}$ $\forall x$, which indeed is a solution, whatever $a\in\mathbb R$
\end{solution}
*******************************************************************************
-------------------------------------------------------------------------------

\begin{problem}[Posted by \href{https://artofproblemsolving.com/community/user/68025}{Pirkuliyev Rovsen}]
	Find all function $f: \mathbb{R^+}\to\mathbb{R^+}$ such that $f(xf(y))=f(xy)+x$ for all $x,y{\ge}0$.
	\flushright \href{https://artofproblemsolving.com/community/c6h608311}{(Link to AoPS)}
\end{problem}



\begin{solution}[by \href{https://artofproblemsolving.com/community/user/29428}{pco}]
	Hem, for me, $\mathbb R^+$ is the set of positive integers ($x>0$) et so contradiction with your "$x,y\ge 0$"

Could you clarify, please ?
\end{solution}



\begin{solution}[by \href{https://artofproblemsolving.com/community/user/64716}{mavropnevma}]
	Anyways, if $0$ is in the domain, then for $y=0$ and $a=f(0)$ we get $f(ax)=a+x$ for all $x$. This forces $a\neq 0$, and yields $f(x) = a + \dfrac {1}{a}x$. 
So then the problem is extremely easy.
\end{solution}



\begin{solution}[by \href{https://artofproblemsolving.com/community/user/29428}{pco}]
	If we consider only $x,y>0$ :\begin{tcolorbox}Find all function $f: \mathbb{R^+}\to\mathbb{R^+}$ such that $f(xf(y))=f(xy)+x$ for all $x,y>0$.\end{tcolorbox}
Let $P(x,y)$ be the assertion $f(xf(y))=f(xy)+x$
Let $a=f(1)$
Let $a_0=a$
Let $A=f(\mathbb R)$

$P(x,\frac 1x)$ $\implies$ $(a,+\infty)\subseteq A$
$P(1,x)$ $\implies$ $f(x)=x+1$ $\forall x\in A$ and so $f(x)=x+1$ $\forall x>a_0$

If $f(x)=x+1$ $\forall x>a_n$ for some $a_n>0$, then :
Let $x>a_n\frac{a+1}{a+2}$ : $P(\frac x{a+1},a+1)$ $\implies$ $f(\frac x{a+1}f(a+1))=f(x)+\frac x{a+1}$
But $f(a+1)=a+2$ and so $f(\frac x{a+1}f(a+1))=f(x\frac{a+2}{a+1})$ and since $x\frac{a+2}{a+1}>a_n$, this is $x\frac{a+2}{a+1}+1$

So $f(x)+\frac x{a+1}=x\frac{a+2}{a+1}+1$ and so $f(x)=x+1$

So $f(x)=x+1$ $\forall x>a_n$ implies $f(x)=x+1$ $\forall x>a_{n+1}=a_n\frac{a+1}{a+2}$

And so $\boxed{f(x)=x+1}$ $\forall x>0$, which indeed is a solution.
\end{solution}



\begin{solution}[by \href{https://artofproblemsolving.com/community/user/49556}{xxp2000}]
	\begin{tcolorbox}Find all function $f: \mathbb{R^+}\to\mathbb{R^+}$ such that $f(xf(y))=f(xy)+x$ for all $x,y{\ge}0$.\end{tcolorbox}

Let $f(1)=a$. 
$P(x,\frac1x)$ implies $[a,\infty)$ is in the range.
$P(1,y):f(f(y))=f(y)+1$ implies $f(x)=x+1,x>a$.
For any $y$, pick $x>\max(\frac ay,\frac a{f(y)})$, we have
$xf(y)+1=xy+x+1$, so $f(y)=y+1,\forall y$.
\end{solution}



\begin{solution}[by \href{https://artofproblemsolving.com/community/user/10045}{socrates}]
	$ f(f(x)f(y))=f(f(x)y)+f(x)=f(xy)+f(x)+y $ so $f(x)+y=f(y)+x$ so $f(x)=x+c...$
\end{solution}
*******************************************************************************
-------------------------------------------------------------------------------

\begin{problem}[Posted by \href{https://artofproblemsolving.com/community/user/154997}{thiennhan97}]
	Find all continuous functions $f:\mathbb{R} \rightarrow \mathbb{R}$ such that
\[f(x+y+xy)=f(x+y)+f(xy)\] for all $x,y \in \mathbb{R}$
	\flushright \href{https://artofproblemsolving.com/community/c6h608705}{(Link to AoPS)}
\end{problem}



\begin{solution}[by \href{https://artofproblemsolving.com/community/user/29428}{pco}]
	\begin{tcolorbox}Find all continuous functions $f:\mathbb{R} \rightarrow \mathbb{R}$ such that
\[f(x+y+xy)=f(x+y)+f(xy)\] for all $x,y \in \mathbb{R}$\end{tcolorbox}
Let $P(x,y)$ be the assertion $f(x+y+xy)=f(x+y)+f(xy)$

$P(0,0)$ $\implies$ $f(0)=0$
Let $x,y\le 0$. The equation $Z^2-xZ+y=0$ has two real roots $u,v$ and $P(u,v)$ implies $f(x+y)=f(x)+f(y)$ $\forall x,y\le 0$
Continuity implies then $f(x)=cx$ $\forall x\le 0$

Let $x\ge 0$, $P(x+1,-1)$ $\implies$ $f(x)=cx$ 

So $\boxed{f(x)=cx}$ $\forall x$, which indeed is a solution, whatever is $c\in\mathbb R$
\end{solution}
*******************************************************************************
-------------------------------------------------------------------------------

\begin{problem}[Posted by \href{https://artofproblemsolving.com/community/user/142747}{yt12}]
	Find all continuous functions $f:\mathbb{R} \rightarrow \mathbb{R}$ such that
$f(x^2+x+3)+2f(x^2-3x+5)=6x^2-10x+17$
	\flushright \href{https://artofproblemsolving.com/community/c6h608759}{(Link to AoPS)}
\end{problem}



\begin{solution}[by \href{https://artofproblemsolving.com/community/user/29428}{pco}]
	\begin{tcolorbox}Find all continuous functions $f:\mathbb{R} \rightarrow \mathbb{R}$ such that
$f(x^2+x+3)+2f(x^2-3x+5)=6x^2-10x+17$\end{tcolorbox}
Let $P(x)$ be the assertion $f(x^2+x+3)+2f(x^2-3x+5)=6x^2-10x+17$

(a) : $P(x)$ $\implies$ $f(x^2+x+3)+2f(x^2-3x+5)=6x^2-10x+17$
(b) : $P(1-x)$ $\implies$ $f(x^2-3x+5)+2f(x^2+x-3)=6x^2-2x+13$
(a)-2(b) : $f(x^2+x+3)=2x^2+2x+3$ and so $f(x)=2x-3$ $\forall x\ge \frac{11}4$

And so $f(x)=2x-3$ $\forall x\ge \frac{11}4$ and $f(x)=h(x)-h(\frac{11}4)+\frac 52$ $\forall x< \frac{11}4$, which indeed is a solution, whatever is $h(x)$ continuous function.
\end{solution}



\begin{solution}[by \href{https://artofproblemsolving.com/community/user/142747}{yt12}]
	Dear PCO,

Could you please kindly explain why you use " x = 1-x " ?

Thanks and Regards,
Yt12
\end{solution}



\begin{solution}[by \href{https://artofproblemsolving.com/community/user/29428}{pco}]
	\begin{tcolorbox}Could you please kindly explain why you use " x = 1-x " ?\end{tcolorbox}
Because moving $x\to 1-x$ moves $x^2+x+3\to x^2-3x+5$


\end{solution}
*******************************************************************************
-------------------------------------------------------------------------------

\begin{problem}[Posted by \href{https://artofproblemsolving.com/community/user/10045}{socrates}]
	Let $ f : \mathbb{R} \to \mathbb{R}$ be a function such that \[\displaystyle{ xf(x+y)\geq yf(x)+f^2(x)   , \ \ \ \forall x,y \in  \mathbb{R}.  }\]
Show that $f(0)=0, \ f(x)>0, \ \forall x>0, \ f(x)<0, \ \forall x<0.$
	\flushright \href{https://artofproblemsolving.com/community/c6h608773}{(Link to AoPS)}
\end{problem}



\begin{solution}[by \href{https://artofproblemsolving.com/community/user/29428}{pco}]
	\begin{tcolorbox}Let $ f : \mathbb{R} \to \mathbb{R}$ be a function such that \[\displaystyle{ xf(x+y)\geq yf(x)+f^2(x)   , \ \ \ \forall x,y \in  \mathbb{R}.  }\]
Show that $f(0)=0, \ f(x)>0, \ \forall x>0, \ f(x)<0, \ \forall x<0.$\end{tcolorbox}
Wrong.
Choose as counter-example $f(x)=0$ $\forall x$
\end{solution}



\begin{solution}[by \href{https://artofproblemsolving.com/community/user/10045}{socrates}]
	Excuse me. I forgot to mention that "f is non-constant". :)
\end{solution}



\begin{solution}[by \href{https://artofproblemsolving.com/community/user/29428}{pco}]
	\begin{tcolorbox}Let $ f : \mathbb{R} \to \mathbb{R}$ be a \begin{bolded}non constant \end{underlined}\end{bolded}function such that \[\displaystyle{ xf(x+y)\geq yf(x)+f^2(x)   , \ \ \ \forall x,y \in  \mathbb{R}.  }\]
Show that $f(0)=0, \ f(x)>0, \ \forall x>0, \ f(x)<0, \ \forall x<0.$\end{tcolorbox}
Let $P(x,y)$ be the assertion $xf(x+y)\ge yf(x)+f(x)^2$

$P(0,0)$ $\implies$ $f(0)=0$
If $x>0$, $P(x,0)$ $\implies$ $xf(x)\ge 0$ and so $f(x)\ge 0$
If $x<0$, $P(x,0)$ $\implies$ $xf(x)\ge 0$ and so $f(x)\le 0$

If $\exists u>0$ such that $f(u)=0$, then $P(u,x-u)$ $\implies$ $f(x)\ge 0$ $\forall x$ and so $f(x)=0$ $\forall x\le 0$
If $\exists v<0$ such that $f(v)=0$, then $P(v,x-v)$ $\implies$ $f(x)\le 0$ $\forall x$ and so $f(x)=0$ $\forall x\ge 0$
So if $\exists u\ne 0$ such that $f(u)=0$, then $f(x)=0$ $\forall x$, impossible since non constant

So $f(x)=0$ $\iff$ $x=0$

Hence the result
\end{solution}



\begin{solution}[by \href{https://artofproblemsolving.com/community/user/10045}{socrates}]
	\begin{tcolorbox}[quote="socrates"]Let $ f : \mathbb{R} \to \mathbb{R}$ be a \begin{bolded}non constant \end{underlined}\end{bolded}function such that \[\displaystyle{ xf(x+y)\geq yf(x)+f^2(x)   , \ \ \ \forall x,y \in  \mathbb{R}.  }\]
Show that $f(0)=0, \ f(x)>0, \ \forall x>0, \ f(x)<0, \ \forall x<0.$\end{tcolorbox}
Let $P(x,y)$ be the assertion $xf(x+y)\ge yf(x)+f(x)^2$

$P(0,0)$ $\implies$ $f(0)=0$
If $x>0$, $P(x,0)$ $\implies$ $xf(x)\ge 0$ and so $f(x)\ge 0$
If $x<0$, $P(x,0)$ $\implies$ $xf(x)\ge 0$ and so $f(x)\le 0$

If $\exists u>0$ such that $f(u)=0$, then $P(u,x-u)$ $\implies$ $f(x)\ge 0$ $\forall x$ and so $f(x)=0$ $\forall x\le 0$
If $\exists v<0$ such that $f(v)=0$, then $P(v,x-v)$ $\implies$ $f(x)\le 0$ $\forall x$ and so $f(x)=0$ $\forall x\ge 0$
[color=#FF0000]So if $\exists u\ne 0$ such that $f(u)=0$, then $f(x)=0$ $\forall x$, impossible since non constant[\/color]

So $f(x)=0$ $\iff$ $x=0$

Hence the result\end{tcolorbox}


Why?
\end{solution}



\begin{solution}[by \href{https://artofproblemsolving.com/community/user/29428}{pco}]
	If $\exists u>0$ such that $f(u)=0$, previous lines imply $f(x)=0$ $\forall x\le 0$
So $\exists v<0$ such that $f(v)=0$ and so previous lines imply $f(x)=0$ $\forall x\ge 0$
So $f(x)=0$ $\forall x$

If $\exists u<0$ such that $f(u)=0$, previous lines imply $f(x)=0$ $\forall x\ge 0$
So $\exists v>0$ such that $f(v)=0$ and so previous lines imply $f(x)=0$ $\forall x\le 0$
So $f(x)=0$ $\forall x$

So, if $\exists u\ne 0$ such that $f(u)=0$, then $f(x)=0$ $\forall x$, impossible since $f(x)$ is supposed non constant
\end{solution}
*******************************************************************************
-------------------------------------------------------------------------------

\begin{problem}[Posted by \href{https://artofproblemsolving.com/community/user/68025}{Pirkuliyev Rovsen}]
	Find all monotone functions $f: \mathbb{R}\to\mathbb{R}$ such that $f(4x)-f(3x)=2x$.
	\flushright \href{https://artofproblemsolving.com/community/c6h608991}{(Link to AoPS)}
\end{problem}



\begin{solution}[by \href{https://artofproblemsolving.com/community/user/86443}{roza2010}]
	[hide]I am pretty sure that only solution is $f(x)=2x+b\ ,\ b\in R$[\/hide]
\end{solution}



\begin{solution}[by \href{https://artofproblemsolving.com/community/user/29428}{pco}]
	\begin{tcolorbox}Find all monotone functions $f: \mathbb{R}\to\mathbb{R}$ such that $f(4x)-f(3x)=2x$.\end{tcolorbox}
Note that monotinicy implies existence and unicity of a right limit at each point. Let $a=\lim_{x\to 0^+}f(x)$
Note that monotinicy implies existence and unicity of a left limit at each point. Let $b=\lim_{x\to 0^-}f(x)$

$f(x)-2x=f(\frac 34x)-2\frac 34x$ and so $f(x)-2x=f((\frac 34)^nx)-2(\frac 34)^nx$

For $x>0$, setting $n\to+\infty$ in this equality implies $f(x)-2x=a$ $\forall x>0$
For $x<0$, setting $n\to+\infty$ in this equality implies $f(x)-2x=b$ $\forall x<0$

Hence the result :
$f(x)=2x+a$ $\forall x>0$
$f(0)=c$
$f(x)=2x+b$ $\forall x<0$
Which indeed is a solution, whatever are $a\ge c\ge b$
\end{solution}



\begin{solution}[by \href{https://artofproblemsolving.com/community/user/64716}{mavropnevma}]
	Clearly $f$ is non-decreasing, since $f(4)-f(3) = 2$ dictates the monotony. The equation may be written as $f(4x) - 8x = f(3x) - 6x$. Consider the function $g$ given by $g(x) = f(x)-2x$; the equation writes now $g(4x) = g(3x)$, or $g(x) = g\left (\dfrac {3}{4} x\right )$. Iterating, that gives $g(x) = g\left (\left (\dfrac {3}{4}\right )^n x\right )$ for any positive integer $n$.

Since $f$ is non-decreasing, it means $\alpha = \inf \{f(x) \mid x>0\} \geq f(0)$ does exist. Then  $g(x) =\lim_{n\to \infty} g\left (\left (\dfrac {3}{4}\right )^n x\right ) = \alpha$ for $x>0$. Similarly, using  $\beta = \sup \{f(x) \mid x<0\} \leq f(0)$, we get $g(x) = \beta$. Thus $f$ is made of two linear branches, and some value $f(0) \in [\beta, \alpha]$.
\end{solution}
*******************************************************************************
-------------------------------------------------------------------------------

\begin{problem}[Posted by \href{https://artofproblemsolving.com/community/user/207996}{HQN}]
	find all function $ f: R\rightarrow\mathbb{R} $ sastisfying 
$ f(x+f(xy))=f(x)+xf(y) $
	\flushright \href{https://artofproblemsolving.com/community/c6h608997}{(Link to AoPS)}
\end{problem}



\begin{solution}[by \href{https://artofproblemsolving.com/community/user/29428}{pco}]
	\begin{tcolorbox}find all function $ f: R\rightarrow\mathbb{R} $ sastisfying 
$ f(x+f(xy))=f(x)+xf(y) $\end{tcolorbox}
$\boxed{f(x)=0}$ $\forall x$ is a solution. So let us from now look only for non allzero solutions.
Let $P(x,y)$ be the assertion $f(x+f(xy))=f(x)+xf(y)$
Let $t$ such that $f(t)\ne 0$. If $t=0$, $P(0,0)$ $\implies$ $f(f(0))=f(0)\ne 0$, so choosing $t=f(0)$ instead, we can WLOG consider $t\ne 0$

If $f(u)=0$ for some $u$, then $P(\frac ut,t)$ $\implies$ $u=0$
$P(-1,-1)$ $\implies$ $f(f(1)-1)=0$ $\implies$ $f(1)=1$ and $f(0)=0$

Let $A=\{x$ such that $f(x)=x\}$. Note that $0\in A$ and $1\in A$

\begin{bolded}Rule R1\end{underlined}\end{bolded} : If $a\in A$, then $P(1,a)$ $\implies$ $a+1\in A$

\begin{bolded}Rule R2\end{underlined}\end{bolded} : If $a\in A,a\ne 0$, then $P(a,\frac 1a)$ $\implies$ $\frac 1a\in A$

If $a\in A,a\ne 1$, then :
$P(a-1,\frac 1{a-1})$ $\implies$ $a=f(a-1)+(a-1)f(\frac 1{a-1})$ $\implies$ $\frac{f(a-1)}{a-1}+f(\frac 1{a-1})=\frac a{a-1}$
$P(\frac 1{a-1},a-1)$ $\implies$ $f(\frac a{a-1})=f(\frac 1{a-1})+\frac {f(a-1)}{a-1}$ $=\frac a{a-1}$ (see previous line)
Hence \begin{bolded}Rule R3\end{underlined}\end{bolded} : if $a\in A,a\ne 1$, then : $\frac a{a-1}\in A$

If $a\in A,a\ne 0, a\ne -1$ : R1 then R2 then R3 then R2 implies $-a\in A$, still true if $a=0$ or $a=-1$
Hence \begin{bolded}Rule R4\end{underlined}\end{bolded} : if $a\in A$; then $-a\in A$.Note that this implies $-1\in A$

If $a\in A,a\ne 0, a\ne 1$ : R2 then R3 then R2 then R4 implies $a-1\in A$, still true if $a=0$ or $a=1$
Hence \begin{bolded}Rule R5\end{underlined}\end{bolded} : if $a\in A$; then $a-1\in A$.

If $f(u)=1$ for some $u$, then $P(u,1)$ $\implies$ $f(u+1)=u+1$ $\implies$ (Rule R5) $f(u)=u$ $\implies$ $u=1$

If $f(u)=f(v)=0$ for some $u,v$, we already got $u=v=0$
If $f(u)=f(v)=c\ne 0$, we know that $u,v\ne 0$. Then :
$P(v,\frac uv)$ $\implies$ $f(v+c)=c+vf(\frac uv)$
$P(v,1)$ $\implies$ $f(v+c)=c+v$
Subtracting, we get $f(\frac uv)=1$ and so $\frac uv=1$ and so $u=v$
So $f(x)$ is injective.

$P(1,x)$ $\implies$ $f(1+f(x))=1+f(x)$ $\implies$ (rule R5) $f(f(x))=f(x)$ $\implies$ (injectivity) $\boxed{f(x)=x}$ $\forall x$, which indeed is a solution.
\end{solution}
*******************************************************************************
-------------------------------------------------------------------------------

\begin{problem}[Posted by \href{https://artofproblemsolving.com/community/user/167245}{TheChainheartMachine}]
	Find all continuous $f: \mathbb{R} \to \mathbb{R}$ for which, given any $a, b, c \in \mathbb{R}$, we have
\[ f(a) + f(b) = f(c) + f(a+b-c). \]
	\flushright \href{https://artofproblemsolving.com/community/c6h609009}{(Link to AoPS)}
\end{problem}



\begin{solution}[by \href{https://artofproblemsolving.com/community/user/29428}{pco}]
	\begin{tcolorbox}Find all continuous $f: \mathbb{R} \to \mathbb{R}$ for which, given any $a, b, c \in \mathbb{R}$, we have
\[ f(a) + f(b) = f(c) + f(a+b-c). \]\end{tcolorbox}
Let $P(a,b,c)$ be the assertion $f(a)+f(b)=f(c)+f(a+b-c)$

$P(x,y,0)$ $\implies$ $(f(x+y)-f(0))=(f(x)-f(0))+(f(y)-f(0))$ and continuity implies 

$\boxed{f(x)=ax+b}$ $\forall x$ which indeed is a solution, whatever are $a,b\in\mathbb R$
\end{solution}
*******************************************************************************
-------------------------------------------------------------------------------

\begin{problem}[Posted by \href{https://artofproblemsolving.com/community/user/202190}{Blitzkrieg97}]
	Find all injective functions $ f : \mathbb{N} \rightarrow \mathbb{R} $ that satisfy :
a) $ f(f(m)+ f(n)) = f(f(m))+ f(n) $ 
b) $ f(1) = 2, f(2) = 4 $
	\flushright \href{https://artofproblemsolving.com/community/c6h609105}{(Link to AoPS)}
\end{problem}



\begin{solution}[by \href{https://artofproblemsolving.com/community/user/29428}{pco}]
	\begin{tcolorbox}Find all injective functions $ f : \mathbb{N} \rightarrow \mathbb{R} $ that satisfy :
a) $ f(f(m)+ f(n)) = f(f(m))+ f(n) $ 
b) $ f(1) = 2, f(2) = 4 $\end{tcolorbox}
Due to $f(f(m))$ in RHS, we get that $f(x)$ is from $\mathbb N\to\mathbb N$
Using (a) with $(1,n)$ and then with $(n,1)$, we get $f(f(n))=f(n)+2$ and so $f(f(m)+f(n))=f(m)+f(n)+2$
Simple induction then implies $f(2n)=2n+2$

So, since injective, for $n$ odd $>1$, we need $f(n)=2p+1$ odd and induction gives $f(2n+1)=2n+3$ $\forall n\ge p$

Considering then the smallest odd integer $k$ such that $f(k)=k+2$, we get with injectivity $k=3$ (else $f(3)$ must be $k$, so $k\ge 7$ and no possible value for $f(5)$)

Hence the unique solution :
$f(1)=2$ and $f(n)=n+2$ $\forall n>1$
\end{solution}
*******************************************************************************
-------------------------------------------------------------------------------

\begin{problem}[Posted by \href{https://artofproblemsolving.com/community/user/216299}{adgj1234}]
	Determine all functions f defined on the set of rational numbers that take rational values for which 
f(2f(x) + f(y)) = 2x + y, 
for each x and y.     
Is the function f(x)  both onto and into function?
	\flushright \href{https://artofproblemsolving.com/community/c6h609409}{(Link to AoPS)}
\end{problem}



\begin{solution}[by \href{https://artofproblemsolving.com/community/user/212018}{Tintarn}]
	Setting $x=y$, you get $f(3f(x))=3x$ and hence $f$ must be surjective because as $x$ runs through $\mathbb{Q}$, also $3x$ runs through $\mathbb{Q}$.
Also, if $f(a)=f(b)$, then we have $3a=f(3f(a))=f(3f(b))=3b$ and thus $a=b$, hence $f$ is injective and thus bijective.
\end{solution}



\begin{solution}[by \href{https://artofproblemsolving.com/community/user/216299}{adgj1234}]
	And what is the function?
\end{solution}



\begin{solution}[by \href{https://artofproblemsolving.com/community/user/29428}{pco}]
	\begin{tcolorbox}Determine all functions f defined on the set of rational numbers that take rational values for which 
f(2f(x) + f(y)) = 2x + y, 
for each x and y.     
Is the function f(x)  both onto and into function?\end{tcolorbox}
Let $P(x,y)$ be the assertion $f(2f(x)+f(y))=2x+y$
$P(x,y)$ immediately shows that $f(x)$ is a bijection.
Let $u$ such that $f(u)=0$

$P(x,u)$ $\implies$ $f(2f(x))=2x+u$
$P(u,y)$ $\implies$ $f(f(y))=y+2u$
So $P(x,y)$ becomes $f(2f(x)+f(y))=f(2f(x))+f(f(y))-3u$ and surjectivity gives $f(x+y)=f(x)+f(y)-3u$

So $f(x)=ax+b$ (since in $\mathbb Q$) and, plugging back in equation, we get $b=0$ and $a=\pm 1$

Hence two solutons :
$f(x)=x$ $\forall x\in\mathbb Q$
$f(x)=-x$ $\forall x\in\mathbb Q$
\end{solution}



\begin{solution}[by \href{https://artofproblemsolving.com/community/user/10045}{socrates}]
	Similar problem (own): 


Determine all functions $f: \Bbb{R}^+ \to \Bbb{R}^+$ such that \[f(2f(x) + f(y)) = 2x + y, \ \ \ \forall x,y\in \Bbb{R}^+.\]
\end{solution}



\begin{solution}[by \href{https://artofproblemsolving.com/community/user/29428}{pco}]
	\begin{tcolorbox}Similar problem (own): 


Determine all functions $f: \Bbb{R}^+ \to \Bbb{R}^+$ such that \[f(2f(x) + f(y)) = 2x + y, \ \ \ \forall x,y\in \Bbb{R}^+.\]\end{tcolorbox}
Let $P(x,y)$ be the assertion $f(2f(x)+f(y))=2x+y$
$f(x)$ is bijective.

$P(x,2)$ $\implies$ $f(2f(x)+f(2))=2x+2$
$P(1,2x)$ $\implies$ $f(2f(1)+f(2x))=2+2x$
And so (subtracting and using injectivity) : $2f(x)=f(2x)+c$ where $c=2f(1)-f(2)$

So $P(\frac x2,y)$ becomes $Q(x,y)$ : $f(f(x)+f(y)+c)=x+y$

$Q(x,1)$ $\implies$ $f(f(x)+f(1)+c)=x+1$
$Q(1,y)$ $\implies$ $f(f(1)+f(y)+c)=1+y$
So $f(f(x)+f(y)+c)=f(f(x)+f(1)+c)+f(f(y)+f(1)+c)-2$

Let $g(x)=f(x+c+2f(1))-2$ and this becomes $g(f(x)+f(y)-2f(1))=g(f(x)-f(1))+g(f(y)-f(1))$
Surjectivity of $f(x)$ implies then $g(x+y)=g(x)+g(y)$ $\forall x,y\ge -f(1)$ and, since lowerbounded, $g(x)=kx$ for some $k$

So $f(x)=ux+v$ for some $u,v$ and plugging this back in original equation, we get $\boxed{f(x)=x}$ $\forall x>0$, which indeed is a solution
\end{solution}
*******************************************************************************
-------------------------------------------------------------------------------

\begin{problem}[Posted by \href{https://artofproblemsolving.com/community/user/213285}{Kezer}]
	Find all functions $f:\mathbb{R} \rightarrow \mathbb{R}$ such that for all $x,y \in \mathbb{R}$:
\[ f(x^2)+f(xy)=f(x)f(y)+yf(x)+xf(x+y) \]
	\flushright \href{https://artofproblemsolving.com/community/c6h609426}{(Link to AoPS)}
\end{problem}



\begin{solution}[by \href{https://artofproblemsolving.com/community/user/29428}{pco}]
	\begin{tcolorbox}Find all functions $f:\mathbb{R} \rightarrow \mathbb{R}$ such that for all $x,y \in \mathbb{R}$:
\[ f(x^2)+f(xy)=f(x)f(y)+yf(x)+xf(x+y) \]\end{tcolorbox}
Let $P(x,y)$ be the assertion $f(x^2)+f(xy)=f(x)f(y)+yf(x)+xf(x+y)$

$P(0,x)$ $\implies$ $f(0)(f(x)+x-2)=0$
If $f(0)\ne 0$, this gives $\boxed{\text{S1 : }f(x)=2-x}$ $\forall x$, which indeed is a solution.
If $f(0)=0$, then :

$P(x,0)$ $\implies$ $f(x^2)=xf(x)$ and so $f(-x)=-f(x)$
$P(x,-x)$ $\implies$ $f(x)(f(x)+x)=0$ and so $\forall x$, either $f(x)=0$; either $f(x)=-x$

Suppose then $\exists u,v\ne 0$ such that $f(u)=0$ and $f(v)=-v$

$P(u,v)$ $\implies$ $f(uv)=uf(u+v)$ and so $f(uv)=f(u+v)=0$ (no other possibilities)
$P(v,u)$ $\implies$ $-v^2+f(uv)=-uv+vf(u+v)$ and so $u=v$, impossible

Hence the only possibilities :
$\boxed{\text{S2 : }f(x)=0}$ $\forall x$, which indeed is a solution.

$\boxed{\text{S3 : }f(x)=-x}$ $\forall x$, which indeed is a solution.
\end{solution}
*******************************************************************************
-------------------------------------------------------------------------------

\begin{problem}[Posted by \href{https://artofproblemsolving.com/community/user/10045}{socrates}]
	Determine all functions $f: \Bbb{R}^+ \to \Bbb{R}^+$ such that \[\displaystyle{\displaystyle{  f(x+y+f(x))=2x+f(y)  , \ \ \ \forall x,y \in  \mathbb{R}^+. }}\]
	\flushright \href{https://artofproblemsolving.com/community/c6h609494}{(Link to AoPS)}
\end{problem}



\begin{solution}[by \href{https://artofproblemsolving.com/community/user/218928}{mihirb}]
	[hide="Hint"] Plugging in 0 for x we get :
$f(f(0)+y) = f(y)$
Pluggin in 0 for y we then get:
$f(f(0)) = f(0)$
[\/hide]
\end{solution}



\begin{solution}[by \href{https://artofproblemsolving.com/community/user/29428}{pco}]
	\begin{tcolorbox} Plugging in 0 for x we get :
$f(f(0)+y) = f(y)$
Pluggin in 0 for y we then get:
$f(f(0)) = f(0)$
\end{tcolorbox}
1) According to me, this is far from being a hint. Just some waste of space.
2) Moreover, this a wrong waste of space; $0\notin\mathbb R^+=\{x>0\}$ and so you can't plug in $0$.
\end{solution}



\begin{solution}[by \href{https://artofproblemsolving.com/community/user/218928}{mihirb}]
	\begin{tcolorbox}[quote="mihirb"] Plugging in 0 for x we get :
$f(f(0)+y) = f(y)$
Pluggin in 0 for y we then get:
$f(f(0)) = f(0)$
\end{tcolorbox}
1) According to me, this is far from being a hint. Just some waste of space.
2) Moreover, this a wrong waste of space; $0\notin\mathbb R^+=\{x>0\}$ and so you can't plug in $0$.\end{tcolorbox}

Thank you. I was being stupid.
\end{solution}



\begin{solution}[by \href{https://artofproblemsolving.com/community/user/13881}{Kurt G\u00f6del}]
	Let $a$ and $b$ be any positive real numbers. 

* $x = y = a$ shows that $f(2a+f(a)) = 2a+f(a)$
* $x = 2a+f(a)$ and $y = b$ shows that $4a+2f(a)+f(b) = f(2a+f(a)+b+f(x)) = f(4a+2f(a)+b)$
* $x = a$ and $y = 3a+b+f(a)$ shows that $f(4a+2f(a)+b) = 2a+f(3a+b+f(a))$
* $x = a$ and $y = 2a+b$ shows that $f(3a+b+f(a)) = 2a+f(2a+b)$.

So: $2f(a)+f(b) = f(2a+b)$. Now let $s$ and $t$ be any positive real numbers. 

* $a = s+t$ and $b = t$ shows that $2f(s+t)+f(t) = f(2s+3t)$
* $a = s$ and $b = 3t$ shows that $f(2s+3t) = 2f(s)+f(3t)$
* $a = b = t$ shows that $f(3t) = 3f(t)$

So $f(s+t) = f(s)+f(t)$ for all positive real numbers. Since the image of $f$ is contained in $\mathbb{R}^+$, it is a classical result that the only solutions of this functional equation are scalar functions $f_a:\mathbb{R}^+\to\mathbb{R}^+:x\mapsto ax$ with $a>0$. Plugging these functions into the original equation shows that only $f_1 = \text{id}$ is the only solution.
\end{solution}
*******************************************************************************
-------------------------------------------------------------------------------

\begin{problem}[Posted by \href{https://artofproblemsolving.com/community/user/10045}{socrates}]
	Determine all continuous  functions $f: \Bbb{R} \to \Bbb{R}$ such that \[\displaystyle{\displaystyle{  f(x+y+f(x))=2x+f(y)  , \ \ \ \forall x,y \in  \mathbb{R}. }}\]
	\flushright \href{https://artofproblemsolving.com/community/c6h609495}{(Link to AoPS)}
\end{problem}



\begin{solution}[by \href{https://artofproblemsolving.com/community/user/29428}{pco}]
	\begin{tcolorbox}Determine all continuous  functions $f: \Bbb{R} \to \Bbb{R}$ such that \[\displaystyle{\displaystyle{  f(x+y+f(x))=2x+f(y)  , \ \ \ \forall x,y \in  \mathbb{R}. }}\]\end{tcolorbox}
Let $g(x)=f(x)+x$ and equation is assertion $P(x,y)$ : $g(g(x)+y)=g(x)+g(y)+2x$

$P(x,x-g(x))$ $\implies$ $g(x-g(x))=-2x$
From there, $g(x)$ is surjective  and $g(u)=0$ implies $u=0$

$P(x,0)$ $\implies$ $g(g(x))=g(x)+2x$ which is a very classical equation whose only continuous solutions are $g(x)=-x$ and $g(x)=2x$ 

And so two solutions : $\boxed{f(x)=-2x}$ $\forall x$ and $\boxed{f(x)=x}$ $\forall x$
\end{solution}
*******************************************************************************
-------------------------------------------------------------------------------

\begin{problem}[Posted by \href{https://artofproblemsolving.com/community/user/68025}{Pirkuliyev Rovsen}]
	Does there exist a function$ f:\mathbb N\to\mathbb N$ such that $f(f(n-1)) = f(n+1) -f(n)$ for all $n \geq 2$?
	\flushright \href{https://artofproblemsolving.com/community/c6h610236}{(Link to AoPS)}
\end{problem}



\begin{solution}[by \href{https://artofproblemsolving.com/community/user/29428}{pco}]
	\begin{tcolorbox}Does there exist a function$ f:\mathbb N\to\mathbb N$ such that $f(f(n-1)) = f(n+1) -f(n)$ for all $n \geq 2$?\end{tcolorbox}
$\implies$ $f(n+1)\ge f(n)+1$ $\forall n\ge 2$ $\implies$ $f(n)\ge n-1$ $\forall n\ge 2$
$\implies$ $f(f(n-1))\ge f(n-1)-1\ge n-3$ $\forall n\ge 4$
$\implies$ $f(n+1)\ge f(n)+n-3\ge 2n-4$ $\forall n\ge 4$
$\implies$ $f(8)\ge 10$

Adding fonctional equations for $n\in\{2,3,..,f(8)-1\}$, we get : $\sum_{n=2}^{f(8)-1}f(f(n-1))=f(f(8))-f(2)$
But, since $f(8)\ge 10$, one of the $f(f(n-1))$ in LHS is $f(f(8))$ and so $LHS>f(f(8))>RHS$, impossible.

So no such function.
\end{solution}
*******************************************************************************
-------------------------------------------------------------------------------

\begin{problem}[Posted by \href{https://artofproblemsolving.com/community/user/10045}{socrates}]
	Determine all functions $f : \mathbb{R}^+ \to \mathbb{R}^+$ such that \[ f^2(x+f(y))=x^2+2f(xy)+y^2 ,\]
for all $x,y \in \mathbb{R}^+$
	\flushright \href{https://artofproblemsolving.com/community/c6h610261}{(Link to AoPS)}
\end{problem}



\begin{solution}[by \href{https://artofproblemsolving.com/community/user/29428}{pco}]
	\begin{tcolorbox}Determine all functions $f : \mathbb{R}^+ \to \mathbb{R}^+$ such that \[ f^2(x+f(y))=x^2+2f(xy)+y^2 ,\]
for all $x,y \in \mathbb{R}^+$\end{tcolorbox}
Let $P(x,y)$ be the assertion $f^2(x+f(y))=x^2+2f(xy)+y^2$
Let $u=f(1)$
I suppose that $f^2(x)=(f(x))(f(x))$ and not $f(f(x))$

If $f(a)=f(b)=c$ for some $a,b,c>0$, then :
$P(1,a)$ $\implies$ $f^2(1+c)=1+2c+a^2$
$P(1,b)$ $\implies$ $f^2(1+c)=1+2c+b^2$
Subtracting, we get $a=b$ (since $>0$) and so $f(x)$ is injective.

$P(1,x)$ $\implies$ $f^2(1+f(x))=1+2f(x)+x^2$
$P(x,1)$ $\implies$ $f^2(x+u)=x^2+2f(x)+1$
Subtracting, we get $f^2(1+f(x))=f^2(x+u)$ and so, since injective, $f(x)=x+(u-1)$

Plugging this back in original equation, we get $u=1$ and so $\boxed{f(x)=x}$ $\forall x>0$
\end{solution}



\begin{solution}[by \href{https://artofproblemsolving.com/community/user/30342}{nicetry007}]
	Once we show $f$ is injective and observe that the RHS is symmetric in $x,y$, we have
$f(x+f(y))^2 = f(f(x) + y)^2 \Rightarrow f(y) = y +f(x) - x \Rightarrow f(x) = x +c$
This implies $f(x) = x$
\end{solution}



\begin{solution}[by \href{https://artofproblemsolving.com/community/user/29428}{pco}]
	\begin{tcolorbox}Once we show $f$ is injective and observe that the RHS is symmetric in $x,y$, we have
$f(x+f(y))^2 = f(f(x) + y)^2 \Rightarrow f(y) = y +f(x) - x \Rightarrow f(x) = x +c$
This implies $f(x) = x$\end{tcolorbox}
Thanks for this quite useful comment of my post.
\end{solution}
*******************************************************************************
-------------------------------------------------------------------------------

\begin{problem}[Posted by \href{https://artofproblemsolving.com/community/user/125553}{lehungvietbao}]
	Problem 1)
Find all functions $f: \mathbb R^{+}\to \mathbb R^{+}$ such that \[f(ab)f(bc)f(ac)f(a+b)f(b+c)f(c+a)=2014 \quad \forall a,b,c> 0\]
Problem 2)
Find all functions $f,g : \mathbb{R}\to \mathbb{R}$ such that :
\[g(f(x+y))=f(x)+(2x+y)g(y), \forall x,y\in\mathbb{R}\]
	\flushright \href{https://artofproblemsolving.com/community/c6h610355}{(Link to AoPS)}
\end{problem}



\begin{solution}[by \href{https://artofproblemsolving.com/community/user/29428}{pco}]
	\begin{tcolorbox}Problem 1)
Find all functions $f: \mathbb R^{+}\to \mathbb R^{+}$ such that $f(ab)f(bc)f(ac)f(a+b)f(b+c)f(c+a)=2014 \quad \forall a,b,c>0$ \end{tcolorbox}
Let $a=\sqrt[6]{2014}$
Let $P(x,y,z)$ be the assertion $f(xy)f(yz)f(zx)f(x+y)f(y+z)f(z+x)=a^6$

$P(1,1,1)$ $\implies$ $f(1)f(2)=a^2$
$P(x,1,1)$ $\implies$ $f(x)f(x+1)=a^2$
$P(x,y,1)$ $\implies$ $f(xy)f(x+y)=a^2$

Let $x,y>0$ and $s>2\sqrt{\max(x,y)}$ so that $s^2-4x>0$ and $s^2-4y>0$
Let $u,v$ the two roots of $X^2-sX+x=0$ : $f(uv)f(u+v)=a^2$ and so $f(x)f(s)=a^2$
Let $w,t$ the two roots of $X^2-sX+y=0$ : $f(wt)f(w+t)=a^2$ and so $f(y)f(s)=a^2$

So $f(x)=f(y)$ $\forall x,y$ and $f(x)$ is constant. Hence the answer : $\boxed{f(x)=\sqrt[6]{2014}}$ $\forall x>0$
\end{solution}



\begin{solution}[by \href{https://artofproblemsolving.com/community/user/212018}{Tintarn}]
	For the second one:
It's from the IMO Shortlist 2011 (A-3):
http://www.artofproblemsolving.com/Forum/viewtopic.php?p=2737643&sid=c58b5ff7d022cf9d3c263b348f3e292f#p2737643
\end{solution}
*******************************************************************************
-------------------------------------------------------------------------------

\begin{problem}[Posted by \href{https://artofproblemsolving.com/community/user/10045}{socrates}]
	Determine all functions $f : \mathbb{R} \to \mathbb{R}$ such that \[ f(a^2+f(b)+c) = af(a)+f(b+c)   , \ \ \ \forall a,b,c \in  \mathbb{R}.\]
	\flushright \href{https://artofproblemsolving.com/community/c6h610510}{(Link to AoPS)}
\end{problem}



\begin{solution}[by \href{https://artofproblemsolving.com/community/user/186874}{Nanas}]
	[hide] \begin{bolded}Problem statement: \end{bolded}Determine all functions $ f :\mathbb{R}\to\mathbb{R} $ such that
 \[ f(a^2+f(b)+c) = af(a)+f(b+c) ,\ \ \ \forall a,b,c\in\mathbb{R}. \] 
____________________________________________


At first we guess only solutions are $f(x)=0$ or $f(x)=x$. Note that the equation looks like Cauchy equation, so we may prove $f$ is a Cauchy equation, and then show that $f(\mathbb{R}_{\geq0}) \subset \mathbb{R}_{\geq0}$ which will suffice .Now one want some thing like $af(a)=f(a^2)$ to do this. We begin investigating the equation  a little. We substitute $a=c=0$ to get $f(f(b))=f(b)$.Now substituting $a=x\quad ;b=x^2 \quad ; c=-x^2$ in the original equation and using the latter equation we get; \[ f(x^2)=xf(x)+f(0)\] from which if we let $x=1$ we get $f(0)=0$. We get then \[f(x^2) = xf(x) \qquad (1)\] Now using this in the original equation we with $a=x \quad;b=0 \quad $ we get \[f(x^2 + c) = f(x^2) + f(c)\], one can now deduce from it the identity $f(x)=-f(x)$ for all real numbers $x$, whence we can prove $f$ is Cauchy function. 

We have now \begin{bolded}an equivalent problem\end{bolded} of finding all Cauchy functions $f$ such that \[f(a^2) = a f(a)  \ \ \ \forall a \in \mathbb{R} \ \ \ (2) \]
\[ f(f(b)) = f(b) \ \ \ \forall b\in \mathbb{R} \ \ \ (3)\]
______
But then we note that \[f(zf(z))=f(z^2) \ \ \ (4)\], so we have letting $a=s+t$ in this latter equation and using the Cauchy condition; \[f(s^2) + 2 f(st) + f(t^2) = f(sf(s)) + f(sf(t)) + f(tf(s)) + f(tf(t))\], using $(4)$ again but with $z=s$ once and $z=t$ another. We get \[f(st)= \frac{f(sf(t))+f(tf(s))}{2} \ \ \ (5) \] Now substituting $t=f(s)$ in the $(5)$ we get; \[f(s^2) = f(sf(s)) = f(f(s)^2) = f(f(s)f(s)) = f(s)^2\] where in the last equality we have used $(2)$ which is the required. So we have $f(x) = f(1) x$ , but substituting that result in (3) say we get $f(1)=0,1$[\/hide]
\end{solution}



\begin{solution}[by \href{https://artofproblemsolving.com/community/user/10045}{socrates}]
	\begin{tcolorbox}At first we guess only solutions are $f(x)=0$ or $f(x)=x$. Note that the equation looks like Cauchy equation, so we may prove $f$ is a Cauchy equation, and then show that $f(\mathbb{R}_{\geq0}) \subset \mathbb{R}_{\geq0}$ which will suffice .Now one want some thing like $af(a)=f(a^2)$ to do this. We begin investigating the equation  a little. We substitute $a=c=0$ to get $f(f(b))=f(b)$.Now substituting $\color{red} a=x\quad ;b=x^2 \quad ; c=-x^2$ in the original equation and using the latter equation we get; \[ f(x^2)=xf(x)+f(0)\] from which if we let $x=1$ we get $f(0)=0$. We get then \[f(x^2) = xf(x) \qquad (1)\] Now using this in the original equation we with $a=x \quad;b=0 \quad $ we get \[f(x^2 + c) = f(x^2) + f(c)\], one can now deduce from it the identity $f(x)=-f(x)$ for all real numbers $x$, whence we can prove $f$ is Cauchy function. 
\end{tcolorbox}


I think it is $a=x\quad ;b=f(0)  \quad ; c=-f(0)$
\end{solution}



\begin{solution}[by \href{https://artofproblemsolving.com/community/user/186874}{Nanas}]
	\begin{tcolorbox}


I think it is $a=x\quad ;b=f(0)  \quad ; c=-f(0)$\end{tcolorbox}

Note that it is previously proved that $f(f(t))= f(t)$, so with the original substitutions; \[f(x^2) = f(f(x^2)) = f(x^2+f(x^2)-x^2) = xf(x) + f(x^2-x^2) = xf(x) + f(0) \].
\end{solution}



\begin{solution}[by \href{https://artofproblemsolving.com/community/user/13881}{Kurt G\u00f6del}]
	Why did you make that last part so complicated? Just prove that $f$ is Cauchy, as you did (by noting that $f(x^2+y) = f(x^2)+f(y)$ and the fact that $f$ is odd), conclude that $f(x) = \gamma x$ for some real number $\gamma$, and then use the equation for $a = c = 0$ and $b = 1$: $\gamma^2 = f(f(1)) = f(1) = \gamma$.
\end{solution}



\begin{solution}[by \href{https://artofproblemsolving.com/community/user/29428}{pco}]
	\begin{tcolorbox}Why did you make that last part so complicated? Just prove that $f$ is Cauchy, as you did (by noting that $f(x^2+y) = f(x^2)+f(y)$ and the fact that $f$ is odd), conclude that $f(x) = \gamma x$ for some real number $\gamma$, and then use the equation for $a = c = 0$ and $b = 1$: $\gamma^2 = f(f(1)) = f(1) = \gamma$.\end{tcolorbox}
You cant directly conclude "Cauchy implies $f(x)=\gamma x$". You need conditions more.
\end{solution}



\begin{solution}[by \href{https://artofproblemsolving.com/community/user/186874}{Nanas}]
	In fact it is not only a hard task to prove that  " If $f$ is Cauchy, then $f(x)=Cx$ ", it cannot be proved nor disproved unless one assume the Axiom of choice, which allow one to construct a counter example by constructing a basis of $\mathbb{R}$ over rational numbers.
\end{solution}
*******************************************************************************
-------------------------------------------------------------------------------

\begin{problem}[Posted by \href{https://artofproblemsolving.com/community/user/10045}{socrates}]
	Determine all functions $f : \mathbb{R} \to \mathbb{R}$ such that \[  f((x+y)f(x))=xf(x)+xf(y) ,\] for all $x,y \in \mathbb{R}$
	\flushright \href{https://artofproblemsolving.com/community/c6h610536}{(Link to AoPS)}
\end{problem}



\begin{solution}[by \href{https://artofproblemsolving.com/community/user/86443}{roza2010}]
	[hide]1) $f\equiv 0$ is one solution!

2) if $f\not\equiv 0$ , for $y=0$ , we get

$f(xf(x))=xf(x)$ , so $f(x)=x$[\/hide]
\end{solution}



\begin{solution}[by \href{https://artofproblemsolving.com/community/user/29428}{pco}]
	\begin{tcolorbox}1) $f\equiv 0$ is one solution!

2) if $f\not\equiv 0$ , for $y=0$ , we get

$f(xf(x))=xf(x)$ , so $f(x)=x$\end{tcolorbox}
In order to write this, you should prove first that $g(x)=xf(x)$ is surjective (which unfortunately is wrong)
\end{solution}



\begin{solution}[by \href{https://artofproblemsolving.com/community/user/29428}{pco}]
	\begin{tcolorbox}Determine all functions $f : \mathbb{R} \to \mathbb{R}$ such that \[  f((x+y)f(x))=xf(x)+xf(y) ,\] for all $x,y \in \mathbb{R}$\end{tcolorbox}
$\boxed{\text{S1 : }f(x)=0\text{   }\forall x}$ is a solution. So let us from now look only for non allzero solutions.

Let $P(x,y)$ be the assertion $f((x+y)f(x))=xf(x)+xf(y)$
Let $u$ such that $f(u)=v\ne 0$
If $f(0)\ne 0$, then $P(0,\frac u{f(0)})$ $\implies$ $f(u)=0$, impossible. So $f(0)=0$

$P(x,-x)$ $\implies$ $f(-x)=-f(x)$ $\forall x\ne 0$, still true when $x=0$ and so $f(x)$ is odd.

(a) : $P(x,0)$ $\implies$ $f(xf(x))=xf(x)$
(b) : $P(-x,2x)$ $\implies$ $-f(xf(x))=xf(x)-xf(2x)$
(c) : $P(x,y-x)$ $\implies$ $f(yf(x))=xf(x)-xf(x-y)$
(d) : $P(-x,x+y)$ $\implies$ $-f(yf(x))=xf(x)-xf(x+y)$
(a)+(b)-(c)-(d) implies $f(2x)=f(x+y)+f(x-y)$ $\forall x\ne 0$, still true when $x=0$
So $f(x)$ is additive

$P(x,0)$ $\implies$ $f(xf(x))=xf(x)$
$P(x,y)$ $\implies$ $f(xf(x))+f(yf(x))=xf(x)+xf(y)$
Subtracting, we get new assertion $Q(x,y)$ : $f(yf(x))=xf(y)$

So $f(f(yf(u)))=f(uf(y))=yf(u)$ and so $f(f(x))=x$ $\forall x$ and $f(x)$ is bijective

$Q(f(x),y)$ becomes $f(xy)=f(x)f(y)$ and $f(x)$ is multiplicative

Additive, multiplicative and bijective implies classicaly $\boxed{\text{S2 : }f(x)=x\text{   }\forall x}$ which indeed is a solution.
\end{solution}
*******************************************************************************
-------------------------------------------------------------------------------

\begin{problem}[Posted by \href{https://artofproblemsolving.com/community/user/61129}{gus135791}]
	N is a set of natural number
f(n) is satiafying the following:

   (1) f:N->N 
   (2) f(n) = f(n+f(n))

(a) if the range of f(n) is finite, prove that f(n) is periodic.
(b) Example f(n) which is not periodic.
	\flushright \href{https://artofproblemsolving.com/community/c6h610559}{(Link to AoPS)}
\end{problem}



\begin{solution}[by \href{https://artofproblemsolving.com/community/user/29428}{pco}]
	\begin{tcolorbox}N is a set of natural number
f(n) is satiafying the following:

   (1) f:N->N 
   (2) f(n) = f(n+f(n))

(a) if the range of f(n) is finite, prove that f(n) is periodic.
(b) Example f(n) which is not periodic.\end{tcolorbox}
a) Easy induction gives $f(n)=f(n+kf(n))$ $\forall k,n\in\mathbb N$
If $|f(\mathbb N)|$ is finite, then this implies $f(n+\text{lcm}(f(\mathbb N)))=f(n)$ and $f(n)$ is indeed periodic.

b) Choose for example $f(n)=10^{g(n)}$ where $g(n)$ is the rank from the right (beginning at $1$) of the rightmost digit $0$ in the decimal representation (beginning with a leading zero) of $n$.
Examples :
$n=103$ $\implies$ $g(n)=2$ $\implies$ $f(n)=100$ $\implies$ $n+f(n)=203$ and $f(n+f(n))=100$
$n=1010$ $\implies$ $g(n)=1$ $\implies$ $f(n)=10$ $\implies$ $n+f(n)=1020$ and $f(n+f(n))=10$
$n=12$ $\implies$ $g(n)=3$ $\implies$ $f(n)=1000$ $\implies$ $n+f(n)=1012$ and $f(n+f(n))=1000$
It's easy to see that $f(n)$ fits the requirements and is unbounded, so non periodic.
\end{solution}



\begin{solution}[by \href{https://artofproblemsolving.com/community/user/61129}{gus135791}]
	Wow. Thanks a lot!. U seems to be a genius
\end{solution}



\begin{solution}[by \href{https://artofproblemsolving.com/community/user/29428}{pco}]
	\begin{tcolorbox}(b) Example f(n) which is not periodic.\end{tcolorbox}
Another simpler example : $f(n)=2^{v_2(n)+1}$
\end{solution}
*******************************************************************************
-------------------------------------------------------------------------------

\begin{problem}[Posted by \href{https://artofproblemsolving.com/community/user/228067}{suxrob}]
	Find all continuous functions f:Q->Q such that F(xy) = F(X)F(Y) - F(X+Y) + 1. for all real values
	\flushright \href{https://artofproblemsolving.com/community/c6h611080}{(Link to AoPS)}
\end{problem}



\begin{solution}[by \href{https://artofproblemsolving.com/community/user/29428}{pco}]
	\begin{tcolorbox}Find all continuous functions f:Q->Q such that F(xy) = F(X)F(Y) - F(X+Y) + 1. for all real values\end{tcolorbox}
What is the meaning of "continuous" for a function from $\mathbb Q\to\mathbb Q$ ??
\end{solution}



\begin{solution}[by \href{https://artofproblemsolving.com/community/user/212018}{Tintarn}]
	http://www.artofproblemsolving.com/Forum/viewtopic.php?p=18081&sid=b033e9cd0dfb3fbd651e1bd14190e653#p18081
It's from "IMO" 1980, Problem 1.
You can prove $f(x)=x+1$ for any rational $x$ without any other constraints.
If you add something like continuity you will get that $f(x)=x+1$ for every real number (if you assume $f$ to be from $\mathbb{R}$ to $\mathbb{R}$)
\end{solution}



\begin{solution}[by \href{https://artofproblemsolving.com/community/user/64716}{mavropnevma}]
	Continuity for a function from $\mathbb{Q}$ to $\mathbb{Q}$ makes perfect sense, since $\mathbb{Q}$ is endowed with the subspace topology $\mathcal{T}_{\mathbb{Q}} = \{U \cap \mathbb{Q}\mid U \textrm{ open in } \mathbb{R}\}$ (it may not be needed in the problem, but that's a different issue).
It's the continuity for a function defined on $\mathbb{Z}$ that makes little sense, since the subspace topology $\mathcal{T}_{\mathbb{Z}} = \{U \cap \mathbb{Z}\mid U \textrm{ open in } \mathbb{R}\}$ is the discrete topology on $\mathbb{Z}$, so every function is continuous.
\end{solution}



\begin{solution}[by \href{https://artofproblemsolving.com/community/user/29428}{pco}]
	\begin{tcolorbox}Find all continuous functions f:Q->Q such that F(xy) = F(X)F(Y) - F(X+Y) + 1. for all real values\end{tcolorbox}
Sorry, but I prefer to forget continuity property (I dis not understood Mavropnevma's precision)
I forget also the precision "for all real value" which seems senseless for a function from $\mathbb q\to\mathbb Q$

Then ;
Let $P(x,y)$ be the assertion $f(xy)=f(x)f(y)-f(x+y)+1$
Let $a=f(1)-1$

$P(0,0)$ $\implies$ $f(0)=1$
$P(0,1)$ $\implies$ $f(1)=a+1$
$P(1,1)$ $\implies$ $f(2)=a^2+a+1$
$P(2,1)$ $\implies$ $f(3)=a^3+a^2+a+1$
$P(3,1)$ $\implies$ $f(4)=a^4+a^3+a^2+a+1$
$P(2,2)$ $\implies$ $a^4=a^2$ and so $a\in\{-1,0,1\}$

1) Case where $a=0$
$P(x-1,1)$ $\implies$ $\boxed{\text{S1 : }f(x)=1\text{  }\forall x\in\mathbb Q}$ which indeed is a solution.

2) Case where $a=1$
$P(x,1)$ $\implies$ $f(x+1)=f(x)+1$ and so $f(x+n)=f(x)+n$ and $f(n)=n+1$
$P(x,n)$ $\implies$ $f(nx)-1=n(f(x)-1)$ and so $f(\frac pq)=\frac pq+1$
And so $\boxed{\text{S2 : }f(x)=x+1\text{  }\forall x\in\mathbb Q}$ which indeed is a solution.

3) Case where $a=-1$
$P(x,1)$ $\implies$ $f(x+1)=1-f(x)$ and so $f(x+2)=f(x)$ and so $f(2)=1$
$P(\frac 12,2)$ $\implies$ $f(1)=1$, impossible. And so no solution in this case.

I dont know if the two solutions are considered as continuous from $\mathbb Q\to\mathbb Q$
\end{solution}



\begin{solution}[by \href{https://artofproblemsolving.com/community/user/29428}{pco}]
	\begin{tcolorbox}You can prove $f(x)=x+1$ for any rational $x$ without any other constraints.\end{tcolorbox}
Are you sure ?
What about $f(x)=1$ $\forall x$ ?
\end{solution}



\begin{solution}[by \href{https://artofproblemsolving.com/community/user/212018}{Tintarn}]
	Ah, sorry I forgot that the original problem from 1980 had the additional condtion $f(1)=2$ which makes the second solution impossible... ;)
\end{solution}



\begin{solution}[by \href{https://artofproblemsolving.com/community/user/64716}{mavropnevma}]
	\begin{tcolorbox}I dont know if the two solutions are considered as continuous from $\mathbb Q\to\mathbb Q$.\end{tcolorbox}
Yes, they do :) As for $\mathbb{R}$, all "elementary" functions are continuous when restricted at $\mathbb{Q}$, since the restriction of a continuous function to a subspace stays continuous. A counterexample is Dirichlet's function, not continuous on $\mathbb{R}$, but constant - thus continuous, on $\mathbb{Q}$.
\end{solution}
*******************************************************************************
-------------------------------------------------------------------------------

\begin{problem}[Posted by \href{https://artofproblemsolving.com/community/user/228170}{sichinel}]
	Find all the injection functions of $ \mathbb{R^*} $ in $ \mathbb{R^*} $ such as

$ f(x+y) \cdot ( f(x)+f(y)) = f(x \cdot y) $

for all $x, y \in \mathbb{R^*}$ with $x+y \ne 0$.
	\flushright \href{https://artofproblemsolving.com/community/c6h611103}{(Link to AoPS)}
\end{problem}



\begin{solution}[by \href{https://artofproblemsolving.com/community/user/29428}{pco}]
	\begin{tcolorbox}Find all the injection functions of $ \mathbb{R^*} $ in $ \mathbb{R^*} $ such as

$ f(x+y) \cdot ( f(x)+f(y)) = f(x \cdot y) $

for all $x, y \in \mathbb{R^*}$ with $x+y \ne 0$.\end{tcolorbox}
Let $P(x,y)$ be the assertion $f(x+y)(f(x)+f(y))=f(xy)$
Let $a=f(1)$

$P(1,1)$ $\implies$ $f(2)=\frac 12$
$P(2,1)$ $\implies$ $f(3)=\frac 1{2a+1}$
$P(3,1)$ $\implies$ $f(4)=\frac 1{2a^2+a+1}$
$P(4,1)$ $\implies$ $f(5)=\frac 1{2a^3+a^2+a+1}$
$P(5,1)$ $\implies$ $f(6)=\frac 1{2a^4+a^3+a^2+a+1}$
$P(2,3)$ $\implies$ $(2a-1)(a-1)(a+1)(2a^2+a+1)=0$ and so $a\in\left\{-1,\frac 12,+1\right\}$

If $a=\frac 12$, we get $f(1)=f(2)$, impossible since injective.
If $a=-1$, we get $f(1)=f(3)$, impossible since injective.
So $a=1$ and $f(n)=\frac 1n$ $\forall n\in\mathbb N$

Let $x>0$, a simple induction using $P(x+n,1)$ implies $f(x+n)=\frac{f(x)}{nf(x)+1}$ and then $P(x,n)$ $\implies$ $f(nx)=\frac{f(x)}n$

Let $x,y>0$
$P(3x,y)$ $\implies$ $f(3x+y)(f(x)+3f(y))=f(xy)$
$P(2x,x+y)$ $\implies$ $f(3x+y)(f(x)+2f(x+y))=f(x(x+y))$
Dividing these two lines, we get $\frac{f(xy)}{f(x(x+y))}=\frac{f(x)+3f(y)}{f(x)+2f(x+y)}$

$P(2x,y)$ $\implies$ $f(2x+y)(f(x)+2f(y))=f(xy)$
$P(x,x+y)$ $\implies$ $f(2x+y)(f(x)+f(x+y))=f(x(x+y))$
Dividing these two lines, we get $\frac{f(xy)}{f(x(x+y))}=\frac{f(x)+2f(y)}{f(x)+f(x+y)}$

So $\frac{f(x)+3f(y)}{f(x)+2f(x+y)}$ $=\frac{f(x)+2f(y)}{f(x)+f(x+y)}$ and so $\frac 1{f(x+y)}=\frac 1{f(x)}+\frac 1{f(y)}$

Plugging this in original equation, we get $f(xy)=f(x)f(y)$ and so $\frac 1{f(x)}$ over $\mathbb R^+$ is injective, additive and multiplicative.
We immediately get $f(x)=\frac 1x$ $\forall x>0$

And extension to $R^-$ is very easy, giving the unique solution $\boxed{f(x)=\frac 1x}$ $\forall x\ne 0$
\end{solution}



\begin{solution}[by \href{https://artofproblemsolving.com/community/user/344350}{soryn}]
	Very nice!!!
\end{solution}
*******************************************************************************
-------------------------------------------------------------------------------

\begin{problem}[Posted by \href{https://artofproblemsolving.com/community/user/228067}{suxrob}]
	Find all continuous functions f:R-> [0;8] such that  F(X2 +Y2) = F(X2 - Y2) + F(2XY). for all real values
	\flushright \href{https://artofproblemsolving.com/community/c6h611110}{(Link to AoPS)}
\end{problem}



\begin{solution}[by \href{https://artofproblemsolving.com/community/user/29428}{pco}]
	\begin{tcolorbox}Find all continuous functions f:R-> [0;8] such that  F(X2 +Y2) = F(X2 - Y2) + F(2XY). for all real values\end{tcolorbox}
I suppose you mean $f(x)$ from $\mathbb R\to[0,+\infty)$
Let $P(x,y)$ be the assertion $f(x^2+y^2)=f(x^2-y^2)+f(2xy)$
$P(0,0)$ $\implies$ $f(0)=0$
$P(0,\sqrt{|x|})$ $\implies$ $f(x)=f(-x)$

Let $g(x)$ from $[0,+\infty)\to[0,+\infty)$ defined as $g(x)=f(\sqrt x)$. $g(x)$ is continuous.

Let $x,y\ge 0$. $P(\sqrt{\frac{\sqrt{x+y}+\sqrt x}2},\sqrt{\frac{\sqrt{x+y}-\sqrt x}2)}$ $\implies$ $f(\sqrt{x+y})=f(\sqrt x)+f(\sqrt y)$

And so $g(x+y)=g(x)+g(y)$ and continuity implies $g(x)=ax$ and $\boxed{f(x)=ax^2}$ $\forall x$, which indeed is a solution, whatever is $a\ge 0$
\end{solution}
*******************************************************************************
-------------------------------------------------------------------------------

\begin{problem}[Posted by \href{https://artofproblemsolving.com/community/user/68025}{Pirkuliyev Rovsen}]
	Determine all function ${f: \mathbb{R}-\ 0}\to\mathbb{R}$ such that $f(\frac{f(y)}{x})=3xy+f(x)$.
	\flushright \href{https://artofproblemsolving.com/community/c6h611482}{(Link to AoPS)}
\end{problem}



\begin{solution}[by \href{https://artofproblemsolving.com/community/user/212018}{Tintarn}]
	$x=1$ implies $f(f(y))=3y+f(1)$.
If $f(y) \ne 0$, then $x=f(y)$ implies $f(1)=3yf(y)+f(f(y))=3yf(y)+3y+f(1)$. Hence $f(y)=0$ or $f(y)=1$ for every $y$.
But $x=y=1$ implies $3=f(f(1))-f(1)$ which is not possible. Thus, no solution for $f$.
\end{solution}



\begin{solution}[by \href{https://artofproblemsolving.com/community/user/29428}{pco}]
	\begin{tcolorbox}Determine all function ${f: \mathbb{R}-\ 0}\to\mathbb{R}$ such that $f(\frac{f(y)}{x})=3xy+f(x)$.\end{tcolorbox}
Let $P(x,y)$ be the assertion $f(\frac{f(y)}{x})=3xy+f(x)$

If $f(1)=0$, $P(1,1)$ is wrong since LHS is not defined
If $f(1)\ne 0$, then $P(1,-\frac{f(1)}3)$ $\implies$ $f(u)=0$ for some $u\ne 0$ and then $P(1,u)$ is wrong since LHS is not defined.

So no solution.
\end{solution}
*******************************************************************************
-------------------------------------------------------------------------------

\begin{problem}[Posted by \href{https://artofproblemsolving.com/community/user/10045}{socrates}]
	Determine all functions $f : \mathbb{R}^+ \to \mathbb{R}^+$ such that \[ \forall x, y \in \mathbb{R}^+ \ , \  \  f(x+yf(x))=f(x)+xf(y).\]
	\flushright \href{https://artofproblemsolving.com/community/c6h611702}{(Link to AoPS)}
\end{problem}



\begin{solution}[by \href{https://artofproblemsolving.com/community/user/29428}{pco}]
	\begin{tcolorbox}Determine all functions $f : \mathbb{R}^+ \to \mathbb{R}^+$ such that \[ \forall x, y \in \mathbb{R}^+ \ , \  \  f(x+yf(x))=f(x)+xf(y).\]\end{tcolorbox}
Let $P(x,y)$ be the assertion $f(x+yf(x))=f(x)+xf(y)$
Let $a=f(1)$
$f(x)$ is obviously strictly increasing and so injective.

If $a<1$, then $P(1,\frac 1{1-a})$ $\implies$ $a=0$, impossible. and so $a\ge 1$
$P(1,\frac{n-1}a)$ $\implies$ $f(\frac{n-1}a)=f(n)-a$
$a\ge 1$ $\implies$  $\frac{n-1}a\le n-1$ and so $f(n)-a=f(\frac{n-1}a)\le f(n-1)$ and so $f(n)\le f(n-1)+a$ $\implies$ $f(n)\le na$

$P(1,x)$ $\implies$ $f(ax+1)=f(x)+a$
$P(1,ax+1)$ $\implies$ $f(a^2x+a+1)=f(x)+2a$
And simple induction gives $f(a^nx+\sum_{k=0}^{n-1}a^k)=f(x)+na$

Setting $x=n$ in above equation, we get $f(a^nn+\sum_{k=0}^{n-1}a^k)=f(n)+na$
But $P(n,1)$ $\implies$ $f(n+f(n))=f(n)+na$ and so, since injective, $f(n)=(a^n-1)n+\sum_{k=0}^{n-1}a^k$

Then $f(n)\le na$ $\implies$ $(a^n-1)n+\sum_{k=0}^{n-1}a^k\le na$ and so $a=1$ and $f(n)=n$

$P(q,\frac pq)$ $\implies$ $f(p+q)=q+qf(\frac pq)$ and so $f(\frac pq)=\frac pq$

And so, since strictly increasing, we get $\boxed{f(x)=x}$ $\forall x>0$, which indeed is a solution
\end{solution}
*******************************************************************************
-------------------------------------------------------------------------------

\begin{problem}[Posted by \href{https://artofproblemsolving.com/community/user/10045}{socrates}]
	Determine all functions $f : \mathbb{R}^+ \to \mathbb{R}^+$ such that \[ \forall x,y \in \Bbb{R}^+ \ , \ \  f(xf(y)+f(x))=2f(x)+xy.\]
	\flushright \href{https://artofproblemsolving.com/community/c6h611705}{(Link to AoPS)}
\end{problem}



\begin{solution}[by \href{https://artofproblemsolving.com/community/user/29428}{pco}]
	\begin{tcolorbox}Determine all functions $f : \mathbb{R}^+ \to \mathbb{R}^+$ such that \[ \forall x,y \in \Bbb{R}^+ \ , \ \  f(xf(y)+f(x))=2f(x)+xy.\]\end{tcolorbox}
Let $P(x,y)$ be the assertion $f(xf(y)+f(x))=2f(x)+xy$
Let $a=f(1)$

$P(1,x)$ $\implies$ $f(f(x)+a)=x+2a$ and so $(2a,+\infty)\subseteq f(\mathbb R)$
$P(1,f(x)+a)$ $\implies$ $f(x+3a)=f(x)+3a$ and so $f(x+3ka)=f(x)+3ka$ $\forall x>0$, $\forall k\in\mathbb Z_{\ge 0}$

Let $x,y>0$ and $n\in\mathbb N$ such that $y+3na>2ax+f(x)$ : $\exists t>0$ such that $y+3na=xf(t)+f(x)$
Comparing $P(x,t)$ with $P(x,t+3a)$, we get $f(y+3na+3ax)=f(y+3na)+3ax$ and so $f(y+3ax)=f(y)+3ax$
So $f(x+y)=x+f(y)$ $\forall x,y>0$ and so $f(x)=x+c$  $\forall x$ and for some $c\in\mathbb R$

Plugging this back in original equation, we get $c=1$ and so the unique solution $\boxed{f(x)=x+1}$ $\forall x$
\end{solution}
*******************************************************************************
-------------------------------------------------------------------------------

\begin{problem}[Posted by \href{https://artofproblemsolving.com/community/user/125553}{lehungvietbao}]
	Let $\mathbb Q$ denote the set of rational numbers. Suppose that the functions $f, g: \mathbb Q\to \mathbb Q$ are strictly monotone increasing functions which attain every rational value. Is it necessarily true that the range of values of the function $f+g$ is also the whole set $\mathbb Q$?
	\flushright \href{https://artofproblemsolving.com/community/c6h611829}{(Link to AoPS)}
\end{problem}



\begin{solution}[by \href{https://artofproblemsolving.com/community/user/29428}{pco}]
	\begin{tcolorbox}Let $\mathbb Q$ denote the set of rational numbers. Suppose that the functions $f, g: \mathbb Q\to \mathbb Q$ are strictly monotone increasing functions which attain every rational value. Is it necessarily true that the range of values of the function $f+g$ is also the whole set $\mathbb Q$?\end{tcolorbox}
Let $a_n$ a strictly increasing sequence of rational numbers such that $a_0=1$ and $\lim_{n\to +\infty}a_n=\sqrt 2$
Let $b_n$ a strictly decreasing sequence of rational numbers such that $b_0=2$ and $\lim_{n\to +\infty}b_n=\sqrt 2$

Let $h(x)$ defined as :
If $x\le a_0=1$ : $h(x)=a_0+b_0-x$
If $x\in[a_k,a_{k+1}]$ : $h(x)=(x-a_k)\frac{b_{k+1}-b_k}{a_{k+1}-a_k}+b_k$ $\forall k\in\mathbb N\cup\{0\}$
$h(\sqrt 2)=\sqrt 2$
If $x\in[b_{k+1},b_k]$ : $h(x)=(x-b_k)\frac{a_{k+1}-a_k}{b_{k+1}-b_k}+a_k$ $\forall k\in\mathbb N\cup\{0\}$
If $x\ge b_0$ : $h(x)=a_0+b_0-x$

Obviously :
$h(x)$ is a strictly decreasing function such that $h(\mathbb Q)=\mathbb Q$

Let then $f(x)=-h(x)$
$f(x)$ is a stricly increasing function such that $f(\mathbb Q)=\mathbb Q$

Let $g(x)=x$
$g(x)$ is a stricly increasing function such that $g(\mathbb Q)=\mathbb Q$

And the equation $f(x)+g(x)=0$ has a unique solution $\sqrt 2$ and so has no rational solution.

So, $f_r,g_r$ (the restrictions of $f,g$ to domain $\mathbb Q$) are two rational strictly increasing surjections whose sum is no longer a surjection.
\end{solution}
*******************************************************************************
-------------------------------------------------------------------------------

\begin{problem}[Posted by \href{https://artofproblemsolving.com/community/user/221848}{yassino}]
	Find all functions f sush : 
 - $ f(2x) =f(x+y)f(y-x) + f(x-y)f(-x-y) $ 
 - $ f(x) \ge 0 $  for all reels .
	\flushright \href{https://artofproblemsolving.com/community/c6h611899}{(Link to AoPS)}
\end{problem}



\begin{solution}[by \href{https://artofproblemsolving.com/community/user/29428}{pco}]
	\begin{tcolorbox}Find all functions f sush : 
 - $ f(2x) =f(x+y)f(y-x) + f(x-y)f(-x-y) $ 
 - $ f(x) \ge 0 $  for all reels .\end{tcolorbox}
$\boxed{\text{S1 : }f(x)=0\text{  }\forall x}$ is a solution. So let us from now look only for non allzero solutions.

Let $P(x,y)$ be the assertion $f(2x)=f(x+y)f(y-x)+f(x-y)f(-x-y)$
Let $a=f(0)$
Let $u$ such that $f(u)\ne 0$

(1) : $P(\frac u2,\frac u2)$ $\implies$ $(a-1)f(u)+af(-u)=0$
(2) : $P(-\frac u2,-\frac u2)$ $\implies$ $af(u)+(a-1)f(-u)=0$
(1-a)*(1)+a*(2) $\implies$ $a=\frac 12$

$P(\frac x2,\frac x2)$ $\implies$ $f(-x)=f(x)$
$P(0,x)$ $\implies$ $f(x)^2=\frac 14$

And so $\boxed{\text{S2 : }f(x)=\frac 12\text{  }\forall x}$ which indeed is a solution
\end{solution}
*******************************************************************************
-------------------------------------------------------------------------------

\begin{problem}[Posted by \href{https://artofproblemsolving.com/community/user/78770}{thuanspdn}]
	Find all function $ f:R\rightarrow R  $  such that
$f(xy)+f(x-y)+f(x+y+1)=xy+2x+1,\forall x,y\in {R}$
	\flushright \href{https://artofproblemsolving.com/community/c6h612067}{(Link to AoPS)}
\end{problem}



\begin{solution}[by \href{https://artofproblemsolving.com/community/user/29428}{pco}]
	\begin{tcolorbox}Find all function $ f:R\rightarrow R  $  such that
$f(xy)+f(x-y)+f(x+y+1)=xy+2x+1,\forall x,y\in {R}$\end{tcolorbox}
Let $P(x,y)$ be the assertion $f(xy)+f(x-y)+f(x+y+1)=xy+2x+1$
Let $a=f(0)$

(a) : $P(x-1,0)$ $\implies$ $f(x-1)+f(x)=2x-1-a$
(b) : $P(x,0)$ $\implies$ $f(x)+f(x+1)=2x+1-a$
(c) : $P(x+1,0)$ $\implies$ $f(x+1)+f(x+2)=2x+3-a$
(d) : $P(x,1)$ $\implies$ $f(x)+f(x-1)+f(x+2)=3x+1$
-(a)+(b)-(c)+(d) : $f(x)=x+a$

Plugging this back in original equation, we get $a=0$ and so the unique solution $\boxed{f(x)=x}$ $\forall x$
\end{solution}
*******************************************************************************
-------------------------------------------------------------------------------

\begin{problem}[Posted by \href{https://artofproblemsolving.com/community/user/78770}{thuanspdn}]
	Find all function $ f:(0;1)\rightarrow R  $  such that
$ f(xy)=xf(x)+yf(y)  ,\forall x,y\in(0;1)$
	\flushright \href{https://artofproblemsolving.com/community/c6h612069}{(Link to AoPS)}
\end{problem}



\begin{solution}[by \href{https://artofproblemsolving.com/community/user/29428}{pco}]
	\begin{tcolorbox}Find all function $ f:(0;1)\rightarrow R  $  such that
$ f(xy)=xf(x)+yf(y)  ,\forall x,y\in(0;1)$\end{tcolorbox}
Let $P(x,y)$ be the assertion $f(xy)=xf(x)+yf(y)$
Let $a=f(\frac 12)$

$P(\frac 12,\frac 12)$ $\implies$ $f(\frac 14)=a$
$P(x,\frac 14)$ $\implies$ $f(\frac x4)=xf(x)+\frac a4$

$P(x,\frac 12)$ $\implies$ $f(\frac x2)=xf(x)+\frac a2$
$P(\frac x2,\frac 12)$ $\implies$ $f(\frac x4)=\frac{x^2}2f(x)+\frac {ax}4+\frac a2$
And so, equating these two expressions for $f(\frac x4)$, we get $xf(x)+\frac a4=\frac{x^2}2f(x)+\frac {ax}4+\frac a2$

And so $f(x)=\frac{a(x+1)}{2x(2-x)}$

Plugging this back in original equation, we get $a=0$ and so $\boxed{f(x)=0}$ $\forall x$
\end{solution}
*******************************************************************************
-------------------------------------------------------------------------------

\begin{problem}[Posted by \href{https://artofproblemsolving.com/community/user/222968}{rkm0959}]
	Find all $f: \mathbb{R}\rightarrow\mathbb{R}$ such that for all $x$ and $y$:

\[f(xf(x) + f(x)f(y) + y - 1) = f(xf(x) + xy) + y - 1\]
	\flushright \href{https://artofproblemsolving.com/community/c6h612312}{(Link to AoPS)}
\end{problem}



\begin{solution}[by \href{https://artofproblemsolving.com/community/user/29428}{pco}]
	\begin{tcolorbox}Find all f R-> R such that for all x and y

f(xf(x) + f(x)f(y) + y - 1) = f(xf(x) + xy) + y - 1\end{tcolorbox}
Let $P(x,y)$ be the assertion $f(xf(x)+f(x)f(y)+y-1)=f(xf(x)+xy)+y-1$
Let $a=f(0)$
Let $u=af(1-a)-a$

$P(0,1-a)$ $\implies$ $f(u)=0$

If $u=1$, then $P(1,1)$ $\implies$ $a=0$ and so $u=af(1-a)-a=0$, impossible.

So $u\ne 1$ and $P(u,\frac 1{1-u})$ $\implies$ $u=0$ and then $P(0,x+1)$ $\implies$ $\boxed{f(x)=x}$ $\forall x$, which indeed is a solution.
\end{solution}



\begin{solution}[by \href{https://artofproblemsolving.com/community/user/213024}{Adrienmath}]
	Taking $x=0$ it's clear that $f$ is surjective. Let $a$ be a real number such that$f(a)=0$. Setting $x=a$ and $y=1$ we get $f(0)=0$ and then it's obvious that $f(x)=x$ is the only solution to the problem ;)
\end{solution}
*******************************************************************************
-------------------------------------------------------------------------------

\begin{problem}[Posted by \href{https://artofproblemsolving.com/community/user/125553}{lehungvietbao}]
	Find all functions $\mathbb R^+\to\mathbb R^+$ such that \[(f(a)+f(b))(f(c)+f(d))=(a+b)(c+d), \quad \forall a,b,c,d\in\mathbb R^+; \quad abcd=1\]
	\flushright \href{https://artofproblemsolving.com/community/c6h612328}{(Link to AoPS)}
\end{problem}



\begin{solution}[by \href{https://artofproblemsolving.com/community/user/29428}{pco}]
	\begin{tcolorbox}Find all functions $\mathbb R^+\to\mathbb R^+$ such that \[(f(a)+f(b))(f(c)+f(d))=(a+b)(c+d), \quad \forall a,b,c,d\in\mathbb R^+; \quad abcd=1\]\end{tcolorbox}
Let $P(x,y,z,t)$ be the assertion $(f(x)+f(y))(f(z)+f(t))=(x+y)(z+t)$, true $\forall x,y,z,t>0$ such that $xyzt=1$

$P(1,1,1,1)$ $\implies$ $f(1)=1$
$P(x,\frac 1x,x,\frac 1x)$ $\implies$ $f(\frac 1x)=x+\frac 1x-f(x)$

$P(x,1,\frac 1x,1)$ $\implies$ $(f(x)+1)(x+\frac 1x-f(x)+1)=(x+1)(\frac 1x+1)$ $\implies$ $(f(x)-x)(f(x)-\frac 1x)=0$

So $\forall x$, either $f(x)=x$, either $f(x)=\frac 1x$

Suppose now that $\exists u,v\ne 1$ such that $f(u)=u$ and $f(v)=\frac 1v$
We immediately get $f(\frac 1u)=\frac 1u$ and $f(\frac 1v)=v$

$P(u,v,\frac 1u,\frac 1v)$ $\implies$ $(u-1)(v-1)=0$, impossible.

And so, either $\boxed{f(x)=x}$ $\forall x$, either $\boxed{f(x)=\frac 1x}$ $\forall x$, which indeed both are solutions.
\end{solution}
*******************************************************************************
-------------------------------------------------------------------------------

\begin{problem}[Posted by \href{https://artofproblemsolving.com/community/user/162032}{MrRTI}]
	Find all function $f : \mathbb{N} \longrightarrow \mathbb{N}$ such that
\[f(f(a)^2+2f(b)^2) = a^2+2b^2\]
	\flushright \href{https://artofproblemsolving.com/community/c6h612601}{(Link to AoPS)}
\end{problem}



\begin{solution}[by \href{https://artofproblemsolving.com/community/user/212018}{Tintarn}]
	I have a solution if we are accepting $0$ to be in the domain of $f$.

Let $P(a,b)$ be the condition that $f(f(a)^2+2f(b)^2) = a^2+2b^2$
$P(a,0)$ directly yields that $f$ is injective.
Obviously there exist integers $x$ and $y$ with $f(x)=0, f(y)=1$.
$P(1,1)$ yields $f(3y^2)=3$.
$P(0,1)$ yields $f(9x^4+2(y^2+2x^2)^2)=2$.
Now, comparing $P(3y^2,x)$ and $P(y,9x^4+2(y^2+2x^2)^2)$ we get
\[9y^4+2x^2=f(9)=y^2+2(9x^4+2(y^2+2x^2)^2)^2\]
and thus $9y^4 \ge 8y^8$ and hence $y=0$ or $y=1$. If $y=0$, then $1=f(y)=f(0)=f(3y^2)=3$. Absurd!
Hence $y=1$ and $x^2+4=(9x^2+2(1+2x^2)^2)^2 \ge 64x^8$ and thus $x=0$.
Now, $f(0)=0, f(1)=1, f(2)=2, f(3)=$ by the results from above.
From now on, it's easy to prove $that f(n)=n$ holds for every $n$ inductively using the identities $(2n+1)^2+2(n-1)^2=(2n-1)^2+2(n+1)^2$ and $(2n+2)^2+2(2n-2)^2=(2n-2)^2+2(2n)^2$.
\end{solution}



\begin{solution}[by \href{https://artofproblemsolving.com/community/user/203965}{wanwan4343}]
	It is Balkan MO 2009 Problem 4 !
\end{solution}



\begin{solution}[by \href{https://artofproblemsolving.com/community/user/29428}{pco}]
	\begin{tcolorbox}Find all function $f : \mathbb{N} \longrightarrow \mathbb{N}$ such that
\[f(f(a)^2+2f(b)^2) = a^2+2b^2\]\end{tcolorbox}
Let $P(x,y)$ be the assertion $f(f(x)^2+2f(y)^2)=x^2+2y^2$
$f(x)$ is injective.
Let $a_n=f(n)^2$

Subtracting $P(n+4,n+1)$ from $P(n,n+3)$ and using injectivity, we get  $a_{n+4}=2a_{n+3}-2a_{n+1}+a_n$

Since characteristic equation is $x^4-2x^3+2x-1=(x-1)^3(x+1)$, this is easily solved as $a_n=an^2+bn+c+d(-1)^n$ 
and so $f(n)=\sqrt{an^2+bn+c+d(-1)^n}$

Plugging this in original equation, we easily get $(a,b,c,d)=(1,0,0,0)$ and so the unique solution $\boxed{f(x)=x}$ $\forall x\in\mathbb N$
\end{solution}



\begin{solution}[by \href{https://artofproblemsolving.com/community/user/64716}{mavropnevma}]
	\begin{tcolorbox}It is Balkan MO 2009 Problem 4 !\end{tcolorbox}
Yes, it is. See [url]http://www.artofproblemsolving.com/Forum/viewtopic.php?p=1484888&sid=ac32d634756510c7707fa58050520960#p1484888[\/url]. 
It seems\begin{bolded} pco \end{bolded}has forgotten this  :) (his proof repeats the one provided there by\begin{bolded} pbornsztein\end{bolded}). Topic - obviously - locked.
\end{solution}
*******************************************************************************
-------------------------------------------------------------------------------

\begin{problem}[Posted by \href{https://artofproblemsolving.com/community/user/202190}{Blitzkrieg97}]
	Prove that there doesn't exist function $f : \mathbb{Z} \rightarrow \mathbb{Z}$ such that $f(x+f(y))=f(x)-y$ for all $x,y\in \mathbb{Z}$
	\flushright \href{https://artofproblemsolving.com/community/c6h612894}{(Link to AoPS)}
\end{problem}



\begin{solution}[by \href{https://artofproblemsolving.com/community/user/29428}{pco}]
	\begin{tcolorbox}Prove that there doesn't exist function $f : \mathbb{Z} \rightarrow \mathbb{Z}$ such that $f(x+f(y))=f(x)-y$ for all $x,y\in \mathbb{Z}$\end{tcolorbox}
$f(x+f(y))=f(x)-y$ $\implies$ (simple inductions) assertion $P(x,y,n)$ : $f(x+nf(y))=f(x)-ny$ $\forall x,y,n\in\mathbb Z$

$P(0,x,f(1))$ $\implies$ $f(f(1)f(x))=f(0)-f(1)x$
$P(0,1,f(x))$ $\implies$ $f(f(x)f(1))=f(0)-f(x)$
Subtracting, we get $f(x)=f(1)x$

Plugging back in original equation, we get $f(1)^2=-1$ and so no solution.
Q.E.D.
\end{solution}



\begin{solution}[by \href{https://artofproblemsolving.com/community/user/202190}{Blitzkrieg97}]
	\begin{tcolorbox}
 $P(x,y,n)$ : $f(x+nf(y))=f(x)-ny$ $\forall x,y,n\in\mathbb Z$
\end{tcolorbox}
can you tell me more detailed,how you get to it?
\end{solution}



\begin{solution}[by \href{https://artofproblemsolving.com/community/user/29428}{pco}]
	\begin{tcolorbox}[quote="pco"]
 $P(x,y,n)$ : $f(x+nf(y))=f(x)-ny$ $\forall x,y,n\in\mathbb Z$
\end{tcolorbox}
can you tell me more detailed,how you get to it?\end{tcolorbox}
Simple inductions.
This is true for $n=0$

Induction 1 :
If $f(x+nf(y))=f(x)-ny$ for some $n\ge 0$, then $f(x+(n+1)f(y))$ $=f((x+nf(y))+f(y))=f(x+nf(y))-y$ $=(f(x)-ny)-y=f(x)-(n+1)y$
Hence $f(x+nf(y))=f(x)-ny$ $\forall n\in\mathbb Z_{\ge 0}$

Induction 2:
If $f(x+nf(y))=f(x)-ny$ for some $n\le 0$, then $f(x)-ny=f(x+nf(y))=f((x+(n-1)f(y))+f(y))$ $=f(x+(n-1)f(y))-y$ and so $f(x+(n-1)f(y))=f(x)-(n-1)y$
Hence $f(x+nf(y))=f(x)-ny$ $\forall n\in\mathbb Z_{\le 0}$
\end{solution}
*******************************************************************************
-------------------------------------------------------------------------------

\begin{problem}[Posted by \href{https://artofproblemsolving.com/community/user/32935}{soruz}]
	Determine all functions $f : \mathbb{R} \to \mathbb{R}$ such that \[ f(x^3+x-2+y)=f(x^3+x-2) + f(y^3), \ \ \ \forall x,y\in \Bbb{R}.\]
	\flushright \href{https://artofproblemsolving.com/community/c6h613065}{(Link to AoPS)}
\end{problem}



\begin{solution}[by \href{https://artofproblemsolving.com/community/user/29428}{pco}]
	\begin{tcolorbox}Determine all functions $f : \mathbb{R} \to \mathbb{R}$ such that \[ f(x^3+x-2+y)=f(x^3+x-2) + f(y^3), \ \ \ \forall x,y\in \Bbb{R}.\]\end{tcolorbox}
Let $P(x,y)$ be the assertion $f(x^3+x-2+y)=f(x^3+x-2)+f(y^3)$.

Let $x\in\mathbb R$
Let $u$ any real solution of equation $Z^3+Z-2-x=0$
Let $v$ any real solution of equation $Z^3-Z-x=0$

$P(u,v)$ $\implies$ $\boxed{f(x)=0}$ $\forall x$, which indeed is a solution.
\end{solution}



\begin{solution}[by \href{https://artofproblemsolving.com/community/user/64716}{mavropnevma}]
	A (bad) joke. The expression $X=x^3+x-2$ is irrelevant, as long as its range is the whole $\mathbb{R}$. It follows that $f(0)=0$, $f(y)=f(y^3)$, $f(X+y) = f(X)+f(y)$, so $2f(X) = f(2X) = f(8X^3) = 8f(X^3) = 8f(X)$, whence $f(X)=0$ for all real $X$.
\end{solution}



\begin{solution}[by \href{https://artofproblemsolving.com/community/user/232511}{mihaim}]
	Can anyone explain the above solution ?
\end{solution}



\begin{solution}[by \href{https://artofproblemsolving.com/community/user/89198}{chaotic_iak}]
	It is clear enough.

Let $X = x^3+x-2$, then the range of $X$ is $\mathbb{R}$ and thus $f(X+y) = f(X) + f(y^3)$ for all $X,y \in \mathbb{R}$. Call that statement $P(X,y)$. Then,
$P(0,0) \implies f(0) = 0$
$P(0,y) \implies f(y) = f(y^3)$, and so $P(X,y)$ becomes $f(X+y) = f(X) + f(y)$.
$P(X,X) \implies f(2X) = f(X) + f(X) = 2f(X)$.
Also, $f(2X) = f((2X)^3) = f(8X^3)$, and due to $f(X+y) = f(X) + f(y)$, we have $f(nx) = nf(x)$ for any integer $n$ (well-known result), so $f(8X^3) = 8f(X^3) = 8f(X)$. Thus $2f(X) = f(2X) = f(8X^3) = 8f(X)$, so $f(X) = 0$ for all $X$ which is a solution.
\end{solution}
*******************************************************************************
-------------------------------------------------------------------------------

\begin{problem}[Posted by \href{https://artofproblemsolving.com/community/user/10045}{socrates}]
	Find all surjective functions $f:\mathbb{R}^+\to \mathbb{R}^+$ such that \[f(x+f(x)+2f(y))=f(2x)+f(2y), \ \ \ \forall x,y \in \mathbb{R}^+.\]
	\flushright \href{https://artofproblemsolving.com/community/c6h613491}{(Link to AoPS)}
\end{problem}



\begin{solution}[by \href{https://artofproblemsolving.com/community/user/29428}{pco}]
	\begin{tcolorbox}Find all surjective functions $f:\mathbb{R}^+\to \mathbb{R}^+$ such that \[f(x+f(x)+2f(y))=f(2x)+f(2y), \ \ \ \forall x,y \in \mathbb{R}^+.\]\end{tcolorbox}
Let $P(x,y)$ be the assertion $f(x+f(x)+2f(y))=f(2x)+f(2y)$

If $f(x)<x$ for some $x>0$, let $y$ such that $f(y)=\frac{x-f(x)}2$. Then $P(x,y)$ $\implies$ $f(2y)=0$, impossible.
So $f(x)\ge x$ $\forall x>0$

If $f(a)=f(b)=c$ for some $a,b,c>0$,then :
Comparaison of $P(1,a)$ and $P(1,b)$ implies $f(2a)=f(2b)$
Comparaison of $P(a,x)$ and $P(b,x)$ implies then $f(a+c+2f(x))=f(b+c+2f(x))$ $\forall x>0$
Surjectivity implies then $f(x+a)=f(x+b)$ $\forall x>c$
If $a\ne b$, this implies that $f(x)$ is periodic from a given point, in contradiction with $f(x)\ge x$ $\forall x>0$
So $f(x)$ is injective

Then, comparaison of $P(x,1)$ with $P(1,x)$ implies $f(x+f(x)+2f(1))=f(1+f(1)+2f(x))$ and injectivity implies $f(x)=x+(f(1)-1)$

Plugging this back in original equation, we get $f(1)=1$ and so the unique solution $\boxed{f(x)=x}$ $\forall x>0$
\end{solution}
*******************************************************************************
-------------------------------------------------------------------------------

\begin{problem}[Posted by \href{https://artofproblemsolving.com/community/user/195015}{Jul}]
	Find all function $g:\mathbb{R}\rightarrow \mathbb{R}$ and such that :
\[g\left [ g(x)-x^2+yz \right ]=g(x)\left [ g(x)-2x^2+2yz \right ]+z^2\left [ y^2-g(y) \right ]\]
\[+y^2\left [ z^2-g(z) \right ]-2x^2yz+x+g(y)g(z)+x^4,\;\forall x,y,z\in \mathbb{R}\]
	\flushright \href{https://artofproblemsolving.com/community/c6h613709}{(Link to AoPS)}
\end{problem}



\begin{solution}[by \href{https://artofproblemsolving.com/community/user/29428}{pco}]
	\begin{tcolorbox}Find all function $g:\mathbb{R}\rightarrow \mathbb{R}$ and such that :
\[g\left [ g(x)-x^2+yz \right ]=g(x)\left [ g(x)-2x^2+2yz \right ]+z^2\left [ y^2-g(y) \right ]\]
\[+y^2\left [ z^2-g(z) \right ]-2x^2yz+x+g(y)g(z)+x^4,\;\forall x,y,z\in \mathbb{R}\]\end{tcolorbox}
Let $P(x,y,z)$ be the assertion $g(g(x)-x^2+yz)$ $=g(x)(g(x)-2x^2+2yz)$ $+z^2(y^2-g(y))$ $+y^2(z^2-g(z))-2x^2yz+x+g(y)g(z)+x^4$

If $g(0)\ne 0$, $P(0,x,0)$ $\implies$ $g(x)=x^2+c$ where $c=\frac{g(g(0))}{g(0)}-g(0)$ which unfortunately is never a solution, whatever is $c\in\mathbb R$
So $g(0)=0$
Let then $h(x)=g(x)-x^2$

$P(0,x,y)$ $\implies$ $h(xy)=h(x)h(y)$

Subtracting $P(0,y,1)$ from $P(x,y,1)$, we get new assertion $Q(x,y)$ : $h(h(x)+y)=x+h(y)$

$Q(x,0)$ $\implies$ $h(h(x))=x$
$Q(h(x),y)$ $\implies$ $h(x+y)=h(x)+h(y)$

So $h(x)$ is additive, multiplicative, and involutive and it's immediate to get $h(x)=x$ $\forall x$

And so $\boxed{g(x)=x^2+x}$ $\forall x$ which indeed is a solution.
\end{solution}
*******************************************************************************
-------------------------------------------------------------------------------

\begin{problem}[Posted by \href{https://artofproblemsolving.com/community/user/125553}{lehungvietbao}]
	1) Find all continuous functions $f \, : \, \mathbb{R} \, \longrightarrow \, \mathbb{R}$ such that :

\[  f(x) + f(2x) + f(4x) = \lfloor 7x \rfloor ,\forall x \in \mathbb{R}\]

2) Find all strictly monotone functions $f:(0,+\infty)\to(0,+\infty)$ such that
\[(x+1)f(\dfrac{y}{f(x)})=f(x+y),\forall x,y>0\]
	\flushright \href{https://artofproblemsolving.com/community/c6h613727}{(Link to AoPS)}
\end{problem}



\begin{solution}[by \href{https://artofproblemsolving.com/community/user/29428}{pco}]
	\begin{tcolorbox}1) Find all continuous functions $f \, : \, \mathbb{R} \, \longrightarrow \, \mathbb{R}$ such that :

\[  f(x) + f(2x) + f(4x) = \lfloor 7x \rfloor ,\forall x \in \mathbb{R}\]\end{tcolorbox}
None, since LHS is continuous while RHS is not
\end{solution}



\begin{solution}[by \href{https://artofproblemsolving.com/community/user/29428}{pco}]
	\begin{tcolorbox}2) Find all strictly monotone functions $f:(0,+\infty)\to(0,+\infty)$ such that
\[(x+1)f(\dfrac{y}{f(x)})=f(x+y),\forall x,y>0\]\end{tcolorbox}
Let $P(x,y)$ be the assertion $f(x+y)=(x+1)f(\frac y{f(x)})$

Let $x>0$. Let $u>\max(0,f(x)-1,xf(1)-x)$

$P(1,\frac{1+u-f(x)}{f(x)})$ $\implies$  $f(\frac {1+u}{f(x)})=2f(\frac{1+u-f(x)}{f(x)f(1)})$
$P(x,u+1)$ $\implies$ $f(x+1+u)=(x+1)f(\frac {1+u}{f(x)})$ $=2(x+1)f(\frac{1+u-f(x)}{f(x)f(1)})$


$P(x,\frac{x+u-xf(1)}{f(1)})$ $\implies$  $f(\frac {x+u}{f(1)})=(x+1)f(\frac{x+u-xf(1)}{f(x)f(1)})$
$P(1,x+u)$ $\implies$ $f(x+1+u)=2f(\frac {x+u}{f(1)})$ $=2(x+1)f(\frac{x+u-xf(1)}{f(x)f(1)})$


So $f(\frac{1+u-f(x)}{f(x)f(1)})=f(\frac{x+u-xf(1)}{f(x)f(1)})$ and, since strictly monotonous, $1+u-f(x)=x+u-xf(1)$ and so $f(x)=x(f(1)-1)+1$

Plugging this back in original equation, we get $f(1)=2$ and the unique solution $\boxed{f(x)=x+1}$
\end{solution}
*******************************************************************************
-------------------------------------------------------------------------------

\begin{problem}[Posted by \href{https://artofproblemsolving.com/community/user/177636}{InvincibleNo0b}]
	Find all functions $ f:\mathbb{Z}\rightarrow\mathbb{Z} $, such that $ f(m+n)+f(m-n)=2f(m)f(n) $
	\flushright \href{https://artofproblemsolving.com/community/c6h614885}{(Link to AoPS)}
\end{problem}



\begin{solution}[by \href{https://artofproblemsolving.com/community/user/29428}{pco}]
	\begin{tcolorbox}Find all functions $ f:\mathbb{Z}\rightarrow\mathbb{Z} $, such that $ f(m+n)+f(m-n)=2f(m)f(n) $\end{tcolorbox}
Let $P(x,y)$ be the assertion $f(x+y)+f(x-y)=2f(x)f(y)$

$P(0,0)$ $\implies$ $f(0)(f(0)-1)=0$ and so $f(0)\in\{0,1\}$
If $f(0)=0$, $P(x,0)$ $\implies$ $\boxed{\text{S1 : }f(x)=0\text{ }\forall x\in\mathbb Z}$ which indeed is a solution.
Let us from now consider that $f(0)=1$

If $f(1)=0$, $P(n+1,1)$ $\implies$ $f(n+2)=-f(n)$ and so
$\boxed{\text{S2 : }f(4n)=1;f(4n+1)=0;f(4n+2)=-1;f(4n+3)=0\text{ }\forall n\in\mathbb Z}$ which indeed is a solution.

If $f(1)\ne 0$, let $a=f(1)+\sqrt{f(1)^2-1}$ : 
$P(n+1,1)$ $\implies$ $f(n+2)=(a+\frac 1a)f(n+1)-f(n)$ and so 
$\boxed{\text{S3 : }f(x)=\frac 12\left(a^x+a^{-x}\right)\text{ }\forall x\in\mathbb Z}$ where  $a=k+\sqrt{k^2-1}$ which indeed is a solution, whatever is $k\in\mathbb Z\setminus\{0\}$
\end{solution}



\begin{solution}[by \href{https://artofproblemsolving.com/community/user/177636}{InvincibleNo0b}]
	wow! thank you, patrick
\end{solution}
*******************************************************************************
-------------------------------------------------------------------------------

\begin{problem}[Posted by \href{https://artofproblemsolving.com/community/user/213024}{Adrienmath}]
	Find all continuous maps $f : R_+ \rightarrow R_+$ such that $f(xf(y)) + f(f(y)) = f(x)f(y) +2$  $\forall x, y >0$
	\flushright \href{https://artofproblemsolving.com/community/c6h615217}{(Link to AoPS)}
\end{problem}



\begin{solution}[by \href{https://artofproblemsolving.com/community/user/29428}{pco}]
	\begin{tcolorbox}Find all continuous maps $f : R_+ \rightarrow R_+$ such that $f(xf(y)) + f(f(y)) = f(x)f(y) +2$  $\forall x, y >0$\end{tcolorbox}
Let $P(x,y)$ be the assertion $f(xf(y))+f(f(y))=f(x)f(y)+2$
Let $a=f(1)$
Clearly, no constant solutions exist.

$P(f(x),1)$ $\implies$ $f(af(x))+f(a)=af(f(x))+2$
$P(a,x)$ $\implies$ $f(af(x))+f(f(x))=f(a)f(x)+2$
Subtracting, we get $f(f(x))=\frac{f(a)}{a+1}f(x)+\frac{f(a)}{a+1}$
$P(1,x)$ $\implies$ $f(f(x))=\frac a2f(x)+1$
Subtracting, we get $\left(\frac{f(a)}{a+1}-\frac a2\right)f(x)+\frac{f(a)}{a+1}-1=0$
So, since $f(x)$ is not constant, we get $\frac{f(a)}{a+1}-\frac a2=0$ and $\frac{f(a)}{a+1}-1=0$ and so $a=2$ and $f(a)=3$

$P(1,x)$ $\implies$ $f(f(x))=f(x)+1$ and simple induction gives $f(n)=n+1$ and so, since continuous, $[1,+\infty)\subseteq f(\mathbb R^+)$
Then $f(f(x))=f(x)+1$ $\implies$ $f(x)=x+1$ $\forall x\ge 1$

Let then $x>0$ and $y>\max(1,\frac 1x-1)$ so that $f(y)=y+1$ and $f(f(y))=y+2$ and $f(xf(y))=f(xy+x)=xy+x+1$ :
$P(x,y)$ $\implies$ $\boxed{f(x)=x+1}$ $\forall x>0$, which indeed is a solution.
\end{solution}
*******************************************************************************
-------------------------------------------------------------------------------

\begin{problem}[Posted by \href{https://artofproblemsolving.com/community/user/68025}{Pirkuliyev Rovsen}]
	Find all function $f: \mathbb{R}\to\mathbb{R}$ such that $f(x)-2f(\frac{x}{2})+f(\frac{x}{4})=x^2$.
	\flushright \href{https://artofproblemsolving.com/community/c6h615311}{(Link to AoPS)}
\end{problem}



\begin{solution}[by \href{https://artofproblemsolving.com/community/user/29428}{pco}]
	\begin{tcolorbox}Find all function $f: \mathbb{R}\to\mathbb{R}$ such that $f(x)-2f(\frac{x}{2})+f(\frac{x}{4})=x^2$.\end{tcolorbox}
Setting $g(x)=f(x)-\frac{16}9x^2-f(0)$, equation is $g(0)=0$ and $g(4x)=2g(2x)-g(x)$

Setting $h(x)=g(2^x)$, we get $h(x+2)=2h(x+1)-h(x)$ and so $h(x)=xp_1(\{x\})+q_1(\{x\})$ where $p_1(x),q_1(x)$ are any fonctions.
Setting $k(x)=g(-2^x)$, we get $k(x+2)=2k(x+1)-k(x)$ and so $k(x)=xp_2(\{x\})+q_2(\{x\})$ where $p_2(x),q_2(x)$ are any fonctions.

\begin{bolded}Hence a general solution\end{underlined}\end{bolded} :

Let $c\in\mathbb R$
Let $p_1(x),p_2(x),q_1(x),q_2(x)$ any functions from $\mathbb R\to\mathbb R$ :

$\forall x>0$ : $f(x)=\frac{16}9x^2+\log_2(x)p_1(\{\log_2 x\})+q_1(\{\log_2 x\})$
$f(0)=c$
$\forall x<0$ : $f(x)=\frac{16}9x^2+\log_2(-x)p_2(\{\log_2 -x\})+q_2(\{\log_2 -x\})$
\end{solution}
*******************************************************************************
-------------------------------------------------------------------------------

\begin{problem}[Posted by \href{https://artofproblemsolving.com/community/user/68025}{Pirkuliyev Rovsen}]
	Find all function $f: \mathbb{R}\to\mathbb{R}$ such that $f(x^2+f(y))=(x-y)^2f(x+y)$.
	\flushright \href{https://artofproblemsolving.com/community/c6h615312}{(Link to AoPS)}
\end{problem}



\begin{solution}[by \href{https://artofproblemsolving.com/community/user/29428}{pco}]
	\begin{tcolorbox}Find all function $f: \mathbb{R}\to\mathbb{R}$ such that $f(x^2+f(y))=(x-y)^2f(x+y)$.\end{tcolorbox}
Let $P(x,y)$ be the assertion $f(x^2+f(y))=(x-y)^2f(x+y)$

Subtracting $P(\frac{x-1}2,\frac{x+1}2)$ from $P(\frac{1-x}2,\frac{x+1}2)$, we get $f(x)=x^2f(1)$

Plugging this back in original equation, we get $f(1)\in\{-1,0\}$ hence the two solutions $\boxed{\text{S1 : }f(x)=0\text{  }\forall x}$ and $\boxed{\text{S2 : }f(x)=-x^2\text{  }\forall x}$
\end{solution}
*******************************************************************************
-------------------------------------------------------------------------------

\begin{problem}[Posted by \href{https://artofproblemsolving.com/community/user/173418}{trkac}]
	Find all functions $f:\mathbb R  \rightarrow \mathbb R$ such that
\[f(f(x)-f(y))+f(2y)=f(x+y)\]
holds for all $x,y \in \mathbb R$.
	\flushright \href{https://artofproblemsolving.com/community/c6h615394}{(Link to AoPS)}
\end{problem}



\begin{solution}[by \href{https://artofproblemsolving.com/community/user/170621}{zachman99323}]
	[hide="Some Progress"]

Taking $x=y=0$ yields $f(0) = 0$. Then taking $y = 0$ yields $f(f(x)) = f(x)$ for all real $x$. Don't really know where to go from here but this seems useful.

[\/hide]
\end{solution}



\begin{solution}[by \href{https://artofproblemsolving.com/community/user/160413}{MohammadMahdi}]
	$(1)$  $x=y=0 \Rightarrow f(f(0)-f(0))+f(0)=f(0) \Rightarrow f(0)=0$

$(2)$  $y=0 \Rightarrow f(f(x))=f(x) \Rightarrow f(f(x)-f(y))=f(x)-f(y)$ 

$\Rightarrow f(x)-f(y)+f(2y)=f(x+y)$ 

$(3)$  $x=0 \Rightarrow f(2y)-f(y)=f(y) \Rightarrow f(2y)=2f(y) \Rightarrow f(x)-f(y)+2f(y)=f(x+y)$ 

$\Rightarrow f(x)+f(y)=f(x+y)$


so the solution is all of Cauchy's Eqaution"s answers.
\end{solution}



\begin{solution}[by \href{https://artofproblemsolving.com/community/user/170621}{zachman99323}]
	How does $f(f(x)) = f(x)$ imply $f(f(x)-f(y)) = f(x)-f(y)$? The first equation only works specifically when the input to $f$ is an output of $f$. Don't you need to show that $f$ is surjective to prove the implication?
\end{solution}



\begin{solution}[by \href{https://artofproblemsolving.com/community/user/29428}{pco}]
	\begin{tcolorbox}...
so the solution is all of Cauchy's Eqaution"s answers.\end{tcolorbox}
Certainly not. For example $f(x)=2x$ is a solution of Cauchy equation but is not a solution of required problem.
\end{solution}



\begin{solution}[by \href{https://artofproblemsolving.com/community/user/224483}{Dadgarnia}]
	\begin{tcolorbox}Find all functions $f:$\begin{bolded}R\end{bolded} $ \rightarrow$ \begin{bolded}R\end{bolded}  such that

$f(f(x)-f(y))+f(2y)=f(x+y)$\end{tcolorbox}
Let $P(x,y)$ be the assertion $f(f(x)-f(y))+f(2y)=f(x+y)$. We get:
$P(x,x)\rightarrow f(0)=0$
$P(x,0)\rightarrow f(f(x))=f(x)$
$P(f(x),-x),P(x,-x)\rightarrow f(f(x)-x)=0$
$P(0,f(x)-x)\rightarrow f(2f(x)-2x)=0$
$P(-f(x),f(x)-x)\rightarrow f(-f(x))=f(-x)$
$P(0,x)\rightarrow f(-x)=f(x)-f(2x)$
$P(\frac{x}{2},-\frac{x}{2})\rightarrow f(-x)=-f(x)$
$P(x,y),P(y,x)\rightarrow f(f(x)-f(y))=\frac{f(2x)-f(2y)}{2}$
$P(x,y)\rightarrow f(2x)+f(2y)=2f(x+y)$
Now implies $y \rightarrow 0$ we get:
$f(2x)=2f(x)\Rightarrow f(x+y)=f(x)+f(y)$
But I can't complete it.
\end{solution}



\begin{solution}[by \href{https://artofproblemsolving.com/community/user/170621}{zachman99323}]
	You can cite cauchy's functional equation
\end{solution}



\begin{solution}[by \href{https://artofproblemsolving.com/community/user/29428}{pco}]
	\begin{tcolorbox}...
But I can't complete it.\end{tcolorbox}
From all the previous posts, the solution is clearly "any solution of Cauchy functional equation such that $f(f(x))=f(x)$"

If you want a general solution, this is a very classical problem and a general solution is :

Let $A,B$ two supplementary subvectorspaces of the $\mathbb Q$-vectorspace $\mathbb R$
Let $a(x)$ from $\mathbb R\to A$ and $b(x)$ from $\mathbb R\to B$ the two projections of $x$ in A and B (so that any real may be written in a unique manner as $x=a(x)+b(x)$)

Then $\boxed{f(x)=a(x)}$
\end{solution}



\begin{solution}[by \href{https://artofproblemsolving.com/community/user/160413}{MohammadMahdi}]
	\begin{tcolorbox}[quote="Dadgarnia"]...
But I can't complete it.\end{tcolorbox}
From all the previous posts, the solution is clearly "any solution of Cauchy functional equation such that $f(f(x))=f(x)$"

If you want a general solution, this is a very classical problem and a general solution is :

Let $A,B$ two supplementary subvectorspaces of the $\mathbb Q$-vectorspace $\mathbb R$
Let $a(x)$ from $\mathbb R\to A$ and $b(x)$ from $\mathbb R\to B$ the two projections of $x$ in A and B (so that any real may be written in a unique manner as $x=a(x)+b(x)$)

Then $\boxed{f(x)=a(x)}$\end{tcolorbox}

can you give a reference to cover these aspects?
\end{solution}



\begin{solution}[by \href{https://artofproblemsolving.com/community/user/29428}{pco}]
	\begin{tcolorbox}can you give a reference to cover these aspects?\end{tcolorbox}
No, but the proof is quite elementary.
Problem : find all functions $f(x)$ from $\mathbb R\to\mathbb R$ such that $f(x+y)=f(x)+f(y)$ and $f(f(x))=f(x)$

1) Claim : A general solution is :
Let $A,B$ any two supplementary subvectorspaces of the $\mathbb Q$-vectorspace $\mathbb R$
Let $a(x)$ from $\mathbb R\to A$ and $b(x)$ from $\mathbb R\to B$ the two projections of $x$ in A and B (so that any real may be written in a unique manner as $x=a(x)+b(x)$)

Then $\boxed{f(x)=a(x)}$

2) Proof that this expression is indeed a solution
That's trivial ($a(x)$ is additive and $a(a(x))=a(x)$)

3) Proof that any solution may be written in the given form (and so we have a general\end{underlined} solution)
Let $f(x)$ from $\mathbb R\to\mathbb R$ such that $f(x+y)=f(x)+f(y)$ and $f(f(x))=f(x)$
Let $A=\{x$ such that $f(x)=x\}$
Let $B=\{x$ such that $f(x)=0\}$
Since $f(x)$ is additive, $A,B$ both are subvectorspaces of the $\mathbb Q$-vectorspace $\mathbb R$
Let $a(x)=f(x)$ : $a(x)\in A$ $\forall x$
Let $b(x)=x-f(x)$ : $b(x)\in B$ $\forall x$
$x=a(x)+b(x)$ and $A\cap B=\{0\}$ and so $A,B$ are indeed supplementary.
Q.E.D.

4) trivial examples
$(A,B)=(\mathbb R,\{0\})$ gives $f(x)=x$ $\forall x$
$(A,B)=(\{0\},\mathbb R)$ gives $f(x)=0$ $\forall x$
\end{solution}



\begin{solution}[by \href{https://artofproblemsolving.com/community/user/173418}{trkac}]
	mod.erased
\end{solution}



\begin{solution}[by \href{https://artofproblemsolving.com/community/user/29428}{pco}]
	\begin{tcolorbox}:)  A bit harder is.

Find all functions $f:$\begin{bolded} R\end{bolded} $\rightarrow $ \begin{bolded}R \end{bolded} such that there exists $x_{0} \in R$ for which

$f(x_{0})=0$ and

$f(f(x)-f(y))+f(2y)=f(x+f(y))$

what happens if we dont consider first condition\end{tcolorbox}
Is it a real olympiad problem ?
Could you kindly confirm us that you have an olympiad-level solution ?

Two doubts :
1) setting $(x,y)=(-f(0),0)$ in second equation, we immediately get $f(f(-f(0))-f(0))=0$ and so first conditon is useless since trivially obtained from the second.

2) besides the trivial $f(x)=x+a$ $\forall x$ and $f(x)=0$ $\forall x$, it is possible to build infinitely many very strange solutions and I did not succeed up to now to find a general form for all of them.

Hence my doubts about the origin of this exercise
\end{solution}



\begin{solution}[by \href{https://artofproblemsolving.com/community/user/173418}{trkac}]
	mod.erased
\end{solution}



\begin{solution}[by \href{https://artofproblemsolving.com/community/user/29428}{pco}]
	\begin{tcolorbox}
ofcourse there are infinetly many solutions

but they all could be characterized in nice manner

if you add one more condition.

and you are right for the condition there is no need for it 

it was due to luck of attention\end{tcolorbox}
So ...
 The problem contains an useless condition
 And, if we add a secret ungiven more condition, we could characterize in nice manner all the solutions of the new secret problem.

What a nice olympiad problem !

I'll play this game.

Suppose the secret one more condition is "And $f(x)=x$ $\forall x$" ..

Then my proposal as nice solution is $f(x)=x$ $\forall x$

Am I right ?
I'm a winner !
\end{solution}



\begin{solution}[by \href{https://artofproblemsolving.com/community/user/173418}{trkac}]
	\begin{tcolorbox}[quote="trkac"]
ofcourse there are infinetly many solutions

but they all could be characterized in nice manner

if you add one more condition.

and you are right for the condition there is no need for it 

it was due to luck of attention\end{tcolorbox}
So ...
 The problem contains an useless condition
 And, if we add a secret ungiven more condition, we could characterize in nice manner all the solutions of the new secret problem.

What a nice olympiad problem !

I'll play this game.

Suppose the secret one more condition is "And $f(x)=x$ $\forall x$" ..

Then my proposal as nice solution is $f(x)=x$ $\forall x$

Am I right ?
I'm a winner !\end{tcolorbox}

OK you are the winner

agess i shouldnt post it

when adding the condition i ment for the first one 
 being $f(x+f(x))=const $and also Cauchys

second one also has similar solution without the condition you have mentined  

but needs a garantee of existing $0$ in the range

the aim was to show the article maby not seen so often about $f(x+f(x))=const.$

agess ive made a mestake .Sorry!!
\end{solution}
*******************************************************************************
-------------------------------------------------------------------------------

\begin{problem}[Posted by \href{https://artofproblemsolving.com/community/user/202360}{ANONYMOUS039}]
	1.Find all functions that $f:\mathbb{Q}\rightarrow\mathbb{R}$ and
$f(nx)=nf(x)$ for all $x\in\mathbb{Q}$	and $n\in\mathbb{Z}$
2.Find all real functions $f$ that
$f(f(x)-y^2)=(x-y)f(x+y)$ for all $x,y\in\mathbb{R}$
	\flushright \href{https://artofproblemsolving.com/community/c6h615595}{(Link to AoPS)}
\end{problem}



\begin{solution}[by \href{https://artofproblemsolving.com/community/user/29428}{pco}]
	\begin{tcolorbox}1.Find all functions that $f:\mathbb{Q}\rightarrow\mathbb{R}$ and
$f(nx)=nf(x)$ for all $x\in\mathbb{Q}$	and $n\in\mathbb{Z}$
\end{tcolorbox}
Let $P(n,x)$ be the assertion $f(nx)=nf(x)$
Let $a=f(1)$
$P(n,1)$ $\implies$ $f(n)=an$
$P(q,\frac pq)$ $\implies$ $f(\frac pq)=a\frac pq$

And so $\boxed{f(x)=ax}$ $\forall x\in\mathbb Q$ which indeed is a solution, whatever is $a\in\mathbb R$
\end{solution}



\begin{solution}[by \href{https://artofproblemsolving.com/community/user/29428}{pco}]
	\begin{tcolorbox}2.Find all real functions $f$ that
$f(f(x)-y^2)=(x-y)f(x+y)$ for all $x,y\in\mathbb{R}$\end{tcolorbox}
Let $P(x,y)$ be the assertion $f(f(x)-y^2)=(x-y)f(x+y)$

Subtracting $P(\frac{x+1}2,\frac{x-1}2)$ from $P(\frac{x+1}2,\frac{1-x}2)$, we get $f(x)=xf(1)$

Plugging this back in original equation, we get $f(1)=0$ and so the unique solution : $\boxed{f(x)=0}$ $\forall x$
\end{solution}
*******************************************************************************
-------------------------------------------------------------------------------

\begin{problem}[Posted by \href{https://artofproblemsolving.com/community/user/25405}{AndrewTom}]
	Determine all functions $f(n)$ from the positive integers to the positive integers which satisfy the following condition: whenever $a, b$ and $c$ are positive integers such that $\frac{1}{a} + \frac{1}{b} = \frac{1}{c}$, then

$\frac{1}{f(a)} + \frac{1}{f(b)} = \frac{1}{f(c)}$.

See here: http://www.bmoc.maths.org\/home\/bmo1-2015.pdf
	\flushright \href{https://artofproblemsolving.com/community/c6h615751}{(Link to AoPS)}
\end{problem}



\begin{solution}[by \href{https://artofproblemsolving.com/community/user/167924}{utkarshgupta}]
	$f(n)=kn$ is a solution where $f(1)=k$

I have not yet completed the solution but here's how I have started
First observe
$\frac{1}{f(a(a+1))}+\frac{1}{f(a+1)}=\frac{1}{f(a)}$
$\implies f(a+1) > f(a)$
Thus $f(x)$ is a strictly increasing function.
Let $f(1)=k$
\end{solution}



\begin{solution}[by \href{https://artofproblemsolving.com/community/user/29428}{pco}]
	\begin{tcolorbox}Determine all functions $f(n)$ from the positive integers to the positive integers which satisfy the following condition: whenever $a, b$ and $c$ are positive integers such that $\frac{1}{a} + \frac{1}{b} = \frac{1}{c}$, then

$\frac{1}{f(a)} + \frac{1}{f(b)} = \frac{1}{f(c)}$.

See here: http://www.bmoc.maths.org\/home\/bmo1-2015.pdf\end{tcolorbox}
Let $\mathbb N$ be the set of all positive integers.
Let $A=\{n\in\mathbb N$ such that $f(nx)=nf(x)$ $\forall x\in\mathbb N\}$ : $1\in A$

$\frac 1{2n}+\frac 1{2n}=\frac 1{n}$ $\implies$ $f(2n)=2f(n)$ $\forall n\in\mathbb N$ and so $2\in A$

If $A\ne \mathbb N$, let $p$ the smallest positive integer such that $p\notin A$. $p\ge 3$
If $p=qr$ with $q,r<p$ is not prime $f(px)=f(qrx)=qf(rx)=qrf(x)$ and so contradiction. So $p$ is an odd prime.
So $p+1=2q$ for some $q<p$ and so $f((p+1)x)=f(2qx)=2f(qx)=2qf(x)=(p+1)f(x)$

Then $\frac 1{p(p+1)x}+\frac 1{(p+1)x}=\frac 1{px}$ implies $\frac 1{f(p(p+1)x)}+\frac 1{f((p+1)x)}=\frac 1{f(px)}$

$\implies$ $\frac 1{(p+1)f(px)}+\frac 1{(p+1)f(x)}=\frac 1{f(px)}$ $\implies$ $f(px)=pf(x)$ and so contradiction

So $A=\mathbb N$ and $f(nx)=nf(x)$ $\forall n,x\in\mathbb N$ and so $f(x)=xf(1)$ $\forall x\in\mathbb N$


And so $\boxed{f(x)=ax}$ $\forall x\in\mathbb N$ which indeed is a solution, whatever is $a\in\mathbb N$
\end{solution}



\begin{solution}[by \href{https://artofproblemsolving.com/community/user/25405}{AndrewTom}]
	Thanks pco.

How do we know\/prove that this is the only solution?
\end{solution}



\begin{solution}[by \href{https://artofproblemsolving.com/community/user/29428}{pco}]
	\begin{tcolorbox}Thanks pco.

How do we know\/prove that this is the only solution?\end{tcolorbox}
I dont understand. My proof is straight from the beginning up to the final family of solutions. There is no alternative path in my proof.
\end{solution}



\begin{solution}[by \href{https://artofproblemsolving.com/community/user/25405}{AndrewTom}]
	Would it make a difference if $f$ were a function form $\mathbb{Q}$ to $\mathbb{Q}$? or from $\mathbb{R}$ to $\mathbb{R}$?
\end{solution}



\begin{solution}[by \href{https://artofproblemsolving.com/community/user/29428}{pco}]
	\begin{tcolorbox}Would it make a difference if $f$ were a function form $\mathbb{Q}$ to $\mathbb{Q}$? or from $\mathbb{R}$ to $\mathbb{R}$?\end{tcolorbox}
If from $\mathbb Q\to\mathbb Q$ or from $\mathbb R\to\mathbb R$ :
$f(x)\ne 0$ $\forall x\ne 0$

Let $g(x)=\frac 1{f(\frac 1x)}$ from $\mathbb Q^*\to\mathbb Q^*$ or from $\mathbb R^*\to\mathbb R^*$
We get $g(x+y)=g(x)+g(y)$ $\forall x,y,x+y\ne 0$

And end of solution is quite easy.
\end{solution}



\begin{solution}[by \href{https://artofproblemsolving.com/community/user/231142}{Dexenberg}]
	Suppose that we already proved that for any $i$ from $\{1,2,...,n-1\}$ we have that $f(i\cdot k)=i \cdot f(k)$, $\forall k \in \{1,2,...n-1\}$
by choosing $a=n-1;b=n\cdot (n-1); c=n$, we get that $f(n)=n \cdot f(1)$, resulting that $f(n \cdot k)=k \cdot f(n)=n\cdot k\cdot f(1)=n \cdot f(k)$ and I think we're done.
\end{solution}
*******************************************************************************
-------------------------------------------------------------------------------

\begin{problem}[Posted by \href{https://artofproblemsolving.com/community/user/153258}{khoa_liv29}]
	Find all functions $f,g,h:R\to R$ such that $f,g,h$ are injective and:
$\begin{cases}f(x+f(y))=g(x)+(y)\\g(x+g(y))=h(x)+f(y)\\h(x+h(y))=f(x)+g(y)\end{cases}$
	\flushright \href{https://artofproblemsolving.com/community/c6h615778}{(Link to AoPS)}
\end{problem}



\begin{solution}[by \href{https://artofproblemsolving.com/community/user/203965}{wanwan4343}]
	See here :
http://www.artofproblemsolving.com/Forum/viewtopic.php?p=3534942&sid=c479f99bacff6388b5425be53b772ec9#p3534942
\end{solution}



\begin{solution}[by \href{https://artofproblemsolving.com/community/user/29428}{pco}]
	I suppose RHS in first line is $g(x)+h(y)$
\begin{tcolorbox}Find all functions $f,g,h:R\to R$ such that $f,g,h$ are injective and:
$\begin{cases}f(x+f(y))=g(x)+h(y)\\g(x+g(y))=h(x)+f(y)\\h(x+h(y))=f(x)+g(y)\end{cases}$\end{tcolorbox}
Second equality implies $g(x+g(0))=h(x)+f(0)$
Setting $x\to x+g(0)$ in first equation, we get $f(x+f(y)+g(0))=h(x)+f(0)+h(y)$
Swapping $x$ and $y$ in the previous line gives $f(y+f(x)+g(0))=h(y)+f(0)+h(x)$
And so $f(x+f(y)+g(0))=f(y+f(x)+g(0))$ and injectivity implies $f(x)=x+a$
And so $g(x)=x+b$ and $h(x)=x+c$

Plugging this back in original system, we get 
$2a=b+c$
$2b=c+a$
$2c=a+b$
And so $a=b=c$

Hence the solutions $\boxed{f(x)=g(x)=h(x)=x+a}$ $\forall x$ and whatever is $a\in\mathbb R$
\end{solution}
*******************************************************************************
-------------------------------------------------------------------------------

\begin{problem}[Posted by \href{https://artofproblemsolving.com/community/user/10045}{socrates}]
	Find all functions $f : \Bbb{Z}\to \Bbb{R}$  such that \[f(x+y) = f(x-1)f(y) + f(y-1)f(x),\] for every integers $x, y.$
	\flushright \href{https://artofproblemsolving.com/community/c6h616011}{(Link to AoPS)}
\end{problem}



\begin{solution}[by \href{https://artofproblemsolving.com/community/user/29428}{pco}]
	\begin{tcolorbox}Find all functions $f : \Bbb{Z}\to \Bbb{R}$  such that \[f(x+y) = f(x-1)f(y) + f(y-1)f(x),\] for every integers $x, y.$\end{tcolorbox}
Let $P(x,y)$ be the assertion $f(x+y)=f(x-1)f(y)+f(y-1)f(x)$
Let $a=f(1)$

If $f(-1)=1$ :
$P(0,0)$ $\implies$ $f(0)=0$
$P(-2,1)$ $\implies$ $1=af(-3)$ and so $a\ne 0$
$P(x+1,1)$ $\implies$ $f(x+2)=af(x)$ and so $\boxed{\text{S1 : }f(2n)=0\text{  and  }f(2n-1)=a^n\text{  }\forall n\in\mathbb Z}$ which indeed is a solution, whatever is $a\in\mathbb R^*$

If $f(-1)\ne 1$ and $f(0)=0$
$P(x,0)$ $\implies$ $\boxed{\text{S2: }f(n)=0\text{  }\forall n\in\mathbb Z}$ which indeed is a solution.

If $f(-1)\ne 1$ and $f(0)\ne 0$ :
Let $c=\frac{f(0)}{1-f(-1)}$ 
$P(x+1,0)$ $\implies$ $f(x+1)=cf(x)$ and so $f(x)=ac^{x-1}$
Plugging this in $c=\frac{f(0)}{1-f(-1)}$, we get $a=\frac {c^2}2$ and so $\boxed{\text{S3 : }f(n)=\frac{c^{n+1}}2\text{  }\forall n\in\mathbb Z}$ which indeed is a solution, whatever is $c\in\mathbb R^*$
\end{solution}
*******************************************************************************
-------------------------------------------------------------------------------

\begin{problem}[Posted by \href{https://artofproblemsolving.com/community/user/10045}{socrates}]
	Determine all functions $f : \mathbb{R}^+ \to \mathbb{R}^+$ such that \[   f(x+yf(x))=f(f(x))+xf(y)   , \ \ \  \forall x,y \in \mathbb{R}^+.\]

Inspired by:  http://www.artofproblemsolving.com/Forum/viewtopic.php?f=36&t=402262
	\flushright \href{https://artofproblemsolving.com/community/c6h616020}{(Link to AoPS)}
\end{problem}



\begin{solution}[by \href{https://artofproblemsolving.com/community/user/29428}{pco}]
	\begin{tcolorbox}Determine all functions $f : \mathbb{R}^+ \to \mathbb{R}^+$ such that \[   f(x+yf(x))=f(f(x))+xf(y)   , \ \ \  \forall x,y \in \mathbb{R}^+.\]\end{tcolorbox}
Let $P(x,y)$ be the assertion $f(x+yf(x))=f(f(x))+xf(y)$

If $f(1)<1$ : $P(1,\frac 1{1-f(1)})$ $\implies$ $f(f(1))=0$, impossible. So $f(1)\ge 1$
If $f(1)>1$ : $P(1,1-\frac 1{f(1)})$ $\implies$ $f(1-\frac 1{f(1)})=0$, impossible. So $f(1)\le 1$
So $f(1)=1$

$P(1,x)$ $\implies$ $f(x+1)=f(x)+1$ and so $f(x+n)=f(x)+n$

Let $x,y>0$. Let $n\in\mathbb N$ such that $n>x-y$

$P(x,\frac{y-x+n}{f(x)})$ $\implies$  $f(y)+n=f(f(x))+xf(\frac{y-x+n}{f(x)})$
$P(x,\frac{y-x+n+f(x)}{f(x)})$ $\implies$ $f(y+f(x))+n=f(f(x))+xf(\frac{y-x+n}{f(x)})+x$
Subtracting, we get new assertion $Q(x,y)$ : $f(y+f(x))=f(y)+x$

$Q(x,1)$ $\implies$ $f(f(x))=x$
$Q(f(x),y)$ $\implies$ $f(x+y)=f(x)+f(y)$ and since $f(x)>0$ $\forall x$, we get $f(x)=f(1)x=x$

Hence the unique solution $\boxed{f(x)=x}$ $\forall x$ which indeed is a solution.
\end{solution}
*******************************************************************************
-------------------------------------------------------------------------------

\begin{problem}[Posted by \href{https://artofproblemsolving.com/community/user/10045}{socrates}]
	Determine all functions $f : \mathbb{R}^+ \to \mathbb{R}^+$ such that \[ f(x^{3}+y^{3})=xf^{2}(x)+yf^{2}(y) , \ \ \  \forall x,y \in \mathbb{R}^+.\]
	\flushright \href{https://artofproblemsolving.com/community/c6h616142}{(Link to AoPS)}
\end{problem}



\begin{solution}[by \href{https://artofproblemsolving.com/community/user/29428}{pco}]
	\begin{tcolorbox}Determine all functions $f : \mathbb{R}^+ \to \mathbb{R}^+$ such that \[ f(x^{3}+y^{3})=xf^{2}(x)+yf^{2}(y) , \ \ \  \forall x,y \in \mathbb{R}^+.\]\end{tcolorbox}
Equation implies $f(2x^3)=2xf(x)^2$ and so $f(x^3+y^3)=\frac 12f(2x^3)+\frac 12f(2y^3)$ and so $f\left(\frac{x+y}2\right)=\frac{f(x)+f(y)}2$

This is a quite classical problem whose only partially bounded solutions are $f(x)=ax+b$ for suitable $a,b$

Plugging this back in original equation, we get $a=1$ and $b=0$ and so $\boxed{f(x)=x}$ $\forall x$, which indeed is a solution.
\end{solution}



\begin{solution}[by \href{https://artofproblemsolving.com/community/user/231688}{sheripqr}]
	\begin{tcolorbox}

This is a quite classical problem .\end{tcolorbox}

Jensen's Equation...
\end{solution}
*******************************************************************************
-------------------------------------------------------------------------------

\begin{problem}[Posted by \href{https://artofproblemsolving.com/community/user/10045}{socrates}]
	Find all functions $f:\mathbb{R} \rightarrow \mathbb{N}$ such that \[f\left(x + \frac{y}{ f(x)}\right) = f\left(y + \frac{1}{f(x)}\right),\] for all $x,y \in \mathbb{R}.$


http://www.artofproblemsolving.com/Forum/viewtopic.php?f=36&t=614822
	\flushright \href{https://artofproblemsolving.com/community/c6h616143}{(Link to AoPS)}
\end{problem}



\begin{solution}[by \href{https://artofproblemsolving.com/community/user/29428}{pco}]
	\begin{tcolorbox}Find all functions $f:\mathbb{R} \rightarrow \mathbb{N}$ such that \[f\left(x + \frac{y}{ f(x)}\right) = f\left(y + \frac{1}{f(x)}\right),\] for all $x,y \in \mathbb{R}.$


http://www.artofproblemsolving.com/Forum/viewtopic.php?f=36&t=614822\end{tcolorbox}
$\boxed{\text{S1 : }f(x)=n\text{  }\forall x}$ is a solution, whatever is $n\in\mathbb N$ So let us from now look only for non constant solutions.
Let $P(x,y)$ be the assertion $f\left(x+\frac y{f(x)}\right)=f(\left(y+\frac 1{f(x)}\right)$
Since $f(x)$ is not constant, let $a$ such that $f(a)\ne 1$

If $f(u)=f(v)=c$ for some $u,v$, then subtracting $P(u,c(x-v))$ from $P(v,c(x-v))$, we get $f(x+u-v)=f(x)$ $\forall x$

Let $t\in\mathbb R$ : $P(a,\frac{tf(a)-af(a)+1}{1-f(a)})$ $\implies$ $f(t+\frac{tf(a)^2-af(a)^2+1}{f(a)-f(a)^2})$ $=f(\frac{tf(a)^2-af(a)^2+1}{f(a)-f(a)^2})$

And so, using property two lines above : $f(x+t)=f(x)$ $\forall x,t$, impossible since non constant.

Hence only solution is constant functions.
\end{solution}
*******************************************************************************
-------------------------------------------------------------------------------

\begin{problem}[Posted by \href{https://artofproblemsolving.com/community/user/179643}{_Mark_01}]
	$f: (0; + \infty) \rightarrow (0; + \infty)$

$ x,y\in\mathbb{R}^+ $

Find all function

$f(x) \cdot f(y  f(x))=f(x+y) $
	\flushright \href{https://artofproblemsolving.com/community/c6h616590}{(Link to AoPS)}
\end{problem}



\begin{solution}[by \href{https://artofproblemsolving.com/community/user/198008}{andreiromania}]
	We first prove that $f(x) \le 1$ for every $x$.
Consider the equation in $y$, $yf(x)=x+y \Leftrightarrow y(f(x)-1)=x$.If there exists a $x$ for which $f(x)>1$,then the equation has a solution $y_{x}>0$.We plug $x$ and $y_{x}$ in the hypothesis to obtain that $f(x)f(y_{x}f(x))=f(x)f(x+y_{x})=f(x+y_{x}) \Rightarrow f(x)=1$,obviously obtaining a contradiction.
Now we obviously have $f(x+y)=f(x)f(yf(x)) \le f(x)$ so obviously $f$ is decreasing.
We now have 3 cases to deal with.
1)There exists $c \in (0,\infty)$ so that $f(c)=1$.We obviously get that $f((0,c])={1}$.Then for $y=c$ and any $x$ we have $f(x+c)=f(x)f(cf(x)) \ge f(c)f(x)=f(x)$,but at the same time from monotony of $f$ we have $f(x+c) \le f(x)$,so we obtain $f(x+c)=f(x)$ for every x;in particular we have that $f((c,2c])=f((0,c])={1}$,so we obtain that $f(2c)=1$.Repeating the procedure ad infinitum we get the solution $f(x)=1$,a constant function.
2)$sup f=1$,but there exists no $c$ for which $f(c)=1$.This actually means that $lim_{x \to 0}(f(x))=1$.
Edit: My solution for this case was incorrect,refer to pco's solution.
3)$sup f=L<1$(Doesnt matter if this supremum is reached or not).Then we have that $f(x+y)=f(x)f(yf(x)) \le Lf(x)$ so applying this repeatedly we get that $f(x+1) \le Lf(x+\frac{n-1}{n}) \le ... \le L^{n}f(x)$ for any $n$ and any $x$,thus we get that $f(x+1)=0$ for any $x$,obviously false.So we have no solutions in this case.


NOTE:Throughout the solution I liberally used the fact that $f$ is decreasing and the fact that $f(x) \le 1$ without mentioning them again.Refer to these two facts if anything in my solution is unclear.
\end{solution}



\begin{solution}[by \href{https://artofproblemsolving.com/community/user/29428}{pco}]
	\begin{tcolorbox}$f: (0; + \infty) \rightarrow (0; + \infty)$

$ x,y\in\mathbb{R}^+ $

Find all function

$f(x) \cdot f(y  f(x))=f(x+y) $\end{tcolorbox}
Let $P(x,y)$ be the assertion $f(x+y)=f(x)f(yf(x))$

If $f(x)>1$ for some $x>0$, then $P(x,\frac x{f(x)-1})$ $\implies$ $f(x)=1$, impossible. So $f(x)\le 1$ $\forall x$
As a consequence, $f(x+y)\le f(x)$ and so $f(x)$ is non increasing.
If $f(t)=1$ for some $t>0$, then $P(t,x)$ $\implies$ $f(x+t)=f(x)$ $\forall x>0$
So, since non increasing, $f(x)$ is constant and so $\boxed{\text{S1 : }f(x)=1\text{  }\forall x>0}$, which indeed is a solution.

If $f(x)<1$ $\forall x>0$, we get that $f(x)$ is decreasing and so injective.

Let $a=\frac 1{f(1)}-1>0$ so that $f(1)=\frac 1{a+1}$

$P(1,(a+1)x+\frac 1{f((a+1)x+1)})$ $\implies$ $f(1+(a+1)x+\frac 1{f((a+1)x+1)})=f(1)f(x+\frac 1{(a+1)f((a+1)x+1)})$
$P(1+(a+1)x,\frac 1{f((a+1)x+1)})$ $\implies$ $f(1+(a+1)x+\frac 1{f((a+1)x+1)})=f(1+(a+1)x)f(1)$

Subtracting and using injectivity, we get $x+\frac 1{(a+1)f((a+1)x+1)}=1+(a+1)x$ and so $(a+1)f((a+1)x+1)=\frac 1{ax+1}$
But $P(1,(a+1)x)$ $\implies$ $f(1+(a+1)x)=f(1)f(x)=\frac{f(x)}{a+1}$

And so $\boxed{\text{S2 : }f(x)=\frac 1{ax+1}\text{  }\forall x>0}$, which indeed is a solution, whatever is $a>0$
\end{solution}



\begin{solution}[by \href{https://artofproblemsolving.com/community/user/215205}{BEHZOD_UZ}]
	see also here: 
http://www.artofproblemsolving.com/Forum/viewtopic.php?p=357658&sid=0d01769e13080278320f3124e91483d4#p357658
\end{solution}
*******************************************************************************
-------------------------------------------------------------------------------

\begin{problem}[Posted by \href{https://artofproblemsolving.com/community/user/159884}{hthtb22}]
	Find $f: \mathbb{R} \rightarrow  \mathbb{R} ; g: \mathbb{R} \rightarrow  \mathbb{R}$ satisfy
i) If $x \ge y$ then $f(x) \ge f(y)$
ii) $f(xy)=g(y)f(x)+f(y)$
	\flushright \href{https://artofproblemsolving.com/community/c6h616642}{(Link to AoPS)}
\end{problem}



\begin{solution}[by \href{https://artofproblemsolving.com/community/user/29428}{pco}]
	\begin{tcolorbox}Find $f: \mathbb{R} \rightarrow  \mathbb{R} ; g: \mathbb{R} \rightarrow  \mathbb{R}$ satisfy
i) If $x \ge y$ then $f(x) \ge f(y)$
ii) $f(xy)=g(y)f(x)+f(y)$\end{tcolorbox}
$\boxed{\text{S1 : }f(x)=0\text{  }\forall x}$ is a solution, whatever is $g(x)$
So let us from now look only for non allzero $f(x)$ functions. Let $u$ such that $f(u)\ne 0$

$f(ux)=g(x)f(u)+f(x)=g(u)f(x)+f(u)$ $\implies$ $g(x)=af(x)+1$ where $a=\frac{g(u)-1}{f(u)}$
Note that this implies that $g(x)$ is monotonous.

So $af(xy)+1=(af(y)+1)(af(x)+1)$ and so $g(xy)=g(x)g(y)$

This is a very classical and simple problem whose monotonous solutions are :
1) $g(x)=0$ $\forall x$ and so $f(x)=c$ constant $\implies$ $\boxed{\text{S2 : }f(x)\equiv c\text{ and }g(x)\equiv 0}$ which indeed is a solution, whatever is $c\in\mathbb R^*$

2) $g(x)=1$ $\forall x$ and then $f(xy)=f(x)+f(y)$ $\implies$ $f(u\times 0)=f(u)+f(0)$ $\implies$ $f(u)=0$, impossible

3) $g(0)=0$ and $g(x)=\text{sign}(x)|x|^{\alpha}$ for some $\alpha \ge 0$ and so :
$\boxed{\text{S3 : }f(0)=-\frac 1a\text{, }g(0)=0\text{,  }f(x)=\frac{\text{sign}(x)|x|^{\alpha}-1}a\text{ and }g(x)=\text{sign}(x)|x|^{\alpha}\text{  }\forall x\ne 0}$ which indeed is a solution, whatever are $a>0,\alpha\ge 0$
\end{solution}
*******************************************************************************
-------------------------------------------------------------------------------

\begin{problem}[Posted by \href{https://artofproblemsolving.com/community/user/78855}{KlausJanick}]
	Determine all functions $f : \mathbb{R}^+ \to \mathbb{R}^+$ such that 

$f\left ( \sqrt{x^{2}+y^{2}} \right )=f\left ( x \right )f\left ( y \right ),\forall x,y> 0$
	\flushright \href{https://artofproblemsolving.com/community/c6h616958}{(Link to AoPS)}
\end{problem}



\begin{solution}[by \href{https://artofproblemsolving.com/community/user/29428}{pco}]
	\begin{tcolorbox}Determine all functions $f : \mathbb{R}^+ \to \mathbb{R}^+$ such that 

$f\left ( \sqrt{x^{2}+y^{2}} \right )=f\left ( x \right )f\left ( y \right ),\forall x,y> 0$\end{tcolorbox}
Let $P(x,y)$ be the assertion $f(\sqrt{x^2+y^2})=f(x)f(y)$

Setting $g(x)=\ln(f(\sqrt x))$, $P(\sqrt x,\sqrt y)$ becomes $g(x+y)=g(x)+g(y)$

Hence the solution $\boxed{f(x)=e^{g(x^2)}}$ $\forall x$, where $g(x)$ is any additive function
\end{solution}
*******************************************************************************
-------------------------------------------------------------------------------

\begin{problem}[Posted by \href{https://artofproblemsolving.com/community/user/78855}{KlausJanick}]
	Find all function $f : \mathbb{R} \to \mathbb{R}$ such that 

$   f\left ( x+f\left ( y \right ) \right )=\left ( f\left ( y \right ) \right )^{2}+2xf\left ( y \right )+f\left ( -x \right )                                                                                         ,\forall x,y\in \mathbb{R}$
	\flushright \href{https://artofproblemsolving.com/community/c6h616963}{(Link to AoPS)}
\end{problem}



\begin{solution}[by \href{https://artofproblemsolving.com/community/user/29428}{pco}]
	\begin{tcolorbox}Find all function $f : \mathbb{R} \to \mathbb{R}$ such that 

$   f\left ( x+f\left ( y \right ) \right )=\left ( f\left ( y \right ) \right )^{2}+2xf\left ( y \right )+f\left ( -x \right )                                                                                         ,\forall x,y\in \mathbb{R}$\end{tcolorbox}
$\boxed{\text{S1 : }f(x)=0\text{  }\forall x}$ is a solution. So let us from now look only for non allzero solutions.
Let $P(x,y)$ be the assertion $f(x+f(y))=f(y)^2+2xf(y)+f(-x)$
Let $u$ such that $f(u)\ne 0$
let $c=f(0)$

$P(\frac{x-f(u)^2}{2f(u)},u)$ $\implies$ $x=f(\frac{x+f(u)^2}{2f(u)})-f(\frac{f(u)^2-x}{2f(u)})$
So any real $x$ may be written in the form $x=f(a)-f(b)$ for some $a,b$ depending on $x$

$P(-f(b),a)$ $\implies$ $f(f(a)-f(b))=f(a)^2-2f(a)f(b)+f(f(b))$
$P(0,b)$ $\implies$ $f(f(b))=f(b)^2+c$
Adding, we get $f(f(a)-f(b))=(f(a)-f(b))^2+c$ and so $\boxed{\text{S2 : }f(x)=x^2+c\text{  }\forall x}$ which indeed is a solution, whatever is $c\in\mathbb R$
\end{solution}
*******************************************************************************
-------------------------------------------------------------------------------

\begin{problem}[Posted by \href{https://artofproblemsolving.com/community/user/78855}{KlausJanick}]
	Find all function $f : \mathbb{R} \to \mathbb{R}$ such that 

$  f\left ( x+f\left ( y \right ) \right )=3f\left ( x \right )+f\left ( y \right )-2x                                                                                         ,\forall x,y\in \mathbb{R}$
	\flushright \href{https://artofproblemsolving.com/community/c6h616973}{(Link to AoPS)}
\end{problem}



\begin{solution}[by \href{https://artofproblemsolving.com/community/user/29428}{pco}]
	\begin{tcolorbox}Find all function $f : \mathbb{R} \to \mathbb{R}$ such that 

$  f\left ( x+f\left ( y \right ) \right )=3f\left ( x \right )+f\left ( y \right )-2x                                                                                         ,\forall x,y\in \mathbb{R}$\end{tcolorbox}
Let $P(x,y)$ be the assertion $f(x+f(y))=3f(x)+f(y)-2x$
Let $a=f(0)$

$P(0,0)$ $\implies$ $f(a)=4a$
$P(a,0)$ $\implies$ $f(2a)=11a$
$P(2a,0)$ $\implies$ $f(3a)=30a$
$P(3a,0)$ $\implies$ $f(4a)=85a$
But also $P(0,a)$ $\implies$ $f(4a)=7a$ and so $a=0$

Then $P(x,0)$ $\implies$ $\boxed{f(x)=x}$ $\forall x$, which indeed is a solution.
\end{solution}



\begin{solution}[by \href{https://artofproblemsolving.com/community/user/212018}{Tintarn}]
	Setting $x=0$ we get $f(f(y))=3f(0)+f(y)$
Setting $x=f(a)$ and $y=b$ we get $f(f(a)+f(b))=9f(0)+3f(a)+f(b)-2a=9f(0)+3f(b)+f(a)-2b$, hence $f(a)-a=f(b)-b$ and thus $f(x)=x+c$. Inserting this back in the original equation we get $c=0$ and hence $f(x)=x$.
\end{solution}
*******************************************************************************
-------------------------------------------------------------------------------

\begin{problem}[Posted by \href{https://artofproblemsolving.com/community/user/197930}{ThisIsART}]
	Find all function $f:\mathbb{R}\to\mathbb{R}$ that satisfy $f(x(y+1)-f(x))=xf(y)$ for all $x,y \in \mathbb{R}$
	\flushright \href{https://artofproblemsolving.com/community/c6h617014}{(Link to AoPS)}
\end{problem}



\begin{solution}[by \href{https://artofproblemsolving.com/community/user/29428}{pco}]
	\begin{tcolorbox}Find all function $f:\mathbb{R}\to\mathbb{R}$ that satisfy $f(x(y+1)-f(x))=xf(y)$ for all $x,y \in \mathbb{R}$\end{tcolorbox}
$\boxed{\text{S1 : }f(x)=0\text{  }\forall x}$ is a solution. So let us rom now look only for non allzero solutions.
Let $P(x,y)$ be the assertion $f(x(y+1)-f(x))=xf(y)$
Let $u$ such that $f(u)\ne 0$

$P(\frac x{f(u)},u)$ $\implies$ $f(x)$ is surjective.
Since surjective, $\exists v,w$ such that $f(v)=-u$ and $f(w)=0$
If $f(-1)=0$, $P(v,-1)$ $\implies$ $f(u)=0$, impossible. So $f(-1)\ne 0$ and $P(x,-1)$ $\implies$ $f(-f(x))=xf(-1)$ and $f(x)$ is injective.

$P(x,w)$ $\implies$ $f(x(w+1)-f(x))=xf(w)=0=f(w)$ and so, since injective, $f(x)=x(w+1)-w$

Plugging this back in original equation, we get $w=0$ and so $\boxed{\text{S2 : }f(x)=x\text{  }\forall x}$ which indeed is a solution.
\end{solution}



\begin{solution}[by \href{https://artofproblemsolving.com/community/user/212018}{Tintarn}]
	$P(x,-1)$ implies $f(-f(x))=xf(-1)$. $P(0,-1)$ shows that there is $z$ such that $f(z)=0$. Then $P(z,-1)$ implies $f(0)=zf(-1)$. So, $f(-1)=0$ or our $z$ must be unique.

First case: $f(-1)=0$. Then also $f(-f(x))=xf(-1)=0$. If $f$ is constantly zero but if there is $x_0$ such that $f(x_0) \ne 0$ then by plugging in $y=x_0$ in the original equation we see that $f$ must be surjective, hence also $-f(x)$ covers all real values and therefore $f(x)=0$ for every $x$ because of $f(-f(x))=0$.

Second case: There is a unique $z$ such that $f(z)=0$.
Then $P(0,-1)$ implies $z=-f(0)$. But we know that $f(0)=zf(-1)=-f(0)f(-1)$. Thus, $f(0)=0$ or $f(-1)=-1$.
If $f(-1)=-1$ then $P(-1,0)$ implies $f(0)=0$. Thus, $f(0)=0$ holds in any case. Now, $P(x,0)$ implies $f(x-f(x))=0$ but for we know that the only point where $f$ is zero is $x=0$ we conclude that $f(x)-x=0$ and thus $f(x)=x$ for every $x$.
\end{solution}
*******************************************************************************
-------------------------------------------------------------------------------

\begin{problem}[Posted by \href{https://artofproblemsolving.com/community/user/197930}{ThisIsART}]
	Find all function $f:\mathbb{R}\to\mathbb{R}$ that satisfy $f(xf(y+1)-f(x))=xf(y)$ for all $x,y \in \mathbb{R}$
	\flushright \href{https://artofproblemsolving.com/community/c6h617153}{(Link to AoPS)}
\end{problem}



\begin{solution}[by \href{https://artofproblemsolving.com/community/user/159884}{hthtb22}]
	+)If $f(x) = 0$
+) Exist $x_0$ satisfy $f(x_0) \ne 0; f(x_0-1) \ne 0$
   Substitute $(x;y) \rightarrow (x;x_0-1)$ 
We get:    $f(-f(x))=xf(x_0-1) \Rightarrow f(x)$ Bijective
 $(x;y) \Rightarrow (1;y) \Leftrightarrow f(f(y+1)-f(1))=f(y) \Rightarrow f(y+1)=f(y)+f(1) \Rightarrow f(0)=0$
 $y \Rightarrow 0$ : $f(xf(1)-f(x))=0=f(0) \Rightarrow f(x)=f(1).x$

Retry: $f(x)=x$
\end{solution}



\begin{solution}[by \href{https://artofproblemsolving.com/community/user/212018}{Tintarn}]
	\begin{tcolorbox} Exist $x_0$ satisfy $f(x_0) \ne 0; f(x_0-1) \ne 0$\end{tcolorbox}
How do you know that? Notice that it could be $f(x)=0$ for every $x \in \mathbb{R} \ \{0\}$ and $f(0)=1$ (which is obviously not a solution) but then there would not be any such $x_0$. So how can you exclude such cases?
\end{solution}



\begin{solution}[by \href{https://artofproblemsolving.com/community/user/197930}{ThisIsART}]
	I agree with tintarn
So any other solution???
\end{solution}



\begin{solution}[by \href{https://artofproblemsolving.com/community/user/29428}{pco}]
	\begin{tcolorbox}Find all function $f:\mathbb{R}\to\mathbb{R}$ that satisfy $f(xf(y+1)-f(x))=xf(y)$ for all $x,y \in \mathbb{R}$\end{tcolorbox}
$\boxed{\text{S1 : }f(x)=0\text{  }\forall x}$ is a solution. So let us from now look only for non allzero solutions.
Let $P(x,y)$ be the assertion $f(xf(y+1)-f(x))=xf(y)$
Let $a=f(0)$

Since $\exists y$ such that $f(y)\ne 0$, we immediately get that $f(x)$ is surjective.

$P(0,0)$ $\implies$ $f(-a)=0$
$P(x,-a-1)$ $\implies$ $f(-f(x))=xf(-a-1)$

If $f(-a-1)\ne 0$, this implies that $f(x)$ is injective and then :
$P(1,x-1)$ $\implies$ $f(f(x)-f(1))=f(x-1)$ $\implies$ $f(x)=x+f(1)-1$ and so :
$\boxed{\text{S2 : }f(x)=x+a\text{  }\forall x}$ which indeed is a solution, whatever is $a\in\mathbb R$

If $f(-a-1)=0$, this implies $f(-f(x))=0$ $\forall x$ and then, since surjective, $f(x)=0$ $\forall x$, impossible in this part of the proof.
\end{solution}



\begin{solution}[by \href{https://artofproblemsolving.com/community/user/197930}{ThisIsART}]
	How can we have it is injective???
\end{solution}



\begin{solution}[by \href{https://artofproblemsolving.com/community/user/64716}{mavropnevma}]
	It was obtained that $f(-f(x)) = xf(-a-1)$ for all $x$. Say $f(-a-1) \neq 0$.
Assume $f(x_1)=f(x_2)$; then $0 = f(-f(x_1))  - f(-f(x_2))  = (x_1-x_2)f(-a-1)$, whence $x_1=x_2$, therefore $f$ is injective. 
Maybe working a little longer on a presented solution, rather than quickly shooting a question, will be more beneficial to you (and others like) ...
\end{solution}
*******************************************************************************
-------------------------------------------------------------------------------

\begin{problem}[Posted by \href{https://artofproblemsolving.com/community/user/195015}{Jul}]
	Find all function $f:R\rightarrow R$  such that : 
                                                                              \[f(x+2f(y))+yf(x+y)=f(xf(y))\]
	\flushright \href{https://artofproblemsolving.com/community/c6h617399}{(Link to AoPS)}
\end{problem}



\begin{solution}[by \href{https://artofproblemsolving.com/community/user/29428}{pco}]
	\begin{tcolorbox}Find all function $f:R\rightarrow R$  such that : 
                                                                              \[f(x+2f(y))+yf(x+y)=f(xf(y))\]\end{tcolorbox}
Let $P(x,y)$ be the assertion $f(x+2f(y))+yf(x+y)=f(xf(y))$
Let $a=f(0)$

If $f(x_1)=f(x_2)=c$ for some $x_1,x_2$, then :
$P(0,x_1)$ $\implies$ $f(2c)+cx_1=a$
$P(0,x_2)$ $\implies$ $f(2c)+cx_2=a$
And subtracting, we get $c(x_1-x_2)=0$

$P(x,0)$ $\implies$ $f(x+2a)=f(ax)$ and so, using previous result, new assertion $Q(x)$ : $f(x+2a)((a-1)x-2a)=0$ $\forall x$

$Q(-2a)$ $\implies$ $a=0$ and then $Q(x)$ becomes $xf(x)=0$ $\forall x$ and so $\boxed{f(x)=0}$ $\forall x$, which indeed is a solution.
\end{solution}
*******************************************************************************
-------------------------------------------------------------------------------

\begin{problem}[Posted by \href{https://artofproblemsolving.com/community/user/232685}{culebarca}]
	[img]http://latex.codecogs.com\/gif.latex?Determine%5C%2C%20function%20%5C%2C%20f%20%5C%2Csatisfying%20%5C%2C%20%5C%2C%20the%5C%2C%20relation%5C%2C%20%3Af%28x%29&plus;f%28%5Cfrac%7B1%7D%7B1-x%7D%29%3D%5Cfrac%7B2%281-2x%29%7D%7Bx%281-x%29%7D[\/img]
	\flushright \href{https://artofproblemsolving.com/community/c6h617495}{(Link to AoPS)}
\end{problem}



\begin{solution}[by \href{https://artofproblemsolving.com/community/user/29428}{pco}]
	\begin{tcolorbox}[img]http://latex.codecogs.com\/gif.latex?Determine%5C%2C%20function%20%5C%2C%20f%20%5C%2Csatisfying%20%5C%2C%20%5C%2C%20the%5C%2C%20relation%5C%2C%20%3Af%28x%29&plus;f%28%5Cfrac%7B1%7D%7B1-x%7D%29%3D%5Cfrac%7B2%281-2x%29%7D%7Bx%281-x%29%7D[\/img]\end{tcolorbox}
As usual, problem badly stated.

What is the domain of $f(x)$ ? : $\mathbb R?$,$\mathbb R\setminus \{1\}?$, $\mathbb R\setminus \{0\}?$, $\mathbb R\setminus \{0,1\}?$
What is the codomain of $f(x)$ ? . Likely $\mathbb R$
What is the domain of functional equation ? : $\mathbb R?$,$\mathbb R\setminus \{1\}?$, $\mathbb R\setminus \{0\}?$, $\mathbb R\setminus \{0,1\}?$

Depending on these answers, solutions may not exist, or exist but with some differences.

Please, copy the entire exact problem you got in your exam. The parts of the problem statement that \begin{bolded}you \end{underlined}\end{bolded}skipped because you thought there were useless may (and usually do) have importance !
\end{solution}



\begin{solution}[by \href{https://artofproblemsolving.com/community/user/29428}{pco}]
	\begin{tcolorbox}[img]http://latex.codecogs.com\/gif.latex?Determine%5C%2C%20function%20%5C%2C%20f%20%5C%2Csatisfying%20%5C%2C%20%5C%2C%20the%5C%2C%20relation%5C%2C%20%3Af%28x%29&plus;f%28%5Cfrac%7B1%7D%7B1-x%7D%29%3D%5Cfrac%7B2%281-2x%29%7D%7Bx%281-x%29%7D[\/img]\end{tcolorbox}
I suppose that domain of $f(x)$ is $[-4,4]$ and domain of functional equation is $\{2\}$

then obviously infinitely many solutions exist.

For example $f(x)=2-x$, or also $f(x)=-3\sin \frac{\pi x}2$ ... and general form of solutions is rather simple to find.
\end{solution}



\begin{solution}[by \href{https://artofproblemsolving.com/community/user/29428}{pco}]
	\begin{tcolorbox}[img]http://latex.codecogs.com\/gif.latex?Determine%5C%2C%20function%20%5C%2C%20f%20%5C%2Csatisfying%20%5C%2C%20%5C%2C%20the%5C%2C%20relation%5C%2C%20%3Af%28x%29&plus;f%28%5Cfrac%7B1%7D%7B1-x%7D%29%3D%5Cfrac%7B2%281-2x%29%7D%7Bx%281-x%29%7D[\/img]\end{tcolorbox}
I suppose now  that domain of $f(x)$ is $\mathbb R\setminus\{0,1\}$
I suppose now that codomain of $f(x)$ is $\mathbb R$
I suppose now that domain of functional equation is $\mathbb R\setminus\{0,1\}$

Let $g,h$, from $\mathbb R\setminus\{0,1\}\to\mathbb R$ defined as $g(x)=\frac 1{1-x}$ and $h(x)=\frac {2(1-2x)}{x(1-x)}$

Note that $g(g(g(x)))$ from $\mathbb R\setminus\{0,1\}\to\mathbb R$ exists and is $g(g(g(x)))=x$. Then :

(a) : $f(x)+f(g(x))=h(x)$
(b) : $f(g(x))+f(g(g(x)))=h(g(x))$
(c) : $f(g(g(x)))+f(x)=h(g(g(x)))$
(a)-(b)+(c) : $\boxed{f(x)=\frac{h(x)-h(g(x))+h(g(g(x)))}2}$

I let you finish the calculus ...
\end{solution}
*******************************************************************************
-------------------------------------------------------------------------------

\begin{problem}[Posted by \href{https://artofproblemsolving.com/community/user/185787}{gobathegreat}]
	Find all functions $ f:\mathbb{R}\rightarrow\mathbb{R} $ such that for all $x,y\in{{\mathbb{R}}}$ holds
$f(x^2)+f(2y^2)=(f(x+y)+f(y))(f(x-y)+f(y))$

\begin{italicized}Proposed by Matija Bucić\end{italicized}
	\flushright \href{https://artofproblemsolving.com/community/c6h617573}{(Link to AoPS)}
\end{problem}



\begin{solution}[by \href{https://artofproblemsolving.com/community/user/173116}{Sardor}]
	It's not very hard!
Here my longer solution:
Let $ P(x,y) $ be the assertion of the function.
From $ P(x,-y)=P(x,y) $, we get $ (f(y)-f(-y))(f(x+y)+f(x-y)+f(y)+f(-y) (*) $.
We have $ P(0,0)  \implies f(0)=0 $ or $ f(0)=\frac{1}{2} $.
Case 1: $ f(0)=0 $
i) $ P(0,x) \implies f(2x^2)=2f(x)(f(x)+f(-x)) $ and  
$ P(x,0) \implies f(x^2)=f(x)^2 $, thus $ f(x)=f(-x) $ or $ f(x)=-f(-x) $.
If $ f $ is odd, the $ f(2x^2)=0 $, so $ f(x)=0 $ for all $ x \in R_0 $ and since $ f $ is odd, we get $ f(x)=0 $, for any $ x $.
Let  $ f $ is even, then $ f(2x^2)=4f(x)^2=4f(x^2) $, so $ f(2x)=4f(x) $, for any $ x $ ( because $ f $ is even).Taking $ x=y $, in original equation, we get $ 3f(x)^2=2f(x)f(2x)=8f(x)^2 $ or $ f(x)=0 $, for any $ x $.

Case 2: $ f(0)=\frac{1}{2} $.
Then  $ P(x,0) \implies  f(x^2)=f(x)^2+f(x)-\frac{1}{4} $ , so $ f $ is even or $ f(x)+f(-x)+1=0 $
a) $ f(x)=f(-x) $. Then $ P(0,x) \implies f(2x^2)=4f(x)^2-\frac{1}{2} $.Hence
$ (f(x+y)+f(y))(f(x-y)+f(y)=f(x^2)+f(2y^2)=f(x)^2+f(x)+4f(y)^2-\frac{3}{4} $.Then taking $ x=y $ in the least equation  we have $ f(2x)=4f(x)-\frac{3}{2} $ (because $ f(x)=-\frac{1}{2} $ isn't solution).So we have $ 4f(x)^2-\frac{1}{2}=f(2x^2)=4f(x^2)-\frac{3}{2}=4f(x)^2+4f(x)-1-\frac{3}{2} $, so $ f(x)=\frac{1}{2} $
b) Let $ f(x)+f(-x)+1=0 $, then by $ (*) $, we get $ f(x+y)+f(x-y)=1 $ for $ y $ not equal to zero, and taking $ y=x $, we get $ f(x)=\frac{1}{2} $, for any $ x $, but $ 0=f(x)+f(-x)+1=2 $, a contradiction.
Answer:
1) $ f(x)=0 $;
2) $ f(x)=\frac{1}{2} $;
\end{solution}



\begin{solution}[by \href{https://artofproblemsolving.com/community/user/29428}{pco}]
	\begin{tcolorbox}It's not very hard!
...
Answer:
1) $ f(x)=0 $;
2) $ f(x)=\frac{1}{2} $;\end{tcolorbox}

And what about $f(x)=x^2$ for example ?
\end{solution}



\begin{solution}[by \href{https://artofproblemsolving.com/community/user/224669}{Mukhammadiev}]
	You are right, Sardor. My solution almost the same as your's. However, $ f(x)=x^2$ holds the aforementioned equation, the equation does not pay the way for this.
\end{solution}



\begin{solution}[by \href{https://artofproblemsolving.com/community/user/168801}{joyce_tan}]
	\begin{tcolorbox}
$ P(x,0) \implies f(x^2)=f(x)^2 $, thus $ f(x)=f(-x) $ or $ f(x)=-f(-x) $.
If $ f $ is odd, the $ f(2x^2)=0 $, so $ f(x)=0 $ for all $ x \in R_0 $ and since $ f $ is odd, we get $ f(x)=0 $, for any $ x $.
Let  $ f $ is even, then ... \end{tcolorbox}

I don't think this line is valid. Even though $ f(x)=f(-x) $ or $ f(x)=-f(-x) $ at any particular x, that doesn't mean $f(x)=f(-x)$ over all x or $f(x)=-f(-x)$ over all x, like you seem to assume in the next few lines.
\end{solution}
*******************************************************************************
-------------------------------------------------------------------------------

\begin{problem}[Posted by \href{https://artofproblemsolving.com/community/user/152589}{rightways}]
	Find all functions $f :\mathbb{R}\to\mathbb{R}$, satisfying the condition

$f(f(x)+x+y)=2x+f(y)$

for any real $x$ and $y$.
	\flushright \href{https://artofproblemsolving.com/community/c6h617711}{(Link to AoPS)}
\end{problem}



\begin{solution}[by \href{https://artofproblemsolving.com/community/user/29428}{pco}]
	\begin{tcolorbox}Find all functions $f :\mathbb{R}\to\mathbb{R}$, satisfying the condition

$f(f(x)+x+y)=2x+f(y)$

for any real $x$ and $y$.\end{tcolorbox}
Let $P(x,y)$ be the assertion $f(f(x)+x+y)=2x+f(y)$

1) problem is equivalent to $f(x)$ additive and $f(f(x))+f(x)=2x$
==============================================
$P(x,-f(x))$ $\implies$ $f(x)=2x+f(-f(x))$ and so $f(x)$ is injective.
$P(0,0)$ $\implies$ $f(f(0))=f(0)$ and so, since injective, $f(0)=0$

$P(x,0)$ $\implies$ $f(f(x)+x)=2x$
$P(-x,x-f(-x))$ $\implies$ $f(x-f(-x))=2x$
And so, since injective, $f(x)+x=x-f(-x)$ and $f(-x)=-f(x)$ and $f(x)$ is odd.

$P(x,-f(x))$ $\implies$ $f(f(x))+f(x)=2x$ (since $f(x)$ is odd) and so $f(x)+x$ is surjective.

$P(x,0)$ $\implies$ $f(f(x)+x)=2x$ and so $P(x,y)$ implies $f((f(x)+x)+y)=f(f(x)+x)+f(y)$
And so, since $f(x)+x$ is surjective, $f(x+y)=f(x)+f(y)$ and $f(x)$ is additive.

And so problem is equivalent to $f(x)$ additive and $f(f(x))+f(x)=2x$
Q.E.D.

2) General solution of $f(x)$ additive and $f(f(x))+f(x)=2x$
===========================================
2.1) Claim about general form
----------------------------------
Let $A,B$ two supplementary subvectorspaces of the $\mathbb Q$-vectorspace $\mathbb R$
let $a(x)$ from $\mathbb R\to A$ and $b(x)$ from $\mathbb R\to B$ the two projections of $x$ in A,B (so that $x=a(x)+b(x)$ in a unique way
Then $f(x)=a(x)-2b(x)$

2.2) Proof that functions in the form of 2.1 indeed are solutions
--------------------------------------------------------------------------
$f(x)=a(x)-2b(x)$ is trivially additive
$f(f(x))=a(x)+4b(x)$ and so $f(f(x))+f(x)=a(x)+4b(x)+a(x)-2b(x)=2(a(x)+b(x))=2x$
Q.E.D.

2.3) Proof that any solution may be put in the form in 2.1, so that this indeed is a general solution
---------------------------------------------------------------------------------------------------------------------
Let $f(x)$ a solution (additive function such that $f(f(x))+f(x)=2x$)
Let $A=\{x\in\mathbb R$ such that $f(x)=x\}$ : $A$ is obviously a subvectorspace of the $\mathbb Q$-vectorspace $\mathbb R$
Let $B=\{x\in\mathbb R$ such that $f(x)=-2x\}$ : $B$ is obviously a subvectorspace of the $\mathbb Q$-vectorspace $\mathbb R$
If $x\in A\cap B$, then $f(x)=x=-2x$ and so $A\cap B=\{0\}$

Let $a(x)=\frac{f(x)+2x}3$ : $f(a(x))=\frac{f(f(x))+2f(x)}3$ $=\frac{f(x)+2x}3$ $=a(x)$ and so $a(x)\in A$

Let $b(x)=\frac{x-f(x)}3$ : $f(b(x))=\frac{f(x)-f(f(x))}3$ $=\frac{2f(x)-2x}3$ $=-2b(x)$ and so $b(x)\in B$

$a(x)+b(x)=x$ and so $A,B$ are supplementary.

And $a(x)-2b(x)=f(x)$
Q.E.D.

2.4) Examples
-----------------
$(A,B)=(\mathbb R,\{0\})$ gives $f(x)=x$
$(A,B)=(\{0\},\mathbb R)$ gives $f(x)=-2x$
And infinitely many other solutions (using axiom of choice and Hamel basis)
\end{solution}



\begin{solution}[by \href{https://artofproblemsolving.com/community/user/188879}{chomk}]
	$a(x)=\frac{f(x)+2x}3$  
does this mean that $a(x)$ is determined by $f(x)$ and $x$?

\end{solution}



\begin{solution}[by \href{https://artofproblemsolving.com/community/user/29428}{pco}]
	\begin{tcolorbox}$a(x)=\frac{f(x)+2x}3$  
does this mean that $a(x)$ is determined by $f(x)$ and $x$?\end{tcolorbox}

In 2.3) I want to prove that any solution may be put in the form given in 2.1

So I choosed $a(x)$ as given and proved that it allows to put the solution in the given form, which is the aim of the paragraph 2.3

Maybe some other $a(x)$ are possible but this form is enough to achieve the goal of paragraph 2.3


\end{solution}



\begin{solution}[by \href{https://artofproblemsolving.com/community/user/188879}{chomk}]
	dear pco. it seems that you choose $f(x)$ as given.  
\end{solution}



\begin{solution}[by \href{https://artofproblemsolving.com/community/user/29428}{pco}]
	\begin{tcolorbox}dear pco. it seems that you choose $f(x)$ as given.\end{tcolorbox}

I dont understand what you wrote. I am sorry.

In paragraph 2.3, I suppose $f(x)$ \begin{bolded}\begin{italicized}given \end{italicized}\end{bolded} and matching all requirements of the functional equation and I proved then that $\exists A,B,a(x),b(x)$ matching all requirements of 2.1 so that the existing $f(x)$ can indeed be put in the claimed form, proving so that the given form is a general solution (since any function in this form is a solution and since any solution can be put in this form).

What exactly is unclear there for you ?

\end{solution}



\begin{solution}[by \href{https://artofproblemsolving.com/community/user/188879}{chomk}]
	\begin{tcolorbox}
So I choosed $a(x)$ as given \end{tcolorbox}

I misunderstood that. sorry 

suppose $f(x)$ given
choose $a(x)$ \begin{bolded}as\end{bolded} given

now it's clear . thank you very much.
\end{solution}
*******************************************************************************
-------------------------------------------------------------------------------

\begin{problem}[Posted by \href{https://artofproblemsolving.com/community/user/195015}{Jul}]
	Find all function $f:\mathbb{R}\rightarrow \mathbb{R}$ and such that :
\[f(2yf(x)+x^2)+f(y^2)=f^2(x+y),\;\forall x,y\in \mathbb{R}\]
	\flushright \href{https://artofproblemsolving.com/community/c6h618048}{(Link to AoPS)}
\end{problem}



\begin{solution}[by \href{https://artofproblemsolving.com/community/user/29428}{pco}]
	\begin{tcolorbox}Find all function $f:\mathbb{R}\rightarrow \mathbb{R}$ and such that :
\[f(2yf(x)+x^2)+f(y^2)=f^2(x+y),\;\forall x,y\in \mathbb{R}\]\end{tcolorbox}
The only constant solutions are $\boxed{\text{S1 : }f(x)=0\text{  }\forall x}$ and $\boxed{\text{S2 : }f(x)=2\text{  }\forall x}$.
So let us from now look only for non constant solutions.
Let $P(x,y)$ be the assertion $f(2yf(x)+x^2)+f(y^2)=f(x+y)^2$

$P(x,0)$ $\implies$ $f(x^2)+f(0)=f(x)^2$
$P(0,x)$ $\implies$ $f(2f(0)x)+f(x^2)=f(x)^2$
Subtracting, we get $f(2f(0)x)=f(0)$ and so $f(0)=0$ since $f(x)$ is not constant.

$P(x,0)$ $\implies$ $f(x^2)=f(x)^2$ and so $f(x)\ge 0$ $\forall x\ge 0$
As a consequence, $f(2yf(x)+x^2)\ge 0$ $\forall x,y\ge 0$ and so $f(x+y)^2\ge f(y)^2$ and $f(x+y)\ge f(y)\ge 0$ $\forall x,y\ge 0$
So $f(x)$ is non decreasing over $[0,+\infty)$

If $f(x)\ne 0$ $\forall x\ne 0$, then $P(x,-\frac x2)$ $\implies$ $f(-xf(x)+x^2)=0$ and so $\boxed{\text{S3 : }f(x)=x\text{  }\forall x}$ which indeed is a solution.

If $f(u)=0$ for some $u\ne 0$, $f(-u)^2=f(u^2)=f(u)^2=0$ and so WLOG consider $u>0$
Let then $x\ge 0$ : $P(u,x)$ $\implies$ $f(x+u)^2=f(x)^2$ and so $f(x+u)=f(x)$ since both are $\ge 0$
As a consequence $f(nu)=0$ and so, since non negative and non decreasing over $[0,+\infty)$ : $f(x)=0$ $\forall x\ge 0$

Then, since $f(-x)^2=f(x)^2$, we get $f(x)=0$ $\forall x$ which is not a solution in this part of the proof.
\end{solution}
*******************************************************************************
-------------------------------------------------------------------------------

\begin{problem}[Posted by \href{https://artofproblemsolving.com/community/user/33274}{toanIneq}]
	Find all function $f : \mathbb{R} \to \mathbb{R}$ such that 
$f\left ( f\left ( x+y \right ) \right )=f\left ( x+y \right )+f(x).f(y)-xy  ,\forall x,y\in \mathbb{R}$
	\flushright \href{https://artofproblemsolving.com/community/c6h618319}{(Link to AoPS)}
\end{problem}



\begin{solution}[by \href{https://artofproblemsolving.com/community/user/29428}{pco}]
	\begin{tcolorbox}Find all function $f : \mathbb{R} \to \mathbb{R}$ such that 
$f\left ( f\left ( x+y \right ) \right )=f\left ( x+y \right )+f(x).f(y)-xy  ,\forall x,y\in \mathbb{R}$\end{tcolorbox}
Let $P(x,y)$ be the assertion $f(f(x+y))=f(x+y)+f(x)f(y)-xy$
$f(x)=0$$\forall x$ is not a solution. So let $u,v$ such that $f(u)=v\ne 0$

If $f(0)=0$ :
$P(x,u)$ $\implies$ $f(f(x+u))=f(x+u)+vf(x)-ux$
$P(x+u,0)$ $\implies$ $f(f(x+u))=f(x+u)$
Subtracting, we get $f(x)=\frac uvx$ and plugging this back in original eaquation, we get $\boxed{f(x)=x\text{  }\forall x}$ 

If $f(0)\ne 0$, let $a=\frac 1{f(0)}\ne 0$ :
$P(x,y)$ $\implies$ $f(f(x+y))=f(x+y)+f(x)f(y)-xy$
$P(x+y,0)$ $\implies$ $f(f(x+y))=f(x+y)+\frac 1af(x+y)$
Subtracting, we get new assertion $Q(x,y)$ : $f(x+y)=af(x)f(y)-axy$

$Q(2,x)$ $\implies$ $f((1+1)+x)=af(1+1)f(x)-a(1+1)x$ $=a^2f(1)f(1)f(x)-a^2f(x)-2ax$
$Q(1+x,1)$ $\implies$ $f((1+x)+1)=af(1+x)f(1)-a(1+x)$ $=a^2f(1)f(1)f(x)-a^2xf(1)-a-ax$
Subtracting, we get $f(x)=x(f(1)-f(0))+f(0)$
Plugging this back in original equation, we get no solution with $f(0)\ne 0$
\end{solution}
*******************************************************************************
-------------------------------------------------------------------------------

\begin{problem}[Posted by \href{https://artofproblemsolving.com/community/user/33274}{toanIneq}]
	Find all function $f : \mathbb{R} \to \mathbb{R}$ such that 
$f\left ( f\left ( x-y \right ) \right )=f ( x )-f(y)+f(x).f(y)-xy  ,\forall x,y\in \mathbb{R}$
	\flushright \href{https://artofproblemsolving.com/community/c6h618320}{(Link to AoPS)}
\end{problem}



\begin{solution}[by \href{https://artofproblemsolving.com/community/user/29428}{pco}]
	\begin{tcolorbox}Find all function $f : \mathbb{R} \to \mathbb{R}$ such that 
$f\left ( f\left ( x-y \right ) \right )=f ( x )-f(y)+f(x).f(y)-xy  ,\forall x,y\in \mathbb{R}$\end{tcolorbox}
Let $P(x,y)$ be the assertion $f(f(x-y))=f(x)-f(y)+f(x)f(y)-xy$
Let $a=f(0)$
Note that $P(x,y)$ implies that $f(x)$ can not be bounded

$P(f(x),f(x))$ $\implies$ $f(f(x))^2=f(x)^2+f(a)$
$P(x,0)$ $\implies$ $f(f(x))=(a+1)f(x)-a$ $\implies$ $f(f(x))^2=(a^2+2a+1)f(x)^2-2a(a+1)f(x)+a^2$
Subtracting, we get $a(a+2)f(x)^2-2a(a+1)f(x)+a^2-f(a)=0$
If $a\ne 0$, this implies that $f(x)$ can only take one or two values, and so is bounded, impossible. So $a=0$

$P(x,x)$ $\implies$ $f(x)^2=x^2$ and so $\forall x$, either $f(x)=x$, either $f(x)=-x$

Suppose now $\exists x,y\ne 0$ such that $f(x)=x$ and $f(y)=-y$
$P(x,y)$ $\implies$ $f(f(x-y))=x+y-2xy$ and so :
either $x-y=x+y-2xy$ and $x=1$
either $y-x=x+y-2xy$ and $y=1$
So :
Either $\boxed{f(x)=x}$ $\forall x$ which indeed is a solution
Either $f(x)=-x$ $\forall x$ which is not a solution
Either $f(x)=x$ $\forall x\ne 1$ and $f(1)=-1$ which is not a solution
Either $f(x)=-x$ $\forall x\ne 1$ and $f(1)=1$ which is not a solution
\end{solution}
*******************************************************************************
-------------------------------------------------------------------------------

\begin{problem}[Posted by \href{https://artofproblemsolving.com/community/user/33274}{toanIneq}]
	Find all function $f :\mathbb{R}\setminus \left \{ 0 \right \}  \to \mathbb{R}$ that satisfies following conditions
a) $ f(1)=1$

b)  $f\left ( \frac{1}{x+y} \right )=f\left ( \frac{1}{x} \right )+f\left ( \frac{1}{y} \right )$

c)  $\left ( x+y \right )f(x+y)=xyf(x)f(y),\forall x,y,xy(x+y)\neq 0$
	\flushright \href{https://artofproblemsolving.com/community/c6h618322}{(Link to AoPS)}
\end{problem}



\begin{solution}[by \href{https://artofproblemsolving.com/community/user/29428}{pco}]
	\begin{tcolorbox}Find all function $f :\mathbb{R}\setminus \left \{ 0 \right \}  \to \mathbb{R}$ that satisfies following conditions
a) $ f(1)=1$

b)  $f\left ( \frac{1}{x+y} \right )=f\left ( \frac{1}{x} \right )+f\left ( \frac{1}{y} \right )$

c)  $\left ( x+y \right )f(x+y)=xyf(x)f(y),\forall x,y,xy(x+y)\neq 0$\end{tcolorbox}
I suppose that condition $\forall x,y,xy(x+y)\ne 0$ in condition c is available too for condition b. Else obviously no solution.

Setting $g(x)$ from $\mathbb R\setminus\{0\}\to\mathbb R$ as $g(x)=f(\frac 1x)$, condition b implies $g(x)$ is additive.

Appling condition c for $x=\frac 1t$ and $y=\frac 1{1-t}$ for $t\notin\{0,1\}$, we get, using $g(1)=1$ and $g(x)$ additive :
$g(t^2)=g(t)^2\ge 0$, still true when $t=1$

So $g(x)$ is non decreasing and so $g(x)=x$ $\forall x\ne 0$ and so $\boxed{f(x)=\frac 1x\text{  }\forall x\ne 0}$ which indeed is a solution.
\end{solution}



\begin{solution}[by \href{https://artofproblemsolving.com/community/user/212018}{Tintarn}]
	This is actually Problem 10 from Baltic Way 1995:
http://www.artofproblemsolving.com/Forum/viewtopic.php?p=2463349&sid=42d145f8744babc224e7170bda99d6e3#p2463349
\end{solution}
*******************************************************************************
-------------------------------------------------------------------------------

\begin{problem}[Posted by \href{https://artofproblemsolving.com/community/user/33274}{toanIneq}]
	Find all $ f: \mathbb{R}\rightarrow \mathbb{R} $   that satisfies following conditions.
a) If $x\neq y$ ,  then     $ f(x)\neq f(y)$

b) $f\left ( xf\left ( y \right ) -y\right )=f\left ( yf\left ( x \right )-x \right )+f(x-y)$
	\flushright \href{https://artofproblemsolving.com/community/c6h618324}{(Link to AoPS)}
\end{problem}



\begin{solution}[by \href{https://artofproblemsolving.com/community/user/29428}{pco}]
	\begin{tcolorbox}Find all $ f: \mathbb{R}\rightarrow \mathbb{R} $   that satisfies following conditions.
a) If $x\neq y$ ,  then     $ f(x)\neq f(y)$

b) $f\left ( xf\left ( y \right ) -y\right )=f\left ( yf\left ( x \right )-x \right )+f(x-y)$\end{tcolorbox}
Let $P(x,y)$ be the assertion $f(xf(y)-y)=f(yf(x)-x)+f(x-y)$
$P(0,0)$ $\implies$ $f(0)=0$
As a consequence, since injective, $f(x)\ne 0$ $\forall x\ne 0$

Let $x\ne 0$ : $P(x,\frac x{f(x)})$ $\implies$ $f\left(xf(\frac x{f(x)})-\frac x{f(x)}\right)=f\left(x-\frac x{f(x)}\right)$ and so, since injective, $xf(\frac x{f(x)})-\frac x{f(x)}=x-\frac x{f(x)}$

$\implies$ $f(\frac x{f(x)})=1$ and so, since injective, $\boxed{f(x)=ax\text{  }\forall x}$ which indeed is a solution, whatever is $a\in\mathbb R\setminus\{0\}$
\end{solution}
*******************************************************************************
-------------------------------------------------------------------------------

\begin{problem}[Posted by \href{https://artofproblemsolving.com/community/user/33274}{toanIneq}]
	Find all $f:\left [ 0;1 \right ]\rightarrow \left [ 0;1 \right ]    $   that satisfies following conditions.
a) $f\left ( x_{1} \right )\neq f\left ( x_{2} \right ),\forall x_{1} \neq x_{2}$

b) $2x-f(x)\in \left [ 0;1 \right ],\forall x\in \left [ 0;1 \right ]$

c) $ f\left ( 2x-f\left ( x \right ) \right )=x$
	\flushright \href{https://artofproblemsolving.com/community/c6h618326}{(Link to AoPS)}
\end{problem}



\begin{solution}[by \href{https://artofproblemsolving.com/community/user/29428}{pco}]
	\begin{tcolorbox}Find all $f:\left [ 0;1 \right ]\rightarrow \left [ 0;1 \right ]    $   that satisfies following conditions.
a) $f\left ( x_{1} \right )\neq f\left ( x_{2} \right ),\forall x_{1} \neq x_{2}$

b) $2x-f(x)\in \left [ 0;1 \right ],\forall x\in \left [ 0;1 \right ]$

c) $ f\left ( 2x-f\left ( x \right ) \right )=x$\end{tcolorbox}
Setting $x=0$ in b), we get $f(0)=0$ and so, since injective, $f(x)\ne 0$ $\forall x\ne 0$
Setting $x=1$ in b), we get $f(1)=1$ and so, since injective, $f(x)\ne 1$ $\forall x\ne 1$

Let then $u\in(0,1)$ and $v=f(u)\in(0,1)$ :
From $f(u)=v$, applying c) for $x=u$, we get $f(2u-v)=u$
From $f(2u-v)=u$, applying c) for $x=2u-v$, we get $f(3u-2v)=2u-v$
And simple induction gives $f((n+1)u-nv)=nu-(n-1)v$
And so $0\le nu-(n-1)v\le 1$ $\forall n\in\mathbb N$

And so $\frac{n-1}n\le \frac uv \le \frac{n-1}n+\frac 1{nv}$
Setting $n\to+\infty$ in this inequality, we get $u=v$ and so $\boxed{f(x)=x\text{  }\forall x\in[0,1]}$ which indeed is a solution.
\end{solution}
*******************************************************************************
-------------------------------------------------------------------------------

\begin{problem}[Posted by \href{https://artofproblemsolving.com/community/user/33274}{toanIneq}]
	Find all $ f,g: \mathbb{R}\rightarrow \mathbb{R} $   that satisfies following conditions.
a) $f(x^{2})-f(y^{2})=\left ( x-y \right )g\left ( x+y \right ),\forall x,y\in \mathbb{R}$

b) $f(-x)=-f(x),\forall x\in \mathbb{R}$
	\flushright \href{https://artofproblemsolving.com/community/c6h618328}{(Link to AoPS)}
\end{problem}



\begin{solution}[by \href{https://artofproblemsolving.com/community/user/29428}{pco}]
	\begin{tcolorbox}Find all $ f,g: \mathbb{R}\rightarrow \mathbb{R} $   that satisfies following conditions.
a) $f(x^{2})-f(y^{2})=\left ( x-y \right )g\left ( x+y \right ),\forall x,y\in \mathbb{R}$

b) $f(-x)=-f(x),\forall x\in \mathbb{R}$\end{tcolorbox}
From b), we get $f(0)=0$
Setting then $y=0$ in a), we get $f(x^2)=xg(x)$ and a) becomes assertion $P(x,y)$ : $xg(x)-yg(y)=(x-y)g(x+y)$

$P(\frac{x+1}2,\frac{x-1}2)$ $\implies$ $\frac{x+1}2g(\frac{x+1}2)-\frac{x-1}2g(\frac{x-1}2)=g(x)$

$P(\frac{x-1}2,\frac{1-x}2)$ $\implies$ $\frac{x-1}2g(\frac{x-1}2)-\frac{1-x}2g(\frac{1-x}2)=(x-1)g(0)$

$P(\frac{1-x}2,\frac{x+1}2)$ $\implies$ $\frac{1-x}2g(\frac{1-x}2)-\frac{x+1}2g(\frac{x+1}2)=-xg(1)$

Adding these three lines, we get $g(x)=(g(1)-g(0))x+g(0)$
Plugging this back in original equations, we get $\boxed{f(x)=g(x)=ax\text{  }\forall x}$ which indeed is a solution, whatever is $a\in\mathbb R$
\end{solution}
*******************************************************************************
-------------------------------------------------------------------------------

\begin{problem}[Posted by \href{https://artofproblemsolving.com/community/user/33274}{toanIneq}]
	Find all $ f: \mathbb{R}\rightarrow \mathbb{R} $   that satisfies following conditions
a)  $f\left ( x^{2}-y \right )=xf(x)-f(y),\forall x,y\in \mathbb{R}$

b)  $xf(x)> 0,\forall x\neq 0$
	\flushright \href{https://artofproblemsolving.com/community/c6h618329}{(Link to AoPS)}
\end{problem}



\begin{solution}[by \href{https://artofproblemsolving.com/community/user/29428}{pco}]
	\begin{tcolorbox}Find all $ f: \mathbb{R}\rightarrow \mathbb{R} $   that satisfies following conditions
a)  $f\left ( x^{2}-y \right )=xf(x)-f(y),\forall x,y\in \mathbb{R}$

b)  $xf(x)> 0,\forall x\neq 0$\end{tcolorbox}
Let $P(x,y)$ be the assertion $f(x^2-y)=xf(x)-f(y)$
$P(0,0)$ $\implies$ $f(0)=0$
$P(0,x)$ $\implies$ $f(-x)=-f(x)$ and $f(x)$ is odd
$P(x,0)$ $\implies$ $f(x^2)=xf(x)$ and so $P(x,y)$ becomes $f(x^2-y)=f(x^2)-f(y)$ and so $f(x-y)=f(x)-f(y)$ $\forall x\ge 0,\forall y$
And, since odd, $f(x+y)=f(x)+f(y)$ $\forall x,y$

$xf(x)>0$ $\forall x\ne 0$ implies $f(x)>0$ $\forall x>0$ and so $f(x)$ is additive and increasing 
and so $\boxed{f(x)=ax\text{  }\forall x}$ which indeed is a solution, whatever is $a>0$
\end{solution}
*******************************************************************************
-------------------------------------------------------------------------------

\begin{problem}[Posted by \href{https://artofproblemsolving.com/community/user/33274}{toanIneq}]
	Find all $ f: \mathbb{R}\rightarrow \mathbb{R} $   that satisfies following conditions
a)  $ f(x)+f(y)\neq 0 $

b)  $  \frac{f(x)-f(x-y)}{f(x)+f(x+y)}+\frac{f(x)-f(x+y)}{f(x)+f(x-y)}=0,\forall x,y\in \mathbb{R}$
	\flushright \href{https://artofproblemsolving.com/community/c6h618330}{(Link to AoPS)}
\end{problem}



\begin{solution}[by \href{https://artofproblemsolving.com/community/user/29428}{pco}]
	\begin{tcolorbox}Find all $ f: \mathbb{R}\rightarrow \mathbb{R} $   that satisfies following conditions
a)  $ f(x)+f(y)\neq 0 $

b)  $  \frac{f(x)-f(x-y)}{f(x)+f(x+y)}+\frac{f(x)-f(x+y)}{f(x)+f(x-y)}=0,\forall x,y\in \mathbb{R}$\end{tcolorbox}
Equation is $f(x-y)^2+f(x+y)^2=2f(x)^2$ and so, setting $g(x)=f(x)^2$ :

$g(\frac{x+y}2)=\frac{g(x)+g(y)}2$ and $g(x)>0$ $\forall x$

This is a quite classical problem whose only solution is $g(x)=a$ $\forall x$ and for some $a>0$

And so $f(x)=\pm\sqrt a$ $\forall x$ and condition a) implies then $f(x)=\sqrt a$ $\forall x$ or $f(x)=-\sqrt a$ $\forall x$

And so $\boxed{f(x)=c\text{  }\forall x}$ which indeed is a solution whatever is $c\in\mathbb R\setminus\{0\}$
\end{solution}
*******************************************************************************
-------------------------------------------------------------------------------

\begin{problem}[Posted by \href{https://artofproblemsolving.com/community/user/33274}{toanIneq}]
	Find all functions $f:\mathbb{R}^{+}\rightarrow \mathbb{R}^{+}  $ such that

$xf\left ( xf(y) \right )=f(f(y)),\forall x,y\in \mathbb{R}^{+}$
	\flushright \href{https://artofproblemsolving.com/community/c6h618333}{(Link to AoPS)}
\end{problem}



\begin{solution}[by \href{https://artofproblemsolving.com/community/user/29428}{pco}]
	\begin{tcolorbox}Find all functions $f:\mathbb{R}^{+}\rightarrow \mathbb{R}^{+}  $ such that

$xf\left ( xf(y) \right )=f(f(y)),\forall x,y\in \mathbb{R}^{+}$\end{tcolorbox}

Let $P(x,y)$ be the assertion $xf(xf(y))=f(f(y))$

$P(\frac x{f(1)},1)$ $\implies$ $f(x)=\frac{f(1)f(f(1))}x$ and so $\boxed{f(x)=\frac ax\text{  }\forall x>0}$ which indeed is a solution, whatever is $a>0$
\end{solution}



\begin{solution}[by \href{https://artofproblemsolving.com/community/user/173116}{Sardor}]
	Let $ P(x,y) $ be the assertion of the function and $ f(1)=a>0 $.
$ P(\frac{1}{a}, 1) \implies  f(a)=1 $;
$ P(x,1) \implies xf(ax)=f(a)=1 $, hence $ f(x)=\frac{a}{x} $ for any $ x \in R^{+} $, where $ a>0 $.
\end{solution}
*******************************************************************************
-------------------------------------------------------------------------------

\begin{problem}[Posted by \href{https://artofproblemsolving.com/community/user/33274}{toanIneq}]
	Find all functions $f:\mathbb{R}^{+}\rightarrow \mathbb{R}^{+}  $ such that

$ f\left ( \frac{f(x)}{y} \right )=yf(y)f(f(x))  ,\forall x,y\in \mathbb{R}^{+}$
	\flushright \href{https://artofproblemsolving.com/community/c6h618334}{(Link to AoPS)}
\end{problem}



\begin{solution}[by \href{https://artofproblemsolving.com/community/user/29428}{pco}]
	\begin{tcolorbox}Find all functions $f:\mathbb{R}^{+}\rightarrow \mathbb{R}^{+}  $ such that

$ f\left ( \frac{f(x)}{y} \right )=yf(y)f(f(x))  ,\forall x,y\in \mathbb{R}^{+}$\end{tcolorbox}
Let $P(x,y)$ be the assertion $f(\frac{f(x)}y)=yf(y)f(f(x))$

$P(1,1)$ $\implies$ $f(1)=1$
$P(1,x)$ $\implies$ $f(\frac 1x)=xf(x)$ and so any real $x>0$ may be written as $x=\frac{f(a)}{f(b)}$ for some real $a,b>0$

$P(a,f(a))$ $\implies$ $f(f(a))=\frac 1{\sqrt{f(a)}}$

$P(a,f(b))$ $\implies$ $f(\frac{f(a)}{f(b)})=f(b)f(f(b))f(f(a))$ $=\sqrt{\frac{f(b)}{f(a)}}$

And so $\boxed{f(x)=\frac 1{\sqrt x}\text{  }\forall x>0}$ which indeed is a solution.
\end{solution}
*******************************************************************************
-------------------------------------------------------------------------------

\begin{problem}[Posted by \href{https://artofproblemsolving.com/community/user/33274}{toanIneq}]
	Find all $ f,g: \mathbb{R}\rightarrow \mathbb{R} $  such that 

$f(x^{3}+2y)+f(x+y)=g(x+2y),\forall x,y\in \mathbb{R}$
	\flushright \href{https://artofproblemsolving.com/community/c6h618335}{(Link to AoPS)}
\end{problem}



\begin{solution}[by \href{https://artofproblemsolving.com/community/user/29428}{pco}]
	\begin{tcolorbox}Find all $ f,g: \mathbb{R}\rightarrow \mathbb{R} $  such that 

$f(x^{3}+2y)+f(x+y)=g(x+2y),\forall x,y\in \mathbb{R}$\end{tcolorbox}
Let $P(x,y)$ be the assertion $f(x^3+2y)+f(x+y)=g(x+2y)$
$P(x,0)$ $\implies$ $g(x)=f(x^3)+f(x)$ and $P(x,y)$ implies new assertion $Q(x,y)$ : $f(x^3+2y)+f(x+y)=f((x+2y)^3)+f(x+2y)$
$Q(0,x)$ $\implies$ $f((2x)^3)=f(x)$ and so $Q(x,y)$ may be written as $R(x,y)$ : $f(x^3+2y)+f(x+y)=f(\frac x2+y)+f(x+2y)$
$R(1,x-\frac 12)$ $\implies$ $f(x+\frac 12)=f(x)$ and so $f(x)$ is periodic
Subtracting $R(x,y)$ from $R(x+1,y)$, we get then $f((x+1)^3+2y)=f(x^3+2y)$

Let then $u-v\ge \frac 14$ :
Let $x$ any real root of $(x+1)^3-x^3=u-v$ and $y=\frac {v-x^3}2$ : last equality means $f(u)=f(v)$

And so $f(x)=f(y)$ $\forall x\ge y+\frac 14$ and it is then elementary to conclude that $f(x)$ is constant.

And so  ${\boxed{f(x)=c\text{  and  }g(x)=2c}\text{  }\forall x}$ which indeed is a solution, whatever is $c\in\mathbb R$
\end{solution}



\begin{solution}[by \href{https://artofproblemsolving.com/community/user/292523}{Mnjr}]
	Very nice solution  :oops: 
\end{solution}
*******************************************************************************
-------------------------------------------------------------------------------

\begin{problem}[Posted by \href{https://artofproblemsolving.com/community/user/33274}{toanIneq}]
	Find all functions $ f,g,h: \mathbb{R}\rightarrow \mathbb{R} $  such that

$f\left ( x+y^{3} \right )+g\left ( x^{3} +y\right )=h(xy),\forall x,y\in \mathbb{R}$
	\flushright \href{https://artofproblemsolving.com/community/c6h618338}{(Link to AoPS)}
\end{problem}



\begin{solution}[by \href{https://artofproblemsolving.com/community/user/29428}{pco}]
	\begin{tcolorbox}Find all functions $ f,g,h: \mathbb{R}\rightarrow \mathbb{R} $  such that

$f\left ( x+y^{3} \right )+g\left ( x^{3} +y\right )=h(xy),\forall x,y\in \mathbb{R}$\end{tcolorbox}
If $f,g,h$ are solution, so are $f(x)-f(0),g(x)-g(0), h(x)-f(0)-g(0)$. So WLOG consider $f(0)=g(0)=0$

Swapping $x,y$ and subtracting from original equation, we get $(f-g)(x+y^3)=(f-g)(x^3+y)$ and so easily $f-g$ constant and so $g(x)=f(x)$ 
Hence assertion $P(x,y)$ : $f(x+y^3)+f(x^3+y)=h(xy)$ 

$P(x,yz)$ $\implies$ $f(x+y^3z^3)+f(x^3+yz)=h(xyz)$ 
$P(y,xz)$ $\implies$ $f(y+x^3z^3)+f(y^3+xz)=h(xyz)$ 
Subtracting, we get $f(x+y^3z^3)+f(x^3+yz)=f(y+x^3z^3)+f(y^3+xz)$
Consider now $v\ne 0$ and $u\ne \pm v$ and $u\ne v^9$ and the system :
$x^3+yz=y+x^3z^3$
$x+y^3z^3=u$
$y^3+xz=v$

$u\ne \pm v$ implies $z\ne \pm 1$ and cancelling $y^3$ between second and third gives $x=\frac{vz^3-u}{z^4-1}$
First lines gives then $y=x^3(z^2+z+1)=\frac{(vz^3-u)^3(z^2+z+1)}{(z^4-1)^3}$

Plugging these values in third line gives $(vz^3-u)^9(z^2+z+1)^3+z(vz^3-u)(z^4-1)^8-v(z^4-1)^9=0$
It's easy to see that greatest degree summand is $(v^9-u)z^{33}$ and that $z=\pm 1$ is not a root
So $\exists x,y,z\ne \pm 1$ matching the system

And so $f(u)=f(v)$ $\forall v\ne 0$ and $u\ne \pm v$ and $u\ne v^9$
Easy step to get that $f(x)$ is constant, and so is $g(x)$ and so is $h(x)$

Hence the answer : $\boxed{f(x)=a\text{ and }g(x)=b\text{ and }h(x)=a+b\text{  }\forall x}$ which indeed is a solution, whatever are $a,b,c\in\mathbb R$
\end{solution}
*******************************************************************************
-------------------------------------------------------------------------------

\begin{problem}[Posted by \href{https://artofproblemsolving.com/community/user/233244}{Illusory}]
	Given $ k \in \mathbb{R} $. Find all $ f: \mathbb{R} \longrightarrow \mathbb{R} $ 
satisfy $ f(yf(x+y)+f(x))=k^2x+kyf(y) $
	\flushright \href{https://artofproblemsolving.com/community/c6h618365}{(Link to AoPS)}
\end{problem}



\begin{solution}[by \href{https://artofproblemsolving.com/community/user/212018}{Tintarn}]
	I have a solution for all cases except $k=0$.
First case: $k=1$. Then the equation writes as:
$f(yf(x+y)+f(x))=x+yf(y)$
$y=0$ implies $f(f(x))=x$ hence $f$ is an involution and therefore a bijection.
Thus, there exists unique $a$ such that $f(a)=0$ and $f(0)=a$.
Inserting $y=a$ in the original equation implies $f(af(x+a)+f(x))=x=f(f(x))$ and hence because of injectivity $af(x+a)=0$ and therefore $a=0$ as $f(x)=0$ for every $x$ is not a solution.
Now, $x=0, y=1$ in the original equation implies $1=f(f(1))=f(1)$ and hence $f(1)=1$.
Finally, $y=1$ in the original equation yields $f(f(x+1)+f(x))=x+1=f(f(x+1))$ and by injectivity $f(x)=0$ for every $x$ but this is obviously not a solution. Hence, in this case no solution exists.
The case $k=-1$ works similarly.

Let us from now on assume $k \ne 0$ and $k^2 \ne 1$.
Then $y=0$ implies $f(f(x))=k^2x$ and therefore $f(k^2x)=k^2f(x)$ and hence $f(0)=0$.
Now, $y=-x$ in the original equation yields $k^2x=f(f(x))=k^2x-kxf(-x)$ and hence $f(x)=0$ for every $x$ which is obviously no solution.

Hence, for $k \ne 0$ the equation has no solution.
In case of $k=0$ the equation writes as $f(yf(x+y)+f(x))=0$ which immediately yields $f(f(x))=0$ and $f(0)=0$. Obviously, $f(x) \equiv 0$ is a solution but I could not yet prove that it is the only one.
\end{solution}



\begin{solution}[by \href{https://artofproblemsolving.com/community/user/233244}{Illusory}]
	Thanks for your interest
\end{solution}



\begin{solution}[by \href{https://artofproblemsolving.com/community/user/29428}{pco}]
	\begin{tcolorbox}In case of $k=0$ the equation writes as $f(yf(x+y)+f(x))=0$ which immediately yields $f(f(x))=0$ and $f(0)=0$. Obviously, $f(x) \equiv 0$ is a solution but I could not yet prove that it is the only one.\end{tcolorbox}
In this case, a lot of non allzero solutions exist.

For example, choose $f(x)=\frac 121_I(x)$ where $I=\{2n+1\text{  }\forall n\in\mathbb Z\}$
\end{solution}
*******************************************************************************
-------------------------------------------------------------------------------

\begin{problem}[Posted by \href{https://artofproblemsolving.com/community/user/68025}{Pirkuliyev Rovsen}]
	Does there exist a function $f(x)$ which is defined for all reals and for which the identities $f(f(x))=x$ and $f(f(x)+1)=1-x$ hold?
	\flushright \href{https://artofproblemsolving.com/community/c6h618394}{(Link to AoPS)}
\end{problem}



\begin{solution}[by \href{https://artofproblemsolving.com/community/user/219179}{huricane}]
	Such function does not exist.Suppose that there exists such function.Then $f(f(x))+f(f(x)+1)=x+(1-x)=1,\forall x\in\mathbb{R}(1)$.Plugging $x=f(t),t\in\mathbb{R}$ in $(1)$ we obtain $f(f(f(t)))+f(f(f(t))+1)=f(t)+f(t+1)=1$.
Therefore $f(t)+f(t+1)=1 \forall t\in\mathbb{R}$.From $f(t)+f(t+1)=f(t+1)+f(t+2)=1$ we get $f(t)=f(t+2)$,which means $t=f(f(t))=f(f(t+2))=t+2 \forall t\in\mathbb{R}$,contradiction.
\end{solution}



\begin{solution}[by \href{https://artofproblemsolving.com/community/user/29428}{pco}]
	\begin{tcolorbox}Does there exist a function $f(x)$ which is defined for all reals and for which the identities $f(f(x))=x$ and $f(f(x)+1)=1-x$ hold?\end{tcolorbox}
$f(x)$ is injective.
Setting $x=\frac 12$ in both equalities, we get $f(f(\frac 12))=f(f(\frac 12)+1)=\frac 12$ and so, since injective, $f(\frac 12)=f(\frac 12)+1$

Hence the result : no such function.
\end{solution}
*******************************************************************************
-------------------------------------------------------------------------------

\begin{problem}[Posted by \href{https://artofproblemsolving.com/community/user/78787}{giaobui}]
	Find all functions $f:\mathbb{R}^{+}\rightarrow \mathbb{R}^{+}  $ such that

$f(xy)f(x+y)=1, \forall x,y> 0$
	\flushright \href{https://artofproblemsolving.com/community/c6h618467}{(Link to AoPS)}
\end{problem}



\begin{solution}[by \href{https://artofproblemsolving.com/community/user/29428}{pco}]
	\begin{tcolorbox}Find all functions $f:\mathbb{R}^{+}\rightarrow \mathbb{R}^{+}  $ such that

$f(xy)f(x+y)=1, \forall x,y> 0$\end{tcolorbox}
Let $P(x,y)$ be the assertion $f(xy)f(x+y)=1$

Let $x,y>0$ and $z> 2\max(\sqrt x,\sqrt y)$
Let $u_1,v_1$ the two distinct positive real roots of $X^2-zX+x=0$ : $P(u_1,v_1)$ $\implies$ $f(x)f(z)=1$
Let $u_2,v_2$ the two distinct positive real roots of $X^2-zX+y=0$ : $P(u_2,v_2)$ $\implies$ $f(y)f(z)=1$

And so $f(x)=f(y)$ $\forall x,y$ and $f(x)$ is constant and so $\boxed{f(x)=1\text{  }\forall x>0}$ which indeed is a solution.
\end{solution}
*******************************************************************************
-------------------------------------------------------------------------------

\begin{problem}[Posted by \href{https://artofproblemsolving.com/community/user/78787}{giaobui}]
	Find all the continuous functions $f : \mathbb{R} \to \mathbb{R}$ such that  $f\left ( \frac{x+y}{2} \right )^{2}=f(x)f(y)$, for all $x,y \in \mathbb{R}$.
	\flushright \href{https://artofproblemsolving.com/community/c6h618471}{(Link to AoPS)}
\end{problem}



\begin{solution}[by \href{https://artofproblemsolving.com/community/user/29428}{pco}]
	\begin{tcolorbox}Find all the continuous functions $f : \mathbb{R} \to \mathbb{R}$ such that  $f\left ( \frac{x+y}{2} \right )^{2}=f(x)f(y)$, for all $x,y \in \mathbb{R}$.\end{tcolorbox}
If $f(u)=0$ for some $u$, then choosing $x=2t-u$ and $y=u$ implies $f(t)=0$ $\forall t$

If $f(x)\ne 0$ $\forall x$, then continuity implies that $f(x)$ has constant sign and since $f(x)$ solution implies $-f(x)$ solution, WLOG say $f(x)>0$ $\forall x$

Let then $g(x)=\ln f(x)$, continuous, and we get $g(\frac{x+y}2)=\frac{g(x)+g(y)}2$ and so (very classical) $g(x)=ax+b$

And so $\boxed{f(x)=c e^{ax}\text{  }\forall x}$ which indeed is a solution, whatever are $a,c\in\mathbb R$
Note that $c=e^b>0$ from previous lines but we can choose also $c=0$ (first solution) or $=-e^b<0$ since $f(x)$ solution implies $-f(x)$ solution.
\end{solution}
*******************************************************************************
-------------------------------------------------------------------------------

\begin{problem}[Posted by \href{https://artofproblemsolving.com/community/user/33274}{toanIneq}]
	Find all the continuous functions $f : \mathbb{R} \mapsto\mathbb{R}$ such that  

 $  f(x^{2}f\left ( x \right )+f\left ( y \right ))=\left ( f\left ( x \right ) \right )^{3}+y $                                                                                                     , $\forall x,y \in \mathbb{R}$
	\flushright \href{https://artofproblemsolving.com/community/c6h618473}{(Link to AoPS)}
\end{problem}



\begin{solution}[by \href{https://artofproblemsolving.com/community/user/29428}{pco}]
	\begin{tcolorbox}Find all the continuous functions $f : \mathbb{R} \mapsto\mathbb{R}$ such that  

 $  f(x^{2}f\left ( x \right )+f\left ( y \right ))=\left ( f\left ( x \right ) \right )^{3}+y $                                                                                                     , $\forall x,y \in \mathbb{R}$\end{tcolorbox}
Let $P(x,y)$ be the assertion $f(x^2f(x)+f(y))=f(x)^3+y$
Let $a=f(0)$

$P(0,x)$ $\implies$ $f(f(x))=x+a^3$ and so $f(x)$ is continuous bijective (and so monotonous) and so $x^2f(x)$ is surjective.
Let $b=f^{-1}(0)$ : $P(x,b)$ $\implies$ $f(x^2f(x))=f(x)^3+b$

So $f(x^2f(x)+f(y))=f(x^2f(x))+f(f(y))-(a^3+b)$ And so, since both $f(x)$ and $x^2f(x)$ are surjective : $f(x+y)=f(x)+f(y)-(a^3+b)$

Continuity implies then $f(x)=ux+v$ for some $u,v$ and plugging this in original equation gives $u=1$ and $v=0$

And so $\boxed{f(x)=x\text{  }\forall x}$ which indeed is a solution.
\end{solution}
*******************************************************************************
-------------------------------------------------------------------------------

\begin{problem}[Posted by \href{https://artofproblemsolving.com/community/user/33274}{toanIneq}]
	Find all the continuous functions$ f:\mathbb{R}^{+}\rightarrow \mathbb{R}^{+}$ such  that

$f\left ( \frac{x+y}{2} \right )+f\left ( \frac{2xy}{x+y} \right ) )=f(x)+f(y),\forall x,y> 0$
	\flushright \href{https://artofproblemsolving.com/community/c6h618477}{(Link to AoPS)}
\end{problem}



\begin{solution}[by \href{https://artofproblemsolving.com/community/user/29428}{pco}]
	\begin{tcolorbox}Find all the continuous functions$ f:\mathbb{R}^{+}\rightarrow \mathbb{R}^{+}$ such  that

$f\left ( \frac{x+y}{2} \right )+f\left ( \frac{2xy}{x+y} \right ) )=f(x)+f(y),\forall x,y> 0$\end{tcolorbox}
Let $x>y$, and the sequences $x_1=x$, $x_{n+1}=\frac{x_n+y_n}2$, $y_1=y$, $y_{n+1}=\frac {2x_ny_n}{x_n+y_n}$

It's immediate to get that $x_n>x_{n+1}>y_{n+1}>y_n$ so that both sequences are convergent and have the same limit.
And since $x_{n+1}y_{n+1}=x_ny_n=xy$, this limit is ${\sqrt {xy}}$

Now, $f(x_{n+1})+f(y_{n+1})=f(x_n)+f(y_n)=...=f(x)+f(y)$ and continuity implies $f(x)+f(y)=2f(\sqrt{xy})$

Setting $g(x)=f(e^x)$, we get $g(\frac{x+y}2)=\frac{g(x)+g(y)}2$, whose only continuous solutions are classicaly $g(x)=ax+b$

So $f(x)=a\ln x + b$ and so, since positive, $a=0$ and $b>0$ and $\boxed{f(x)=b\text{  }\forall x}$ which indeed is a solution, whatever is $b>0$
\end{solution}



\begin{solution}[by \href{https://artofproblemsolving.com/community/user/211185}{toto1234567890}]
	With no continuity we get this.
$ f(x)+f(y)=2f(\sqrt{xy}) $
\end{solution}



\begin{solution}[by \href{https://artofproblemsolving.com/community/user/29428}{pco}]
	\begin{tcolorbox}With no continuity we get this.
$ f(x)+f(y)=2f(\sqrt{xy}) $\end{tcolorbox}
Could you show me how you get this without continuity ?
I personnaly used continuity for this (since I took the limit). I'm interested in the method you used without continuity.
Thanks in advance.
\end{solution}



\begin{solution}[by \href{https://artofproblemsolving.com/community/user/211185}{toto1234567890}]
	Take x,y,z,w and apply the condition in $ f(x)+f(y)+f(z)+f(w) $ two times like $ (x,y)(z,w) ,(x,z)(y,w) $. Then let $ y=w=\sqrt{xz} $ you will get 
$ \forall x\in (\frac{1}{4}y,4y): f(x)+f(y)=2f(\sqrt{xy}) $. So, done. :P
\end{solution}



\begin{solution}[by \href{https://artofproblemsolving.com/community/user/29428}{pco}]
	\begin{tcolorbox}Take x,y,z,w and apply the condition in $ f(x)+f(y)+f(z)+f(w) $ two times like $ (x,y)(z,w) ,(x,z)(y,w) $. Then let $ y=w=\sqrt{xz} $ you will get 
$ \forall x\in (\frac{1}{4}y,4y): f(x)+f(y)=2f(\sqrt{xy}) $. So, done. :P\end{tcolorbox}
I'm really sorry but I dont succeed :

$(x,y)(z,w)$ gives $f(x)+f(y)+f(z)+f(w)$ $=f(\frac{x+y}2)+f(\frac{2xy}{x+y})+f(\frac{z+w}2)+f(\frac{2zw}{z+w})$
$(x,z)(y,w)$ gives $f(x)+f(y)+f(z)+f(w)$ $=f(\frac{x+z}2)+f(\frac{2xz}{x+z})+f(\frac{y+w}2)+f(\frac{2yw}{y+w})$

Setting  $y=w=\sqrt{xz}$, this becomes :
$f(x)+f(z)+2f(\sqrt{xz})$ $=f(\frac{x+\sqrt{xz}}2)+f(\frac{2x\sqrt{xz}}{x+\sqrt{xz}})+f(\frac{z+\sqrt{xz}}2)+f(\frac{2z\sqrt{xz}}{z+\sqrt{xz}})$
$f(x)+f(z)+2f(\sqrt{xz})$ $=f(\frac{x+z}2)+f(\frac{2xz}{x+z})+2f(\sqrt{xz})$

Hemmmm, and now ?
\end{solution}



\begin{solution}[by \href{https://artofproblemsolving.com/community/user/211185}{toto1234567890}]
	Oh, I didn't say the essential thing. :P
In $ f(x)+f(y)+f(z)+f(w) =f(\frac{x+y}{2})+f(\frac{2xy}{x+y})+f(\frac{z+w}{2})+f(\frac{2zw}{z+w}) $ apply $ (\frac{x+y}{2},\frac{z+w}{2}) $ and $ 
(\frac{2xy}{x+y},\frac{2zw}{z+w}) $ .
And the same thing in  $ f(x)+f(y)+f(z)+f(w)=f(\frac{x+z}{2})+f(\frac{2xz}{x+z})+f(\frac{y+w}{2})+f(\frac{2yw}{y+w}) $ then we put in $ y=w=\sqrt{xz} $ then, we get the desired result $ \forall x\in (\frac{1}{4}y,4y): f(x)+f(y)=2f(\sqrt{xy}) $ :) A good problem.
\end{solution}



\begin{solution}[by \href{https://artofproblemsolving.com/community/user/29428}{pco}]
	Hmmmmm, maybe.
I clearly consider my own solution far simpler.
\end{solution}



\begin{solution}[by \href{https://artofproblemsolving.com/community/user/211185}{toto1234567890}]
	Well, I just meant that we don't need continuity for $ f(x)+f(y)=2f(\sqrt{xy}) $ .  :)
\end{solution}
*******************************************************************************
-------------------------------------------------------------------------------

\begin{problem}[Posted by \href{https://artofproblemsolving.com/community/user/33274}{toanIneq}]
	Find all the continuous functions$  f:\left [ 0;1 \right ]\rightarrow \mathbb{R}    $ such  that

$f(x)=\frac{1}{2}\left ( f\left ( \frac{x}{2} \right )+f\left ( \frac{1+x}{2} \right ) \right ),\forall x\in \left [ 0;1 \right ]$
	\flushright \href{https://artofproblemsolving.com/community/c6h618478}{(Link to AoPS)}
\end{problem}



\begin{solution}[by \href{https://artofproblemsolving.com/community/user/29428}{pco}]
	\begin{tcolorbox}Find all the continuous functions$  f:\left [ 0;1 \right ]\rightarrow \mathbb{R}    $ such  that

$f(x)=\frac{1}{2}\left ( f\left ( \frac{x}{2} \right )+f\left ( \frac{1+x}{2} \right ) \right ),\forall x\in \left [ 0;1 \right ]$\end{tcolorbox}
Let $P(x)$ be the assertion $f(x)=\frac 12\left(f(\frac x2)+f(\frac {1+x}2)\right)$

$f(x)$ is continuous an so bounded over $[0,1]$. Let then $M=\max_{x\in[0,1]}f(x)$ and $m=\min_{x\in[0,1]}f(x)$

Let $u\in[0,1]$ such that $f(u)=M$. $P(u)$ implies $f(\frac u2)=f(\frac{1+u}2)=M$ and continuity implies $f(0)=f(1)=M$

Let $v\in[0,1]$ such that $f(v)=m$. $P(v)$ implies $f(\frac v2)=f(\frac{1+v}2)=m$ and continuity implies $f(0)=f(1)=m$

And so $m=M=c$ and $\boxed{f(x)=c\text{  }\forall x\in[0,1]}$ which indeed is a solution whatever is $c\in\mathbb R$
\end{solution}
*******************************************************************************
-------------------------------------------------------------------------------

\begin{problem}[Posted by \href{https://artofproblemsolving.com/community/user/33274}{toanIneq}]
	Find all the continuous functions$  f: [ 0;1  ]\rightarrow \mathbb{R}    $ such  that

$f ( x  )\geq 2xf ( x^{2}) ,\forall x\in  [ 0;1  ]$
	\flushright \href{https://artofproblemsolving.com/community/c6h618479}{(Link to AoPS)}
\end{problem}



\begin{solution}[by \href{https://artofproblemsolving.com/community/user/29428}{pco}]
	\begin{tcolorbox}Find all the continuous functions$  f: [ 0;1  ]\rightarrow \mathbb{R}    $ such  that

$f ( x  )\geq 2xf ( x^{2}) ,\forall x\in  [ 0;1  ]$\end{tcolorbox}
Setting $x=0$ in equation, we get $f(0)\ge 0$
Setting $x=1$ in equation, we get $f(1)\le 0$

If $f(x)>0$ for some $\in(0,1)$, then consider the increasing sequence $x_n=x^{2^{-n}}$ :
$f(x_{n+1})\ge 2x_{n+1}f(x_n)$ and so $f(x_n)>0$ $\forall n$
Since $x_n$ is increasing with limit $1$, $x_{n+1}>\frac 12$ from a given point and then $f(x_{n+1})>f(x_n)>0$ from this point.
Continuity implies then $f(1)>0$, impossible. 
So $f(x)\le 0$ $\forall x\in(0,1)$ and so $f(x)\le 0$ $\forall x\in[0,1]$ and so $f(0)=0$

If $f(x)<0$ for some $x\in(0,\frac 12)$,  then consider the decreasing sequence $x_n=x^{2^{n}}<\frac 12$ :
$f(x_n)\ge 2x_nf(x_{n+1})>f(x_{n+1})$  and so $f(x_n)$ is a negative decreasing sequence and continuity im^plies $f(0)<0$, impossible.

So $f(x)=0$ $\forall x\in[0,\frac 12)$
And from there, it's easy to get $\boxed{f(x)=0\text{  }\forall x\in[0,1]}$ which indeed is a solution
\end{solution}
*******************************************************************************
-------------------------------------------------------------------------------

\begin{problem}[Posted by \href{https://artofproblemsolving.com/community/user/33274}{toanIneq}]
	Find all the continuous functions $f:\mathbb{R}^{+}\rightarrow \mathbb{R}^{+} $ such that  

$f\left (\frac{1}{f\left ( xy \right )}  \right )=f(x)f(y),\forall x,y> 0$
	\flushright \href{https://artofproblemsolving.com/community/c6h618481}{(Link to AoPS)}
\end{problem}



\begin{solution}[by \href{https://artofproblemsolving.com/community/user/209597}{Element118}]
	Did you mean $ f :\mathbb{R}^+\mapsto\mathbb{R}^+ $?
\end{solution}



\begin{solution}[by \href{https://artofproblemsolving.com/community/user/29428}{pco}]
	\begin{tcolorbox}Find all the continuous functions $f:\mathbb{R}^{+}\rightarrow \mathbb{R}^{+} $ such that  

$f\left (\frac{1}{f\left ( xy \right )}  \right )=f(x)f(y),\forall x,y> 0$\end{tcolorbox}
Let $g(x)=\frac 1{f(x)}$, continuous from $\mathbb R^+\to\mathbb R^+$. We get $g(g(xy))=g(x)g(y)$
As a consequence $g(g(xy))=g(xy)g(1)$ and so $g(1)g(xy)=g(x)g(y)$ which immediately gives, with continuity, $g(x)=c x^a$
Plugging this in original equation, we get two possibilities : $g(x)=1$ and $g(x)=x$ and so two solutions 

$\boxed{\text{S1 : }f(x)=1\text{  }\forall x}$ and $\boxed{\text{S2 : }f(x)=\frac 1x\text{  }\forall x}$
\end{solution}
*******************************************************************************
-------------------------------------------------------------------------------

\begin{problem}[Posted by \href{https://artofproblemsolving.com/community/user/33274}{toanIneq}]
	Find all the continuous functions $f : \mathbb{R} \mapsto\mathbb{R}$ such that  

$xf(x)-yf(y)=(x-y)f(x+y),\forall x,y\in \mathbb{R}$
	\flushright \href{https://artofproblemsolving.com/community/c6h618483}{(Link to AoPS)}
\end{problem}



\begin{solution}[by \href{https://artofproblemsolving.com/community/user/29428}{pco}]
	\begin{tcolorbox}Find all the continuous functions $f : \mathbb{R} \mapsto\mathbb{R}$ such that  

$xf(x)-yf(y)=(x-y)f(x+y),\forall x,y\in \mathbb{R}$\end{tcolorbox}
Posted a lot of times. Maybe you could walk a little thru the forum before flooding with old problems ...

Let $P(x,y)$ be the assertion $xf(x)-yf(y)=(x-y)f(x+y)$

$P(\frac{x+1}2,\frac{x-1}2)$ $\implies$ $\frac{x+1}2f(\frac{x+1}2)-\frac{x-1}2f(\frac{x-1}2)=f(x)$

$P(\frac{x-1}2,\frac{1-x}2)$ $\implies$ $\frac{x-1}2f(\frac{x-1}2)-\frac{1-x}2f(\frac{1-x}2)=(x-1)f(0)$

$P(\frac{1-x}2,\frac{x+1}2)$ $\implies$ $\frac{1-x}2f(\frac{1-x}2-\frac{x+1}2f(\frac{x+1}2)=-xf(1)$

Adding these three lines, we get $f(x)=x(f(1)-f(0))+f(0)$ and so $\boxed{f(x)=ax+b\text{  }\forall x}$ which indeed is a solution, whatever are $a,b\in\mathbb R$
And, btw, no need for continuity here.
\end{solution}
*******************************************************************************
-------------------------------------------------------------------------------

\begin{problem}[Posted by \href{https://artofproblemsolving.com/community/user/33274}{toanIneq}]
	Find all the  continuous functions $f:\left [ 0;1 \right ]\rightarrow \mathbb{R}  $   that satisfies following conditions.

a) $f(0)=f(1)=0$

b)$f\left ( \frac{x+y}{2} \right )\leq f(x)+f(y),\forall x,y\in \left [ 0;1 \right ]$
	\flushright \href{https://artofproblemsolving.com/community/c6h618502}{(Link to AoPS)}
\end{problem}



\begin{solution}[by \href{https://artofproblemsolving.com/community/user/29428}{pco}]
	\begin{tcolorbox}Find all the  continuous functions $f:\left [ 0;1 \right ]\rightarrow \mathbb{R}  $   that satisfies following conditions.

a) $f(0)=f(1)=0$

b)$f\left ( \frac{x+y}{2} \right )\leq f(x)+f(y),\forall x,y\in \left [ 0;1 \right ]$\end{tcolorbox}
Setting $y=0$, we get $f(x)\ge f(\frac x2)$ and so $f(x)\ge f(\frac x{2^n})$ and so, with continuity, $f(x)\ge f(0)=0$

If $f(x)=f(y)=0$ then we get $0\le f(\frac{x+y}2)\le 0$ and so $f(\frac {x+y}2)=0$

So, simple induction, starting with $f(0)=f(1)=0$ gives $f(\frac n{2^m})=0$ $\forall m\in\mathbb N$, $\forall n\in\{0,1,2,...,2^m\}$

And continuity implies then $\boxed{f(x)=0\text{  }\forall x\in[0,1]}$ which indeed is a solution.
\end{solution}
*******************************************************************************
-------------------------------------------------------------------------------

\begin{problem}[Posted by \href{https://artofproblemsolving.com/community/user/33274}{toanIneq}]
	Find all the continuous functions $ f: [ -1;1  ]\rightarrow  [ -1;1  ]  $ such that  
                                                                          
$f ( 2x^{2}  -1)=f ( x  ),\forall x\in  [ -1;1  ]$
	\flushright \href{https://artofproblemsolving.com/community/c6h618507}{(Link to AoPS)}
\end{problem}



\begin{solution}[by \href{https://artofproblemsolving.com/community/user/29428}{pco}]
	\begin{tcolorbox}Find all the continuous functions $ f: [ -1;1  ]\rightarrow  [ -1;1  ]  $ such that  
                                                                          
$f ( 2x^{2}  -1)=f ( x  ),\forall x\in  [ -1;1  ]$\end{tcolorbox}

So $f(\cos x)=f(\cos \frac x2)$ $\forall x$ and so $f(\cos x)=f(\cos\frac x{2^n})$ $\forall x$, $\forall n\in\mathbb N$

Setting $n\to+\infty$ and using continuity, we get $f(\cos x)=f(1)$ $\forall x$ and so $\boxed{f(x)=c\text{  }\forall x\in[-1,1]}$ which indeed is a solution, whatever is $c\in[-1,1]$
\end{solution}
*******************************************************************************
-------------------------------------------------------------------------------

\begin{problem}[Posted by \href{https://artofproblemsolving.com/community/user/33274}{toanIneq}]
	Find all function $f : \mathbb{R} \to \mathbb{R}$ such that 

$ f(x^{3}+2y)=f(y^{3}+2x),\forall x,y\in \mathbb{R}$
	\flushright \href{https://artofproblemsolving.com/community/c6h618513}{(Link to AoPS)}
\end{problem}



\begin{solution}[by \href{https://artofproblemsolving.com/community/user/29428}{pco}]
	\begin{tcolorbox}Find all function $f : \mathbb{R} \to \mathbb{R}$ such that 

$ f(x^{3}+2y)=f(y^{3}+2x),\forall x,y\in \mathbb{R}$\end{tcolorbox}
System $x^3+2y=u$ and $y^3+2x=v$ is equivalent to $y=\frac{u-x^3}2$ and $(u-x^3)^3+16x-8v=0$ and so always have at least a real solution.

So $f(u)=f(v)$ $\forall u,v$ and $\boxed{f(x)=c}$ $\forall x$ which indeed is a solution, whatever is $c\in\mathbb R$
\end{solution}
*******************************************************************************
-------------------------------------------------------------------------------

\begin{problem}[Posted by \href{https://artofproblemsolving.com/community/user/218396}{vi1lat}]
	Find value $a$ , where sum of real roots of equation $\frac{f(a)x^2 +1}{x^2+g(a)}=\sqrt{\frac{xg(a)-1}{f(a)-x}}$ was minimum value , and $f(a)=a^2-a\sqrt{20} +23$ , $g(a)=1,5a^2-a\sqrt{20}+24$
	\flushright \href{https://artofproblemsolving.com/community/c6h618688}{(Link to AoPS)}
\end{problem}



\begin{solution}[by \href{https://artofproblemsolving.com/community/user/218396}{vi1lat}]
	does anyone have solutions?
\end{solution}



\begin{solution}[by \href{https://artofproblemsolving.com/community/user/29428}{pco}]
	\begin{tcolorbox}does anyone have solutions?\end{tcolorbox}
Since you posted in this "proposed and own" category, this means that you have a solution and are just looking for another one.
Please, post your own solution, please.

Hereunder is mine

\begin{tcolorbox}Find value $a$ , where sum of real roots of equation $\frac{f(a)x^2 +1}{x^2+g(a)}=\sqrt{\frac{xg(a)-1}{f(a)-x}}$ was minimum value , and $f(a)=a^2-a\sqrt{20} +23$ , $g(a)=1,5a^2-a\sqrt{20}+24$\end{tcolorbox}
For easier writing, let $u=f(a)=a^2-a\sqrt{20}+23$ and $v=g(a)=\frac 32a^2-a\sqrt{20}+24$
1) some preliminary results
1.1 : $u=(a-\sqrt 5)^2+18\ge 18$
1.2 : $v=\frac 32(a-\frac{2\sqrt 5}3)^2+\frac{62}3\ge \frac{62}3$
1.3 : using previous results, we trivially have $uv\ge 3$ and so $\frac 1v<\frac 2v<\frac {2u}3<u$
1.4 : $4u < v^2$
$4v^2-16u=(3a^2-4a\sqrt{5}+48)^2-16(a^2-2a\sqrt{5}+23)$ $=9a^4-24\sqrt 5a^3+352a^2-352\sqrt 5a+1936$
$=(3a^2-4a\sqrt 5+44)^2+8a^2$ $>0$
Q.E.D.
1.5 : $2u^2 > 9v$
$4u^2-18v=4(a^2-2a\sqrt 5+23)^2-9(3a^2-4a\sqrt 5+48)$ $=4a^4-16\sqrt 5a^3+237a^2-332\sqrt 5a+1684$
$=(2a^2-4a\sqrt 5+37)^2+9(a-2\sqrt 5)^2+135$ $>0$
Q.E.D

2) solution
2.1 get an equivalent cubic
Equation is $\frac{ux^2+1}{x^2+v}=\sqrt{\frac{vx-1}{u-x}}$ and is only defined when $x\in[\frac 1v,u)$ (remember $uv>3>1$)

$\iff$ $x-\frac{ux^2+1}{x^2+v}=x-\sqrt{\frac{vx-1}{u-x}}$

$\iff$ $\frac{x^3-ux^2+vx-1}{x^2+v}$ $=\frac{x^2-\frac{vx-1}{u-x}}{x+\sqrt{\frac{vx-1}{u-x}}}$ $=-\frac{x^3-ux^2+vx-1}{(u-x)(x+\sqrt{\frac{vx-1}{u-x}})}$

$\iff$ $(x^3-ux^2+vx-1)\left(\frac 1{x^2+v}+\frac{1}{(u-x)(x+\sqrt{\frac{vx-1}{u-x}})}\right)=0$

$\iff$ $x\in[\frac 1v,u)$ and $x^3-ux^2+vx-1=0$ (other factor is $>0$)

2.2 Prove that equivalent cubic has no complex root and that all real roots are in the good interval
Let $P(x)=x^3-ux^2+vx-1$

$P(\frac 1v)=\frac 1{v^3}-\frac u{v^2}=\frac{1-uv}{v^3}$ $<0$ (see 1.3 above)

$P(\frac 2v)=\frac 8{v^3}-\frac{4u}{v^2}+1$ $=\frac 8{v^3}+\frac {v^2-4u}{v^2}$ $>0$ (see 1.4 above)

$P(\frac{2u}3)=\frac{8u^3}{27}-\frac{4u^3}9+\frac{2uv}3-1$ $=\frac{2u(9v-2u^2)}{27}-1$ $<0$ (see 1.5 above)

$P(u)=uv-1$ $>0$ (see 1.3 above)

So we have indeed three distinct real roots in the suitable interval and so the sum of all real roots is exactly the sum of all roots and  is $u$ whose minimum is $18$ when $\boxed{a=\sqrt 5}$ (see 1.1 above)
\end{solution}



\begin{solution}[by \href{https://artofproblemsolving.com/community/user/218396}{vi1lat}]
	Sorry, but I haven't solution.
\end{solution}



\begin{solution}[by \href{https://artofproblemsolving.com/community/user/29428}{pco}]
	\begin{tcolorbox}Sorry, but I haven't solution.\end{tcolorbox}
You are welcome.
Glad to have helped you.

And please, respect the usage of each forum. Posting in "proposed and own" means that you have a solution.
\end{solution}
*******************************************************************************
-------------------------------------------------------------------------------

\begin{problem}[Posted by \href{https://artofproblemsolving.com/community/user/195015}{Jul}]
	Find all function : $f:\mathbb{R}\rightarrow \mathbb{R}$ and such that :
\[f(xf(x)+y^2)=x^2+yf(y),\;\forall x,y\in \mathbb{R}\]
	\flushright \href{https://artofproblemsolving.com/community/c6h618752}{(Link to AoPS)}
\end{problem}



\begin{solution}[by \href{https://artofproblemsolving.com/community/user/29428}{pco}]
	\begin{tcolorbox}Find all function : $f:\mathbb{R}\rightarrow \mathbb{R}$ and such that :
\[f(xf(x)+y^2)=x^2+yf(y),\;\forall x,y\in \mathbb{R}\]\end{tcolorbox}
Let $P(x,y)$ be the assertion $f(xf(x)+y^2)=x^2+yf(y)$

$P(0,0)$ $\implies$ $f(0)=0$
$P(0,x)$ $\implies$ $f(x^2)=xf(x)$
$P(x,0)$ $\implies$ $f(xf(x))=x^2$
$P(xf(x),0)$ $\implies$ $f(xf(x)f(xf(x)))=x^2f(x)^2$ $\implies$ $f(x^3f(x))=x^2f(x)^2$
$P(x^2,0)$ $\implies$  $f(x^2f(x^2))=x^4$ $\implies$ $f(x^3f(x))=x^4$

And so $f(x)^2=x^2$ $\forall x\ne 0$, still true when $x=0$ and so $\forall x$ : either $f(x)=x$, either $f(x)=-x$
Suppose now that $\exists a,b\ne 0$ such that $f(a)=a$ and $f(b)=-b$
$P(a,b)$ $\implies$ $f(a^2+b^2)=a^2-b^2$ and so :
either $a^2+b^2=a^2-b^2$ and so $b=0$ impossible
either $-(a^2+b^2)=a^2-b^2$ and so $a=0$, impossible

And so :
either $\boxed{f(x)=x}$ $\forall x$ which indeed is a solution
either $\boxed{f(x)=-x}$ $\forall x$ which indeed is a solution
\end{solution}
*******************************************************************************
-------------------------------------------------------------------------------

\begin{problem}[Posted by \href{https://artofproblemsolving.com/community/user/68025}{Pirkuliyev Rovsen}]
	If $a{\in}Z$ is a given number, then determine all injective functions $f:  \mathbb{Z}\to\mathbb{Z}$ for which $f(f(x))=f(x)+a$ for all $x{\in}Z$.
	\flushright \href{https://artofproblemsolving.com/community/c6h618911}{(Link to AoPS)}
\end{problem}



\begin{solution}[by \href{https://artofproblemsolving.com/community/user/89198}{chaotic_iak}]
	EDIT: Failed attempt.

Fix an integer $n$. Let $(u_i)$ be the sequence $u_0 = n, u_{k+1} = f(u_k)$ for all nonnegative integer $k$. Let $x = u_{k-1}$ for a positive integer $k$, then $u_{k+1} = u_k + a$ from the condition. Also, $u_{k+2} = u_{k+1} + a$ (set $k \leftarrow k+1$), so $u_{k+2} - 2u_{k+1} + u_k = 0$. This gives the characteristic polynomial $\lambda^2 - 2\lambda + 1 = 0$, so $\lambda = 1, 1$, and so $u_k = A(1^k) + B(1^k \cdot k)$ for some $A,B$. So $u_k = A + Bk$. Since $u_0 = n$, we have $A = n$.

Substituting back to the original equation $u_2 = u_1 + a$, we have $(n+2B) = (n+B) + a$, so $B = a$ for any $n$. Thus $f(n) = u_1 = n+a$ for any $n$, which can be verified to work.
\end{solution}



\begin{solution}[by \href{https://artofproblemsolving.com/community/user/29428}{pco}]
	\begin{tcolorbox}Fix an integer $n$. Let $(u_i)$ be the sequence $u_0 = n, u_{k+1} = f(u_k)$ for all nonnegative integer $k$. Let $x = u_{k-1}$ for a positive integer $k$, then $u_{k+1} = u_k + a$ from the condition. Also, $u_{k+2} = u_{k+1} + a$ (set $k \leftarrow k+1$), so $u_{k+2} - 2u_{k+1} + u_k = 0$. This gives the characteristic polynomial $\lambda^2 - 2\lambda + 1 = 0$, so $\lambda = 1, 1$, and so $u_k = A(1^k) + B(1^k \cdot k)$ for some $A,B$. So $u_k = A + Bk$. Since $u_0 = n$, we have $A = n$.

Substituting back to the original equation $u_2 = u_1 + a$, we have $(n+2B) = (n+B) + a$, so $B = a$ for any $n$. Thus $f(n) = u_1 = n+a$ for any $n$, which can be verified to work.\end{tcolorbox}
You got $u_{k+1}=u_k+a$ $\forall k\ge 1$ (and NOT for $k=0$) and so your conclusion is not available for $n=0$ and all you got is $f^{[n]}(x)=f(x)+(n-1)a$ $\forall n\ge 1$. That's all.
\end{solution}



\begin{solution}[by \href{https://artofproblemsolving.com/community/user/89198}{chaotic_iak}]
	Oops, that's true. And that statement is also obvious by induction.

Back to the drawing board...
\end{solution}



\begin{solution}[by \href{https://artofproblemsolving.com/community/user/29428}{pco}]
	\begin{tcolorbox}If $a{\in}Z$ is a given number, then determine all injective functions $f:  \mathbb{Z}\to\mathbb{Z}$ for which $f(f(x))=f(x)+a$ for all $x{\in}Z$.\end{tcolorbox}
Let $P(x)$ be the assertion $f(f(x))=f(x)+a$

1) If $a=0$
==========
Equation is $f(f(x))=f(x)$ $\forall x\in \mathbb Z$ and so, since injective, $\boxed{f(x)=x}$ $\forall x\in\mathbb Z$, which indeed is a solution

2) If $a\ne 0$
==========
Let $A=\{x\in\mathbb Z$ such that $f(x)=x+a\}$
$f(\mathbb Z)\in A$
If $x\in A$, then $P(x)$ $\implies$ $x+a\in A$

Let $B=\{x\in A$ such that $x-a\notin A\}$
$|B|\le |a|$
If $x\notin A$, $f(x)\in B$ and so at most $|a|$ such $x$
But $x\notin A$ $\implies$ $f(x)-ka\notin A$ $\forall k\in\mathbb Z$ and so contradiction

So $|B|=0$ and so $x\in A$ $\forall x$ and so $\boxed{f(x)=x+a}$ $\forall x\in\mathbb Z$, which indeed is a solution.
\end{solution}
*******************************************************************************
-------------------------------------------------------------------------------

\begin{problem}[Posted by \href{https://artofproblemsolving.com/community/user/195015}{Jul}]
	Find all function $f:\mathbb{R}\rightarrow \mathbb{R}$ and such that 
\[f(x^3+2f(y))=f^3(x)+y+f(y),\;\forall x,y\in \mathbb{R}\]
	\flushright \href{https://artofproblemsolving.com/community/c6h618937}{(Link to AoPS)}
\end{problem}



\begin{solution}[by \href{https://artofproblemsolving.com/community/user/233560}{alisherianvar}]
	\[Easy\: \: \: \: to\: \: \: \: prove \: \: \: \: that\: \: \: \: \: the \: \: \: \: \: function \: \: \: is \: \: \: injective\]
\end{solution}



\begin{solution}[by \href{https://artofproblemsolving.com/community/user/29428}{pco}]
	\begin{tcolorbox}Find all function $f:\mathbb{R}\rightarrow \mathbb{R}$ and such that 
\[f(x^3+2f(y))=f^3(x)+y+f(y),\;\forall x,y\in \mathbb{R}\]\end{tcolorbox}
Let $P(x,y)$ be the assertion $f(x^3+2f(y))=f(x)^3+y+f(y)$
$f(x)$ is injective
$P(\sqrt[3]{-x^3-2f(-x^3)},-x^3)$ $\implies$ $f(\sqrt[3]{-x^3-2f(-x^3)})=x$ and so $f(x)$ is surjective.
Let then $a=f(0)$ and $u$ such that $f(u)=0$

$P(0,y)$ $\implies$ $f(2f(y))=a^3+y+f(y)$
$P(x,u)$ $\implies$ $f(x^3)=f(x)^3+u$
$P(0,u)$ $\implies$ $a=a^3+u$
And so $f(x^3+2f(y))=f(x^3)+f(2f(y)-a$ and so, since surjective, $f(x+y)=f(x)+f(y)-a$ and so $f(x)-a$ is additive.

$P(x+p,u)$ $\implies$ $f(x^3+3px^2+3p^2x+p^3)$ $=(f(x)+pf(1)-a)^3+u$
$\implies$ $p^3(f(1)^3-f(1)+a)$ $+3p^2(f(1)^2(f(x)-a)-f(x)+a)$ $+3p(f(1)(f(x)-a)^2-f(x^2)+a)$ $+(f(x)-a)^3+u-f(x^3)=0$
This a cubic which is zero for any rational and so each coefficient is zero.
So $f(1)(f(x)-a)^2-f(x^2)+a=0$ and so $f(x^2)=f(1)(f(x)-a)^2+a$  and so $f(x)$ is lower bounded over $\mathbb R^+$ and so is continuous an $f(x)=\alpha x+a$
Plugging this back in original equation, we get $\alpha=1$ and $\beta=0$ and so $\boxed{f(x)=x}$ $\forall x$
\end{solution}
*******************************************************************************
-------------------------------------------------------------------------------

\begin{problem}[Posted by \href{https://artofproblemsolving.com/community/user/32935}{soruz}]
	Determine all functions $f : \mathbb{N} \to \mathbb{N}$ such that \[ x^2+2f(xy)+y^2=f^2(x+y), \ \ \ \forall x,y\in \Bbb{N}.\]
	\flushright \href{https://artofproblemsolving.com/community/c6h619008}{(Link to AoPS)}
\end{problem}



\begin{solution}[by \href{https://artofproblemsolving.com/community/user/29428}{pco}]
	\begin{tcolorbox}Determine all functions $f : \mathbb{N} \to \mathbb{N}$ such that \[ x^2+2f(xy)+y^2=f^2(x+y), \ \ \ \forall x,y\in \Bbb{N}.\]\end{tcolorbox}
Let $P(x,y)$ be the assertion $x^2+2f(xy)+y^2=f(x+y)^2$

If $f(x)=x$ for some $x$, then $P(x,1)$ $\implies$ $f(x+1)=x+1$
$P(2,2)$ $\implies$ $f(4)=4$ and so $f(x)=x$ $\forall x\ge 4$
$P(3,1)$ $\implies$ $f(3)=3$
$P(2,1)$ $\implies$ $f(2)=2$
$P(1,1)$ $\implies$ $f(1)=1$
And so $\boxed{f(x)=x}$ $\forall x\in\mathbb N$, which indeed is a solution.
\end{solution}



\begin{solution}[by \href{https://artofproblemsolving.com/community/user/233377}{souheibsolver}]
	let $ x=0 $ and $ y=0 $
so $ f(0)^2=2f(0) \Rightarrow  $   $ f(0)=0 $  or   $ f(0)=2 $
if $ f(0)=2 $ 
$ P(x,0) \Rightarrow f(x)^2=x^2+4 \Rightarrow f(x)=\sqrt{x^2+4 } $ which doesn't always give natural values


so $ f(0)=0 $
$ P (x,0) \Rightarrow f(x)^2=x^2 \Rightarrow f(x)=x $

so $ f(x)=x $        $ \forall x\in \mathbb{N} $
\end{solution}



\begin{solution}[by \href{https://artofproblemsolving.com/community/user/65499}{PIRISH}]
	\begin{tcolorbox}let $ x=0 $ and $ y=0 $
so $ f(0)^2=2f(0) \Rightarrow  $   $ f(0)=0 $  or   $ f(0)=2 $
if $ f(0)=2 $ 
$ P(x,0) \Rightarrow f(x)^2=x^2+4 \Rightarrow f(x)=\sqrt{x^2+4 } $ which doesn't always give natural values


so $ f(0)=0 $
$ P (x,0) \Rightarrow f(x)^2=x^2 \Rightarrow f(x)=x $

so $ f(x)=x $        $ \forall x\in \mathbb{N} $\end{tcolorbox}
 0  is not natural number
\end{solution}



\begin{solution}[by \href{https://artofproblemsolving.com/community/user/65499}{PIRISH}]
	\begin{tcolorbox}[quote="soruz"]Determine all functions $f : \mathbb{N} \to \mathbb{N}$ such that \[ x^2+2f(xy)+y^2=f^2(x+y), \ \ \ \forall x,y\in \Bbb{N}.\]\end{tcolorbox}
Let $P(x,y)$ be the assertion $x^2+2f(xy)+y^2=f(x+y)^2$

If $f(x)=x$ for some $x$, then $P(x,1)$ $\implies$ $f(x+1)=x+1$
$P(2,2)$ $\implies$ $f(4)=4$ and so $f(x)=x$ $\forall x\ge 4$
$P(3,1)$ $\implies$ $f(3)=3$
$P(2,1)$ $\implies$ $f(2)=2$
$P(1,1)$ $\implies$ $f(1)=1$
And so $\boxed{f(x)=x}$ $\forall x\in\mathbb N$, which indeed is a solution.\end{tcolorbox}


If $f(x)=x$ for some $x$, then $P(x,1)$ $\implies$ $f(x+1)=x+1$
If $f(x+1)=x+1$ for some $x$, then $P(x,1)$ $\implies$ $f(x)=x$
$P(2,2)$ $\implies$ $f(4)=4$ and so $f(x)=x$ $\forall x\in\mathbb N$, which indeed is a solution.  :)
\end{solution}
*******************************************************************************
-------------------------------------------------------------------------------

\begin{problem}[Posted by \href{https://artofproblemsolving.com/community/user/68025}{Pirkuliyev Rovsen}]
	Determine all functions $f: \mathbb{R}\to\mathbb{R}$ for which $f(0)=1$ and which have the primitive function $F: \mathbb{R}\to\mathbb{R}$ such that $F(x)=f(x)+\sin^4{x}$ for all $x{\in}R$.
	\flushright \href{https://artofproblemsolving.com/community/c6h619021}{(Link to AoPS)}
\end{problem}



\begin{solution}[by \href{https://artofproblemsolving.com/community/user/29428}{pco}]
	\begin{tcolorbox}Determine all functions $f: \mathbb{R}\to\mathbb{R}$ for which $f(0)=1$ and which have the primitive function $F: \mathbb{R}\to\mathbb{R}$ such that $F(x)=f(x)+\sin^4{x}$ for all $x{\in}R$.\end{tcolorbox}
1) find a specific solution for $g(x)=g'(x)+\sin^4x$
$\sin^4(x)=\left(\frac{1-\cos 2x}2\right)^2$ $=\frac 14-\frac 12\cos 2x+\frac 14\cos^22x$ $=\frac 14-\frac 12\cos 2x+\frac 14\frac{1+\cos 4x}2$ $=\frac 38-\frac 12\cos 2x+\frac 18\cos 4x$ 

1.1) find a specific solution for $h_1(x)=h_1'(x)+1$
We easily get $h_1(x)=1$

1.2) find a specific solution for $h_2(x)=h_2'(x)+\cos 2x$
Setting $h_2(x)=a\sin 2x + b\cos 2x$, we get $h_2(x)=-\frac 25\sin 2x+\frac 15\cos 2x$

1.3) find a specific solution for $h_3(x)=h_3'(x)+\cos 4x$
Setting $h_3(x)=a\sin 4x + b\cos 4x$, we get $h_3(x)=-\frac 4{17}\sin 4x+\frac 1{17}\cos 4x$

1.4) find a specific solution for $g(x)=g'(x)+\sin^4x$
Obviously, choosing $g(x)=\frac 38h_1(x)-\frac 12h_2(x)+\frac 18h_3(x)$ is a solution.
So $g(x)=\frac 38+\frac 15\sin 2x-\frac 1{10}\cos 2x-\frac 1{34}\sin 4x+\frac 1{136}\cos 4x$

2) find all solutions for $F(x)=F'(x)+\sin^4x$
We trivially have then $F(x)=ke^x+\frac 38+\frac 15\sin 2x-\frac 1{10}\cos 2x-\frac 1{34}\sin 4x+\frac 1{136}\cos 4x$

And so $f(x)=F'(x)=ke^x+\frac 25\cos 2x+\frac 1{5}\sin 2x-\frac 2{17}\cos 4x-\frac 1{34}\sin 4x$

3) Solution to the given problem
Adding the condition $f(0)=1$, we get $k=\frac{61}{85}$

Hence the solution $\boxed{f(x)=\frac{61}{85}e^x+\frac 25\cos 2x+\frac 1{5}\sin 2x-\frac 2{17}\cos 4x-\frac 1{34}\sin 4x}$
\end{solution}
*******************************************************************************
-------------------------------------------------------------------------------

\begin{problem}[Posted by \href{https://artofproblemsolving.com/community/user/212909}{ARCH999}]
	Find all $f:R_0 \to R_0$ satisfying:
(a) $f(xf(y))f(y)=f(x+y)$, for all $x,y \ge 0$;
(b) $f(2)=0$;
(c) $f(x) \ne 0$, for $0 \le x < 2$.
	\flushright \href{https://artofproblemsolving.com/community/c6h619236}{(Link to AoPS)}
\end{problem}



\begin{solution}[by \href{https://artofproblemsolving.com/community/user/29428}{pco}]
	\begin{tcolorbox}Find all $f:R_0 \to R_0$ satisfying:
(a) $f(xf(y))f(y)=f(x+y)$, for all $x,y \ge 0$;
(b) $f(2)=0$;
(c) $f(x) \ne 0$, for $0 \le x < 2$.\end{tcolorbox}
What is $R_0$ ?
\end{solution}



\begin{solution}[by \href{https://artofproblemsolving.com/community/user/212909}{ARCH999}]
	Non-negative reals.
\end{solution}



\begin{solution}[by \href{https://artofproblemsolving.com/community/user/29428}{pco}]
	\begin{tcolorbox}Find all $f:\mathbb R_{\ge 0} \to \mathbb R_{\ge 0}$ satisfying:
(a) $f(xf(y))f(y)=f(x+y)$, for all $x,y \ge 0$;
(b) $f(2)=0$;
(c) $f(x) \ne 0$, for $0 \le x < 2$.\end{tcolorbox}
Let $P(x,y)$ be the assertion $f(xf(y))f(y)=f(x+y)$
If $x\ge 2$, $P(x-2,2)$ $\implies$ $f(x)=0$ and so $f(x)=0$ $\iff$ $x\ge 2$

If $x<2$ : 
Let $y\ge 2-x$ : $P(y,x)$ $\implies$ $f(yf(x))=0$ $\implies$ $f(x)\ge \frac 2y$ $\implies$ $f(x)\ge \frac 2{2-x}$
Let $y < 2-x$ : $P(y,x)$ $\implies$ $f(yf(x))\ne 0$ $\implies$ $f(x)<\frac 2y$ $\implies$ $f(x)\le \frac 2{2-x}$

And so $\boxed{f(x)=\frac 2{2-x}\text{  }\forall x\in[0,2)\text{  and  }f(x)=0\text{  }\forall x\ge 2}$ which indeed is a solution.
\end{solution}



\begin{solution}[by \href{https://artofproblemsolving.com/community/user/212909}{ARCH999}]
	But how can we get $f(x)\ge \frac{2}{2-x}$ from $f(x)\ge \frac{2}{y}$ in the first case because we can conclude so if only $y \le 2-x$ ,but then it's a contradiction. Example: $x=1.5, y=0.6$. 
Similar for the second case.......
\end{solution}



\begin{solution}[by \href{https://artofproblemsolving.com/community/user/29428}{pco}]
	\begin{tcolorbox}But how can we get $f(x)\ge \frac{2}{2-x}$ from $f(x)\ge \frac{2}{y}$ in the first case because we can conclude so if only $y \le 2-x$ ,but then it's a contradiction. Example: $x=1.5, y=0.6$. 
Similar for the second case.......\end{tcolorbox}
For the first, choose $y=2-x\ge 2-x$ and so $f(x)\ge\frac 2y$ becomes $f(x)\ge \frac 2{2-x}$

For the second, choose $y=2-x-\frac 1n<2-x$ and so $f(x)<\frac 2y$ becomes $f(x)< \frac 2{2-x-\frac 1n}$ and, setting $n\to+\infty$,  $f(x)\le \frac 2{2-x}$
\end{solution}
*******************************************************************************
-------------------------------------------------------------------------------

\begin{problem}[Posted by \href{https://artofproblemsolving.com/community/user/212909}{ARCH999}]
	Find all functions $f:R \to [0, \infty )$ such that :
$f(x^2 + y^2) = f(x^2-y^2) + f(2xy)$.
	\flushright \href{https://artofproblemsolving.com/community/c6h619238}{(Link to AoPS)}
\end{problem}



\begin{solution}[by \href{https://artofproblemsolving.com/community/user/29428}{pco}]
	\begin{tcolorbox}Find all functions $f:R \to [0, \infty )$ such that :
$f(x^2 + y^2) = f(x^2-y^2) + f(2xy)$.\end{tcolorbox}
Let $P(x,y)$ be the assertion $f(x^2+y^2)=f(x^2-y^2)+f(2xy)$
$P(0,0)$ $\implies$ $f(0)=0$
$P(0,x)$ $\implies$ $f(x^2)=f(-x^2)$ and so $f(x)$ is even.
Let then $g(x)$ from $\mathbb R_{\ge 0}\to\mathbb R_{\ge 0}$ defined as $g(x)=f(\sqrt x)=f(-\sqrt x)$

Let $u,v\ge 0$ : $P(\sqrt{\frac{\sqrt{u+v}+\sqrt u}2},\sqrt{\frac{\sqrt{u+v}-\sqrt u}2})$ $\implies$ $g(u+v)=g(u)+g(v)$ and so $g(x)=ax$ for some $a\ge 0$ (note that $g(x)$ is non negative and so non decreasing)

And so $\boxed{f(x)=ax^2}$ $\forall x$, which indeed is a solution, whatever is $a\ge 0$
\end{solution}



\begin{solution}[by \href{https://artofproblemsolving.com/community/user/212909}{ARCH999}]
	Please tell me how you got the idea for the substitution.................. :(
Also how can you conclude $g(x)=ax$ when it has not been said that it is continuous?

[hide]Actually, I'm learning functional equations now and I need to know the full detail. So, please help me.......[\/hide]
\end{solution}



\begin{solution}[by \href{https://artofproblemsolving.com/community/user/29428}{pco}]
	\begin{tcolorbox}Also how can you conclude $g(x)=ax$ when it has not been said that it is continuous?\end{tcolorbox}
It is $\ge 0$, so lower bounded. (and btw non-decreasing). That is enough to conclude.
\end{solution}
*******************************************************************************
-------------------------------------------------------------------------------

\begin{problem}[Posted by \href{https://artofproblemsolving.com/community/user/153258}{khoa_liv29}]
	Find all functions $f:R\to R$ such that:
 $f(xy+f(x))=xf(y)+f(x)$
	\flushright \href{https://artofproblemsolving.com/community/c6h619265}{(Link to AoPS)}
\end{problem}



\begin{solution}[by \href{https://artofproblemsolving.com/community/user/29428}{pco}]
	\begin{tcolorbox}Find all functions $f:R\to R$ such that:
 $f(xy+f(x))=xf(y)+f(x)$\end{tcolorbox}
$\boxed{\text{S1 : }f(x)=0\text{  }\forall x}$ is a solution. So let us from now look only for non allzero solutions.
Let $P(x,y)$ be the assertion $f(xy+f(x))=xf(y)+f(x)$
Let $a$ such that $f(a)\ne 0$
Let $u=f(-1)+1$

$P(-1,-1)$ $\implies$ $f(u)=0$
$P(x,u)$ $\implies$ $f(f(x)+ux)=f(x)$
$P(x,f(x)+ux)$ $\implies$ $f(x(f(x)+ux)+f(x))=(x+1)f(x)$
$P(f(x)+ux,x)$ $\implies$ $f(x(f(x)+ux)+f(x))=(f(x)+ux+1)f(x)$
Subtracting, we get $f(x)(f(x)+(u-1)x)=0$ $\forall x$ and so $\forall x$, either $f(x)=0$, either $f(x)=(1-u)x$
Note that this implies that $f(0)=0$ and $f(a)=(1-u)a$ and so $u\ne 1$ and $a\ne 0$
$P(a,0)$ $\implies$ $f((1-u)a)=(1-u)a$ and so so $u=0$
If $\exists v\ne 0$ such that $f(v)=0$, then $P(a,v)$ $\implies$ $f(av+(1-u)a)=(1-u)a$ and, since $RHS\ne 0$, we get $(1-u)(av+(1-u)a))=(1-u)a$ and so $u=v$, impossible since $u=0$

So $\boxed{\text{S2 : }f(x)=x\text{  }\forall x}$ which indeed is a solution.
\end{solution}



\begin{solution}[by \href{https://artofproblemsolving.com/community/user/153258}{khoa_liv29}]
	Thanks for your nice solution. :D This is my solution: (this one is based on yours :D)
Let $P(x,y)$ be the assertion $f(xy+f(x))=xf(y)+f(x)$
$P(x,0) \Rightarrow f(f(x))=f(x)+xf(0)$
$\bullet$ If $f(0)=0$ we have $f(f(x))=f(x)$
$P(f(x),x) \Rightarrow f(f(x).x+f(x))=f(x)^2+f(x)$
$P(x,f(x)) \Rightarrow f(xf(x)+f(x))=xf(x)+f(x)$
Thus we have $f(x)^2=xf(x) \Rightarrow f(x)=0$ or $f(x)=x$
If there exist $a,b \in R, a \neq b \neq 0$: $P(a,b) \Rightarrow ab+b=b$, we have contradiction.
Thus $f(x)=x$ for all $x$ and $f(x)=0$ for all $x$ are the solutions.
$\bullet$ If $f(0) \neq 0$ we have $f(x)$ is injective.
$P(0,y) \Rightarrow f(f(0))=f(0) \Rightarrow f(0)=0$, we have contradiction.
So $f(x)=x$ for all $x$ and $f(x)=0$ for all $x$ are all the solutions.
\end{solution}
*******************************************************************************
-------------------------------------------------------------------------------

\begin{problem}[Posted by \href{https://artofproblemsolving.com/community/user/185304}{aymas}]
	Let $f$ be an injective function from $\mathbb R$ to $\mathbb R$ such that for every sequence $(x_n)_n$, if  the sequence $(f(x_n))_n$ converges to $f(a)$ then $(x_n)_n$ is convergent.
Is it necessary that $(x_n)_n$ converges to $ a $?
	\flushright \href{https://artofproblemsolving.com/community/c7h570617}{(Link to AoPS)}
\end{problem}



\begin{solution}[by \href{https://artofproblemsolving.com/community/user/97235}{iarnab_kundu}]
	My example $f(x)=\begin{cases}0\text{ if } x=1\\ 1\text{ if } x=0\\ x\text{ otherwise }\end{cases}$.

[url=http://www.artofproblemsolving.com/Forum\/memberlist.php?mode=viewprofile&u=107451]Learner94's example[\/url] $f(x)\begin{cases}x\text{ if } x\text{ is}\text{ an}\text{ integer}\\x+1\text{ otherwise }\end{cases}$.
\end{solution}



\begin{solution}[by \href{https://artofproblemsolving.com/community/user/29428}{pco}]
	\begin{tcolorbox}My example $f(x)=\begin{cases}0\text{ if } x=1\\ 1\text{ if } x=0\\ x\text{ otherwise }\end{cases}$.
\end{tcolorbox}
Wrong example :

Let the sequence $x_{2n}=-\frac 1n$ and $x_{2n+1}=1$

The sequence $f(x_n)$ converges towards $f(1)$ while $x_n$ does not converge.  So this function does not match all the requirements of the problem.
\end{solution}



\begin{solution}[by \href{https://artofproblemsolving.com/community/user/29428}{pco}]
	\begin{tcolorbox}[url=http://www.artofproblemsolving.com/Forum\/memberlist.php?mode=viewprofile&u=107451]Learner94's example[\/url] $f(x)\begin{cases}x\text{ if } x\text{ is}\text{ an}\text{ integer}\\x+1\text{ otherwise }\end{cases}$.\end{tcolorbox}
Wrong example :

Let the sequence $x_{2n}=\frac 1n$ and $x_{2n+1}=1$

The sequence $f(x_n)$ converges towards $f(1)$ while $x_n$ does not converge.  So this function does not match all the requirements of the problem
\end{solution}



\begin{solution}[by \href{https://artofproblemsolving.com/community/user/29428}{pco}]
	\begin{tcolorbox}Let $f$ be an injective function from $\mathbb R$ to $\mathbb R$ such that for every sequence $(x_n)_n$, if  the sequence $(f(x_n))_n$ converges to $f(a)$ then $(x_n)_n$ is convergent.
Is it necessary that $(x_n)_n$ converges to $ a $?\end{tcolorbox}
Let any sequence $x_n$ such that $f(x_n)$ converges towards $f(a)$

Let the sequence $y_n$ defined as : $y_{2n}=x_n$ and $y_{2n+1}=a$ : the sequence $f(y_n)$ converges towards $f(a)$

So the sequence $y_n$ converges. But the sequence $y_n$ can only converge towards $a$.

So the sequence $x_n$ converges towards $a$.

And no need of injectivity (which is an immediate consequence of the convergence property)
\end{solution}



\begin{solution}[by \href{https://artofproblemsolving.com/community/user/185304}{aymas}]
	that is the idea pco . well done
\end{solution}



\begin{solution}[by \href{https://artofproblemsolving.com/community/user/29428}{pco}]
	\begin{tcolorbox}Edit: it should read f(x) = x +1 for integers and f(x) = x otherwise. (not my fault )\end{tcolorbox}

Wrong example :

Let the sequence $x_{2n}=\frac 1n$ and $x_{2n+1}=-1$ : the sequence $f(x_n)$ converges towards $f(-1)$ while $x_n$ does not converge.

So the function does not match all the requirements of the problem.
\end{solution}



\begin{solution}[by \href{https://artofproblemsolving.com/community/user/107451}{Learner94}]
	yeah true.. i messed it up :P
\end{solution}



\begin{solution}[by \href{https://artofproblemsolving.com/community/user/185304}{aymas}]
	about your remarque  pco i proved during the test the seconde condition implies directly the injectivity and by the same manner i proved the unicity of  the  of adherence value of sequence but as my teacher told me the injectivite was a supplémentary condition to maintain the unicity of the  of adherence values of sequence .for proving the injectivity suppose that there existe a and b such that a!=b and f(a)=f(b) then considere the sequence x_(2n)=a and x_(2n+1)=b then f(x_n)=f(a) convergente but x_n is divergente contradiction . So neccessary f is injective
\end{solution}



\begin{solution}[by \href{https://artofproblemsolving.com/community/user/185304}{aymas}]
	another alternative .
Let $f$ be an injective function from $R$ to $R$ .
let $(x_n)_n$ be a convegente sequence such that $(f(x_n))_n$ converge to $f(a)$
is it necessary that $(x_n)_n$ converge to $a ?$
\end{solution}



\begin{solution}[by \href{https://artofproblemsolving.com/community/user/29428}{pco}]
	\begin{tcolorbox}another alternative .
Let $f$ be an injective function from $R$ to $R$ .
let $(x_n)_n$ be a convegente sequence such that $(f(x_n))_n$ converge to $f(a)$
is it necessary that $(x_n)_n$ converge to $a ?$\end{tcolorbox}
Obviously not : see the two examples given previously by iarnab_kundu and Learner94 (which, now, are usable)
\end{solution}



\begin{solution}[by \href{https://artofproblemsolving.com/community/user/64716}{mavropnevma}]
	The reason your teacher invoked for the presence of injectivity in the statement of the problem is a fallacy. Who cares about the possible adherence values of the sequence $(x_n)$, when the only time we deal with such a sequence is when $(f(x_n))$ is convergent to some $f(a)$? in which case \begin{bolded}we are told \end{bolded}$(x_n)$ is convergent, thus it has only one adherence point (its limit).

Notice that if we are just told $(f(x_n))$ is convergent, forcing $(x_n)$ to be convergent to some limit $\lim_{n\to \infty} x_n =\ell$, it does not necessarily follow that $\lim_{n\to \infty} f(x_n) =f(\ell)$, even if we know $f$ to be injective. An example is any function with a jump discontinuity.
\end{solution}
*******************************************************************************
-------------------------------------------------------------------------------

\begin{problem}[Posted by \href{https://artofproblemsolving.com/community/user/171969}{mehrdad1st}]
	Find all functions $f:\mathbb R\to \mathbb R$ such that 
\[f(xf(y))+f(yf(x))=2xy \quad \forall x,y\in \mathbb R\]
	\flushright \href{https://artofproblemsolving.com/community/q1h570625}{(Link to AoPS)}
\end{problem}



\begin{solution}[by \href{https://artofproblemsolving.com/community/user/190093}{KamalDoni}]
	setting x=y=1 we find that f((1)) = 1 let's take f(1)=k then 
f(kf(y))+f(y)=2ky (it's obvious that f(0)=0)  then we can say that function is injective  but
by setting x=1 and y=k we find that f($\ k^2$)=k then by injection k^2=1 . So we proved that k=1 or k=-1 , in the first we take f(x)=x the second f(x)=-x
\end{solution}



\begin{solution}[by \href{https://artofproblemsolving.com/community/user/29428}{pco}]
	\begin{tcolorbox}..., in the first we take f(x)=x the second f(x)=-x\end{tcolorbox}
Uhhh ? How ?
Thanks for any explanation more.
\end{solution}
*******************************************************************************
-------------------------------------------------------------------------------

\begin{problem}[Posted by \href{https://artofproblemsolving.com/community/user/25405}{AndrewTom}]
	Find all polynomials

$f(x) = a_{0} + a_{1}x + ... + a_{n}x^{n}$

satisfying the equation 

$f(x^{2}) = (f(x))^{2}$

for all real numbers $x$.
	\flushright \href{https://artofproblemsolving.com/community/q1h571651}{(Link to AoPS)}
\end{problem}



\begin{solution}[by \href{https://artofproblemsolving.com/community/user/31915}{Batominovski}]
	Hint: Any root of $f$, if exists, must be $0$.  

EDIT: This hint assumes that $f$ is not identically zero.
\end{solution}



\begin{solution}[by \href{https://artofproblemsolving.com/community/user/29428}{pco}]
	\begin{tcolorbox}Hint: Any root of $f$, if exists, must be $0$.\end{tcolorbox}
What about $-1, 1, i, -i,$ ... and a lot of others ....

This is not the good hint. Good hint is "identification of two highest degrees coefficients"
\end{solution}



\begin{solution}[by \href{https://artofproblemsolving.com/community/user/31915}{Batominovski}]
	If $z \neq 0$ is a root of $f$, then, for any integer $k \geq 0$, \begin{bolded}every\end{bolded} $\left(2^k\right)$-th root of $z$ will be a root of $f$ as well.  Why is my hint not good?  The only omission in my original comment is that $f$ is assumed to be nonzero.  By the way, my hint works even if $f$ is an entire function satisfying the same functional equation.  So, in a sense, it is better than equating coefficients.
\end{solution}



\begin{solution}[by \href{https://artofproblemsolving.com/community/user/29428}{pco}]
	\begin{tcolorbox}If $z \neq 0$ is a root of $f$, then, for any integer $k \geq 0$, \begin{bolded}every\end{bolded} $\left(2^k\right)$-th root of $z$ will be a root of $f$ as well.  ...\end{tcolorbox}
You're quite right. I considered only $x^{2^n}$ while, as you suggest, $x^{2^{-n}}$ kills it.

And your hint indeed is better than mine.

Sorry :blush:
\end{solution}
*******************************************************************************
-------------------------------------------------------------------------------

\begin{problem}[Posted by \href{https://artofproblemsolving.com/community/user/174113}{victory1204}]
	Find all f:Z->Z such that
f(m+n)+f(mn-1)=f(m)f(n)+a ;a is not equal to 0
	\flushright \href{https://artofproblemsolving.com/community/q1h573996}{(Link to AoPS)}
\end{problem}



\begin{solution}[by \href{https://artofproblemsolving.com/community/user/192463}{arkanm}]
	Let $P(x,y)$ be the assertion $f(x+y)+f(xy-1)=f(x)f(y)+a$

$P(x,0)\implies f(x)+f(-1)=f(x)f(0)+a\implies f(x)(1-f(0))=a-f(-1)$
There are two cases :

\begin{bolded}Case 1.\end{bolded} $f(0)=1$, and so $f(-1)=a$
For this case, I'm getting these weird polynomials :
$f(0)=1$
$f(\pm 1)=a$
$f(\pm 2)=a^2+a-1$
$f(\pm 3)=a^3+a^2-a$
$f(\pm 4)=a^4+a^3-2a^2+1$
$f(\pm 5)=a^5+a^4-3a^3-a^2+3a$
$f(\pm 6)=a^6+a^5-4a^4-2a^3+5a^2+a-1$
and etc.

The second case is easy, just let $f(x)=c\in \bf Z$ and then solve.
\end{solution}



\begin{solution}[by \href{https://artofproblemsolving.com/community/user/29428}{pco}]
	\begin{tcolorbox}Find all f:Z->Z such that
f(m+n)+f(mn-1)=f(m)f(n)+a ;a is not equal to 0\end{tcolorbox}
Let $P(x,y)$ be the assertion $f(x+y)+f(xy-1)=f(x)f(y)+a$

1) Constant solutions
==============
If $x^2-2x+a=0$ has integer solutions (in fact if $a=1-n^2$), then we have the solutions :
$\boxed{S_1 : a=1-n^2\text{ gives }f(x)=1+n\text{  }\forall n}$

$\boxed{S_2 : a=1-n^2\text{ gives }f(x)=1-n\text{  }\forall n}$

2)Non constant solutions
=================
Let $b=f(1)$

2.1) $f(0)=1$ and $f(-1)=a$ and $b=a$ and $a\in\{-2,-1,1,2\}$
--------------------------------------------------------------------------
If $f(0)\ne 1$ : $P(x,0)$ $\implies$ $f(-1)-a=f(x)(f(0)-1)$ and so $f(x)$ is contant, impossible. So $f(0)=1$ and $f(-1)=a$

$P(-1,1)$ $\implies$  $f(-2)=ab+a-1$
$P(-1,-1)$ $\implies$  $f(-2)=a^2+a-1$
Subtracting, we get $b=a$ (remember $a\ne 0$)

$f(0)=1$
$f(1)=a$
$P(1,1)$ $\implies$ $f(2)=a^2+a-1$
$P(2,1)$ $\implies$ $f(3)=a^3+a^2-a$
$P(3,1)$ $\implies$ $f(4)=a^4+a^3-2a^2+1$
$P(4,1)$ $\implies$ $f(5)=a^5+a^4-3a^3-a^2+3a$
$P(3,2)$ $\implies$ $(a-2)(a-1)(a+1)(a+2)=0$
Q.E.D.

2.2) Case where $a=b=-2$ 
----------------------------
$f(0)=1$
$f(1)=-2$
$P(x+1,1)$ $\implies$ $f(x+2)=-2f(x+1)-f(x)-2$
Simple induction gives :

$\boxed{S_3 : a=-2\text{ gives }f(2n)=1\text{  and }f(2n+1)=-2\text{  }\forall n}$

which indeed is a solution

2.3) Case where $a=b=-1$ 
---------------------------
$f(0)=1$
$f(1)=-1$
$P(x+1,1)$ $\implies$ $f(x+2)=-f(x+1)-f(x)-1$
Simple induction gives :

$\boxed{S_4 : a=-1\text{ gives }f(3n)=1\text{  and }f(3n+1)=-1\text{  and }f(3n+2)=-1\text{  }\forall n}$

which indeed is a solution

2.4) Case where $a=b=1$ 
------------------------
$f(0)=1$
$f(1)=1$
$P(x+1,1)$ $\implies$ $f(x+2)=-f(x+1)-f(x)-1$
Simple induction gives $f(n)=1$, which indeed is a solution, already noted in the paragraph "constant solutions"

2.5) Case where $a=b=2$ 
-------------------------------
$f(0)=1$
$f(1)=2$
$P(x+1,1)$ $\implies$ $f(x+2)=2f(x+1)-f(x)+2$
Simple induction gives :

$\boxed{S_5 : a=2\text{ gives }f(n)=n^2+1\text{  }\forall n}$

which indeed is a solution
\end{solution}



\begin{solution}[by \href{https://artofproblemsolving.com/community/user/174113}{victory1204}]
	How about a=0? Is there a solution?
\end{solution}



\begin{solution}[by \href{https://artofproblemsolving.com/community/user/29428}{pco}]
	\begin{tcolorbox}How about a=0? Is there a solution?\end{tcolorbox}
Uhhh ?, this is exactly the same exercise. You should have found alone from the previous one , according to me.

Let $P(x,y)$ be the assertion $f(x+y)+f(xy-1)=f(x)f(y)$

1) Constant solutions
==============
We immediately get two constant solutions :
$\boxed{S_1 : f(x)=0\text{   }\forall x}$

$\boxed{S_2 : f(x)=2\text{   }\forall x}$

2)Non constant solutions
=================
2.1) $f(0)=1$ and $f(-1)=0$ and $f(1)\in \{-1,0,2\}$
-------------------------------------------------
Let $b=f(1)$
If $f(0)\ne 1$ : $P(x,0)$ $\implies$ $f(-1)=f(x)(f(0)-1)$ and so $f(x)$ is contant, impossible. So $f(0)=1$ and $f(-1)=0$

$P(x+1,1)$ $\implies$ $f(x+2)=bf(x+1)-f(x)$ and so :
$f(-1)=0$
$f(0)=1$
$f(1)=b$
$f(2)=b^2-1$
$f(3)=b^3-2b$
$f(4)=b^4-3b^2+1$

$P(2,2)$ $\implies$ $b(b+1)(b-2)=0$ and so $b\in\{-1,0,2\}$
Q.E.D.

2.2) Case where $f(1)=-1$ 
--------------------------
$f(0)=1$
$f(1)=-1$
$P(x+1,1)$ $\implies$ $f(x+2)=-f(x+1)-f(x)$
Simple induction gives :

$\boxed{S_3 : f(3n)=1\text{  and }f(3n+1)=-1\text{  and }f(3n+2)=0\text{  }\forall n}$

which indeed is a solution

2.3) Case where $f(1)=0$ 
--------------------------
$f(0)=1$
$f(1)=0$
$P(x+1,1)$ $\implies$ $f(x+2)=-f(x)$
Simple induction gives :

$\boxed{S_4 : f(2n)=(-1)^n\text{  and }f(2n+1)=0\text{  }\forall n}$

which indeed is a solution

2.4) Case where $f(1)=2$ 
--------------------------
$f(0)=1$
$f(1)=2$
$P(x+1,1)$ $\implies$ $f(x+2)=2f(x+1)-f(x)$
Simple induction gives :

$\boxed{S_5 : f(n)=n+1\text{  }\forall n}$

which indeed is a solution
\end{solution}
*******************************************************************************
-------------------------------------------------------------------------------

\begin{problem}[Posted by \href{https://artofproblemsolving.com/community/user/125553}{lehungvietbao}]
	Find all functions $f,g:\mathbb{R}\to\mathbb{R}$ such that $f$ is strictly increasing and \[f(xy)=f(x)g(y)+f(y)\quad \forall x,y\in \mathbb{R}\]
	\flushright \href{https://artofproblemsolving.com/community/q1h578361}{(Link to AoPS)}
\end{problem}



\begin{solution}[by \href{https://artofproblemsolving.com/community/user/29428}{pco}]
	\begin{tcolorbox}Find all functions $f,g:\mathbb{R}\to\mathbb{R}$ such that $f$ is strictly increasing and \[f(xy)=f(x)g(y)+f(y)\quad \forall x,y\in \mathbb{R}\]\end{tcolorbox}
Let $P(x,y)$ be the assertion $f(xy)=f(x)g(y)+f(y)$

If $f(0)=0$, then $P(0,x)$ $\implies$ $f(x)=0$ $\forall x$, impossible since $f(x)$ is strictly increasing. So $f(0)=\ne 0$
Let $a=\frac 1{f(0)}$

$P(0,x)$ $\implies$ $g(x)=1-af(x)$ and functional equation becomes :
$g(x)$ strictly monotonous and $g(0)=0$ and new assertion $Q(x,y)$ : $g(xy)=g(x)g(y)$
And so very classicaly $g(x)=\text{sign}(x)|x|^v$ $\forall x$ and for any $v>0$

And so $\boxed{(f(x),g(x))=\left(u(\text{sign}(x)|x|^v-1),\text{sign}(x)|x|^v\right)}$ $\forall x$ which indeed is a solution, whatever are $u,v>0$
\end{solution}
*******************************************************************************
-------------------------------------------------------------------------------

\begin{problem}[Posted by \href{https://artofproblemsolving.com/community/user/208238}{phamngocsonyb}]
	Determine all functions $f:Z - Z$ where $Z$ is the set of integers, such that 

$f(m+f(f(n)))=-f(f(m+1))-n$ for all integers $m$ and $n$
	\flushright \href{https://artofproblemsolving.com/community/q1h580160}{(Link to AoPS)}
\end{problem}



\begin{solution}[by \href{https://artofproblemsolving.com/community/user/29428}{pco}]
	\begin{tcolorbox}Determine all functions $f:Z - Z$ where $Z$ is the set of integers, such that 

$f(m+f(f(n)))=-f(f(m+1))-n$ for all integers $m$ and $n$\end{tcolorbox}
Let $P(x,y)$ be the assertion $f(x+f(f(y)))=-f(f(x+1))-y$
$f(x)$ is bijective

$P(0,y)$ $\implies$ $f(f(f(y)))=a-y$ for some $a=-f(f(1))$
Using surjectivity, let $u$ such that $f(f(u+1))=0$ : $P(u,f(y))$ $\implies$  $f(u+a-y)=-f(y)$

So $P(x,y)$ may be written $f(x+f(f(y)))=f(u+a-f(x+1))-y$ and so $P(x,0)$ becomes $f(x+f(f(0)))=f(u+a-f(x+1))$

Then, injectivity implies $f(x)=c-x$ for some $c$ and, plugging this back in original equation, we get $c=-1$

Hence the unique solution $\boxed{f(x)=-x-1}$
\end{solution}
*******************************************************************************
-------------------------------------------------------------------------------

\begin{problem}[Posted by \href{https://artofproblemsolving.com/community/user/32404}{epsilon07}]
	$\overline{abc}$ is a three digit number, the roots of the polynomial $ ax^2 + bx +c $  are integers, and these integer roots divide the $3$-digit number $\overline{abc}$. How many different $\overline{abc}$ can you find?
	\flushright \href{https://artofproblemsolving.com/community/q1h581335}{(Link to AoPS)}
\end{problem}



\begin{solution}[by \href{https://artofproblemsolving.com/community/user/29428}{pco}]
	\begin{tcolorbox}$\overline{abc}$ is a three digit number, the roots of the polynomial $ ax^2 + bx +c $  are integers, and these integer roots divide the $3$-digit number $\overline{abc}$. How many different $\overline{abc}$ can you find?\end{tcolorbox}
So quadratic is $px^2+p(m+n)x+pmn$ where $-m,-n$ are the two negative integers root.

Constraints are $p(m+n),pmn\in\{1,2,3,4,5,6,7,8,9\}$ and $m|10p(n+10)$ and $n|10p(m+10)$
WLOG $m\ge n$

First constraint gives very few values $(m,n,p)$ : 
$(9,1,1),(8,1,1),(7,1,1),(6,1,1),(5,1,1)$,
$(4,2,1),(4,1,1)$,
$(3,3,1),(3,2,1),(3,1,1),(3,1,2)$,
$(2,2,1),(2,2,2),(2,1,1)(2,1,2),(2,1,3)$,
$(1,1,1),(1,1,2),(1,1,3),(1,1,4)$

Second constraint reduces this short list to :
$(5,1,1),(4,2,1),(3,2,1),(2,2,1),(1,1,1),(1,1,2),(1,1,3),(1,1,4),(2,1,1),(2,1,2),(2,2,2),(2,1,3)$

And so $\boxed{\overline{abc}\in\{165,168,156,144,121,242,363,484,132,264,288,396\}}$
\end{solution}
*******************************************************************************
-------------------------------------------------------------------------------

\begin{problem}[Posted by \href{https://artofproblemsolving.com/community/user/195510}{AHZOLFAGHARI}]
	Find all Continuous function ( R to R) :

[size=150]f(1-f(x))=1-x[\/size]
	\flushright \href{https://artofproblemsolving.com/community/q1h583154}{(Link to AoPS)}
\end{problem}



\begin{solution}[by \href{https://artofproblemsolving.com/community/user/29428}{pco}]
	\begin{tcolorbox}Find all Continuous function ( R to R) :

[size=150]f(1-f(x))=1-x[\/size]\end{tcolorbox}
Setting $f(x)=1-g(x)$, equation is $g(g(x))=x$ and so a trivial problem with infinitely many continuous solutions.
\end{solution}
*******************************************************************************

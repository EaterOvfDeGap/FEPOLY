-------------------------------------------------------------------------------

\begin{problem}[Posted by \href{https://artofproblemsolving.com/community/user/2}{Valentin Vornicu}]
	Let $P(x)$ be a polynomial of degree $n > 1$ with integer coefficients and let $k$ be a positive integer. Consider the polynomial $Q(x) = P(P(\ldots P(P(x)) \ldots ))$, where $P$ occurs $k$ times. Prove that there are at most $n$ integers $t$ such that $Q(t) = t$.
	\flushright \href{https://artofproblemsolving.com/community/c6h101487}{(Link to AoPS)}
\end{problem}



\begin{solution}[by \href{https://artofproblemsolving.com/community/user/11309}{ychjae}]
	Oops :oops:
\end{solution}



\begin{solution}[by \href{https://artofproblemsolving.com/community/user/4683}{Nima Ahmadi Pour}]
	My solution to this one:
If $a_{1}, a_{2},\ldots, a_{k}$ are points of a periodic orbit of $P$ ($P(a_{i})=a_{i+1}$) we have \[a_{2}-a_{1}\mid a_{3}-a_{2}\mid \ldots\mid a_{1}-a_{k}\mid a_{2}-a_{1}\] So we have $|a_{i+1}-a_{i}|$ is a constant. But this is impossible for $k>2$. (Guess why)
So all periodic orbits are of degree $1$ or $2$.
If we have no $2$nd degree periodic orbit we are done ($P(x)=x$ has at most $n$ roots)
If we have two $2$nd degree orbits like $(a,b)$ and $(c,d)$, then we have \[a-c\mid b-d\mid a-c\] So we have $|a-c|=|b-d|$ and similarly $|a-d|=|b-c|$ which easily proves that $a+b=c+d$. If we have a $2$nd degree orbit like $(a,b)$ and a fixed point $t$ then similarly we would have $t+t=a+b$. So now let just $u=a+b$ then all periodic points are the roots of equation $x+P(x)=u$ which obviously has at most $n$ roots.
\end{solution}



\begin{solution}[by \href{https://artofproblemsolving.com/community/user/13790}{dule_00}]
	\begin{tcolorbox}My solution to this one:
If $a_{1}, a_{2},\ldots, a_{k}$ are points of a periodic orbit of $P$ ($P(a_{i})=a_{i+1}$) we have \[a_{2}-a_{1}\mid a_{3}-a_{2}\mid \ldots\mid a_{1}-a_{k}\mid a_{2}-a_{1}\] \end{tcolorbox}

Can you explain me this, please?
\end{solution}



\begin{solution}[by \href{https://artofproblemsolving.com/community/user/10512}{mathmanman}]
	It follows from the well-known lemma that, for all integers $a, b$ ($a \neq b$), $a-b | P(a)-P(b)$.
\end{solution}



\begin{solution}[by \href{https://artofproblemsolving.com/community/user/13790}{dule_00}]
	O, thank you  :blush:
\end{solution}



\begin{solution}[by \href{https://artofproblemsolving.com/community/user/141}{Soarer}]
	\begin{tcolorbox} then $P(P(t))=t$ iff $P(t)\in S$\end{tcolorbox}

huh? why?
\end{solution}



\begin{solution}[by \href{https://artofproblemsolving.com/community/user/11309}{ychjae}]
	\begin{tcolorbox}[quote="ychjae"] then $P(P(t))=t$ iff $P(t)\in S$\end{tcolorbox}

huh? why?\end{tcolorbox}
Sorry, I make a very big mistake here :oops: .
\end{solution}



\begin{solution}[by \href{https://artofproblemsolving.com/community/user/1174}{darktreb}]
	My question: does this problem still hold over the reals? 

When I first saw it, I intuitively thought that the maximum possible cycle should be 2, because if you consider the graph of an arbitrary polynomial $f(x)$ of degree greater than 1, and the line $y = x$, then while you're iterating you're basically jumping from points on the function, to the line $y = x$, then back to the function, etc. So you're going around in a rectangular motion, and you either start diverging, start converging, or in the special case of an iteration of length 2, you create a perfect rectangle ($P(a) = b, P(b) = a$). Oh yeah and there's the case $P(a) = a$ as well, of course.

Since we are given a positive integer $k$, and the convergence case would require infinitely many iterations, my first instinct is that the problem still holds over real polynomials with real coefficients.

EDIT: Courtesy of Thomas Mildorf and Mathematica I see that it does not hold over all reals. Just pounding out the resulting polynomials from applying $P(P(x))$ for instance will give many distinct roots....
\end{solution}



\begin{solution}[by \href{https://artofproblemsolving.com/community/user/8980}{julien_santini}]
	Maybe we should exploit the fact that $P$ divides iterations of $P$. 
In the easiest case, $P^{(2)}=QP$, where $P^{(k)}=PoPo \dots oP$ ($k>1$ times). Now $P^{(2)}(t)=t=Q(t)P(t) \Rightarrow P(t)$ divides $t$. But also $P(t)=Q(P(t))t$, and t divides $P(t)$, meaning $|P(t)|=|t|$. Now we have at most $2n$ such $t$. How do you finish it for this case ?
\end{solution}



\begin{solution}[by \href{https://artofproblemsolving.com/community/user/18317}{Centy}]
	I feel slightly robbed by this question. For the case where the cycle length of $P(P\ldots P(x) \ldots ))$ is 3 or more, I quoted a famous problem out of from the USAMO in the 1970s, proved in Engel's Problem Solving.

I got nothing for quoting it whereas I would have got 3 for proving it.
\end{solution}



\begin{solution}[by \href{https://artofproblemsolving.com/community/user/1598}{Arne}]
	I also think some contestants got a serious advantage here. The problem from the USAMO is really famous, and it's also well known that any fixed point of $P \circ P \circ \cdots \circ P$ must be a fixed point of $P \circ P$.
\end{solution}



\begin{solution}[by \href{https://artofproblemsolving.com/community/user/8628}{bodom}]
	is this solution right?please read it and tell me :blush: because if it is then again this problem is not hard at all.if it's wrong then :blush:  :blush:  :blush: if $k=1$ it is obvious.if $k=2$
let's say that there are  $n+1$(if there are more it is even better) integers $x$ such that $P(P(x))=x$.let $a_{1}> a_{2}>... > a_{(}n+1 )$ be those $n+1$ integers.we clearly have$a_{i}-a_{(}i+1) \mid P(a_{i})-P(a_{(}i+1)) \mid P(P(a_{i}))-P(P(a_{(}i+1)))=a_{i}-a_{(}i+1)$so $| P(a_{i})-P(a_{(}i+1)) |=a_{i}-a_{(}i+1)$.adding this for all $i=1,2,...,n$ we get that $|P(a_{1})-P(a_{2})|+....+|P(a_{n})-P(a_{(}n+1))|=a_{1}-a_{(}n+1)= |P(a_{1})-P(_{(}n+1))|$.

but$|P(a_{1})-P(a_{2})|+....+|P(a_{n})-P(a_{(}n+1))| \ge |P(a_{1})-P(_{(}n+1))|$.since in the last one we have equality then all $|P(a_{i})-P(_{(}i+1))|$have the same sign.
1)$P(a_{i})-P(_{(}i+1))>0$ =>$P(a_{i})-a_{i}=k$ for all $i=1,2,..,n+1$ =>$P(x)-x-k=0$ for n+1 numbers but this is a contradiction since P's degree is n
2)$P(a_{i})-P(a_{(}i+1))<0$=>$P(x)+x-k=0$ for n+1 numbers and again contradiction.
if $k>2$ again we easily get in the same way that $| P(a_{i})-P(a_{i}+1) |=a_{i}-a_{i}+1$ and this is just case $k=2$ 
i can't find any mistakes.are there any?
\end{solution}



\begin{solution}[by \href{https://artofproblemsolving.com/community/user/651}{Pascual2005}]
	well in Fact it is not a "hard" problem if you are used to polinomials! Obviouslyiy you have integeer coefficients you must use $a-b|p(a)-p(b)$ and later analitic properties: in this case analitic properties didnt help sou you only needed the divisibility and inequalities and some classical facts ( a polinomial of degree $n$ has $n$ roots)

And finally the key idea was considering the numbers $p_{k+1}(x)-p_{k}(x)$ which was typical since their sum telescope to what we wanted!

I didnt like about this problem some thinks, for example its ancestors

[url]http://www.kalva.demon.co.uk\/usa\/usa74.html[\/url]

However, It  is not too bad if someone knewed that problem, it just told him there is no period of lenght 3, which I think many people thought whitout knowing it...

Edit: I didnt read whats before my post... 
\end{solution}



\begin{solution}[by \href{https://artofproblemsolving.com/community/user/3431}{manuel}]
	indeed, the "ancestror" that pascual says is a very famous and wellknown usamo problem.
in every olympiad training they teach it...
\end{solution}



\begin{solution}[by \href{https://artofproblemsolving.com/community/user/285}{harazi}]
	What you forgot to say is that the problem appeared in a kind of romanian TST some very long time ago for k=2. And believe me, this case is not much more difficult than this problem.
\end{solution}



\begin{solution}[by \href{https://artofproblemsolving.com/community/user/21129}{nik}]
	I think there is simpler solution.
x-y | P(x)-P(y)
But also P(x)-P(y) | Q(x)-Q(y)
so, if Q(x)=x and Q(y)=y,
x-y | P(x)-P(y)
P(x)-P(y) | x-y. Then everything is clear.
\end{solution}



\begin{solution}[by \href{https://artofproblemsolving.com/community/user/8628}{bodom}]
	that's what i did.but after $|P(a_{i})-P(a_{j})|=|a_{i}-a_{j}|$ it's not so clear and you must see that ineq in order to conclude about the sign. 
\end{solution}



\begin{solution}[by \href{https://artofproblemsolving.com/community/user/21129}{nik}]
	It follows:
 If there are n+1 roots of Q(t) = t,  a1 < a2 < a3 <... < a(n+1)
then we must have :
 P(a1) < P(a2) <... < P(a(n+1)), 
or P(a1) > P(a2) > ... > P(a(n+1)), 
because if for some i<j<k P(a(i)), P(a(j)), P(a(k))  aren't sorted, we get contradiction. After this it's similar to another solutions.

I think it's easier than author's solution, and problem isn't hard.
\end{solution}



\begin{solution}[by \href{https://artofproblemsolving.com/community/user/8628}{bodom}]
	to nik:what's your argument for that contradiction?
ps:indeed there are solutions much easier then the author's
\end{solution}



\begin{solution}[by \href{https://artofproblemsolving.com/community/user/21129}{nik}]
	Contradiction to |P(x)-P(y)|=|x-y|.
If we write it for a(i),a(j)   a(j),a(k) , a(i),a(k) , the sum of the fist two must be equal to the third one. It's possible only in the case they're sorted.
\end{solution}



\begin{solution}[by \href{https://artofproblemsolving.com/community/user/11748}{campos}]
	i know it would seem like a commentary from a "bad loser", since i didn't solve this problem at the test, but this problem looks much more theoric than an olympiad-kind one... don't you think the same? i would have preferred another kind of problem 

the purpose of the post is just to hear (read) some commentaries on that, instead of starting a discussion in which everybody will finish angry with everybody else.
\end{solution}



\begin{solution}[by \href{https://artofproblemsolving.com/community/user/1598}{Arne}]
	I think this problem is fine. It's beautiful and challenging, I don't see what's wrong with it.
\end{solution}



\begin{solution}[by \href{https://artofproblemsolving.com/community/user/2639}{Peter}]
	My main objection for this problem would be:\begin{tcolorbox}and it's also well known that any fixed point of $P \circ P \circ \cdots \circ P$ must be a fixed point of $P \circ P$.\end{tcolorbox}and this makes the problem really easy, showing independence of $k$. Moreover, it's rather easy to "guess" this, and state it accurately even if you didn't know the theorem. (which is kind of weak, but I'm sure it happens.)
\end{solution}



\begin{solution}[by \href{https://artofproblemsolving.com/community/user/1598}{Arne}]
	Yes, I knew that fact, so I solved it very quickly...

But apparently this problem was not easy, since very few contestants solved it...
\end{solution}



\begin{solution}[by \href{https://artofproblemsolving.com/community/user/4243}{ondrob}]
	I found this problem, maybe no easy, but certainly not hard. The only gun I was about to use was $x-y$ divide $P(x)-P(y)$. And it was quite straighforward with that... $x-y$ divide $|P(x)-P(y)|$ and it divide $Q(x)-Q(y)=x-y$ (when you suppose such $x,y$ for contradiction). so $|P(x)-P(y)|=|x-y|$ and you almost've got it :)
\end{solution}



\begin{solution}[by \href{https://artofproblemsolving.com/community/user/20415}{who}]
	if we fix k is n best possible?
\end{solution}



\begin{solution}[by \href{https://artofproblemsolving.com/community/user/4243}{ondrob}]
	2who: I don't know whether I understand what are you asking for, but for every $k$ and every $n$ you can find such a polynomial $P(x)$, that $Q(x)=P_{k}(x)=x$ for $n$ integer values of $x$, just take: $P(x)=x+x(x-1)(x-2)\cdots(x-(n-1))$ and $x\in\{0,1,\ldots,n-1\}$
\end{solution}



\begin{solution}[by \href{https://artofproblemsolving.com/community/user/6612}{marko avila}]
	I  Liked  the problem   but i  think  it definitely  gives advantages to people  who  have  experiences  with  polynomials  then  others.  i  dont have experience  with  polynomials.  but  then  again  it  ocurred  to me to  use  a-b |p(a)-p(b)  but for  some reason  i  thought that  this property  is too  known and that  the  psc wouldnt dare to  put  a problem  like that  on the imo.   i  was wrong  and  this  cost  me  my  bornze medal . so  my  advice  to anyone on the imo  is :  dont treat  the imo  as a competition  and  never  think  strategically  because  it can  kill  , moreover i  think  the whole point  of the imo  is to promote creativeness and  not  strategy. 
anyway i still  have next  year.   gold medal  for sure!!  (its not impossible)  mexico  got its  first  gold  medal  this year!!!!!!!!!
\end{solution}



\begin{solution}[by \href{https://artofproblemsolving.com/community/user/1372}{Ravi B}]
	\begin{tcolorbox}The problem from the USAMO is really famous, and it's also well known that any fixed point of $P \circ P \circ \cdots \circ P$ must be a fixed point of $P \circ P$.\end{tcolorbox}
Yes, for example, that latter fact is basically Problem A6 from the Putnam 2000 contest:
[url]http://www.kalva.demon.co.uk\/putnam\/putn00.html[\/url].

And the result was known before that.
\end{solution}



\begin{solution}[by \href{https://artofproblemsolving.com/community/user/3236}{test20}]
	\begin{tcolorbox}I also think some contestants got a serious advantage here. The problem from the USAMO is really famous, and it's also well known that any fixed point of $P \circ P \circ \cdots \circ P$ must be a fixed point of $P \circ P$.\end{tcolorbox}

But Sharkosky's theorem does not give you integrality.
\end{solution}



\begin{solution}[by \href{https://artofproblemsolving.com/community/user/64273}{trigg}]
	I like this problem 
it is a good but not hard problem 
I think it is a good problem of polynomials in number theory :)
\end{solution}



\begin{solution}[by \href{https://artofproblemsolving.com/community/user/46163}{Altunshukurlu}]
	http://www.4shared.com\/file\/145687043\/937791d5\/New_Word_2007_Document.html
 
Here is my solution ... I hope that it is true ...
\end{solution}



\begin{solution}[by \href{https://artofproblemsolving.com/community/user/191127}{sayantanchakraborty}]
	For the sake of a contradiction let $t_1>t_2>\cdots>t_{n+1}$ be integers for which $Q(t)=t$.
Then $i_i-t_j \mid P(t_i)-P(t_j) \mid P(P(t_i))-P(P(t_j)) \cdots \mid P^k(t_i)-P^k(t_j)=t_i-t_j$
so $P(t_i)-P(t_j)=\pm(t_i-t_j)$
Now it is easy to show that either $P(t_i)-P(t_{i+1})=t_i-t_{i+1} \forall i=1,2,\cdots,n$ or $P(t_i)-P(t_{i+1})=t_{i+1}-t_i \forall i=1,2,\cdots,n$(For this we use the ordering we have assumed).

But both the system of equations yeild $P(t_i)=t_i \forall i=1,2,\cdots,n+1$ which contradicts the fact that $P(x)$ is a polynomial of degree $n$.
\end{solution}



\begin{solution}[by \href{https://artofproblemsolving.com/community/user/198687}{Legend-crush}]
	let $a_1<a_2< .... < a_l $ be the fixed points of Q, and suppose $l\geq n+1$  and  $S=\lbrace a_i\/ \ i\in [|1,l|] \rbrace $
\[(\forall (i,j)) \  a_i-a_j|P(a_i)-P(a_j) and P(a_i)-P(a_j)|Q(a_i)-Q(a_j) \]
it follows that  $ (\forall (i,j)) \  |a_i-a_j|=|P(a_i)-P(a_j)| \Rightarrow (\forall (i,j)) \ |\frac{P(a_i)-P(a_j)}{a_i-a_j}|=1$
we can easily deduce that $ (\forall (i,j)) \ \frac{P(a_i)-P(a_j)}{a_i-a_j}=1 or  (\forall (i,j)) \ \frac{P(a_i)-P(a_j)}{a_i-a_j}=-1$
WLOG $ (\forall (i,j)) \ \frac{P(a_i)-P(a_j)}{a_i-a_j}=1$ then by the mean value theorem\end{underlined} \[ (\forall i\in [|1,l-1|])(\exists c\in ]a_i,a_{i+1}[)  \ P'(c)=1 \]
Thus the equation $P'(x)=1$ has $l-1\geq n$ solution.  but P' has degree n-1 . Hence we found a contradiction.
As a conclusion: Q has at most n fixed points
\end{solution}



\begin{solution}[by \href{https://artofproblemsolving.com/community/user/165750}{mathdebam}]
	Well I see a lot of "it is easy from here to prove this afterwards" kind of sentences which are not actually satisfying.Hence I am giving a solution with all the "dumb" steps and no step jumps. :p
Let the required fixed points of $Q$ be $a_1,a_2,..a_{n+1}$.
Now clearly $P(a_i)-P(a_j)|P^{k-1}(a_i)-P^{k-1}(a_j)$ as the polynomial is of integer coefficients.(note here $k \ge 2$)
But $P^{k-1}(a_i)-P^{k-1}(a_j)|P^{k}(a_i)-P^k(a_j)=a_i-a_j|P(a_i)-P(a_j).$
Hence $P(a_i)-P(a_j)=|a_i-a_j|$.Fine upto here.But then there are a lot of "ordering" and other arguments which I do not understand quite. :p
Hence let $P(a_i)-P(a_j)=a_i-a_j$ and $P(a_i)-P(a_k)=a_k-a_i$.Thus $P(a_k)-P(a_j)=2a_i-a_j-a_k$.Now WLOG let $P(a_j)-P(a_k)=a_j-a_k$We get $a_i=a_j$.Similarly the other case.Hence this case cannot happen.
So either $P(a_i)-P(a_j)=a_i-a_j$ or $P(_i)-P(a_j)=a_j-a_i$ for all $i,j$.
So either the polynomial $P(x)+x$or $P(x)-x$ is a constant polynomial.But that contradicts the lower bound of the degree of $P$. Thus done.
And yes it was a bit too easy for an $P5$ if I have not mistaken above. :p
\end{solution}



\begin{solution}[by \href{https://artofproblemsolving.com/community/user/64716}{mavropnevma}]
	\begin{tcolorbox}And yes it was a bit too easy for an $P5$ if I have not mistaken above. :p\end{tcolorbox}
Well, the average score on it was $1.183$. One has to go back to 1996 to find a P5 with lower average ($0.493$), with the averages for the P2 higher. 
If one goes forward, with the exceptions of P5 of 2010 ($0.930$) and P2 of 2011 ($0.652$), all other averages are again higher.

So the IMO community belies your feeling; in fact it was one of the most difficult P5\/P2 from the last twenty years!
\end{solution}



\begin{solution}[by \href{https://artofproblemsolving.com/community/user/165750}{mathdebam}]
	Actually scores may be really misleading and people loose marks due to strange claims which they make and do not prove.And also the fact that it had an idea or a lemma which was previously in some USAMO has made maximum contestant to prove by using that lemma and thus neglecting the easier way.Anyways I may have made a mistake as well as this concept of "easy" or "hard" is a very much relative one.
\end{solution}



\begin{solution}[by \href{https://artofproblemsolving.com/community/user/243741}{anantmudgal09}]
	Nice! :)

Suppose that the polynomial $Q(x)-x$ vanishes at $n+1$ distinct integer points $x_1<x_2<\dots<x_{n+1}$ and note that $x_i-x_j \mid P(x_i)-P(x_j)$ for all distinct $i$ and $j$. Observe that $$P(x_i)-P(x_j) \mid P^k(x_i)-P^k(x_j)=Q(x_i)-Q(x_j)=x_i-x_j$$ and it follows that $|P(x_i)-P(x_j)|=|x_i-x_j|$ for all $i$ and $j$. We obtain $$\sum_{i=1}^n |P(x_{i+1})-P(x_i)|=\sum_{i=1}^n (x_{i+1}-x_i)=(x_{n+1}-x_1)=|P(x_{n+1})-P(x_1)|=|\sum_{i=1}^n (P(x_{i+1})-P(x_i))|$$ yielding that the numbers $(P(x_{i+1})-P(x_i))$ ($i=1,2,\dots, n+1$) have the same sign. Thus, $P(x_i)=x_i$ for $i=1,2,\dots, n+1$ or $P(x_i)=-x_i$ for $i=1,2,\dots, n+1$. This is false as each of the polynomials $P(x)+x$ and $P(x)-x$ has degree $n \ge 2$ so they cannot have $n+1$ roots.

\end{solution}



\begin{solution}[by \href{https://artofproblemsolving.com/community/user/345008}{Kayak}]
	Please proofread my solution :help:
[hide=Solution] We first prove the case for $k =1,2$, then we show that all cases are exactly equivalent to this. Also, we overuse the fact $a-b | P(a) -P(b)$.

\begin{bolded}Claim-1\end{bolded}: The problem holds true for $k=1,2$.
[hide=Proof] $k=1$ is trivial, because $P(x)-x$ can't assume more than $Deg[P]$ roots. Now, for $k=2$, note that the integers satisfying $P(P(x)) = x$ can be either fipo (short for fixed point, $P(x) = x$), or twin (the integes satisfying $P(P(x)) = x$, but $x \neq P(x)$). 

If two distinct roots of $P(x)-x$ are integral, we claim no twin satisfies $P(P(x))$, which is [hide=indeed true.] Let the the two fipos be $P(f_1) = f_1$ and $P(f_2) = f_2$ with $f_1 \neq f_2$. Assume otherwise, such that $P(P(t)) = t$, $P(t) = \bar{t} \neq t$. Now, from the fact, we have $t-f_1 | P(t) - P(f_1) = \bar{t} - f_1 | P(\bar{t}) - P(f_1) = t-f_1$, so $t-f_1 = \pm {\bar{t} - f_1}$ (minus case is impossible) $ = f_1 - \bar{t} \Rightarrow t + \bar{t} = 2f_1$. Proceeding analogusly, we have $t + \bar{t} = 2f_2 \neq 2f_1 = 2f_1$, a contradiction. [\/hide], so all are fipos, so again applying FTA to $P(x)-x$, we're done.

So assume at most one fipo satisfies $P(P(x)) = x$ (so there must be atleast one twin satisfying it). We now prove that any integer satisfying the condition satisfies $P(x)+x = constant$, which is [hide=indeed true] Take the twins to be $t, \bar{t}$ such that $P(t) = \bar{t} \neq t$ and $P(\bar{t}) = t$, and any one other integer $x, y = P(x)$ such that $P(P(x)) = x$ (they might not be different)

Now note that applying the fact two times gives $t-x = \pm (\bar{t}-y)$. Assume for the sake of contradiction, $t-x = \bar{t} - y$, so $t - \bar{t} = x-y$. But applying the fact again, we get $\bar{t} -x = \pm(t - y)$.  Now, the positivie sign can't happen (otherwise it would imply $(t, \bar{t}) = (x,y)$), so $t - \bar{t} = y-x = \bar{t} - t \Rightarrow t = \bar{t} \Rightarrow \text{Contradiction}$. [\/hide] So, all "pairwise" sums $P(x)+x$ is constant, so applying FTA on $G(x) := P(x)+x-constant$, we prove the claim [\/hide]

\begin{bolded} Claim-2 \end{bolded}  The problem is equivalent to proving Claim-1
[hide=Proof] Let $Q_k(x)$ denote the $k$ fold iteration of $P$, e.g $Q_3(x) = P(P(P(x)))$.  We prove that $Q_k(x) = x \Rightarrow Q_2(x) = x$. For $0<j<k$. Let $d_i = Q_{i+1} - Q_{ i}$. Now, applying the fact $k$ times, we have $|d_j| = |d_i| := d $, for any $0 < i, j <k$, so the claim is obvious. [\/hide][\/hide]
\end{solution}



\begin{solution}[by \href{https://artofproblemsolving.com/community/user/277546}{realquarterb}]
	[hide=Sketch] So the idea here is to use the lemma $a-b|P(a)-P(b)$. A simple usage of this lemma gives us $|{P(a_{i+1})-P(a_{i})}|=|{a_{i+1}-a_{i}}|$. It follows that P(X)-X+C=0 for n+1 X, which is a contradiction by the degree of the polynomial. [\/hide]
\end{solution}
*******************************************************************************
-------------------------------------------------------------------------------

\begin{problem}[Posted by \href{https://artofproblemsolving.com/community/user/89511}{dawn_pingpong}]
	This may be posted at the wrong place, so if this is posted wrongly moderator please help to move:P

$a$ is a real root of the equation $x^5 - x^3 + x-2 = 0$. Find $\lfloor a^6\rfloor$
	\flushright \href{https://artofproblemsolving.com/community/c6h463498}{(Link to AoPS)}
\end{problem}



\begin{solution}[by \href{https://artofproblemsolving.com/community/user/29428}{pco}]
	\begin{tcolorbox}This may be posted at the wrong place, so if this is posted wrongly moderator please help to move:P

$a$ is a real root of the equation $x^5 - x^3 + x-2 = 0$. Find $\lfloor a^6\rfloor$\end{tcolorbox}
Let $f(x)=x^5-x^3+x-2$

$a^5-a^3+a-2=0$ $\implies$ $a\ne 0$ and $a^6=a^4-a^2+2a=\frac{2-a}a+2a=-1+2(a+\frac 1a)$

$f'(x)=5x^4-3x^2+1=5\left(x^2-\frac 3{10}\right)^2+\frac{11}{20}>0$ and $f(x)$ is increasing
$f(1)<0$ and $f(2)>0$ and so $a\in(1,2)$ and so $a+\frac 1a\in(2,\frac 52)$ and so $a^6=-1+2(a+\frac 1a)\in(3,4)$

And so $\boxed{\left\lfloor a^6\right\rfloor=3}$
\end{solution}



\begin{solution}[by \href{https://artofproblemsolving.com/community/user/124745}{Uzbekistan}]
	Easy.

$x^5-x^3+x=2\implies x^2(x^5-x^3+x)=2x^2\implies x^7-(x^5-x^3)=2x^2\implies x^7-(2-x)=2x^2\implies x^7+x=2x^2+2\implies x(x^6+1)=2(x^2+1)>2\cdot 2x\implies x^6+1>4\implies x^6>3$ 
and  $x^5+x=2+x^3\implies \frac{x^5+x}{x^3}=\frac{2+x^3}{x^3}=x^2+\frac{1}{x^2}\geq 2\implies \frac{2+x^3}{x^3}\geq 2\implies x^3\leq 2\implies x^6\leq 4$ 
Then $3<x^6<4\implies \boxed{\left\lfloor a^{6}\right\rfloor=3} $
\end{solution}



\begin{solution}[by \href{https://artofproblemsolving.com/community/user/29428}{pco}]
	\begin{tcolorbox}$x^3\leq 2\implies x^6\leq 4$\end{tcolorbox}
What about $x^3=-5$ ?
\end{solution}



\begin{solution}[by \href{https://artofproblemsolving.com/community/user/124745}{Uzbekistan}]
	\begin{tcolorbox}[quote="Uzbekistan"]$x^3\leq 2\implies x^6\leq 4$\end{tcolorbox}
What about $x^3=-5$ ?\end{tcolorbox}
Yes you are right!  But you can see second part $x^6>3$ 
\end{solution}



\begin{solution}[by \href{https://artofproblemsolving.com/community/user/29428}{pco}]
	\begin{tcolorbox}[quote="pco"]\begin{tcolorbox}$x^3\leq 2\implies x^6\leq 4$\end{tcolorbox}
What about $x^3=-5$ ?\end{tcolorbox}
Yes you are right!  But you can see second part $x^6>3$ \end{tcolorbox}
So first part proved that $x^6\ge 0$ and second part proved that $x^6>3$. So .... what ?
\end{solution}
*******************************************************************************
-------------------------------------------------------------------------------

\begin{problem}[Posted by \href{https://artofproblemsolving.com/community/user/80177}{smart of math}]
	find the number of the third degree polynomials $P(x) = ax^3 + bx^2 + cx + d$ such that the coefficients $a,b,c,d$ belong to the set ${0,1,2,3}$ and $P(2)$ not equal $12$ .
	\flushright \href{https://artofproblemsolving.com/community/c6h463583}{(Link to AoPS)}
\end{problem}



\begin{solution}[by \href{https://artofproblemsolving.com/community/user/29428}{pco}]
	\begin{tcolorbox}find the number of the third degree polynomials $P(x) = ax^3 + bx^2 + cx + d$ such that the coefficients $a,b,c,d$ belong to the set ${0,1,2,3}$ and $P(2)$ not equal $12$ .\end{tcolorbox}
Considering $a>0$, the only such polynomials with $P(2)=12$ are obviously :
$x^3+x^2$
$x^3+2x$
$x^3+x+2$
$x^3+4$

And since we have a total of $3\times 4^3$ third degree polynomials whose all coefficients $\in\{0,1,2,3\}$, the required number is

$192-4=\boxed{188}$
\end{solution}



\begin{solution}[by \href{https://artofproblemsolving.com/community/user/80177}{smart of math}]
	\begin{tcolorbox}[quote="smart of math"]find the number of the third degree polynomials $P(x) = ax^3 + bx^2 + cx + d$ such that the coefficients $a,b,c,d$ belong to the set ${0,1,2,3}$ and $P(2)$ not equal $12$ .\end{tcolorbox}
Considering $a>0$, the only such polynomials with $P(2)=12$ are obviously :
$x^3+x^2$
$x^3+2x$
$x^3+x+2$
$x^3+4$

And since we have a total of $3\times 4^3$ third degree polynomials whose all coefficients $\in\{0,1,2,3\}$, the required number is

$192-4=\boxed{188}$\end{tcolorbox}


nice but $4 \in\{0,1,2,3\}$
\end{solution}



\begin{solution}[by \href{https://artofproblemsolving.com/community/user/29428}{pco}]
	\begin{tcolorbox}[quote="pco"]\begin{tcolorbox}find the number of the third degree polynomials $P(x) = ax^3 + bx^2 + cx + d$ such that the coefficients $a,b,c,d$ belong to the set ${0,1,2,3}$ and $P(2)$ not equal $12$ .\end{tcolorbox}
Considering $a>0$, the only such polynomials with $P(2)=12$ are obviously :
$x^3+x^2$
$x^3+2x$
$x^3+x+2$
$x^3+4$

And since we have a total of $3\times 4^3$ third degree polynomials whose all coefficients $\in\{0,1,2,3\}$, the required number is

$192-4=\boxed{188}$\end{tcolorbox}


nice but $4 \in\{0,1,2,3\}$\end{tcolorbox}
I suppose you mean $4\notin\{0,1,2,3\}$

If so, I'm sorry for my little mistake and result becomes $192-3=\boxed{189}$
\end{solution}
*******************************************************************************
-------------------------------------------------------------------------------

\begin{problem}[Posted by \href{https://artofproblemsolving.com/community/user/10045}{socrates}]
	Find all polynomials $P(x)$ such that there is some polynomial $Q(x)$ so that the following equation holds: \[P\left(x^{2} \right)+ Q\left(x \right)=P\left(x \right)+x^{5}Q\left(x \right)\]
	\flushright \href{https://artofproblemsolving.com/community/c6h467869}{(Link to AoPS)}
\end{problem}



\begin{solution}[by \href{https://artofproblemsolving.com/community/user/29428}{pco}]
	\begin{tcolorbox}Find all polynomials $P(x)$ such that there is some polynomial $Q(x)$ so that the following equation holds: \[P\left(x^{2} \right)+ Q\left(x \right)=P\left(x \right)+x^{5}Q\left(x \right)\]\end{tcolorbox}
So we are looking for polynomials $P(x)$ such that $x^5-1|P(x^2)-P(x)$

$\iff$ $P(e^{i\frac{4k\pi}5})=P(e^{i\frac{2k\pi}5})$ for $k\in\{0,1,2,3,4\}$

$\iff$ $P(e^{i\frac{2k\pi}5})=c$ for $k\in\{1,2,3,4\}$

$\iff$ $x^4+x^3+x^2+x+1|P(x)-c$

Hence the answer : $\boxed{P(x)=c+(x^4+x^3+x^2+1)A(x)}$ for any $x\in\mathbb R$ and any $A(x)\in\mathbb R[X]$ which indeed are solutions
\end{solution}



\begin{solution}[by \href{https://artofproblemsolving.com/community/user/121558}{Bigwood}]
	Maybe it is straightforward that $P(x^2)-P(x)\equiv P'(x^2)-P'(x)(mod\ x^5-1)$ when $P(x)\equiv P'(x)(mod\ x^5-1)$. Then you have only to calculate. (Hint; Consider the case of $P(x)=x^n$.)
\end{solution}



\begin{solution}[by \href{https://artofproblemsolving.com/community/user/29428}{pco}]
	\begin{tcolorbox}Maybe it is straightforward that $P(x^2)-P(x)\equiv P'(x^2)-P'(x)(mod\ x^5-1)$ when $P(x)\equiv P'(x)(mod\ x^5-1)$. Then you have only to calculate. (Hint; Consider the case of $P(x)=x^n$.)\end{tcolorbox}
Could you kindly give us your complete (up to the end of calculus) straightforward proof, please ?
\end{solution}



\begin{solution}[by \href{https://artofproblemsolving.com/community/user/121558}{Bigwood}]
	$x^{2n}-x^n\equiv x^2-x,x^4-x^2,x-x^3,x^3-x^4,0 (mod\ x^5-1)$, and the $RHS$ depends on $n (mod\ 5)$.
Let $P(x)\equiv ax^4+bx^3+cx^2+dx+e\ (\ mod\ x^5-1)$. 
$P(x^2)-P(x)\equiv (c-a)x^4+(a-b)x^3+(d-c)x^2+(b-d)x\ (mod\ x^5-1)$. Then we get $a=b=c=d$ because $P(x^2)-P(x)\equiv 0$.
Hence the result; $P(x)=R(x)(x^4+x^3+x^2+x+1)+c$.
\end{solution}
*******************************************************************************
-------------------------------------------------------------------------------

\begin{problem}[Posted by \href{https://artofproblemsolving.com/community/user/92334}{vanstraelen}]
	Given a polynomial $V(x) = x^{3}+2x^{2}+x+ c $.
a) To prove: there exists an infinite number of polynomials $x^{2}+p x +q$ with roots $\alpha$ and $\beta$, so that $V(\alpha)=V(\beta)$.
b) Calculate $p$ and $q$ to find that polynomial $x^{2}+p x +q$.
	\flushright \href{https://artofproblemsolving.com/community/c6h468092}{(Link to AoPS)}
\end{problem}



\begin{solution}[by \href{https://artofproblemsolving.com/community/user/29428}{pco}]
	\begin{tcolorbox}Given a polynomial $V(x) = x^{3}+2x^{2}+x+ c $.
a) To prove: there exists an infinite number of polynomials $x^{2}+p x +q$ with roots $\alpha$ and $\beta$, so that $V(\alpha)=V(\beta)$.
b) Calculate $p$ and $q$ to find that polynomial $x^{2}+p x +q$.\end{tcolorbox}
I suppose we consider only real roots.

$V(a)=V(b)$ $\iff$ $(a-b)((a+b)^2+2(a+b)-ab+1)=0$ and so two cases :

Either $a=b$ and so two equal roots $\iff$ $p^2-4q=0$

Either $a\ne b$ and so $p^2-4q>0$ and $p^2-2p-q+1=0$ $\iff$ $q=(p-1)^2$ with $q<\frac{p^2}4$

And so $\boxed{q=\frac {p^2}4\text{ or }\left(p\in(\frac 23,2)\text{ and }q=(p-1)^2\right)}$
\end{solution}



\begin{solution}[by \href{https://artofproblemsolving.com/community/user/92334}{vanstraelen}]
	Can you clarify: $p\in(\frac{2}{3},2) $ ?
\end{solution}



\begin{solution}[by \href{https://artofproblemsolving.com/community/user/29428}{pco}]
	\begin{tcolorbox}Can you clarify: $p\in(\frac{2}{3},2) $ ?\end{tcolorbox}
We have $q=(p-1)^2$ and $p^2-4q>0$ 

So $p^2-4(p-1)^2>0$

So $(2-p)(3p-2)>0$

So $p\in(\frac 23,2)$
\end{solution}
*******************************************************************************
-------------------------------------------------------------------------------

\begin{problem}[Posted by \href{https://artofproblemsolving.com/community/user/132970}{alis678}]
	Let $f\in \mathbb{R}[X]$ a polynomial with the degree greater than or equal to $2$. If $f$ is divided by $x+1$ we get the remainder $2$ and it is also given that $(x+1)f(x)+xf(x+3)=1,\ \forall x$. Find the remainder obtained when $f$ is divided by $x^2-x-2$.
	\flushright \href{https://artofproblemsolving.com/community/c6h468250}{(Link to AoPS)}
\end{problem}



\begin{solution}[by \href{https://artofproblemsolving.com/community/user/29428}{pco}]
	\begin{tcolorbox}Let $f\in \mathbb{R}[X]$ a polynomial with the degree greater than or equal to $2$. If $f$ is divided by $x+1$ we get the remainder $2$ and it is also given that $(x+1)f(x)+xf(x+3)=1,\ \forall x$. Find the remainder obtained when $f$ is divided by $x^2-x-2$.\end{tcolorbox}
From "If $f$ is divided by $x+1$ we get the remainder $2$", we get $f(-1)=2$
From "$(x+1)f(x)+xf(x+3)=1,\ \forall x$, we get (set $x=-1$) $f(2)=-1$

Writing $f(x)=(x^2-x-2)g(x)+ax+b$, we get :
$x=-1$ $\implies$ $-a+b=2$
$x=2$ $\implies$ $2a+b=-1$

So $a=-1$ and $b=1$ and the requested remainder is $\boxed{-x+1}$
\end{solution}



\begin{solution}[by \href{https://artofproblemsolving.com/community/user/61082}{Pain rinnegan}]
	The solution given by pco is correct, but the problem is wrong as stated. Can you notice the error?
\end{solution}



\begin{solution}[by \href{https://artofproblemsolving.com/community/user/29428}{pco}]
	\begin{tcolorbox}The solution given by pco is correct, but the problem is wrong as stated. Can you notice the error?\end{tcolorbox}
Ahhhhhh yes 
No such polynomial exists!

Just looking at the highest degree summand in $(x+1)f(x)+xf(x+3)=1$ shows it.
\end{solution}
*******************************************************************************
-------------------------------------------------------------------------------

\begin{problem}[Posted by \href{https://artofproblemsolving.com/community/user/114585}{anonymouslonely}]
	Find all polynomials such that $ f(x^{2})=f^{2}(x)+2f(x) $ and the function $ g $ defined on the real numbers and real valued such that $ g(x)=f(x) $ is bijective.
	\flushright \href{https://artofproblemsolving.com/community/c6h469554}{(Link to AoPS)}
\end{problem}



\begin{solution}[by \href{https://artofproblemsolving.com/community/user/31919}{tenniskidperson3}]
	Replace $x$ with $-x$ to see that $f^2(x)-2f(x)=f^2(-x)-2f(-x)$.  That is, $(f(x)-f(-x))(f(x)+f(-x)-2)=0$.  So one of the polynomials $f(x)-f(-x)$ or $f(x)+f(-x)-2$ has an infinite number of roots, and is thus identically 0.  It can't be $f(x)-f(-x)$ because $f$ is bijective from $\mathbb{R}$ to $\mathbb{R}$.  So that means that $f(x)+f(-x)-2=0$ for all $x$.  In particular, $f(0)=1$.  But letting $x=0$ in the original equation gives $f(0)=f^2(0)-2f(0)$ which means $1=1-2$ or $-1=1$, which is not true.  Therefore there are no such polynomials.

Unless $f^2(x)$ means $f(f(x))$ in which case the polynomial must have degree 2, but then it is still not bijective.  So there are no solutions either way you read the notation.
\end{solution}



\begin{solution}[by \href{https://artofproblemsolving.com/community/user/114585}{anonymouslonely}]
	yes...sorry.... I made a mistake. look at this now.
\end{solution}



\begin{solution}[by \href{https://artofproblemsolving.com/community/user/29428}{pco}]
	\begin{tcolorbox}Find all polynomials such that $ f(x^{2})=f^{2}(x)+2f(x) $ and the function $ g $ defined on the real numbers and real valued such that $ g(x)=f(x) $ is bijective.\end{tcolorbox}
So $f^2(x)+2f(x)=f^2(-x)+2f(-x)$ and so $(f(x)-f(-x))(f(x)+f(-x)+2)=0$ and since $f(x)$ is injective, $f(x)+f(-x)+2=0$

So $f(x)+1$ is an odd polynomial and we can write $f(x)=xh(x^2)-1$ where $h(x)\in\mathbb R[X]$

Plugging this in original equation, we get $h(x^4)=h(x^2)^2$ and so (since polynomial) $h(x^2)=h(x)^2$

This is a very classical equation whose only polynomial solutions are $h(x)=0$ and $h(x)=x^n$ (with $n\in\mathbb N\cup\{0\}$)

$h(x)=0$ implies $f(x)=-1$ which is not a solution

$h(x)=x^n$ implies $\boxed{f(x)=x^{2n+1}-1}$ which indeed is a solution
\end{solution}
*******************************************************************************
-------------------------------------------------------------------------------

\begin{problem}[Posted by \href{https://artofproblemsolving.com/community/user/29034}{newsun}]
	Find $m, n, p $ be positive integer numbers such that the polynomial $ x^{3m}-x^{3n+1}+x^{3p+2} $ divisible by $ x^2-x+1 $
	\flushright \href{https://artofproblemsolving.com/community/c6h469836}{(Link to AoPS)}
\end{problem}



\begin{solution}[by \href{https://artofproblemsolving.com/community/user/29428}{pco}]
	\begin{tcolorbox}Find $m, n, p $ be positive integer numbers such that the polynomial $ x^{3m}-x^{3n+1}+x^{3p+2} $ divisible by $ x^2-x+1 $\end{tcolorbox}
The two roots of $x^2-x+1$ are $e^{i\frac{\pi}3}$ and $e^{-i\frac{\pi}3}$

So we want : 
1) $e^{im\pi}-e^{i\frac{\pi}3+in\pi}$ $+e^{i\frac{2\pi}3+ip\pi}=0$
which implies  $(-1)^m-(-1)^ne^{i\frac{\pi}3}$ $+(-1)^pe^{i\frac{2\pi}3}=0$ and so $m\equiv n\equiv p\pmod 2$


2) $e^{-im\pi}-e^{-i\frac{\pi}3-in\pi}$ $+e^{-i\frac{2\pi}3-ip\pi}=0$
which implies  $(-1)^m-(-1)^ne^{-i\frac{\pi}3}$ $+(-1)^pe^{-i\frac{2\pi}3}=0$ and so again $m\equiv n\equiv p\pmod 2$

Hence the answer $\boxed{m\equiv n\equiv p\pmod 2}$
\end{solution}



\begin{solution}[by \href{https://artofproblemsolving.com/community/user/140796}{mathbuzz}]
	put x=-w and x=-w^2 in the polynomial and try
\end{solution}
*******************************************************************************
-------------------------------------------------------------------------------

\begin{problem}[Posted by \href{https://artofproblemsolving.com/community/user/89198}{chaotic_iak}]
	Let $P$ be a polynomial with real coefficients. Find all functions $f : \mathbb{R} \rightarrow \mathbb{R}$ such that there exists a real number $t$ such that
\[f(x+t) - f(x) = P(x)\]
for all $x \in \mathbb{R}$.
	\flushright \href{https://artofproblemsolving.com/community/c6h470219}{(Link to AoPS)}
\end{problem}



\begin{solution}[by \href{https://artofproblemsolving.com/community/user/29428}{pco}]
	\begin{tcolorbox}Let $P$ be a polynomial with real coefficients. Find all functions $f : \mathbb{R} \rightarrow \mathbb{R}$ such that there exists a real number $t$ such that
\[f(x+t) - f(x) = P(x)\]
for all $x \in \mathbb{R}$.\end{tcolorbox}
Choose any $t\ne 0$ and define $f(x)$ as any function you want over $[0,t)$ and by the induction formulas :
$f(x+t)=f(x)+P(x)$
$f(x-t)=f(x)-P(x)$

And you get all the solutions, built piece per piece.
\end{solution}



\begin{solution}[by \href{https://artofproblemsolving.com/community/user/96532}{dgrozev}]
	\begin{tcolorbox}
Choose any $t\ne 0$ and define $f(x)$ as any function you want over $[0,t)$ and by the induction formulas :
$f(x+t)=f(x)+P(x)$
$f(x-t)=f(x)-P(x)$

And you get all the solutions, built piece per piece.\end{tcolorbox}
I think it is just a typo, but it should be:
...
$f(x-t)=f(x)-P(x-t)$.
\end{solution}
*******************************************************************************
-------------------------------------------------------------------------------

\begin{problem}[Posted by \href{https://artofproblemsolving.com/community/user/127400}{MSMS}]
	Find a polynomial satisfying:
$\left\{ \begin{array}{l}
 Q\left( 2 \right) = 12 \\ 
 Q\left( {{x^2}} \right) = {x^2}\left( {{x^2} + 1} \right)Q\left( x \right),\forall x \in R \\ 
 \end{array} \right.$
Thanks you!
	\flushright \href{https://artofproblemsolving.com/community/c6h470956}{(Link to AoPS)}
\end{problem}



\begin{solution}[by \href{https://artofproblemsolving.com/community/user/29428}{pco}]
	\begin{tcolorbox}Find a polynomial satisfying:
$\left\{ \begin{array}{l}
 Q\left( 2 \right) = 12 \\ 
 Q\left( {{x^2}} \right) = {x^2}\left( {{x^2} + 1} \right)Q\left( x \right),\forall x \in R \\ 
 \end{array} \right.$
Thanks you!\end{tcolorbox}
So $Q(x)$ is an even polynomial with degree 4 and so $Q(x)=ax^4+bx^2+c$

And so $ax^8+bx^4+c=x^2(x^2+1)(ax^4+bx^2+c)$ and so $c=0$ and $b=-a$ and we get $Q(x)=ax^4-ax^2$

And $Q(2)=12$ gives the unique solution $\boxed{Q(x)=x^4-x^2}$
\end{solution}



\begin{solution}[by \href{https://artofproblemsolving.com/community/user/127400}{MSMS}]
	\begin{tcolorbox}So  is an even polynomial with degree 4 and so $Q\left( x \right) = a{x^4} + b{x^2} + c$\end{tcolorbox}
Pco! Why? Please explain it?
\end{solution}



\begin{solution}[by \href{https://artofproblemsolving.com/community/user/29428}{pco}]
	\begin{tcolorbox}[quote]So  is an even polynomial with degree 4 and so $Q\left( x \right) = a{x^4} + b{x^2} + c$\end{tcolorbox}
Pco! Why? Please explain it?\end{tcolorbox}
1) Replacing $x$ by $-x$ in second equation, we get $x^2(x^2+1)Q(x)=x^2(x^2+1)Q(-x)$ and so $Q(x)$ is even

2) Considering that $Q(x)$ is not constant (since $Q(x)=2$ $\forall x$ is not a solution) and so that degree of $Q(x)$ is $n>0$ and looking at second equation, we get degree of LHS is $2n$ while degree of RHS is $n+4$
So $2n=n+4$ and so $n=4$

So  $Q(x)$ is an even polynomial with degree 4
\end{solution}
*******************************************************************************
-------------------------------------------------------------------------------

\begin{problem}[Posted by \href{https://artofproblemsolving.com/community/user/43631}{mathwizarddude}]
	Let $f(x),g(x)$ be nonzero polynomials, with $f(x^2+x+1)=f(x)g(x)$. Show that $f(x)$ has even degree.
	\flushright \href{https://artofproblemsolving.com/community/c6h471552}{(Link to AoPS)}
\end{problem}



\begin{solution}[by \href{https://artofproblemsolving.com/community/user/29428}{pco}]
	\begin{tcolorbox}Let $f(x),g(x)$ be nonzero polynomials, with $f(x^2+x+1)=f(x)g(x)$. Show that $f(x)$ has even degree.\end{tcolorbox}
If $f$ has an odd degree, then it has at least one real root $u$.
Considering then the increasing sequence $x_0=u$ and $x_{n+1}=x_n^2+x_n+1$, we see that $f(x_n)=0$ $\forall x$ and so that $f(x)$ has infinitely many distinct real roots, which is impossible since $f(x)$ is not the zero polynomial.
Q.E.D.
\end{solution}



\begin{solution}[by \href{https://artofproblemsolving.com/community/user/64716}{mavropnevma}]
	And to show that there exist such polynomials with even degree larger than $0$, it is enough to see that for $f(x) = x^2+1$, $g(x) = f(x+1)$, we have $f(x^2+x+1) = (x^2+1)((x+1)^2 + 1) = f(x)g(x)$.
\end{solution}
*******************************************************************************
-------------------------------------------------------------------------------

\begin{problem}[Posted by \href{https://artofproblemsolving.com/community/user/145455}{faithjeff1993}]
	let f(x) and g(x) be quadratic function with real coefficient, for x>0,if g(x) is a integer, then f(x) is a integer, prove that there exist integers M,N, such that f(x)=Mg(x)+N
	\flushright \href{https://artofproblemsolving.com/community/c6h472459}{(Link to AoPS)}
\end{problem}



\begin{solution}[by \href{https://artofproblemsolving.com/community/user/145455}{faithjeff1993}]
	who can help me?
\end{solution}



\begin{solution}[by \href{https://artofproblemsolving.com/community/user/29428}{pco}]
	\begin{tcolorbox}let f(x) and g(x) be quadratic function with real coefficient, for x>0,if g(x) is a integer, then f(x) is a integer, prove that there exist integers M,N, such that f(x)=Mg(x)+N\end{tcolorbox}
Nobody can help you since your problem is obviously wrong.

Choose as counter-example $f(x)=x^2+x+1$ and $g(x)=x^2$
\end{solution}



\begin{solution}[by \href{https://artofproblemsolving.com/community/user/69901}{dinoboy}]
	@pco

Notice that $x = \sqrt{2}$ has $g(x)$ is an integer while $f(x)$ is not, so your counterexample does not work.
\end{solution}



\begin{solution}[by \href{https://artofproblemsolving.com/community/user/29428}{pco}]
	@dinoboy

You are quite right.

Sorry both for this wrong post.  :blush:
\end{solution}



\begin{solution}[by \href{https://artofproblemsolving.com/community/user/69901}{dinoboy}]
	First of all the condition $x > 0$ is useless.  It only introduces the complication that there are finitely many exceptions when we consider all real $x$.

First remark multiplying $g(x)$ by $-1$ does not change the result, nor does adding an integer to it (applying the same transformations on $f$ has the same result of not changing the truthfulness of the result). Furthermore, if we change $f(x), g(x)$ both into $f(x-a), g(x-a)$ for a real number $a$ the truthfulness is not changed.

Hence we shift $g$ to be symmetric about the origin and be of the form $g(x) = ax^2 + b$ for $a,b \ge 0$ and we can assume $f$ is a quadratic with positive leading coefficient which has no real roots. It follows for all sufficiently large $c$ a positive integer, we have $g(\pm \sqrt{(c-b)\/a})$ are integers and hence $f(\pm \sqrt{(c-b)\/a})$ are integers as well. Let $x_c = \sqrt{(c-b)\/a}$ and $y_c = f(x_c), z_c = f(-x_c)$ when $c$ is big enough. Let $f(x) = dx^2 + ex + f$. Remark that $z_n - y_n$ is always an integer.
But $z_n - y_n = 2ex_n$. Hence for all $x_n$, we need $2ex_n$ to be an integer. But this is obviously false unless $e \neq 0$ because $2ex_{n+1} - 2ex_n$ approaches $0$ as $n$ grows large, hence $e=0$.

Therefore we have reduced the problem to $g(x) = ax^2 + b, f(x) = dx^2 + f$. We need for all big enough integers $c$ that $\frac{d(c-b)}{a} + f$ is an integer. This means $\frac{d}{a}$ is an integer. Hence it suffices for $\frac{-db}{a} + f$ is an integer. By shifting $f(x)$ appropriately we can get it to equal $0$, as $f = (d\/a) \cdot b$. It follows $f(x) = g(x) \cdot (d\/a)$ where $d\/a$ is an integer, and hence we are done.

EDIT : Thanks bzprules, fixed the typo.
\end{solution}



\begin{solution}[by \href{https://artofproblemsolving.com/community/user/145455}{faithjeff1993}]
	elegant!
many thx!
\end{solution}



\begin{solution}[by \href{https://artofproblemsolving.com/community/user/60729}{GlassBead}]
	Oh, you have one small typo:
\begin{tcolorbox}
But $z_n - y-n = 2ex_n$. Hence for all $x_n$, we need $2ex_n$ to be an integer.
\end{tcolorbox}
This should be $y_n$.
\end{solution}
*******************************************************************************
-------------------------------------------------------------------------------

\begin{problem}[Posted by \href{https://artofproblemsolving.com/community/user/132519}{siavosh}]
	assume $ P(x)= x^3+ax^2+bx+c$ which $ ab=9c ,   b<0$ show $P$ have three different root in real number.
	\flushright \href{https://artofproblemsolving.com/community/c6h473926}{(Link to AoPS)}
\end{problem}



\begin{solution}[by \href{https://artofproblemsolving.com/community/user/29428}{pco}]
	\begin{tcolorbox}assume $ P(x)= x^3+ax^2+bx+c$ which $ ab=9c ,   b<0$ show $P$ have three different root in real number.\end{tcolorbox}
let $P(x)=x^3+ax^2+bx+\frac{ab}9$

Then $P(\frac {x-a}3)=\frac{x^3+3(3b-a^2)x+2a(a^2-3b)}{27}$

Applying Cardano's method to $x^3+3(3b-a^2)x+2a(a^2-3b)=0$ with $p=3(3b-a^2)$ and $q=2a(a^2-3b)$, we get $27q^2+4p^3=324b(a^2-3b)^2<0$

hence the result.
\end{solution}
*******************************************************************************
-------------------------------------------------------------------------------

\begin{problem}[Posted by \href{https://artofproblemsolving.com/community/user/127783}{Sayan}]
	Consider the equation $x^5+x=10$. Show that
(a) the equation has only one real root;
(b) this root lies between $1$ and $2$;
(c) this root must be irrational.
	\flushright \href{https://artofproblemsolving.com/community/c6h475185}{(Link to AoPS)}
\end{problem}



\begin{solution}[by \href{https://artofproblemsolving.com/community/user/29428}{pco}]
	\begin{tcolorbox}Consider the equation $x^5+x=10$. Show that
(a) the equation has only one real root;
(b) this root lies between $1$ and $2$;
(c) this root must be irrational.\end{tcolorbox}
$f(x)=x^5+x-10$ is stricly increasing and so at most one real root.
$f(1)<0$ and $f(2)>0$ and so at least one real root in $(1,2)$

So exactly one real root $r$ and $r\in(1,2)$

If $r=\frac pq$ with $q>0$ and $\gcd(|p|,q)=1$, then $p^5+pq^4-10q^5=0$ and so $q|p^5$ and so $q=1$ and $r\in\mathbb Z$, impossible since $1<r<2$
\end{solution}



\begin{solution}[by \href{https://artofproblemsolving.com/community/user/123851}{ctumeo}]
	a) $x^5=-x+10$
$y=x^5$ and $y=-x+10$ intersect  only in a point
b) Assign values and is easy to verify
c) Rational roots lies between divisors of $10$. Easy to verify that can't be
\end{solution}



\begin{solution}[by \href{https://artofproblemsolving.com/community/user/140796}{mathbuzz}]
	let $f(x)=x^5+x-10$ , then $f '(x)=5x^4+1>0$ for all $x \in R$ .so , the function is strictly increasing over $R$
hence there exist at most one real root. $f(1)=-8$ ,$f(2)=24$ , so , there exists some $c \in (1,2) $ such that        $f(c)=0$. so , there exists only one real root which is in $(1,2)$. if possible ,let the root be rational. then the root must be an integer which lies between 1 and 2  ,which is not possible. hence the root is irrational :D  
\end{solution}
*******************************************************************************
-------------------------------------------------------------------------------

\begin{problem}[Posted by \href{https://artofproblemsolving.com/community/user/87452}{Agung}]
	Define the functions \[ f(x) = x^5+5x^4+5x^3+5x^2+1\] \[g(x) =x^5+5x^4+3x^3-5x^2-1\] 
Find all prime numbers $p$ for which there exists a natural number $ 0 < x < p $, such that both $f(x)$ and $g(x)$ are divisible by $p$, and for each such $p$, find all such $x$.
	\flushright \href{https://artofproblemsolving.com/community/c6h476005}{(Link to AoPS)}
\end{problem}



\begin{solution}[by \href{https://artofproblemsolving.com/community/user/29428}{pco}]
	\begin{tcolorbox}Define the functions \[ f(x) = x^5+5x^4+5x^3+5x^2+1\] \[g(x) =x^5+5x^4+3x^3-5x^2-1\] 
Find all prime numbers $p$ for which there exists a natural number $ 0 < x < p $, such that both $f(x)$ and $g(x)$ are divisible by $p$, and for each such $p$, find all such $x$.\end{tcolorbox}
$p|f(x)$ and $p|g(x)$ imply $p|f(x)+g(x)=2x^3(x+1)(x+4)$ 
$p=2$ is not a solution since both $f(x)$ and $g(x)$ are odd when $x\in\mathbb N$ and since $x\not\equiv 0\pmod p$, we get :

either $x\equiv -1\pmod p$ and then $f(x)\equiv 5\pmod p$ and so $p=5$, which indeed is a solution and so $x=4$

either $x\equiv -4\pmod p$ and then $f(x)\equiv 17\pmod p$ and so $p=17$, which indeed is a solution and so $x=13$

Hence the result $\boxed{(x,p)\in\{(4,5),(13,17)\}}$
\end{solution}
*******************************************************************************
-------------------------------------------------------------------------------

\begin{problem}[Posted by \href{https://artofproblemsolving.com/community/user/122611}{oty}]
	Find all polynomials $P\in \mathbb{K}[X]$ s.t $P'$ divide $P$ .
	\flushright \href{https://artofproblemsolving.com/community/c6h476428}{(Link to AoPS)}
\end{problem}



\begin{solution}[by \href{https://artofproblemsolving.com/community/user/76247}{yugrey}]
	Well, all the roots of P' must be roots of P, and thus double roots of P.

Thus, the answer is all polynomials whose roots all have multiplicity greater than one, and only those.

I believe that by K(x) you mean C(x).
\end{solution}



\begin{solution}[by \href{https://artofproblemsolving.com/community/user/29428}{pco}]
	\begin{tcolorbox}Find all polynomials $P\in \mathbb{K}[X]$ s.t $P'$ divide $P$ .\end{tcolorbox}
$P(x)$ is not a constant polynomial in order sentence "$P'$ divides $P$" being meaningful.

So $P(x)=(ax+b)P'(x)$ with $a\ne 0$ (looking at degrees of $P$ and $P'$) and so $\frac{P'}P=\frac 1{ax+b}$ for all $x$ but roots of $P$

This last equation implies $P=\alpha(x+c)^{\frac 1a}$

Hence the answer : $\boxed{P(x)=a(x+b)^n}$ for any $n\in\mathbb N$

@yugrey : your condition is necessary, but not sufficient. Choose for example $P(x)=x^2(x+1)^2$ which fits your requirements but $P'(x)=2x(x+1)(2x+1)$ does not divide $P(x)$
\end{solution}
*******************************************************************************
-------------------------------------------------------------------------------

\begin{problem}[Posted by \href{https://artofproblemsolving.com/community/user/29034}{newsun}]
	Prove that $p(x)=x^4+ax^3+bx^2+cx+d >0 $ forall $ x \in \mathbb{R}$ then p(x) can be reprsented as sum of two polynomials of degrê 2

[color=#FF0000][Mod: Give your topics a better title than "help."][\/color]
	\flushright \href{https://artofproblemsolving.com/community/c6h476649}{(Link to AoPS)}
\end{problem}



\begin{solution}[by \href{https://artofproblemsolving.com/community/user/29428}{pco}]
	\begin{tcolorbox}Prove that $p(x)=x^4+ax^3+bx^2+cx+d >0 $ forall $ x \in \mathbb{R}$ then p(x) can be reprsented as sum of two polynomials of degrê 2\end{tcolorbox}
Certainly not : a sum of two polynomials of degree 2 is at most of degree 2.
\end{solution}



\begin{solution}[by \href{https://artofproblemsolving.com/community/user/29034}{newsun}]
	I mean sum of square
\end{solution}



\begin{solution}[by \href{https://artofproblemsolving.com/community/user/29034}{newsun}]
	Please help me pco!
\end{solution}



\begin{solution}[by \href{https://artofproblemsolving.com/community/user/29428}{pco}]
	\begin{tcolorbox}Prove that $p(x)=x^4+ax^3+bx^2+cx+d >0 $ forall $ x \in \mathbb{R}$ then p(x) can be reprsented as sum of squares of two polynomials of degrê 2\end{tcolorbox}
Since $P(x)>0$ $\forall x$, we can write $P(x)=(x-z_1)(x-\overline{z_1})(x-z_2)(x-\overline{z_2})$ for some $z_1,z_2\in\mathbb  C\setminus\mathbb R$

Writing then $P(x)=A(x)^2+B(x)^2$, we get :
$A(z_1)=\pm iB(z_1)$
$A(z_2)=\pm iB(z_2)$

Choosing for example $\pm=+1$ in both cases, we get $A(z_1)-iB(z_1)=A(z_2)-iB(z_2)=0$ and so $A(x)-iB(x)=k(x-z_1)(x-z_2)$

Let then $z_1=Re(z_1)+i Im(z_1)$ and $z_2=Re(z_2)+i Im(z_2)$ and we get :

So $A(x)=k(x-Re(z_1))(x-Re(z_2))-Im(z_1)Im(z_2)$
And $B(x)=-k(x-Re(z_1))Im(z_2)-k(x-Re(z_2))Im(z_1)$

Setting $k=1$, we indeed get 

$P(x)=((x-Re(z_1))(x-Re(z_2))-Im(z_1)Im(z_2))^2$ $+((x-Re(z_1))Im(z_2)+(x-Re(z_2))Im(z_1))^2$

It's easy to check back this equality by setting $x=z_1$ and $x=z_2$ (and so also $x=\overline{z_1}$ and $x=\overline{z_2}$) in it.
Hence the result
\end{solution}



\begin{solution}[by \href{https://artofproblemsolving.com/community/user/29034}{newsun}]
	You are so kind. Thanks a lots, pco!
\end{solution}



\begin{solution}[by \href{https://artofproblemsolving.com/community/user/29428}{pco}]
	\begin{tcolorbox}You are so kind. Thanks a lots, pco!\end{tcolorbox}
Thanks

But you should just notice that in my proof, one polynomial is of degree $2$ and the other is of degree $1$. So I did not exactly answered your problem (I considered that the problem asked for degrees $\le 2$).

I did not try solving for both polynomial having same degree $2$
\end{solution}



\begin{solution}[by \href{https://artofproblemsolving.com/community/user/9882}{Virgil Nicula}]
	\begin{bolded}See\end{bolded} PP14\end{underlined} from [url=http://www.artofproblemsolving.com\/blog\/64515][color=darkred]\begin{bolded}here\end{bolded}[\/color][\/url].
\end{solution}



\begin{solution}[by \href{https://artofproblemsolving.com/community/user/29428}{pco}]
	\begin{tcolorbox}\begin{bolded}See\end{bolded} PP14\end{underlined} from [url=http://www.artofproblemsolving.com\/blog\/64515][color=darkred]\begin{bolded}here\end{bolded}[\/color][\/url].\end{tcolorbox}
That is my solution and does not show that the degree of both polynomials is $2$.
\end{solution}



\begin{solution}[by \href{https://artofproblemsolving.com/community/user/9882}{Virgil Nicula}]
	\begin{bolded}@ pco\end{bolded} That is a classical proof and it appears and in my book "\begin{bolded}Numere Complexe, Ed. Scorpion 7, Bucuresti, 1994\end{underlined}\end{bolded}". 
We can prove easily that both polynomials $m$ and $n$ (for what $p=m^2+n^2$ ) have same degree $s$ , where the degree of $p$ is $n=2s$ . A polynomial with real coefficients and odd degree has at least a real root. If a polynomial has real coefficients, then for any $z\in\mathbb C$ we have $p(z)=0\iff p(\overline z)=0$ . If $p(x)\ge 0$ for any $x\in\mathbb R$ , then each its real root (if exists) has even multiplicity. Thus can suppose w.l.o.g. $p(x)>0$ for any $x\in\mathbb R$ a.s.o.

\begin{bolded}Application\end{underlined}.\end{bolded} If $p\in\mathbb R[X]$ such that  for any $x\in\mathbb R$ we have $p(x)\ge 0$ , then for any matrix $A\in\mathbb M_n(\mathbb R)$ we have $\det p(A)\ge 0$ .

\begin{bolded}Proof\end{underlined}.\end{bolded} Indeed, from upper property get that exist $\{m.n\}\subset\mathbb R[X]$ so that $p=m^2+n^2$ . Hence and $p(A)=m^2(A)+n^2(A)$ . Since the matrices $m(A)$ , $n(A)$ are permutably, obtain that $p(A)=[m(A)+i\cdot n(A)]\cdot [m(A)-i\cdot n(A)]$ . Denote $U=m(A)+i\cdot n(A)$ , where $U\in\mathrm M_n(\mathbb R)$ . Observe that $\overline U=m(A)-i\cdot n(A)$ and $\det\overline U=\overline {\det U}$ . In conclusion, $p(A)=U\cdot\overline U$ and $\det p(A)=\det \left(U\cdot\overline U\right)=$ $\det U\cdot\det \overline U=$ $\det U\cdot\overline{\det U}=$ $|\det U|^2\ge 0$ , i.e. $\det p(A)\ge 0$ .
\end{solution}



\begin{solution}[by \href{https://artofproblemsolving.com/community/user/31919}{tenniskidperson3}]
	Pco, to finish the proof, I'm pretty sure all you have to do is say that if $P(x)$ and $Q(x)$ work but $P(x)$ has degree less than 2, then consider the polynomials $\frac{P(x)+Q(x)}{\sqrt{2}}$ and $\frac{Q(x)-P(x)}{\sqrt{2}}$.  Is that right?
\end{solution}
*******************************************************************************
-------------------------------------------------------------------------------

\begin{problem}[Posted by \href{https://artofproblemsolving.com/community/user/88852}{myceliumful}]
	Determine all polynomials $ P(x) $ with real coefficients such that $ P(x^2)=P^2(x) $.
	\flushright \href{https://artofproblemsolving.com/community/c6h476928}{(Link to AoPS)}
\end{problem}



\begin{solution}[by \href{https://artofproblemsolving.com/community/user/29428}{pco}]
	\begin{tcolorbox}Determine all polynomials $ P(x) $ with real coefficients such that $ P(x^2)=P^2(x) $.\end{tcolorbox}
$P(x)=0$ $\forall x$ is a solution. Let us from now look only for non zero polynomials.

$P(x)=x^nP_1(x)$ for some $n\in\mathbb N\cup\{0\}$ and $P_1(0)\ne 0$

The equation becomes then $P_1(x^2)=P_1^2(x)$ and so $P_1(x)$ is either odd, either even and so $P_1(x)$ is even (since $P_1(0)\ne 0$) and so $P_1(x)=P_2(x^2)$

The equation becomes then $P_2(x^4)=P_2^2(x^2)$ and so $P_2(x^2)=P_2^2(x)$ and so $P_2(x)$ is even and $P_2(x)=P_3(x^2)$

and since $P_k(0)\ne 0$, the only solution at the end of this process is $P_k(x)=1$ $\forall x$

\begin{bolded}Hence the answers\end{underlined}\end{bolded} :
$P(x)=0$ $\forall x$
$P(x)=x^n$ $\forall x$ and for any integer $n\ge 0$
\end{solution}
*******************************************************************************
-------------------------------------------------------------------------------

\begin{problem}[Posted by \href{https://artofproblemsolving.com/community/user/88852}{myceliumful}]
	Determine all polynomials $ P(x) $ with real coefficients such that $ P(x^2)=P(x)P(x+1) $.
	\flushright \href{https://artofproblemsolving.com/community/c6h476929}{(Link to AoPS)}
\end{problem}



\begin{solution}[by \href{https://artofproblemsolving.com/community/user/29428}{pco}]
	\begin{tcolorbox}Determine all polynomials $ P(x) $ with real coefficients such that $ P(x^2)=P(x)P(x+1) $.\end{tcolorbox}
The only constant solutions are $P(x)=0$ $\forall x$ and $P(x)=1$ $\forall x$. Let us from now look only for non constant polynomials.
$x\in\mathbb C$ root $\implies$ $x^2$ root too and so the only complex roots are $0$ or have modulus $1$.
$x\in\mathbb C$ root $\implies$ $(x-1)^2$ root too and so (see previous line) $(x-1)^2=0$ or $|x-1|=1$

So real roots can only be $0,1$ and non real roots are such that $|z|=|z-1|=1$ and so $z=e^{\pm i\frac{\pi}3}$ but then $(z-1)^2$ is no longer a root.

So $P(x)=ax^n(x-1)^m$  and, plugging back in original equation, we get $a=1$ and $m=n$

Hence the solutions :
$P(x)=0$ $\forall x$
$P(x)=x^n(x-1)^n$ $\forall x$ and for any integer $n\ge 0$
\end{solution}
*******************************************************************************
-------------------------------------------------------------------------------

\begin{problem}[Posted by \href{https://artofproblemsolving.com/community/user/88852}{myceliumful}]
	Suppose $ P(x) $ is a polynomial such that $ P(x-1)+P(x+1)=2P(x) $ for all real $ x $.
Prove that $ P(x) $ has degree at most $ 1 $.
	\flushright \href{https://artofproblemsolving.com/community/c6h476930}{(Link to AoPS)}
\end{problem}



\begin{solution}[by \href{https://artofproblemsolving.com/community/user/29428}{pco}]
	\begin{tcolorbox}Suppose $ P(x) $ is a polynomial such that $ P(x-1)+P(x+1)=2P(x) $ for all real $ x $.
Prove that $ P(x) $ has degree at most $ 1 $.\end{tcolorbox}
Immediate induction gives $P(x+n)=nP(x+1)-(n-1)P(x)$ $\forall$ integer $n\ge 0$

And so $\frac{P(x+n)}n=P(x+1)-\frac{n-1}nP(x)$

Setting $n\to +\infty$ in the above line, we get that $RHS$ has a finite limit and so $LHS$ must also have and so degree of $P(x)$ is at most $1$
Q.E.D.
\end{solution}
*******************************************************************************
-------------------------------------------------------------------------------

\begin{problem}[Posted by \href{https://artofproblemsolving.com/community/user/143628}{MANMAID}]
	let p(x) be a polynomial in R such that p(x)>=0 for all real values of x ,prove that 
 p(x)=q1(x)^2+q2(x)^2+....+qn(x)^2
  for some polynoial q1,q2,...qn (qk a polynomial, k is in suffix)
	\flushright \href{https://artofproblemsolving.com/community/c6h478007}{(Link to AoPS)}
\end{problem}



\begin{solution}[by \href{https://artofproblemsolving.com/community/user/29428}{pco}]
	\begin{tcolorbox}let p(x) be a polynomial in R such that p(x)>=0 for all real values of x ,prove that 
 p(x)=q1(x)^2+q2(x)^2+....+qn(x)^2
  for some polynoial q1,q2,...qn (qk a polynomial, k is in suffix)\end{tcolorbox}
Already posted.

1) If $P(x)$ is a non negative constant, the result is immediate.

2) If $P(x)$ has no real root :
 $P(x)$  has an even degree $2n$ and $2n$ complex roots $\{z_i\}_{i=1}^n$ and $\{\overline{z_i}\}_{i=1}^n$
So that $P(x)=a\prod_i(x-z_i)\prod_i(x-\overline{z_i})$ with $a>0$

Then $P(x)=\left(\sqrt a\frac{\prod_i(x-z_i)+\prod_i(x-\overline{z_i})}2\right)^2$ $+\left(i\sqrt a\frac{\prod_i(x-z_i)-\prod_i(x-\overline{z_i})}2\right)^2$

And it's easy to see that the two polynomials inside parenthesis $\in\mathbb R[X]$ since they both are equal to their conjugate.

3)  If $P(x)$ has some real roots.
Then all real roots have an even multiplicity and we can write $P(x)=A(x)^2Q(x)$ where $Q(x)>0$ has no real root and so, looking at previous paragraphs, is sum of two squares.
And so is $P(x)$
\end{solution}
*******************************************************************************
-------------------------------------------------------------------------------

\begin{problem}[Posted by \href{https://artofproblemsolving.com/community/user/72957}{JSGandora}]
	Find all polynomials $P(x,y)$ in two variables such that for any $x$ and $y$, $P(x+y,y-x)=P(x,y)$.
	\flushright \href{https://artofproblemsolving.com/community/c6h478931}{(Link to AoPS)}
\end{problem}



\begin{solution}[by \href{https://artofproblemsolving.com/community/user/143628}{MANMAID}]
	note that P(x,y)=P(2y,-2x)=....=p(16x,16y)        .....(1)

  also note that P(x,y) has no real roots. 
  
expand P(x,y) ,then we get all (i+j)=0 where i is power of x, j is power of y ,as P(x,y) is a polynomial then i>=0 ,j>=0  sow get i=j=0 

                               hence P(x,y)=c      ,  where c is a constant.
\end{solution}



\begin{solution}[by \href{https://artofproblemsolving.com/community/user/69901}{dinoboy}]
	Why does $P(x,y)$ have no real roots? And why does $i+j=0$?
\end{solution}



\begin{solution}[by \href{https://artofproblemsolving.com/community/user/143628}{MANMAID}]
	\begin{tcolorbox}Why does $P(x,y)$ have no real roots? And why does $i+j=0$?\end{tcolorbox}
if you expand P(x,y) & P(16x,16y) you can get this result by comparing both equation 
   
      check (a,b) be real roots of the function, which does not exist.
\end{solution}



\begin{solution}[by \href{https://artofproblemsolving.com/community/user/69901}{dinoboy}]
	You're going to have to elaborate a lot more because I'm not seeing why...
And also there are possibly real roots of the polynomial, for instance when $P(x,y) = 0$
\end{solution}



\begin{solution}[by \href{https://artofproblemsolving.com/community/user/29428}{pco}]
	\begin{tcolorbox}Find all polynomials $P(x,y)$ in two variables such that for any $x$ and $y$, $P(x+y,y-x)=P(x,y)$.\end{tcolorbox}
Setting $x\to x+y$ and $y\to x-y$ in functional equation, we get $P(2x,-2y)=P(x+y,x-y)=P(x,y)$ and so So $P(x,y)=P(\frac x{2^n},\frac y{(-2)^n})$

Setting $n\to +\infty$ in this equation and using continuity (since polynomial), we get $\boxed{P(x,y)=a}=P(0,0)$ which indeed is a solution
\end{solution}



\begin{solution}[by \href{https://artofproblemsolving.com/community/user/64716}{mavropnevma}]
	In fact, setting $x\to x+y$ and $y\to x-y$ in the functional equation, we get $P(2x,-2y)=P(x+y,x-y)=P(y,x)$, but this does not matter, since in fact by iterating we can obtain $P(x,y) = P(16x,16y)$, with same continuation.
\end{solution}
*******************************************************************************
-------------------------------------------------------------------------------

\begin{problem}[Posted by \href{https://artofproblemsolving.com/community/user/144185}{drEdrE}]
	Let $ f:\mathbb{R} \rightarrow \mathbb{R}$ be a function such that $f=ax^2+bx+c$. Prove that  $|A| \neq 3 $, where $A =\{x\in\mathbb{R} | f(f(x))=x\}$ . ( I denoted |A| the number of elements from A.)
	\flushright \href{https://artofproblemsolving.com/community/c6h478974}{(Link to AoPS)}
\end{problem}



\begin{solution}[by \href{https://artofproblemsolving.com/community/user/140796}{mathbuzz}]
	if i am not misreading it , it is easy
note that f(f(x))-x=0 is a 4th degree equation. with all coefficients. being real ( it is quite clear from the fact that the domain of f and range of f are R ,according to the problem)so, if n(A)=3 , then , there are 3 real roots and another non real one. then the sum of the roots would be non real ,which leads to a clear contradiction. 
\end{solution}



\begin{solution}[by \href{https://artofproblemsolving.com/community/user/144185}{drEdrE}]
	but what if the 4th root is equal to the 3rd, or to the 2nd\/ 1st ?
you must prove that the multiplicity order of each root is 1
\end{solution}



\begin{solution}[by \href{https://artofproblemsolving.com/community/user/29428}{pco}]
	\begin{tcolorbox}Let $ f:\mathbb{R} \rightarrow \mathbb{R}$ be a function such that $f=ax^2+bx+c$. Prove that  $|A| \neq 3 $, where $A =\{x\in\mathbb{R} | f(f(x))=x\}$ . ( I denoted |A| the number of elements from A.)\end{tcolorbox}
If $a=0$, then $f(f(x))-x=0$ $\iff$ $(b+1)((b-1)x+c)=0$ and $|A|\in\{0,1,+\infty\}$ and so $|A|\ne 3$
If $a\ne 0$ :

Let $B=\{x\in\mathbb R$ such that $f(x)=x\}$. $|B|\le 2$
Let $C=\{x\in\mathbb R$ such that $f(f(x))=x$ and $f(x)\ne x\}$. If $u\in C$, then $f(u)\ne u\in C$ and so $|C|$ is finite and even
$B\cap C=\emptyset$ and $B\cup C=A$ and so $|A|=|B|+|C|$

If $|C|=0$, then $|A|=|B|\le 2$
If $|C|\ne 0$, let $a,b=f(a)\ne a\in C$ : $f(a)-a=b-a$ and $f(b)-b=a-b$ and so $f(x)-x$ is a degree 2 polynomial and changes its sign over $\mathbb R$ and so $|B|=2$ and so $|A|$ is even

Hence the result
\end{solution}
*******************************************************************************
-------------------------------------------------------------------------------

\begin{problem}[Posted by \href{https://artofproblemsolving.com/community/user/145866}{krawatte}]
	Find all polynomials $P(x)=a_nx^n+a_{n-1}x^{n-1}+...+a_1x+a_0$ with \begin{bolded}integer\end{bolded} coefficients such that the roots of $P(x)$ are $a_{n-1}, a_{n-2}^2,...,a_1^{n-1},a_0^n$.
	\flushright \href{https://artofproblemsolving.com/community/c6h479084}{(Link to AoPS)}
\end{problem}



\begin{solution}[by \href{https://artofproblemsolving.com/community/user/29428}{pco}]
	\begin{tcolorbox}Find all polynomials $P(x)=a_nx^n+a_{n-1}x^{n-1}+...+a_1x+a_0$ with \begin{bolded}integer\end{bolded} coefficients such that the roots of $P(x)$ are $a_{n-1}, a_{n-2}^2,...,a_1^{n-1},a_0^n$.\end{tcolorbox}
The sentence "the roots are ..." with exactly $n$ numbers means that all the roots are real numbers (maybe not distinct).
I suppose too that $a_n\ne 0$ although it's not clear in the problem sentence.

Let $k\in[0,n]$ such that $a_k\ne 0$ and $a_i=0$ $\forall 0\le i<k$ and so $P(x)=x^{k}Q(x)$ where $Q(x)=\sum_{i=k}^na_ix^{i-k}$

If $k=n$, this gives polynomial $a_nx^n$ which indeed is a solution.
If $k=n-1$, this gives the polynomial $a_nx^n+a_{n-1}x^{n-1}$ and so the solutions $-x^n+ax^{n-1}$

If $n\ge k+2$, roots of $Q(x)$ are $a_{n-1},a_{n-2}^2,...,a_k^{n-k}$ and so $a_na_{n-1}a_{n-2}^2...a_k^{n-k}=(-1)^{n-k}a_k$
So $a_na_{n-1}a_{n-2}^2...a_{k+1}^{n-k-1}a_k^{n-k-1}=(-1)^{n-k}$ and $a_i=\pm 1$ $\forall i\in[k,n]$

So all remaining coefficients are $\pm 1$
So all remaining roots are $\pm 1$ and $Q(x)=\pm(x-1)^p(x+1)^q$ with $p+q=n-k\ge 2$ is a polynomial whose all coefficients are $\pm 1$

Looking at coefficient of $x^{p+q-1}$ we get $\pm(p-q)$ and so $p-q=\pm 1$ and :
either $Q(x)=\pm(x^2-1)^r(x-1)$ with $2r+1=n-k\ge 2$ and all coefficients are $\pm 1$
either $Q(x)=\pm(x^2-1)^r(x+1)$ with $2r+1=n-k\ge 2$ and all coefficients are $\pm 1$

$r\ge 1$ and looking then at coefficient of $x^{2r-2}$, we find $\pm r$ and so $r=1$ and we get $Q(x)=\pm(x^2-1)(x\pm 1)$
And so the only two solutions $x^3-x^2-x+1$ and $-x^3+x^2+x-1$

\begin{bolded}Synthesis of solutions\end{underlined} \end{bolded}:
$P(x)=ax^n$ for any $n\ge 1$
$P(x)=-x^n+ax^{n-1}$ for any $n\ge 1$
$P(x)=x^n-x^{n-1}-x^{n-2}+x^{n-3}$  for any $n\ge 3$
$P(x)=-x^n+x^{n-1}+x^{n-2}-x^{n-3}$  for any $n\ge 3$
\end{solution}
*******************************************************************************
-------------------------------------------------------------------------------

\begin{problem}[Posted by \href{https://artofproblemsolving.com/community/user/72731}{goodar2006}]
	Let $g(x)$ be a polynomial of degree at least $2$ with all of its coefficients positive. Find all functions $f:\mathbb R^+ \longrightarrow \mathbb R^+$ such that
\[f(f(x)+g(x)+2y)=f(x)+g(x)+2f(y) \quad \forall x,y\in \mathbb R^+.\]

\begin{italicized}Proposed by Mohammad Jafari\end{italicized}
	\flushright \href{https://artofproblemsolving.com/community/c6h479267}{(Link to AoPS)}
\end{problem}



\begin{solution}[by \href{https://artofproblemsolving.com/community/user/29428}{pco}]
	\begin{tcolorbox}$g(x)$ is a polynomial of degree at least $2$ with all of it's coefficients positive. Find all functions $f:\mathbb R^+ \longrightarrow \mathbb R^+$ such that for all $x,y\in \mathbb R^+$

                                                                                  $f(f(x)+g(x)+2y)=f(x)+g(x)+2f(y)$\end{tcolorbox}
It's rather easy to show that $f(x)=x$ is the only continuous solution.
Could you kindly confirm us that the problem must be solved without continuity constraint ?
\end{solution}



\begin{solution}[by \href{https://artofproblemsolving.com/community/user/64868}{mahanmath}]
	\begin{tcolorbox}
Could you kindly confirm us that the problem must be solved without continuity constraint ?\end{tcolorbox}
Yes , It's can be solved without any other restrictions . :)
\end{solution}



\begin{solution}[by \href{https://artofproblemsolving.com/community/user/115063}{PhantomR}]
	Does $0\in \mathbb{R_+}$ in this problem?
\end{solution}



\begin{solution}[by \href{https://artofproblemsolving.com/community/user/72731}{goodar2006}]
	\begin{tcolorbox}Does $0\in \mathbb{R_+}$ in this problem?\end{tcolorbox}

I only translated what was written on the exam paper, but I think the answer to your question is no.
\end{solution}



\begin{solution}[by \href{https://artofproblemsolving.com/community/user/93837}{jjax}]
	\begin{tcolorbox}$g(x)$ is a polynomial of degree at least $2$ with all of its coefficients positive. Find all functions $f:\mathbb R^+ \longrightarrow \mathbb R^+$ such that for all $x,y\in \mathbb R^+$

                                                                                  $f(f(x)+g(x)+2y)=f(x)+g(x)+2f(y)$\end{tcolorbox}

Let $P(x,y):f(f(x)+g(x)+2y)=f(x)+g(x)+2f(y)$.

We will show that for some $c>0$ and $M>0$, for all $x>M$ we have $f(x)+c=f(x+c)$.
[hide]Consider a positive number $a$. Clearly, as $g$ is unbounded, there is some $b$ such that $g(b)>f(a)+g(a)$, so $f(b)+g(b)>f(a)+g(a)$.
Comparing $P(a,y),P(b,y)$ and eliminating the term $2f(y)$ gives
$f(f(b)+g(b)+2y)-f(f(a)+g(a)+2y)=f(b)+g(b)-f(a)-g(a)$.
Thus, for $c=f(b)+g(b)-f(a)-g(a)>0$, we have $f(x)+c=f(x+c)$ for all sufficiently large $x>M$.[\/hide]

Let $d=g(x_0 +c)-g(x_0)$, for some $x_0 >M$.

We show that $f(r)+d=f(r+d)$ for all reals $r$.
[hide]$P(x_0+c,y):f(f(x_0+c)+g(x_0+c)+2y)=f(x_0+c)+g(x_0+c)+2f(y)$.
For $y>M$ this becomes $f(f(x_0)+g(x_0+c)+2y)=f(x_0)+g(x_0+c)+2f(y)$.
Subtracting from this $P(x_0,y)$ gives $f(f(x_0)+g(x_0+c)+2y)-f(f(x_0)+g(x_0)+2y))=g(x_0+c)-g(x_0)$.
Then $f(z+d)-f(z)=d$ for all sufficiently large $z>N$.
Now consider any positive real $r$. Choose $X$ large enough such that $f(X)+g(X)>g(X)>N$.
Comparing $P(X,r+d)$ and $P(X,r)$ gives $f(r)+d=f(r+d)$.[\/hide]

Since some interval $[k, \infty )$ is contained in $ \{ g(x+c)-g(x), x>M \}$, we have
$f(r)+d=f(r+d)$ for any $d \geq  k$.
Consider any $p>0$. Since $f(r+p)+k=f(r+p+k)=f(r)+p+k$, we obtain that $f(r+p)=f(r)+p$ for any $r,p>0$.
Thus $f(x)=x+a$ for some $a$. A quick check shows that $f(x)=x$ is the only solution.

My thoughts:
[hide]This solution is truly distasteful; it's a rehash of the solution for China TST 2011 Quiz 2 - D1 - P1.
[url]http://www.artofproblemsolving.com/Forum/viewtopic.php?f=38&t=398427[\/url]
It's a common trick to obtain $f(x+c)=f(x)+b$ for constants $c,b$ for all $x$ then spam it everywhere.
In this particular solution, the number inside the $f$ (in this case, it's $c$) is fudged using the polynomial $g$.[\/hide]
\end{solution}



\begin{solution}[by \href{https://artofproblemsolving.com/community/user/72731}{goodar2006}]
	Congrats! Awesome!

During the exam, only three people (out of sixteen!) could solve it!
\end{solution}



\begin{solution}[by \href{https://artofproblemsolving.com/community/user/94615}{Pedram-Safaei}]
	put $h(x)=f(x)+g(x)$.because $g$ is a polynomial so $h$ is not constant now if $r,s$ are two positive reals such that we have $h(r)-h(s)=2T$ for some positive $T$ we have:$h(s)+2y+2T=h(r)+2y$ so with take an $f$ we have $f(y+T)=f(y)+T$ so for natural $n$ ,$f(y+nT)=f(y)+nT$ now for an arbitrary $x$ choose $y$ such that $y+h(x)=nT$ for some $n\geq2$ so we have $2f(y)+h(x)=f(2y+h(x))=f(y+nT)=f(y)+nT$ so $f(y)=y$.
now because $h(x+T)-h(x)=T+g(x+T)-g(x)$ so there exist $c$ such that it is surjective after $c$ so for any $l\geqq$ we have $f(y+l)=f(y)+l$ now with use the fact that $f$ has a fixed point we have $f(y)=y$ for any positive $y$.
\end{solution}



\begin{solution}[by \href{https://artofproblemsolving.com/community/user/82334}{bappa1971}]
	Let $P(x,y)$ be the given assertion. Obviously, $h(x)=f(x)+g(x)$ is not a constant function,

Outline of my solution:
Take some $x, y$ such that $h(x)>h(z)$ and take $d=\frac{h(x)-h(z)}{2}$
Then, $P(x,y)$ and $P(z,y+d)$ imply $f(y+d)=f(y)+d$
So, the function $f(x)-x$ is periodic and hence bounded.
That is, $|f(x)-x|<M$ for some real $M$.

But, $f(y)-y=\frac{f(2y+h(x))-(2y+h(x))}{2}=\frac{f(4y+3h(x))-(4y+3h(x))}{4}$
$=\cdots =\frac{f(2^n y + (2^n - 1)h(x)) - (2^n y + (2^n - 1)h(x)) }{2^n}$
So, $|f(y)-y|<\frac{M}{2^n}$  for large $n$.
So, $f(y)=y$, done!
\end{solution}



\begin{solution}[by \href{https://artofproblemsolving.com/community/user/29428}{pco}]
	\begin{tcolorbox}So, the function $f(x)-x$ is periodic and hence bounded.\end{tcolorbox}
"periodic implies bounded" is true only if you have continuity (there are a lot of periodic unbounded non continuous functions)
And we dont have.
\end{solution}



\begin{solution}[by \href{https://artofproblemsolving.com/community/user/334227}{reveryu}]
	\begin{tcolorbox}
now because $h(x+T)-h(x)=T+g(x+T)-g(x)$ so there exist $c$ such that it is surjective after $c$ so for any $l\geqq$ we have $f(y+l)=f(y)+l$ now with use the fact that $f$ has a fixed point we have $f(y)=y$ for any positive $y$.\end{tcolorbox}

Can anyone gives a detailed explanation for these lines please?


\end{solution}



\begin{solution}[by \href{https://artofproblemsolving.com/community/user/243741}{anantmudgal09}]
	This is a really nice problem! :)

\begin{tcolorbox}Let $g(x)$ be a polynomial of degree at least $2$ with all of its coefficients positive. Find all functions $f:\mathbb R^+ \longrightarrow \mathbb R^+$ such that
\[f(f(x)+g(x)+2y)=f(x)+g(x)+2f(y) \quad \forall x,y\in \mathbb R^+.\]

\begin{italicized}Proposed by Mohammad Jafari\end{italicized}\end{tcolorbox}

Let $h(x) \overset{\text{def}}{:=} f(x)-x$, $p(x) \overset{\text{def}}{:=} x+g(x)$ and $t(x) \overset{\text{def}}{:=} h(x)+p(x)$ for all $x>0$; then we obtain $$h(h(x)+p(x)+2y)=2h(y)$$ for all $x,y>0$. Swap $x$ with a variable $z>0$; then $$h(h(x)+p(x)+2y)=h(h(z)+p(z)+2y)=2h(y)$$ for all $x,y,z>0$. Notice that if $h$ is injective, then $f(x)=-g(x)+c$ which is false since $g(x) \rightarrow \infty$ as $x \rightarrow \infty$ while $f(x)>0$ always holds. 

Pick $a,b$ with $t(a) \ne t(b)$ and set $c=|t(a)-t(b)|$ then for all $x>N=\min(t(a), t(b), 0)$ we have $h(x)=h(x+c)$. Now put $y=\varepsilon$ and $x$ to be large enough so that $g(x)>N$ (hence $t(x) \ge p(x)-x=g(x)>N$), and substitute $y$ with $\varepsilon+c$; then $f(c+\varepsilon)=f(\varepsilon)$. Thus, $h$ is periodic with period $c$; i.e. $h(x+c)=h(x)$ for all $x>0$.

Now notice that $h(x+c)+g(x+c)-h(x)-g(x)=g(x+c)-g(x)$ is a period of $h$, for all $x>0$. Thus, all points in an entire interval are periods of $h$ proving $h$ is the constant function. Clearly, $h(h(1)+p(1)+2)=2h(1)$ shows that $h \equiv 0$. Hence $f$ is the identity function and it clearly works! $\blacksquare$
\end{solution}
*******************************************************************************
-------------------------------------------------------------------------------

\begin{problem}[Posted by \href{https://artofproblemsolving.com/community/user/127783}{Sayan}]
	Prove that the polynomial equation $x^{8}-x^{7}+x^{2}-x+15=0$ has no real solution.
	\flushright \href{https://artofproblemsolving.com/community/c6h479486}{(Link to AoPS)}
\end{problem}



\begin{solution}[by \href{https://artofproblemsolving.com/community/user/76247}{yugrey}]
	This factors partially as $(x)(x-1)(x^6+1)+15$.

If $x\le 0$ or $x\ge 1$, then $(x)(x-1)\ge 0$, and $x^6+1\ge 1$, so $(x)(x-1)(x^6+1)$ is nonnegative, and thus the polynomial is at least $15$.

I claim that for values in this range from $0$ to $1$, we also have no solution.

Note $1\le x^6+1\le 2$ then.

Also $0\ge x(x-1)\ge -1\/4$.

Thus, $x(x-1)(x^6+1)$ is then at most $2$ in magnitude, so it is at least $-2$, and the polynomial is at least $13>0$, so we are done.
\end{solution}



\begin{solution}[by \href{https://artofproblemsolving.com/community/user/97235}{iarnab_kundu}]
	This is true for even

$x^8 - x^7 + x^2 - x + 1$

in (0,1) rewrite it as $x^2 - x^7 + 1- x + x^8 > 0$
\end{solution}



\begin{solution}[by \href{https://artofproblemsolving.com/community/user/29428}{pco}]
	\begin{tcolorbox}Prove that the polynomial equation $x^{8}-x^{7}+x^{2}-x+15=0$ has no real solution.\end{tcolorbox}
Just for "fun" (yugrey's solution is quite nicer) :

$x^8-x^7+x^2-x+15=$ $\frac{(128x^4-64x^3-16x^2-8x-9)^2+4(16x^2-11x+121)^2+60(x-49)^2+43055}{2^{14}}$ $>0$
\end{solution}



\begin{solution}[by \href{https://artofproblemsolving.com/community/user/140796}{mathbuzz}]
	it is quite easy to show that any x<0 can not be a root . also , it is easy to show that any x>1 cant be a root.
also 1 and 0 are not roots. so , the only possibility we are left with is 0<x<1 i.e. x is a positive proper fraction.
but in that case , it is quite obvious that $(x^8-x^7+x^2-x)>-15$.so, this case is also not possible. so, no real root at all :D
\end{solution}



\begin{solution}[by \href{https://artofproblemsolving.com/community/user/99639}{tuan119}]
	:oops_sign:  :oops_sign: 
http://www.artofproblemsolving.com/Forum/viewtopic.php?f=296&t=479386
\end{solution}



\begin{solution}[by \href{https://artofproblemsolving.com/community/user/97235}{iarnab_kundu}]
	\begin{bolded}Lemma :\end{bolded} If $\alpha$ is a real root of some real polynomial $f(x)=a_nx^n+a_{n-1}x^{n-1}+\cdots+a_1x+a_0$,
Then $\alpha\in\left[-M,M\right]$ where \[M_1=1+\max\{\mid a_n\mid,\mid a_{n-1}\mid,\cdots,\mid a_0\mid\}\] \[M_2=\max\{1,\mid a_n\mid +\mid a_{n-1}\mid +\cdots+\mid a_0\mid\}\]
\[M=\min\{M_1,M_2\}\]

\begin{bolded} Main Proof :\end{bolded} Let $P(x)=x^8-x^7+x^2-x+15$

By descartes rule of signs we get the number of positive roots $\in\{1,3,5\}$ and that there can be no negative roots.
Again, applying \begin{italicized}Lemma\end{italicized} above we get any real root $\alpha\in\left[-16,16\right]$.

Hence any root of P(x) must belong to the closed interval $\left[0,2\right]$.

Computing $P(0), P(1), P(2)$ we get all of them are positive.
Thus,the total number of roots $\in\left( 0,1\right)$ and $\in\left( 1,2\right)$ must be even in each of the intervals;
which is a clear contradiction to the fact that there are $5,3$ or $1$ real roots total.

So, $P(x)$ has no real roots.

Edited
\end{solution}



\begin{solution}[by \href{https://artofproblemsolving.com/community/user/29428}{pco}]
	\begin{tcolorbox}...
\begin{bolded} Main Proof :\end{bolded} Let $P(x)=x^8-x^7+x^2-x+15$

By descartes rule of signs we get the number of positive roots $\in\{1,3,5\}$
...\end{tcolorbox}
According to me, application here of Descartes rule of signs indicates $0,2,4$ positive roots
(and, btw, writing $1,3,5$ and concluding with $0$ is a contradiction)
\end{solution}



\begin{solution}[by \href{https://artofproblemsolving.com/community/user/97235}{iarnab_kundu}]
	Yes PCO you are correct...

I had some serious understanding problem of Descartes Rule of Signs :read: ...

My solution is wrong :wallbash_red:   ...
\end{solution}



\begin{solution}[by \href{https://artofproblemsolving.com/community/user/51470}{Potla}]
	If $x<0,$ note that $x^8+(-x^7)+x^2+(-x)>0,$ so the polynomial cannot have any negative roots.
If $x\geq 0,$ then note that from AM-GM inequality we have:
$\left\{\begin{aligned}& \frac 78 x^8+\frac 18\geq x^7\\& x^2+\frac 14\geq x\end{aligned}\right\} ;$
Thus $\frac 78x^8-x^7+x^2-x+\frac 38>0;$ so again, the original polynomial cannot have any positive roots.
\end{solution}



\begin{solution}[by \href{https://artofproblemsolving.com/community/user/118094}{Manolescu}]
	iarnab_kundu 
Can you prove your lemma please?
Thank you!
\end{solution}



\begin{solution}[by \href{https://artofproblemsolving.com/community/user/107451}{Learner94}]
	The lemma posted was slightly wrong. Here is my proof. 

If $z$ is a complex root of a monic polynomial $x^n+a_{n-1}x^{n-1}+\cdots+a_1x+a_0$ with complex coefficients then $|z| \le M $ where \[M_1=1+\max\{\mid a_{n-1}\mid,\cdots,\mid a_0\mid\}\] \[M_2=\max\{1,\mid a_{n-1}\mid +\cdots+\mid a_0\mid\}\]
\[M=\min\{M_1,M_2\}\]


Proof : 

1) We will show $|z| \le M_1$. Let $\max\{\mid a_{n-1}\mid,\mid a_{n-1}\mid,\cdots,\mid a_0\mid\} = p $

Note that if $z$ is a root then 

$|z|^n = |a_{n-1} z^{n-1} +\ldots + a_0| \le |a_{n-1}| |z|^{n-1} + \ldots +|a_0| \le p (|z|^{n-1} + \ldots + 1) = p \left ( \frac{|z|^n - 1}{|z| - 1} \right ) $

So we get $ |z|^n  \le p \left ( \frac{|z|^n - 1}{|z| - 1} \right )$

which rewrites as $|z|^{n+1} \le |z|^n (p+1) - p < |z|^n (p+1)$; implying $|z| < p+1$, qed

2) we will show $|z| \le M_2$. This splits in two cases.

i) if $1 > |a_{n-1}| + \ldots + |a_0|$, we have to show $|z| < 1$.

On the contrary assume that $|z| \ge 1$

Then we have 

$|z|^n = |a_{n-1} z^{n-1} +\ldots + a_0| \le |a_{n-1}| |z|^{n-1} + \ldots +|a_0| \le |z|^n (|a_{n-1}+ \ldots + |a_0| ) < |z|^n$, which is a contradiction.


ii) if $|a_{n-1}| + \ldots + |a_0| \ge 1$, we will show $|z| \le |a_{n-1}| + \ldots + |a_0| $. On the contrary assume that $|z| > |a_{n-1}| + \ldots + |a_0|$. Note that this implies $|z| > |a_{n-1}| + \ldots + |a_0| \ge 1$, so $|z| >1$

Then $|z|^n \le |a_{n-1}| |z|^{n-1} + \ldots +|a_0| < |z|^{n-1} (|a_{n-1}| + \ldots + |a_0| ) < |z|^{n}$, giving $|z|^n < |z|^{n}$, which contradicts our assumption that $|z| >1$, so we are done.
\end{solution}



\begin{solution}[by \href{https://artofproblemsolving.com/community/user/85351}{tzhang1}]
	Is there a non inequality way to solve this problem?
\end{solution}



\begin{solution}[by \href{https://artofproblemsolving.com/community/user/29428}{pco}]
	\begin{tcolorbox}Is there a non inequality way to solve this problem?\end{tcolorbox}
My solution above is a non-inequality way.
\end{solution}



\begin{solution}[by \href{https://artofproblemsolving.com/community/user/119445}{Sampro}]
	\begin{tcolorbox}[quote="Sayan"]Prove that the polynomial equation $x^{8}-x^{7}+x^{2}-x+15=0$ has no real solution.\end{tcolorbox}
Just for "fun" (yugrey's solution is quite nicer) :

$x^8-x^7+x^2-x+15=$ $\frac{(128x^4-64x^3-16x^2-8x-9)^2+4(16x^2-11x+121)^2+60(x-49)^2+43055}{2^{14}}$ $>0$\end{tcolorbox}

How did you arrive at this?
\end{solution}



\begin{solution}[by \href{https://artofproblemsolving.com/community/user/29428}{pco}]
	Write $x^8-x^7+x^2-x+15-(x^4+ax^3+bx^2+cx+d)^2=P(x)$
Let us make degree of $P$ even and the smallest possible.

Cancelling $x^7$ in LHS implies $2a=1$ and so $a=-\frac 12$
Cancelling $x^6$ in LHS implies $a^2+2b=0$ and so $b=-\frac 18$
Cancelling $x^5$ in LHS implies $2c+2ab=0$ and so $c=-\frac 1{16}$
Cancelling $x^4$ in LHS is not clever since then $P(x)$ would be of odd degree and we want it always $>0$

So we got up to now :

$x^8-x^7+x^2-x+15=(x^4-\frac 12x^3-\frac 18x^2-\frac 1{16}x+d)^2$ $+x^4(-2d-\frac 5{64})+x^3(d-\frac 1{64})+x^2(\frac d4+1-\frac 1{256})+x(\frac d8-1)+(15-d^2)$

Let us then write 
$x^4(-2d-\frac 5{64})+x^3(d-\frac 1{64})+x^2(\frac d4+1-\frac 1{256})+x(\frac d8-1)+(15-d^2)$ $=u(x^2+vx+w)^2+Q(x)$ with $u>0$ and $Q(x)$ polynomial of degree $2$ or $0$

Identifying $x^4$ summands gives $u=-2d-\frac 5{64}$ and so $d<-\frac 5{128}$

Identifying $x^3$ summands gives $d-\frac 1{64}=2uv$ and so $v=-\frac{64d-1}{256d+10}$

And we get then $Q(x)=x^2(-uv^2-2uw+\frac d4+1-\frac 1{256})+x(-2uvw+\frac d8-1)+(15-d^2-uw^2)$

And we must choose $d,w$ such that :
$d<-\frac 5{128}$ (in order to have $u>0$)
Quadratic is always positive and so :
a) $-uv^2-2uw+\frac d4+1-\frac 1{256}>0$ and so $-\frac{v^2}2+\frac 1{2u}(\frac d4+1-\frac 1{256})>w$
b) Discriminant of quadratic is $<0$

My example is with $d=-\frac 9{128}$ (and so $u=\frac 1{16}$ and $v=-\frac {11}{16}$) and $w=\frac{121}{16}$ but many other combinaisons are possible

For example, with $d=-\frac 6{128}$ and $w=2$ we get 

$x^8-x^7+x^2-x+15=$ $\frac{5(128x^4-64x^3-16x^2-8x-6)^2+1280(x^2-2x+2)^2+11(80x-41)^2+1205009}{5\times 2^{14}}$

And a lot of other possibilities
\end{solution}



\begin{solution}[by \href{https://artofproblemsolving.com/community/user/210672}{Chirantan}]
	I don't know but cant we just use descartes sign rule
\end{solution}



\begin{solution}[by \href{https://artofproblemsolving.com/community/user/79541}{szl6208}]
	For #1
We have
\[x^8-x^7+x^2-x+15=\frac{(1+x^6)(x-1)^2}{2}+\frac{x^2(x^2+1)(x^2-1)^2}{2}+\frac{x^4}{2}+\frac{29}{2}\]
\end{solution}



\begin{solution}[by \href{https://artofproblemsolving.com/community/user/213278}{shmm}]
	The following equations is equivalent to last equation.
Solve in positive numbers $x^{12}+x^4+1=x^9+x$

\end{solution}



\begin{solution}[by \href{https://artofproblemsolving.com/community/user/296265}{Roshesh}]
	The given equation is similar to..
(x-1)(x)(x+1)(1+x+x^2+x^3+x^4+x^5)=15
Let us assume the equation has a real root...
Now since at least one of  x-1, x, x+1 will be divisible by 3 we will divide both the sides of the eqn by 3..
Now on right side we have a prime no. & on left side we have Its Factors other than 1& itself..
Contradiction!!!!!
\end{solution}



\begin{solution}[by \href{https://artofproblemsolving.com/community/user/29428}{pco}]
	\begin{tcolorbox}T... Now since at least one of  x-1, x, x+1 will be divisible by 3 ...\end{tcolorbox}
And what about non-integer roots ?


\end{solution}



\begin{solution}[by \href{https://artofproblemsolving.com/community/user/296107}{kitun}]
	my solution:
clearly applying descartes rule above has no negative roots but can have positive roots. also 0 and 1 are not roots. So we consider cases 0<x<1 and x>1
for x>1 we  have x^8>x^7 and x^2>x,  using  these inequalities x^8-x^7 + x^2 -x +15>0 so no roots greater than 1.
for x<1, absolute value of x^8-x^7 is always lesser than 2, same for x^2-x,  so so x^8-x^7+x^2-x+15>0 once again.
hence proved
\end{solution}



\begin{solution}[by \href{https://artofproblemsolving.com/community/user/212515}{adityaguharoy}]
	Simply compare values ...


 \begin{tcolorbox}it is quite easy to show that any x<0 can not be a root . also , it is easy to show that any x>1 cant be a root.
also 1 and 0 are not roots. so , the only possibility we are left with is 0<x<1 i.e. x is a positive proper fraction.
but in that case , it is quite obvious that $(x^8-x^7+x^2-x)>-15$.so, this case is also not possible. so, no real root at all :D\end{tcolorbox}

And here is the easy solution!!
\end{solution}



\begin{solution}[by \href{https://artofproblemsolving.com/community/user/296107}{kitun}]
	\begin{tcolorbox}Simply compare values ...


 \begin{tcolorbox}it is quite easy to show that any x<0 can not be a root . also , it is easy to show that any x>1 cant be a root.
also 1 and 0 are not roots. so , the only possibility we are left with is 0<x<1 i.e. x is a positive proper fraction.
but in that case , it is quite obvious that $(x^8-x^7+x^2-x)>-15$.so, this case is also not possible. so, no real root at all :D\end{tcolorbox}

And here is the easy solution!!\end{tcolorbox}

similar to me i see
\end{solution}



\begin{solution}[by \href{https://artofproblemsolving.com/community/user/212515}{adityaguharoy}]
	You used and argued about Descartes Rule.. which won't prove to be a credit in this case !!
\end{solution}



\begin{solution}[by \href{https://artofproblemsolving.com/community/user/296107}{kitun}]
	\begin{tcolorbox}You used and argued about Descartes Rule.. which won't prove to be a credit in this case !!\end{tcolorbox}

why?
\end{solution}



\begin{solution}[by \href{https://artofproblemsolving.com/community/user/212515}{adityaguharoy}]
	Because simple solutions are worth more credit !!
and you may need to prove Descartes Rule in case you use it !!
\end{solution}



\begin{solution}[by \href{https://artofproblemsolving.com/community/user/296107}{kitun}]
	well mathbuzz told x<0 cannot be root. can u tell me how to do that (without descartes)?
\end{solution}



\begin{solution}[by \href{https://artofproblemsolving.com/community/user/212515}{adityaguharoy}]
	Suppose x<0 p(x) has all the terms >0 in it so it is >0.
\end{solution}



\begin{solution}[by \href{https://artofproblemsolving.com/community/user/410853}{bhattacharya301}]
	\begin{tcolorbox}If $x<0,$ note that $x^8+(-x^7)+x^2+(-x)>0,$ so the polynomial cannot have any negative roots.
If $x\geq 0,$ then note that from AM-GM inequality we have:
$\left\{\begin{aligned}& \frac 78 x^8+\frac 18\geq x^7\\& x^2+\frac 14\geq x\end{aligned}\right\} ;$
Thus $\frac 78x^8-x^7+x^2-x+\frac 38>0;$ so again, the original polynomial cannot have any positive roots.\end{tcolorbox}
    

I am sorry, but how? 
 \frac 78 x^8+\frac 18\geq x^7\\
\end{solution}



\begin{solution}[by \href{https://artofproblemsolving.com/community/user/410853}{bhattacharya301}]
	I mean where is the application of am - gm inequality in (7\/8)x^8  + (1\/8)>= x^7








\end{solution}



\begin{solution}[by \href{https://artofproblemsolving.com/community/user/391068}{TuZo}]
	\begin{tcolorbox}Prove that the polynomial equation $x^{8}-x^{7}+x^{2}-x+15=0$ has no real solution.\end{tcolorbox}

\begin{bolded}The simplest solution:\end{bolded}
1) If $x\geq1$, we can vrite $x^7(x-1)+x(x-1)+15>0$
2) If $x\leq0$, put $-x$ instead of $x$, and we get: $x^8+x^7+x^2+x+15>0$
3) If $x\in(0,1)$, we can vrite: $x^8+x^2(1-x^5)+(1-x)+14>0$
Done  :D 

\end{solution}



\begin{solution}[by \href{https://artofproblemsolving.com/community/user/134102}{jonny}]
	\begin{tcolorbox}The following equations is equivalent to last equation.
Solve in positive numbers $x^{12}+x^4+1=x^9+x$\end{tcolorbox}

[hide = Solution]$$x^{12}-x^9+x^4-x+1 = \frac{1}{2}\bigg[2x^{12}-2x^9+2x^4-2x+2\bigg]$$

$$ = \frac{1}{2}\bigg[x^{12}-2x^9+x^6+x^{12}+\frac{1}{4}-x^6+2\left(x^4+\frac{1}{4}-x^2\right)+x^2+x^2-2x+1+\frac{1}{4}\bigg]$$

$$ = \frac{1}{2}\bigg[\left(x^{12}-x^6\right)^2+\left(x^6-\frac{1}{2}\right)^2+2\left(x^2-\frac{1}{2}\right)^2+x^2+(x-1)^2+\frac{1}{4}\bigg]>0\;\forall x \in \mathbb{R}$$[\/hide]
\end{solution}



\begin{solution}[by \href{https://artofproblemsolving.com/community/user/134102}{jonny}]
	\begin{tcolorbox}Prove that the polynomial equation $x^{8}-x^{7}+x^{2}-x+15=0$ has no real solution.\end{tcolorbox}

[hide = Alternate] We can write $x^8-x^7+x^2-x+15 = \frac{1}{2}\bigg[2x^8-2x^7+2x^2-2x+30\bigg]$

$ = \frac{1}{2}\bigg[x^8+x^6+x^2-2x+1+x^8+x^6-2x^6-2x^4+x^4+x^2+x^4+29\bigg]$

$ = \frac{1}{2}\bigg[(x^6+1)(x-1)^2+(x^3-1)^2(x^2+1)+x^4+29\bigg]>0\;\forall x \in \mathbb{R}$[\/hide]
\end{solution}



\begin{solution}[by \href{https://artofproblemsolving.com/community/user/346843}{jrc1729}]
	\begin{tcolorbox}
[hide = Alternate] We can write $x^8-x^7+x^2-x+15 = \frac{1}{2}\bigg[2x^8-2x^7+2x^2-2x+30\bigg]$

$ = \frac{1}{2}\bigg[x^8+x^6+x^2-2x+1+x^8+x^6-2x^6-2x^4+x^4+x^2+x^4+29\bigg]$

$ = \frac{1}{2}\bigg[(x^6+1)(x-1)^2+(x^3-1)^2(x^2+1)+x^4+29\bigg]>0\;\forall x \in \mathbb{R}$[\/hide]\end{tcolorbox}
What's the motivation behind the factorization. I mean, how did you factorize it. :o

\end{solution}
*******************************************************************************
-------------------------------------------------------------------------------

\begin{problem}[Posted by \href{https://artofproblemsolving.com/community/user/126258}{edwed18}]
	Determine all polynomials such that \[(P{}'(x))^2=cP(x)P{}''(x)\] for some constant $c$.
	\flushright \href{https://artofproblemsolving.com/community/c6h480262}{(Link to AoPS)}
\end{problem}



\begin{solution}[by \href{https://artofproblemsolving.com/community/user/29428}{pco}]
	If $c=0$, we get $P(x)=a$ constant, which indeed is a solution.

If $c\ne 0$, the equation implies $\frac {P'}{P}=c\frac{P''}{P'}$ for all $x$ where $P(x)P'(x)\ne 0$ and so $|P|=\alpha|P'|^c$ on any interval where $P(x)P'(x)\ne 0$

And so $P'(x)|P(x)|^{-\frac 1c}=\beta$ on any interval where $P(x)P'(x)\ne 0$

$c=1$ gives a non polynomial solution.

$c\notin\{0,1\}$ gives then $|P(x)|^{1-\frac 1c}=ux+v$ on any interval where $P(x)P'(x)\ne 0$

So $|P(x)|=(ux+v)^{\frac c{c-1}}$ on any interval where $P(x)P'(x)\ne 0$

And so, since polynomial, $\boxed{P(x)=a(x+b)^n}$ for any non non negative integer $n\ne 1$, which indeed is a solution (and than $c=\frac n{n-1}$)

(note that setting $a$ out of parenthesis avoid to add a $\pm$ in front of it)
\end{solution}
*******************************************************************************
-------------------------------------------------------------------------------

\begin{problem}[Posted by \href{https://artofproblemsolving.com/community/user/125553}{lehungvietbao}]
	Prove that  $x^{5}-5x^{4}+30x^{3}-50x^{2}+55x-21=0$ has a unique real solution $x=1+\sqrt[5]{2}-\sqrt[5]{4}+\sqrt[5]{8}-\sqrt[5]{16}$.
	\flushright \href{https://artofproblemsolving.com/community/c6h480318}{(Link to AoPS)}
\end{problem}



\begin{solution}[by \href{https://artofproblemsolving.com/community/user/29428}{pco}]
	\begin{tcolorbox}Prove that  $x^{5}-5x^{4}+30x^{3}-50x^{2}+55x-21=0$ has a unique real solution $x=1+\sqrt[5]{2}-\sqrt[5]{4}+\sqrt[5]{8}-\sqrt[5]{16}$.\end{tcolorbox}
Derivative is $5x^4-20x^3+90x^2-100x+55$ $=5(x^2-2x+3)^2+10(2x-1)^2>0$ and so exactly one real root for this equation.

Writing then $a=\sqrt[5]2$ and $x=1+a-a^2+a^3-a^4$, we get $x-2=-1+a-a^2+a^3-a^4$ $=-\frac{(-a)^5-1}{-a-1}$ $=-\frac 3{a+1}$

$\implies$ $a+1=\frac 3{2-x}$ $\implies$ $a=\frac {x+1}{2-x}$ $\implies$ $(x+1)^5=2(2-x)^5$

And developping this expression gives the required result (remember we know that there is a unique real root)
\end{solution}



\begin{solution}[by \href{https://artofproblemsolving.com/community/user/125553}{lehungvietbao}]
	Can you  explain?  thank you very much !
\end{solution}



\begin{solution}[by \href{https://artofproblemsolving.com/community/user/89198}{chaotic_iak}]
	Try to actually read it first; I don't see anything that needs clearer explanation if you work a little.

First, pco finds the derivative. As it's always larger than 0, then the function is strictly increasing and hence only one real root exists.

Now, denoting $a = \sqrt[5]{2}$ for easier writing, pco changed the form of $x-2$ ($x$ being the given root) to a geometrical series which yields the equality. Performing a little algebra, an expression of $a$ in $x$ is found. Replacing back the value of $a$, taking the fifth power, and another algebra will yield back the original equation, hence $x$ is indeed a real root.

Finally, as we have shown that it only has one real root, the problem is solved.
\end{solution}
*******************************************************************************
-------------------------------------------------------------------------------

\begin{problem}[Posted by \href{https://artofproblemsolving.com/community/user/145648}{perfect_square}]
	Prove that: $ \sqrt[3]{2}+ \sqrt[3]{3}+ \sqrt[3]{5}$ is a irrational number
	\flushright \href{https://artofproblemsolving.com/community/c6h480792}{(Link to AoPS)}
\end{problem}



\begin{solution}[by \href{https://artofproblemsolving.com/community/user/31919}{tenniskidperson3}]
	[hide="hint"]The sum of two algebraic integers is an algebraic integer, and $\sqrt[3]{2}+\sqrt[3]{3}+\sqrt[3]{5}$ can be shown to not be a rational integer.[\/hide]
\end{solution}



\begin{solution}[by \href{https://artofproblemsolving.com/community/user/145648}{perfect_square}]
	\begin{tcolorbox}[hide="hint"]The sum of two algebraic integers is an algebraic integer, and $\sqrt[3]{2}+\sqrt[3]{3}+\sqrt[3]{5}$ can be shown to not be a rational integer.[\/hide]\end{tcolorbox}
Right!!!
But I'm waiting for an elemantary solution!!!
Have you some solution???
\end{solution}



\begin{solution}[by \href{https://artofproblemsolving.com/community/user/29428}{pco}]
	\begin{tcolorbox}Prove that: $ \sqrt[3]{2}+ \sqrt[3]{3}+ \sqrt[3]{5}$ is a irrational number\end{tcolorbox}
I think that tenniskidperson3'hint may be used as a rather elementary solution :

First note that $\sqrt[3]2\in(\frac 54,\frac 43)$ and $\sqrt[3]3\in(\frac 43,\frac 32)$ and $\sqrt[3]5\in(\frac 32,2)$ and so $\sqrt[3]2+\sqrt[3]3+\sqrt[3]5\in(\frac{49}{12},\frac{29}6)$ $\subset (4,5)$ and so is not an integer.

Let $j=e^{\frac{2i\pi}3}$ 

Let $P(x)\in\mathbb Z[X]$
Consider $Q(y)=P(x-y)$ as a polynomial in $y$ whose all coefficients are in $\mathbb Z[x]$
So $Q(y)Q(jy)Q(j^2y)$ is a polynomial in $y^3$ whose all coefficients are in $\mathbb Z[x]$
So : if $P(x)$ is a monic polynomial $\in\mathbb Z[X]$, so is $R(x)=P(x-\sqrt[3]n)P(x-j\sqrt[3]n)P(x-j^2\sqrt[3]n)$

Let then $P(x)=x^3-2$ : $P(x)$ is a monic polynomial $\in\mathbb Z[X]$ and $\sqrt[3]2$ is a root of $P(x)$
Let then $P_1(x)=P(x-\sqrt[3]3)P(x-j\sqrt[3]3)P(x-j^2\sqrt[3]3)$ : $P_1(x)$ is a monic polynomial $\in\mathbb Z[X]$ and $\sqrt[3]2+\sqrt[3]3$ is a root of $P_1(x)$
Let then $P_2(x)=P_1(x-\sqrt[3]5)P_1(x-j\sqrt[3]5)P_1(x-j^2\sqrt[3]5)$ : $P_2(x)$ is a monic polynomial $\in\mathbb Z[X]$ and $\sqrt[3]2+\sqrt[3]3+\sqrt[3]5$ is a root of $P_2(x)$

So, if $\sqrt[3]2+\sqrt[3]3+\sqrt[3]5\in\mathbb Q$, then $\sqrt[3]2+\sqrt[3]3+\sqrt[3]5\in\mathbb Z$, which is wrong.
\end{solution}



\begin{solution}[by \href{https://artofproblemsolving.com/community/user/145648}{perfect_square}]
	\begin{tcolorbox}...So $Q(y)Q(jy)Q(j^2y)$ is a polynomial in $y^3$ whose all coefficients are in $\mathbb Z[x]$
...Let then $P_1(x)=P(x-\sqrt[3]3)P(x-j\sqrt[3]3)P(x-j^2\sqrt[3]3)$ : $P_1(x)$ is a monic polynomial $\in\mathbb Z[X]$ and $\sqrt[3]2+\sqrt[3]3$ is a root of $P_1(x)$
...Let then $P_2(x)=P_1(x-\sqrt[3]5)P_1(x-j\sqrt[3]5)P_1(x-j^2\sqrt[3]5)$ : $P_2(x)$ is a monic polynomial $\in\mathbb Z[X]$ and $\sqrt[3]2+\sqrt[3]3+\sqrt[3]5$ is a root of $P_2(x)$...
\end{tcolorbox}
A great idea!!!
I have some idea. One of them is quite same your proof, but I haven't solved it yet. Another is too long. But it can be generalized. Because your proof is only useful for monic polynominal.
\end{solution}
*******************************************************************************
-------------------------------------------------------------------------------

\begin{problem}[Posted by \href{https://artofproblemsolving.com/community/user/123851}{ctumeo}]
	Find all real polinomials f and g so that:

$(x^2+x+1)*f(x^2-x+1)=(x^2-x+1)*g(x^2+x+1)$

for all real x.

                  Thanks
	\flushright \href{https://artofproblemsolving.com/community/c6h481769}{(Link to AoPS)}
\end{problem}



\begin{solution}[by \href{https://artofproblemsolving.com/community/user/86443}{roza2010}]
	$f(x)=g(x)=Cx$
\end{solution}



\begin{solution}[by \href{https://artofproblemsolving.com/community/user/123851}{ctumeo}]
	\begin{tcolorbox}$f(x)=g(x)=Cx$\end{tcolorbox}

Yes, this is the solution, but I need explanation.
                                                             Thanks
\end{solution}



\begin{solution}[by \href{https://artofproblemsolving.com/community/user/86443}{roza2010}]
	\begin{tcolorbox} but I need explanation.
                                                             \end{tcolorbox}
Try alike: $x^2+x+1=a,x^2-x+1=b$
$af(b)=bg(a)$
$deg(f)=m, deg(g)=n$ etc
\end{solution}



\begin{solution}[by \href{https://artofproblemsolving.com/community/user/123851}{ctumeo}]
	\begin{tcolorbox}[quote="ctumeo"] but I need explanation.
                                                             \end{tcolorbox}
Try alike: $x^2+x+1=a,x^2-x+1=b$
$af(b)=bg(a)$
$deg(f)=m, deg(g)=n$ etc\end{tcolorbox}

Thanks for your answers.
Can you give me further explanation?
Clearly m=n, but how can I demonstrate that m=1?
\end{solution}



\begin{solution}[by \href{https://artofproblemsolving.com/community/user/86443}{roza2010}]
	equaling coefficients...
\end{solution}



\begin{solution}[by \href{https://artofproblemsolving.com/community/user/123851}{ctumeo}]
	\begin{tcolorbox}equaling coefficients...\end{tcolorbox}

Sorry, perhaps it isn't easy for me.
I write f and g in polynomial form, multiply, and I put their coefficient equal.
But now? Can you, please, explain it to me in detail?
                                                                      Thanks
\end{solution}



\begin{solution}[by \href{https://artofproblemsolving.com/community/user/123851}{ctumeo}]
	Please,
Can anyone explain this to me in details?
                                                     Thanks
\end{solution}



\begin{solution}[by \href{https://artofproblemsolving.com/community/user/64716}{mavropnevma}]
	The situation is not as banal as it may seem. For example, if the equation were to be $(x+1)f(x)=xg(x+1)$, then the general solution would be $(f(x), g(x)) = (xp(x+1), xp(x))$, for an arbitrary polynomial $p(x)$.

Now, the polynomial $\varphi(x) = x^2 - x + 1$ is \begin{bolded}special\end{bolded}, since we have $x^2 + x + 1 = \varphi(x+1) = \varphi(-x)$. Our functional equation writes $\varphi(x+1)f(\varphi(x)) = \varphi(x)g(\varphi(x+1))$.
Since $\gcd(\varphi(x), \varphi(x+1)) = 1$, it follows $\varphi(x) \mid f(\varphi(x))$ and $\varphi(x+1) \mid g(\varphi(x+1))$, therefore $x\mid f(x)$ and $x \mid g(x)$, i.e. $f(x)=xf_0(x)$ and $g(x)=xg_0(x)$.

The equation now writes $f_0(\varphi(x)) = g_0(\varphi(x+1))$. So $f_0(\varphi(x)) = g_0(\varphi(-x))$, hence $f_0(\varphi(-x)) = g_0(\varphi(x))$, i.e. $f_0(\varphi(x+1)) = g_0(\varphi(x))$, and so $f_0(\varphi(x+2)) = g_0(\varphi(x+1)) = f_0(\varphi(x)$. This writes as $(f_0 \circ \varphi)(x+2) = (f_0 \circ \varphi)(x)$, leading to $f_0 \circ \varphi = C$ constant, hence $f_0 = C$ constant. Similarly $g_0 = C$, and so the solution to the equation is $f(x) = g(x) = Cx$.
\end{solution}



\begin{solution}[by \href{https://artofproblemsolving.com/community/user/29428}{pco}]
	\begin{tcolorbox}Find all real polinomials f and g so that:

$(x^2+x+1)*f(x^2-x+1)=(x^2-x+1)*g(x^2+x+1)$

for all real x.\end{tcolorbox}
Another writing of the same method :
Let $z$ a complex root of $x^2-x+1=0$. Pluging $z$ in equation implies $f(0)=0$ and so $f(x)=xa(x)$
Let $z$ a complex root of $x^2+x+1=0$. Pluging $z$ in equation implies $g(0)=0$ and so $g(x)=xb(x)$

And equation becomes $a(x^2-x+1)=b(x^2+x+1)$ $\forall x$ (even for roots of $x^2\pm x+1=0$ because continuity).

Let $u(x)=a(x+\frac 34)$ and $v(x)=b(x+\frac 34)$ and the equations becomes $u((x-\frac 12)^2)=v((x+\frac 12)^2)$ and so $u(x^2)=v((x+1)^2)$ $\forall x$

Let then the polynomial $h(x)=u(x^2)=v((x+1)^2)$ : $h(x)=h(-x)$ and $h(x)=h(-x-2)$ and so $h(x)=h(x+2)$ 

But the only periodic polynomial is the constant one. So $h(x)=c$ and, since polynomials, $u(x),v(x),a(x),b(x)$ are constant.

And so $\boxed{f(x)=g(x)=cx}$ $\forall x$
\end{solution}
*******************************************************************************
-------------------------------------------------------------------------------

\begin{problem}[Posted by \href{https://artofproblemsolving.com/community/user/92753}{WakeUp}]
	Given is the polynomial $P(x)$ and the numbers $a_1,a_2,a_3,b_1,b_2,b_3$ such that $a_1a_2a_3\not=0$. Suppose that for every $x$, we have
\[P(a_1x+b_1)+P(a_2x+b_2)=P(a_3x+b_3)\]
Prove that the polynomial $P(x)$ has at least one real root.
	\flushright \href{https://artofproblemsolving.com/community/c6h481894}{(Link to AoPS)}
\end{problem}



\begin{solution}[by \href{https://artofproblemsolving.com/community/user/29428}{pco}]
	\begin{tcolorbox}Given is the polynomial $P(x)$ and the numbers $a_1,a_2,a_3,b_1,b_2,b_3$ such that $a_1a_2a_3\not=0$. Suppose that for every $x$, we have
\[P(a_1x+b_1)+P(a_2x+b_2)=P(a_3x+b_3)\]
Prove that the polynomial $P(x)$ has at least one real root.\end{tcolorbox}
Suppose that such a polynomial $P(x)$ exists with no real roots.

$P(x)$ has a constant sign and so WLOG say $P(x)>0$ $\forall x$ (changing $P(x)\to -P(x)$ if necessary)
$P(x)$ has an even degree (since no real roots) and has a global minimum $m>0$
Let $x$ such that $P(a_3x+b_3)=m$ (possible since $a_3\ne 0$).

Then both $P(a_1x+b_1)$ and $P(a_2x+b_2)$ are $<m$, which is impossible. (and we just needed $a_3\ne 0$ and not $a_1a_2a_3\ne 0$)

Hence the required result.
\end{solution}



\begin{solution}[by \href{https://artofproblemsolving.com/community/user/143832}{ilya_kachan}]
	\begin{tcolorbox}Given is the polynomial $P(x)$ and the numbers $a_1,a_2,a_3,b_1,b_2,b_3$ such that $a_1a_2a_3\not=0$. Suppose that for every $x$, we have
\[P(a_1x+b_1)+P(a_2x+b_2)=P(a_3x+b_3)\]
Prove that the polynomial $P(x)$ has at least one real root.\end{tcolorbox}
Let's $P(x)=\sum_{i=0}^{n}c_ix^i$, $c_n \ne 0$, $n \in \mathbb{N}$. Since $c_n(a_jx+b_j)^n=c_na_j^nx^n+...$ then equating the coefficients of $x^n$ in the original equation we get: $a_1^n+a_2^n=a_3^n$.
If $a_1=a_2=a_3=a$ then $2a^n=a^n$ and $a=0$ - a contradiction.
So $a_1 \ne a_3$ or $a_2 \ne a_3$. WLOG $a_2 \ne a_3$. For $x=\frac{b_3-b_2}{a_2-a_3}$ we have: $a_2x+b_2=a_3x+b_3 \implies P(a_2x+b_2)=P(a_3x+b_3) \implies P(a_1x+b_1)=0$. Thus $x=\frac{a_1b_3-a_1b_2+a_2b_1-a_3b_1}{a_2-a_3}$ is the root of $P$.
\end{solution}
*******************************************************************************
-------------------------------------------------------------------------------

\begin{problem}[Posted by \href{https://artofproblemsolving.com/community/user/41061}{meganyot}]
	A polynomial $f(x)$ has a remainder of $p_1$ if it is divided by $(x-m)$ and a remainder of $p_2$ if it is divided by $(x-n)$
Another polynomial $g(x)$ has a remainder of $q_1$ if it is divided by $(x-m)$ and a remainder of $q_2$ if it is divided by $(x-n)$

Find the remainder of $h(x) = f(x) g(x)$ when it is divided by $(x-m)(x-n)$
	\flushright \href{https://artofproblemsolving.com/community/c6h482499}{(Link to AoPS)}
\end{problem}



\begin{solution}[by \href{https://artofproblemsolving.com/community/user/29428}{pco}]
	\begin{tcolorbox}A polynomial $f(x)$ has a remainder of $p_1$ if it is divided by $(x-m)$ and a remainder of $p_2$ if it is divided by $(x-n)$
Another polynomial $g(x)$ has a remainder of $q_1$ if it is divided by $(x-m)$ and a remainder of $q_2$ if it is divided by $(x-n)$

Find the remainder of $h(x) = f(x) g(x)$ when it is divided by $(x-m)(x-n)$\end{tcolorbox}
So $f(x)=u(x)(x-m)(x-n)+ax+b$ with $am+b=p_1$ and $an+b=p_2$ and so $a=\frac{p_1-p_2}{m-n}$ and $b=\frac{p_2m-p_1n}{m-n}$

And $g(x)=v(x)(x-m)(x-n)+cx+d$ with $cm+d=q_1$ and $cn+d=q_2$ and so $c=\frac{q_1-q_2}{m-n}$ and $d=\frac{q_2m-q_1n}{m-n}$

And so $h(x)=f(x)g(x)=$ $(x-m)(x-n)(u(x)v(x)(x-m)(x-n)$ $+u(x)(cx+d)+v(x)(ax+b)+ac)$ $+(ax+b)(cx+d)-ac(x-m)(x-n)$

Hence the answer : $(ad+bc+ac(m+n))x+bd-acmn$ $=\boxed{\frac{(p_1q_1-p_2q_2)x+p_2q_2m-p_1q_1n}{m-n}}$
\end{solution}



\begin{solution}[by \href{https://artofproblemsolving.com/community/user/52424}{talkinaway}]
	[hide]Applying the remainder theorem on both poylnomials at both points:
$f(m) = p_1$ and $f(n) = p_2$
$g(m) = q_1$ and $g(n) = q_2$

Using the definition of $h(x)$, we get:
$h(m) = p_1q_1$
$h(n) = p_2q_2$

We want to find the linear function $R(x) = ax + b$ such that:

$h(x) = (x-m)(x-n)Q(x) + R(x)$ for some quotient $Q(x)$ (which isn't important) and for some remainder $R(x)$ (which IS important).

Plug in $x=m$ and $x=n$ again, and we get:

$h(m) = R(m)$ and $h(n) = R(n)$

$p_1q_1 = am + b$ and $p_2q_2 = an + b$

Believe it or not, that's two equations with "only" two variables to eliminate:  $a$ and $b$.  Everything else is a parameter\/"given".  Equating the $b$s, we get:

$p_1q_1 - am = p_2q_2 - an$

$an - am = p_2q_2 - p_1q_1$

$a = \frac{p_2q_2 - p_1q_1}{n-m}$

Plug that back into $p_1q_1 = am + b$, and you get:

$p_1q_1 = m\frac{p_2q_2 - p_1q_1}{n-m} + b$

$b =  \frac{p_1q_1n - p_1q_1m + mp_1q_1 - mp_2q_2}{n-m}$

The two middle terms in the numerator cancel.  

So, we  have $R(x) = \boxed{\frac{p_2q_2 - p_1q_1}{n-m} \cdot x + \frac{np_1q_1 - mp_2q_2}{n-m}}$[\/hide]

Beaten, but too much text to waste, and maybe a different way of explaining it.  Plus, mine's backwards. :)

EDIT:  I just noticed there's another, perhaps more "elegant", way to express the answer:

[hide]$\frac{p_1q_1(x-n) - p_2q_2(x-m)}{m-n}$

Note that if $x=m$, the answer reduces to ${\frac{p_1q_1(m-n)}{m-n}} = p_1q_1$, and that if $x=n$, the answer reduces to $\frac{-p_2q_2(n-m)}{m-n} = p_2q_2$.  This makes sense - since $h(m) = p_1q_1$, the remainder of $h(x)$ when divided by $m$ should be $p_1q_1$.[\/hide]
\end{solution}
*******************************************************************************
-------------------------------------------------------------------------------

\begin{problem}[Posted by \href{https://artofproblemsolving.com/community/user/103227}{shohvanilu}]
	Find all polynomials such that for all $x \in R$ satisfies the equation
$(x+1)P(x-1)+(x-1)P(x+1)=2xP(x)$
	\flushright \href{https://artofproblemsolving.com/community/c6h482794}{(Link to AoPS)}
\end{problem}



\begin{solution}[by \href{https://artofproblemsolving.com/community/user/29428}{pco}]
	\begin{tcolorbox}Find all polynomials such that for all $x \in R$ satisfies the equation
$(x+1)P(x-1)+(x-1)P(x+1)=2P(x)$\end{tcolorbox}
Just looking at highest degree summand in equation immediately implies the unique solution $P(x)=0$ $\forall x$
\end{solution}



\begin{solution}[by \href{https://artofproblemsolving.com/community/user/124745}{Uzbekistan}]
	first... we dont know that $q(-1) = 0$, rather, we know that $q(1) = q(0) = 0$

$q(-1) = p(-1) - p(-2) = -p(-2)$ but we know nothing of $p(-2)$ (plus, checking your $q(x) = C(x^2-1)$ in the original $(x+1)q(x) = (x-1)q(x+1)$ shows that it is not correct)

hence, using an adjusted form of ur reasoning, $q(x) = Cx(x-1)$

checking this in $(x+1)q(x) = (x-1)q(x+1)$ we see that this holds

then, again using your reasoning since $p(x)$ has degree at most $3$ (this is from setting $p(x) = a_nx^n + a_{n-1}x^{n-1} + \ldots + a_1x + a_0$, expanding, and noting that the $x^{n-1}$ terms don't cross out), we know

$p(x) - p(x-1) = x(x-1) = x^2-x$

setting $p(x) = ax^3 + bx^2 + cx + d$, we get:

$p(x) - p(x-1) = a(3x^2 - 3x + 1) + b(2x-1) + c = x^2 - x$

equating coefficients, we get:

$a = \frac{1}{3}$, $b = 0$, and $c = - \frac{1}{3}$ (again, all scaled by $C$)

so:

$p(x) = \frac{1}{3}x^3 - \frac{1}{3}x + d$ and letting $x = 0$, we know that since $p(0) = 0$, then $d= 0$ and finally that:

$p(x) = \frac{1}{3}x^3 - \frac{1}{3}x$ is a solution (checking we see that it holds)

also, i dont know if we have to include the fact, from $(x+1)q(x) = (x-1)q(x+1)$, when $q(x) = 0$, in which case we know $p(x) = p(x-1) = k$ for some constant $k$

checking this, we see that when $p(x)$ is a constant our original expression holds as well

thus we have our two solutions for $p(x)$?
\end{solution}



\begin{solution}[by \href{https://artofproblemsolving.com/community/user/29428}{pco}]
	\begin{tcolorbox} $p(x) = \frac{1}{3}x^3 - \frac{1}{3}x$ is a solution (checking we see that it holds)\end{tcolorbox}
Huh ?
Just saying "checking we see that it holds" is not enough. Better to really check 

Let us publicly check :

$P(x)=\frac 13x(x^2-1)$

$(x+1)P(x-1)=\frac 13(x+1)(x-1)(x^2-2x)$

$(x-1)P(x+1)=\frac 13(x-1)(x+1)(x^2+2x)$

$(x+1)P(x-1)+(x-1)P(x+1)=\frac 23x^2(x^2-1)$ $\ne 2P(x)=\frac 23x(x^2-1)$

And so, "checking we see that \begin{bolded}it does not hold\end{underlined}\end{bolded}"
\end{solution}



\begin{solution}[by \href{https://artofproblemsolving.com/community/user/64716}{mavropnevma}]
	Which points to the fact that the OP probably made a typo, and the relation in fact is $(x+1)P(x-1)+(x-1)P(x+1)=2xP(x)$. 
Ha, ha, I know what \begin{bolded}pco\end{bolded} thinks about this "second-guessing" :)

Then $f(x) = \Delta P(x) = P(x) - P(x-1)$ satisfies $(x+1)f(x) = (x-1)f(x+1)$, and checking valueas at $x = \pm 1$ readily yields $f(x) = kx(x-1)$. Coefficient identification now gives $P(x) = \dfrac {k} {3} (x^3-x) + C$ as the general family of solutions.
\end{solution}



\begin{solution}[by \href{https://artofproblemsolving.com/community/user/29428}{pco}]
	\begin{tcolorbox}Which points to the fact that the OP probably made a typo, and the relation in fact is $(x+1)P(x-1)+(x-1)P(x+1)=2xP(x)$. 
Ha, ha, I know what \begin{bolded}pco\end{bolded} thinks about this "second-guessing" :)\end{tcolorbox}
Indeed :)
\end{solution}



\begin{solution}[by \href{https://artofproblemsolving.com/community/user/150618}{dilnozaxon}]
	It is INDIA TST
\end{solution}



\begin{solution}[by \href{https://artofproblemsolving.com/community/user/103227}{shohvanilu}]
	\begin{tcolorbox}It is INDIA TST
 \end{tcolorbox}
No. I think that it was wrong.  :maybe:
\end{solution}
*******************************************************************************
-------------------------------------------------------------------------------

\begin{problem}[Posted by \href{https://artofproblemsolving.com/community/user/43631}{mathwizarddude}]
	Let $f(x)=\sqrt{4+\sqrt{4-x}}$. Solve for x in $f(f(x))=x$.
	\flushright \href{https://artofproblemsolving.com/community/c6h482932}{(Link to AoPS)}
\end{problem}



\begin{solution}[by \href{https://artofproblemsolving.com/community/user/150728}{talgarmo}]
	f(x)=-x^4+8x^2-12    is it the right answer?
\end{solution}



\begin{solution}[by \href{https://artofproblemsolving.com/community/user/29428}{pco}]
	\begin{tcolorbox}Let $f(x)=\sqrt{4+\sqrt{4-x}}$. Solve for x in $f(f(x))=x$.\end{tcolorbox}
Here is a complete ugly calculus proof giving the unique root (finding a root is trivial and the real problem is uniqueness).
I did not find anything more clever.
Could you kindly post your own proof of uniqueness, please ?
Thanks in advance.


Finding a root is quite simple since we at least have the roots of $f(x)=x$ :

$\sqrt{4+\sqrt{4-x}}=x$
$\iff$ (squaring) $x\in[0,4]$ and $\sqrt{4-x}=x^2-4$
$\iff$ (squaring) $x\in[2,4]$ and $x^4-8x^2+x+12=0$ $\iff$ $(x^2-x-3)(x^2+x-4)=0$ 
And so a first solution $\frac{1+\sqrt{13}}2$

The real problem is to see if there exists some roots of $f(f(x))$ which are not roots of $f(x)=x$
$\sqrt{4+\sqrt{4-\sqrt{4+\sqrt{4-x}}}}=x$
$\iff$ (squaring) $x\in[0,4]$ and $\sqrt{4-\sqrt{4+\sqrt{4-x}}}=x^2-4$
$\iff$ (squaring) $x\in[2,4]$ and $\sqrt{4+\sqrt{4-x}}=(x^2-2)(6-x^2)$
$\iff$ (squaring) $x\in[2,\sqrt 6]$ and ${\sqrt{4-x}}=(x^4-8x^2+14)(x^4-8x^2+10)$
$\iff$ (squaring) $x\in[2,\sqrt {4+\sqrt 2}]$ and $(x^4-8x^2+14)^2(x^4-8x^2+10)^2+x-4=0$
This a degree $16$ polynomial and we know that $(x^2-x-3)(x^2+x-4)$ must divide it. and so we get a degree $12$ polynomial equation :

$h(x)=x^{12}-24x^{10}$ $-x^9+228x^8$ $+16x^7-1087x^6$ $-88x^5+2720x^4$ $+191x^3-3380x^2$ $-136x+1633=0$
$h(-3)>0$
$h(-\frac 52)<0$ and so at least a root $x_1\in(-3,-\frac 52)$
$h(-\frac 73)>0$ and so at least a root $x_2\in(-\frac 52,-\frac 73)$
$h(-2)<0$ and so at least a root $x_3\in(-\frac 73,-2)$
$h(-\frac 53)>0$ and so at least a root $x_4\in(-2,-\frac 53)$
$h(-\frac 32)<0$ and so at least a root $x_5\in(-\frac 53,-\frac 32)$
$h(0)>0$ and so at least a root $x_6\in(-\frac 32,0)$
$h(\frac 54)<0$ and so at least a root $x_7\in(0,\frac 54)$
$h(\frac 32)>0$ and so at least a root $x_8\in(\frac 54,\frac 32)$
$h(2)<0$ and so at least a root $x_9\in(\frac 32,2)$
$h(\frac 52)>0$ and so at least a root $x_{10}\in(2,\frac 52)$
$h(\frac {28}{11})<0$ and so at least a root $x_{11}\in(\frac 52,\frac {28}{11})$
$h(3)>0$ and so at least a root $x_{12}\in(\frac {28}{11},3)$

And so fortunately $h(x)$ has all its $12$ roots real.
$x_9<2<x_{10}$ and $x_9<\sqrt{4+\sqrt 2}<\frac 52<x_{11}$ and so it remains to locate $\sqrt{4+\sqrt 2}$ in front of $x_{10}$

$h(\frac 73)<0$ and so $x_9<\frac 73<x_{10}$ and so $x_9<2<\sqrt{4+\sqrt 2}<\frac 73<x_{10}$
And so no real root of $h(x)$ in $[2,\sqrt {4+\sqrt 2}]$

Hence the unique root of the original problem $\boxed{x=\frac{1+\sqrt{13}}2}$
\end{solution}



\begin{solution}[by \href{https://artofproblemsolving.com/community/user/43631}{mathwizarddude}]
	Bravo, pco! I don't see a better way to prove uniqueness either...  :(
\end{solution}



\begin{solution}[by \href{https://artofproblemsolving.com/community/user/126258}{edwed18}]
	[hide="Proof of uniqueness"]

For any solution, since each square root is positive, we must have $0$ less than or equal to $x$ less than or equal to $4$.
Considering each nested radical in turn, from the innermost outwards, we see also that $f(f(x))$ is strictly increasing over this range.
We have also: $f(f(0)) = 2.29$ and $f(f(4)) = 2.33$, correct to two decimal places.
We conclude that the graph of $y = f(f(x))$, for 0 less than or equal to $x$ less than or equal to $4$, is almost a straight line, with average slope approximately $0.01$.

Since $f(f(x))$ is a continuous function, it follows that $y = f(f(x))$ intersects the line $y = x$, for $0$ less than or equal to $x$ less than or equal to $4$.
Furthermore, $y = f(f(x)$ will intersect $y = x$ precisely once, provided that no section of $y = f(f(x))$ has a slope greater than $1$.
This is clearly the case, and may be verified, if necessary, by differentiating $f(f(x))$ with respect to $x$. [\/hide]
\end{solution}
*******************************************************************************
-------------------------------------------------------------------------------

\begin{problem}[Posted by \href{https://artofproblemsolving.com/community/user/25405}{AndrewTom}]
	Find all values of $m$ such that the eqaution

$m(x-a)^{3} -(x+a)^{2} =0$

has three distinct roots.
	\flushright \href{https://artofproblemsolving.com/community/c6h483583}{(Link to AoPS)}
\end{problem}



\begin{solution}[by \href{https://artofproblemsolving.com/community/user/81664}{hyperspace.rulz}]
	Is this an equation in $a$ or $x$? And if so, what is the role of the other variable in the equation?
\end{solution}



\begin{solution}[by \href{https://artofproblemsolving.com/community/user/149377}{Ramonjms}]
	\begin{tcolorbox}Find all values of $m$ such that the eqaution

$m(x-a)^{3} -(x+a)^{2} =0$

has three distinct roots.\end{tcolorbox}

There is a theorem that says if a polynomial equation has a double root then the derivative of this polynomial has the same root (if you don't know, prove it!). hence:

$ 3m{(x-a)}^2 - 2{(x+a)} = 0 $
$ 3m{(x-a)}^2 = 2{(x+a)} $

Dividing the first equation given by the equation above:

$ x=-5a $

This is the unique common root between this two equations. Then:

$m(-5a-a)^{3} -(-5a+a)^{2} =0$

$ m=-\frac{2}{27a} $

If there is three equal roots we can do derivative again:

$ 6m{(x-a)} - 2 = 0 $

deviding the first derivative by the second we find the value of x which can be the triple root:

$ x=-3a $

Replacing:

$m(-3a-a)^{3} -(-3a+a)^{2} =0$

$m=-\frac{1}{16a} $

then the values for m wich gives two or three equal roots to the equation is:

$ -\frac{1}{16a} , -\frac{2}{27a} $
\end{solution}



\begin{solution}[by \href{https://artofproblemsolving.com/community/user/25405}{AndrewTom}]
	Thanks. The equation is in $x$ and $a$ is a (positive) constant.

My solution is as follows:

Consider the curve $m= \frac{(x+a)^{2}}{(x-a)^{2}}$

We get three distinct roots iff the horizontal line $y=m$ crosses the curve in three distinct points.

Vertical asymptote: $x=a$; when $x=0$,  $y= \frac{-1}{a}$; when $y=0$  $x=-a$.

The derivative is zero for $x=-a$ and $x=-5a$,

The x-axis is tangential to the curve at $(-a, 0)$ and $(-5a, \frac{-2}{27a})$ is a local minimum.

So we get three distinct rots for $\frac{-2}{27a} < m < 0$.

I wonder if there are alternative (and better) solutions which is why I posted it here and not in the calculus forum.
\end{solution}



\begin{solution}[by \href{https://artofproblemsolving.com/community/user/29428}{pco}]
	Here is my (calculus) method, more complex, in my opinion, than yours :

$m\ne 0$ else the equation has a unique root $-a$

Then the equation is a cubic and a cubic $f(x)$ has three distinct root iff $f'(x)$ has two distinct roots $r_1,r_2$ and $f(r_1)f(r_2)<0$

$f(x)=m(x-a)^3-(x+a)^2$ and so $f'(x)=3m(x-a)^2-2(x+a)$ $=3mx^2-2x(3am+1)+3ma^2-2a$

Discriminant is $4((3am+1)^2-3m(3ma^2-2a))=4(12am+1)$ and so we first need $12am+1>0$

Let us write $12am+1=u^2$ with $u>0$ and the two roots of $f'(x)$ are $r_1=\frac{3am+1- u}{3m}$ and $r_2=\frac{3am+1+ u}{3m}$

$f(r_1)=m\left(\frac{1- u}{3m}\right)^3-\left(\frac{6am+1- u}{3m}\right)^2$

Writing $6am=\frac{u^2-1}2$ and simplifying, we get $108m^2f(r_1)=(u-1)^3(1+3u)$

$f(r_2)=m\left(\frac{1+ u}{3m}\right)^3-\left(\frac{6am+1+ u}{3m}\right)^2$

Writing $6am=\frac{u^2-1}2$ and simplifying, we get $108m^2f(r_2)=(u+1)^3(3u-1)$

And so the constraint is $(u^2-1)^3(9u^2-1)<0$ and so $u^2=12am+1\in(\frac 19,1)$

Hence the answer $\boxed{am\in\left(-\frac 2{27},0\right)}$
\end{solution}
*******************************************************************************
-------------------------------------------------------------------------------

\begin{problem}[Posted by \href{https://artofproblemsolving.com/community/user/127130}{Megadeth}]
	Show that for any two real numbers $a\neq b$ the polynomial $ P(x) = (a-b)x^{n-1} + (a^2 - b^2)x^{n-2} + \cdots + (a^{n}- b^{n})  $ has not more than one real root.
	\flushright \href{https://artofproblemsolving.com/community/c6h485853}{(Link to AoPS)}
\end{problem}



\begin{solution}[by \href{https://artofproblemsolving.com/community/user/114585}{anonymouslonely}]
	but if $ n $ is odd then it will have an even number of real roots so... are you sure?
\end{solution}



\begin{solution}[by \href{https://artofproblemsolving.com/community/user/29428}{pco}]
	\begin{tcolorbox}Show that for any two real numbers $a\neq b$ the polynomial $ P(x) = (a-b)x^{n-1} + (a^2 - b^2)x^{n-2} + \cdots + (a^{n}- b^{n})  $ has only one real root.\end{tcolorbox}
Yes, as anonymouslonely has shown, this is a wrong problem :

Choose $n=1$ and obviously the polynomial is $2(a-b)$ and has no root

Choose $n=3$ and obviously the polynomial is $(a-b)(x^2+(a+b)x+a^2+ab+b^2)$ and has no real root
\end{solution}



\begin{solution}[by \href{https://artofproblemsolving.com/community/user/127130}{Megadeth}]
	\begin{tcolorbox}but if $ n $ is odd then it will have an even number of real roots so... are you sure?\end{tcolorbox}

Sorry I had written it wrong
\end{solution}



\begin{solution}[by \href{https://artofproblemsolving.com/community/user/127130}{Megadeth}]
	\begin{tcolorbox}[quote="Megadeth"]Show that for any two real numbers $a\neq b$ the polynomial $ P(x) = (a-b)x^{n-1} + (a^2 - b^2)x^{n-2} + \cdots + (a^{n}- b^{n})  $ has only one real root.\end{tcolorbox}
Yes, as anonymouslonely has shown, this is a wrong problem :

Choose $n=1$ and obviously the polynomial is $2(a-b)$ and has no root

Choose $n=3$ and obviously the polynomial is $(a-b)(x^2+(a+b)x+a^2+ab+b^2)$ and has no real root\end{tcolorbox}

Sorry , i had written it Wrong
\end{solution}
*******************************************************************************
-------------------------------------------------------------------------------

\begin{problem}[Posted by \href{https://artofproblemsolving.com/community/user/110552}{youarebad}]
	Let $P(x)$ be a polynomial with integer coefficient. Show that if $Q(x) = P(x) + 12$ has at least six distinct integer roots, then $P(x)$ has no integer roots.
	\flushright \href{https://artofproblemsolving.com/community/c6h485927}{(Link to AoPS)}
\end{problem}



\begin{solution}[by \href{https://artofproblemsolving.com/community/user/29428}{pco}]
	\begin{tcolorbox}Let $P(x)$ be a polynomial with integer coefficient. Show that if $Q(x) = P(x) + 12$ has at least six distinct integer roots, then $P(x)$ has no integer roots.\end{tcolorbox}
So $P(x)=(x-a_1)(x-a_2)(x-a_3)(x-a_4)(x-a_5)(x-a_6)H(x)-12$

And $P(r)=0$ $\implies$ $(r-a_1)(r-a_2)(r-a_3)(r-a_4)(r-a_5)(r-a_6)H(r)=12$

And so $12$ must be the product of six distinct integers, which is impossible
\end{solution}



\begin{solution}[by \href{https://artofproblemsolving.com/community/user/151117}{expotential}]
	What if $\ H(r) $ removes some of intergers $\ (r-a_{i}) $   $\ (i=1,2,3,...,5,6) $ ?
I think $\ H(r) $ can be a rational number.
For example, $\ (x-2)(x-4)(x-8) \times (x-\frac{1}{2}) $ (rough way)
Therefore, we have to prove there is no this case.

Edit : sorry. It's trivial $\ H(r) $ is interger. :blush:
\end{solution}



\begin{solution}[by \href{https://artofproblemsolving.com/community/user/29428}{pco}]
	\begin{tcolorbox}What if $\ H(r) $ removes some of intergers $\ (r-a_{i}) $   $\ (i=1,2,3,...,5,6) $ ?
I think $\ H(r) $ can be a rational number.
For example, $\ (x-2)(x-4)(x-8) \times (x-\frac{1}{2}) $ (rough way)
Therefore, we have to prove there is no this case.\end{tcolorbox}
$H(x)\in\mathbb Z[X]$
\end{solution}
*******************************************************************************
-------------------------------------------------------------------------------

\begin{problem}[Posted by \href{https://artofproblemsolving.com/community/user/141363}{alibez}]
	Find all polynomials $p$ such that $p(x^{2})= p(x)^{2}$.
	\flushright \href{https://artofproblemsolving.com/community/c6h487150}{(Link to AoPS)}
\end{problem}



\begin{solution}[by \href{https://artofproblemsolving.com/community/user/86443}{roza2010}]
	$p(x)=x^c,c={\text{const}}$
\end{solution}



\begin{solution}[by \href{https://artofproblemsolving.com/community/user/29428}{pco}]
	\begin{tcolorbox}Find all polynomials $p$ such that $p(x^{2})= p(x)^{2}$.\end{tcolorbox}
If $P(x)=c$ constant, we get $c=c^2$ and so the two solutions $P(x)=0$ $\forall x$ and $P(x)=1$ $\forall x$

If $P(x)$ is not constant, let $a_nx^n$ with $n>0$ and $a_n\ne 0$ the greatest degree summand of $P(x)$
Comparing the greatest degrees summands in $P(x^2)$ and $P(x)^2$, we get $a_n=1$

If $P(x)$ is not constant and has a unique summand, we get  $P(x)=x^n$ which indeed are solutions.

If $P(x)$ is not constant and has at least two summands, let $x^n+a_px^p$ with $n>p\ge 0$ and $a_p\ne 0$ the two greatest degrees summands.
Then the two greatest degrees summands of $P(x^2)$ are $x^{2n}+a_px^{2p}$ while the two greatest degrees summands of $P(x)^2$ are $x^{2n}+2a_px^{n+p}$, impossible.

\begin{bolded}Hence the solutions\end{underlined}\end{bolded} :
$P(x)=0$ $\forall x$
$P(x)=x^n$ $\forall x$ and for any integer $n\ge 0$
\end{solution}



\begin{solution}[by \href{https://artofproblemsolving.com/community/user/109704}{dien9c}]
	\begin{tcolorbox}Find all polynomials $p$ such that $p(x^{2})= p(x)^{2}$.\end{tcolorbox}
Use it we can solve the follow problem
Find all polynomias $P(x)$ such that
\[P(x^2)+x(3P(x)+P(-x))=P(x)^2+2x^2\]
\end{solution}



\begin{solution}[by \href{https://artofproblemsolving.com/community/user/112168}{sir_hoang}]
	\begin{tcolorbox}[quote="alibez"]Find all polynomials $p$ such that $p(x^{2})= p(x)^{2}$.\end{tcolorbox}
Use it we can solve the follow problem
Find all polynomias $P(x)$ such that
\[P(x^2)+x(3P(x)+P(-x))=P(x)^2+2x^2\]\end{tcolorbox}
VMO 2006 :)
\[P(x) = x;P(x) = 2x;P(x) = 2x + 1;P(x) = {x^{2n + 1}} + x;P(x) = {x^{2n}} + 2x\]
\end{solution}
*******************************************************************************
-------------------------------------------------------------------------------

\begin{problem}[Posted by \href{https://artofproblemsolving.com/community/user/99931}{jaspion}]
	Find all rational numbers $ p, q, r$ such that
$pcos\frac{\pi}{7}+qcos\frac{2\pi}{7}+rcos\frac{3\pi}{7}=1$
	\flushright \href{https://artofproblemsolving.com/community/c6h488611}{(Link to AoPS)}
\end{problem}



\begin{solution}[by \href{https://artofproblemsolving.com/community/user/29428}{pco}]
	\begin{tcolorbox}Find all rational numbers $ p, q, r$ such that
$pcos\frac{\pi}{7}+qcos\frac{2\pi}{7}+rcos\frac{3\pi}{7}=1$\end{tcolorbox}
For easier writing, let $c_k=\cos\frac{k\pi}7$

It's easy to get that $c_1-c_2+c_3=\frac 12$ and so that a solution is $(p,q,r)=(2,-2,2)$

So any other solution is $(2+a,-2+b,2+c)$ such that $ac_1+bc_2+cc_3=0$
But $c_3=4c_1^3-3c_1$ and $c_2=2c_1^2-1$
And so $c_1$ is root of $4cx^3+2bx^2+(a-3c)x-b=0$

But it's not very difficult to establish that minimal monic polynomial of $\mathbb Q[X]$ whose $c1$ is solution is $x^3-\frac 12x^2-\frac 12x+\frac 18$

And so, looking at coefficients of degrees $2$ and $0$, we get that $a=b=c=0$

And so the unique solution $\boxed{(p,q,r)=(2,-2,2)}$
\end{solution}
*******************************************************************************
-------------------------------------------------------------------------------

\begin{problem}[Posted by \href{https://artofproblemsolving.com/community/user/127783}{Sayan}]
	Let $n$ be a positive integer and let $(1+iT)^n=f(T)+ig(T)$ where $i$ is the square root of $-1$, and $f$ and $g$ are polynomials with real coefficients. Show that for any real number $k$ the equation $f(T)+kg(T)=0$ has only real roots.
	\flushright \href{https://artofproblemsolving.com/community/c6h488784}{(Link to AoPS)}
\end{problem}



\begin{solution}[by \href{https://artofproblemsolving.com/community/user/29428}{pco}]
	\begin{tcolorbox}Let $n$ be a positive integer and let $(1+iT)^n=f(T)+ig(T)$ where $i$ is the square root of $-1$, and $f$ and $g$ are polynomials with real coefficients. Show that for any real number $k$ the equation $f(T)+kg(T)=0$ has only real roots.\end{tcolorbox}
$f(x)=\frac{(1+ix)^n+(1-ix)^n}2$ and $g(x)=i\frac{(1-ix)^n-(1+ix)^n}2$

And so $f(x)+kg(x)=(1+ix)^n\frac{1-ki}2+(1-ix)^n\frac{1+ki}2$

1) case where $k\ne 0$ 
================
Equation is $(1+ix)^n(1-ki)+(1-ix)^n(1+ki)=0$ and so, since $x=-i$ is not a solution :

$\left(\frac{1+ix}{1-ix}\right)^n=\frac{1+ki}{-1+ki}$ $=e^{it}$ for some real $t$ which is not an integer multiple of $\pi$

And so $\frac{1+ix}{1-ix}=e^{ia_k}$ where $\{a_k=\frac{t+2k\pi}n\}_{k=0}^{n-1}$ is an arithmetic sequence of real numbers with common difference $\frac{2\pi}n$ 

And so $x=\tan\frac{a_k}2$ (which is defined since no $a_k$ is an integer multiple of $\pi$)

And so $f(x)+kg(x)$ is a real polynomial with degree $n$ and we got $n$ distinct real roots.
Hence the result.

2) case where $k=0$
==============
Equation is $(1+ix)^n+(1-ix)^n=0$ and so, since $x=-i$ is not a solution :

$\left(\frac{1+ix}{1-ix}\right)^n=e^{i\pi}$

And so $\frac{1+ix}{1-ix}=e^{i\frac{(2k+1)\pi}n}$ wher $k=0,..,n-1$

And so, if $n$ is odd, $n-1$ real solutions $\tan \frac{(2k+1)\pi}{2n}$ where $k=0,...,n-1$ except $k=\frac{n-1}2$

And, if $n$ is even, $n$ real solutions $\tan \frac{(2k+1)\pi}{2n}$ where $k=0,...,n-1$

And since $f(x)$ is a polynomial of degree $n$ if $n$ is even, and of degree $n-1$ if $n$ is odd, we got the result.
\end{solution}



\begin{solution}[by \href{https://artofproblemsolving.com/community/user/140796}{mathbuzz}]
	PCO wrote -- $f(x)=\frac{(1+ix)^n+(1-ix)^n}2$ and $g(x)=i\frac{(1-ix)^n-(1+ix)^n}2$
after this , we can just take mod and it turns out that ,
$\mid 1+ix \mid =\mid 1-ix \mid$ which trivially gives that $x$ is real .
\end{solution}
*******************************************************************************
-------------------------------------------------------------------------------

\begin{problem}[Posted by \href{https://artofproblemsolving.com/community/user/10045}{socrates}]
	Does there exist a polynomial $P(x)$ such that
a) $P(\sin x) = \cos 2004 x$

b) $P(\sin x) = \cos 2005 x$  for all $ x \in  \Bbb{R}?$
	\flushright \href{https://artofproblemsolving.com/community/c6h489396}{(Link to AoPS)}
\end{problem}



\begin{solution}[by \href{https://artofproblemsolving.com/community/user/29428}{pco}]
	\begin{tcolorbox}Does there exist a polynomial $P(x)$ such that
a) $P(\sin x) = \cos 2004 x$  for all $ x \in  \Bbb{R}?$\end{tcolorbox}It's easy to show that $\exists H(x)\in\mathbb R[X]$ such that $\cos 1002x=H(\cos x)$ $\forall x$
Just choose then $P(x)=H(1-2x^2)$

\begin{tcolorbox}Does there exist a polynomial $P(x)$ such that
b) $P(\sin x) = \cos 2005 x$  for all $ x \in  \Bbb{R}?$\end{tcolorbox}$x=0$ $\implies$ $P(0)=1$
$x=\pi$ $\implies$ $P(0)=-1$
And so no such polynomial.
\end{solution}



\begin{solution}[by \href{https://artofproblemsolving.com/community/user/64716}{mavropnevma}]
	That polynomial $H$ is in fact $T_{1002}$, the Tchebysheff polynomyal defined by $T_{1002}(t) = \cos(1002\arccos t)$, so for $x=\arccos t$ we have $T_{1002} (\cos x) = \cos 1002 x$.
\end{solution}
*******************************************************************************
-------------------------------------------------------------------------------

\begin{problem}[Posted by \href{https://artofproblemsolving.com/community/user/125907}{mikolez}]
	Let $P(x)$ be a real quadratic trinomial, so that for all $x\in \mathbb{R}$ the inequality $P(x^3+x)\geq P(x^2+1)$ holds. Find the sum of the roots of $P(x)$.

\begin{italicized}Proposed by A. Golovanov, M. Ivanov, K. Kokhas\end{italicized}
	\flushright \href{https://artofproblemsolving.com/community/c6h489900}{(Link to AoPS)}
\end{problem}



\begin{solution}[by \href{https://artofproblemsolving.com/community/user/64716}{mavropnevma}]
	Let $P(x) = ax^2+bx+c$. Then $\Delta(x) = P(x^3+x) - P(x^2+1) =$ $ P(x(x^2+1)) - P(x^2+1) =$ $ (x-1)(x^2+1)(a(x+1)(x^2+1) + b)$. Since we need $\Delta(x) \geq 0$ for all $x$, we need $x-1 \mid a(x+1)(x^2+1) + b$, i.e. $4a + b = 0$. Therefore $b= -4a$, and so by Viète's relation, the sum of the roots of $P(x)$ is $\boxed{x_1 + x_2 = -\dfrac {b} {a} = 4}$.
Indeed $\Delta(x) = a(x-1)^2(x^2+1)(x^2+2x +3) \geq 0$ for all $x$ (since the discriminant of $x^2+2x +3$ is negative), provided we take $a>0$. Therefore all such polynomials are of the form $P(x) = k^2(x^2 - 4x + C)$.
\end{solution}



\begin{solution}[by \href{https://artofproblemsolving.com/community/user/29428}{pco}]
	\begin{tcolorbox}Let $P(x)$ be a quadratic trinomial, so that for all $x\in \mathbb{R}$ the inequality $P(x^3+x)\geq P(x^2+1)$ holds. Find the sum of the roots of $P(x)$.\end{tcolorbox}
Another method (a bit too late) :)

Writing $f(x)=P(x^3+x)-P(x^2+1)$, we get $f(x)\ge 0$, $C_{\infty}$ and $f(1)=0$ and so $f'(1)=0$ and so $P'(2)=0$

Writing $P(x)=ax^2+bx+c$ with $a\ne 0$, then $P'(2)=0$ becomes $4a+b=0$ and so $-\frac ba=4$

And so, if such a quadratic exists the sum of roots is $\boxed{4}$

Btw, such a polynomial exists. For example $P(x)=x^2-4x$
\end{solution}
*******************************************************************************
-------------------------------------------------------------------------------

\begin{problem}[Posted by \href{https://artofproblemsolving.com/community/user/134102}{jonny}]
	Prove that $x^{12}-x^{9}+x^4-x+1 > 0$
	\flushright \href{https://artofproblemsolving.com/community/c6h490139}{(Link to AoPS)}
\end{problem}



\begin{solution}[by \href{https://artofproblemsolving.com/community/user/67669}{minhphuc.v}]
	Consider $x \ge 1$ and $x<1$. It is easy to say that 
$ x^{12}-x^{9}+x^{4}-x+1 > 0 $
\end{solution}



\begin{solution}[by \href{https://artofproblemsolving.com/community/user/134102}{jonny}]
	\begin{bolded}'Thanks' \end{bolded}

But I want to solve Using Completing Square Method

Like $x^{12}-x^9+x^4-x+1 = A^2+B^2+C^2>0\forall x\in \mathbb{R}$
\end{solution}



\begin{solution}[by \href{https://artofproblemsolving.com/community/user/10153}{10000th User}]
	After much algebraic play and testing out, this is what I've gotten, though it still doesn't prove much how the polynomial is positive for all reals:

$\begin{aligned}x^{12}-x^9+x^4-x+1
&=(x^{12}-2x^9+x^6)+(x^9-2x^6+x^3)+(x^6+x^4-x^3-x+1)\\
&=x^6(x^3-1)^2+x^3(x^3-1)^2+(x^6+x^4-x^3-x+1)\\
&=x^6(x^3-1)^2+x^3(x^3-1)^2+(x^6-2x^3+1)+(x^4+x^3-x)\\
&=x^6(x^3-1)^2+x^3(x^3-1)^2+(x^3-1)^2+(x^4+x^3-x)\\
&=(x^6+x^3+1)(x^3-1)^2+(x^4-2x^2+1)+(x^3+2x^2-x-1)\\
&=(x^6+x^3+1)(x^3-1)^2+(x^2-1)^2+(x^3+2x^2-x-1)\\
&=\left(x^3+\frac12\right)^2(x^3-1)^2+\frac34(x^3-1)^2+(x^2-1)^2+(x^3+2x^2-x-1)\\
&=\left(x^3+\frac12\right)^2(x^3-1)^2+(x^2-1)^2+\left(\frac34x^6-\frac12x^3+2x^2-x-\frac14\right)\\
&=\left(x^3+\frac12\right)^2(x^3-1)^2+(x^2-1)^2+\frac34\left(x^6-\frac23x^3+\frac19\right)+\left(2x^2-x-\frac14-\frac1{12}\right)\\
&=\left(x^3+\frac12\right)^2(x^3-1)^2+(x^2-1)^2+\frac34\left(x^3-\frac13\right)^2+2\left(x^2-\frac12x+\frac1{16}\right)-\left(\frac14+\frac1{12}+\frac18\right)\\
&=\left(x^3+\frac12\right)^2(x^3-1)^2+(x^2-1)^2+\frac34\left(x^3-\frac13\right)^2+2\left(x-\frac14\right)^2-\frac{11}{24}\\
\end{aligned}$
\end{solution}



\begin{solution}[by \href{https://artofproblemsolving.com/community/user/29428}{pco}]
	\begin{tcolorbox}Prove that $x^{12}-x^{9}+x^4-x+1 > 0$\end{tcolorbox}
$x^{12}-x^9+x^4-x+1=$ $\left(x^6-\frac{x^3}2-\frac 18\right)^2$ $+\left(x^2-\frac x{16}-\frac 12\right)^2$ $+\frac{255}{256}\left(x-\frac{136}{255}\right)^2$ $+\frac{433}{960}$
\end{solution}



\begin{solution}[by \href{https://artofproblemsolving.com/community/user/10153}{10000th User}]
	:o  That is amazing pco! I'm impressed  :omighty:
\end{solution}



\begin{solution}[by \href{https://artofproblemsolving.com/community/user/139608}{ArianitZeqiri}]
	but with it  you can say that \[x^12-x^9+x^4-x+1\geq 0\]
it can be 0 you can't say that it is >0 .
\end{solution}



\begin{solution}[by \href{https://artofproblemsolving.com/community/user/29428}{pco}]
	\begin{tcolorbox}but with it  you can say that \[x^12-x^9+x^4-x+1\geq 0\]
it can be 0 you can't say that it is >0 .\end{tcolorbox}
I dont understand what you mean. With my equality, I got $x^{12}-x^9+x^4-x+1\ge \frac{433}{960}>0$
\end{solution}



\begin{solution}[by \href{https://artofproblemsolving.com/community/user/20909}{pankajsinha}]
	\begin{tcolorbox}[quote="jonny"]Prove that $x^{12}-x^{9}+x^4-x+1 > 0$\end{tcolorbox}
$x^{12}-x^9+x^4-x+1=$ $\left(x^6-\frac{x^3}2-\frac 18\right)^2$ $+\left(x^2-\frac x{16}-\frac 12\right)^2$ $+\frac{255}{256}\left(x-\frac{136}{255}\right)^2$ $+\frac{433}{960}$\end{tcolorbox}

Just great.
But how can one ever think of something like this.
\end{solution}



\begin{solution}[by \href{https://artofproblemsolving.com/community/user/10153}{10000th User}]
	I was able to explain to myself how he chose the setup for $\left(x^6-\frac{x^3}2-\frac18\right)^2$ because it takes care of the terms $x^{12}-x^9$ and vanishing the $x^6$ terms. The same can be deduced for the term $-\frac{x}{16}$ to eliminate the terms $x^4-\frac18x^3$, but how he got $-\frac12$ in that second square still beats me.... after that the rest is just cake-simple complete the square to finish up.
\end{solution}



\begin{solution}[by \href{https://artofproblemsolving.com/community/user/29428}{pco}]
	As 10000th User said, choosing $x^6-\frac {x^3}2-\frac 18$ is rather normal. And also the $x^2-\frac x{16}$ and indeed I found them in that way.

For the $-\frac 12$, what I did is :

$x^{12}-x^9+x^4-x+1$ $-\left(x^6-\frac{x^3}2-\frac 18\right)^2$ $-\left(x^2-\frac x{16}+a\right)^2$ $=-(2a+\frac 1{256})x^2+(\frac a8-1)x+\frac{63}{64}-a^2$

And so we just have to choose $a$ such that this quadratic has a positive leading coefficient and a negative discriminant. So :

$a<-\frac 1{512}$ and $(\frac a8-1)^2+4(2a+\frac 1{256})(\frac {63}{64}-a^2)<0$

$\iff$ $a<-\frac 1{512}$ and $-8a^3+\frac{61a}8+\frac{4159}{4096}<0$

And studying this cubic shows that some suitable values exist in $(-1,0)$ so that $a=-\frac 12$ is a "natural" try.

 ============================================

Btw, notice that this equality is my answer to the OP question "But I want to solve Using Completing Square Method Like $x^{12}-x^9+x^4-x+1 = A^2+B^2+C^2>0\forall x\in \mathbb{R}$"

The simpliest solution to the original problem is the one suggested by minhphuc.v in his\/her post :

If $x\ge 1$, then $x^{12}\ge x^9$ and $x^4\ge x$ and so $x^{12}-x^9+x^4-x+1$ $=(x^{12}-x^9)+(x^4-x)+1\ge 1>0$
If $x< 1$, then $x^{12}\ge 0$ and $x^9\le x^4$ and so $x^{12}-x^9+x^4-x+1$ $=x^{12}+(x^4-x^9)+(1-x)> 0$
\end{solution}
*******************************************************************************

-------------------------------------------------------------------------------

\begin{problem}[Posted by \href{https://artofproblemsolving.com/community/user/115}{A1lqdSchool}]
	Find all functions $f,g,h: \mathbb R \to \mathbb R$ such that
\[ f(x+y^3)+g(x^3+y)=h(xy)\]
for all $x,y \in \mathbb R$.
	\flushright \href{https://artofproblemsolving.com/community/c6h4100}{(Link to AoPS)}
\end{problem}



\begin{solution}[by \href{https://artofproblemsolving.com/community/user/36491}{Adriana N.}]
	Find all functions $ f,g,h: R->R$ such that:
$ f(x+y^3)+g(x^3+y)=h(xy)$,for all $ x,y$ in $ R$
\end{solution}



\begin{solution}[by \href{https://artofproblemsolving.com/community/user/29428}{pco}]
	\begin{tcolorbox}Find all functions $ f,g,h: R - > R$ such that:
$ f(x + y^3) + g(x^3 + y) = h(xy)$,for all $ x,y$ in $ R$\end{tcolorbox}

Here is a rather ugly solution. I hope someone will propose a better one :

1) $ f(x)=g(x)+c$
Swapping $ x$ and $ y$, we get $ f(x+y^3)+g(x^3+y)=f(y+x^3)+g(y^3+x)$.
Let $ k(x)=f(x)-g(x)$. We get $ k(x+y^3)=k(x^3+y)$
The system of equations : $ x+y^3=u$ and $ x^3+y=v$ is equivalent to :
$ y=v-x^3$ and $ x+(v-x^3)^3=u$
Since the degree of the second equation is odd ($ 9$), this system always has real solution and then $ k(u)=k(v)$ $ \forall u,v$
So $ k(x)=c$ constant
Q.E.D.

2) $ f(x)=$constant
The new equation is now $ f(x+y^3)+f(x^3+y)=h(xy)-c$
And so, replacing $ x$ with $ \frac xz$ and $ y$ with $ yz$ : $ f(\frac xz+y^3z^3)+f(\frac{x^3}{z^3}+yz)=h(xy)-c$

And so : $ f(x+y^3)+f(x^3+y)=$ $ f(\frac xz+y^3z^3)+f(\frac{x^3}{z^3}+yz)$ $ \forall x,y$, $ \forall z\neq 0$

Let now $ y=\frac{x^3(z^2+z+1)}{z^3}$. This value implies $ x^3+y=\frac{x^3}{z^3}+yz$ and the above equality becomes :

$ f(x+y^3)=f(\frac xz+y^3z^3)$ $ \forall x$ $ \forall z\neq 0$ and with $ y=\frac{x^3(z^2+z+1)}{z^3}$

Then we'll try to see if the following system has real solutions :
$ y=\frac{x^3(z^2+z+1)}{z^3}$
$ x+y^3=u$
$ \frac xz+y^3z^3=v$

This system is equivalent to :
$ y=\frac{x^3(z^2+z+1)}{z^3}$
$ x+y^3=u$
$ \frac xz+y^3z^3=v$ $ \iff$ $ \frac xz+(u-x)z^3=v$ $ \iff$ $ x=z\frac{v-uz^3}{1-z^4}$ if we consider now $ z\notin\{-1,0,+1\}$

So :
$ x=z\frac{v-uz^3}{1-z^4}$
$ y=\frac{x^3(z^2+z+1)}{z^3}$
$ z\frac{v-uz^3}{1-z^4}+\left(\frac{(z\frac{v-uz^3}{1-z^4})^3(z^2+z+1)}{z^3}\right)^3=u$

This last equation may be written :

$ P(z)=z(v-uz^3)(1-z^4)^8+(v-uz^3)^9(z^2+z+1)^3-u(1-z^4)^9=0$
The highest power term of $ P(z)$ is $ (v-u^9)z^{33}$ and so this equation always has a solution (odd degree) whenever $ v\neq u^9$
If we want that this solution $ \notin\{-1,0,+1\}$, we also need to add some constraints :
$ P(0)\neq 0$ $ \iff$ $ v^9-u\neq 0$
$ P(1)\neq 0$ $ \iff$ $ (v-u)^9\neq 0$
$ P(-1)\neq 0$ $ \iff$ $ (v+u)^9\neq 0$

So we have $ f(u)=f(v)$ $ \forall u,v$ such that $ v\neq u^9$ and $ u\neq v^9$ and $ u\neq -v$

From this, it's immediate to conclude $ f(u)=f(v)$ $ \forall u,v$
Q.E.D.

3) Synthesis of solutions :
$ f(x)=a$
$ g(x)=b$
$ h(x)=a+b$
\end{solution}
*******************************************************************************
-------------------------------------------------------------------------------

\begin{problem}[Posted by \href{https://artofproblemsolving.com/community/user/11106}{tranthanhnam}]
	Find all functions $f: \mathbb R \to \mathbb R$ such that \[ f( xf(x) + f(y) ) = f^2(x) + y \] for all $x,y\in \mathbb R$.
	\flushright \href{https://artofproblemsolving.com/community/c6h56415}{(Link to AoPS)}
\end{problem}



\begin{solution}[by \href{https://artofproblemsolving.com/community/user/3598}{Agrippina}]
	Possibly I'm missing something obvious, but -- 

surely setting $x=0$ would imply that $y=f(0)-f(0)^2$ for all $y$?
\end{solution}



\begin{solution}[by \href{https://artofproblemsolving.com/community/user/1598}{Arne}]
	I'm sure he meant $f(xf(x) + f(y)) = f(x)^2 + y$.
\end{solution}



\begin{solution}[by \href{https://artofproblemsolving.com/community/user/13}{enescu}]
	That's strange..:) I thought that this problem is already well known, since it is the only problem proposed twice to the same international competition. First, it was Problem 4 (assumed the most difficult) of Balkan Math Olympiad 1997, proposed by Bulgaria. Second, it was Problem 1 (assumed the easiest, but that's normal, now) of Balkan Math Olympiad 2000, proposed by Albania ;)
\end{solution}



\begin{solution}[by \href{https://artofproblemsolving.com/community/user/68920}{prester}]
	\begin{tcolorbox}Find all functions $ f: \mathbb R \to \mathbb R$ such that
\[ f( xf(x) + f(y) ) = f^2(x) + y\]
for all $ x,y\in \mathbb R$.\end{tcolorbox}

Denote with $ P(x,y)$ the $ f( xf(x) + f(y) ) = f^2(x) + y \qquad (o)$

$ P(0,y)$ $ \implies f(f(y))=f^2(0)+y\qquad(*)$

Therefore $ f(x)$ is a bijection of $ \mathbb {R}$.

So $ \exists u\in\mathbb{R}$ such that $ f(u)=0$

$ P(u,y)$ $ \implies f(f(y))=y\qquad(**)$

So $ f(0)=0$ and $ u=0$

$ P(f(x),y)$ $ \implies f(xf(x) + f(y))=x^2+y\qquad (***)$

Combining $ (o)$ and $ (***)$ we have $ f^2(x)=x^2 \qquad \forall x\in\mathbb R$

Therefore let $ x_0 \in \mathbb{R}$ we have either $ f(x_0)=x_0$ or $ f(x_0)=-x_0$

Let $ x_0 \in \mathbb{R}$ such that $ f(x_0)=-x_0$. From $ (**)$ we have $ f(f(x_0))=f(-x_0)=x_0$
and this excludes solutions as $ f(x)=|x|$ or $ f(x)=-|x|$.

Now we must exclude other solutions that are not continuos over $ \mathbb{R}$ 

Let $ x_0,y_0 \in \mathbb{R}$ with $ f(x_0)=x_0$ and $ f(y_0)=-y_0$.

$ P(x_0,-y_0)$ $ \implies f(x_0^2+y_0)=x_0^2-y_0$ and this is false because $ f(x)=x$ or $ f(x)=-x$. 

$ P(y_0,x_0)$ $ \implies f(-y_0^2+x_0)=y_0^2+x_0$ and this is false because $ f(x)=x$ or $ f(x)=-x$. 

So we may conclude that $ \forall x\in \mathbb{R}$ we have either $ f(x)=x$ or $ f(x)=-x$

Hence the solution are 
${ 1) f(x)=x\qquad \forall x\in\mathbb{R}}$
${ 2) f(x)=-x\qquad \forall x\in\mathbb{R}}$
\end{solution}



\begin{solution}[by \href{https://artofproblemsolving.com/community/user/98620}{luca-97}]
	First look at that if x = 0 → f (f (and)) = c + f (0) where c is a constant fact is f(0)square.
Then if we do and y =-(c) →f (f(-c)) = 0 then there is a real such that f (a) = 0
Now if x = a: f (af (a) + f (and)) = y → f (f (and)) = and → c = 0 → f (0) = 0 now if y = 0:
F (xf (x)) = (f (x)) 2 = f (f ((f (x)) 2)) by the injectivity xf (x) = f (f (x) 2) changing x for f (x)
f (x) f (f (x)) = f ((f (f (x))) 2) = f (x) = f (f (x) 2) →x2 = f (x) 2 and if we assume that there is a f (n) = n, f (m) = - m
where n, m ≠0 → x = n, and m =
f(N2-m) = n2 + m → n4 + m2 - 2nm2 = n4 + m2 + 2nm2 → 4nm2 = 0 (→←).
This leads wings solutions f (x) = x, f (x) =-x.
\end{solution}



\begin{solution}[by \href{https://artofproblemsolving.com/community/user/84677}{andreass}]
	Well, after proving that $f(x)^2=x^2$  $(*)$ one could square the original functional equation and get
$f(xf(x)+f(y))^2=f(x)^4+2f(x)^2y+y^2$
$\Rightarrow x^2f(x)^2+2xf(x)f(y)+f(y)^2=x^2f(x)^2+2f(x)^2y+y^2$.
This leaves us with $xf(x)f(y)=f(x)^2y \Rightarrow \frac{f(x)}{x}=\frac{f(y)}{y}$ assuming $x$ and $y$ are non-zero.
Hence $f(x)=kx$  $\forall x\in R$ where $k$ is a constant (for $0$ which was excluded above we already know that $f(0)=0$) and by plugging it back to $(*)$ we get that either $k=1$ or $k=-1$ hence $f(x)\equiv x$ or $f(x)\equiv -x$ and by a simple substitution to the original equation we see that both are valid.
\end{solution}



\begin{solution}[by \href{https://artofproblemsolving.com/community/user/61542}{AwesomeToad}]
	Does $f^2(x)$ here mean $f(f(x))$ or $(f(x))^2$?
\end{solution}



\begin{solution}[by \href{https://artofproblemsolving.com/community/user/141397}{subham1729}]
	Putting $x=0$ we get it's a bijective function. so exist a such that $f(a)=0$
it's easy to show now $a=0$
so we have $f(f(x))=x$ and $f(xf(x))=f^2(x)$
put $x$ in place of $f(x)$ get $f^2(x)=x^2$
not directly but it can be said easily $f(x)=x ,-x$
\end{solution}



\begin{solution}[by \href{https://artofproblemsolving.com/community/user/82678}{hEatLove}]
	\begin{tcolorbox}Well, after proving that $f(x)^2=x^2$  $(*)$ one could square the original functional equation and get
$f(xf(x)+f(y))^2=f(x)^4+2f(x)^2y+y^2$
$\Rightarrow x^2f(x)^2+2xf(x)f(y)+f(y)^2=x^2f(x)^2+2f(x)^2y+y^2$.
This leaves us with $xf(x)f(y)=f(x)^2y \Rightarrow \frac{f(x)}{x}=\frac{f(y)}{y}$ assuming $x$ and $y$ are non-zero.
Hence $f(x)=kx$  $\forall x\in R$ where $k$ is a constant (for $0$ which was excluded above we already know that $f(0)=0$) and by plugging it back to $(*)$ we get that either $k=1$ or $k=-1$ hence $f(x)\equiv x$ or $f(x)\equiv -x$ and by a simple substitution to the original equation we see that both are valid.\end{tcolorbox}

That's correct . But you should prove injectivity first!. But it's easy . Congratz anyway :P
\end{solution}



\begin{solution}[by \href{https://artofproblemsolving.com/community/user/64716}{mavropnevma}]
	\begin{tcolorbox}Putting $x=0$ we get it's a bijective function. so exist a such that $f(a)=0$
it's easy to show now $a=0$
so we have $f(f(x))=x$ and $f(xf(x))=f^2(x)$
put $x$ in place of $f(x)$ get $f^2(x)=x^2$
not directly but it can be said easily $f(x)=x ,-x$\end{tcolorbox}
You are trivializing the whole problem; the issue is to prove that from $|f(x)| = |x|$ follows either $f(x) = x$ for all $x$, or $f(x) = -x$ for all $x$, and others struggle to prove just this, by various methods, while you dismiss it by your "it can be said easily".
\end{solution}



\begin{solution}[by \href{https://artofproblemsolving.com/community/user/141397}{subham1729}]
	\begin{tcolorbox}[quote="subham1729"]Putting $x=0$ we get it's a bijective function. so exist a such that $f(a)=0$
it's easy to show now $a=0$
so we have $f(f(x))=x$ and $f(xf(x))=f^2(x)$
put $x$ in place of $f(x)$ get $f^2(x)=x^2$
not directly but it can be said easily $f(x)=x ,-x$\end{tcolorbox}
You are trivializing the whole problem; the issue is to prove that from $|f(x)| = |x|$ follows either $f(x) = x$ for all $x$, or $f(x) = -x$ for all $x$, and others struggle to prove just this, by various methods, while you dismiss it by your "it can be said easily".\end{tcolorbox}

I know but those parts are really easy and boaring to write.
\end{solution}



\begin{solution}[by \href{https://artofproblemsolving.com/community/user/64716}{mavropnevma}]
	So then why do you \begin{bolded}bother\end{bolded} posting that at all? Just to tell us that, if not bored, you \begin{bolded}could\end{bolded} finish the proof? What contribution does your post bring towards a solution that will be meaningful and interesting for those that read what you wrote?
\end{solution}



\begin{solution}[by \href{https://artofproblemsolving.com/community/user/139996}{Faustus}]
	Subham, I think, it would be easy to let the trivial steps, that you gently missed, to trickle off the aura of your intelligence through your keyboard instead of unnecessarily quoting the post above above you(which was made for the same reason). Things that are 'trivial' and 'boaring'( :) ) for you are not the same for all. 
Anyway, the thread was revived for a question:
\begin{tcolorbox}Does $f^2(x)$ here mean $f(f(x))$ or $(f(x))^2$?\end{tcolorbox}
the answer to which is $f^2(x)=(f(x))^2$

\begin{tcolorbox}Sometimes, try to live the world through other's eyes, you might as well see a Roger Penrose in a Barack Obama\end{tcolorbox}
Peace.
Faustus
 :yinyang:
\end{solution}



\begin{solution}[by \href{https://artofproblemsolving.com/community/user/154765}{dgpiano}]
	why doesn't (f(x))^2=x^2 directly imply f(x)=x, -x?? Why do you have to go through all that hassle just to prove this statement?
\end{solution}



\begin{solution}[by \href{https://artofproblemsolving.com/community/user/89198}{chaotic_iak}]
	It might be possible that $f(1) = 1$ while $f(2) = -2$. It is true that $f(x) = x \vee -x \forall x$, but it is not true that $f(x) = x \forall x$ or $f(x) = -x \forall x$ (yet; you have to prove it).
\end{solution}



\begin{solution}[by \href{https://artofproblemsolving.com/community/user/206595}{spikerboy}]
	\begin{tcolorbox}Find all functions $f: \mathbb R \to \mathbb R$ such that \[ f( xf(x) + f(y) ) = f^2(x) + y \] for all $x,y\in \mathbb R$.\end{tcolorbox}
Let,$P(x,y)$ be the assertion.$P(0,y) \implies f(f(y))=f^2(0)+y$ take $y=-f^2(0)$ and $c=f(y) \; \; \implies f(c)=0$
$P(c,c) \;\; \implies f(0)=c$.and $P(0,c) \implies c=c^2+c \implies c=0$.we get, $f(0)=0$.so putting $x=0$ we get, $f(f(y))=y$ so $f$ is surjection.$P(x,0) \implies f(xf(x))=f^2(x)$.Let, $f(x_1)=x$. putting back into the previous equation we get,$f(x_{1}f(x_{1}))=x^2_1 \implies f^2(x_1)=x^2_1 \;\; \forall x_1$.so $f(1)=1$ or $-1$.$P(1,y) \implies f(1+f(y))=1+y \implies(1+f(y))^2=(1+y)^2  \implies f(y)=y$ and $P(-1,y) \implies f(-1+f(y))= -1+y \implies f(y)=-y \;\; \forall y$.so the solutions are $f(x)=x$ or $-x$
\end{solution}



\begin{solution}[by \href{https://artofproblemsolving.com/community/user/172163}{joybangla}]
	This is way too easy. As above we denote $P(x,y)$ as the assertion. Now $P(0,y)\implies f(f(y))=y+(f(0))^2$. Thus $f$ is a bijection. Now take $f(s)=0$. 
$P(s,y)\implies f(f(y))=y$. Now $P(f(x),y)\implies f(xf(x)+f(y))=(f(f(x)))^2+y=x^2+y\implies f(x)^2=x^2$. Now suppose $\exists x,y$ such that $f(x)=x,f(y)=-y$. Then $P(x,y)\implies f(x^2-y)=x^2+y$. Then $(x^2+y)^2=(x^2-y)^2\implies x^2y=0$. If $x=0$ or $y=0$ then $f(x)=-x$ also holds and $f(y)=y$ also holds. So we see that either $f\equiv \text{id}_{\mathbb{R}}$ or 
$f\equiv -\text{id}_{\mathbb{R}}$. And they both work.
\begin{tcolorbox}[quote="spikerboy"][quote="tranthanhnam"]Find all functions $f: \mathbb R \to \mathbb R$ such that \[ f( xf(x) + f(y) ) = f^2(x) + y \] for all $x,y\in \mathbb R$.\end{tcolorbox}
Let,$P(x,y)$ be the assertion.$P(0,y) \implies f(f(y))=f^2(0)+y$ take $y=-f^2(0)$ and $c=f(y) \; \; \implies f(c)=0$
$P(c,c) \;\; \implies f(0)=c$.and $P(0,c) \implies c=c^2+c \implies c=0$.we get, $f(0)=0$.so putting $x=0$ we get, $f(f(y))=y$ so $f$ is surjection.$P(x,0) \implies f(xf(x))=f^2(x)$.Let, $f(x_1)=x$. putting back into the previous equation we get,$f(x_{1}f(x_{1}))=x^2_1 \implies f^2(x_1)=x^2_1 \;\; \forall x_1$.so $f(1)=1$ or $-1$.$P(1,y) \implies f(1+f(y))=1+y \implies(1+f(y))^2=(1+y)^2  \implies f(y)=y$ and $P(-1,y) \implies f(-1+f(y))= -1+y \implies f(y)=-y \;\; \forall y$.so the solutions are $f(x)=x$ or $-x$\end{tcolorbox}
Nice solution man you didn't need the separate cases even to show that $f(x)=x,f(y)=-y$ doesnot happen. BTW sorry for spamming  :blush:  :oops: [\/hide]
\end{solution}



\begin{solution}[by \href{https://artofproblemsolving.com/community/user/64716}{mavropnevma}]
	\begin{tcolorbox}[quote="tranthanhnam"]Find all functions $f: \mathbb R \to \mathbb R$ such that \[ f( xf(x) + f(y) ) = f^2(x) + y \] for all $x,y\in \mathbb R$.\end{tcolorbox}
Let,$P(x,y)$ be the assertion.$P(0,y) \implies f(f(y))=f^2(0)+y$ take $y=-f^2(0)$ and $c=f(y) \; \; \implies f(c)=0$
$P(c,c) \;\; \implies f(0)=c$.and $P(0,c) \implies c=c^2+c \implies c=0$.we get, $f(0)=0$.so putting $x=0$ we get, $f(f(y))=y$ so $f$ is surjection.$P(x,0) \implies f(xf(x))=f^2(x)$.Let, $f(x_1)=x$. putting back into the previous equation we get,$f(x_{1}f(x_{1}))=x^2_1 \implies f^2(x_1)=x^2_1 \;\; \forall x_1$.so $f(1)=1$ or $-1$.$P(1,y) \implies f(1+f(y))=1+y \implies(1+f(y))^2=(1+y)^2  \implies f(y)=y$ and $P(-1,y) \implies f(-1+f(y))= -1+y \implies f(y)=-y \;\; \forall y$.so the solutions are $f(x)=x$ or $-x$\end{tcolorbox}
From your third line "Let, $f(x_1)=x$ ..." it's all nonsense. You should review that ...
\end{solution}



\begin{solution}[by \href{https://artofproblemsolving.com/community/user/206595}{spikerboy}]
	\begin{tcolorbox}
From your third line "Let, $f(x_1)=x$ ..." it's all nonsense. You should review that ...\end{tcolorbox}
Thanks mavropnevma for your most valuable suggestion.I will review on my "nonsense" stuffs.
\end{solution}



\begin{solution}[by \href{https://artofproblemsolving.com/community/user/173116}{Sardor}]
	Here my solution(I think it's beautiful):
Let $ P(x,y) $ be the assertion of $ f(xf(x)+f(y))=f(x)^2+y $.It's eas to see that $ f $ is injective $ (*) $.Let $ f(0)=c $.Then $ P(0,y) \implies f(f(y))=c^2+y  \implies f $ is surjective $ (**) $.From $ (*)+(**) \implies f $ is bijective $ (\#) $.Let $ a\in R $ such that $ f(a)=0 $.Then, $ P(a,y) \implies y=f(f(y))=c^2+y \implies c=0 $ i.e $ f(0)=0 $.Let $ Q(x) $ be the assertion $ f(xf(x))=f(x)^2 $.Then
$ Q(f(x)) \implies f(x)^2=f(xf(x))=f(f(x)x)=f(f(x)f(f(x)))=f(f(x))^2=x^2 $ i.e $ f(x)^2=x^2 $. Hence there ard two answers:
1. $ f(x)=x $
2. $ f(x)=-x $.
\end{solution}



\begin{solution}[by \href{https://artofproblemsolving.com/community/user/31919}{tenniskidperson3}]
	And again, Sardor, you miss the possible case where $f(1)=1$ and $f(2)=-2$.  You need to prove that if $f(a)=a$ for some nonzero $a$, then $f(b)=b$ FOR ALL $b$.  You made this easier on yourself by showing that the only other possibility is $f(b)=-b$, but it still takes some work.  Also, this exact pothole has been posted and discussed many times in the above discussion, and it might be worth your time to read it.
\end{solution}



\begin{solution}[by \href{https://artofproblemsolving.com/community/user/130234}{vsathiam}]
	\begin{tcolorbox}And again, Sardor, you miss the possible case where $f(1)=1$ and $f(2)=-2$.  You need to prove that if $f(a)=a$ for some nonzero $a$, then $f(b)=b$ FOR ALL $b$.  You made this easier on yourself by showing that the only other possibility is $f(b)=-b$, but it still takes some work.  Also, this exact pothole has been posted and discussed many times in the above discussion, and it might be worth your time to read it.\end{tcolorbox}

Formally, the pothole's called "The Pointwise Value Trap". Also, there was a nice technique in this problem where you find a value u in the domain of f such that f(u) = 0 and use it. That's pretty neat. (Also, it's common)

To solve it, look at P(1,y) for f(1) = 1 and f(1) = -1 resp.
\end{solution}
*******************************************************************************
-------------------------------------------------------------------------------

\begin{problem}[Posted by \href{https://artofproblemsolving.com/community/user/9956}{duongqua}]
	Find all functions $f: \mathbb N^* \to \mathbb N^*$ such that
 \[f(f(m)f(n))=f(n)+f(m),\quad \forall m,n \in\mathbb N^{*}.\]
	\flushright \href{https://artofproblemsolving.com/community/c6h56559}{(Link to AoPS)}
\end{problem}



\begin{solution}[by \href{https://artofproblemsolving.com/community/user/29428}{pco}]
	\begin{tcolorbox}find $ f: N^{*}\to\ N^{*}$ such that:
 $ f(f(m)f(n)) = f(n) + f(m) \forall m,n \in N^{*}$\end{tcolorbox}

[hide="My solution"]
Let $ \mathbb A = f(\mathbb N)$. The equation is $ f(xy) = x + y$ $ \forall x,y\in\mathbb A$

So, if $ x,y,z,t\in\mathbb A$ are such that $ xy = zt$, then $ f(xy) = f(zt)$ and so $ x + y = z + t$ and it is easy to conclude that either $ (x,y) = (z,t)$, either $ (x,y) = (t,z)$

$ u,v\in\mathbb A$ $ \implies$ $ u + v\in\mathbb A$ (since $ f(uv) = u + v$) and so $ pu + qv\in\mathbb A$ $ \forall p,q\in\mathbb N_0$

Let then $ u,v\in\mathbb A$ and :
let $ x = uu + uv$
let $ y = u + 2v$
let $ z = uu + 2uv$
let $ t = u + v$

All these four numbers are in the form $ pu + qv$ and so belong to $ \mathbb A$ and we have $ xy = zt = u(u + v)(u + 2v)$ and so :
Either $ (x,y) = (z,t)$ and so $ z - x = uv = 0$, which is impossible
Either $ (x,y) = (t,z)$ and so $ x - t = (u - 1)(u + v) = 0$ and so $ u = 1$

So $ u,v\in\mathbb A$ $ \implies$ $ u = 1$ and so $ \mathbb A = \{1\}$, which is obviously wrong ($ f(x) = 1$ is not a solution).

So no solution.
[\/hide]
\end{solution}
*******************************************************************************
-------------------------------------------------------------------------------

\begin{problem}[Posted by \href{https://artofproblemsolving.com/community/user/1991}{orl}]
	Let $ {\mathbb Q}^ +$ be the set of positive rational numbers. Construct a function $ f : {\mathbb Q}^ + \rightarrow {\mathbb Q}^ +$ such that
\[ f(xf(y)) = \frac {f(x)}{y}
\]
for all $ x$, $ y$ in $ {\mathbb Q}^ +$.
	\flushright \href{https://artofproblemsolving.com/community/c6h60742}{(Link to AoPS)}
\end{problem}



\begin{solution}[by \href{https://artofproblemsolving.com/community/user/26}{grobber}]
	It suffices to construct such a function satisfying $f(ab)=f(a)f(b),\ \forall a,b\in\mathbb Q^+\ (*)$ (this implies $f(1)=1$) and $f(f(x))=\frac 1x,\ \forall x\in\mathbb Q^+\ (**)$.

All we need to do is define $f(p_i)$ s.t. $(*)$ whenever $x=p_i$ for some $i\ge 1$, where $(p_n)_{n\ge 1}$ is the sequence of primes, and then extend it to the rest of $\mathbb Q^+$ so that $(**)$ holds. Then it's clear that $(*)$ will automatically hold.
\end{solution}



\begin{solution}[by \href{https://artofproblemsolving.com/community/user/4527}{Amir.S}]
	as Grobber said(he didn't prove it) we have $ f(ab) = f(a)f(b)$ , this  implies $ f(\prod_{i = 1}^np_i^{\alpha_i}) = \prod_{i = 1}^nf(p_i)^{\alpha_i}$ , hence we must deifne the function on all primes, let $ p_i$ denote the $ i - th$ prime number we define $ f$ as:
$ f(p_{2k - 1}) = p_{2k}\ ,\ f(p_{2k}) = \frac {1}{p_{2k - 1}}$
this function satisfies the problem , clearly.
\end{solution}



\begin{solution}[by \href{https://artofproblemsolving.com/community/user/28136}{Rofler}]
	So how do you extend to Q?
\end{solution}



\begin{solution}[by \href{https://artofproblemsolving.com/community/user/38170}{aznlord1337}]
	$ f(f(y)) = f(1)\/y$. This implies that $ f$ is injective. $ f(f(1)) = f(1) \longrightarrow f(1) = 1$

Therefore $ f(f(y)) = 1\/y$. Let $ y=f(y)$, so $ f(x\/y) = f(x)\/f(y)$. Then $ f(1\/f(y)) = f(1)\/f(f(y)) = y$

From the original equation, letting $ y=1\/f(y)$ implies $ f(xy) = f(x)f(y)$. A function on primes like Amir's works.
\end{solution}



\begin{solution}[by \href{https://artofproblemsolving.com/community/user/41655}{triplebig}]
	I don't understand how you can conclude that $ f$ is injective, can anyone please share some light?
\end{solution}



\begin{solution}[by \href{https://artofproblemsolving.com/community/user/29428}{pco}]
	\begin{tcolorbox}I don't understand how you can conclude that $ f$ is injective, can anyone please share some light?\end{tcolorbox}
$ f(y_1) = f(y_2)$ $ \implies$ $ f(xf(y_1)) = f(xf(y_2))$ $ \implies$ $ \frac {f(x)}{y_1} = \frac {f(x)}{y_2}$ $ \implies$ $ y_1 = y_2$ (since $ f(x)\neq 0$)
\end{solution}



\begin{solution}[by \href{https://artofproblemsolving.com/community/user/41655}{triplebig}]
	Got it, thank you for the help
\end{solution}



\begin{solution}[by \href{https://artofproblemsolving.com/community/user/72819}{Dijkschneier}]
	\begin{tcolorbox}So how do you extend to Q?\end{tcolorbox}
Can sameone answer to this, please ?
\end{solution}



\begin{solution}[by \href{https://artofproblemsolving.com/community/user/29428}{pco}]
	\begin{tcolorbox}[quote="Rofler"]So how do you extend to Q?\end{tcolorbox}
Can sameone answer to this, please ?\end{tcolorbox}

There is no need for extension : the problem is just for Q+ and we know that any positive rational may be written in a unique manner as the product of prime numbers raised to integer powers.

What do you want more ?
\end{solution}



\begin{solution}[by \href{https://artofproblemsolving.com/community/user/72819}{Dijkschneier}]
	Thank you.
\end{solution}



\begin{solution}[by \href{https://artofproblemsolving.com/community/user/95245}{CPT_J_H_Miller}]
	Sorry to revive this topic, but can someone please explain why this doesn't work? :

Note that from the above we've already established that $f(xy)=f(x)f(y)$ and $f$ is injective.
Also, from $ f(f(y)) = \frac{1}{y} $, we know that $ f $ is surjective, therefore $ f^{-1}(x) $ exists for all positive rationals $ x $.

So set $ f^{-1}(y)$ as $ y $ in $ f(f(y)) = \frac{1}{y} \Rightarrow f(y)f^{-1}(y) = 1 $

Thus set $ f^{-1}(x) $ as $ x $ and $ y $ as $ y $ into the original equation and we obtain:
$ f(f^{-1}(x)f(y)) = \frac{x}{y} $
$ \Rightarrow f^{-1}(x)f(y) = \frac{x}{y} $
$ \Rightarrow \frac{f(y)}{f(x)} = \frac{x}{y} $
$ \Rightarrow f(x) = \frac{1}{x} $ $\forall$  $ x \in \mathbb{Q}^{+} $
which is obviously not a solution to the equation.

Can someone please explain what went wrong? Thanks.
\end{solution}



\begin{solution}[by \href{https://artofproblemsolving.com/community/user/64716}{mavropnevma}]
	\begin{tcolorbox}Thus set $ f^{-1}(x) $ as $ x $ and $ y $ as $ y $ into the original equation and we obtain:
$ f(f^{-1}(x)f(y)) = \frac{x}{y} $
$ \Rightarrow f^{-1}(x)f(y) = \frac{x}{y} $.\end{tcolorbox}
The implication is abusive; from $f(A) = B$ you infer $A=B$.
\end{solution}



\begin{solution}[by \href{https://artofproblemsolving.com/community/user/95245}{CPT_J_H_Miller}]
	Argh... silly mistake... yes it should be $ f(f^{-1}(x)f(y)) = xf(f(y)) = \frac{x}{y} $.
Thanks!
\end{solution}



\begin{solution}[by \href{https://artofproblemsolving.com/community/user/74734}{flare}]
	I was able to determine the conditions for the function, but not able to construct it. 
Out of curiosity, how many points would I get for this (on the actual thing I would probably spend time finding it since the conditions take a very small time to find, but...)?
\end{solution}



\begin{solution}[by \href{https://artofproblemsolving.com/community/user/141397}{subham1729}]
	:mad: Plugging in x = 1 we get f(f(y)) = f(1)\/y and hence f(y1) = f(y2)
implies y1 = y2 i.e. that the function is bijective. Plugging in y = 1 gives
us f(xf(1)) = f(x) ⇒ xf(1) = x ⇒ f(1) = 1. Hence f(f(y)) = 1\/y.
Plugging in y = f(z) implies 1\/f(z) = f(1\/z). Finally setting y = f(1\/t)
into the original equation gives us f(xt) = f(x)\/f(1\/t) = f(x)f(t).
Conversely, any functional equation on Q+ satisfying (i) f(xt) = f(x)f(t)
and (ii) f(f(x)) = 1
x for all x, t ∈ Q+ also satisfies the original functional
equation: f(xf(y)) = f(x)f(f(y)) = f(x)
y . Hence it suffices to find
a function satisfying (i) and (ii).
We note that all elements q ∈ Q+ are of the form q = $n
i=1 pai
i where
pi are prime and ai ∈ Z. The criterion (a) implies f(q) = f($n
i=1 pai
i ) = $n
i=1 f(pi)ai . Thus it is sufficient to define the function on all primes. For
the function to satisfy (b) it is necessary and sufficient for it to satisfy
f(f(p)) = 1
p for all primes p. Let qi denote the i-th smallest prime. We
define our function f as follows:
f(q2k−1) = q2k, f(q2k) =
1
q2k−1
, k ∈ N .
Such a function clearly satisfies (b) and along with the additional condition
f(xt) = f(x)f(t) it is well defined for all elements of Q+ and it satisfies
the original functional equation. :P  :mad:  :mad:
\end{solution}



\begin{solution}[by \href{https://artofproblemsolving.com/community/user/223099}{MathPanda1}]
	\begin{tcolorbox}as Grobber said(he didn't prove it) we have $ f(ab) = f(a)f(b)$ , this implies $ f(\prod_{i = 1}^np_i^{\alpha_i}) = \prod_{i = 1}^nf(p_i)^{\alpha_i}$ , hence we must deifne the function on all primes, let $ p_i$ denote the $ i - th$ prime number we define $ f$ as:
$ f(p_{2k - 1}) = p_{2k}\ ,\ f(p_{2k}) = \frac {1}{p_{2k - 1}}$
 this function satisfies the problem , clearly.\end{tcolorbox}

What is the motivation for constructing such a function? Thank you very much!
\end{solution}



\begin{solution}[by \href{https://artofproblemsolving.com/community/user/130234}{vsathiam}]
	\begin{tcolorbox}[quote=Amir.S]as Grobber said(he didn't prove it) we have $ f(ab) = f(a)f(b)$ , this implies $ f(\prod_{i = 1}^np_i^{\alpha_i}) = \prod_{i = 1}^nf(p_i)^{\alpha_i}$ , hence we must deifne the function on all primes, let $ p_i$ denote the $ i - th$ prime number we define $ f$ as:
$ f(p_{2k - 1}) = p_{2k}\ ,\ f(p_{2k}) = \frac {1}{p_{2k - 1}}$
 this function satisfies the problem , clearly.\end{tcolorbox}

What is the motivation for constructing such a function? Thank you very much!\end{tcolorbox}

First you try out some algebraic methods: none of them are fruitful. Then you note that the problem said construction, which implies a numbertheoretic approach. This immediately applies looking for multiplicity and a way to define f(1), which just happen to be related to each other. (Shows that you are on the right track). Then you obtain the relation:

$f(f(p)) = \frac{1}{p}$ for all primes.

So it is clear that you cannot manipulate the power of prime p to get from an exponent of 1 to -1 in two steps over $\mathbb{Q^{+}}$. So you have to manipulate the primes in some other method, with a group of elements acting as a medium. (In other words, f(f(p)) maps A $\rightarrow$ B $\rightarrow$ C, where you know that {p} = A, B is unknown, and {$\frac{1}{p}$} = C.

This suggests bipartitioning the set of primes, which suggests considering the sets {$p_{2k}$}, {$p_{2k-1}$}, {$\frac{1}{p_{2k}}$} and {$\frac{1}{p_{2k-1}}$}. Playing around with directed arrows that map elements between the sets gives you the function.
\end{solution}
*******************************************************************************
-------------------------------------------------------------------------------

\begin{problem}[Posted by \href{https://artofproblemsolving.com/community/user/1991}{orl}]
	Find all functions $f$ defined on the non-negative reals and taking non-negative real values such that: $f(2)=0,f(x)\ne0$ for $0\le x<2$, and $f(xf(y))f(y)=f(x+y)$ for all $x,y$.
	\flushright \href{https://artofproblemsolving.com/community/c6h60776}{(Link to AoPS)}
\end{problem}



\begin{solution}[by \href{https://artofproblemsolving.com/community/user/4036}{Megus}]
	Just a note: generalization of this problem appeared in IMC 2000 as problem 5.

http://www.mathlinks.ro/Forum/viewtopic.php?p=357658#p357658
\end{solution}



\begin{solution}[by \href{https://artofproblemsolving.com/community/user/2520}{towersfreak2006}]
	\begin{tcolorbox}Find all functions $f$ defined on the non-negative reals and taking non-negative real values such that: $f(2)=0,f(x)\ne0$ for $0\le x<2$, and $f(xf(y))f(y)=f(x+y)$ for all $x,y$.\end{tcolorbox}Can someone confirm this? The Engel book says no solution...
[hide="Solution"]
For all $x+y\ge2$, there exists a nonnegative real $t$ such that $t+2=x+y$ and $f(x+y)=f(t+2)=f(tf(2))f(2)=0$. Hence, $f(x)=0$ for all $x\ge 2$.

For $x<2$, take $0=f(2)=f((2-x)+(x))=f((2-x)f(x))f(x)$.
Since $f(x)\neq0$ for $x<2$, then $f((2-x)f(x))=0$.
Thus, $(2-x)f(x)\ge2\Leftrightarrow f(x)\ge\frac{2}{2-x}$ for all $x<2$.

Suppose $f(a)>\frac{2}{2-a}$ so that $f(a)=\frac{2}{2-a-\epsilon}$ for some $\epsilon>0$ and $0\le a<2$.
Then $f\left((2-a-\epsilon)f(a)\right)f(a)=f(2-\epsilon)$. The left hand side is equivalent to $f\left((2-a-\epsilon)\left(\frac{2}{2-a-\epsilon}\right)\right)f(a)=f(2)=0$, but the right hand side is nonzero, since $2-\epsilon<2$ and $f(x)\neq0$ for all $x<2$.

Hence, $f(x)=\frac{2}{2-x}$ for $0\le a<2$.
For $x+y<2$, \begin{eqnarray*}f(xf(y))f(y) &=& f\left((x)\left(\frac{2}{2-y}\right)\right)\left(\frac{2}{2-y}\right)\\ &=&\left(\frac{2}{2-\frac{2x}{2-y}}\right)\left(\frac{2}{2-y}\right)\\ &=&\frac{4}{4-2y-2x}\\ &=&\frac{2}{2-y-x}\\ f(x+y)&=&\frac{2}{2-y-x}.\end{eqnarray*}
Indeed, $f(xf(y))f(y)=f(x+y)$, so the desired function is \[ \begin{equation*}f(x)=\begin{cases}\frac{2}{2-x},& 0\le x<2\\0,& x\ge2.\end{cases}\end{equation*}  \][\/hide]
\end{solution}



\begin{solution}[by \href{https://artofproblemsolving.com/community/user/432}{darij grinberg}]
	More or less the same problem [url=http://www.mathlinks.ro/Forum/viewtopic.php?t=5439]has been given to us in the 5th German TST 2004[\/url], only made a bit harder by stating that $f\left(x\right)\neq 0$ holds for all 0 < x < 2 (instead of $0\leq x<2$), so that we had to prove that $f\left(0\right)\neq 0$ by an additional argument.

Towersfreak2006's solution is correct, and is more or less identic with the solution I gave on the exam. Engel is sometimes really strange about giving solutions in his book, so "no solution" should neither mean that the problem is too hard, nor that it is too easy. But there is always MathLinks, Kalva, ...

  Darij
\end{solution}



\begin{solution}[by \href{https://artofproblemsolving.com/community/user/15035}{Ilthigore}]
	Consider $ 0\leq y < 2$.

\[ x: =\frac{2}{f(y)} \Rightarrow f(\frac{2}{f(y)}+y)=f(2)f(y)=0 
\Rightarrow \frac{2}{f(y)}+y \geq 2 \Rightarrow f(y) \leq \frac{2}{2-y} \\
x: =(2-y) \Rightarrow f((2-y)f(y))f(y)=0 \Rightarrow (2-y)f(y)\geq 2 \Rightarrow f(y) \geq \frac{2}{2-y}\]
\end{solution}



\begin{solution}[by \href{https://artofproblemsolving.com/community/user/25787}{olorin}]
	Let's find all $ f: [0,\infty)\to [0,\infty)$ with
$ (*)$ $ f(xf(y))f(y) = f(x + y)$ for all $ x,y\ge 0$

First assume $ f(x) = 0$ for some $ x\ge 0$ and take $ c: = \inf\{x\ge 0: f(x) = 0\}\ge 0$.
Then for all $ z > c$ we find $ y < z$ with $ f(y) = 0$, 
and for $ x = z - y$ in $ (*)$ we get $ f(z) = f(x + y) = f(xf(y))f(y) = 0$.
This gives $ f(x) = 0$ for all $ x > c$ and $ f(x)\not = 0$ for all $ 0\le x < c$.
Then for all $ 0\le y < c$ we have $ f(y)\not = 0$ and
$ x > c - y\Rightarrow f(x + y) = 0 \Rightarrow f(xf(y)) = 0 \Rightarrow f(y)\ge {c\over x}$
and
$ 0 < x < c - y\Rightarrow f(x + y)\not = 0 \Rightarrow f(xf(y))\not = 0 \Rightarrow f(y)\le {c\over x}$
for all $ x$.
So $ f(y)\ge \sup_{x > c - y}{c\over x} = {c\over c - y}$ and $ f(y)\le \inf_{0 < x < c - y}{c\over x} = {c\over c - y}$ for all $ 0\le y < c$.

This gives $ f(x) = {c\over c - x}$ for all $ 0\le x < c$, 
which leaves to find $ f(c)$.

Assume $ f(c)\not = 0$.
Then for all $ x > 0$ and $ y = c$ in $ (*)$ we get 
$ f(xf(c))f(c) = f(x + c) = 0\Rightarrow f(xf(c)) = 0\Rightarrow xf(c)\ge c$.
So $ c\le \inf_{x > 0}xf(c) = 0$, and therefore $ c = 0$ and $ f(0)\not = 0$.
But $ x = y = 0$ in $ (*)$ gives $ f(0)^2 = f(0)\Rightarrow f(0) = 1$.
So $ f(x) = \{ {1\text{ for }x = 0\atop 0\text{ for }x\not = 0}$.

Otherwise $ f(c) = 0$ and
$ f(x) = \{ {{c\over c - x}\text{ for }0\le x < c\atop 0\text{ for }x\ge c}$
for some $ c\ge 0$.
(This is $ f\equiv 0$ for $ c = 0$)


Finally assume $ f(x) > 0$ for all $ x\ge 0$.
If $ f(y) > 1$ for some $ y\ge 0$, then for $ x = {y\over f(y) - 1}\ge 0$ we have $ xf(y) = x + y$,
and $ (*)$ gives $ 1 < f(y) = {f(x + y)\over f(xf(y))} = 1$. Contradiction!
So $ f(x)\le 1$ for all $ x\ge 0$, and $ (*)$ gives $ f(y)\ge f(xf(y))f(y) = f(x + y)$ for all $ x,y\ge 0$.
So $ f$ is monotonically decreasing.

Assume $ f(a) = f(b)$ for some $ 0\le a < b$.
Then by $ (*)$ we get $ f(x + a) = f(xf(a))f(a) = f(xf(b))f(b) = f(x + b)$ for all $ x\ge 0$.
For $ x = n(b - a)$ we get $ f(n(b - a) + a) = f(n(b - a) + b) = f((n + 1)(b - a) + a)$ for all $ n\in\mathbb{N}_0$,
and so by induction $ f(a) = f(n(b - a) + a)$ for all $ n\in\mathbb{N}_0$.
But then for all $ x\ge a$ we find $ n\in\mathbb{N}_0$ with $ x < n(b - a) + a$,
and then $ f(a)\ge f(x)\ge f(n(b - a) + a) = f(a)$.
So $ f(x) = f(a)$ for all $ x\ge a$,
and for all $ y\ge 0$ we can find $ x > 0$ with $ xf(y)\ge a$ and $ x + y\ge a$,
and using $ (*)$ we get $ f(y) = {f(x + y)\over f(xf(y))} = {f(a)\over f(a)} = 1$.

This gives $ f\equiv 1$,  or otherwise $ f$ is strictly monotonically decreasing.
Assume the latter.
Then replacing $ x$ by $ {x\over f(y)}$ in $ (*)$ we get
$ f(x)f(y) = f({x\over f(y)} + y)$ for all $ x,y\ge 0$,
and using the symmetry of the left side we get
$ f({x\over f(y)} + y) = f({y\over f(x)} + x)$ and $ {x\over f(y)} + y = {y\over f(x)} + x$ for all $ x,y\ge 0$.
Then for $ y = 1$ and $ c: = {1\over f(1)} - 1\ge 0$ we get
$ f(x) = {1\over 1 + cx}$ for all $ x\ge 0$.
(This is $ f\equiv 1$ for $ c = 0$)


We therfore get $ 3$ types of solutions of $ (*)$:
$ f(x) = \{ {1\text{ for }x = 0\atop 0\text{ for }x\not = 0}$
or
$ f(x) = \{ {{c\over c - x}\text{ for }0\le x < c\atop 0\text{ for }x\ge c}$
or
$ f(x) = {1\over 1 + cx},x\ge 0$
for some $ c\ge 0$.
Easy to see, that these are legit.
\end{solution}



\begin{solution}[by \href{https://artofproblemsolving.com/community/user/68920}{prester}]
	\begin{tcolorbox}Find all functions $ f$ defined on the non-negative reals and taking non-negative real values such that: $ f(2) = 0,f(x)\ne0$ for $ 0\le x < 2$, and $ f(xf(y))f(y) = f(x + y)$ for all $ x,y$.\end{tcolorbox}

Looking at this old IMO problem I get a strange result. If I put $ y=2$ in the property $ f(xf(y))f(y) = f(x + y)$ this implies $ \implies f(x+2)=0$ for all $ x\ge0$. This should imply that $ f(x)=0$ for all $ x\ge0$. Where do I get wrong? The problem statement is correct?
\end{solution}



\begin{solution}[by \href{https://artofproblemsolving.com/community/user/29428}{pco}]
	\begin{tcolorbox}[quote="orl"]Find all functions $ f$ defined on the non-negative reals and taking non-negative real values such that: $ f(2) = 0,f(x)\ne0$ for $ 0\le x < 2$, and $ f(xf(y))f(y) = f(x + y)$ for all $ x,y$.\end{tcolorbox}

Looking at this old IMO problem I get a strange result. If I put $ y = 2$ in the property $ f(xf(y))f(y) = f(x + y)$ this implies $ \implies f(x + 2) = 0$ for all $ x\ge0$. This should imply that $ f(x) = 0$ for all $ x\ge0$. Where do I get wrong? The problem statement is correct?\end{tcolorbox}

No : $ f(x + 2) = 0$ for all $ x\ge 0$ implies $ f(x)=0$ $ \forall x\ge 2$ and not $ f(x) = 0$ for all $ x\ge 0$

So the required function is non zero on $ [0,2)$ and zero on $ [2,+\infty)$
\end{solution}



\begin{solution}[by \href{https://artofproblemsolving.com/community/user/68920}{prester}]
	\begin{tcolorbox}[quote="prester"][quote="orl"]Find all functions $ f$ defined on the non-negative reals and taking non-negative real values such that: $ f(2) = 0,f(x)\ne0$ for $ 0\le x < 2$, and $ f(xf(y))f(y) = f(x + y)$ for all $ x,y$.\end{tcolorbox}

Looking at this old IMO problem I get a strange result. If I put $ y = 2$ in the property $ f(xf(y))f(y) = f(x + y)$ this implies $ \implies f(x + 2) = 0$ for all $ x\ge0$. This should imply that $ f(x) = 0$ for all $ x\ge0$. Where do I get wrong? The problem statement is correct?\end{tcolorbox}

No : $ f(x + 2) = 0$ for all $ x\ge 0$ implies $ f(x) = 0$ $ \forall x\ge 2$ and not $ f(x) = 0$ for all $ x\ge 0$

So the required function is non zero on $ [0,2)$ and zero on $ [2, + \infty)$\end{tcolorbox}

Ok pco, I got a stupid result.  :blush:  $ x=y-2 \ge 0$ $ \implies  f(y)=0 \;\;\;\forall y\ge 2$
Thanks for your quick remark
\end{solution}
*******************************************************************************
-------------------------------------------------------------------------------

\begin{problem}[Posted by \href{https://artofproblemsolving.com/community/user/1991}{orl}]
	$f$ and $g$ are real-valued functions defined on the real line. For all $x$ and $y, f(x+y)+f(x-y)=2f(x)g(y)$. $f$ is not identically zero and $|f(x)|\le1$ for all $x$. Prove that $|g(x)|\le1$ for all $x$.
	\flushright \href{https://artofproblemsolving.com/community/c6h60824}{(Link to AoPS)}
\end{problem}



\begin{solution}[by \href{https://artofproblemsolving.com/community/user/29428}{pco}]
	\begin{tcolorbox}$ f$ and $ g$ are real-valued functions defined on the real line. For all $ x$ and $ y, f(x + y) + f(x - y) = 2f(x)g(y)$. $ f$ is not identically zero and $ |f(x)|\le1$ for all $ x$. Prove that $ |g(x)|\le1$ for all $ x$.\end{tcolorbox}

Let $ y\in\mathbb R$ :

If $ g(y)=\pm 1$, the result ($ |g(y)|\leq 1$) is achieved.
If $ g(y)\neq \pm 1$ :

We have $ f(x+y)=2g(y)f(x)-f(x-y)$ and so $ f(x+ny)=\frac{f(x)-rf(x+y)}{1-r^2}r^n+r\frac{f(x+y)-rf(x)}{1-r^2}r^{-n}$ with $ r$ any root of $ X^2-2g(y)X+1=0$

From this, it's obvious that $ |f(x)|\leq 1$ can be true only if the roots are complex, and so if the discriminant of the quadratic $ X^2-2g(y)X+1=0$ is negative.

So $ g(y)^2< 1$

Q.E.D.
\end{solution}



\begin{solution}[by \href{https://artofproblemsolving.com/community/user/112}{Diogene}]
	An other one ..  :D . Let's denote $ 0 < S = sup(|f(x)|) < \infty$ the superior bound of $ |f|$, so 
$ \forall_{x,y} ,2|f(x)|.|g(y) |\leq 2S \Longrightarrow \forall_y, S.|g(y)|\leq S \Longrightarrow \forall_y ,|g(y)|\leq 1$ ..  :cool:
\end{solution}



\begin{solution}[by \href{https://artofproblemsolving.com/community/user/29428}{pco}]
	\begin{tcolorbox}An other one ..  :D . Let's denote $ 0 < S = sup(|f(x)|) < \infty$ the superior bound of $ f$, so 
$ \forall_{x,y} ,2|f(x)|.|g(y) |\leq 2S \Longrightarrow \forall_y, S.|g(y)|\leq S \Longrightarrow \forall_y ,|g(y)|\leq 1$ ..  :cool:\end{tcolorbox}

Yessss, quick and nice!

Congrats !   :)
\end{solution}



\begin{solution}[by \href{https://artofproblemsolving.com/community/user/35129}{Zhero}]
	\begin{tcolorbox}An other one ..  :D . Let's denote $ 0 < S = sup(|f(x)|) < \infty$ the superior bound of $ |f|$, so 
$ \forall_{x,y} ,2|f(x)|.|g(y) |\leq 2S \Longrightarrow \forall_y, S.|g(y)|\leq S \Longrightarrow \forall_y ,|g(y)|\leq 1$ ..  :cool:\end{tcolorbox}
I don't really understand your solution. Could you clarify on that? I understand that $ |f(x)| |g(y)| \leq S$, but how does it follows that $ |g(y)| \leq 1$? We don't have that $ f(x) \geq S$.  :?:
\end{solution}



\begin{solution}[by \href{https://artofproblemsolving.com/community/user/10153}{10000th User}]
	\begin{tcolorbox}[quote="Diogene"]An other one ..  :D . Let's denote $ 0 < S = sup(|f(x)|) < \infty$ the superior bound of $ |f|$, so 
$ \forall_{x,y} ,2|f(x)|.|g(y) |\leq 2S \Longrightarrow \forall_y, S.|g(y)|\leq S \Longrightarrow \forall_y ,|g(y)|\leq 1$ ..  :cool:\end{tcolorbox}
I don't really understand your solution. Could you clarify on that? I understand that $ |f(x)| |g(y)| \leq S$, but how does it follows that $ |g(y)| \leq 1$? We don't have that $ f(x) \geq S$.  :?:\end{tcolorbox}Ya but if we have $ |f(x)\parallel{}g(y)|\le S$ and $ |f(x)|\le S$ for all x, $ |g(y)|\le 1$ for all y, otherwise $ S$ contradicts the definition of $ \sup$. Think that $ S$ is not simply any upper bound but it is the \begin{italicized}least\end{italicized} upper bound. That's what makes Diogene's solution valid.
\end{solution}



\begin{solution}[by \href{https://artofproblemsolving.com/community/user/35129}{Zhero}]
	Oh, I see; because if we have some $ c = |g(y)| > 1$, then $ |f(x)| \leq \frac{S}{c}$ for all $ x$, with $ \frac{S}{c} < S$, a contradiction. 

That's a very nice solution. And thanks for explaining.
\end{solution}



\begin{solution}[by \href{https://artofproblemsolving.com/community/user/130521}{mathsluver}]
	Let f take on the value 1 at k.Put x=k in given equation.We are left with g=f(k+y)+f(k-y)\/2
\end{solution}



\begin{solution}[by \href{https://artofproblemsolving.com/community/user/144234}{askhamath}]
	let $ |f(z)| $ is max 
then $ 2|f(z)|\ge |f(z+y)|+|f(z-y)|\ge |f(z+y)+f(z-y)|=2|f(z)||g(y)| $
since $ f(z)\ne 0 $ then $ |g(x)|\le 1 $ for all $ x $
\end{solution}
*******************************************************************************
-------------------------------------------------------------------------------

\begin{problem}[Posted by \href{https://artofproblemsolving.com/community/user/17894}{X-man}]
	Is there a function $f: \mathbb N \rightarrow \mathbb N$ such that for every $n \in \mathbb N$, \[ f(f(n-1)) = f(n+1) - f(n) \, ?\]
	\flushright \href{https://artofproblemsolving.com/community/c6h84610}{(Link to AoPS)}
\end{problem}



\begin{solution}[by \href{https://artofproblemsolving.com/community/user/29428}{pco}]
	\begin{tcolorbox}Is there a function $ f: \mathbb N \rightarrow \mathbb N$ such that for every $ n \in \mathbb N$ is
\[ f(f(n - 1)) = f(n + 1) - f(n)
\]
\end{tcolorbox}
[hide="My solution"]
$ f(n)\in\mathbb N$ $ \implies$ $ f(n)\geq 1$ $ \forall n$
Then $ f(n+1)=f(n)+f(f(n-1))$ $ \forall n>1$ $ \implies$ $ f(n+1)>f(n)$ $ \forall n>1$
So $ f(n)\geq n-1$ $ \forall n$
So $ f(f(n))\geq n-2$ $ \forall n$

So $ f(n+1)\geq f(n)+n-3\geq n-1+n-3=2n-4$ $ \forall n>1$ and so $ f(8)\geq 10$ and so $ f(f(8))\geq f(10)$
But $ f(f(8))=f(10)-f(9)$ and so $ f(f(8))<f(10)$

Hence a contradiction.
Hence no such $ f(x)$ exists.

[\/hide]
\end{solution}
*******************************************************************************
-------------------------------------------------------------------------------

\begin{problem}[Posted by \href{https://artofproblemsolving.com/community/user/23682}{thachpbc}]
	Find all functions $ f : \mathbb{R} \to \mathbb{R}$ such that
\[f(x - f(y)) = f(x) + f(f(y)), \quad \forall x,y\in\mathbb R.\]
	\flushright \href{https://artofproblemsolving.com/community/c6h114884}{(Link to AoPS)}
\end{problem}



\begin{solution}[by \href{https://artofproblemsolving.com/community/user/29428}{pco}]
	\begin{tcolorbox}Find all functions f : R-> R and:
f(x -f(y)) = f(x) + f(f(y)) all x,y in R.\end{tcolorbox}

So $ f(x - f(y)) = f(x) + f(f(y))$ $ \forall x,y\in\mathbb R$

Not very hard, but interesting ...

[hide="My solution"]
1) \begin{bolded}basics elements \end{underlined}\end{bolded}:
$ P(x,y)$ : $ f(x - f(y)) = f(x) + f(f(y))$
Let $ f(0) = a$

$ P(f(x),x)$ $ \implies$ $ f(f(x)) = \frac a2$ and so $ f(x) = \frac a2$ $ \forall x\in I = f(\mathbb R)$  (** edited **)
$ P(2f(x),x)$ $ \implies$ $ f(2f(x)) = 0$ and so $ 0\in I = f(\mathbb R)$ and so $ f(0) = \frac a2$ but $ f(0) = a$ $ \implies$ $ a = 0$

So we have : 
$ f(f(x)) = 0$ $ \forall x$
$ f(x - f(y)) = f(x)$ $ \forall x,y$

From the second equation, it's easy to conclude that $ f(x + a) = f(x)$ $ \forall x\in\mathbb R$ and $ \forall a\in A$ where $ A$ is the smallest additive subgroup generated by elements of $ I = f(\mathbb R)$ and this is enough to establish the general solution :

2) \begin{bolded}general solution \end{underlined}\end{bolded}:
Let $ A$ any additive subgroup of $ \mathbb R$
Let $ \sim$ the relation $ x\sim y$ $ \iff$ $ x - y\in A$
Since $ A$ is an additive group, $ \sim$ is an equivalence relation. Let then $ c(x)$ any choice function which associates to a real $ x$ a representative (unique per class) of it's equivalence class.
Let then $ h(x)$ a function from $ c(\mathbb R)\to A$ such that $ h(c(0)) = 0$

Then $ f(x) = h(c(x))$ $ \forall x$

3) \begin{bolded}proof that any function defined as in 2) is a solution \end{underlined}\end{bolded}:
$ f(\mathbb R)\subseteq A$ $ \implies$ $ f(y)\in A$ $ \forall y$ $ \implies$ $ c(f(y)) = c(0)$ $ \implies$ $ f(f(y)) = h(c(f(y))) = h(c(0)) = 0$
$ f(y)\in A$ $ \implies$ $ - f(y)\in A$ $ \implies$ $ x - f(y)\sim x$ $ \implies$ $ c(x - f(y)) = c(x)$ $ \implies$ $ f(x - f(y)) = f(x)$

And so $ f(x - f(y)) = f(x) + f(f(y))$
Q.E.D

4) \begin{bolded}proof that any solution is in the form defined in 2\end{underlined}).\end{bolded}
Just choose as $ A$ the smallest additive subgroup generated by elements of $ I = f(\mathbb R)$ and the conclusion is immediate.

5) \begin{bolded}some examples\end{underlined}.\end{bolded}
5.1) $ A = \{0\}$, then $ c(x) = x$ and $ h(x) = 0$ $ \implies$ $ f(x) = 0$ $ \forall x$

5.2) $ A = \mathbb Z$, with $ c(x) = x - [x]$ and $ h(x) = [10\sin(x)]$ $ \implies$ $ f(x) = [10\sin(x - [x])]$

5.3) $ A = \mathbb Q$ with $ h(c(\pi)) = 1$, $ h(c(\sqrt 2)) = 2$ and $ h(x) = 0$ anywhere else $ \implies$
$ f(\pi + a) = 1$ $ \forall a\in\mathbb Q$
$ f(\sqrt 2 + a) = 2$ $ \forall a\in\mathbb Q$
$ f(x) = 0$ anywhere else.


...

[\/hide]
\end{solution}



\begin{solution}[by \href{https://artofproblemsolving.com/community/user/38215}{bboypa}]
	\begin{tcolorbox} \begin{bolded}basics elements \end{underlined}\end{bolded}:
[...]

$ P(f(x),f(x))$ $ \implies$ $ f(f(x)) = \frac a2$ \end{tcolorbox}

Is P(f(x),x)?

However too beautiful, i think you are the best with functional equation  :o
\end{solution}



\begin{solution}[by \href{https://artofproblemsolving.com/community/user/29428}{pco}]
	\begin{tcolorbox} Is P(f(x),x)?\end{tcolorbox}
Sure, you're right. Thanks for the remark. I've edited my post.

\begin{tcolorbox} However too beautiful, i think you are the best with functional equation  :o\end{tcolorbox}
Thanks  :blush:
\end{solution}
*******************************************************************************
-------------------------------------------------------------------------------

\begin{problem}[Posted by \href{https://artofproblemsolving.com/community/user/23937}{Lee Sang Hoon}]
	Find all functions $f: \mathbb R \to \mathbb R$ such that
(1) $f(x+1)=f(x)+1$ for all $x \in \mathbb R$, and
(2) For any non-zero real $x$,
\[\frac{f(x)}{x^2}=f\left(\frac 1x\right).\]
	\flushright \href{https://artofproblemsolving.com/community/c6h122010}{(Link to AoPS)}
\end{problem}



\begin{solution}[by \href{https://artofproblemsolving.com/community/user/29428}{pco}]
	\begin{tcolorbox}find all function f : R→R such that
(1)f(x+1)=f(x)+1
(2)f(x)\/x^2=f(1\/x)\end{tcolorbox}

Very interesting ...

[hide="My solution"]

Let $ \sim$ the relation defined as : "$ x\sim y$ $ \iff$ either $ x$ and $ y$ both have a finite continued fraction, either both have an infinite one with the same trailing part from some point"

Obviously, the relation $ "\sim"$ is an equivalence relation. Let $ \mathcal C(x)$ be the equivalence class of $ x$.

Let $ k_n(x)=x+n$ ($ n\in\mathbb Z$) and let $ i(x)=\frac{1}{x}$
Let us define a "path" from $ x$ to $ y$ as any sequence $ p_x^y=k_a\circ k_b\circ i\circ k_c \circ i ...$ of compositions of $ k_n$ and $ i$ such that $ y=p_x^y(x)$. 

Clearly, $ x\sim y$ $ \iff$ $ \exists$ a path $ p_x^y$

If we know a path from $ x$ to $ y$, then, using $ f(k_n(x))=k_n(f(x)$ and $ f(i(x))=\frac{1}{x^2}f(x)$, we can compute $ f(y)$ from $ f(x)$

Now are the three important properties :
P0 : If $ f(x)=x$ for some $ x$, then $ f(y)=y$ $ \forall y\sim x$
P1 : If the continued fraction of $ x$ is infinite periodic, then $ f(x)=x$
P2 : If the continued fraction of $ x$ is finite or infinite not periodic, all the paths from $ x\to y$ give the same relation between $ f(x)$ and $ f(y)$

1) P0 : If $ f(x)=x$ for some $ x$, then $ f(y)=y$ $ \forall y\sim x$
Take any path from $ x\to y$. Since $ f(k_n(x))=k_n(f(x))=k_n(x)$ and $ f(i(x))=\frac{1}{x^2}f(x)=\frac{1}{x}=i(x)$ it's easy to show with induction on elements of the path that $ f(y)=y$
So $ f(x)=x$ $ \forall x\in\mathcal C(x)$ and the two rules are always true over $ \mathcal C(x)$

2) P1 : If the continued fraction of $ x$ is infinite periodic, then $ f(x)=x$
Let $ y=[0;\overline{a_1,a_2,...,a_j}]$ (where $ a_1,...,a_j$ is the periodic part (starting anywhere).
Using as much as necessary the fact that $ f([b_0;b_1,b_2...])=b_0+\frac{1}{[b_1;b_2...]^2}f([b_1;b_2...])$, we have so $ f(y)=\alpha f(y)+\beta$ with $ \alpha < 1$ ($ \alpha$ and $ \beta$ depend on $ y$).
So $ f(y)$ has an unique value for $ y$ given. And, since $ f(y)=y$ is clearly a solution, this is the unique one.
So, using $ P0$, $ f(x)=x$.
Q.E.D.

3) If the continued fraction of $ x$ is finite or infinite not periodic, all the paths from $ x\to y$ give the same relation between $ f(x)$ and $ f(y)$
Let us call a path "canonical" if it contains neither subsequence $ i\circ i$ nor $ f_a\circ f_b$. Clearly there is a unique canonical path from $ x$ to $ y$.
Replacing $ k_n\circ k_m$ by $ k_{n+m}$ in a path gives the same relation between $ f(x)$ and $ f(y)$
Suppressing $ i\circ i$ in a path gives too the same relation between $ f(x)$ and $ f(y)$
Unicity of caninical path gives the final result.
Q.E.D.

And now the final result :
========================
Let $ c(x)$ a choice function associating to any $ x$ a representant (unique per class) of its equivalence class.
Let $ h(x)$ any function from $ c(\mathbb R)\to \mathbb R$
 
$ \forall x$ whose continued fraction is infinite periodic, $ f(x)=x$
For any other $ x\in c(\mathbb R)$, $ f(x)=h(x)$
For any other $ x$, take any path $ p_{c(x)}^x$ and compute $ f(x)$ from $ f(c(x))=h(c(x))$

And so infinitely many solutions.

Nota : If we required continuity, then :
We have $ f(n)=n+f(0)$ and $ f(\frac{1}{n})=\frac{n+f(0)}{n^2}$ and so $ \lim_{n\to +\infty}f(\frac{1}{n})=0$ and so (continuity) $ f(0)=0$
So (using $ P0$), $ f(x)=x$ $ \forall x\in\mathcal C(0)=\mathbb Q$
So, using continuity again and since $ \mathbb Q$ is dense in $ \mathbb R$, $ f(x)=x$ $ \forall x\in \mathbb R$

[\/hide]
\end{solution}



\begin{solution}[by \href{https://artofproblemsolving.com/community/user/55355}{N.N.Trung}]
	Dear pco,
My knowledge in algebra is not good as yours. And I will be pleasant for a elementary solution ...  
\end{solution}



\begin{solution}[by \href{https://artofproblemsolving.com/community/user/29428}{pco}]
	\begin{tcolorbox}Dear pco,
My knowledge in algebra is not good as yours. And I will be pleasant for a elementary solution ...  \end{tcolorbox}

I did not find any elementary solution (in the case where $ f(x)$ is not continuous) and, looking at the result, I'm not sure we can find one.  :(
\end{solution}



\begin{solution}[by \href{https://artofproblemsolving.com/community/user/33633}{ScienceJar}]
	\begin{tcolorbox}find all function f : R→R such that
(1)f(x+1)=f(x)+1
(2)f(x)\/x^2=f(1\/x)\end{tcolorbox}

[hide]
if f(0)=a then a=f(0)=f(-1+1)=f(-1)+1=f(-2+1)+1=f(-2)+2=f(-3+1)+2=f(-3)+3=.........=f(n)-n 
and since a=f(0)=f(1)-1=f(2)-2=f(3)-3=f(4)-4=.........=f(n)-n
thus a=f(n)-n or f(n)=n+a for all integers n
Thus f(x) is any function such that the integers follow f(x)=x+a; this function we will call Intish(x) so f(x)=Intish(x)+a

In (2)
f(x)-x^2*f(1\/x)=0 so:
Intish(x)+a-x^2*Intish(1\/x)-x^2*a=0  but if x is an integer then Intish[x]=x and more simplifying can be produced:
x+a-x^2*Intish(1\/x)-x^2*a=0 or further simplified for Intish(1\/x):
Intish(1\/x)=a\/x2-1\/x-a so if we solve for as x->Infinity:
Intish(0)=-a 

and finally:
a = f(0) {for integers} = Intish(0)+a = a-a = 0 
a = 0

so at least for all integers x the function f(x) = x satisfy (1) + (2) {which can easily be varified for all reals by pluging f(x)=x into both (1) + (2)}

Now Intish(x) must be defined to see if any more functions exist that f(x)=x for all reals:
if x is a fraction then 1\/x is a real number so by implementing (2):
f(x)=x^2*f(1\/x)=x^2*(1\/x)=x

but to prove that f(x)=x is the only function for real irrationals a irrational x such that 0<x<1 {or by (1) it has been reduced until it was less then one} then
f(x)=x^2f(1\/x) but f(1\/x)>1 so if 1\/x=z+r where z>1 and 0<r<1 then
f(x)=x^2f(z+r)=x^2(f(r)+z)=x^2(f(r)+1\/x-r)=x+x^2(f(r)-r)=x+x^2(r^2*f(1\/r)-r); and if 1\/r=a+b such that a>1 and 0<b<1 then
f(x)=x+x^2(r^2*f(1\/r)-r)=x+x^2(r^2*f(a+b)-r)=x+x^2(r^2*(f(b)+a)-r)=x+x^2(r^2*(f(b)+1\/r-b)-r)=x+x^2(r^2*(f(b)-b)+r-r)=x+x^2(r^2*(f(b)-b))
Notice that "x+x^2(f(r)-r)" was simplified to x+x^2(r^2*(f(b)-b)) and that if I picked any random two variables to equal 1\/b where one was between 0 and one and the other greater then 1 then the simplification would then be x+x^2*r^2*b^2*(f(random variable between 0 and 1)- (random variable between 0 and 1)) then at infinity we would get the simplification x+x^2*r^2*b^2*(variable between 0 and 1)^2*(another variable between 0 and 1)^2*............
It is clear that the second term would approach 0 because of the continuous multiplication of a number between 0 and 1 squared. 
thus the equations would become f(x)=x+0=x and thus the only function which satisfies the given two conditions is:

\begin{bolded} f(x)=x \end{bolded}

[\/hide]
\end{solution}



\begin{solution}[by \href{https://artofproblemsolving.com/community/user/29428}{pco}]
	\begin{tcolorbox} Notice that "x+x^2(f(r)-r)" was simplified to x+x^2(r^2*(f(b)-b)) and that if I picked any random two variables to equal 1\/b where one was between 0 and one and the other greater then 1 then the simplification would then be x+x^2*r^2*b^2*(f(random variable between 0 and 1)- (random variable between 0 and 1)) then at infinity we would get the simplification x+x^2*r^2*b^2*(variable between 0 and 1)^2*(another variable between 0 and 1)^2*............
It is clear that the second term would approach 0 because of the continuous multiplication of a number between 0 and 1 squared. 
thus ... \end{tcolorbox}

This "thus ..." needs the function to be continuous.
So, you only proved that the only \begin{bolded}continuous \end{bolded}\end{underlined}solution is $ f(x)=x$

But other non continuous solutions exist (see my post)
\end{solution}



\begin{solution}[by \href{https://artofproblemsolving.com/community/user/33633}{ScienceJar}]
	I see your point it would have infinite solutions since continuity is not given. 

That is very cool. Thanks :)
\end{solution}
*******************************************************************************
-------------------------------------------------------------------------------

\begin{problem}[Posted by \href{https://artofproblemsolving.com/community/user/5820}{N.T.TUAN}]
	Prove that there does not exist a function $f: \mathbb R\to\mathbb R$ such that $f(f(x))=x^2-2$ for all $x\in\mathbb R$.
	\flushright \href{https://artofproblemsolving.com/community/c6h126371}{(Link to AoPS)}
\end{problem}



\begin{solution}[by \href{https://artofproblemsolving.com/community/user/14052}{t0rajir0u}]
	I have seen many problems like this.  A question:  can we characterize all reals $a$ for which there does not exist $f$ such that

$f(f(x)) = x^{2}+a$

?
\end{solution}



\begin{solution}[by \href{https://artofproblemsolving.com/community/user/5820}{N.T.TUAN}]
	\begin{tcolorbox}I have seen many problems like this.  A question:  can we characterize all reals $a$ for which there does not exist $f$ such that

$f(f(x)) = x^{2}+a$

?\end{tcolorbox}

$f(f(x)) = ax^{2}+bx+c$ , not exists function , characterize $a,b,c$. This forum has it, but I can't find it  :(
\end{solution}



\begin{solution}[by \href{https://artofproblemsolving.com/community/user/13}{enescu}]
	[url]http://www.mathlinks.ro/Forum/viewtopic.php?t=14151[\/url]
\end{solution}



\begin{solution}[by \href{https://artofproblemsolving.com/community/user/18275}{fermat3}]
	the  idia in this problem that we can see points fixs like this F(F(x))=x^2 -1996. :)
\end{solution}



\begin{solution}[by \href{https://artofproblemsolving.com/community/user/29721}{Erken}]
	In fact, a more general proposition holds:

For any set $ A$, if a function $ g: A\rightarrow A$ has exactly two fixed points $ \{a,b\}$ and $ g(g(x))$ has exactly four fixed points $ \{a,b,c,d\}$, then there is no function $ f$, such that $ g = f(f(x))$.

Let's check whether, the condition holds for the function $ g = x^2 -2$ - it has two fixed points $ \{2,-1\}$. The equation $ x^2-2=x$ has exactly two roots, and the equation, $ x^4-4x^2+2=x$, had the following fixed points $ \{2,-1,\frac{-1+\sqrt{5}}{2}, \frac{-1-\sqrt{5}}{2}\}$.

Let's try to give an answer to the \begin{bolded}t0rajir0u\end{bolded}'s question:

Clearly the first condition must be that $ 1-4a > 0$, or $ a<\frac{1}{4}$. That's not hard to check that:

$ x^4+2a\cdot x^2 + a^2 + a - x = (x^2+x+a+1)(x^2-x+a)$

So the second condition is $ 1-(4a+4)>0$, or $ a<-\frac{3}{4}$. Combining it with the previous condition, we conclude that if $ \boxed{a<-\frac{3}{4}}$, then there is no such function.

That's not a complete answer, clearly, but at least something.

\begin{bolded}Proof of the proposition:
\end{bolded}
Assume that $ g(c)=x$, for some, $ x\in A$. Then $ g(x)=c$,  then $ g(g(x))=g(c)=x$, so $ x$ is a fixed point for $ g$, therefore, $ x$ is $ c$ or $ d$. As long as $ g(c)\neq c$, we conclude that $ g(c)=d$ and analogously $ g(d)=c$.

Now suppose that $ g(x) = f(f(x))$. Then obviously, $ \boxed{f(g(x))=g(f(x))}$.

Let's consider $ f(a)$. Clearly $ f(a)=f(g(a))=g(f(a))$, so $ f(a)$ is either $ a$ or $ b$. The same holds for $ f(b)$. Let consider the following equation:
 $ f(d)=f(g(c))=g(f(c))=g(f(g(d)))=g(g(f(d)))$. So $ f(d)$ is either of $ \{a,b,c,d\}$ and the same statement holds for $ f(c)$.

Consider the following cases:

$ f(c)=a$:

Then $ f(f(c))=f(a)=g(c) = d$, which is a contradiction, because, as we've previously established $ f(a)$ is either $ a$ or $ b$. 

The same way leads to the impossibility of $ f(c)=b$. Now if $ f(c)=c$, then $ f(c)=f(f(c))=g(c)=d$, contradiction. So the last case is $ f(c)=d$.

Then $ g(d)=g(f(c))=f(g(c))=f(d)=f(f(c))=g(c)=d$, contradiction.

So finally the lemma is proven.
\end{solution}



\begin{solution}[by \href{https://artofproblemsolving.com/community/user/50028}{hophinhan}]
	Copyed from a book ...

\begin{bolded}Solution.\end{bolded} After some attempts we can see that none of the first three methods leads to a progress.
Notice that the function g of the right-hand side has exactly 2 fixed points and that the function g ◦g
has exactly 4 fixed points. Now we will prove that there is no function f such that f ◦ f = g. Assume
the contrary. Let a,b be the fixed points of g, and a,b,c,d the fixed points of g ◦ g. Assume that
g(c)= y. Then c= g(g(c))= g(y), hence g(g(y))= g(c)= y and y has to be on of the fixed points of
g ◦g. If y = a then from a = g(a)= g(y)= c we get a contradiction. Similarly y 6= b, and since y 6= c
we get y = d. Thus g(c) = d and g(d) = c. Furthermore we have g( f (x)) = f ( f ( f (x))) = f (g(x)).
Let x0 ∈ {a,b}. We immediately have f (x0) = f (g(x0)) = g( f (x0)), hence f (x0) ∈ {a,b}. Similarly
if x1 ∈ {a,b,c,d} we get f (x1) ∈ {a,b,c,d}, and now we will prove that this is not possible. Take
first f (c) = a. Then f (a) = f ( f (c)) = g(c) = d which is clearly impossible. Similarly f (c) 6= b and
f (c) 6=c (for otherwise g(c)=c) hence f (c)=d. Howeverwe then have f (d)= f ( f (c)) =g(c)=d,
which is a contradiction, again. This proves that the required f doesn’t exist. △
\end{solution}



\begin{solution}[by \href{https://artofproblemsolving.com/community/user/29428}{pco}]
	\begin{tcolorbox}Prove that NOT exists function $ f: \mathbb R\to\mathbb R$ such that $ f(f(x)) = x^{2} - 2\forall x\in\mathbb R$.\end{tcolorbox}

Beside pretty Erken's demo, just notice that there exists infinitely many $ f: [2,+\infty)\to[2,+\infty)$ such that $ f(f(x)) = x^{2} - 2\forall x\in[2,+\infty)$
\end{solution}



\begin{solution}[by \href{https://artofproblemsolving.com/community/user/143628}{MANMAID}]
	Suppose that $f(x)$ has a root $c$ , then $f(0)=c^2-2\Rightarrow f(x)$ has only one real root$\Rightarrow f(x)$ is monotony increasing or decreasing. 
Now consider one case: 
Since $x^2-2$ is a two degree eq. and cuts $x$- axis two times then there is two interval where $f(x)$ is $0$. But $f$ has at most one root. This is a contradiction. So there does exist such $f$.
\end{solution}



\begin{solution}[by \href{https://artofproblemsolving.com/community/user/84645}{s7o0ory}]
	How about the function f(a+a^(-1))=a^(sqrt(2))+a^(-sqrt(2)), where a is a real number or a complex number on the unit circle? One can prove that this function is well defined from R to R and that it satisfies the equation! Any explanations please?
\end{solution}
*******************************************************************************
-------------------------------------------------------------------------------

\begin{problem}[Posted by \href{https://artofproblemsolving.com/community/user/19490}{behemont}]
	Find all functions $ f: \mathbb{R^+} \to \mathbb{R^+}$ satisfying the equation
\[f(a^3+b^3)=(f(a)+b)(f(a^2)-af(b)+b^2),\]
for all $a,b \in \mathbb{R}^+$.
	\flushright \href{https://artofproblemsolving.com/community/c6h178896}{(Link to AoPS)}
\end{problem}



\begin{solution}[by \href{https://artofproblemsolving.com/community/user/19490}{behemont}]
	i can't solve it and need help..pco? 
\end{solution}



\begin{solution}[by \href{https://artofproblemsolving.com/community/user/18420}{aviateurpilot}]
	$ f(0) = f(0)^2$ we take $ k = f(0)$
if $ k = 0$ then $ f(b^3) = (k + b)(k + b^2) = b^3$
then $ \forall x\in R^ + : \ f(x) = x$
if $ k = 1$ then $ f(b^3) = (1 + b)(1 + b^2) = 1 + b + b^2 + b^3$
so $ 4=f(1+0^3)=(f(1)+0)(f(1)-f(0)+0^2)=(4+0)(4-1)$ gives contradiction
so $ f(x)\equiv x$
\end{solution}



\begin{solution}[by \href{https://artofproblemsolving.com/community/user/19490}{behemont}]
	\begin{tcolorbox}$ f(0) = f(0)^2$ we take $ k = f(0)$
if $ k = 0$ then $ f(b^3) = (k + b)(k + b^2) = b^3$
then $ \forall x\in R^ + : \ f(x) = x$
if $ k = 1$ then $ f(b^3) = (1 + b)(1 + b^2) = 1 + b + b^2 + b^3$
so $ 4 = f(1 + 0^3) = (f(1) + 0)(f(1) - f(0) + 0^2) = (4 + 0)(4 - 1)$ gives contradiction
so $ f(x)\equiv x$\end{tcolorbox}

sorry aviateurpilot, but we don't know what is $ f(0)$.. :P
\end{solution}



\begin{solution}[by \href{https://artofproblemsolving.com/community/user/18420}{aviateurpilot}]
	\begin{tcolorbox}[quote="aviateurpilot"]$ f(0) = f(0)^2$ we take $ k = f(0)$
if $ k = 0$ then $ f(b^3) = (k + b)(k + b^2) = b^3$
then $ \forall x\in R^ + : \ f(x) = x$
if $ k = 1$ then $ f(b^3) = (1 + b)(1 + b^2) = 1 + b + b^2 + b^3$
so $ 4 = f(1 + 0^3) = (f(1) + 0)(f(1) - f(0) + 0^2) = (4 + 0)(4 - 1)$ gives contradiction
so $ f(x)\equiv x$\end{tcolorbox}

sorry aviateurpilot, but we don't know what is $ f(0)$.. :P\end{tcolorbox}
hhhh but we know that $ f(0)^2 = f(0)$
so if $ f(0) = 0$ i have found that $ f(x) = 0$
if $ f(0) = 0$ i get contradiction,
plz read ma solution again,  :D

 :arrow: me i konw that $ \mathbb{R}^+=\{x\in\mathbb{R}|\ x\ge 0\}$
\end{solution}



\begin{solution}[by \href{https://artofproblemsolving.com/community/user/19490}{behemont}]
	i woudn't agree on that one..$ \mathbb{R}^+$ means only positive numbers, and not $ 0$... :!:
\end{solution}



\begin{solution}[by \href{https://artofproblemsolving.com/community/user/19490}{behemont}]
	anyone please :( ??
\end{solution}



\begin{solution}[by \href{https://artofproblemsolving.com/community/user/29428}{pco}]
	\begin{tcolorbox}Find all functions $ f: \mathbb{R^ + } \to \mathbb{R^ + }$ satisfying the equation

$ f(a^3 + b^3) = (f(a) + b)(f(a^2) - af(b) + b^2)$ $ ;$$ \forall a,b \in \mathbb{R}^ +$\end{tcolorbox}

The difficulty here is the impossibility to use $ x=0$ or $ y=0$ since neither the function, neither the assertion are available for these values.

Here is a rather ugly demo (I would be interested in anything better).

Let $ P(x,y)$ be the assertion $ f(x^3+y^3)=(f(x)+y)(f(x^2)-xf(y)+y^2)$

$ P(x,y)$ $ \implies$ $ f(x^3+y^3)=(f(x)+y)(f(x^2)-xf(y)+y^2)$
$ P(y,x)$ $ \implies$ $ f(x^3+y^3)=(f(y)+x)(f(y^2)-yf(x)+x^2)$

From these two lines, we get $ f(x^2)=\frac{(f(y)+x)(f(y^2)-yf(x)+x^2)}{f(x)+y}+xf(y)-y^2$

So $ Q(x,y)$ : $ f(x^2)=\frac{f(x)((x-y)f(y)-y(x+y))+f(y)f(y^2)+f(y)x(x+y)+xf(y^2)+x^3-y^3)}{f(x)+y}$

Let $ f(1)=a$
$ P(1,1)$ $ \implies$ $ f(2)=a+1$
$ Q(2,1)$ $ \implies$ $ f(4)=2(a+1)$

$ Q(x,1)$ $ \implies$ $ f(x^2)=\frac{f(x)((a-1)x-(a+1))+x^3+ax^2+2ax+a^2-1}{f(x)+1}$
$ Q(x,2)$ $ \implies$ $ f(x^2)=\frac{f(x)((a-1)x-2(a+3))+x^3+(a+1)x^2+4(a+1)x+2a^2+4a-6}{f(x)+2}$

Equating the two RHS gives :
$ \frac{f(x)((a-1)x-(a+1))+x^3+ax^2+2ax+a^2-1}{f(x)+1}$ $ =\frac{f(x)((a-1)x-2(a+3))+x^3+(a+1)x^2+4(a+1)x+2a^2+4a-6}{f(x)+2}$

$ f(x)(f(x)+2)((a-1)x-(a+1))$ $ +(f(x)+2)(x^3+ax^2+2ax+a^2-1)$ $ =f(x)(f(x)+1)((a-1)x-2(a+3))$ $ +(f(x)+1)(x^3+(a+1)x^2+4(a+1)x+2a^2+4a-6)$

$ f(x)^2((a-1)x-(a+1))+$ $ f(x)(x^3+ax^2+2(2a-1)x+a^2-2a-3)+2x^3+2ax^2+4ax+2a^2-2$ $ =f(x)^2((a-1)x-2(a+3))+$ $ f(x)(x^3+(a+1)x^2+(5a+3)x+2a^2+2a-12)+x^3+(a+1)x^2+4(a+1)x+2a^2+4a-6$

$ f(x)^2(a+5)-f(x)(x^2+(a+5)x+a^2+4a-9)+x^3+(a-1)x^2-4x-4a+4=0$

$ (f(x)-x-a+1)(f(x)(a+5)-x^2+4)=0$

And so : $ \forall x$, either $ f(x)=x+a-1$, either $ f(x)=\frac{x^2-4}{a+5}$

The second possibility is impossible for $ x\leq 2$ since then we would have $ f(x)<0$ and so :

$ \forall x\leq 2$, $ f(x)=x+a-1$

Consider then $ x\leq 1$ : $ x^2\leq 2$ and $ x^3+1\leq 2$ so we can use $ f(x)=x+a-1$ in $ P(x,1)$ which gives $ (a-a^2)(x-1)=0$
And so $ a=1$ ($ a=0$ is impossible since $ f(x)>0$ $ \forall x$)

So, $ \forall x\leq 2$ $ f(x)=x$

Using then $ Q(x,1)$ for any $ x\leq 2$, we get $ f(x^2)=\frac{-2f(x)+x^3+x^2+2x}{f(x)+1}$ $ =\frac{-2x+x^3+x^2+2x}{x+1}=x^2$

And so $ f(x)=x$ $ \forall x\leq 4$ ...

And so $ f(x)=x$ $ \forall x$
And it's easy to verify that this function actually fits the initial requirements.
\end{solution}
*******************************************************************************
-------------------------------------------------------------------------------

\begin{problem}[Posted by \href{https://artofproblemsolving.com/community/user/30374}{hoangclub}]
	Find all function $f: \mathbb R^{+} \to \mathbb R^{+}$ such that for every real  $ x>0$ and $ z \in (0,1)$, we have \[(1- z)  f(x)= f\left( \frac{(1- z)f(xz)}{z}\right).\]
	\flushright \href{https://artofproblemsolving.com/community/c6h183666}{(Link to AoPS)}
\end{problem}



\begin{solution}[by \href{https://artofproblemsolving.com/community/user/29428}{pco}]
	\begin{tcolorbox}Find all function satisfying: $ f$:$ R^ +$-->$ R^ +$
           with every $ x$>0 and $ z$ $ \in$ (0;1) then (1-$ z$) $ f(x)$=$ f$($ f(xz)$(1-$ z$)\/$ z$).\end{tcolorbox}

Let $ f(x)=xg(x)$. The equation becomes $ (1-z)xg(x)=xzg(xz)\frac{1-z}{z}g(xzg(xz)\frac{1-z}{z})$ and so $ g(x)=g(xz)g(g(xz)(x-xz))$
Let $ X=x(1-z)$ and $ Y=xz$. The equation becomes $ g(X+Y)=g(Y)g(Xg(Y))$

So we have to find all functions $ g(x)$ : $ \mathbb R^+\to\mathbb R^+$ such that assertion $ P(x,y)$ : $ g(x+y)=g(y)g(xg(y))$ is true $ \forall x>0,y>0$

If $ \exists a$ such that $ g(a)>1$, then $ P(\frac{a}{g(a)-1},a)$ $ \implies$ $ g(\frac{ag(a)}{g(a)-1})=g(a)g(\frac{ag(a)}{g(a)-1})$ and so $ g(a)=1$

So $ g(x)\leq 1$ $ \forall x$ and so $ g(x+y)\leq g(y)$ and so $ g(x)$ is non decreasing.

If $ g(a)=g(b)$ for some $ a>b$ then $ P(a-b,b)$ $ \implies$ $ g(a)=g(b)g((a-b)g(b))$ and so $ \exists u$ such that $ g(u)=1$.
Then $ P(x,u)$ $ \implies$ $ g(x+u)=g(x)$ and so $ g(x)$ is constant (since non decreasing) and so $ g(x)=1$ $ \forall x$

If $ g(x)$ is not the constant function, then it's injective.

Then $ P(\frac{x}{g(y)},y)$ $ \implies$ $ g(\frac{x}{g(y)}+y)=g(x)g(y)$ and $ P(\frac{y}{g(x)},x)$ $ \implies$ $ g(\frac{y}{g(x)}+x)=g(y)g(x)$

So $ g(\frac{x}{g(y)}+y)=g(\frac{y}{g(x)}+x)$ and, since $ g(x)$ is injective : $ \frac{x}{g(y)}+y=\frac{y}{g(x)}+x$

And so $ \frac{x}{\frac{1}{g(x)}-1}$ $ =\frac{y}{\frac{1}{g(y)}-1}$ and so $ \exists c$ such that $ \frac{x}{\frac{1}{g(x)}-1}=\frac{1}{c}$

And so $ g(x)=\frac{1}{1+cx}$


And so the solutions of the initial equation :
$ f(x)=x$

$ f(x)=\frac{x}{1+cx}$

And it's easy to check back that these two solutions fit the initial requirements.
\end{solution}
*******************************************************************************
-------------------------------------------------------------------------------

\begin{problem}[Posted by \href{https://artofproblemsolving.com/community/user/3182}{Kunihiko_Chikaya}]
	Find all functions $ f : \mathbb{R} \to \mathbb{R}$ such that \[ f(x+y)f(f(x)-y)=xf(x)-yf(y)\] for all $ \ x,y \in \mathbb{R}.$
	\flushright \href{https://artofproblemsolving.com/community/c6h188108}{(Link to AoPS)}
\end{problem}



\begin{solution}[by \href{https://artofproblemsolving.com/community/user/32006}{Qruni}]
	[hide="First solution  :D "]
$ f(x)=x$
$ f(x)=0$
[\/hide]
\end{solution}



\begin{solution}[by \href{https://artofproblemsolving.com/community/user/16909}{limsk1}]
	$ f(0) = c$, $ f(x) = 0$ for all real number $ x$ except 0 could be solution too :P
\end{solution}



\begin{solution}[by \href{https://artofproblemsolving.com/community/user/29721}{Erken}]
	I'll rewrite my solution in another way:
Solution:
$ x = 0$ in the main equation gives us:
$ f(y)f(f(0) - y) = - yf(y)$,
so it is either $ f(y) = 0$,or $ f(c - y) = - y$,where $ c = f(0)$.
Hence $ f(c - y) = - y$,for all $ y\in\mathbb{R}$,such that $ f(y)\neq 0$.
$ y\rightarrow c - y$,so $ f(y) = y - c$,for all $ y$,such that $ f(c - y)\neq 0$.
If $ f(x) = x - c$,for some $ x\neq c$ then we make substitution
$ y = 0$:
$ (x - c)f(x - c) = x(x - c)$.
If $ f(x - c) = x$ and $ x\neq 0$,then $ c = 0$.
If $ x = 0$,then $ c = 0$.
So $ f(x) = x$,if $ f( - x)\neq 0$.
Suppose that there exist $ x_0,y_0\in\mathbb{R}$,such that $ f(x_0) = 0,f(y_0) = y_0,y_0\neq 0$.
If we put $ x = x_0,y = y_0$:
$ f(x_0 + y_0)f( - y_0) = - y_0^2$
It is obvious that $ f( - y_0),f(x_0 + y_0)\neq 0$,thus $ f(x_0 + y_0) = x_0 + y_0$,hence $ x_0 = 0$.
So if $ f(x) = x$,for some $ x\in\mathbb{R}$,then it is true for all $ x\in\mathbb{R}$.
If $ f(y) = 0$,then it is true for all $ y\in\mathbb{R}$,except $ y = 0$.
So the answer is $ f(y) = y$ and $ f(y) = 0,f(0) = c,y\neq 0$.
\end{solution}



\begin{solution}[by \href{https://artofproblemsolving.com/community/user/3182}{Kunihiko_Chikaya}]
	\begin{tcolorbox}Very strange,that an easy problem like this was at Japan MOF.\end{tcolorbox}

I agree with you.
\end{solution}



\begin{solution}[by \href{https://artofproblemsolving.com/community/user/29721}{Erken}]
	\begin{tcolorbox}[quote="Erken"]Very strange,that an easy problem like this was at Japan MOF.\end{tcolorbox}

I agree with you.\end{tcolorbox}
Where is the other problems?
Thank you.
\end{solution}



\begin{solution}[by \href{https://artofproblemsolving.com/community/user/31988}{ringos}]
	\begin{tcolorbox}
now since $ f$ is injective and $ f(2x)\neq 0,$,then
$ f(f(x) - x) = 0 = f(0)$.
\end{tcolorbox}

Why is $ f$ injective?

In fact, there are strange solutions:

$ f(x) = 0 (x \neq 0), f(0) = c$
\end{solution}



\begin{solution}[by \href{https://artofproblemsolving.com/community/user/29721}{Erken}]
	\begin{tcolorbox}[quote="Erken"]
now since $ f$ is injective and $ f(2x)\neq 0,$,then
$ f(f(x) - x) = 0 = f(0)$.
\end{tcolorbox}

Why is $ f$ injective?

In fact, there are strange solutions:

$ f(x) = 0 (x \neq 0), f(0) = c$\end{tcolorbox}
It is injective,because $ f(f(x)) = x$.
Oh yes,i forgot to consider $ f(0)$,when $ f(x)\equiv 0$,for all $ x\in\mathbb{R},x\neq 0$.So $ f(0)=c,f(x)=0,\forall x\in\mathbb{R},x\neq 0$ is a solution too.Thank you.
\end{solution}



\begin{solution}[by \href{https://artofproblemsolving.com/community/user/31988}{ringos}]
	\begin{tcolorbox}[quote="ringos"]\begin{tcolorbox}
now since $ f$ is injective and $ f(2x)\neq 0,$,then
$ f(f(x) - x) = 0 = f(0)$.
\end{tcolorbox}

Why is $ f$ injective?

In fact, there are strange solutions:

$ f(x) = 0 (x \neq 0), f(0) = c$\end{tcolorbox}
It is injective,because $ f(f(x)) = x$.
Oh yes,i forgot to consider $ f(0)$,when $ f(x)\equiv 0$,for all $ x\in\mathbb{R},x\neq 0$.So $ f(0) = c,f(x) = 0,\forall x\in\mathbb{R},x\neq 0$ is a solution too.Thank you.\end{tcolorbox}

$ f(f(x)) = x$ for some $ x$, not for all $ x$.
\end{solution}



\begin{solution}[by \href{https://artofproblemsolving.com/community/user/3182}{Kunihiko_Chikaya}]
	\begin{tcolorbox}[quote="kunny"]\begin{tcolorbox}Very strange,that an easy problem like this was at Japan MOF.\end{tcolorbox}

I agree with you.\end{tcolorbox}
Where is the other problems?
Thank you.\end{tcolorbox}

Here are.

[url=http://www.mathlinks.ro/viewtopic.php?t=188109]Problem 1[\/url]
[url=http://www.mathlinks.ro/viewtopic.php?t=188111]Problem 3[\/url]
[url=http://www.mathlinks.ro/viewtopic.php?t=188110]Problem 5[\/url]
\end{solution}



\begin{solution}[by \href{https://artofproblemsolving.com/community/user/29721}{Erken}]
	I've posted another solution,i hope it is clear now.See above.
\end{solution}



\begin{solution}[by \href{https://artofproblemsolving.com/community/user/31750}{primoz2}]
	This problem isnt so easy as it looks like, because just like ringos wrote f(f(x))=x just for some x.  I will post my solution as soon as i have time. I just hope i didnt make a mistake.
\end{solution}



\begin{solution}[by \href{https://artofproblemsolving.com/community/user/29721}{Erken}]
	\begin{tcolorbox}This problem isnt so easy as it looks like, because just like ringos wrote f(f(x))=x just for some x.  I will post my solution as soon as i have time. I just hope i didnt make a mistake.\end{tcolorbox}
I've already changed my solution,i didn't use $ f(f(x))=x$,i replaced the old proof,by new,see above.
\end{solution}



\begin{solution}[by \href{https://artofproblemsolving.com/community/user/30326}{quangpbc}]
	\begin{tcolorbox}I've posted another solution,i hope it is clear now.See above.\end{tcolorbox}

[color=green]Where is the first one, Erken?[\/color]
\end{solution}



\begin{solution}[by \href{https://artofproblemsolving.com/community/user/29721}{Erken}]
	I replaced it by a new proof few minutes ago.
\end{solution}



\begin{solution}[by \href{https://artofproblemsolving.com/community/user/30326}{quangpbc}]
	[color=green]Uhm, and what about it ? Did you have a mistake with this solution? Show it again please, friend.[\/color]
\end{solution}



\begin{solution}[by \href{https://artofproblemsolving.com/community/user/31988}{ringos}]
	\begin{tcolorbox}
If $ x = 0$,then $ c = 0$.
\end{tcolorbox}

Could you explain this part?

I couldn't solve this problem in the competition.
\end{solution}



\begin{solution}[by \href{https://artofproblemsolving.com/community/user/3182}{Kunihiko_Chikaya}]
	\begin{tcolorbox}This problem isnt so easy as it looks like, because just like ringos wrote f(f(x))=x just for some x.  I will post my solution as soon as i have time. I just hope i didnt make a mistake.\end{tcolorbox}

At first sight, I thought that this problem is not difficult, but when I began to solve, it seemed to be unable to solve so easily.
\end{solution}



\begin{solution}[by \href{https://artofproblemsolving.com/community/user/29721}{Erken}]
	\begin{tcolorbox}[quote="Erken"]
If $ x = 0$,then $ c = 0$.
\end{tcolorbox}

Could you explain this part?

I couldn't solve this problem in the competition.\end{tcolorbox}
We have that $ f(x)=x-c$,so if $ x=0$,then $ f(0)=-c=-f(0)$,hence $ c=0$.So in any case $ c=0$.
\end{solution}



\begin{solution}[by \href{https://artofproblemsolving.com/community/user/31988}{ringos}]
	\begin{tcolorbox}We have that $ f(x) = x - c$,so if $ x = 0$,then $ f(0) = - c = - f(0)$,hence $ c = 0$.So in any case $ c = 0$.\end{tcolorbox}

Thank you, I understood. Nice solution.
\end{solution}



\begin{solution}[by \href{https://artofproblemsolving.com/community/user/29721}{Erken}]
	\begin{tcolorbox}[color=green]Uhm, and what about it ? Did you have a mistake with this solution? Show it again please, friend.[\/color]\end{tcolorbox}
No,i think this solution is correct,friend. :)
P.S:I've got a nice number of posts:$ 666$. 
\end{solution}



\begin{solution}[by \href{https://artofproblemsolving.com/community/user/22187}{Jumbler}]
	\begin{tcolorbox}
If $ f(x) = x - c$,for some $ x\neq c$ then we make substitution
$ y = 0$:
$ (x - c)f(x - c) = x(x - c)$.
\end{tcolorbox}

Where do you substitute $ y=0$?
\end{solution}



\begin{solution}[by \href{https://artofproblemsolving.com/community/user/29721}{Erken}]
	\begin{tcolorbox}[quote="Erken"]
If $ f(x) = x - c$,for some $ x\neq c$ then we make substitution
$ y = 0$:
$ (x - c)f(x - c) = x(x - c)$.
\end{tcolorbox}

Where do you substitute $ y = 0$?\end{tcolorbox}
In the main equation
\end{solution}



\begin{solution}[by \href{https://artofproblemsolving.com/community/user/32448}{thanhnam2902}]
	Thank for your post. Thank you very much.
\end{solution}



\begin{solution}[by \href{https://artofproblemsolving.com/community/user/3182}{Kunihiko_Chikaya}]
	[hide="Answer"]$ f(x)=x$ and $ f(x)=\left\{
\begin{array}{ll}
c\ (x=0)\ c: arbiterary\ real\ numbers  &\quad  \\
0\ (x\neq 0) &\quad
\end{array}
\right.$[\/hide]
\end{solution}



\begin{solution}[by \href{https://artofproblemsolving.com/community/user/29721}{Erken}]
	\begin{tcolorbox}[hide="Answer"]$ f(x) = x$ and $ f(x) = \left\{ \begin{array}{ll} c\ (x = 0)\ c: arbiterary\ real\ numbers & \quad \\
0\ (x\neq 0) & \quad \end{array} \right.$[\/hide]\end{tcolorbox}
I've already solved this problem,see the previous list 
\end{solution}



\begin{solution}[by \href{https://artofproblemsolving.com/community/user/1147}{stergiu}]
	\begin{tcolorbox}I'll rewrite my solution in another way:
Solution:
$ x = 0$ in the main equation gives us:
$ f(y)f(f(0) - y) = - yf(y)$,
so it is either $ f(y) = 0$,or $ f(c - y) = - y$,where $ c = f(0)$................
......................................
\end{tcolorbox}

 I'm afraid that this conclution is not valid. Generally, if $ f(x)g(x) = 0 ,   \forall x \in \mathbb R$ , we can not take $ f(x) = 0 , \forall x \in \mathbb R$ or $ g(x) = 0 , \forall x \in \mathbb R$ .Do you have something more in mind ?

 Babis
\end{solution}



\begin{solution}[by \href{https://artofproblemsolving.com/community/user/10088}{silouan}]
	See here (at the last post ) for a nice straightforward solution 

http://www.mathlinks.ro/viewtopic.php?t=194064
\end{solution}



\begin{solution}[by \href{https://artofproblemsolving.com/community/user/1147}{stergiu}]
	\begin{tcolorbox}See here (at the last post ) for a nice straightforward solution 

http://www.mathlinks.ro/viewtopic.php?t=194064\end{tcolorbox}

 Thank you Silouan!
 Good luck by the IMO 2008 !!!

 \begin{bolded}NOTE\end{bolded}

 The solution at the last post(in the greek forum) is in english , so everybody can read it.

 Babis
\end{solution}



\begin{solution}[by \href{https://artofproblemsolving.com/community/user/50028}{hophinhan}]
	\begin{tcolorbox}Find all functions $ f : \mathbb{R} \mapsto \mathbb{R}$ such that $ f(x + y)f(f(x) - y) = xf(x) - yf(y)$ for all $ \ x,y \in \mathbb{R}.$\end{tcolorbox}

+ $ f = 0$

+ $ f \neq 0$

$ y = 0\ = > \ f(x).f(f(x)) = x.f(x) \ = > \ f(f(x)) = x \ \ \ (1)$

$ y: = f(x)\ = > \ f(x + f(x)).f(0) = x.f(x) - f(x).f(f(x))\ \ \ (2)$

$ (1)\ \ vs\ \ (2) \ \ f(0) = 0$

$ x = 0\ = > \ f(y).f( - y) = - y.f(y) \ = > \ f( - y) = - y\ = > \ f(x) = x$

Two solutions :  $ f(x) = 0 \ \ ; \ \ f(x) = x$
\end{solution}



\begin{solution}[by \href{https://artofproblemsolving.com/community/user/29428}{pco}]
	\begin{tcolorbox} 
Two solutions :  $ f(x) = 0 \ \ ; \ \ f(x) = x$\end{tcolorbox}

You should read the entire thread.
There exist other solutions :

$ f(0)=a\neq 0$ and $ f(x)=0$ $ \forall x\neq 0$
\end{solution}



\begin{solution}[by \href{https://artofproblemsolving.com/community/user/29721}{Erken}]
	\begin{tcolorbox}[quote="Erken"]I'll rewrite my solution in another way:
Solution:
$ x = 0$ in the main equation gives us:
$ f(y)f(f(0) - y) = - yf(y)$,
so it is either $ f(y) = 0$,or $ f(c - y) = - y$,where $ c = f(0)$................
......................................
\end{tcolorbox}

 I'm afraid that this conclution is not valid. Generally, if $ f(x)g(x) = 0 , \forall x \in \mathbb R$ , we can not take $ f(x) = 0 , \forall x \in \mathbb R$ or $ g(x) = 0 , \forall x \in \mathbb R$ .Do you have something more in mind ?

 Babis\end{tcolorbox}

Please read it through more carefully, I didn't make such a conlusion.
\end{solution}



\begin{solution}[by \href{https://artofproblemsolving.com/community/user/40564}{mehmetcantu}]
	we have 2 case first is f(x)=0 second is not.
for second;
let y=x so we have $ f(2x).f(f(x)-x)=0$ and plugging y=0 we have $ f(f(x))=x$
so taking f of two sides in first equation we have     $ f(x)=f(0)+x$ writing it to first function we get $ f(x)=x$ or from first case $ f(x)=0$
\end{solution}



\begin{solution}[by \href{https://artofproblemsolving.com/community/user/60032}{Stephen}]
	I didn't know that this problem is from Japan.

Thank you for posting the problem that I wanted.
\end{solution}



\begin{solution}[by \href{https://artofproblemsolving.com/community/user/191127}{sayantanchakraborty}]
	Oh!!I must mention that there are too many useless posts of this problem.
\end{solution}



\begin{solution}[by \href{https://artofproblemsolving.com/community/user/184652}{CanVQ}]
	\begin{tcolorbox}Find all functions $ f : \mathbb{R} \mapsto \mathbb{R}$ such that $ f(x+y)f(f(x)-y)=xf(x)-yf(y)\quad (1)$ for all $ \ x,y \in \mathbb{R}.$\end{tcolorbox}
Let $a=f(0).$ Replacing $y=0,$ we get \[f(x)\cdot f\big(f(x)\big)=x\cdot f(x),\quad \forall x \in \mathbb R. \quad (2)\] Now, replacing $y=f(x)$ and using $(2),$ we get \[a\cdot f\big(x+f(x)\big)=0.\quad (3)\] There are two cases to consider:

\begin{bolded}Case 1: \end{bolded}$\mathbf{a=0}.$ Replacing $x=0$ in $(1),$ we get \[f(y)\cdot f({-y})=-y\cdot f(y),\quad \forall y \in \mathbb R. \quad (4)\] Replacing $y$ by $-y$ in $(4),$ we get \[-y\cdot f(y)=f(y)\cdot f({-y})=y\cdot f({-y}),\] so by combining with the fact $f(0)=0,$ we get \[f({-y})=-f(y),\quad \forall y \in \mathbb R. \quad (5)\] Using this result in $(4),$ we obtain \[f(y)=0\vee f(y)=y,\quad \forall y \in \mathbb R. \quad (6)\] Assume that, there exist $u,\,v \ne 0$ such that $f(u)=u$ and $f(v)=0.$ Replacing $x=u,\,y=v$ in $(1),$ we get \[f(u+v)\cdot f(u-v)=u^2.\] Since $u^2 \ne 0,$ we must have $f(u+v) \ne 0$ and $f(u-v)\ne 0.$ Therefore, $f(u+v)=u+v,\, f(u-v)=u-v.$ It follows that \[u^2=(u+v)(u-v)=u^2-v^2,\] and hence $v=0,$ a contradiction. So, we must have $f(x)=0,\, \forall x \in \mathbb R$ or $f(x)=x,\, \forall x \in \mathbb R.$ It is easy to check that these functions satisfy our condition.

\begin{bolded}Case 2: $\mathbf{ a\ne 0}.$\end{bolded} From $(3),$ we get \[f\big( x+f(x)\big) =0,\quad \forall x \in \mathbb R. \quad (7)\] Now, by replacing $x$ by $x+f(x),$ $y$ by $x$ in $(1),$ we get \[f\big(2x+f(x)\big)\cdot f({-x})=-x\cdot f(x),\quad \forall x \in \mathbb R. \quad (8)\] Replacing $y$ by $x+ f(x),$ we also have \[f\big( 2x+f(x)\big)\cdot f({-x})=x\cdot f(x),\quad \forall x \in \mathbb R. \quad (9)\] Combining $(8)$ and $(9),$ we get \[f(x)=0,\quad \forall x \ne 0.\] Therefore, \[f(x)=\begin{cases} 0 &\text{if } x \ne 0 \\ a &\text{if } x =0 \end{cases}\] It is easy to check that this function satisfies the given condition.
\end{solution}



\begin{solution}[by \href{https://artofproblemsolving.com/community/user/212043}{Mikasa}]
	Let us denote the given statement by $P(x,y)$. Then,
$P(x,f(x))\Rightarrow f(x+f(x))f(0)=xf(x)-f(x)f(f(x))$.
$P(x,0)\Rightarrow f(x)f(f(x))=xf(x)$.
So, $f(x+f(x))f(0)=0 \forall x\in \mathbb{R}$.
So we have three cases here.

\begin{bolded}Case 1:\end{bolded} When $f(x+f(x))=0 \forall x\in \mathbb{R}$.
$P(x+f(x),-x)\Rightarrow f(f(x))f(x)=xf(-x)$. But $f(f(x))f(x)=xf(x)$. So, $f(x)=f(-x)\forall x\in \mathbb{R}$.
$P(x,0)\Rightarrow f(0)f(x+f(x))=0=2xf(x)$. This yields us the solution:
$f(x)=0\forall x\neq 0$ and $f(0)=c\in \mathbb{R}$. In particular, if $c=0$, then it is the constant solution to the given equation.

\begin{bolded}Case 2:\end{bolded} When $f(x+f(x))\neq 0\forall x\neq 0$.
Then $f(0)=0$. The only constant solution belongs to the family of solution in case $1$. So let $f$ be non-constant.
$P(x,-x)\Rightarrow f(x)=-f(-x)\forall x\in \mathbb{R}$.
$P(0,y)\Rightarrow f(y)(f(y)-y)=0 \forall y\in \mathbb{R}.....(*)$
Now since $f$ is non-constant, there exists $d\neq 0$ such that $f(d)\neq 0$. Then from equation$(*)$, $f(d)=d$.
If there is a $k\neq 0$ such that $f(k)=0$, then,
$P(k,d-k)\Rightarrow df(k-d)=(d-k)f(k-d)\Rightarrow kf(k-d)=0\Rightarrow f(k-d)=0$. So,
$P(k-d,d)\Rightarrow 0=-df(d)\Rightarrow d^{2}=0\Rightarrow d=0$ which is a contradiction.
So, $r=0$ if and only if $f(r)=0$. Thus from $(*)$ and from $f(0)=0$, we have that $f(y)=y\forall y\in \mathbb{R}$.

\begin{bolded}Case 3:\end{bolded}  When $\exists a,b$ such that $f(a+f(a))=0$ and $f(b+f(b))\neq 0$, where both $a,b\neq 0$.
According to our previous results, $f(b+f(b))f(0)=0$. This means that $f(0)=0$. So,
$P(0,y)\Rightarrow f(y)[f(y)-y]=0$. Set $y=b+f(b)$, then we have $f(b+f(b))=b+f(b)$.
Let $d=b+f(b)$ and $k=a+f(a)$. Now unless $k=0$, we will have $d=0$, i.e. $f(b+f(b))=b+f(b)=0$ which contradicts our assumption. So, $k=f(a)+a=0$. Thus,
$P(a+f(a),-a)\Rightarrow f(a)=f(-a)$. Then $f(-a)=f(a)=-a$. So, 
$P(-a,a)\Rightarrow 0=-af(-a)-af(a)=2a^{2}\Rightarrow a=0$ which is also a contradiction. So no such $a,b$ exist.
Thus either $f(x+f(x))=0\forall x\in \mathbb{R}$, or $f(x+f(x))$ is non-zero for all non-zero $x$.

So finally the solutions to the given equations are:
$f(x)=0\forall x\neq 0$ and $f(0)=c\in \mathbb{R}$
or,
$f(y)=y\forall y\in \mathbb{R}$.

It is easy to check that these are indeed the solutions.
\end{solution}
*******************************************************************************
-------------------------------------------------------------------------------

\begin{problem}[Posted by \href{https://artofproblemsolving.com/community/user/25101}{radio}]
	Let $ A$ denote the set of nonnegative integers. Find all functions $ f: A\rightarrow A$ such that

$ 2f(m^2+n^2)=f(m)^2+f(n)^2$, for all $ m,n\in A$, and

$ f(m^2)\geq f(n^2)$, for all $ m,n\in A$ such that $ m\geq n$.
	\flushright \href{https://artofproblemsolving.com/community/c6h192868}{(Link to AoPS)}
\end{problem}



\begin{solution}[by \href{https://artofproblemsolving.com/community/user/29428}{pco}]
	\begin{tcolorbox}Let $ A$ denote the set of nonnegative integers. Find all functions $ f: A\rightarrow A$ such that

$ 2f(m^2 + n^2) = f(m)^2 + f(n)^2$, for all $ m,n\in A$, and

$ f(m^2)\geq f(n^2)$, for all $ m,n\in A$ such that $ m\geq n$.\end{tcolorbox}

[hide="My solution"]

1) \begin{bolded}the only two constant solutions are \end{bolded}\end{underlined}$ f(m)=0$ $ \forall m\in\mathbb N_0$ and $ f(m)=1$ $ \forall m\in\mathbb N_0$
Quite obvious

2) \begin{bolded}$ f(x)$ is non decreasing\end{bolded}\end{underlined}
$ a\geq b$ $ \implies$ $ f(a^2)\geq f(b^2)$ $ \implies$ $ f(a)^2+f(0)^2=2f(a^2+0^2)\geq 2f(b^2+0^2)=f(b)^2+f(0)^2$ $ \implies$ $ f(a)\geq f(b)$
Q.E.D

3) \begin{bolded}Non constant solutions are strictly increasing\end{bolded}\end{underlined}
Let $ a$ such that $ f(a+1)=f(a)$.
Then $ f(x^2+a^2)=f(x^2+(a+1)^2)$ and so $ f(n)$ is constant, since non decreasing, over $ [a^2+x^2,(a+1)^2+x^2]$ $ \forall x$.
So, for instance $ f(a^2+4)=f(a^2+5)$ and we can consider $ a\geq 3$ (else replace $ a$ with $ a^2+4$)

Then, if, for some $ k\leq a$, $ f(n)$ is constant over $ [a^2,a^2+k^2]$, and since we know that $ f(n)$ is constant over $ [a^2+k^2,(a+1)^2+k^2]$, we get that $ f(n)$ is constant over $ [a^2,(a+1)^2+k^2]$. And, since $ a^2+(k+1)^2\leq (a+1)^2+k^2$, we get that $ f(n)$ is constant over $ [a^2,a^2+(k+1)^2]$
So, a simple induction shows that $ f(n)$ is constant over $ [a^2,a^2+(a+1)^2]$.
Since $ a\geq 3$ $ \implies$ $ (a+2)^2\leq a^2+(a+1)^2$, we get $ f((a+2)^2)=f(a^2)$
And so $ f(a+2)=f(a)$ (using $ f(a+2)^2+f(0)^2=2f((a+2)^2+0^2)$ $ =2f(a^2+0^2)=f(a)^2+f(0)^2$)

So, $ f(a)=f(a+1)$ $ \implies$ $ f(b)=f(b+1)$ for some $ b\geq 3$ and then $ f(b)=f(b+2)$ and so $ f(n)$ is constant over $ [b,+\infty)$
Then, let $ m$ such that $ m^2\geq b$ : $ f(m^2+n^2)=f(m^2+0^2)$ and so $ f(n)=f(0)$ $ \forall n$

So, if $ f(n)$ is not a constant function, $ f(a+1)>f(a)$ $ \forall a$
Q.E.D

4) \begin{bolded}The unique non constant solution is $ f(n)=2n$\end{bolded}\end{underlined}
$ 2f(0^2+0^2)=2f(0)^2$ $ \implies$ $ f(0)=0$ or $ f(0)=1$
If $ f(0)=1$, $ 2f(1^2+0^2)=f(0)^2+f(1)^2$ $ \implies$ $ f(1)=1=f(0)$ so $ f(n)$ constant (see point 3 above)
If $ f(0)=0$, $ 2f(1^2+0^2)=f(0)^2+f(1)^2$ $ \implies$ $ f(1)=1=f(0)$ and $ f(n)$ is constant, or $ f(1)=2$

Let then $ f(0)=0$ and $ f(1)=2$ : 
$ 2f(m^2+1^2)=f(m)^2+4$ and so $ f(m)$ is even for all $ m$.

Using $ f(2m^2)=f(m)^2$, it's immediate to show with induction that $ f(u_n)=2u_n$ for $ u_n=2^{2^n-1}$

And, since $ f(n)$ is strictly increasing, even, and $ f(u)=2u$ for some $ u$ as great as we want, we obviously have $ f(n)=2n$ $ \forall n$

5) \begin{bolded}synthesis \end{bolded}\end{underlined}:
Solutions are :
$ f(n)=0$
$ f(n)=1$
$ f(n)=2n$

[\/hide]
\end{solution}



\begin{solution}[by \href{https://artofproblemsolving.com/community/user/62517}{mohamed-01-01}]
	for $ m = n = 0$  so $ f(0) = f^2(0)$ ==>  $ f(0) = 0$ or $ f(0) = 1$
   
        1)if $ f(0) = 1$  ==> $ 2f(1) = f^2(1) + f^2(0) = f^2(1) + 1$ so $ f(1) = 1$ 

 let Un suite such that $ U0 = 1$  and $ U_{n + 1} = 2(U_n)^2$

so $ 2f(U_{n + 1}) = 2f^2(U_n)$  BY RECURRENCE YOU CAN PROVE $ f(u_n) = 1$ for all n

now you must prove that f is increasing 

$ f^2(m + 1) + f^2(0) = f((m + 1)^2) \geq f(m^2) = f^2(m) + f^2(0)$ so $ f(m + 1) \geq f(m)$

for all $ x \in [U_n;U_{n + 1}] f(U_n)\geq f(x) \geq f(U_{n + 1}$ so $ f(x) = 1$==>$ f = 1$

      2)if $ f(0) = 0$ so 2f(1)=f²(1) so $ f(1) = 0or 2$

   2-a) if $ f(1) = 0$  ===> f(Un)=0 and f is increasing so f=0

   2-b) if$ f(1) = 2$   is easy to prove $ f(U_{n}) = 2U_n$ by recurrence

prove that f is strict increassing f²(m+1)=2f((m+1)²) $ \geq$ 2f(m²+1) because f is increassing so so $ f(m) = > m$

$ f^2(m + 1) \geq f(m^2) + f^2(1)$ ==>$ f(m + 1) > f(m)$
 
for to include f(n)=2nyou must to prove $ f(m + 1) \geq f(m) + 2$     <==>   

$ f(m + 1) > f(m) + 2$ (because f(m) is integer) 


$ f^2(m + 1) = 2f((m + 1))^2 + 0^2) = 2(f(m^2 + 2m + 1))$  you have f is strictly increasing 

so $ 2(f(m^2 + 2m + 1))\geq 2(f(m^2 + 1) + 2m) \geq f^2(m) + f^2(1) + 2f(m) = (f(m) + 1)^2 + 3 > (f(m) + 1)^2$

so $ f(m + 1) > f(m) + 1 so f(m + 1) \geq f(m) + 2$ ce que nous permet de conclure que $ f(n) = 2n$
\end{solution}
*******************************************************************************
-------------------------------------------------------------------------------

\begin{problem}[Posted by \href{https://artofproblemsolving.com/community/user/44055}{discredit}]
	Find all continuous functions $ f: \mathbb{R}\rightarrow\mathbb{R}$ such that
\[ f(x+f(y+f(z)))=f(x)+f(f(y))+f(f(f(z)))\]
for all reals $ x,y$, and $z$.
	\flushright \href{https://artofproblemsolving.com/community/c6h214066}{(Link to AoPS)}
\end{problem}



\begin{solution}[by \href{https://artofproblemsolving.com/community/user/29428}{pco}]
	\begin{tcolorbox}Find all continuous functions $ f: \mathbb{R}\rightarrow\mathbb{R}$ such that

$ f(x + f(y + f(z))) = f(x) + f(f(y)) + f(f(f(z)))$

for all reals $ x,y,z$\end{tcolorbox}

[hide="My solution"]
Let $ P(x,y,z)$ be the assertion $ f(x + f(y + f(z))) = f(x) + f(f(y)) + f(f(f(z)))$

$ f(x) = 0$ is obviously a solution. So we'll now look for non-zero solutions.

1) non-zero solutions are such that $ f(f(a))\neq 0$ for some real $ a$
Suppose $ f(f(x)) = 0$ $ \forall x$ then $ P(x,y,f(u))$ $ \implies$ $ f(x + f(y)) = f(x)$ $ \forall x,y$
So, we have a periodic continuous fonction, with period as little as we want, and so a constant, and so $ f(x) = 0$ $ \forall x$
Q.E.D.

2) non-zero solutions are surjective.
If $ f(f(f(0)))\neq 0$, then $ P(x,f(0),0)$ $ \implies$ $ f(x + u) = f(x) + v$, with $ u = f(2f(0))$ and $ v = 2f(f(f(0)))\neq 0$
If $ f(f(f(0))) = 0$, then, using 1) above, $ \exists a$ such that $ f(f(a))\neq 0$. Then $ P(x,a,0)$ $ \implies$ $ f(x + u) = f(x) + v$ with $ u = f(a + f(0))$ and $ v = f(f(a))\neq 0$

From $ f(x + u) = f(x) + v$ and $ v\neq 0$, we get $ f(nu) = f(0) + nv$ $ \forall n\in \mathbb Z$ and so, since $ f(x)$ is continuous, $ f(x)$ is surjective.
Q.E.D.

3) The only non-zero solutions are $ f(x) = ux$ and $ f(x) = - 2x + b$
Let $ b = f(0)$
Since $ f(x)$ is surjective, the initial assertion may be written $ f(x + f(y + z)) = f(x) + f(f(y)) + f(f(z))$
Setting $ x = 0$, we get $ f(f(y + z)) = f(0) + f(f(y)) + f(f(z))$ and so, since $ f(f(x))$ is continuous, a classical Cauchy equation and : $ f(f(x)) = ax - b$
So $ f(b) = - b$

Putting this in $ f(x + f(y + z)) = f(x) + f(f(y)) + f(f(z))$, we get $ f(x + f(y + z)) = f(x) + a(y + z) - 2b$ and so :
$ Q(x,y)$ : $ f(x + f(y)) = f(x) + ay - 2b$
Notice here that $ a\neq 0$. Else $ f(x + f(y)) = f(x) - 2b$ and so (taking $ y = f^{[ - 1]}(0)$) $ b = 0$ and $ f(f(x)) = 0$ $ \forall x$ and so (using point 1) $ f(x) = 0$

$ Q(ax,b)$ $ \implies$ $ f(ax - b) = f(ax) + ab - 2b$
$ Q(0,f(x))$ $ \implies$ $ f(ax - b) = af(x) - b$
Equating the two RHS implies $ af(x) = f(ax) + (a - 1)b$

$ Q(ax,f(y))$ $ \implies$ $ f(ax + ay - b) = f(ax) + af(y) - 2b$ $ = f(ax) + f(ay) + (a - 3)b$
$ Q(ax + ay,b)$ $ \implies$ $ f(ax + ay - b)) = f(ax + by) + (a - 2)b$
Equating the two RHS implies $ f(ax + ay) = f(ax) + f(ay) - b$ and so, since $ a\neq 0$, $ g(x + y) = g(x) + g(y)$ with $ g(x) = f(x) - b$
Hence $ f(x) = ux + b$

Pluging this back in $ P(x,y,z)$, we get $ (u + 2)b = 0$
Q.E.D.

So the solutions are :
=====================
$ f(x) = 0$
$ f(x) = ux$
$ f(x) = - 2x + b$

[\/hide]
\end{solution}
*******************************************************************************
-------------------------------------------------------------------------------

\begin{problem}[Posted by \href{https://artofproblemsolving.com/community/user/44055}{discredit}]
	Prove that there exist infinitely many functions $ f$ such that the domain is the set of natural numbers and
\[f(nf(k)+kf(n))=f(n^2+k^2)f(n+k-1)\]
for any two natural numbers $ k$ and $n$.
	\flushright \href{https://artofproblemsolving.com/community/c6h214232}{(Link to AoPS)}
\end{problem}



\begin{solution}[by \href{https://artofproblemsolving.com/community/user/29428}{pco}]
	\begin{tcolorbox}Prove that there exist infinitely many functions $ f$ such that the domain is the set of natural numbers and

$ f(nf(k) + kf(n)) = f(n^2 + k^2)f(n + k - 1)$

for any natural numbers $ k,n$.\end{tcolorbox}
[hide="My solution"]
Let $ f_a(x)$, for any $ a\in\mathbb N$ defined as :
$ f_a(x)=1$ for all odd $ x$
$ f_a(x)=2a$ for all even $ x$

$ f_a(x)$ is a solution ($ nf(k)+kf(n)$ is always even, so LHS is $ 2a$ and $ n^2+k^2$ and $ n+k-1$ are either even - odd, either odd - even, so RHS is always $ 2a$ too).

And there are infinitely many different $ f_a$, hence the result.

[\/hide]
\end{solution}
*******************************************************************************
-------------------------------------------------------------------------------

\begin{problem}[Posted by \href{https://artofproblemsolving.com/community/user/33274}{toanIneq}]
	Find all continuous functions $f: \mathbb R \to \mathbb R$ satisfying the equality
\[ f(x+f(y+f(z)))=f(x)+f(f(y))+f(f(f(z)))\] for all real $ x,y,z$.
	\flushright \href{https://artofproblemsolving.com/community/c6h219275}{(Link to AoPS)}
\end{problem}



\begin{solution}[by \href{https://artofproblemsolving.com/community/user/29428}{pco}]
	\begin{tcolorbox}Find all continuous functions $ f: R\rightarrow R$ satisfying the equality:
$ f(x + f(y + f(z))) = f(x) + f(f(y)) + f(f(f(z)))$ for all real $ x,y,z$\end{tcolorbox}

1) Case 1 : $ f(f(y))+f(f(f(z)))=0$ $ \forall y,z$
So we have $ f(x+f(y+f(z)))=f(x)$ $ \forall y,z$

1.1) $ f(x)=c$ constant.
Then $ f(f(y))+f(f(f(z)))=2c=0$ and we have $ f(x)=0$, which is actually a solution

1.2) $ f(x)$ is not a constant
Then consider $ a<b$ such that $ [a,b]\subset f(\mathbb R)$ : $ f(y+f(z))$ may take any value in $ [a,b]$ and so $ f(x+c)=f(x)$ $ \forall x,\forall c\in[a,b]$ and so $ f(x)=$constant, hence a contradiction.

2)$ \exists y,z$ such that $ f(f(y))+f(f(f(z)))=v\neq 0$
Let then $ u=f(y+f(z))$
We have $ f(x+u)=f(x)+v$ and so $ f(x+ku)=f(x)+kv$ $ \forall k\in\mathbb Z$ and so $ f(\mathbb R)=\mathbb R$ ($ f(x)$ is a surjective function).

Then, for any real $ a$, there exists at least one real $ y$ such that $ a=f(y+f(0))$ and we can write :
$ f(x+a)=f(x)+f(f(y))+f(f(f(0)))$
$ f(0+a)=f(0)+f(f(y))+f(f(f(0)))$

And, subtracting these two lines : $ f(x+a)-f(a)=f(x)-f(0)$

And so $ f(x+y)=f(x)+f(y)-f(0)$ $ \forall x,y$

And, since $ f(x)$ is continuous, the only solutions of this Cauchy-like equation are $ f(x)=ax+b$

Putting this back in the original equation, we get $ (a+2)b=0$

And so the solutions :

$ f(x)=ax$
$ f(x)=b-2x$
\end{solution}
*******************************************************************************
-------------------------------------------------------------------------------

\begin{problem}[Posted by \href{https://artofproblemsolving.com/community/user/1991}{orl}]
	Let $ a, b \in \mathbb{N}$  with $ 1 \leq a \leq b,$ and $ M = \left[\frac {a + b}{2} \right].$ Define a function $ f: \mathbb{Z} \mapsto \mathbb{Z}$ by
\[ f(n) = \begin{cases} n + a, & \text{if } n \leq M, \\
n - b, & \text{if } n >M. \end{cases}
\]
Let $ f^1(n) = f(n),$ $ f_{i + 1}(n) = f(f^i(n)),$ $ i = 1, 2, \ldots$ Find the smallest natural number $ k$ such that $ f^k(0) = 0.$
	\flushright \href{https://artofproblemsolving.com/community/c6h220954}{(Link to AoPS)}
\end{problem}



\begin{solution}[by \href{https://artofproblemsolving.com/community/user/29428}{pco}]
	\begin{tcolorbox}Let $ a, b \in \mathbb{N}$  with $ 1 \leq a \leq b,$ and $ M = \left[\frac {a + b}{2} \right].$ Define a function $ f: \mathbb{Z} \mapsto \mathbb{Z}$ by
\[ f(n) = \begin{cases} n + a, & \text{if } n \leq M, \\
n - b, & \text{if } n \geq M. \end{cases}
\]
Let $ f^1(n) = f(n),$ $ f_{i + 1}(n) = f(f^i(n)),$ $ i = 1, 2, \ldots$ Find the smallest natural number $ k$ such that $ f^k(0) = 0.$\end{tcolorbox}

I suppose we must read "$ n - b$, if $ n > M$" (instead of "$ n\geq M$")

[hide="My solution"]

If $ k(x)$ is the least cycle for $ x$, we have $ f^{k(x)}(x) = x$, then we have $ p(x)a - q(x)b = 0$ where $ p(x)$ is the number of "$ + a$" moves and $ q(x)$ the number of "$ - b$" moves for the least cycle (and so $ p(x) + q(x) = k(x)$)

And so the least values possible for $ p(x)$ and $ q(x)$ are $ p(x) = \frac {b}{\gcd(a,b)}$ and $ q(x) = \frac {a}{\gcd(a,b)}$

So $ k(x)\geq \frac {a + b}{\gcd(a,b)}$

Now, if $ x = 0$, $ f^i(0)$ can only take values in $ ([\frac {a + b}{2}] - b,[\frac {a + b}{2}] + a]$, all multiples of $ \gcd(a,b)$ (since the starting value $ 0$ is such a multiple), so $ \frac {a + b}{\gcd(a,b)}$ values.

So a cycle is encountered in the first $ \frac {a + b}{\gcd(a,b)}$ moves. As a consequence, this cycle includes $ 0$ (else we would have a cycle whose length would be $ < \frac {a + b}{\gcd(a,b)}$

And so the result : $ k(0) = \frac {a + b}{\gcd(a,b)}$

(Remark : with some other departure points, the cycle may not involve the departure point and so $ k(x) > \frac {a + b}{\gcd(a,b)}$)

[\/hide]
\end{solution}
*******************************************************************************
-------------------------------------------------------------------------------

\begin{problem}[Posted by \href{https://artofproblemsolving.com/community/user/44209}{Vova-LFML}]
	Find all functions $f: \mathbb Z \to \mathbb Z$  such that \[f(x + y + f(y))=f(x) + 2y ,\] for all integers $x$ and $y$.
	\flushright \href{https://artofproblemsolving.com/community/c6h226387}{(Link to AoPS)}
\end{problem}



\begin{solution}[by \href{https://artofproblemsolving.com/community/user/29428}{pco}]
	\begin{tcolorbox}Find all f (x + y + f(y))=f(x) + 2y , f : Z:Z, x,y : Z\end{tcolorbox}

$ f(x+y+f(y))=f(x)+2y$ $ \implies$ $ Q(x,y,p)$ : $ f(x+p(y+f(y))=f(x)+2py$ $ \forall x, y, p\in\mathbb Z$

$ Q(0,x,1+f(1))$ : $ f((1+f(1))(x+f(x))=f(0)+2(1+f(1))x$
$ Q(0,1,x+f(x))$ : $ f((x+f(x))(1+f(1))=f(0)+2(x+f(x))$

And so $ 2(1+f(1))x=2(x+f(x))$

And so $ f(x)=f(1)x$

Putting back this value in the original equation, we get $ f(1)(f(1)+1)x=2x$ and so $ f(1)=1$ or $ f(1)=-2$

Hence the two solutions :
$ f(x)=x$
$ f(x)=-2x$
\end{solution}
*******************************************************************************
-------------------------------------------------------------------------------

\begin{problem}[Posted by \href{https://artofproblemsolving.com/community/user/22227}{V.V.}]
	Find all functions $ f: \mathbb{R}\to\mathbb{R}$ that 

$ y(x+y)f(x)\sin{x}+x(x+y)f(y)\sin{y}=xyf(-x-y)\sin{(x+y)}$ 

for all $ x$, $ y\in\mathbb{R}$.
	\flushright \href{https://artofproblemsolving.com/community/c6h228323}{(Link to AoPS)}
\end{problem}



\begin{solution}[by \href{https://artofproblemsolving.com/community/user/29428}{pco}]
	\begin{tcolorbox}Find all functions $ f: \mathbb{R}\to\mathbb{R}$ that 

$ y(x + y)f(x)\sin{x} + x(x + y)f(y)\sin{y} = xyf( - x - y)\sin{(x + y)}$ 

for all $ x$, $ y\in\mathbb{R}$.\end{tcolorbox}

Strange problem :)

[hide="My solution"]

Let $ x,y$ such that $ xy(x + y)\neq 0$. Then dividing the equation by $ xy(x + y)$, we get $ g(x) + g(y) = g( - x - y)$ with $ g(x) = \frac {f(x)\sin(x)}{x}$ 

Let then $ a\neq 0$ : $ (a)( - \frac {a}{2})(\frac {a}{2})\neq 0$ $ \implies$ $ g(a) + g( - \frac a2)$ ${ = g( -a + \frac a2})$ $ = g( - \frac a2)$ and so $ g(a)=0$

So $ \forall x\neq 0$, $ g(x) = 0$ $ \implies$ $ \forall x\neq 0$ : $ \frac {f(x)\sin(x)}{x} = 0$ $ \implies$ $ \forall x\neq 0$ : $ f(x)\sin(x) = 0$ $ \implies$ $ \forall x\neq k\pi$ : $ f(x) = 0$

And it's easy to see that $ f(k\pi)$ may have any value.

So the solutions are :
Let $ h(x)$ any function from $ \mathbb Z\to\mathbb R$ :

$ f(k\pi) = h(k)$ $ \forall k\in\mathbb Z$
$ f(x) = 0$ $ \forall x\neq k\pi$

And it's easy to check that this solution indeed fits the initial requirements.

[\/hide]
\end{solution}
*******************************************************************************
-------------------------------------------------------------------------------

\begin{problem}[Posted by \href{https://artofproblemsolving.com/community/user/48949}{MathVietNam}]
	1. Find all function $f: \mathbb R \to \mathbb R$, satisfying $f(x+f(y))=f(x)+ \sin y$ for all real numbers $x$ and $y$.

2. Find all function $f: \mathbb N \to \mathbb N$ such that $f(f(f(n))) +f(f(n)) +f(n) = 3n$ for all $n \in \mathbb N$.
	\flushright \href{https://artofproblemsolving.com/community/c6h230044}{(Link to AoPS)}
\end{problem}



\begin{solution}[by \href{https://artofproblemsolving.com/community/user/29428}{pco}]
	\begin{tcolorbox}1.Find all function f: R into R, satisfying: f(x+f(y))=f(x)+ Siny, for all real numberx,y\end{tcolorbox}

No such function exists :

Let $ P(x,y)$ the assertion $ f(x+f(y))=f(x)+ \sin y$

1) $ f(\mathbb R)=\mathbb R$
If $ a\in f(\mathbb R)$, $ \exists b$ such that $ a=f(b)$. Then  $ P(b,y)$ $ \implies$ $ a+\sin y\in f(\mathbb R)$, and so $ [a-1,a+1]\subset f(\mathbb R)$
An immediate induction shows that $ f(\mathbb R)=\mathbb R$ and $ f(x)$ is a surjection.

2) $ f(\mathbb R)\neq\mathbb R$
Let $ a\in\mathbb R$. Since $ f(x)$ is a surjection, $ \exists b$ such that $ f(b)=a$
Then $ P(0,b)$ $ \implies$ $ f(a)=f(0)+ \sin b$ $ \in[f(0)-1,f(0)+1]$
So $ f(\mathbb R)\subseteq[f(0)-1,f(0)+1]$

Since 1. and 2. are in contradiction, no such function exists.
\end{solution}



\begin{solution}[by \href{https://artofproblemsolving.com/community/user/38215}{bboypa}]
	\begin{tcolorbox}2.Find all function f :N into N ,Such that: f(f(f(n))) +f(f(n)) +f(n) = 3n\end{tcolorbox}

$ f(n)$ is injective, so if $ n_0$ is the least non fixed point we would have lhs>rhs  

ps. beautiful part 1 on the image of f! :D
\end{solution}



\begin{solution}[by \href{https://artofproblemsolving.com/community/user/61870}{replay}]
	For n = 1, we easily get f(1) = 1. Suppose that, for n < k, we have f(n) =n. We prove that f(k) = k. if p = f(k) < k then by the induction  
hypothesis f(p) = p = f(k), and this contradicts the injectivity of f. If f(k) > k, then f[f(k)] >= k. 
If we had f(f(k)) < k, then, as before, we would get the contradiction  
Similarly, we have f{f[f(k)]} > k. Hence, f{f[f(k)]} + f[f(k)] + f(k) > 3k, which contradicts the original condition. Thus f(k) = k.
\end{solution}
*******************************************************************************
-------------------------------------------------------------------------------

\begin{problem}[Posted by \href{https://artofproblemsolving.com/community/user/43824}{jagdish}]
	Let $ f: \mathbb R\longrightarrow \mathbb R$ be a continuous function such that
i) $ f(2010)=2009$, and
ii) $ f(x)\cdot f^{4}(x)=1$, $ \forall x\in \mathbb R$.

Find $ f(2008)$. 

Notice that in this problem, we denote by $f^4(x)$ the composition of $f(x)$ with itself four times. That is, $f^4(x)=f(f(f(f(x))))$.
	\flushright \href{https://artofproblemsolving.com/community/c6h239200}{(Link to AoPS)}
\end{problem}



\begin{solution}[by \href{https://artofproblemsolving.com/community/user/29428}{pco}]
	If we add, as mathVNpro suggested, the condition "$ f(x)$ continuous", then :

We have $ f(f(f(x)))=\frac{1}{x}$ $ \forall x\in A=f(\mathbb R)$ and so we also have $ f(\frac{1}{x})=\frac{1}{f(x)}$ $ \forall x\in A$

Notice that $ 0\notin A$ and $ 2009\in A$, so $ A\subseteq\mathbb R^{+*}$

So $ f(x)$ is an injective function from A$ \to$ A, so monotonous, so decreasing (else $ f(f(f(x)))$ would be increasing).

Let $ a\in A$
Suppose $ f(a)>\frac{1}{a}$ : Then $ \frac{1}{a}=f(f(f(a)))>f(f(\frac{1}{a}))$ (since $ f(f(x))$ is strictly increasing). So  $ f(a)>\frac{1}{a}>f(f(\frac{1}{a}))$  and so $ f(a)>f(f(\frac{1}{a}))$.

But $ f(a)>f(f(\frac{1}{a}))$ implies $ a<f(\frac{1}{a})=\frac{1}{f(a)}$ (since $ f(x)$ is strictly decreasing) and so $ f(a)<\frac{1}{a}$
Similarly $ f(a)<\frac{1}{a}$ $ \implies$ $ f(a)>\frac{1}{a}$

So $ f(x)=\frac{1}{x}$ $ \forall x\in A$

Since $ 2009\in A$, $ \frac{1}{2009}=f(f(f(2009)))\in A$, so $ [\frac{1}{2009},2009]\subseteq A$, and so $ 2008\in A$

So $ f(2008)=\frac{1}{2008}$

==========================
Notice that such a function exists :

$ f(x)=2009$ $ \forall x< \frac{1}{2009}$

$ f(x)=\frac{1}{x}$ $ \forall x\in[\frac{1}{2009},2009]$

$ f(x)=(2009-\frac{1}{2009})(x-2010)+2009$ $ \forall x\in(2009,2010)$

$ f(x)=2009$ $ \forall x\geq 2010$
\end{solution}
*******************************************************************************
-------------------------------------------------------------------------------

\begin{problem}[Posted by \href{https://artofproblemsolving.com/community/user/39313}{rahulakaneo}]
	Find a function $f: \mathbb N \to \mathbb N$ such that \[f(f(x))=1993\cdot x^{1945}\] for all $x \in \mathbb N$.
	\flushright \href{https://artofproblemsolving.com/community/c6h252203}{(Link to AoPS)}
\end{problem}



\begin{solution}[by \href{https://artofproblemsolving.com/community/user/29428}{pco}]
	\begin{tcolorbox}Find a function f(x) such that f(f(x))=1993*(x^1945) for all x in N(The Set Of Natural Numbers) ?\end{tcolorbox}

This is a classical problem with classical solution :

[hide="My solution"]
Let $ u(x)=1993x^{1945}$.
Let $ A=u(\mathbb N)$
Let $ B=\mathbb N\backslash A$

$ u(x)$ is injective and $ u(x)>x$ $ \forall x>0$
So, for any $ x>0$, there is a unique couple of integer $ n(x)\geq 0$ and $ b(x)>0$ such that $ b(x)\in B$ and $ x=u^{[n(x)]}(b(x))$ where $ u^{[m]}$ is the composition $ u\circ u\circ u ... \circ u$ $ m$ times (with $ u^{[0]}(x)=x$ and $ u^{[1]}(x)=u(x)$)

Cardinal($ B$) is infinite and so we can split $ B$ in two infinite equipotent sets $ B_1$ and $ B_2$ :
$ B_1\cup B_2=B$
$ B_1\cap B_2=\emptyset$
$ \exists$ a bijection  $ h(x): B_1\to B_2$

Then we can easily define $ f(x)$ as :

If $ b(x)\in B_1$ : $ f(x)=u^{[n(x)]}(h(b(x)))$

If $ b(x)\in B_2$ : $ f(x)=u^{[n(x)+1]}(h^{[-1]}(b(x)))$
[\/hide]
\end{solution}
*******************************************************************************
-------------------------------------------------------------------------------

\begin{problem}[Posted by \href{https://artofproblemsolving.com/community/user/32134}{bambaman}]
	Is it possible to find all solutions to the functional equation $ f(x + y) + f(x - y) = 2f(x)f(y)$? Is there a non-constant solution other than $ \cos(x)$?
	\flushright \href{https://artofproblemsolving.com/community/c6h256322}{(Link to AoPS)}
\end{problem}



\begin{solution}[by \href{https://artofproblemsolving.com/community/user/49556}{xxp2000}]
	Let $ \theta(x)$ satisfy Cauchy equation, i.e.
$ \theta(x+y)=\theta(x)+\theta(y)$.

Then you have at least two way to define $ f$:
1) $ f(x)=\cos(\theta(x))$

2) $ f(x)=\frac{e^{\theta(x)}+e^{-\theta(x)}}2$.

Hence, there are infinitely many non-constant solutions.
\end{solution}



\begin{solution}[by \href{https://artofproblemsolving.com/community/user/32134}{bambaman}]
	Cool, thanks. is there a way to show that these are the only 2 ways to define $ f$? (they are equivalent if we take Cauchy equation in C, but I am still interested in real $ f$). I am trying some substitutions by I can't come up with something good.
I want to show that if $ f$ is bounded, then we can deduce that $ f$ satisfies the first definition (if it is bounded it clearly can't satisfy the second definition).
\end{solution}



\begin{solution}[by \href{https://artofproblemsolving.com/community/user/14836}{zhubin846152}]
	f(x) is an even function, and f(0)=1;

if f(x) is a 2-derivable function.

f"(x) =( f(x+h)-2f(x)+f(x-h) )\/ 2h^2 = ( f(x)(f(h)-1) ) \/ h^2 = f''(0)f(x) 

y" = cy
Then we can solve the problem via the differetial equation above.
\end{solution}



\begin{solution}[by \href{https://artofproblemsolving.com/community/user/3182}{Kunihiko_Chikaya}]
	The problem doesn't say "differentiable"". 
\end{solution}



\begin{solution}[by \href{https://artofproblemsolving.com/community/user/40503}{Flakky}]
	This equation is very popular. It's called \begin{bolded}D'Alembert functional equation.\end{bolded}
\end{solution}



\begin{solution}[by \href{https://artofproblemsolving.com/community/user/62475}{hqthao}]
	i think zhubin846152 is wrong because we can't derivative a function equation when we couldn't know it can or cannot be derivative
\end{solution}



\begin{solution}[by \href{https://artofproblemsolving.com/community/user/29428}{pco}]
	\begin{tcolorbox}Cool, thanks. is there a way to show that these are the only 2 ways to define $ f$? (they are equivalent if we take Cauchy equation in C, but I am still interested in real $ f$). I am trying some substitutions by I can't come up with something good.
I want to show that if $ f$ is bounded, then we can deduce that $ f$ satisfies the first definition (if it is bounded it clearly can't satisfy the second definition).\end{tcolorbox}

It's rather easy to show these are the only two families of \begin{bolded}continuous \end{bolded}\end{underlined}solutions (see [url]http://www.mathlinks.ro/Forum/viewtopic.php?p=1490388&search_id=1697160437#1490388[\/url]
\end{solution}



\begin{solution}[by \href{https://artofproblemsolving.com/community/user/46171}{tuandokim}]
	oh,I have the same problem about this :D
how many f(n): Z to Z
satisfied these conditions????
$ 1,|f(n)\le 2009|$(or k...) for every $ n \in Z$
$ 2,f(m + n) + f(m - n) = 2f(m)f(n)$ for every $ m,n \in Z$
\end{solution}



\begin{solution}[by \href{https://artofproblemsolving.com/community/user/29428}{pco}]
	\begin{tcolorbox}oh,I have the same problem about this :D
how many f(n): Z to Z
satisfied these conditions????
$ 1,|f(n)\le 2009|$(or k...) for every $ n \in Z$
$ 2,f(m + n) + f(m - n) = 2f(m)f(n)$ for every $ m,n \in Z$\end{tcolorbox}
This one is quite simple :

$ n=0$ $ \implies$ $ f(m)=f(m)f(0)$ and so :
either $ f(0)=0$ and we have a first solution $ f(n)=0$ $ \forall n$
either $ f(0)=1$. Then :

$ m=0$ $ \implies$ $ f(n)+f(-n)=2f(n)$ and so $ f(-n)=f(n)$
Since $ |f(n)|\le k$, $ \exists a$ such that $ k\geq f(a)\geq f(n)$ $ \forall n$ and we have $ f(a)\geq f(0)=1$

Then $ f(n+a)+f(n-a)=2f(a)f(n)$ $ \implies$ $ f(n)\leq 1$ else $ f(n+a)+f(n-a)>2f(a)$, which is impossible since $ f(n+a)\leq f(a)$ and $ f(n-a)\leq f(a)$
So $ f(n)\leq 1$ $ \forall n$

Now, if $ \exists n$ such that $ f(n)<-1$, then $ f(n+n)+f(0)=2f(n)^2>2$ and so $ f(2n)>1$ which is impossible.

So $ f(n)\in\{-1,0,+1\}$ $ \forall n$

Then :
1) $ f(1)=-1$ $ \implies$, using $ f(n+1)=-2f(n)-f(n-1)$ the sequence is, for $ n\geq 0$ : $ 1,-1,+1,-1,+1, ...$
2) $ f(1)=0$ $ \implies$, using $ f(n+1)=-f(n-1)$ the sequence is, for $ n\geq 0$ : $ 1,0,-1,0,1, 0, -1, ...$
3) $ f(1)=1$ $ \implies$, using $ f(n+1)=2f(n)-f(n-1)$ the sequence is, for $ n\geq 0$ : $ 1, 1, 1, 1, ...$

And we can easily check that these 3 sequences indeed fit the initial requirements.

Hence the answer :
If $ k=0$ only one sequence exists : $ 0, 0, 0, 0, 0, 0, ...$
If $ k>0$ four sequences exist :
$ 0, 0, 0, 0, 0, 0, ...$
$ 1,-1,+1,-1,+1, ...$
$ 1,0,-1,0,1, 0, -1, ...$
$ 1, 1, 1, 1, ...$
\end{solution}



\begin{solution}[by \href{https://artofproblemsolving.com/community/user/46171}{tuandokim}]
	nice :D
but if I change the condition:
f(n) from Z to R
can you solve it too?
\end{solution}



\begin{solution}[by \href{https://artofproblemsolving.com/community/user/29428}{pco}]
	\begin{tcolorbox}nice :D
but if I change the condition:
f(n) from Z to R
can you solve it too?\end{tcolorbox}

No, and nobody can since we know that $ f(n)=\cos(an)$ is solution, it's clear that there are infinitely many solutions and that we cant answer to the question  "how many f ...".
\end{solution}



\begin{solution}[by \href{https://artofproblemsolving.com/community/user/29428}{pco}]
	\begin{tcolorbox}[quote="tuandokim"]nice :D
but if I change the condition:
f(n) from Z to R
can you solve it too?\end{tcolorbox}

No, and nobody can since we know that $ f(n) = \cos(an)$ is solution, it's clear that there are infinitely many solutions and that we cant answer to the question  "how many f ...".\end{tcolorbox}

Btw, if the question is "find all functions $ f(n)$ from $ \mathbb Z\to\mathbb R$ such that $ |f(n)|\leq k$ $ \forall n$ and $ f(m + n) + f(m - n) = 2f(m)f(n)$ $ \forall m,n$", then :

$ f(n) = 0$ is solution
If $ k < 1$, this is the only solution (since $ f(0) = 0$ and $ f(n) = 0$ $ \forall n$ or $ f(0) = 1 > k$).
If $ k\geq 1$, then the same method as above implies $ f(n)\in[ - 1,1]$ $ \forall n$

Then $ \exists u$ such that $ f(1) = \cos u$
Using then $ f(2) = 2f(1)^2 - 1$ we get $ f(2) = \cos (2u)$

Then, a quick induction allow us to write $ f(n) = \cos(nu)$ $ \forall n$ :
1) it's true $ \forall n\in[0,2^1]$
2) suppose it's true $ \forall n\in[0,2^p]$, then 
Let $ n\in[2^p,2^{p + 1}]$ : $ n = 2^p + m$ with $ m\in[0,2^p]$ and we have :

$ f(2^p + m) + f(2^p - m) = 2f(2^p)f(m)$

Since $ 2^p\in[0,2^p]$, we get $ f(2^p) = \cos(2^pu)$
Since $ m\in[0,2^p]$, we get $ f(m) = \cos(mu)$
Since $ 2^p - m\in[0,2^p]$, we get $ f(2^p - m) = \cos((2^p - m)u)$

And so $ f(n) + \cos((2^p - m)u) = 2\cos(2^pu)\cos(mu)$

And so $ f(n) = 2\cos(2^pu)\cos(mu) - \cos((2^p - m)u)$ $ = \cos(2^pu + mu) = \cos(nu)$

And, since $ f( - n) = f(n)$, we get the result :

If $ k < 1$, the only solution is $ f(n) = 0$ $ \forall n$

If $ k\geq 1$, then the solutions are :
$ fn) = 0$
$ f(n) = \cos(nu)$ for any real $ u$.

==============================
In fact it's easy to show that all solutions to $ f(m + n) + f(m - n) = 2f(m)f(n)$ when $ f(n)$ is from $ \mathbb Z\to\mathbb R$ are :
$ f(n) = 0$
$ f(n) = \cos(\alpha n)$
$ f(n) = \cosh(\alpha n)$
$ f(n)=(-1)^n\cosh(\alpha n)$
\end{solution}



\begin{solution}[by \href{https://artofproblemsolving.com/community/user/46171}{tuandokim}]
	\begin{tcolorbox}\begin{tcolorbox}[quote="tuandokim"]nice :D
but if I change the condition:
f(n) from Z to R
can you solve it too?\end{tcolorbox}

No, and nobody can since we know that $ f(n) = \cos(an)$ is solution, it's clear that there are infinitely many solutions and that we cant answer to the question  "how many f ...".\end{tcolorbox}

Btw, if the question is "find all functions $ f(n)$ from $ \mathbb Z\to\mathbb R$ such that $ |f(n)|\leq k$ $ \forall n$ and $ f(m + n) + f(m - n) = 2f(m)f(n)$ $ \forall m,n$", then :

$ f(n) = 0$ is solution
If $ k < 1$, this is the only solution (since $ f(0) = 0$ and $ f(n) = 0$ $ \forall n$ or $ f(0) = 1 > k$).
If $ k\geq 1$, then the same method as above implies $ f(n)\in[ - 1,1]$ $ \forall n$

Then $ \exists u$ such that $ f(1) = \cos u$
Using then $ f(2) = 2f(1)^2 - 1$ we get $ f(2) = \cos (2u)$

Then, a quick induction allow us to write $ f(n) = \cos(nu)$ $ \forall n$ :
1) it's true $ \forall n\in[0,2^1]$
2) suppose it's true $ \forall n\in[0,2^p]$, then 
Let $ n\in[2^p,2^{p + 1}]$ : $ n = 2^p + m$ with $ m\in[0,2^p]$ and we have :

$ f(2^p + m) + f(2^p - m) = 2f(2^p)f(m)$

Since $ 2^p\in[0,2^p]$, we get $ f(2^p) = \cos(2^pu)$
Since $ m\in[0,2^p]$, we get $ f(m) = \cos(mu)$
Since $ 2^p - m\in[0,2^p]$, we get $ f(2^p - m) = \cos((2^p - m)u)$

And so $ f(n) + \cos((2^p - m)u) = 2\cos(2^pu)\cos(mu)$

And so $ f(n) = 2\cos(2^pu)\cos(mu) - \cos((2^p - m)u)$ $ = \cos(2^pu + mu) = \cos(nu)$

And, since $ f( - n) = f(n)$, we get the result :

If $ k < 1$, the only solution is $ f(n) = 0$ $ \forall n$

If $ k\geq 1$, then the solutions are :
$ fn) = 0$
$ f(n) = \cos(nu)$ for any real $ u$.

==============================
In fact it's easy to show that all solutions to $ f(m + n) + f(m - n) = 2f(m)f(n)$ when $ f(n)$ is from $ \mathbb Z\to\mathbb R$ are :
$ f(n) = 0$
$ f(n) = \cos(\alpha n)$
$ f(n) = \cosh(\alpha n)$
$ f(n) = ( - 1)^n\cosh(\alpha n)$\end{tcolorbox}
thank you for a nice solution:D
\end{solution}
*******************************************************************************
-------------------------------------------------------------------------------

\begin{problem}[Posted by \href{https://artofproblemsolving.com/community/user/45762}{FelixD}]
	Find all monotonous functions $ f: \mathbb{R} \to \mathbb{R}$ that satisfy the following functional equation:
\[f(f(x)) = f( - f(x)) = f(x)^2.\]
	\flushright \href{https://artofproblemsolving.com/community/c6h257007}{(Link to AoPS)}
\end{problem}



\begin{solution}[by \href{https://artofproblemsolving.com/community/user/29428}{pco}]
	\begin{tcolorbox}Find all monotonous functions $ f: \mathbb{R} \to \mathbb{R}$ that satisfy the following functional equation:
$ f(f(x)) = f( - f(x)) = f(x)^2.$\end{tcolorbox}

This implies $ f(x)=f(-x)$ $ \forall x\in f(\mathbb R)$. So, since $ f(x)$ is monotonous, $ f(x)=c$ $ \forall x\in f(\mathbb R)$ and, since $ f(x)=x^2$ $ \forall x\in f(\mathbb R)$, we get $ c=c^2$

And the only solutions are :
$ f(x)=0$
$ f(x)=1$
\end{solution}



\begin{solution}[by \href{https://artofproblemsolving.com/community/user/34607}{keyree10}]
	\begin{tcolorbox}This implies $ f(x) = f( - x) \forall x\in f(\mathbb R)$\end{tcolorbox}

Why is this true?
\end{solution}



\begin{solution}[by \href{https://artofproblemsolving.com/community/user/54874}{lizhi}]
	put x=f(x),we get f(x^2)=f^2(x)=f(-x^2) 
thus f(x)=f(-x), $ \forall x\in \mathbb{R}$
but f(x) is monotonous,so we get f(x)=c,c=0 or 1
\end{solution}



\begin{solution}[by \href{https://artofproblemsolving.com/community/user/29428}{pco}]
	\begin{tcolorbox}[quote="pco"]This implies $ f(x) = f( - x) \forall x\in f(\mathbb R)$\end{tcolorbox}

Why is this true?\end{tcolorbox}

Hello keyree10!

If $ x\in f(\mathbb R)$, $ \exists y$ such that $ x=f(y)$ and so $ f(f(y))=f(x)$ and $ f(-f(y))=f(-x)$ and, since $ f(f(y))=f(-f(y))$, we get $ f(x)=-f(x)$ $ \forall x\in f(\mathbb R)$
\end{solution}



\begin{solution}[by \href{https://artofproblemsolving.com/community/user/10035}{Altheman}]
	\begin{tcolorbox}[quote="FelixD"]Find all monotonous functions $ f: \mathbb{R} \to \mathbb{R}$ that satisfy the following functional equation:
$ f(f(x)) = f( - f(x)) = f(x)^2.$\end{tcolorbox}

This implies $ f(x) = f( - x)$ $ \forall x\in f(\mathbb R)$. So, since $ f(x)$ is monotonous, $ f(x) = c$ $ \forall x\in f(\mathbb R)$ and, since $ f(x) = x^2$ $ \forall x\in f(\mathbb R)$, we get $ c = c^2$

And the only solutions are :
$ f(x) = 0$
$ f(x) = 1$\end{tcolorbox}

It seems to me that all you proved was that $ f(x)\in \{0,1\}$ for all $ x\in f(\mathbb{R})$ and also since $ f(\mathbb{R})\subset\{0,1\}$ it doesn't seem like much has been done.

I see that you are very good at these problems but could you give a solution with more detail?

One step:
WLOG $ f$ is non decreasing.

Suppose that $ x,y\in f(\mathbb{R})$ such that $ x\ge y$. Then $ f(x)\ge f(y)$, also $ -y\ge -x$ so $ f(-y)\ge f(-x)$ so $ f(y)\ge f(x)$. Therefore $ f(x)=f(y)$ for all $ x,y\in f(\mathbb{R})$.
\end{solution}



\begin{solution}[by \href{https://artofproblemsolving.com/community/user/29428}{pco}]
	\begin{tcolorbox} [I see that you are very good at these problems but could you give a solution with more detail?
\end{tcolorbox}

 :blush: Youre're right, I was a little bit too quick. Sorry for this. I forgot $ -1$ :

I think you agree with $ f(x)=f(-x)$ $ \forall x\in f(\mathbb R)$ so we have, since $ f(x)$ is monotonous, $ f(y)=f(x)$ $ \forall y\in [-|x|,+|x|]$ and $ f(0)=f(x)$

So, if $ a \in f(\mathbb R)$ and $ b \in f(\mathbb R)$ we have $ f(0)=f(a)$ and $ f(0)=f(b)$ and so $ f(a)=f(b)$

So $ f(x)=c$ $ \forall x\in f(\mathbb R)$
And, since $ c\in f(\mathbb R)$, $ f(c)=c$ but also $ f(c)=c^2$ so either $ c=0$, either $ c=1$

1) if $ c=0$ : we know that :
$ \forall x\in f(\mathbb R)$, $ f(x)=0$
But, $ \forall x\in f(\mathbb R)$, $ f(x)=x^2$
so $ \forall x\in f(\mathbb R)$, $ x=0$ and so $ f(\mathbb R)=\{0\}$ and so $ f(x)=0$ $ \forall x$

2) If $ c=1$ : we know that :
$ \forall x\in f(\mathbb R)$, $ f(x)=1$
But, $ \forall x\in f(\mathbb R)$, $ f(x)=x^2$
so $ \forall x\in f(\mathbb R)$, $ x^2=1$ and so $ f(\mathbb R)\subseteq\{-1,+1\}$

and , since $ f(-1)=f(1)=1$, we have 5 solutions (remember $ f(x)$ is monotonous):

$ f(x)=1$ $ \forall x$
Let $ a<-1$ and then $ f(x)=-1$ $ \forall x\leq a$ and $ f(x)=1$ $ \forall x>a$
Let $ a\leq -1$ and then $ f(x)=-1$ $ \forall x< a$ and $ f(x)=1$ $ \forall x\geq a$
Let $ a>1$ and then $ f(x)=+1$ $ \forall x<a$ and $ f(x)=-1$ $ \forall x\geq a$
Let $ a\geq 1$ and then $ f(x)=+1$ $ \forall x\leq a$ and $ f(x)=-1$ $ \forall x>a$


I think it's ok now  :blush:
\end{solution}
*******************************************************************************
-------------------------------------------------------------------------------

\begin{problem}[Posted by \href{https://artofproblemsolving.com/community/user/25853}{DocEmBr}]
	Determine all injective and continuous functions $f: \mathbb R \to \mathbb R$ with $ f(1)=1$ and \[ f(2x-f(x))=x\] for all real $x$.
	\flushright \href{https://artofproblemsolving.com/community/c6h261054}{(Link to AoPS)}
\end{problem}



\begin{solution}[by \href{https://artofproblemsolving.com/community/user/29428}{pco}]
	\begin{tcolorbox}Determine all injective and continuous functions $ f: R\rightarrow R$ with $ f(1) = 1$ and $ f(2x - f(x)) = x$\end{tcolorbox}

[hide="My solution"]
$ f(x) = x$ is obviously a solution and we'll show this is the only solution.

Suppose $ \exists a\in\mathbb R$ such that $ b = f(a)\neq a$ : we get $ f(2a - b) = a$ and, with induction : $ f(n(a - b) + b) = (n - 1)(a - b) + b$  $ \forall n\in \mathbb N$. Using then the fact that $ f(x)$ is obviously surjective, so bijective, we get $ f^{ - 1}(x) = 2x - f(x)$ and  so $ f(a_n) = a_n + (b - a)$ (where $ a_n = n(a - b) + b$)  $ \forall n\in \mathbb Z$

$ \forall x$, $ \exists n\in\mathbb Z$ such that $ x\in[a_n,a_{n + 1}]$ (or $ [a_{n + 1},a_n]$, but it does not matter), so $ f(x)\in[f(a_n),f(a_{n + 1})]$ since $ f(x)$ is monotonous (since injective and continuous). So $ f(x)\in[a_n + (b - a),a_{n + 1} + (b - a)]$

So $ |f(x) - x|\leq |f(x) - f(a_n)| + |f(a_n) - a_n| + |a_n - x|$ $ \leq 3|b - a|$

And, since $ f(1) - 1 = 0$, $ f(a) - a\neq 0$ and $ f(x) - x$ is continuous, we can find some $ a,b = f(a)$ such that $ |b - a|$ may be as little as we want.

And so $ f(x) = x$ $ \forall x$
[\/hide]
\end{solution}
*******************************************************************************
-------------------------------------------------------------------------------

\begin{problem}[Posted by \href{https://artofproblemsolving.com/community/user/46787}{moldovan}]
	Find all the continuous functions $ f: \mathbb{R} \rightarrow \mathbb{R}$ such that:

$ f(x+y)+f(x-y)=2f(x)f(y)$, for all $ x,y \in \mathbb{R}$
	\flushright \href{https://artofproblemsolving.com/community/c6h262802}{(Link to AoPS)}
\end{problem}



\begin{solution}[by \href{https://artofproblemsolving.com/community/user/29428}{pco}]
	\begin{tcolorbox}Find all the continuous functions $ f: \mathbb{R} \rightarrow \mathbb{R}$ such that:

$ f(x + y) + f(x - y) = 2f(x)f(y)$, for all $ x,y \in \mathbb{R}$\end{tcolorbox}
Let $ P(x,y)$ the assertion $ f(x + y) + f(x - y) = 2f(x)f(y)$

$ P(0,0)$ $ \implies$ $ 2f(0) = f(0)^2$ and so either $ f(0) = 0$, either $ f(0) = 1$

$ f(0) = 0$ and $ P(x,0)$ $ \implies$ $ f(x) = 0$ $ \forall x$ and this is a first solution.

So we'll now consider $ f(0) = 1$.
$ P(0,x)$ $ \implies$ $ f(x) + f( - x) = 2f(x)$ and so $ f( - x) = f(x)$ and $ f(x)$ is even.
$ P(x,x)$ $ \implies$ $ f(2x) = 2f(x)^2 - 1$

Case 1 : $ \exists b$ such that $ f(b) < 1$.
======
Suppose now that $ \exists b$ such that $ f(b) < 1$. Since $ f( - x) = f(x)$, wlog consider $ b > 0$.
Obviously, the sequence $ u_0 = b < 1$, $ u_{n + 1} = 2u_n^2 - 1$ contains negative numbers and so (since $ f(0) = 1$ and $ f(x)$ is continuous), there exists $ c > 0$ such that $ f(c) = 0$. So the set $ Z = \{x > 0$ such that $ f(x) = 0\}$ has an infimum "$ a$" and $ f(a) = 0$ (since $ f(x)$ is continuous)

So we have $ f(0) = 1$, $ f(a) = 0$ and $ f(x) > 0$ $ \forall x\in[0,a)$
Then $ P(x + 2a,x)$ $ \implies$ $ f(x + 2a) = - f(x)$ and so $ f(x + 4a) = f(x)$ and so $ f(x)$ is a periodic function. 

It's easy to show with induction over $ n$ (using $ P(\frac {a}{2^n},\frac {a}{2^n})$ and the fact that $ f(\frac {a}{2^n})\geq 0$ $ \forall n\geq 0$) that $ f(\frac {a}{2^n}) = \cos(\frac {\pi}{2^{n + 1}})$ $ \forall n\geq 0$

It's then easy to show with induction over $ p$ (using $ P(p\frac {a}{2^n},\frac {a}{2^n})$) that $ f(p\frac {a}{2^n}) = \cos(p\frac {\pi}{2^{n + 1}})$ $ \forall n\geq 0$ and $ \forall p\in[0,2^n]$

And, since the set $ \{p\frac {a}{2^n}$, $ \forall n\geq 0$ and $ \forall p\in[0,2^n]\}$ is dense in $ [0,a]$ and since $ f(x)$ is continuous :

$ f(x) = \cos(\frac {\pi x}{2a})$ $ \forall x\in[0,a]$
Now, since $ f(x + 2a) = - f(x)$, since $ f(x + 4a) = f(x)$ and since $ f( - x) = f(x)$, we have :

$ f(x) = \cos(\frac {\pi x}{2a})$ $ \forall x\in\mathbb R$

and it's easy to check that this function matches the original equation.

Case 2 : $ f(x)\geq 1$ $ \forall x\mathbb R$
======
$ f(1)\geq 1$ and so $ \exists u \geq 0$ such that $ f(1) = \cosh(u)$

It's easy to show with induction over $ n$ (using $ P(\frac {1}{2^n},\frac {1}{2^n})$, and the fact that $ f(\frac {1}{2^n})\geq 1$ $ \forall n\geq 0$) that $ f(\frac {1}{2^n}) = \cosh(\frac {u}{2^n})$ $ \forall n\geq 0$

It's easy to show with induction over $ n$ (using $ P(2^n,2^n)$, and the fact that $ f(2^n)\geq 1$ $ \forall n\geq 0$) that $ f(2^n) = \cosh(u2^n)$ $ \forall n\geq 0$

It's then easy to show with induction over $ p$ (using $ P(p2^k,2^k)$) that $ f(p2^k) = \cosh(pu2^k)$ $ \forall k\in\mathbb Z$ $ \forall p\in\mathbb N$

And, since the set $ \{p2^k$, $ \forall k\in\mathbb Z$ $ \forall p\in\mathbb N\}$ is dense in $ \mathbb R^ +$ and since $ f(x)$ is continuous :

$ f(x) = \cosh(ux)$ $ \forall x\in\mathbb R +$

And, since $ f(0) = 1$ and $ f( - x) = f(x)$, $ f(x) = \cosh(ux)$ $ \forall x\in\mathbb R$
and it's easy to check that this function matches the original equation.

Synthesis :
=========
All solutions are :
$ f(x) = 0$
$ f(x) = \cos(\alpha x)$
$ f(x) = \cosh(\alpha x)$

(and the solution $ f(x) = 1$ is obtained thru $ \alpha = 0$)
\end{solution}
*******************************************************************************
-------------------------------------------------------------------------------

\begin{problem}[Posted by \href{https://artofproblemsolving.com/community/user/29876}{ozgurkircak}]
	Find all functions $ f: \mathbb{R^+}\rightarrow \mathbb{R^+}$ satisfying \[ f(xf(x)+f(y))=(f(x))^2+y\] for all $ x,y>0.$
	\flushright \href{https://artofproblemsolving.com/community/c6h267843}{(Link to AoPS)}
\end{problem}



\begin{solution}[by \href{https://artofproblemsolving.com/community/user/49556}{xxp2000}]
	Here is a solution. Hopefully someone can propose a shorter one.

Let $ f(a)=b$ with $ a,b>0$.
$ P(a,y): f(ab+f(y))=b^2+y$. (F1)
$ P(x,a): f(xf(x)+b)=f(x)^2+a$.  (F2)
Replace $ y$ in (F1) with $ ab+f(y)$, we get
$ f(ab+b^2+y)=ab+b^2+f(y)$.
Let $ c=ab+b^2$, we get
$ f(c+y)=f(y)+c$  (F3)

1) $ f(u+v)=f(u+b)+f(v+ab)+d,\forall u,v>0$, where $ d$ is some constant.
From (F1), we know $ f$ is injective and $ (b^2,\infty)\subset Im(f)$.
Then from (F2), we see $ f(xf(x)+b)=t$ has solution whenever $ t>b^4+a$.
Also, $ f(ab+f(y))=t$ has solution whenever $ t>b^2$.
From (F3), we have $ f(y+nc)=f(y)+nc$, where $ n\in\mathbb N$.
For any fixed $ u,v>0$, we can find $ n,m\in\mathbb N$ such that
$ f(u+b+nc)=f(u+b)+nc>b^4+a$ and let $ r$ solve $ f(rf(r)+b)=f(u+b+nc)$
and
$ f(v+ab+mc)=f(v+ab)+mc>b^2$ and let $ q$ solve $ f(ab+f(q))=f(v+ab+mc)$.

Since $ f$ is injective, we have
$ rf(r)+b=u+b+nc$ and $ ab+f(q)=v+ab+mc$.
$ P(r,q): f(rf(r)+f(q))=f(r)^2+q=f(rf(r)+b)-a+f(ab+f(q))-b^2$ implies
$ f(u+nc+v+mc)=f(u+b+nc)+f(v+ab+mc)+d$, where $ d=-a-b^2$
Use (F3), we get
$ f(u+v)=f(u+b)+f(v+ab)+d$.

2) $ f(u+v)=f(u)+f(v)+e,\forall u,v>0$, where $ e$ is constant
Let $ n\in\mathbb N$ be such that $ nc>b$, we can apply $ (u+nc,v)$ and $ (u+nc-b,v+b)$ to 1),
$ f(u+nc+v)=f(u+nc+b)+f(v+ab)+d=f(u+nc)+f(v+ab+b)\Rightarrow f(u+b)-f(u)=f(v+ab+b)-f(v+ab)$.
RHS does not depend on $ u$. So we have $ f(u+b)-f(u)=e_1$ with constant $ e_1$.
Similarly $ f(v+ab)=f(v)+e_2$ with constant $ e_2$.
Then 1) implies 2)

3) $ f(u+v)=f(u)+f(v),\forall u,v>0$
Let $ g(x)=f(x)+e$. We see $ g$ satisfies Cauchy equation.
So for $ r\in\mathbb Q$, $ g(r)=rg(1)$ or $ f(r)=-e+rg(1)=-e(1-r)+rf(1)>0$.
Obviously $ e\le0$ and $ g(1)>0$. 

Suppose $ e<0$.
(F1) can be rewritten as $ g(g(x))=x+e_3$, where $ e_3$ is constant.
$ g(g(r))=rg(g(1))=r+e_3,\forall  r\in\mathbb Q$
Hence, $ g(g(1))=1,e_3=0$ and $ g(g(x))=x$
For $ r,q\in\mathbb Q$,
$ P(r,q): f(rf(r)+f(q))=g(rg(r)-re+g(q)-e)-e=f(r)^2+q=(g(r)-e)^2+q$ 
or
$ r^2g(g(1))+rg(-e)+g(g(q))+g(-e)-e=(rg(1)-e)^2+q$ 
or
$ r^2+rg(-e)=r^2g(1)^2-2eg(1)r+e^2+e-g(-e)$.
Now, we treat both sides as polynomial of $ r$.
$ g(1)=1,g(-e)=-2e,e^2+e=g(-e)$ implies
$ e=-3,g(3)=6\ne 3g(1)$. Contradiction! So $ e=0$ and 3) holds.

4) $ f(x)=x$
3) implies $ f(r)=rf(1)$. Then $ f(f(r))=rf(f(1))$.
Hence, (F1) implies $ rf(f(1))+f(ab)=b^2+r\Rightarrow f(ab)=b^2$
Since $ a$ is arbitrary, we have 
$ f(xf(x))=f(x)^2$ (F4)
Then the 3) and (F1) imply $ f(f(y))=y$ (F5)
Replace $ x$ with $ f(x)$ in (F4), we get
$ f(xf(x))=f(f(x))^2=x^2=f(x)^2$.
Hence, $ f(x)=x$.
\end{solution}



\begin{solution}[by \href{https://artofproblemsolving.com/community/user/29428}{pco}]
	\begin{tcolorbox}Here is a solution. Hopefully someone can propose a shorter one.\end{tcolorbox}

Hello!. I have one question  about your demo and a proposal for another one (not very short too)

\begin{tcolorbox} 2) $ f(u + v) = f(u) + f(v) + e,\forall u,v > 0$, where $ e$ is constant
Let $ n\in\mathbb N$ be such that $ nc > b$, we can apply $ (u + nc,v)$ and $ (u + nc - b,v + b)$ to 1),
$ f(u + nc + v) = f(u + nc + b) + f(v + ab) + d = f(u + nc) + f(v + ab + b)\Rightarrow f(u + b) - f(u) = f(v + ab + b) - f(v + ab)$.
RHS does not depend on $ u$. So we have $ f(u + b) - f(u) = e_1$ with constant $ e_1$.
Similarly $ f(v + ab) = f(v) + e_2$ with constant $ e_2$.
Then 1) implies 2)\end{tcolorbox}

we miss a "$ +d$" at the end of  $ f(u + nc + v) = f(u + nc + b) + f(v + ab) + d = f(u + nc) + f(v + ab + b)$ (but no consequence).
But I dont understand "Similarly $ f(v + ab) = f(v) + e_2$ with constant $ e_2$"

Here is my own proposal :

1) Original assertion $ P0(x,y)$ : $ f(xf(x)+f(y))=(f(x))^2+y$

Let $ a=f(1)$ and $ b=f(f(1))$

2) $ f(x)$ is an injective function :
If $ f(y_1)=f(y_2)$, subtracting $ P0(x,y_1)$ from $ P0(x,y_2)$ implies $ y_1=y_2$.
Q.E.D.

3) new assertion $ P1(x,y)$ : $ f(xf(x)+a^2+y)=(f(x))^2+a+f(y)$
$ P0(x,f(1)+f(y))$ $ \implies$ $ f(xf(x)+f(f(1)+f(y)))=(f(x))^2+f(1)+f(y)$. But (using $ P0(1,y)$) : $ f(f(1)+f(y))=(f(1))^2+y$.
Q.E.D.

4) new assertion $ S(x)$ : $ f(x+a^2+a)=f(x)+a^2+a$
This is an immediate result from $ P1(1,x)$
Q.E.D.

5) new assertion $ Q(x)$ : $ (f(x))^2=f(xf(x))+a^2-b$ 
$ P1(x,a)$ $ \implies$ $ f(xf(x)+a^2+a)=(f(x))^2+a+b$
$ S(xf(x))$ $ \implies$ $ f(xf(x)+a^2+a)=f(xf(x))+a^2+a$
These two equalities have the same left member. So : $ (f(x))^2+a+b=f(xf(x))+a^2+a$
Q.E.D.

6) new assertion $ P2(x,y)$ : $ f(xf(x)+f(y))=xf(x)+y+a^2-b$
This is an immediate result from using $ Q(x)$ in right part of $ P0(x,y)$
Q.E.D.

7) new assertion $ R(x)$ : $ (f(x))^2 = xf(x) + a^2 - a$
$ P2(x,a)$ : $ f(xf(x)+b)=xf(x)+a+a^2-b$
$ P2(1,xf(x))$ : $ f(a+f(xf(x)))=a+xf(x)+a^2-b$

These two equalities have the same right member. So, since $ f(x)$ is injective : $ xf(x)+b=a+f(xf(x))$ and so :
$ 0=-f(xf(x))+xf(x)-a+b$
Adding this assertion to $ Q(x)$, we get the required result.
Q.E.D.

8) new assertion : $ f(x)=x$
$ R(x+a^2+a)$ $ \implies$ $ (f(x+a^2+a))^2 = (x+a^2+a)f(x+a^2+a) + a^2-a$.
But ($ S(x)$) : $ f(x+a^2+a)=f(x)+a^2+a$ and so $ (f(x))^2+2(a^2+a)f(x)+(a^2+a)^2$ $ = xf(x)+(a^2+a)x+$ $ (a^2+a)f(x)+(a^2+a)^2 + a^2-a$, so :
$ (f(x))^2 = xf(x)+(a^2+a)x -(a^2+a)f(x) + a^2-a$
Subtracting $ R(x)$ from this, we get :
$ (a^2+a)f(x)=(a^2+a)x$
Hence the result (since $ a^2+a>0$)

9) The only solution to this equation is $ f(x)=x$
This is a required condition (from 8) and it's easy to verify that this mandatory form fit the equation.
Q.E.D
\end{solution}



\begin{solution}[by \href{https://artofproblemsolving.com/community/user/49556}{xxp2000}]
	I mean "similarly" by applying $ (u,v+mc)$ and $ (u+ab,v+mc-ab)$ to 1), where $ m\in\mathbb N$ such that $ mc>ab$.
\end{solution}



\begin{solution}[by \href{https://artofproblemsolving.com/community/user/29428}{pco}]
	Ok, quite clear, thanks

Next question (I hope I dont bother you : I try to understand the full demo) :

\begin{tcolorbox} 
3) $ f(u + v) = f(u) + f(v),\forall u,v > 0$
Let $ g(x) = f(x) + e$. We see $ g$ satisfies Cauchy equation.
So for $ r\in\mathbb Q$, $ g(r) = rg(1)$ or $ f(r) = - e + rg(1) = - e(1 - r) + rf(1) > 0$.
Obviously $ e\le0$ and $ g(1) > 0$. \end{tcolorbox}
OK

\begin{tcolorbox} Suppose $ e < 0$.
(F1) can be rewritten as $ g(g(x)) = x + e_3$, where $ e_3$ is constant.
$ g(g(r)) = rg(g(1)) = r + e_3,\forall r\in\mathbb Q$
Hence, $ g(g(1)) = 1,e_3 = 0$ and $ g(g(x)) = x$
\end{tcolorbox}

Here, I have again a question : I dont understand why (F1) can be rewritten as $ g(g(x)) = x + e_3$
(F1) is $ f(ab+f(y))=b^2+y$. So, writing $ h(x)=f(x)+ab$, we get $ h(h(x))=b^2+ab+x$ but, IMHO,  this $ h(x)$ is different (with a constant) from $ g(x)$ and we dont have $ h(x+y)=h(x)+h(y)$
\end{solution}



\begin{solution}[by \href{https://artofproblemsolving.com/community/user/49556}{xxp2000}]
	Using 2),
$ g(g(x))=f(f(x)+e)+e=f(f(x)+ab)-f(ab-e)-e+e=x+b^2-f(ab-e)$.
\end{solution}



\begin{solution}[by \href{https://artofproblemsolving.com/community/user/29428}{pco}]
	\begin{tcolorbox}Using 2),
$ g(g(x)) = f(f(x) + e) + e = f(f(x) + ab) - f(ab - e) - e + e = x + b^2 - f(ab - e)$.\end{tcolorbox}


That's OK (and $ f(ab-e)$ is meaningful since $ e<0$, so $ ab-e>0$)

No more questions, thanks.
And I do agree with your demo !

:)
\end{solution}



\begin{solution}[by \href{https://artofproblemsolving.com/community/user/49556}{xxp2000}]
	I figure out there is still one flaw in my post above. we have to make sure $ f(x)+e>0$, which can be shown below.
$ g(rx)=rg(x)\Rightarrow f(rx)=-e+rg(x)>0\Rightarrow g(x)>\frac er\Rightarrow g(x)\ge0,\forall x>0$. If $ g(x)=0$, then $ g(2x)=0$ and $ f(x)=f(2x)$ conflicts with injective. So $ g(x)>0,\forall x>0$.

Thanks for your comments. pco.
\end{solution}
*******************************************************************************
-------------------------------------------------------------------------------

\begin{problem}[Posted by \href{https://artofproblemsolving.com/community/user/19819}{matex (L)(L)(L)}]
	Prove that there does not exist a continuous function $f: \mathbb R \to \mathbb R$ such that for all real $x$,
\[ f(x-f(x)) =x\/2.\]
	\flushright \href{https://artofproblemsolving.com/community/c6h271187}{(Link to AoPS)}
\end{problem}



\begin{solution}[by \href{https://artofproblemsolving.com/community/user/49383}{limes123}]
	Very nice problem! Here's my solution (hope it's correct)
[hide="Solution"]I'll prove that if $ a\in \mathbb{R}$ then $ \exists_{x_0\in \mathbb{R}} a=x_0-f(x_0)$. Assume it's not true, and let $ a\in \mathbb{R}$ be such that $ a\neq x-f(x)$ for all real $ x$. Since $ g(x)=x-f(x)$ is continous function we have $ g(x)>a$ or $ g(x)<a$ for all real $ x$. Assume WLOG that $ g(x)>a\iff x-f(x)>a$. Setting $ x=x-f(x)$ we get $ (x-f(x))-\frac{x}{2}>a\iff \frac{x}{2}-f(x)>a$ and again $ x=x-f(x)$ gives $ f(x)<-2a$ which is a contradiction, since f is surjective. Assume that $ f(b)=f(c)$ for some real $ b\neq c$. Since $ b=x_1-f(x_1)$ and $ c=x_2-f(x_2)$ for some real $ x_1,x_2$ we have $ \frac{x_1}{2}=f(x_1-f(x_1))=f(b)=f(c)=f(x_2-f(x_2))=\frac{x_2}{2}$, hence $ x_1=x_2$ and $ b=c$, contradiction. This means that f is injective, hence it's increasing (or decreasing, but I can assume WLOG since proof of second case is analogous). Let $ r_1>r_2$ be two real numbers. Since f is increasing we have $ f(r_1)>f(r_2)$ and $ f(r_1-f(r_1))=\frac{r_1}{2}>\frac{r_2}{2}=f(r_2-f(r_2))\Rightarrow r_1-f(r_1)>r_2-f(r_2)\Rightarrow f(\frac{r_1}{2}-f(r_1))=\frac{r_1-f(r_1)}{2}>\frac{r_2-f(r_2)}{2}=f(\frac{r_2}{2}-f(r_2))\Rightarrow f(-\frac{f(r_1)}{2})=\frac{\frac{r_1}{2}-f(r_1)}{2}>\frac{\frac{r_2}{2}-f(r_2)}{2}=f(-\frac{f(r_2)}{2})\Rightarrow f(r_1)<f(r_2)$ which is a contradiction.[\/hide]
\end{solution}



\begin{solution}[by \href{https://artofproblemsolving.com/community/user/29428}{pco}]
	\begin{tcolorbox}Very nice problem! Here's my solution (hope it's correct)
[hide="Solution"]I'll prove that if $ a\in \mathbb{R}$ then $ \exists_{x_0\in \mathbb{R}} a = x_0 - f(x_0)$. Assume it's not true, and let $ a\in \mathbb{R}$ be such that $ a\neq x - f(x)$ for all real $ x$. Since $ g(x) = x - f(x)$ is continous function we have $ g(x) > a$ or $ g(x) < a$ for all real $ x$. Assume WLOG that $ g(x) > a\iff x - f(x) > a$. Setting $ x = x - f(x)$ we get $ (x - f(x)) - \frac {x}{2} > a\iff \frac {x}{2} - f(x) > a$ and again $ x = x - f(x)$ gives $ f(x) < - 2a$ which is a contradiction, since f is surjective. Assume that $ f(b) = f(c)$ for some real $ b\neq c$. Since $ b = x_1 - f(x_1)$ and $ c = x_2 - f(x_2)$ for some real $ x_1,x_2$ we have $ \frac {x_1}{2} = f(x_1 - f(x_1)) = f(b) = f(c) = f(x_2 - f(x_2)) = \frac {x_2}{2}$, hence $ x_1 = x_2$ and $ b = c$, contradiction. This means that f is injective, hence it's increasing (or decreasing, but I can assume WLOG since proof of second case is analogous). Let $ r_1 > r_2$ be two real numbers. Since f is increasing we have $ f(r_1) > f(r_2)$ and $ f(r_1 - f(r_1)) = \frac {r_1}{2} > \frac {r_2}{2} = f(r_2 - f(r_2))\Rightarrow r_1 - f(r_1) > r_2 - f(r_2)\Rightarrow f(\frac {r_1}{2} - f(r_1)) = \frac {r_1 - f(r_1)}{2} > \frac {r_2 - f(r_2)}{2} = f(\frac {r_2}{2} - f(r_2))\Rightarrow f( - \frac {f(r_1)}{2}) = \frac {\frac {r_1}{2} - f(r_1)}{2} > \frac {\frac {r_2}{2} - f(r_2)}{2} = f( - \frac {f(r_2)}{2})\Rightarrow f(r_1) < f(r_2)$ which is a contradiction.[\/hide]\end{tcolorbox}

Very nice solution (and quite correct, according to me). Here is another one (very similar, using the same ideas) :

We have (applying 3 times the original equation) :

(A1) : $ f(a) = b$
(A2) : $ f(a - b) = \frac {a}{2}$
(A3) : $ f(\frac {a - 2b}{2}) = \frac {a - b}{2}$
(A4) : $ f( - \frac {b}{2}) = \frac {a - 2b}{4}$

Then : $ f(a) = f(c) = b$ $ \implies$ (looking at A4) : $ f( - \frac {b}{2}) = \frac {a - 2b}{4} = \frac {c - 2b}{4}$ and so $ a = c$ and $ f(x)$ is injective, so monotonous.

Using $ a = 0$ and comparing (A3) and (A4), we get $ f( - b) = f( - \frac {b}{2})$ and so $ b = 0$ since $ f(x)$ is injective. So $ f(0) = 0$

Since $ f(x)$ is monotonous and $ f(0) = 0$, $ \frac {f(x)}{x}$ has a constant sign for any $ x\neq 0$

So, using (A1) to (A4), we get that $ \frac {b}{a}$, $ \frac {a}{2(a - b)}$, $ \frac {a - b}{a - 2b}$ and $ - \frac {a - 2b}{2b}$ all have the same sign, which is impossible since their product is $ - \frac {1}{4}$, negative

Hence the contradiction.
\end{solution}
*******************************************************************************
-------------------------------------------------------------------------------

\begin{problem}[Posted by \href{https://artofproblemsolving.com/community/user/25714}{TaiPan~SP!}]
	Does there exist $ f: \mathbb{R} \rightarrow \mathbb{R}$ such that $ f(0)>0$ and $ f(x+y) \geq f(x)+yf(f(x))$ for all $ x,y \in \mathbb{R}$?
	\flushright \href{https://artofproblemsolving.com/community/c6h271856}{(Link to AoPS)}
\end{problem}



\begin{solution}[by \href{https://artofproblemsolving.com/community/user/25714}{TaiPan~SP!}]
	Anyone? Any ideas? :maybe:  Anything helps, please!
\end{solution}



\begin{solution}[by \href{https://artofproblemsolving.com/community/user/29428}{pco}]
	\begin{tcolorbox}Does there exist $ f: \mathbb{R} \rightarrow \mathbb{R}$ such that $ f(0) > 0$ and $ f(x + y) \geq f(x) + yf(f(x))$ for all $ x,y \in \mathbb{R}$?\end{tcolorbox}

Let $ P(x,y)$ be the assertion $ f(x + y)\geq f(x) + yf(f(x))$

Let $ u > 0$.

$ P(x,u)$ $ \implies$ $ \frac {f(x + u) - f(x)}{u}\geq f(f(x)$
$ P(x + u, - u)$ $ \implies$ $ f(f(x + u)) \geq\frac {f(x + u) - f(x)}{u}$

And so the new assertion $ Q(x,u)$ : $ f(f(x + u)) \geq\frac {f(x + u) - f(x)}{u}\geq f(f(x))$ $ \forall x$, $ \forall u > 0$ and so $ f(f(x))$ is a non decreasing function.

Let's suppose $ f(f(x))\leq 0$ $ \forall x$, then $ Q(x,u)$ shows that $ \frac {f(x + u) - f(x)}{u}\leq 0$ $ \forall u > 0$ and so $ f(x)$ is a non increasing function. So $ f(f(x))\leq 0$ $ \implies$ $ f(f(x)) < f(0)$ $ \implies$ $ f(x) > 0$ $ \forall x$, in contradiction with $ f(f(x))\leq 0$ $ \forall x$.

So, $ \exists a$ such that $ f(f(a)) > 0$ and so (since $ f(f(x))$ is non decreasing), $ f(f(x)) > 0$ $ \forall x\geq a$

Then $ P(a,x)$ $ \implies$ $ f(x + a)\geq xf(f(a)) + f(a)$ and so $ \lim_{x\to + \infty}f(x) = + \infty$ and so $ \lim_{x\to + \infty}f(f(x)) = + \infty$.

So, $ \exists b$ such that $ f(f(b)) > 2$ and then $ P(b,x)$ $ \implies$ $ f(x + b)\geq xf(f(b)) + f(b)$ and so $ f(x + b) > 2x + f(b)$ $ \forall x > 0$

So, $ \exists c$ such that $ f(f(x)) > 0$ and $ f(x) > x + 1 > 0$ $ \forall x > c$

Then $ \forall x > c$ $ P(x,f(x) - x)$ $ \implies$ $ f(f(x))\geq f(x) + (f(x) - x)f(f(x))$ and so $ 0\geq f(x) + (f(x) - x - 1)f(f(x))$.

But this is impossible, since, $ \forall x > c$ :  $ f(f(x)) > 0$, $ f(x) > 0$ and $ f(x) > x + 1$

So, no such function exists.
\end{solution}
*******************************************************************************
-------------------------------------------------------------------------------

\begin{problem}[Posted by \href{https://artofproblemsolving.com/community/user/49383}{limes123}]
	Find all continous functions $ f: \mathbb{R}\rightarrow \mathbb{R}$ satisfying $ f(f(x))=f(x)+2x$ for $ x\in \mathbb{R}$.
	\flushright \href{https://artofproblemsolving.com/community/c6h271914}{(Link to AoPS)}
\end{problem}



\begin{solution}[by \href{https://artofproblemsolving.com/community/user/29428}{pco}]
	\begin{tcolorbox}Find all continous functions $ f: \mathbb{R}\rightarrow \mathbb{R}$ satisfying $ f(f(x)) = f(x) + 2x$ for $ x\in \mathbb{R}$.\end{tcolorbox}

From the equation, we get $ x=\frac{f(f(x))-f(x)}{2}$ and so $ f(x)$ is a bijection and so is strictly monotonous (since continuous).
We have $ f(f(0))=f(0)$ and, since $ f(x)$ is injective, $ f(0)=0$

So, $ \frac{f(x)}{x}$has a constant sign $ \forall x\in\mathbb{R}^*$ ($ f(x)$ strictly monotonous and $ f(0)=0$)  

Let $ a\neq 0$, $ b=f(a)$ and the two sequences :
$ u_0=a$, $ u_1=b$, $ u_{n+2}=u_{n+1}+2u_n$

$ v_0=b$, $ v_1=a$, $ v_{n+2}=\frac{v_n-v_{n+1}}{2}$

We clearly have $ f(u_n)=u_{n+1}$ and $ f(v_{n+1})=v_n$

We also have : 
$ u_n=\frac{a+b}{3}2^n + \frac{2a-b}{3}(-1)^n$

$ v_n=2\frac{a+b}{3}\frac{1}{2^n} - \frac{2a-b}{3}(-1)^n$

So :

$ \frac{f(u_n)}{u_n}$ $ =\frac{u_{n+1}}{u_n}$ $ =\frac{(a+b)2^{n+1} + (2a-b)(-1)^{n+1}}{(a+b)2^{n} + (2a-b)(-1)^{n}}$

$ \frac{f(v_n)}{v_n}$ $ =\frac{v_{n-1}}{v_n}$ $ =\frac{2(a+b)\frac{1}{2^{n-1}} - (2a-b)(-1)^{n-1}}{2(a+b)\frac{1}{2^{n}} - (2a-b)(-1)^{n}}$

Now, If $ a+b\neq 0$ : $ \frac{f(u_n)}{u_n}$ is $ >0$ for $ n$ great enough (limit is $ +2$), but, if $ 2a-b\neq 0$, $ \frac{f(v_n)}{v_n}$ is $ <0$ for $ n$ great enough (limit is $ -1$). But this is impossible, since we know that $ \frac{f(x)}{x}$has a constant sign $ \forall x\in\mathbb{R}^*$.

So, either $ a+b=0$, either $ 2a-b=0$

So, $ \forall a\in\mathbb{R}^*$, either $ f(a)=-a$, either $ f(a)=2a$.

And, since $ \frac{f(x)}{x}$has a constant sign $ \forall x\in\mathbb{R}^*$:
Either $ f(x)=-x$ $ \forall x\in\mathbb{R}$
Either $ f(x)=2x$ $ \forall x\in\mathbb{R}$

And it's immediate to verifiy that these two functions actually match the initial equation.
\end{solution}
*******************************************************************************
-------------------------------------------------------------------------------

\begin{problem}[Posted by \href{https://artofproblemsolving.com/community/user/38553}{Agr_94_Math}]
	Find all $ f: \mathbb{N} \cup \{0\} \to \mathbb{N} \cup \{0\}$ such that
1. $ f(m^2+n^2)=f(m)^2+f(n)^2$ for $ m,n \in \mathbb{N_0}$, and
2. $ f(1)> 0$.
	\flushright \href{https://artofproblemsolving.com/community/c6h272002}{(Link to AoPS)}
\end{problem}



\begin{solution}[by \href{https://artofproblemsolving.com/community/user/29428}{pco}]
	\begin{tcolorbox}Find all $ f: \mathbb{N}_0 \to \mathbb{N}_0$ such that
1. $ f(m^2 + n^2) = f(m)^2 + f(n)^2$ for $ m,n \in \mathbb{N}_0$.
2. $ f(1) > 0$.\end{tcolorbox}

I hope somebody has a shorter demo than mine :)

We immediatly have $ f(0)=0$ and then $ f(n^2)=f(n)^2$

From the equality $ (2p-1)^2 + (p-3)^2 = (2p-3)^2 + (p+1)^2$, we get $ f((2p-1)^2+(p-3)^2)=f((2p-3)^2+(p+1)^2)$ and so :

$ (E1)$ : $ f(2p-1)^2 = f(2p-3)^2 + f(p+1)^2 - f(p-3)^2$ (notice that if $ p\geq 4$, we have $ 2p-1>$ all other args of $ f(x)$ in RHS)

From the equality $ (2p)^2 + (p-5)^2 = (2p-4)^2 + (p+3)^2$, we get similarly :

$ (E2)$ : $ f(2p)^2 = f(2p-4)^2 + f(p+3)^2 - f(p-5)^2$ (notice that if $ p\geq 4$, we have $ 2p>$ all other args of $ f(x)$ in RHS)

From these two equalities, we see that the knowledge of $ f(n)$ for $ n\in[0,6]$ implies the full knowledge of $ f(n)$ $ \forall n\in\mathbb{N}_0$

Then :
$ f(0)=0$
$ f(1)=a$
$ f(2)=f(1^2+1^2)=2a$
$ f(4)=f(2^2)=4a^2$
$ f(5)=f(2^2+1^2)=4a^2+a$
$ f(8)=f(2^2+2^2)=8a^2$

$ f(7^2+1^2)=f(5^2+5^2)$ $ \implies$ $ f(7)^2=2f(5)^2-f(1)^2 = 32a^4+16a^3+a^2$
$ f(7^2+4^2)=f(8^2+1^2)$ $ \implies$ $ f(7)^2=f(8)^2+f(1)^2-f(4)^2=48a^4+a^2$

So $ 32a^4+16a^3+a^2=48a^4+a^2$ and $ a=0$ or $ a=1$, and so $ a=1$ (statement 2 says $ a>0$).

Then :
$ f(0)=0$
$ f(1)=1$
$ f(2)=2$ (since $ f(2)=2a$, from above)
$ f(3)=b$
$ f(4)=4$ (since $ f(4)=4a^2$, from above)
$ f(5)=5$ (since $ f(5)=4a^2+a$, from above)
$ f(6)=d$
$ f(7)=7$ (since $ f(7)^2=48a^4+a^2$, from above)
$ f(8)=8$ (since $ f(8)=8a^2$, from above)

$ f(25)=f(5^2)=f(5)^2=25$
$ f(25)=f(3^2+4^2)=b^2+f(4)^2=b^2+16$ and so $ b^2+16=25$ and $ b=3$

$ f(9)=f(3^2)=b^2=9$

$ f(85)=f(9^2+2^2)=9^2+2^2=85$
$ f(85)=f(7^2+6^2)=f(7^2)+d^2=d^2+49$ and so $ d^2+49=85$ and so $ f(6)=d=6$

So $ f(n)=n$ $ \forall n\in[0,6]$

Since $ (E1)$ and $ (E2)$ show that $ f(n)$ is uniquely defined when values in $ [0,6]$ are known, and since we just saw there is one unique set of possible values of $ f(n)$ when $ n\in[0,6]$, there is a unique solution to this equation.

And since $ f(n)=n$ $ \forall n\in\mathbb{N}_0$ is obviously a solution, this is the unique solution.
\end{solution}



\begin{solution}[by \href{https://artofproblemsolving.com/community/user/32726}{Differ}]
	Your squares identities at the beginning were absolutely ingenious. I have never seen an uniqueness solution for functional equations before.

Also, $ f(1) = f(1)^2$ leads to $ f(1) = 1$ from the problem restriction.
\end{solution}



\begin{solution}[by \href{https://artofproblemsolving.com/community/user/29428}{pco}]
	\begin{tcolorbox} Also, $ f(1) = f(1)^2$ leads to $ f(1) = 1$ from the problem restriction.\end{tcolorbox}

Aaaaaahhh! Sure. It's immediate (I needed nearly 10 lines to prove this ! :blush: )
\end{solution}



\begin{solution}[by \href{https://artofproblemsolving.com/community/user/38553}{Agr_94_Math}]
	Nice solution pco.Thanks for your patience.
Here is my solution:
 $ f: \mathbb{N}_{0}\to\mathbb{N}_{0}$ 
$ f(m^{2} + n^{2}) = f(m)^{2} + f(n)^{2}$ (1)
$ f(1) > 0$.(2)
Substitute  $ m = n = 0$ in (1). We get 
$ f(0) = 2 f(0)^2$. THis implies that $ f(0)(2f(0)-1)=0$ which gives us that $ f(0)=0 or \frac{1}{2}$. But $ f(0) \ne \frac{1}{2}$ as the function is defined only for whole numbers while $ \frac{1}{2}$ is not a whole number. this implies that $ f(0) = 0$ which leads to $ f(m)^2 = f(m^2)$.
We can write (1) as $ f(m^2 + n^2) = f(m)^2 + f(n)^2 = f(m^2) + f(n^2)$.
Observe that $ f(1) = f(1^2) = f(1)^2$. Since $ f(1) > 0$,  we get that $ f(1) = 1$.
THis leads to:
$ f(2) = f(1^2 + 1^2) = f(1)^2 + f(1)^2 = 1 + 1 = 2$.
$ f(4) = f(2)^2 = 4$.
$ f(5) = f(2)^2 + f(1)^2 = 5$
$ f(8) = 2 f(2)^2 = 8$.

Also, notice that $ f(5)^2 = 25 = f(5^2) = f(3)^2 + f(4)^2 = f(3)^2 + 16.$
This implies that $ f(3) = 3$.
THis further leads to 
$ f(9) = f(3^2) = f(3)^2 = 9$
$ f(10) = f(3^2 + 1^2) = 10$
Now we know that $ 2* 5^2 = 7^2 + 1$.
We already know $ f(5)$ and $ f(1)$ and hence we compute $ f(7)$ with the given values as $ 7$.
Also, $ (10,6,8)$ form a Pythagorean triplet.
THis implies that $ f(10^2) = 10^2 = f(6)^2 + f(8)^2 = f(6)^2 + 8^2$ which leads to $ f(6) = 6$.
With this, we come to a conclusion taht $ f(n) = n$ for all $ n \le 10$.

Now with the following identites
$ (5t + 1)^2 + 2^2 = (3t - 1)^2 + (4t + 2)^2$
$ (5t + 2)^2 + 1^2 = (4t + 1)^2 + (3t + 2)^2$
$ (5t + 3)^2 + 1^2 = (4t + 3)^2 + (3t + 1)^2$
$ (5t + 4)^2 + 2^2 = (4t + 2)^2 + (3t + 4)^2$
$ (5t + 5)^2 = (4t + 4)^2 + (3t + 3)^2$,

for $ t \ge 3$, notice that each term on the $ RHS$ does not exceed any term of the $ LHS$. Hence, we can go inductively in steps of $ 5$.
FOr $ t = 2$, $ 11^2 + 2^2 = 10^2 + 5^2$
                 $ 12^2 + 1^2 = 9^2 + 8^2$
                 $ 13^2 + 1^2 = 11^2 + 7^2$
                 $ 14^2 + 2^2 = 10^2 + 10^2$
                 $ 15^2 = 12^2 + 9^2$.
With these identities.  we can find $ f(n)$ by knowing the values of $ f$ of smaller numbers.
Hence, $ f(n) = n$ for all $ n \in \mathbb{N}_{0}$ .
\end{solution}



\begin{solution}[by \href{https://artofproblemsolving.com/community/user/29428}{pco}]
	\begin{tcolorbox} ...
Now with the following identites
$ (5t + 1)^2 + 2^2 = (3t - 1)^2 + (4t + 2)^2$
$ (5t + 2)^2 + 1^2 = (4t + 1)^2 + (3t + 2)^2$
$ (5t + 3)^2 + 1^2 = (4t + 3)^2 + (3t + 1)^2$
$ (5t + 4)^2 + 2^2 = (4t + 2)^2 + (3t + 4)^2$
$ (5t + 5)^2 = (4t + 4)^2 + (3t + 3)^2$,
...\end{tcolorbox}

Yes, quite similar solutions, just based on different sets of square identities.
Nice.
\end{solution}
*******************************************************************************
-------------------------------------------------------------------------------

\begin{problem}[Posted by \href{https://artofproblemsolving.com/community/user/60652}{DCTPKTCSPHN}]
	Find all $f: \mathbb R \to \mathbb R$ such that $ f(f(x) + y) = 2x+ f(f(y) - x)$  for all $ x,y\in \mathbb R$.
	\flushright \href{https://artofproblemsolving.com/community/c6h272194}{(Link to AoPS)}
\end{problem}



\begin{solution}[by \href{https://artofproblemsolving.com/community/user/53798}{mathema*}]
	Hi :) !!!

$ f(f(x) + y) = 2x + f(f(y) - x)$    $ (E)$

$ Let \ \ \ f(0) = a$

than we put: $ x = 0 \ \ \ and \ \ \ y = x$

than: $ f(a + x) = f(f(x))$ ...  $ (1)$

we put in $ (E)$ :$ y = 0$ than we have: $ f(f(x)) = 2x + f(a - x)$...$ (2)$

of $ (1)$ and $ (2)$:

$ f(a + x) - f(a - x) = 2x \Leftrightarrow f(a + x) - (a + x) = f(a - x) - (a - x)$(3)

we put $ g(x) = f(x) - x$ than  into (3) we have: $ g(a + x) = g(a - x)$

than:

$ f(x) = x + h(x)$ with $ h$ is all function axial symetrique where $ (D): x = f(0) = a$ is the $ axe$ $ of$ $ symetrie$ ($ i.e$ $ h(2f(0) - x) = h(x))$  

than the only function $ h$ verified $ (E)$ is the fonction constante $ h(x) = b$

finaly:
$ f(x) = x+b$ 
PS: i edited my poste sorry because i forgot the onstante function
\end{solution}



\begin{solution}[by \href{https://artofproblemsolving.com/community/user/44083}{jgnr}]
	\begin{tcolorbox}
$ f(x) = x + h(x)$ with $ h$ is all function axial symetrique where $ (D): x = f(0) = a$ is the $ axe$ $ of$ $ symetrie$ ($ i.e$ $ h(2f(0) - x) = h(x))$  

than the only function $ h$ verified $ (E)$ is the fonction nul $ h(x) = 0$
\end{tcolorbox}I don't understand how do you deduce the $ h(x)=0$. Can you explain?
\end{solution}



\begin{solution}[by \href{https://artofproblemsolving.com/community/user/53798}{mathema*}]
	\begin{tcolorbox}[quote="mathema*"]
$ f(x) = x + h(x)$ with $ h$ is all function axial symetrique where $ (D): x = f(0) = a$ is the $ axe$ $ of$ $ symetrie$ ($ i.e$ $ h(2f(0) - x) = h(x))$  

than the only function $ h$ verified $ (E)$ is the fonction nul $ h(x) = 0$
\end{tcolorbox}I don't understand how do you deduce the $ h(x) = 0$. Can you explain?\end{tcolorbox}


sorry i edited this see my answer:
OK if we replace f into original equation funtional we find that $ h(x) = b$ !!!!
\end{solution}



\begin{solution}[by \href{https://artofproblemsolving.com/community/user/29428}{pco}]
	\begin{tcolorbox}[quote="Johan Gunardi"]\begin{tcolorbox}
$ f(x) = x + h(x)$ with $ h$ is all function axial symetrique where $ (D): x = f(0) = a$ is the $ axe$ $ of$ $ symetrie$ ($ i.e$ $ h(2f(0) - x) = h(x))$  

than the only function $ h$ verified $ (E)$ is the fonction nul $ h(x) = 0$
\end{tcolorbox}I don't understand how do you deduce the $ h(x) = 0$. Can you explain?\end{tcolorbox}

OK if we replace f into original equation funtional we find that $ h(x) = 0$ !!!!\end{tcolorbox}

I dont think so. If you replace f by $ x + h(x)$ in the original equation, you find :
$ x+h(x)+y+h(x+h(x)+y)=2x+y+h(y)-x+h(y+h(y)-x)$

So:
$ h(x)+h(x+h(x)+y)=h(y)+h(y+h(y)-x)$

And not really $ h(x)=0$ !
\end{solution}



\begin{solution}[by \href{https://artofproblemsolving.com/community/user/29428}{pco}]
	\begin{tcolorbox}Find all $ f: R\rightarrow R$ such that $ : f(f(x) + y) = 2x + f(f(y) - x)$  $ \forall x,y\in R$\end{tcolorbox}

Let $ P(x,y)$ be the equation : $ f(f(x) + y) = 2x + f(f(y) - x)$

$ P(\frac {f(0) - x}{2}, - f(\frac {f(0) - x}{2}))$ $ \implies$ $ f(0) = f(0) - x + f(something)$ and so $ f(x)$ is a surjection.

Since $ f(x)$ is surjective, for x given, let then $ c$ such that $ f(c) = 0$ and $ u$ such that $ f(u) = x + c$

Then $ P( c,u)$ $ \implies$ $ x + c = 2c + f(x)$ and so $ f(x) = x - c$

And it's easy to verify that this necessary condition is sufficient.

The solutions are $ f(x) = x - c$ for any c.
\end{solution}



\begin{solution}[by \href{https://artofproblemsolving.com/community/user/53798}{mathema*}]
	\begin{tcolorbox}\end{tcolorbox}[quote="mathema*"][quote="Johan Gunardi"][quote="mathema*"]
$ f(x) = x + h(x)$ with $ h$ is all function axial symetrique where $ (D): x = f(0) = a$ is the $ axe$ $ of$ $ symetrie$ ($ i.e$ $ h(2f(0) - x) = h(x))$  

than the only function $ h$ verified $ (E)$ is the fonction nul $ h(x) = 0$
\end{tcolorbox}I don't understand how do you deduce the $ h(x) = 0$. Can you explain?\end{tcolorbox}

OK if we replace f into original equation funtional we find that $ h(x) = 0$ !!!!\end{tcolorbox}\begin{tcolorbox}

I dont think so. If you replace f by $ x + h(x)$ in the original equation, you find :
$ x + h(x) + y + h(x + h(x) + y) = 2x + y + h(y) - x + h(y + h(y) - x)$

So:
$ h(x) + h(x + h(x) + y) = h(y) + h(y + h(y) - x)$

And not really $ h(x) = 0$ !\end{tcolorbox}

yes we can find the constant fuction by resolution of this equation functional!!!
\end{solution}



\begin{solution}[by \href{https://artofproblemsolving.com/community/user/29428}{pco}]
	\begin{tcolorbox}$ h(x) + h(x + h(x) + y) = h(y) + h(y + h(y) - x)$

yes we can find the constant fuction by resolution of this equation functional!!!\end{tcolorbox}

Please, show us.
\end{solution}



\begin{solution}[by \href{https://artofproblemsolving.com/community/user/29876}{ozgurkircak}]
	Here is another solution:
$ y=-f(x)$ $ \Longrightarrow$ $ f(0)=2x+f(f(-f(x))-x)$ which means that f is surjective.
So there is $ x_0$ s.t $ f(x_0)=0.$
$ x=x_0$ $ \Longrightarrow$ $ f(y)=2x_0+f(f(y)-x_0)$ or
$ f(f(y)-x_0)=(f(y)-x_0)-x_0$ Since f is surjective $ f(y)-x_0$ spans all real numbers.
So $ f(x)=x-x_0$ and that satisfies the original equation.
\end{solution}



\begin{solution}[by \href{https://artofproblemsolving.com/community/user/29428}{pco}]
	\begin{tcolorbox}Here is another solution:
$ y = - f(x)$ $ \Longrightarrow$ $ f(0) = 2x + f(f( - f(x)) - x)$ which means that f is surjective.
So there is $ x_0$ s.t $ f(x_0) = 0.$
$ x = x_0$ $ \Longrightarrow$ $ f(y) = 2x_0 + f(f(y) - x_0)$ or
$ f(f(y) - x_0) = (f(y) - x_0) - x_0$ Since f is surjective $ f(y) - x_0$ spans all real numbers.
So $ f(x) = x - x_0$ and that satisfies the original equation.\end{tcolorbox}

Yes, quite good.

But it's very near from mine :)
\end{solution}



\begin{solution}[by \href{https://artofproblemsolving.com/community/user/44674}{Allnames}]
	It is a well-known  problem which was proposed in IMO shortlist 2002 (problem A1,page 21). See the link http://www.mathlinks.ro/Forum/viewtopic.php?t=15588
The main idea is proving that $ f$ is surjective.
\end{solution}
*******************************************************************************
-------------------------------------------------------------------------------

\begin{problem}[Posted by \href{https://artofproblemsolving.com/community/user/60652}{DCTPKTCSPHN}]
	Find all $f: \mathbb Z \to \mathbb Z$ such that $f(x+y+f(y))=f(x)+2y$ for all $ x,y\in \mathbb Z$.
	\flushright \href{https://artofproblemsolving.com/community/c6h272196}{(Link to AoPS)}
\end{problem}



\begin{solution}[by \href{https://artofproblemsolving.com/community/user/53798}{mathema*}]
	Hi :) !!!

$ f(x + y + f(y)) = f(x) + 2y$ $ (E)$

$ let$ $ f(0) = a \in \mathbb{Z}$

than $ let$ $ y = 0$:

$ f(x + a) = f(x)$ 

put $ x = - a$ we find $ f(0) = f( - a) = a$ 

in $ (E)$ put $ x = a$ and $ y = - a$ we find $ f(a - a + f( - a)) = f(a) - 2a %Error. "Rightlong" is a bad command.
f(a) = f(a) - 2a \Rightarrow a = 0$

than $ f(0) = 0$ we returd to $ (E)$:

$ let \ \ \ x = 0$:

$ (E)$ become's:

$ f(y + f(y)) = 2y$ 

$ f$ are bijective

$ let$ $ g(y) = f(y) + y$ than $ g(g(y)) = f(g(y)) + g(y) = 2y + g(y) = 3y + f(y) = 2y + g(y)$

than $ g^{[n]}(y) = \frac {(2)^n - ( - 1)^n}{3}f(y) + \frac {2^{n + 1} - ( - 1)^{n + 1}}{3}y$

than .....etc

in conclusion we can find: 

$ f(x) = x$ and $ f(x) = - 2x$ $ \forall x \in \mathbb{Z}$
\end{solution}



\begin{solution}[by \href{https://artofproblemsolving.com/community/user/29428}{pco}]
	\begin{tcolorbox}Find all $ f: Z\rightarrow Z$ such that $ : f(x + y + f(y)) = f(x) + 2y$  $ \forall x,y\in Z$\end{tcolorbox}
Let $ P(x,y)$ be the equation $ f(x+y+f(y))=f(x)+2y$

Let $ a\in\mathbb{Z}$, let $ u=a+f(a)$ and $ v=2a$. We have $ f(x+u)=f(x)+v$ and so, with easy induction, $ f(x+nu)=f(x)+nv$ $ \forall n\in\mathbb{Z}$
For $ n=v$, we get $ f(x+uv)=f(x)+v^2$

Then, $ P(x-v,y+u)$ $ \implies$ $ f(x-v+y+u+f(y+u))=f(x-v)+2y+2u$, and so $ f(x+y+f(y))+v=f(x-v)+2y+2u$

So we have : $ f(x+y+f(y))=f(x-v)+2y+2u-v$
But : $ f(x+y+f(y))=f(x)+2y$ 
So : $ f(x)+2y = f(x-v)+2y+2u-v$
So : $ f(x)=f(x-v)+2u-v$ and so, with easy induction : $ f(x+nv)=f(x)+n(2u-v)$ $ \forall n\in\mathbb{Z}$
For $ n=u$, we get $ f(x+uv)=f(x)+u(2u-v)$

But : $ f(x+uv)=f(x)+v^2$

So $ v^2=u(2u-v)$

So $ 4a^2=2(a+f(a))f(a)$

So $ (f(a)-a)(f(a)+2a)=0$ $ \forall a\in\mathbb{Z}$ and so either $ f(a)=a$, either $ f(a)=-2a$

So, $ \forall x$, either $ f(x)=x$, either $ f(x)=-2x$

Assume now that there exist $ x\neq 0$ such that $ f(x)=x$ and $ y\neq 0$ such that $ f(y)=-2y$.

$ P(x,y)$ $ \implies$ $ f(x-y)=x+2y$
If $ f(x-y)=x-y$, we get $ x-y=x+2y$ and so $ y=0$
If $ f(x-y)=-2(x-y)$, we get $ -2x+2y=x+2y$ and so $ x=0$

So, either $ f(x)=x$ $ \forall x$, either $ f(x)=-2x$ $ \forall x$

And it's easy to verify that these two expressions both are solutions of the initial equation.
\end{solution}



\begin{solution}[by \href{https://artofproblemsolving.com/community/user/60652}{DCTPKTCSPHN}]
	\begin{tcolorbox}[quote="DCTPKTCSPHN"]Find all $ f: Z\rightarrow Z$ such that $ : f(x + y + f(y)) = f(x) + 2y$  $ \forall x,y\in Z$\end{tcolorbox}
Let $ P(x,y)$ be the equation $ f(x + y + f(y)) = f(x) + 2y$

Let $ a\in\mathbb{Z}$, let $ u = a + f(a)$ and $ v = 2a$. We have $ f(x + u) = f(x) + v$ and so, with easy induction, $ f(x + nu) = f(x) + nv$ $ \forall n\in\mathbb{Z}$
For $ n = v$, we get $ f(x + uv) = f(x) + v^2$

Then, $ P(x - v,y + u)$ $ \implies$ $ f(x - v + y + u + f(y + u)) = f(x - v) + 2y + 2u$, and so $ f(x + y + f(y)) + v = f(x - v) + 2y + 2u$

So we have : $ f(x + y + f(y)) = f(x - v) + 2y + 2u - v$
But : $ f(x + y + f(y)) = f(x) + 2y$ 
So : $ f(x) + 2y = f(x - v) + 2y + 2u - v$
So : $ f(x) = f(x - v) + 2u - v$ and so, with easy induction : $ f(x + nv) = f(x) + n(2u - v)$ $ \forall n\in\mathbb{Z}$
For $ n = u$, we get $ f(x + uv) = f(x) + u(2u - v)$

But : $ f(x + uv) = f(x) + v^2$

So $ v^2 = u(2u - v)$

So $ 4a^2 = 2(a + f(a))f(a)$

So $ (f(a) - a)(f(a) + 2a) = 0$ $ \forall a\in\mathbb{Z}$ and so either $ f(a) = a$, either $ f(a) = - 2a$

So, $ \forall x$, either $ f(x) = x$, either $ f(x) = - 2x$

Assume now that there exist $ x\neq 0$ such that $ f(x) = x$ and $ y\neq 0$ such that $ f(y) = - 2y$.

$ P(x,y)$ $ \implies$ $ f(x - y) = x + 2y$
If $ f(x - y) = x - y$, we get $ x - y = x + 2y$ and so $ y = 0$
If $ f(x - y) = - 2(x - y)$, we get $ - 2x + 2y = x + 2y$ and so $ x = 0$

So, either $ f(x) = x$ $ \forall x$, either $ f(x) = - 2x$ $ \forall x$

And it's easy to verify that these two expressions both are solutions of the initial equation.\end{tcolorbox}
Pco: you have nice solution ,I agree with you :)
\end{solution}
*******************************************************************************
-------------------------------------------------------------------------------

\begin{problem}[Posted by \href{https://artofproblemsolving.com/community/user/46840}{behdad.math.math}]
	Find all functions $ f : \mathbb R \longrightarrow \mathbb R$ such that for every $ x\in \mathbb R$,
\[ f\left(\frac{1+x}{1-x}\right) = af(x),\]
where $a$ is a real constant.
	\flushright \href{https://artofproblemsolving.com/community/c6h272552}{(Link to AoPS)}
\end{problem}



\begin{solution}[by \href{https://artofproblemsolving.com/community/user/29428}{pco}]
	\begin{tcolorbox}you are right dear pco.I saw this problem in a book and now i am sure that this problem is wrong.

but if $ f : R \longrightarrow R$ ,then?\end{tcolorbox}
Let $ E(x)$ be the statement $ f(\frac {1 + x}{1 - x}) = af(x)$
Obviously, this statement can not be verified for $ x = 1$. I'll consider that it is verified for any other $ x$.

Let $ g(x) = \frac {1 + x}{1 - x}$. We have : $ g(g(g(g(x)))) = x$

So :

1) If $ a = 0$, we have the trivial solution $ f( - 1) = c$ and $ f(x) = 0$ $ \forall x\neq - 1\in\mathbb{R}$

2) If $ a\neq 0$, we have $ f(g(g(g(g(x)))) = a^4f(x) = f(x)$ And so :

2.1) Either $ f(x) = 0$ $ \forall x\notin\{ - 1,0, + 1\}$ and we have the trivial solution :
$ f( - 1) = c$
$ f(0) = ac$
$ f( + 1) = a^2c$
$ f(x) = 0$ $ \forall x\notin\{ - 1,0, + 1\}$

2.2) Either $ f(u)\neq 0$ for some $ u\notin\{ - 1,0, + 1\}$ and so we have $ a = 1$ or $ a = - 1$
$ g(x)$ has the following other properties :

$ g(]0, + 1[) = ] + 1, + \infty[$
$ g(] + 1, + \infty[) = ] - \infty, - 1[$
$ g(] - \infty, - 1[) = ] - 1,0[$
$ g(] - 1,0[) = ]0, + 1[$

So we can easily build solutions $ f(x)$:

Let $ h(x)$ be any real function defined in $ (0,1)$

$ \forall x\in(0,1)$ $ f(x) = h(x)$
$ \forall x\in(1, + \infty)$ $ f(x) = ah(\frac {x - 1}{x + 1})$
$ \forall x\in( - \infty, - 1)$ $ f(x) = h( - \frac {1}{x})$
$ \forall x\in( - 1,0)$ $ f(x) = ah(\frac {1 + x}{1 - x})$

$ f( - 1) = c$
$ f(0) = ac$
$ f( + 1) = c$
And it's easy to verify that this mandatory form is a solution.

===================================
Synthesis :

Since cases 1 and 2.1 are the same, we have :

S1) For any value $ a$, we have the solution :
$ f( - 1) = c$
$ f(0) = ac$
$ f( + 1) = a^2c$
$ f(x) = 0$ $ \forall x\notin\{ - 1,0, + 1\}$

S2) For $ a = 1$ or $ a = - 1$, we have the solutions :
Let $ h(x)$ be any real function defined in $ (0,1)$

$ \forall x\in(0,1)$ $ f(x) = h(x)$
$ \forall x\in(1, + \infty)$ $ f(x) = ah(\frac {x - 1}{x + 1})$
$ \forall x\in( - \infty, - 1)$ $ f(x) = h( - \frac {1}{x})$
$ \forall x\in( - 1,0)$ $ f(x) = ah(\frac {1 + x}{1 - x})$

$ f( - 1) = c$
$ f(0) = ac$
$ f( + 1) = a^2c = c$
\end{solution}
*******************************************************************************
-------------------------------------------------------------------------------

\begin{problem}[Posted by \href{https://artofproblemsolving.com/community/user/46840}{behdad.math.math}]
	Find all functions $ h : \mathbb Z \longrightarrow \mathbb Z$ that for every $ x,y \in \mathbb Z$,
\[h(x+y) + h(xy)= h(x)h(y) = 1.\]
	\flushright \href{https://artofproblemsolving.com/community/c6h272555}{(Link to AoPS)}
\end{problem}



\begin{solution}[by \href{https://artofproblemsolving.com/community/user/29428}{pco}]
	\begin{tcolorbox}Find all functions $ h : Z \longrightarrow Z$ that for every $ x,y \in R$ :

$ h(x + y) + h(xy)$ $ =$ $ h(x)h(y) + 1$\end{tcolorbox}

Try to be careful on your statements : If $ h(x)$ is only defined on $ \mathbb{Z}$, the equation cannot be defined "for every $ x,y \in R$ "
\end{solution}



\begin{solution}[by \href{https://artofproblemsolving.com/community/user/29428}{pco}]
	\begin{tcolorbox}Find all functions $ h : Z \longrightarrow Z$ that for every $ x,y \in R$ :

$ h(x + y) + h(xy)$ $ =$ $ h(x)h(y) + 1$\end{tcolorbox}

I will modify the problem as :

Find all functions $ h(x)$ : $ \mathbb{Z}\longrightarrow\mathbb{Z}$ such that $ E(x,y)$ : $ h(x + y) + h(xy) = h(x)h(y) + 1$ is true $ \forall x,y\in\mathbb{Z}$

$ E(0,0)$ $ \implies$ $ h(0) = 1$

$ E(1, - 1)$ $ \implies$ $ h( - 1)(h(1) - 1) = 0$ and so either $ h(1) = 1$, either $ h( - 1) = 0$

1) $ h(1) = 1$
$ E(x - 1,1)$ $ \implies$ $ h(x) = 1$ and we can verify that this is a solution

2) $ h(1)\neq 1$ and $ h( - 1) = 0$
$ E( - 2,1)$ $ \implies$ $ h( - 2)(1 - h(1)) = 1$ and so $ 1 - h(1)$ divides 1 and so $ h(1) = 0$ or $ h(1) = 2$

2.1) $ h(1) = 2$
$ E(x,1)$ $ \implies$ $ h(x + 1) = h(x) + 1$ and so $ h(x) = x + 1$ and we can verify that this is a solution.

2.2) $ h(1) = 0$
$ E(x,1)$ $ \implies$ $ h(x + 1) = 1 - h(x)$ and so $ h(x + 2) = h(x)$ and so $ h(x) = 0$ $ \forall x$ odd and $ h(x) = 1$ $ \forall x$ even and we can verify that this is a solution.

3) synthesis:
We have 3 solutions :
$ h(x) = 1$
$ h(x) = x + 1$
$ h(x) = 0$ $ \forall x$ odd and $ h(x) = 1$ $ \forall x$ even
\end{solution}
*******************************************************************************
-------------------------------------------------------------------------------

\begin{problem}[Posted by \href{https://artofproblemsolving.com/community/user/57717}{linda2005}]
	i) Find all functions $ f$ continuous on $ \mathbb{R}$ such that :

\[ f ( f ( f (x))) + f (x) = 2x, \;\forall x \in \mathbb{R}.\]
ii) Find all continuous functions $ f : \mathbb{R} \to \mathbb{R}$ such that :
\[ 9 f (8x) - 9 f (4x) + 2 f (2x) = 100x, \;\forall x \in \mathbb{R}.\]
iii) Find all the functions $ f : \mathbb{Z} \to \mathbb{Z}$ satisfying the following conditions:
[color=white]....[\/color]\begin{bolded}-\end{bolded} $ f \big( f (m - n)\big) = f (m^2) + f (n) - 2n f (m)$,[color=white]... [\/color]$ \forall m, n \in \mathbb{Z}$.
[color=white]....[\/color]\begin{bolded}-\end{bolded} $ f (1) > 0$.
	\flushright \href{https://artofproblemsolving.com/community/c6h272657}{(Link to AoPS)}
\end{problem}



\begin{solution}[by \href{https://artofproblemsolving.com/community/user/29428}{pco}]
	\begin{tcolorbox}  Find all functions $ f$ continuous on $ \mathbb{R}$ such that :
\[ f ( f ( f (x))) + f (x) = 2x, \;\forall x \in \mathbb{R}.
\]\end{tcolorbox}

$ f(x_1)=f(x_2)$ $ \implies$ $ f(f(f(x_1)))+f(x_1)=f(f(f(x_2)))+f(x_2)$ and so $ x_1=x_2$.  So $ f(x)$ is injective, so monotonous, since continuous.

If $ f(x)$ is decreasing, $ f(f(x))$ is increasing, $ f(f(f(x)))$ decreasing and $ f(f(f(x)))+f(x)$ decreasing, which is impossible ($ 2x$ is increasing). So $ f(x)$ is increasing.

Then, if $ f(a)<a$, so $ f(f(a))<f(a)<a$ and then $ f(f(f(a)))<f(f(a))<a$ and $ f(f(f(a)))+f(a)<2a$, which is impossible.
Then, if $ f(a)>a$, so $ f(f(a))>f(a)>a$ and then $ f(f(f(a)))>f(f(a))>a$ and $ f(f(f(a)))+f(a)>2a$, which is impossible.

So $ f(a)=a$ and the only solution is $ f(x)=x$. And we can immediatly verify that this solution is OK.
\end{solution}



\begin{solution}[by \href{https://artofproblemsolving.com/community/user/29428}{pco}]
	\begin{tcolorbox} ii) Find all continuous functions $ f : \mathbb{R} \to \mathbb{R}$ such that :
\[ 9 f (8x) - 9 f (4x) + 2 f (2x) = 100x, \;\forall x \in \mathbb{R}.
\]\end{tcolorbox}

The equation may be written $ 9(f(8x)-\frac{5}{2}(8x))-9(f(4x)-\frac{5}{2}(4x))+2(f(2x)-\frac{5}{2}(2x))=0$

We can then write $ g(x)=f(x)-\frac{5x}{2}$ and we have $ 9g(8x)-9g(4x)+2g(2x)=0$, or also :

$ g(2x)=\frac{9}{2}(g(4x)-g(8x))$

So, we can create the sequence :
$ u_0=g(a)$
$ u_1=g(\frac{a}{2})$
$ u_{n+2}=\frac{9}{2}(u_{n+1}-u_n)$

Obviously, $ u_n=g(\frac{a}{2^n})$

And we have $ u_n=\alpha 3^n + \beta (\frac{3}{2})^n$ for some $ \alpha$ and $ \beta$

But, when $ n\longrightarrow +\infty$, this expression may have a finite limite ($ g(0)$, since $ f(x)$ and $ g(x)$ are continuous) only if $ \alpha=\beta=0$

So $ g(x)=0$ and $ f(x)=\frac{5x}{2}$ and we can easily verify that this solution works.
\end{solution}



\begin{solution}[by \href{https://artofproblemsolving.com/community/user/29428}{pco}]
	\begin{tcolorbox} iii) Find all the functions $ f : \mathbb{Z} \to \mathbb{Z}$ satisfying the following conditions:
[color=white]....[\/color]\begin{bolded}-\end{bolded} $ f \big( f (m - n)\big) = f (m^2) + f (n) - 2n f (m)$,[color=white]... [\/color]$ \forall m, n \in \mathbb{Z}$.
[color=white]....[\/color]\begin{bolded}-\end{bolded} $ f (1) > 0$.\end{tcolorbox}

According to me, there are no solutions to this equation :

Let $ P(m,n)$ be $ f(f(m-n))=f(m^2)+f(n)-2nf(m)$

Let $ f(0)=a$. $ P(0,0)$ gives $ f(a)=2a$

$ P(n,0)$ $ \implies$ $ f(f(n))=f(n^2)+a$ 
$ P(0,-n)$ $ \implies$ $ f(f(n))=a+f(-n)+2an$

And so $ f(n^2)=f(-n)+2an$. Putting this in $ P(x,y)$, we get :

$ Q(m,n)$ : $ f(f(m-n))=f(-m)+2am+f(n)-2nf(m)$

$ Q(n,n)$ $ \implies$ $ 2a=f(-n)+2an-(2n-1)f(n)$ and so $ f(-n)=-2a(n-1)+(2n-1)f(n)$. Putting this in $ Q(m,n)$, we get :

$ R(m,n)$ : $ f(f(m-n))=2a+(2m-2n-1)f(m)+f(n)$

Then $ R(n+1,n)$ $ \implies$ $ f(f(1))=2a+f(n+1)+f(n)$ and so $ f(n+1)=k-f(n)$ and $ f(n+2)=f(n)$
So $ f(n)$ would take only two values but $ P(1,n)$ would be $ f(f(1-n))=f(1)+f(n)-2nf(1)$, so $ 2nf(1)=-f(f(1-n))+f(1)+f(n)$

And this last equation is impossible since left part is unbounded ($ f(1)>0$) and right part is bounded (since $ f(x)$ can take only two values).
\end{solution}



\begin{solution}[by \href{https://artofproblemsolving.com/community/user/15024}{Farenhajt}]
	[hide="Problem 3"]Putting $ m = 0, n = 0$ into the equation, we get $ f(f(0)) = 2f(0)$

Putting $ m = 1, n = 1$ we get $ f(f(0)) = 0$

Hence $ 0 = f(f(0)) = 2f(0)\implies f(0) = 0$

Putting $ n = m$ we get $ 0 = f(m^2) + f(m) - 2mf(m)\quad (1)$

Putting $ n = m - 1$ we get $ f(f(1)) = f(m^2) + f(m - 1) - 2(m - 1)f(m)\quad(2)$

Subtracting $ (2)$ from $ (1)$, with notation $ D: = f(f(1))$, we get

$ - D = f(m) - f(m - 1) - 2f(m)\iff f(m) + f(m - 1) - D = 0$

Hence $ f(m - 1) + f(m - 2) - D = 0$, so $ f(m) = f(m - 2)$

Therefore $ f(m) = \{\begin{matrix}0 & \mathrm{m\,\, even} \\
C & \mathrm{m\,\, odd}\end{matrix}$ for some integer $ C$.

But if we put $ m$ odd, $ n$ odd in the initial equation, we get $ 0 = 2C(1 - n)$ which is impossible as the condition $ f(1) > 0$ yields $ C>0$.

Therefore no such function exists.[\/hide]
\end{solution}
*******************************************************************************
-------------------------------------------------------------------------------

\begin{problem}[Posted by \href{https://artofproblemsolving.com/community/user/51029}{mathVNpro}]
	Find all functions $ f: \mathbb R^2 \to \mathbb R$ such that it qualifies both conditions below:
i)  $ f(x,y)+f(y,z)+f(z,x)=x^2+y^2+z^2$, for all reals $ x,y,z$, and
ii) $ f(x^2+y-f(0,x),0)=x+f^2(y,0)-f(0,y^2-f(0,y))$, for all reals $ x,y$.
	\flushright \href{https://artofproblemsolving.com/community/c6h272732}{(Link to AoPS)}
\end{problem}



\begin{solution}[by \href{https://artofproblemsolving.com/community/user/29428}{pco}]
	\begin{tcolorbox}Find all fuction $ f(x,y)$ such that it qualifies both condition below:
i)  $ f(x,y) + f(y,z) + f(z,x) = x^2 + y^2 + z^2$, for all $ x,y,z$ real numbers
ii) $ f(x^2 + y - f(0,x),0) = x + f^2(y,0) - f(0,y^2 - f(0,y))$, for all $ x,y$ real numbers\end{tcolorbox}

Let $ g(x,y)=f(x,y)-y^2$

The two conditions may be writen :
A1(x,y,z) : $ g(x,y)+g(y,z)+g(z,x)=0$
A2(x,y) : $ g(y-g(0,x),0)=x+g^2(y,0)-g(0,-g(0,y))-g^2(0,y)$

A1(0,0,0) $ \implies$ $ g(0,0)=0$
A1(x,0,0) $ \implies$ $ g(x,0)=-g(0,x)$

Using this last property in A2, we get :
A3(x,y) : $ g(y+g(x,0),0)=x+g(g(y,0),0)$

Let then $ h(x)=g(x,0)$. A3 becomes :
A4(x,y) : $ h(y+h(x))=x+h(h(y))$ with $ h(0)=0$

A4(x,0) $ \implies$ $ h(h(x))=x$
A4(0,x) $ \implies$ $ h(x)=h(h(x))$ 

So $ h(x)=x$ and $ g(x,0)=x$

A1(x,y,0) $ \implies$ $ g(x,y)+g(y,0)+g(0,x)=0$, so $ g(x,y)+g(y,0)-g(x,0)=0$, and so $ g(x,y)=x-y$

And so $ f(x,y)=x-y+y^2$

And it is easy to verify that this necessary condition works.
\end{solution}
*******************************************************************************
-------------------------------------------------------------------------------

\begin{problem}[Posted by \href{https://artofproblemsolving.com/community/user/9882}{Virgil Nicula}]
	Does the exist a function $ f: \mathbb R^* \rightarrow \mathbb R$ so that for every $ x\ne 0$, we have
\[ f(f(x))=-\frac 1x \, ?\]
	\flushright \href{https://artofproblemsolving.com/community/c6h272801}{(Link to AoPS)}
\end{problem}



\begin{solution}[by \href{https://artofproblemsolving.com/community/user/29428}{pco}]
	\begin{tcolorbox}Isn't integer part ...\end{tcolorbox}
Ok, so $ f(f(x))=-\frac{1}{x}$ $ \forall x\neq 0$

We have $ f(f(f(f(x))))=x$ and so we just have to organize all non null reals in sequences $ \{a,b,-\frac{1}{a},-\frac{1}{b}\}$

Here is a general solution :

Let $ \mathbb{A}$ and $ \mathbb{B}$ two equipotent subsets of $ \mathbb{R^*^+}$ such that $ \mathbb{A}\cup\mathbb{B}=\mathbb{R^*^+}$ and $ \mathbb{A}\cap\mathbb{B}=\emptyset$

Let $ \mathbb{C}=\{x$ such that $ -\frac{1}{x}\in\mathbb{A}\}$
Let $ \mathbb{D}=\{x$ such that $ -\frac{1}{x}\in\mathbb{B}\}$

Let $ h(x)$ any bijective function from $ \mathbb{A}\longrightarrow\mathbb{B}$
Let $ k(x)$ the reciprocal of $ h(x)$
Let $ e(x)$ any function from $ \mathbb{A}\longrightarrow\{-1,+1\}$

All solutions are of the form :

$ \forall x\in\mathbb{A}$ : $ f(x)=e(x)h(x)^{e(x)}$ (to be clear : either $ h(x)$, either $ -\frac{1}{h(x)}$)
$ \forall x\in\mathbb{B}$ : $ f(x)=-e(k(x))k(x)^{-e(k(x))}$
$ \forall x\in\mathbb{C}$ : $ f(x)=-e(-\frac{1}{x})h(-\frac{1}{x})^{-e(-\frac{1}{x})}$
$ \forall x\in\mathbb{D}$ : $ f(x)=e(k(-\frac{1}{x}))k(-\frac{1}{x})^{e(k(-\frac{1}{x}))}$
\end{solution}



\begin{solution}[by \href{https://artofproblemsolving.com/community/user/9882}{Virgil Nicula}]
	Thanks. Please, can you present us a concret example ? Thank you.
\end{solution}



\begin{solution}[by \href{https://artofproblemsolving.com/community/user/29428}{pco}]
	\begin{tcolorbox}Thanks. Please, can you present us a concret example ? Thank you.\end{tcolorbox}

Ok :

Let $ \mathbb{A}=\{x>0$ such that $ [-x]$ is odd$ \}$  (where $ [.]$ is the integer part).
Let $ \mathbb{B}=\{x>0$ such that $ [-x]$ is even$ \}$
Let $ h(x)=x+1$
Let $ k(x)=x-1$
Let $ e(x)=+1$

Then :

$ \forall x>0$ such that $ [-x]$ is odd, $ f(x)=x+1$

$ \forall x>0$ such that $ [-x]$ is even, $ f(x)=-\frac{1}{x-1}$

$ \forall x<0$ such that $ [\frac{1}{x}]$ is odd, $ f(x)=\frac{x}{1-x}$

$ \forall x<0$ such that $ [\frac{1}{x}]$ is even, $ f(x)=-\frac{x+1}{x}$

Verification :

1) $ \forall x>0$ such that $ [-x]$ is odd :
$ f(x)=x+1$ and $ [-f(x)]=[-x-1]=[-x]-1$. 
So $ f(x)>0$ and $ [-f(x)]$ is even.
So $ f(f(x))=-\frac{1}{f(x)-1}=-\frac{1}{x}$


2) $ \forall x>0$ such that $ [-x]$ is even :
$ f(x)=-\frac{1}{x-1}$
So $ f(x)<0$ ($ [-x]$ even implies $ x>1$) and $ [\frac{1}{f(x)}]=[1-x]=1+[-x]$ is odd.
So $ f(f(x))=\frac{f(x)}{1-f(x)}=\frac{-\frac{1}{x-1}}{1+\frac{1}{x-1}}=-\frac{1}{x}$


3) $ \forall x<0$ such that $ [\frac{1}{x}]$ is odd
$ f(x)=\frac{x}{1-x}$
So $ f(x)<0$ and $ [\frac{1}{f(x)}]=[\frac{1-x}{x}]=[\frac{1}{x}]-1$ is even
So $ f(f(x))=-\frac{f(x)+1}{f(x)}=-\frac{\frac{x}{1-x}+1}{\frac{x}{1-x}}=-\frac{1}{x}$


4) $ \forall x<0$ such that $ [\frac{1}{x}]$ is even
$ f(x)=-\frac{x+1}{x}$
So $ f(x)>0$ ($ [\frac{1}{x}]$ even with $ x<0$ means $ \frac{1}{x}<-1$) and $ [-f(x)]=1+[\frac{1}{x}]$ is odd.
So $ f(f(x))=f(x)+1=-\frac{1}{x}$

Q.E.D.
\end{solution}



\begin{solution}[by \href{https://artofproblemsolving.com/community/user/29428}{pco}]
	\begin{tcolorbox}Thanks. Please, can you present us a concret example ? Thank you.\end{tcolorbox}

Here is another one, simpler than the previous, I think

Let $ \mathbb{A}=]0,1[$.
Let $ \mathbb{B}=[1,+\infty[$
Let $ h(x)=\frac{1}{2x}$ if $ x=2^{-n}$ for some integer $ n\geq 1$, and $ h(x)=\frac{1}{x}$ else.
Let $ e(x)=+1$

Then :

$ \forall x\in]0,1[$ such that $ x=2^{-n}$ for some integer $ n\geq 1$, $ f(x)=\frac{1}{2x}$

$ \forall x\in]0,1[$ such that $ x\neq 2^{-n}$ for all integers $ n\geq 1$, $ f(x)=\frac{1}{x}$

$ \forall x\in[1,+\infty[$ such that $ x=2^n$ for some integer $ n\geq 0$, $ f(x)=-2x$

$ \forall x\in[1,+\infty[$ such that $ x\neq 2^n$ for any integer $ n\geq 0$, $ f(x)=-x$

$ \forall x\in]-\infty,-1[$ such that $ x=-2^n$ for some integer $ n\geq 1$, $ f(x)=\frac{2}{x}$

$ \forall x\in]-\infty,-1[$ such that $ x\neq -2^n$ for any integer $ n\geq 1$, $ f(x)=\frac{1}{x}$

$ \forall x\in[-1,0[$ such that $ x=-2^{-n}$ for some integer $ n\geq 0$, $ f(x)=-\frac{x}{2}$

$ \forall x\in[-1,0[$ such that $ x\neq -2^{-n}$ for any integer $ n\geq 0$, $ f(x)=-x$

Verification is nearly immediate (I'll give it if some reader encounters problem)
\end{solution}



\begin{solution}[by \href{https://artofproblemsolving.com/community/user/9882}{Virgil Nicula}]
	Thank you, \begin{bolded}Patrick\end{bolded}, for your effort to explain us on two nice examples.
\end{solution}
*******************************************************************************
-------------------------------------------------------------------------------

\begin{problem}[Posted by \href{https://artofproblemsolving.com/community/user/60032}{Stephen}]
	Let $P(x)= x^5+x^2+1$, and $Q(x)= x^2-2009$. The roots of $P(x)=0$ are $r_1, r_2, r_3, r_4$, and $r_5$. Find the value of \[|Q(r_1)Q(r_2)Q(r_3)Q(r_4)Q(r_5)|.\]
	\flushright \href{https://artofproblemsolving.com/community/c6h272959}{(Link to AoPS)}
\end{problem}



\begin{solution}[by \href{https://artofproblemsolving.com/community/user/29428}{pco}]
	\begin{tcolorbox}Let P(x)=$ x^5 + x^2 + 1$, and Q(x)=$ x^2 - 2009$.

The roots of P(x)=0 is r1, r2, r3, r4, and r5.

Find the value of |q(r1)q(r2)q(r3)q(r4)q(r5)|.\end{tcolorbox}

$ P(x)=(x-r_1)(x-r_2)(x-r_3)(x-r_4)(x-r_5)$

$ q(r_i)=(\sqrt{2009}-r_i)(-\sqrt{2009}-r_i)$

So $ |q(r_1)q(r_2)q(r_3)q(r_4)q(r_5)|=|P(\sqrt{2009})P(-\sqrt{2009})|$ $ =|(2010+2009^2\sqrt{2009})(2010-2009^2\sqrt{2009})|$ $ =|2010^2-2009^5|=2009^5-2010^2$
\end{solution}



\begin{solution}[by \href{https://artofproblemsolving.com/community/user/60032}{Stephen}]
	\begin{tcolorbox}[quote="Stephen"]Let P(x)=$ x^5 + x^2 + 1$, and Q(x)=$ x^2 - 2009$.

The roots of P(x)=0 is r1, r2, r3, r4, and r5.

Find the value of |q(r1)q(r2)q(r3)q(r4)q(r5)|.\end{tcolorbox}

$ P(x) = (x - r_1)(x - r_2)(x - r_3)(x - r_4)(x - r_5)$

$ q(r_i) = (\sqrt {2009} - r_i)( - \sqrt {2009} - r_i)$

So $ |q(r_1)q(r_2)q(r_3)q(r_4)q(r_5)| = |P(\sqrt {2009})P( - \sqrt {2009})|$ $ = |(2010 + 2009^2\sqrt {2009})(2010 - 2009^2\sqrt {2009})|$ $ = |2010^2 - 2009^5| = 2009^5 - 2010^2$\end{tcolorbox}

Nice solution, pco!
\end{solution}
*******************************************************************************
-------------------------------------------------------------------------------

\begin{problem}[Posted by \href{https://artofproblemsolving.com/community/user/58367}{theSA}]
	For function $ f: \mathbb{R} \to \mathbb{R}$ given that $ f(x^2 +x +3) +2 \cdot f(x^2 - 3x + 5) = 6x^2 - 10x +17$, calculate $ f(2009)$.
	\flushright \href{https://artofproblemsolving.com/community/c6h272967}{(Link to AoPS)}
\end{problem}



\begin{solution}[by \href{https://artofproblemsolving.com/community/user/29428}{pco}]
	\begin{tcolorbox}For function $ f: \mathbb{R} \to \mathbb{R}$ given that $ f(x^2 + x + 3) + 2 \cdot f(x^2 - 3x + 5) = 6x^2 - 10x + 17$. Calculate $ f(2009)$.\end{tcolorbox}

Quite nice ...

First, write $ f(x)=g(x-2009)+2x-3$. The original equation becomes :

$ g(x^2+x+3-2009)+2(x^2+x+3)-3$ $ +2(g(x^2-3x+5-2009)+2(x^2-3x+5)-3)$ $ =6x^2-10x+17$, so :

$ g(P1(x))+2g(P2(x))=0$ with $ P1(x)=x^2+x-2006$ and $ P2(x)=x^2-3x-2004$

One root of $ P_1(x)$ is $ r=\frac{-1-\sqrt{8025}}{2}$ and $ P_2(r)=4+2\sqrt{8025}=a$
One root of $ P_2(x)$ is $ s=\frac{+3+\sqrt{8025}}{2}$ and $ P_1(s)=4+2\sqrt{8025}=a$

Putting $ r$ then $ s$ in $ g(P1(x))+2g(P2(x))=0$, we get :

$ x=r$ $ \implies$ $ g(0)+2g(a)=0$
$ x=s$ $ \implies$ $ g(a)+2g(0)=0$

So $ g(0)=0$ and $ f(2009)=g(2009-2009)+2*2009-3=4015$

So, if we have at least one solution to the original equation, $ f(2009)=4015$

But we know that $ g(x)=0$ is a solution of $ g(P1(x))+2g(P2(x))=0$, so $ f(x)=2x-3$ is a solution of the original equation. So $ f(2009)$ exists and  $ f(2009)=4015$ for all the solutions.
\end{solution}



\begin{solution}[by \href{https://artofproblemsolving.com/community/user/89670}{man111}]
	If $f(x)$ is a polyniomial. then $f(x)$ must be in the form of $f(x) =ax+b$

Now put value of $f(x)$ in equation

$f(x^2+x+3)+2.f(x^2-3x+5) = 6x^2-10x+17$

$ax^2+ax+3a+b+2ax^2-6ax+10a+2b = 6x^2-10x+17$

Now equating Coeff. , We get

$a=2$ and $b=-3$

so $f(x) = ax+b = 2x-3$

so $f(2009) = 4015$
\end{solution}



\begin{solution}[by \href{https://artofproblemsolving.com/community/user/8638}{me@home}]
	man111, please reread the first line of your posting.
You addressed the case in which $f$ is a polynomial. This is not given in the problem, hence your solution is incomplete.
\end{solution}



\begin{solution}[by \href{https://artofproblemsolving.com/community/user/60529}{silvergrasshopper}]
	Because we only want to calculate the value of $f(2009)$, is it true that without loss of generality we can assume that the function $f(x)$ is a polynomial?
\end{solution}



\begin{solution}[by \href{https://artofproblemsolving.com/community/user/8638}{me@home}]
	We don't know \begin{italicized}a priori\end{italicized} that $f(2009)$ is independent of the solution $f$; this must be proven as well. Since pco has a correct solution, you can see that this statement is proven during the proof.
\end{solution}



\begin{solution}[by \href{https://artofproblemsolving.com/community/user/24037}{elim}]
	For $x = -a$ we have $f(a^2-a+3)+2f(a^2+3a+5)=6a^2+10a+17\quad(1)$ 
while $x=a+1$ then $f(a^2+3a+5)+2f(a^2-a+3)=6a^2+2a+13\;\,\quad(2)$

This means that $f(a^2-a+3) = 2(a^2-a+3)-3$ and so 
$f(t) = 2t-3\quad \left(\forall t \ge 11\/4 = \min\{a^2-a+3 \mid a\in\mathbb{R}\}\right)$

Therefore $f(2009) = 4015$
\end{solution}



\begin{solution}[by \href{https://artofproblemsolving.com/community/user/24037}{elim}]
	Based on my previous post, the general solution of the original function equation is

$f(x) =\left\{\begin{array}{ll} 2x-3,  &  x \ge 11\/4\\
g(x)  &  x < 11\/4 \end{array}
\right.\quad\quad \left(g\in \mathbb{R}^{(-\infty,11\/4)}\right)$
\end{solution}
*******************************************************************************
-------------------------------------------------------------------------------

\begin{problem}[Posted by \href{https://artofproblemsolving.com/community/user/54772}{enndb0x}]
	Let $ f: [0,1]\rightarrow \mathbb{R}$ be a function such that 
a) $ f(1)=1$,
b) $ f(x)\geq 0$ for all $ x\in [0,1]$, and
c) If $ x,y$ and $ x+y$ all lie in $ [0,1]$, then $ f(x+y)\geq f(x)+f(y)$.

Prove that $ f(x)\leq 2x , \forall x\in [0,1].$
	\flushright \href{https://artofproblemsolving.com/community/c6h273514}{(Link to AoPS)}
\end{problem}



\begin{solution}[by \href{https://artofproblemsolving.com/community/user/29428}{pco}]
	\begin{tcolorbox}It may be posted before but ...

Let $ f: [0,1]\rightarrow \mathbb{R}$ be a function such that 
$ a. f(1) = 1$
$ b.f(x)\geq 0$ for all $ x\in [0,1]$,
$ c.$if $ x,y$ and $ x + y$ all lie in $ [0,1]$,then $ f(x + y)\geq f(x) + f(y)$.
Prove that $ f(x)\leq 2x , \forall x\in [0,1].$\end{tcolorbox}

Quite nice problem ...

From (c), we get that $ f(x+0)\geq f(x)+f(0)$ and so $ f(0)=0$ since $ f(x)\geq 0$, so $ f(x)\leq 2x$ for $ x=0$

Suppose now $ \exists a\in (0,1]$ such that $ f(a)>2a$

Let $ n\in\mathbb{N}\cup\{0\}$ such that $ \frac{1}{2}\leq 2^na < 1$. Let $ b=2^na$

From (c), we get $ f(2x)\geq 2f(x)$ $ \forall x\in[0,\frac{1}{2}]$ So $ f(b)=f(2^na)\geq$ $ 2^nf(a)>2^{n+1}a=2b$.

So we have $ b\geq\frac{1}{2}$ such that $ f(b)>2b$.

Then, $ f(b)+f(1-b)\leq f(1)$ $ \implies$ $ f(1-b)\leq f(1)-f(b) < 1 - 2b \leq 0$, which is impossible.

So, such $ a$ does not exist.
\end{solution}



\begin{solution}[by \href{https://artofproblemsolving.com/community/user/54772}{enndb0x}]
	Hi,
Why $ f(2^{n}a)\geq 2^{n}f(a)$
\end{solution}



\begin{solution}[by \href{https://artofproblemsolving.com/community/user/29428}{pco}]
	\begin{tcolorbox}Hi,
Why $ f(2^{n}a)\geq 2^{n}f(a)$\end{tcolorbox}

If $ x\leq\frac{1}{2}$ we have $ f(x+x)\geq f(x)+f(x)$, and so $ f(2x)\geq 2f(x)$

If $ x\leq\frac{1}{4}$ we have $ 2x\leq\frac{1}{2}$ and so $ f(4x)=f(2(2x))\geq 2f(2x) \geq 2(2f(x))=4f(x)$

An immediate induction gives $ f(2^nx)\geq 2^nf(x)$ $ \forall x\in[0,1],n\in\mathbb{N}\cup\{0\}$ such that $ 2^nx \leq 1$
\end{solution}
*******************************************************************************
-------------------------------------------------------------------------------

\begin{problem}[Posted by \href{https://artofproblemsolving.com/community/user/45765}{popolux}]
	Find all functions $ f: \mathbb{R}\rightarrow \mathbb{R}$ such that for all reals $x$ and $y$,
\[f(x+y)f(x-y)=f(x^2)-f(y^2).\]
	\flushright \href{https://artofproblemsolving.com/community/c6h273627}{(Link to AoPS)}
\end{problem}



\begin{solution}[by \href{https://artofproblemsolving.com/community/user/29428}{pco}]
	\begin{tcolorbox}Find all functions $ f: \mathbb{R}\rightarrow \mathbb{R}$ such that $ \forall x,y\in\mathbb{R},f(x + y)f(x - y) = f(x^2) - f(y^2)$
(no continuity assumption on f)\end{tcolorbox}

Yes, it's a nice problem.

Let $ P(x,y)$ be the assertion $ f(x+y)f(x-y)=f(x^2)-f(y^2)$

$ P(0,0)$ $ \implies$ $ f(0)=0$
$ P(x,0)$ $ \implies$ $ f(x^2)=f(x)^2$
If $ f(a)=0$, then $ f(-a)^2=f((-a)^2)=f(a^2)=f(a)^2=0$ and $ f(-a)=-f(a)$ 
If $ f(a)\neq 0$, then $ P(0,a)$ $ \implies$ $ f(a)f(-a)=-f(a^2)= -f(a)^2$ and so $ f(-a)=-f(a)$
So $ f(-x)=-f(x)$ $ \forall x$

Then, from $ f(x^2)=f(x)^2$, we get $ f(x)=f(\sqrt x)^2$ $ \forall x\geq 0$ and so $ f(x)\geq 0$ $ \forall x\geq 0$
Then, from $ f(-x)=-f(x)$, we have $ f(x)\leq 0$ $ \forall x\leq 0$

Then, for $ x\geq y\geq 0$, we have $ x+y\geq x-y\geq 0$, so $ f(x+y)\geq 0$ and $ f(x-y)\geq 0$, so $ f(x)^2\geq f(y)^2$ and so $ f(x)\geq f(y)$ since these are two positive numbers.
Then, for $ 0\geq x\geq y$, we have $ -y\geq -x\geq 0$ and so $ f(-y)\geq f(-x)$, so $ -f(y)\geq -f(x)$ and so $ f(x)\geq f(y)$
Then, for $ x\geq 0\geq y$, we obviously have $ f(x)\geq 0 \geq f(y)$

So, $ f(x)$ is  a non decreasing function.

Then, $ P(1,0)$ $ \implies$ $ f(1)^2=f(1)$ and so either $ f(1)=0$, either $ f(1)=1$

1) $ f(1)=0$
Since $ f(x)$ is non decreasing, we get $ f(x)=0$ $ \forall x\in[0,1]$
Then $ P(x+1,x)$ $ \implies$ $ f(2x+1)f(1)=f(x+1)^2-f(x)^2=0$ and so $ f(x+1)^2=f(x)^2$ $ \forall x$
An immediate induction implies $ f(x)=0$ $ \forall x$ and it's easy to check that this value fit the equation.

2) $ f(1)=1$
Let $ f(2)=a$ (we have $ a=f(2)\geq f(1)=1$
$ P(2,1)$ $ \implies$ $ f(3)=a^2-1$
$ P(3,1)$ $ \implies$ $ af(4)=(a^2-1)^2-1=a^2(a^2-2)$ and so $ f(4)=a(a^2-2)$ (since $ a\geq 1$)
But we also have $ f(4)=f(2)^2=a^2$ and so $ a^2=a(a^2-2)$ and so $ a=-1$ or $ a=2$ so $ f(2)=2$

Then, an easy induction, using $ P(n,1)$ shows immediately that $ f(n)=n$ $ \forall n\in \mathbb{Z}$

We'll now show that $ f(x)=x$ $ \forall x$.
First, we must establish that $ |f(x)-x|<1$.
Suppose we have, for some $ x$ : $ |f(x)-x|>1$. Then we have, for some $ n\in\mathbb{Z}$, either $ f(x)<n<x$, either $ f(x)>n>x$, but :
$ f(x)<n<x$ $ \implies$ $ x>n$ and $ f(n)>f(x)$, which is impossible, since $ f(x)$ is non decreasing
$ f(x)>n>x$ $ \implies$ $ x<n$ and $ f(n)<f(x)$, which is impossible, since $ f(x)$ is non decreasing
Q.E.D.

Suppose now that $ \exists u$ such that $ f(u)\neq u$. We'll show first that it exists then some $ w>2$ such that $ f(w)\neq w$
Let $ v=|u|$. Obviously, we have $ f(v)\neq v$ and $ v>0$

$ P(v+2,v)$ $ \implies$ $ 2f(2v+2)=f(v+2)^2-f(v)^2$
If $ f(2v+2)=2v+2$ and $ f(v+2)=v+2$, then we have $ f(v)^2=(v+2)^2-2(2v+2)=v^2$ and $ f(v)=v$, which is wrong (choice of v).
So let $ w=v+2$ if $ f(v+2)\neq v+2$ and $ w=2v+2$ if $ f(v+2)=v+2$ (and so $ f(2v+2)\neq 2v+2$

So we have now $ w>2$ such that $ f(w)\neq w$ Let $ b=f(w)-w$. Since $ |f(x)-x|<1$ $ \forall x$, we have $ |b|<1$
We have $ f(w^2)-w^2=b(2w+b)$ and, since $ w>2$ and $ |b|<1$, we have $ 2w+b > 2$ and $ |f(w^2)-w^2|>2|b|$
repeating this process, we'll obtain some t such that $ |f(t)-t|>1$, which is impossible.

So, no $ u$ such that $ f(u)\neq u$ exists and $ f(x)=x$ (which indeed match the original assertion).

So the only two solutions are :
$ f(x)=0$
$ f(x)=x$
\end{solution}



\begin{solution}[by \href{https://artofproblemsolving.com/community/user/45765}{popolux}]
	Nice proof!(sorry for my late answer)
Mine was more complicated.However,i think i can simplify the end.Indeed,you proved that f is monotone and that f(n)=n.We can easily deduce that $ f(x)\/x\rightarrow 1$ when $ x\rightarrow\infty$.So,if we fix x>1,then $ \left(f(x)\/x\right)^{2^n}=f(x^{2^n})\/x^{2^n}\rightarrow 1$ when $ n\rightarrow\infty$,so f(x)=x.
\end{solution}



\begin{solution}[by \href{https://artofproblemsolving.com/community/user/29428}{pco}]
	\begin{tcolorbox}Nice proof!(sorry for my late answer)
Mine was more complicated.However,i think i can simplify the end.Indeed,you proved that f is monotone and that f(n)=n.We can easily deduce that $ f(x)\/x\rightarrow 1$ when $ x\rightarrow\infty$.So,if we fix x>1,then $ \left(f(x)\/x\right)^{2^n} = f(x^{2^n})\/x^{2^n}\rightarrow 1$ when $ n\rightarrow\infty$,so f(x)=x.\end{tcolorbox}

Oh yes ! quite nice improvement!

Thanks :)
\end{solution}
*******************************************************************************
-------------------------------------------------------------------------------

\begin{problem}[Posted by \href{https://artofproblemsolving.com/community/user/34189}{tdl}]
	Assume that $ f(x),g(x)$ are continuous function on $ \mathbb R$ and $ f(g(x))=g(f(x)),\ \forall x\in \mathbb R$. Prove that if the equation $ f(x)=g(x)$ doesn't have real roots, then the equation $ f(f(x))=g(g(x))$ doesn't have real roots either.
	\flushright \href{https://artofproblemsolving.com/community/c6h273920}{(Link to AoPS)}
\end{problem}



\begin{solution}[by \href{https://artofproblemsolving.com/community/user/29428}{pco}]
	\begin{tcolorbox}Assume that $ f(x),g(x)$ are continue function on $ R$ and $ f(g(x)) = g(f(x))\ \forall x\in R$. Prove that: If equation $ f(x) = g(x)$ hasn't real root then equation $ f(f(x)) = g(g(x))$ hasn't real root too.\end{tcolorbox}

Let $ h(x) = f(x) - g(x)$. $ h(x)$ is continuous.

Suppose $ \exists a\in \mathbb{R}$ such that $ f(f(a)) = g(g(a))$

$ h(f(a)) = f(f(a)) - g(f(a)) = g(g(a)) - g(f(a))$ $ = - (g(f(a)) - g(g(a)))$ $ = - (f(g(a)) - g(g(a)))$ $ = - h(g(a))$

And so, since $ h(x)$ is continous, $ h(x)$ has a root $ b\in [f(a),g(a)]$ and so $ f(b) = g(b)$

Hence the result : if equation $ f(x) = g(x)$ has no real root then equation $ f(f(x)) = g(g(x))$ has no real root too
\end{solution}
*******************************************************************************
-------------------------------------------------------------------------------

\begin{problem}[Posted by \href{https://artofproblemsolving.com/community/user/40002}{Ahiles}]
	Denote by $ S$ the set of all positive integers. Find all functions $ f: S \rightarrow S$ such that
\[ f (f^2(m) + 2f^2(n)) = m^2 + 2 n^2\]
for all $ m,n \in S$.

\begin{italicized}Bulgaria\end{italicized}
	\flushright \href{https://artofproblemsolving.com/community/c6h274322}{(Link to AoPS)}
\end{problem}



\begin{solution}[by \href{https://artofproblemsolving.com/community/user/36373}{No Reason}]
	Does S contain 0 ? :?:
\end{solution}



\begin{solution}[by \href{https://artofproblemsolving.com/community/user/246}{pbornsztein}]
	First, it is clear that $ f$ is injective.

Then, note that we always have $ (x+3)^2 + 2x^2 = (x-1)^2 +2 (x+2)^2$. Back to the original equation, it follows that $ f^2(x+3) + 2f^2(x) = f^2(x-1)^2 +2f^2 (x+2)$.

Let $ u_n = f^2(n)$. Thus $ u_{n+3} - 2u_{n+2} + 2u_n - u_{n-1} = 0$ for all $ n>1$.
Solving this linear relation, we get that $ u_n = an^2 + bn + c + d(-1)^n$ for all $ n > 0$ for some constant $ a,b,c,d$.
Moreover, from the initial equation, we have $ 2u_1 + u_5 = 3u_3$ from which we deduce that $ b = 0$.

Thus, $ u_n = an^2  + c + d(-1)^n$ and is the square of an integer. It is a classical result that $ a$ is a square of an integer and $ c=d=0$.
Thus, for some positive integer $ k$, we have $ f(n) = kn$ for all $ n>0$.

Back to the initial equation, we deduce that $ f(n) = n$ for all $ n>0$, which is clearly a solution.

Pierre.
\end{solution}



\begin{solution}[by \href{https://artofproblemsolving.com/community/user/29428}{pco}]
	Quite nice ! :)
\end{solution}



\begin{solution}[by \href{https://artofproblemsolving.com/community/user/15487}{Umut Varolgunes}]
	I think it is quite better than the official solution. Good work pbornsztein, you increased my respect to the problem  
\end{solution}



\begin{solution}[by \href{https://artofproblemsolving.com/community/user/9054}{cefer}]
	\begin{tcolorbox}Denote by $ S$ the set of all positive integers. Find all functions $ f: S \rightarrow S$ such that

$ f \bigg(f^2(m) + 2f^2(n)\bigg) = m^2 + 2 n^2$ for all $ m,n \in S$.

[color=darkred]\begin{italicized}Bulgaria.\end{italicized}[\/color]\end{tcolorbox}

First let $ m=n=1$ to get
$$f(3f(1)^2)=3\, .$$
Then $ (m+2n)^2+2(m-n)^2=(m-2n)^2+2(m+n)^2$ and injectivity gives
 \begin{align*}
 f(m+2n)^2+2f(|m-n|)^2&=f(|m-2n|)^2+2f(m+n)^2\,
\end{align*}
From now on we will use only this equation. Plug $ m=f(1)^2$ and $ n=2f(1)^2$ to get 
$$f(5f(1)^2)^2+2f(f(1)^2)^2=3f(3f(1)^2)^2=27\, .$$
By solving equation $ a^2+2b^2=27$  in positive integers we get the following cases

\begin{bolded}Case 1:\end{bolded} If $ f(5f(1)^2)=3=f(f(1)^2) $, then by injectivity 
$$5f(1)^2=f(1)^2\Longrightarrow f(1)=0$$
which is a contradiction.

\begin{bolded}Case 2:\end{bolded}  If $ f(5f(1)^2)=5$ and $ f(f(1)^2)=1$, then by letting  $ m=3f(1)^2$ and $ n=f(1)^2$ we get
\begin{align*}
 f(5f(1)^2)^2+2f(2f(1)^2)^2&=f(f(1)^2)^2+2f(4f(1)^2)^2\\
12 + f(2f(1)^2)^2 &= f(4f(1)^2)^2\, .
\end{align*}
So, $ f(2f(1)^2)=2$ and $ f(4f(1)^2)=4$

Now, use induction to prove that $ f(kf(1)^2)=k$ for all $k \in S$.\\

$ (1)$ For $ k=1,2,3,4,5$ it was shown above.
$ (2)$ Assume the claim holds up to some $k > 5$.
$ (3)$ By taking $ m=(k-2)f(1)^2$ and $ n=f(1)^2$ we get
\begin{align*}
f(kf(1)^2)^2+2f((k-3)f(1)^2)^2&=f((k-4)f(1)^2)^2+2f((k-1)f(1)^2)^2\\
f(kf(1)^2)^2+2(k-3)^2&=(k-4)^2+2(k-1)^2\\
f(kf(1)^2)&=k\, .
\end{align*}
This completes induction.

Since $ f(1)$ is a positive integer
$$ f(f(1)f(1)^2)=f(1) \Longrightarrow f(1)=1\,.$$
So $ f(k)=k$ for all $k\in S$.

Hope all calculations are correct...
\end{solution}



\begin{solution}[by \href{https://artofproblemsolving.com/community/user/54128}{YOGRRR}]
	\begin{tcolorbox}
Hope all calculations are correct...\end{tcolorbox}

I think, there is no mistake in official solution :P
\end{solution}



\begin{solution}[by \href{https://artofproblemsolving.com/community/user/9054}{cefer}]
	\begin{tcolorbox}[quote="cefer"]
Hope all calculations are correct...\end{tcolorbox}

I think, there is no mistake in official solution :P\end{tcolorbox}

Hmm... How can I know this is same as official one :)  ? Anyway, thanks a lot.
\end{solution}



\begin{solution}[by \href{https://artofproblemsolving.com/community/user/43536}{nguyenvuthanhha}]
	\begin{italicized}pbornsztein , how to prove $ a$ is a perfect square , $ d = c = 0$ ?\end{italicized}
\end{solution}



\begin{solution}[by \href{https://artofproblemsolving.com/community/user/34466}{je4ko}]
	Well it's not so difficult: you want $ an^2+c+d(-1)^n$ to be a perfect square for all natural $ n$. It's obvious that $ a$ is nonnegative (we assume it's positive because $ a=0$ gives as f to be a constant function which is not a solution). Now we shall be interested what is happening only over the even $ n$. Let $ c+d=b$. So $ an^2+b$ is a perfect square for all even $ n$. Let $ n$ is even and $ an^2+b=p^2$, $ a(2n)^2+b=4an^2+b=q^2$. Then $ 4an^2+4b=(2p)^2$. If $ b$ is not equal 0 then we have 2 cases. 
1) $ b>0$ and we get $ 3b=(2p)^2-q^2$. Since $ b>0$ we get $ 2p>q$ so $ 3b>=(q+1)^2-q^2=2q+1$. Now it's clear that $ q>n$, so for large $ n$ we get a contadiction. 
2) $ b<0$ - in the same way for large $ n$ we get contradiction.
Thus we must have $ b=0$ and so $ a$ must be a square of integer. Since a is a square and $ c+d=0$ it is now easy to get $ c=d=0$  
\end{solution}



\begin{solution}[by \href{https://artofproblemsolving.com/community/user/64273}{trigg}]
	I think it is the best problem of BMO-2009  I like it but you can get 1 point from finding f is surjective I think it is very simple
\end{solution}



\begin{solution}[by \href{https://artofproblemsolving.com/community/user/9054}{cefer}]
	\begin{tcolorbox}I think it is the best problem of BMO-2009  I like it but you can get 1 point from finding f is surjective I think it is very simple\end{tcolorbox}

I think you mean injective  
\end{solution}



\begin{solution}[by \href{https://artofproblemsolving.com/community/user/64273}{trigg}]
	yes,thank you :blush:
\end{solution}



\begin{solution}[by \href{https://artofproblemsolving.com/community/user/43536}{nguyenvuthanhha}]
	Proof of pbornsztein's result can be found here

http://www.artofproblemsolving.com/Forum/viewtopic.php?f=591&t=290821
\end{solution}



\begin{solution}[by \href{https://artofproblemsolving.com/community/user/114585}{anonymouslonely}]
	\begin{tcolorbox}
Let $ u_n = f^2(n)$. Thus $ u_{n+3} - 2u_{n+2} + 2u_n - u_{n-1} = 0$ for all $ n>1$.
Solving this linear relation, we get that $ u_n = an^2 + bn + c + d(-1)^n$ for all $ n > 0$ for some constant $ a,b,c,d$.
\end{tcolorbox} why these constants exist? in specially $ d $ ?
\end{solution}



\begin{solution}[by \href{https://artofproblemsolving.com/community/user/292293}{Road_to_gold_at_IMO2016}]
	\begin{tcolorbox}First, it is clear that $ f$ is injective.

Then, note that we always have $ (x+3)^2 + 2x^2 = (x-1)^2 +2 (x+2)^2$. Back to the original equation, it follows that $ f^2(x+3) + 2f^2(x) = f^2(x-1)^2 +2f^2 (x+2)$.

Let $ u_n = f^2(n)$. Thus $ u_{n+3} - 2u_{n+2} + 2u_n - u_{n-1} = 0$ for all $ n>1$.
Solving this linear relation, we get that $ u_n = an^2 + bn + c + d(-1)^n$ for all $ n > 0$ for some constant $ a,b,c,d$.
Moreover, from the initial equation, we have $ 2u_1 + u_5 = 3u_3$ from which we deduce that $ b = 0$.

Thus, $ u_n = an^2  + c + d(-1)^n$ and is the square of an integer. It is a classical result that $ a$ is a square of an integer and $ c=d=0$.
Thus, for some positive integer $ k$, we have $ f(n) = kn$ for all $ n>0$.

Back to the initial equation, we deduce that $ f(n) = n$ for all $ n>0$, which is clearly a solution.

Pierre.\end{tcolorbox}


Sorry but how to solve this linear relation for 4 elements?
\end{solution}



\begin{solution}[by \href{https://artofproblemsolving.com/community/user/237384}{Wolowizard}]
	\begin{bolded}Very nice solution\end{underlined}\end{bolded}:
Let $A(m,n)$ be assertion of \\
$\[ f (f^2(m) + 2f^2(n)) = m^2 + 2 n^2\]$.\\
First thing is to show that $f$ is injective.
Assume $f(a)=f(b)$.\\
From $A(m,a)$,$A(m,b)$ it holds that $a=b$ that is $f $is injective.\\
Assume $f(1)=a$.\\
$A(1,1)$ implies $f(3a^2)=3$.\\
If we look for numbers $m_1,m_2,n_1,n_2$ such that $m_1^2+2n_1^2=m_2^2+2n_2^2$ we see that \\
$(5,1,3,3), (1,4,5,2)$ are such numbers.
Now $A(5a^2,1a^2)$ ,$A(3a^2,3a^2)$ , $A(1a^2,4a^2)$ and $A(5a^2,2a^2)$ imply that \\
$f(2a^2)=2,f(5a^2)=5,f(a^2)=1,f(4a^2)=4$(basic computation )\\
Now since $x^2+2(x+3)^2=(x+4)^2+2(x+1)^2$ assuming that for $n\le x+4$ ,$f(na^2)=n$ it implies that 
from $A(xa^2,2(x+3)a^2)$,$A((x+4)a^2,(x+1)a^2)$ that $f((x+4)a^2)=x+4$.\\
So we have $f(na^2)=n$ but now plugging in \\
$n=a$ we have $f(a^3)=a=f(1)$ $\implies a=1$ $\implies$\\
$\boxed{f(n)=n}$ which indeed is solution.

\end{solution}
*******************************************************************************
-------------------------------------------------------------------------------

\begin{problem}[Posted by \href{https://artofproblemsolving.com/community/user/39993}{Rose_joker}]
	Show that there is no function $ f: \mathbb R \to \mathbb R$ such that  for all $ x,y\in \mathbb R$,
\[ f(f(x+y))=f(x)+f(y)-|x-y|.\]
	\flushright \href{https://artofproblemsolving.com/community/c6h274477}{(Link to AoPS)}
\end{problem}



\begin{solution}[by \href{https://artofproblemsolving.com/community/user/29428}{pco}]
	\begin{tcolorbox}Show that there is none function from $ R$ to $ R$ such that 
for all $ x,y\in R$
$ f(f(x + y)) = f(x) + f(y) - |x - y|$\end{tcolorbox}

Let $ P(x,y)$ the assertion $ f(f(x+y))=f(x)+f(y)-|x-y|$

$ P(x,x)$ $ \implies$ $ f(f(2x))=2f(x)$

$ P(2,0)$ $ \implies$ $ f(f(2))=f(2)+f(0)-2$ and so  $ (E1)$ : $ 2f(1)=f(2)+f(0)-2$
$ P(-2,0)$ $ \implies$ $ f(f(-2))=f(-2)+f(0)-2$ and so  $ (E2)$ : $ 2f(-1)=f(-2)+f(0)-2$
$ P(1,-1)$ $ \implies$ $ f(f(0))=f(1)+f(-1)-2$ and so  $ (E3)$ : $ 4f(0)=2f(1)+2f(-1)-4$
$ P(2,-2)$ $ \implies$ $ f(f(0))=f(2)+f(-2)-4$ and so  $ (E4)$ : $ f(2)+f(-2)=2f(0)+4$

Adding the four equalities $ (E1)$ to $ (E4)$ implies $ 0=-4$

So, no such function exists.
\end{solution}
*******************************************************************************
-------------------------------------------------------------------------------

\begin{problem}[Posted by \href{https://artofproblemsolving.com/community/user/51029}{mathVNpro}]
	Find all functions $ f: \mathbb R \longrightarrow \mathbb R$ such that 
\[ xf(y) - yf(x) = f\left(\frac {y}{x}\right),\] for all $ y\in \mathbb R$ and $ x\in \mathbb R\ 
\{0\}$.
	\flushright \href{https://artofproblemsolving.com/community/c6h274494}{(Link to AoPS)}
\end{problem}



\begin{solution}[by \href{https://artofproblemsolving.com/community/user/29428}{pco}]
	\begin{tcolorbox}Let $ f: \mathbb R \longrightarrow \mathbb R$. Find all $ f$ such that: 
$ xf(y) - yf(x) = f(\frac {y}{x})$, for all $ y\in \mathbb R$ and $ x\in \mathbb R\ \{0\}$.\end{tcolorbox}

Let $ P(x,y)$ be the assertion $ xf(y)-yf(x)=f(\frac{y}{x})$

$ P(1,1)$ $ \implies$ $ f(1)=0$
$ P(2,0)$ $ \implies$ $ f(0)=0$

Let $ x\neq 0$. Then :
$ P(x,1)$ $ \implies$ $ f(\frac{1}{x})=-f(x)$

$ P(\frac{1}{x},2)$ $ \implies$ $ f(2x)=\frac{1}{x}f(2)-2f(\frac{1}{x})$ $ =\frac{1}{x}f(2)+2f(x)$

$ P(\frac{1}{2},x)$ $ \implies$ $ f(2x)=\frac{1}{2}f(x)-xf(\frac{1}{2})$ $ =\frac{1}{2}f(x)+xf(2)$

So $ \frac{1}{x}f(2)+2f(x)=\frac{1}{2}f(x)+xf(2)$

And $ f(x)=\frac{2f(2)}{3}(x - \frac{1}{x})$

So we have :
$ f(0)=0$
$ f(x)=a(x - \frac{1}{x})$ $ \forall x\in \mathbb{R}^*$, and for any real $ a$.

And it's easy to verify that this function match the original assertion.
\end{solution}
*******************************************************************************
-------------------------------------------------------------------------------

\begin{problem}[Posted by \href{https://artofproblemsolving.com/community/user/40922}{mehdi cherif}]
	Find all functions $ f: \mathbb{R}\rightarrow: \mathbb{R^{+}}$ such that
\[f^3\left(\frac{x+y}{2}\right)=f(x)f(y)\sqrt{\frac{f^2(x)+f^2(y)}{2}},\] for all positive reals $ x$ and $ y$.
	\flushright \href{https://artofproblemsolving.com/community/c6h274544}{(Link to AoPS)}
\end{problem}



\begin{solution}[by \href{https://artofproblemsolving.com/community/user/112}{Diogene}]
	Trivial : $ y=x \Longrightarrow f(x)=|f(x)|$ , $ y=-x\Longrightarrow f^3(0)=f^3(x)\Longrightarrow f(x)=f(0) \geq 0$
:cool:
\end{solution}



\begin{solution}[by \href{https://artofproblemsolving.com/community/user/29428}{pco}]
	\begin{tcolorbox}Trivial : $ y = x \Longrightarrow f(x) = |f(x)|$ , $ y = - x\Longrightarrow f^3(0) = f^3(x)\Longrightarrow f(x) = f(0) \geq 0$
:cool:\end{tcolorbox}

Maybe trivial, but surely not thru your demo :

$ y = - x\Longrightarrow f^3(0) = f(x)f(-x)\sqrt{\frac{f(x)^2+f(-x)^2}{2}}$

You seem to think that $ |f(-x)|=|f(x)|$ which is absolutely not obvious.

And, btw, $ f(x) = |f(x)|$ is an immediate conclusion of $ f(\mathbb{R})\subseteq\mathbb{R}^+$
\end{solution}



\begin{solution}[by \href{https://artofproblemsolving.com/community/user/29428}{pco}]
	\begin{tcolorbox}find all functions $ f: \mathbb{R}\rightarrow: \mathbb{R^{ + }}$ such that :

$ f^3(\frac {x + y}{2}) = f(x)f(y)\sqrt {\frac {f^2(x) + f^2(y)}{2}}$ for all $ x$ and $ y$

have fun :)\end{tcolorbox}

Hello Mehdi Cherif !

I just have one ugly solution. And I hope there is something simpler :)

Obviously $ f(x)=c>0$ is a solution. So, let's search for non constant solutions.

Let $ g(x)=\frac{f^2(x)}{2f^2(0)}$ We have :

$ P(x,y)$ : $ 2g^3(\frac{x+y}{2})=g(x)g(y)(g(x)+g(y))$. and $ g(0)=\frac{1}{2}$

1) If $ g(x)$ is not constant, card($ g(\mathbb{R})$)$ =+\infty$
Demonstration :

Let $ u$ and $ a\neq 0$ such that $ g(u)\neq g(u+a)$
$ P(u,u+2a)$ $ \implies$ $ 2g^3(u+a)=g(u)g(u+2a)(g(u)+g(u+2a))$ which may be written :

$ 2=\frac{g(u)}{g(u+a)}\frac{g(u+2a)}{g(u+a)}(\frac{g(u)}{g(u+a}+\frac{g(u+2a)}{g(u+a)})$

From this, we get :
${ g(u)}>g(u+a)$ $ \implies$ $ g(u+a)>g(u+2a)$
${ g(u)}<g(u+a)$ $ \implies$ $ g(u+a)<g(u+2a)$

And so the sequence $ g(u+na)$ is strictly monotonous and hence all $ g(u+na)$ are different.
Q.E.D;

Now :
Let $ a\neq 0$
Let $ u=g(a)$
Let $ v=g(2a)$
Let $ w=g(-a)$
Let $ t=g(-2a)$

$ P(2a,0)$ $ \implies$ $ 8u^3 = v(2v+1)$
$ P(-2a,0)$ $ \implies$ $ 8w^3 = t(2t+1)$
$ P(-a,a)$ $ \implies$ $ 1 = 4uw(u+w)$
$ P(-2a,2a)$ $ \implies$ $ 1 = 4vt(v+t)$

This system of four equations with four unknown values has a limited set of solutions (demonstration follows).
Hence $ u=g(a)$ can take a limited set of values
Hence card($ g(\mathbb{R})$) is a finite number
Hence $ g(x)$ is a constant function (from point 1) above)

Hence solutions of initial equation are constant functions.
Q.E.D.

=======================
Here is the ugly part of my demo :
$ E1$ : $ 8u^3 = v(2v+1)$
$ E2$ : $ 8w^3 = t(2t+1)$
$ E3$ : $ 1 = 4uw(u+w)$
$ E4$ : $ 1 = 4vt(v+t)$

Showing that this set of equation has a limited set of solutions seems trivial but need some calculus ... :

From $ E3$, we get $ (u+w)^3=u^3+w^3+\frac{3}{4}$ and so $ 1=64u^3w^3(u+w)^3=64u^3w^3(u^3+w^3+\frac{3}{4})$
So, using $ E1$ and $ E2$ :

$ 8=vt(2v+1)(2t+1)(v(2v+1)+t(2t+1)+6)$

$ 8=vt(4vt+2v+2t+1)(2v^2+2t^2+v+t+6)$

$ 8=vt(4vt+2(v+t)+1)(2(v+t)^2 -4vt+(v+t)+6)$

Let then $ v+t=z$ and $ vt=\frac{1}{4z}$ (using $ E4$) :

$ 8=\frac{1}{4z}(\frac{1}{z}+2z+1)(2z^2 -\frac{1}{z}+z+6)$

$ (2z^2+z+1)(2z^3+z^2+6z-1)=32z^3$

And this, obviously, gives a limited set of values for $ z$. And each value of $ z$ gives one value of $ vt$ (thru $ E4$), so at most two values of $ v$ and $ t$, and so values of $ u$ and $ w$.

Q.E.D.
\end{solution}
*******************************************************************************
-------------------------------------------------------------------------------

\begin{problem}[Posted by \href{https://artofproblemsolving.com/community/user/51029}{mathVNpro}]
	Suppose $ f(x)$ is odd and strictly increase. Prove that:
If $ a,b,c\in \mathbb R$, such that $ a+b+c=0$, then $ f(a)f(b)+f(b)f(c)+f(c)f(a)\le 0$.
	\flushright \href{https://artofproblemsolving.com/community/c6h274550}{(Link to AoPS)}
\end{problem}



\begin{solution}[by \href{https://artofproblemsolving.com/community/user/29428}{pco}]
	\begin{tcolorbox}Suppose $ f(x)$ is odd and strictly increase. Prove that:
If $ a,b,c\in \mathbb R$, such that $ a + b + c = 0$, then $ f(a)f(b) + f(b)f(c) + f(c)f(a)\le 0$.\end{tcolorbox}

wlog say $ |c|=\max(|a|,|b|,|c|)$. So $ c=-a-b$ and $ c$, $ -a$ and $ -b$ all have the same sign, wlog say $ \leq 0$. So :

$ f(a)+f(b)\geq f(b)\geq 0$ (since $ f(x)$ is increasing and since $ f(0)=0$ since $ f(x)$ is odd).
$ f(a+b)\geq f(a)\geq 0$ (same reason )

Multiplying these two lines :

$ f(a+b)(f(a)+f(b)) \geq f(a)f(b)$ and so $ f(a)f(b) -f(a)f(a+b) - f(b)f(a+b) \leq 0$ and so :

$ f(a)f(b) + f(a)f(c) + f(b)f(c) \leq 0$
\end{solution}
*******************************************************************************
-------------------------------------------------------------------------------

\begin{problem}[Posted by \href{https://artofproblemsolving.com/community/user/51029}{mathVNpro}]
	Let $ f,g$ be continuous: $ [0,1]\longrightarrow [0,1]$ such that: 
$ f(g(x))=g(f(x))$, for all $ x\in [0,1]$. Suppose $ f$ is strictly increase.
Prove that: There exists $ a\in [0,1]$ such that $ f(a)=g(a)=a$.
	\flushright \href{https://artofproblemsolving.com/community/c6h274551}{(Link to AoPS)}
\end{problem}



\begin{solution}[by \href{https://artofproblemsolving.com/community/user/29428}{pco}]
	\begin{tcolorbox}Let $ f,g$ be continuous: $ [0,1]\longrightarrow [0,1]$ such that: 
$ f(g(x)) = g(f(x))$, for all $ x\in [0,1]$. Suppose $ f$ is strictly increase.
Prove that: There exists $ a\in [0,1]$ such that $ f(a) = g(a) = a$.\end{tcolorbox}

Nice problem !

Since $ g(0)\geq 0$ and $ g(1)\leq 1$, and $ g(x)$ continuous $ \exists u\in[0,1]$ such that $ g(u)=u$

$ g(f(u))=f(g(u))=f(u)$ and so $ f(u)$ is also a fixed point of $ g(x)$

Let then the sequence $ a_0=u$ and ${ a_{n+1}=f(a_n})$ $ \forall n\geq 0$. We have $ g(a_n)=a_n$ $ \forall n\geq 0$
If $ a_1\geq a_0$, $ a_n$ is a non decreasing sequence (since $ f(x)$ is strictly increasing).
If $ a_1\leq a_0$, $ a_n$ is a non increasing sequence (since $ f(x)$ is strictly increasing).

In either case :
Either $ \exists p$ such that $ a_{p+1}=a_p$, so $ f(a_p)=a_p=g(a_p)$ Q.E.D
Either such $ p$ does not exist and so $ a_n$ has a limit $ L$ in $ [0,1]$. Then, since $ g(a_n)=a_n$, $ f(a_n)=a_{n+1}$ and $ f(x)$ and $ g(x)$ are continuous, we have $ f(L)=L=g(L)$ Q.E.D.
\end{solution}



\begin{solution}[by \href{https://artofproblemsolving.com/community/user/51029}{mathVNpro}]
	\begin{tcolorbox}[quote="mathVNpro"]Let $ f,g$ be continuous: $ [0,1]\longrightarrow [0,1]$ such that: 
$ f(g(x)) = g(f(x))$, for all $ x\in [0,1]$. Suppose $ f$ is strictly increase.
Prove that: There exists $ a\in [0,1]$ such that $ f(a) = g(a) = a$.\end{tcolorbox}

Nice problem !

Since $ g(0)\geq 0$ and $ g(1)\leq 1$, and $ g(x)$ continuous $ \exists u\in[0,1]$ such that $ g(u) = u$

$ g(f(u)) = f(g(u)) = f(u)$ and so $ f(u)$ is also a fixed point of $ g(x)$

Let then the sequence $ a_0 = u$ and ${ a_{n + 1} = f(a_n})$ $ \forall n\geq 0$. We have $ g(a_n) = a_n$ $ \forall n\geq 0$
If $ a_1\geq a_0$, $ a_n$ is a non decreasing sequence (since $ f(x)$ is strictly increasing).
If $ a_1\leq a_0$, $ a_n$ is a non increasing sequence (since $ f(x)$ is strictly increasing).

In either case :
Either $ \exists p$ such that $ a_{p + 1} = a_p$, so $ f(a_p) = a_p = g(a_p)$ Q.E.D
Either such $ p$ does not exist and so $ a_n$ has a limit $ L$ in $ [0,1]$. Then, since $ g(a_n) = a_n$, $ f(a_n) = a_{n + 1}$ and $ f(x)$ and $ g(x)$ are continuous, we have $ f(L) = L = g(L)$ Q.E.D.\end{tcolorbox}

Nice solution \begin{bolded}pco\end{bolded}, much shorter than mine. \begin{bolded}Here is mine approach\end{bolded}:
Denote by $ h(x) = g(x) - x$. Therefore, $ h(x)$ will be continuos on $ [0,1]$ and $ h(0) = g(0)\ge 0$; 
$ h(1) = g(1) - 1\le 0 \Longrightarrow$ \begin{bolded}Exist\end{bolded} $ x_0\in [0,1]$ such that $ h(x_0) = 0\Longrightarrow g(x_0) = x_0$.
\begin{bolded}Case $ 1$\end{bolded}\end{underlined}: $ f(x_0) = x_0$. Put $ a = x_0 \Longrightarrow f(a) = g(a) = a$ (Done!)
\begin{bolded}Case $ 2$\end{bolded}\end{underlined}: $ f(x_0)$ is different from $ x_0$.
Consider a sequence $ \{x_n\}$: $ x_1 = f(x_0)$; $ x_{n + 1} = f(x_n)$, $ n\ge1$.
If $ f(x_0) < x_0$ then $ \{x_n\}$ strictly increases, otherwise, $ \{x_n\}$ strictly decrease.
It is so obvious that $ x_n\in [0,1]$, hence in both probabilities, $ \lim\limits_{n\rightarrow \infty}x_n = a\in [0,1]\Longrightarrow$ 
${ f(a) = f(\lim\limits_{n\rightarrow \infty}x_n}) = \lim\limits_{n\rightarrow \infty}f(x_n) = \lim x_{n + 1} = \lim x_n = a$, $ (1)$.
In the other hand, $ g(x_n) = g(f(x_{n - 1})) = g(f(x_{n - 1}))$
$ \Longrightarrow g(x_1) = f(g(x_0)) = f(x_0) = x_1\Longrightarrow....\Longrightarrow g(x_n) = f(g(x_{n - 1}) = f(x_{n - 1}) = x_n$.
$ \Longrightarrow g(a) = g(\lim x_n) = \lim g(x_n) = \lim x_n = a$, $ (2)$.
Combine $ (1),(2)$, we get $ f(a) = g(a) = a$.
Our proof is completed. 
\end{solution}



\begin{solution}[by \href{https://artofproblemsolving.com/community/user/29428}{pco}]
	\begin{tcolorbox} Nice solution \begin{bolded}pco\end{bolded}, much shorter than mine. \end{tcolorbox}

Thanks. But our two solutions are exactly the same, even though i wrote mine in a shorter way. :)
\end{solution}



\begin{solution}[by \href{https://artofproblemsolving.com/community/user/53051}{vinhhop}]
	\begin{tcolorbox}Let $ f,g$ be continuous: $ [0,1]\longrightarrow [0,1]$ such that: 
$ f(g(x))=g(f(x))$, for all $ x\in [0,1]$. Suppose $ f$ is strictly increase.
Prove that: There exists $ a\in [0,1]$ such that $ f(a)=g(a)=a$.\end{tcolorbox}
I found a problem without condition "$ f$ is strictly increase". Please tell me how I can solve it.
\end{solution}
*******************************************************************************
-------------------------------------------------------------------------------

\begin{problem}[Posted by \href{https://artofproblemsolving.com/community/user/25184}{jedaihan}]
	Find all $ f: \mathbb{R} \to \mathbb{R}$ such that:
(i) $ [f(x)]=[x]$ for all $ x \in \mathbb{R}$
(ii) $ f(x+1)=f(x)+1$ for all $ x \in \mathbb{R}$
(iii) $ f(x)f(\frac{1}{x})=1$ for all $ x \in \mathbb{R} -\{0\}$
	\flushright \href{https://artofproblemsolving.com/community/c6h275384}{(Link to AoPS)}
\end{problem}



\begin{solution}[by \href{https://artofproblemsolving.com/community/user/29428}{pco}]
	\begin{tcolorbox}Find all $ f: \mathbb{R} \to \mathbb{R}$ such that:
(i) $ [f(x)] = [x]$ for all $ x \in \mathbb{R}$
(ii) $ f(x + 1) = f(x) + 1$ for all $ x \in \mathbb{R}$
(iii) $ f(x)f(\frac {1}{x}) = 1$ for all $ x \in \mathbb{R} - \{0\}$\end{tcolorbox}

Since $ f(1)^2 = 1$ and $ [f(1)] = [1]$, we get $ f(1) = 1$ and, using $ f(x + 1) = f(x) + 1$, $ f(n) = n$

Consider $ u > 0$ and $ x = [a_0;a_1;a_2; ...]$ a normalized continued fraction representation of $ x$ (not ending with 1, except for $ x = 1$) 
Consider then $ v = f(u) = [b_0;b_1;b_2; ...]$ a normalized continued fraction representation of $ f(x)$

$ [f(u)] = $ $ \implies$ $ a_0 = b_0$

$ f(u - a_0) = f(u) - a_0$ $ \implies$ $ f([0;a_1;a_2; ...] = [0;b_1;b_2; ...]$

Then, since $ f(\frac {1}{x}) = \frac {1}{f(x)}$, $ f([a_1;a_2;...]) = [b_1;b_2;...]$

And it's easy to show with induction that $ a_i = b_i$ $ \forall i$ (using, for rational numbers, whose represention is limited, the fact that $ f(n) = n$)

So $ f(x) = x$ $ \forall x > 0$
So $ f(x) = x$ $ \forall x$ (use $ f(1 + 0) = 1 + f(0)$ for $ f(0)$, for example)
\end{solution}
*******************************************************************************
-------------------------------------------------------------------------------

\begin{problem}[Posted by \href{https://artofproblemsolving.com/community/user/59576}{4865550150}]
	Given that $ f: [0,2] \rightarrow \mathbb{R}$ satisfies
(i) For all $ x \in [0,2]$, there always exist $ f(2 - x) = f(x)$ such that $ f(x) \geq 1, f(1) = 3$, and
(ii) For all $ x,y \in [1,2]$, if $ x + y \geq 3$, then $ f(x) + f(y) \leq f(x + y - 2) + 1$.

Prove that
(1) For all $n \in \mathbb N$, \[ f \left(\frac {1}{3^n} \right) \leq \frac {2}{3^n} + 1.\]
(2) For all $ x \in [1,2]$,
\[1 \leq f(x) \leq 13 - 6x.\]
	\flushright \href{https://artofproblemsolving.com/community/c6h275520}{(Link to AoPS)}
\end{problem}



\begin{solution}[by \href{https://artofproblemsolving.com/community/user/29428}{pco}]
	\begin{tcolorbox}  For all $ x \in [0,2]$, there always exist $ f(2 - x) = f(x)$ such that....\end{tcolorbox}

What could be the meaning of this phrase ?

is it "$ \forall x\in[0,2]$ $ f(2-x)=f(x)$ and ..." ??
\end{solution}



\begin{solution}[by \href{https://artofproblemsolving.com/community/user/59576}{4865550150}]
	Sorry. My English is very bad.

(i) means that
 
$ \forall x \in [0,2] , f(2-x)=f(x) \geq 1$ and $ f(1)=3$.
\end{solution}



\begin{solution}[by \href{https://artofproblemsolving.com/community/user/29428}{pco}]
	\begin{tcolorbox}Given that $ f: [0,2] \rightarrow \textrm{R}$ which satisfies
[list](i) For all $ x \in [0,2]$, $ f(2 - x) = f(x)$, $ f(x) \geq 1$ and $ f(1) = 3$,

(ii) For all $ x,y \in [1,2]$, if $ x + y \geq 3$, then $ f(x) + f(y) \leq f(x + y - 2) + 1$.[\/list]

Prove that

(1) $ f \left(\frac {1}{3^n} \right) \leq \frac {2}{3^n} + 1$ for all $ n \in N^ +$,

(2) when $ x \in [1,2]$,      $ \; \; \; 1 \leq f(x) \leq 13 - 6x$.\end{tcolorbox}

1) first question :
Let $ n>0$
Let $ a_n^p=f(\frac{p}{3^n})-1$
Let two positive integers $ p$ and $ q$ such that $ p+q\leq 3^n$ 

Then $ a_n^p+a_n^q=f(\frac{p}{3^n})-1+f(\frac{q}{3^n})-1$ $ =f(2-\frac{p}{3^n})+f(2-\frac{q}{3^n})-2$
We have then $ 2-\frac{p}{3^n}+2-\frac{q}{3^n}$ $ =4-\frac{p+q}{3^n}\geq 3$ and so :

$ a_n^p+a_n^q\leq f(4-\frac{p+q}{3^n}-2)+1-2$ $ =a_n^{p+q}$

From this inequality $ a_n^p+a_n^q\leq a_n^{p+q}$ $ \forall p,q,n>0$ such that $ p+q\leq 3^n$, it's immediate to get : $ pa_n^1\leq a_n^p$ and $ 3^na_n^1\leq a_n^{3^n}$

And so : $ 3^n(f(\frac{1}{3^n})-1)\leq f(\frac{3^n}{3^n})-1=2$

And so : $ f(\frac{1}{3^n})\leq \frac{2}{3^n}+1$ $ \forall n\geq 0$ (notice it's true for $ n=0$).

2) second question :
$ \forall n\geq 0$, $ \forall x\in[2-\frac{1}{3^p},2-\frac{1}{3^{p+1}}]$, let $ y=4-x-\frac{1}{3^p}$. We have then $ y\in[1,2]$ and $ x+y\geq 3$. So :

$ f(x)+f(y)\leq f(x+y-2)+1$ $ =f(2-\frac{1}{3^p})+1$ $ =f(\frac{1}{3^p})+1$ $ =2+\frac{2}{3^p}$.

Then, since $ f(y)\geq 1$ : 
$ \forall x\in[2-\frac{1}{3^p},2-\frac{1}{3^{p+1}}]$ : $ f(x)\leq 1 + \frac{2}{3^p}$

And, since $ 1 + \frac{2}{3^p}$ $ =13 - 6(2-\frac{1}{3^{p+1}})$ :

$ \forall x\in[1,2)$ : $ f(x)\leq 13-6x$

For $ x=2$, we have $ f(2)+f(1)\leq f(2+1-2)-1$ and so $ f(2)+f(1)\leq f(1)+1$ and so $ f(2)\leq 1$ and so $ f(2)=1$

So : $ \forall x\in[1,2]$ : $ f(x)\leq 13-6x$
\end{solution}
*******************************************************************************
-------------------------------------------------------------------------------

\begin{problem}[Posted by \href{https://artofproblemsolving.com/community/user/46039}{ll931110}]
	(1) Find all function $ f: [0;1] \rightarrow [0;1]$ which satisfies:
1\/ $ f(x_1) \neq f(x_2) \forall x_1 \neq x_2$
2\/ $ 2x - f(x) \in [0;1] \forall x \in [0;1]$
3\/ $ f(2x - f(x)) = x$

(2) Does there exist a function $ f: \mathbb R \rightarrow \mathbb R$ satisfying these below conditions:
1\/ There exists $ M \ge 0$ such that $ -M \le f(x) \le M$
2\/ $ f(1) = 1$
3\/ If $ x \neq 0$ then $ f(x + \frac{1}{x^2}) = f(x) + f^2(\frac{1}{x})$.

(3) Find all function $ f,g: \mathbb R \rightarrow \mathbb R$ satisfying
\[ f(x + g(y)) = xf(y) - yf(x) + g(x) , \quad \forall x,y \in \mathbb R.\]

(4) Find all functions $ f: \mathbb R \rightarrow \mathbb R$ satisfying
1\/ $ f(x_1) \neq f(x_2)  \forall x_1 \neq x_2$, and
2\/ For all real numbers $x$ and $y$ with $x \neq y$,
\[ f\left(\frac{x + y}{x - y}\right) = \frac{f(x) + f(y)}{f(x) - f(y)}.\]
	\flushright \href{https://artofproblemsolving.com/community/c6h275567}{(Link to AoPS)}
\end{problem}



\begin{solution}[by \href{https://artofproblemsolving.com/community/user/29428}{pco}]
	\begin{tcolorbox}Problem 1:
Find all function $ f: [0;1] \rightarrow [0;1]$ which satisfies:
1\/ $ f(x_1) \neq f(x_2) \forall x_1 \neq x_2$
2\/ $ 2x - f(x) \in [0;1] \forall x \in [0;1]$
3\/ $ f(2x - f(x)) = x$
\end{tcolorbox}

Conditions 2. + 3. are enough to conclude $ f(x) = x$ :

Let $ f(a) = b$ and the following sequence :
$ u_0 = a$
$ u_1 = 2a - b$
$ u_{n + 2} = 2u_{n + 1} - u_n$

We obviously have $ f(u_{n + 1}) = u_n$ $ \forall n$ and $ u_n\in[0,1]$ $ \forall n$

But clearly $ u_n = n(a - b) + a$ and so $ a = b$, else $ u_n\in[0,1]$ $ \forall n$ would be wrong.
\end{solution}



\begin{solution}[by \href{https://artofproblemsolving.com/community/user/46039}{ll931110}]
	Thanks pco. I thought it when I was solving this problem, but at that time I wasn't sure my solution is correct.

What about other problems? Who can solve them?
\end{solution}



\begin{solution}[by \href{https://artofproblemsolving.com/community/user/29428}{pco}]
	\begin{tcolorbox}Problem 2:
Does there exist a function $ f: R \rightarrow R$ satisfying these below conditions:
1\/ There exists $ M \ge 0$ such that $ - M \le f(x) \le M$
2\/ $ f(1) = 1$
3\/ If $ x \neq 0$ then $ f(x + \frac {1}{x^2}) = f(x) + f^2(\frac {1}{x})$\end{tcolorbox}

$ f(1) = 1$ $ \implies$ $ f(2) = f(1 + \frac {1}{1^2}) = f(1) + f^2(\frac {1}{1}) = 2$

Let then $ u$ such that $ f(u)\geq 2$

If $ f(\frac {1}{u})\geq 1 - f(u)$, let $ v = \frac {1}{u} + u^2$.
$ f(v) = f(\frac {1}{u}) + f^2(u)$ $ \geq 1 - f(u) + f^2(u)\geq f(u) + 1$

If $ f(\frac {1}{u}) < 1 - f(u)$, let $ v = u + \frac {1}{u^2}$.
$ f(\frac {1}{u}) < 1 - f(u) < 0$ $ \implies$ $ f^2(\frac {1}{u}) > f^2(u) - 2f(u) + 1$
$ f(v) = f(u) + f^2(\frac {1}{u})$ $ > f^2(u) - f(u) + 1\geq f(u) + 1$

So, if $ \exists u$ such that $ f(u)\geq 2$, then $ \exists v$ such that $ f(v)\geq f(u) + 1$

So, since $ f(2) = 2$, $ f(x)$ may be as great as we want, which is impossible, since $ f(x)\leq M$ $ \forall x$

So, no such function exists.
\end{solution}



\begin{solution}[by \href{https://artofproblemsolving.com/community/user/29428}{pco}]
	\begin{tcolorbox}Problem 4:
Find all functionj $ f: R \rightarrow R$ satisfying
1\/ $ f(x_1) \neq f(x_2) \forall x_1 \neq x_2$
2\/ $ f(\frac {x + y}{x - y}) = \frac {f(x) + f(y)}{f(x) - f(y)} \forall x \neq y$\end{tcolorbox}
Let $ P(x,y)$ be the assertion $ f(\frac{x+y}{x-y})=\frac{f(x)+f(y)}{f(x)-f(y)}$

$ P(x,0)$ $ \implies$ $ f(1)=\frac{f(x)+f(0)}{f(x)-f(0)}$ and so $ f(x)(f(1)-1)=f(0)(f(1)+1)$ and so, since $ (f(x)$ cant be a constant, since injective) :
$ f(0)=0$
$ f(1)=1$
$ P(0,1)$ $ \implies$ $ f(-1)=-1$

Compare now $ P(x,1)$ and $ P(xy,y)$ for $ x\neq 1$ and $ y\neq 0$. The two LHS are equal, so are the two RHS : $ \frac{f(x)+1}{f(x)-1}=\frac{f(xy)+f(y)}{f(xy)-f(y)}$ And so :

$ f(x)f(xy)-f(x)f(y)+f(xy)-f(y)$ $ =f(xy)f(x)+f(y)f(x)-f(xy)-f(y)$ and so $ \boxed{f(xy)=f(x)f(y)}$  and this is still true for $ x=1$ or $ y=0$

Let then $ f(2)=a$. $ P(x,1)$ may then be written (using $ f(xy)=f(x)f(y)$) : $ f(x+1)=f(x-1)\frac{f(x)+1}{f(x)-1}$ and so allow us to compute :
$ f(2)=a$
$ f(3)=\frac{a+1}{a-1}$
$ f(4)=a^2$
$ f(5)=\frac{a^2+1}{(a-1)^2}$
$ f(6)=a(a^2-a+1)$
Writing then $ f(6)=f(2)f(3)$ and using the fact that $ f(x)$ is injective (so $ a\neq 0$), we get $ a=2$.

Let now $ A=\frac{x^2+y^2+2xy}{x^2+y^2-2xy}$ for $ x\neq 0$, $ y\neq 0$ and $ x\neq y$

Using $ P(x^2+y^2,2xy)$ we obtain : $ f(A)=\frac{f(x^2+y^2)+2f(x)f(y)}{f(x^2+y^2)-2f(x)f(y)}$ (we needed $ f(2)=2$ in order to write this).

But $ A=(\frac{x+y}{x-y})^2$ and so $ f(A)=f^2(\frac{x+y}{x-y})$ $ =\frac{f^2(x)+2f(x)f(y)+f^2(y)}{f^2(x)-2f(x)f(y)+f^2(y)}$

Equating these two expressions give us $ f(x^2+y^2)=f(x^2)+f(y^2)$, which is obviously also true if $ x=0$, or $ y=0$, or $ x=y$

So $ f(x+y)=f(x)+f(y)$ $ \forall x,y\geq 0$ and, writing $ f(x)=f(x+y)-f(y)$, we have $ \boxed{f(x+y)=f(x)+f(y)\forall x,y}$

So we have a classical Cauchy equation with the complementary condition $ f(xy)=f(x)f(y)$ which gives, for any $ a>0$ $ f(x+a)=f(x)+f((\sqrt a)^2)=f(x)+f^2(\sqrt a)>f(x)$. So $ f(x)$ is strictly increasing and so $ f(x)=xf(1)$

Hence the solution $ f(x)=x$
\end{solution}



\begin{solution}[by \href{https://artofproblemsolving.com/community/user/46039}{ll931110}]
	@pco: I think I have another way for Problem 4 (at first, thanks for your help)

From $ f(xy) = f(x).f(y)$, replacing $ y$ by $ \frac{1}{x}$ into 2\/ gives
$ f(\frac{x^2 + 1}{x^2 - 1}) = \frac{(f(x))^2 + 1}{(f(x))^2 - 1}$ (1)

However, we have $ f(\frac{x^2 + 1}{x^2 - 1}) = \frac{(f(x^2) + 1}{(f(x^2) - 1}$ (2)
Combining (1) and (2) gives $ f(x^2) = (f(x))^2 \forall x$, which means $ f(x) > 0 \forall x > 0$, which yields f is strictly increasing.

So we have $ f(x) = x^a$, and return 2\/ gives $ a = 1$. So $ f(x) = x$
\end{solution}



\begin{solution}[by \href{https://artofproblemsolving.com/community/user/29428}{pco}]
	\begin{tcolorbox} From $ f(xy) = f(x).f(y)$, replacing $ y$ by $ \frac {1}{x}$ into 2\/ gives
$ f(\frac {x^2 + 1}{x^2 - 1}) = \frac {(f(x))^2 + 1}{(f(x))^2 - 1}$ (1)

However, we have $ f(\frac {x^2 + 1}{x^2 - 1}) = \frac {(f(x^2) + 1}{(f(x^2) - 1}$ (2)
Combining (1) and (2) gives $ f(x^2) = (f(x))^2 \forall x$, \end{tcolorbox}

You dont need this step (using $ y=\frac{1}{x}$) : $ f(xy)=f(x)f(y)$ immediately implies $ f(x^2)=f^2(x)$

\begin{tcolorbox} $ f(x^2) = (f(x))^2 \forall x$, which means $ f(x) > 0 \forall x > 0$, which yields f is strictly increasing.

So we have $ f(x) = x^a$, and return 2\/ gives $ a = 1$. So $ f(x) = x$ \end{tcolorbox}

And you're right. Well done :)

Be careful about the distinction $ x>0$ and $ x<0$
\end{solution}



\begin{solution}[by \href{https://artofproblemsolving.com/community/user/46039}{ll931110}]
	Sorry, I'm so silly :D.

In fact, we only need to consider whether $ x > 0$, because we have $ f(-x) = -f(x) \forall x$.

Can you help me to solve Problem 3, pco? Thanks a lot
\end{solution}
*******************************************************************************
-------------------------------------------------------------------------------

\begin{problem}[Posted by \href{https://artofproblemsolving.com/community/user/62101}{palmita}]
	Given that $ f$ : $ \mathbb{R} \to \mathbb{R}$ which satisfying : \[ f(2x)=2f(x)\sqrt{1+f^2(x)}\; , \forall x\in\mathbb{R}\quad (E) \] 
$ i)$ Solve the equation $ (E)$ by assuming $ f$ is differentiable at the point $ 0$.

$ ii)$ Solve the equation $ (E)$ in general case.
	\flushright \href{https://artofproblemsolving.com/community/c6h275670}{(Link to AoPS)}
\end{problem}



\begin{solution}[by \href{https://artofproblemsolving.com/community/user/29428}{pco}]
	\begin{tcolorbox}Hello,

Given that $ f$ : $ \mathbb{R} \to \mathbb{R}$ which satisfying :
\[ f(2x) = 2f(x)\sqrt {1 + f^2(x)}\; , \forall x\in\mathbb{R}\quad (E)
\]
$ i)$ Solve the equation $ (E)$ by assuming $ f$ is differentiable at the point $ 0$.

$ ii)$ Solve the equation $ (E)$ in general case.\end{tcolorbox}

Notice that $ a=2b\sqrt{1+b^2}$ $ \implies$ $ b=sign(a)\sqrt{\frac{\sqrt{1+a^2}-1}{2}}$ and so $ f(x)$ is fully determined by the knowleged of the function in for example $ (-2,-1]\cup[1,2)$

Notice also that if $ f(x)=\sinh(u)$, $ f(2x)=\sinh(2u)$ and $ f(\frac{x}{2})=\sinh(\frac{u}{2})$

So a general solution of this equation is :

Let $ u_1(x)$ and $ u_2(x)$ any real functions defined in $ [1,2)$. Then :

$ \forall x<0$ : $ f(x)=\sinh(xu_1(2^{\{\log_2 |x|\}}))$ (where $ \{y\}$ is the fractional part of $ y$, so is in $ [0,1)$).
$ f(0)=0$
$ \forall x>0$ : $ f(x)=\sinh(xu_2(2^{\{\log_2 |x|\}}))$ (where $ \{y\}$ is the fractional part of $ y$, so is in $ [0,1)$).


In order to have a solution differentiable at 0 :
1) We  need $ u_1$ and $ u_2$ to have max and min values over $ [1,2)$ (in order to have continuity)
2) We need to have $ \limits_{x\to 0}\frac{f(x)}{x} = c$

$ \lim_{x\to 0}\frac{f(x)}{x}$ =$ \lim_{x\to 0}\frac{\sinh(xu_i(2^{\{\log_2 |x|\}}))}{x}=c$ $ \implies$ $ \lim_{x\to 0}u_i(2^{\{\log_2 |x|\}}) = c$ and so $ u_1(x)=u_2(x)=c$

And so $ f(x)=\sinh(cx)$
\end{solution}



\begin{solution}[by \href{https://artofproblemsolving.com/community/user/29428}{pco}]
	Here is another general form for all the solutions, equivalent to the previous one, but simpler to write :

Let $ g(x)$ any real function defined over $ (-1,1)$. Then $ f(0)=0$ and $ f(x)=\sinh(x g(sign(x)\{\log_2(|x|)\}))$ $ \forall x\neq 0$ (with $ \{y\}=$ fractional part of $ y$).
\end{solution}
*******************************************************************************
-------------------------------------------------------------------------------

\begin{problem}[Posted by \href{https://artofproblemsolving.com/community/user/41147}{FantasyLover}]
	For odd prime numbers $ p$, find all functions $ f: \mathbb{Z}\rightarrow\mathbb{Z}$ satisfying following conditions:

(1) If $ m\equiv n\mod p$ , $ f(m)=f(n)$.
(2) For all integers $ m$ and $ n$, $ f(mn)=f(m)f(n)$.
	\flushright \href{https://artofproblemsolving.com/community/c6h275931}{(Link to AoPS)}
\end{problem}



\begin{solution}[by \href{https://artofproblemsolving.com/community/user/29428}{pco}]
	\begin{tcolorbox}For odd prime numbers $ p$, find all functions $ f: \mathbb{Z}\rightarrow\mathbb{Z}$ satisfying following conditions:

(1) If $ m\equiv n\mod p$ , $ f(m) = f(n)$.
(2) For all integers $ m$ and $ n$, $ f(mn) = f(m)f(n)$.\end{tcolorbox}

$ f(n) = 1$ $ \forall n$ is a trivial solution.
$ f(n) = 0$ $ \forall n$ is a trivial solution.
$ f(n) = 1$ $ \forall n\neq 0\pmod p$ and $ f(n)=0$ $ \forall n=0\pmod p$ is a trivial solution.

If $ f(u)=0$ for some $ u\neq 0\pmod p$, then let $ y=\frac{v}{u}\pmod p$. then $ v=uy\pmod p$ and so $ f(v)=f(u)f(y)=0$ $ \forall v$

Consider now we have $ m$ such that $ f(m)\notin\{0,1\}$ 
$ f(0) = f(m)f(0)$ $ \implies$ $ f(0) = 0$.
$ f(m*1) = f(m)f(1)$ $ \implies$ $ f(1) = 1$
Let $ x\neq 0\pmod p$. Exists $ y$ such that $ xy = 1\pmod p$ so $ f(xy) = f(1)$ so $ f(x)f(y) = 1$ and so $ f(x)\in\{ - 1,1\}$ $ \forall x\neq0\pmod p$ (since $ f(x)\in\mathbb Z$ and $ f(x)|1$). So $ f(m) = - 1$

Let now $ u\neq 0\pmod p$ a quadratic residue mod p. we have $ u = v^2\pmod p$ and so $ f(u) = (f(v)^2 = 1$
So $ m$ is not a quadratic residue mod p.
Let $ w$ a non quadratic residue mod p. $ mw$ is a quadratic residue and so $ f(mw) = 1$. But $ f(mw) = f(m)f(w) = - f(w)$ and so $ f(w) = - 1$

So $ f(x)$ is the Legendre symbol $ \left(\frac {x}{p}\right)$

And all the solutions are 

$ f(x) = 0$

$ f(x) = 1$

$ f(x) = 0$ $ \forall x=0\pmod p$ and $ f(x)=1$ $ \forall x\neq 0\pmod p$

$ f(x) = \left(\frac {x}{p}\right)$
\end{solution}
*******************************************************************************
-------------------------------------------------------------------------------

\begin{problem}[Posted by \href{https://artofproblemsolving.com/community/user/56873}{duythuc_lqd}]
	Find all functions $f: \mathbb R \setminus \{\pm 1\} \to \mathbb R$ satisfying
\[ f(1-2x) = 2 f\left(\frac 1x \right) + 1, \quad \forall x \in \mathbb R, x \neq 0,\pm 1.\]
	\flushright \href{https://artofproblemsolving.com/community/c6h275982}{(Link to AoPS)}
\end{problem}



\begin{solution}[by \href{https://artofproblemsolving.com/community/user/29428}{pco}]
	Ok, I consider that the question is :
Find all functions $ f$ : $ \mathbb R\backslash\{ - 1, + 1\}\to\mathbb R$ such that $ f(1 - 2x) = 2f(\frac {1}{x}) + 1$ $ \forall x\in\mathbb R\backslash\{ - 1,0, + 1\}$


$ f(1 - 2x) = 2f(\frac {1}{x}) + 1$ $ \forall x\in\mathbb R\backslash\{ - 1,0, + 1\}$ $ \Leftrightarrow$ $ f(x) = 2f(\frac {2}{1 - x}) + 1$ $ \forall x\in\mathbb R\backslash\{ - 1, + 1, + 3\}$

Let $ g(x) = f(x) + 1$ and $ h(x) = \frac {2}{1 - x}$

The problem is now : Find all functions $ g$ : $ \mathbb R\backslash\{ - 1, + 1\}\to\mathbb R$ such that $ g(x) = 2g(h(x))$  $ \forall x\in\mathbb R\backslash\{ - 1, + 1, + 3\}$

We'll have (with good conditions over $ x$) : $ g(x) = 2^ng(h(h(h( ...$n times$ ...(x)))) ... )$. And clearly the $ n^{th}$ composit of $ h(x)$ is an homographic function.

Let then the sequence $ a_n$ defined $ \forall n\in\mathbb Z$ as :
$ a_0 = 0$
$ a_1 = 2$
$ \forall n > 2$ : $ a_n = a_{n - 1} - 2a_{n - 2}$

$ \forall n < 0$ : $ a_n = \frac {a_{n + 1} - a_{n + 2}}{2}$

It's easy to check that $ a_n = \frac {\sqrt 2}{2}(b^n - (\frac { - 1}{b})^n$ (where $ b = 1 + \sqrt 2$) $ \forall n\in\mathbb Z$ and that $ a_n\neq 0$ $ \forall n\in\mathbb Z^*$.

Let then the family of functions defined for any ${ k\in\mathbb Z}$ as : $ h_k(x) = 2\frac {a_{k - 1}x - a_k}{a_kx - a_{k + 1}}$ ($ h_0$ : $ \mathbb R\to \mathbb R$; $ h_k$ : $ \mathbb R\backslash\{\frac {a_{k + 1}}{a_k}\}\to\mathbb R$)

It's easy to check that $ h_0(x) = x$ and $ h_{k + 1} = h\circ h_k$ $ \forall k\in\mathbb Z$. So, $ g(x) = 2^kg(h_k(x))$
It's also easy to check that, $ \forall k\neq 0$, $ h_k(x)$ has no fixed point (the equation $ h_k(x) = x$ implies a quadratic whose discriminant is $ - \frac {7a_k^2}{4} < 0$

Let's then the relation $ \sim$ defined as $ x\sim y$ $ \iff$ $ \exists$ a unique $ k\in \mathbb Z$ such that $ y = h_k(x)$
This relation is an equivalence one :
$ x\sim x$ : $ h_0(x) = x$ and no other $ k$ is available since $ \forall k\neq 0$, $ h_k(x)$ has no fixed point.
$ x\sim y$ $ \implies$ $ y\sim x$ : $ y = h_k(x)$ $ \iff$ $ x = h_{ - k}(x)$ and unicity is obtained thru the no_fixed_points property.
$ x\sim y$ and $ y\sim z$ $ \implies$ $ x\sim z$ :  $ y = h_i(x)$ and $ z = h_j(z)$ imply $ z = h_{i + j}(x)$ and unicity is obtained thru the no_fixed_points property.

So we can define a function $ k(x,y)$ defined in each equivalence class of $ \sim$ as $ y = h_{k(x,y}(x)$
Let $ C(x)$ the equivalence class of $ x$ and $ A = C( - 1)$

We have $ - 1 = h(3)$ and $ 1 = h( - 1)$ and so $ \{ - 1, + 1, + 3\}\subset A$

Let $ r(x)$ any choice function which associates to each real $ x$ a representative of its equivalence class (unique per class).

Then :
$ \forall x\notin A$, we have $ x = h_{k(r(x),x)}(r(x))$ and so $ g(x) = 2^{k(x,r(x))}g(r(x))$ and so we just have to define $ g(r(x))$ as we want.

The situation for $ A$ is a little bit different :
We have $ 0 = h_4(3)$ but we dont have $ g(0) = 2^4g(h_4(3))$ since we dont have $ g(1) = 2g(h(3))$
So we can choose g(x) freely for $ x = 3$ and for $ x = 0$.

So here is the general solution :
================================
Let $ U = \{r(x)$ $ \forall x\notin A = C( - 1)\}\cup\{0,3\}$
Let $ m(x)$ any function from $ U\to\mathbb R$

$ \forall x\notin C( - 1)$ : $ f(x) = 2^{k(x,r(x))}m(r(x)) - 1$
$ \forall x\in C( - 1)\backslash\{ - 1, + 1\}$ such that $ k(0,x)\geq 0$ : $ f(x) = 2^{k(x,0)}m(0) - 1$
$ \forall x\in C( - 1)\backslash\{ - 1, + 1\}$ such that $ k(0,x) < 0$ : $ f(x) = 2^{k(x,3)}m(3) - 1$

And obviously this is a general solution.

If we want a continuous solution :
================================

If we suppose $ g(x)$ continuous in $ \mathbb R\backslash\{ - 1, + 1\}$, and since $ \lim_{k\to - \infty}h_k(x) = -2b$, we have, using the continuity and $ k\to - \infty$, $ g(x) = 0$ $ \forall x\in\mathbb R\backslash A$. Then, since $ A$ is a countable set, we can always find, in any neighborhood of an element of A, elements of $ \mathbb R\backslash A$. And so $ g(x) = 0$ $ \forall x\in R\backslash\{ - 1, + 1\}$ (remember $ g(x)$ is not defined in $ \{ - 1, + 1\}$ And so the only continuous function solution of the initial problem is $ f(x) = - 1$
\end{solution}
*******************************************************************************
-------------------------------------------------------------------------------

\begin{problem}[Posted by \href{https://artofproblemsolving.com/community/user/45699}{MNrule}]
	Let $ f : \mathbb{N}\rightarrow \mathbb{N}$ be a function such that $ f(n + 1) > f(n)$ and $ f(f(n)) = 3n$ for all $n \in \mathbb N$. Prove that $ f(1) = 2$ and $ f(1458) = 2189$.
	\flushright \href{https://artofproblemsolving.com/community/c6h276200}{(Link to AoPS)}
\end{problem}



\begin{solution}[by \href{https://artofproblemsolving.com/community/user/1430}{JBL}]
	\begin{tcolorbox}Let $ f : \mathbb{N}\rightarrow \mathbb{N}$\end{tcolorbox}  This is a sure sign that your post does not belong in the calculus forum  :roll: 
[color=green]Moved.[\/color]

There are a very small number of things you can possibly do in this problem.  So, start doing them.  It may help you to think about what is special about 1458.
\end{solution}



\begin{solution}[by \href{https://artofproblemsolving.com/community/user/29428}{pco}]
	\begin{tcolorbox} [This is a sure sign that your post does not belong in the calculus forum  :roll: 
[color=green]Moved.[\/color].\end{tcolorbox}

I dont think it's an olympiad problem.
\end{solution}



\begin{solution}[by \href{https://artofproblemsolving.com/community/user/54383}{EastyMoryan}]
	Because we know $ f(n+1) > f(n)$, we know that if $ a > b$, $ f(a) > f(b)$.

Lemma 1: $ a \le f(a) \le 3a$.
To prove this, assume $ f(a) > 3a$. Then, $ f(f(a))>3a$ which contradicts the second statement of the question. Thus, $ f(a) \le 3a$. And if $ a \neq 0$, if $ f(a) = 3a$, $ f(f(a)) > 3a$, which is a contradiction, showing that the right equality only holds when $ a = 0$. For the left side, if $ f(a) < a$, then $ f(f(a)) < f(a) < 3a$, a contradiction. Clearly, equality also only holds at $ a = 0$.

From Lemma 1, we know that $ f(0)=0$.

Now, $ 1 < f(1) < 3$, thus $ f(1) = 2$, and $ f(2) = 3$. Because $ f(f(2)) = 3(2)$, $ f(3) = 6$.

$ f(0)=0$
$ f(1)=2$
$ f(2)=3$
$ f(3)=6$

$ f(f(3))=3(3)$
$ f(6)=9$

$ f(f(6))=18$
$ f(9)=18$

$ f(f(9))=27$
$ f(18)=27$

$ f(f(18))=54$
$ f(27)=54$

$ f(f(27))=108$
$ f(54)=108$

$ f(f(54))=162$
$ f(108)=162$

$ f(f(108))=324$
$ f(162)=324$

$ f(f(162))=486$
$ f(324)=486$

$ f(f(324))=972$
$ f(486)=972$

$ f(f(486))=1458$
$ f(972)=1458$

$ f(f(972))=2916$
$ f(1458)=2916$
\end{solution}



\begin{solution}[by \href{https://artofproblemsolving.com/community/user/29428}{pco}]
	\begin{tcolorbox}Let $ f : \mathbb{N}\rightarrow \mathbb{N}$ such that $ f(n + 1) > f(n)$ and $ f(f(n)) = 3n$, then prove that $ f(1) = 2$ and $ f(1458) = 2189$\end{tcolorbox}
The problem is wrong (it's $ 2187$ instead of $ 2189$) and not an olympiad one :

$ f(1)\neq 1$ else this would imply $ f(f(1))=1$ and we know $ f(f(1))=3$
So $ f(1)\geq 2$ and, since $ f(n+1)>f(n)$, we have $ f(n)\geq n+1$ and so $ 3=f(f(1))\geq f(1)+1$, so $ f(1)\leq 2$ and so $ f(1)=2$

$ f(f(n))=3n$ $ \implies$ $ f(3n)=3f(n)$ and so $ f(1458)=f(3^62)=3^6f(2)=729f(f(1))=729*3=2187\neq 2189$
\end{solution}



\begin{solution}[by \href{https://artofproblemsolving.com/community/user/44674}{Allnames}]
	How will you do with this property?
Find all functions satisfying
\[ f: \mathbb N \to \mathbb N
\]

\[ f(f(n)) = pn
\]
where $ p\in \mathbb P$
[hide="pco"]I want to express my admiration to you, pco.Your solving functional equation skill is too good.Can you share with me some experiences :) [\/hide]
\end{solution}



\begin{solution}[by \href{https://artofproblemsolving.com/community/user/1430}{JBL}]
	\begin{tcolorbox}[quote="JBL"] [This is a sure sign that your post does not belong in the calculus forum  :roll: 
[color=green]Moved.[\/color].\end{tcolorbox}

I dont think it's an olympiad problem.\end{tcolorbox}  I'm not the one who decided where to move it ;)  (The green text was inserted by a moderator.)


@Allnames: note that pco didn't characterize all functions, he just showed that any such function must have certain particular properties and values.  The given functional equation does not, for example, determine the value of $ f(4)$, and we have a great deal of freedom in choosing many values of the function.  There is probably no simpler characterization of these functions than as the result of all consistent subsets of choices.
\end{solution}



\begin{solution}[by \href{https://artofproblemsolving.com/community/user/29428}{pco}]
	\begin{tcolorbox}How will you do with this property?
Find all functions satisfying
\[ f: \mathbb N \to \mathbb N
\]

\[ f(f(n)) = pn
\]
where $ p\in \mathbb P$\end{tcolorbox}
So, the new problem you are looking for is "Let $ p$ a prime integer, find all functions : $ \mathbb N\to\mathbb N$ such that $ f(f(n)) = pn$

Let $ g_p(n)$ : $ \mathbb N\to\mathbb N\cup\{0\}$ the function which associates to any positive integer $ n$ the power of prime $ p$ in the prime factors decomposition of $ n$.
Let $ A_k = \{x\in\mathbb N$ such that $ g_p(x) = k\}$
$ f(f(n)) = pn$ implies $ f(pn) = pf(n)$ and so $ f(p^kn) = p^kf(n)$ and so we need to find $ f(x)$ over the set $ A_0$

Let $ x\in A_0$ : either $ f(x)\in A_0$ either $ f(x)\in A_1$ (else $ p^2|f(x)$, so $ p^2|f(f(x)) = px$ and $ p|x$).
Let then $ U = \{x\in A_0$ such that $ f(x)\in A_0\}$ and $ V = \{x\in A_0$ such that $ f(x)\in A_1\}$

Obviously :
$ \forall x\in U$ : $ f(x)\in V$
$ \forall x\in V$ : $ f(x)\in A_1$ and $ \frac {f(x)}{p}\in U$ (since $ f(\frac {f(x)}{p}) = \frac {f(f(x))}{p} = x\in A_0$

So $ f(x)$ is a bijection from $ U\to V$ (with inverse function being $ g(x) = \frac {f(x)}{p}$)

And so, we have found a general solution of the problem.

General solution :
=================
Let $ g_p(n)$ : $ \mathbb N\to\mathbb N\cup\{0\}$ the function which associates to any positive integer $ n$ the power of prime $ p$ in the prime factors decomposition of $ n$.
Let $ A_0 = \{x\in\mathbb N$ such that $ g_p(x) = 0\}$
Let $ U$ and $ V$ two equipotent subsets of $ A_0$ such that $ U\cap V = \emptyset$ and $ U \cup V = A_0$
Let $ h(x)$ a bijection from $ U\to V$ and $ h^{[ - 1]}(x)$ : $ V\to U$ its inverse function.

Then $ f(x)$ may be defined as :
$ \forall x\in A_0$ : 
If $ x\in U$ : $ f(x) = h(x)$
If $ x\in V$ : $ f(x) = ph^{[ - 1]}(x)$
$ \forall x\in A_k$ with $ k > 0$ : $ f(x) = p^kf(\frac {x}{p^k})$

Demo 1 : This form is a solution :
=================================
$ \forall x\in U$ : $ f(x) = h(x)\in V$ and so $ f(f(x)) = ph^{[ - 1]}(f(x)) = ph^{[ - 1]}(h(x)) = px$
$ \forall x\in V$ : $ f(x) = ph^{[ - 1]}(x)\in A_1$ and so $ f(f(x)) = pf(\frac {f(x)}{p})$ $ = pf(h^{[ - 1]}(x))$. But $ h^{[ - 1]}(x)\in U$ and so $ f(h^{[ - 1]}(x)) = h(h^{[ - 1]}(x)) = x$. So $ f(f(x)) = px$

$ \forall x\in A_k$ with $ k > 0$ : $ f(x) = p^kf(\frac {x}{p^k})$. But $ \frac {x}{p^k}\in A_0$ and so :
Either $ \frac {x}{p^k}\in U$, so $ f(\frac {x}{p^k}) = h(\frac {x}{p^k})\in V$ and so $ f(x)\in A_k$, so $ f(f(x)) = p^kf(\frac {f(x)}{p^k})$ $ = p^kf(f(\frac {x}{p^k}))$ $ = p^kp\frac {x}{p^k}$ (since $ \frac {x}{p^k}\in A$) and so $ f(f(x)) = px$

Either $ \frac {x}{p^k}\in V$, so $ f(\frac {x}{p^k}) = ph^{[ - 1]}(\frac {x}{p^k})\in A_1$ and so $ f(x)\in A_{k + 1}$, so $ f(f(x)) = p^{k + 1}f(\frac {f(x)}{p^{k + 1}})$ $ = p^{k + 1}f(\frac {f(\frac {x}{p^k})}{p})$ $ = p^{k + 1}f(h^{[ - 1]}(\frac {x}{p^k})$ $ = p^{k + 1}h(h^{[ - 1]}(\frac {x}{p^k})$ $ = p^{k + 1}\frac {x}{p^k}$ $ = px$
Q.E.D.

Demo 2 : All solutions are in this form :
========================================
This is the beginning of this post :
Take $ U = {x\in A_0}$ such that $ f(x)\in A_0\}$ and $ V = {x\in A_0}$ such that $ f(x)\in A_1\}$
Take $ h(x) = f(x)$ $ \forall x\in U$

Some examples :
==============
1) 
Let $ q\neq p$ a prime number
Let $ U = \{x$ such that $ g_p(x) = 0$ and $ g_q(x)$ is even$ \}$
Let $ V = \{x$ such that $ g_p(x) = 0$ and $ g_q(x)$ is odd$ \}$
Let $ h(x) = qx$

Then $ f(x)$ is :
Let $ x = p^mq^nz$ (with $ z\neq 0\pmod p$ and $ z\neq 0\pmod q$ :
If $ n$ is even, then $ f(x) = p^mq^{n + 1}z$
If $ n$ is odd, then $ f(x) = p^{m + 1}q^{n - 1}z$

2)
(This example suppose $ p\neq 2$)
Let $ m_p(x)$ the reminder of the division of $ x$ by $ p$
Let $ U = \{x$ such that $ m_p(x)$ is odd$ \}$
Let $ V = \{x$ such that $ m_p(x)\neq 0$ and even$ \}$
Let $ h(x) = x + 1$

Then $ f(x)$ is :
Let $ x = p^mz$ (with $ z\neq 0\pmod p$)
If $ m_p(z)$ is odd, then $ f(x) = p^m(z + 1)$
If $ m_p(z)$ is even, then $ f(x) = p^{m + 1}(z - 1)$

 And so on ...
\end{solution}



\begin{solution}[by \href{https://artofproblemsolving.com/community/user/44674}{Allnames}]
	Thank you pco  :o . The initial property requires $ f$ is a strictly increasing function. But I think we can find out $ f$ without that condition. And  I am right!   .
\end{solution}



\begin{solution}[by \href{https://artofproblemsolving.com/community/user/23345}{Mij}]
	\begin{tcolorbox}Let $ f : \mathbb{N}\rightarrow \mathbb{N}$ such that $ f(n + 1) > f(n)$ and $ f(f(n)) = 3n$, then prove that $ f(1) = 2$ and $ f(1458) = 2187$\end{tcolorbox}
This problem is essentially the same as British Math Olympiad 1992 Round 1 Problem 5. However, in the BMO problem you find the value of $ f(1992)$ instead of proving $ f(1) = 2$ and $ f(1458) = 2187$.
\end{solution}
*******************************************************************************
-------------------------------------------------------------------------------

\begin{problem}[Posted by \href{https://artofproblemsolving.com/community/user/60652}{DCTPKTCSPHN}]
	Find all functions $ f: \mathbb R^+ \rightarrow \mathbb R^+$ such that
\[f(xf(y))=f(xy)+x\]
holds for all $x,y\in \mathbb R^+$.
	\flushright \href{https://artofproblemsolving.com/community/c6h276360}{(Link to AoPS)}
\end{problem}



\begin{solution}[by \href{https://artofproblemsolving.com/community/user/29428}{pco}]
	\begin{tcolorbox}Find all $ f: R^ + \rightarrow R^ +$ such that
$ f(xf(y)) = f(xy) + x$ $ \forall x,y\in R^ +$\end{tcolorbox}

$ x=1$ $ \implies$ $ f(f(y))=f(y)+1$ and so $ f(x)=x+1$ $ \forall x\in f(\mathbb R^+)$
Let $ u=f(1)$ and $ x>u$ : $ f((x-u)f(\frac{1}{x-u}))=f(1)+x-u=x$ and so $ x\in f(\mathbb R^+)$ and so  $ f(x)=x+1$

So : $ \exists u>0$ such that $ \forall x>u$ : $ f(x)=x+1$

Let then $ v=\frac{u^2}{u+1}$. If $ x>v$, we get $ x>\frac{u^2}{u+1}$ and so $ x\frac{u+1}{u}>u$ and so $ \frac{x}{u}>u-x$.
Let then $ a>0$ such that $ \frac{x}{u}>a>u-x$ and $ b=\frac{x}{a}$. We have $ b>u$ and $ ab+a>u$

We have $ f(af(b))=f(ab)+a$. Since $ b>u$, $ f(b)=b+1$ and $ af(b)=ab+a$ and, since $ ab+a>u$, $ f(af(b))=f(ab+a)=ab+a+1$, and so $ ab+a+1=f(ab)+a$ and so $ f(x)=x+1$

So : $ \forall x>\frac{u^2}{u+1}$ : $ f(x)=x+1$

So, let the sequence $ a_0=u$, $ a_{n+1}=\frac{a_n^2}{a_n+1}$. We have $ f(x)=x+1$ $ \forall x>a_n$ $ \forall n$

So, since $ a_n$ is a decreasing sequence whose limit is 0, we get :

$ f(x)=x+1$ $ \forall x>0$
\end{solution}



\begin{solution}[by \href{https://artofproblemsolving.com/community/user/29876}{ozgurkircak}]
	here is another solution. [url]http://www.mathlinks.ro/viewtopic.php?t=266578[\/url]
\end{solution}
*******************************************************************************
-------------------------------------------------------------------------------

\begin{problem}[Posted by \href{https://artofproblemsolving.com/community/user/22793}{April}]
	How many times changes the sign of the function \[ f(x)=\cos x\cos\frac{x}{2}\cos\frac{x}{3}\cdots\cos\frac{x}{2009}\] at the interval $ \left[0, \frac{2009\pi}{2}\right]$?
	\flushright \href{https://artofproblemsolving.com/community/c6h276374}{(Link to AoPS)}
\end{problem}



\begin{solution}[by \href{https://artofproblemsolving.com/community/user/29428}{pco}]
	\begin{tcolorbox}How many times changes the sign of the function
\[ f(x) = \cos x\cos\frac {x}{2}\cos\frac {x}{3}\cdots\cos\frac {x}{2009}
\]
at the interval $ \left[0, \frac {2009\pi}{2}\right]$?\end{tcolorbox}

Very nice problem !

[hide="My solution proposal"]
We are looking for the number of roots of $ f(x) = 0$ in $ (0,\frac {2009\pi}{2})$ whose order is odd. (I take the open interval since there is no sign change at the bounds, even if this is a root).

Roots are $ x = kp\frac {\pi}{2}$ for any $ k\in[1,2009]$ and odd $ p$ such that $ 0 < kp < 2009$

So roots are $ r_n = n\frac {\pi}{2}$ with $ n\in[1,2008]$ and n having some odd divisors.

Order of the root $ r_n$ is the number of odd divisors of $ n$. 

So, if $ n = 2^m\prod_{odd\cdot prime\cdot p_i}p_i^{k_i}$, the order of $ r_n$ is $ \prod (k_i + 1)$
So the only roots whose order is odd are roots whose all $ k_i$ are even and so roots which may be written $ 2^ma^2$ (with odd a).

So the required number $ X$ is the number of integers in $ [1,2008]$ of the form $ n^2$ or $ 2n^2$

So $ X = [\sqrt {2008}] + [\sqrt {1004}] = 75$
[\/hide]
\end{solution}
*******************************************************************************
-------------------------------------------------------------------------------

\begin{problem}[Posted by \href{https://artofproblemsolving.com/community/user/54046}{SUPERMAN2}]
	Find all polynomial $ P(x)$ with real coefficients satisfying $$ P^2(x)+P^2(x-1)=2(P(x)-x)^2.$$
Note that by $ P^2(x)$ we mean $P(x)\cdot P(x)$.
	\flushright \href{https://artofproblemsolving.com/community/c6h276670}{(Link to AoPS)}
\end{problem}



\begin{solution}[by \href{https://artofproblemsolving.com/community/user/29428}{pco}]
	\begin{tcolorbox}Find all polynomial $ P(x)$ with real coefficients satisfying $ P^2(x) + P^2(x - 1) = 2(P(x) - x)^2$
Note:$ P^2(x) = P(x)P(x)$\end{tcolorbox}

We can write this equality as $ P^2(x)-(P(x)-x)^2=(P(x)-x)^2-P^2(x-1)$ and so :

$ \boxed{x(2P(x)-x)=(P(x)+P(x-1)-x)(P(x)-P(x-1)-x)}$

If the degree of $ P(x)$ is $ n>1$, LHS has degree $ n+1$, but $ P(x)+P(x-1)-x$ has degree $ n$ and so $ P(x)-P(x-1)-x$ has degree $ 1$ and so $ P(x)$ has degree 2. So $ P(x)$ has degree $ 0,1$ or $ 2$.

Let then $ P(x)=ax^2+bx+c$ and so $ P(x-1)=ax^2+(b-2a)x+a-b+c$. The equation becomes :

$ x(2ax^2+(2b-1)x+2c)=(2ax^2+(2b-2a-1)x+a-b+2c)((2a-1)x-a+b)$

Equality between powers of $ x^3$ is $ 2a=2a(2a-1)$ and so $ a=0$ or $ a=1$

$ a=0$ $ \implies$ $ x((2b-1)x+2c)=((2b-1)x-b+2c)(b-x)$. Comparing $ x^2$ terms implies $ b=\frac{1}{2}$ and then $ 2cx=(-\frac{1}{2}+2c)(\frac{1}{2}-x)$ which is impossible.

So $ a=1$ and we have : $ x(2x^2+(2b-1)x+2c)=(2x^2+(2b-3)x+1-b+2c)(x-1+b)$ and so :

$ (2b-1)x^2+2cx=(4b-5)x^2+(4+2c+2b^2-6b)x  +(b-1) (1-b+2c)$ and so :

$ (2b-4)x^2+(4+2b^2-6b)x  +(b-1) (1-b+2c) = 0$ $ \implies$ $ b=2$ and $ c=\frac{1}{2}$

And so the only polynomial such that $ P^2(x) + P^2(x - 1) = 2(P(x) - x)^2$ is $ \boxed{P(x)=x^2+x+\frac{1}{2}}$
\end{solution}



\begin{solution}[by \href{https://artofproblemsolving.com/community/user/53406}{stephencheng}]
	\begin{tcolorbox}[quote="SUPERMAN2"]Find all polynomial $ P(x)$ with real coefficients satisfying $ P^2(x) + P^2(x - 1) = 2(P(x) - x)^2$
Note:$ P^2(x) = P(x)P(x)$\end{tcolorbox}

And so the only polynomial such that $ P^2(x) + P^2(x - 1) = 2(P(x) - x)^2$ is $ \boxed{P(x) = x^2 + x + \frac {1}{2}}$\end{tcolorbox}

\begin{bolded}Solution:\end{bolded}

The condition $ \Leftrightarrow P^2(x) - 4xP(x) + 2x^2 = P^2(x - 1)$
$ \Leftrightarrow (P(x) + P(x - 1) - 2x)(P(x) - P(x - 1) - 2x) = 2x^2$

So $ P(x)$ is of degree $ 0,1,2$

\begin{bolded}Case 1:\end{bolded} $ P(x)$ is of degree $ 0,1$ and it's easy to check that there are no solutions in this case.

\begin{bolded}Case 2:\end{bolded} $ P(x)$ is of degree $ 2$
Let $ P(x) = ax^2 + bx + c$, then

$ P(x) - P(x - 1) - 2x = a(2x - 1) + b - 2x$ must be of degree $ 0$, so $ a = 1$

Now $ L.H.S. = (b - 1)(P(x) + P(x - 1) - 2x) = (b - 1)(2x^2 + (2b - 4)x + (2c - b + 1)) = 2x^2$

Coefficient of $ x$ is $ 0$ $ \Rightarrow 2b - 4 = 0 \Rightarrow P(x) = x^2 + 2x + \frac {1}{2}$

Sorry, pco . Also the mistake is now corrected now.
\end{solution}



\begin{solution}[by \href{https://artofproblemsolving.com/community/user/29428}{pco}]
	\begin{tcolorbox} ... $ \implies$ $ b = 2$ and $ c = \frac {1}{2}$

And so the only polynomial such that $ P^2(x) + P^2(x - 1) = 2(P(x) - x)^2$ is $ \boxed{P(x) = x^2 + x + \frac {1}{2}}$\end{tcolorbox}

Just a typo error. $ b = 2$ so :

the only polynomial such that $ P^2(x) + P^2(x - 1) = 2(P(x) - x)^2$ is $ \boxed{P(x) = x^2 + 2x + \frac {1}{2}}$

And I checked it ... twice :)

And your own error is here :
\begin{tcolorbox} ... Now $ L.H.S. = (b - 1)(P(x) + P(x - 1) - 2x) = (b - 1)(2x^2 + (2b - 2)x + (2c - b + 1)) = 2x^2$ \end{tcolorbox}

Coefficient of $ x$ in LHS is $ 2b-4$ and not $ 2b-2$
\end{solution}
*******************************************************************************
-------------------------------------------------------------------------------

\begin{problem}[Posted by \href{https://artofproblemsolving.com/community/user/34189}{tdl}]
	Suppose that the function $ f: \mathbb R \rightarrow \mathbb R$ satisfies \[{ f\left(\frac {x + y}{x - y}\right) = \frac {f(x) + f(y)}{f(x) -f( y)}},\] for all reals $x$ and $y$ with $x\neq y$.

a) Find $ f(2)$
b) Find $ f(x)$
	\flushright \href{https://artofproblemsolving.com/community/c6h276874}{(Link to AoPS)}
\end{problem}



\begin{solution}[by \href{https://artofproblemsolving.com/community/user/54383}{EastyMoryan}]
	$ P(1, - 1) \rightarrow f(0) = \frac {f(0)}{f(2)} \rightarrow f(0) = 0$
$ P(1,0) \rightarrow f(1) = \frac {f(1)}{f(1)} \rightarrow f(1) = 1$

Let $ y = 1 - x$
$ f(\frac {1}{2x - 1}) = \frac {1}{f(2x - 1)}$
So, by setting $ a = \frac {1}{2x - 1}$ we see that $  f(a)f (\frac {1}{a} ) = 1 $ (1)

I'm pretty sure the only functions that satisfy this are $ f(x) = x$ (2) and $ f(x) = \frac {1}{x}$, the second of which is not defined for $ x = - y$, and thus isn't the desired function, but I'm unsure how to go from (1) to (2).

Suffice it to say,$ f(2) = 2$and $ f(x) = x$.
\end{solution}



\begin{solution}[by \href{https://artofproblemsolving.com/community/user/46039}{ll931110}]
	\begin{tcolorbox}$ P(1, - 1) \rightarrow f(0) = \frac {f(0)}{f(2)} \rightarrow f(0) = 0$
$ P(1,0) \rightarrow f(1) = \frac {f(1)}{f(1)} \rightarrow f(1) = 1$

Let $ y = 1 - x$
$ f(\frac {1}{2x - 1}) = \frac {1}{f(2x - 1)}$
So, by setting $ a = \frac {1}{2x - 1}$ we see that $  f(a)f (\frac {1}{a} ) = 1 $ (1)

I'm pretty sure the only functions that satisfy this are $ f(x) = x$ (2) and $ f(x) = \frac {1}{x}$, the second of which is not defined for $ x = - y$, and thus isn't the desired function, but I'm unsure how to go from (1) to (2).

Suffice it to say,$ f(2) = 2$and $ f(x) = x$.\end{tcolorbox}

I think you're mistaken at the beginning of your solution
$ P(1, - 1) \rightarrow f(0) = \frac {f(0)}{f(2)} \rightarrow f(0) = 0$ is incorrect. $ f(2) = 1$ satisfies this equation too

It's easy to verify that $ f(xy) = f(x).f(y) \forall x,y \in R$, but it isn't enough information to continue.
\end{solution}



\begin{solution}[by \href{https://artofproblemsolving.com/community/user/31917}{daniel73}]
	For any two reals $ u,v$ with $ u\neq0$, take $ x=\frac{u(v+1)}{2}$, $ y=\frac{u(v-1)}{2}$, and $ \frac{x+y}{x-y}=v$, $ x-y=u$, $ x+y=uv$, or $ f(u)f(v)=f(uv)$ is necessary for any two reals, not both zero.  Assume now that $ g$ satisfies $ g(u)g(v)=g(uv)$ for any two reals $ u,v$, not both zero.  Taking $ v=\frac{1}{u}$, then $ g(v)g(\frac{1}{v})=1$, or if $ x\neq y$, then ${ g(\frac{x+y}{x-y}})=\frac{g(x+y)}{g(x-y)}$, or it is equivalent to substitute in the problem statement the given condition by $ f(x)f(y)=f(xy)$ for any two reals $ x,y$, not both zero.  Now this solution, appart from the trivial solution $ f(x)=0$, has the infinitely many solutions $ f(x)=x^a$ for any real $ a$, as substitution clearly shows.  If I remember correctly, there are no more solutions if we add the condition that $ f$ is continuous, but there are also infinitely many solutions where $ f$ is not continuous...
\end{solution}



\begin{solution}[by \href{https://artofproblemsolving.com/community/user/34189}{tdl}]
	Sorry! There is a mistake in my problem! I have fixed it now!  :blush:
\end{solution}



\begin{solution}[by \href{https://artofproblemsolving.com/community/user/29428}{pco}]
	\begin{tcolorbox}Let a function $ f: R \rightarrow R$ satisfy: ${ f(\frac {x + y}{x - y}) = \frac {f(x) + f(y)}{f(x) - f( y)}\ \forall x\neq y}$.
a) Find $ f(2)$
b) Find $ f(x)$\end{tcolorbox}

[Second part of this demo from ll931110's idea]

Let $ P(x,y)$ be the assertion $ f(\frac {x + y}{x - y}) = \frac {f(x) + f(y)}{f(x) - f(y)}$

$ P(x,0)$ $ \implies$ $ f(1) = \frac {f(x) + f(0)}{f(x) - f(0)}$ and so $ f(x)(f(1) - 1) = f(0)(f(1) + 1)$ and so, since $ (f(x)$ cant be a constant, else RHS of initial assertion would be undefined) :
$ f(0) = 0$
$ f(1) = 1$
$ P(0,1)$ $ \implies$ $ f( - 1) = - 1$

Compare now $ P(x,1)$ and $ P(xy,y)$ for $ x\neq 1$ and $ y\neq 0$. The two LHS are equal, so are the two RHS : $ \frac {f(x) + 1}{f(x) - 1} = \frac {f(xy) + f(y)}{f(xy) - f(y)}$ And so :

$ f(x)f(xy) - f(x)f(y) + f(xy) - f(y)$ $ = f(xy)f(x) + f(y)f(x) - f(xy) - f(y)$ and so $ \boxed{f(xy) = f(x)f(y)}$  and this is still true for $ x = 1$ or $ y = 0$

Then $ f(x^2) = f^2(x)$ and $ f(x) > 0$ $ \forall x > 0$

Let then $ x > y > 0$. $ \frac {x + y}{x - y} > 0$, so $ f(\frac {x + y}{x - y}) > 0$, so $ \frac {f(x) + f(y)}{f(x) - f(y)} > 0$, so, since $ f(x) > 0$ and $ f(y) > 0$ : $ f(x) - f(y) > 0$ and so $ f(x) > f(y)$. So $ f(x)$ is a strictly increasing function for $ x > 0$

But $ f(xy) = f(x)f(y)$ and $ f(x)$ strictly increasing for $ x > 0$ implies easily $ f(x) = x^c$ $ \forall x > 0$ (thanks ll931110).

Then $ P(2,1)$ $ \implies$ $ 3^c = \frac {2^c + 1}{2^c - 1}$ and so $ 6^c - 2^c - 3^c = 1$. It's easy to verify that $ g(x) = 6^x - 3^x - 2^x - 1$ is a strictly decreasing negative function for $ x < 0$ and strictly increasing function for $ x > 0$. So $ 1$ is the unique zero of this function and so $ c = 1$

And $ f(x) = x$ $ \forall x > 0$
And, since $ f(0) = 0$ and $ f( - x) = f( - 1)f(x) = - f(x)$ : $ f(x) = x$ $ \forall x$

And, btw, $ f(2) = 2$
\end{solution}
*******************************************************************************
-------------------------------------------------------------------------------

\begin{problem}[Posted by \href{https://artofproblemsolving.com/community/user/51029}{mathVNpro}]
	Find all functions $ f: (0,\infty)\longrightarrow (0,\infty)$ such that:
i) $ f(x) \in (1, \infty)$ for each $ x\in (0,1)$, and
ii) For all $x,y \in (0,\infty)$,
\[ f(xf(y)) = yf(x).\]
	\flushright \href{https://artofproblemsolving.com/community/c6h277031}{(Link to AoPS)}
\end{problem}



\begin{solution}[by \href{https://artofproblemsolving.com/community/user/16261}{Rust}]
	Let $ y = \frac {1}{f(x)}$, then exist $ z = xf(\frac {1}{f(x)})$, suth that $ f(z) = 1$.
$ y = z$ give $ f(x) = zf(x)\to z = 1$. Therefore $ f(\frac {1}{f(x)})\equiv \frac 1x.$
$ x = 1$ give $ f(f(y))\equiv y$. Therefore $ f$ is bijective and ($ y = \frac 1x$) $ f(\frac 1x) = \frac {1}{f(x)}$.
It give $ f(x) > 1,x < 1,$ and $ f(x) < 1,x > 1$, therefore $ f(x) = x$ only for $ x = 1$.
$ y = x$ give $ f(xf(x))\equiv xf(x)$. therefore $ xf(x)\equiv 1$ or $ f(x) = \frac 1x$.
\end{solution}



\begin{solution}[by \href{https://artofproblemsolving.com/community/user/29428}{pco}]
	\begin{tcolorbox} ...
It give $ f(0,1)\to (1,\infty), f(1,\infty)\to (0,1)$ and $ f(x) = x\to x = 1$.
...\end{tcolorbox}

I liked this.
:)
\end{solution}



\begin{solution}[by \href{https://artofproblemsolving.com/community/user/16261}{Rust}]
	I edit my post.
\end{solution}



\begin{solution}[by \href{https://artofproblemsolving.com/community/user/29428}{pco}]
	\begin{tcolorbox}I edit my post.\end{tcolorbox}

There was no irony at all in my post and I think your previous writing was very easy to understand.

Sorry if you thought I was ironic.  :oops:
\end{solution}
*******************************************************************************
-------------------------------------------------------------------------------

\begin{problem}[Posted by \href{https://artofproblemsolving.com/community/user/43461}{mathson}]
	Find all polynomials $ f: \mathbb{R} \to \mathbb{R}$ satisfy \[ f(a-b) + f(b-c) + f(c-a) = 2f(a+b+c)\]
for all $ a,b,c \in \mathbb{R}$ with $ ab+bc+ac = 0$.
	\flushright \href{https://artofproblemsolving.com/community/c6h277054}{(Link to AoPS)}
\end{problem}



\begin{solution}[by \href{https://artofproblemsolving.com/community/user/29428}{pco}]
	\begin{tcolorbox}Find all polynomials $ f: \mathbb{R} \to \mathbb{R}$ satisfy
\[ f(a - b) + f(b - c) + f(c - a) = 2f(a + b + c)
\]
for all $ a,b,c \in \mathbb{R}$ with $ ab + bc + ac = 0$.\end{tcolorbox}

$ a = b = 0$ and $ c = x$ $ \implies$ $ f(x) = f( - x) + f(0)$ but $ a = b = 0$ and $ c = - x$ $ \implies$ $ f( - x) = f(x) + f(0)$ and so $ f(x)$ is even.

Take then $ a = 3x$, $ b = 6x$ and $ c = - 2x$. The condition $ ab + bc + ca = 0$ is true and we get $ f( - 3x) + f(8x) + f( - 5x) = 2f(7x)$ $ \forall x$ or $ f(3x) + f(8x) + f(5x) = 2f(7x)$

Equating coefficients of the two polynomials (LHS and RHS), we get $ 3^k + 8^k + 5^k = 2\cdot 7^k$ $ \forall k$ such that some $ a_kx^k$ with $ a_k\neq 0$ is in $ f(x)$ ($ k$ is even)

This equality is impossible for $ k\geq 6$ since then, $ LHS > 8^k > 2\cdot 7^k$

Checking explicitely values $ k = 0$, $ k = 2$ and $ k = 4$ gives two matches for $ k = 2$ and $ k = 4$ and we need only to check if these two mandatory values fit the equation.

[hide="Some rather ugly checking, to be improved"]
1) check for $ f(x) = x^2$
$ (a - b)^2 + (b - c)^2 + (c - a)^2 =$ $ 2(a^2 + b^2 + c^2) - 2(ab + bc + ca)$ $ = 2(a^2 + b^2 + c^2) + 4(ab + bc + ca)$ $ = 2(a + b + c)^2$
Q.E.D

2) check for $ f(x) = x^4$
I had no courage to develop these polynomials and I suggest another check :
Any $ (a,b,c)$ such that $ ab + bc + ca = 0$ may be written $ ((x + 1)y,xy(x + 1), - xy)$ so we just have to check for $ (x + 1,x(x + 1), - x)$ so :
is $ (x^2 - 1)^4 + (x^2 + 2x)^4 + (2x + 1)^4 = 2(x^2 + x + 1)^4$ true for any $ x$ ?
Instead of developing, I suggest to consider that $ RHS - LHS$ is a polynomial whose degree is at most $ 7$ ($ x^8$ term obviously disappears) and so check this equality for 8 differents values. And, since replacing $ x$ by $ - 1 - x$ clearly maintains the equality, I suggest to check for 4 values :
For $ x = 0$, LHS is $ 2$ and RHS is $ 2$, so we have equality for $ x = 0$ and for $ x = - 1$
For $ x = 1$, LHS is $ 162$ and RHS is $ 162$, so we have equality for $ x = 1$ and for $ x = - 2$
For $ x = 2$, LHS is $ 4802$ and RHS is $ 4802$, so we have equality for $ x = 2$ and for $ x = - 3$
For $ x = 3$, LHS is $ 57122$ and RHS is $ 57122$, so we have equality for $ x = 3$ and for $ x = - 4$
So we have equality for any $ x$ (we have a polynomial whose degree is at most $ 7$ with at least $ 8$ different roots).
[\/hide]

And so the answer is $ \boxed{f(x) = \alpha x^4 + \beta x^2}$ for any real $ \alpha$ and $ \beta$
\end{solution}



\begin{solution}[by \href{https://artofproblemsolving.com/community/user/9049}{nsato}]
	This is 1994 IMO, #2:
[url]http://www.artofproblemsolving.com/Forum/viewtopic.php?t=14021[\/url]
\end{solution}
*******************************************************************************
-------------------------------------------------------------------------------

\begin{problem}[Posted by \href{https://artofproblemsolving.com/community/user/43461}{mathson}]
	1) Find the close form of
\[ T(x) = 3 \sum_{k = 0}^{n} \binom{3n}{3k}.
\]
Please, I need algebric solution & put it step by step (Anyone have books of generating functions ?).

2) Let $ f: \mathbb{R} \to \mathbb{R}$ (decreasing function), and $ a,b,c \in \mathbb{R}^ +$ satisfy
\[ a + \sqrt 2 = b + \sqrt 3 = c + \sqrt 5 = k.
\]
Prove that
\[ f\left(a\sqrt 3 + b\sqrt 5 + c\sqrt 2 \right) \ge f(k^2).
\]
	\flushright \href{https://artofproblemsolving.com/community/c6h277283}{(Link to AoPS)}
\end{problem}



\begin{solution}[by \href{https://artofproblemsolving.com/community/user/29428}{pco}]
	\begin{tcolorbox} 2) Let $ f: \mathbb{R} \to \mathbb{R}$, and $ a,b,c \in \mathbb{R}^ +$ satisfy
\[ a + \sqrt 2 = b + \sqrt 3 = c + \sqrt 5 = k.
\]
Prove that
\[ f\left(a\sqrt 3 + b\sqrt 5 + c\sqrt 2 \right) \ge f(k^2).
\]
\end{tcolorbox}

This is obviously wrong.

As soon as $ k>\sqrt 5$ is such that $ k^2\neq (k-\sqrt 2)\sqrt 3 + (k-\sqrt 3)\sqrt 5 + (k-\sqrt 5)\sqrt 2$ you can choose a $ f(x)$ such that  $ f(k^2)< f(a\sqrt 3 + b\sqrt 5 + c\sqrt 2)$
\end{solution}



\begin{solution}[by \href{https://artofproblemsolving.com/community/user/29428}{pco}]
	\begin{tcolorbox}1) Find the close form of
\[ T(x) = 3 \sum_{k = 0}^{n} \binom{3n}{3k}.
\]
Please, I need algebric solution & put it step by step (Anyone have books of generating functions ?).
\end{tcolorbox}

Sorry, but I dont know what is an "algebraic" solution. Hope this will fit your requirements :

Let $ S(x) = (1 + x)^{3n} = \sum_{k = 0}^{3n}\binom{3n}{k}x^{k}$

$ S(x) = \sum_{k = 0,k = 0\pmod 3}^{3n}\binom{3n}{k}x^{k}$ $ + \sum_{k = 0,k = 1\pmod 3}^{3n}\binom{3n}{k}x^{k}$ $ + \sum_{k = 0,k = 2\pmod 3}^{3n}\binom{3n}{k}x^{k}$

Let then complex number (cubic root of 1) $ j = e^{\frac {2i\pi}{3}}$. Since $ j^3 = 1$, we can write : $ j^{k} = j^{k\pmod 3}$

Let then complex number (another cubic root of 1) $ j^2 = e^{\frac {4i\pi}{3}}$. Since $ (j^2)^3 = 1$, we can also write : $ (j^2)^{k} = (j^2)^{k\pmod 3}$

$ (1 + 1)^{3n} = S(1) = \sum_{k = 0,k = 0\pmod 3}^{3n}\binom{3n}{k}$ $ + \sum_{k = 0,k = 1\pmod 3}^{3n}\binom{3n}{k}$ $ + \sum_{k = 0,k = 2\pmod 3}^{3n}\binom{3n}{k}$

$ (1 + j)^{3n} = S(j) = \sum_{k = 0,k = 0\pmod 3}^{3n}\binom{3n}{k}$ $ + \sum_{k = 0,k = 1\pmod 3}^{3n}\binom{3n}{k}j$ $ + \sum_{k = 0,k = 2\pmod 3}^{3n}\binom{3n}{k}j^2$

$ (1 + j^2)^{3n} = S(j^2) = \sum_{k = 0,k = 0\pmod 3}^{3n}\binom{3n}{k}$ $ + \sum_{k = 0,k = 1\pmod 3}^{3n}\binom{3n}{k}j^2$ $ + \sum_{k = 0,k = 2\pmod 3}^{3n}\binom{3n}{k}j$

Notice that $ (1 + j)^3 = ( - j^2)^3 = - 1$ and $ (1 + j^2)^3 = ( - j)^3 = - 1$ and add these 3 equalities and remember $ 1 + j + j^2 = 0$ and you get :

$ 2^{3n} + 2( - 1)^n$ $ = 3\sum_{k = 0,k = 0\pmod 3}^{3n}\binom{3n}{k}$ $ = 3\sum_{k = 0}^{n}\binom{3n}{3k}$


Hence the result : $ 3\sum_{k = 0}^{n}\binom{3n}{3k}$ $ = 2^{3n} + 2( - 1)^n$
\end{solution}



\begin{solution}[by \href{https://artofproblemsolving.com/community/user/43461}{mathson}]
	\begin{tcolorbox}[quote="mathson"] 2) Let $ f: \mathbb{R} \to \mathbb{R}$, and $ a,b,c \in \mathbb{R}^ +$ satisfy
\[ a + \sqrt 2 = b + \sqrt 3 = c + \sqrt 5 = k.
\]
Prove that
\[ f\left(a\sqrt 3 + b\sqrt 5 + c\sqrt 2 \right) \ge f(k^2).
\]
\end{tcolorbox}

This is obviously wrong.

As soon as $ k > \sqrt 5$ is such that $ k^2\neq (k - \sqrt 2)\sqrt 3 + (k - \sqrt 3)\sqrt 5 + (k - \sqrt 5)\sqrt 2$ you can choose a $ f(x)$ such that  $ f(k^2) < f(a\sqrt 3 + b\sqrt 5 + c\sqrt 2)$\end{tcolorbox}

Sorry, I forgot to write : $ f$ is decreasing function.

By the way, Your solution for the second one is elegant.
\end{solution}



\begin{solution}[by \href{https://artofproblemsolving.com/community/user/29428}{pco}]
	\begin{tcolorbox} 2) Let $ f: \mathbb{R} \to \mathbb{R}$ (decreasing function), and $ a,b,c \in \mathbb{R}^ +$ satisfy
\[ a + \sqrt 2 = b + \sqrt 3 = c + \sqrt 5 = k.
\]
Prove that
\[ f\left(a\sqrt 3 + b\sqrt 5 + c\sqrt 2 \right) \ge f(k^2).
\]
\end{tcolorbox}

You have $ a=k-\sqrt 2$, $ b= k- \sqrt 3$ and $ c = k - \sqrt 5$

Since $ f(x)$ is a decreasing function, your requirement is to show that $ a\sqrt3 + b\sqrt 5 + c\sqrt 2\leq k^2$ $ \forall k>\sqrt 5$ (in order $ a,b,c\in\mathbb R^+$) so that :

$ (k - \sqrt 2)\sqrt 3 + (k - \sqrt 3)\sqrt 5 + (k - \sqrt 5)\sqrt 2 \leq k^2$

$ k^2 - k(\sqrt 3 + \sqrt 5 + \sqrt 2) + \sqrt 6 + \sqrt{15} + \sqrt{10} \geq 0$

The  minimum value of LHS is reached for $ k=\frac{\sqrt 3 + \sqrt 5 + \sqrt 2}{2}$ and its value is :

$ m=\sqrt 6 + \sqrt{15} + \sqrt{10} - \frac{(\sqrt 3 + \sqrt 5 + \sqrt 2)^2}{4}$

$ 4m=4\sqrt 6 + 4\sqrt{15} + 4\sqrt{10} -$ $ (3 + 5 + 2 + 2\sqrt 6 + 2\sqrt {10} + 2\sqrt {15})$

$ 4m=2\sqrt 6 + 2\sqrt{15} + 2\sqrt{10} - 10$

And, since $ 2\sqrt{15} > 6$ and $ 2\sqrt{10} > 6$, we obviously have $ 2m > 12 - 10 > 0$

Q.E.D.

(but it's not an olympiad problem) :(
\end{solution}
*******************************************************************************
-------------------------------------------------------------------------------

\begin{problem}[Posted by \href{https://artofproblemsolving.com/community/user/49444}{Xaenir}]
	Let $a$ and $b$ be given real numbers such that $1<a^3<b$. Find all continuous functions $ f: (a,b)\to \mathbb R$ satisfying \[ f(xyz)=f(x)+f(y)+f(z),\] for all $x,y,z \in  (a,b)$ such that $xyz < b$.
	\flushright \href{https://artofproblemsolving.com/community/c6h277426}{(Link to AoPS)}
\end{problem}



\begin{solution}[by \href{https://artofproblemsolving.com/community/user/29428}{pco}]
	\begin{tcolorbox}Find all continuous functions $ f: (a,b)\to R$ satisfying $ f(xyz) = f(x) + f(y) + f(z)$,$ \forall x,y,z$ in $ (a,b)$
and $ 1 < a^3 < b$\end{tcolorbox}

Rather interesting problem requiring a lot of rigor. :)

Ok, so the problem is :
Find all continuous functions $ f : (a,b)\to\mathbb R$ such that $ f(xyz)=f(x)+f(y)+f(z)$ $ \forall x,y,z\in(a,b)$ such that $ xyz\in(a,b)$ with $ 1<a<a^3<b$

I prefer working with additions rather than multiplications, so I suggest to use $ g(x)=f(e^x)$
$ g(x)$ is continuous and defined in $ (c,d)$ with $ c=\ln(a)$ and $ d=\ln(b)$ and $ 0<c<3c<d$

So the new problem is :
Find all continuous functions $ g : (c,d)\to\mathbb R$ such that $ g(x+y+z)=g(x)+g(y)+g(z)$ $ \forall x,y,z\in(c,d)$ such that $ x+y+z\in(c,d)$ with $ 0<c<3c<d$

Let $ x_0\in(c,\frac{d-c}{2})$
Let $ u\in(0,d-c-2x_0)$
Let $ p,q\in\mathbb N$ such that $ x_0+\frac{p}{q}u<d-c-x_0$

Let $ n\in\mathbb N_0$ such that $ x_0+(n+1)\frac{u}{q} < d-c-x_0$ ($ n$ exists since $ x_0+\frac{u}{q}\leq$ $ x_0+u<x_0+(d-c-2x_0)=d-c-x_0)$
With such $ n$, we have $ d>d-2x_0-(n+1)\frac{u}{q} > c$. Then : Let $ z\in(c,d-2x_0-(n+1)\frac{u}{q})$

$ g(x_0+(x_0+(n+1)\frac{u}{q})+z)=g(x_0)+g(x_0+(n+1)\frac{u}{q})+g(z)$
$ g((x_0+\frac{u}{q})+(x_0+n\frac{u}{q})+z)=g(x_0+\frac{u}{q})+g(x_0+n\frac{u}{q})+g(z)$

And so $ g(x_0+(n+1)\frac{u}{q})=g(x_0+n\frac{u}{q})+(g(x_0+\frac{u}{q})-g(x_0))$

And so $ g(x_0+k\frac{u}{q})=g(x_0)+k(g(x_0+\frac{u}{q})-g(x_0))$ $ \forall k\in\mathbb N_0$ such that $ x_0+k\frac{u}{q} < d-c-x_0$

So :
$ k=p$ $ \implies$ $ g(x_0+p\frac{u}{q})=g(x_0)+p(g(x_0+\frac{u}{q})-g(x_0))$
$ k=q$ $ \implies$ $ g(x_0+q\frac{u}{q})=g(x_0)+q(g(x_0+\frac{u}{q})-g(x_0))$

$ \implies$ $ g(x_0+\frac{p}{q}u)=g(x_0)+\frac{p}{q}(g(x_0+u)-g(x_0))$

$ \implies$ $ g(x_0+r\cdot u)=g(x_0)+r(g(x_0+u)-g(x_0))$ $ \forall r\in\mathbb Q^+$ such that $ x_0+r\cdot u<d-c-x_0$

Using continuity, we get : $ g(x_0+y u)=g(x_0)+y(g(x_0+u)-g(x_0))$ $ \forall y\in\mathbb R^+$ such that $ x_0+y\cdot u<d-c-x_0$

$ \implies$ $ g(x_0+y)=g(x_0)+y\frac{g(x_0+u)-g(x_0)}{u}$ $ \forall y\in\mathbb R^+$ such that $ x_0+y<d-c-x_0$

$ \implies$ $ g(x)=g(x_0)+(x-x_0)\frac{g(x_0+u)-g(x_0)}{u}$ $ \forall x\in [x_0,d-c-x_0)$

$ \implies$ $ g(x)=\alpha x + \beta$ $ \forall x\in [x_0,d-c-x_0)$ (and $ \alpha$ and $ \beta$ dont depend on $ u$)

$ \implies$ $ \boxed{g(x)=\alpha x + \beta \forall x\in (c,d-2c)}$

Since $ d>3c$, we have $ c<\frac{d}{3}<d-2c$ and so :
$ \forall x\in(c,\frac{d}{3})$, we know that ${ x\in (c,d-2c)}$ and that $ 3x\in(c,d)$. So : $ g(3x)=3g(x)=3\alpha x+3\beta$

So, $ \boxed{g(x)=\alpha x + 3\beta \forall x\in (3c,d)}$ and we have three cases :

1) $ d-2c > 3c$
In this case, the two intervals $ (c,d-2c)$ and $ (3c,d)$ have a common part and so $ \beta=0$ and $ g(x)=\alpha x$ $ \forall x\in (c,d)$
And these conditions are obviously enough.

2) $ d-2c < 3c$
In this case, the two intervals $ (c,d-2c)$ and $ (3c,d)$ have no common part and so :
$ g(x)=\alpha x + \beta \forall x\in (c,d-2c)$
$ g(x)=\alpha x + 3\beta \forall x\in (3c,d)$
$ g(x)$ may have any value for $ x\in[d-2c,3c]$ (in respect with continuity)
And these conditions are enough since $ x+y+z\in(c,d)$ $ \implies$ $ x,y,z\in(c,d-2c)$

3) $ d-2c=3c$
We are in the same situation than in point 2 but the continuity at $ d-2c=3c$ implies $ \beta=0$ and so we are in the case 1).

Final solution :
==============
1) If $ 1 < a < a^3 < a^5\leq b$ : $ f(x)=\alpha\ln(x)$ ($ \alpha\in\mathbb R$)

2) If $ 1 < a < a^3 < b < a^5$ : Let $ \alpha,\beta\in\mathbb R$

$ \forall x\in(a,\frac{b}{a^2})$ : $ f(x)=\alpha\ln(x)+\beta$

$ \forall x\in(a^3,b)$ : $ f(x)=\alpha\ln(x)+3\beta$

$ \forall x\in[\frac{b}{a^2},a^3]$ : $ f(x)$ may have any value in respect with continuity
\end{solution}
*******************************************************************************
-------------------------------------------------------------------------------

\begin{problem}[Posted by \href{https://artofproblemsolving.com/community/user/49444}{Xaenir}]
	Find all continuous functions $ f: \mathbb{R} \to \mathbb{R}$ such that \[ f\left( {\frac {{x + y}} {2}} \right) = \sqrt {f(x)f(y)},\] for all reals $x$ and $y$.
	\flushright \href{https://artofproblemsolving.com/community/c6h277431}{(Link to AoPS)}
\end{problem}



\begin{solution}[by \href{https://artofproblemsolving.com/community/user/53406}{stephencheng}]
	\begin{tcolorbox}Find all continuous functions $ f: \mathbb{R} \to \mathbb{R}$such that $ f\left( {\frac {{x + y}} {2}} \right) = \sqrt {f(x)f(y)}$      , $ \forall x,y$\end{tcolorbox}

If we can find $ x$ such that $ f(x)=0$, then put it into the equation we get $ f(x)=0$ for all $ x$ as a solution.

If not, then put $ p=\frac{x+y}{2}$ we get $ f(p)$ is positive for all $ p$


Now $ \sqrt{f(x+y)f(0)}=f(\frac{x+y}{2})=\sqrt{f(x)f(y)}$

So $ f(x+y)f(0)=f(x)f(y)$
$ \Leftrightarrow In (f(x+y))+In (f(0))=In (f(x))+In(f(y))$

Now let $ g(x)=In (f(x))-In (f(0))$ the equation becomes:

$ g(x)+g(y)=g(x+y)$

As $ g(x)$ is continuous (because $ f(x)$ is continuous), $ g(x)=cx$ (as Cauchy) , where $ c$ is a constant.

Let $ b=f(0)$, we have $ f(x)=e^{cx+In(f(0))}$

Let $ k=e^c$, $ t=f(0)$


We get $ f(x)=tk^x$ is the solution after checking, where $ t$ is any real constant and $ k$ is any non-zero constant.
\end{solution}



\begin{solution}[by \href{https://artofproblemsolving.com/community/user/29428}{pco}]
	\begin{tcolorbox}Find all continuous functions $ f: \mathbb{R} \to \mathbb{R}$such that $ f\left( {\frac {{x + y}} {2}} \right) = \sqrt {f(x)f(y)}$      , $ \forall x,y$\end{tcolorbox}

We clearly have $ f(x)\geq 0$
If $ f(x_0)=0$ for some $ x_0$, we have obviously $ f(x)=f(\frac{x_0+(2x-x_0)}{2})=\sqrt{f(x_0)f(2x-x_0)}=0$ $ \forall x$

Consider now $ f(x)>0$ $ \forall x$

$ f(\frac{(x+y)+0}{2})=\sqrt{f(x+y)f(0)}$ and so $ f(x+y)=\frac{1}{f(0)}f(x)f(y)$

This is a classical equation whose the only continuous solutions are $ f(x)=a\cdot e^{bx}$

Plugging back in the original equation, we get the solutions :

$ f(x)=a\cdot e^{bx}$ $ \forall a\geq 0$
\end{solution}



\begin{solution}[by \href{https://artofproblemsolving.com/community/user/62475}{hqthao}]
	you can take log two side, then you have Cauchy equation
\end{solution}



\begin{solution}[by \href{https://artofproblemsolving.com/community/user/29428}{pco}]
	\begin{tcolorbox}you can take log two side, then you have Cauchy equation\end{tcolorbox}

You must first show that $ f(x)\neq 0$

And, btw, this is exactly what Stephencheng did.  
\end{solution}
*******************************************************************************
-------------------------------------------------------------------------------

\begin{problem}[Posted by \href{https://artofproblemsolving.com/community/user/49444}{Xaenir}]
	Let $K$ be a given real number. Find all continuous functions $ f: \mathbb{R} \to \mathbb{R}$ such that that continuity is not demanded at $ x=K$ for $f$ and \[ f\left( {\frac {{x + y}} {{1 + (xy\/K^2 )}}} \right) = f(x)f(y),\] for all reals $x$ and $y$ such that $xy\neq -K^2$.
	\flushright \href{https://artofproblemsolving.com/community/c6h277432}{(Link to AoPS)}
\end{problem}



\begin{solution}[by \href{https://artofproblemsolving.com/community/user/29428}{pco}]
	\begin{tcolorbox}Find all continuous functions $ f: \mathbb{R} \to \mathbb{R}$ satisfying:$ f\left( {\frac {{x + y}} {{1 + (xy\/K^2 )}}} \right) = f(x)f(y)$ ,$ \forall x,y$\end{tcolorbox}

Notice that this equality can never be verified $ \forall x,y$ since it cant be verified for $ (K,-K)$
If we consider that the problem is modified by adding $ \forall x,y$ \begin{bolded}such that\end{underlined}\end{bolded} $ xy\neq -K^2$, then :

$ y=-x\notin\{-K,+K\}$ $ \implies$ $ f(0)=f(x)f(y)$ and so $ f(x)=c$ and so $ c=c^2$ and only two solutions :

$ f(x)=0$
$ f(x)=1$
\end{solution}



\begin{solution}[by \href{https://artofproblemsolving.com/community/user/49444}{Xaenir}]
	Obviously $ xy \ne  - K^2$
I can't understand how you got $ f(0)=f(x)f(y)$
Anyway , I got the answer $ \left( {\frac{{K + x}}
{{K - x}}} \right)^a$
\end{solution}



\begin{solution}[by \href{https://artofproblemsolving.com/community/user/29428}{pco}]
	\begin{tcolorbox}Obviously $ xy \ne - K^2$\end{tcolorbox}
OK, better to write it.
:)
\begin{tcolorbox}
I can't understand how you got $ f(0) = f(x)f(y)$\end{tcolorbox}

nvm, it's a mistake ...  :blush:
\end{solution}



\begin{solution}[by \href{https://artofproblemsolving.com/community/user/29428}{pco}]
	\begin{tcolorbox}Obviously $ xy \ne - K^2$
I can't understand how you got $ f(0) = f(x)f(y)$
Anyway , I got the answer $ \left( {\frac {{K + x}} {{K - x}}} \right)^a$\end{tcolorbox}

This solution is not continuous for $ x = K$.

This solution is not defined for $ |x| > |K|$ and $ a\notin\mathbb Q$

Some other solutions exist (including all continuous).

I'll post my solution in the next 24h.
\end{solution}



\begin{solution}[by \href{https://artofproblemsolving.com/community/user/29428}{pco}]
	\begin{tcolorbox}Find all continuous functions $ f: \mathbb{R} \to \mathbb{R}$ satisfying:$ f\left( {\frac {{x + y}} {{1 + (xy\/K^2 )}}} \right) = f(x)f(y)$ ,$ \forall x,y$\end{tcolorbox}

Obviously $ f(x)=0$ $ \forall x$ and $ f(x)=1$ $ \forall x$ are solutions (and are the only constant solutions).
Let us look now for non constant solutions.

We'll show that no non constant solution exist (no non constant solution is continuous for $ x=K$). So I'll consider that \begin{bolded}there is a second error in the problem statement\end{bolded}\end{underlined} and that continuity is not demanded for $ x=K$


First get rid of $ K$ (I suppose $ K>0$) : 

$ f(\frac {Kx + Ky} {1 + (KxKy\/K^2 )}) = f(Kx)f(Ky)$ $ \implies$ $ f(K\frac {x + y} {1 + xy}) = f(Kx)f(Ky)$

Let $ g(x)=f(Kx)$ : $ g(\frac {x + y} {1 + xy}) = g(x)g(y)$

1) \begin{bolded}Some preliminary results\end{bolded}\end{underlined} : 
$ y=1$ and $ x\neq -1$ $ \implies$ $ g(x)g(1)=g(1)$ and so either $ g(1)=0$, either $ g(x)=1$ $ \forall x\neq -1$. So $ g(1)=0$
$ y=-1$ and $ x\neq 1$ $ \implies$ $ g(x)g(-1)=g(-1)$ and so either $ g(-1)=0$, either $ g(x)=1$ $ \forall x\neq 1$. So $ g(-1)=0$

If $ g(x_0)=0$ for some $ |x_0|\neq 1$, then, using $ y=\frac{z-x_0}{1-x_0z}$, we get $ f(x_0)f(y)=f(z)$ and so $ f(x)=0$ $ \forall x\neq \frac{1}{x_0}$ and so (since we are studying non constant solutions) $ g(x)\neq 0$ $ \forall x\notin \{-1,+1\}$

Then, $ \forall x,y\in\mathbb R^*$ :

$ g(x)g(\frac{1}{y})=$ $ g(\frac {x + \frac{1}{y}} {1 + \frac{x}{y}})$ $ =g(\frac {1 + xy} {x + y})$

$ g(y)g(\frac{1}{x})=$ $ g(\frac {y + \frac{1}{x}} {1 + \frac{y}{x}})$ $ =g(\frac {1 + xy} {x + y})$

So $ g(x)g(\frac{1}{y})=g(y)g(\frac{1}{x})$, which implies $ \frac{g(x)}{g(\frac{1}{x})}$ $ =\frac{g(y)}{g(\frac{1}{y})}$ $ =c$

So $ g(x)=c\cdot g(\frac{1}{x})$ $ \forall x\neq 0$ and since then $ g(\frac{1}{x})=c\cdot g(x)$, we get $ c=+1$ or $ c=-1$.

So :
either $ g(x)=g(\frac{1}{x})$ $ \forall x\neq 0$
either $ g(x)=-g(\frac{1}{x})$ $ \forall x\neq 0$


2) \begin{bolded}Computation of non constant $ g(x)$ for $ |x|<1$\end{underlined}\end{bolded} : $ g(x)=\left(\frac{1+x}{1-x}\right)^b$ $ \forall x\in(-1,+1)$
Consider $ x,y\in(-1,1)$ : $ \exists! u,v\in\mathbb R$ such that $ x=\tanh(u)$ and $ y=\tanh(v)$ Then :

$ g(\tanh(u+v))=g(\tanh(u)g(\tanh(v))$ $ \forall x,y\in \mathbb R$ and, since $ g(\tanh(x))$ is continuous and non constant : 

$ g(\tanh(x))=e^{ax}$ $ \forall x\in\mathbb R$

$ \implies$ $ g(x)=e^{a\tanh^{[-1]}(x)}$ $ \forall x\in(-1,+1)$

$ \implies$ $ g(x)=\left(\frac{1+x}{1-x}\right)^b$ $ \forall x\in(-1,+1)$ (with $ b=\frac{a}{2})$

Notice here that this function has no limit when$ x\to +1$ and so the function cant be continuous in $ 1$.

\begin{bolded}So the only continuous solution are constant solutions $ g(x)=0$ and $ g(x)=1$ $ \forall x$\end{underlined}\end{bolded}

In case of second error in the problem statement (pleaaaase be more rigorous in your statements) allowing the function not to be continuous at $ +1$, let's go on :)

3) All solutions for $ g(x)$
We have as mandatory conditions for non constant solutions :
$ g(x)=\left(\frac{1+x}{1-x}\right)^b$ $ \forall x\in(-1,+1)$

But $ g(\frac{1}{x})=\epsilon\cdot g(x)$ (with $ \epsilon=1$ or $ \epsilon=-1$) and so :

$ g(x)=\epsilon\left(\frac{1+\frac{1}{x}}{1-\frac{1}{x}}\right)^b$ $ =\epsilon\left(\frac{x+1}{x-1}\right)^b$ $ \forall x\notin[-1,+1]$ and so :

Case $ \epsilon=+1$
$ g(x)=\left|\frac{1+x}{1-x}\right|^b$ $ \forall x\notin\{-1,+1\}$
$ g(1)=0$
$ g(-1)=0$
The continuity at $ x=1$ cant be obtain.
The continuity at $ x=-1$ may be obtained iff $ b>0$
And it's easy to check that this solution matches the original equation.

Case $ \epsilon=-1$
$ g(x)=\left(\frac{1+x}{1-x}\right)^b$ $ \forall x\notin\{-1,+1\}$ but then we need $ b\in\mathbb N$ in order to have power of negative values meaningful (I dont want to discuss about $ b\in\mathbb Q)$
$ g(1)=0$
$ g(-1)=0$
The continuity at $ x=1$ cant be obtain.
The continuity at $ x=-1$ may be obtained iff $ b>0$
And it's easy to check that this solution matches the original equation.


4) \begin{bolded}Synthesis : All solutions of the required <modified> problem\end{bolded}\end{underlined}  :
Modified problem :
Find all functions $ f: \mathbb{R} \to \mathbb{R}$, continuous in $ \mathbb R\backslash\{K\}$ and satisfying:$ f\left( {\frac {{x + y}} {{1 + (xy\/K^2 )}}} \right) = f(x)f(y)$ , (K>0) $ \forall x,y$ such that $ xy\neq -K^2$

Solutions :
$ f(x)=0$ $ \forall x$
$ f(x)=1$ $ \forall x$

$ f(x)=\left|\frac{K+x}{K-x}\right|^b$ $ \forall x\neq K$ and for any $ b\in\mathbb R+$
$ f(K)=0$

$ f(x)=\left(\frac{K+x}{K-x}\right)^n$ $ \forall x\neq K$ and for $ n\in\mathbb N$
$ f(K)=0$
\end{solution}
*******************************************************************************
-------------------------------------------------------------------------------

\begin{problem}[Posted by \href{https://artofproblemsolving.com/community/user/49444}{Xaenir}]
	Find all continuous functions$ f: \mathbb R \to \mathbb R$ which satisfies
\[ f(x + y + axy) = f(x)f(y)
\]
for all $x, y \in \mathbb R$.
	\flushright \href{https://artofproblemsolving.com/community/c6h277434}{(Link to AoPS)}
\end{problem}



\begin{solution}[by \href{https://artofproblemsolving.com/community/user/29428}{pco}]
	\begin{tcolorbox}Find all continuous functions$ f: R \to R$ which satisfy:
\[ f(x + y + axy) = f(x)f(y)
\]

\[ \forall x,y
\]
\end{tcolorbox}

We have the obvious constant solutions $ f(x)=0$ and $ f(x)=1$.  Let us now look for non constant solutions :
If $ a=0$, we get (since $ f(x)$ is continuous) the standard non constant solution  $ f(x)=e^{bx}$

If $ a\neq 0$, let $ g(x)=f(\frac{x-1}{a})$

So $ f(x)=g(ax+1)$ and the initial equation becomes $ g((ax+1)(ay+1))=g(ax+1)g(ay+1)$, so $ g(xy)=g(x)g(y)$ and so $ g(x)=x$ and so $ f(x)=ax+1$

As a synthesis :
Constant solutions are $ f(x)=0$ and $ f(x)=1$

Non constant solutions are :
If $ a=0$, $ f(x)=e^{bx}$, for any real $ b\neq 0$

If $ a\neq 0$, $ f(x)=ax+1$
\end{solution}
*******************************************************************************
-------------------------------------------------------------------------------

\begin{problem}[Posted by \href{https://artofproblemsolving.com/community/user/49444}{Xaenir}]
	Find all continuous functions $f: \mathbb R \to \mathbb R$ satisfying \[ f(x + y) = \frac{{f(x) + f(y) + 2f(x)f(y)}}{{1 - f(x)f(y)}},\]
for all $x, y \in \mathbb R$.
	\flushright \href{https://artofproblemsolving.com/community/c6h277437}{(Link to AoPS)}
\end{problem}



\begin{solution}[by \href{https://artofproblemsolving.com/community/user/29428}{pco}]
	\begin{tcolorbox}Find all continuous functions$ f: R \to R$
 satisfying:$ f(x + y) = \frac {{f(x) + f(y) + 2f(x)f(y)}} {{1 - f(x)f(y)}}$,$ \forall x,y$\end{tcolorbox}

The keypoint here is that, in order this equation be defined $ \forall x,y$, we need to have $ f(x)f(y)\neq 1$ $ \forall x,y$.

As a consequence, we know that $ |f(x)|\neq 1$ $ \forall x$

Let $ x=y=0$. We have $ f(0)(1-f^2(0))=2f(0)+2f^2(0)$ and so $ f(0)(f(0)+1)^2=0$

Since $ f(0)=-1$ is impossible ($ |f(x)|\neq 1$ $ \forall x$), we get $ f(0)=0$.

So, since continuous, $ -1<f(x)<1$ $ \forall x$ 

Then $ y=x$ implies $ f(2x)=\frac{2f(x)}{1-f(x)}$

Now consider $ a < f(x) < b$ $ \forall x$ with $ -1\leq a < 0 < b\leq 1$.

then $ a < f(2x) < b$ implies $ a < \frac{2f(x)}{1-f(x)} < b$ and, since $ 1 - f(x)>0$ : 


$ a - af(x) < 2f(x) < b - bf(x)$ and so $ \frac{a}{a+2} < f(x) < \frac{b}{b+2}$

So $ u_n < f(x) < v_n$ $ \forall x\in \mathbb R$ $ \forall n\in \mathbb N$ where the sequences $ u_n$ and $ v_n$ are :
$ u_1=-1$ and $ u_{n+1}=\frac{u_n}{u_n+2}$
$ v_1=+1$ and $ v_{n+1}=\frac{v_n}{v_n+2}$

And, since these two sequences converge towards 0, $ f(x)=0$ $ \forall x$ and we immediatly check that this function fits the initial equation.
\end{solution}
*******************************************************************************
-------------------------------------------------------------------------------

\begin{problem}[Posted by \href{https://artofproblemsolving.com/community/user/49444}{Xaenir}]
	Find all functions $ f: \mathbb Z \to \mathbb Z$ such that 
\[f(m + n) + f(mn - 1) = f(m)f(n) + 2 ,\]
for all $m,n \in \mathbb Z$.
	\flushright \href{https://artofproblemsolving.com/community/c6h277444}{(Link to AoPS)}
\end{problem}



\begin{solution}[by \href{https://artofproblemsolving.com/community/user/29428}{pco}]
	\begin{tcolorbox}Find all functions $ f: Z \to Z$ such that 
$ %Error. "eqalign" is a bad command.
{ \& f(m + n) + f(mn - 1) = f(m)f(n) + 2 \cr$;$ \forall m,n \in Z \cr}$\end{tcolorbox}

$ n=0$ $ \implies$ $ f(m)+f(-1)=f(m)f(0)+2$

If $ f(0)\neq 1$, this implies $ f(x)=c$ and so $ 2c=c^2+2$, which is impossible. So $ f(0)=1$ and $ f(-1)=2$

Let then $ f(1)=a$

$ n=1$ $ \implies$ $ f(m+1)=af(m)-f(m-1)+2$ and this allow us to compute all values of $ f(x)$. Let's go :

$ f(-1)=2$
$ f(0)=1$
$ f(1)=a$
$ f(2)=a^2+1$
$ f(3)=a^3+2$
$ f(4)=a^4-a^2+2a+1$
$ f(5)=a^5-2a^3+2a^2+a$

$ m=n=2$ $ \implies$ $ f(4)+f(3)=f(2)^2+2$ $ \iff$ $ (a^4-a^2+2a+1)+(a^3+2)=(a^2+1)^2+2$ $ \iff$ $ a(a-1)(a-2)=0$ and so $ a\in\{0,1,2\}$

$ m=2,n=3$ $ \implies$ $ f(5)+f(5)=f(3)f(2)+2$ $ \iff$ $ 2(a^5-2a^3+2a^2+a)=(a^3+2)(a^2+1)+2$
Checking this equality with $ a=0$, $ a=1$ and $ a=2$ implies $ a=2$

So $ f(m+1)=2f(m)-f(m-1)+2$ which is easily solved as $ \boxed{f(m)=m^2+1}$

And it's easy to check that this solution fits the initial equation.
\end{solution}
*******************************************************************************
-------------------------------------------------------------------------------

\begin{problem}[Posted by \href{https://artofproblemsolving.com/community/user/49444}{Xaenir}]
	Find all integers $ k$ for which there does not exist a function $ f: \mathbb N \to \mathbb Z$ which satisfies
(a) $ f(1995)=1996$, and
(b) For all integers $x$ and $y$,
\[ f(xy)=f(x)+f(y)+kf(\gcd(x,y)).\]
	\flushright \href{https://artofproblemsolving.com/community/c6h277446}{(Link to AoPS)}
\end{problem}



\begin{solution}[by \href{https://artofproblemsolving.com/community/user/29428}{pco}]
	\begin{tcolorbox}For which integers $ k$, does ther exist a function $ f: N \to Z$ which satisfies:
$ (a)$ $ f(1995) = 1996$
$ (b)$ $ f(xy) = f(x) + f(y) + kf(gcd(x,y))$ ; $ \forall x,y \in Z$\end{tcolorbox}

I suppose we must read $ \forall x,y \in N$ and not $ \forall x,y \in Z$. If so, let $ x\in\mathbb N$ :

$ f(x^2)=f(x)+f(x)+kf(\gcd(x,x))=$ $ (k+2)f(x)$
$ f(x^3)=f(x^2)+f(x)+kf(\gcd(x,x^3))=$ $ (k+2)f(x)+f(x)+kf(x)=$ $ (2k+3)f(x)$
$ f(x^4)=f((x^2)^2)=(k+2)f(x^2)=(k+2)^2f(x)$
$ f(x^4)=f(x^3\cdot x)=f(x^3)+f(x)+f(\gcd(x^3,x))=$ $ (2k+3)f(x)+f(x)+kf(x)=$ $ (3k+4)f(x)$

So $ (k+2)^2f(x)=(3k+4)f(x)$
So $ k(k+1)f(x)=0$ $ \forall x$

$ f(1995)\neq 0$ $ \implies$ $ k=0$ or $ k=-1$

It exists a function $ f(x)$ for $ k=0$ :
$ f(1)=0$
$ f(\prod p_k^{n_k})=\sum n_kf(p_k)$
$ f(3)+f(5)+f(7)+f(19)=1996$

It exists a function $ f(x)$ for $ k=-1$ :
$ f(1)=0$
$ f(\prod p_k^{n_k})=\sum f(p_k)$
$ f(3)+f(5)+f(7)+f(19)=1996$

So the answer is $ k\in\{-1,0\}$
\end{solution}
*******************************************************************************
-------------------------------------------------------------------------------

\begin{problem}[Posted by \href{https://artofproblemsolving.com/community/user/55184}{mathematikos}]
	Find all functions $f, g: \mathbb R \to \mathbb R$ such that
\[f(x+g(x))=g(x)\]
holds for all real $x$.
	\flushright \href{https://artofproblemsolving.com/community/c6h277490}{(Link to AoPS)}
\end{problem}



\begin{solution}[by \href{https://artofproblemsolving.com/community/user/29126}{MellowMelon}]
	Was going to ask for the domain and codomain, but in fact the answer is completely unchanged as long as it's something with addition and subtraction.

Let $ h(x) = g(x) + x$, so the equation becomes $ f(h(x)) = h(x) - x$. If $ h(a) = h(b)$, then $ f(h(a)) = f(h(b))$, so $ h(a) - a = h(b) - b$ and $ a = b$. Thus $ h$ is injective; we show all of these work. For each $ x$ in the domain of $ h$, define $ f(h(x))$ to be $ h(x) - x$; this is well-defined since $ h$ is injective. This defines $ f$ on the range of $ h$; on anything outside the range define $ f$ arbitrarily. This gets the entire solution set, which happens to be quite large.
\end{solution}



\begin{solution}[by \href{https://artofproblemsolving.com/community/user/29428}{pco}]
	\begin{tcolorbox}HI!
find all functions $ (f,g)$ such that:
$ f(x + g(x)) = g(x)$
thanx.\end{tcolorbox}

Let $ h(x)=x+g(x)$

$ h(x)=h(y)$ $ \implies$ $ f(h(x))=f(g(y))$ $ \implies$ $ g(x)=g(y)$ $ \implies$ $ h(x)-g(x)=h(y)-g(y)$ $ \implies$ $ x=y$ $ \implies$ $ h(x)$ is injective.

Let then $ A=h(\mathbb R)$. $ h(x)$ is then a bijection from $ \mathbb R \to A$ and so $ \exists$ inverse function $ h^{[-1]}$ from $ A\to\mathbb R$

And we have $ g(x)=h(x)-x$ and $ f(x)=g(h^{[-1]}(x))$ $ \forall x\in A$

\begin{bolded}And this is obviously the general solution \end{underlined}\end{bolded}:
Take any injective function $ h(x)$ from $ \mathbb R\to\mathbb R$
Let $ A=h(\mathbb R)$ and the inverse function $ h^{[-1]}$ from $ A\to\mathbb R$

$ \forall x\in \mathbb R$ : $ g(x)=h(x)-x$
$ \forall x\in A$ : $ f(x)=g(h^{[-1]}(x))$
$ \forall x\notin A$ : $ f(x)=$any value.

The fact that this is a solution is obvious.
The fact that all solutions are of this form is a consequence of the beginning of this post.

Hope this will help you  
\end{solution}
*******************************************************************************
-------------------------------------------------------------------------------

\begin{problem}[Posted by \href{https://artofproblemsolving.com/community/user/43461}{mathson}]
	Find all functions $ P: \mathbb{R} \to \mathbb{R}$ satisfys
\[ P(a+b+c) = P(a+b-c+a^2) + P(a-b+c+c^2) + P(-a+b+c+b^2) + 2ab + 2bc + 2ac\]
for all $ a,b,c \in \mathbb{R}$.
	\flushright \href{https://artofproblemsolving.com/community/c6h278156}{(Link to AoPS)}
\end{problem}



\begin{solution}[by \href{https://artofproblemsolving.com/community/user/29428}{pco}]
	\begin{tcolorbox}Find all functions $ P: \mathbb{R} \to \mathbb{R}$ satisfys
\[ P(a + b + c) = P(a + b - c + a^2) + P(a - b + c + c^2) + P( - a + b + c + b^2) + 2ab + 2bc + 2ac
\]
for all $ a,b,c \in \mathbb{R}$.\end{tcolorbox}

None :

$ a=2, b=\frac{7}{2}, c=2$ $ \implies$ $ P(\frac{15}{2})=P(\frac{15}{2})+P(\frac{9}{2})+P(\frac{63}{4})+36$ $ \implies$ $ P(\frac{9}{2})+P(\frac{63}{4})=-36$

$ a=-\frac{9}{2}, b=0, c=0$ $ \implies$ $ P(-\frac{9}{2})=P(\frac{63}{4})+P(-\frac{9}{2})+P(\frac{9}{2})+0$ $ \implies$ $ P(\frac{63}{4})+P(\frac{9}{2})=0$

Hence a contradiction.
\end{solution}
*******************************************************************************
-------------------------------------------------------------------------------

\begin{problem}[Posted by \href{https://artofproblemsolving.com/community/user/60962}{Anni}]
	Find all functions $f: \mathbb R \to \mathbb R$ such that
 \[ f(x)=f\left(\frac{x}{2}\right)+\frac{x}{2}f'(x)\]
for all $x \in \mathbb R$.
	\flushright \href{https://artofproblemsolving.com/community/c6h278400}{(Link to AoPS)}
\end{problem}



\begin{solution}[by \href{https://artofproblemsolving.com/community/user/29428}{pco}]
	So we have to solve $ f(x) = f(\frac x2) + \frac x2f'(x)$ (and $ f(x)$ is continuous and it's derivative exists all over $ \mathbb R$)

Let $ f(x)$ any solution, let $ u\neq 0$ and let $ g(x) = f(x) + (f(0) - f(u))\frac xu - f(0)$. Obviously, $ g(x)$ is a solution and $ g(u) = g(0) = 0$
Let then $ M = \max_{x\in[0,u]}g(x)$ (I write $ [0,u]$ even if $ u < 0$) and any $ x_0\in[0,u]$ such that $ g(x_0) = M$ ($ M$ and $ x_0$ exist since $ g(x)$ is continuous)
If $ M\neq 0$, $ x_0\neq 0$ and $ f'(x_0) = 0$. We have then $ f(x_0) = f(\frac {x_0}{2}) = M$ and so $ f'(\frac {x_0}{2}) = 0$. 
An immediate induction give us $ f(\frac {x_0}{2^n}) = M$ and so $ M = 0$ (since $ g(0) = 0$ and $ g(x)$ is continuous at $ 0$).

Same, Let then $ m = \min_{x\in[0,u]}g(x)$ and any $ x_1\in[0,u]$ such that $ g(x_1) = m$. The same method implies $ m = 0$

So $ g(x) = 0$ $ \forall x\in[0,u]$. and so $ f(x) = \frac {f(u) - f(0)}{u}x + f(0)$ $ \forall x\in[0,u]$

So  $ \frac {f(u) - f(0)}{u} = \frac {f(v) - f(0)}{v}=a$ $ \forall u,v$ with same sign.

So $ f(x) = ax + b$ $ \forall x\geq 0$ and $ f(x) = cx + d$ $ \forall x\leq 0$. But continuity at $ 0$ implies $ b = d$ and existence of $ f'(0)$ implies $ a = c$

And so the only solutions are $ f(x) = ax + b$ $ \forall x\in\mathbb R$ (and we easily verify that these solutions fit).
\end{solution}
*******************************************************************************
-------------------------------------------------------------------------------

\begin{problem}[Posted by \href{https://artofproblemsolving.com/community/user/60962}{Anni}]
	Find all functions $f: \mathbb R \to \mathbb R$ such that
\[ f(x+f(y))=y+f(x+1),\] 
for all $x,y \in \mathbb R$.
	\flushright \href{https://artofproblemsolving.com/community/c6h278402}{(Link to AoPS)}
\end{problem}



\begin{solution}[by \href{https://artofproblemsolving.com/community/user/29428}{pco}]
	\begin{tcolorbox}find all f:R-R such that
  
    f(x+f(y))=y+f(x+1)\end{tcolorbox}

Let $ g(x) = f(x) - 1$ the equation becomes $ g(x + g(y)) = y + g(x)$

$ x = 0$ $ \implies$ $ g(g(y)) = y + g(0)$ and so $ g(x)$ is bijective.
Then $ y = 0$ $ \implies$ $ g(x + g(0)) = g(x)$ and so $ g(0) = 0$ since $ g(x)$ is bijective and we have $ g(g(x)) = x$

Then $ y = g(z)$ $ \implies$ $ g(x + z) = g(x) + g(z)$

So we have $ g(x + y) = g(x) + g(y)$ and $ g(g(x)) = x$

The first part is a classical Cauchy equation and we conclude :

The only continuous solutions are $ f(x) = x + 1$ and $ f(x) = 1 - x$
Infinitely many non continuous solutions (with AC) exist.

In fact all solutions may be presented in a unique method :
Let $ \mathbb R$ considered as $ \mathbb Q$-vector space and $ A$ any $ \mathbb Q$-vector subspace.
Let $ B$ any supplementary (hope this is the good english word) vector subspace of $ A$ ($ B$ always exists with Axiom of Choice)

Let $ a(x)$ and $ b(x)$ the projection functions of $ x$ in $ A$ and $ B$.

Then $ f(x) = 1 + a(x) - b(x)$

$ A = \mathbb R$ gives $ f(x) = x + 1$
$ A = \{0\}$ gives $ f(x) = 1 - x$
any other $ A$ give non continuous solutions.

edit \end{underlined}: sorry, but there are a lot of other non continuous solutions
\end{solution}
*******************************************************************************
-------------------------------------------------------------------------------

\begin{problem}[Posted by \href{https://artofproblemsolving.com/community/user/45765}{popolux}]
	Find all functions $ f: \mathbb{R}\rightarrow \mathbb{R}$ such that for all $x,y\in\mathbb{R}$,
\[f(xy(x+y)) = f(x)f(y)(f(x)+f(y)).\]
	\flushright \href{https://artofproblemsolving.com/community/c6h278426}{(Link to AoPS)}
\end{problem}



\begin{solution}[by \href{https://artofproblemsolving.com/community/user/45765}{popolux}]
	No ideas?Must I post my solution?
\end{solution}



\begin{solution}[by \href{https://artofproblemsolving.com/community/user/29428}{pco}]
	\begin{tcolorbox}Find all functions $ f: \mathbb{R}\rightarrow \mathbb{R}$ such that $ \forall x,y\in\mathbb{R},f(xy(x + y)) = f(x)f(y)(f(x) + f(y))$\end{tcolorbox}

Here is a rather long and not very nice solution. Dont hesitate to post a nicer one :)

Let $ P(x,y)$ be the assertion $ f(xy(x+y))=f(x)f(y)(f(x)+f(y))$

1) Constant solutions are $ f(x)=0$, or $ f(x)=\frac{1}{\sqrt 2}$ and $ f(x)=-\frac{1}{\sqrt 2}$
Just solve the equation $ c=2c^3$

2) Non constant solutions are such that $ f(0)=0$
$ P(0,0)$ $ \implies$ $ f(0)=2f(0)^3$ $ \implies$ $ f(0)\in\{-\frac{1}{\sqrt 2},0,+\frac{1}{\sqrt 2}\}$
$ P(x,0)$ $ \implies$ $ f(0)=f(x)f(0)(f(x)+f(0))$
If $ f(0)\neq 0$, we get $ f(0)=\frac{\epsilon}{\sqrt 2}$ (where $ \epsilon=\pm 1$) and $ f(x)^2+f(x)f(0)-1=0$ and so :
Either $ f(x)=\frac{\epsilon}{\sqrt 2}$, either $ f(x)=\frac{-2\epsilon}{\sqrt 2}$

But $ P(x,x)$ $ \implies$ $ f(2x^3)=2f(x)^3$. So, $ f(x)=\frac{-2\epsilon}{\sqrt 2}$ would imply $ f(2x^3)=\frac{-8\epsilon}{\sqrt 2}$, which is impossible.
So $ f(x)=f(0)$ and $ f(x)$ is constant.
Q.E.D.

3) In non constant solutions, $ f(a)=0$ $ \iff$ $ a=0$
We already know that $ f(0)=0$ (point 2)
If $ f(a)=0$ for some $ a\neq 0$, then $ P(x,a)$ $ \implies$ $ f(ax(x+a))=0$ $ \forall x$
Since the equation $ ax(x+a)=b$ has solutions for any $ b$ such that $ a^4+4ab\geq 0$, we get $ f(b)=0$ $ \forall b$ such that $ a^4+4ab\geq 0$
So $ f(x)=0$ $ \forall x$ with same sign as $ a$.
But $ b=-\frac{a^3}{8}$ is such that $ a^4+4ab=a^4-\frac{a^4}{2}\geq 0$ and so $ f(b)=0$ with $ b$ opposite sign than $ a$
So $ f(x)=0$ $ \forall x$ with same sign as $ b$, so with opposite signe as $ a$
Q.E.D.

4) Non constant solutions are injective
Let $ x\neq 0$ : $ P(x,-x)$ $ \implies$ $ f(0)=f(x)f(-x)(f(x)+f(-x))$ and so $ f(-x)=-f(x)$ (since neither $ f(x)=0$, neither $ f(-x)=0$, according to point 3).
So, suppose $ f(a)=f(b)$, then $ f(-b)=-f(a)$ and $ P(a,-b)$ $ \implies$ $ f(-ab(a-b))=0$ and so $ ab(a-b)=0$
So, either $ a=b$, either $ a=0$ and so $ f(b)=0$ and $ b=0$, either $ b=0$ and so $ f(a)=0$ and so $ a=0$
Q.E.D.

5) Non constant solutions are such that $ f(x)=x$ $ \forall x\in\mathbb Q$ or $ f(x)=-x$ $ \forall x\in\mathbb Q$
5.1) $ f(2x)=2f(x)$
$ P(x,x)$ $ \implies$ $ f(2x^3)=2f(x)^3$
$ P(2x,-x)$ $ \implies$ $ f(2x(-x)(2x-x))=f(2x)f(-x)(f(2x)+f(-x))$ $ \implies$ $ -f(2x^3)=-f(2x)f(x)(f(2x)-f(x))$
$ \implies$ $ 2f(x)^3=f(2x)f(x)(f(2x)-f(x))$
$ \implies$ $ f(x)(f(2x)-2f(x))(f(2x)+f(x))=0$. So :
Either $ f(x)=0$ and so $ x=0$ and so $ f(2x)=2f(x)=0$
Either $ f(2x)=2f(x)$
Either $ f(2x)=-f(x)=f(-x)$ and so $ 2x=-x$ since $ f(x)$ is injective and so $ x=0$ and $ f(2x)=2f(x)$
Q.E.D.

5.2) $ f(x^3)=f(x)^3$ and, as a consequence, $ f(1)=1$ or $ f(1)=-1$
$ P(x,x)$ $ \implies$ $ f(2x^3)=2f(x)^3$. But $ f(2x^3)=2f(x^3)$
Q.E.D.

5.3) $ f(nx)=nf(x)$ $ \forall x$, $ \forall n\in \mathbb N$
This is true for $ n=1$
Suppose it's true for $ n$. Then $ P(x,nx)$ $ \implies$ $ f(n(n+1)x^3)=f(nx)f(x)(f(nx)+f(x))$
$ \implies$ $ nf((n+1)x^3)=n(n+1)f(x)^3$
$ \implies$ $ f((n+1)x^3)=(n+1)f(x^3)$
$ \implies$ $ f((n+1)x)=(n+1)f(x)$ $ \forall x$
Q.E.D

5.4) $ f(x)=x$ $ \forall x\in\mathbb Q$ or $ f(x)=-x$ $ \forall x\in\mathbb Q$
We have $ f(nx)=nf(x)$ $ \forall n\in\mathbb N$
Since we also have $ f(-x)=-f(x)$, we get $ f(nx)=nf(x)$ $ \forall n\in\mathbb Z$
$ f(p\frac xq)=pf(\frac xq)$ and $ f(x)=f(q\frac xq)=qf(\frac xq)$ $ \implies$ $ f(ax)=af(x)$ $ \forall a\in\mathbb Q$
And since $ f(1)=1$ or $ f(1)=-1$, we get the result.

6) Non constant solutions are such that $ f(x)=x$ $ \forall x\in\mathbb R$ or $ f(x)=-x$ $ \forall x\in\mathbb R$
If $ f(x)$ is solution, $ -f(x)$ is solution too. So Wlog consider solutions where $ f(1)=1$ and $ f(x)=x$ $ \forall x\in\mathbb Q$

Let $ a,b$ such that $ b^4+4ab\geq 0$. Then the equation $ xb(x+b)=a$ has solutions and, for $ x$ such that $ xb(x+b)=a$, we have :
$ f(a)=f(x)f(b)(f(x)+f(b))$ and so the equation $ X^2f(b)+Xf(b)^2-f(a)$ has solutions and so $ f(b)^4+4f(a)f(b)\geq 0$

So $ b^4+4ab\geq 0$ $ \implies$ $ f(b)^4+4f(a)f(b)\geq 0$
From this, we can get three important conclusions :
6.1) $ b>0$ $ \implies$ $ f(b)>0$
Suppose $ b>0$ such that $ f(b)<0$. It's then easy to find a rational $ a>0$ great enough to have $ b^4+4ab\geq 0$ and $ f(b)^4+4f(a)f(b)=f(b)^4+4af(b)< 0$

6.2) $ f(x)\leq x$ $ \forall x>0$
Consider now $ a>0$ and $ b<0\in\mathbb Q$ such that $ -\frac{b^3}{4}\geq a$. Then $ b^4+4ab\geq 0$ and so $ f(b)^4+4f(a)f(b)\geq 0$ and so $ -\frac{b^3}{4}\geq f(a)$  (since $ f(b)=b$).

So $ -\frac{b^3}{4}\geq a$ $ \implies$ $ -\frac{b^3}{4}\geq f(a)$ and, since we can always find rational $ b$ such that $ -\frac{b^3}{4}$ is as near of $ a$ as we want : $ f(a)\leq a$ $ \forall a>0$

6.3) $ f(x)\geq x$ $ \forall x>0$
Consider now $ b>0$ and $ a<0\in\mathbb Q$ such that $ -\frac{b^3}{4}\leq a$. Then $ b^4+4ab\geq 0$ and so $ f(b)^4+4f(a)f(b)\geq 0$ and so $ f(b)^4+4af(b)\geq 0$. So, since $ f(b)>0$ (see 6.1) : $ -\frac{f(b)^3}{4}\leq a$

And so, since we can always find rational $ a$ such that $ a$ is as near of $ -\frac{b^3}{4}$ as we want : $ -\frac{f(b)^3}{4}\leq -\frac{b^3}{4}$ and so $ f(b)^3\geq b^3$ and so $ f(b)\geq b$


So 6.2 and 6.3 imply $ f(x)=x$ $ \forall x>0$ and so $ f(x)=x$ $ \forall x$

7) Synthesis of solutions :
$ f(x)=0$
$ f(x)=\frac{1}{\sqrt 2}$
$ f(x)=-\frac{1}{\sqrt 2}$
$ f(x)=x$
$ f(x)=-x$
\end{solution}
*******************************************************************************
-------------------------------------------------------------------------------

\begin{problem}[Posted by \href{https://artofproblemsolving.com/community/user/46787}{moldovan}]
	Let $ f: \mathbb{R} \rightarrow \mathbb{R}$ be a function such that for all $ x,y \in \mathbb{R}$, $ f(x^3+y^3)=(x+y)(f(x)^2-f(x)f(y)+f(y)^2).$ Prove that for all $ x \in \mathbb{R}, f(1996x)=1996f(x)$.
	\flushright \href{https://artofproblemsolving.com/community/c6h278795}{(Link to AoPS)}
\end{problem}



\begin{solution}[by \href{https://artofproblemsolving.com/community/user/29428}{pco}]
	\begin{tcolorbox}Let $ f: \mathbb{R} \rightarrow \mathbb{R}$ be a function such that for all $ x,y \in \mathbb{R}$, $ f(x^3 + y^3) = (x + y)(f(x)^2 - f(x)f(y) + f(y)^2).$ Prove that for all $ x \in \mathbb{R}, f(1996x) = 1996f(x)$.\end{tcolorbox}

Let $ P(x,y)$ be the assertion $ f(x^3 + y^3) = (x + y)(f(x)^2 - f(x)f(y) + f(y)^2)$

$ P(1, - 1)$ $ \implies$ $ f(0) = 0$
$ P(\sqrt [3]x,0)$ $ \implies$ $ f(x) = \sqrt [3]xf(x)^2$. So $ f(x)$ has the same sign as $ x$

Suppose then $ f(ax) = af(x)$ $ \forall x$ for some real $ a > 0$.
$ P(\sqrt [3]ax,0)$ $ \implies$ $ f(ax^3) = \sqrt [3]axf(\sqrt [3]ax)^2$ $ \implies$ $ af(x^3) = \sqrt [3]axf(\sqrt [3]ax)^2$
$ P(x,0)$ $ \implies$ $ f(x^3) = xf(x)^2$ and so, with line above : $ axf(x)^2 = \sqrt [3]axf(\sqrt [3]ax)^2$ 
So $ f(\sqrt [3]ax)^2 = \sqrt [3]{a^2}f(x)^2$ and, since $ a > 0$ and we know that $ f(x)$ has the same sign as $ x$ : $ f(\sqrt [3]ax) = \sqrt [3]{a}f(x)$

$ P(\sqrt [3]ax,x)$ $ \implies$ $ f((a + 1)x^3) = (\sqrt [3]a + 1)x(\sqrt [3]{a^2}f(x)^2 - \sqrt [3]af(x)^2 + f(x)^2)$ $ = (a + 1)xf(x)^2$
And, since $ P(x,0)$ implies $ f(x^3) = xf(x)^2$ we get $ f((a + 1)x^3) = (a + 1)f(x^3)$

So, if $ f(ax) = af(x)$ $ \forall x$, then $ f((a + 1)x) = (a + 1)f(x)$ $ \forall x$

And so, since $ f(1\cdot x) = 1\cdot f(x)$, we get $ f(nx) = nf(x)$ $ \forall x$, $ \forall n\in\mathbb N$

Hence the result.
\end{solution}
*******************************************************************************
-------------------------------------------------------------------------------

\begin{problem}[Posted by \href{https://artofproblemsolving.com/community/user/40002}{Ahiles}]
	$ f(x)$ and $ g(x)$ are two polynomials with nonzero degrees and integer coefficients, such that $ g(x)$ is a divisor of $ f(x)$ and the polynomial $ f(x)+2009$ has $ 50$ integer roots. Prove that the degree of $ g(x)$ is at least $ 5$.
	\flushright \href{https://artofproblemsolving.com/community/c6h279980}{(Link to AoPS)}
\end{problem}



\begin{solution}[by \href{https://artofproblemsolving.com/community/user/53406}{stephencheng}]
	\begin{tcolorbox}[color=darkblue]$ f(x)$ and $ g(x)$ are two polynomials with nonzero degrees and integer coefficients, such that $ g(x)$ is a divisor of $ f(x)$ and the polynomial $ f(x) + 2009$ has $ 50$ integer roots. Prove that the degree of $ g(x)$ is at least $ 5$.[\/color]\end{tcolorbox}

What do you mean by $ g(x)$ is a divisor of $ f(x)$?
\end{solution}



\begin{solution}[by \href{https://artofproblemsolving.com/community/user/154}{Myth}]
	Apparently bad translation.
\end{solution}



\begin{solution}[by \href{https://artofproblemsolving.com/community/user/40002}{Ahiles}]
	What do you mean by "bad translation"????

I ment $ f(x) = g(x)h(x)$
\end{solution}



\begin{solution}[by \href{https://artofproblemsolving.com/community/user/154}{Myth}]
	I meant the following thing:
 does the example $ f(x)=(x-1)^{50}(x+2009)-2009$, $ g(x)=x$ satisfy the condition and $ deg\ g=1$ ?
\end{solution}



\begin{solution}[by \href{https://artofproblemsolving.com/community/user/40002}{Ahiles}]
	Of course not

We can't gave

$ (x-1)^{50}(x+2009)-2009=x h(x)$

And those roots are different...
\end{solution}



\begin{solution}[by \href{https://artofproblemsolving.com/community/user/3236}{test20}]
	\begin{tcolorbox}Apparently bad translation.\end{tcolorbox}
Yes.
I guess that when he writes "[...] and the polynomial f(x)+2009 has 50 integer roots"
he actually wants to say "[...] and the polynomial f(x)+2009 has 50 PAIRWISE DISTINCT integer roots"
\end{solution}



\begin{solution}[by \href{https://artofproblemsolving.com/community/user/10233}{caffeineboy}]
	Write $ f(x) = g(x)h(x)$ and $ h$ has integer coefficients. Then for the 50 integer roots of $ f$, $ g(x)$ is one of the twelve integers dividing 2009. By pidgeonhole, g takes on one of those 12 integers 5 times. Since $ g$ is non constant, g must have degree $ \geq 5$.

PS: Basically the same as http://www.artofproblemsolving.com/Forum/viewtopic.php?t=217338
\end{solution}



\begin{solution}[by \href{https://artofproblemsolving.com/community/user/29428}{pco}]
	\begin{tcolorbox}Write $ f(x) = g(x)h(x)$ and $ h$ has integer coefficients.\end{tcolorbox}

I dont understand why $ h$ has integer coefficients. We know that $ f(x)$ and $ g(x)$ have, but all we know so far is that $ h(x)$ has rational coefficients
\end{solution}



\begin{solution}[by \href{https://artofproblemsolving.com/community/user/29428}{pco}]
	\begin{tcolorbox}Of course not

We can't gave

$ (x - 1)^{50}(x + 2009) - 2009 = x h(x)$

\end{tcolorbox}

Btw, we can obviously write $ f(x)=(x - 1)^{50}(x + 2009) - 2009 = x h(x)$ since $ f(0)=0$
\end{solution}



\begin{solution}[by \href{https://artofproblemsolving.com/community/user/10233}{caffeineboy}]
	\begin{tcolorbox}I dont understand why $ h$ has integer coefficients. We know that $ f(x)$ and $ g(x)$ have, but all we know so far is that $ h(x)$ has rational coefficients\end{tcolorbox}

I'm assuming that (as in the IMC problem) divisibility meant divisibility in $ \mathbb{Z}[x]$, because if $ h$ doesn't need to have integer coefficients, I think the problem is false. (We needed $ h$ to take integer values on those 50 roots because otherwise we have no restrictions on $ g$ other then being a divisor $ f$ and having integer coefficients, which isn't enough to determine anything about $ g$)
\end{solution}



\begin{solution}[by \href{https://artofproblemsolving.com/community/user/29428}{pco}]
	\begin{tcolorbox} I'm assuming that (as in the IMC problem) divisibility meant divisibility in $ \mathbb{Z}[x]$\end{tcolorbox}

So, as Stephencheng said, maybe Ahiles could rewrite his problem :

Must we consider all roots are different pairwise ?

Must we consider that "divisor" means that all three polynomial $ f,g$ and $ h$ are in $ \mathbb Z[X]$, and not only $ f$ and $ g$ ?

Ahiles ? :?:
\end{solution}



\begin{solution}[by \href{https://artofproblemsolving.com/community/user/40002}{Ahiles}]
	[color=darkblue]\begin{bolded}pco\end{bolded}, \begin{bolded}test20\end{bolded}'

CLEARLY roots are different.. we don't say that $ (x - 1)^{50}$ has $ 50$ roots equal to 1... It has just one root....[\/color]
\end{solution}



\begin{solution}[by \href{https://artofproblemsolving.com/community/user/29428}{pco}]
	\begin{tcolorbox}[color=darkblue]\begin{bolded}pco\end{bolded}, \begin{bolded}test20\end{bolded}'

CLEARLY roots are different.. we don't say that $ (x - 1)^{50}$ has $ 50$ roots equal to 1... It has just one root....[\/color]\end{tcolorbox}

 
\end{solution}



\begin{solution}[by \href{https://artofproblemsolving.com/community/user/53406}{stephencheng}]
	\begin{tcolorbox}[quote="caffeineboy"] I'm assuming that (as in the IMC problem) divisibility meant divisibility in $ \mathbb{Z}[x]$\end{tcolorbox}

So, as Stephencheng said, maybe Ahiles could rewrite his problem :

Must we consider all roots are different pairwise ?

Must we consider that "divisor" means that all three polynomial $ f,g$ and $ h$ are in $ \mathbb Z[X]$, and not only $ f$ and $ g$ ?

Ahiles ? :?:\end{tcolorbox}

Actually , just like pco, I also thought that $ h(x)$ was not given to be of integer coefficients. And to prove that $ k h(x)$, where $ k$ is the gcd of the coefficients of $ g(x)$,  must be of integer coefficients is exactly the same as to prove the Gauss's Lemma.

[url]http://en.wikipedia.org\/wiki\/Gauss%27s_lemma_(polynomial)[\/url]
\end{solution}



\begin{solution}[by \href{https://artofproblemsolving.com/community/user/3236}{test20}]
	\begin{tcolorbox}[color=darkblue]\begin{bolded}pco\end{bolded}, \begin{bolded}test20\end{bolded}'
CLEARLY roots are different.. we don't say that $ (x - 1)^{50}$ has $ 50$ roots equal to 1... It has just one root....[\/color]\end{tcolorbox}

My dear child, it seems that for you there is still a lot of mathematics to learn....
\end{solution}



\begin{solution}[by \href{https://artofproblemsolving.com/community/user/40002}{Ahiles}]
	\begin{tcolorbox}[quote="Ahiles"][color=darkblue]\begin{bolded}pco\end{bolded}, \begin{bolded}test20\end{bolded}'
CLEARLY roots are different.. we don't say that $ (x - 1)^{50}$ has $ 50$ roots equal to 1... It has just one root....[\/color]\end{tcolorbox}

My dear child, it seems that for you there is still a lot of mathematics to learn....\end{tcolorbox}

[size=150][color=darkblue]It's not very polite to be so sarcastic......[\/color][\/size]
\end{solution}



\begin{solution}[by \href{https://artofproblemsolving.com/community/user/29428}{pco}]
	OK, calm down. We are sorry.

You must know that very often, when one say that a polynomial has $ n$ roots, these roots may be identical. For example, it's very often said that a polynomial of degree $ n$ always has $ n$ roots in $ \mathbb C$, which obviously includes identical roots.

So, when you want to avoid such a confusion, just try to be more precise and write "distinct roots", for example.

Now, it's clear that your problem deals with distinct roots.

Then, could you say us if the divisibility is considered in $ \mathbb Z[X]$ or not ? (for example, do you consider that $ 2x - 2$ is a "divisor" of $ (x - 1)(x + 1)$ ?)

I suggest you try to avoid be bothered when someone ask some precisions on a problem statement. We all are from different countries and some "not so precise" statements may not be understood in the same way everywhere.

Thanks for your attention, and sorry again if we bothered you.
\end{solution}



\begin{solution}[by \href{https://artofproblemsolving.com/community/user/40002}{Ahiles}]
	I've only translated the proposed problem... I don't now why, but all contestants understood the problem... 
Of course $ h \in \mathbb{Z}[x]$. A don't think that the problem can be solved if it is not....
\end{solution}



\begin{solution}[by \href{https://artofproblemsolving.com/community/user/53406}{stephencheng}]
	\begin{tcolorbox}I've only translated the proposed problem... I don't now why, but all contestants understood the problem... 
Of course $ h \in \mathbb{Z}[x]$. A don't think that the problem can be solved if it is not....\end{tcolorbox}

Even if it is not, the problem can still be solved.  :D
\end{solution}



\begin{solution}[by \href{https://artofproblemsolving.com/community/user/29428}{pco}]
	\begin{tcolorbox} I don't now why, but all contestants understood the problem... \end{tcolorbox}

Ok, sorry for my remarks. The only conclusion is that my level is quite below the level of all contestants. I'll avoid any remark on your posts in the future. And I'll keep in mind that "a polynomial with degree $ n$ always has $ n$ roots in $ \mathbb C$" is wrong, since, for example, $ (z-1)^{50}$ only has one root. 

Thanks for your clever advices.
\end{solution}



\begin{solution}[by \href{https://artofproblemsolving.com/community/user/44083}{jgnr}]
	[hide="Solution"]Let $ f(x)+2009=(x-a_1)(x-a_2)(x-a_3)\ldots(x-a_{50})$ and $ f(x)=g(x)h(x)$. So $ g(x)h(x)=(x-a_1)(x-a_2)\ldots(x-a_{50})-2009$.

So $ g(a_1),g(a_2),\ldots,g(a_{50})$ divides 2009. But $ 2009=7^2\cdot41$ has $ 2\cdot(3\cdot2)=12$ integer divisors (including the negative divisors). By pigeonhole principle, there exist $ \left\lceil\frac{50}{12}\right\rceil=5$ values $ a_i,a_j,a_k,a_l,a_m$ such that $ g(a_i)=g(a_j)=g(a_k)=g(a_l)=g(a_m)$. Thus $ g(x)=k(x)(x-a_i)(x-a_j)(x-a_k)(x-a_l)(x-a_m)$, so the degree of $ g(x)$ is at least 5.[\/hide]
\end{solution}
*******************************************************************************
-------------------------------------------------------------------------------

\begin{problem}[Posted by \href{https://artofproblemsolving.com/community/user/54046}{SUPERMAN2}]
	Find all the functions $f: \mathbb R \to \mathbb R$ that satisfy \[ f(xy)(f(x) - f(y)) = (x - y)f(x)f(y) , \quad \forall x,y \in \mathbb R.\]
	\flushright \href{https://artofproblemsolving.com/community/c6h280764}{(Link to AoPS)}
\end{problem}



\begin{solution}[by \href{https://artofproblemsolving.com/community/user/22804}{nayel}]
	Let $ f(1)=k$. Substituting $ y=1$ yields $ f(x)(f(x)-k)=(x-1)kf(x)$. Hence $ f(x)=0$ for all x, or $ f(x)-k=k(x-1)$ id est $ f(x)=kx$. These are all possible solutions.
\end{solution}



\begin{solution}[by \href{https://artofproblemsolving.com/community/user/56329}{Anavel_Gato}]
	\begin{tcolorbox}Let $ f(1) = k$. Substituting $ y = 1$ yields $ f(x)(f(x) - k) = (x - 1)kf(x)$. Hence $ f(x) = 0$ for all x, or $ f(x) - k = k(x - 1)$ id est $ f(x) = kx$. These are all possible solutions.\end{tcolorbox}

How did you deny the case that $ f(x)=0$ for some $ x$ and $ f(x)=kx$ for other $ x$ ?
\end{solution}



\begin{solution}[by \href{https://artofproblemsolving.com/community/user/29428}{pco}]
	\begin{tcolorbox} How did you deny the case that $ f(x) = 0$ for some $ x$ and $ f(x) = kx$ for other $ x$ ?\end{tcolorbox}

He can't.

General solution is :
Let $ A$ and $ B$ a splitting of $ \mathbb R$ such that :

$ A\cup B = \mathbb R$
$ A\cap B = \emptyset$
$ \forall x\in A, \forall y\in B$ : $ xy\in A$
$ \forall x\in B, \forall y\in B$ : $ xy\in B$

Then :
$ \forall x\in A$ : $ f(x) = 0$
$ \forall x\in B$ : $ f(x) = kx$

Example : $ A = \mathbb R\backslash \mathbb Q$ and $ B = \mathbb Q$
$ \forall x\notin \mathbb Q$ : $ f(x) = 0$
$ \forall x\in \mathbb Q$ : $ f(x) = kx$
\end{solution}



\begin{solution}[by \href{https://artofproblemsolving.com/community/user/22804}{nayel}]
	\begin{tcolorbox}How did you deny the case that $ f(x) = 0$ for some $ x$ and $ f(x) = kx$ for other $ x$ ?\end{tcolorbox}

 :wallbash:  I'm really sorry. pco is right, I can't. This is the same mistake I made in IMO last year... :(
\end{solution}
*******************************************************************************
-------------------------------------------------------------------------------

\begin{problem}[Posted by \href{https://artofproblemsolving.com/community/user/55355}{N.N.Trung}]
	Let $ f: [0;1] \rightarrow \mathbb{R}$ be a function such that:
$ (i)$ $ f(1) = 1$,
$ (ii)$ $ f(x) \ge 0$ for all $ x \in [0,1]$,
$ (iii)$ if $ x,y$ and $ x + y$ all lie in $ [0,1]$, then $ f(x + y) \ge f(x) + f(y)$.
Prove that $ f(x) \le 2x$ for all $ x \in [0;1]$.
	\flushright \href{https://artofproblemsolving.com/community/c6h281056}{(Link to AoPS)}
\end{problem}



\begin{solution}[by \href{https://artofproblemsolving.com/community/user/29428}{pco}]
	\begin{tcolorbox}Let $ f: [0;1] \rightarrow \mathbb{R}$ be a function such that:
$ (i)$ $ f(1) = 1$,
$ (ii)$ $ f(x) \ge 0$ for all $ x \in [0,1]$,
$ (iii)$ if $ x,y$ and $ x + y$ all lie in $ [0,1]$, then $ f(x + y) \ge f(x) + f(y)$.
Prove that $ f(x) \le 2x$ for all $ x \in [0;1]$.\end{tcolorbox}

Let $ a\in[0,1]$ such that $ f(a)>2a$
If $ a\leq\frac 12$, and using $ f(x + y) \ge f(x) + f(y)$, we get $ f(2a)\geq 2f(a) > 4a=2(2a)$. Using this process as much as necessary, we get $ b\in(\frac 12,1]$ such that $ f(b)>2b>1$
Then $ f(1)=f(b+(1-b))\ge f(b)+f(1-b)>1+0$ (remember $ f(x) \ge 0$ for all $ x \in [0,1]$) which is a contradiction with $ f(1)=1$

So no such $ a$ exists and $ f(x)\leq 2x$ $ \forall x\in[0,1]$
\end{solution}
*******************************************************************************
-------------------------------------------------------------------------------

\begin{problem}[Posted by \href{https://artofproblemsolving.com/community/user/44674}{Allnames}]
	Let $ f,g$ :$ \mathbb N^* \to \mathbb N^*$ be two functions such that for all $n \in \mathbb N^*$ \[ |f(n)-n|\le 2004\sqrt n,\]  and \[ n^2+g^2(n)=2f^2(n).\] Prove that if $ f$ or $ g$ is surjective, then these functions have infinitely many fixed points.
	\flushright \href{https://artofproblemsolving.com/community/c6h281636}{(Link to AoPS)}
\end{problem}



\begin{solution}[by \href{https://artofproblemsolving.com/community/user/44674}{Allnames}]
	\begin{tcolorbox}Let $ f,g$ :$ \mathbb N^* \to \mathbb N^*$ be two functions such that $ |f(n) - n|\le 2004\sqrt n$
and $ n^2 + g^2(n) = 2f^2(n)$. Prove that if $ f$ or $ g$ is surjective, then these functions have infinitely many fixed points.\end{tcolorbox}
No one?. pco where ae you?
\end{solution}



\begin{solution}[by \href{https://artofproblemsolving.com/community/user/16261}{Rust}]
	Solution is
$ f(n)=c(a^2+b^2),n=|a^2-b^2-2ab|,g(n)=c|a^2-b^2+2ab|, b\in Z, a,b\in N$.
Therefore $ |cb|<1002$. If $ f(n)=p=3\mod 4$, then $ c=p,a=1,b=0\to g(n)=n=c=f(n)$.
If $ g(n)=p=3,5\mod 8$, then $ c=p=g(n)=f(n)=n$.
\end{solution}



\begin{solution}[by \href{https://artofproblemsolving.com/community/user/29428}{pco}]
	\begin{tcolorbox}Solution is
$ f(n) = c(a^2 + b^2),n = |a^2 - b^2 - 2ab|,g(n) = c|a^2 - b^2 + 2ab|, b\in Z, a,b\in N$.
Therefore $ |cb| < 1002$. If $ f(n) = p = 3\mod 4$, then $ c = p,a = 1,b = 0\to g(n) = n = c = f(n)$.
If $ g(n) = p = 3,5\mod 8$, then $ c = p = g(n) = f(n) = n$.\end{tcolorbox}
 
\end{solution}



\begin{solution}[by \href{https://artofproblemsolving.com/community/user/44674}{Allnames}]
	My solution:
1) Denote $ A_m = \{x\in N^*|f(x) = m\}$ with $ m$ is a given natural number..If $ A$ has many infinitely elements thus $ 2m^2 = x^2 + g^2(x)$.It is impossible!
2) We choose $ n_0$ satisfying $ f(n_0) = 1$. because $ n_0^2 + g^2(n_0) = 2$ implies $ n_0 = 1$ and $ g(1) = 1$. It means $ f(x)$; and $ g(x)$ have at least one fixed point.
Assume that the set of fixed points of $ f(x)$ is finite. Let $ n_0$ be the maximum value of those elements.
According to 1) we must have there exists some $ n_1 > n_0$ such that for all $ n\ge n_1$ then $ f(n)\ge n$.
Because $ |f(n) - n|\le 2004\sqrt n$, so when $ n$ is big enough then $ f(n)\le n + 2004\sqrt n\le n + 1$.
It means $ f(n) = n$ or $ f(n) = n + 1$.
If $ f(n) = n + 1$ hence $ g^2(n) + 2 = (n + 2)^2$ impossible.
So that $ f(n) = n$ and $ n > n_0$ contradiction. It is able to complete my proof.
It  does not seem to be right. What do you think?
\end{solution}



\begin{solution}[by \href{https://artofproblemsolving.com/community/user/16261}{Rust}]
	\begin{tcolorbox}Because $ |f(n) - n|\le 2004\sqrt n$, so when $ n$ is big enough then $ f(n)\le n + 2004\sqrt n\le n + 1$.\end{tcolorbox}
Why?
It means $ f(n) = n$ or $ f(n) = n + 1$.
\begin{tcolorbox}It  does not seem to be right. What do you think?\end{tcolorbox}
You don't use, that $ (n,g(n),f(n)$ is natural solution for $ x^2+y^2=2z^2\to x=c(a^2-b^2-2ab|),y=c(|a^2-b^2+2ab|),z=c(a^2+b^2).$
\end{solution}
*******************************************************************************
-------------------------------------------------------------------------------

\begin{problem}[Posted by \href{https://artofproblemsolving.com/community/user/34189}{tdl}]
	Find all monotoic functions $f: \mathbb R \to \mathbb R$ so that \[ f(x) + 2x = f(f(x))\] for all reals $x$.
	\flushright \href{https://artofproblemsolving.com/community/c6h282499}{(Link to AoPS)}
\end{problem}



\begin{solution}[by \href{https://artofproblemsolving.com/community/user/16261}{Rust}]
	There are infinetelymane solutions.
1)Let $ x$ is fixed point, then $ 2x = 0$. Because $ f(0)$ fixed point we have unique fixed point $ x = 0$.
2) If $ f(x) = 0$, then $ x = 0$. Therefore defined function $ g(x) = \frac {f(x)}{x}, x\not = 0$ and $ (g(xg(x)) - 1)g(x) = 2$.
If $ g(x)\equiv const$, then $ g(x) = - 1$ or $ g(x) = 2$.
Let $ g_2(x)\ge 2$ in interval $ (x_1,x_2), x_1 > 0, x_2 = g(x_1)x_1$ and $ g_2(x)x$ is monototonic. Then we get
$ g(y) = 1 + \frac {2}{g(x)},y = xg(x)$ and define $ g(x)$ in interval $ (x_2,x_3),x_3 = g(x_2)x_2 > x_2 = g(x_1)x_1$.
Then we can define function in interval $ (x_3,x_4)$ e.t.c.
For example, we can consider any decrease function $ f_0(x),f_0(0)=0,x>0$, suth that $ f_0(x)+2x$ increase
and define $ f(x)=f_0(x),x\ge 0, f(x)=x+2f_0^{-1}(x),x<0$.
Or consider any increase function $ f_0(x), 1<x<2$, suth that $ f_0(1)=2, f_0(2)=4$ and define
$ f_1(x)=f_0(x), 1\le x\le 2, f_1(x)=x+2f_0^{-1}(x), 2\le x\le 4$.
Define $ f(x)=sign(x) 4^{[ln_4|x|}f_1(4^{\{ln_4|x|\}})$. We can take distinct functions $ f_0(x),f_1(x)$ for distinct intervals $ (4^n,4^{n+1})$.
\end{solution}



\begin{solution}[by \href{https://artofproblemsolving.com/community/user/29428}{pco}]
	\begin{tcolorbox}There are infinetelymane solutions.
1)Let $ x$ is fixed point, then $ 2x = 0$. Because $ f(0)$ fixed point we have unique fixed point $ x = 0$.
2) If $ f(x) = 0$, then $ x = 0$. Therefore defined function $ g(x) = \frac {f(x)}{x}, x\not = 0$ and $ (g(xg(x)) - 1)g(x) = 2$.
If $ g(x)\equiv const$, then $ g(x) = - 1$ or $ g(x) = 2$.
Let $ g_2(x)\ge 2$ in interval $ (x_1,x_2), x_1 > 0, x_2 = g(x_1)x_1$ and $ g_2(x)x$ is monototonic. Then we get
$ g(y) = 1 + \frac {2}{g(x)},y = xg(x)$ and define $ g(x)$ in interval $ (x_2,x_3),x_3 = g(x_2)x_2 > x_2 = g(x_1)x_1$.
Then we can define function in interval $ (x_3,x_4)$ e.t.c.
For example, we can consider any decrease function $ f_0(x),f_0(0) = 0,x > 0$, suth that $ f_0(x) + 2x$ increase
and define $ f(x) = f_0(x),x\ge 0, f(x) = x + 2f_0^{ - 1}(x),x < 0$.
Or consider any increase function $ f_0(x), 1 < x < 2$, suth that $ f_0(1) = 2, f_0(2) = 4$ and define
$ f_1(x) = f_0(x), 1\le x\le 2, f_1(x) = x + 2f_0^{ - 1}(x), 2\le x\le 4$.
Define $ f(x) = sign(x) 4^{[ln_4|x|}f_1(4^{\{ln_4|x|\}})$. We can take distinct functions $ f_0(x),f_1(x)$ for distinct intervals $ (4^n,4^{n + 1})$.\end{tcolorbox}


I'm sorry, Rust : I've not fully analyzed your demo but it seems to me that your conclusion is wrong :

I can easily show that the only decreasing solution is $ f(x)=-x$ and your conclusion seems to say that we can have a lot of others.
I think that the only increasing soluton is $ f(x)=2x$. I did not succeed showing this till now but I can show that if an increasing solution is different from $ f(x)=2x$, then it is continuous at $ 0$ and has discontinuity points in any interval owning $ 0$. And this constraint does not appear in your conclusion.
\end{solution}



\begin{solution}[by \href{https://artofproblemsolving.com/community/user/34189}{tdl}]
	\begin{tcolorbox}I can easily show that the only decreasing solution is $ f(x) = - x$\end{tcolorbox}
Please give it more detail! Thanks!
\end{solution}



\begin{solution}[by \href{https://artofproblemsolving.com/community/user/29428}{pco}]
	\begin{tcolorbox}[quote="pco"]I can easily show that the only decreasing solution is $ f(x) = - x$\end{tcolorbox}
Please give it more detail! Thanks!\end{tcolorbox}

If $ \exists u$ such that $ f(u) = u$, then $ u + 2u = u$ and so $ u = 0$
$ f(0) + 2\cdot 0 = f(f(0))$ and so $ f(f(0)) = f(0)$ and so (see above) $ f(0) = 0$

$ f(x)$ is injective and so, since monotonous, is either increasing, either decreasing. And so we know that the sign of $ \frac {f(x)}{x}$ is constant all over $ \mathbb R^*$ (since $ f(0) = 0$) : positive for any increasing solution and negative for any decreasing function.

Let then $ a\neq 0$ and $ b = f(a)$.
Let the sequence $ u_n$ defined as $ u_0 = a$, $ u_1 = b$ and $ u_{n + 2} = u_{n + 1} + 2u_n$
It's easy to show with induction that $ u_{n + 1} = f(u_n)$ and that $ u_n = \frac {2a - b}{3}( - 1)^n + \frac {a + b}{3}2^n$
So, if $ b\neq - a$, we have $ \lim_{n\to + \infty}\frac {u_{n + 1}}{u_n} = 2 > 0$ and so, since  $ \frac {u_{n + 1}}{u_n} = \frac {f(u_n)}{u_n}$, we get that $ f(x)$ is increasing.
So, if $ f(x)$ is decreasing, $ b = f(a) = - a$ $ \forall a$ and so $ f(x) = - x$ $ \forall x$.
Q.E.D.

Note : in order to use the same mechanism for showing that the only increasing solution is $ f(x)=2x$, it would be easy to have $ f(x)$ continuous or surjective. Since we dont have, it's more difficult.
\end{solution}
*******************************************************************************
-------------------------------------------------------------------------------

\begin{problem}[Posted by \href{https://artofproblemsolving.com/community/user/43824}{jagdish}]
	If $f(0) =1$, $f(1) =2$, and \[f(x) = \frac{f(x+1)+f(x+2)}{2},\] for all integers $x \geq 0$, then find $f(100)$.
	\flushright \href{https://artofproblemsolving.com/community/c6h282531}{(Link to AoPS)}
\end{problem}



\begin{solution}[by \href{https://artofproblemsolving.com/community/user/29428}{pco}]
	\begin{tcolorbox}\begin{bolded}(1) If f(0) =1,f(1) =2 and f(x) = 1\/2{f(x+1)+f(x+2)}. Then find f(100) = \end{bolded}\end{tcolorbox}

This is not a functional equation. This is just a sequence computation :

$ u_0=1$
$ u_1=2$
$ u_{n+2}=2u_n-u_{n+1}$

So $ u_n=\frac 43 -\frac 13(-2)^n$

So $ f(100)=u_{100}=\frac 43 -\frac 13(-2)^{100}$
\end{solution}



\begin{solution}[by \href{https://artofproblemsolving.com/community/user/43824}{jagdish}]
	\begin{bolded}Sir can you explain me 
 
how can i get Un=4\/3-1\/3(-2)^-n from  Un+2 = 2Un-Un+1.\end{bolded}
\end{solution}



\begin{solution}[by \href{https://artofproblemsolving.com/community/user/29428}{pco}]
	\begin{tcolorbox}\begin{bolded}Sir can you explain me 
 
how can i get Un=4\/3-1\/3(-2)^-n from  Un+2 = 2Un-Un+1.\end{bolded}\end{tcolorbox}

Hello,

This is a classical method to solve sequences like $ u_{n+2}=\alpha u_{n+1} + \beta u_{n}$ :
You look first for solutions in the form $ u_n=z^n$, which implies to solve $ z^2=\alpha z + \beta$

And then all the solutions are in the form $ az_1^n + b z_2^n$ with $ z_1$ and $ z_2$ solutions of $ z^2=\alpha z + \beta$

Here, the equation to solve is $ z^2=2-z$ whose roots are $ 1$ and $ -2$ and so $ u_n=a + b(-2)^n$
You just then have to compute $ a$ and $ b$ from $ u_0$ and $ u_1$ and you get the resuired result.

In this case, you can also use a direct method : $ u_{n+2}-u_{n+1}=(-2)(u_{n+1}-u_n)$ and so $ u_{n+2}-u_{n+1}=(-2)^{n+1}(u_1-u_0)$ ... and so on
\end{solution}
*******************************************************************************
-------------------------------------------------------------------------------

\begin{problem}[Posted by \href{https://artofproblemsolving.com/community/user/34380}{math10}]
	Let $ f(x)=x^4-x^3+8ax^2-ax+a^2$. Find all real number $ a$ such that $f(x)=0$ has four different positive solutions.
	\flushright \href{https://artofproblemsolving.com/community/c6h282819}{(Link to AoPS)}
\end{problem}



\begin{solution}[by \href{https://artofproblemsolving.com/community/user/29428}{pco}]
	\begin{tcolorbox}Let $ f(x) = x^4 - x^3 + 8ax^2 - ax + a^2$ .Find all real number $ a$ ,suth that: $ f(x) = 0$ has four different positive solution.\end{tcolorbox}

Consider $ a\neq 0$ (since $ a = 0$ gives $ 3$ similar roots $ 0$).
We have $ f(x) = \frac {x^4}{a^2}f(\frac ax)$ $ \forall x\neq 0$ and so $ z\neq 0$ root $ \implies$ $ \frac az$ root too.

So we can try to write $ f(x) = (x^2 - ux + a)(x^2 - vx + a)$ which implies $ u + v = 1$ and $ uv = 6a$
In order to have $ 4$ distinct real roots, we need $ u,v$ be different real numbers such that $ u^2 > 4a$ and $ v^2 > 4a$  and so :
1) $ x^2 - x + 6a$ must have two real roots (distinct) and so $ a < \frac {1}{24}$ and so $ u = \frac {1 - \sqrt {1 - 24a}}{2}$ and $ v = \frac {1 + \sqrt {1 - 24a}}{2}$

2) $ u^2 > 4a$ and $ v^2 > 4a$ and so $ \frac {1 - 12a - \sqrt {1 - 24a}}{2} > 4a$ and $ \frac {1 - 12a + \sqrt {1 - 24a}}{2} > 4a$ and so :
$ 1 - 20a > \sqrt {1 - 24a}$ and so (since $ a < \frac {1}{24}$ implies $ 1 - 20a > 0$) : $ 1 - 40a + 400a^2 > 1 - 24a$ and so $ 25a^2 > a$.

And so $ f(x)$ has $ 4$ (maybe not distinct) real roots $ \iff$ $ a\in( - \infty,0]\cup[\frac {1}{25},\frac {1}{24}]$
We know that $ a\in\{0,\frac {1}{25},\frac {1}{24}\}$ implies multiple roots. It remains to check if out of these values, roots are distinct.

And this is easy to check :
If $ a\notin\{0,\frac {1}{25},\frac {1}{24}\}$, $ u\neq v$ and both polynomial $ x^2 - ux + a$ and $ x^2 - vx + a$ have two distinct roots. It remains to check if some root $ r$ may be root of both polynomials. But in such a case, both $ r$ and $ \frac ar$ would be root of the two polynomials and these two polynomials would be identical, and so $ u = v$.

Hence the result : $ f(x)$ has $ 4$ distinct real roots $ \iff$ $ a\in( - \infty,0)\cup(\frac {1}{25},\frac {1}{24})$

More, we have the $ 4$ roots :

$ \frac {1 - \sqrt {1 - 24a} - \sqrt {2(1 - 20a - \sqrt {1 - 24a})}}{4}$

$ \frac {1 - \sqrt {1 - 24a} + \sqrt {2(1 - 20a - \sqrt {1 - 24a})}}{4}$

$ \frac {1 + \sqrt {1 - 24a} - \sqrt {2(1 - 20a + \sqrt {1 - 24a})}}{4}$

$ \frac {1 + \sqrt {1 - 24a} + \sqrt {2(1 - 20a + \sqrt {1 - 24a})}}{4}$

Btw, I forgot that the problem required that roots must be positive. So we need $ u,v$ and $ a$ all positive, and so $ a > 0$ (since $ a > 0$ implies $ u + v > 0$ and $ uv > 0$). And so :
$ f(x)$ has $ 4$ distinct positive real roots $ \iff$ $ a\in(\frac {1}{25},\frac {1}{24})$
\end{solution}
*******************************************************************************
-------------------------------------------------------------------------------

\begin{problem}[Posted by \href{https://artofproblemsolving.com/community/user/64127}{kazekage92}]
	(2005. 5) Find all functions $f: \mathbb R \to \mathbb R$ such that \[f(x+y)=f(x)f(y)f(xy)\] for all reals $x$ and $y$.

(2004.4) Find all $f: \mathbb R \to \mathbb R$ such that  \[f(x^2 + y) = 2 f(x) + f(y^2)\] for all reals $x$ and $y$.
	\flushright \href{https://artofproblemsolving.com/community/c6h282852}{(Link to AoPS)}
\end{problem}



\begin{solution}[by \href{https://artofproblemsolving.com/community/user/48133}{hotheadhacker_mathslover}]
	hey someone answer these questions please~I would like to learn how to solve them too
\end{solution}



\begin{solution}[by \href{https://artofproblemsolving.com/community/user/15487}{Umut Varolgunes}]
	it's really interesting that you achieved to find the most unrelated place to write this question. congratulations!
\end{solution}



\begin{solution}[by \href{https://artofproblemsolving.com/community/user/48133}{hotheadhacker_mathslover}]
	those are questions from the malaysian national mathematics olympiad so i do think its in the correct place to post?ain's it so?anyway how do we solve such questions?HELP!!!
\end{solution}



\begin{solution}[by \href{https://artofproblemsolving.com/community/user/45167}{Bugi}]
	You put problems in respective sections (Algebra,Geometry,...) and whole olympiads with links to problems in this section.
\end{solution}



\begin{solution}[by \href{https://artofproblemsolving.com/community/user/54383}{EastyMoryan}]
	Let $ P(x,y)$ be the assertion that $ f(x+y)=f(x)f(y)f(xy)$

$ P(0,0)\rightarrow f(0)=f(0)^3$
So $ f(0)={-1,0,1}$
$ P(0,x)\rightarrow f(x)=f(0)f(x)^2$

CASE 1:
$ f(0)=-1$
$ f(x)=-f(x)^2$
$ f(x)[f(x)+1]=0$
This gives us $ f(x)=0$ and $ f(x)=-1$

CASE 2:
$ f(0)=1$
$ f(x)=f(x)^2$
$ f(x)[f(x)-1]=0$
This gives us $ f(x)=1$

CASE 3:
$ f(0)=0$
$ P(0,x)\rightarrow f(x)=0$

So the only solutions are $ f(x)=0$, $ f(x)=1$, and $ f(x)=-1$
\end{solution}



\begin{solution}[by \href{https://artofproblemsolving.com/community/user/15487}{Umut Varolgunes}]
	how would we understand that those are some malaysian olympiad problems if you don't write this? moreover, still the wrong place. just post those problems to algebra section and see how you get helped.
\end{solution}



\begin{solution}[by \href{https://artofproblemsolving.com/community/user/54383}{EastyMoryan}]
	Let $ P(x,y)$ be the assertion that $ f(x^2 + y) = 2f(x) + f(y^2)$

$ P(0,0)\rightarrow f(0) = 3f(0)\rightarrow f(0) = 0$
$ P(x,0)\rightarrow f(x^2) = 2f(x)$
$ P(0,x)\rightarrow f(x) = f(x^2)$
$ 2f(x) = f(x)$
$ f(x) = 0$, $ \forall x\in \mathbb R$

Let $ Z(x,y)$ be the assertion that $ f(xy)=f(x)f\left (\frac{m}{y}\right )+f(y)f\left (\frac{m}{x}\right )$, $ \forall x,y\in\mathbb R^+$

$ Z(1,1)\rightarrow \frac{1}{2}=f(m)$

So that seems fairly simple, unless I'm missing something...
\end{solution}
*******************************************************************************
-------------------------------------------------------------------------------

\begin{problem}[Posted by \href{https://artofproblemsolving.com/community/user/32128}{birzhan}]
	Find all $ f: \mathbb{R} \rightarrow  \mathbb{R}$ such that
1) $ f(1)=1$
2) for all non-zero reals $x$ and $y$, \[ f(x+y)=f(x)+f(y).\]
3) for all non-zero real $x$, \[ f(x)\cdot f\left( \frac 1x \right) =1.\]
	\flushright \href{https://artofproblemsolving.com/community/c6h282920}{(Link to AoPS)}
\end{problem}



\begin{solution}[by \href{https://artofproblemsolving.com/community/user/54383}{EastyMoryan}]
	$ f(1+1)=f(1)+f(1)$
$ f(2)=2$
$ f(2+1)=f(2)+f(1)$
$ f(3)=3$

By induction, we can see that $ f(n)=f(n-1)+f(1)=n$, $ \forall n \in \mathbb{N}$

$ f(2-1)=f(2)+f(-1)$
$ 1=2+f(-1)$
$ f(-1)=-1$

$ f(x-(x-1))=f(x)+f(1-x)$
$ 1-x=f(1-x)$
Thus:
$ f((1-x)-1)=f(1-x)+f(-1)$
$ f(-x)=1-x-1=-x$
And therefore:
$ f(x)=x$,  $ \forall x \neq 0 \in \mathbb{Z}$

Now, since $ f(x)f\left (\frac{1}{x}\right )=1$ and $ f(x)=x$
$ f\left(\frac{1}{x}\right)=\frac{1}{x}$,  $ \forall x \neq 0 \in\mathbb{Z}$

Clearly for any non-zero integer $ a$, $ \frac{a}{x}$ is equal to a sum of multiple $ \left(\frac{1}{x}\right )$'s, and from $ f\left (\frac{1}{x}+\frac{1}{x}\right )=\frac{2}{x}$ using induction we can see that $ f\left (\frac{a}{x}\right )=\frac{a}{x}$.

Thus, $ \forall x\in\mathbb{Q}$, $ f(x)=x$

Thus the only (almost-) continuous function (except at 0 which is not in the domain) that satisfies it is $ f(x)=x$, but multiple non-continuous solutions exist. (I suppose $ f(x)=\frac{x^2}{x}$ satisfies all of the conditions...)

If we assume that the restriction on 0 only applies to the third line, and not the second, then $ f(x)=x$ is the only continuous solution.
\end{solution}



\begin{solution}[by \href{https://artofproblemsolving.com/community/user/32128}{birzhan}]
	$ f(x)=x$ for $ x \in \mathbb{Q}$ follows from first and second conditions by Cauchy function.But for proving it for $ \mathbb{R}$ you need some kind of continuity, bound  or something like that
\end{solution}



\begin{solution}[by \href{https://artofproblemsolving.com/community/user/29428}{pco}]
	\begin{tcolorbox}Find all $ f: \mathbb{R} \rightarrow \mathbb{R}$ such that
$ f(1) = 1$
$ f(x + y) = f(x) + f(y)$
$ f(x)*f(1\/x) = 1$
for all $ x,y \in \mathbb{R} \setminus 0$\end{tcolorbox}

We obviously have $ f(x)=x$ $ \forall x\in\mathbb Q$ (Cauchy's equation).

We also know that $ f(x)\neq 0$ $ \forall x\neq 0$

Let then $ x\notin\{0,1\}$.
$ \frac{1}{x(x-1)}+\frac{1}{x}=\frac{1}{x-1}$ $ \implies$ $ f(\frac{1}{x(x-1)})+f(\frac{1}{x})=f(\frac{1}{x-1})$ 

$ \implies$ $ \frac{1}{f(x^2)-f(x)}+\frac{1}{f(x)}=\frac{1}{f(x)-1}$

$ \implies$ $ \frac{1}{f(x^2)-f(x)}=\frac{1}{f(x)(f(x)-1)}$

$ \implies$ $ f(x^2)=f(x)^2$ $ \forall x\notin\{0,1\}$

$ \implies$ $ f(x^2)=f(x)^2$ $ \forall x$

$ \implies$ $ f(x)>0$ $ \forall x>0$

$ \implies$ $ f(x)$ is an increasing function ($ f(x+y)=f(x)+f(y)>f(x)$ $ \forall y>0$)

$ \implies$ $ f(x)=x$ $ \forall x\in\mathbb R$
\end{solution}
*******************************************************************************
-------------------------------------------------------------------------------

\begin{problem}[Posted by \href{https://artofproblemsolving.com/community/user/34380}{math10}]
	Find all functions $ f: \mathbb N^2 \to \mathbb R$ satisfying
\[f(x,x)=x, \quad  f(x,y)=f(y,x),\quad \text{and} \quad(x+y)f(x,y)=yf(x,x+y)\]
for all positive integers $x$ and $y$.
	\flushright \href{https://artofproblemsolving.com/community/c6h283367}{(Link to AoPS)}
\end{problem}



\begin{solution}[by \href{https://artofproblemsolving.com/community/user/29428}{pco}]
	\begin{tcolorbox}Find all function $ f$,defined onthe sets of ordered pairs of positive integers,satisfying properties:
 $ f(x,x) = x; f(x,y) = f(y,x)$ and $ (x + y)f(x,y) = yf(x,x + y)$\end{tcolorbox}

Must we understand that $ f(x)$ is from $ \mathbb N^2\to\mathbb R$ ? (my question is about image)
\end{solution}



\begin{solution}[by \href{https://artofproblemsolving.com/community/user/34380}{math10}]
	Yes,$ f: \mathbb{N}^{2}\to\mathbb{R}$
\end{solution}



\begin{solution}[by \href{https://artofproblemsolving.com/community/user/29428}{pco}]
	\begin{tcolorbox}Find all function $ f$,defined onthe sets of ordered pairs of positive integers,satisfying properties:
 $ f(x,x) = x; f(x,y) = f(y,x)$ and $ (x + y)f(x,y) = yf(x,x + y)$\end{tcolorbox}

Let $ P(x,y)$ be the assertion $ (x + y)f(x,y) = yf(x,x + y)$

1) There is at most one solution
If $ x=y$, $ f(x,x)=x$
$ P(1,x)$ $ \implies$ $ (1+x)f(1,x) = xf(1,1+x)$ $ \implies$ $ f(1,x+1)=\frac{x+1}xf(1,x)$ $ \implies$ $ f(1,x)=x$
If $ x\neq y$, Wlog say $ x<y$, then $ yf(x,y-x) = (y-x)f(x,y)$ and so $ f(x,y)=\frac{y}{y-x}f(x,y-x)$
And so we have a finite descent method with and ending situiation either $ f(1,x)$, either $ f(x,x)$
Q.E.D

2) $ f(x,y)=lcm(x,y)=\frac{xy}{\gcd(x,y)}$ fits the equation
$ lcm(x,x)=x$
$ lcm(x,y)=lcm(y,x)$
$ \gcd(x,x+y)=\gcd(x,y)$ $ \implies$ $ lcm(x,x+y)=\frac{x(x+y)}{\gcd(x,x+y)}$ $ =\frac{x(x+y)}{\gcd(x,y)}$ $ =\frac{x+y}{y}\frac{xy}{\gcd(x,y)}$ $ =\frac{x+y}{y}lcm(x,y)$
Q.E.D.

3) Hence the only solution is $ f(x,y)=lcm(x,y)=\frac{xy}{\gcd(x,y)}$
\end{solution}



\begin{solution}[by \href{https://artofproblemsolving.com/community/user/34380}{math10}]
	Yes,you had right.
This is nice problem,nice solution 
\end{solution}
*******************************************************************************
-------------------------------------------------------------------------------

\begin{problem}[Posted by \href{https://artofproblemsolving.com/community/user/34380}{math10}]
	Let $ \mathbb R^*$ denote the set of non-zero real numbers. Find all functions $ f: \mathbb R^*\rightarrow \mathbb R^*$ such that
\[ f(x^2+y)=(f(x))^2+\frac{f(xy)}{f(x)}\]
for every pair of non-zero real numbers $ x$ and $y$ with $ x^2+y\not= 0$.
	\flushright \href{https://artofproblemsolving.com/community/c6h283517}{(Link to AoPS)}
\end{problem}



\begin{solution}[by \href{https://artofproblemsolving.com/community/user/29428}{pco}]
	\begin{tcolorbox}Let $ R^*$ denote the set of nonzero real number.Find all functions $ f: R*\rightarrow R*$ such that:
 $ f(x^2 + y) = (f(x))^2 + \frac {f(xy)}{f(x)}$ for every pair of nonzero real numbers $ x,y$ with $ x^2 + y\not = 0$\end{tcolorbox}

Let $ P(x,y)$ the assertion $ f(x^2+y)=f(x)^2+\frac{f(xy)}{f(x)}$

1) $ f(1)=1$ and $ f(-x)=-f(x)$ $ \forall x\neq 0$
Let $ f(1)=a$
Let $ f(-1)=b$

$ P(1,1)$ $ \implies$ $ f(2)=a^2+1$
$ P(-1,1)$ $ \implies$ $ f(2)=b^2+1$ and so $ b=\epsilon a$ with $ \epsilon\in\{-1,+1\}$

$ P(1,x)$ $ \implies$ $ f(x+1)=a^2+\frac{f(x)}{a}$
$ P(-1,x)$ $ \implies$ $ f(x+1)=a^2+\epsilon\frac{f(-x)}{a}$ and so $ f(-x)=\epsilon f(x)$

Then :
$ P(x,1)$ $ \implies$ $ f(x^2+1)=f(x)^2+1$
$ P(1,x^2)$ $ \implies$ $ f(x^2+1)=a^2+\frac{f(x^2)}{a}$ and so $ f(x^2)=af(x)^2+a-a^3$

$ P(x,-1)$ $ \implies$ $ f(x^2-1)=f(x)^2+\epsilon$
$ P(1,x^2-1)$ $ \implies$ $ f(x^2)=a^2+\frac{f(x^2-1)}{a}$ and so $ f(x^2)=\frac{f(x)^2}{a}+a^2+\frac{\epsilon}{a}$

Comparing these two values for $ f(x^2)$, we get $ f(x)^2(a-\frac 1a)=a^3+a^2-a+\frac{\epsilon}{a}$
So, either $ a=1$ (and then $ \epsilon=-1$), either $ f(x)^2$ is a constant, and so $ f(x^2)$ is a constant and so $ f(x)$ is a constant $ c$ for $ x>0$. But this would imply $ c=c^2+1$, which is impossible.
Hence the result.

2) $ f(x)=x$ $ \forall x\in \mathbb R^*$
$ P(1,x)$ $ \implies$ $ f(x+1)=f(x)+1$ $ \forall x\neq -1$ And so $ f(x+n)=f(x)+n$ $ \forall n\mathbb Z$, $ \forall x\neq n$
An immediate consequence is $ f(n)=n$ $ \forall n\in\mathbb Z^*$

From $ f(x^2)=af(x)^2+a-a^3$ above, we get $ f(x^2)=f(x)^2$

Then $ P(x,n)$ $ \implies$ $ f(x^2+n)=f(x)^2+\frac{f(nx)}{n}$ But $ f(x^2+n)=f(x^2)+n=f(x)^2+n$ and so $ f(nx)=nf(x)$
This give us immediately $ f(x)=x$ $ \forall x\in\mathbb Q^*$

But $ f(x^2)=f(x)^2$ shows that $ f(x)>0$ $ \forall x>0$. So $ f(x^2+y)>f(x^2)$ $ \forall x,y>0$ and so $ f(x)$ is increasing for $ x>0$

So $ f(x)=x$ $ \forall x>0$ and, since $ f(-x)=-f(x)$, we get $ f(x)=x$ $ \forall x\in \mathbb R^*$
\end{solution}



\begin{solution}[by \href{https://artofproblemsolving.com/community/user/34380}{math10}]
	nice solution :D
\end{solution}
*******************************************************************************
-------------------------------------------------------------------------------

\begin{problem}[Posted by \href{https://artofproblemsolving.com/community/user/44674}{Allnames}]
	Find all continuous functions $f: \mathbb R \to \mathbb R$ in each of the following cases (each equation is supposed to be true for all reals $x$):
1) $f(f(x)) = 3f(x) - 2x$.
2) $f(f(x)) = f(x) - x$.
3) $f(f(f(x))) = 2f(f(x)) + f(x) - 2$.
4) $f(f(f(x))) = 4f(x) + 1$.
	\flushright \href{https://artofproblemsolving.com/community/c6h284033}{(Link to AoPS)}
\end{problem}



\begin{solution}[by \href{https://artofproblemsolving.com/community/user/29428}{pco}]
	\begin{tcolorbox}A.Find all continuous  functions satisfying:
1)
\[ f: \mathbb R \to \mathbb R
\]

\[ f(f(x)) = 3f(x) - 2f(x)
\]
\end{tcolorbox}

Maybe it is $ 3f(x)-2x$ instead of $ 3f(x)-2f(x)$ ??
\end{solution}



\begin{solution}[by \href{https://artofproblemsolving.com/community/user/44674}{Allnames}]
	I have just edited, pco!. The first seems to be solved by the same method you used in the link above. But I can't pass the step when $ f(x)$ is a increasing function ( I had another proof, but long and very ugly).
Especially. I want you to help me solve the fourth (  I have not solved it, till now)
\end{solution}



\begin{solution}[by \href{https://artofproblemsolving.com/community/user/29428}{pco}]
	\begin{tcolorbox}A.Find all continuous  functions satisfying:
4)
\[ f: \mathbb R \to \mathbb R
\]

\[ f(f(f(x))) = 4f(x) + 1
\]
\end{tcolorbox}

1) some infos on $ f(\mathbb R)$
We obviously have $ f(f(x))=4x+1$ $ \forall x\in f(\mathbb R)$
If $ a>-\frac 13\in f(\mathbb R)$, then $ 4a+1>a$ and $ 4a+1\in f(\mathbb R)$ and so $ [a,+\infty)\subseteq f(\mathbb R)$
For same reason : If $ b<-\frac 13\in f(\mathbb R)$, then $ (-\infty,b]\subseteq f(\mathbb R)$

So :
Either $ f(\mathbb R)=\mathbb R$
Either $ f(\mathbb R)=\{-\frac 13\}$
Either $ f(\mathbb R)=[a,+\infty)$ or $ (a,+\infty)$ for some $ a\geq -\frac 13$
Either $ f(\mathbb R)=(-\infty,a]$ or $ (-\infty,a)$ for some $ a\leq -\frac 13$

2) $ f(x)$ is injective, so monotonous, for $ x\in f(\mathbb R)$
This is an immediate consequence of $ f(f(x))=4x+1$ $ \forall x\in f(\mathbb R)$

As first consequence, if $ f(\mathbb R)=[a,+\infty)$ or $ (a,+\infty)$ for some $ a\geq -\frac 13$, $ f(x)$ is increasing over $ (a,+\infty)$.
Else, $ lim_{x\to +\infty}f(x)=L\geq a$ and $ f(f(x))=4x+1$ would be impossible when $ x\to +\infty$

As second consequence, $ f(\mathbb R)=(-\infty,a]$ or $ (-\infty,a)$ for some $ a\leq -\frac 13$, $ f(x)$ is increasing over $ (-\infty,a)$.
Else, $ lim_{x\to -\infty}f(x)=L\leq a$ and $ f(f(x))=4x+1$ would be impossible when $ x\to -\infty$

3) Solutions
3.1) $ f(\mathbb R)=\{-\frac 13\}$
Then, obviously, $ f(x)=-\frac 13$ $ \forall x$
And this indeed is a solution.

3.2) $ f(\mathbb R)=\mathbb R$
Then $ f(x)$ is a continuous monotonous bijection and $ f(f(x))=4x+1$ $ \forall x\in\mathbb R$
This is a classical equation with infinitely many continuous solutions (including trivial $ 2x+\frac 13$ and $ -2x-1$)
Solutions may be built piece by piece (I'm not quite sure for decreasing case).

3.3) $ f(\mathbb R)=[a,+\infty)$ or $ (a,+\infty)$ for some $ a\geq -\frac 13$
Choose any $ b\in (a,4a+1)$ and buid (classical piece by piece method) an increasing continuous fonction from $ [a,+\infty)$ $ \to[b,+infty)$
Choose any values for $ f(x)$ when $ x\leq a$ in respect with rules :
$ f(x)$ is continuous
$ f(a)=b$
$ [a,b)$ (or $ (a,b)$) $ \subseteq f((-\infty,a])$

Example :
$ f(\mathbb R)=[0,\+\infty)$ with :
$ f(x)=2x+\frac 13$ $ \forall x\geq 0$
$ f(x)=\frac{1+\sin(x)}{3}$ $ \forall x\leq 0$

3.4) $ f(\mathbb R)=(-\infty,a]$ or $ (-\infty,a)$ for some $ a\leq -\frac 13$
Same construction as in 3.3 above.
\end{solution}



\begin{solution}[by \href{https://artofproblemsolving.com/community/user/29428}{pco}]
	\begin{tcolorbox}A.Find all continuous  functions satisfying:
1)
\[ f: \mathbb R \to \mathbb R
\]

\[ f(f(x)) = 3f(x) - 2x
\]
\end{tcolorbox}
1) $ f(x)$ is an increasing bijection
$ f(f(x))=3f(x)-2x$ immediately implies that $ f(x)$ is injective, so monotonous.
Since $ f(x)$ is continuous monotonous, $ f(f(x))$ is continuous increasing, so is $ f(f(x))+2x$ and so is $ f(x)$
If $ \exists L$ such that $ \lim_{x\to +\infty}f(x)=L_1$, then $ f(f(x))=3f(x)-2x$ would be false when $ x\to +\infty$
If $ \exists L$ such that $ \lim_{x\to -\infty}f(x)=L_2$, then $ f(f(x))=3f(x)-2x$ would be false when $ x\to -\infty$
So $ f(\mathbb R)=\mathbb R$ and $ f(x)$ is surjective.
Q.E.D.

2) If $ f(a)=b$, then $ f(2a-b)=2a-b$
If $ a=b$, this is immediate.
If $ f(a)=b\neq a$, let the sequence $ u_n=2a-b+(b-a)2^n$ $ \forall n\in \mathbb Z$
It's easy to see that $ u_{n+2}=3u_{n+1}-2u_n$ and, thru induction,  that $ f(u_n)=u_{n+1}$ $ \forall n\in\mathbb Z$ (caution : for $ n<0$, we need first to show that $ f(x)$ is a bijection and we did it in step 1).

Then $ f(2a-b+(b-a)2^n)=2a-b+(b-a)2^{n+1}$ and, setting $ n\to -\infty$, we get the result.

3) if $ f(a)=a$ and $ f(b)=b$ for some $ a<b$, then $ f(x)=x$ $ \forall x\in [a,b]$
Let $ c\in(a,b)$ and consider the line $ g(x)=2x-c$
Since $ f(a)=a$, $ f(b)=b$ and $ f(x)$ and $ g(x)$ are continuous, and since $ c\in(a,b)$ $ \exists x_0$ such that $ f(x_0)=g(x_0)$
So $ f(x_0)=2x_0-c$ and then, using step 2 above, $ f(2x_0-(2x_0-c))=2x_0-(2x_0-c)$ and so $ f(c)=c$
Q.E.D.

4) Solutions
Let $ A=\{x$ such that $ f(x)=x\}$
Using step 2, we know that $ A\neq\emptyset$
Using step 3, we know that :
Either $ A=\mathbb R$
Either $ A=(-\infty,b]$ for some real $ b$
Either $ A=[a,+\infty)$ for some real $ a$
Either $ A=[a,b]$ for some real $ a\leq b$

4.1) $ A=\mathbb R$
So $ f(x)=x$ $ \forall x$ and this indeed is a solution (always check back)

4.2) $ A=(-\infty,b]$ for some real $ b$
So $ f(x)=x$ $ \forall x\leq b$ and $ f(x)\neq x$ $ \forall x>b$
Let then $ u>b$ and $ v=f(u)$
Using now step 2, we have $ f(2u-v)=2u-v$ and so $ 2u-v\leq b$
But, using the sequence $ u_n=2u-v+(v-u)2^n$ as defined in step 2, we have :
$ u_n=2u-v+(v-u)2^n$
$ u_{n+1}=2u-v+(v-u)2^{n+1}$
and so $ f(u_n)=2u_n-(2u-v)$
So, if $ 2u-v<b$, setting $ n\to -\infty$, we get some $ u_n$ as near of $ 2u-v$ as we want with $ f(u_n)\neq u_n$
And this is impossible since $ f(x)=x$ $ \forall x\leq b$
So $ 2u-v\geq b$
Hence $ 2u-v=b$
And so $ f(x)=2x-b$ $ \forall x\geq b$
and this indeed is a solution (checking back)

4.3) $ A=[a,+\infty)$ for some real $ a$
Using the same method than in 4.2 above, we get :
$ f(x)=x$ $ \forall x\geq a$
$ f(x)=2x-a$ $ \forall x\leq a$

4.4) $ A=[a,b]$ for some real $ a\leq b$
Using the same method than in 4.2 above, we get :
$ f(x)=2x-a$ $ \forall x\leq a$
$ f(x)=x$ $ \forall x\in\mathbb[a,b]$
$ f(x)=2x-b$ $ \forall x\geq b$
(if $ a=b$, we have the solution $ f(x)=2x-a$ $ \forall x$)
\end{solution}



\begin{solution}[by \href{https://artofproblemsolving.com/community/user/29428}{pco}]
	\begin{tcolorbox}A.Find all continuous  functions satisfying:
2)
\[ f: \mathbb R \to \mathbb R
\]

\[ f(f(x)) = f(x) - x
\]
\end{tcolorbox}

$ f(x)$ is injective and continuous, so monotonous.
So $ f(f(x))$ is increasing, so $ f(f(x))+x$ is increasing, so $ f(x)$ is increasing.

$ f(f(0))=f(0)$ and so $ f(0)$ is a fixed point of $ f(x)$
$ f(u)=u$ $ \implies$ $ u=0$ and the only fixed point of $ f(x)$ is $ 0$. So $ f(0)=0$

Let then $ x>0$, so $ f(f(x))=f(x)-x<f(x)$
But, since $ f(x)$ is increasing, $ f(f(x))<f(x)$ $ \implies$ $ f(x)<x$ and so $ f(f(x))=f(x)-x<0$, but this is in contradiction with $ f(x)$ increasing and $ f(0)=0$

So no solution.
\end{solution}
*******************************************************************************
-------------------------------------------------------------------------------

\begin{problem}[Posted by \href{https://artofproblemsolving.com/community/user/44674}{Allnames}]
	Find all functions $f: \mathbb  R\to \mathbb R$ such that
\[ f(f(x) + y) = f(f(x) - y) + 4f(x)y
\]
for all $ x,y \in \mathbb R$.
	\flushright \href{https://artofproblemsolving.com/community/c6h284036}{(Link to AoPS)}
\end{problem}



\begin{solution}[by \href{https://artofproblemsolving.com/community/user/44674}{Allnames}]
	\begin{tcolorbox}Find all functions such that
\[ f: \mathbb R\to \mathbb R\]

\[ f(f(x) + y) = f(f(x) - y) + 4f(x)y\]
for all $ x,y \in \mathbb R$\end{tcolorbox}
No ideas. Where are you, pco ???? :maybe:
\end{solution}



\begin{solution}[by \href{https://artofproblemsolving.com/community/user/29428}{pco}]
	\begin{tcolorbox}Find all functions such that
\[ f: \mathbb R\to \mathbb R\]

\[ f(f(x) + y) = f(f(x) - y) + 4f(x)y\]
for all $ x,y \in \mathbb R$\end{tcolorbox}

Let $ P(x,y)$ be the assertion $ P(x,y)$ : $ f(f(x) + y) = f(f(x) - y) + 4f(x)y$

$ f(x) = 0$ $ \forall x$ is a trivial solution. Let us now consider that $ \exists c$ such that $ f(c)\neq 0$. Then :

Let $ a\in\mathbb R$
Let $ u = f(c) - \frac {a}{8f(c)}$ and  $ v = f(c) + \frac {a}{8f(c)}$

$ P(c,\frac {a}{8f(c)})$ $ \implies$ $ f(v) = f(u) + \frac a2$

$ P(u,x - f(u))$ $ \implies$ $ f(x) = f(2f(u) - x) + 4f(u)(x - f(u))$

$ P(v,2f(u) - f(v) - x)$ $ \implies$ $ f(2f(u) - x) = f(2(f(v) - f(u)) + x) + 4f(v)(2f(u) - f(v) - x)$ $ = f(x + a) + (2f(u) + a)(2f(u) - a - 2x)$

So $ f(x) = f(x + a) + (2f(u) + a)(2f(u) - a - 2x) + 4f(u)(x - f(u))$ $ = f(x + a) - a^2 - 2ax$

So $ f(x) - x^2 = f(x + a) - (x + a)^2$ $ \forall x,a\in\mathbb R$

So $ f(x) - x^2$ is constant and $ f(x) = x^2 + b$
Plugging back this necessary condition in the original equation, we find that this indeed is a solution.

Hence the full set of solutions :
$ f(x) = 0$ $ \forall x$
$ f(x) = x^2 + b$ $ \forall x$, for any real $ b$
\end{solution}



\begin{solution}[by \href{https://artofproblemsolving.com/community/user/44674}{Allnames}]
	\begin{tcolorbox}[quote="Allnames"]Find all functions such that
\[ f: \mathbb R\to \mathbb R\]

\[ f(f(x) + y) = f(f(x) - y) + 4f(x)y\]
for all $ x,y \in \mathbb R$\end{tcolorbox}

Let $ P(x,y)$ be the assertion $ P(x,y)$ : $ f(f(x) + y) = f(f(x) - y) + 4f(x)y$

$ f(x) = 0$ $ \forall x$ is a trivial solution. Let us now consider that $ \exists c$ such that $ f(c)\neq 0$. Then :

Let $ a\in\mathbb R$
Let $ u = f(c) - \frac {a}{8f(c)}$ and  $ v = f(c) + \frac {a}{8f(c)}$

$ P(c,\frac {a}{8f(c)})$ $ \implies$ $ f(v) = f(u) + \frac a2$

$ P(u,x - f(u))$ $ \implies$ $ f(x) = f(2f(u) - x) + 4f(u)(x - f(u))$

$ P(v,2f(u) - f(v) - x)$ $ \implies$ $ f(2f(u) - x) = f(2(f(v) - f(u)) + x) + 4f(v)(2f(u) - f(v) - x)$ $ = f(x + a) + (2f(u) + a)(2f(u) - a - 2x)$

So $ f(x) = f(x + a) + (2f(u) + a)(2f(u) - a - 2x) + 4f(u)(x - f(u))$ $ = f(x + a) - a^2 - 2ax$

So $ f(x) - x^2 = f(x + a) - (x + a)^2$ $ \forall x,a\in\mathbb R$

So $ f(x) - x^2$ is constant and $ f(x) = x^2 + b$
Plugging back this necessary condition in the original equation, we find that this indeed is a solution.

Hence the full set of solutions :
$ f(x) = 0$ $ \forall x$
$ f(x) = x^2 + b$ $ \forall x$, for any real $ b$\end{tcolorbox}
Amazing proof , dear pco. I shouldn't post mine now.
\end{solution}



\begin{solution}[by \href{https://artofproblemsolving.com/community/user/29721}{Erken}]
	you should've really looked through the problems base, besides [url=http://www.mathlinks.ro/viewtopic.php?p=827178#827178]BMO problems[\/url]are quite popular   .
\end{solution}
*******************************************************************************
-------------------------------------------------------------------------------

\begin{problem}[Posted by \href{https://artofproblemsolving.com/community/user/46787}{moldovan}]
	Fro each real number $ r$ let $ T_r$ be the transformation of the plane that takes the point $ (x,y)$ into the point $ (2^r x,r 2^r x +2^r y)$. Let $ F=\{ T_r | r \in \mathbb{R} \}$. Find all curves $ y=f(x)$ whose graphs remains unchanged by every transformation in $ F$.
	\flushright \href{https://artofproblemsolving.com/community/c6h284588}{(Link to AoPS)}
\end{problem}



\begin{solution}[by \href{https://artofproblemsolving.com/community/user/29428}{pco}]
	\begin{tcolorbox}Fro each real number $ r$ let $ T_r$ be the transformation of the plane that takes the point $ (x,y)$ into the point $ (2^r x,r 2^r x + 2^r y)$. Let $ F = \{ T_r | r \in \mathbb{R} \}$. Find all curves $ y = f(x)$ whose graphs remains unchanged by every transformation in $ F$.\end{tcolorbox}

We hare looking for functions such that $ f(2^rx) = r2^rx + 2^rf(x)$, so for example $ f(2x) = 2x + 2f(x)$. 

This is a classical functional equation whose general solution may be written as :

$ f(0) = 0$
$ f(x) = x(\log_2(x) + u_1(\{\log_2(x)\})$ $ \forall x > 0$ with $ \{x\} =$ fractional part of $ x$ and $ u_1(x)$ any function from $ [0,1)\to\mathbb R$
$ f(x) = x(\log_2( - x) + u_2(\{\log_2( - x)\})$ $ \forall x < 0$ with $ \{x\} =$ fractional part of $ x$ and $ u_2(x)$ any function from $ [0,1)\to\mathbb R$

And it is easy to check back that these solutions fit the general requirement $ f(2^rx) = r2^rx + 2^rf(x)$ $ \forall r\in\mathbb R$
\end{solution}



\begin{solution}[by \href{https://artofproblemsolving.com/community/user/29428}{pco}]
	\begin{tcolorbox}Fro each real number $ r$ let $ T_r$ be the transformation of the plane that takes the point $ (x,y)$ into the point $ (2^r x,r 2^r x + 2^r y)$. Let $ F = \{ T_r | r \in \mathbb{R} \}$. Find all curves $ y = f(x)$ whose graphs remains unchanged by every transformation in $ F$.\end{tcolorbox}

I'm sorry but my previous post, as many of you noticed in your replies, was erroneous since it considered that $ r\in\mathbb Z$.

For $ r\in\mathbb R$, we have :
$ f(2^rx) = r2^rx + 2^rf(x)$ and so $ f(ax) = a\log_2(a)x + af(x)$ $ \forall x$, $ \forall a > 0$ and so :
Taking $ x = + 1$ we get $ f(a) = a\log_2(a) + af(1)$ $ \forall a > 0$
Taking $ x = - 1$ we get $ f( - a) = - a\log_2(a) + af( - 1)$ $ \forall a > 0$
Taking $ x = 0$ we get $ f(0) = 0$

And so the solution must be in the form :
$ f(0) = 0$
$ f(x) = x\log_2(|x|) + ux$ $ \forall x > 0$
$ f(x) = x\log_2(|x|) + vx$ $ \forall x < 0$

And it is easy to check that this form fit the initial requirements :
For $ x = 0$ , we have $ 0 = 0 + 2^r0$
For $ x > 0$, we have also $ 2^rx > 0$ and $ f(2^rx) = 2^rx\log_2(2^r|x|) + u2^rx$ $ = 2^r(x\log_2(|x|) + ux) + r2^rx$ $ = r2^rx + 2^rf(x)$
For $ x < 0$, we have also $ 2^rx < 0$ and $ f(2^rx) = 2^rx\log_2(2^r|x|) + v2^rx$ $ = 2^r(x\log_2(|x|) + vx) + r2^rx$ $ = r2^rx + 2^rf(x)$

And sorry, Moldovan, for the wrong previous solution. I hope this did not create problems for you.
\end{solution}
*******************************************************************************
-------------------------------------------------------------------------------

\begin{problem}[Posted by \href{https://artofproblemsolving.com/community/user/46787}{moldovan}]
	Suppose that $ u$ is a real parameter with $ 0<u<1$. Define $ f(x)=0$ if $ 0 \le x \le u$, and $ f(x)=1-(\sqrt{ux}+\sqrt{(1-u)(1-x)})^2$ if $ u \le x \le 1$, and define the sequence $ u_n$ recursively by $ u_1=f(1)$ and $ u_n=f(u_{n-1})$ for all $ n >1$. Show that there exists a positive integer $ k$ for which $ u_k=0$.
	\flushright \href{https://artofproblemsolving.com/community/c6h285072}{(Link to AoPS)}
\end{problem}



\begin{solution}[by \href{https://artofproblemsolving.com/community/user/29428}{pco}]
	\begin{tcolorbox}Suppose that $ u$ is a real parameter with $ 0 < u < 1$. Define $ f(x) = 0$ if $ 0 \le x \le u$, and $ f(x) = 1 - (\sqrt {ux} + \sqrt {(1 - u)(1 - x)})^2$ if $ u \le x \le 1$, and define the sequence $ u_n$ recursively by $ u_1 = f(1)$ and $ u_n = f(u_{n - 1})$ for all $ n > 1$. Show that there exists a positive integer $ k$ for which $ u_k = 0$.\end{tcolorbox}

The key here is to see that $ f(x)<x$ $ \forall x>0$. So $ u_n$ is a positive non increasing sequence, so converging towards the only fixed point of $ f(x)$, so $ 0$. 
So, for $ n$ great enough, $ u_n\leq u$ and $ u_{n+1}=0$
\end{solution}
*******************************************************************************
-------------------------------------------------------------------------------

\begin{problem}[Posted by \href{https://artofproblemsolving.com/community/user/58541}{apratimgtr}]
	Find all functions $f: \mathbb R \to \mathbb R$ satisfying
\[ f(x+y)=f(x)f(y)f(xy)\] for all $ x,y\in \mathbb R$.
	\flushright \href{https://artofproblemsolving.com/community/c6h285206}{(Link to AoPS)}
\end{problem}



\begin{solution}[by \href{https://artofproblemsolving.com/community/user/29428}{pco}]
	\begin{tcolorbox}Find all $ f: R\rightarrow R$ satisfying, 
$ f(x + y) = f(x)f(y)f(xy)$ for all $ x,y\in R$

I can't guess the function,please give me a hint.\end{tcolorbox}

[hide="Some hint"]
Compute $ f(x + y + z)$
[\/hide]

[hide="Solution"]
If $ \exists u$ such that $ f(u) = 0$, then $ P(x - u,u)$ $ \implies$ $ f(x) = f(x - u)f(u)f((x - u)u) = 0$ and we get the solution $ f(x) = 0$ $ \forall x$

Suppose now $ f(x)\neq 0$ $ \forall x$

$ f(x + y + z) = f(x + y)f(z)f(xz + yz)$ $ = f(x)f(y)f(xy)f(z)$ $ f(xz)f(yz)f(xyz^2)$

So $ f(x + y + z) = f(x)f(y)f(z)f(xy)f(xz)f(yz)f(xyz^2)$

So (since no factor is zero) $ f(xyz^2) = f(x^2yz) = f(xy^2z)$ $ \forall x,y,z$ and so $ f(x) = a$ $ \forall x\neq 0$ ,with $ a = a^3$
The case $ f(0)$ is easy to solve and gives $ f(0)^2 = 1$. Considering then $ P(x,-x)$, we get that $ f(0)=a$ too.

Hence the solutions :
$ f(x) = 0$ $ \forall x$
$ f(x) = - 1$ $ \forall x$
$ f(x) = + 1$ $ \forall x$
[\/hide]
\end{solution}



\begin{solution}[by \href{https://artofproblemsolving.com/community/user/58541}{apratimgtr}]
	Thank you,
\end{solution}
*******************************************************************************
-------------------------------------------------------------------------------

\begin{problem}[Posted by \href{https://artofproblemsolving.com/community/user/59009}{eureka-123}]
	If  \[ f(x+y+1)=\left(\sqrt{f(x)}+\sqrt{f(y)}\right)^{2}\] for all reals $x$ and $y$ and $f(0)=1$, determine $f(x)$.
	\flushright \href{https://artofproblemsolving.com/community/c6h285629}{(Link to AoPS)}
\end{problem}



\begin{solution}[by \href{https://artofproblemsolving.com/community/user/32725}{SAPOSTO}]
	Maybe $ f\left(x\right)=\left(x+1\right)^{2}$, but I haven't proven it!
\end{solution}



\begin{solution}[by \href{https://artofproblemsolving.com/community/user/59009}{eureka-123}]
	plz someone start it......just give me a hint atlest...
i am getting f(x)=1   :P
\end{solution}



\begin{solution}[by \href{https://artofproblemsolving.com/community/user/29428}{pco}]
	\begin{tcolorbox}easy question..but answer is not matching :( 

If  $ f(x + y + 1) = (\sqrt {f(x)} + \sqrt {f(y)})^{2}$ and f(0)=1 for all x,y $ \epsilon R$.Then determine f(x)\end{tcolorbox}

Obviously $ f(x)\geq 0$ $ \forall x\in \mathbb R$
Let then $ f(x)=g(x)^2$. We get $ |g(x+y+1)|=|g(x)|+|g(y)|$ $ \forall x,y$

Let then $ h(x)=|g(x-1)|$. We get $ h(x+y+2)=h(x+1)+h(y+1)$ And so $ h(x+y)=h(x)+h(y)$ and $ h(x)\geq 0$. This last constraint implies that $ h(x)$ is non decreasing and so that the solution is $ h(x)=cx$. But then, we need to have $ c=0$ in order to have $ h(x)\geq 0$ $ \forall x$

So $ f(x)=0$ $ \forall x$
And so no solution exists since we have the constraint $ f(0)=1$
\end{solution}



\begin{solution}[by \href{https://artofproblemsolving.com/community/user/52090}{Dumel}]
	\begin{tcolorbox}We get $ |g(x + y + 1)| = |g(x)| + |g(y)|$ $ \forall x,y$ \end{tcolorbox}it should be $ |g(x + y + 1)|^2 = |g(x)| + |g(y)|$
my solution:
1. it's easy to prove that $ f(-1)=0$
2. substitution $ y=-x-2$
so $ f(x)=0$ =contradiction
\end{solution}



\begin{solution}[by \href{https://artofproblemsolving.com/community/user/29428}{pco}]
	\begin{tcolorbox}[quote="pco"]We get $ |g(x + y + 1)| = |g(x)| + |g(y)|$ $ \forall x,y$ \end{tcolorbox}it should be $ |g(x + y + 1)|^2 = |g(x)| + |g(y)|$
\end{tcolorbox}

No, we have $ g(x + y + 1)^2 = (|g(x)| + |g(y)|)^2$ and so $ |g(x + y + 1)| = |g(x)| + |g(y)|$

Btw, your solution is good
... and better than mine :)
\end{solution}



\begin{solution}[by \href{https://artofproblemsolving.com/community/user/59009}{eureka-123}]
	so whats the final answer???????????

ia m getting confused  :(
\end{solution}



\begin{solution}[by \href{https://artofproblemsolving.com/community/user/29428}{pco}]
	\begin{tcolorbox}so whats the final answer???????????

ia m getting confused  :(\end{tcolorbox}

The answer is ... no such function exists.
\end{solution}



\begin{solution}[by \href{https://artofproblemsolving.com/community/user/59009}{eureka-123}]
	kkk...............thanx pico and everyone else.. :)
\end{solution}
*******************************************************************************
-------------------------------------------------------------------------------

\begin{problem}[Posted by \href{https://artofproblemsolving.com/community/user/59009}{eureka-123}]
	Determine all continuous functions $f,g,h: \mathbb R \to \mathbb R$ satisfying the relation \[f(x+y)=g(x)+h(y)\] for all reals $x$ and $y$.
	\flushright \href{https://artofproblemsolving.com/community/c6h285633}{(Link to AoPS)}
\end{problem}



\begin{solution}[by \href{https://artofproblemsolving.com/community/user/45922}{Kirill}]
	$ f(x+y)=g(x)+h(y)=g(y)+h(x)$;
$ g(x)+h(y)=g(y)+h(x)$;
$ h(x)-g(x)=h(0)-g(0)=A=const$;
$ h(x)=g(x)+A$ - we will put it to $ f(x+y)=g(x)+h(y)$:

$ f(x+y)=g(x)+g(y)+A$;
After $ y=0$: $ f(x)=g(x)+B$, $ B=A+g(0)$;
$ f(x+y)=f(x)+f(y)-2B+A$;

We will try to find continuous solutions of $ f(x+y)=f(x)+f(y)+C$.
After taking substitution $ v(x)=f(x)+C$ we will have:
$ v(x+y)=v(x)+v(y)$, where $ v$ is continous.

It's well-known, that $ v(x)=kx$ is solution of last equation (with $ v$ - continous).
So, $ f(x)=kx+b$, $ k, b \in R$.
It's easy to check that this class of functions is don't  break any conditions (for $ f(x)=kx+b$ we can take $ g(x)=kx+\frac{b}{2}$, $ h(x)=kx+\frac{b}{2}$).
\end{solution}



\begin{solution}[by \href{https://artofproblemsolving.com/community/user/59009}{eureka-123}]
	looks ok... :)
\end{solution}



\begin{solution}[by \href{https://artofproblemsolving.com/community/user/29428}{pco}]
	\begin{tcolorbox}$ f(x + y) = g(x) + h(y) = g(y) + h(x)$;
$ g(x) + h(y) = g(y) + h(x)$;
$ h(x) - g(x) = h(0) - g(0) = A = const$;
$ h(x) = g(x) + A$ - we will put it to $ f(x + y) = g(x) + h(y)$:

$ f(x + y) = g(x) + g(y) + A$;
After $ y = 0$: $ f(x) = g(x) + B$, $ B = A + g(0)$;
\end{tcolorbox}

Your solution is OK but the beginning may be shorter :
$ y=0$ $ \implies$ $ g(x)=f(x)+a$
$ x=0$ $ \implies$ $ h(y)=f(y)+b$
And so $ f(x+y)=f(x)+f(y)+c$
\end{solution}
*******************************************************************************
-------------------------------------------------------------------------------

\begin{problem}[Posted by \href{https://artofproblemsolving.com/community/user/59009}{eureka-123}]
	Find all continuous functions $f: \mathbb R \to \mathbb R$ such that \[f(x-y)=f(x)f(y)+g(x)g(y)\] for all reals $x$ and $y$.
	\flushright \href{https://artofproblemsolving.com/community/c6h285634}{(Link to AoPS)}
\end{problem}



\begin{solution}[by \href{https://artofproblemsolving.com/community/user/29428}{pco}]
	\begin{tcolorbox}Find all continuous solutions of f(x-y)=f(x)f(y)+g(x)g(y)\end{tcolorbox}

That's a rather classical equation.

I suppose that $ f(x)$ and $ g(x)$ both are continuous.

Let $ P(x)$ be the assertion $ f(x-y)=f(x)f(y)+g(x)g(y)$

1) Get rid of trivial constant solutions.
If $ f(x)=c_1$, then $ g(x)g(y)=c_1-c_1^2$ and so $ g(x)=c_2$
Same, If $ g(x)=c_3$, then $ P(x,x)$ implies $ f(x)^2=f(0)-c_3^2$ and so $ f(x)=c_4$ (since continuous).

So $ f(x)$ constant $ \iff$ $ g(x)$ constant and we have the families of solutions :

$ f(x)=c$ and $ g(x)=d$ with $ c,d$ any real numbers such that $ c=c^2+d^2$

2) Suppose now neither $ f(x)$, neither $ g(x)$ are constant fonctions
2.1) $ f(0)=1$, $ g(0)=0$ and $ f(x)^2+g(x)^2=1$ $ \forall x$
$ P(x,0)$ $ \implies$ $ f(x)=f(x)f(0)+g(x)g(0)$
If $ g(0)\neq 0$, this implies $ g(x)=\alpha f(x)$ but then $ P(x,x)$ would imply $ f(0)=f(x)^2(1+\alpha^2)$ and so $ f(x)$ constant.
So $ g(0)=0$
Then $ P(0,0)$ $ \implies$ $ f(0)=f(0)^2$ and so $ f(0)=0$ or $ f(0)=1$. But, if $ f(0)=0$, $ P(x,x)$ would imply $ f(x)^2+g(x)^2=0$ and so $ f(x)$ and $ g(x)$ constant
So $ f(0)=1$
And then $ P(x,x)$ $ \implies$ $ f(x)^2+g(x)^2=1$
Q.E.D.

2.2) $ f(x)$ is an even function and $ g(x)$ is an odd function.
$ P(0,x)$ $ \implies$ $ f(-x)=f(x)$ and $ f(x)$ is an even function.
Then, since $ g(x)^2=1-f(x)^2$, we get that $ g(x)^2=g(-x)^2$
Suppose then that exist a non empty interval $ (u,v)$ such that $ g(x)=g(-x)$ $ \forall x\in(u,v)$
$ P(x,-x)$ $ \implies$ $ f(2x)=f^2(x)+g(x)g(-x)$ and so $ f(2x)=1$ $ \forall x\in(u,v)$
As a consequence $ g(2x)=0$ $ \forall x\in(u,v)$

Let then $ y\in(u,v)$. $ P(x,2y)$ $ \implies$ $ f(x-2y)=f(x)$ $ \forall x$, $ \forall y\in(u,v)$ and this clearly implies $ f(x)$ constant
So no such interval $ (u,v)$ exist and $ g(-x)=g(x)$ $ \forall x$

2.3) $ \exists a>0$ such that $ f(a)=0$ and $ f(x)>0$ $ \forall x\in[0,a)$
$ P(x,-x)$ $ \implies$ $ f(2x)=f(x)^2-g(x)^2$ $ =2f(x)^2-1$
Since $ f(0)=1$ and $ f(x)$ is a non constant even function, it exists $ u>0$ such that $ f(u)<1$
If $ 1>f(u)>0$, we can build the sequences $ u_n=2^nu$ and $ v_n=f(u_n)=2f(u_{n-1})^2-1$ and it's easy to show that it exists some $ v_i\leq 0$
So, since $ f(x)$ is even and continuous, $ \exists x_0>0$ such that $ f(x_0)=0$

Let now $ A=\{x>0$ such that $ f(x)=0\}$ and let $ a=\inf(A)$; Since $ f(x)$ is continuous, $ f(a)=0$ and $ f(x)\neq 0$ $ \forall x\in[0,a)$
And, since $ f(0)>0$, $ f(x)>0$ $ \forall x\in[0,a)$
Q.E.D.

2.4) Knowledge of $ f(x)$ for $ x\in[0,a]$ gives exactly full knowledge of two solutions.
First, consider that $ g(x)$ solution implies $ -g(x)$ solution.
Second, $ f(a)=0$ implies $ g(a)^2=1$ 
So, from there, wlog consider $ g(a)=1$ (and so $ g(x)>0$ $ \forall x\in(0,a]$)

We have $ f(a)=0$ and $ g(a)=1$ and so $ f(-a)=0$ and $ g(-a)=-1$
Then, $ P(x,-a)$ $ \implies$ $ f(x+a)=-g(x)$
So $ f(2a)=-g(a)=-1$ and $ g(2a)=0$ and $ f(-2a)=-1$ and $ g(-2a)=0$
Then, $ P(x,-2a)$ $ \implies$ $ f(x+2a)=-f(x)$

So, if $ f(x)$ is defined over $ [0,a]$ :
Then $ f(x)^2+g(x)^2=1$ and $ g(x)\geq 0$ over $ [0,a]$ $ \implies$ $ g(x)$ defined over $ [0,a]$
Then $ f(x+a)=-g(x)$ $ \implies$ $ f(x)$ defined  over $ [0,2a]$
Then $ f(x+2a)=-f(x)$ $ \implies$ $ f(x)$ defined  over $ [0,+\infty)$
Then $ f(-x)=f(x)$ $ \implies$ $ f(x)$ defined  over $ -\infty,+\infty)$
Then $ g(x)=-f(x+a)$ $ \implies$ $ g(x)$ defined  over $ -\infty,+\infty)$
Q.E.D

2.5) $ f(x)=\cos(\frac{\pi}{2a}x)$ $ \forall x$
We obviously have $ f(x)=\cos(\frac{\pi}{2a}x)$ for $ x=a$
Using then $ f(2x)=2f(x)^2-1$ and $ f(x)>0$ $ \forall x\in[0,a)$, it's easy to show :
$ f(x)=\cos(\frac{\pi}{2a}x)$ and $ g(x)=\sin(\frac{\pi}{2a}x)$ for $ x=\frac{a}{2^k}$, $ \forall k\in\mathbb N\cup\{0\}$ (remember we said $ g(a)=1$)

It's then easy with induction to show that :
$ f(x)=\cos(\frac{\pi}{2a}x)$ and $ g(x)=\sin(\frac{\pi}{2a}x)$ for $ x=\frac{pa}{2^k}$, $ \forall k\in\mathbb N\cup\{0\}$ $ \forall$ integer $ p\in[0,2^k]$

And continuity gives :
$ f(x)=\cos(\frac{\pi}{2a}x)$ and $ g(x)=\sin(\frac{\pi}{2a}x)$ $ \forall x\in[0,a]$

And 2.4 above gives the final result :
$ f(x)=\cos(\frac{\pi}{2a}x)$ and $ g(x)=\sin(\frac{\pi}{2a}x)$ $ \forall x$


3) Synthesis of solutions :
$ f(x)=c$ and $ g(x)=d$ with $ c,d$ any real numbers such that $ c=c^2+d^2$

$ f(x)=\cos(\frac{\pi}{2a}x)$ and $ g(x)=\sin(\frac{\pi}{2a}x)$ $ \forall x$ and any real $ a$

$ f(x)=\cos(\frac{\pi}{2a}x)$ and $ g(x)=-\sin(\frac{\pi}{2a}x)$ $ \forall x$ and any real $ a$
\end{solution}



\begin{solution}[by \href{https://artofproblemsolving.com/community/user/59009}{eureka-123}]
	yeah its a classsical eqn..with a trivial answwer...............but can anyone give  a  shorter proof to it????
\end{solution}



\begin{solution}[by \href{https://artofproblemsolving.com/community/user/29428}{pco}]
	\begin{tcolorbox}yeah its a classsical eqn..with a trivial answwer...............but can anyone give  a  shorter proof to it????\end{tcolorbox}

You're welcome
\end{solution}



\begin{solution}[by \href{https://artofproblemsolving.com/community/user/29428}{pco}]
	\begin{tcolorbox}yeah its a classsical eqn..with a trivial answwer...............but can anyone give  a  shorter proof to it????\end{tcolorbox}

Here is a shorter solution. I hope you'll enjoy it :

Constant solutions are $ f(x) = c$ and $ g(x) = d$ with $ c,d$ any real numbers such that $ c = c^2 + d^2$. Let us look now for non constant continuous solutions :

$ y = 0$ gives $ f(x)(1 - f(0)) = g(x)g(0)$ and so, for non constant solutions, $ f(0) = 1$ and $ g(0) = 0$. As a consequence $ f(x)^2 + g(x)^2 = 1$

$ x = 0$ $ \implies$ $ f( - y) = f(y)$ and so $ f(x)$ is an even function. Then, since $ g(x)^2 = 1 - f(x)^2$, $ g( - x)^2 = g(x)^2$ $ \forall x$. But if, $ g(x) = g( - x)$ over a non empty interval, $ f(2x) = f(x)^2 + g(x)^2 = 1$ over a non empty interval and so $ g(x) = 0$ over a non empty interval, from which it is easy to show that $ f$ and $ g$ are constant functions. So $ g( - x) = - g(x)$ $ \forall x$

From $ f(2x) = 2f(x)^2 - 1$, $ f(x)$ non constant and $ f(0) = 1$, we get $ f(x) = 0$ for some $ x$ and, since $ f(x)$ is continuous, $ \exists a > 0$ such that $ f(\frac {\pi}{2a}) = 0$ and $ f(x) > 0$ $ \forall x\in[0,\frac {\pi}{2a})$

From $ f(0) = \cos(a\times 0)$, and $ f(\frac {\pi}{2a}) = \cos(a\frac {\pi}{2a})$ and $ f(2x) = 2f(x)^2 - 1$, it's immediate to conclude $ f(x) = \cos(ax)$ $ \forall x = \frac {\pi}{2^na}$, then  $ f(x) = \cos(ax)$ $ \forall x = \frac {k\pi}{2^na}$, then, using continuity, $ f(x) = \cos(ax)$ $ \forall x$

Hence the non constant solutions : $ (f,g) = (\cos(ax),\sin(ax))$ and $ (\cos(ax), - \sin(ax))$
\end{solution}
*******************************************************************************
-------------------------------------------------------------------------------

\begin{problem}[Posted by \href{https://artofproblemsolving.com/community/user/34380}{math10}]
	Find all fuction $f: \mathbb N \to \mathbb R$ such that $ f(2) = 2$, $ f(nm) = f(n)f(m)$ for all $m,n\in \mathbb N$, and $ f(n + 1) > f(n)$ for all $ n\in \mathbb N$.
	\flushright \href{https://artofproblemsolving.com/community/c6h285966}{(Link to AoPS)}
\end{problem}



\begin{solution}[by \href{https://artofproblemsolving.com/community/user/53406}{stephencheng}]
	\begin{tcolorbox}Find all fuction $ f: N\longrightarrow R$ such that:$ f(2) = 2$ ; $ f(nm) = f(n)f(m)$ for $ \forall m,n\in N$ and $ f(n + 1) > f(n)$ for $ \forall n\in N$\end{tcolorbox}


We have $ f(1) = 1$, then $ f(n) \ge 1$ for all $ n$.

Clearly $ f(n^k) = [f(n)]^k$ , where $ n,k$ are positive integers.


Assume that $ f(n) > n$ for some $ n$, and let $ In(f(n)) - In(n) = c$ is a positive constant then

$ f(2^{x + 1}) = 2^{x + 1} > f(n^k) > n^k \ge 2^{x} = f(2^x)$ is true for any positive integer $ k$ and corresponding integer $ x$.

$ \Rightarrow (x + 1)In \ 2 > k In (f(n)) > k In (n) > x In \ 2$

$ \Rightarrow (x + 1) In \ 2 - x In \ 2 = In \ 2 \ge kc$ for any $ k$ , which is impossible as $ k$ can be made arbitrarily large, contradiction.

Similarly we assume that $ f(n) < n$ for some $ n$ and get contradiction.

$ \Rightarrow f(n) = n$ for all $ n$ is the only solution.

P.S. Sorry, pco . Typo corrected.
\end{solution}



\begin{solution}[by \href{https://artofproblemsolving.com/community/user/29428}{pco}]
	\begin{tcolorbox} Assume that $ f(n) > n$ for some $ n$, and let $ f(n) - n = c$ is a positive constant then

$ f(2^{x + 1}) = 2^{x + 1} > f(n^k) > n^k \ge 2^{x} = f(2^x)$ is true for any positive integer $ k$ and corresponding integer $ x$.

$ \Rightarrow (x + 1)In \ 2 > k In (f(n)) > k In (n) > x In \ 2$

$ \Rightarrow (x + 1) In \ 2 - x In \ 2 = In \ 2 \ge kc$ for any $ k$ , \end{tcolorbox}

$ (x + 1)In \ 2 > k In (f(n)) > k In (n) > x In \ 2$ $ \implies$ $ (x + 1) In \ 2 - x In \ 2 = In \ 2 \ge k(\ln(f(n))-\ln(n))$

How did you get from there $ (x + 1) In \ 2 - x In \ 2 = In \ 2 \ge kc$ ?
\end{solution}
*******************************************************************************
-------------------------------------------------------------------------------

\begin{problem}[Posted by \href{https://artofproblemsolving.com/community/user/48364}{cnyd}]
	Find all functions $ f : \mathbb{R}\to\mathbb{R}$ such that for any $ x,y\in \mathbb R$,
\[ f(x^2 - y^2) = (x - y)\cdot [f(x) + f(y)].\]
	\flushright \href{https://artofproblemsolving.com/community/c6h285970}{(Link to AoPS)}
\end{problem}



\begin{solution}[by \href{https://artofproblemsolving.com/community/user/29428}{pco}]
	\begin{tcolorbox}$ f: R\rightarrow R$ $ ,$     $ \forall x,y\in\ R$            $ f(x^2 - y^2) = (x - y).[f(x) + f(y)]$\end{tcolorbox}

[hide="My solution"]
Let $ P(x,y)$ the assertion $ f(x^2-y^2)=(x-y)(f(x)+f(y))$

$ P(x,x)$ $ \implies$ $ f(0)=0$
$ P(x,0)$ $ \implies$ $ f(x^2)=xf(x)$
$ P(-x,0)$ $ \implies$ $ f(x^2)=-xf(-x)$ $ \implies$ $ f(-x)=-f(x)$
$ P(x,-y)$ $ \implies$ $ f(x^2-y^2)=(x+y)(f(x)-f(y))$

So $ (x-y)(f(x)+f(y))=(x+y)(f(x)-f(y))$ and so $ xf(y)=yf(x)$ and so $ \frac{f(x)}{x}=\frac{f(y)}{y}$ $ \forall x,y\neq 0$

So $ f(x)=ax$ $ \forall x$ and it is easy to check back that this solution fits the initial equation.
[\/hide]
\end{solution}



\begin{solution}[by \href{https://artofproblemsolving.com/community/user/48364}{cnyd}]
	$ \frac{f(x)}{x}=\frac{f(y)}{y}$ $ \implies$  $ f(x) = ax$ Could you explain better?
\end{solution}



\begin{solution}[by \href{https://artofproblemsolving.com/community/user/29428}{pco}]
	\begin{tcolorbox}$ \frac {f(x)}{x} = \frac {f(y)}{y}$ $ \implies$  $ f(x) = ax$ Could you explain better?\end{tcolorbox}

$ \forall x\neq 0$ $ \frac{f(x)}{x}=\frac{f(1)}{1}$ and so $ f(x)=f(1)x$ $ \forall x$
\end{solution}



\begin{solution}[by \href{https://artofproblemsolving.com/community/user/48364}{cnyd}]
	aah thanks   :D
\end{solution}
*******************************************************************************
-------------------------------------------------------------------------------

\begin{problem}[Posted by \href{https://artofproblemsolving.com/community/user/48364}{cnyd}]
	Find all functions $f: (0, + \infty)\to(0, + \infty)$ such that for all $x,y\in (0, + \infty)$,
\[ f(f(x) + y) = x \cdot f(1 + xy).\]
	\flushright \href{https://artofproblemsolving.com/community/c6h285976}{(Link to AoPS)}
\end{problem}



\begin{solution}[by \href{https://artofproblemsolving.com/community/user/29428}{pco}]
	\begin{tcolorbox}$ \forall x,y\in\((0, + \infty))$
$ f(f(x) + y) = x.f(1 + xy)$
$ f: (0, + \infty))\rightarrow (0, + \infty))$\end{tcolorbox}

Let $ P(x)$ be the assertion $ f(f(x)+y)=xf(1+xy)$

$ P(\frac{u}{f(2)},\frac{f(2)}{u})$ $ \implies$ $ f(\text{something})=u$ and $ f(x)$ is surjective.

Let then $ x>y>0$ :
If $ f(x)>f(y)$, the number $ \frac{f(x)-f(y)}{x-y}\in(0,+\infty)$ and so :

$ P(x,y\frac{f(x)-f(y)}{x-y})$ $ \implies$ $ f(\frac{xf(x)-yf(y)}{x-y})$ $ =xf(1+xy\frac{f(x)-f(y)}{x-y})$

$ P(y,x\frac{f(x)-f(y)}{x-y})$ $ \implies$ $ f(\frac{xf(x)-yf(y)}{x-y})$ $ =yf(1+xy\frac{f(x)-f(y)}{x-y})$

And so $ x=y$, hence contradiction.

So $ x>y>0$ $ \implies$ $ f(x)\leq f(y)$ and so $ f(x)$ is non increasing and, since surjective, continuous.

As a consequence (since non increasing surjective), $ \lim_{x\to +\infty}f(x)=0$

Let then $ z>0$ and $ x_0$ such that $ f(x_0z)<1$
Let then $ x>x_0$. We have $ f(xz)\leq f(x_0z)<1$ and so $ z(1-f(xz))>0$
$ P(x,z(1-f(xz)))$ $ \implies$ $ f(f(x)+z(1-f(xz)))=xf(1+xz(1-f(xz)))$
$ P(xz,1-f(xz))$ $ \implies$ $ f(1)=xzf(1+xz(1-f(xz)))$

And so $ f(f(x)+z(1-f(xz)))=\frac{f(1)}{z})$

Raise now $ x\to +\infty$. Since $ f(x)$ is continuous, LHS$ \to f(z)$ and we get $ f(z)=\frac{f(1)}{z}$
Putting back in the original equation, we get $ f(1)=1$

And so the solution is $ f(x)=\frac 1x$
\end{solution}
*******************************************************************************
-------------------------------------------------------------------------------

\begin{problem}[Posted by \href{https://artofproblemsolving.com/community/user/46787}{moldovan}]
	Find all functions $ f: \mathbb{R} \rightarrow \mathbb{R}$ such that for all real numbers $ x,y$:

$ x f(x)-yf(y)=(x-y)f(x+y)$.
	\flushright \href{https://artofproblemsolving.com/community/c6h286211}{(Link to AoPS)}
\end{problem}



\begin{solution}[by \href{https://artofproblemsolving.com/community/user/29428}{pco}]
	\begin{tcolorbox}Find all functions $ f: \mathbb{R} \rightarrow \mathbb{R}$ such that for all real numbers $ x,y$:

$ x f(x) - yf(y) = (x - y)f(x + y)$.\end{tcolorbox}

Let $ P(x,y)$ the assertion $ xf(x)-yf(y)=(x-y)f(x+y)$

1) Obviously, the solutions set is a $ \mathbb R$-vector space ($ f$ solution implies $ \lambda f$ solution and $ f,g$ solutions imply $ f+g$ solution)

2) Dimension of solution vector-space is at most 2:
If three independant solutions $ h_1,h_2$ and $ h_3$ exist, it's easy to build a non all-zero solution $ f=\alpha h_1+\beta h_2+\gamma h_3$ which has at least two zeros.

Then, if $ f(u)=f(v)=0$ for some $ u\neq v$ :

$ P(x,u-x)$ $ \implies$ $ xf(x)=(u-x)f(u-x)$
$ P(x+v-u,u-x)$ $ \implies$ $ (x+v-u)f(x+v-u)=(u-x)f(u-x)$

And so $ Q(x)$ : $ xf(x)=(x+v-u)f(x+v-u)$

Then $ P(x+v-u,y)$ $ \implies$ $ (x+v-u)f(x+v-u)-yf(y)=(x+v-u-y)f(x+v-u+y)$ and so $ xf(x)-yf(y)=(x+v-u-y)f(x+v-u+y)$ and so :
$ (x-y)f(x+y)=(x+v-u-y)f(x+v-u+y)$
But $ Q(x+y)$ $ \implies$ $ (x+y)f(x+y)=(x+y+v-u)f(x+y+v-u)$ 
Subtracting these two equalities, we get $ 2yf(x+y)=2yf(x+y+v-u)$ and so : $ f(x+v-u)=f(x)$

Then $ Q(x)$ becomes $ xf(x)=(x+v-u)f(x+v-u)$ $ =(x+v-u)f(x)$ and so $ f(x)=0$
Q.E.D.

3) All solutions are $ f(x)=ax+b$
$ f(x)=1$ is obviously a solution
$ f(x)=x$ is obviously a solution
So we have a two-vectors basis of our vector space.
Hence the result.
\end{solution}



\begin{solution}[by \href{https://artofproblemsolving.com/community/user/112}{Diogene}]
	An easier solution :   
$ (x=1,y=-1) \Longrightarrow f_{(1)}+f_{(-1)}=2f_{(0)}$
$ (x= X+1,y=-1)\Longrightarrow (X+1)f_{(X+1)}+f_{(-1)}=(X+2)f_{(X)}$
$ (x=X,y=1) \Longrightarrow Xf_{(X)}-f_{(1)}=(X-1)f_{(X+1)}$
Finally, it's easy to obtain $ f_{(X)}= (f_{(1)}-f_{(0)})X + f_{(0)}$
 :cool:
\end{solution}



\begin{solution}[by \href{https://artofproblemsolving.com/community/user/42653}{Bacteria}]
	Let $ x \neq0$ and $ y=0$. Then $ x f(x)=x^2$, and $ f(x)=x$ for all nonzero x. We can let $ f(0)$ be anything.

Am I missing something here?
\end{solution}



\begin{solution}[by \href{https://artofproblemsolving.com/community/user/29428}{pco}]
	@Bacteria : Yes. You're missing something : $ y=0$ $ \implies$ $ xf(x)=xf(x)$ and not $ xf(x)=x^2$


@Diogene : nice and quick! Congrats !
\end{solution}



\begin{solution}[by \href{https://artofproblemsolving.com/community/user/42653}{Bacteria}]
	Oh wow, I saw $ (x+y)$ not $ f(x+y)$  :oops: .
\end{solution}
*******************************************************************************
-------------------------------------------------------------------------------

\begin{problem}[Posted by \href{https://artofproblemsolving.com/community/user/52090}{Dumel}]
	Find all functions $f: \mathbb R^{+} \to \mathbb R^{+}$ that satisfy \[ f(xf(y))+f(yf(x))=2xy\] for all $x,y >0$.
	\flushright \href{https://artofproblemsolving.com/community/c6h312394}{(Link to AoPS)}
\end{problem}



\begin{solution}[by \href{https://artofproblemsolving.com/community/user/46488}{Raja Oktovin}]
	Let $ f(1)=a$.
Substituting $ (x,y)=(1,1)$, then $ f(f(1))=1$ so $ f(a)=1$.
Substituting $ (x,y)=(a,a)$, then $ f(a)=a^2$ so $ a^2=1$ so $ a=1$.
Thus $ f(1)=1$.
Substituting $ (x,y)=(x,1)$, then $ f(f(x))+f(x)=2x$ for all $ x \in \mathbb{R}_+$.
Now, by recursion we have that $ f(x)=x$ for all $ x \in \mathbb{R}_+$.
Done.
\end{solution}



\begin{solution}[by \href{https://artofproblemsolving.com/community/user/29428}{pco}]
	I think, Raja Oktovin, that your phrase "by recursion, we have ..." is a little bit too short. Here is a more precise version of your demo (IMHO) :

Let $ P(x,y)$ be the assertion $ f(xf(y)) + f(yf(x)) = 2xy$

$ P(1,1)$ $ \implies$ $ f(f(1)) = 1$
$ P(f(1),f(1))$ $ \implies$ $ f(f(1)) = f(1)$ and so $ f(1) = 1$
$ P(x,1)$ $ \implies$ $ f(f(x)) = - f(x) + 2x$

Using then the sequence $ a_0 = x$, $ a_1 = f(x)$ and $ a_{n + 2} = - a_{n + 1} + 2a_n$, we get $ f^{[n]}(x) = a_n = \frac 13(2x + f(x) + (x - f(x))( - 2)^n)$

And now, we can say that if $ f(x)\ne x$ for some $ x$, then $ RHS$ will be negative for some $ n$ great enough and so $ f^{[n]}(x) < 0$ for some $ n,x$, in contradiction with the fact that $ f(x)$ is from $ \mathbb R^ + \to\mathbb R^ +$

Hence your [good] result $ \boxed{f(x) = x}$ $ \forall x>0$
\end{solution}
*******************************************************************************
-------------------------------------------------------------------------------

\begin{problem}[Posted by \href{https://artofproblemsolving.com/community/user/34380}{math10}]
	Let $f: \mathbb R \to \mathbb R$ be a function such that \[|f(x + y) - f(x)- f(y)|\leq 1\] for all $ x,y \in \mathbb R$. Prove that there exists a function $g: \mathbb R \to \mathbb R$ with $ |f(x) - g(x)| \leq 1$ and $ g(x +y)= g(x)+g(y)$ for all reals $x$ and $y$.
	\flushright \href{https://artofproblemsolving.com/community/c6h312802}{(Link to AoPS)}
\end{problem}



\begin{solution}[by \href{https://artofproblemsolving.com/community/user/37364}{kihe_freety5}]
	g(x)=$ \lim_{n\to\infty}{ \frac{f(2^nx)}{2^n}}$
\end{solution}



\begin{solution}[by \href{https://artofproblemsolving.com/community/user/34380}{math10}]
	\begin{tcolorbox}g(x)=$ \lim_{n\to\infty}{\frac {f(2^nx)}{2^n}}$\end{tcolorbox}
can you post full solution?

Ps Khai:Anh nho bai nay thay Quoc giao roi ma tim mai khong thay cai to loi giai dau.
\end{solution}



\begin{solution}[by \href{https://artofproblemsolving.com/community/user/37364}{kihe_freety5}]
	setting x=y=$ 2^nx$ we attain |$ \frac {f(2^{m + 1}x)}{2^{m + 1}} - \frac {f(2^{m}x)}{2^m}$|$ \le \frac{1}{2^{m + 1}}$
so sum of $ \frac {f(2^{m + 1}x)}{2^{m + 1}} - \frac {f(2^{m}x)}{2^m}$ with m starts 0 to infinite 
because the values of the terms are bounded by the geometric series $ \frac {1}{2}, \frac{1}{4}...$ which sums to 1, this sum converges absolutely and is bounded by 1 as well. on the other hand sum equals 
$ \lim_{n\to\infty} {\frac {f(2^{n}x)}{2^n}} - f(x)$
so we attains |$ f(x) - g(x)$|$ \le 1$
 prove  g(x+y)=g(x)+g(y)
$ g(x + y) - g(x) - g(y) = \lim_{n\to\infty} {\frac {f(2^{n}{x + y})}{2^n}} - \lim_{n\to\infty} {\frac {f(2^{n}x)}{2^n}} - \lim_{n\to\infty} {\frac {f(2^{n}y)}{2^n}} = \lim_{n\to\infty} {\frac {f(2^{n}{x + y})}{2^n} - \frac {f(2^{n}x)}{2^n} - \frac {f(2^{n}y)}{2^n}}$
from the given | $ f(2^{n}{x + y}) - f(2^{n}x) - f(2^{n}y)$| $ \le 1$ for any n
so we have g(x+y)=g(x)+g(y)
\end{solution}



\begin{solution}[by \href{https://artofproblemsolving.com/community/user/29428}{pco}]
	\begin{tcolorbox}setting x=y=$ 2^nx$ we attain |$ \frac {f(2^{m + 1}x)}{2^{m + 1}} - \frac {f(2^{m}x)}{2^m}$|$ \le \frac {1}{2^{m + 1}}$
so sum of $ \frac {f(2^{m + 1}x)}{2^{m + 1}} - \frac {f(2^{m}x)}{2^m}$ with m starts 0 to infinite 
because the values of the terms are bounded by the geometric series $ \frac {1}{2}, \frac {1}{4}...$ which sums to 1, this sum converges absolutely and is bounded by 1 as well. on the other hand sum equals 
$ \lim_{n\to\infty} {\frac {f(2^{n}x)}{2^n}} - f(x)$
so we attains |$ f(x) - g(x)$|$ \le 1$
 prove  g(x+y)=g(x)+g(y)
$ g(x + y) - g(x) - g(y) = \lim_{n\to\infty} {\frac {f(2^{n}{x + y})}{2^n}} - \lim_{n\to\infty} {\frac {f(2^{n}x)}{2^n}} - \lim_{n\to\infty} {\frac {f(2^{n}y)}{2^n}} = \lim_{n\to\infty} {\frac {f(2^{n}{x + y})}{2^n} - \frac {f(2^{n}x)}{2^n} - \frac {f(2^{n}y)}{2^n}}$
from the given | $ f(2^{n}{x + y}) - f(2^{n}x) - f(2^{n}y)$| $ \le 1$ for any n
so we have g(x+y)=g(x)+g(y)\end{tcolorbox}

That's very nice! 

I had a doubt about the existence of $ \lim_{n\to\infty} {\frac {f(2^{n}x)}{2^n}}$ but |$ \frac {f(2^{m + 1}x)}{2^{m + 1}} - \frac {f(2^{m}x)}{2^m}$|$ \le \frac {1}{2^{m + 1}}$ indeed is enough to show this existence.


Congrats !
\end{solution}



\begin{solution}[by \href{https://artofproblemsolving.com/community/user/34380}{math10}]
	\begin{tcolorbox}setting x=y=$ 2^nx$ we attain |$ \frac {f(2^{m + 1}x)}{2^{m + 1}} - \frac {f(2^{m}x)}{2^m}$|$ \le \frac {1}{2^{m + 1}}$
so sum of $ \frac {f(2^{m + 1}x)}{2^{m + 1}} - \frac {f(2^{m}x)}{2^m}$ with m starts 0 to infinite 
because the values of the terms are bounded by the geometric series $ \frac {1}{2}, \frac {1}{4}...$ which sums to 1, this sum converges absolutely and is bounded by 1 as well. on the other hand sum equals 
$ \lim_{n\to\infty} {\frac {f(2^{n}x)}{2^n}} - f(x)$
so we attains |$ f(x) - g(x)$|$ \le 1$
 prove  g(x+y)=g(x)+g(y)
$ g(x + y) - g(x) - g(y) = \lim_{n\to\infty} {\frac {f(2^{n}{x + y})}{2^n}} - \lim_{n\to\infty} {\frac {f(2^{n}x)}{2^n}} - \lim_{n\to\infty} {\frac {f(2^{n}y)}{2^n}} = \lim_{n\to\infty} {\frac {f(2^{n}{x + y})}{2^n} - \frac {f(2^{n}x)}{2^n} - \frac {f(2^{n}y)}{2^n}}$
from the given | $ f(2^{n}{x + y}) - f(2^{n}x) - f(2^{n}y)$| $ \le 1$ for any n
so we have g(x+y)=g(x)+g(y)\end{tcolorbox}
nice  :D 
Thank you very much.
\end{solution}
*******************************************************************************
-------------------------------------------------------------------------------

\begin{problem}[Posted by \href{https://artofproblemsolving.com/community/user/19427}{TRAN THAI HUNG}]
	Find all continuous functions $f: \mathbb R \to \mathbb R$ such that
\[ f(2x - y) = 2f(x) - f(y), \quad \forall x,y\in \mathbb R.\]
	\flushright \href{https://artofproblemsolving.com/community/c6h313064}{(Link to AoPS)}
\end{problem}



\begin{solution}[by \href{https://artofproblemsolving.com/community/user/29428}{pco}]
	\begin{tcolorbox}Find f(x) continuous in R and
$ {\rm{f(2x - y) = 2f(x) - f(y)}}\forall {\rm{x,y}} \in {\rm{R}}$:blush:\end{tcolorbox}
Let $ P(x,y)$ be the assertion $ f(2x - y) = 2f(x) - f(y)$

If $ f(x)$ is solution, $ f(x) - f(0)$ is too. So we'll only look for solutions such that $ f(0) = 0$

$ P(x,0)$ $ \implies$ $ f(2x) = 2f(x)$
$ P(0,y)$ $ \implies$ $ f( - y) = - f(y)$

So $ 2f(x) - f(y) = f(2x) + f( - y)$ and so $ f(2x + ( - y)) = f(2x) + f( - y)$ and so $ f(x + y) = f(x) + f(y)$ and so $ f(x) = ax$ since $ f(x)$ is continuous (Cauchy).

So $ f(x) = ax + b$ and it's easy to check back that these functions indeed are solutions.
\end{solution}
*******************************************************************************
-------------------------------------------------------------------------------

\begin{problem}[Posted by \href{https://artofproblemsolving.com/community/user/19427}{TRAN THAI HUNG}]
	Find all $ f: \mathbb{R}\to\mathbb{R}$ such that for all $x,y \in \mathbb R$,
\[ f(x)f(y) - f(x + y) = \sin(x) \sin(y).\]
	\flushright \href{https://artofproblemsolving.com/community/c6h313250}{(Link to AoPS)}
\end{problem}



\begin{solution}[by \href{https://artofproblemsolving.com/community/user/29428}{pco}]
	\begin{tcolorbox}Find $ f(x)$ continuous in $ R$ and
$ f(x)f(y) - f(x + y) = sin(x)sin(y)$ for all $ x,y \in R$:?:\end{tcolorbox}

Let $ P(x,y)$ be the assertion $ f(x)f(y)-f(x+y)=\sin(x)\sin(y)$

$ P(x,0)$ $ \implies$ $ f(x)(f(0)-1)=0$ and, since $ f(x)$ cant be always zero, $ f(0)=1$

$ P(\frac{\pi}2,-\frac{\pi}2)$ $ \implies$ $ f(\frac{\pi}2)f(-\frac{\pi}2))=0$ and so 

Either $ f(\frac{\pi}2)=0$ and then $ P(x-\frac{\pi}2,\frac{\pi}2)$ $ \implies$ $ f(x)=\cos(x)$ 
Either $ f(-\frac{\pi}2)=0$ and then $ P(x+\frac{\pi}2,-\frac{\pi}2)$ $ \implies$ $ f(x)=\cos(x)$ 

And it's easy to check back that this function indeed is a solution.

Hence the unique solution : $ \boxed{f(x)=\cos(x)}$

And, btw, the "continuity" constraint is useless.
\end{solution}
*******************************************************************************
-------------------------------------------------------------------------------

\begin{problem}[Posted by \href{https://artofproblemsolving.com/community/user/19427}{TRAN THAI HUNG}]
	Find all continuous functions $ f: \mathbb{R}\to\mathbb{R}$ such that for all $x,y \in \mathbb R$,
\[ xf(x) - yf(y) = (x - y)f(x+  y).\]
	\flushright \href{https://artofproblemsolving.com/community/c6h313264}{(Link to AoPS)}
\end{problem}



\begin{solution}[by \href{https://artofproblemsolving.com/community/user/29428}{pco}]
	\begin{tcolorbox}Find the $ f(x)$continuous in $ R$and
$ xf(x) - yf(y) = (x - y)f(x + y)$:)\end{tcolorbox}

The equation may be written, for $ x\ne 0$ and $ y\ne x$ : $ \frac {f(x + y) - f(y)}x = \frac {f(x) - f(y)}{x - y}$

Setting then $ x\to 0$, this shows that $ f(x)$ is differentiable at least on $ \mathbb R^*$ and that $ f'(x) = \frac {f(x) - f(0)}x$ $ \forall x\ne 0$

And so $ \boxed{f(x) = ax + b}$ (using continuity at $ 0$), which indeed is a solution
\end{solution}



\begin{solution}[by \href{https://artofproblemsolving.com/community/user/19427}{TRAN THAI HUNG}]
	\begin{tcolorbox}[quote="TRAN THAI HUNG"]Find the $ f(x)$continuous in $ R$and
$ xf(x) - yf(y) = (x - y)f(x + y)$:)\end{tcolorbox}

The equation may be written, for $ x\ne 0$ and $ y\ne x$ : $ \frac {f(x + y) - f(y)}x = \frac {f(x) - f(y)}{x - y}$

Setting then $ x\to 0$, this shows that $ f(x)$ is differentiable at least on $ \mathbb R^*$ and that $ f'(x) = \frac {f(x) - f(0)}x$ $ \forall x\ne 0$

And so $ \boxed{f(x) = ax + b}$ (using continuity at $ 0$), which indeed is a solution\end{tcolorbox}
It should be  $ y\to 0$ right? :)
\end{solution}



\begin{solution}[by \href{https://artofproblemsolving.com/community/user/29428}{pco}]
	\begin{tcolorbox}[quote="pco"]\begin{tcolorbox}Find the $ f(x)$continuous in $ R$and
$ xf(x) - yf(y) = (x - y)f(x + y)$:)\end{tcolorbox}

The equation may be written, for $ x\ne 0$ and $ y\ne x$ : $ \frac {f(x + y) - f(y)}x = \frac {f(x) - f(y)}{x - y}$

Setting then $ x\to 0$, this shows that $ f(x)$ is differentiable at least on $ \mathbb R^*$ and that $ f'(x) = \frac {f(x) - f(0)}x$ $ \forall x\ne 0$

And so $ \boxed{f(x) = ax + b}$ (using continuity at $ 0$), which indeed is a solution\end{tcolorbox}
It should be  $ y\to 0$ right? :)\end{tcolorbox}

No, better to use $ x\to 0$ and the result is $ f'(y) = \frac {f(y) - f(0)}y$ that I translated to $ f'(x) = \frac {f(x) - f(0)}x$
\end{solution}



\begin{solution}[by \href{https://artofproblemsolving.com/community/user/19427}{TRAN THAI HUNG}]
	Can you tell me how from
$ f'(x) =\frac{f(x)-f(0)}{x}$
we have $ \boxed{f(x) = ax+b}$ :) ?
\end{solution}



\begin{solution}[by \href{https://artofproblemsolving.com/community/user/29428}{pco}]
	\begin{tcolorbox}Can you tell me how from
$ f'(x) = \frac {f(x) - f(0)}{x}$
we have $ \boxed{f(x) = ax + b}$ :) ?\end{tcolorbox}

The straight way is $ \frac {f'(x)}{f(x) - f(0)} = \frac 1x$ and then integration, logs, and ... result. But you need to deal with signs and absolute values .... Here is a simpler method:

Considering $ x\in\mathbb R^*$ : 

$ f'(x) = \frac {f(x) - f(0)}{x}$

$ \frac {xf'(x) - f(x) + f(0)}{x^2} = 0$

$ (\frac {f(x) - f(0)}{x})' = 0$

$ \frac {f(x) - f(0)}{x} = a$

$ f(x) = ax + f(0)$

And continuity at 0 ends the proof.

To be full honest, the equation "$ f(x)$ continuous on $ \mathbb R$, differentiable on $ \mathbb R^*$ and $ f'(x) = \frac {f(x) - f(0)}{x}$ $ \forall x\ne 0$" gives as a result :
$ f(x) = ax + b$ $ \forall x\ge 0$
$ f(x) = cx + b$ $ \forall x\le 0$
But, using our own original equation, it's easy to show that $ a = c$ (use $ x = 1$ and $ y = - 1$ in the equation, for example)
\end{solution}



\begin{solution}[by \href{https://artofproblemsolving.com/community/user/29876}{ozgurkircak}]
	Here is another solution.
$ x=-y \neq0$ gives $ xf(x)+xf(-x)=2xf(0) \Longrightarrow f(x)+f(-x)=2f(0)$
$ y=-y$ gives $ xf(x)+yf(-y)=(x+y)f(x-y)$
If we replace $ f(-y)=2f(0)-f(y)$ we get $ xf(x)-yf(y)+2yf(0)=(x+y)f(x-y)$ 
That is $ (x-y)f(x+y)+2yf(0)=(x+y)f(x-y)$
Set $ x=\frac{u+v}{2}, y=\frac{u-v}{2}$ we get 
$ \frac{f(v)-f(0)}{v}=\frac{f(u)-f(0)}{u}$
Since this holds for any $ u,v$ we have $ \frac{f(x)-f(0)}{x}$ is constant. 
Therefore, $ f(x)=ax+b$ which satisfy the equation.
\end{solution}
*******************************************************************************
-------------------------------------------------------------------------------

\begin{problem}[Posted by \href{https://artofproblemsolving.com/community/user/64682}{KDS}]
	Prove that there is no function $ f: (0,\infty) \to (0,\infty)$ so that for all $ x,y>0$ we have $ f(x+y)\geq f(x)+yf(f(x))$
	\flushright \href{https://artofproblemsolving.com/community/c6h313430}{(Link to AoPS)}
\end{problem}



\begin{solution}[by \href{https://artofproblemsolving.com/community/user/29428}{pco}]
	See http://www.mathlinks.ro/Forum/viewtopic.php?t=6531
\end{solution}
*******************************************************************************
-------------------------------------------------------------------------------

\begin{problem}[Posted by \href{https://artofproblemsolving.com/community/user/43536}{nguyenvuthanhha}]
	Let $ k$  be a fixed positive integer, $ k \ge 2$. Find all continuous functions $ f : (0, + \infty ) \to \mathbb{R}$ such that 
\[ f \left( \sqrt [k]{\frac {x^{k} + y^{k}}{2}} \ \right) + f \left(\sqrt [k]{\frac {2}{\frac {1}{x^{k}} + \frac {1}{y^{k}}}} \ \right) = f(x) + f(y) ,\quad \forall x ,y \in ( 0 , + \infty ).\]
	\flushright \href{https://artofproblemsolving.com/community/c6h313486}{(Link to AoPS)}
\end{problem}



\begin{solution}[by \href{https://artofproblemsolving.com/community/user/29428}{pco}]
	\begin{tcolorbox}\begin{italicized}    Problem :\end{italicized}

Let $ k$  be a fixed positive integer , $ k \ge 2$

  Find all continuous function $ f : ( 0 ; + \infty ) \to \mathbb{R}$ such that :

     $ f \left( \sqrt [k]{\frac {x^{k} + y^{k}}{2}} \ \right) + f \left(\sqrt [k]{\frac {2}{\frac {1}{x^{k}} + \frac {1}{y^{k}}}} \ \right) = f(x) + f(y) \ \forall x ;y \in ( 0 ; + \infty )$\end{tcolorbox}

Let $ f(x) = g(x^k)$ and the equation becomes $ g(\frac {x + y}2) + g(\frac {2xy}{x + y}) = g(x) + g(y)$

Let then $ x\ge y$ and the two sequences : $ u_1 = x$, $ v_1 = y$, $ u_{n + 1} = \frac {u_n + v_n}2$ and $ v_{n + 1} = \frac {2u_nv_n}{u_n + v_n}$ 

It(s easy to see that $ g(u_n) + g(v_n) = g(x) + g(y)$, that $ u_nv_n = xy$, that $ u_n$ is a non increasing sequence while $ v_n$ is a non decreasing one and that $ u_n$ and $ v_n$ both have limit $ \sqrt {xy}$

So, continuity implies $ g(x) + g(y) = 2g(\sqrt {xy})$

Let then $ h(x) = g(e^x)$ ($ h: \mathbb R\to\mathbb R$). We get $ h(\frac {x + y}2) = \frac {h(x) + h(y)}2$

This a very classical equation whose solution, when we have continuity constraint is $ h(x) = ax + b$, so $ g(x) = a\ln(x) + b$

and so $ \boxed{f(x) = a\ln(x) + b}$ and it's easy to check back that these functions indeed are solutions
\end{solution}
*******************************************************************************
-------------------------------------------------------------------------------

\begin{problem}[Posted by \href{https://artofproblemsolving.com/community/user/63660}{Victory.US}]
	Determine all functions $f: \mathbb R \to \mathbb R$ such that
\[f(x-f(y))=2f(x)+x+f(y)\quad, \forall x,y \in \mathbb R.\]
	\flushright \href{https://artofproblemsolving.com/community/c6h313494}{(Link to AoPS)}
\end{problem}



\begin{solution}[by \href{https://artofproblemsolving.com/community/user/29428}{pco}]
	\begin{tcolorbox}determine all fuction $ R \to R$ such that :
$ f(x - f(y)) = 2f(x) + x + f(y), \forall x,y \in R$\end{tcolorbox}

Let $ P(x,y)$ be the assertion $ f(x-f(y))=2f(x)+x+f(y)$
Let $ a=f(0)$

$ P(f(-2a),-a)$ $ \implies$ $ f(f(-2a)-f(-a))=2f(f(-2a))+f(-2a)+f(-a)$
$ P(f(-2a),-2a)$ $ \implies$ $ a=2f(f(-2a))+2f(-2a)$ and the previous line becomes $ f(f(-2a)-f(-a))=a+f(-a)-f(-2a)$

$ P(-a,0)$ $ \implies$ $ f(-2a)=2f(-a)$ and so $ f(-2a)-f(-a)=f(-a)$ and the previous line becomes $ f(f(-a))=a-f(-a)$

$ P(f(-a),-a)$ $ \implies$ $ a=2f(f(-a))+2f(-a)$ and so $ f(f(-a))=\frac a2-f(-a)$

Comparing the two previous lines, we get $ a=0$

Then $ P(x,0)$ $ \implies$ $ f(x)=-x$ and it is easy to verify that this indeed is a solution.

Hence the unique solution $ \boxed{f(x)=-x}$ $ \forall x$
\end{solution}



\begin{solution}[by \href{https://artofproblemsolving.com/community/user/63660}{Victory.US}]
	$ f(x - f(y)) = 2f(x) + x + f(y)$, $ \forall x,y \in R$
then $ f(x-f(x))-2f(x)-f(y)=x$
set $ y=y_0$ then  $ f(x-f(y_0))-2f(x)-f(y_0)=x$

$ x \in R$ thus forall $ t\in R$ there exist $ u,v$ such that $ t=f(u)-2f(v)$

it 's nice way,too :P
\end{solution}



\begin{solution}[by \href{https://artofproblemsolving.com/community/user/29428}{pco}]
	\begin{tcolorbox}$ f(x - f(y)) = 2f(x) + x + f(y)$, $ \forall x,y \in R$
then $ f(x - f(x)) - 2f(x) - f(y) = x$
set $ y = y_0$ then  $ f(x - f(y_0)) - 2f(x) - f(y_0) = x$

$ x \in R$ thus forall $ t\in R$ there exist $ u,v$ such that $ t = f(u) - 2f(v)$\end{tcolorbox}
I'm sorry but I dont understand at all your nice way  :blush:  :

The second line is obviously wrong and must be either  $ f(x - f(x)) - 3f(x) = x$, either  $ f(x - f(y)) - 2f(x) - f(y) = x$

The third line is just the problem statement.

I dont understand how you got $ t = f(u) - 2f(v)$ (but it's easy to get $ t = f(u) - 3f(v)$

And I dont understand hou you get the result $ f(x)=-x$ from $ t = f(u) - 2f(v)$ (if you get another proof for) or even from $ t = f(u) - 3f(v)$ (which is easy to prove)

Could you please, give us some precisions.
Thanks.
\end{solution}



\begin{solution}[by \href{https://artofproblemsolving.com/community/user/63660}{Victory.US}]
	i made some mistake In latex.sorry :blush: 
it must be $ f(x - f(y)) - 2f(x) - f(y) = x$
Let $ y = 0$ then denote $ f(0) = a$ we have $ f(x - a) - 2f(x) = x + a$ ,for all $ x\in R$

the right have value in $ R$ then the left hve value in $ R$, then for all $ t \in R$ there exist $ u,v$ and $ f(t) = f(u) - 2f(v)$                                              .(1)

take $ x$ by $ f(y)$ then$ f(f(x)) = - f(x) + \frac {a}{2}$   (2)
thus take $ x$ by $ f(x)$ then $ f(f(x) - f(y)) = - (f(x) - f(y)) + a$
take $ x = f(x) - f(y)$ then $ f(f(x) - 2f(y)) = - (f(x) - 2f(y)) + 2a$    .(3)

from (1) and (3) we have $ f(t) = f(f(u) - 2f(v)) = - (f(x) - 2f(y)) + 2a = - t + a$
which $ f(x) = - x + 2a$, for all x in R. and from (2) lead to $ a = 0$ then $ f(x) = - x$

$ f(x) = - x$ solution to the equation.
\end{solution}



\begin{solution}[by \href{https://artofproblemsolving.com/community/user/29428}{pco}]
	\begin{tcolorbox}i made some mistake In latex.sorry :blush: 
it must be $ f(x - f(y)) - 2f(x) - f(y) = x$
Let $ y = 0$ then denote $ f(0) = a$ we have $ f(x - a) - 2f(x) = x + a$ ,for all $ x\in R$

the right have value in $ R$ then the left hve value in $ R$, then for all $ t \in R$ there exist $ u,v$ and $ f(t) = f(u) - 2f(v)$                                              .(1)

take $ x$ by $ f(y)$ then$ f(f(x)) = - f(x) + \frac {a}{2}$   (2)
thus take $ x$ by $ f(x)$ then $ f(f(x) - f(y)) = - (f(x) - f(y)) + a$
take $ x = f(x) - f(y)$ then $ f(f(x) - 2f(y)) = - (f(x) - 2f(y)) + 2a$    .(3)

from (1) and (3) we have $ f(t) = f(f(u) - 2f(v)) = - (f(x) - 2f(y)) + 2a = - t + a$
which $ f(x) = - x + 2a$, for all x in R. and from (2) lead to $ a = 0$ then $ f(x) = - x$

$ f(x) = - x$ solution to the equation.\end{tcolorbox}

Ok, I undertand.
Thanks.
\end{solution}
*******************************************************************************
-------------------------------------------------------------------------------

\begin{problem}[Posted by \href{https://artofproblemsolving.com/community/user/48278}{Dimitris X}]
	Let $ f: \mathbb{R}\to\mathbb{R}$ be a function so that
\[f(x - f(y)) - f(y - f(x)) = 2f(f(x) - f(y)), \quad \forall x,y \in \mathbb{R}.\]
Prove that $ f(x - f(x)) = 0$ for all $x \in \mathbb{R}$.
	\flushright \href{https://artofproblemsolving.com/community/c6h313557}{(Link to AoPS)}
\end{problem}



\begin{solution}[by \href{https://artofproblemsolving.com/community/user/29428}{pco}]
	\begin{tcolorbox}Let a function $ f: \mathbb{R}\to\mathbb{R}$ so that:

${ f(x - f(y)) - f(y - f(x)) = 2f(f(x) - f(y)),\forall x,y \in \mathbb{R}}$

Prove that:
$ f(x - f(x)) = 0,\forall x \in \mathbb{R}$\end{tcolorbox}

Let $ P(x,y)$ be the assertion $ f(x - f(y)) - f(y - f(x)) = 2f(f(x) - f(y))$

$ P(x,x)$ $ \implies$ $ f(0) = 0$

$ P(0,x)$ $ \implies$ $ f( - f(x)) - f(x) = 2f( - f(x))$ and so $ f( - f(x)) = - f(x)$
$ P(x,0)$ $ \implies$ $ f(x) - f( - f(x)) = 2f(f(x))$ and so, using previous line : $ f(f(x)) = f(x)$

Then $ P(x,f(x))$ implies the required result $ f(x - f(x)) = 0$
\end{solution}



\begin{solution}[by \href{https://artofproblemsolving.com/community/user/48278}{Dimitris X}]
	\begin{tcolorbox}[quote="Dimitris X"]Let a function $ f: \mathbb{R}\to\mathbb{R}$ so that:

${ f(x - f(y)) - f(y - f(x)) = 2f(f(x) - f(y)),\forall x,y \in \mathbb{R}}$

Prove that:
$ f(x - f(x)) = 0,\forall x \in \mathbb{R}$\end{tcolorbox}

Let $ P(x,y)$ be the assertion $ f(x - f(y)) - f(y - f(x)) = 2f(f(x) - f(y))$

$ P(x,x)$ $ \implies$ $ f(0) = 0$

$ P(0,x)$ $ \implies$ $ f( - f(x)) - f(x) = 2f( - f(x))$ and so $ f( - f(x)) = - f(x)$
$ P(x,0)$ $ \implies$ $ f(x) - f( - f(x)) = 2f(f(x))$ and so, using previous line : $ f(f(x)) = f(x)$

Then $ P(x,f(x))$ implies the required result $ f(x - f(x)) = 0$\end{tcolorbox}

That was my solution exactly during the contest...
Easy but nice question.....

Dimitris
\end{solution}
*******************************************************************************
-------------------------------------------------------------------------------

\begin{problem}[Posted by \href{https://artofproblemsolving.com/community/user/48364}{cnyd}]
	Find all functions $f: \mathbb N \to \mathbb N$ such that
\[f(x^{2}+f(y))=xf(x)+y\]
for all $x, y \in \mathbb N$.
	\flushright \href{https://artofproblemsolving.com/community/c6h313579}{(Link to AoPS)}
\end{problem}



\begin{solution}[by \href{https://artofproblemsolving.com/community/user/45762}{FelixD}]
	What is $ \mathbb{N}$?^^
\end{solution}



\begin{solution}[by \href{https://artofproblemsolving.com/community/user/32886}{dgreenb801}]
	the positive integers
\end{solution}



\begin{solution}[by \href{https://artofproblemsolving.com/community/user/29428}{pco}]
	\begin{tcolorbox}$ f: \mathbb{N} \mapsto \mathbb{N}$  (not $ \mathbb{N}_{0})$

$ f(x^{2} + f(y)) = xf(x) + y$\end{tcolorbox}

Let $ P(x,y)$ be the assertion $ f(x^2+f(y))=xf(x)+y$
Let $ a=f(1)$

$ P(1,x)$ $ \implies$ $ f(f(x)+1)=x+a$
$ P(1,f(x)+1)$ $ \implies$ $ f(f(f(x)+1)+1)=f(x)+a+1$, and, using previous line : $ f(x+a+1)=f(x)+a+1$ and so $ f(x+p(a+1))=f(x)+p(a+1)$ $ \forall p\in\mathbb N_0,x\in\mathbb N$

Then $ P(x+a+1,y)$ $ \implies$ $ f(x^2+f(y)+(a+1)(a+1+2x))=(x+a+1)f(x+a+1)+f(y)$. Using then the property established in the previous line :

$ f(x^2+f(y))+(a+1)(a+1+2x)=(x+a+1)(f(x)+a+1)+f(y)$
$ \iff$ $ f(x^2+f(y))+(a+1)^2+2x(a+1)=xf(x)+(a+1)^2+(a+1)f(x)+x(a+1)+f(y)$
$ \iff$ $ f(x^2+f(y))+x(a+1)=xf(x)+(a+1)f(x)+f(y)$

Subtracting $ P(x,y)$ from the above line, we get $ x(a+1)=(a+1)f(x)$ and so $ f(x)=x$ since $ a+1>1$

It's immediate to check back that this function indeed is a solution. Hence the unique solution : $ \boxed{f(x)=x}$ $ \forall x\in\mathbb N$
\end{solution}
*******************************************************************************
-------------------------------------------------------------------------------

\begin{problem}[Posted by \href{https://artofproblemsolving.com/community/user/19427}{TRAN THAI HUNG}]
	Find all functions $f: \mathbb R \to \mathbb R$ such that
\[ f(f(x)+2y)=x+f(2y)\]
for all $x, y \in \mathbb R$.
	\flushright \href{https://artofproblemsolving.com/community/c6h313709}{(Link to AoPS)}
\end{problem}



\begin{solution}[by \href{https://artofproblemsolving.com/community/user/29428}{pco}]
	\begin{tcolorbox}Find f(x) R to R 
and $ f(f(x) + 2y) = x + f(2y)$ :)\end{tcolorbox}

Let $ P(x,y)$ be the assertion $ f(f(x) + 2y) = x + f(2y)$
Let $ a = f(0)$

$ P(0,\frac x2)$ $ \implies$ $ f(x + a) = f(x)$
$ P(x + a,y)$ $ \implies$ $ f(f(x + a) + 2y) = x + a + f(2y)$ and so $ f(f(x) + 2y) = x + f(2y) + a$ and so $ a = 0$

Then $ P(x,0)$ $ \implies$ $ f(f(x)) = x$
And $ P(f(x),\frac y2)$ $ \implies$ $ f(x + y) = f(x) + f(y)$

And so all solutions are involutary solutions of Cauchy's equation.
And it's easy to check that all involutary solutions of Cauchy's equation are indeed solutions.

Hence the answer : The set of solutions is the set of all  involutary solutions of Cauchy's equation (infinitely many).
If we add the continuity constraint, we get only $ f(x) = x$ and $ f(x) = - x$
\end{solution}



\begin{solution}[by \href{https://artofproblemsolving.com/community/user/48278}{Dimitris X}]
	Is it valid???

Let $ x_1,x_2 \in \mathbb{R}$ so that 
$ f(x_1) = f(x_2) \Longleftrightarrow f(x_1) + 2y = f(x_2) + 2y \Longleftrightarrow f(f(x_1) + 2y) = f(f(x_2) + 2y) \Longleftrightarrow x_1 + f(2y) = x_2 + f(2y) \Longleftrightarrow x_1 = x_2$.

So f is one to one function.

So if we put $ x = y = 0$ in the equation we take:

$ f(f(0)) = f(0) \Longleftrightarrow f(0) = 0$.

Now setting $ y - - > 0$ we take:

$ f(f(x)) = x$.

Now setting $ x - - > f(x)$ and $ w - - > 2y$ we take:

$ f(x + w) = f(x) + f(w)$......
\end{solution}



\begin{solution}[by \href{https://artofproblemsolving.com/community/user/29428}{pco}]
	\begin{tcolorbox}Is it valid???\end{tcolorbox}

Sure. You just miss the conclusion
\end{solution}



\begin{solution}[by \href{https://artofproblemsolving.com/community/user/48278}{Dimitris X}]
	\begin{tcolorbox}[quote="Dimitris X"]Is it valid???\end{tcolorbox}

Sure. You just miss the conclusion\end{tcolorbox}

I did a preview before posting and i saw your solution with the conclusion....There is no need to write the same.....   :) 
I've post it because it is a different(and easier i think :) ) way.....

Thank you for your reply....
\end{solution}



\begin{solution}[by \href{https://artofproblemsolving.com/community/user/29428}{pco}]
	\begin{tcolorbox} 
I did a preview before posting and i saw your solution with the conclusion....There is no need to write the same..... 
I've post it because it is a different(and easier i think ) way.....\end{tcolorbox}

Ahhh OK.
 :)
\end{solution}



\begin{solution}[by \href{https://artofproblemsolving.com/community/user/52090}{Dumel}]
	\begin{tcolorbox}And it's easy to check that all involutary solutions of Cauchy's equation are indeed solutions. \end{tcolorbox}are you sure that each solution of Cauchy's equation satisfies $ f(f(x)) = x$  :?:
\end{solution}



\begin{solution}[by \href{https://artofproblemsolving.com/community/user/29428}{pco}]
	\begin{tcolorbox}[quote="pco"]And it's easy to check that all involutary solutions of Cauchy's equation are indeed solutions. \end{tcolorbox}are you sure that each solution of Cauchy's equation satisfies $ f(f(x)) = x$  :?:\end{tcolorbox}

I'm sure of the contrary. That's the reason for which I wrote "all \begin{bolded}involutary \end{bolded}\end{underlined}solutions of Cauchy's equation are indeed solutions"  :)
\end{solution}



\begin{solution}[by \href{https://artofproblemsolving.com/community/user/52090}{Dumel}]
	it was a very stupid question  :blush: I wanted to change it but you were faster   
It took me a few minutes to find what 'involutary' means
my english...
\end{solution}
*******************************************************************************
-------------------------------------------------------------------------------

\begin{problem}[Posted by \href{https://artofproblemsolving.com/community/user/64868}{mahanmath}]
	1) Find all $f: \mathbb R \to \mathbb R$ such that
\[(x^2 + xy + y^2)(f(x) - f(y)) = f(x^3) - f(y^3)\]
for all $x,y \in \mathbb R$.

2) Find all $f: \mathbb R \to \mathbb R$ such that
\[ (f(x^2) + f(xy) + f(y^2))(x - y) = f(x^3) - f(y^3)\]
for all $x,y \in \mathbb R$.
	\flushright \href{https://artofproblemsolving.com/community/c6h313756}{(Link to AoPS)}
\end{problem}



\begin{solution}[by \href{https://artofproblemsolving.com/community/user/29428}{pco}]
	\begin{tcolorbox}Hi ! :D  
Find all $ f: R\rightarrow R$ such that

1) $ (x^2 + xy + y^2)(f(x) - f(y)) = f(x^3) - f(y^3)$\end{tcolorbox}

Let $ P(x,y)$ be the assertion $ f(x^3) - f(y^3) = (f(x) - f(y))(x^2 + xy + y^2)$

If $ f(x)$ is solution, so is $ f(x) + c$. So we'll look for solutions such that $ f(0) = 0$

Then $ P(x,0)$  $ \implies$ $ f(x^3) = x^2f(x)$ and the equation becomes $ x^2f(x) - y^2f(y) = (f(x) - f(y))(x^2 + xy + y^2)$ $ \iff$ $ y(x + y)f(x) = x(x + y)f(y)$ 

Setting $ y = 1$, this implies $ (x + 1)f(x) = x(x + 1)f(1)$ and so $ f(x) = xf(1)$ $ \forall x\ne - 1$
Then $ P(2, - 1)$ $ \implies$ $ f( - 1) = - f(1)$ and so $ f(x) = xf(1)$ $ \forall x$

And it's easy to check thas this indeed is a solution.

Hence the solutions $ \boxed{f(x) = ax + b}$

\begin{tcolorbox}Find all $ f: R\rightarrow R$ such that

2) $ f(x^2) + f(xy) + f(y^2)(x - y) = f(x^3) - f(y^3)$\end{tcolorbox}

Setting $ y=x$, we get $ 2f(x^2)  = 0$ and so $ f(x)=0$ $ \forall x\ge 0$

The equation becomes  $ f(xy) = f(x^3) - f(y^3)$
Using $ y<0$ and $ x=y^2>0$, we get $ f(y^3) = - f(y^3)$ and so $ f(x)=0$ $ \forall x< 0$

Hence the unique solution $ \boxed{f(x)=0}$ $ \forall x$
\end{solution}



\begin{solution}[by \href{https://artofproblemsolving.com/community/user/64868}{mahanmath}]
	Sorry . I have made a typo in part 2 and I edit it now . :blush: 
I mean $ (f(x^{2}) + f(xy) + f(y^{2}))(x - y) = f(x^{3}) - f(y^{3})$
\end{solution}



\begin{solution}[by \href{https://artofproblemsolving.com/community/user/48413}{pfanni}]
	2)
$ y=0$ gives: $ xf(x^2)+2xf(0)=f(x^3)-f(0)$
Set now: $ x=1 \Rightarrow f(0)=0$.

So we have $ xf(x^2)=f(x^3)$ and therefore $ f(-x)=-f(x)$.

Using this in the first equation gives: $ (x-y)f(xy)+xf(y^2)-yf(x^2)=0$
Writing the same equation with $ -y$ gives : $ -(x+y)f(xy)+xf(y^2)+yf(x^2)=0$

Adding both lines and $ y=1$ gives us the solution: $ \boxed{f(x)=f(1)x}$
\end{solution}



\begin{solution}[by \href{https://artofproblemsolving.com/community/user/29428}{pco}]
	\begin{tcolorbox}2)
$ y = 0$ gives: $ xf(x^2) + 2xf(0) = f(x^3) - f(0)$
Set now: $ x = 1 \Rightarrow f(0) = 0$.

So we have $ xf(x^2) = f(x^3)$ and therefore $ f( - x) = - f(x)$.

Using this in the first equation gives: $ (x - y)f(xy) + xf(y^2) - yf(x^2) = 0$
Writing the same equation with $ - y$ gives : $ - (x + y)f(xy) + xf(y^2) + yf(x^2) = 0$

Adding both lines and $ y = 1$ gives us the solution: $ \boxed{f(x) = f(1)x}$\end{tcolorbox}

Quite nice and ok (I did not succeed up to now to show $ f(0)=0$)

Just a detail : all your proof lead only to mandatory condition. So $ f(x)$ must be $ xf(1)$ but you need to check back in the original equation to show that this indeed is a solution (sometimes, it may lead to constraints on $ f(1)$ for example).
\end{solution}
*******************************************************************************
-------------------------------------------------------------------------------

\begin{problem}[Posted by \href{https://artofproblemsolving.com/community/user/64868}{mahanmath}]
	Determine all functions $f: \mathbb R \to \mathbb R$ such that for all reals $x$ and $y$,
\[ f(xf(x) + f(y))=(f(x))^2 + y.\]
	\flushright \href{https://artofproblemsolving.com/community/c6h313761}{(Link to AoPS)}
\end{problem}



\begin{solution}[by \href{https://artofproblemsolving.com/community/user/67111}{Abdek}]
	\begin{tcolorbox}Hi !  :) 
Find all $ f: R\rightarrow R$ such that :

$ f(xf(x) + f(y)) = (f(x))^2 + y$\end{tcolorbox}

Let $ G(x,y)$ be the assertion of $ f(xf(x) + f(y)) = (f(x))^2 + y$

$ G(0,0) \implies f(a) = a^2$ with $ f(0) = a$

$ G(0,y) \implies f(f(y)) = a^2 + y \implies f(c) = 0$ with $ c = f( - a^2)$
$ G(c,y) \implies ff(y) = y$
$ G(f(x),y) \implies f(xf(x) + f(y)) = x^2 + y$

which implies that $ f(x)^2 = x^2$

and hence the solution are $ f(x) = \pm x$
\end{solution}



\begin{solution}[by \href{https://artofproblemsolving.com/community/user/29428}{pco}]
	\begin{tcolorbox}Hi !  :) 
Find all $ f: R\rightarrow R$ such that :

$ f(xf(x) + f(y)) = (f(x))^2 + y$\end{tcolorbox}

Let $ P(x,y)$ be the assertion $ f(xf(x)+f(y))=f(x)^2+y$
Let $ a=f(0)$

$ P(0,x)$ $ \implies$ $ f(f(x))=x+a^2$ and so $ f(x)$ is a bijection and $ \exists u$ such that $ f(u)=0$
$ P(u,x)$ $ \implies$ $ f(f(x))=x$ and so $ a=0$
$ P(x,0)$ $ \implies$ $ f(xf(x))=f(x)^2$
$ P(f(x),0)$ $ \implies$ $ f(xf(x))=x^2$

And so $ f(x)^2=x^2$ 

If $ f(1)=1$, if $ \exists x$ such that $ f(x)=-x$, then $ P(1,f(x))$ $ \implies$ $ f(x+1)=-x+1$ and so either $ x+1=-x+1$, either $ x+1=x-1$, and so $ x=0$ and so $ f(x)=x$ $ \forall x$
If $ f(1)=-1$, if $ \exists x$ such that $ f(x)=x$, then $ P(1,f(x))$ $ \implies$ $ f(x-1)=x+1$ and so either $ x-1=x+1$, either $ -x+1=x+1$, and so $ x=0$ and so $ f(x)=x$ $ \forall x$

And it is easy to check back that these two functions indeed are solutions.

Hence the two solutions $ f(x)=x$ $ \forall x$ and $ f(x)=-x$ $ \forall x$

@abdek : $ f(x)^2=x^2$ need some more proof to deduce the two solutions. You must show that there are no solutions in which $ f(2)=2$ while $ f(3)=-3$, for example.
\end{solution}



\begin{solution}[by \href{https://artofproblemsolving.com/community/user/48278}{Dimitris X}]
	Let $ P(x,y)$ the assertion of $ f(xf(x)+f(y))=(f(x))^2+y$.

Let $ f(0)=s$.

Let $ y_1,y_2 \in \mathbb{R}$ so that 
$ f(y_1)=f(y_2) \Longrightarrow xf(x)+f(y_1)=xf(x)+f(y_2) \Longrightarrow f(xf(x)+f(y_1))=f(xf(x)+f(y_2)) \Longrightarrow (f(x))^2+y_1=(f(x))^2+y_2 \Longrightarrow y_1=y_2$

$ P(0,0) \Longrightarrow f(s)=s^2$


$ P(s,0) \Longrightarrow f(sf(s)+s)= s^2$.

But $ s^2=f(s)$ so 
$ f(s^3+s)=f(s)$ and because f is 1-1 function we take $ s^3+s=s \Longleftrightarrow s=0$.

$ P(x,0) \Longrightarrow f(xf(x))=(f(x))^2$

$ P(f(x),0) \Longrightarrow f(xf(x))=x^2$.

So $ (f(x))^2=x^2$....

Now look at pco's post.... :)
\end{solution}
*******************************************************************************
-------------------------------------------------------------------------------

\begin{problem}[Posted by \href{https://artofproblemsolving.com/community/user/67949}{aktyw19}]
	Find all functions $f: \mathbb R \to \mathbb R$ such that
\[ f(x+y)=f(x)+f(y)-2f(xy)\]
for all reals $x$ and $y$.
	\flushright \href{https://artofproblemsolving.com/community/c6h314018}{(Link to AoPS)}
\end{problem}



\begin{solution}[by \href{https://artofproblemsolving.com/community/user/29428}{pco}]
	\begin{tcolorbox}Find all  $ f : R \rightarrow R$
 $ f(x + y) = f(x) + f(y) - 2f(xy)$\end{tcolorbox}

$ f(x + y + z) = f(x + (y + z)) = f(x) + f(y + z)$ $ - 2f(xy + xz) = f(x) + f(y) +$ $ f(z) - 2f(yz) - 2f(xy)$ $ - 2f(xz) + 4f(x^2yz)$

$ f(x + y + z) = f((x + y) + z) = f(x + y) + f(z)$ $ - 2f(xz + yz) = f(x) + f(y)$ $ + f(z) - 2f(xy) - 2f(xz)$ $ - 2f(yz) + 4f(xyz^2)$

And so $ f(x^2yz) = f(xyz^2)$ $ \forall x,y,z$

For $ uv\ne 0$, setting $ x = \sqrt [3]{\frac {u^2}v}$ and $ y = 1$ and $ z = \sqrt [3]{\frac {v^2}u}$, then $ f(x^2yz) = f(xyz^2)$ becomes $ f(u) = f(v) = c$ $ \forall u\ne 0,v\ne0$

Plugging this in the original equation, we get $ c =0$ and so $ f(x) = 0$ $ \forall x\ne 0$
Then, using $ x = 1$ and $ y = - 1$, we get $ f(0) = 0$

And so the unique solution $ f(x) = 0$ $ \forall x$
\end{solution}



\begin{solution}[by \href{https://artofproblemsolving.com/community/user/67949}{aktyw19}]
	thanks  
\end{solution}
*******************************************************************************
-------------------------------------------------------------------------------

\begin{problem}[Posted by \href{https://artofproblemsolving.com/community/user/67949}{aktyw19}]
	Find all continuous functions $ f: (0; \infty ) \to \mathbb R$ such that
\[f(f(x)) = x f(x)\]
for all reals $x>0$.
	\flushright \href{https://artofproblemsolving.com/community/c6h314046}{(Link to AoPS)}
\end{problem}



\begin{solution}[by \href{https://artofproblemsolving.com/community/user/29428}{pco}]
	\begin{tcolorbox}Find all $ f: (0; \infty ) \rightarrow R$
$ f(f(x)) = x f(x)$\end{tcolorbox}

Besides the two trivial solutions $ x^{\frac{1+\sqrt 5}2}$ and $ x^{\frac{1-\sqrt 5}2}$, I'm pretty sure there are infinitely many solutions.

Are you sure there is no other constraint (for example continuity) in this problem ?.
Thanks for checking ...
\end{solution}



\begin{solution}[by \href{https://artofproblemsolving.com/community/user/67949}{aktyw19}]
	ok let f continuity
\end{solution}



\begin{solution}[by \href{https://artofproblemsolving.com/community/user/29428}{pco}]
	\begin{tcolorbox}Find all $ f: (0; \infty ) \rightarrow R$
$ f(f(x)) = x f(x)$\end{tcolorbox}

Plus, since aktyw19 is so kind with us (I wonder what was exactly the problem he got in his class \/ contest \/ book ) :

\begin{tcolorbox}ok let f continuity\end{tcolorbox}

$ f(x) > 0$, else $ f(f(x))$ would not be defined for some $ x$
$ f(a) = f(b)$ $ \implies$ $ f(f(a)) = f(f(b))$ $ \implies$ $ af(a) = bf(b)$ $ \implies$ $ a = b$ (since $ f(a) = f(b) > 0$) and so $ f(x)$ is injective.
If $ \lim_{x\to + \infty}f(x) = c > 0$ we have a contradiction when using $ x\to + \infty$ in $ f(f(x)) = xf(x)$ ($ LHS\to f(c)$ while $ RHS\to + \infty$)
If $ \lim_{x\to 0 + }f(x) = c > 0$ we have a contradiction when using $ x\to + \infty$ in $ f(f(x)) = xf(x)$ ($ LHS\to f(c)$ while $ RHS\to 0$)

$ f(x)$, as a continuous injective function is monotonous.
Considering the previous lines about limits, it's immediate to establish then that $ f((0, + \infty)) = (0, + \infty))$ and that $ f(x)$ is a bijection.

Then, let $ f^{[ - 1]}(a) = b$ : $ f(b) = a$ and $ f(f(b)) = bf(b)$ so $ b = \frac {f(f(b))}{f(b)} = \frac {f(a)}a$ and we get $ f(\frac {f(x)}x) = x$

Now let $ a\ne 1 > 0$ and $ b = \frac {\ln(f(a))}{\ln(a)}$ (such that $ f(a) = a^b$)

Using $ f(f(x)) = xf(x)$, it's easy to establish that $ f(a^{u_n}) = a^{u_{n + 1}}$ $ \forall n\ge 0$ where $ u_n$ is the fibonacci sequence starting with $ u_0 = 1$ and $ u_1 = b$

Using $ f(\frac {f(x)}x) = x$, it's easy to establish that $ f(a^{u_n}) = a^{u_{n + 1}}$ $ \forall n\le 0$ where $ u_n$ is the fibonacci sequence (extended to negative index) with $ u_0 = 1$ and $ u_1 = b$

Let then $ u = \frac {1 + \sqrt 5}2$ and $ v = \frac {1 - \sqrt 5}2$, we have $ u_n = \frac {(b - v)u^n - (b - u)v^n}{u - v}$ $ \forall n\in\mathbb Z$

Since $ f(x)$ is strictly monotonous, $ \frac {f(x) - f(y)}{x - y}$, when $ x\ne y$, has a constant sign ($ +$ is increasing, $ -$ is decreasing).

So $ A_n = \frac {f(a^{u_{n + 1}}) - f(a^{u_n})}{a^{u_{n + 1}} - a^{u_n}}$ has a constant sign.

$ A_n = \frac {a^{u_{n + 2}} - a^{u_{n + 1}}}{a^{u_{n + 1}} - a^{u_n}}$ $ = a^{u_{n - 1}}\frac {a^{u_n} - 1}{a^{u_{n - 1}} - 1}$ and so sign of $ A_n$ is sign of $ \frac {u_n}{u_{n - 1}}$

So sign of $ A_n$ is sign of $ B_n = \frac {(b - v)u^n - (b - u)v^n}{(b - v)u^{n - 1} - (b - u)v^{n - 1}}$

If $ b\ne v$ $ \lim_{n\to + \infty}B_n = u > 0$ and so the only decreasing solution may be obtained when $ b = v$ $ \forall a\ne 1$ and so $ f(x) = x^v$, which indeed is a solution.
(notice that we obtained $ f(x) = x^v$ only for $ x\ne 1$ but we extend this at $ f(1)$ thru continuity)

If $ b\ne u$ $ \lim_{n\to - \infty}B_n = v < 0$ and so the only increasing solution may be obtained when $ b = u$ $ \forall a\ne 1$ and so $ f(x) = x^u$, which indeed is a solution.
(notice that we obtained $ f(x) = x^u$ only for $ x\ne 1$ but we extend this at $ f(1)$ thru continuity)

Hence the only two continuous solutions :

$ f(x) = x^{\frac {1 + \sqrt 5}2}$ (the unique increasing solution)

$ f(x) = x^{\frac {1 - \sqrt 5}2}$ (the unique decreasing solution)
\end{solution}



\begin{solution}[by \href{https://artofproblemsolving.com/community/user/67949}{aktyw19}]
	thanks  
\end{solution}
*******************************************************************************
-------------------------------------------------------------------------------

\begin{problem}[Posted by \href{https://artofproblemsolving.com/community/user/56873}{duythuc_lqd}]
	Find all function $ f: \mathbb R \to \mathbb R$ that satisfy \[ f(x + \cos(2007y))=f(x)+2007\cos(f(y)),\quad \forall x,y\in \mathbb R.\]
	\flushright \href{https://artofproblemsolving.com/community/c6h314367}{(Link to AoPS)}
\end{problem}



\begin{solution}[by \href{https://artofproblemsolving.com/community/user/29428}{pco}]
	\begin{tcolorbox}Find all function $ f: R \to R$ satisfy $ f(x + cos(2007y)) = f(x) + 2007cos(f(y)), \forall x,y\in R$\end{tcolorbox}

Let $ P(x,y)$ be the assertion $ f(x + \cos(2007y)) = f(x) + 2007\cos(f(y))$

Subtracting $ P(0,y)$ from $ P(x,y)$, we get $ f(x + \cos(2007y)) = f(x) + f(\cos(2007y)) - f(0)$

Using then $ g(x) = f(x) - f(0)$, we get $ g(0) = 0$ and $ g(x + y) = g(x) + g(y)$ $ \forall x, \forall y\in[ - 1,1]$

From this, it is easy to show thru induction that $ g(py) = pg(y)$ $ \forall y\in[ - 1,1]$ $ \forall p\in\mathbb Z$
Then a new induction establishes $ g(x + py) = g(x) + pf(y) = g(x + py)$, $ \forall x$, $ \forall y\in[ - 1,1]$ and $ \forall p\in\mathbb Z$

And so $ g(x + y) = g(x) + g(y)$ $ \forall x,y$

But $ P(0,x)$ $ \implies$ $ f(\cos(2007x)) = f(0) + 2007\cos(f(x))$ and so $ g(x)\in[ - 2007, + 2007]$ $ \forall x\in[ - 1,1]$

So $ g(x)$ is a solution of Cauchy's equation bounded on $ [ - 1, + 1]$ and so $ g(x) = ax$ and $ f(x) = ax + b$

Plugging this back in the original equation, we get $ a\cos(2007y) = 2007\cos(ay + b)$ and so the solutions :

$ f(x) = 2007x + 2k\pi$
$ f(x) = -2007x + (2k+1)\pi$
\end{solution}
*******************************************************************************
-------------------------------------------------------------------------------

\begin{problem}[Posted by \href{https://artofproblemsolving.com/community/user/67949}{aktyw19}]
	Find all values the parameter $ a$ and for which there exists a function $f: \mathbb R \to \mathbb R$ for which
\[f(x^2+y+f(y))=(f(x))^2+ay\]
for all $x,y \in \mathbb R$.
	\flushright \href{https://artofproblemsolving.com/community/c6h314375}{(Link to AoPS)}
\end{problem}



\begin{solution}[by \href{https://artofproblemsolving.com/community/user/29428}{pco}]
	\begin{tcolorbox}Find all values the parameter $ a$ and for which there exists a function $ f: R \rightarrow R$ 
$ f(x^2 + y + f(y)) = (f(x))^2 + ay$.\end{tcolorbox}

I'm sorry for my long rather ugly proof and I hope somebody will provide a nicer one  :blush: 

Let $ P(x,y)$ be the assertion $ f(x^2 + y + f(y)) = f(x)^2 + ay$
Let $ c = f(0)$

If $ a = 0$, we have solutions, for example $ f(x) = 0$ $ \forall x$.
So we'll consider from now that $ a\ne 0$.

1) $ f(x)$ is a surjection
===============
Just choose $ y$ such that RHS take any value you want.

2) $ f(x)$ is an injection, so a bijection and $ f(0) = 0$ and $ f( - x) = - f(x)$ $ \forall x$
=================================================
2.1) $ f(u) = f(v)\le 0$ $ \implies$ $ u = v$
--------------------------------------------
Let $ x\le 0$. Since $ f(x)$ is a surjection, let $ y$ such that $ f(y) = x$. 

Then $ P(\sqrt { - x},y)$ $ \implies$ $ x = f(\sqrt { - x})^2 + ay$ and so $ y = \frac {x - f(\sqrt { - x})^2}a$ and so $ f(u) = f(v) = x\le 0$ $ \implies$ $ u = v = \frac {x - f(\sqrt { - x})^2}a$

2.2) $ f(u) = f(v)$ $ \implies$ $ u^2 = v^2$
---------------------------------------------
Let $ f(u) = f(v)$. We can choose $ y$ such that $ f(u)^2 + ay \le 0$ and so $ f(u^2 + y + f(y)) = f(v^2 + y + f(y)) = f(u)^2 + ay\le 0$ and so $ u^2 = v^2$
Q.E.D.

2.3) $ f( - x) = \pm f(x)$
-----------------------
Just compare $ P(x,y)$ and $ P( - x,y)$

2.4) $ x = 0$ $ \iff$ $ f(x) = 0$
----------------------------
Let $ x < 0$ and $ \ne f(0)$. Since $ f(x)$ is a surjection, let $ u$ such that $ f(u) = x$. According to 2.3, either $ f( - u) = f(u)$, either $ f( - u) = - f(u)$
If $ f( - u) = f(u) = x < 0$, according to 2.1, we get $ - u = u$ and so $ u = 0$, which is wrong since $ f(u) = x\ne f(0)$
So $ f( - u) = - f(u)$. Then :

$ P(0,u)$ $ \implies$ $ f(u + f(u)) = c^2 + au$
$ P(0, - u)$ $ \implies$ $ f( - u - f(u)) = c^2 - au$ and so either $ c^2 + au = c^2 - au$ and so $ u = 0$, wrong, either $ c^2 + au = - c^2 + au$ and so $ c = 0$
So $ x = 0$ $ \implies$ $ f(x) = 0$

Now, if $ f(x) = 0$, we get $ f(x) = f(0)$ and, according to 2.2, $ x^2 = 0^2$ and so $ x = 0$.
Q.E.D

2.5) $ f( - x) = - f(x)$ $ \forall x$
-------------------------------
$ f( - 0) = - f(0) = 0$
Suppose now $ \exists u\ne 0$ such that $ f( - u) = f(u)\ne 0$. Then, since $ f(x)$ is a surjection, $ \exists w$ such that $ f(w) = - f(u)$
Then , if $ f( - w) = - f(w)$, we get that $ f( - w) = f(u)$ and so $ w^2 = u^2$ and $ w = \pm u$ which is impossible since $ f(u) = f( - u)$ while $ f(w) = - f(u)\ne f(u)$. 
So $ f( - w) = f(w) = - f(u) = - f( - u)$ but this is also impossible since one of the two values $ f(u)$ or $ f(w)$ is negative :
If $ f(u) = f( - u) < 0$, according to 2.1, $ u = - u$ and so $ u = 0$, wrong
If $ f(w) = f( - w) < 0$, according to 2.1, $ w = - w$ and so $ w = 0$ and $ f(u) = - f(w) = 0$ and so, according to 2.4, $ u = 0$, wrong

So $ f( - u) = - f(u)$
Q.E.D.

2.6) $ f(x)$ is injective
------------------------
From 2.2, we get that $ f(u) = f(v)$ $ \implies$ $ u^2 = v^2$. 
If $ u = - v$, we have then $ f(v) = - f(u) = f(u)$ and so $ u = v = 0$ 
So $ f(u) = f(v)$ $ \implies$ $ u = v$
Q.E.D.

3) $ f(x) = x$ $ \forall x$
=============
$ P(x,0)$ $ \implies$ $ f(x^2) = f(x)^2$
$ P(0,x)$ $ \implies$ $ f(x + f(x)) = ax$

So (transforming $ P(x,y)$) : $ f(x^2 + y + f(y)) = f(x^2) + f(y + f(y))$

$ P(0,\frac {f(x)}a)$ $ \implies$ $ f(\frac {f(x)}a + f(\frac {f(x)}a)) = f(x)$ and, since $ f(x)$ is injective : $ \frac {f(x)}a + f(\frac {f(x)}a) = x$

So $ x + f(x)$ is surjective and "$ f(x^2 + y + f(y)) = f(x^2) + f(y + f(y))$" may be written :

$ f(x + y) = f(x) + f(y)$ $ \forall x\ge 0$, $ \forall y$

And, since $ f( - x) = - f(x)$ : $ f(x + y) = f(x) + f(y)$ $ \forall x,y$

So we have a solution of Cauchy's equation which is $ \ge 0$ $ \forall x\ge 0$ (since $ f(x^2) = f(x)^2$)
So $ f(x) = bx$

Plugging back in the equation, we get $ b = 1$ (since $ a\ne 0$) and so $ a = 2$

4) Solution
=======
The required values are $ a\in\{0,2\}$
\end{solution}



\begin{solution}[by \href{https://artofproblemsolving.com/community/user/67949}{aktyw19}]
	thanks for the detailed solution
\end{solution}
*******************************************************************************
-------------------------------------------------------------------------------

\begin{problem}[Posted by \href{https://artofproblemsolving.com/community/user/63660}{Victory.US}]
	Determine all function $f: \mathbb R \to \mathbb R$ such that
\[ f(x^2f(x)+f(y))=(f(x))^3+y, \quad \forall x,y\in \mathbb R.\]
	\flushright \href{https://artofproblemsolving.com/community/c6h314427}{(Link to AoPS)}
\end{problem}



\begin{solution}[by \href{https://artofproblemsolving.com/community/user/48278}{Dimitris X}]
	Very nice!!!
I hope my solution is right....

\begin{italicized}Proof\end{underlined}\end{italicized}

First of all we can prove that $ f$ is ''1-1'' function.

Let $ y_1,y_2 \in \mathbb{R}$ so that 

$ f(y_1) = f(y_2) \Longrightarrow x^2f(x) + f(y_1) = x^2f(x) + f(y_2) \Longrightarrow f(x^2f(x) + f(y_1)) = f(x^2f(x) + f(y_2)) \Longrightarrow (f(x))^3 + y_1 = (f(x))^3 + y_2 \Longrightarrow y_1 = y_2$.

Let $ P(x,y)$ the assertion of $ f(x^2f(x) + f(y)) = (f(x))^3 + y$.

$ P(0,0) \Longrightarrow f(s) = s^3$

$ P(0,x) \Longrightarrow f(f(x)) = s^3 + x$

$ P(x,0) \Longrightarrow f(x^2f(x) + s) = (f(x))^3$.

Setting int the last $ x - - > f(x)$ we take:

$ f(x^2(s^3 + x) + s) = (s^3 + x)^3$.

Now in the last setting $ x = 0$ we take:

$ f(s) = s^9$.

So 
$ s^3 = s^9$.

So 

$ i)s = 0$.

we take:

$ f(f(x)) = x$ and $ f(x^2f(x)) = (f(x))^3$

so for $ x - - > f(x)$ we take $ f(x^2 \cdot x) = x^3 \Longrightarrow f(x^3) = x^3$, so ${ f(x) = x,\forall x \in \mathbb{R}}$.

we can easily check that this value is true..

$ ii)s = 1$

we take from $ f(x^2f(x) + s) = (f(x))^3$ and $ f(f(x)) = s^3 + x:$

$ f(x^2f(x) + 1) = (f(x))^3$. and $ f(f(x)) = x + 1$.

So from this two we take by setting $ x - - > f(x)$ in the first:

$ f(x^3 + x^2 + 1) = (x + 1)^3$.

Now setting in the last $ x = 0$ we take $ f(1) = 1$.

But we already know that $ f(0) = 1$ and f is 1-1 function $ \Longrightarrow$ contradiction

$ iii)s = - 1$

working as in the second we take:

$ f(x^3 - x^2 - 1) = (x - 1)^3$.

for $ x = 0$ we take $ f( - 1) = - 1$ 
So $ f( - 1) = f(0)$ and f ''1-1'' $ \Longrightarrow$ contradiction.

So finaly we take that 

$ \boxed{f(x) = x},\forall x\in \mathbb{R}$.
 :)

Were did you find this functional equation \begin{bolded}Victory.US\end{bolded}
\end{solution}



\begin{solution}[by \href{https://artofproblemsolving.com/community/user/29428}{pco}]
	\begin{tcolorbox}Very nice!!!
I hope my solution is right....

\begin{italicized}Proof\end{underlined}\end{italicized}

...
$ P(0,x) \Longrightarrow f(f(x)) = s^3 + x$

$ P(x,0) \Longrightarrow f(x^2f(x) + s) = (f(x))^3$.

Setting int the last $ x - - > f(x)$ we take:

$ f(x^2(s^3 + x) + s) = (s^3 + x)^3$.

\end{tcolorbox}

I'm affraid there is an error here.
Setting $ x\to f(x)$ in $ f(x^2f(x) + s) = (f(x))^3$ gives $ f(f(x)^2(s^3 + x) + s) = (s^3 + x)^3$ instead of $ f(x^2(s^3 + x) + s) = (s^3 + x)^3$

:(
\end{solution}



\begin{solution}[by \href{https://artofproblemsolving.com/community/user/29428}{pco}]
	\begin{tcolorbox}Determine all function $ f: R\to R$ such that
                  $ f(x^2f(x) + f(y)) = (f(x))^3 + y$ ,$ \forall x;y\in R$\end{tcolorbox}

Let $ P(x,y)$ be the assertion $ f(x^2f(x)+f(y))=f(x)^3+y$

1) $ f(x)$ is bijective 
$ f(y_1)=f(y_2)$ $ \implies$ $ f(x^2f(x)+f(y_1))=f(x^2f(x)+f(y_2))$ $ \implies$ $ f(x)^3+y_1=f(x)^3+y_2$ $ \implies$ $ y_1=y_2$
So $ f(x)$ is injective.

$ P(0,u-f(0)^3)$ $ \implies$ $ f(\text{something})=u$ 
So $ f(x)$ is surjective.

2) $ f(f(x))=x$
Since $ f(x)$ is a bijection, $ \exists u$ such that $ f(u)=0$
$ P(u,x)$ $ \implies$ $ f(f(x))=x$

3) $ x^2f(x)$ is surjective
Let $ u\in\mathbb R$ and $ v$ such that $ f(v)= \sqrt[3]{f(u)-f(0)}$
$ P(v,f(0))$ $ \implies$ $ f(v^2f(v))=f(v)^3+f(0)=f(u)$ $ \implies$ (since $ f(x)$ is injective) $ v^2f(v)=u$
Q.E.D

4) $ f(x+y)=f(x)+f(y)-f(0)$
$ P(x,f(0))$ $ \implies$ $ f(x^2f(x))=f(x)^3+f(0)$
So $ P(x,f(y))$ may be written $ f(x^2f(x)+y)=f(x^2f(x))+f(y)-f(0)$
And, since $ x^2f(x)$ is surjective : $ f(x+y)=f(x)+f(y)-f(0)$

5) Solution
Let $ g(x)=f(x)-f(0)$. from 4. above, we get : $ g(x+y)=g(x)+g(y)$
If $ g(x)$ is continuous (not in the problem statement, but in the post title), it's a continuous solution of Cauchy's equation
So $ g(x)=ax$
So $ f(x)=ax+b$
Plugging back in the original equation, we get $ \boxed{f(x)=x}$ $ \forall x$
\end{solution}
*******************************************************************************
-------------------------------------------------------------------------------

\begin{problem}[Posted by \href{https://artofproblemsolving.com/community/user/34524}{Algadin}]
	Find all continous functions $f: \mathbb R \to \mathbb R$ which satisfy
\[ f(x+y)=f(x+f(y))\]
for all real numbers $x$ and $y$.
	\flushright \href{https://artofproblemsolving.com/community/c6h314628}{(Link to AoPS)}
\end{problem}



\begin{solution}[by \href{https://artofproblemsolving.com/community/user/46039}{ll931110}]
	\begin{tcolorbox}Find all continous function $ f: R\to R$ which satisfy:
$ f(x + y) = f(x + f(y)) (1)$\end{tcolorbox}

Since (1), put x = 0 gives $ f(f(y)) = f(y)$ for all real number y, so $ f(x) = x$ for all x in the value set of f
Let $ a$ and $ b$, respectively, be the minimum and maximum of $ f$ ($ a$ and $ b$ might be infinitive)
- If $ a = b$, then f is constant.
- If $ a \neq b$:
  + If $ a$ is finite, then $ a = lim_{x \rightarrow a} f(x) = f(lim_{x \rightarrow a} x) = f(a)$
So $ f(a) = a$, and since $ f$ is continuous, so there exists a real number $ d$ such that
$ f(x) = x \forall x \in [a; a + 2d]$
Now we prove that $ f(a - s) \ge a + d \forall s \in (0;d]$
Indeed, suppose there exists $ s$ such that $ f(a - s) = a + q < a + d$, we have
$ a = f(a - s + s) = f(f(a - s) + s) = f(a + q + s) = a + q + s$ (since $ q + s < 2d$)
$ \rightarrow q + s = 0$, which is obviously wrong

So $ f(a - s) \ge a + d \forall x \in (0; d]$. But $ f(a) = a$, which contradicts the Intermediate Value Theorem. So our suppose is wrong and $ a$ is infinite. Similarity, $ b$ is infinite, which means $ f(x) = x$ for all real number x

Conclusion:
There are two functions satisfying the functional equation is $ f(x) = c$ and $ f(x) = x$
\end{solution}



\begin{solution}[by \href{https://artofproblemsolving.com/community/user/29428}{pco}]
	\begin{tcolorbox}Find all continous function $ f: R\to R$ which satisfy:
$ f(x + y) = f(x + f(y))$\end{tcolorbox}

Let $ g(x)=f(x)-x$. The equation becomes $ f(x)=f(x+g(y))$

If $ g(x)=c$ (constant), then $ f(x)=x+c$ and so, plugging back in original equation, $ c=0$ and $ f(x)=x$ which, indeed, is a solution.

If $ g(x)$ is non constant, and continuous, $ \exists [a,b]\subset g(\mathbb R)$ with $ b>a$. So $ f(x)=f(x+g(y))$ implies $ f(x)$ constant over $ [x+a,x+b]$ $ \forall x$ and so $ f(x)$ constant over $ \mathbb R$, which indeed is a solution.

Hence the two solutions :
$ f(x)=c$ $ \forall x$
$ f(x)=x$ $ \forall x$
\end{solution}



\begin{solution}[by \href{https://artofproblemsolving.com/community/user/46039}{ll931110}]
	Nice solution, pco. And here is another solution of my friend:

It's easy to check that $ f(x) = x$ and $ f(x) = c$ satisfying the equation
Suppose that $ f(x) \neq x$, then there exists a real number $ y_0$ such that $ f(y_0) - y_0 = k \neq 0$
And we have
$ f(x + y_0) = f(x + f(y_0) \rightarrow f(x) = f(x + k)$, implying $ f$ is periodic.
Since $ f$ is continuous and periodic, so there exists a real number $ T > 0$ be the smallest period of $ f$

Since $ f(x + y) = f(x + f(y)$, it implies that $ f$ is periodic, and $ f(y) - y$ is the period. So $ |f(y) - y| = T.h$, where $ h \in N$
Because $ f$ is continuous, so $ lim_{y \rightarrow y_0} [f(y) - y] = f(y_0) - y_0 = T.s$, implying there exists $ d > 0$ such that
$ |[f(y) - y] - [f(y_0) - y_0]| < T \forall y \in (y_0 - d; y_0 + d)$, implying $ f(y) - y = f(y_0) - y_0$
Finally, we have $ f(x) = x + c$. Putting back into the first equation implies $ c = 0$

Conclusion:
There are two functions satisfying the functional equation, namely $ f(x) = x$ and $ f(x) = c$
\end{solution}
*******************************************************************************
-------------------------------------------------------------------------------

\begin{problem}[Posted by \href{https://artofproblemsolving.com/community/user/19427}{TRAN THAI HUNG}]
	Find all functions $f: \mathbb R \to \mathbb R$ such that
\[f(xy+x+y)=f(xy)+f(x)+f(y)\] for all $ x,y \in \mathbb R$.
	\flushright \href{https://artofproblemsolving.com/community/c6h314645}{(Link to AoPS)}
\end{problem}



\begin{solution}[by \href{https://artofproblemsolving.com/community/user/29428}{pco}]
	\begin{tcolorbox}Find $ f: R \to R$ sastisfy
$ f(xy + x + y) = f(xy) + f(x) + f(y)$ for all $ x,y \in R$\end{tcolorbox}

Are you sure there is not a supplementary constraint (for example continuity ?)

If not, you have infinitely many solutions (at least all solutions of Cauchy's equation).
\end{solution}



\begin{solution}[by \href{https://artofproblemsolving.com/community/user/19427}{TRAN THAI HUNG}]
	\begin{tcolorbox}[quote="TRAN THAI HUNG"]Find $ f: R \to R$ sastisfy
$ f(xy + x + y) = f(xy) + f(x) + f(y)$ for all $ x,y \in R$\end{tcolorbox}

Are you sure there is not a supplementary constraint (for example continuity ?)

If not, you have infinitely many solutions (at least all solutions of Cauchy's equation).\end{tcolorbox}
 :oops: Sorry, pco. That is my mistakes. I forgot the 'continuity' :( .Forgive me :blush:
\end{solution}



\begin{solution}[by \href{https://artofproblemsolving.com/community/user/29428}{pco}]
	\begin{tcolorbox}Find $ f: R \to R$ sastisfy
$ f(xy + x + y) = f(xy) + f(x) + f(y)$ for all $ x,y \in R$\end{tcolorbox}
+ "$ f(x)$" continuous

Let $ P(x,y)$ be the assertion $ f(xy+x+y)=f(xy)+f(x)+f(y)$

$ P(0,0)$ $ \implies$ $ f(0)=0$

Let $ x,y\in[0,1]$ such that $ x+y\le 1$ :

$ P(xy,\frac{x+y}{xy+1})$ $ \implies$ $ f(xy+x+y)=f(\frac{xy(x+y)}{xy+1})+$ $ f(xy)+f(\frac{x+y}{u+1})$ and so, subtracting $ P(x,y)$ :

$ f(x)+f(y)=f(\frac{xy(x+y)}{xy+1})+f(\frac{x+y}{xy+1})$

Let then the sequences $ u_n$ and $ v_n$ defined as :
$ u_0=x$, $ v_0=y$, $ u_{n+1}=\frac{u_n+v_n}{u_nv_n+1}$ and $ v_{n+1}=u_n+v_n-u_{n+1}$

It's easy to show that $ v_n$ is a decreasing sequence whose limit is $ 0$ while $ x_n$ is increasing up to $ x+y$

Then $ f(x)+f(y)=f(u_n)+f(v_n)$ and continuity implies $ f(x+y)=f(x)+f(y)$ $ \forall x,y,x+y\in[0,1]$

Then $ f(x)=f(1)x$ $ \forall x\in[0,1]$ (it's not exactly Cauchy but the same mechanism may be used).

Then $ P(x,1)$ $ \implies$ $ f(2x+1)=2f(x)+f(1)$ and so $ f(x)=f(1)x$ $ \forall x\in[0,3]$ and so on ... $ f(x)=x$ $ \forall x\ge 0$
Then $ P(x,-x)$ $ \implies$ $ f(-x)=-f(x)$ and so $ f(x)=x$ $ \forall x$

And it's easy to check that this indeed is a solution.
\end{solution}



\begin{solution}[by \href{https://artofproblemsolving.com/community/user/19427}{TRAN THAI HUNG}]
	\begin{tcolorbox}[quote="TRAN THAI HUNG"]Find $ f: R \to R$ sastisfy
$ f(xy + x + y) = f(xy) + f(x) + f(y)$ for all $ x,y \in R$\end{tcolorbox}
+ "$ f(x)$" continuous

Let $ P(x,y)$ be the assertion $ f(xy + x + y) = f(xy) + f(x) + f(y)$

$ P(0,0)$ $ \implies$ $ f(0) = 0$

Let $ x,y\in[0,1]$ such that $ x + y\le 1$ :

$ P(xy,\frac {x + y}{xy + 1})$ $ \implies$ $ f(xy + x + y) = f(\frac {xy(x + y)}{xy + 1}) +$ $ f(xy) + f(\frac {x + y}{u + 1})$ and so, subtracting $ P(x,y)$ :

$ f(x) + f(y) = f(\frac {xy(x + y)}{xy + 1}) + f(\frac {x + y}{xy + 1})$

Let then the sequences $ u_n$ and $ v_n$ defined as :
$ u_0 = x$, $ v_0 = y$, $ u_{n + 1} = \frac {u_n + v_n}{u_nv_n + 1}$ and $ v_{n + 1} = u_n + v_n - u_{n + 1}$

It's easy to show that $ v_n$ is a decreasing sequence whose limit is $ 0$ while $ x_n$ is increasing up to $ x + y$

Then $ f(x) + f(y) = f(u_n) + f(v_n)$ and continuity implies $ f(x + y) = f(x) + f(y)$ $ \forall x,y,x + y\in[0,1]$

Then $ f(x) = f(1)x$ $ \forall x\in[0,1]$ (it's not exactly Cauchy but the same mechanism may be used).

Then $ P(x,1)$ $ \implies$ $ f(2x + 1) = 2f(x) + f(1)$ and so $ f(x) = f(1)x$ $ \forall x\in[0,3]$ and so on ... $ f(x) = x$ $ \forall x\ge 0$
Then $ P(x, - x)$ $ \implies$ $ f( - x) = - f(x)$ and so $ f(x) = x$ $ \forall x$

And it's easy to check that this indeed is a solution.\end{tcolorbox}
I just wonder how you approach to choose $ P(xy,\frac{x+y}{xy+1})$
and can we get f(x)= ax directly from $ f(x+y)=f(x)+f(y)$$ \forall x,y,x+y\in[0,1]$
\end{solution}



\begin{solution}[by \href{https://artofproblemsolving.com/community/user/29428}{pco}]
	\begin{tcolorbox} I just wonder how you approach to choose $ P(xy,\frac {x + y}{xy + 1})$...\end{tcolorbox}

I tried to write $ xy+x+y=uv+u+v$ and so $ y=\frac{uv+u+v-x}{x+1}$

Then $ P(x,\frac{uv+u+v-x}{x+1})$ $ \implies$ $ f(uv+u+v)=f(x\frac{uv+u+v-x}{x+1})+f(x)+f(\frac{uv+u+v-x}{x+1})$
And $ P(u,v)$ $ \implies$ $ f(uv+u+v)=f(uv)+f(u)+f(v)$

So $ f(uv)+f(u)+f(v)$ $ =f(x\frac{uv+u+v-x}{x+1})+f(x)+f(\frac{uv+u+v-x}{x+1})$

And then I tried to equate one left term with one right term. And the only interesting was $ x=uv$ ...

\begin{tcolorbox} ... and can we get f(x)= ax directly from $ f(x + y) = f(x) + f(y)$$ \forall x,y,x + y\in[0,1]$\end{tcolorbox}

Induction gives $ f(\frac pq)=pf(\frac 1q)$ $ \forall 1\le p\le q$ and so $ f(\frac 1q)=\frac{f(1)}q$ and so $ f(\frac pq)=f(1)\frac pq$ and continuity gives the conclusion
\end{solution}
*******************************************************************************
-------------------------------------------------------------------------------

\begin{problem}[Posted by \href{https://artofproblemsolving.com/community/user/67111}{Abdek}]
	Find all fuctions $ f: \mathbb{N^{*}} \longrightarrow \mathbb{N^*}$ satisfying $f(1)=1$ and 
\[f(a^2+b^2+c^2+d^2)=f(a)^2+f(b)^2+f(c)^2+f(d)^2\]
for all $(a,b,c,d) \in \mathbb{N^*}^4$.
	\flushright \href{https://artofproblemsolving.com/community/c6h314704}{(Link to AoPS)}
\end{problem}



\begin{solution}[by \href{https://artofproblemsolving.com/community/user/46039}{ll931110}]
	\begin{tcolorbox}Find all fuctions $ f: \mathbb{N^{*}} \longrightarrow \mathbb{N^*}$ Satisfying:

$ \forall (a,b,c,d) \in \mathbb{N^*}^4$

$ f(a^2 + b^2 + c^2 + d^2) = f(a)^2 + f(b)^2 + f(c)^2 + f(d)^2$ and $ f(1) = 1$\end{tcolorbox}

To simply our proof, denote $ P(x,y,z,t) = f^2(x) + f^2(y) + f^2(z) + f^2(t)$

From $ f(x^2 + y^2 + z^2 + t^2) = P(x,y,z,t)$ we obtain this following result:
If $ a_1,...,a_8$ be positive integers satisfying $ a_1^2 + ... + a_4^2 = a_5^2 + ... + a_8^2$ and $ f(a_i) = a_i \forall 1 \le i \le 7$, then $ f(a_8) = a_8$

Now we notice that $ P(2k + 1,k - 2,1,1) = P(2k - 1,k + 2,1,1)$ and $ P(2k,k - 5,1,1) = P(2k - 4,k + 3,1,1)$, so we will have $ f(n) = n \forall n \in N*$ if $ f(n) = n \forall 1 \le n \le 10$
- $ f(4) = 4f^2(1) \rightarrow f(4) = 4$
- $ P(5,2,1,1) = P(4,4,1,1)$ and $ P(5,1,1,1) = P(4,2,2,2)$, so $ f(2) = 2$ and $ f(5) = 5$
- $ P(3,3,3,1) = P(5,1,1,1) \rightarrow f(3) = 3$
- $ P(2,3,5,5) = P(1,1,5,6) \rightarrow f(6) = 6$
- $ f(7) = f^2(2) + 3f^2(1) \rightarrow f(7) = 7$
- $ P(8,1,1,1) = P(7,4,1,1) \rightarrow f(8) = 8$
- $ P(9,2,1,1) = P(7,6,1,1) \rightarrow f(9) = 9$
- $ P(10,10,3,1) = P(9,8,8,1) \rightarrow f(10) = 10$

And we conclude the only function is $ f(n) = n$
\end{solution}



\begin{solution}[by \href{https://artofproblemsolving.com/community/user/62562}{shoki}]
	we can easily get f(0)=0 and f(1)=1, f(2)=2,f(3)=3,f(4)=4
now we prove the rest by induction and we use the fact that every integer can be written as the sum of four squares:
we know that f(k)=k for all k=1,...,n-1 ; n>4
now we know that for some a,b,c,d we have n=a²+b²+c²+d² which implies f(n)=f(a)²+f(b)²+f(c)²+f(d)²
and since a,b,c,d < n so we get f(n)=a²+b²+c²+d²=n
\end{solution}



\begin{solution}[by \href{https://artofproblemsolving.com/community/user/29428}{pco}]
	\begin{tcolorbox}we can easily get f(0)=0 and f(1)=1, f(2)=2,f(3)=3,f(4)=4
now we prove the rest by induction and we use the fact that every integer can be written as the sum of four squares:
we know that f(k)=k for all k=1,...,n-1 ; n>4
now we know that for some a,b,c,d we have n=a²+b²+c²+d² which implies f(n)=f(a)²+f(b)²+f(c)²+f(d)²
and since a,b,c,d < n so we get f(n)=a²+b²+c²+d²=n\end{tcolorbox}

No. The only difficulty of this problem was that it involved $ \mathbb N^*$. So no $ f(0)$ and the equality is true only for elements $ >0$ and Lagrange theorem is true only for squares $ >=0$, so cant be used here. 
Sorry.
\end{solution}



\begin{solution}[by \href{https://artofproblemsolving.com/community/user/62562}{shoki}]
	sorry but i didn't understand what u said,(sorry for my poor english)
can u explain plz?
thx
(as i know $ N^*=N \cup \{0 \}$ ! plz tell me where i'm wrong)
\end{solution}



\begin{solution}[by \href{https://artofproblemsolving.com/community/user/67111}{Abdek}]
	\begin{tcolorbox}sorry but i didn't understand what u said,(sorry for my poor english)
can u explain plz?
thx
(as i know $ N^* = N \cup \{0 \}$ ! plz tell me where i'm wrong)\end{tcolorbox}

$ \mathbb{N^*}=\mathbb{N}-\{0\}$
\end{solution}



\begin{solution}[by \href{https://artofproblemsolving.com/community/user/62562}{shoki}]
	:!:   that's very amazing it seems that in iran $ N$ is different from the other countries! :huh: 
but i'm sure that $ N^*=N \cup \{0 \}$ and it follows that $ 0\not \in N$ (at least here it is so!)
anyway,i'm really sorry ... :(
\end{solution}



\begin{solution}[by \href{https://artofproblemsolving.com/community/user/29428}{pco}]
	\begin{tcolorbox}:!:   that's very amazing it seems that in iran $ N$ is different from the other countries! :huh: 
but i'm sure that $ N^* = N \cup \{0 \}$ and it follows that $ 0\not \in N$ (at least here it is so!)
anyway,i'm really sorry ... :(\end{tcolorbox}

According to me, $ \mathbb N^*$ is a notation used in countries where $ 0\in\mathbb N$ in order to describe $ \mathbb N - \{0\}$

Dont you use $ \mathbb R^*$ in order to describe $ \mathbb R-\{0\}$ ?
\end{solution}



\begin{solution}[by \href{https://artofproblemsolving.com/community/user/67111}{Abdek}]
	Nice solution ll931110   :) 

In my solution I used the fact that 

$ n^2 + n^2 + 3^2 + 1^2 = (n + 1)^2 + (n - 1)^2 + 2^2 + 2^2$

and it follows that $ f(n + 1)^2 - f(n)^2 = f(n)^2 - f(n - 1)^2 + f(3)^2 + f(1)^2 - 2f(2)^2$

we can easly get by replacing $ n$ by $ 3$ $ : f(3)^2 - f(2)^2 = 5 \implies f(3) = 3$ and $ f(2) = 2$ since $ 5$ is prime 

and hence $ f(3)^2 + 1 - 2f(2)^2 > 0$ By the induction we can clealry see that $ f(n + 1)^2 - f(n)^2 > 0$ or $ f(n + 1)\ge f(n) + 1$

which guid us to $ f(n) \ge n$

we know that $ f(4^{2^n - 1}) = 4^{2^n - 1}$ (easy to prove by induction) hence if there exists some $ k < 4^{2^n - 1}$ such that $ f(k) > k \implies f(k + 1) > k + 1$ then $ f(4^{2^n - 1}) > 4^{2^n - 1}$ since $ f$ is increasing which is absurd 

Mr pco Can you verify this solution ? please  
\end{solution}



\begin{solution}[by \href{https://artofproblemsolving.com/community/user/46039}{ll931110}]
	\begin{tcolorbox}Nice solution ll931110   :) 

In my solution I used the fact that 

$ n^2 + n^2 + 3^2 + 1^2 = (n + 1)^2 + (n - 1)^2 + 2^2 + 2^2$

and it follows that $ f(n + 1)^2 - f(n)^2 = f(n)^2 - f(n - 1)^2 + f(3)^2 + f(1)^2 - 2f(2)^2$

we can easly get by replacing $ n$ by $ 3$ $ : f(3)^2 - f(2)^2 = 5 \implies f(3) = 3$ and $ f(2) = 2$ since $ 5$ is prime 

and hence $ f(3)^2 + 1 - 2f(2)^2 > 0$ By the induction we can clealry see that $ f(n + 1)^2 - f(n)^2 > 0$ or $ f(n + 1)\ge f(n) + 1$

which guid us to $ f(n) \ge n$

we know that $ f(4^{2^n - 1}) = 4^{2^n - 1}$ (easy to prove by induction) hence if there exists some $ k < 4^{2^n - 1}$ such that $ f(k) > k \implies f(k + 1) > k + 1$ then $ f(4^{2^n - 1}) > 4^{2^n - 1}$ since $ f$ is increasing which is absurd \end{tcolorbox}

We may make it simpler. Since $ n^2 + n^2 + 3^2 + 1^2 = (n + 1)^2 + (n - 1)^2 + 2^2 + 2^2$ implying $ f^2(n + 1) = 2f^2(n) - f^2(n - 1) + 2 = (n + 1)^2 \rightarrow f(n + 1) = n + 1$ (by induction)
\end{solution}
*******************************************************************************
-------------------------------------------------------------------------------

\begin{problem}[Posted by \href{https://artofproblemsolving.com/community/user/69304}{marin.bancos}]
	Let $ a$ and $b$ be fixed real numbers which are not zero at the same time. Find all functions $ f: \mathbb R\to \mathbb R$ such that
\[ af(xy)+bf(xz)-f(x)f(yz)\geq \frac{1}{4}\cdot(a+b)^2\]
holds for all $ x,y,z\in \mathbb R$.
	\flushright \href{https://artofproblemsolving.com/community/c6h314746}{(Link to AoPS)}
\end{problem}



\begin{solution}[by \href{https://artofproblemsolving.com/community/user/29428}{pco}]
	\begin{tcolorbox}Let $ a,b$ real fixed numbers, no both of them are zero.
Find all functions $ f: R\longrightarrow R$ such that
$ af(xy) + bf(xz) - f(x)f(yz)\geq \frac {1}{4}\cdot(a + b)^2$
holds for all $ x,y,z\in R$\end{tcolorbox}

Let $ s = a + b$
Let $ P(x,y,z)$ be the assertion $ af(xy) + bf(xz) - f(x)f(yz)\ge \frac {s^2}4$

$ P(0,0,0)$ $ \implies$ $ f(0)(s - f(0))\ge \frac {s^2}4$ $ \iff$ $ 4f(0)s - 4f(0)^2\ge s^2$ $ \iff$ $ 0\ge (s - 2f(0))^2$ and so $ f(0) = \frac s2$

$ P(1,1,1)$ $ \implies$ $ f(1)(s - f(1))\ge \frac {s^2}4$ $ \iff$ $ 4f(1)s - 4f(1)^2\ge s^2$ $ \iff$ $ 0\ge (s - 2f(1))^2$ and so $ f(1) = \frac s2$

$ P(x,0,0)$ $ \implies$ $ a\frac s2 + b\frac s2 - f(x)\frac s2\ge \frac {s^2}4$ $ \iff$ $ 2sf(x)\le s^2$

$ P(x,1,1)$ $ \implies$ $ af(x) + bf(x) - f(x)\frac s2\ge \frac {s^2}4$ $ \iff$ $ 2sf(x)\ge s^2$

So $ 2sf(x) = s^2$

If $ s\ne 0$, we get $ f(x) = \frac s2$ $ \forall x$ and this indeed is a solution.

If $ s = 0$, we get $ b = - a\ne 0$ and $ f(0) = f(1) = 0$ and then :
$ P(1,x,1)$ $ \implies$ $ af(x)\ge 0$
$ P(1,1,x)$ $ \implies$ $ - af(x)\ge 0$
And so $ f(x) = 0 = \frac s2$ $ \forall x$ which indeed is a solution.

Hence the answer : $ \boxed{f(x) = \frac {a+b}2}$ $ \forall x$
\end{solution}



\begin{solution}[by \href{https://artofproblemsolving.com/community/user/69304}{marin.bancos}]
	Dear Patrick,
Well done! Excellent! This is the solution to my functional inecuation. I hope you've enjoyed this problem.
\end{solution}



\begin{solution}[by \href{https://artofproblemsolving.com/community/user/68025}{Pirkuliyev Rovsen}]
	Bravo Patrick     super solution  !!!!!!!!!!!!!!!!!!!!!!
\end{solution}
*******************************************************************************
-------------------------------------------------------------------------------

\begin{problem}[Posted by \href{https://artofproblemsolving.com/community/user/43536}{nguyenvuthanhha}]
	Find all continuous functions $ f : \mathbb{R^+} \to\mathbb{R^+}$ such that
\[ f( f(xy) - xy ) + xf(y) + yf(x)  = f(xy) + f(x)f(y), \quad \forall x,y   \in  \mathbb{R^+}.\]
	\flushright \href{https://artofproblemsolving.com/community/c6h314979}{(Link to AoPS)}
\end{problem}



\begin{solution}[by \href{https://artofproblemsolving.com/community/user/29428}{pco}]
	\begin{tcolorbox}\begin{italicized}Find alll continuous function $ f \ : \ \mathbb{R^ + } \mapsto \mathbb{R^ + }$ such that :

        $ f( f(xy) - xy ) + xf(y) + yf(x) \ = \ f(xy) + f(x)f(y) \ \forall \ x;y \ \in \ \mathbb{R^ + }$\end{italicized}\end{tcolorbox}

Let $ P(x,y)$ be the assertion $ f(f(xy)-xy)+xf(y)+yf(x)=f(xy)+f(x)f(y)$
$ P(x,y)$ may be written : $ f(f(xy)-xy)-(f(xy)-xy)=(f(x)-x)(f(y)-y)$

In order to $ f(f(xy)-xy)$ be defined $ \forall x,y$, we need to have $ f(x)>x$ $ \forall x$

Let then $ g(x)=f(x)-x$, continuous function from $ \mathbb R^+\to\mathbb R^+$. We got $ g(g(xy))=g(x)g(y)$
Using $ y=1$, we get $ g(g(x))=g(1)g(x)$ and so $ g(g(xy))=g(1)g(xy)$ and so $ g(1)g(xy)=g(x)g(y)$

This last equation is very classical and continuity implies $ g(x)=ax^b$ for any $ a>0$ and any $ b$. Plugging this in $ g(g(x))=g(1)g(x)$, we get two solutions : $ g(x)=1$ or $ g(x)=ax$

And so $ \boxed{f(x)=x+1}$ or $ \boxed{f(x)=ax}$ ($ a>1$) and it is easy to check back that these two functions indeed are solutions.
\end{solution}
*******************************************************************************
-------------------------------------------------------------------------------

\begin{problem}[Posted by \href{https://artofproblemsolving.com/community/user/59935}{hana1122}]
	Find all continues functions $ f: \mathbb R\to\mathbb R$ such that for all $ x,y,z\in\mathbb R$,
\[f(x+f(y+f(z)))=f(x)+f(f(y))+f(f(f(z))).\]
	\flushright \href{https://artofproblemsolving.com/community/c6h315246}{(Link to AoPS)}
\end{problem}



\begin{solution}[by \href{https://artofproblemsolving.com/community/user/29428}{pco}]
	\begin{tcolorbox}Find all continues functions $ f: \mathbb R\to\mathbb R$ such that for all $ x,y,z\in\mathbb R$,
$ f(x + f(y + f(z))) = f(x) + f(f(y)) + f(f(f(z)))$\end{tcolorbox}

Let $ P(x,y,z)$ be the assertion $ f(x + f(y + f(z))) = f(x) + f(f(y)) + f(f(f(z)))$

$ P(0,0,0)$ $ \implies$ $ f(0) + f(f(0)) = 0$ and so either $ f(0) = f(f(0)) = 0$, either $ f(0)$ and $ f(f(0))$ have opposite signs and $ \exists u$ between them such that $ f(u) = 0$

So $ \exists u$ such that $ f(u) = 0$

$ P(x,y,u)$ $ \implies$ $ f(x + f(y)) = f(x) + f(f(y)) + f(f(0))$ and so $ g(x + f(y)) = g(x) + g(f(y))$ where $ g(x) = f(x) + f(f(0)) = f(x) - f(0)$. So :

$ g(x + y) = g(x) + g(y)$ $ \forall x\in\mathbb R,\forall y\in f(\mathbb R)$ and also (using $ x - y$ instead of $ x$) :
$ g(x - y) = g(x) - g(y)$ $ \forall x\in\mathbb R,\forall y\in f(\mathbb R)$

If $ f(x) = c$ constant, then $ P(x,y,z)$ $ \implies$ $ c = 3c$ and so $ f(x) = 0$ $ \forall x$, which indeed is a solution.

If $ f(x)$ is not constant, then $ \exists b\ne0$ and $ y$ such that $ g(y) = b\ne 0$ and an immediate induction gives :
$ g(x + ny) = g(x) + nb$ $ \forall x,\forall n\in\mathbb N$
$ g(x - ny) = g(x) - nb$ $ \forall x,\forall n\in\mathbb N$

and so, since $ g(x)$ is continuous,  $ g(\mathbb R) = \mathbb R$ and so $ f(\mathbb R) = \mathbb R$ and so :

$ g(x + y) = g(x) + g(y)$ $ \forall x,y$

And so $ g(x) = ax$ (continuous solutions of Cauchy) and $ f(x) = ax + b$

Plugging this back in the original equation, we get $ b(a + 2) = 0$ hence the solutions :

$ \boxed{f(x) = ax}$ and $ \boxed{f(x) = b - 2x}$
\end{solution}
*******************************************************************************
-------------------------------------------------------------------------------

\begin{problem}[Posted by \href{https://artofproblemsolving.com/community/user/63326}{diegu}]
	For any positive integer $n$, we define
\[f( n) = n +\max \left\{ m \in \mathbb N \cup \{0\}: 2 ^{2 ^{m}} \leq  n\cdot 2^{n}\right\}.\]
Find the image of function $f$.
	\flushright \href{https://artofproblemsolving.com/community/c6h316308}{(Link to AoPS)}
\end{problem}



\begin{solution}[by \href{https://artofproblemsolving.com/community/user/29428}{pco}]
	\begin{tcolorbox}for any positive integer,we have:
$ f( n) = n$ +$ max \{ m \in N : 2 ^{2 ^{m} } \leq n\cdot 2^{n}\}$

find the image of function f\end{tcolorbox}

Remark\end{underlined}: just notice that in order $ f(1)$ be defined, the $ \mathbb N$ inside the definition must include $ 0$, else $ max \{ m \in N : 2 ^{2 ^{m} } \leq 1\cdot 2^{1}\}$ would not be defined :(
So I consider the problem rewritten as : 
"Let $ \mathbb N$ be the set of positive integers and $ f(n) : \mathbb N\to\mathbb N$ such that $ \forall n\in\mathbb N$ :  $ f( n) = n$ +$ max \{ m \in \mathbb N\cup\{0\} : 2 ^{2 ^{m} } \leq n\cdot 2^{n}\}$. Find $ f(\mathbb N)$"

Then :
$ f(n) = n + [\log_2(n + \log_2(n))]$  and so :

$ \forall n\in[2^k,2^{k + 1} - k)$ : $ f(n) = n + k$
$ \forall n\in[2^{k + 1} - k,2^{k + 1})$ : $ f(n) = n + k + 1$

So $ f([2^k,2^{k + 1}) = [2^k + k,2^{k + 1} - 1]\cup[2^{k + 1} + 1,2^{k + 1} + k]$

So $ f(\mathbb N) = \mathbb N\backslash\{2^k,\forall k\in\mathbb N\}$
\end{solution}
*******************************************************************************
-------------------------------------------------------------------------------

\begin{problem}[Posted by \href{https://artofproblemsolving.com/community/user/25405}{AndrewTom}]
	Find all functions $f: \mathbb R \to \mathbb R$ such that for all reals $x$ and $y$,
\[f((x-y)^{2})=(f(x))^{2}-2xf(y) +y^{2}.\]
	\flushright \href{https://artofproblemsolving.com/community/c6h316317}{(Link to AoPS)}
\end{problem}



\begin{solution}[by \href{https://artofproblemsolving.com/community/user/29428}{pco}]
	\begin{tcolorbox}Find all functions $ f(x)$ such that, for all real $ x$ and $ y$, $ f(x)$ is real and

$ f((x - y)^{2}) = (f(x))^{2} - 2xf(y) + y^{2}$.\end{tcolorbox}

Let $ P(x,y)$ be the assertion $ f((x-y)^2)=f(x)^2-2xf(y)+y^2$
Let $ f(0)=a$

$ P(0,0)$ $ \implies$ $ a=a^2$ and so $ a\in\{0,1\}$
$ P(0,x)$ $ \implies$ $ f(x^2)=a^2+x^2=x^2+a$ and so $ f(x)=x+a$ $ \forall x\ge 0$

Let then $ x\ge 0$ :
$ (x-y)^2\ge 0$ and so $ f((x-y)^2)=(x-y)^2+a$
$ x\ge 0$ and so $ f(x)=x+a$ and so $ f(x)^2=(x+a)^2$

Then $ P(x,y)$ becomes $ (x-y)^2+a=(x+a)^2-2xf(y)+y^2$ and so $ 2xf(y)=2x(y+a)$ and so $ f(x)=x+a$ $ \forall x$

And it is easy to check back that these two functions indeed are solutions. Hence the two solutions :

$ f(x)=x$ $ \forall x$
$ f(x)=x+1$ $ \forall x$
\end{solution}



\begin{solution}[by \href{https://artofproblemsolving.com/community/user/25405}{AndrewTom}]
	Patrick, I am confused by your third paragraph. I thought you meant to write "Let then $ x<0$" instead of "Let then $ x\ge0$" to deal with the case where $ x$ is negative and the next line follows from that. But then you have $ x\ge0$ again in the third line of this third paragraph. I can't see that you've shown it for negative values of $ x$. Help!
\end{solution}



\begin{solution}[by \href{https://artofproblemsolving.com/community/user/29428}{pco}]
	\begin{tcolorbox}Patrick, I am confused by your third paragraph. I thought you meant to write "Let then $ x < 0$" instead of "Let then $ x\ge0$" to deal with the case where $ x$ is negative and the next line follows from that. But then you have $ x\ge0$ again in the third line of this third paragraph. I can't see that you've shown it for negative values of $ x$. Help!\end{tcolorbox}

At the end of second paragraph, I got $ f(x) = x + a$ $ \forall x\ge 0$

So, in the third paragraph, I considered $ x\ge 0$ so that $ f((x - y)^2) = (x - y)^2 + a$ (since $ (x - y)^2\ge 0)$ and $ f(x) = x + a$ (since $ x\ge 0$

Using these two values, I can rewite $ P(x,y)$ as $ (x - y)^2 + a = (x + a)^2 - 2xf(y) + y^2$ $ \forall x\ge 0$, $ \forall y$

And so $ 2x(y + a) = 2xf(y)$ (remember $ a^2 = a$)  $ \forall x\ge 0$, $ \forall y$ hence the result (choose $ x = 1$, for example) : $ f(y) = y + a$ $ \forall y$ (that I immediately rewrote as $ f(x)=x+a$ $ \forall x$)
\end{solution}



\begin{solution}[by \href{https://artofproblemsolving.com/community/user/25405}{AndrewTom}]
	Thanks Patrick, it's quite clear to me now.
\end{solution}
*******************************************************************************
-------------------------------------------------------------------------------

\begin{problem}[Posted by \href{https://artofproblemsolving.com/community/user/68719}{MJ GEO}]
	Find all $ f: \mathbb R\to\mathbb R$ such that \[f(x)+f(y)=f(f(x)f(y))\] for all $x,y\in\mathbb R$.
	\flushright \href{https://artofproblemsolving.com/community/c6h316340}{(Link to AoPS)}
\end{problem}



\begin{solution}[by \href{https://artofproblemsolving.com/community/user/29428}{pco}]
	\begin{tcolorbox}Find all $ f(x)$ from $ \mathbb R\to\mathbb R$ such that $ f(x)+f(y)=f(f(x)f(y))$ $ \forall x,y\in\mathbb R$\end{tcolorbox}

The equation may be written : Assertion $ P(x,y)$ : $ f(xy)=x+y$ $ \forall x,y\in f(\mathbb R)$

So, $ x,y\in f(\mathbb R)$ $ \implies$ $ x+y\in f(\mathbb R)$ and an immediate induction shows that $ nx\in f(\mathbb R)$ $ \forall x\in f(\mathbb R)$, $ \forall n\in\mathbb N$ (set of positive integers)

Then $ P(px,qy)$ $ \implies$ $ f(pqxy)=px+qy$
And $ P(pqx,y)$ $ \implies$ $ f(pqxy)=pqx+y$

And so $ pqx+y=px+qy$ $ \forall x,y\in f(\mathbb R)$, $ \forall p,q\in\mathbb N$ and so $ x=y=0$

So $ f(\mathbb R)=\{0\}$ and so the unique solution :

$ f(x)=0$ $ \forall x$ (which is indeed a solution)
\end{solution}
*******************************************************************************
-------------------------------------------------------------------------------

\begin{problem}[Posted by \href{https://artofproblemsolving.com/community/user/68719}{MJ GEO}]
	Find all $ f: \mathbb R\to\mathbb R$ such that \[ f(xy) = f(f(x)+f(y))\] for all $x,y\in\mathbb R$.
	\flushright \href{https://artofproblemsolving.com/community/c6h316343}{(Link to AoPS)}
\end{problem}



\begin{solution}[by \href{https://artofproblemsolving.com/community/user/29428}{pco}]
	\begin{tcolorbox}Find all functions $ f(x)$ from $ \mathbb R\to\mathbb R$ such that $ f(xy)=f(f(x)+f(y))$ $ \forall x,y$\end{tcolorbox}

Let $ P(x,y)$ be the assertion $ f(xy)=f(f(x)+f(y))$

1) $ \exists a\ne 0$ such that $ f(a)=f(0)$
==========================
if $ f(0)\ne 0$, then $ P(0,0)$ $ \implies$ $ f(0)=f(2f(0))$ and $ a=2f(0)\ne 0$ fullfills the requirement
If $ f(0)=0$, then let $ u\ne 0$
If $ f(u)=0$, then $ f(u)=f(0)$ and $ u\ne 0$ fullfills the requirement
If $ f(u)\ne 0$, $ P(u,0)$ $ \implies$ $ f(0)=f(f(u))$ and $ f(u)\ne 0$ fullfills the requirement
Q.E.D.

2) $ f(1)=f(0)$
===========
Let $ a\ne 0$ such that $ f(a)=f(0)$ (such a real exists, according to point 1. above)

$ P(a,\frac 1a)$ $ \implies$ $ f(1)=f(f(a)+f(\frac 1a))$

$ P(0,\frac 1a)$ $ \implies$ $ f(0)=f(f(0)+f(\frac 1a))$

And since $ f(a)=f(0)$, we get $ f(0)=f(1)$
Q.E.D.

3) $ f(x)=f(0)$ $ \forall x$
==============
$ P(0,x)$ $ \implies$ $ f(0)=f(f(0)+f(x))$
$ P(1,x)$ $ \implies$ $ f(x)=f(f(1)+f(x))$
And since $ f(0)=f(1)$, according to point 2. above, $ f(x)=f(0)$
Q.E.D

Hence the unique solution : $ \boxed{f(x)=c}$ constant, $ \forall x$
\end{solution}
*******************************************************************************
-------------------------------------------------------------------------------

\begin{problem}[Posted by \href{https://artofproblemsolving.com/community/user/68719}{MJ GEO}]
	Find all $ f: \mathbb R\to\mathbb R$ such that \[ f(x)f(yf(x)-1)=x^2f(y)-f(x)\] for all $x,y\in\mathbb R$.
	\flushright \href{https://artofproblemsolving.com/community/c6h316347}{(Link to AoPS)}
\end{problem}



\begin{solution}[by \href{https://artofproblemsolving.com/community/user/68719}{MJ GEO}]
	ANY BODY????? WHERE ARE YOU??????
\end{solution}



\begin{solution}[by \href{https://artofproblemsolving.com/community/user/29428}{pco}]
	\begin{tcolorbox}ANY BODY????? WHERE ARE YOU??????\end{tcolorbox}

Huh !!!!

You make an up on a topic after $ 23$ minutes ??????

Wait at least $ 4$ or $ 5$ days before first up.
 :rotfl:

And if you want some mathlinkers find interest in your posts :
1) use Latex
2) thank those who gave you hints \/ solution \/ attention
\end{solution}



\begin{solution}[by \href{https://artofproblemsolving.com/community/user/68719}{MJ GEO}]
	sorry you are righ.I ashamed :oops:
\end{solution}



\begin{solution}[by \href{https://artofproblemsolving.com/community/user/68719}{MJ GEO}]
	Please help me.I need the solution of this quastion today.its realy important for me.
\end{solution}



\begin{solution}[by \href{https://artofproblemsolving.com/community/user/29428}{pco}]
	\begin{tcolorbox}Please help me.I need the solution of this quastion today.its realy important for me.\end{tcolorbox}

You posted in "Algebra Proposed & own problems" so you are supposed to already have the solution.

If you dont have the solution, post, next time (dont post twice this one), in "unsolved problems"
\end{solution}



\begin{solution}[by \href{https://artofproblemsolving.com/community/user/29428}{pco}]
	\begin{tcolorbox}Find all functions $ f(x)$ from $ \mathbb R\to\mathbb R$ such that $ f(x)f(yf(x)-1)=x^2f(y)-f(x)$ $ \forall x,y$\end{tcolorbox}

Let $ P(x,y)$ be the assertion $ f(x)f(yf(x)-1)=x^2f(y)-f(x)$

$ P(1,1)$ $ \implies$ $ f(1)f(f(1)-1)=0$. Then :
If $ f(1)=0$ : $ P(1,x)$ $ \implies$ $ 0=f(x)$ and we got the solution $ f(x)=0$ $ \forall x$
If $ f(1)\ne 0$ and $ f(1)\ne 1$ then let $ a=f(1)-1\ne 0$. We got $ f(a)=0$ and so $ P(a,x)$ $ \implies$ $ 0=a^2f(x)$ and we got again the solution $ f(x)=0$ $ \forall x$

So we'll now consider $ f(1)=1$
$ P(1,x)$ $ \implies$ $ f(x-1)=f(x)-1$ and so $ f(n)=n$ $ \forall n\in\mathbb Z$

If $ f(u)=0$ then $ P(u,1)$ $ \implies$ $ u^2=0$  and so $ u=0$ 

Then $ f(yf(x)-1)=f(xf(x))-1$ and $ P(x,y)$ becomes : $ Q(x,y)$ : $ f(x)f(yf(x))=x^2f(y)$

$ Q(x,1)$ $ \implies$ $ f(x)f(f(x))=x^2$ and so $ f(f(x))=\frac {x^2}{f(x)}$ $ \forall x\ne 0$

Let $ x\ne 0$ then $ Q(x+1,1)$ $ \implies$ $ f(x+1)f(f(x+1))=(x+1)^2$ and so $ (f(x)+1)(f(f(x))+1)=(x+1)^2$ and so $ (f(x)+1)(\frac {x^2}{f(x)}+1)=(x+1)^2$

$ \implies$ $ (f(x)+1)(x^2+f(x))=f(x)(x^2+2x+1)$

$ \implies$ $ (x^2+1)f(x) + f(x)^2+x^2=f(x)(x^2+2x+1)$

$ \implies$ $ f(x)^2-2xf(x)+x^2=0$

$ \implies$ $ (f(x)-x)^2=0$ $ \implies$ $ f(x)=x$, which, indeed, is a solution

Hence the two solutions :
$ f(x)=0$ $ \forall x$
$ f(x)=x$ $ \forall x$
\end{solution}



\begin{solution}[by \href{https://artofproblemsolving.com/community/user/68719}{MJ GEO}]
	thank you for this nice and lovely solution 
\end{solution}
*******************************************************************************
-------------------------------------------------------------------------------

\begin{problem}[Posted by \href{https://artofproblemsolving.com/community/user/68719}{MJ GEO}]
	Let $ f(x)$ and $ g(x)$ be $ \mathbb Z\to\mathbb Z$ functions such that \[ f(m)+f(n)=g(m^2+n^2)\] and \[ -2\le f(n+2)-f(n)\le 2\] for all integers $m$ and $n$. What is the greatest possible cardinality of $ f(\mathbb Z$)?
	\flushright \href{https://artofproblemsolving.com/community/c6h316348}{(Link to AoPS)}
\end{problem}



\begin{solution}[by \href{https://artofproblemsolving.com/community/user/68719}{MJ GEO}]
	is that problem realy hard?  :(
\end{solution}



\begin{solution}[by \href{https://artofproblemsolving.com/community/user/29428}{pco}]
	\begin{tcolorbox}Let $ f(x)$ and $ g(x)$ funxtions from $ \mathbb Z\to\mathbb Z$ such that $ f(m)+f(n)=g(m^2+n^2)$ and $ -2\le f(n+2)-f(n)\le 2$.
What is the greatest cardinal possible for $ f(\mathbb Z$)? \end{tcolorbox}

I did not succed up to now to end this problem. Please, MJ GEO, could you give us your solution ?

What I did : at most $ 6$ elements. But I only succeed to find a three elements solutions (so it remains to show that this is the maximum or to find a solution with $ 4,5$ or $ 6$).

1) We have at most 6 elements :
$ f(ab+1)+f(a-b)=g(a^2b^2+a^2+b^2+1)$
$ f(ab-1)+f(a+b)=g(a^2b^2+a^2+b^2+1)$

So $ f(ab+1)-f(ab-1)=f(a+b)-f(a-b)$ and, since $ |LHS|\le 2$, we get that $ |f(a+b)-f(a-b)|\le 2$ and so $ |f(n+2p)-f(n)|\le 2$
So $ f(2p)$ can take at most $ 3$ values and $ f(2p+1)$ can also take at most $ 3$ values.

2) there exists solutions with $ 3$ elements :
$ f(4p)=0$
$ f(4p+2)=1$
$ f(2p+1)=3$

$ m^2+n^2\pmod{16}\in\{0,1,2,4,5,8,9,10,13\}$ and :

$ g(16p)=0$
$ g(16p+1)=3$
$ g(16p+2)=6$
$ g(16p+4)=1$
$ g(16p+5)=4$
$ g(16p+8)=2$
$ g(16p+9)=3$
$ g(16p+10)=6$
$ g(16p+13)=4$
\end{solution}



\begin{solution}[by \href{https://artofproblemsolving.com/community/user/68719}{MJ GEO}]
	i will post my solution 3 days later.
\end{solution}



\begin{solution}[by \href{https://artofproblemsolving.com/community/user/29428}{pco}]
	\begin{tcolorbox}i will post my solution 3 days later.\end{tcolorbox}

 :rotfl:
\end{solution}



\begin{solution}[by \href{https://artofproblemsolving.com/community/user/45762}{FelixD}]
	What's so funny, pco?^^
\end{solution}



\begin{solution}[by \href{https://artofproblemsolving.com/community/user/29428}{pco}]
	\begin{tcolorbox}What's so funny, pco?^^\end{tcolorbox}

MJ GEO posted many many problems for which he says he has solutions ($ 22$ this week-end in algebra subforum)
MJ GEO is in a great hurry to have alternative solutions for these problems (he posts a up on these problems 15-30 mn after no answer and begs many times for answers : "\begin{italicized}Please help me.I need the solution of this quastion today.its realy important for me\end{italicized}")
MJ GEO received answers to a lot of these problems (unfortunetly not all of them up to now) in 1\/2h-4h delay (I personaly gave solutions to 13-14 of them)

But when I ask a hint \/ solution for one of them, he answers : "3 days". 
So we must give him a second solution in 15 mn but he'll give us some help in 3 days.

I found this very funny...

... and begin to think that the class during which the solutions to his homework will be given is next wednesday ... :)
\end{solution}



\begin{solution}[by \href{https://artofproblemsolving.com/community/user/29428}{pco}]
	\begin{tcolorbox}i will post my solution 3 days later.\end{tcolorbox}
Please, could you post your solution now.

Thanks.
\end{solution}
*******************************************************************************
-------------------------------------------------------------------------------

\begin{problem}[Posted by \href{https://artofproblemsolving.com/community/user/68719}{MJ GEO}]
	Find all functions $ f,g : \mathbb R\to\mathbb R$ such that $ f(g(x))=x^3$ and $ g(f(x))=x^4$ for all real $x$.
	\flushright \href{https://artofproblemsolving.com/community/c6h316350}{(Link to AoPS)}
\end{problem}



\begin{solution}[by \href{https://artofproblemsolving.com/community/user/29428}{pco}]
	\begin{tcolorbox}Find all couples $ f(x),g(x)$ of functions from $ \mathbb R\to\mathbb R$ such that $ f(g(x))=x^3$ and $ g(f(x))=x^4$ $ \forall x$\end{tcolorbox}

From $ f(g(x))=x^3$, we get that $ g(x)$ is injective and that $ f(x)$ is surjective
From $ g(f(x))=x^4$ and $ f(x)$ surjective, we get that $ g(\mathbb R)=\mathbb R^+_0$
From $ g(f(x))=x^4$ and $ g(f(-x))=x^4$ and $ g(x)$ injective, we get that $ f(x)=f(-x)$ $ \forall x$
From $ g(f(x))=x^4$, we get that $ f(a)=f(b)$ $ \implies$ $ a^4=b^4$ and so $ a=b$ or $ a=-b$

So $ g(x)$ is a bijection from $ \mathbb R\to\mathbb R^+_0$
And the restriction $ f_r(x)$ of $ f(x)$ to $ \mathbb R^+_0$ is a bijection from $ \mathbb R^+_0\to\mathbb R$

Let then $ h(x)$ from $ \mathbb R\to\mathbb R^+_0$ the inverse of $ f_r(x)$ (so a bijection too)

From $ f_r(g(x))=x^3$, we get that $ g(x)=h(x^3)$
From $ g(f_r(x))=x^4$, we get that $ g(x)=h(x)^4$

So $ h(x^3)=h(x)^4$ $ \forall x\in\mathbb R$ and so :
$ h(-1)=h(-1)^4$ and $ h(-1)\in\{0,1\}$ (remember that $ h(\mathbb R)=\mathbb R^+_0$)
$ h(0)=h(0)^4$ and $ h(0)\in\{0,1\}$ (remember that $ h(\mathbb R)=\mathbb R^+_0$)
$ h(1)=h(1)^4$ and $ h(1)\in\{0,1\}$ (remember that $ h(\mathbb R)=\mathbb R^+_0$)

And so at least two of the three values $ h(-1),h(0)$ and $ h(1)$ must be equal but this is impossible since $ h(x)$ is a bijection.

So no solution to this equation.
\end{solution}



\begin{solution}[by \href{https://artofproblemsolving.com/community/user/68719}{MJ GEO}]
	thank you so much nice solution  
\end{solution}
*******************************************************************************
-------------------------------------------------------------------------------

\begin{problem}[Posted by \href{https://artofproblemsolving.com/community/user/68719}{MJ GEO}]
	Find all $ f: \mathbb R\to\mathbb R$ such that \[f(x^2-y^2)=xf(x)-yf(y)\] for all $x,y\in\mathbb R$.
	\flushright \href{https://artofproblemsolving.com/community/c6h316395}{(Link to AoPS)}
\end{problem}



\begin{solution}[by \href{https://artofproblemsolving.com/community/user/29428}{pco}]
	\begin{tcolorbox}Find all functions $ f(x)$ from $ \mathbb R\to\mathbb R$ such that $ f(x^2-y^2)=xf(x)-yf(y)$ \end{tcolorbox}

\begin{bolded}Please \end{bolded}\end{underlined}:
1) use Latex
2) post in "unsolved"
3) use lowercase (uppercase means shouting)

Let $ P(x,y)$ be the assertion $ f(x^2-y^2)=xf(x)-yf(y)$
$ P(x,x)$ $ \implies$ $ f(0)=0$
$ P(x,0)$ $ \implies$ $ f(x^2)=xf(x)$ and so $ P(x,y)$ becomes $ f(x^2-y^2)=f(x^2)-f(y^2)$

So $ f(x-y)=f(x)-f(y)$ $ \forall x,y\ge 0$
So $ f(0-y)=f(0)-f(y)$ and so $ f(-x)=-f(x)$ 
So $ f(x+y)=f(x)+f(y)$ $ \forall x,y$

Using then $ f(x^2)=xf(x)$ with $ x+y$, we get $ f((x+y)^2)=(x+y)f(x+y)$ and so $ f(x^2)+2f(xy)+f(y^2)=xf(x)+yf(y)+xf(y)+yf(x)$

So $ 2f(xy)=xf(y)+yf(x)$

Setting $ y=1$ in this equation, we get $ 2f(x)=xf(1)+f(x)$ and so $ f(x)=f(1)x$ and it's easy to check back that this mandatory form is sufficient.

Hence the answer : $ f(x)=ax$ $ \forall x$
\end{solution}



\begin{solution}[by \href{https://artofproblemsolving.com/community/user/68719}{MJ GEO}]
	1)I cant use LATX
2)i solved this problem
\end{solution}



\begin{solution}[by \href{https://artofproblemsolving.com/community/user/29428}{pco}]
	\begin{tcolorbox}1)I cant use LATX
2)i solved this problem\end{tcolorbox}

1) : look at others posts : just enclose your formulas between dollars sign ... everyone can do this.
2) : Ok, sorry.
3) : thanks
 
\end{solution}



\begin{solution}[by \href{https://artofproblemsolving.com/community/user/68719}{MJ GEO}]
	how can i write F from R to R in latex
\end{solution}



\begin{solution}[by \href{https://artofproblemsolving.com/community/user/29428}{pco}]
	\begin{tcolorbox}how can i write F from R to R in latex\end{tcolorbox}

Let <dollar>f(x)<dollar> from R to R :  
Let $ f(x)$ from R to R  

Let <dollar>f(x)<dollar> from <dollar>R\to R<dollar> : 
Let $ f(x)$ from $ R\to R$

Let <dollar>f(x)<dollar> from <dollar>\mathbb R\to\mathbb R<dollar> : 
Let $ f(x)$ from $ \mathbb R\to\mathbb R$

Let <dollar>f(x)=x^2+\frac{x+1}{x^3-2}<dollar> :
Let $ f(x)=x^2+\frac{x+1}{x^3-2}$

Just put cursor over images and look at the popup (browse the forum to find a lot of examples)
\end{solution}



\begin{solution}[by \href{https://artofproblemsolving.com/community/user/68719}{MJ GEO}]
	thanks too much 
\end{solution}
*******************************************************************************
-------------------------------------------------------------------------------

\begin{problem}[Posted by \href{https://artofproblemsolving.com/community/user/58583}{b555}]
	Find all functions $f: \mathbb Z \to \mathbb Z$ which satisfy \[f(m+ f(n)) = f(m) + n\] for all integers $m$ and $n$.
	\flushright \href{https://artofproblemsolving.com/community/c6h316414}{(Link to AoPS)}
\end{problem}



\begin{solution}[by \href{https://artofproblemsolving.com/community/user/40922}{mehdi cherif}]
	for $ m = 0$  $ f(f(n)) = n + f(0)$ so $ f$ is bijective

$ n = 0$ $ \implies f(m + f(0)) = f(m) \implies f(0) = 0$

replace in the original equation $ n$ by $ f(n)$  , $ f(m + n) = f(m) + f(n)$
wich is cauchy equation so $ f(n) = an \forall n\in\mathbb{Z}$ with $ a\in { - 1,1}$
\end{solution}



\begin{solution}[by \href{https://artofproblemsolving.com/community/user/29428}{pco}]
	\begin{tcolorbox}Find all functions f : Z --> Z which satisfy f(m+ f(n)) = f(m) + n for all m, n belong to Z.\end{tcolorbox}

$ m=0$ $ \implies$ $ f(f(n))=n+f(0)$
$ n=f(p)$ $ \implies$ $ f(m+p+f(0))=f(m)+f(p)$
$ m=x-f(0)$ an $ p=y-f(0)$ $ \implies$ $ f(x+y-f(0))=f(x-f(0))+f(y-f(0))$

Let $ g(x)=f(x-f(0))$. this last equation becomes $ g(x+y)=g(x)+g(y)$ and so $ g(x)=ax$ and s $ f(x)=ax+b$

Plugging this in the original equation, we get $ a(m+an+b)+b=am+b+n$ and so $ a=\pm 1$ and $ b=0$

Hence the two solutions :
$ f(n)=n$
$ f(n)=-n$
\end{solution}
*******************************************************************************
-------------------------------------------------------------------------------

\begin{problem}[Posted by \href{https://artofproblemsolving.com/community/user/68719}{MJ GEO}]
	Find all functions $f: \mathbb N \to \mathbb N$ such that for every positive integer $n$,
\[ f(f(n)) + f(n) = 2n+2001 \quad \text{or} \quad 2n + 2002.\]
	\flushright \href{https://artofproblemsolving.com/community/c6h316526}{(Link to AoPS)}
\end{problem}



\begin{solution}[by \href{https://artofproblemsolving.com/community/user/29428}{pco}]
	\begin{tcolorbox}$ f$ from $ N$ to $ n$ and $ f(f(n)) + f(n) = 2n + 2001 or 2n + 2002$\end{tcolorbox}

Suppose we have $ ax + b\le f(x)\le cx + d$ $ \forall x\in\mathbb N$. Then :

$ af(x) + b\le f(f(x))\le cf(x) + d$

$ (a + 1)f(x) + b\le f(f(x)) + f(x)\le (c + 1)f(x) + d$

Then, $ f(f(x)) + f(x)\le 2x + 2002$ $ \implies$ $ (a + 1)f(x) + b\le 2x + 2002$ $ \implies$ $ f(x)\le \frac {2}{a + 1}x + \frac {2002 - b}{a + 1}$

Same, $ f(f(x)) + f(x)\ge 2x + 2001$ $ \implies$ $ (c + 1)f(x) + d\ge 2x + 2001$ $ \implies$ $ f(x)\ge \frac {2}{c + 1}x + \frac {2001 - d}{c + 1}$

So $ ax + b\le f(x)\le cx + d$ $ \forall x$ $ \implies$ $ \frac {2}{c + 1}x + \frac {2001 - d}{c + 1}$ $ \le f(x)\le$ $ \frac {2}{a + 1}x + \frac {2002 - b}{a + 1}$ $ \forall x$

And since $ 1\le f(x)\le 2x + 2001$, we can build four sequences $ a_n,b_n,c_n$ and $ d_n$ :
$ a_1 = 0$, $ b_1 = 1$, $ c_1 = 2$ and $ d_1 = 2001$

$ a_{n + 1} = \frac {2}{c_n + 1}$

$ b_{n + 1} = \frac {2001 - d_n}{c_n + 1}$

$ c_{n + 1} = \frac {2}{a_n + 1}$

$ d_{n + 1} = \frac {2002 - b_n}{a_n + 1}$

And $ a_nx + b_n\le f(x)\le c_nx + d_n$ $ \forall x\in\mathbb N$

It's easy to show that $ \lim_{n\to + \infty}a_n = 1$, $ \lim_{n\to + \infty}b_n = 666 + \frac 23$, $ \lim_{n\to + \infty}c_n = 1$ and $ \lim_{n\to + \infty}d_n = 667 + \frac 23$

So $ x + 666 + \frac 23\le f(x)\le x + 667 + \frac 23$ $ \forall x\in\mathbb N$

And so the unique possibility, which indeed is a solution : $ \boxed{f(n) = n + 667}$ $ \forall n\in\mathbb N$
\end{solution}



\begin{solution}[by \href{https://artofproblemsolving.com/community/user/62562}{shoki}]
	see Pen K7
\end{solution}
*******************************************************************************
-------------------------------------------------------------------------------

\begin{problem}[Posted by \href{https://artofproblemsolving.com/community/user/68719}{MJ GEO}]
	Find all functions $f: \mathbb N \to \mathbb N$ for which
\[ f(f(m) + f(n)) = m + n\]
holds for all positive integers $m$ and $n$.
	\flushright \href{https://artofproblemsolving.com/community/c6h316531}{(Link to AoPS)}
\end{problem}



\begin{solution}[by \href{https://artofproblemsolving.com/community/user/44083}{jgnr}]
	[hide]Fix $ m$, and let $ a,b\in\mathbb{N}$ such that $ f(a)=f(b)$. We have $ f(f(m)+f(a))=f(f(m)+f(b))$, so $ m+a=m+b$ and $ a=b$. Thus $ f$ is injective. We have $ f(f(m+1)+f(n-1))=m+1+n-1=m+n=f(f(m)+f(n))$. Therefore $ f(m+1)+f(n-1)=f(m)+f(n)$ or $ f(m+1)-f(m)=f(n)-f(n-1)$. Substitute $ m=x,n-1=y$, we get $ f(x+1)-f(x)=f(y+1)-f(y)$ for all positive integers $ x,y$. Thus there exists a constant $ k$ such that $ f(x+1)-f(x)=k$, or $ f(x+1)=f(x)+k$. Clearly $ k>0$ (otherwise $ f(x)<0$ for some $ x$ or $ f$ is constant which are both impossible).

$ f(x)=f(1)+(x-1)k$.

$ 3=f(f(2)+f(1))=f(2f(1)+k)=f(1)+(2f(1)+k-1)k=(2k+1)f(1)+k^2-k\ge 2k+1+k^2-k=k^2+k+1$

So $ k=1$.

$ m+n=f(f(m)+f(n))=f(2f(1)+m+n-2)=3f(1)+m+n-3$. Hence $ f(1)=1$. We get $ f(x)=1+(x-1)1=x$.

Answer: $ \boxed{f(x)=x,\forall x\in\mathbb{N}}$[\/hide]
\end{solution}



\begin{solution}[by \href{https://artofproblemsolving.com/community/user/29428}{pco}]
	\begin{tcolorbox}Fix $ m$, and let $ a,b\in\mathbb{N}$ such that $ f(a) = f(b)$. We have $ f(f(m) + f(a)) = f(f(m) + f(b))$, so $ m + a = m + b$ and $ a = b$. Thus $ f$ is injective. We have $ f(f(m + 1) + f(n - 1)) = m + 1 + n - 1 = m + n = f(f(m) + f(n))$. Therefore $ f(m + 1) + f(n - 1) = f(m) + f(n)$ or $ f(m + 1) - f(m) = f(n) - f(n - 1)$. Substitute $ m = x,n - 1 = y$, we get $ f(x + 1) - f(x) = f(y + 1) - f(y)$ for all positive integers $ x,y$. Thus there exists a constant $ k$ such that $ f(x + 1) - f(x) = k$, or $ f(x + 1) = f(x) + k$. Clearly $ k > 0$ (otherwise $ f(x) < 0$ for some $ x$ or $ f$ is constant which are both impossible).

$ f(x) = f(1) + (x - 1)k$.

$ 3 = f(f(2) + f(1)) = f(2f(1) + k) = f(1) + (2f(1) + k - 1)k = (2k + 1)f(1) + k^2 - k\ge 2k + 1 + k^2 - k = k^2 + k + 1$

So $ k = 1$.

$ m + n = f(f(m) + f(n)) = f(2f(1) + m + n - 2) = 3f(1) + m + n - 3$. Hence $ f(1) = 1$. We get $ f(x) = 1 + (x - 1)1 = x$.

Answer: $ \boxed{f(x) = x,\forall x\in\mathbb{N}}$\end{tcolorbox}

Quite OK, but you should be a little bit quicker in the end : from $ f(x) = f(1) + (x - 1)k$, you get $ f(x) = ax + b$ and you just have to plug back in the original equation to obtain $ a^2 = 1$ and $ b = 0$ and so $ f(x) = x$
\end{solution}
*******************************************************************************
-------------------------------------------------------------------------------

\begin{problem}[Posted by \href{https://artofproblemsolving.com/community/user/68719}{MJ GEO}]
	Find all functions $f: \mathbb N \to \mathbb N$ for which 
\[ f(n) + f(n+1) = f(n+2)f(n+3)-1996\]
holds for all positive integers $n$.
	\flushright \href{https://artofproblemsolving.com/community/c6h316532}{(Link to AoPS)}
\end{problem}



\begin{solution}[by \href{https://artofproblemsolving.com/community/user/29428}{pco}]
	\begin{tcolorbox}$ f$ from $ N$ to $ N$ and $ f(n) + f(n + 1) = f(n + 2)f(n + 3) - 1996$\end{tcolorbox}

$ f(n) + f(n + 1) = f(n + 2)f(n + 3) - 1996$
$ f(n+1) + f(n + 2) = f(n + 3)f(n + 4) - 1996$

Subtracting : $ f(n)-f(n+2)=f(n+3)((f(n+2)-f(n+4)$ and so :

$ f(1)-f(3)=f(4)f(6)f(8)...f(2p+2)(f(2p+1)-f(2p+3))$ $ \forall p\ge 1$
$ f(2)-f(4)=f(5)f(7)f(9)...f(2p+3)(f(2p+2)-f(2p+4))$ $ \forall p\ge 1$

Considering the first line, we conclude that :
either $ f(2p+1)=f(2p+3)$ from a given point and then $ f(2p+1)=c$ $ \forall p>p_0$
either $ f(2p+2)=1$ and $ f(2p+1)-f(2p+3)=c$ from a given point 

Combining with the same analysis for second line, we conclude that, from a given point, we have :
either $ a,b,a,b,a,b,....$
either $ 1,k,1,k+c,1,k+2c,1,k+3c, ...$

Using then the original equation, we can write :
$ a+b=ab-1996$ and so $ (a-1)(b-1)=1997$ and so $ a=2$ and $ b=1998$ (or the reverse)
$ 1+k=1(k+c)-1996$ and so $ c=1997$

So the function ends with :
either $ 2,1998,2,1998,2,1998,...$
either $ 1,k,1,k+1997,1,k+2\cdot 1997, ...$

It's then immediate to show that if this is true from a given point, it's true from the beginning. Hence the four solutions :

$ \{f(1),f(2),f(3),...\}=\{2,1998,2,1998,2,1998, ...\}$
$ \{f(1),f(2),f(3),...\}=\{1998,2,1998,2,1998,2 ...\}$
$ \{f(1),f(2),f(3),...\}=\{1,a,1,a+1997,1,a+2\cdot 1997,1,a+3\cdot 1997, ...\}$
$ \{f(1),f(2),f(3),...\}=\{a,1,a+1997,1,a+2\cdot 1997,1,a+3\cdot 1997,1 ...\}$
\end{solution}



\begin{solution}[by \href{https://artofproblemsolving.com/community/user/68719}{MJ GEO}]
	thanks.did you think on my polynomials problems?
\end{solution}



\begin{solution}[by \href{https://artofproblemsolving.com/community/user/29428}{pco}]
	\begin{tcolorbox}thanks.did you think on my polynomials problems?\end{tcolorbox}

Since you already have solutions, we are in no hurry to solve these problems ...
\end{solution}



\begin{solution}[by \href{https://artofproblemsolving.com/community/user/68719}{MJ GEO}]
	yes.you are right no hurry :)
\end{solution}
*******************************************************************************
-------------------------------------------------------------------------------

\begin{problem}[Posted by \href{https://artofproblemsolving.com/community/user/46039}{ll931110}]
	Determine all functions $f: \mathbb R \to \mathbb R$ satisfying
\[ f(x^2 + f(y)) = y + f^2(x), \quad \forall x,y \in \mathbb R.\]
	\flushright \href{https://artofproblemsolving.com/community/c6h316533}{(Link to AoPS)}
\end{problem}



\begin{solution}[by \href{https://artofproblemsolving.com/community/user/66394}{reason}]
	$ R$ means $ \mathbb{R^{*}}$ ?
\end{solution}



\begin{solution}[by \href{https://artofproblemsolving.com/community/user/46039}{ll931110}]
	\begin{tcolorbox}$ R$ means $ \mathbb{R^{*}}$ ?\end{tcolorbox}

$ R$ means the set of real number (including 0)
And finally, I have solved it by myself  .
\end{solution}



\begin{solution}[by \href{https://artofproblemsolving.com/community/user/29428}{pco}]
	\begin{tcolorbox}I think this functional equation is not difficult, but I haven't solved it yet  :mad: 

\begin{italicized}Determine all functions $ f: R \rightarrow R$ satisfying
$ f(x^2 + f(y)) = y + f^2(x)$\end{italicized}\end{tcolorbox}

Let $ P(x,y)$ be the assertion $ f(x^2+f(y))=y+f(x)^2$
Let $ f(0)=a$

$ P(0,x)$ $ \implies$ $ f(f(x))=x+a^2$ and so $ f(x)$ is bijective

Let then $ u$ such that $ f(u)=0$
$ P(u,u)$ $ \implies$ $ f(u^2)=u$ and so $ f(f(u^2))=0$ But $ P(0,u^2)$ $ \implies$ $ f(f(u^2))=u^2+a^2$ so $ u^2+a^2=0$ and $ u=a=0$

$ P(x,0)$ $ \implies$ $ f(x^2)=f(x)^2$ and so $ P(x,y)$ may be rewritten as $ Q(x,y)$ : $ f(x^2+f(y))=y+f(x^2)$

$ Q(x,f(y))$ $ \implies$ $ f(x^2+y)=f(y)+f(x^2)$ and so $ f(x+y)=f(x)+f(y)$ $ \forall x\ge 0,\forall y$

It's then immediate to find $ f(-x)=-f(x)$ and so $ f(x+y)=f(x)+f(y)$ $ \forall x,y$
And since $ f(x^2)=f(x)^2$, we get that $ f(x)\ge 0$ $ \forall x\ge 0$ 

So solutions are solutions of Cauchy's equation and have a lower bound on $ \mathbb R^+$. So $ f(x)$ is continuous and is $ cx$

Plugging back in the original equation, we get $ c=1$ and the unique solution $ \boxed{f(x)=x}$ $ \forall x$
\end{solution}



\begin{solution}[by \href{https://artofproblemsolving.com/community/user/66394}{reason}]
	there is my solution:


$ P(x,y): f(x^{2} + f(y)) = y + (f(x))^{2}$

$ P(0,y)\Rightarrow f(f(y)) = y + (f(0))^{2}\Rightarrow$ $ f$ is bijective.

we have $ f(x^{2} + f(y)) = f(( - x)^{2} + f(y)) \Rightarrow (f(x))^{2} = (f( - x))^{2}$
 
since f is injective $ \Rightarrow$ $ f(x) = - f( - x)$ ($ f$ is odd).

we have $ f(0) = 0$ so $ f(f(y)) = y$ and $ f(x^{2}) = (f(x))^{2} > 0$

so $ f(\mathbb{R}) > 0$ $ \forall x\in\mathbb{R^{*}}$.

Let $ u = x^{2}$ and $ v = f(y)$ so $ f(u + v) = f(u) + f(v)$.

$ \Rightarrow$ $ f(x) = x$ $ \forall x\in{\mathbb{Q + }}$.

since $ f(x) > 0$ we have $ f(u + v) = f(u) + f(v)\ge f(u)$ $ \Rightarrow$ $ f$ is increasing in $ \mathbb{R + }$. so since $ f$ is odd we conclude finally that: $ f(x) = x$ $ \forall x\in\mathbb{R}$. :)
\end{solution}
*******************************************************************************
-------------------------------------------------------------------------------

\begin{problem}[Posted by \href{https://artofproblemsolving.com/community/user/68719}{MJ GEO}]
	Find all functions $f: \mathbb Z \to \mathbb Z$ for which
\[ f(m+f(f(n))=-f(f(m+1))-n\]
holds for all integers $m$ and $n$.
	\flushright \href{https://artofproblemsolving.com/community/c6h316534}{(Link to AoPS)}
\end{problem}



\begin{solution}[by \href{https://artofproblemsolving.com/community/user/29428}{pco}]
	\begin{tcolorbox}$ f$ from $ Z$ to $ Z$ and $ f(m + f(f(n)) = - f(f(m + 1)) - n$\end{tcolorbox}

Let $ P(x,y)$ be the assertion $ f(x+f(f(y))=-f(f(x+1))-y$

$ P(0,x)$ $ \implies$ $ f(f(f(x)))=-x-f(f(1))$ and so $ f(x)$ is a bijection and so is $ f(f(x))$ and $ \exists a$ such that $ f(f(a))=1$

Let $ x\in\mathbb Z$. Since $ f(x)$ is a bijection, $ \exists y$ such that $ f(y)=x$. Then : $ P(y-1,a)$ $ \implies$ $ x=-f(x)-a$ and so $ f(x)=-x-a$

Plugging this back in the original equation, we get $ a=1$ and the unique solution : $ \boxed{f(n)=-(n+1)}$ $ \forall n\in\mathbb Z$
\end{solution}



\begin{solution}[by \href{https://artofproblemsolving.com/community/user/68719}{MJ GEO}]
	why there is and a that $ f(f(a))=1$
\end{solution}



\begin{solution}[by \href{https://artofproblemsolving.com/community/user/29428}{pco}]
	\begin{tcolorbox}why there is and a that $ f(f(a)) = 1$\end{tcolorbox}

Because $ f(x)$ is a bijection, and so $ f(f(x))$ is also a bijection, and so is a surjection.
\end{solution}



\begin{solution}[by \href{https://artofproblemsolving.com/community/user/68719}{MJ GEO}]
	now i confused becasue of geometry problem and cant understand you
\end{solution}



\begin{solution}[by \href{https://artofproblemsolving.com/community/user/29428}{pco}]
	\begin{tcolorbox}now i confused becasue of geometry problem and cant understand you\end{tcolorbox}
Let $ P(x,y)$ be the assertion $ f(x + f(f(y)) = - f(f(x + 1)) - y$

In a more detailed way :

$ P(0,x)$ $ \implies$ $ f(f(f(x))) = - x - f(f(1))$  So :
1) $ f(x)$ is injective:
$ f(a)=f(b)$ $ \implies$ $ f(f(f(a)))=f(f(f(b)))$ $ \implies$ $ - a - f(f(1))=- b - f(f(1))$ $ \implies$ $ a=b$ and so $ f(x)$ is injective.

2) $ f(x)$ is surjective :
Using $ x=-u-f(f(1))$ in $ f(f(f(x))) = - x - f(f(1))$  , we get $ f(f(f(-u-f(f(f1))))) = u$ and so $ \forall u$ $ \exists v=f(f(-u-f(f(f1))))$ such that $ f(v)=u$ and so $ f(x)$ is a surjection


and so $ f(x)$ is a bijection
Then $ f(x)$ bijection $ \implies$ $ f(f(x))$ is too a bijection.

Since $ f(f(x))$ is a bijection, it's a surjective function and so $ \exists a$ such that $ f(f(a)) = 1$

Is it OK ?

And the end now :
Let $ x\in\mathbb Z$. Since $ f(x)$ is a bijection, $ \exists y$ such that $ f(y) = x$. Then : $ P(y - 1,a)$ $ \implies$ $ x = - f(x) - a$ and so $ f(x) = - x - a$

Plugging this back in the original equation, we get $ a = 1$ and the unique solution : $ \boxed{f(n) = - (n + 1)}$ $ \forall n\in\mathbb Z$


What was your own solution ?
\end{solution}
*******************************************************************************
-------------------------------------------------------------------------------

\begin{problem}[Posted by \href{https://artofproblemsolving.com/community/user/68719}{MJ GEO}]
	Find all functions $f: \mathbb R \to \mathbb R$ such that for any real $x$ and $y$, 
\[ f(x+y)=f(x)+f(y)\]
and
\[f(x)=x^2f\left( \frac 1x \right).\]
	\flushright \href{https://artofproblemsolving.com/community/c6h316568}{(Link to AoPS)}
\end{problem}



\begin{solution}[by \href{https://artofproblemsolving.com/community/user/29428}{pco}]
	\begin{tcolorbox}$ f$ from $ R$ to $ R$,$ f(x + y) = f(x) + f(y),f(x) = x^2f(\frac 1x)$\end{tcolorbox}

$ f(1)=f(\frac{x}{x+1}+\frac 1{x+1})$ $ =f(\frac{x}{x+1})+f(\frac 1{x+1})$ $ =\frac{x^2}{(x+1)^2}f(\frac{x+1}{x})$ $ +\frac 1{(x+1)^2}f(x+1)$ 

So $ f(1)(x+1)^2=x^2f(1+\frac 1x)+f(x+1)$ $ =x^2f(1)+x^2f(\frac 1x)+f(x)+f(1)$ $ =x^2f(1)+2f(x)+f(1)$

So $ f(x)=f(1)x$ $ \forall x$ and it's immediate to verify back that this necessary form is sufficient.

Hence the solution : $ \boxed{f(x)=ax}$ $ \forall x$
\end{solution}
*******************************************************************************
-------------------------------------------------------------------------------

\begin{problem}[Posted by \href{https://artofproblemsolving.com/community/user/39373}{M.A}]
	Find all functions $f: \mathbb R \to \mathbb R$ such that
\[ f(2x)+f(1-x)=5x+9\]
holds for all real $x$.
	\flushright \href{https://artofproblemsolving.com/community/c6h317255}{(Link to AoPS)}
\end{problem}



\begin{solution}[by \href{https://artofproblemsolving.com/community/user/29428}{pco}]
	\begin{tcolorbox}Find all the functions $ f : R \to R$  such that:
\[ f(2x) + f(1 - x) = 5x + 9\]
\end{tcolorbox}

A trivial solution is $ f(x)=5x+2$

All other solutions are $ 5x+2+g(x)$ where $ g(x)$ is such that $ g(2x)+g(1-x)=0$

Let then $ h(x)=g(x+\frac 23)$ . we get $ h(2(x-\frac 13))+h(\frac 13-x)=0$ and so $ h(2x)=-h(-x)$

This implies $ h(4x)=h(x)$ and so we get a general solution for $ h(x)$ :

Let $ u(x)$ any function defined on $ [1,4)$. h(x) is defined as :
$ \forall x>0$ : $ h(x)=u(4^{\{\log_4(x)\}})$
$ h(0)=0$
$ \forall x<0$ $ h(x)=-u(4^{\{\log_4(-2x)\}})$

Which may also be written, for $ x\ne 0$ : $ h(x)=s(x)u(4^{\{\log_4(\frac{3-s(x)}2|x|)\}})$ where $ s(x)=$ sign of $ x$ ($ \pm 1$)

Hence the general solution for the required equation :

Let $ u(x)$ any function defined on $ [1,4)$. f(x) is defined as :

If $ x=\frac 23$ : $ f(x)=\frac{16}3$

If $ x\ne\frac 23$ : $ f(x)=5x+2+s(x-\frac 23)u(4^{\{\log_4(\frac{3-s(x-\frac 23)}2|x-\frac 23|)\}})$ where $ s(x)=$ sign of $ x$ ($ \pm 1$)
\end{solution}



\begin{solution}[by \href{https://artofproblemsolving.com/community/user/48278}{Dimitris X}]
	Sorry for the spam but if it is an ''easy function'' why did you post it in \begin{bolded}unsolved\end{bolded} senction M.A???
\end{solution}



\begin{solution}[by \href{https://artofproblemsolving.com/community/user/39373}{M.A}]
	It just looks easy my friend  :)
\end{solution}
*******************************************************************************
-------------------------------------------------------------------------------

\begin{problem}[Posted by \href{https://artofproblemsolving.com/community/user/58355}{NikolayKaz}]
	Find all functions $ f: \mathbb N_0 \to \mathbb R$ that satisfy the following condition for all non-negative integers $x$ and $y$:
\[f(x+y)+f(x-y)=f(3x).\]
	\flushright \href{https://artofproblemsolving.com/community/c6h317326}{(Link to AoPS)}
\end{problem}



\begin{solution}[by \href{https://artofproblemsolving.com/community/user/12955}{spanferkel}]
	\begin{tcolorbox}Find all function $ f: N_0 \Rightarrow R$ that satisfy the following condition

$ f(x + y) + f(x - y) = f(3x)$

[hide]
After few easy steps I got that $ f(a) + f(b) = f(a + b)$ , it's cauchy's equation so i get the only family of solutions is f(x)=cx. Substituting this gives me f(x)=0 but i'm not sure if i did it right

[\/hide]\end{tcolorbox}

$ f(3) + f(1) = f(2) + f(2)\qquad = f(6)$ 
$ f(2) + f(0) = f(3)$ 
$ f(4) + f(2) = f(5) + f(1)\qquad = f(9)$ 
$ .\qquad\ \ \ f(6) = f(4) + f(0)$ 

Adding it all up yields $ f(6) = f(5)$.  Further,

$ f(6) + f(2)\ = f(5) + f(3) = f(12) \Longrightarrow$ $ f(2) = f(3)\Longrightarrow f(0) = 0$ 
$ f(6) + f(0)\ = f(5) + f(1) = f(9) \Longrightarrow$ $ f(0) = f(1) = 0$ 


Now, from the first two above we get $ f(1) = f(2)$. So we have $ f(1) + f(2k + 1) = f(2) + f(2k) = f(3) + f(2k - 1)$ for $ k\ge1$, i.e.$ f(x) = 0\forall x\in\mathbb N_0$  is the only solution :(
\end{solution}



\begin{solution}[by \href{https://artofproblemsolving.com/community/user/29428}{pco}]
	\begin{tcolorbox}Find all function $ f: N_0 \Rightarrow R$ that satisfy the following condition

$ f(x + y) + f(x - y) = f(3x)$

[hide]
After few easy steps I got that $ f(a) + f(b) = f(a + b)$ , it's cauchy's equation so i get the only family of solutions is f(x)=cx. Substituting this gives me f(x)=0 but i'm not sure if i did it right

[\/hide]\end{tcolorbox}

$ y=0$ $ \implies$ $ f(3x)=2f(x)$ and so $ f(0)=0$
$ y=x$ $ \implies$ $ f(3x)=f(2x)$ and so $ f(2x)=2f(x)$
So $ f(6x)=4f(x)$
Using then $ x=2y$ in the original equation, we get $ f(3y)+f(y)=f(6y)$ and so $ 2f(y)+f(y)=4f(y)$ and so $ f(y)=0$ 

And the unique solution is $ f(x)=0$ $ \forall x$
\end{solution}
*******************************************************************************
-------------------------------------------------------------------------------

\begin{problem}[Posted by \href{https://artofproblemsolving.com/community/user/29876}{ozgurkircak}]
	Find all functions $f: \mathbb R \to \mathbb R$ such that
\[ f(x - f(y)) = 4f(x) - f(y) - 4x\] for all $ x,y \in \mathbb{R}.$
	\flushright \href{https://artofproblemsolving.com/community/c6h317348}{(Link to AoPS)}
\end{problem}



\begin{solution}[by \href{https://artofproblemsolving.com/community/user/73589}{mathmen}]
	$ f(x)=\frac{4}{3}x$
\end{solution}



\begin{solution}[by \href{https://artofproblemsolving.com/community/user/29428}{pco}]
	\begin{tcolorbox}Find all $ f: R\rightarrow R$ such that:
\[ f(x - f(y)) = 4f(x) - f(y) - 4x\]
for all $ x,y \in \mathbb{R}.$\end{tcolorbox}

Let $ P(x,y)$ be the assertion $ f(x - f(y)) = 4f(x) - f(y) - 4x$
Let $ I = f(\mathbb R)$

$ P(f(x),x)$ $ \implies$ $ f(0) = 4f(f(x)) - 5f(x)$ and so $ 4f(f((x)) = 5f(x) + f(0)$
$ P(f(x),y)$ $ \implies$ $ f(f(x) - f(y)) = 4f(f(x)) - f(y) - 4f(x)$ and so, using previous line : $ f(f(x) - f(y)) = f(x) - f(y) + f(0)$
$ \implies$ $ f(x) - f(y) + f(0)\in I$ $ \forall x,y$
$ \implies$ $ f(x) - (f(y) - f(z) + f(0)) + f(0) = f(x) - f(y) + f(z)\in I$ $ \forall x,y,z$
$ \implies$ $ (f(x) - f(y) + f(x)) - (4f(x) - f(y) - 4x) + (f(x) - f(y) + f(x)) = 4x - f(y)\in I$ $ \forall x,y$
$ \implies$ $ 0\in I$ (just set $ x = \frac {f(y)}4$ in previous line).

Using then $ y$ such that $ f(y) = 0$ in $ P(x,y)$, we get $ f(x) = 4f(x) - 4x$ and so $ f(x) = \frac {4x}3$ $ \forall x$

Then, plugging this back in original equation, we get that this necessary condition does not fit the equation.

Hence $ \boxed{\text{no solution for this functional equation}}$
\end{solution}



\begin{solution}[by \href{https://artofproblemsolving.com/community/user/30342}{nicetry007}]
	$ f(x - f(y)) = 4f(x) - f(y) -4x -------------(1)$

$ x = f(0)$ in $ (1) \Rightarrow f(f(0) - f(y)) = 4f(f(0)) - f(y) - f(0) = c - f(y)$ where $ c = 4f(f(0)) - f(0)$.

$ x = 0$ in $ (1) \Rightarrow f(- f(y)) = 4f(0) - f(y) = d - f(y)$ where $ d = 4f(0)$.

Setting $ x = f(0) - f(u), y = 0$ in $ (1)$, we get

$ f(-f(u)) = 4f(f(0) - f(u)) - f(0) -4(f(0) - f(u))$ 

$ \Rightarrow d - f(u) = 4(c - f(u)) -f(0) -4f(0) + 4f(u)$

$ \Rightarrow f(u) = d + 5f(0) -4c \Rightarrow f$ is a constant function.  But $ (1)$ does not have any constant function as a solution.
Hence, the given functional equation does not have any solution.
\end{solution}



\begin{solution}[by \href{https://artofproblemsolving.com/community/user/29428}{pco}]
	\begin{tcolorbox}$ f(x - f(y)) = 4f(x) - f(y) - 4x - - - - - - - - - - - - - (1)$

$ x = f(0)$ in $ (1) \Rightarrow f(f(0) - f(y)) = 4f(f(0)) - f(y) - f(0) = c - f(y)$ where $ c = 4f(f(0)) - f(0)$.

$ x = 0$ in $ (1) \Rightarrow f( - f(y)) = 4f(0) - f(y) = d - f(y)$ where $ d = 4f(0)$.

Setting $ x = f(0) - f(u), y = 0$ in $ (1)$, we get

$ f( - f(u)) = 4f(f(0) - f(u)) - f(0) - 4(f(0) - f(u))$ 

$ \Rightarrow d - f(u) = 4(c - f(u)) - f(0) - 4f(0) + 4f(u)$

$ \Rightarrow f(u) = d + 5f(0) - 4c \Rightarrow f$ is a constant function.  But $ (1)$ does not have any constant function as a solution.
Hence, the given functional equation does not have any solution.\end{tcolorbox}

Nice ! 
:)
\end{solution}



\begin{solution}[by \href{https://artofproblemsolving.com/community/user/68920}{prester}]
	$ \forall x,y \in \mathbb{R}$ we have $ 4x = 4f(x)-f(y)-f(x-f(y))$

So the $ f(x)$ is surjective. Hence $ \exists u \in \mathbb{R}$ such that $ f(u)=0$.

$ y=u$ $ \implies f(x)=4f(x)-4x \implies f(x)=\frac{4x}3$

$ f(x)=\frac{4x}3$ is not a solution of original equation

\begin{bolded}Hence there are no solution.\end{bolded} 

\begin{italicized}(I am not sure that we can assume that $ f(x)$ is surjective....what do you think about it?)\end{italicized}
\end{solution}



\begin{solution}[by \href{https://artofproblemsolving.com/community/user/29428}{pco}]
	\begin{tcolorbox}$ \forall x,y \in \mathbb{R}$ we have $ 4x = 4f(x) - f(y) - f(x - f(y))$

So the $ f(x)$ is surjective. ...

\begin{italicized}(I am not sure that we can assume that $ f(x)$ is surjective....what do you think about it?)\end{italicized}\end{tcolorbox}

$ 4x = 4f(x) - f(y) - f(x - f(y))$ does not imply that $ f(x)$ is surjective  :)
\end{solution}



\begin{solution}[by \href{https://artofproblemsolving.com/community/user/42653}{Bacteria}]
	Let $ a,c$ be arbitrary real numbers and say $ f(a)=b$ and $ f(c)=d$. Take y=a to get:
$ f(x-f(a))=4f(x)-f(a)-4x$, so $ f(x-b)=4f(x)-b-4x$
Take y=c to get:
$ f(x-f(c))=4f(x)-f(c)-4x$ so $ f(x-d)=4f(x)-d-4x$
Take x as x-b and y=c to get:
$ f(x-b-f(c))=4f(x-b)-f(c)-4(x-b)$ so $ f(x-b-d)=4(4f(x)-b-4x)-d-4(x-b)=16f(x)-20x-d$ 
Take x as x-d and y=a to get:
$ f(x-d-f(a))=4f(x-d)-f(a)-4(x-d)$ so $ f(x-d-b)=4(4f(x)-d-4x)-b-4(x-d)=16f(x)-20x-b$ 
Thus
$ 16f(x)-20x-d=f(x-b-d)=f(x-d-b)=16f(x)-20x-b$ so $ b=d$.
Therefore $ f$ is a constant function, ie $ f(x)=b$ for every real x. Thus $ b=4b-b-4x$, so $ 4x=2b$ and $ x=b\/2$ for every real x; contradiction. Thus no such function exists.
\end{solution}



\begin{solution}[by \href{https://artofproblemsolving.com/community/user/86345}{namdan}]
	\begin{tcolorbox}[quote="ozgurkircak"]Find all $ f: R\rightarrow R$ such that:
\[ f(x - f(y)) = 4f(x) - f(y) - 4x\]
for all $ x,y \in \mathbb{R}.$\end{tcolorbox}

Let $ P(x,y)$ be the assertion $ f(x - f(y)) = 4f(x) - f(y) - 4x$
Let $ I = f(\mathbb R)$

$ P(f(x),x)$ $ \implies$ $ f(0) = 4f(f(x)) - 5f(x)$ and so $ 4f(f((x)) = 5f(x) + f(0)$
$ P(f(x),y)$ $ \implies$ $ f(f(x) - f(y)) = 4f(f(x)) - f(y) - 4f(x)$ and so, using previous line : $ f(f(x) - f(y)) = f(x) - f(y) + f(0)$
$ \implies$ $ f(x) - f(y) + f(0)\in I$ $ \forall x,y$
$ \implies$ $ f(x) - (f(y) - f(z) + f(0)) + f(0) = f(x) - f(y) + f(z)\in I$ $ \forall x,y,z$
$ \implies$ $ (f(x) - f(y) + f(x)) - (4f(x) - f(y) - 4x) + (f(x) - f(y) + f(x)) = 4x - f(y)\in I$ $ \forall x,y$
$ \implies$ $ 0\in I$ (just set $ x = \frac {f(y)}4$ in previous line).

Using then $ y$ such that $ f(y) = 0$ in $ P(x,y)$, we get $ f(x) = 4f(x) - 4x$ and so $ f(x) = \frac {4x}3$ $ \forall x$

Then, plugging this back in original equation, we get that this necessary condition does not fit the equation.

Hence $ \boxed{\text{no solution for this functional equation}}$\end{tcolorbox}
I do not understand, why we can for $f(y) = 0$
Thanks you pco!
\end{solution}



\begin{solution}[by \href{https://artofproblemsolving.com/community/user/29428}{pco}]
	\begin{tcolorbox}
I do not understand, why we can for $f(y) = 0$
Thanks you pco!\end{tcolorbox}
Because, just line above, I proved that $0\in I$
\end{solution}



\begin{solution}[by \href{https://artofproblemsolving.com/community/user/86345}{namdan}]
	Why we can not infer $4x-f(y)$ immediately?
\end{solution}



\begin{solution}[by \href{https://artofproblemsolving.com/community/user/29428}{pco}]
	\begin{tcolorbox}Why we can not infer $4x-f(y)$ immediately?\end{tcolorbox}
If you want some help, make some effort.

What is the meaning of this question ? 
What is the meaning of "infer a quantity" ? I understand "infer a proposition \/ equality \/ inequality" from another ?

Make some full sentences and I'll try to answer.
\end{solution}



\begin{solution}[by \href{https://artofproblemsolving.com/community/user/82334}{bappa1971}]
	$g(x)=x-f(x)$ implies $g\left( x-y+g(y) \right) = 4g(x)+x$
Then, $y=x$ implies $g(g(x))=4g(x)+x$, hence $g$ is injective.
Again, $g(g(0))=4g(0)+0=g(0-x+g(x))$. So injectivity of $g$ implies $g(x)=x+g(0)$, which is not a solution indeed.
So, there is no solution.
\end{solution}



\begin{solution}[by \href{https://artofproblemsolving.com/community/user/86345}{namdan}]
	I want to ask you: Why we prove $ 4x-f(y) \in I \forall x,y $?
Thanhks you!
\end{solution}



\begin{solution}[by \href{https://artofproblemsolving.com/community/user/29428}{pco}]
	\begin{tcolorbox}I want to ask you: Why we prove $ 4x-f(y) \in I \forall x,y $?
Thanhks you!\end{tcolorbox}
Just above the line where I wrote that $4x-f(y)\in I$, I proved that $f(x)-f(y)+f(z)\in I$ $\forall x,y,z$

And just as the beginning of the line where I wrote that $4x-f(y)\in I$, I wrote :
$4x-f(y)=$ $(f(x)-f(y)+f(x))-(4f(x)-f(y)-4x)+(f(x)-f(y)+f(x))$

And since $f(x)-f(y)+f(x)\in I$ and $4f(x)-f(y)-4x=f(x-f(y))\in I$, we get the required result.

Btw, bappa1971's solution is quite shorter and nicer than mine :)
\end{solution}



\begin{solution}[by \href{https://artofproblemsolving.com/community/user/86345}{namdan}]
	$I=f(\mathbb{R})$: Valued function
We have $f(x-f(y))\in I$ so $f(f(x)-f(y))\in I$?
\end{solution}



\begin{solution}[by \href{https://artofproblemsolving.com/community/user/29428}{pco}]
	\begin{tcolorbox}$I=f(\mathbb{R})$: Valued function
We have $f(x-f(y))\in I$ so $f(f(x)-f(y))\in I$?\end{tcolorbox}

$f($anything$)\in I$ by definition of $I$
\end{solution}



\begin{solution}[by \href{https://artofproblemsolving.com/community/user/105288}{trenkialabautroj}]
	I think namdan try to ask you for your idea to solve this problem,pco.
\end{solution}



\begin{solution}[by \href{https://artofproblemsolving.com/community/user/86345}{namdan}]
	I don't know here:
$f(x-f(y))\in I$ so $f(f(x)-f(y))\in I$?
It's true for all cases?
Thanks all.
\end{solution}



\begin{solution}[by \href{https://artofproblemsolving.com/community/user/105288}{trenkialabautroj}]
	\begin{tcolorbox}I don't know here:
$f(x-f(y))\in I$ so $f(f(x)-f(y))\in I$?
It's true for all cases?
Thanks all.\end{tcolorbox}
just replace $x$ by  $f(x)$
\end{solution}



\begin{solution}[by \href{https://artofproblemsolving.com/community/user/105288}{trenkialabautroj}]
	Pco explained for you $f(anything)\in I$. I can't understand you
\end{solution}
*******************************************************************************
-------------------------------------------------------------------------------

\begin{problem}[Posted by \href{https://artofproblemsolving.com/community/user/51029}{mathVNpro}]
	Find all strictly increasing functions $f: \mathbb N \to \mathbb N$ such that
\[ f(n+f(n))=2f(n), \quad \forall n\in \mathbb {N}.\]
	\flushright \href{https://artofproblemsolving.com/community/c6h317477}{(Link to AoPS)}
\end{problem}



\begin{solution}[by \href{https://artofproblemsolving.com/community/user/44083}{jgnr}]
	[hide]$ f(n+f(n))=2f(n)$

$ n+f(n)+f(n+f(n))=n+3f(n)$

$ f(n+f(n)+f(n+f(n))=f(n+3f(n))$

$ 2f(n+f(n))=f(n+3f(n))$

$ 4f(n)=f(n+3f(n))\ge f(n+3f(n)-1)+1\ge f(n+3f(n)-2)+2\ge\ldots\ge f(n+1)+3f(n)-1\ge f(n)+3f(n)=4f(n)$

Equality holds, so $ f(n+1)=f(n)+1$. Thus $ f(n)=n+a$ for all $ n$. Easy to check the this function satisfies the given conditions for any $ a\ge0$.[\/hide]
\end{solution}



\begin{solution}[by \href{https://artofproblemsolving.com/community/user/37364}{kihe_freety5}]
	we have $ f(n+1) \ge f(n)+1$ by induction we attain $ f(m+n) \ge f(n)+m$
we choose m=f(n) we have $ f(n+f(n)) \ge 2f(n)$ so f(n+m)=f(m)+f(n) with every natural m,n....
\end{solution}



\begin{solution}[by \href{https://artofproblemsolving.com/community/user/29428}{pco}]
	\begin{tcolorbox}Find all strictly increase function $ f$ from $ \mathbb {N}$ to $ \mathbb {N}$ such that:
\[ f(n + f(n)) = 2f(n), \forall n\in \mathbb {N}\]
\end{tcolorbox}

If $ \exists u$ such that $ f(u+1)>f(u)+1$ then $ f(u+f(u))>f(u)+f(u)$, impossible. So $ f(u+1)=f(u)+1$ $ \forall u$. hence the unique solution $ f(x)=x+a$
\end{solution}



\begin{solution}[by \href{https://artofproblemsolving.com/community/user/53575}{subhanrustemli}]
	f(n)>f(n-1)
 f(n)≥f(n-1)+1
 f(n)-n≥f(n-1) -(n-1)≥f(n-2)-(n-2)≥.............≥f(1)-1
 thus f(n+f(n))-(n+f(n))≥f(n)-n 
 f(n+f(n))≥2f(n)
 but we have f(n+f(n))=2f(n)
 equality holds f(n)-n=f(n-1)-(n-1)=.........=f(1)-1
 f(n)-n=f(1)-1  f(n)=n+c proven :roll:
\end{solution}
*******************************************************************************
-------------------------------------------------------------------------------

\begin{problem}[Posted by \href{https://artofproblemsolving.com/community/user/61814}{caubetoanhoc94}]
	Find all functions $f: (0,1) \to \mathbb R$ such that
\[f(xyz) = xf(x) + yf(y) + zf(z)\]
for all $x,y,z \in (0,1)$.
	\flushright \href{https://artofproblemsolving.com/community/c6h317539}{(Link to AoPS)}
\end{problem}



\begin{solution}[by \href{https://artofproblemsolving.com/community/user/29428}{pco}]
	\begin{tcolorbox}Find all function:$ (0,1) \to R$ such that:
$ f(xyz) = xf(x) + yf(y) + zf(z)$\end{tcolorbox}

$ f(xyztu)=f((xyz)tu)=xyzf(xyz)+tf(t)+uf(u)$ $ =x^2yzf(x)+xy^2zf(y)+xyz^2f(z)+tf(t)+uf(u)$

$ f(xyztu)=f(x(yzt)u)=xf(x)+yztf(yzt)+uf(u)$ $ =xf(x)+y^2ztf(y)+yz^2tf(z)+yzt^2f(t)+uf(u)$

Equating implies $ (x^2yz-x)f(x)+(xy^2z-y^2zt)f(y)$ $ +(xyz^2-yz^2t)f(z)+(t-yzt^2)f(t)=0$

Setting $ y=z=t=\frac 12$ in this equation, we get $ (x^2-4x)f(x)+(x+\frac 54)f(\frac 12)=0$

And so $ f(x)=a\frac{4x+5}{x^2-4x}$

Plugging this back in the equation, we get $ a=0$

Hence the unique solution : $ f(x)=0$ $ \forall x\in(0,1)$
\end{solution}
*******************************************************************************
-------------------------------------------------------------------------------

\begin{problem}[Posted by \href{https://artofproblemsolving.com/community/user/46163}{Altunshukurlu}]
	Find all functions $f: \mathbb R^{+} \to \mathbb R^{+}$ such that for all positive reals $x$ and $y$,
\[ f(x^3+y)=f(x)^3+\frac{f(xy)}{f(x)}.\]
	\flushright \href{https://artofproblemsolving.com/community/c6h318112}{(Link to AoPS)}
\end{problem}



\begin{solution}[by \href{https://artofproblemsolving.com/community/user/29428}{pco}]
	\begin{tcolorbox}let f;R+ -> R+  and f{x^3 + y}={f(x)}^3 + {f(xy)}\/{f(x)} find all such functions for x,y>0\end{tcolorbox}
Let $ P(x,y)$ be the assertion $ f(x^3+y)=f(x)^3+\frac{f(xy)}{f(x)}$
Let $ f(1)=a$


1) $ f(1)=1$
$ P(1,x)$ $ \implies$ $ f(x+1)=a^3+\frac{f(x)}a$ and so $ f(x+n)=\frac{f(x)}{a^n}+a^3+a^2+a+1+a^{-1}+...a^{4-n}$ so :
$ f(2)=a^3+1$
$ f(9)=f(1+8)=\frac{f(1)}{a^8}+a^3+a^2+a+1+a^{-1}+...a^{-4}$ $ =a^{-7}+a^{-4}+a^{-3}+...+a^2+a^3$
But $ P(2,1)$ $ \implies$ $ f(9)=f(2)^3+\frac{f(2)}{f(2)}$ $ =(a^3+1)^3+1$

So $ a^{-7}+a^{-4}+a^{-3}+...+a^2+a^3=(a^3+1)^3+a^2+a^{-1}$ $ =a^9+3a^6+3a^3+2$

So $ a^{16}+3a^{13}+2a^{10}-a^9-a^8+a^7-a^6-a^5-a^4-a^3-1=0$

So $ (a-1)(a^{15}+a^{14}+a^{13}+4a^{12}+4a^{11}+4a^{10}$ $ +6a^9+5a^8+4a^7+5a^6+4a^5+$ $ 3a^4+2a^3+a^2+a+1)=0$

And so $ a=1$ (since the other factor is $ >0$

2) $ f(u+v)=f(u)+f(v)$ $ \forall u,v>0$
$ P(1,x)$ $ \implies$ $ f(x+1)=f(x)+1$
$ P(x,y+1)$ $ \implies$ $ f(x^3+y+1)=f(x)^3+\frac{f(xy+x)}{f(x)}$ $ =f(x^3+y)+1$ and so $ f(xy+x)=f(xy)+f(x)$
Setting then $ x=v$ and $ y=\frac uv$, we get $ f(u+v)=f(u)+f(v)$

3) $ f(x)=x$ $ \forall x$
From $ f(1)=1$ and $ f(u+v)=f(u)+f(v)$ $ \forall u,v>0$, it's immediate to conclude :
$ f(n)=n$ $ \forall n\in\mathbb N$
$ f(\frac 1n)=\frac 1n$ $ \forall n\in\mathbb N$
$ f(p)=p$ $ \forall p\in\mathbb Q^+$
$ f(x)$ is non decreasing
And so $ f(x)=x$ $ \forall x\in\mathbb R^+$

And it is easy to check back that this necessary condition is sufficient ($ f(x)=x$ indeed is a solution)

Hence the unique solution : $ \boxed{f(x)=x}$
\end{solution}



\begin{solution}[by \href{https://artofproblemsolving.com/community/user/46163}{Altunshukurlu}]
	Thanks for the solution ... 
\begin{bolded}a^{3}+1=a \end{bolded} if this equation has a root in \begin{bolded}{R+} \end{bolded} then this root can be a solution to the function
\end{solution}



\begin{solution}[by \href{https://artofproblemsolving.com/community/user/29428}{pco}]
	\begin{tcolorbox}Thanks for the solution ... 
\begin{bolded}a^{3}+1=a \end{bolded} if this equation has a root in \begin{bolded}{R+} \end{bolded} then this root can be a solution to the function\end{tcolorbox}
As I previously proved : $ f(x)=x$ is the only solution. So $ f(x)=c$ cant be.

And a quick look at $ x^3-x+1=0$ shows that this equation has no positive real root.
\end{solution}



\begin{solution}[by \href{https://artofproblemsolving.com/community/user/46163}{Altunshukurlu}]
	Thanks again !!  :)
\end{solution}
*******************************************************************************
-------------------------------------------------------------------------------

\begin{problem}[Posted by \href{https://artofproblemsolving.com/community/user/58367}{theSA}]
	Find all functions $f: \mathbb N \to \mathbb N$ such that \[ f(f(m) + f(n)) = m + n + 3\] for all positive integers $m$ and $n$.
	\flushright \href{https://artofproblemsolving.com/community/c6h318344}{(Link to AoPS)}
\end{problem}



\begin{solution}[by \href{https://artofproblemsolving.com/community/user/29428}{pco}]
	\begin{tcolorbox}Find all functions  $ f: N \to N$  such that  $ f(f(m) + f(n)) = m + n + 3$.\end{tcolorbox}

The functional equation shows that $ f(n)$ is injective.

Then, $ f(f(m + 1) + f(n)) = m + n + 4 = f(f(m) + f(n + 1))$ and so, since $ f$ is injective : $ f(m + 1) + f(n) = f(m) + f(n + 1)$, which may be written :
$ f(m + 1) - f(m) = f(n + 1) - f(n)$ $ \forall m,n$

So $ f(n + 1) - f(n) = a$, constant and so $ f(n) = an + b$

Plugging this back in original equation, we get $ a = b = 1$ and so $ \boxed{f(n) = n + 1}$ 
(there is another solution $ a = - 1$ and $ b = - 3$, and so $ f(n) = - n - 3$, but not from $ \mathbb N\to\mathbb N$)
\end{solution}
*******************************************************************************
-------------------------------------------------------------------------------

\begin{problem}[Posted by \href{https://artofproblemsolving.com/community/user/73589}{mathmen}]
	Find all functions $f: \mathbb R \to \mathbb R$ such that
\[ f(x + f(y)) = x^2 + f(y)^2\]
for all $ x,y \in \mathbb{R}.$
	\flushright \href{https://artofproblemsolving.com/community/c6h318378}{(Link to AoPS)}
\end{problem}



\begin{solution}[by \href{https://artofproblemsolving.com/community/user/29428}{pco}]
	\begin{tcolorbox}find all function such that ;
$ (\forall x,y \in \mathbb{R}); f(x + f(y)) = x^2 + f(y)^2$\end{tcolorbox}
Let $ P(x,y)$ be the assertion $ f(x + f(y)) = x^2 + f(y)^2$

$ P(0,x)$ $ \implies$ $ f(f(x))=f(x)^2$ and so $ f(x)=x^2$ $ \forall x\in f(\mathbb R)$

Let then $ y\in\mathbb R$ and $ a\ne 0$ such that $ a>f(y)^2-f(y)$. Let $ z=\sqrt{a+f(y)-f(y)^2}$ : $ P(z,y)$ $ \implies$ $ f(z + f(y)) = z^2 + f(y)^2=a+f(y)$

So $ a+f(y)\in f(\mathbb R)$ and so $ f(a+f(y))=(a+f(y))^2$ and so $ a^2 + f(y)^2=(a+f(y))^2$ and so $ af(y)=0$ and so $ f(y)=0$

Hence the unique solution : $ f(x)=0$ $ \forall x$
\end{solution}



\begin{solution}[by \href{https://artofproblemsolving.com/community/user/66394}{reason}]
	\begin{tcolorbox}[quote="mathmen"]find all function such that ;
$ (\forall x,y \in \mathbb{R}); f(x + f(y)) = x^2 + f(y)^2$\end{tcolorbox}
Let $ P(x,y)$ be the assertion $ f(x + f(y)) = x^2 + f(y)^2$

$ P(0,x)$ $ \implies$ $ f(f(x)) = f(x)^2$ and so $ f(x) = x^2$ $ \forall x\in f(\mathbb R)$

Let then $ y\in\mathbb R$ and $ a\ne 0$ such that $ a > f(y)^2 - f(y)$. Let $ z = \sqrt {a + f(y) - f(y)^2}$ : $ P(z,y)$ $ \implies$ $ f(z + f(y)) = z^2 + f(y)^2 = a + f(y)$

So $ a + f(y)\in f(\mathbb R)$ and so $ f(a + f(y)) = (a + f(y))^2$ and so $ a^2 + f(y)^2 = (a + f(y))^2$ and so $ af(y) = 0$ and so $ f(y) = 0$

Hence the unique solution : $ f(x) = 0$ $ \forall x$\end{tcolorbox}

$ f(x)=0$  verify the initial condition?
\end{solution}



\begin{solution}[by \href{https://artofproblemsolving.com/community/user/68920}{prester}]
	\begin{tcolorbox}[quote="mathmen"]find all function such that ;
$ (\forall x,y \in \mathbb{R}); f(x + f(y)) = x^2 + f(y)^2$\end{tcolorbox}
Let $ P(x,y)$ be the assertion $ f(x + f(y)) = x^2 + f(y)^2$

$ P(0,x)$ $ \implies$ $ f(f(x)) = f(x)^2$ and so $ f(x) = x^2$ $ \forall x\in f(\mathbb R)$

Let then $ y\in\mathbb R$ and $ a\ne 0$ such that $ a > f(y)^2 - f(y)$. Let $ z = \sqrt {a + f(y) - f(y)^2}$ : $ P(z,y)$ $ \implies$ $ f(z + f(y)) = z^2 + f(y)^2 = a + f(y)$

So $ a + f(y)\in f(\mathbb R)$ and so $ f(a + f(y)) = (a + f(y))^2$ and so $ a^2 + f(y)^2 = (a + f(y))^2$ and so $ af(y) = 0$ and so $ f(y) = 0$

Hence the unique solution : $ f(x) = 0$ $ \forall x$\end{tcolorbox}

Let $ P(x,y)$ be the assertion $ f(x + f(y)) = x^2 + f(y)^2$

Let $ f(0) = a$

$ P(x - a,0)$ $ \implies$ $ f(x) = (x - a)^2 + a^2$

Hence $ f(0) = a$ $ \implies 2a^2 = a$ (edited since previous error)

If $ a = 0$ $ \implies f(x) = x^2$ but it does not verifiy the inital condition

If $ a = + \frac12$ $ \implies f(x) = x^2 - x + \frac12$ but it does not verifiy the inital condition

So there are no solution...
\end{solution}



\begin{solution}[by \href{https://artofproblemsolving.com/community/user/29428}{pco}]
	\begin{tcolorbox}[quote="pco"][quote="mathmen"]find all function such that ;
$ (\forall x,y \in \mathbb{R}); f(x + f(y)) = x^2 + f(y)^2$\end{tcolorbox}
Let $ P(x,y)$ be the assertion $ f(x + f(y)) = x^2 + f(y)^2$

$ P(0,x)$ $ \implies$ $ f(f(x)) = f(x)^2$ and so $ f(x) = x^2$ $ \forall x\in f(\mathbb R)$

Let then $ y\in\mathbb R$ and $ a\ne 0$ such that $ a > f(y)^2 - f(y)$. Let $ z = \sqrt {a + f(y) - f(y)^2}$ : $ P(z,y)$ $ \implies$ $ f(z + f(y)) = z^2 + f(y)^2 = a + f(y)$

So $ a + f(y)\in f(\mathbb R)$ and so $ f(a + f(y)) = (a + f(y))^2$ and so $ a^2 + f(y)^2 = (a + f(y))^2$ and so $ af(y) = 0$ and so $ f(y) = 0$

Hence the unique solution : $ f(x) = 0$ $ \forall x$\end{tcolorbox}

$ f(x) = 0$  verify the initial condition?\end{tcolorbox}

 :blush:  :rotfl:  How stupid I am !!!

So ... no solution.  :blush:

Thanks for the correction
\end{solution}
*******************************************************************************
-------------------------------------------------------------------------------

\begin{problem}[Posted by \href{https://artofproblemsolving.com/community/user/74020}{irantst}]
	Find all functions $f: \mathbb N \to \mathbb N$ such that
\[f(x+f(x^2+y))=2yf(x+y)\]
holds true for all positive integers $x$ and $y$.
	\flushright \href{https://artofproblemsolving.com/community/c6h318642}{(Link to AoPS)}
\end{problem}



\begin{solution}[by \href{https://artofproblemsolving.com/community/user/29428}{pco}]
	\begin{tcolorbox}find all functions f:N--N
f(x+f(x^2+y))=2yf(x+y)
 \end{tcolorbox}
There are no such functions. It seemed obvious to me but I needed a rather heavy method to show it. I hope (not so much :) ) that somebody will find an elementary one.

I consider, as usual in Mathlinks, that $ 0\notin\mathbb N$

Let $ P(x,y)$ be the assertion $ f(x+f(x^2+y))=2yf(x+y)$

1) $ f(a)=f(b)$ and $ a,b>1$ implies $ a=b$
$ a>1$ and $ P(1,a-1)$ $ \implies$ $ f(1+f(a))=2(a-1)f(a)>0$
$ b>1$ and $ P(1,b-1)$ $ \implies$ $ f(1+f(b))=2(b-1)f(b)>0$
So $ a>1$ and $ b>1$ and $ f(a)=f(b)$ $ \implies$ $ a-1=b-1$ and so $ a=b$
Q.E.D.

2) $ f(x^2+2y)>f((x+2y-1)^2+1)$ $ \forall x,y\in\mathbb N$
$ P(x,2y)$ $ \implies$ $ f(x+f(x^2+2y))=4yf(x+2y)$
Let $ u(x)=x+f(x^2+1)\ge 2$ (notice that $ u(x)>x$)
$ x+2y-1>0$ and so $ P(x+2y-1,1)$ $ \implies$ $ f(u(x+2y-1))=2f(x+2y)$
$ x+2y-1>0$ and $ u(x+2y-1)-y>x+2y-1-y>0$ and  so :
$ P(u(x+2y-1)-y,y)$ $ \implies$ $ f(u(x+2y-1)-y+f((u(x+2y-1)-y)^2+y))=2yf(u(x+2y-1))$ $ =4yf(x+2y)$

So $ f(x+f(x^2+2y))=f(u(x+2y-1)-y+f((u(x+2y-1)-y)^2+y))$
We have $ x+f(x^2+2y)>1$ and $ u(x+2y-1)-y+f((u(x+2y-1)-y)^2+y)>u(x+2y-1)-y>x+2y-1-y\ge 1$
So we can apply the pseudo-injectivity (see 1. above) and :

$ x+f(x^2+2y)=u(x+2y-1)-y+f((u(x+2y-1)-y)^2+y)$
$ \iff$ $ x+f(x^2+2y)=(x+2y-1)+f((x+2y-1)^2+1)-y+f((u(x+2y-1)-y)^2+y)$
$ \iff$ $ f(x^2+2y)=y-1+f((x+2y-1)^2+1)+f((u(x+2y-1)-y)^2+y)$ $ >f((x+2y-1)^2+1)$
Q.E.D.

3) There are no solution
Let $ m=\inf(f(\mathbb N\backslash\{1,2,3,4,5\}))$ and $ a\ge 6$ any positive integer such that $ f(a)=m$
3.1) If $ a$ is odd
Let $ b=\frac{a-1}2\ge 3$
Using property established in 2. above with $ x=1$ and $ y=b$, we get $ f(a)>f(4b^2+1)$ and so $ f(4b^2+1)<m$
And since $ 4b^2+1\ge 37$, this is impossible (see definition of $ m$)

3.2) If $ a$ is even
Let $ b=\frac a2-2\ge 1$
Using property established in 2. above with $ x=2$ and $ y=b$, we get $ f(a)>f((2b+1)^2+1)$ and so $ f((2b+1)^2+1)<m$
And since $ (2b+1)^2+1\ge 10$, this is impossible (see definition of $ m$)

Q.E.D.


Hence no solution for this equation.
\end{solution}



\begin{solution}[by \href{https://artofproblemsolving.com/community/user/74020}{irantst}]
	tanks
It was very nice soloution
very good
 :)
\end{solution}



\begin{solution}[by \href{https://artofproblemsolving.com/community/user/43536}{nguyenvuthanhha}]
	\begin{italicized}Excellent Solution !\end{italicized} :first:
\end{solution}



\begin{solution}[by \href{https://artofproblemsolving.com/community/user/74020}{irantst}]
	yes it's very nice
pco is god of algebra
\end{solution}
*******************************************************************************
-------------------------------------------------------------------------------

\begin{problem}[Posted by \href{https://artofproblemsolving.com/community/user/51029}{mathVNpro}]
	Find all functions $ f: \mathbb {Z}\to\mathbb {Z}$ for which we have $f(0)=1$ and 
\[ f(f(n))=f(f(n+2)+2)=n,\quad \forall n\in \mathbb {Z}.\]
	\flushright \href{https://artofproblemsolving.com/community/c6h319146}{(Link to AoPS)}
\end{problem}



\begin{solution}[by \href{https://artofproblemsolving.com/community/user/29428}{pco}]
	\begin{tcolorbox}Find all functions $ f:$ $ \mathbb {Z}\mapsto \mathbb {Z}$ for which we have:
\[ f(0) = 1, f(f(n)) = f(f(n + 2) + 2) = n, \forall n\in \mathbb {Z}\]
\end{tcolorbox}

$ f(f(n)) = n$ shows that $ f(n)^$ is injective. So $ f(f(n)) = f(f(n + 2) + 2)$ implies $ f(n + 2) = f(n) - 2$ and so :
$ f(2p + 1) = f(1) - 2p$
$ f(2p) = f(0) - 2p$

Then $ f(0) = 1$ implies $ f(1) = f(f(0)) = 0$ and so $ f(n) = 1 - n$ $ \forall n$ and a quick check confirms that this indeed is a solution.

Hence the unique solution : $ \boxed{f(n) = 1 - n}$
\end{solution}
*******************************************************************************
-------------------------------------------------------------------------------

\begin{problem}[Posted by \href{https://artofproblemsolving.com/community/user/70914}{Gaoranger}]
	Find all functions $ f: \mathbb {R}\to\mathbb {R}$ for which we have
\[ (x+y)(f(x)-f(y))=(x-y)f(x+y), \quad \forall x,y \in \mathbb R.\]
	\flushright \href{https://artofproblemsolving.com/community/c6h319317}{(Link to AoPS)}
\end{problem}



\begin{solution}[by \href{https://artofproblemsolving.com/community/user/29428}{pco}]
	\begin{tcolorbox}Find all functions $ f: \mathbb {R}\mapsto \mathbb {R}$ for which we have:

$ (x + y)(f(x) - f(y)) = (x - y)f(x + y)$\end{tcolorbox}

Let $ P(x,y)$ be the assertion $ (x+y)(f(x)-f(y))=(x-y)f(x+y)$

Let $ x\in\mathbb R^*$ :

$ P(x+1,x-1)$ $ \implies$ $ f(x+1)-f(x-1)=\frac{f(2x)}x$

$ P(x-1,3-x)$ $ \implies$ $ f(x-1)-f(3-x)=(x-2)f(2)$

$ P(3-x,x+1)$ $ \implies$ $ f(3-x)-f(x+1)=(1-x)\frac{f(4)}2$

Summing these three lines implies : $ \frac{f(2x)}x+(x-2)f(2)+(1-x)\frac{f(4)}2=0$ and so $ f(2x)=\left(\frac{f(4)}2-f(2)\right)x^2+\left(2f(2)-\frac{f(4)}2\right)x$

And so $ f(x)=ax^2+bx$ $ \forall x\ne 0$.

Plugging back in the original equation, we get that this indeed is a solution. And since $ P(1,-1)$ $ \implies$ $ f(0)=0$, we get :

$ \boxed{f(x)=ax^2+bx}$ $ \forall x$
\end{solution}
*******************************************************************************
-------------------------------------------------------------------------------

\begin{problem}[Posted by \href{https://artofproblemsolving.com/community/user/29214}{HTA}]
	Find all functions $f: \mathbb Z \to \mathbb Z$ which satisfy
\[ f(m+n) + f(m)f(n) = f(mn+1 )\]
for all integers $m$ and $n$.
	\flushright \href{https://artofproblemsolving.com/community/c6h319361}{(Link to AoPS)}
\end{problem}



\begin{solution}[by \href{https://artofproblemsolving.com/community/user/29428}{pco}]
	\begin{tcolorbox}find f : Z -> Z :

$ f(m + n) + f(m)f(n) = f(mn + 1 )$\end{tcolorbox}
Let $ P(x,y)$ be the assertion $ f(x+y)+f(x)f(y)=f(xy+1)$

The only constant solution $ f(x)=c$ is such that $ c+c^2=c$ and so is $ f(x)=0$ $ \forall x$. So we'll from now consider that $ f(x)$ is not a constant function.

Then, $ P(x,0)$ $ \implies$ $ f(x)(f(0)+1)=f(1)$ and, since $ f(x)$ is not constant, we need $ f(0)=-1$ and so $ f(1)=0$
$ P(1-x,-1)$ $ \implies$ $ f(-x)+f(1-x)f(-1)=f(x)$ and so $ f(x)-f(-x)=f(1-x)f(-1)$
$ P(1+x,-1)$ $ \implies$ $ f(x)+f(1+x)f(-1)=f(-x)$ and so $ f(-x)-f(x)=f(1+x)f(-1)$
And so (adding these two lines) : $ f(-1)(f(1+x)+f(1-x))=0$

1) Let $ f(-1)=a\ne 0$
================
Then the last line above implies $ f(x+1)+f(1-x)=0$
And, since $ P(x,-1)$ $ \implies$ $ f(x-1)+af(x)=f(1-x)$, we get $ f(x-1)+af(x)=-f(x+1)$ and so :
$ f(0)=-1$, $ f(1)=0$ and $ f(n+2)=-af(n+1)-f(n)$
This gives $ f(2)=1$, $ f(3)=-a$, $ f(4)=a^2-1$ and $ f(5)=2a-a^3$
Then $ P(2,2)$ $ \implies$ $ f(4)+f(2)^2=f(5)$ and so $ a^2-1+1=2a-a^3$ $ \iff$ $ a(a-1)(a+2)=0$ hence $ a\in\{-2,1\}$ :

1.1) $ a=1$
-----------
So $ f(0)=-1$, $ f(1)=0$ and $ f(n+2)=-f(n+1)-f(n)$
This gives $ f(2)=1$ and $ f(3)=-1$ and so $ (f(0),f(1),f(2),f(3),f(4),f(5),...)$ $ =(-1,0,1,-1,0,1, ...)$
So $ f(n)=\text{mod}(n,3)-1$ and it is easy to check that this indeed is a solution.

1.2) $ a=-2$
-------------
So $ f(0)=-1$, $ f(1)=0$ and $ f(n+2)=2f(n+1)-f(n)$ and so $ f(n)=n-1$
And it is easy to check that this indeed is a solution.

2) Consider now that $ f(-1)=0$
=======================
$ P(n,2)$ $ \implies$ $ f(2n+1)=f(n+2)+f(2)f(n)$
$ P(2n,2)$ $ \implies$ $ f(4n+1)=f(2n+2)+f(2n)f(2)$
$ P(n,4)$ $ \implies$ $ f(4n+1)=f(n+4)+f(n)f(4)$
And so, from the two lines above : $ f(2n+2)=f(n+4)+f(n)f(4)-f(2n)f(2)$

So :
$ f(2n+1)=f(n+2)+f(2)f(n)$
$ f(2n+2)=f(n+4)+f(n)f(4)-f(2n)f(2)$
And this shows that knowledge of $ f(0),f(1),f(2),f(3),f(4),f(5)$ and $ f(6)$ is enough to know $ f(n)$ $ \forall n\ge 0$

And, since $ P(n+1,-1)$ $ \implies$ $ f(-n)=f(n)+f(n+1)f(-1)$ and so $ f(-n)=f(n)$, we get full knowledge of $ f(n)$ $ \forall 
n\in\mathbb Z$

We know that : $ f(-1)=0$, $ f(0)=-1$ and $ f(1)=0$. Let $ f(2)=b$


$ P(2,-2)$ $ \implies$ $ f(0)+f(2)f(-2)=f(-3)$ and so $ f(3)=f(-3)=b^2-1$
$ P(2,-3)$ $ \implies$ $ f(-1)+f(2)f(-3)=f(-5)$ and so $ f(5)=f(-5)=b^3-b$
$ P(2,2)$ $ \implies$ $ f(4)+f(2)^2=f(5)$ and so $ f(4)=f(-4)=b^3-b^2-b$
$ P(2,3)$ $ \implies$ $ f(5)+f(2)f(3)=f(7)$ and so $ f(7)=f(-7)=2b^3-2b$
$ P(4,-2)$ $ \implies$ $ f(2)+f(4)f(-2)=f(-7)$ and so $ f(7)=f(-7)=b^4-b^3-b^2+b$

So $ b^4-b^3-b^2+b=2b^3-2b$ $ \iff$ $ b^4-3b^3-b^2+3b=0$ $ \iff$ $ (b-3)(b-1)b(b+1)=0$ Hence $ b\in\{-1,0,1,3\}$ :

2.1) $ b=-1$
--------------
We got up to now : $ (f(0),f(1),f(2),f(3),f(4),f(5),f(7))$ $ =(-1,0,b,b^2-1,b^3-b^2-b,b^3-b,2b^3-2b)$ and so :
$ (f(0),f(1),f(2),f(3),f(4),f(5),f(7))$ $ =(-1,0,-1,0,-1,0,0)$

$ P(3,-3)$ $ \implies$ $ f(0)+f(3)f(-3)=f(-8)$ and so $ f(8)=-1$
Using $ f(2n+2)=f(n+4)+f(n)f(4)-f(2n)f(2)$ with $ n=3$, we get $ f(8)=f(7)+f(3)f(4)-f(6)f(2)$ and so $ f(6)=-1$ 

So $ (f(0),f(1),f(2),f(3),f(4),f(5),f(6))$ $ =(-1,0,-1,0,-1,0,-1)$
And this lead us to test $ f(2p)=-1$, $ f(2p+1)=0$, which, indeed, is a solution.
So this is the unique solution in this paragraph ($ b=-1$)


2.2) $ b=0$
------------
We got up to now : $ (f(0),f(1),f(2),f(3),f(4),f(5),f(7))$ $ =(-1,0,b,b^2-1,b^3-b^2-b,b^3-b,2b^3-2b)$ and so :
$ (f(0),f(1),f(2),f(3),f(4),f(5),f(7))$ $ =(-1,0,0,-1,0,0,0)$

$ P(3,-3)$ $ \implies$ $ f(0)+f(3)f(-3)=f(-8)$ and so $ f(8)=0$
Using $ f(2n+2)=f(n+4)+f(n)f(4)-f(2n)f(2)$ with $ n=4$, we get $ f(10)=f(8)+f(4)^2-f(8)f(2)$ and so $ f(10)=0$
$ P(3,3)$ $ \implies$ $ f(6)+f(3)^2=f(10)$ and so $ f(6)=-1$

So $ (f(0),f(1),f(2),f(3),f(4),f(5),f(6))$ $ =(-1,0,0,-1,0,0,-1)$
And this lead us to test $ f(3p)=-1$, $ f(3p+1)=0$ and $ f(3p+2)=0$, which, indeed, is a solution.
So this is the unique solution in this paragraph ($ b=0$)

2.3) $ b=1$
------------
We got up to now : $ (f(0),f(1),f(2),f(3),f(4),f(5),f(7))$ $ =(-1,0,b,b^2-1,b^3-b^2-b,b^3-b,2b^3-2b)$ and so :
$ (f(0),f(1),f(2),f(3),f(4),f(5),f(7))$ $ =(-1,0,1,0,-1,0,0)$

$ P(3,-3)$ $ \implies$ $ f(0)+f(3)f(-3)=f(-8)$ and so $ f(8)=-1$
Using $ f(2n+2)=f(n+4)+f(n)f(4)-f(2n)f(2)$ with $ n=3$, we get $ f(8)=f(7)+f(3)f(4)-f(6)f(2)$ and so $ f(8)=-f(6)$
And so $ f(6)=1$

So $ (f(0),f(1),f(2),f(3),f(4),f(5),f(6))$ $ =(-1,0,1,0,-1,0,1)$
And it's obvious we must test the solution $ f(2p+1)=0$ and $ f(2p)=(-1)^{p+1}$, which, indeed, is a solution.
So this is the unique solution in this paragraph ($ b=1$)

2.4) $ b=3$
-------------
We got up to now : $ (f(0),f(1),f(2),f(3),f(4),f(5),f(7))$ $ =(-1,0,b,b^2-1,b^3-b^2-b,b^3-b,2b^3-2b)$ and so :
$ (f(0),f(1),f(2),f(3),f(4),f(5),f(7))$ $ =(-1,0,3,8,15,24,48)$

Using $ f(2n+2)=f(n+4)+f(n)f(4)-f(2n)f(2)$ with $ n=3$, we get $ f(8)=f(7)+f(3)f(4)-f(6)f(2)$ and so $ f(8)=168-3f(6)$ 
Using $ f(2n+2)=f(n+4)+f(n)f(4)-f(2n)f(2)$ with $ n=-4$, we get $ f(6)=f(0)+f(4)f(4)-f(8)f(2)$ and so $ f(6)=224-3f(8)$ 
From these two equalities, we get $ f(6)=35$ and $ f(8)=63$
Then $ f(0),f(1),f(2),f(3),f(4),f(5)$ and $ f(6)$ are all known and at most one function may get these values. And obviously we must test $ f(n)=n^2-1$, which, indeed, is a solution.
So this is the unique solution in this paragraph ($ b=3$)


3) Synthesis of solutions : We got $ 7$ different solutions :
============================================================
From introduction : $ f(x)=0$ $ \forall x$ : 
   $ (f(0),f(1),f(2),....)$ $ =(0,0,0,0,0,0,0,....)$

From 2.3 : $ f(2p)=(-1)^{p+1}$ and $ f(2p+1)=0$ : 
   $ (f(0),f(1),f(2),....)$ $ =(-1,0,1,0,-1,0,1,0,....)$

From 2.1 : $ f(2p)=-1$, $ f(2p+1)=0$ : 
   $ (f(0),f(1),f(2),....)$ $ =(-1,0,-1,0,-1,0,-1,0,....)$

From 2.2 : $ f(3p)=-1$, $ f(3p+1)=0$ and $ f(3p+2)=0$ : 
   $ (f(0),f(1),f(2),....)$ $ =(-1,0,0,-1,0,0,-1,0,0,....)$

From 1.1 : $ f(3p)=-1$, $ f(3p+1)=0$ and $ f(3p+2)=1$ : 
   $ (f(0),f(1),f(2),....)$ $ =(-1,0,1,-1,0,1,-1,0,1,....)$

From 1.2 : $ f(n)=n-1$ $ \forall n$

From 2.4 : $ f(n)=n^2-1$ $ \forall n$
\end{solution}



\begin{solution}[by \href{https://artofproblemsolving.com/community/user/125880}{sinhaarunabh}]
	\begin{tcolorbox}[quote="HTA"]find f : Z -> Z :

$ f(m + n) + f(m)f(n) = f(mn + 1 )$\end{tcolorbox}
Let $ P(x,y)$ be the assertion $ f(x+y)+f(x)f(y)=f(xy+1)$

The only constant solution $ f(x)=c$ is such that $ c+c^2=c$ and so is $ f(x)=0$ $ \forall x$. So we'll from now consider that $ f(x)$ is not a constant function.

Then, $ P(x,0)$ $ \implies$ $ f(x)(f(0)+1)=f(1)$ and, since $ f(x)$ is not constant, we need $ f(0)=-1$ and so $ f(1)=0$
$ P(1-x,-1)$ $ \implies$ $ f(-x)+f(1-x)f(-1)=f(x)$ and so $ f(x)-f(-x)=f(1-x)f(-1)$
$ P(1+x,-1)$ $ \implies$ $ f(x)+f(1+x)f(-1)=f(-x)$ and so $ f(-x)-f(x)=f(1+x)f(-1)$
And so (adding these two lines) : $ f(-1)(f(1+x)+f(1-x))=0$

1) Let $ f(-1)=a\ne 0$
================
Then the last line above implies $ f(x+1)+f(1-x)=0$
And, since $ P(x,-1)$ $ \implies$ $ f(x-1)+af(x)=f(1-x)$, we get $ f(x-1)+af(x)=-f(x+1)$ and so :
$ f(0)=-1$, $ f(1)=0$ and $ f(n+2)=-af(n+1)-f(n)$
This gives $ f(2)=1$, $ f(3)=-a$, $ f(4)=a^2-1$ and $ f(5)=2a-a^3$
Then $ P(2,2)$ $ \implies$ $ f(4)+f(2)^2=f(5)$ and so $ a^2-1+1=2a-a^3$ $ \iff$ $ a(a-1)(a+2)=0$ hence $ a\in\{-2,1\}$ :

1.1) $ a=1$
-----------
So $ f(0)=-1$, $ f(1)=0$ and $ f(n+2)=-f(n+1)-f(n)$
This gives $ f(2)=1$ and $ f(3)=-1$ and so $ (f(0),f(1),f(2),f(3),f(4),f(5),...)$ $ =(-1,0,1,-1,0,1, ...)$
So $ f(n)=\text{mod}(n,3)-1$ and it is easy to check that this indeed is a solution.

1.2) $ a=-2$
-------------
So $ f(0)=-1$, $ f(1)=0$ and $ f(n+2)=2f(n+1)-f(n)$ and so $ f(n)=n-1$
And it is easy to check that this indeed is a solution.

2) Consider now that $ f(-1)=0$
=======================
$ P(n,2)$ $ \implies$ $ f(2n+1)=f(n+2)+f(2)f(n)$
$ P(2n,2)$ $ \implies$ $ f(4n+1)=f(2n+2)+f(2n)f(2)$
$ P(n,4)$ $ \implies$ $ f(4n+1)=f(n+4)+f(n)f(4)$
And so, from the two lines above : $ f(2n+2)=f(n+4)+f(n)f(4)-f(2n)f(2)$

So :
$ f(2n+1)=f(n+2)+f(2)f(n)$
$ f(2n+2)=f(n+4)+f(n)f(4)-f(2n)f(2)$
And this shows that knowledge of $ f(0),f(1),f(2),f(3),f(4),f(5)$ and $ f(6)$ is enough to know $ f(n)$ $ \forall n\ge 0$

And, since $ P(n+1,-1)$ $ \implies$ $ f(-n)=f(n)+f(n+1)f(-1)$ and so $ f(-n)=f(n)$, we get full knowledge of $ f(n)$ $ \forall 
n\in\mathbb Z$

We know that : $ f(-1)=0$, $ f(0)=-1$ and $ f(1)=0$. Let $ f(2)=b$


$ P(2,-2)$ $ \implies$ $ f(0)+f(2)f(-2)=f(-3)$ and so $ f(3)=f(-3)=b^2-1$
$ P(2,-3)$ $ \implies$ $ f(-1)+f(2)f(-3)=f(-5)$ and so $ f(5)=f(-5)=b^3-b$
$ P(2,2)$ $ \implies$ $ f(4)+f(2)^2=f(5)$ and so $ f(4)=f(-4)=b^3-b^2-b$
$ P(2,3)$ $ \implies$ $ f(5)+f(2)f(3)=f(7)$ and so $ f(7)=f(-7)=2b^3-2b$
$ P(4,-2)$ $ \implies$ $ f(2)+f(4)f(-2)=f(-7)$ and so $ f(7)=f(-7)=b^4-b^3-b^2+b$

So $ b^4-b^3-b^2+b=2b^3-2b$ $ \iff$ $ b^4-3b^3-b^2+3b=0$ $ \iff$ $ (b-3)(b-1)b(b+1)=0$ Hence $ b\in\{-1,0,1,3\}$ :

2.1) $ b=-1$
--------------
We got up to now : $ (f(0),f(1),f(2),f(3),f(4),f(5),f(7))$ $ =(-1,0,b,b^2-1,b^3-b^2-b,b^3-b,2b^3-2b)$ and so :
$ (f(0),f(1),f(2),f(3),f(4),f(5),f(7))$ $ =(-1,0,-1,0,-1,0,0)$

$ P(3,-3)$ $ \implies$ $ f(0)+f(3)f(-3)=f(-8)$ and so $ f(8)=-1$
Using $ f(2n+2)=f(n+4)+f(n)f(4)-f(2n)f(2)$ with $ n=3$, we get $ f(8)=f(7)+f(3)f(4)-f(6)f(2)$ and so $ f(6)=-1$ 

So $ (f(0),f(1),f(2),f(3),f(4),f(5),f(6))$ $ =(-1,0,-1,0,-1,0,-1)$
And this lead us to test $ f(2p)=-1$, $ f(2p+1)=0$, which, indeed, is a solution.
So this is the unique solution in this paragraph ($ b=-1$)


2.2) $ b=0$
------------
We got up to now : $ (f(0),f(1),f(2),f(3),f(4),f(5),f(7))$ $ =(-1,0,b,b^2-1,b^3-b^2-b,b^3-b,2b^3-2b)$ and so :
$ (f(0),f(1),f(2),f(3),f(4),f(5),f(7))$ $ =(-1,0,0,-1,0,0,0)$

$ P(3,-3)$ $ \implies$ $ f(0)+f(3)f(-3)=f(-8)$ and so $ f(8)=0$
Using $ f(2n+2)=f(n+4)+f(n)f(4)-f(2n)f(2)$ with $ n=4$, we get $ f(10)=f(8)+f(4)^2-f(8)f(2)$ and so $ f(10)=0$
$ P(3,3)$ $ \implies$ $ f(6)+f(3)^2=f(10)$ and so $ f(6)=-1$

So $ (f(0),f(1),f(2),f(3),f(4),f(5),f(6))$ $ =(-1,0,0,-1,0,0,-1)$
And this lead us to test $ f(3p)=-1$, $ f(3p+1)=0$ and $ f(3p+2)=0$, which, indeed, is a solution.
So this is the unique solution in this paragraph ($ b=0$)

2.3) $ b=1$
------------
We got up to now : $ (f(0),f(1),f(2),f(3),f(4),f(5),f(7))$ $ =(-1,0,b,b^2-1,b^3-b^2-b,b^3-b,2b^3-2b)$ and so :
$ (f(0),f(1),f(2),f(3),f(4),f(5),f(7))$ $ =(-1,0,1,0,-1,0,0)$

$ P(3,-3)$ $ \implies$ $ f(0)+f(3)f(-3)=f(-8)$ and so $ f(8)=-1$
Using $ f(2n+2)=f(n+4)+f(n)f(4)-f(2n)f(2)$ with $ n=3$, we get $ f(8)=f(7)+f(3)f(4)-f(6)f(2)$ and so $ f(8)=-f(6)$
And so $ f(6)=1$

So $ (f(0),f(1),f(2),f(3),f(4),f(5),f(6))$ $ =(-1,0,1,0,-1,0,1)$
And it's obvious we must test the solution $ f(2p+1)=0$ and $ f(2p)=(-1)^{p+1}$, which, indeed, is a solution.
So this is the unique solution in this paragraph ($ b=1$)

2.4) $ b=3$
-------------
We got up to now : $ (f(0),f(1),f(2),f(3),f(4),f(5),f(7))$ $ =(-1,0,b,b^2-1,b^3-b^2-b,b^3-b,2b^3-2b)$ and so :
$ (f(0),f(1),f(2),f(3),f(4),f(5),f(7))$ $ =(-1,0,3,8,15,24,48)$

Using $ f(2n+2)=f(n+4)+f(n)f(4)-f(2n)f(2)$ with $ n=3$, we get $ f(8)=f(7)+f(3)f(4)-f(6)f(2)$ and so $ f(8)=168-3f(6)$ 
Using $ f(2n+2)=f(n+4)+f(n)f(4)-f(2n)f(2)$ with $ n=-4$, we get $ f(6)=f(0)+f(4)f(4)-f(8)f(2)$ and so $ f(6)=224-3f(8)$ 
From these two equalities, we get $ f(6)=35$ and $ f(8)=63$
Then $ f(0),f(1),f(2),f(3),f(4),f(5)$ and $ f(6)$ are all known and at most one function may get these values. And obviously we must test $ f(n)=n^2-1$, which, indeed, is a solution.
So this is the unique solution in this paragraph ($ b=3$)


3) Synthesis of solutions : We got $ 7$ different solutions :
============================================================
From introduction : $ f(x)=0$ $ \forall x$ : 
   $ (f(0),f(1),f(2),....)$ $ =(0,0,0,0,0,0,0,....)$

From 2.3 : $ f(2p)=(-1)^{p+1}$ and $ f(2p+1)=0$ : 
   $ (f(0),f(1),f(2),....)$ $ =(-1,0,1,0,-1,0,1,0,....)$

From 2.1 : $ f(2p)=-1$, $ f(2p+1)=0$ : 
   $ (f(0),f(1),f(2),....)$ $ =(-1,0,-1,0,-1,0,-1,0,....)$

From 2.2 : $ f(3p)=-1$, $ f(3p+1)=0$ and $ f(3p+2)=0$ : 
   $ (f(0),f(1),f(2),....)$ $ =(-1,0,0,-1,0,0,-1,0,0,....)$

From 1.1 : $ f(3p)=-1$, $ f(3p+1)=0$ and $ f(3p+2)=1$ : 
   $ (f(0),f(1),f(2),....)$ $ =(-1,0,1,-1,0,1,-1,0,1,....)$

From 1.2 : $ f(n)=n-1$ $ \forall n$

From 2.4 : $ f(n)=n^2-1$ $ \forall n$\end{tcolorbox}
How long did this take to write?
\end{solution}
*******************************************************************************
-------------------------------------------------------------------------------

\begin{problem}[Posted by \href{https://artofproblemsolving.com/community/user/57591}{KMK00009}]
	Find all functions $g: \mathbb R \to \mathbb R$ such that for all real numbers $x$ and $y$,
\[g(x+y)+g(x)g(y)=g(xy)+g(x)+g(y).\]
	\flushright \href{https://artofproblemsolving.com/community/c6h320016}{(Link to AoPS)}
\end{problem}



\begin{solution}[by \href{https://artofproblemsolving.com/community/user/40525}{bokagadha}]
	[hide="Solution"]
Plug in $ x=y=0$. We see that:
\[ g(0) + [g(0)]^2 = 3g(0)\]

Solving, we see that we have two cases: $ g(0) = 0$ or $ g(0) = 2$.

Case 1: $ g(0) = 2$

Plug in $ y = y$ and $ x = 0$. We see that:
\[ g(y) + 2g(y) = 2 + 2 + g(y)\]

Solving for that, we see that the only possible solution is that $ g(y) = 2$.

Case 2: $ g(0) = 0$

Plug in $ y=y$ and $ x=0$. We see that:
\[ g(y) + 0 = 0 + 0 + g(y)\]

Obviously, we see that this restriction that $ g(y) = g(y)$ holds for every function. So we see that as long as $ g(0) = 0$, any function $ g(x)$ is a solution to this functional equation.

To summarize, all of the solutions are: $ g(y) = 2$ and any function $ g(y)$ such that $ g(0) = 0$.
[\/hide]
\end{solution}



\begin{solution}[by \href{https://artofproblemsolving.com/community/user/70365}{Maharjun}]
	\begin{tcolorbox}
Plug in y=y and x=0. We see that:
$ g(y) + 0 = 0 + 0 + g(y)$\end{tcolorbox}

you are right when you say that this holds for every function.
however note that this is only the case when you have plugged x = 0 and y = y

if you plug x = 1 and y = y we get

$ \& g\left( {y + 1} \right) + g\left( y \right)g\left( 1 \right) = g\left( y \right) + g\left( y \right) + g\left( 1 \right) \\
\& \Rightarrow g\left( {y + 1} \right) + g\left( y \right)g\left( 1 \right) = 2\cdot\left( {g\left( y \right)} \right) + g\left( 1 \right)$

can you say that this is true for all functions $ g\left( {y} \right)$?.
remember, a function is required which satisfies the give condition for \begin{bolded}all real numbers $ x$ and $ y$\end{bolded}
\end{underlined}
hence your logic is incomplete and solution incorrect.

e.g. consider the function
\[ g\left( x \right) = \left\{ {
   6,\forall x \ne 0,  \\ 
   0,x = 0  \\ 
 } \right\}\]
$ \& g\left( 2 \right) + g\left( 1 \right)g\left( 1 \right) = 6 + 36 = 42  \\
  \& g\left( {1\cdot1} \right) + g\left( 1 \right) + g\left( 1 \right) = 6 + 6 + 6 = 18  \\
  \& LHS \ne RHS $
\end{solution}



\begin{solution}[by \href{https://artofproblemsolving.com/community/user/29428}{pco}]
	\begin{tcolorbox}Find all functions g:R->R such that for any real numbers x and y:
g(x+y)+g(x)g(y)=g(xy)+g(x)+g(y)\end{tcolorbox}
Let $ P(x,y)$ be the assertion $ f(x+y)+f(x)f(y)=f(xy)+f(x)+f(y)$
Let $ f(1)=a$

$ P(1,1)$ $ \implies$ $ f(2)=3a-a^2$
$ P(2,2)$ $ \implies$ $ f(2)^2=2f(2)$ and so $ (3a-a^2)^2=2(3a-a^2)$ $ \iff$ $ a(a-1)(a-2)(a-3)=0$ and so :

1) either $ a=0$
===========
Then $ P(x,1)$ $ \implies$ $ f(x+1)=2f(x)$
So $ P(x,y+1)$ $ \implies$ $ f(x+y+1)+f(x)f(y+1)=f(xy+x)+f(x)+f(y+1)$ and so $ 2f(x+y)+2f(x)f(y)=f(xy+x)+f(x)+2f(y)$
And since $ P(x,y)$ $ \implies$ $ 2f(x+y)+2f(x)f(y)=2f(xy)+2f(x)+2f(y)$, we get (subtracting these two equalities) :
$ 2f(xy)+f(x)=f(xy+x)$

Set $ x\ne 0$ and $ y=\frac 1x$ in this equality and we get $ f(x+1)=f(x)$ and, since we algeady got $ f(x+1)=2f(x)$ :
$ f(x)=0$ $ \forall x\ne 0$
Then $ P(1,-1)$ $ \implies$ $ f(0)=0$ and so $ f(x)=0$ $ \forall x$
Which, indeed, is a solution of our equation.


2) either $ a=1$
==========
Then  $ P(x,1)$ $ \implies$ $ f(x+1)=f(x)+1$ and so $ f(n)=n$ $ \forall n\in\mathbb Z$
Then $ P(x,y+1)$ $ \implies$  $ f(x+y+1)+f(x)f(y+1)=f(xy+x)+f(x)+f(y+1)$ and so $ f(x+y)+f(x)f(y)=f(xy+x)+f(y)$
Subtracting $ P(x,y)$ from this equality, we get $ f(xy+x)=f(xy)+f(x)$

Setting $ x=u\ne 0$ and $ y=\frac vu$ in this equality, we get $ f(u+v)=f(u)+f(v)$ $ \forall u\ne 0,v$ and, since $ f(0)=0$ :
$ f(x+y)=f(x)+f(y)$ $ \forall x,y$. Setting this in $ P(x,y)$, we get also :
$ f(xy)=f(x)f(y)$ 
and this is a classical functional equation whose unique solution is $ f(x)=x$ $ \forall x$
Which, indeed, is a solution of our equation.

3) either $ a=2$
==========
Then  $ P(x,1)$ $ \implies$ $ f(x+1)=2$ and so $ f(x)=2$ $ \forall x$
Which, indeed, is a solution of our equation.

3) either $ a=3$
==========
Then $ P(1,1)$ $ \implies$ $ f(2)=0$
Then $ P(x,1)$ $ \implies$ $ f(x+1)=3-f(x)$ and so $ f(x+2)=3-f(x+1)=f(x)$
But $ P(x,2)$ $ \implies$ $ f(x+2)=f(2x)+f(x)$
And so $ f(2x)=0$, which is impossible since, for example, $ f(2\frac 12)=3$


4) Synthesis of solutions
=========================
We got three solutions :
$ f(x)=0$ $ \forall x$
$ f(x)=2$ $ \forall x$
$ f(x)=x$ $ \forall x$
\end{solution}



\begin{solution}[by \href{https://artofproblemsolving.com/community/user/70365}{Maharjun}]
	\begin{tcolorbox}
Setting $ x=u\ne 0$ and $ y=\frac vu$ in this equality, we get $ f(u+v)=f(u)+f(v) \forall u\ne 0,v$ and, since $ f(0)=0$ : 
\end{tcolorbox}

some questions:
why put the extra condition of $ u\ne v$
why would $ f(0)=0$ have any implications. because, since u,v are completely random real numbers, hence the result $ f(x+y)=f(x)+f(y)$ $ \forall x,y$. should be true anyway. is it just to account for the case of $ f\left( {0 + 0} \right) = f\left( 0 \right) + f\left( 0 \right)$

if someone could kindly explain, id be most grateful
\end{solution}



\begin{solution}[by \href{https://artofproblemsolving.com/community/user/29428}{pco}]
	\begin{tcolorbox}[quote="pco"]
Setting $ x = u\ne 0$ and $ y = \frac vu$ in this equality, we get $ f(u + v) = f(u) + f(v) \forall u\ne 0,v$ and, since $ f(0) = 0$ : 
\end{tcolorbox}

some questions:
why put the extra condition of $ u\ne v$
why would $ f(0) = 0$ have any implications. because, since u,v are completely random real numbers, hence the result $ f(x + y) = f(x) + f(y)$ $ \forall x,y$. should be true anyway. is it just to account for the case of $ f\left( {0 + 0} \right) = f\left( 0 \right) + f\left( 0 \right)$

if someone could kindly explain, id be most grateful\end{tcolorbox}
1) There is no extra condition $ u\ne v$. I just wanted to write : $ \forall u\ne 0,\forall v$

2) We need $ u\ne 0$ in order to allow $ y=\frac vu$ And so we established $ f(u+v)=f(u)+f(v)$ only for $ u\ne 0$. Then, $ f(0)=0$ implies $ f(0+v)=f(0)+f(v)$ and so the equatlity $ f(u+v)=f(u)+f(v)$, which was first established only when $ u\ne 0$ is also true for $ u=0$ and so is true $ \forall u,v$
\end{solution}



\begin{solution}[by \href{https://artofproblemsolving.com/community/user/330078}{Delray}]
	\begin{tcolorbox}[quote="KMK00009"]Find all functions g:R->R such that for any real numbers x and y:
g(x+y)+g(x)g(y)=g(xy)+g(x)+g(y)\end{tcolorbox}
Let $ P(x,y)$ be the assertion $ f(x+y)+f(x)f(y)=f(xy)+f(x)+f(y)$
Let $ f(1)=a$

$ P(1,1)$ $ \implies$ $ f(2)=3a-a^2$
$ P(2,2)$ $ \implies$ $ f(2)^2=2f(2)$
\end{tcolorbox}
How do you get rid of $f(4)$?
\end{solution}



\begin{solution}[by \href{https://artofproblemsolving.com/community/user/29428}{pco}]
	\begin{tcolorbox}[quote=pco]Let $ P(x,y)$ be the assertion $ f(x+y)+f(x)f(y)=f(xy)+f(x)+f(y)$
...$ P(2,2)$ $ \implies$ $ f(2)^2=2f(2)$
\end{tcolorbox}
How do you get rid of $f(4)$?\end{tcolorbox}
$f(2+2)$ in LHS is cancelled by $f(2\times 2)$ in RHS



\end{solution}



\begin{solution}[by \href{https://artofproblemsolving.com/community/user/330078}{Delray}]
	\begin{tcolorbox}[quote=Delray]\begin{tcolorbox}Let $ P(x,y)$ be the assertion $ f(x+y)+f(x)f(y)=f(xy)+f(x)+f(y)$
...$ P(2,2)$ $ \implies$ $ f(2)^2=2f(2)$
\end{tcolorbox}
How do you get rid of $f(4)$?\end{tcolorbox}
$f(2+2)$ in LHS is cancelled by $f(2\times 2)$ in RHS\end{tcolorbox}

Thanks so much! Not sure why I missed that :)
\end{solution}
*******************************************************************************
-------------------------------------------------------------------------------

\begin{problem}[Posted by \href{https://artofproblemsolving.com/community/user/73589}{mathmen}]
	Find all function $f: \mathbb Q \to \mathbb Q$ such that for all rationals $x$ and $y$,
\[ \left(f(x)\right)^{f(y)}=\left(f(y)\right)^{f(x)}.\]
	\flushright \href{https://artofproblemsolving.com/community/c6h320076}{(Link to AoPS)}
\end{problem}



\begin{solution}[by \href{https://artofproblemsolving.com/community/user/29428}{pco}]
	\begin{tcolorbox}find all function $ f;\mathbb{Q} \to \mathbb{Q}$
such that

$ (\forall x,y \in \mathbb{Q})$; $ f(x)^{f(y)} = f(y)^{f(x)}$\end{tcolorbox}
The equation $ x^y = y^x$ with $ x,y\in\mathbb Q$ is well known and has the general solution :

either $ x = y$

either $ x = \left(\frac {n + 1}n\right)^n$ and $ y = \left(\frac {n + 1}n\right)^{n + 1}$ where $ n\in\mathbb Z^*$ and, since $ g(n) = \left(\frac {n + 1}n\right)^n$ is injective on $ \mathbb Z^*$, we have the solutions :

1) $ f(x) = c$ constant

2) Let $ A$ and $ B$ a split of $ \mathbb Q$ :
$ f(x) = \left(\frac {n + 1}n\right)^n$ $ \forall x\in A$ and $ f(x) = \left(\frac {n + 1}n\right)^{n + 1}$ $ \forall x\in B$, for any $ n\in\mathbb Z^*$
\end{solution}
*******************************************************************************
-------------------------------------------------------------------------------

\begin{problem}[Posted by \href{https://artofproblemsolving.com/community/user/73589}{mathmen}]
	Find all functions $f: \mathbb N \to \mathbb N$ such that for all positive integers $x$ and $y$,
\[f(x^2 + y^2) = f(x)^2 + f(y)^2.\]
	\flushright \href{https://artofproblemsolving.com/community/c6h320099}{(Link to AoPS)}
\end{problem}



\begin{solution}[by \href{https://artofproblemsolving.com/community/user/29428}{pco}]
	\begin{tcolorbox}find all function $ f;\mathbb{N}^* \to \mathbb{N}*$

such that $ \forall x;y \in \mathbb{N}^*$
$ f(x^2 + y^2) = f(x)^2 + f(y)^2$\end{tcolorbox}
Once again, IMHO, a crazy problem. It seems that, since some days, we get a lot of abnormal problems that I thought impossible to find in a real contest \/ book \/ class. And I wonder if all these problems are maybe just invented by posters.

Here is a solution I gave in another forum :
The difficulty here is the fact that the function is not defined for $ 0$ :

Let $ f(1)=a$
$ f(2)=f(1^2+1^2)=f(1)^2+f(1)^2=2a^2$
$ f(5)=f(2^2+1^2)=f(2)^2+f(1)^2=4a^4+a^2$

Computation of $ f(7)$ :
$ f(50)=f(7^2+1^2)$
$ f(50)=f(5^2+5^2)$
So $ f(7)^2+a^2=2(4a^4+a^2)^2$
and $ f(7)=\sqrt{2(4a^4+a^2)^2-a^2}$

$ f(8)=f(2^2+2^2)=f(2)^2+f(2)^2=8a^4$

$ f(10)=f(3^2+1^2)=f(3)^2+a^2$

$ f(13)=f(3^2+2^2)=f(3)^2+4a^4$

Computation of $ f(3)$ :
$ f(125)=f(10^2+5^2)=f(10)^2+f(5)^2$
$ f(125)=f(11^2+2^2)=f(11)^2+f(2)^2$
and so $ f(11)^2+4a^4=(f(3)^2+a^2)^2+(4a^4+a^2)^2$
so $ f(11)^2=(f(3)^2+a^2)^2+(4a^4+a^2)^2-4a^4$


$ f(170)=f(11^2+7^2)$
$ f(170)=f(13^2+1^2)$
and so $ f(11)^2+2(4a^4+a^2)^2-a^2=(f(3)^2+4a^4)^2+a^2$
So $ f(11)^2=(f(3)^2+4a^4)^2+2a^2-2(4a^4+a^2)^2$

Comparing the equation  $ f(125)$ with the equation $ f(170)$, we get :
$ (f(3)^2+a^2)^2+(4a^4+a^2)^2-4a^4=(f(3)^2+4a^4)^2+2a^2-2(4a^4+a^2)^2$
$ (f(3)^2+a^2)^2=(f(3)^2+4a^4)^2+4a^4+2a^2-3(4a^4+a^2)^2$
$ (4a^2-1)f(3)^2=16a^6+12a^4-1$
$ f(3)=2a^2+1$

Synthesis up to now :

$ f(1)=a$
$ f(2)=2a^2$
$ f(3)=2a^2+1$
$ f(5)=4a^4+a^2$
$ f(7)=\sqrt{2(4a^4+a^2)^2-a^2}$
$ f(8)=8a^4$
$ f(10)=4a^4+5a^2+1$
$ f(13)=12a^2+1$


$ f(65)=f(7^2+4^2)$
$ f(65)=f(8^2+1^2)$
and so $ f(8)^2+f(1)^2=f(7)^2+f(4)^2$
so $ 64a^8+a^2=2(4a^4+a^2)^2-a^2+f(4)^2$
and $ f(4)^2=64a^8+a^2-2(4a^4+a^2)^2+a^2$
$ f(4)=\sqrt{32a^8-16a^6-2a^4+2a^2}$


$ f(185)=f(11^2+8^2)$
$ f(185)=f(13^2+4^2)$
and so $ f(11)^2+f(8 )^2=f(13)^2+f(4)^2$
$ f(11)=\sqrt{-32a^8-16a^6+142a^4+26a^2+1}$

$ f(130)=f(9^2+7^2)$
$ f(130)=f(11^2+3^2)$
and so $ f(9)^2+f(7)^2=f(11)^2+f(3)^2$
so $ f(9)^2=-32a^8-16a^6+142a^4+26a^2+1+(2a^2+1)^2-2(4a^4+a^2)^2+a^2$
$ f(9)=\sqrt{-64a^8-32a^6+144a^4+31a^2+2}$

$ f(85)=f(6^2+7^2)$
$ f(85)=f(9^2+2^2)$
and so $ f(6)^2+f(7)^2=f(9)^2+f(2)^2$
$ f(6)=\sqrt{-96a^8-48a^6+146a^4+32a^2+2}$

$ f(200)=f(10^2+10^2)$
$ f(200)=f(14^2+2^2)$
and so $ f(14)^2+f(2)^2=2f(10)^2$
$ f(14)=\sqrt{2(4a^4+5a^2+1)^2-4a^4}$

And finally, the equation allowing to find $ a=f(1)$
==================================================
$ f(205)=f(13^2+6^2)$
$ f(205)=f(14^2+3^2)$
and so $ f(13)^2+f(6)^2=f(14)^2+f(3)^2$
$ (12a^2+1)^2-96a^8-48a^6+146a^4+$ $ 32a^2+2=2(4a^4+5a^2+1)^2-4a^4+(2a^2+1)^2$
$ 144a^4+24a^2+1-96a^8-48a^6+146a^4$ $ +32a^2+2=32a^8+50a^4+2+80a^6+$ $ 16a^4+20a^2-4a^4+4a^4+4a^2+1$
$ 128a^8+128a^6-220a^4-36a^2=0$
$ 32a^6+32a^4-55a^2-9=0$
$ (a^2-1)(32a^4+64a^2+9)=0$
And so $ a=1$, unique positive integer root of this equation.

This implies $ f(n)=n$ $ \forall n\in\{1,2,3,4,5,6,7,8,9,10,11,13,14\}$

Now, we can use the equalities  :
$ (2n-1)^2+(n-3)^2=(2n-3)^2+(n+1)^2$
$ (2n)^2+(n-5)^2=(2n-4)^2+(n+3)^2$

Which allow to find $ f(n)$ for any $ n$ as soon as we know $ f(n)$ for $ n\in\{1,2,3,4,5,6\}$

And so $ \boxed{f(n)=n}$ $ \forall n$
\end{solution}



\begin{solution}[by \href{https://artofproblemsolving.com/community/user/43536}{nguyenvuthanhha}]
	\begin{tcolorbox} and so $ f(13)^2 + f(6)^2 = f(14)^2 + f(3)^2$
$ (12a^2 + 1)^2 - 96a^8 - 48a^6 + 146a^4 +$ $ 32a^2 + 2 = 2(4a^4 + 5a^2 + 1)^2 - 4a^4 + (2a^2 + 1)^2$


  \end{tcolorbox}

\begin{italicized}   Very brilliant idea    but I think in that step , it should be :

   $ ( 8a^4 + 4 a^2 + 1)^2 - 96a^8 - 48a^6 + 146a^4 +$ $ 32a^2 + 2 = 2(4a^4 + 5a^2 + 1)^2 - 4a^4 + (2a^2 + 1)^2$

   Because if you can prove $ f(13) = 12a^2 + 1$ as you said in the previous step than :

    $ f(13) = 12a^2 + 1 = (f(3))^2 + 4a^4 = 8a^4 + 4a^2 + 1$ , so it's really trivial that $ a = 1$
  
    Am I right ?
            
   \end{italicized}
\end{solution}



\begin{solution}[by \href{https://artofproblemsolving.com/community/user/29428}{pco}]
	\begin{tcolorbox}[quote="pco"] and so $ f(13)^2 + f(6)^2 = f(14)^2 + f(3)^2$
$ (12a^2 + 1)^2 - 96a^8 - 48a^6 + 146a^4 +$ $ 32a^2 + 2 = 2(4a^4 + 5a^2 + 1)^2 - 4a^4 + (2a^2 + 1)^2$


  \end{tcolorbox}

\begin{italicized}   Very brilliant idea    but I think in that step , it should be :

   $ ( 8a^4 + 4 a^2 + 1)^2 - 96a^8 - 48a^6 + 146a^4 +$ $ 32a^2 + 2 = 2(4a^4 + 5a^2 + 1)^2 - 4a^4 + (2a^2 + 1)^2$

   Because if you can prove $ f(13) = 12a^2 + 1$ as you said in the previous step than :

    $ f(13) = 12a^2 + 1 = (f(3))^2 + 4a^4 = 8a^4 + 4a^2 + 1$ , so it's really trivial that $ a = 1$
  
    Am I right ?
            
   \end{italicized}\end{tcolorbox}

AAAaaaaaaaargh ! D**** !
I dont know from where I got $ f(13) = 12a^2 + 1$ and so the good value is  $ f(13) = (f(3))^2 + 4a^4 = 8a^4 + 4a^2 + 1$ and we dont have equation here.

But this error crashed down all my computations :

This modifies the computation of $ f(11)$ which becomes $ f(11) = \sqrt {32a^8 + 48a^6 + 30a^4 + 10a^2 + 1}$

Which modifies then the computation of $ f(9)$ which becomes $ f(9) = \sqrt {32a^6 + 32a^4 + 15a^2 + 2}$

Which modifies then the computation of $ f(6)$ which becomes $ f(6) = \sqrt { - 32a^8 + 16a^6 + 34a^4 + 16a^2 + 2}$

And then the final equation $ f(13)^2 + f(6)^2 = f(14)^2 + f(3)^2$ becomes ... an identity :(

And so I no longer got $ a$ :( 

Thanks for your careful reading. I'll edit my post and try to find $ a$ in another way. \begin{bolded}edited \end{bolded}\end{underlined}: sorry, but my original post is too old to be edited.
What a pity :)
\end{solution}



\begin{solution}[by \href{https://artofproblemsolving.com/community/user/29428}{pco}]
	I got it !

We obtained $ f(6) = \sqrt { - 32a^8 + 16a^6 + 34a^4 + 16a^2 + 2}$

And it is very easy to see that  $ - 32a^8 + 16a^6 + 34a^4 + 16a^2 + 2 < 0$ $ \forall a\ge 2$ and so the only solution may be $ a = 1$

And so a correct (I hope) solution :
\begin{tcolorbox}find all function $ f;\mathbb{N}^* \to \mathbb{N}*$

such that $ \forall x;y \in \mathbb{N}^*$
$ f(x^2 + y^2) = f(x)^2 + f(y)^2$\end{tcolorbox}
Once again, IMHO, a crazy problem. It seems that, since some days, we get a lot of abnormal problems that I thought impossible to find in a real contest \/ book \/ class. And I wonder if all these problems are maybe just invented by posters.

Here is a solution I gave in another forum :
The difficulty here is the fact that the function is not defined for $ 0$ :

Let $ f(1) = a$
$ f(2) = f(1^2 + 1^2) = f(1)^2 + f(1)^2 = 2a^2$
$ f(5) = f(2^2 + 1^2) = f(2)^2 + f(1)^2 = 4a^4 + a^2$

Computation of $ f(7)$ :
$ f(50) = f(7^2 + 1^2)$
$ f(50) = f(5^2 + 5^2)$
So $ f(7)^2 + a^2 = 2(4a^4 + a^2)^2$
and $ f(7) = \sqrt {2(4a^4 + a^2)^2 - a^2}$

$ f(8) = f(2^2 + 2^2) = f(2)^2 + f(2)^2 = 8a^4$

$ f(10) = f(3^2 + 1^2) = f(3)^2 + a^2$

$ f(13) = f(3^2 + 2^2) = f(3)^2 + 4a^4$

Computation of $ f(3)$ :
$ f(125) = f(10^2 + 5^2) = f(10)^2 + f(5)^2$
$ f(125) = f(11^2 + 2^2) = f(11)^2 + f(2)^2$
and so $ f(11)^2 + 4a^4 = (f(3)^2 + a^2)^2 + (4a^4 + a^2)^2$
so $ f(11)^2 = (f(3)^2 + a^2)^2 + (4a^4 + a^2)^2 - 4a^4$


$ f(170) = f(11^2 + 7^2)$
$ f(170) = f(13^2 + 1^2)$
and so $ f(11)^2 + 2(4a^4 + a^2)^2 - a^2 = (f(3)^2 + 4a^4)^2 + a^2$
So $ f(11)^2 = (f(3)^2 + 4a^4)^2 + 2a^2 - 2(4a^4 + a^2)^2$

Comparing the equation  $ f(125)$ with the equation $ f(170)$, we get :
$ (f(3)^2 + a^2)^2 + (4a^4 + a^2)^2 - 4a^4 = (f(3)^2 + 4a^4)^2 + 2a^2 - 2(4a^4 + a^2)^2$
$ (f(3)^2 + a^2)^2 = (f(3)^2 + 4a^4)^2 + 4a^4 + 2a^2 - 3(4a^4 + a^2)^2$
$ (4a^2 - 1)f(3)^2 = 16a^6 + 12a^4 - 1$
$ f(3) = 2a^2 + 1$

Synthesis up to now :

$ f(1) = a$
$ f(2) = 2a^2$
$ f(3) = 2a^2 + 1$
$ f(5) = 4a^4 + a^2$
$ f(7) = \sqrt {2(4a^4 + a^2)^2 - a^2}$
$ f(8) = 8a^4$
$ f(10) = 4a^4 + 5a^2 + 1$
$ f(13) = (f(3))^2 + 4a^4 = 8a^4 + 4a^2 + 1$ 


$ f(65) = f(7^2 + 4^2)$
$ f(65) = f(8^2 + 1^2)$
and so $ f(8)^2 + f(1)^2 = f(7)^2 + f(4)^2$
so $ 64a^8 + a^2 = 2(4a^4 + a^2)^2 - a^2 + f(4)^2$
and $ f(4)^2 = 64a^8 + a^2 - 2(4a^4 + a^2)^2 + a^2$
$ f(4) = \sqrt {32a^8 - 16a^6 - 2a^4 + 2a^2}$


$ f(185) = f(11^2 + 8^2)$
$ f(185) = f(13^2 + 4^2)$
and so $ f(11)^2 + f(8 )^2 = f(13)^2 + f(4)^2$
$ f(11) = \sqrt {32a^8 + 48a^6 + 30a^4 + 10a^2 + 1}$

$ f(130) = f(9^2 + 7^2)$
$ f(130) = f(11^2 + 3^2)$
and so $ f(9)^2 + f(7)^2 = f(11)^2 + f(3)^2$
so $ f(9)^2 = - 32a^8 - 16a^6 + 142a^4 + 26a^2 + 1 + (2a^2 + 1)^2 - 2(4a^4 + a^2)^2 + a^2$
$ f(9) = \sqrt {32a^6 + 32a^4 + 15a^2 + 2}$


$ f(85) = f(6^2 + 7^2)$
$ f(85) = f(9^2 + 2^2)$
and so $ f(6)^2 + f(7)^2 = f(9)^2 + f(2)^2$
$ f(6) = \sqrt { - 32a^8 + 16a^6 + 34a^4 + 16a^2 + 2}$

And it is very easy to see that  $ - 32a^8 + 16a^6 + 34a^4 + 16a^2 + 2 < 0$ $ \forall a\ge 2$ and so the only solution may be $ a = 1$

This implies $ f(n) = n$ $ \forall n\in\{1,2,3,4,5,6,7,8,9,10,11,13\}$

Now, we can use the equalities  :
$ (2n - 1)^2 + (n - 3)^2 = (2n - 3)^2 + (n + 1)^2$
$ (2n)^2 + (n - 5)^2 = (2n - 4)^2 + (n + 3)^2$

Which allow to find $ f(n)$ for any $ n$ as soon as we know $ f(n)$ for $ n\in\{1,2,3,4,5,6\}$

And so $ \boxed{f(n) = n}$ $ \forall n$
\end{solution}



\begin{solution}[by \href{https://artofproblemsolving.com/community/user/71459}{x164}]
	Just a question:
What's the definition of $ \mathbb{N}^\ast$?

I've always thought that
\[ \mathbb{N} = \{ 1,2,3,\ldots \} = \mathbb{Z}_{>0}\] and
\[ \mathbb{N}^\ast = \{ 0,1,2,3,\ldots \} = \mathbb{Z}_{\geq 0}\]
But according to pco $ 0 \not\in \mathbb{N}^\ast$. Could you please explain this?
\end{solution}



\begin{solution}[by \href{https://artofproblemsolving.com/community/user/29428}{pco}]
	\begin{tcolorbox}Just a question:
What's the definition of $ \mathbb{N}^\ast$?

I've always thought that
\[ \mathbb{N} = \{ 1,2,3,\ldots \} = \mathbb{Z}_{ > 0}\]
and
\[ \mathbb{N}^\ast = \{ 0,1,2,3,\ldots \} = \mathbb{Z}_{\geq 0}\]
But according to pco $ 0 \not\in \mathbb{N}^\ast$. Could you please explain this?\end{tcolorbox}

As far as I know :

In some countries, (France, Morocco, ... for example), $ 0\in\mathbb N$ and $ \mathbb N^*=\mathbb N\backslash \{0\}$

In most other countries (and on Mathlink), $ 0\notin\mathbb N$ and $ \mathbb N_0=\mathbb N\cup\{0\}$. I did not think that $ \mathbb N^*$ was used in these countries.

I saw for the first time this problem in a forum in Morocco and this was the reason for which I supposed this was the first explanation.
\end{solution}
*******************************************************************************
-------------------------------------------------------------------------------

\begin{problem}[Posted by \href{https://artofproblemsolving.com/community/user/61709}{mkn19}]
	Find all functions $f: \mathbb R \to \mathbb R$ such that for all reals $x$ and $y$, 
\[f(x^2-y^2)=xf(x)-yf(y).\]
	\flushright \href{https://artofproblemsolving.com/community/c6h320299}{(Link to AoPS)}
\end{problem}



\begin{solution}[by \href{https://artofproblemsolving.com/community/user/29428}{pco}]
	\begin{tcolorbox}Find all real valued functions defined on the reals such that for every real x,y: $ f(x^2 - y^2) = xf(x) - yf(y)$\end{tcolorbox}
Let $ P(x,y)$ be the assertion $ f(x^2-y^2)=xf(x)-yf(y)$

$ P(0,0)$ $ \implies$ $ f(0)=0$
$ P(x,0)$ $ \implies$ $ f(x^2)=xf(x)$ and so $ P(x,y)$ becomes $ f(x^2-y^2)=f(x^2)-f(y^2)$
$ P(0,x)$ $ \implies$ $ f(-x^2)=-xf(x)=-f(x^2)$ 
These two lines imply that $ f(x+y)=f(x)+f(y)$ $ \forall x,y$

Then $ P(x+1,0)$ $ \implies$ $ f(x^2+2x+1)=(x+1)f(x+1)$ so $ f(x^2)+2f(x)+f(1)=xf(x)+f(x)+xf(1)+f(1)$ and, since $ f(x^2)=xf(x)$, we get : $ f(x)=xf(1)$
And it is immediate to check back that this indeed is a solution.

Hence the general solution : $ \boxed{f(x)=ax}$
\end{solution}



\begin{solution}[by \href{https://artofproblemsolving.com/community/user/101044}{erfan_Ashorion}]
	we know that f(x+y)=f(x)+f(y) by the proof up;)!!so we can use coshi function and see that f(x)=cx!!sorry for my bad english!!
\end{solution}



\begin{solution}[by \href{https://artofproblemsolving.com/community/user/72731}{goodar2006}]
	\begin{tcolorbox}we know that f(x+y)=f(x)+f(y) by the proof up;)!!so we can use coshi function and see that f(x)=cx!!sorry for my bad english!!\end{tcolorbox}

no, it's not true. if for a function $f$ we have $f(x+y)=f(x)+f(y)$ and it's \begin{bolded}continuous\end{bolded} then we can apply cauchy's theorem to state that $f(x)=ax$ for all real variables $x$.
\end{solution}



\begin{solution}[by \href{https://artofproblemsolving.com/community/user/66201}{basketball9}]
	Also from f(x)=0 and xf(x)=f(x^2) we can divide by x^2 to and have g(y)=f(y)\/y, but since g(x)=g(x^2) g is a constant say a and thus g(x)=a=f(x)\/x and therefore f(x)=ax
\end{solution}



\begin{solution}[by \href{https://artofproblemsolving.com/community/user/95245}{CPT_J_H_Miller}]
	Sorry to revive this topic, but can someone please explain why we can conclude from $ g(x) = g(x^2)$ that $ g $ is a constant?
\end{solution}



\begin{solution}[by \href{https://artofproblemsolving.com/community/user/233377}{souheibsolver}]
	These two lines imply that $ f(x+y)=f(x)+f(y) \forall x,y $

here you've proved it just when one of $ x $ and $ y $ is negative and the other is positive
you haven't done it when they're from the same signal
\end{solution}



\begin{solution}[by \href{https://artofproblemsolving.com/community/user/29428}{pco}]
	\begin{tcolorbox}These two lines imply that $ f(x+y)=f(x)+f(y) \forall x,y $

here you've proved it just when one of $ x $ and $ y $ is negative and the other is positive
you haven't done it when they're from the same signal\end{tcolorbox}
Read more carefully. The second line of "these two lines" prove that $f(x)$ is odd.So ....
\end{solution}



\begin{solution}[by \href{https://artofproblemsolving.com/community/user/163459}{SMOJ}]
	Why does this problem look so familiar? http://www.artofproblemsolving.com/Forum/viewtopic.php?p=337857&sid=fb2d18932f71cab5bf436781078d7a5d#p337857
\end{solution}



\begin{solution}[by \href{https://artofproblemsolving.com/community/user/231508}{john10}]
	Sorry to revive . @ Pco can you explain more how did you get cauchy's equation . :blush: 
These two lines imply that $ f(x+y)=f(x)+f(y)$ $ \forall x,y$


\end{solution}



\begin{solution}[by \href{https://artofproblemsolving.com/community/user/29428}{pco}]
	\begin{tcolorbox}Sorry to revive . @ Pco can you explain more how did you get cauchy's equation . :blush: 
These two lines imply that $ f(x+y)=f(x)+f(y)$ $ \forall x,y$\end{tcolorbox}

Let $Q(x,y)$ be the assertion $f(x^2-y^2)=f(x^2)-f(y^2)$

Let $x\ge 0$ and $y\ge 0$ : $Q(\sqrt{x+y},\sqrt y)$ $\implies$ $f(x+y)=f(x)+f(y)$
Let $x\ge 0$ and $y\le 0$ : $Q(\sqrt x,\sqrt{-y})$ $\implies$ $f(x+y)=f(x)-f(-y)=f(x)+f(y)$
Let $x\le 0$ and $y\le 0$ : $Q(\sqrt{-x-y},\sqrt {-y})$ $\implies$ $f(x+y)=f(x)+f(y)$
Let $x\le 0$ and $y\ge 0$ : $Q(\sqrt y,\sqrt{-x})$ $\implies$ $f(x+y)=f(y)-f(-x)=f(x)+f(y)$

Hence the claim

\end{solution}



\begin{solution}[by \href{https://artofproblemsolving.com/community/user/221848}{yassino}]
	\begin{tcolorbox}[quote=john10]Sorry to revive . @ Pco can you explain more how did you get cauchy's equation . :blush: 
These two lines imply that $ f(x+y)=f(x)+f(y)$ $ \forall x,y$\end{tcolorbox}

Let $Q(x,y)$ be the assertion $f(x^2-y^2)=f(x^2)-f(y^2)$

Let $x\ge 0$ and $y\ge 0$ : $Q(\sqrt{x+y},\sqrt y)$ $\implies$ $f(x+y)=f(x)+f(y)$
Let $x\ge 0$ and $y\le 0$ : $Q(\sqrt x,\sqrt{-y})$ $\implies$ $f(x+y)=f(x)-f(-y)=f(x)+f(y)$
Let $x\le 0$ and $y\le 0$ : $Q(\sqrt{-x-y},\sqrt {-y})$ $\implies$ $f(x+y)=f(x)+f(y)$
Let $x\le 0$ and $y\ge 0$ : $Q(\sqrt y,\sqrt{-x})$ $\implies$ $f(x+y)=f(y)-f(-x)=f(x)+f(y)$

Hence the claim\end{tcolorbox}
@Pco , in the seconde case , although y is negative you substitute it by $ \sqrt{-y} $ ( wich is positive) .Similary in the third and last case ? Can you explain please 
\end{solution}



\begin{solution}[by \href{https://artofproblemsolving.com/community/user/29428}{pco}]
	\begin{tcolorbox}@Pco , in the seconde case , although y is negative you substitute it by $ \sqrt{-y} $ ( wich is positive) .Similary in the third and last case ? Can you explain please\end{tcolorbox}
$Q(x,y)$ is true $\forall x,y\in\mathbb R$, positive or negative.

So the only caution is to be sure that the square roots are defined ...
And they are in each subcase.


\end{solution}



\begin{solution}[by \href{https://artofproblemsolving.com/community/user/238224}{Verit}]
	Sorry to revive yet again... but what about $f(x) = 0 \, \forall x \in \mathbb{R}?$
\end{solution}



\begin{solution}[by \href{https://artofproblemsolving.com/community/user/238224}{Verit}]
	It did not say that $f$ has to be injective.
\end{solution}



\begin{solution}[by \href{https://artofproblemsolving.com/community/user/155347}{Wave-Particle}]
	\begin{tcolorbox}Sorry to revive yet again... but what about $f(x) = 0 \, \forall x \in \mathbb{R}?$\end{tcolorbox}

That works and \begin{bolded}pco\end{bolded} addressed that function in his solution.

$f(x)=ax$, let $a=0$ and we have $f(x)=0$.
\end{solution}



\begin{solution}[by \href{https://artofproblemsolving.com/community/user/29428}{pco}]
	\begin{tcolorbox}we know that f(x+y)=f(x)+f(y) by the proof up;)!!so we can use coshi function and see that f(x)=cx!!sorry for my bad english!!\end{tcolorbox}

As said thousands of times : Cauchy does not allow to directly conclude linearity from additivity.
Some conditions more are needed (mononotne, continuous, locally lowerbounded or upperbounded, ...)

In this example, $f(-x^2)=-f(x^2)$ shows that function is upperbounded over $\mathbb R^-$ and it would have been sufficient to conclude linearity. I prefered the elementary two lines proof I gave.


\end{solution}



\begin{solution}[by \href{https://artofproblemsolving.com/community/user/391072}{akmathworld}]
	Why do we follow all the time that f(x+y)=f(x)+f(y) and f(0)=0??
\end{solution}
*******************************************************************************
-------------------------------------------------------------------------------

\begin{problem}[Posted by \href{https://artofproblemsolving.com/community/user/63660}{Victory.US}]
	Determine all continuous functions $f: \mathbb R \to \mathbb R$ such that
\[ f(x + f(y)) = f(x) +y^n\]
for all $x,y \in \mathbb R$ and all positive integers $n$.
	\flushright \href{https://artofproblemsolving.com/community/c6h320304}{(Link to AoPS)}
\end{problem}



\begin{solution}[by \href{https://artofproblemsolving.com/community/user/43269}{JoeBlow}]
	Are you sure there are solutions for any $ n$?  For even $ n$, for example, unless I've made some mistake we seem to get a quick contradiction:

Taking $ x=0$ implies that $ f$ is unbounded on $ \mathbb{R}$.  Taking $ y=0$ implies either $ f$ is periodic or $ f(0)=0$, but a continuous periodic function cannot be unbounded, so $ f(0)=0$ and $ f(f(y))=y^n$.  Take $ x=-f(y)$, then $ f(-f(y))=-y^n=-f(f(y))$, so $ f$ is surjective and $ -f(x)=f(-x)$ for all $ x\in \mathbb{R}$.  But if $ n$ is even, then $ f(-x-f(y))=f(-x)+(-y)^n=-f(x)+|y|^n\neq -f(x)-|y|^n=-f(x+f(y))$, contradiction if $ y\neq 0$.
\end{solution}



\begin{solution}[by \href{https://artofproblemsolving.com/community/user/29428}{pco}]
	\begin{tcolorbox}Determine all function $ f$ defined and continuous on $ R$ such that
$ f(x + f(y)) = f(x) + y^n$, $ \forall x,y \in R; n \in N^*$\end{tcolorbox}

Let $ P(x,y)$ be the assertion $ f(x + f(y)) = f(x) + y^n$

As Joeblow said : If $ f(0)\ne 0$, $ f(x)$ is continuous, unbounded (see $ \lim_{y\to+\infty}f(x+f(y))$) and periodic (see $ P(x,0)$), which is impossible. So $ f(0)=0$

Then $ P(0,x)$ $ \implies$ $ f(f(x))=x^n$ and so $ f(x^n)=f(x)^n$
$ P(x^n,f(y))$ $ \implies$ $ f(x^n+y^n)=f(x^n)+f(y^n)$ and so $ f(x+y)=f(x)+f(y)$ $ \forall x,y\ge 0$

This last equation implies immediately $ f(x)=f(1)x$ $ \forall$ non-negative rational $ x$ and continuity gives $ f(x)=ax$ $ \forall x\ge 0$

Using then $ y\ge 0$ and $ x\ge\max(0,-ay)$, $ P(x,y)$ implies $ a(x+ay)=ax+y^n$ and so $ a^2=1$ and $ n=1$ and it is easy to see that $ x$ and $ -x$ are indeed solutions.

So, there exist solutions only for $ n=1$ and then the only two solutions are $ f(x)=x$ $ \forall x$ and $ f(x)=-x$ $ \forall x$
\end{solution}
*******************************************************************************
-------------------------------------------------------------------------------

\begin{problem}[Posted by \href{https://artofproblemsolving.com/community/user/69938}{imedeen}]
	Determine all continuous functions $f: \mathbb R \to \mathbb R$ such that
\[ f(x + y + xy) = f(x) + f(y) + f(xy)\]
happens for all $x,y \in \mathbb R$.
	\flushright \href{https://artofproblemsolving.com/community/c6h320307}{(Link to AoPS)}
\end{problem}



\begin{solution}[by \href{https://artofproblemsolving.com/community/user/29428}{pco}]
	\begin{tcolorbox}Determine all function $ f$ defined and continuous on $ R$ such that
$ f(x + y + xy) = f(x) + f(y) + f(xy)$ , $ \forall x,y \in R$\end{tcolorbox}

I think some shorter method might exist :

Let $ P(x,y)$ be the assertion $ f(x+y+xy)=f(x)+f(y)+f(xy)$

$ P(0,0)$ $ \implies$ $ f(0)=0$
$ P(x,-1)$ $ \implies$ $ f(-x)=-f(x)$
$ P(1,1)$ $ \implies$ $ f(3)=3f(1)$
$ P(x,1)$ $ \implies$ $ f(2x+1)=2f(x)+f(1)$

$ P(x,3)$ $ \implies$ $ f(4x+3)=f(x)+f(3)+f(3x)=f(3x)+3f(1)+f(x)$
$ P(2x+1,1)$ $ \implies$ $ f(4x+3)=2f(2x+1)+f(1)$ $ =4f(x)+3f(1)$
Comparing these two lines, we get $ f(3x)=3f(x)$

$ P(x,-3)$ $ \implies$ $ f(-3-2x)=f(x)+f(-3)+f(-3x)$ and so $ f(2x+3)=2f(x)+3f(1)$ $ =(2f(x)+f(1))+2f(1)$ $ =f(2x+1)+2f(1)$

So $ f(x+2)=f(x)+2f(1)$

$ P(1,-2)$ $ \implies$ $ f(-3)=f(1)+f(-2)+f(-2)$ and so $ f(2)=2f(1)$
$ P(x,-2)$ $ \implies$ $ f(-2-x)=f(x)+f(-2)+f(-2x)$ and so $ f(x+2)=f(2x)-f(x)+2f(1)$ and, since $ f(x+2)=f(x)+2f(1)$, we get : $ f(2x)=2f(x)$

So $ f(x+1)=f(x)+f(1)$ and $ f(x+n)=f(x)+nf(1)$ and $ f(n)=nf(1)$

Then $ P(x,n)$ $ \implies$ $ f((n+1)x)=f(x)+f(nx)$ and so $ f(nx)=nf(x)$

This immediately gives $ f(x)=f(1)x$ $ \forall x\in\mathbb Q$ and continuity implies then $ \boxed{f(x)=ax}$ $ \forall x$
And it is easy to check back that this indeed is a solution.
\end{solution}



\begin{solution}[by \href{https://artofproblemsolving.com/community/user/44887}{Mathias_DK}]
	\begin{tcolorbox}Determine all function $ f$ defined and continuous on $ R$ such that
$ f(x + y + xy) = f(x) + f(y) + f(xy)$ , $ \forall x,y \in R$\end{tcolorbox}
As in pco's solution let $ P(x,y)$ be the assertion that $ f(x+y+xy)=f(x)+f(y)+f(xy)$.
$ P(0,0): f(0)=0$
$ P(x,-x): f(x) = -f(-x)$
$ P(1,1): f(3)=3f(1)$

$ P(1,y): f(1+2y) = f(1)+2f(y)$
$ P(3,y): f(3+4y) = f(3)+f(y)+f(3y) = 3f(1)+f(y)+f(3y)$
$ P(1,2y+1): f(3+4y) = f(1) + 2f(2y+1)$ $ = f(1) + 2(f(1)+2f(y)) = 3f(1)+4f(y)$ and hence $ f(3y)=3f(y)$

$ P(3,1): f(7) = f(3)+f(1)+f(3) = 7f(1)$
$ P(7,y): f(7+8y) = f(7) + f(y) + f(7y)$

$ P(1,4y+3): f(7+8y) = f(1) + 2f(4y+3)$ $ = f(1) + 2(3f(1)+4f(y)) = 7f(1)+8f(y)$ and hence $ f(7y) = 7f(y)$.

Now we have $ f(0)=0, f(x)=-f(-x), f(3x)=3f(x), f(7x) = 7f(x)$, and this is enough to prove that $ f(x) = ax$ for some $ a \in \mathbb{R}$. Let $ f(1)=a$. Since $ f(0)=0$ and the rest of the function is uniquely defined by $ f(x), x \in [1;3)$ it is enough to prove that $ f(x) = ax \forall x \in [1;3)$.

From $ f(3x)=3f(x), f(7x) = 7f(x)$ we easily get $ f(3^nx)=3^nf(x), n \in \mathbb{Z}$ and $ f(7^nx) = 7^nf(x), n \in \mathbb{Z}$.

From $ f(3^nx)=3^nf(x)$ we get: $ f(x) = 3^{\left \lfloor \log_3(x) \right \rfloor} f( x 3^{-\left \lfloor \log_3(x) \right \rfloor} )$.

Since $ f(7^n) = 7^n f(1) = 7^n a$ we have $ 7^n a = f(7^n) = 3^{\left \lfloor \log_3(7^n) \right \rfloor} f( 7^n \cdot 3^{-\left \lfloor \log_3(7^n) \right \rfloor} )$.

Let $ g(n) = \frac{7^n}{3^{\left \lfloor \log_3(7^n) \right \rfloor}}$. Then $ f(g(n)) = ag(n)$. Let $ h(n) = \log_3(g(n)) \iff$

$ h(n) = \log_3(7) \cdot n -\left \lfloor \log_3(7) \cdot n \right \rfloor$. Let $ \alpha = \log_3(7)$ which is clearly irrational, then $ h(n) = \alpha n - \left \lfloor \alpha n \right \rfloor = \{ \alpha n \}$, where $ \{x\} = x - \lfloor x \rfloor$ denotes the fractional part of $ x$.

Since $ \alpha$ is irrational we can easily prove that in any interval $ (a;b) \subset (0;1)$ there exists a $ n$ such that $ h(n) \in (a;b)$. So for any $ (a;b) \in (1;3)$ there exists $ n$ such that $ g(n) \in (a;b)$. Because of continouity we find that $ f(x) = ax, 1 < x < 3$, and hence $ f(x) = ax \forall x$ using $ f(3x)=3f(x)$ and $ f(x)=-f(-x)$. So $ f(x) = ax \forall x \in \mathbb{R}$ is the only solution - and it is easily seen to be a solution.
\end{solution}
*******************************************************************************
-------------------------------------------------------------------------------

\begin{problem}[Posted by \href{https://artofproblemsolving.com/community/user/63660}{Victory.US}]
	Let $ c > 0$ . Determine all functions function $f: \mathbb R \to \mathbb R$ which satisfy
 i) $f(x + y) = f(x)\cdot f(y)$ for all $x,y \in \mathbb R$, and
 ii) $|f(x)| \le c$ for all $x \in [ -1,1]$.
	\flushright \href{https://artofproblemsolving.com/community/c6h320308}{(Link to AoPS)}
\end{problem}



\begin{solution}[by \href{https://artofproblemsolving.com/community/user/29428}{pco}]
	\begin{tcolorbox}Let $ c > 0$ . Determine function $ f$ satisfy
 $ i) f(x + y) = f(x).f(y), \forall x,y \in R$
 $ ii)|f(x)| \le c, \forall x \in [ - 1,1]$\end{tcolorbox}

1) If $ \exists a$ such that $ f(a)=0$, then $ f((x-a)+a)=f(x-a)f(a)=0$ and we get $ f(x)=0$ $ \forall x$, which, indeed, is a solution.

2) If $ f(x)\ne 0$ $ \forall x$, let $ g(x)=\ln(|f(x)|)$. We got $ g(x+y)=g(x)+g(y)$ and $ g(x)$ is a solution of Cauchy equation.
Let then $ a\in\mathbb R$ and $ x\in[a-1,a+1]$. We get $ f(a)=f(x)f(a-x)$ and so $ |f(a)|\le c|f(x)|$ and so $ g(x)>g(a)-\ln(c)$ $ \forall x\in[a-1,a+1]$
So $ g(x)$ has a lower limit on a non empty open interval and so is continuous (classical result of Cauchy equation) and must be $ \lambda x$

So $ |f(x)|=e^{\lambda x}$ and since $ f(x)=f(\frac x2+\frac x2)=f(\frac x2)^2>0$, we get $ f(x)=e^{\lambda x}$

Plugging back in initial requirements we get that this is a solution as soon as $ |\lambda|\le \ln c$

Hence the answer :

If $ c<1$ the only solution is $ f(x)=0$
If $ c=1$ the only solutions are $ f(x)=0$ and $ f(x)=1$
If $ c>1$ the solutions are $ f(x)=0$ and $ f(x)=e^{\lambda x}$ where $ |\lambda|\le \ln c$
\end{solution}
*******************************************************************************
-------------------------------------------------------------------------------

\begin{problem}[Posted by \href{https://artofproblemsolving.com/community/user/73589}{mathmen}]
	Find all functions $f: \mathbb R \to \mathbb R$ such that \[ f( xf(x) + f(y) ) = f(x)^2 + y \] for all $x,y\in \mathbb R$.
	\flushright \href{https://artofproblemsolving.com/community/c6h320331}{(Link to AoPS)}
\end{problem}



\begin{solution}[by \href{https://artofproblemsolving.com/community/user/29428}{pco}]
	\begin{tcolorbox}Find all real valued functions defined on the reals such that for every real $ x,y$: 

$ f(xf(x) + f(y)) = f(x)^2 + y$\end{tcolorbox}

Let $ P(x,y)$ be the assertion $ f(xf(x)+f(y))=f(x)^2+y$

$ P(0,x)$ $ \implies$ $ f(f(x))=x+f(0)^2$ and so $ f(x)$ is bijective and so $ \exists a$ such that $ f(a)=0$. 
Then $ P(a,x)$ $ \implies$ $ f(f(x))=x$ and $ f(0)=0$

$ P(f(x),y)$ $ \implies$ $ f(f(x)f(f(x))+f(y))=f(f(x))^2+y$ and so $ f(xf(x)+f(y))=x^2+y$ and so (comparing with $ P(x,y)$) : $ f(x)^2=x^2$
So : $ \forall x$, either $ f(x)=x$, either $ f(x)=-x$

Suppose now $ \exists a$ such $ f(a)=-a$ and $ \exists b$ such that $ f(b)=b$

$ P(a,b)$ $ \implies$ $ f(-a^2+b)=a^2+b$ and so either $ -a^2+b=a^2+b$, either $ a^2-b=a^2+b$ $ \implies$ either $ a=0$, either $ b=0$

So either $ f(x)=x$ $ \forall x$, either $ f(x)=-x$ $ \forall x$ and it is easy to check back that these two solutions both fit the requirements.

Hence the two solutions :
$ f(x)=x$ $ \forall x$
$ f(x)=-x$ $ \forall x$
\end{solution}



\begin{solution}[by \href{https://artofproblemsolving.com/community/user/64716}{mavropnevma}]
	\begin{tcolorbox}$ f(f(x))=x+f(0)^2$ and so $ f(x)$ is bijective ...\end{tcolorbox}
1. Assume $f(x_1) = f(x_2)$, then $f(f(x_1)) = f(f(x_2))$, and so $x_1 + f(0)^2 = x_2 + f(0)^2$, hence $x_1 = x_2$, hence injectivity.
2. Let $y\in \mathbb{R}$; then take $x = f(y-f(0)^2)$ and get $f(x) = y$, hence surjectivity.

A simpler argument is that from $f\circ f = \textrm{id} + \textrm{constant}$ follows immediately $f$ bijective, since RHS is bijective.

EDIT. This was in answer to some question about bijectivity of $f$, made by user eraydin, who then withdrew his entry, as this was posted.
\end{solution}



\begin{solution}[by \href{https://artofproblemsolving.com/community/user/98555}{dr_Civot}]
	See here http://www.artofproblemsolving.com/Forum/viewtopic.php?p=348686&sid=8659ca050356ad91f5b7f8d940a4f4ed#p348686
\end{solution}
*******************************************************************************
-------------------------------------------------------------------------------

\begin{problem}[Posted by \href{https://artofproblemsolving.com/community/user/69938}{imedeen}]
	Determine all continuous functions $f: \mathbb R \to \mathbb R$ such that:
\[f(2009f(x) + f(y)) = 2009x + y\]
holds true for all reals $x$ and $y$.
	\flushright \href{https://artofproblemsolving.com/community/c6h320387}{(Link to AoPS)}
\end{problem}



\begin{solution}[by \href{https://artofproblemsolving.com/community/user/29428}{pco}]
	\begin{tcolorbox}Determine continuous function $ f: R\to R$ such that:

$ f(2009f(x) + f(y)) = 2009x + y$ ,$ \forall x,y \in R$\end{tcolorbox}

[hide="A solution"]
Let $ P(x,y)$ be the assertion $ f(2009f(x) + f(y)) = 2009x + y$
Let $ a = 2010f(0)$

$ P(0,0)$ $ \implies$ $ f(a) = 0$
$ P(a,a)$ $ \implies$ $ f(0) = 2010a = 2010^2f(0)$ and so $ f(0) = 0$ and $ a = 0$

$ P(0,x)$ $ \implies$ $ f(f(x)) = x$
$ P(f(x),0)$ $ \implies$ $ f(2009x) = 2009f(x)$
$ P(f(x),f(y))$ $ \implies$ $ f(2009x + y) = 2009f(x) + f(y)$ $ = f(2009x) + f(y)$

And so $ f(x)$ is a continuous solution of Cauchy's equation and so $ f(x) = cx$. Plugging back in the original equation, we get $ c^2 = 1$ and so two solutions :

$ f(x) = x$ $ \forall x$
$ f(x) = - x$ $ \forall x$
[\/hide]
\end{solution}
*******************************************************************************
-------------------------------------------------------------------------------

\begin{problem}[Posted by \href{https://artofproblemsolving.com/community/user/63660}{Victory.US}]
	Determine all continuous functions $f: \mathbb R \to \mathbb R$ such that
\[f(x)+2f(x^2)+f(x^4)=0\]
holds for all $ x \in \mathbb R$.
	\flushright \href{https://artofproblemsolving.com/community/c6h320460}{(Link to AoPS)}
\end{problem}



\begin{solution}[by \href{https://artofproblemsolving.com/community/user/29428}{pco}]
	\begin{tcolorbox}Determine all continuous function $ R\to R$ such that
$ f(x) + 2f(x^2) + f(x^4) = 0$ ,$ \forall x \in R$\end{tcolorbox}

Let $ g(x)=f(x)+f(x^2)$. $ g(x)$ is continuous and the original equation becomes $ g(x)+g(x^2)=0$ so :

$ g(-x)=g(x)$
$ g(0)=0$
$ g(x)=(-1)^ng(x^{2^{-n}})$ $ \forall x\ge 0$. Setting $ n\to +\infty$ and using continuity, we get $ g(x)=0$ $ \forall x\ge 0$
And so $ g(x)=0$ $ \forall x$

So $ g(x)=f(x)+f(x^2)=0$ and the same proof gives $ \boxed{f(x)=0}$ $ \forall x$
\end{solution}



\begin{solution}[by \href{https://artofproblemsolving.com/community/user/44887}{Mathias_DK}]
	\begin{tcolorbox}[quote="Victory.US"]Determine all continuous function $ R\to R$ such that
$ f(x) + 2f(x^2) + f(x^4) = 0$ ,$ \forall x \in R$\end{tcolorbox}

Let $ g(x) = f(x) + f(x^2)$. $ g(x)$ is continuous and the original equation becomes $ g(x) + g(x^2) = 0$ so :

$ g( - x) = g(x)$
$ g(0) = 0$
$ g(x) = ( - 1)^ng(x^{2^{ - n}})$ $ \forall x\ge 0$. Setting $ n\to + \infty$ and using continuity, we get $ g(x) = 0$ $ \forall x\ge 0$
And so $ g(x) = 0$ $ \forall x$

So $ g(x) = f(x) + f(x^2) = 0$ and the same proof gives $ \boxed{f(x) = 0}$ $ \forall x$\end{tcolorbox}
You should remember to write $ g(1) = 0$ as $ 2^{ - n} \to 0, n \to \infty$ and therefore $ x^{2^{ - n}} \to 1, n \to \infty$ and therefore $ g(x) = g(1) \forall x > 0$.
\end{solution}



\begin{solution}[by \href{https://artofproblemsolving.com/community/user/68920}{prester}]
	\begin{tcolorbox}[quote="pco"][quote="Victory.US"]Determine all continuous function $ R\to R$ such that
$ f(x) + 2f(x^2) + f(x^4) = 0$ ,$ \forall x \in R$\end{tcolorbox}

Let $ g(x) = f(x) + f(x^2)$. $ g(x)$ is continuous and the original equation becomes $ g(x) + g(x^2) = 0$ so :

$ g( - x) = g(x)$
$ g(0) = 0$
$ g(x) = ( - 1)^ng(x^{2^{ - n}})$ $ \forall x\ge 0$. Setting $ n\to + \infty$ and using continuity, we get $ g(x) = 0$ $ \forall x\ge 0$
And so $ g(x) = 0$ $ \forall x$

So $ g(x) = f(x) + f(x^2) = 0$ and the same proof gives $ \boxed{f(x) = 0}$ $ \forall x$\end{tcolorbox}
You should remember to write $ g(1) = 0$ as $ 2^{ - n} \to 0, n \to \infty$ and therefore $ x^{2^{ - n}} \to 1, n \to \infty$ and therefore $ g(x) = g(1) \forall x > 0$.\end{tcolorbox}

Dear pco and Mathias_DK, what about the factor $ ( - 1)^n?$ When $ n \to \infty$ it has no limit....so how do you assert that when $ n \to \infty$ the $ g(x) \to g(1)?$....
\end{solution}



\begin{solution}[by \href{https://artofproblemsolving.com/community/user/29428}{pco}]
	\begin{tcolorbox}...Dear pco and Mathias_DK, what about the factor $ ( - 1)^n?$ When $ n \to \infty$ it has no limit....so how do you assert that when $ n \to \infty$ the $ g(x) \to g(1)?$....\end{tcolorbox}

I just said that $ g(x) = 0$ and this is an immediate conclusion of continuity and $ g(x) = ( - 1)^ng(x^{2^{ - n}})$ :

When $ n\to + \infty$, RHS has a limit ($ g(x)$) and so $ g(x)=g(1) = 0$ else RHS would swap between $ g(1)$ and $ - g(1)$.
\end{solution}
*******************************************************************************
-------------------------------------------------------------------------------

\begin{problem}[Posted by \href{https://artofproblemsolving.com/community/user/43536}{nguyenvuthanhha}]
	Find all functions $ f : \mathbb{R} \to \mathbb{R}$ such that
\[ \lim_{x \to 0 } \frac{f(x)}{x} =  1\]
and 
\[ f(x+y) = f(x) + f(y) + 2x^2 + 3xy + 2y^2, \quad  \forall x , y  \in  \mathbb{R}.\]
	\flushright \href{https://artofproblemsolving.com/community/c6h320962}{(Link to AoPS)}
\end{problem}



\begin{solution}[by \href{https://artofproblemsolving.com/community/user/71459}{x164}]
	If we fill in $ x=0$ in the second equation we get:
\[ f(y) = f(0) + f(y) + 2y^2,\]
so $ y^2 = -\frac{1}{2}f(0)$ for all $ y \in \mathbb{R}$, contradiction.
So no such functions exist.
\end{solution}



\begin{solution}[by \href{https://artofproblemsolving.com/community/user/29428}{pco}]
	\begin{tcolorbox}If we fill in $ x = 0$ in the second equation we get:
\[ f(y) = f(0) + f(y) + 2y^2,\]
so $ y^2 = - \frac {1}{2}f(0)$ for all $ y \in \mathbb{R}$, contradiction.
So no such functions exist.\end{tcolorbox}
indeed :)
Very quick !
\end{solution}



\begin{solution}[by \href{https://artofproblemsolving.com/community/user/29214}{HTA}]
	Well , just in case there was a typo , i may assume that the function defined in R+ and have value on R+
putting x=y gives 
$ f(2x) = 2f(x) + 7x^2$
$ \frac {f(2x)}{2x} = \frac {f(x)}{x} + \frac {7}{2}x$

So
$ \frac {f(x)}{x} = \frac {f(\frac {x}{2})}{\frac {x}{2}} + \frac {7}{2}\frac {x}{2} = ... = \frac {f(\frac {x}{2^n})}{\frac {x}{2^n}} + \frac {7x}{2}(\frac {1}{2} + \frac {1}{2^2} + ... + \frac {1}{2^n} ) \rightarrow 1 + \frac {7}{2}x$
So $ f(x) = x + \frac {7}{2}x^2$

We notice that this solution doesnt work so there is no solution
\end{solution}
*******************************************************************************
-------------------------------------------------------------------------------

\begin{problem}[Posted by \href{https://artofproblemsolving.com/community/user/71459}{x164}]
	Let $ n \in \mathbb{Z}_{ > 0}$ be a positive integer.

Find all strictly increasing functions $ f: \mathbb{Z}_{\geq 0} \rightarrow \mathbb{Z}_{\geq 0}$ such that the equation
\[ \frac {f(x)}{k^n} = k - x\]
has an integral solution $ x$ for all $ k \in \mathbb{Z}_{ > 0}$.
	\flushright \href{https://artofproblemsolving.com/community/c6h321004}{(Link to AoPS)}
\end{problem}



\begin{solution}[by \href{https://artofproblemsolving.com/community/user/73693}{pluristiq}]
	let $ x_k$ denote a solution to $ \frac {f(x)}{k^n} = k - x$. it's clear that $ x_k\in [0,k]$, and in fact, as shown below, for $ k\geq 2$, $ x_k\in [1,k - 1]$.

it's easy to check that $ f(x) = (x + 1)^n$ satisfies the conditions: it's strictly increasing, and we may take $ x_k = k - 1$. now we'll show that this is the only function.

by considering $ k = 1$, there's an $ x$ s.t. $ f(x) = 1 - x$. we can't have $ x = 1$ or else $ f(1) = 0$, in which case $ f(0) < f(1)$ is impossible. thus $ x = 0$, and $ f(0) = 1$. this is our base case.

now assume that $ f(x) = (x + 1)^n$ for $ x = 0, 1, ..., r$, $ r\geq 0$.

note that for $ k\geq 2$, the RHS of the equation $ \frac {f(x)}{k^n} = k - x$ is at least $ k - (k - 1) = 1$ and at most $ k - 1$. this is because we cannot have $ x = k$ (or else $ f(k) = 0$, which, as above, is impossible) and we cannot have $ x = 0$ (or else $ f(0) = k^{n + 1}$>1, contradicting $ f(0) = 1$).

it then follows that for every $ k\geq 2$, $ k^n\leq f(x_k) < k^{n + 1}$, where $ x_k\in [1,k - 1]$ (*)

now, by (*), $ (r + 2)^n\leq f(x_{r + 2}) < (r + 2)^{n + 1}$ and $ x_{r + 2}\in [1,r + 1]$. by the inductive hypothesis, $ f(0), ..., f(r)$ are all $ \leq (r + 1)^n < (r + 2)^n$. thus $ x_{r + 2} = r + 1$, from which we get $ \frac {f(r + 1)}{(r + 2)^n} = r + 2 - (r + 1) = 1$, i.e. $ f(r + 1) = (r + 2)^n$, completing the induction.
\end{solution}



\begin{solution}[by \href{https://artofproblemsolving.com/community/user/44887}{Mathias_DK}]
	\begin{tcolorbox}Let $ n \in \mathbb{Z}_{ > 0}$ be a positive integer.

Find all strictly increasing functions $ f: \mathbb{Z}_{\geq 0} \rightarrow \mathbb{Z}_{\geq 0}$ such that the equation
\[ \frac {f(x)}{k^n} = k - x\]
has an integral solution $ x$ for all $ k \in \mathbb{Z}_{ > 0}$.\end{tcolorbox}
For every $ k$ let $ h : \mathbb{N} \to \mathbb{N}_0$ be the solution such that:
$ \frac {f(h(k))}{k^n} = k - h(k)$
Let $ g(k) = \frac {f(h(k))}{k^n}$. Obviously $ g(k) \in \mathbb{N}_0$.
We know that $ h(k) \le k$, and it is easy to prove that $ h(k)$ is injective. (If $ h(k_1) = h(k_2)$ then $ k_1 - \frac {f(h(k_1))}{k_1^n} = k_2 - \frac {f(h(k_1))}{k_2^n}$)
If $ h(n) = n$ for some $ n$ then $ f(n) = 0$, but $ n \ge 1$ and $ f(n) > f(0)$ is impossible. Hence $ h(k) \le k - 1$. Since $ h$ is injective we easily find $ h(k) = k - 1 \forall k \ge 1$ and therefore $ f(k - 1) = k^n \forall k \ge 1 \iff f(x) = (x + 1)^n \forall x \ge 0$.
So the only solution is $ f(x) = (x + 1)^n \forall x \ge 0$ which is easily seen to be a solution.

We can replace the "strictly increasing function $ f : \mathbb{N}_0 \to \mathbb{N}_0$" condition with "any function $ f : \mathbb{N}_0 \to \mathbb{N}$".
\end{solution}



\begin{solution}[by \href{https://artofproblemsolving.com/community/user/29428}{pco}]
	\begin{tcolorbox} ...We know that $ h(k) \le k$, and it is easy to prove that $ h(k)$ is injective. (If $ h(k_1) = h(k_2)$ then $ k_1 - \frac {f(h(k_1))}{k_1^n} = k_2 - \frac {f(h(k_1))}{k_2^n}$)
\end{tcolorbox}

And why would "$ k_1 - \frac {f(h(k_1))}{k_1^n} = k_2 - \frac {f(h(k_1))}{k_2^n}$" be impossible ?
\end{solution}



\begin{solution}[by \href{https://artofproblemsolving.com/community/user/65556}{voong}]
	\begin{tcolorbox}[quote="Mathias_DK"] ...We know that $ h(k) \le k$, and it is easy to prove that $ h(k)$ is injective. (If $ h(k_1) = h(k_2)$ then $ k_1 - \frac {f(h(k_1))}{k_1^n} = k_2 - \frac {f(h(k_1))}{k_2^n}$)
\end{tcolorbox}

And why would "$ k_1 - \frac {f(h(k_1))}{k_1^n} = k_2 - \frac {f(h(k_1))}{k_2^n}$" be impossible ?\end{tcolorbox}

because $ p(x) = x - \frac {k}{x^n}$ is strictly increasing in our condition.

( $ h(k_1) = h(k_2)$ , then $ f(h(k_1)) = f(h(k_2)) = k$)
\end{solution}



\begin{solution}[by \href{https://artofproblemsolving.com/community/user/29428}{pco}]
	\begin{tcolorbox}[quote="pco"][quote="Mathias_DK"] ...We know that $ h(k) \le k$, and it is easy to prove that $ h(k)$ is injective. (If $ h(k_1) = h(k_2)$ then $ k_1 - \frac {f(h(k_1))}{k_1^n} = k_2 - \frac {f(h(k_1))}{k_2^n}$)
\end{tcolorbox}

And why would "$ k_1 - \frac {f(h(k_1))}{k_1^n} = k_2 - \frac {f(h(k_1))}{k_2^n}$" be impossible ?\end{tcolorbox}

because $ p(x) = x - \frac {k}{x^n}$ is strictly increasing in our condition.

( $ h(k_1) = h(k_2)$ , then $ f(h(k_1)) = f(h(k_2)) = k$)\end{tcolorbox}
Ohh yes. I forgot that $ f>0$. Sorry for my stupid question and thanks for your answer.
\end{solution}



\begin{solution}[by \href{https://artofproblemsolving.com/community/user/301514}{sa2001}]
	For $k = 1$, we have
$$f(a) + a = 1$$
 for some non-negative integer $a$. $a$ here is of course less than $2$. $a = 1$ would imply $f(1) = 0$, which would mean $f(0) < 0$, which is not possible. Therefore, we get $a = 0$ and $f(0) = 1$.
Now we prove that $f(j-1) = j^n$ for all natural numbers $j$ using strong induction on $j$.
The base case $j = 1$ is done.
Assume that all natural $j$ less than $m$ satisfy our induction hypothesis.
Then, considering $k = m+1$ we get
$$f(b) = (m+1)^n(m+1-b)$$
for some natural number $b$. As $f(0) > 0$, $f(b) > 0$. As $(m+1)^n$ must divide $f(b)$, $f(b) \geq (m+1)^n$. This means $b > (m-1)$ from our induction hypothesis. Also, as $(m+1-b)$ needs to be positive, $b < (m+1)$,  which means $b = m$, which shows that
$$f(m) = (m+1)^n$$
as required.
It's easy to check that this satisfies the given conditions.
\end{solution}
*******************************************************************************
-------------------------------------------------------------------------------

\begin{problem}[Posted by \href{https://artofproblemsolving.com/community/user/51029}{mathVNpro}]
	Find all funtion $ f$ from positive real numbers to real numbers such that
\[ xf\left (x + \frac {1}{y}\right ) + yf(y) + \frac {y}{x} = yf\left (y + \frac {1}{x}\right ) + xf(x) + \frac {x}{y}\]
holds for all positive reals $x$ and $y$.
	\flushright \href{https://artofproblemsolving.com/community/c6h321307}{(Link to AoPS)}
\end{problem}



\begin{solution}[by \href{https://artofproblemsolving.com/community/user/29428}{pco}]
	\begin{tcolorbox}Find all funtion $ f$ from positive real numbers to real numbers such that:
\[ xf\left (x + \frac {1}{y}\right ) + yf(y) + \frac {y}{x} = yf\left (y + \frac {1}{x}\right ) + xf(x) + \frac {x}{y}\]
\end{tcolorbox}

Let $ f(x) = g(x) + x$ and the let $ P(x,y)$ be the new equation :

$ P(x,y)$ : $ xg(x + \frac 1y) + yg(y) = yg(y + \frac 1x) + xg(x)$
Let $ g(1) = a$

$ P(\frac 1x,\frac 1y)$ $ \implies$ $ \frac 1xg(\frac 1x + y) + \frac 1yg(\frac 1y) = \frac 1yg(\frac 1y + x) + \frac 1xg(\frac 1x)$

Multiplying by $ xy$, we get $ yg(\frac 1x + y) + xg(\frac 1y) = xg(\frac 1y + x) + yg(\frac 1x)$

Adding to $ P(x,y)$, we get new assertion $ Q(x,y)$ : $ yg(y) + xg(\frac 1y) = xg(x) + yg(\frac 1x)$

$ Q(x,1)$ $ \implies$ $ g(\frac 1x) = a(x + 1) - xg(x)$

Replacing $ g(\frac 1x)$ by $ a(x + 1) - xg(x)$ and $ g(\frac 1y)$ by $ a(y + 1) - yg(y)$ in $ Q(x,y)$, we get :

$ yg(y) + x(a(y + 1) - yg(y)) = xg(x) + y(a(x + 1) - xg(x))$ and so :

$ (1 - x)(yg(y) - a) = (1 - y)(xg(x) - a)$ 

So, $ \forall x,y\ne 1$ : $ \frac {yg(y) - a}{1 - y} = \frac {xg(x) - a}{1 - x}$ and so $ \frac {xg(x) - a}{1 - x} = c$ and :

$ \forall x\ne 1$ : $ g(x) = \frac {a + c(1 - x)}x$ and since this equality is still  true when $ x = 1$, we get :

$ g(x) = \frac {\alpha x + \beta}x$ $ \forall x$ and it is easy to check back that this indeed matches $ P(x,y)$

Hence the solution $ f(x) = x + g(x) = \boxed{\frac {x^2 + \alpha x + \beta}x}$
\end{solution}
*******************************************************************************
-------------------------------------------------------------------------------

\begin{problem}[Posted by \href{https://artofproblemsolving.com/community/user/69184}{sondhtn}]
	1) Find $ f(x,y)$ such that $ 2009 f'_{x}+2010 f'_{y}=0$.

2) Find the minimum and the maximum value of $ f(x,y)=x^2+y^2-3xy+x+y$, where $ (x,y)\in \{(x,y): x^2+y^2\leq 16\}$.
	\flushright \href{https://artofproblemsolving.com/community/c6h321431}{(Link to AoPS)}
\end{problem}



\begin{solution}[by \href{https://artofproblemsolving.com/community/user/29428}{pco}]
	\begin{tcolorbox}1)Find $ f(x,y)$ such that $ 2009 f'_{x} + 2010 f'_{y} = 0$\end{tcolorbox}

We at least have $ f(x)=h(2010ax-2009ay+b)$ where $ h(x)$ is any differentiable function.

But I'm not sure for now that these are the only solutions
\end{solution}



\begin{solution}[by \href{https://artofproblemsolving.com/community/user/29428}{pco}]
	\begin{tcolorbox}1)Find $ f(x,y)$ such that $ 2009 f'_{x} + 2010 f'_{y} = 0$\end{tcolorbox}

here is a general solution :

Let $ g(x,y)=f(y,\frac{2010y-x}{2009})$

We get $ f(x,y)=g(2010x-2009y,x)$ and :

$ f'_x(x,y)=2010g'_x(2010x-2009y,x)+g'_y(2010x-2009y,x)$
$ f'_y(x,y)=-2009g'_x(2010x-2009y,x)$

And the equation $ 2009f'_x(x,y)+2010f'_y(x,y)=0$ becomes $ g'_y(2010x-2009y,x)=0$ and so $ g'_y(x,y)=0$ and so $ g(x,y)=h(x)$ 

Hence the answer $ \boxed{f(x,y)=h(2010x-2009y)}$ where $ h(x)$ is any differentiable function
\end{solution}
*******************************************************************************
-------------------------------------------------------------------------------

\begin{problem}[Posted by \href{https://artofproblemsolving.com/community/user/63660}{Victory.US}]
	Determine all continuous functions $f: \mathbb R \to \mathbb R$ such that
i) $ f(0) \ne 2$ and $f(2) \ne 0$, and
ii) For all reals $x$ and $y$, \[ f(x+y) \ne f(x) \cdot f(y) =f(xy)+f(x)+f(y).\]
	\flushright \href{https://artofproblemsolving.com/community/c6h321703}{(Link to AoPS)}
\end{problem}



\begin{solution}[by \href{https://artofproblemsolving.com/community/user/29428}{pco}]
	\begin{tcolorbox}Determine all continuos function $ R\to R$ such that :
$ i)$ $ f(0) \ne 2;f(2) \ne 0$
$ ii)$ $ f(x + y) \ne f(x).f(y) = f(xy) + f(x) + f(y)$ ,$ \forall x,y \in R$\end{tcolorbox}

Setting $ x=y=0$ in $ f(x).f(y) = f(xy) + f(x) + f(y)$, we get $ f(0)(f(0)-3)=0$ but $ f(0)=0$ would imply $ f(0+0)=f(0)f(0)$ which is supposed false. So $ f(0)=3$

Setting then $ y=0$ in $ f(x).f(y) = f(xy) + f(x) + f(y)$, we get $ 3f(x)=f(x)+6$ and so $ f(x)=3$, which indeed is a solution.

Hence the answer : $ \boxed{f(x)=3}$ $ \forall x\in\mathbb R$
\end{solution}
*******************************************************************************
-------------------------------------------------------------------------------

\begin{problem}[Posted by \href{https://artofproblemsolving.com/community/user/3182}{Kunihiko_Chikaya}]
	Let $ f(x)$ be a continuous function such that $ f(x) = \frac {f(x + y) + f(x - y)}{2}$ for all $ x,\ y\in\mathbb{R}$. 

Prove that $ \int_a^b f(x)\ dx = \frac {b - a}{2}\{f(a) + f(b)\}\ (a < b)$.
	\flushright \href{https://artofproblemsolving.com/community/c7h321469}{(Link to AoPS)}
\end{problem}



\begin{solution}[by \href{https://artofproblemsolving.com/community/user/25405}{AndrewTom}]
	It looks like the equation defines a straight line, the mid-value being the mid-point of the line segment joining $ f(x-y)$ to $ f(x+y)$ and so the integral is the area of a trapezium.
\end{solution}



\begin{solution}[by \href{https://artofproblemsolving.com/community/user/55088}{mathxl}]
	Yeah its Jensens functional equation with solution f(x)=ax+b
\end{solution}



\begin{solution}[by \href{https://artofproblemsolving.com/community/user/3182}{Kunihiko_Chikaya}]
	Could you show your work in detail, guys? :)
\end{solution}



\begin{solution}[by \href{https://artofproblemsolving.com/community/user/29428}{pco}]
	\begin{tcolorbox}Could you show your work in detail, guys? :)\end{tcolorbox}

Hello kunny! Here is a simple way :

Let $ P(x,y)$ be the assertion $ f(x)=\frac{f(x+y)+f(x-y)}2$
Let $ g(x)=f(x)-f(0)$

Obviously, $ g(x)$ is too a solution and so matches $ P(x,y)$ and, since $ g(0)=0$ :

$ P(0,x)$ $ \implies$ $ g(-x)=-g(x)$
$ P(x,x)$ $ \implies$ $ g(2x)=2g(x)$

$ P(x,x+2y)$ $ \implies$ $ g(x)=\frac{g(2x+2y)+g(-2y)}2$ $ =\frac{2g(x+y)-2g(y)}2$ and so $ g(x+y)=g(x)+g(y)$

This is a Cauchy equation whose only continuous solutions are $ g(x)=ax$, which indeed are solutions of the original equation.

Hence the general solution $ f(x)=ax+b$

Hence the result.
\end{solution}



\begin{solution}[by \href{https://artofproblemsolving.com/community/user/3182}{Kunihiko_Chikaya}]
	Thank you, pco.
\end{solution}
*******************************************************************************
-------------------------------------------------------------------------------

\begin{problem}[Posted by \href{https://artofproblemsolving.com/community/user/26965}{Shishkin}]
	Find all functions $f: \mathbb Q \to \mathbb Q$  such that $ f(x^{2}+y+f(xy)) = 3+(x+f(y)-2)f(x)$ for all $x,y \in \mathbb Q$.
	\flushright \href{https://artofproblemsolving.com/community/q2h167309}{(Link to AoPS)}
\end{problem}



\begin{solution}[by \href{https://artofproblemsolving.com/community/user/29428}{pco}]
	\begin{tcolorbox}Find all functions $ f$ $ Q\to Q$  such that $ f(x^{2} + y + f(xy)) = 3 + (x + f(y) - 2)f(x)$.\end{tcolorbox}

I have a rather complex solution to prove $ f(x)=x+1$ :

Let $ P(x,y)$ be the property : $ f(x^{2} + y + f(xy)) = 3 + (x + f(y) - 2)f(x)$
Let $ f(0)=a$

1) $ P(0,x)$ implies $ f(x+a)=af(x)+3-2a$ and so :
$ \boxed{f(x+na)=a^nf(x)+(3-2a)(1+a+a^2+\ldots+a^{n-1})}$

2) $ P(x,0)$ and $ P(-x,0)$ imply $ f(x^2+a)=(x+a-2)f(x)+3$ and $ f(x^2+a)=(-x+a-2)f(-x)+3$ and so :
$ \boxed{(x+a-2)f(x)=(-x+a-2)f(-x)}$

3) $ P(1,0)$ implies $ f(a+1)=(a-1)f(1)+3$ but $ 1)$ implies $ f(a+1)=af(1)+3-2a$ and so $ af(1)+3-2a=(a-1)f(1)+3$ and :
$ \boxed{f(1)=2a}$

4) If $ a=2$, then $ 1)$ gives $ f(2n)=2^n+1$ and $ 1)+3)$ gives $ f(2n+1)=3\times 2^n+1$
But then $ P(2,2)$ gives $ f(6+f(4))=12$, so $ f(11)=12$ but $ f(11)=f(2\times 5+1)=3\times 2^5+1=97$ and so $ a\neq 2$
If $ a=0$, then $ P(0,0)$ gives $ f(0)=a=3$ and so $ a\neq 0$
If $ a=3$, then $ 2)$ above with $ x=1$ gives $ 2f(1)=0$ but $ 3)$ said $ f(1)=2a=6$ and so :
$ \boxed{a\notin \{0,2,3\}}$

5) Using $ x=a-2$ in $ 2)$ above, we get $ 2(a-2)f(a-2)=0$ and so, since, from $ 4)$, $ a\neq 2$ :
$ \boxed{f(a-2)=0}$

6) Using $ x=1$ in $ 2)$ above gives $ f(-1)=2a\frac{a-1}{a-3}$ (remember with $ 4)$ above that $ a\neq 3$).
Then $ P(-1,-1)$ gives $ f(2a)=3 + (2a\frac{a-1}{a-3} - 3)2a\frac{a-1}{a-3}$
But $ 1)$ gives $ f(2a)=a^3-2a^2+a+3$ and so :
$ a^3-2a^2+a+3=3 + (2a\frac{a-1}{a-3} - 3)2a\frac{a-1}{a-3}$ which gives :
$ a(a-1)(a^3-11a^2+25a-27)= 0$

It's rather easy to see that $ a^3-11a^2+25a-27$ has no rational root and so :
$ \boxed{f(0)=a=1}$

7) Then $ 1)$ gives $ f(x+n)=f(x)+n$ and $ f(n)=n+1$
Then, Let $ p$ and $ q$ coprimes positive integers.
$ P(q,\frac{p}{q})$ gives $ f(q^2+\frac{p}{q}+p+1)=3 + (q + f(\frac{p}{q}) - 2)(q+1)$
But $ f(q^2+\frac{p}{q}+p+1)=q^2+p+1+f(\frac{p}{q})$ (since $ f(x+n)=f(x)+n$) and so :

$ q^2+p+1+f(\frac{p}{q})=3 + (q + f(\frac{p}{q}) - 2)(q+1)$

$ p+q=qf(\frac{p}{q}$ and so $ f(\frac{p}{q})=\frac{p}{q}+1$

So $ f(x)=x+1$ $ \forall x\in\mathbb{Q}^{+}$
Using $ 2)$, we have then $ f(-x)=-x+1$ $ \forall x\in\mathbb{Q}^{+}$

So $ f(x)=x+1$ $ \forall x\in\mathbb{Q}$

And putting back this expression in original equation, we find that this necessary condition is sufficient.

And the only solution is $ \boxed{f(x)=x+1}$
\end{solution}



\begin{solution}[by \href{https://artofproblemsolving.com/community/user/43536}{nguyenvuthanhha}]
	\begin{italicized}A very good solution , Pco . You really have talent in Algebra .  \end{italicized}
\end{solution}



\begin{solution}[by \href{https://artofproblemsolving.com/community/user/52090}{Dumel}]
	\begin{tcolorbox}
It's rather easy to see that $ a^3 - 11a^2 + 25a - 27$ has no rational root and so :
$ \boxed{f(0) = a = 1}$\end{tcolorbox}you're wrong
\end{solution}



\begin{solution}[by \href{https://artofproblemsolving.com/community/user/29428}{pco}]
	\begin{tcolorbox}[quote="pco"]
It's rather easy to see that $ a^3 - 11a^2 + 25a - 27$ has no rational root and so :
$ \boxed{f(0) = a = 1}$\end{tcolorbox}you're wrong\end{tcolorbox}

Please, could you kindly show me where I am wrong ?

Is there a rational root to $ a^3 - 11a^2 + 25a - 27$ ? (please give us)

Or is my conclusion $ a=1$ wrong ?, and why ?
\end{solution}



\begin{solution}[by \href{https://artofproblemsolving.com/community/user/52090}{Dumel}]
	let $ g(x) = x^3 - 11x^2 + 25x - 27$
$ - 27 = g(0) < 0$
and
$ g(11) > 0$
$ g$ is a continuous funcion so there exist $ x_0$ in $ (0,11)$ such that $ g(x_0) = 0$

edit: oh oh I forgot that  $ f$ is $ Q\to Q$  and all roots of g are surd so your solution is (almost) correct :-)
\end{solution}



\begin{solution}[by \href{https://artofproblemsolving.com/community/user/29428}{pco}]
	\begin{tcolorbox}let $ g(x) = x^3 - 11x^2 + 25x - 27$
$ - 27 = g(0) < 0$
and
$ g(11) > 0$
$ g$ is a continuous funcion so there exist $ x_0$ in $ (0,11)$ such that $ g(x_0) = 0$

edit: oh oh I forgot that  $ f$ is $ Q\to Q$  and all roots of g are surd so your solution is (almost) correct :-)\end{tcolorbox}

Yes,  :) , I said that $ x^3 - 11x^2 + 25x - 27$ had no \begin{bolded}rational \end{underlined}\end{bolded}root.

And, why the "almost" word in your post ?
What else is not correct in my solution ?
\end{solution}



\begin{solution}[by \href{https://artofproblemsolving.com/community/user/52090}{Dumel}]
	"almost" was just due to the word "rational"  
\end{solution}



\begin{solution}[by \href{https://artofproblemsolving.com/community/user/177508}{mathuz}]
	:roll: 
Oh...ho, I  found very nice solution of the problem!
Let  $g(x)=f(x)-1$  and  $ g:Q\rightarrow Q $.  Then from original equation  we have \[ g(x^2+y+g(xy)+1)=2+(x+g(y)-1)(g(x)+1) .\]
(1) $g(0)=0; $  really it's true!
(2)  from (1),  $ g(1)=1 $  and 
\[ g(x+1)=g(x)+1 \] and \[ g(x^2)=x+g(x)(x-1) (*)\]
for any ratsional $x$.
Hence, $ g(n)=n $ any integer $n$.    Let $x=\frac{m}{n} $ some integers $m$ and $n$,  then  at $P(x,n)$, \rightarrow $ $g(x)=x $ for any ratsional numbers $x$.
Therefore, $f(x)=x+1$, any ratsional $x$.
\end{solution}



\begin{solution}[by \href{https://artofproblemsolving.com/community/user/177508}{mathuz}]
	Sorry, I have latex mistake. Original version:  
 :roll: 
Oh...ho, I  found very nice solution of the problem!
Let  $g(x)=f(x)-1$  and  $ g:Q\rightarrow Q $.  Then from original equation  we have \[ g(x^2+y+g(xy)+1)=2+(x+g(y)-1)(g(x)+1) .\]
(1) $g(0)=0; $  really it's true!
(2)  from (1),  $ g(1)=1 $  and 
\[ g(x+1)=g(x)+1 \] and \[ g(x^2)=x+g(x)(x-1) (*)\]
for any ratsional $x$.
Hence, $ g(n)=n $ any integer $n$.    Let $x=\frac{m}{n} $ some integers $m$ and $n$,  then from $(*)$, at $P(x,n)$, $\rightarrow $ $g(x)=x $ for any ratsional numbers $x$.
Therefore, $f(x)=x+1$, any ratsional $x$.
\end{solution}



\begin{solution}[by \href{https://artofproblemsolving.com/community/user/288210}{tenplusten}]
	\begin{tcolorbox}Sorry, I have latex mistake. Original version:  
 :roll: 
Oh...ho, I  found very nice solution of the problem!
Let  $g(x)=f(x)-1$  and  $ g:Q\rightarrow Q $.  Then from original equation  we have \[ g(x^2+y+g(xy)+1)=2+(x+g(y)-1)(g(x)+1) .\]
(1) $g(0)=0; $  really it's true!
(2)  from (1),  $ g(1)=1 $  and 
\[ g(x+1)=g(x)+1 \] and \[ g(x^2)=x+g(x)(x-1) (*)\]
for any ratsional $x$.
Hence, $ g(n)=n $ any integer $n$.    Let $x=\frac{m}{n} $ some integers $m$ and $n$,  then from $(*)$, at $P(x,n)$, $\rightarrow $ $g(x)=x $ for any ratsional numbers $x$.
Therefore, $f(x)=x+1$, any ratsional $x$.\end{tcolorbox}

I would like to see your proof for both claims???? 
\end{solution}
*******************************************************************************
-------------------------------------------------------------------------------

\begin{problem}[Posted by \href{https://artofproblemsolving.com/community/user/9882}{Virgil Nicula}]
	Does the exist a function $ f: \mathbb R^* \rightarrow \mathbb R$ so that for every $ x\ne 0$, we have
\[ f(f(x))=-\frac 1x \, ?\]
	\flushright \href{https://artofproblemsolving.com/community/q2h272801}{(Link to AoPS)}
\end{problem}



\begin{solution}[by \href{https://artofproblemsolving.com/community/user/29428}{pco}]
	\begin{tcolorbox}Isn't integer part ...\end{tcolorbox}
Ok, so $ f(f(x))=-\frac{1}{x}$ $ \forall x\neq 0$

We have $ f(f(f(f(x))))=x$ and so we just have to organize all non null reals in sequences $ \{a,b,-\frac{1}{a},-\frac{1}{b}\}$

Here is a general solution :

Let $ \mathbb{A}$ and $ \mathbb{B}$ two equipotent subsets of $ \mathbb{R^*^+}$ such that $ \mathbb{A}\cup\mathbb{B}=\mathbb{R^*^+}$ and $ \mathbb{A}\cap\mathbb{B}=\emptyset$

Let $ \mathbb{C}=\{x$ such that $ -\frac{1}{x}\in\mathbb{A}\}$
Let $ \mathbb{D}=\{x$ such that $ -\frac{1}{x}\in\mathbb{B}\}$

Let $ h(x)$ any bijective function from $ \mathbb{A}\longrightarrow\mathbb{B}$
Let $ k(x)$ the reciprocal of $ h(x)$
Let $ e(x)$ any function from $ \mathbb{A}\longrightarrow\{-1,+1\}$

All solutions are of the form :

$ \forall x\in\mathbb{A}$ : $ f(x)=e(x)h(x)^{e(x)}$ (to be clear : either $ h(x)$, either $ -\frac{1}{h(x)}$)
$ \forall x\in\mathbb{B}$ : $ f(x)=-e(k(x))k(x)^{-e(k(x))}$
$ \forall x\in\mathbb{C}$ : $ f(x)=-e(-\frac{1}{x})h(-\frac{1}{x})^{-e(-\frac{1}{x})}$
$ \forall x\in\mathbb{D}$ : $ f(x)=e(k(-\frac{1}{x}))k(-\frac{1}{x})^{e(k(-\frac{1}{x}))}$
\end{solution}



\begin{solution}[by \href{https://artofproblemsolving.com/community/user/9882}{Virgil Nicula}]
	Thanks. Please, can you present us a concret example ? Thank you.
\end{solution}



\begin{solution}[by \href{https://artofproblemsolving.com/community/user/29428}{pco}]
	\begin{tcolorbox}Thanks. Please, can you present us a concret example ? Thank you.\end{tcolorbox}

Ok :

Let $ \mathbb{A}=\{x>0$ such that $ [-x]$ is odd$ \}$  (where $ [.]$ is the integer part).
Let $ \mathbb{B}=\{x>0$ such that $ [-x]$ is even$ \}$
Let $ h(x)=x+1$
Let $ k(x)=x-1$
Let $ e(x)=+1$

Then :

$ \forall x>0$ such that $ [-x]$ is odd, $ f(x)=x+1$

$ \forall x>0$ such that $ [-x]$ is even, $ f(x)=-\frac{1}{x-1}$

$ \forall x<0$ such that $ [\frac{1}{x}]$ is odd, $ f(x)=\frac{x}{1-x}$

$ \forall x<0$ such that $ [\frac{1}{x}]$ is even, $ f(x)=-\frac{x+1}{x}$

Verification :

1) $ \forall x>0$ such that $ [-x]$ is odd :
$ f(x)=x+1$ and $ [-f(x)]=[-x-1]=[-x]-1$. 
So $ f(x)>0$ and $ [-f(x)]$ is even.
So $ f(f(x))=-\frac{1}{f(x)-1}=-\frac{1}{x}$


2) $ \forall x>0$ such that $ [-x]$ is even :
$ f(x)=-\frac{1}{x-1}$
So $ f(x)<0$ ($ [-x]$ even implies $ x>1$) and $ [\frac{1}{f(x)}]=[1-x]=1+[-x]$ is odd.
So $ f(f(x))=\frac{f(x)}{1-f(x)}=\frac{-\frac{1}{x-1}}{1+\frac{1}{x-1}}=-\frac{1}{x}$


3) $ \forall x<0$ such that $ [\frac{1}{x}]$ is odd
$ f(x)=\frac{x}{1-x}$
So $ f(x)<0$ and $ [\frac{1}{f(x)}]=[\frac{1-x}{x}]=[\frac{1}{x}]-1$ is even
So $ f(f(x))=-\frac{f(x)+1}{f(x)}=-\frac{\frac{x}{1-x}+1}{\frac{x}{1-x}}=-\frac{1}{x}$


4) $ \forall x<0$ such that $ [\frac{1}{x}]$ is even
$ f(x)=-\frac{x+1}{x}$
So $ f(x)>0$ ($ [\frac{1}{x}]$ even with $ x<0$ means $ \frac{1}{x}<-1$) and $ [-f(x)]=1+[\frac{1}{x}]$ is odd.
So $ f(f(x))=f(x)+1=-\frac{1}{x}$

Q.E.D.
\end{solution}



\begin{solution}[by \href{https://artofproblemsolving.com/community/user/29428}{pco}]
	\begin{tcolorbox}Thanks. Please, can you present us a concret example ? Thank you.\end{tcolorbox}

Here is another one, simpler than the previous, I think

Let $ \mathbb{A}=]0,1[$.
Let $ \mathbb{B}=[1,+\infty[$
Let $ h(x)=\frac{1}{2x}$ if $ x=2^{-n}$ for some integer $ n\geq 1$, and $ h(x)=\frac{1}{x}$ else.
Let $ e(x)=+1$

Then :

$ \forall x\in]0,1[$ such that $ x=2^{-n}$ for some integer $ n\geq 1$, $ f(x)=\frac{1}{2x}$

$ \forall x\in]0,1[$ such that $ x\neq 2^{-n}$ for all integers $ n\geq 1$, $ f(x)=\frac{1}{x}$

$ \forall x\in[1,+\infty[$ such that $ x=2^n$ for some integer $ n\geq 0$, $ f(x)=-2x$

$ \forall x\in[1,+\infty[$ such that $ x\neq 2^n$ for any integer $ n\geq 0$, $ f(x)=-x$

$ \forall x\in]-\infty,-1[$ such that $ x=-2^n$ for some integer $ n\geq 1$, $ f(x)=\frac{2}{x}$

$ \forall x\in]-\infty,-1[$ such that $ x\neq -2^n$ for any integer $ n\geq 1$, $ f(x)=\frac{1}{x}$

$ \forall x\in[-1,0[$ such that $ x=-2^{-n}$ for some integer $ n\geq 0$, $ f(x)=-\frac{x}{2}$

$ \forall x\in[-1,0[$ such that $ x\neq -2^{-n}$ for any integer $ n\geq 0$, $ f(x)=-x$

Verification is nearly immediate (I'll give it if some reader encounters problem)
\end{solution}



\begin{solution}[by \href{https://artofproblemsolving.com/community/user/9882}{Virgil Nicula}]
	Thank you, \begin{bolded}Patrick\end{bolded}, for your effort to explain us on two nice examples.
\end{solution}
*******************************************************************************
-------------------------------------------------------------------------------

\begin{problem}[Posted by \href{https://artofproblemsolving.com/community/user/60962}{Anni}]
	Find all functions $f: \mathbb R \to \mathbb R$ such that
 \[ f(x)=f\left(\frac{x}{2}\right)+\frac{x}{2}f'(x)\]
for all $x \in \mathbb R$.
	\flushright \href{https://artofproblemsolving.com/community/q2h278400}{(Link to AoPS)}
\end{problem}



\begin{solution}[by \href{https://artofproblemsolving.com/community/user/29428}{pco}]
	So we have to solve $ f(x) = f(\frac x2) + \frac x2f'(x)$ (and $ f(x)$ is continuous and it's derivative exists all over $ \mathbb R$)

Let $ f(x)$ any solution, let $ u\neq 0$ and let $ g(x) = f(x) + (f(0) - f(u))\frac xu - f(0)$. Obviously, $ g(x)$ is a solution and $ g(u) = g(0) = 0$
Let then $ M = \max_{x\in[0,u]}g(x)$ (I write $ [0,u]$ even if $ u < 0$) and any $ x_0\in[0,u]$ such that $ g(x_0) = M$ ($ M$ and $ x_0$ exist since $ g(x)$ is continuous)
If $ M\neq 0$, $ x_0\neq 0$ and $ f'(x_0) = 0$. We have then $ f(x_0) = f(\frac {x_0}{2}) = M$ and so $ f'(\frac {x_0}{2}) = 0$. 
An immediate induction give us $ f(\frac {x_0}{2^n}) = M$ and so $ M = 0$ (since $ g(0) = 0$ and $ g(x)$ is continuous at $ 0$).

Same, Let then $ m = \min_{x\in[0,u]}g(x)$ and any $ x_1\in[0,u]$ such that $ g(x_1) = m$. The same method implies $ m = 0$

So $ g(x) = 0$ $ \forall x\in[0,u]$. and so $ f(x) = \frac {f(u) - f(0)}{u}x + f(0)$ $ \forall x\in[0,u]$

So  $ \frac {f(u) - f(0)}{u} = \frac {f(v) - f(0)}{v}=a$ $ \forall u,v$ with same sign.

So $ f(x) = ax + b$ $ \forall x\geq 0$ and $ f(x) = cx + d$ $ \forall x\leq 0$. But continuity at $ 0$ implies $ b = d$ and existence of $ f'(0)$ implies $ a = c$

And so the only solutions are $ f(x) = ax + b$ $ \forall x\in\mathbb R$ (and we easily verify that these solutions fit).
\end{solution}
*******************************************************************************
-------------------------------------------------------------------------------

\begin{problem}[Posted by \href{https://artofproblemsolving.com/community/user/43536}{nguyenvuthanhha}]
	Let $ a , b$ be fixed real numbers and $ a < 0$. Let $ f: \mathbb{R} \ \to \ \mathbb{R}$ be a function such that 
\[f(f(x)) \ = \ ax + b ,\quad  \forall   x \in \mathbb{R}.\]
Prove that there are infinitely many value $x_0$ such that $f$ is discontinuous at $x_0$.
	\flushright \href{https://artofproblemsolving.com/community/c6h286483}{(Link to AoPS)}
\end{problem}



\begin{solution}[by \href{https://artofproblemsolving.com/community/user/29428}{pco}]
	\begin{tcolorbox}\begin{italicized}Let $ a , b$ be fixed real numbers , $ a < 0$

 $ f$ be a function , $ f: \mathbb{R} \ \to \ \mathbb{R}$  such that :

   $ f(f(x)) \ = \ ax + b \ \forall \ x \ \in \ \mathbb{R}$

Prove that there are infinite many value $ x_0$ such that :

  $ f$ is discontinuous at $ x_0$\end{italicized}\end{tcolorbox}

First, get rid of $ b$ :
Let $ u=\frac{b}{a-1}$ and $ f(x)=g(x+u)-u$. The equation becomes $ g(g(x+u))-u=ax+b=a(x+u)+b-au$ and so $ g(g(x))=ax$

$ g(g(x))=ax$ $ \implies$ $ g(x)$ is a bijection
$ g(g(x))=ax$ $ \implies$ $ g(ax)=ag(x)$ and so $ g(0)=0$. So, since $ g(x)$ is injective, $ g(x)\neq 0$ $ \forall x\neq 0$

Let then $ u>0$ and $ v=g(u)$ and $ w=g(v)=au$ and $ t=g(w)=av$
If $ v>0$, we have $ u>0$, $ v>0$, $ g(u)>0$ and $ g(v)<0$ and so a discontinuity point in $ (u,v)$
If $ v<0$, we have $ v<0$, $ w<0$, $ g(v)<0$ and $ g(w)>0$ and so a discontinuity point in $ (v,w)$
And so we have at least one discontinuity point $ x_0\neq 0$

Then 2 cases :

1) $ a\neq -1$
Then $ g(a^nx)=a^ng(x)$ and so, if $ x_0$ is a discontinuity point of $ g(x)$, $ a^nx_0$ is too. and so we have infinitely many discontinuity points.

2) $ a=-1$
Suppose we have a finite number of discontinuity points.
Consider the set of intervals $ (u,v)$ (including $ (u,+\infty)$ and $ (-\infty,v)$) where $ g(x)$ is continuous (with $ u,v$ discontinuity points or $ \pm\infty)$)

It's easy to see that we have an even number of such positive intervals, so an odd number of positive discontinuity points.
But, if $ u$ is a discontinuity point, $ v=g(u)$ is too and so are $ -u$ and $ -v$, so the set of positive discontinuity points need to be even.
Hence the contradiction.
And so we have infinitely many discontinuity points.
\end{solution}
*******************************************************************************
-------------------------------------------------------------------------------

\begin{problem}[Posted by \href{https://artofproblemsolving.com/community/user/43536}{nguyenvuthanhha}]
	Find all function $ f : \mathbb{R} \ \to \  \mathbb{R}$ such that :
\[ f(x^2 + f(y) - y) \ = \ (f(x))^2 - f(y), \quad \forall  x; y  \in \mathbb{R}.\]
	\flushright \href{https://artofproblemsolving.com/community/c6h286774}{(Link to AoPS)}
\end{problem}



\begin{solution}[by \href{https://artofproblemsolving.com/community/user/29428}{pco}]
	\begin{tcolorbox}\begin{italicized}Find all function $ f : \mathbb{R} \ \to \ \mathbb{R}$ such that :

   $ f(x^2 + f(y) - y) \ = \ (f(x))^2 - f(y) \ \ \forall \ \ x; y \in \mathbb{R}$\end{italicized}\end{tcolorbox}

Let $ P(x,y)$ be the assertion $ f(x^2+f(y)-y)=f(x)^2-f(y)$

Comparing $ P(x,y)$ and $ P(-x,y)$ implies $ f(x)^2=f(-x)^2$
Suppose now $ \exists u$ such that $ f(-u)=-f(u)$
$ P(0,u)$ $ \implies$ $ f(f(u)-u)=f(0)^2-f(u)$
$ P(0,-u)$ $ \implies$ $ f(-f(u)+u)=f(0)^2+f(u)$
But, since either $ f(-f(u)+u)=f(f(u)-u)$, either $ f(-f(u)+u)=-f(f(u)-u)$, we get that either $ f(u)=0$, either $ f(0)=0$ and so two cases :

\begin{bolded}Case 1\end{bolded} : \end{underlined}$ f(0)\neq 0$ and so $ f(-u)=-f(u)$ $ \implies$ $ f(u)=0$
So $ f(-x)=f(x)$ $ \forall x$
Suppose now that $ \exists u$ such that $ f(u)<0$
$ P(\sqrt{-f(u)},u)$ $ \implies$ $ f(-u)=f(\sqrt{-f(u)})^2-f(u)$ $ \implies$ $ 2f(u)=f(\sqrt{-f(u)})^2\geq 0$, hence contradiction.
So $ f(x)\geq 0$ $ \forall x$
So $ f(x)^2\geq f(y)$ $ \forall x,y$ 
So $ f(x)\in[0,M]$ is bounded.

Comparing $ P(x,y)$ and $ P(x,-y)$, we get that $ f(x^2+f(y)-y)=f(x^2+f(y)+y)$ $ \forall x,y$
So $ f(x)=f(x+2y)$ $ \forall x\geq M$, $ \forall y>0$ and so $ f(x)=c$ $ \forall x\geq M$
Now, for any $ y$, it is possible to choose $ x$ such that $ x^2+f(y)-y>M$ and $ x>M$.
Then $ P(x,y)$ $ \implies$ $ c=c^2-f(y)$ $ \forall y$

And so we get the solution $ f(x)=c$ $ \forall x$ with $ c=c^2-c$
And so $ \boxed {f(x)=2\forall x}$ (we dont consider the solution $ f(x)=0$ here since in the case 1 we have $ f(0)\neq 0$

\begin{bolded}Case 2\end{bolded} :\end{underlined} $ f(0)=0$
Then $ P(x,0)$ $ \implies$ $ f(x^2)=f(x)^2$ and so $ f(x)\geq 0$ $ \forall x\geq 0$
Then, suppose $ f(u)>0$, then $ P(0,u)$ $ \implies$ $ f(f(u)-u)=-f(u)< 0$ and so $ f(u)-u<0$ and so $ u>f(u)>0$
So, $ f(-u)=-f(u)$ $ \forall u$. Else, we'd get a $ v<0$ such that $ f(v)>0$

Then $ P(x,x^2)$ $ \implies$ $ f(f(x^2))=f(x)^2-f(x^2)=0$ And so $ f(f(x))=0$ $ \forall x\geq 0$ and so (since odd) $ f(f(x))=0$ $ \forall x$
Let then $ y>0$, and so $ f(y)\geq 0$
Then $ P(\sqrt{\frac{f(y)}{2}},f(y))$ $ \implies$ $ f(-\frac{f(y)}{2})=f(\frac{f(y)}{2})^2\geq 0$ and so $ f(y)\leq 0$ and so $ f(y)=0$

And so the only solution in this case n^° 2 : $ \boxed{f(x)=0\forall x}$
\end{solution}



\begin{solution}[by \href{https://artofproblemsolving.com/community/user/52090}{Dumel}]
	\begin{tcolorbox}  
\begin{bolded}Case 1\end{bolded} : \end{underlined}$ f(0)\neq 0$ and so $ f( - u) = - f(u)$ $ \implies$ $ f(u) = 0$
So $ f( - x) = f(x)$ $ \forall x$\end{tcolorbox}Why?  :huh:
\end{solution}



\begin{solution}[by \href{https://artofproblemsolving.com/community/user/29428}{pco}]
	\begin{tcolorbox}[quote="pco"]  
\begin{bolded}Case 1\end{bolded} : \end{underlined}$ f(0)\neq 0$ and so $ f( - u) = - f(u)$ $ \implies$ $ f(u) = 0$
So $ f( - x) = f(x)$ $ \forall x$\end{tcolorbox}Why?  :huh:\end{tcolorbox}

In the four first lines, we shown :

That $ f(x)^2=f(-x)^2$ and so that $ \forall x$ either $ f(-x)=f(x)$, either $ f(-x)=-f(x)$
Then, we said that if $ f(-u)=-f(u)$ for some $ u$, then either $ f(0)=0$, either $ f(u)=0$
Then, in case 1, we condidered $ f(0)\neq 0$. So, if $ f(-u)=-f(u)$, then $ f(u)=0$
So, $ \forall x$ either $ f(-x)=f(x)$, either $ f(-x)=-f(x)$ but then $ f(x)=0$ and so $ f(-x)=-f(x)=0=f(x)$

So, in case 1, we have $ f(-x)=f(x)$ $ \forall x$
\end{solution}



\begin{solution}[by \href{https://artofproblemsolving.com/community/user/54383}{EastyMoryan}]
	Nevermind this ><
\end{solution}



\begin{solution}[by \href{https://artofproblemsolving.com/community/user/29428}{pco}]
	\begin{tcolorbox}Nevermind this ><\end{tcolorbox}

I dont understand what this mean, sorry
\end{solution}



\begin{solution}[by \href{https://artofproblemsolving.com/community/user/7594}{Xantos C. Guin}]
	He probably posted something, then realized that it was a mistake or something, then edited the post rather than deleting it.
\end{solution}
*******************************************************************************
-------------------------------------------------------------------------------

\begin{problem}[Posted by \href{https://artofproblemsolving.com/community/user/46787}{moldovan}]
	A function $ f: \mathbb{N} \rightarrow \mathbb{N}$ satisfies:
$ (a)$ $ f(ab)=f(a)f(b)$ whenever $ a$ and $ b$ are coprime;
$ (b)$ $ f(p+q)=f(p)+f(q)$ for all prime numbers $ p$ and $ q$.
Prove that $ f(2)=2,f(3)=3$ and $ f(1999)=1999.$
	\flushright \href{https://artofproblemsolving.com/community/c6h286832}{(Link to AoPS)}
\end{problem}



\begin{solution}[by \href{https://artofproblemsolving.com/community/user/29428}{pco}]
	\begin{tcolorbox}A function $ f: \mathbb{N} \rightarrow \mathbb{N}$ satisfies:
$ (a)$ $ f(ab) = f(a)f(b)$ whenever $ a$ and $ b$ are coprime;
$ (b)$ $ f(p + q) = f(p) + f(q)$ for all prime numbers $ p$ and $ q$.
Prove that $ f(2) = 2,f(3) = 3$ and $ f(1999) = 1999.$\end{tcolorbox}

Let $ p$ an odd prime : $ 2,p$ coprime $ \implies$ $ f(2p)=f(2)f(p)$ but $ p$ prime $ \implies$ $ f(p+p)=f(p)+f(p)$ and so $ f(2)=2$

$ f(2+2)=2f(2)$ and so $ f(4)=4$
$ f(5)=f(3+2)=f(3)+f(2)=f(3)+2$
$ f(7)=f(5+2)=f(5)+f(2)=f(3)+4$

$ f(12)=f(3)f(4)=4f(3)$
$ f(12)=f(5+7)=f(5)+f(7)=2f(3)+6$

So $ 4f(3)=2f(3)+6$ and $ f(3)=3$ and so $ f(5)=5$ and $ f(7)=7$

$ 3$ and $ 5$ coprime $ \implies$ $ f(15)=f(3\times 5)=f(3)f(5)=15$
$ 13$ and $ 2$ primes $ \implies$ $ f(15)=f(13+2)=f(13)+f(2)$ $ \implies$ $ f(13)=13$
$ 11$ and $ 2$ primes $ \implies$ $ f(13)=f(11+2)=f(11)+f(2)$ $ \implies$ $ f(11)=11$
$ 11$ and $ 3$ coprime $ \implies$ $ f(33)=f(3\times 11)=f(3)f(11)=33$
$ 31$ and $ 2$ primes $ \implies$ $ f(33)=f(31+2)=f(31)+f(2)$ $ \implies$ $ f(31)=31$
$ 29$ and $ 2$ primes $ \implies$ $ f(31)=f(29+2)=f(29)+f(2)$ $ \implies$ $ f(29)=29$
$ 13$ and $ 2$ coprime $ \implies$ $ f(26)=f(2\times 13)=f(2)f(13)=26$
$ 23$ and $ 3$ primes $ \implies$ $ f(26)=f(23+3)=f(23)+f(3)=f(23)+3$ $ \implies$ $ f(23)=23$
$ 23$ and $ 29$ coprime $ \implies$ $ f(667)=f(23\times 29)=f(23)f(29)=667$
$ 667$ and $ 3$ coprime $ \implies$ $ f(2001)=f(3\times 667)=f(3)f(667)=2001$
$ 1999$ and $ 2$ primes $ \implies$ $ f(2001)=f(1999+2)=f(1999)+2$ $ \implies$ $ f(1999)=1999$
\end{solution}
*******************************************************************************
-------------------------------------------------------------------------------

\begin{problem}[Posted by \href{https://artofproblemsolving.com/community/user/46787}{moldovan}]
	Determine all functions $ f: \mathbb{N} \rightarrow \mathbb{N}$ which satisfy:

$ f(x+f(y))=f(x)+y$ for all $ x,y \in \mathbb{N}$.
	\flushright \href{https://artofproblemsolving.com/community/c6h287034}{(Link to AoPS)}
\end{problem}



\begin{solution}[by \href{https://artofproblemsolving.com/community/user/29428}{pco}]
	\begin{tcolorbox}Determine all functions $ f: \mathbb{N} \rightarrow \mathbb{N}$ which satisfy:

$ f(x + f(y)) = f(x) + y$ for all $ x,y \in \mathbb{N}$.\end{tcolorbox}

$ f(x+f(y))=f(x)+y$ $ \implies$ $ f(x+nf(y))=f(x)+ny$
So $ f(x+f(z)f(y))=f(x)+f(z)y$ and, for symetry between $ z$ and $ y$ : $ f(x)+f(z)y=f(x)+zf(y)$ and so $ \frac{f(y)}{y}=\frac{f(z)}{z}$ and so $ f(x)=ax$

Plugging back in the original equation, we get $ a=1$ and so $ f(x)=x$ $ \forall x$
\end{solution}
*******************************************************************************
-------------------------------------------------------------------------------

\begin{problem}[Posted by \href{https://artofproblemsolving.com/community/user/46787}{moldovan}]
	For each nonzero integer $ n$ find all functions $ f: \mathbb{R} - \{-3,0 \} \rightarrow \mathbb{R}$ satisfying:

$ f(x+3)+f \left( -\frac{9}{x} \right)=\frac{(1-n)(x^2+3x-9)}{9n(x+3)}+\frac{2}{n}$ for all $ x \not= 0,-3.$

Furthermore, for each fixed $ n$ find all integers $ x$ for which $ f(x)$ is an integer.
	\flushright \href{https://artofproblemsolving.com/community/c6h288025}{(Link to AoPS)}
\end{problem}



\begin{solution}[by \href{https://artofproblemsolving.com/community/user/29428}{pco}]
	\begin{tcolorbox}For each nonzero integer $ n$ find all functions $ f: \mathbb{R} - \{ - 3,0 \} \rightarrow \mathbb{R}$ satisfying:

$ f(x + 3) + f \left( - \frac {9}{x} \right) = \frac {(1 - n)(x^2 + 3x - 9)}{9n(x + 3)} + \frac {2}{n}$ for all $ x \not = 0, - 3.$\end{tcolorbox}
If the problem statement is correct, no solution exists since the equation must be true for $ x=3$ but then $ f(-\frac 9x)=f(-3)$ is not defined.
So I think we must understand that $ f: \mathbb R\backslash\{0,+3\}\to\mathbb R$

With this modification, this equation may be written $ f(x)+f(\frac{-9}{x-3})=\frac{n-1}{n}(-\frac{x-3}{9}+\frac 1x)+\frac 2n$ $ \forall x\notin\{0,3\}$ and so :

Let then $ g(x)=\frac{-9}{x-3}$ (notice that we have $ g(g(g(x)))=x$ $ \forall x\notin\{0,3\}$

The equation is $ f(x)+f(g(x))=\frac{n-1}{n}(\frac{1}{g(x)}+\frac{1}{x})+\frac 2n$ $ \forall x\notin\{0,3\}$

So :
$ f(x)+f(g(x))=\frac{n-1}{n}(\frac{1}{g(x)}+\frac{1}{x})+\frac 2n$ $ \forall x\notin\{0,3\}$

$ f(g(x))+f(g(g(x)))=\frac{n-1}{n}(\frac{1}{g(g(x))}+\frac{1}{g(x)})+\frac 2n$ $ \forall x\notin\{0,3\}$

$ f(g(g(x)))+f(x)=\frac{n-1}{n}(\frac{1}{x}+\frac{1}{g(g(x))})+\frac 2n$ $ \forall x\notin\{0,3\}$

And so, (adding the first and the third and subtracting the second) : $ \boxed{f(x)=\frac{n-1}{nx}+\frac 1n}$ $ \forall x\notin\{0,3\}$
And it is immediate to verify that this necessary condition is also sufficient.

\begin{tcolorbox}Furthermore, for each fixed $ n$ find all integers $ x$ for which $ f(x)$ is an integer.\end{tcolorbox}

$ f(x)=\frac{x+n-1}{nx}$ can be an integer only :
For $ n\notin\{-2,1\}$ : if $ x\in\{-(n-1),1\}$
For $ n=1$ : $ \forall x\notin\{0,3\}$
For $ n=-2$ : if $ x=1$
\end{solution}
*******************************************************************************
-------------------------------------------------------------------------------

\begin{problem}[Posted by \href{https://artofproblemsolving.com/community/user/46787}{moldovan}]
	Find all functions $ f: \mathbb{Z} - \{ 0 \} \rightarrow \mathbb{Q}$ satisfying:

$ f \left( \frac{x+y}{3} \right)=\frac {f(x)+f(y)}{2},$ whenever $ x,y,\frac{x+y}{3} \in \mathbb{Z} - \{ 0 \}.$
	\flushright \href{https://artofproblemsolving.com/community/c6h288038}{(Link to AoPS)}
\end{problem}



\begin{solution}[by \href{https://artofproblemsolving.com/community/user/29428}{pco}]
	\begin{tcolorbox}Find all functions $ f: \mathbb{Z} - \{ 0 \} \rightarrow \mathbb{Q}$ satisfying:

$ f \left( \frac {x + y}{3} \right) = \frac {f(x) + f(y)}{2},$ whenever $ x,y,\frac {x + y}{3} \in \mathbb{Z} - \{ 0 \}.$\end{tcolorbox}

Replacing $ x$ with $ x+z$ and $ y$ with $ y-z$, we get $ f(x+z)+f(y-z)=f(x)+f(y)$ $ \forall z\in\mathbb Z\backslash\{-x,y\}$, $ \forall x,y,\frac{x+y}3\in\mathbb Z\backslash\{0\}$

So $ f(x+z)=f(x)+g_i(z)$ $ \forall z\in\mathbb Z\backslash\{-x\}$, $ \forall x\in\mathbb Z\backslash\{0\}$, where $ i\in\{0,1,2\}$ is $ x\pmod 3$
So $ f(x+zt)=f(x)+tg_i(z)$
So $ g_i(z)=c_iz$
So $ f(x+z)=f(x)+c_iz$ and :

$ f(z+1)=f(1)+c_1z$
$ f(z+2)=f(2)+c_2z$
$ f(z+3)=f(3)+c_0z$

But :
$ f(z+2)=f(1)+c_1(z+1)$
$ f(z+3)=f(1)+c_1(z+2)$

So $ c_0=c_1=c_2=a$ and $ f(x)=ax+b$

Plugging back this in the original equation, we get $ a=0$ and :

$ f(x)=b$ $ \forall x\neq 0$
\end{solution}
*******************************************************************************
-------------------------------------------------------------------------------

\begin{problem}[Posted by \href{https://artofproblemsolving.com/community/user/46787}{moldovan}]
	Find all functions $ f: \mathbb{R} \rightarrow \mathbb{R}$ such that for all $ x,y,z$ it holds that:

$ f(x+f(y+z))+f(f(x+y)+z)=2y.$
	\flushright \href{https://artofproblemsolving.com/community/c6h288281}{(Link to AoPS)}
\end{problem}



\begin{solution}[by \href{https://artofproblemsolving.com/community/user/29428}{pco}]
	\begin{tcolorbox}Find all functions $ f: \mathbb{R} \rightarrow \mathbb{R}$ such that for all $ x,y,z$ it holds that:

$ f(x + f(y + z)) + f(f(x + y) + z) = 2y.$\end{tcolorbox}

Let $ x=z=t-f(0)$, $ y=f(0)-t$. The equation becomes $ f(t)=f(0)-t$

Plugging back in the equation, we verify that this solution fits the requirements and so $ \boxed{f(x)=a-x}$
\end{solution}



\begin{solution}[by \href{https://artofproblemsolving.com/community/user/46787}{moldovan}]
	very nice solution  
\end{solution}



\begin{solution}[by \href{https://artofproblemsolving.com/community/user/52090}{Dumel}]
	my solution:
1. $ x=z=0$
2. $ x=z=-y$
\end{solution}
*******************************************************************************
-------------------------------------------------------------------------------

\begin{problem}[Posted by \href{https://artofproblemsolving.com/community/user/50645}{stvs_f}]
	Find a function $f: \mathbb N \to \mathbb N$ such that for all positive integers $n$, \[ f(f(n))=n^2.\]
	\flushright \href{https://artofproblemsolving.com/community/c6h289799}{(Link to AoPS)}
\end{problem}



\begin{solution}[by \href{https://artofproblemsolving.com/community/user/29428}{pco}]
	\begin{tcolorbox}find $ F: N \to N$ such that :
$ \forall n \in N$ :
\[ f(f(n)) = n^2\]
\end{tcolorbox}

Let $ A=\{x$ such that $ x$ is a natural number but not the square of a natural number$ \}$

For any $ n>1$, let $ h(n)\in A$ and $ p(n)\in\mathbb N\cup\{0\}$ such that $ n=h(n)^{2^{p(n)}}$

Obviously $ h(n)$ and $ p(n)$ exist and are unique for any $ n>1$

Let $ U$ and $ V$ a split ($ U\cup V=A$ and $ U\cap V=\emptyset$) of $ A$ in two equinumerous subsets and $ b(x)$ a bijection from $ U\to V$ (such subsets exist since $ A$ is an infinite set).


Then the general solution of the required equation is :

$ f(1)=1$
$ \forall n$ such that $ h(n)\in U$ : $ f(n)=b(h(n))^{2^{p(n)}}$
$ \forall n$ such that $ h(n)\in V$ : $ f(n)=b^{-1}(h(n))^{2^{p(n)+1}}$
\end{solution}
*******************************************************************************
-------------------------------------------------------------------------------

\begin{problem}[Posted by \href{https://artofproblemsolving.com/community/user/37877}{every}]
	Let $ f$ be a function on natural numbers $ N->N$ with the following properties:
1.$ (f(2n)+f(2n+1)+1)(f(2n+1)-f(2n)-1)=3(1+2f(n))$
2.$ f(2n)\ge f(n)$
for all natural numbers $ n$
Determine all values of $ n$ such that $ f(n)\le 2009$
	\flushright \href{https://artofproblemsolving.com/community/c6h289830}{(Link to AoPS)}
\end{problem}



\begin{solution}[by \href{https://artofproblemsolving.com/community/user/29428}{pco}]
	\begin{tcolorbox}Let $ f$ be a function on natural numbers $ N - > N$ with the following properties:
1.$ (f(2n) + f(2n + 1) + 1)(f(2n + 1) - f(2n) - 1) = 3(1 + 2f(n))$
2.$ f(2n)\ge f(n)$
for all natural numbers $ n$
Determine all values of $ n$ such that $ f(n)\le 2009$\end{tcolorbox}

$ RHS > 0$ $ \implies$ $ f(2n + 1) - f(2n) - 1 > 0$
If $ f(2n + 1) - f(2n) - 1\geq 3$, then $ f(2n) + f(2n + 1) + 1\geq 2f(2n) + 5$ and $ LHS\geq 6f(2n) + 15$ so $ 3(1 + 2f(n))\geq 6f(2n) + 15$ and so $ f(n)\geq f(2n) + 2$, which is wrong.

So $ 3 > f(2n + 1) - f(2n) - 1 > 0$

If $ f(2n + 1) - f(2n) - 1 = 2$, then LHS is even while RHS is odd.

So $ f(2n + 1) - f(2n) - 1 = 1$ $ \implies$ $ f(2n + 1) = f(2n) + 2$ $ \implies$ the equation becomes $ 2f(2n) + 3 = 3 + 6f(n)$ and so $ f(2n) = 3f(n)$

So $ f(2n) = 3f(n)$ and $ f(2n + 1) = 3f(n) + 2$

So knowledge of $ f(1)$ is enough to determine $ f(n)$ $ \forall n$ but we cant go further since we have no information about $ f(1)$. For example, $ f(1) = 10000$ would imply that no value of $ n$ would be such that $ f(n)\leq 2009$ while $ f(1)=1$ would imply $ f(n)\leq 2009$ $ \forall n\in[1,127]$ and $ f(n)>2009$ $ \forall n>127$
\end{solution}



\begin{solution}[by \href{https://artofproblemsolving.com/community/user/44674}{Allnames}]
	\begin{tcolorbox}Let $ f$ be a function on natural numbers $ N - > N$ with the following properties:
1.$ (f(2n) + f(2n + 1) + 1)(f(2n + 1) - f(2n) - 1) = 3(1 + 2f(n))$
2.$ f(2n)\ge f(n)$
for all natural numbers $ n$
Determine all values of $ n$ such that $ f(n)\le 2009$\end{tcolorbox}
No No, every. You posted a problem in MYM 45th anniversary contest. It is the greatest contest during  a decade of the Mathematics and Young magazine!. And now the contest is still running. So posting all of problems should be prohibited.
Dear moderators! Please lock this topic and warm every ( he posted many many problems like that)
However , dear pco. I think you should check your solution more carefully. I think your result isn't right!. Read my pm !
\end{solution}
*******************************************************************************
-------------------------------------------------------------------------------

\begin{problem}[Posted by \href{https://artofproblemsolving.com/community/user/28454}{danilorj}]
	Find all functions $f: \mathbb R \to \mathbb R$ such that \[ f(x+y+f(y))=4x-f(x)+f(3y)\] for all $x, y \in \mathbb R$.
	\flushright \href{https://artofproblemsolving.com/community/c6h291010}{(Link to AoPS)}
\end{problem}



\begin{solution}[by \href{https://artofproblemsolving.com/community/user/29428}{pco}]
	\begin{tcolorbox}Find all functions f: $ \Re\to\Re$ such: $ f(x + y + f(y)) = 4x - f(x) + f(3y)$.\end{tcolorbox}

Let $ P(x,y)$ be the assertion $ f(x+y+f(y))=4x-f(x)+f(3y)$
Let $ f(0)=a$

$ P(0,0)$ $ \implies$ $ f(a)=0$
$ P(a,0)$ $ \implies$ $ f(2a)=5a$
$ P(2a,0)$ $ \implies$ $ f(3a)=4a$
$ P(0,a)$ $ \implies$ $ f(3a)=a$

So $ 4a=a$ and $ a=0$.

Then $ P(x,0)$ $ \implies$ $ f(x)=4x-f(x)$ and so $ \boxed{f(x)=2x}$ 
And putting back this value in the original equation shows that this indeed is a solution.
\end{solution}
*******************************************************************************
-------------------------------------------------------------------------------

\begin{problem}[Posted by \href{https://artofproblemsolving.com/community/user/46488}{Raja Oktovin}]
	Let $ \mathbb{R}^ +$ be the set of all positive real numbers. Find all functions $ f: \mathbb{R}^ + \rightarrow \mathbb{R}^ +$ satisfying
\[ f(x)f(y) = f(xy) + f\left(\frac {x}{y}\right)\]
for all positive real numbers $ x$ and $ y$.
	\flushright \href{https://artofproblemsolving.com/community/c6h291897}{(Link to AoPS)}
\end{problem}



\begin{solution}[by \href{https://artofproblemsolving.com/community/user/29428}{pco}]
	\begin{tcolorbox}Let $ \mathbb{R}^ +$ be the set of all positive real numbers. Find all functions $ f: \mathbb{R}^ + \rightarrow \mathbb{R}^ +$ satisfying
\[ f(x)f(y) = f(xy) + f\left(\frac {x}{y}\right)\]
for all positive real numbers $ x$ and $ y$.\end{tcolorbox}

At least we have the solutions $ f(x)=2\cosh(g(\ln(x)))$ where $ g(x)$ is any solution of the Cauchy equation $ g(x+y)=g(x)+g(y)$

With continuous solutions $ g(x)=ax$, we get $ x^a+x^{-a}$ for any real $ a$

The difficulty (for me) is to see if these are the only solutions or not ...
To be continued.
\end{solution}



\begin{solution}[by \href{https://artofproblemsolving.com/community/user/29428}{pco}]
	Hello Raja Oktovin,

Could you now give us your solution, please ?
I give up and it seems no one is searching again.

Thanks.
\end{solution}



\begin{solution}[by \href{https://artofproblemsolving.com/community/user/46488}{Raja Oktovin}]
	I can't do this problem so I posted this here...
Actually this topic subject is actually "help me..." 
So... someone please...  :(
\end{solution}



\begin{solution}[by \href{https://artofproblemsolving.com/community/user/1276}{Nukular}]
	A reduction:

Let $ f(x) = g(\log x)$ (doable since $ f$ is defined on $ \mathbb{R}^ +$), and lettng $ a = \log x, b = \log y$,

$ g(a)g(b) = g(a + b)g(a - b).$

Substituting $ b = 0$, $ g(a)g(b) = 2g(a)$, so since $ g(a) > 0$, we know $ g(0) = 2$.

Switching the roles of $ a$ and $ b$, it is evident $ g( - a) = g(a)$. 

However, here's the big issue: suppose $ \sigma : \mathbb{R} \to \mathbb{R}$ is an additive function,
i.e. one satisfying the Cauchy equation $ \sigma(x + y) = \sigma(x) + \sigma(y)$; then
$ g(\sigma a)g(\sigma b) = g(\sigma a + \sigma b) + g (\sigma a - \sigma b) = g(\sigma(a + b)) + g(\sigma(a - b))$.

In other words, if $ g$ is a solution, so is $ g \circ \sigma$; so we're already going to have that amazingly large
degree of freedom.

Not to say it's not solvable (it probably is), but there are definitely an infinitude of ugly solution functions $ g$ if
there are any nontrivial ones.

\begin{tcolorbox}[quote="Raja Oktovin"]Let $ \mathbb{R}^ +$ be the set of all positive real numbers. Find all functions $ f: \mathbb{R}^ + \rightarrow \mathbb{R}^ +$ satisfying
\[ f(x)f(y) = f(xy) + f\left(\frac {x}{y}\right)\]
for all positive real numbers $ x$ and $ y$.\end{tcolorbox}

At least we have the solutions $ f(x) = 2\cosh(g(\ln(x)))$ where $ g(x)$ is any solution of the Cauchy equation $ g(x + y) = g(x) + g(y)$

With continuous solutions $ g(x) = ax$, we get $ x^a + x^{ - a}$ for any real $ a$

The difficulty (for me) is to see if these are the only solutions or not ...
To be continued.\end{tcolorbox}

Actually, can you show that $ f(x) \ge 2$ all the time? And, if so, does your reduction $ f(x) = 2 \cosh(g \ln(x))$ imply $ g(x+y) = g(x) + g(y)$? If so, that's
a full solution right there.
\end{solution}



\begin{solution}[by \href{https://artofproblemsolving.com/community/user/1276}{Nukular}]
	Actually, might have something from my previous work there... 

So, subbing in $ b = 1$,

$ g(b+1) - g(1)g(b) + g(b-1) = 0$,

or $ g(x) - \alpha g(x-1) + g(x-2) = 0$, where $ \alpha = g(1)$. This gives a linear recurrence
with $ g(0) = 2$ and $ g(1) = \alpha$, which has solution $ g(n) = \kappa^n + \kappa^{-n}$, where
$ \kappa + \kappa^{-1} = \alpha$. If $ \alpha < 2$, then $ \kappa$ is a unit complex number, and $ \kappa^n + \kappa^{-n}$ is eventually negative; this cannot be allowed to happen, so $ \alpha \ge 2$ always.

 Let $ \kappa$ float as a function of $ \alpha$.

 This is necessarily true for any solution $ g$. 

Now let $ h(x) : = g(kx)$; this is also a solution, and so $ h(n) = \kappa^n + \kappa^{-n}$ (where
$ \kappa + \kappa^{-1} = h(1) = g(k)$) This incidentally shows that $ g(x) \ge 2$ for every $ x$.

Therefore $ g(nk) = \kappa(g(k))^n + \kappa(g(k))^{-n}$ (Here $ \kappa$ is a function -- excuse
the abuse of notation).

So, if we know $ g$ on any value $ k$, we also know it on any of its integral multiples. In fact, it is the case that $ g(k)$ is recoverable from $ g(nk)$ from the above -- the function taking $ \kappa$ to $ \kappa^n + \kappa^{-n}$ is invertible modulo the action $ \kappa \mapsto 1\/\kappa$ -- from which $ g(qk)$ can be determined for all rational $ q$. 

So, we're done... and $ g$ is determined from an arbitrary positive function on basis elements of the space of reals over the rationals. 

(If you dig thru the above and actually write out the construction, it's probably something along the lines of what pco wrote above... but what this does do is show we've gotten all the solutions through that line of reasoning.)
\end{solution}



\begin{solution}[by \href{https://artofproblemsolving.com/community/user/29428}{pco}]
	@Raja Oktovin :
\begin{tcolorbox}I can't do this problem so I posted this here...
\end{tcolorbox}
Ok, thanks. Better to post in "Algebra unsolved Problems", next time, IMHO.

@Nukular :
\begin{tcolorbox}So, we're done... and  is determined from an arbitrary positive function on basis elements of the space of reals over the rationals. 
\end{tcolorbox}

Thanks a lot for your kind remarks, though I'm not sure to have understood all your conclusions :

From the basis $ g(a)g(b)=g(a+b)+g(a-b)$, we get that this is the classical D'Alembert equation (without the continuity requirement) and the solutions $ 2\cos(ax)$ and $ 2\cosh(ax)$ for $ x\in\mathbb Q$ are classical results. The first can be eliminated since we are looking for positive solutions.

So $ f(x)=2\cosh(a\ln(x))$ $ \forall x\in\mathbb Q$ is a normal result deduced from D'Alembert equation.
And $ f(x)=2\cosh(\sigma(\ln(x)))$ is a family of solution, as I already stated.

But I dont see where (or if) you prove that this is is the unique family, which was my question.

Sorry if I misunderstood or did not see all your demo.
\end{solution}



\begin{solution}[by \href{https://artofproblemsolving.com/community/user/1276}{Nukular}]
	The argument has a couple of key lemmas (and, yes, is missing a last step):

Solutions for $ f(x)f(y) = f(xy) + f(x\/y)$ are directly in bijective correspondence to positive functions $ g$ for which $ g(x + y) + g(x - y) = g(x)g(y)$; so I'm
going to stick with that latter equation (*), and solve there. (Eliminates the $ \ln$ term for simplicity).

Let $ \kappa : [2,\infty) \to [1,\infty)$ be defined such that $ \kappa(x) + \kappa(x)^{ - 1} = x$.

Lemma 1) For any solution $ g$, $ g(n) = \kappa(g(1))^n + \kappa(g(1))^{ - n}$ (**) for all integers $ n$. Furthermore, $ g(1) \ge 2$.

I proved this in my earlier post, by considering the recurrence $ g(n + 2) - g(1)g(n + 1) + g(n) = 0$.
 
Lemma 2) If $ \sigma$ is a solution to the Cauchy equation, then $ g \circ \sigma$ is a solution to (*) whenever $ g$ is a solution.

Lemma 3) If $ g$ is a solution to (*), then $ g(kn) = \kappa(g(k))^n + \kappa(g(k))^{ - n}$ (***) for all reals $ k$; furthermore, $ g(k) \ge 2$ for all real $ k$.
Proof: $ \sigma(x) = kx$ is a solution to the Cauchy equation; letting $ h = g \circ \sigma$ in equation (**) gives equation (***).

Lemma 4) $ g(p\/q)$ is determined completely by $ g(1)$, for all integers $ p,q$.
Proof: We have $ \kappa(g(p\/q))^q + \kappa(g(p\/q))^{ - q} = g(p) = \kappa(g(1))^p + \kappa(g(1))^{ - p}$. There is only one solution for $ g(p\/q)$ from this 
equation.

Lemma 5) $ g(kx)$ is determined uniquely, given $ g(x)$, for any rational $ k$ and any real $ x$.
Proof: Same trick as before, apply Lemma 4 to $ g \circ (y \mapsto xy)$.

------------

Yeah, and I have to still add a lemma which gives $ g(x + y)$ from $ g(x)$ and $ g(y)$, provided $ x$ and $ y$ are independent over $ \mathbb{Q}$. Will 
work on that next.

EDIT: Was not aware of this being a classical D'Alembert equation; so most of this is probably just reproducing what you already know. But, at least
it's true that we know $ g$ is at least determined by it's footprint on $ \mathbb{Q} -$projective equivalence classes in $ \mathbb{R}$.

EDIT 2: Yes, sadly, I believe you get a single additional degree of freedom when you add up two reals from differing rational lines... might be a 
ton more solutions; though hopefully not :).

EDIT 3: This can probably be solved if you consider $ g$ to be a positive function from $ \mathbb{Q}^2 \to \mathbb{R}$; if you get more than two degrees of freedom of solutions, then we're going to have a ton here.
\end{solution}



\begin{solution}[by \href{https://artofproblemsolving.com/community/user/1276}{Nukular}]
	Heh: or, you could notice

$ g(a+b)g(a-b) = g(2a) + g(2b)$

and 

$ g(a)g(b) = g(a+b)+g(a-b)$.

Which should let you get values for both $ g(a+b)$ and $ g(a-b)$. So you probably can build it up once you know $ g$ on a basis.

Back to this when I've had some sleep.
\end{solution}



\begin{solution}[by \href{https://artofproblemsolving.com/community/user/1276}{Nukular}]
	Or not: 

Sketch: 

Use the equations $ g(a)g(b) = g(a+b)g(a-b)$ and $ g(a+b)g(a-b) = g(2a) + g(2b)$ to lift $ g$ from disjoint
a vector subspace $ V \subset \mathbb{R}$ to $ V \oplus \mathbb{Q}b$. However, this requires a choice, since
unless one of $ a,b$ is zero, then $ g(a+b)$ and $ g(a-b)$ are different, and switching the 
values of $ g(a+b)$ and $ g(a-b)$ can be accomplished by a reflection across $ b^\perp$, without interfering
with other values of $ g$. Once $ g(a+b)$ is determined, so is $ g(a + qb)$ for rational $ q$, and therefore so is $ g(ra + qb)$ for rational $ r$.

So, given the value of $ g$ on each basis element of $ \mathbb{R}$ over $ \mathbb{Q}$, 
we must also choose, for each basis element (other than the first), an orientation for $ g$ to grow in --
i.e. whether $ g(a+b)$ or $ g(a-b)$ is the larger. This is a complete family of solutions for $ g$.

Yeah, lots of axiom of choice, but I think it works this time around.
\end{solution}



\begin{solution}[by \href{https://artofproblemsolving.com/community/user/29428}{pco}]
	\begin{tcolorbox} This is a complete family of solutions for $ g$.\end{tcolorbox}
Hemmm, thanks a lot ...
I'll try to read all your demo when it will be finished ...

Is your family THE complete family of solutions ?

And, if yes, is your family equal to $ \{2\cosh(\sigma(\ln(x)))\}$ where $ \sigma(x)$ is any solution of $ f(x + y) = f(x) + f(y)$ ?
This was my question and I dont understand if you answered it or not.
Sorry again if I misunderstood.
\end{solution}



\begin{solution}[by \href{https://artofproblemsolving.com/community/user/1276}{Nukular}]
	What I have shown: 

$ g$ is completely determined from its values on a basis for $ \mathbb{R}$ over $ \mathbb{Q}$, along with 
a direction for all but one basis element; this "direction" tells us whether $ g(a + b)$ or $ g(a - b)$ is the larger.
For any values $ \ge 2$ chosen on basis elements, along with choices of direction, a unique $ g$ can be 
constructed. So, this is the entire set of such functions $ g$.

However, this "choice of direction" is not substantial, since $ g$ is an even function; so it rather chooses
an automorphism of $ \mathbb{R}$.

So... yes, I your $ 2\cosh(\sigma(\ln x))$ is a complete family of functions, provided that my earlier
work computes $ 2 \cosh (\ln x)$ as the only rational solution (I get a \begin{italicized} unique \end{italicized} solution, so
if you are correct these would coincide).
\end{solution}



\begin{solution}[by \href{https://artofproblemsolving.com/community/user/1276}{Nukular}]
	Yes, assuming that nice family $ g(x) = 2\cosh (x)$ for rationals;

$ g(x+y) + g(x-y) = g(x)g(y) : = ab$
$ g(x+y)g(x-y) = g(2x)g(2y) : = a^2 + b^2 -4$,

solving gives that 

$ g(x+y) = \frac{ab \pm \sqrt{(a^2-4)(b^2-4)}}{2} = 2\cosh(x)\cosh(y) \pm 2\sinh(x)\sinh(y) = 2 \cosh(x \pm y)$,

but that $ \pm$ depends on our choice of $ \sigma$, since $ \cosh$ is an even function.

So, definitively, the solution set is $ f(x) = 2\cosh(\sigma(\ln x))$.

I appreciate the constant adversarial criticism to make me provide a simpler more constructive solution  .
\end{solution}



\begin{solution}[by \href{https://artofproblemsolving.com/community/user/44674}{Allnames}]
	We see that $ f(x) = f(\frac {1}{x})$; $ f(1) = 1$;$ f(x^2) = f^2(x) - 2$. Thus $ f(x) > \sqrt {2 + \sqrt {2 + ...}} = 2$ when the number of root tends to the infinite . It means $ f(x) > 2$ for all positive numbers $ x$.
Notice that $ f(x^2) + f(y^2) = f(xy)f(\frac {x}{y})$. It suffices to write $ x^2 = xy\frac {x}{y}$ and $ y^2 = \frac {xy}{\frac {x}{y}}$ thus $ f^2(x) + f^2(y) - 4 = f(x^2) + f(y^2) = f(xy)(f(x)(f(y) - f(xy))$
Now it implies $ f^2(x) + f^2(y) + f^2(xy) = f(x)f(y)f(xy) + 4$
We know some techniques when proving inequality , now we deduce the existence of continuous function $ g$:$ (0;\infty )\to [1;\infty)$ such that $ f(x) = g(x) + \frac {1}{g(x)}$ (since $ f(x) > 2$ above).
The rest is easy enough; $ g(x) = x^k$ is immediately solution. Thus $ f(x) = x^k + \frac {1}{x^k}$
\end{solution}



\begin{solution}[by \href{https://artofproblemsolving.com/community/user/1276}{Nukular}]
	\begin{tcolorbox}We see that $ f(x) = f(\frac {1}{x})$; $ f(1) = 1$;$ f(x^2) = f^2(x) - 2$. Thus $ f(x) > \sqrt {2 + \sqrt {2 + ...}} = 2$ when the number of root tends to the infinite . It means $ f(x) > 2$ for all positive numbers $ x$.
Notice that $ f(x^2) + f(y^2) = f(xy)f(\frac {x}{y})$. It suffices to write $ x^2 = xy\frac {x}{y}$ and $ y^2 = \frac {xy}{\frac {x}{y}}$ thus $ f^2(x) + f^2(y) - 4 = f(x^2) + f(y^2) = f(xy)(f(x)(f(y) - f(xy))$
Now it implies $ f^2(x) + f^2(y) + f^2(xy) = f(x)f(y)f(xy) + 4$
We know some techniques when proving inequality , now we deduce the existence of continuous function $ g$:$ (0;\infty )\to [1;\infty)$ such that $ f(x) = g(x) + \frac {1}{g(x)}$ (since $ f(x) > 2$ above).
The rest is easy enough; $ g(x) = x^k$ is immediately solution. Thus $ f(x) = x^k + \frac {1}{x^k}$\end{tcolorbox}

Sadly, $ f$ is not required to be continuous -- see the work above for a full solution (modulo
some standard vector space knowledge of $ \mathbb{R}$ over $ \mathbb{Q}$) .
\end{solution}



\begin{solution}[by \href{https://artofproblemsolving.com/community/user/29721}{Erken}]
	a quick side note: with a condition of continuity the problem appeared in one of the Russian Olympiads. It was either National Olympiad or St. Petersburg olympiad.
\end{solution}



\begin{solution}[by \href{https://artofproblemsolving.com/community/user/44674}{Allnames}]
	\begin{tcolorbox}a quick side note: with a condition of continuity the problem appeared in one of the Russian Olympiads. It was either National Olympiad or St. Petersburg olympiad.\end{tcolorbox}
In my notebook; It appeared as St. Petersburg Olympiad problem. And I posted my solution for it ( without reading carefully the condition). I got a badly mistake!
\end{solution}
*******************************************************************************
-------------------------------------------------------------------------------

\begin{problem}[Posted by \href{https://artofproblemsolving.com/community/user/46787}{moldovan}]
	Find all functions $ f: \mathbb{R}^{+}\rightarrow \mathbb{R}$ that satisfy:

$ f(x)f(y)=f(xy)+2005 \left( \frac{1}{x}+\frac{1}{y}+2004 \right)$ for all $ x,y>0.$
	\flushright \href{https://artofproblemsolving.com/community/c6h291916}{(Link to AoPS)}
\end{problem}



\begin{solution}[by \href{https://artofproblemsolving.com/community/user/29428}{pco}]
	\begin{tcolorbox}Find all functions $ f: \mathbb{R}^{ + }\rightarrow \mathbb{R}$ that satisfy:

$ f(x)f(y) = f(xy) + 2005 \left( \frac {1}{x} + \frac {1}{y} + 2004 \right)$ for all $ x,y > 0.$\end{tcolorbox}

$ y=1$ $ \implies$ $ f(x)(f(1)-1)=2005(2005+\frac 1x)$. So $ f(1)\neq 1$ and $ f(x)=a(2005+\frac 1x)$

Plugging back in the original equation, we get $ a=1$ and the unique solution $ \boxed{f(x)=2005+\frac 1x}$
\end{solution}
*******************************************************************************
-------------------------------------------------------------------------------

\begin{problem}[Posted by \href{https://artofproblemsolving.com/community/user/46787}{moldovan}]
	Find all monotone (not necessarily strictly) functions $ f: \mathbb{R}^{+}_0\rightarrow \mathbb{R}$ such that:

$ f(x+y)-f(x)-f(y)=f(xy+1)-f(xy)-f(1) \forall x,y \ge 0$;

$ f(3)+3f(1)=3f(2)+f(0).$
	\flushright \href{https://artofproblemsolving.com/community/c6h291928}{(Link to AoPS)}
\end{problem}



\begin{solution}[by \href{https://artofproblemsolving.com/community/user/29428}{pco}]
	\begin{tcolorbox}Find all monotone (not necessarily strictly) functions $ f: \mathbb{R}^{ + }_0\rightarrow \mathbb{R}$ such that:

$ f(x + y) - f(x) - f(y) = f(xy + 1) - f(xy) - f(1) \forall x,y \ge 0$;

$ f(3) + 3f(1) = 3f(2) + f(0).$\end{tcolorbox}
1) Let us solve the easier equation $ (E1)$ :
"Find all functions $ g(x)$ from $ \mathbb N\to\mathbb R$ such that : $ g(2x+y)-g(2x)-g(y)=g(2y+x)-g(2y)-g(x)$ $ \forall x,y\in\mathbb N$"

The set $ \mathbb S$ of solutions is a $ \mathbb R$-vector space.
Setting $ y=1$, we get $ g(2x+1)=g(2x)+g(1)+g(x+2)-g(2)-g(x)$
Setting $ y=2$, we get $ g(2x+2)=g(2x)+g(2)+g(x+4)-g(4)-g(x)$
From these two equations, we see that knowledge of $ g(1),g(2),g(3),g(4)$ and $ g(6)$ gives knowledge of $ g(x)$ $ \forall x\in\mathbb N$ and so dimension of $ \mathbb S$ is at most $ 5$.
But the $ 5$ functions below are independant solutions :
$ g_1(x)=1$
$ g_2(x)=x$
$ g_3(x)=x^2$
$ g_4(x)=1$ if $ x=0\pmod 2$ and $ g_4(x)=0$ if $ x\neq 0\pmod 2$
$ g_5(x)=1$ if $ x=0\pmod 3$ and $ g_5(x)=0$ if $ x\neq 0\pmod 3$
And the general solution of $ (E1)$ is $ g(x)=a\cdot x^2+b\cdot x+c+d\cdot g_4(x)+e\cdot g_5(x)$

2) Solutions of the original equation :
Let $ P(x,y)$ be the assertion $ f(x+y)-f(x)-f(y)=f(xy+1)-f(xy)-f(1)$
In a first time, let us forget the monotonous constraint and the equation about $ f(0)$

Comparing $ P(xy,z)$ and $ P(xz,y)$, we get $ Q(x,y,z)$ : $ f(xy+z)-f(xy)-f(z)=f(xz+y)-f(xz)-f(y)$
Let then $ p$ a positive integer.
$ Q(2,\frac mp,\frac np)$ $ \implies$ $ f(\frac{2m+n}{p})-f(\frac{2m}{p})-f(\frac np)=f(\frac{2n+m}{p})-f(\frac{2n}{p})-f(\frac mp)$

So $ f(\frac xp)$ is a solution of $ (E1)$ and so $ f(\frac xp)=a_p\cdot x^2+b_p\cdot x+c_p+d_p\cdot g_4(x)+e_p\cdot g_5(x)$ $ \forall x\in\mathbb N$
Choosing $ x=kp$, it's easy to see that $ a_p=\frac{a}{p^2}$, then that $ b_p=\frac bp$
Choosing $ x=2kp$, $ x=3kp$ and $ x=6kp$, it's easy to see that $ c_p=c$ and $ d_p=e_p=0$

And so $ f(\frac xp)=a(\frac xp)^2+b(\frac xp)+c$ $ \forall x,p\in\mathbb N$
And so $ f(x)=ax^2+bx+c$ $ \forall x\in\mathbb Q^{+*}$

Now, we must remember that $ f(x)$ is monotonous, and so $ f(x)=ax^2+bx+c$ $ \forall x\in\mathbb R^{+*}$
The condition "monotonous" implies too that $ ab\geq 0$
The condition $ f(0)=f(3)+3f(1)-3f(2)$ implies $ f(0)=c$

And so the necessary conditions are :  $ \boxed{f(x)=ax^2+bx+c}$ $ \forall x\in\mathbb R_0^+$ with $ ab\geq 0$

And it's immediate to check that these necessary conditions are indeed sufficient.
\end{solution}
*******************************************************************************
-------------------------------------------------------------------------------

\begin{problem}[Posted by \href{https://artofproblemsolving.com/community/user/54046}{SUPERMAN2}]
	a) Find all function $f: \mathbb R \to \mathbb R$ satisfying \[ f(x+f(xy))=f(x)+xf(y)\] for all $x,y \in \mathbb R$.
b) Find all function $f: \mathbb R \to \mathbb R$ satisfying \[f(x+f(x)f(y))=f(x)+xf(y)\] for all $x,y \in \mathbb R$.
	\flushright \href{https://artofproblemsolving.com/community/c6h292835}{(Link to AoPS)}
\end{problem}



\begin{solution}[by \href{https://artofproblemsolving.com/community/user/29428}{pco}]
	\begin{tcolorbox}Problem:
a)Find all function $ f: R \to R$ satisfying $ f(x + f(xy)) = f(x) + xf(y)$
\end{tcolorbox}

Let $ P(x,y)$ be the assertion $ f(x+f(xy))=f(x)+xf(y)$

We obviously have a solution $ f(x)=0$ $ \forall x$. So we'll now consider only solutions non all-zero.

1) $ f(0)=0$, $ f(1)=1$ and $ f(x)\neq 0$ $ \forall x\neq 0$
If $ f(x)=0$ $ \forall x\neq 0$, Then $ P(0,0)$ implies $ f(f(0))=f(0)$ and so $ f(0)=0$. So $ \exists v\neq 0$ such that $ f(v)\neq 0$
Suppose now that $ \exists u$ such that $ f(u)=0$. Then $ P(\frac uv,v)$ $ \implies$ $ f(\frac uv+f(u))=f(\frac uv) +\frac uvf(v)$ $ \implies$ $ uf(v)=0$ $ \implies$ $ u=0$
But $ P(-1,-1)$ $ \implies$ $ f(-1+f(1))=0$ and so $ f(1)=1$ and $ f(0)=0$
Q.E.D.

2) $ f(x)$ is injective
Suppose $ f(a)=f(b)$ with $ a,b\neq 0$ and $ a\neq b$ (we already know that $ f(a)=0$ $ \implies$ $ a=0$)
Let $ x\neq 0$. Comparing $ P(x,\frac ax)$ and $ P(x,\frac bx)$, we get that $ f(\frac ax)=f(\frac bx)$ $ \forall x\neq 0$
Let then $ u\neq 0$ (and so $ f(u)\neq 0$) and let $ x=\frac {a-b}{f(u)}\neq 0$  so that $ \frac ax=\frac bx + f(u)$ :
$ f(\frac ax)=f(\frac bx)$ $ \implies$ $ f(\frac bx)=f(\frac bx+f(u))$
Then $ P(\frac bx,\frac{ux}b)$ $ \implies$ $ f(\frac bx)=f(\frac bx+f(u))$ $ =f(\frac bx)+\frac bxf(\frac {ux}{b})$ and so contradiction since $ a,b,x,u\neq 0$
So $ a=b$
Q.E.D.

3) $ f(x)=x$ $ \forall x$
$ P(1,x+f(xy))$ $ \implies$ $ f(1+f(x)+xf(y))=1+f(x)+xf(y)$
Setting $ x=-1$ in this equation and writing $ f(-1)+1=c$, we get $ f(c-f(y))=c-f(y)$
Setting then $ y=c-f(z)$ in this last equation, we get $ f(f(z))=f(z)$
And, since $ f(x)$ is injective, this implies $ f(z)=z$
Q.E.D.

Hence the two solutions :
$ f(x)=0$ $ \forall x$
$ f(x)=x$ $ \forall x$
\end{solution}



\begin{solution}[by \href{https://artofproblemsolving.com/community/user/43536}{nguyenvuthanhha}]
	\begin{italicized}That's official solution :\end{italicized} 
\end{solution}



\begin{solution}[by \href{https://artofproblemsolving.com/community/user/54046}{SUPERMAN2}]
	Thanks a lot.
Could you please send me the links for downloading the problems and solutions from AMM,nguyenvuthanhha?Thanks.
\end{solution}
*******************************************************************************
-------------------------------------------------------------------------------

\begin{problem}[Posted by \href{https://artofproblemsolving.com/community/user/66394}{reason}]
	Find all functions $f: \mathbb Q \to \mathbb R$ such that
\[f(x+1)=3f(x)+1,\]
for all $x \in \mathbb Q$.
	\flushright \href{https://artofproblemsolving.com/community/c6h293185}{(Link to AoPS)}
\end{problem}



\begin{solution}[by \href{https://artofproblemsolving.com/community/user/44659}{uglysolutions}]
	I think there is a condition missing here.

Suppose we know the value of $ f(x)$ for some rational number $ x$. Using the condition in the problem statement, we can find (probably by proving some formula using induction) the value of $ f(x+n)$ for every integer $ n$. However, this doesn't tell us anything at all about the value of $ f(y)$ unless $ x-y$ is an integer. So if $ y$ is a rational number such that $ \{y\} \neq \{x\}$ (brackets indicate fractional part), we can arbitrarily define the value of $ f(y)$, independently of $ f(x)$.

And we can do the same thing for all rational numbers in $ [0,1)$. So there are infinite functions which satisfy this equation, actually there are as many of them as functions $ f: [0,1) \rightarrow \mathbb R$.
\end{solution}



\begin{solution}[by \href{https://artofproblemsolving.com/community/user/29428}{pco}]
	No information is missing.
As uglysolution said, we just have infinitely many solutions and the possibility of giving general solutions :

From $ f(x+1)=3f(x)+1$, it's easy to show that $ f(x+n)=3^nf(x)+\frac{3^n-1}{2}$ $ \forall n\in\mathbb Z$

Then, a general solution is :

$ f(x)=3^xh(\{x\})+\frac{3^{[x]}-1}{2}$  where $ h(x)$ is any function defined on $ [0,1)$, $ [x]$ is the integer part of $ x$ and $ \{x\}$ the fractional part of $ x$.
\end{solution}
*******************************************************************************
-------------------------------------------------------------------------------

\begin{problem}[Posted by \href{https://artofproblemsolving.com/community/user/48290}{artofproblemsolving}]
	Find all functions $f: \mathbb R \to \mathbb R$ that \[f(f(x+y))=f(x+y)+f(x)+f(y)+xy\] holds for all $x,y \in \mathbb R$.
	\flushright \href{https://artofproblemsolving.com/community/c6h293642}{(Link to AoPS)}
\end{problem}



\begin{solution}[by \href{https://artofproblemsolving.com/community/user/29428}{pco}]
	\begin{tcolorbox}find all function f:R->R that f(f(x+y))=f(x+y)+f(x)+f(y)+xy\end{tcolorbox}

Strange problem. Where is it coming from ?

There are no solution to this equation :

Let $ P(x,y)$ be the assertion $ f(f(x + y)) = f(x + y) + f(x) + f(y) + xy$
Let $ a = f(0)$
Let $ \mathbb A = f(\mathbb R)$

1) $ Q(x)$ : $ f(x) = 2x + a$ $ \forall x\in\mathbb A$
$ P(x,0)$ $ \implies$ $ f(f(x)) = 2f(x) + a$, hence the result


2) $ R(x,y)$ : $ f(x + y) = f(x) + f(y) + xy - a$ $ \forall x,y$
$ P(x + y,0)$ $ \implies$ $ f(f(x + y)) = 2f(x + y) + a$ and so, comparing with $ P(x,y)$, we get the result.

3) $ S(x)$ : $ f(x + a) = - x^2 - (a - 2)x + 3a$ $ \forall x\in\mathbb A$
$ R(x,x + a)$ $ \implies$ $ f(2x + a) = f(x) + f(x + a) + x^2 + ax - a$. Then, if $ x\in\mathbb A$, $ Q(x)$ gives $ f(x) = 2x + a$ and so $ 2x + a\in\mathbb A$ and so $ Q(2x + a)$ gives $ f(2x + a) = 4x + 3a$. So $ 4x + 3a = 2x + a + f(x + a) + x^2 + ax - a$
Q.E.D.

4) Contradiction :
$ R(x,a)$ $ \implies$ $ f(x + a) = f(x) + f(a) + ax - a$ $ \forall x$
$ P(0,0)$ $ \implies$ $ f(a) = 3a$
So $ f(x + a) = f(x) + ax + 2a$ $ \forall x$
So $ f(x + a) = (a + 2)x + 3a$ $ \forall x\in\mathbb A$

But, comparing this result with $ S(x)$, we get $ - x^2 - (a - 2)x + 3a = (a + 2)x + 3a$ $ \forall x\in\mathbb A$ and so :

$ x^2 + 2ax = 0$ $ \forall x\in\mathbb A$ and so $ \mathbb A\subseteq\{0, - 2a\}$
And this is obviously impossible since it would mean that $ f(f(x + y)) - f(x) - f(y) - f(x + y)$ would be bounded, which is impossible since this quantity is equal to $ xy$

There are surely shorter paths to contradiction. 
But at least we have one.  :)
\end{solution}
*******************************************************************************
-------------------------------------------------------------------------------

\begin{problem}[Posted by \href{https://artofproblemsolving.com/community/user/66394}{reason}]
	Find all functions $f: \mathbb R \to [-1,1]$ such that for all $x,y \in \mathbb R$,
\[f(x+y)=f'(x)f'(y)-f(x)f(y).\]
	\flushright \href{https://artofproblemsolving.com/community/c6h294664}{(Link to AoPS)}
\end{problem}



\begin{solution}[by \href{https://artofproblemsolving.com/community/user/44083}{jgnr}]
	I think problems involving derivative does not belong to the olympiad section.
\end{solution}



\begin{solution}[by \href{https://artofproblemsolving.com/community/user/29428}{pco}]
	\begin{tcolorbox}Hi!!
find all f such that:

$ f(x + y) = f'(x)f'(y) - f(x)f(y)$

$ f: IR - - - - - - - - - > [ - 1,1]$

thanx.\end{tcolorbox}

Let $ P(x,y)$ be the assertion $ f(x+y)=f'(x)f'(y)-f(x)f(y)$

If $ f'(x)=0$ $ \forall x$, we get $ f(x)=c$ and $ c=-c^2$, hence two trivial solutions $ f(x)=0$ $ \forall x$ and $ f(x)=-1$ $ \forall x$

We'll now consider that $ f(x)$ is not a constant function. So $ \exists c$ such that $ f'(c)\neq 0$, $ P(x,c)$ $ \implies$  $ f'(x)=\frac{1}{f(c)}(f(x+c)+f(x)f(c))$ and so $ f(x)$ is $ C_{\infty}$

Then, derivating $ P(x,y)$ along $ x$, we get $ f'(x+y)=f"(x)f'(y)-f'(x)f(y)$
Same, derivating $ P(x,y)$ along $ y$, we get $ f'(x+y)=f'(x)f"(y)-f(x)f'(y)$

Equating these two equations, we get $ (f"(x)+f(x))f'(y)=(f"(y)+f(y))f'(x)$

And so either $ f(x)+f"(x)=0$ $ \forall x$, either $ f'(x)=a(f(x)+f"(x))$ $ \forall x$ for some $ a$

The first case give us $ f(x)=\alpha\cos(x+\beta)$ and, plugging back in the original equation, we get $ f(x)=-\cos(x)$

The second case, after some longer calculus, and using the fact that $ f(x)\in[-1,+1]$ implies no bounded non constant solutions.

Hence the solutions :
$ f(x)=0$ 
$ f(x)=-1$
$ f(x)=-\cos(x)$
\end{solution}



\begin{solution}[by \href{https://artofproblemsolving.com/community/user/66394}{reason}]
	Thanx for the answer pco  :) 
 
But I have a question: What do you mean by $ C_{\infty}$ ? :oops: 

thanx again.
\end{solution}



\begin{solution}[by \href{https://artofproblemsolving.com/community/user/29428}{pco}]
	\begin{tcolorbox}Thanx for the answer pco  :) 
 
But I have a question: What do you mean by $ C_{\infty}$ ? :oops: 

thanx again.\end{tcolorbox}

I mean that the $ n^{th}$ derivative $ f^n(x)$ exists and is continuous for any $ n$
\end{solution}



\begin{solution}[by \href{https://artofproblemsolving.com/community/user/66394}{reason}]
	Thanx a lot  :)
\end{solution}
*******************************************************************************
-------------------------------------------------------------------------------

\begin{problem}[Posted by \href{https://artofproblemsolving.com/community/user/64682}{KDS}]
	Find all function $ f: \mathbb R \to [0,+\infty)$ such that \[ f(x^2+y^2)=f(x^2-y^2)+f(2xy), \quad \forall x,y \in \mathbb R.\]
	\flushright \href{https://artofproblemsolving.com/community/c6h295339}{(Link to AoPS)}
\end{problem}



\begin{solution}[by \href{https://artofproblemsolving.com/community/user/9911}{Albanian Eagle}]
	This has definitely been posted before. I can't find a link so I'll write up a solution.

$ y=0\implies f(0)=0$ and $ x=0\implies f(a)=f(-a)\forall a\in \mathbb{R}$.
Let's denote $ g(x)=f(\sqrt{x})$ where $ g: [0,\infty)\to[0,\infty)$
For any $ a,b$ that are non negative the system 
\[ a=x^2-y^2,b=2xy\]
always has a solution in $ (x,y)\in \mathbb{R}^2$
And therefore $ g(a^2+b^2)=g(a^2)+g(b^2)$ or
\[ g(x+y)=g(x)+g(y)\]
which is a Cauchy functional equation and therefore we get that
\[ g(x)=ax,\forall x\in [0,\infty)\cup \mathbb{Q}\]
translating back to our equation this simply means that
\[ f(x)=ax^2\forall x\in \mathbb{Q}\]
And the final observation is that $ f$ is monotone so therefore the only solutions to $ f$ are functions of the form $ ax^2$ where $ a\in [0,\infty)$
\end{solution}



\begin{solution}[by \href{https://artofproblemsolving.com/community/user/29428}{pco}]
	\begin{tcolorbox}This has definitely been posted before. I can't find a link so I'll write up a solution.

$ y = 0\implies f(0) = 0$ and $ x = 0\implies f(a) = f( - a)\forall a\in \mathbb{R}$.
Let's denote $ g(x) = f(\sqrt {x})$ where $ g: [0,\infty)\to[0,\infty)$
For any $ a,b$ that are non negative the system
\[ a = x^2 - y^2,b = 2xy\]
always has a solution in $ (x,y)\in \mathbb{R}^2$
And therefore $ g(a^2 + b^2) = g(a^2) + g(b^2)$ or
\[ g(x + y) = g(x) + g(y)\]
which is a Cauchy functional equation and therefore we get that
\[ g(x) = ax,\forall x\in [0,\infty)\cup \mathbb{Q}\]
translating back to our equation this simply means that
\[ f(x) = ax^2\forall x\in \mathbb{Q}\]
And the final observation is that $ f$ is monotone so therefore the only solutions to $ f$ are functions of the form $ ax^2$ where $ a\in [0,\infty)$\end{tcolorbox}

Nothing allow you to conclude that $ f(x)$ is monotone (and your conclusion $ f(x)=ax^2$ shows the contrary).

But $ g(x+y)=g(x)+g(y)$ with $ g(x)$ : $ \mathbb R_0^+\to\mathbb R_0^+$ is enough to conclude $ g(x)=ax$ ($ g(x)$ is monotonous).
\end{solution}



\begin{solution}[by \href{https://artofproblemsolving.com/community/user/9911}{Albanian Eagle}]
	$ f(x^2+y^2)\geq f(x^2-y^2)$ so $ f$ is monotone in $ \mathbb{R}_{+}$ is what I meant  :)
\end{solution}
*******************************************************************************
-------------------------------------------------------------------------------

\begin{problem}[Posted by \href{https://artofproblemsolving.com/community/user/66394}{reason}]
	Find all functions $f: \mathbb R \to \mathbb R$ such that
\[f(x)+f(y)+f(z)=f(x)f(y)f(z)\]
happens for all reals $x,y$, and $z$.
	\flushright \href{https://artofproblemsolving.com/community/c6h295384}{(Link to AoPS)}
\end{problem}



\begin{solution}[by \href{https://artofproblemsolving.com/community/user/29428}{pco}]
	\begin{tcolorbox}Hi!!
find all functions:
$ f(x) + f(y) + f(z) = f(x)f(y)f(z)$

$ f: \mathbb{R} - - - - - - - - > \mathbb{R}$

thanx.\end{tcolorbox}

$ f(z)(f(x)f(y)-1)=f(x)+f(y))$, and so :
either $ f(x)f(y)=1$ $ \forall x,y$ and so $ f(x)$ is constant
either $ \exists u,v$ such that $ f(u)f(v)-1\neq 0$ and so $ f(z)=\frac{f(u)+f(v)}{f(u)f(v)-1}$ is constant too.

So $ f(x)=a$ and $ 3a=a^3$ and so three solutions :

$ f(x)=0$
$ f(x)=\sqrt 3$
$ f(x)=-\sqrt 3$

Is it a real problem found in any contest \/ book \/ ... or just something you invented ?
\end{solution}



\begin{solution}[by \href{https://artofproblemsolving.com/community/user/66394}{reason}]
	this problem is for $ x+y+z=\pi$ but I edited it.
\end{solution}



\begin{solution}[by \href{https://artofproblemsolving.com/community/user/7594}{Xantos C. Guin}]
	That restriction makes a very significant difference. 

By removing that restriction, you allow $ z$ to be anything instead of $ \pi - (x+y)$. 

With the restriction, $ f(x) = \tan(x)$ is a solution (and possibly others, I don't have time to investigate now).  

BTW: use "f: \mathbb{R} \to \mathbb{R}" to produce "$ f: \mathbb{R} \to \mathbb{R}$" instead of $ -------->$.
\end{solution}



\begin{solution}[by \href{https://artofproblemsolving.com/community/user/35129}{Zhero}]
	I don't think it can be $ f(x) = \tan x$; the domain is all reals. We won't be able to plug in $ \frac{\pi}{2}$.
\end{solution}



\begin{solution}[by \href{https://artofproblemsolving.com/community/user/45922}{Kirill}]
	\begin{tcolorbox}Hi!!
find all functions:
$ f(x) + f(y) + f(z) = f(x)f(y)f(z)$

$ f: \mathbb{R} - - - - - - - - > \mathbb{R}$

thanx.\end{tcolorbox}
For $ x = y = z$ we will have: $ 3f(x) = (f(x))^3$, so $ D(f) = \{0,\sqrt{3}, - \sqrt{3}\}$.
Let $ f(\alpha ) = a$ and $ f(\beta ) = b$, and $ a\neq b$. For $ x = y = \alpha$ and $ z = \beta$:
$ 2a + b = a^2b$ (1).
And for $ x = y = \beta$ and $ z = \alpha$:
$ 2b + a = b^2a$ (2).
(1)-(2) gives $ ab = 1$. Contadiction (because $ a,b\in \{0,\sqrt{3}, - \sqrt{3}\}$).
So, $ f(x)$ - constant for all $ x$, and three constants available.
\end{solution}
*******************************************************************************
-------------------------------------------------------------------------------

\begin{problem}[Posted by \href{https://artofproblemsolving.com/community/user/48290}{artofproblemsolving}]
	Find all functions $f: \mathbb R \to \mathbb R$ such that
\[f(f(x)y+x)=xf(y)+f(x)\]
for all $x,y \in \mathbb R$.
	\flushright \href{https://artofproblemsolving.com/community/c6h295583}{(Link to AoPS)}
\end{problem}



\begin{solution}[by \href{https://artofproblemsolving.com/community/user/29428}{pco}]
	\begin{tcolorbox}find all function f such that f: R->R, $ f(f(x)y+x)=xf(y)+f(x)$\end{tcolorbox}

Let $ P(x,y)$ be the assertion $ f(f(x)y+x)=xf(y)+f(x)$

$ f(x)=0$ $ \forall x$ is a trivial solution. So consider now $ \exists c$ such that $ f(c)\neq 0$

1) $ f(x)=0$ $ \iff$ $ x=0$
==================
$ P(1,0)$ $ \implies$ $ f(1)=f(1)+f(0)$ and so $ f(0)=0$
Suppose $ \exists u$ such that $ f(u)=0$. Then $ P(u,c)$ $ \implies$ $ 0=uf(c)$ $ \implies$ $ u=0$
Q.E.D.

2) $ f(-1)=-1$ and $ f(1)=1$ and $ f(-x)=-f(x)$ $ \forall x$
=======================================
$ P(-1,-1)$ $ \implies$ $ f(-f(-1)-1)=0$ $ \implies$ $ f(-1)=-1$
$ P(1,-1)$ $ \implies$ $ f(1-f(1))=f(1)-1$
$ P(1-f(1),1)$ $ \implies$ $ 0=-(1-f(1))^2$ and so $ f(1)=1$

$ P(-1,x-1)$ $ \implies$ $ f(-x)=-f(x-1)-1$
$ P(1,x-1)$ $ \implies$ $ f(x)=f(x-1)+1$
And so $ f(-x)=-f(x)$

3) $ f(\frac{1}{f(x)})=\frac 1x$ $ \forall x\neq 0$
=================================
Let $ x\neq 0$ (and so $ f(x)\neq 0$)
$ P(1,x)$ $ \implies$ $ f(x+1)=f(x)+1$
$ P(x,\frac{1}{f(x)})$ $ \implies$ $ f(x+1)=f(x)+xf(\frac{1}{f(x)})$
So $ 1=xf(\frac{1}{f(x)})$
Q.E.D.

4) $ f(x)$ and $ x$ have same sign $ \forall x\neq 0$
==================================
Let $ x\neq 0$ (and so $ f(x)\neq 0$)
$ P(x,-\frac{x}{f(x)})$ $ \implies$ $ 0=xf(-\frac{x}{f(x)})+f(x)$ $ \implies$ $ f(-\frac{x}{f(x)})=-\frac{f(x)}{x}$

$ P(\frac{1}{f(x)},\frac{-x^2-x}{f(x)})$ $ \implies$ $ f(f(\frac{1}{f(x)})\frac{-x^2-x}{f(x)}+\frac{1}{f(x)})$ $ =\frac{f(\frac{-x^2-x}{f(x)})}{f(x)}+f(\frac{1}{f(x)})$

$ \implies$ $ f(\frac{-x}{f(x)})$ $ =\frac{xf(\frac{-x^2-x}{f(x)})+f(x)}{xf(x)}$ $ =\frac{f(f(x)\frac{-x^2-x}{f(x)}+x)}{xf(x)}$

$ \implies$ $ f(\frac{-x}{f(x)})$ $ =\frac{f(-x^2)}{xf(x)}$

Comparing with the first line, we get $ -\frac{f(x)}{x}=\frac{f(-x^2)}{xf(x)}$

And so ${ -(f(x))^2=f(-x^2)}$ and also $ f(x^2)=(f(x))^2$

So $ f(x)>0$ $ \forall x>0$ and $ f(x)<0$ $ \forall x<0$
Q.E.D.

5) $ f(x)=x$
=========
$ P(x,-1)$ $ \implies$ $ f(x-f(x))=f(x)-x$
So $ f(x)=x$, else we would have $ f(u)=-u$ for some $ u=x-f(x)\neq 0$ and so a contradiction with point 4) above.
Q.E.D


Hence the two solutions for this equation :
$ f(x)=0$ $ \forall x$
$ f(x)=x$ $ \forall x$
\end{solution}



\begin{solution}[by \href{https://artofproblemsolving.com/community/user/43536}{nguyenvuthanhha}]
	Excellent solution :first:
\end{solution}



\begin{solution}[by \href{https://artofproblemsolving.com/community/user/29428}{pco}]
	\begin{tcolorbox}Excellent solution :first:\end{tcolorbox}

Thanks :)
\end{solution}



\begin{solution}[by \href{https://artofproblemsolving.com/community/user/66525}{kyungissac}]
	Excellent indeed. 
\end{solution}



\begin{solution}[by \href{https://artofproblemsolving.com/community/user/30342}{nicetry007}]
	Hi There, 

  My solution is similar to PCO's but differs after (2). I am only including the difference.

From PCO's solution, we have 

$ f(f(x)y + x) = xf(y) + f(x) - - - - - - - - - - - - - (1)$

1)$ f(x) = 0 \iff x = 0$
==================
$ P(1,0) \implies f(1) = f(1) + f(0)$ and so $ f(0) = 0$
Suppose $ \exists u$ such that $ f(u) = 0$. Then $ P(u,c) \implies 0 = uf(c) \implies u = 0$
Q.E.D.

2) $ f( - 1) = - 1$ and $ f(1) = 1$ and $ f( - x) = - f(x) \forall x$
=======================================
$ P( - 1, - 1) \implies f( - f( - 1) - 1) = 0 \implies f( - 1) = - 1$
$ P(1, - 1) \implies f(1 - f(1)) = f(1) - 1$
$ P(1 - f(1),1) \implies 0 = - (1 - f(1))^2$ and so $ f(1) = 1$

$ P( - 1,x - 1) \implies f( - x) = - f(x - 1) - 1$
$ P(1,x - 1) \implies f(x) = f(x - 1) + 1$
And so $ f( - x) = - f(x)$
************************************************
MY SOLUTION STARTS HERE: 

We have $ f(x + 1) = f(x) + 1 \Rightarrow f(n) = n \;\;\forall \;n \in \mathbb{Z}$
Set $ x = 2$ in $ (1)$.

$ f(2y + 2) = 2f(y) + 2$ 

$ \Rightarrow f(2y) + 2 = 2f(y) + 2$ 

$ \Rightarrow f(2y) = 2f(y) - - - - - - (2)$

Set $ x = - x, y = - y$ in $ (1)$ and use $ f( - x) = - f(x)$.

$ f(f(x)y - x) = xf(y) - f(x)$ 

$ \Rightarrow f(f(x)y - x) + 2f(x) = xf(y) + f(x)$ 

$ \Rightarrow f(f(x)y - x) + f(2x) = f(f(x)y + x) - - - - - - (3)$

We note that $ f(x)y - x$ can assume any real value as long as $ x\neq 0$. 
We let $ u = f(x)y - x$ and $ v = 2x$ and $ x\neq 0$ in $ (3)$.

$ f(u) + f(v) = f(u + v) \;\;\forall u,v \in \mathbb{R}\setminus\{0\} - - - - - - (4)$

We note that the above equation holds even when $ u$ or $ v$ or both are 0.
Hence, we have 

$ f(u) + f(v) = f(u + v) \;\;\forall u,v \in \mathbb{R} - - - - - - (5)$

$ \implies f(q) = qf(1) = q \;\;\forall q \in \mathbb{Q}$

Using $ (5)$ in $ (1)$, we get

$ f(f(x)y) + f(x) = xf(y) + f(x) \implies f(f(x)y) = xf(y) - - - (6)$ 

$ \implies f(f(x)) = x \;\;(\;\text{ set } y = 1 \text{ in } (6))$

$ \implies f(xy) = f(x)f(y) \;\;(\;\text{ set } x = f(x) \text { in } (6))$.

$ \implies f(x^2) = (f(x))^2 \geq 0 \;\;(\;\text{ set } x = y \text { in the above equation})$.

Setting $ v > 0$ in $ (5)$ gives us $ f(u + v) > f(u)$. We note that $ f$ is an increasing function and $ f(q) = q \;\;\forall q \in \mathbb{Q}$.

Hence, $ f(x) = x\;\;\forall x \in \mathbb{R}$.
\end{solution}



\begin{solution}[by \href{https://artofproblemsolving.com/community/user/29428}{pco}]
	Quite OK, congrats  :)
\end{solution}
*******************************************************************************
-------------------------------------------------------------------------------

\begin{problem}[Posted by \href{https://artofproblemsolving.com/community/user/48364}{cnyd}]
	Find all functions $ g : \mathbb{R}\to \mathbb{R}$ which satisfies
\[g(x+y)+g(x) \cdot g(y)=g(xy)+g(x)+g(y)\]
for all reals $x$ and $y$.
	\flushright \href{https://artofproblemsolving.com/community/c6h295605}{(Link to AoPS)}
\end{problem}



\begin{solution}[by \href{https://artofproblemsolving.com/community/user/29428}{pco}]
	\begin{tcolorbox}$ g : \mathbb{R}\mapsto\mathbb{R}$
$ g(x + y) + g(x).g(y) = g(xy) + g(x) + g(y)$\end{tcolorbox}

Let $ P(x,y)$ be the assertion $ g(x + y) + g(x)g(y) = g(xy) + g(x) + g(y)$

$ P(x,0)$ $ \implies$ $ g(0)(g(x) - 2) = 0$ and so a first solution $ g(x) = 2$ $ \forall x$. From now, consider that $ \exists c$ such that $ g(c)\neq 2$, then $ g(0) = 0$

Let $ g(1) = a$
$ P(x,1)$ $ \implies$ $ g(x + 1) = (2 - a)g(x) + a$
So $ g(2) = 3a - a^2$ But $ P(2,2)$ $ \implies$ $ g(2)^2 = 2g(2)$ and so $ 3a - a^2 = 0$ or $ 3a - a^2 = 2$ and so $ a\in\{0,1,2,3\}$

1) If $ a = 0$ : $ g(x + 1) = 2g(x)$, so $ g(x + n) = 2^ng(x)$ and $ g(n) = 0$. Then $ P(x,2)$ $ \implies$ $ g(2x) = 3g(x)$ and so $ g(4x) = 9g(x)$  but $ P(x,4)$ $ \implies$ $ g(4x) = 15g(x)$ and so we get $ g(x) = 0$ $ \forall x$

2) If $ a = 1$ : $ g(x + 1) = g(x) + 1$. Subtracting then $ P(x,\frac yx)$ from $ P(x,\frac yx + 1)$, we get $ g(x + y) = g(x) + g(y)$ and so $ g(xy) = g(x)g(y)$. So we have a Cauchy equation and a function such that $ g(x^2) = (g(x))^2$, so $ g(x) > 0$ $ \forall x > 0$ and so $ g(x)$ is monotonous, so is $ x$.

3) If $ a = 2$ : $ g(x + 1) = 2$ ****\begin{bolded}edited \end{bolded}\end{underlined}: "and we got the same constant solution $ g(x) = 2$ we already found" $ \to$ "and we got a contradiction since we have $ g(0)=0$ in this part of the demo. (thanks Rafikafi for the remark) ***.

4) If $ a = 3$, we got $ g(x + 1) = 3 - g(x)$ and so $ g(x + 2) = g(x)$. Then $ P(x,2)$ $ \implies$ $ g(x + 2) = g(2x) + g(x)$ and so $ g(2x) = 0$  $ \forall x$, and so contradiction (since $ g(1) = 3$)


5) Synthesis
Hence the solutions :
$ g(x) = 0$
$ g(x) = 2$
$ g(x) = x$
\end{solution}



\begin{solution}[by \href{https://artofproblemsolving.com/community/user/66275}{Rafikafi}]
	Nice solution, there's just a tiny thing:

At point 3) you actually have a contradiction since $ g(0)=0$, otherwise you would have forgotten to check whether $ g(x)=2$ if  $ x \neq 0$ and $ g(0)=0$ is a solution, which is not.
\end{solution}



\begin{solution}[by \href{https://artofproblemsolving.com/community/user/29428}{pco}]
	\begin{tcolorbox}Nice solution, there's just a tiny thing:

At point 3) you actually have a contradiction since $ g(0) = 0$, otherwise you would have forgotten to check whether $ g(x) = 2$ if  $ x \neq 0$ and $ g(0) = 0$ is a solution, which is not.\end{tcolorbox}

Yes, you're perfectly right. I edited my solution.
Thanks for the careful reading :)
\end{solution}
*******************************************************************************
-------------------------------------------------------------------------------

\begin{problem}[Posted by \href{https://artofproblemsolving.com/community/user/369}{Leon}]
	Let $ f$ be a function defined on the set of non-negative integers and taking values in the same
set. Given that

(a) $ \displaystyle x - f(x) = 19\left[\frac{x}{19}\right] - 90\left[\frac{f(x)}{90}\right]$ for all non-negative integers $ x$;

(b) $ 1900 < f(1990) < 2000$,

find the possible values that $ f(1990)$ can take.
(Notation : here $ [z]$ refers to largest integer that is $ \leq z$, e.g. $ [3.1415] = 3$).
	\flushright \href{https://artofproblemsolving.com/community/c6h295763}{(Link to AoPS)}
\end{problem}



\begin{solution}[by \href{https://artofproblemsolving.com/community/user/29428}{pco}]
	\begin{tcolorbox}Let $ f$ be a function defined on the set of non-negative integers and taking values in the same
set. Given that

(a) $ \displaystyle x - f(x) = 19\left[\frac {x}{19}\right] - 90\left[\frac {f(x)}{90}\right]$ for all non-negative integers $ x$;

(b) $ 1900 < f(1990) < 2000$,

find the possible values that $ f(1990)$ can take.
(Notation : here $ [z]$ refers to largest integer that is $ \leq z$, e.g. $ [3.1415] = 3$).\end{tcolorbox}

$ x-f(x)=19[\frac x{19}]-90[\frac{f(x)}{90}]$ $ =19(\frac x{19}-\{\frac x{19}\})$ $ -90(\frac{f(x)}{90}-\{\frac{f(x)}{90}\})$ $ \iff$ $ 19\{\frac x{19}\}=90\{\frac{f(x)}{90}\})$ $ \iff$ $ \{\frac{f(x)}{90}\}=\frac{19}{90}\{\frac x{19}\}$

$ \iff$ $ \frac{f(x)}{90}=h(x)+\frac{19}{90}\{\frac x{19}\}$ where $ h(x)=[\frac{f(x)}{90}]\in\mathbb N_0$ and so $ f(x)=90h(x)+19\{\frac x{19}\}$

so $ f(1990)=90n+19\{\frac {1990}{19}\}=90n+14$ and, since $ 1900<f(1990)<2000$, $ f(1990)\in\{1904,1994\}$
\end{solution}
*******************************************************************************
-------------------------------------------------------------------------------

\begin{problem}[Posted by \href{https://artofproblemsolving.com/community/user/66394}{reason}]
	Find all functions $f: \mathbb{R}\to\mathbb{R}$ such that
\[f(f(x)+f(y)+2xy)=yf(x)+xf(y)\]
for all $x,y \in \mathbb R$.
	\flushright \href{https://artofproblemsolving.com/community/c6h295794}{(Link to AoPS)}
\end{problem}



\begin{solution}[by \href{https://artofproblemsolving.com/community/user/29428}{pco}]
	\begin{tcolorbox}find all f such that:

$ f(f(x) + f(y) + 2xy) = yf(x) + xf(y)$

$ f: \mathbb{IR}\to\mathbb{IR}$

thanx\end{tcolorbox}

Wow, I found this one difficult. I hope somebody will find a simpler solution than mine (and I'm very surprised that such a problem was given in some contest).

Let $ P(x,y)$ be the assertion $ f(f(x)+f(y)+2xy)=yf(x)+xf(y)$


1) Consider first that $ f(0)\neq 0$
$ P(x,0)$ $ \implies$ $ f(f(x)+f(0))=xf(0)$ $ \implies$ $ f(x)$ is bijective.
Then $ P(\frac 12,\frac 12)$ $ \implies$ $ f(2f(\frac 12)+\frac 12)=f(\frac 12)$ and, since $ f(x)$ is bijective, $ 2f(\frac 12)+\frac 12 =\frac 12$ and so $ f(\frac 12)=0$
$ P(0,0)$ $ \implies$ $ f(2f(0))=0$ and so, since $ f(x)$ is bijective and $ f(\frac 12)=0$, we get $ f(0)=\frac 14$


$ P(x,\frac 12)$ $ \implies$ new assertion $ Q(x)$ : $ f(f(x)+x)=\frac 12f(x)$
$ P(x,0)$ $ \implies$ new assertion $ R(x)$ : $ f(f(x)+\frac 14)=\frac x4$

Then $ Q(f(2x)+\frac 14)$ $ \implies$ $ f(f(f(2x)+\frac 14)+f(2x)+\frac 14)=\frac 12f(f(2x)+\frac 14)$ $ \implies$ $ f(\frac x2+f(2x)+\frac 14)=\frac x4$ $ =f(f(x)+\frac 14)$ and so $ f(2x)=f(x)-\frac x2$

But $ R(f(x)+\frac 14)$ $ \implies$ $ f(f(f(x)+\frac 14)+\frac 14)=\frac {f(x)}4+\frac 1{16}$ $ \implies$ $ f(\frac {x+1}4)=\frac {f(x)}4+\frac 1{16}$. Using then twice the fact that $ f(2u)=f(u)-\frac u2$, we get $ f(x+1)=\frac{f(x)}4-\frac{3x}8-\frac 5{16}$

So we got :
$ U(x)$ : $ f(2x)=f(x)-\frac x2$
$ V(x)$ : $ f(x+1)=\frac{f(x)}4-\frac{3x}8-\frac 5{16}$

From this we can compute $ f(2x+2)$ in two different ways :
$ V(x+1)$ $ \implies$ $ f(x+2)=\frac{f(x+1)}4-\frac{3(x+1)}8-\frac 5{16}$ $ =\frac{f(x)}{16}-\frac{3x}{32}-\frac 5{64}-\frac{3(x+1)}8-\frac 5{16}$ $ =\frac{f(x)}{16}-\frac{15x}{32}-\frac {49}{64}$
Using $ 2x$ instead of $ x$ in this lats equality, we get $ f(2x+2)=\frac{f(2x)}{16}-\frac{15x}{16}-\frac {49}{64}$ $ =\frac{f(x)}{16}-\frac {x}{32}-\frac{15x}{16}-\frac {49}{64}$ and so :

$ f(2x+2)=\frac{f(x)}{16}-\frac{31x}{32}-\frac {49}{64}$

But $ U(x+1)$ $ \implies$ $ f(2x+2)=f(x+1)-\frac {x+1}2$ and so $ f(2x+2)=\frac{f(x)}4-\frac{3x}8-\frac 5{16}-\frac {x+1}2$ and so $ f(2x+2)=\frac{f(x)}4-\frac{7x}8-\frac {13}{16}$ 

And so $ \frac{f(x)}{16}-\frac{31x}{32}-\frac {49}{64}$ $ =\frac{f(x)}4-\frac{7x}8-\frac {13}{16}$

And $ f(x)=\frac 14-\frac{x}{2}$

Plugging this mandatory condition in the original equation, we can easily see that this indeed is a solution.

2) Consider now that $ f(0)=0$
(1) : $ P(x,0)$ $ \implies$ $ f(f(x))=0$ $ \forall x$
(2) : $ P(-\frac 12,f(-\frac 12))$ $ \implies$ $ f(-\frac 12)=0$ (using 1)
(3) : $ P(x,-\frac 12)$ $ \implies$ $ f(f(x)-x)=-\frac{f(x)}{2}$ (using 2)
(4) : $ P(f(x),-\frac 12)$ $ \implies$ $ f(-f(x))=0$ $ \forall x$ (using 1 and 2)
(5) : $ P(\frac 12,-f(\frac 12))$ $ \implies$ $ f(\frac 12)=0$ (using 1)
(6) : $ P(x,\frac 12)$ $ \implies$ $ f(f(x)+x)=\frac{f(x)}2$ (using 5)
(7) : $ P(x,f(y))$ $ \implies$ $ f(f(x)+2xf(y))=f(x)f(y)$ (using 1) and so $ f(f(x)f(y))=0$ $ \forall x,y$ (using 1 again)

(8) : If $ f(x)\neq 0$, then $ P(-\frac{1}{2f(x)},f(x)f(-\frac{1}{2f(x)}))$ $ \implies$ $ f(-\frac{1}{2f(x)})=0$ (using 7). Then, since $ f(f(x)-x)=-\frac{f(x)}{2}\neq 0$ (using 3), and replacing $ x$ with $ f(x)-x$, we get $ f(\frac{1}{f(x)})=0$

(9) : If $ f(x)\neq 0$, $ P(y,\frac{1}{f(x)})$ $ \implies$ $ f(f(y)+\frac{2y}{f(x)})=\frac{f(y)}{f(x)}$ (using 8) and so $ f(\frac{f(y)}{f(x)})=0$ $ \forall x,y$ such that $ f(x)\neq 0$

$ P(x,x)$ $ \implies$ $ f(2f(x)+2x^2)=2xf(x)$ and so (using 6) $ f(z)=xf(x)$ with $ z=2f(x)+2x^2+f(2f(x)+2x^2)$

Using then 9, we get $ f(x)\neq 0$ $ \implies$ $ f(\frac{f(z)}{f(x)})=0$ with the above $ z$ such that $ f(z)=xf(x)$ and so $ f(x)=0$, so contradiction and so $ f(x)=0$ $ \forall x$

3) hence the solutions :
$ f(x)=0$ $ \forall x$
$ f(x)=\frac 14-\frac{x}{2}$ $ \forall x$
\end{solution}



\begin{solution}[by \href{https://artofproblemsolving.com/community/user/40922}{mehdi cherif}]
	good problem and amazing solution , Patrick   

\begin{bolded}the mine :\end{bolded} (simpler , i think ! )

\begin{tcolorbox}
Let $ P(x,y)$ be the assertion $ f(f(x) + f(y) + 2xy) = yf(x) + xf(y)$


1) Consider first that $ f(0)\neq 0$
$ P(x,0)$ $ \implies$ $ f(f(x) + f(0)) = xf(0)$ $ \implies$ $ f(x)$ is bijective.
Then $ P(\frac 12,\frac 12)$ $ \implies$ $ f(2f(\frac 12) + \frac 12) = f(\frac 12)$ and, since $ f(x)$ is bijective, $ 2f(\frac 12) + \frac 12 = \frac 12$ and so $ f(\frac 12) = 0$
$ P(0,0)$ $ \implies$ $ f(2f(0)) = 0$ and so, since $ f(x)$ is bijective and $ f(\frac 12) = 0$, we get $ f(0) = \frac 14$

$ P(x,\frac 12)$ $ \implies$ new assertion $ Q(x)$ : $ f(f(x) + x) = \frac 12f(x)$
$ P(x,0)$ $ \implies$ new assertion $ R(x)$ : $ f(f(x) + \frac 14) = \frac x4$

Then $ Q(f(2x) + \frac 14)$ $ \implies$ $ f(f(f(2x) + \frac 14) + f(2x) + \frac 14) = \frac 12f(f(2x) + \frac 14)$ $ \implies$ $ f(\frac x2 + f(2x) + \frac 14) = \frac x4$ $ = f(f(x) + \frac 14)$ and so $ f(2x) = f(x) - \frac x2$
\end{tcolorbox}

we have previously proved that : $ \boxed{f(2x)=f(x)-\frac{x}{2}}$ : $ M(x)$

but we have : $ R(2f(x))\rightarrow f(f(2f(x))+\frac 14)=\frac{f(x)}{2}$

but from $ Q(x)$ we have $ f(f(x)+x)=\frac{f(x)}{2}$

so : $ f(f(2f(x))+\frac 14)=f(f(x)+x)\implies f(2f(x))+\frac 14=f(x)+x$

and we have $ M(f(x))\implies f(2f(x))=f(f(x))-\frac{f(x)}{2}$

so : $ f(2f(x))=f(f(x))-\frac{f(x)}{2}=f(x)+x-\frac 14$

$ \iff \boxed{f(f(x))-\frac{3}{2}f(x)-x+\frac 14=0}$  $ (E)$

now $ (E)$ can easily be solved  and $ f(x)=\frac{-x}{2}+\frac{1}{4}$

2) for $ f(0)=0$ 

$ P(x,0)\implies f(f(x))=0$

$ P(\frac{-1}{2},f(\frac{-1}{2}))\implies f^2(\frac{-1}{2})=0 \implies f(\frac{-1}{2})=0$

$ P(x,\frac{-1}{2})\implies f(f(x)-x)=-\frac{f(x)}{2}\implies f(-\frac{f(x)}{2})=0$

$ P(1,-\frac{f(1)}{2})\implies f(f(1)-f(1))=\frac{-f^2(1)}{2}\implies f(1)=0$

$ P(x,1)\implies f(f(x)+2x)=f(x)$ $ (*)$

now put $ g(x)=f(x)+2x$

$ (*)\iff g(g(x))-3g(x)+2x=0$   $ (E')$

$ (E')$ is also easy to solve , $ g(x)=2x$ $ \implies f(x)=0$

:)

Mehdi
\end{solution}



\begin{solution}[by \href{https://artofproblemsolving.com/community/user/29428}{pco}]
	Hello Mehdi !

\begin{tcolorbox} $ \boxed{f(f(x)) - \frac {3}{2}f(x) - x + \frac 14 = 0}$  $ (E)$

now $ (E)$ can easily be solved  and $ f(x) = \frac { - x}{2} + \frac {1}{4}$\end{tcolorbox}

Huh, could you give us the easy solution to this equation ?




\begin{tcolorbox} $ (*)\iff g(g(x)) - 3g(x) + 2x = 0$   $ (E')$

$ (E')$ is also easy to solve , $ g(x) = 2x$ $ \implies f(x) = 0$\end{tcolorbox}

And also the easy solution to this one ?

Thanks
\end{solution}



\begin{solution}[by \href{https://artofproblemsolving.com/community/user/40922}{mehdi cherif}]
	\begin{tcolorbox}

Huh, could you give us the easy solution to this equation ?

\end{tcolorbox}

not easy , but known  :) 

we can use this result : http://www.mathlinks.ro/viewtopic.php?p=1222512#1222512

Mehdi

 :)
\end{solution}



\begin{solution}[by \href{https://artofproblemsolving.com/community/user/29428}{pco}]
	\begin{tcolorbox}[quote="pco"]

Huh, could you give us the easy solution to this equation ?

\end{tcolorbox}

not easy , but known  :) 

we can use this result : http://www.mathlinks.ro/viewtopic.php?p=1222512#1222512

Mehdi

 :)\end{tcolorbox}

No, you cant use this result, because the key point of the demonstration in the link you gave is that $ f(x)>0$. And we have no such constraint here.
\end{solution}



\begin{solution}[by \href{https://artofproblemsolving.com/community/user/29428}{pco}]
	\begin{tcolorbox}[quote="pco"]

Huh, could you give us the easy solution to this equation ?

\end{tcolorbox}

not easy , but known  :) 

:)\end{tcolorbox}

Not known by be ...  :( 

So you mean that equation $ f(f(x))-bf(x)-cx=0$ always has as unique solutions $ f(x)=rx$ with $ r$ root of $ x^2-bx-c=0$  ?

If so, it's obviously wrong. Chosse for example $ b=0$ and $ c=1$ : $ f(x)=x$ and $ f(x)=-x$ are certainly not the unique solutions of $ f(f(x))=x$
\end{solution}



\begin{solution}[by \href{https://artofproblemsolving.com/community/user/40922}{mehdi cherif}]
	you're right , i'm sorryyy  :(
\end{solution}
*******************************************************************************
-------------------------------------------------------------------------------

\begin{problem}[Posted by \href{https://artofproblemsolving.com/community/user/66525}{kyungissac}]
	Find all $ f: \mathbb R\rightarrow \mathbb R$ such that \[ f(x-y) = f(x)+f(y)+xy\] for all $x\in \mathbb R$ and all $ y\in \{f(x) \vert x\in \mathbb R\}$.
	\flushright \href{https://artofproblemsolving.com/community/c6h296383}{(Link to AoPS)}
\end{problem}



\begin{solution}[by \href{https://artofproblemsolving.com/community/user/29428}{pco}]
	\begin{tcolorbox}Find all $ f: R\rightarrow R$ such that $ f(x - y) = f(x) + f(y) + xy$ for $ \forall x\in R$ and $ \forall y\in \{f(x) \vert x\in R\}$.\end{tcolorbox}

Just to be sure it's meaningful to continue to search : are you sure the problem statement is correct ? (and where did you find it? , to be sure there is a solution).

My currently situation :

We have the two trivial solutions $ f(x) = 0$ and $ f(x) = - \frac {x^2}2$

Then, writing $ f(x) = g(x) - \frac {x^2}2$, we get $ g(x - y) = g(x) + g(y)$ $ \forall x\in \mathbb R$, $ \forall y\in f(\mathbb R)$

From there, it's "rather" easy to show that $ g(x) = 0$ $ \forall x\in f(\mathbb R)$ and so that $ g(x + y) = g(x)$ $ \forall x\in \mathbb R$, $ \forall y\in f(\mathbb R)$

But it's not so easy to show (and maybe it's wrong) that the only solution would be $ g(x) = 0$ $ \forall x$. With continuity constraint, it would be ok, but we have not.
\end{solution}



\begin{solution}[by \href{https://artofproblemsolving.com/community/user/66525}{kyungissac}]
	\begin{tcolorbox}[quote="kyungissac"]Find all $ f: R\rightarrow R$ such that $ f(x - y) = f(x) + f(y) + xy$ for $ \forall x\in R$ and $ \forall y\in \{f(x) \vert x\in R\}$.\end{tcolorbox}

Just to be sure it's meaningful to continue to search : are you sure the problem statement is correct ? (and where did you find it? , to be sure there is a solution).

My currently situation :

We have the two trivial solutions $ f(x) = 0$ and $ f(x) = - \frac {x^2}2$

Then, writing $ f(x) = g(x) - \frac {x^2}2$, we get $ g(x - y) = g(x) + g(y)$ $ \forall x\in \mathbb R$, $ \forall y\in f(\mathbb R)$

From there, it's "rather" easy to show that $ g(x) = 0$ $ \forall x\in f(\mathbb R)$ and so that $ g(x + y) = g(x)$ $ \forall x\in \mathbb R$, $ \forall y\in f(\mathbb R)$

But it's not so easy to show (and maybe it's wrong) that the only solution would be $ g(x) = 0$ $ \forall x$. With continuity constraint, it would be ok, but we have not.\end{tcolorbox}

My friend(2005 IMO gold medalist) gave me a set of problems. This was one of them. I'm sorry I do not know the source. :maybe:
\end{solution}



\begin{solution}[by \href{https://artofproblemsolving.com/community/user/29876}{ozgurkircak}]
	the problem is ok. it is given in korea mo final 2002. see the solution by xxp2000
[url]http://www.mathlinks.ro/viewtopic.php?t=260366[\/url]
\end{solution}



\begin{solution}[by \href{https://artofproblemsolving.com/community/user/29428}{pco}]
	\begin{tcolorbox}the problem is ok. it is given in korea mo final 2002. see the solution by xxp2000
[url]http://www.mathlinks.ro/viewtopic.php?t=260366[\/url]\end{tcolorbox}

Ahhh yes. Thanks for the link.

And, btw, pretty solution by xxp2000 !  :)
\end{solution}



\begin{solution}[by \href{https://artofproblemsolving.com/community/user/29876}{ozgurkircak}]
	here is another nice problem that can be solved by the same idea.
Find all functions  defined on real numbers and taking real values such that  for all real numbers  \[ (f(x))^2+2yf(x)+f(y)=f(y+f(x))\]
\end{solution}



\begin{solution}[by \href{https://artofproblemsolving.com/community/user/29428}{pco}]
	\begin{tcolorbox}here is another nice problem that can be solved by the same idea.
Find all functions  defined on real numbers and taking real values such that  for all real numbers
\[ (f(x))^2 + 2yf(x) + f(y) = f(y + f(x))\]
\end{tcolorbox}

Yes, quite nice method. Thanks for having shown :

Let $ P(x,y)$ be the assertion $ f(x)^2+2yf(x)+f(y)=f(y+f(x))$

$ \boxed{f(x)=0}$ $ \forall x$ is a solution. Let us now consider that $ \exists a$ such that $ f(a)\neq 0$

$ P(x,0)$ $ \implies$ $ f(f(x))=f(0)+f(x)^2$

$ P(y,f(x)-f(y))$ $ \implies$ $ f(y)^2+2f(y)(f(x)-f(y))+f(f(x)-f(y))=f(f(x))$ $ =f(0)+f(x)^2$ $ \implies$ $ f(f(x)-f(y))=f(0)+(f(x)-f(y))^2$

$ P(a,\frac{u-f(a)^2}{2f(a)})$ $ \implies$ $ f(\frac{u+f(a)^2}{2f(a)})-f(\frac{u-f(a)^2}{2f(a)})=u$

And so $ f(x)-f(y)$ can take any value we want and, from $ f(f(x)-f(y))=f(0)+(f(x)-f(y))^2$, we can deduce $ f(x)=f(0)+x^2$ $ \forall x$

And, plugging back in the original equation, we check that $ \boxed{f(x)=x^2+b}$ is indeed a solution.

Nice method (which works with non surjective functions)
\end{solution}
*******************************************************************************
-------------------------------------------------------------------------------

\begin{problem}[Posted by \href{https://artofproblemsolving.com/community/user/48364}{cnyd}]
	Find all continuous functions $ f: \mathbb{R}\to\mathbb{R}$ for which
\[ f(xy)=f\left(\frac{x^{2}+y^{2}}{2}\right)-f\left(\frac{x^{2}-y^{2}}{2}\right), \quad \forall x,y \in \mathbb R.\]
	\flushright \href{https://artofproblemsolving.com/community/c6h296599}{(Link to AoPS)}
\end{problem}



\begin{solution}[by \href{https://artofproblemsolving.com/community/user/29428}{pco}]
	\begin{tcolorbox}$ f: \mathbb{R}\mapsto\mathbb{R}$ and $ f$ is continuous
$ f(xy) = f(\frac {x^{2} + y^{2}}{2}) - f(\frac {x^{2} - y^{2}}{2})$\end{tcolorbox}

[hide="My solution"]
Let $ P(x,y)$ be the assertion $ f(xy) = f(\frac {x^2 + y^2}2) - f(\frac {x^2 - y^2}2)$

$ P(0,0)$ $ \implies$ $ f(0) = 0$
$ P(0,\sqrt {2|x|})$ $ \implies$ $ f(|x|) = f( - |x|)$ $ \implies$ $ f(x) = f( - x)$

For $ x\geq 0$, let us define $ g(x) = f(\sqrt x)$. $ g(x)$ is continuous. Then, for $ x,y\geq 0$, $ P(x,y)$ becomes : $ g(x^2y^2) = g(\frac {x^4 + y^4}4 + \frac {x^2y^2}2) - g(\frac {x^4 + y^4}4 - \frac {x^2y^2}2)$

For $ u,v\geq 0$, let $ x = \sqrt {\sqrt {u + v} + \sqrt v}$ and $ y = \sqrt {\sqrt {u + v} - \sqrt v}$ and insert these values in the above equation. We get : $ g(u) = g(u + v) - g(v)$

So, since $ g(x)$ is continuous,  $ g(x) = ax$ $ \forall x\geq 0$ and so $ f(x) = ax^2$ $ \forall x\geq 0$ and, since $ f(x) = f( - x)$ : $ \boxed{f(x) = ax^2}$ $ \forall x$ and it's immediate to check that this mandatory form indeed is a solution.
[\/hide]
\end{solution}
*******************************************************************************
-------------------------------------------------------------------------------

\begin{problem}[Posted by \href{https://artofproblemsolving.com/community/user/43536}{nguyenvuthanhha}]
	Find all functions $ f : \mathbb{R}  \ \to \ \mathbb{R}$ such that
\[f((x+1)f(y)) \ = \ y (f(x) +1) , \quad\forall x,y \in \mathbb{R}.\]
	\flushright \href{https://artofproblemsolving.com/community/c6h297003}{(Link to AoPS)}
\end{problem}



\begin{solution}[by \href{https://artofproblemsolving.com/community/user/29428}{pco}]
	\begin{tcolorbox}\begin{italicized}Find all function $ f : \mathbb{R} \ \to \ \mathbb{R}$ such that :

   $ f((x + 1)f(y)) \ = \ y (f(x) + 1) \ \forall \ \ x ; y \ \in \ \mathbb{R}$

 [hide]I think the problem will become very easy if we can prove $ f(1) = 1$ but i can't reach that arguement [\/hide]\end{italicized}\end{tcolorbox}
Thanks for the gift :)

[hide="My solution"]
Let $ P(x,y)$ be the assertion $ f((x + 1)f(y)) = y(f(x) + 1)$
Let $ a = f(1)$
$ f(x) = 0$ $ \forall x$ is not a solution. So $ \exists c$ such that $ f(c)\neq 0$

If $ f(0)\neq 0$, $ P(\frac {c}{f(0)} - 1,0)$ $ \implies$ $ f(c) = 0$, which is wrong. So $ f(0) = 0$
$ P(0,x)$ $ \implies$ $ f(f(x)) = x$ and $ f(x)$ is an involution, and so is bijective.
$ P( - 1,1)$ $ \implies$ $ f( - 1) = - 1$
Since $ f( - 1) = - 1$ and $ f(0) = 0$ and $ f(x)$ is bijective, $ a = f(1)\notin\{ - 1,0\}$

Since $ f(f(x)) = x$, we get $ f(a) = 1$
$ P(x - 1,a)$ $ \implies$ $ f(x) = a(f(x - 1) + 1)$ and so $ f(x - 1) + 1 = \frac {f(x)}a$
$ P(x - 1,f(y))$ $ \implies$ $ f(xy) = f(y)(f(x - 1) + 1)$ $ = \frac {f(x)f(y)}{a}$

We can compute then $ f(4)$ in two ways :
$ P(1,a)$ $ \implies$ $ f(2) = a^2 + a$
$ P(2,a)$ $ \implies$ $ f(3) = a^3 + a^2 + a$
$ P(3,a)$ $ \implies$ $ f(4) = a^4 + a^3 + a^2 + a$

But, using $ x = y = 2$ in $ f(xy) = \frac {f(x)f(y)}{a}$ implies $ f(4) = \frac {(a^2 + a)^2}a = a^3 + 2a^2 + a$

So $ a^4 + a^3 + a^2 + a = a^3 + 2a^2 + a$ and so $ a^4 = a^2$ and so $ a\in\{ - 1,0,1\}$ but we already know that $ a\notin\{ - 1,0\}$ and so $ a = 1$

So $ f(xy) = f(x)f(y)$
Then $ P(x,f(y))$ $ \implies$ $ f((x + 1)y) = f(y)(f(x) + 1) = f(x)f(y) + f(y)$ $ = f(xy) + f(y)$ and $ f(xy + y) = f(xy) + f(y)$ and so $ f(u + v) = f(u) + f(v)$

Since we also have $ f(xy) = f(x)f(y)$ $ \implies$ $ f(x^2) = f(x)^2$, we get that $ f(x)$ and $ x$ have same signs and so $ f(x)$ is increasing.

Hence $ f(x) = x$ (increasing solution of Cauchy equation with $ f(1)=1$) and it's immediate to verify that this mandatory result indeed is a solution.
[\/hide]
\end{solution}



\begin{solution}[by \href{https://artofproblemsolving.com/community/user/40922}{mehdi cherif}]
	\begin{tcolorbox}\begin{italicized}Find all function $ f : \mathbb{R} \ \to \ \mathbb{R}$ such that :

   $ f((x + 1)f(y)) \ = \ y (f(x) + 1) \ \forall \ \ x ; y \ \in \ \mathbb{R}$

 I think the problem will become very easy if we can prove $ f(1) = 1$ but i can't reach that arguement \end{italicized}\end{tcolorbox}

let $ P(x,y)$ be the assertion $ f((x + 1)f(y)) = y(f(x) + 1) \forall x,y$.

$ P( - 1,0)\implies f(0) = 0$

$ P( - 1, 1)\implies f( - 1) = - 1$

$ P(0,y)\implies f(f(y)) = y$

$ P( - 2, - 1)\implies f( - f( - 1)) = f(1) = - (f( - 2) + 1)$ so $ f( - 2) = - 1 - f(1)$ $ (1)$

$ P( - 2,f(1))\implies f(( - 2 + 1)f(f(1))) = f( - 1) = - 1 = f(1)f( - 2) + f(1)$  $ (2)$

from $ (1)$ and $ (2)$ we have $ f(1)^2 = 1$

and since $ f(f(1)) = 1$ , $ f(1) = 1$

.....

 
\end{solution}



\begin{solution}[by \href{https://artofproblemsolving.com/community/user/29428}{pco}]
	\begin{tcolorbox}[quote="nguyenvuthanhha"]\begin{italicized}Find all function $ f : \mathbb{R} \ \to \ \mathbb{R}$ such that :

   $ f((x + 1)f(y)) \ = \ y (f(x) + 1) \ \forall \ \ x ; y \ \in \ \mathbb{R}$

 I think the problem will become very easy if we can prove $ f(1) = 1$ but i can't reach that arguement \end{italicized}\end{tcolorbox}

let $ P(x,y)$ be the assertion $ f((x + 1)f(y)) = y(f(x) + 1) \forall x,y$.

$ P( - 1,0)\implies f(0) = 0$

$ P( - 1, 1)\implies f( - 1) = - 1$

$ P(0,y)\implies f(f(y)) = y$

$ P( - 2, - 1)\implies f( - f( - 1)) = f(1) = - (f( - 2) + 1)$ so $ f( - 2) = - 1 - f(1)$ $ (1)$

$ P( - 2,f(1))\implies f(( - 2 + 1)f(f(1))) = f( - 1) = - 1 = f(1)f( - 2) + f(1)$  $ (2)$

from $ (1)$ and $ (2)$ we have $ f(1)^2 = 1$

and since $ f(f(1)) = 1$ , $ f(1) = 1$

.....

 \end{tcolorbox}

Yess, nicer and quicker than my own path to $ f(1)=1$
Congrats :)
\end{solution}



\begin{solution}[by \href{https://artofproblemsolving.com/community/user/40922}{mehdi cherif}]
	thank's  :blush:
\end{solution}



\begin{solution}[by \href{https://artofproblemsolving.com/community/user/43536}{nguyenvuthanhha}]
	\begin{italicized}Yes , after prove $ f(1) = 1$

   We have $ f(x + 1) = f(x) + 1$ ( put $ y = 1$ into the initial equation )

 Consider $ P( x - 1; f(y) ) \rightarrow f(xy) = f(x) f(y)$

  $ f(x + y) = f \left(y \left( 1 + \frac {x}{y} \right) \right) = f \left( \left( 1 + \frac {x}{y} \right) f(f(y))\right) = f(y)\left( f \left( \frac {x}{y} \right) + 1 \right) = f(x) + f(y) \forall x ; y \ \in \ \mathbb{R} ; y \ \neq \ 0$
  
But $ f(0) = 0$
 So we have $ f(xy) = f(x) f(y) ; f(x + y) = f(x) + f(y) \ \forall x ; y \ \in \ \mathbb{R}$

  By a well - known lemma , we have known that $ f(x) = x \forall x \ \in \ \mathbb{R}$
or $ f(x) = 0 \forall x \ \in \ \mathbb{R}$

  After an easy check , we can eliminate the case $ f(x) = 0 \forall x \ \in \ \mathbb{R}$

 and $ f(x) = x \forall x \ \in \ \mathbb{R}$ is the unique root of the functional equation  \end{italicized}
\end{solution}
*******************************************************************************
-------------------------------------------------------------------------------

\begin{problem}[Posted by \href{https://artofproblemsolving.com/community/user/29214}{HTA}]
	Find all functions $f: \mathbb R \to \mathbb R$ satisfying 
\[f(x) + f(y) \leq f(x+y), \quad \forall x,y \in \mathbb R,\]
and \[\lim_{x\to 0}\frac{f(x)}{x} = 1.\]
	\flushright \href{https://artofproblemsolving.com/community/c6h297362}{(Link to AoPS)}
\end{problem}



\begin{solution}[by \href{https://artofproblemsolving.com/community/user/29428}{pco}]
	\begin{tcolorbox}Find all function $ f : R - > R$ satisfying 

$ f(x) + f(y) \leq f(x + y)$

and $ lim_{x - > 0}\frac {f(x)}{x} = 1$\end{tcolorbox}

[hide="My solution"]
Let $ P(x,y)$ be the assertion $ f(x)+f(y)\leq f(x+y)$

It's immediate to establish with induction that $ f(nx)\geq nf(x)$ $ \forall n\in\mathbb N$, and so $ f(x)\geq nf(\frac xn)$ and so $ f(x)\geq x\frac{f(\frac xn)}{\frac xn}$

Setting $ n\to +\infty$, we get  $ f(x)\geq x$ $ \forall x$ and so $ f(x)+f(-x)\geq x-x=0$ $ \forall x$

But then : $ P(x,0)$ $ \implies$ $ f(0)\leq 0$ and $ P(x,-x)$ $ \implies$ $ f(x)+f(-x)\leq f(0)\leq 0$

So $ f(x)+f(-x)=0$ $ \forall x$, so $ f(-x)=-f(x)$ and $ P(-x,-y)$ $ \implies$ $ -f(x)-f(y)\leq -f(x+y)$ and so $ f(x)+f(y)\geq f(x+y)$

So $ f(x+y)=f(x)+f(y)$, then $ f(0)=0$ and so $ f(x)$ is continous at $ 0$.

And it is a classical result that any solution of Cauchy's equation which is continuous at one point is continuous over $ \mathbb R$, so is $ ax$

And here, $ a=1$ in order to get the right limit at $ 0$ and so $ \boxed{f(x)=x}$ is the unique solution (and is, indeed, a solution)
[\/hide]
\end{solution}
*******************************************************************************
-------------------------------------------------------------------------------

\begin{problem}[Posted by \href{https://artofproblemsolving.com/community/user/48364}{cnyd}]
	Find all functions $ f: \mathbb{N}_{0}\rightarrow\mathbb{N}_{0}$ such that
\[f(m + f(n)) = f(f(m)) + f(n)\]
for all $m,n\in\mathbb{N}_{0}$.
	\flushright \href{https://artofproblemsolving.com/community/c6h297535}{(Link to AoPS)}
\end{problem}



\begin{solution}[by \href{https://artofproblemsolving.com/community/user/35243}{earldbest}]
	I'm not very good with functional equations, so please check if I took wrong\/too many steps.  :D 

let $ P(m,n)$ denote the assertion $ f[m + f(n)] = f[f(m)] + f(n).$

$ P(0,0) \Rightarrow f(0) = 0.$
$ P(n,0) \Rightarrow f(m) = f(f(m)).$

suppose for some $ m$, $ f(m) = k \Rightarrow f(k) = k.$
then, $ P(k,m) \Rightarrow f(2k) = f(k) + f(k).$
This is the Cauchy functional equation, and since $ f(k) = k$, therefore $ f(x) = x$ for the given. :nhl:
\end{solution}



\begin{solution}[by \href{https://artofproblemsolving.com/community/user/48364}{cnyd}]
	i think we get $ f(n.k) = nk$  for $ \forall n\in\mathbb{N}_{0}$
could you explain better ?
\end{solution}



\begin{solution}[by \href{https://artofproblemsolving.com/community/user/29428}{pco}]
	\begin{tcolorbox}$ f: \mathbb{N}_{0}\rightarrow\mathbb{N}_{0}$
$ f[m + f(n)] = f[f(m)] + f(n)$ for $ \forall m,n\in\mathbb{N}_{0}$\end{tcolorbox}

Let $ P(x,y)$ be the assertion $ f(x + f(y)) = f(f(x)) + f(y)$

$ f(x) = 0$ $ \forall x$ and $ f(x) = x$ $ \forall x$  are two trivial solutions and we'll now search for other solutions :

Let $ A = f(\mathbb N_0)$
Let $ u = \gcd($all non zero elements of $ A)$
$ P(0,0)$ $ \implies$ $ f(0) = 0$
$ P(0,x)$ $ \implies$ $ f(f(x)) = f(x)$ and so $ f(x) = x$ $ \forall x\in A$ and so $ f(x + y) = x + y$ $ \forall x,y\in A$ and so $ f(mx + ny) = mx + ny$ ** edited ** : "$ \forall x,y,m,n\in\mathbb N_0$" ==> $ \forall x,y\in A$, $ \forall m,n\in\mathbb N_0$

We know that $ \exists x_0,y_0\in A^*$ such that $ \gcd(x,y) = u$. 
So $ \exists m,n,m',n'\in\mathbb N_0$ such that $ m'x_0 + n'y_0 = mx_0 + ny_0 + u$

Then $ P(u,mx_0 + nx_0)$ $ \implies$ $ f(u) = u$ and so $ A = \{nu, n\in\mathbb N_0\}$

So, up to now, we know that $ f(m)$ is always a multiple of $ u$ and that $ f(m) = m$ for any multiple of $ u$.
We also know that $ f(m + ku) = f(m) + ku$ and so it remains to define $ f(m)$ for $ m\in[1,u - 1]$ in order to have a full definition for $ f$.
But we'll now show that $ f$ may take any values (multiple of $ u$) for $ m\in[1,u - 1]$ :

Let $ g(m)$ any function from $ [0,u - 1]\to\mathbb N_0$ such that $ g(0) = 0$ and $ f(n)$ defined as :

$ \forall a\in[0,u - 1],\forall b$ : $ f(a + ub) = u(g(a) + b))$

Then let $ m = a + ub$ and $ n = c + ud$ :
$ f(n) = u(g(c) + d)$
$ m + f(n) = a + u(b + g(c) + d)$ and so $ f(m + f(n)) = u(g(a) + b + g(c) + d)$
$ f(m) = u(g(a) + b)$ and so $ f(f(m)) = u(g(a) + b)$ and so $ f(f(m)) + f(n) = y(g(a) + b + g(c) + d) = f(m + f(n))$

Hence the results :
$ f(x) = 0$ $ \forall x$
$ f(x) = u\cdot g(u\{\frac xu\}) + u[\frac xu]$ where $ u$ is any positive integer and $ g(m)$ any function from $ [0,u - 1]\to\mathbb N_0$ such that $ g(0) = 0$.
(Notice that $ u = 1$ gives the solution $ f(x) = x$)
\end{solution}



\begin{solution}[by \href{https://artofproblemsolving.com/community/user/29428}{pco}]
	\begin{tcolorbox}... please check if I took wrong\/too many steps.  ...\end{tcolorbox}

First, you can get $ f(x+y)=f(x)+f(y)$, but only for $ x,y\in f(\mathbb N_0)$ and not for any $ x,y$
Second, Cauchy equation implies $ f(x)=ax$ only if it is true at least over $ \mathbb Q$, so you could not use this conclusion here.
\end{solution}



\begin{solution}[by \href{https://artofproblemsolving.com/community/user/35243}{earldbest}]
	\begin{tcolorbox}

First, you can get $ f(x + y) = f(x) + f(y)$, but only for $ x,y\in f(\mathbb N_0)$ and not for any $ x,y$
Second, Cauchy equation implies $ f(x) = ax$ only if it is true at least over $ \mathbb Q$, so you could not use this conclusion here.\end{tcolorbox}


oh, ok.  :D
\end{solution}



\begin{solution}[by \href{https://artofproblemsolving.com/community/user/29876}{ozgurkircak}]
	Cuneyt what is the source of this problem?
\end{solution}
*******************************************************************************
-------------------------------------------------------------------------------

\begin{problem}[Posted by \href{https://artofproblemsolving.com/community/user/54134}{Lousin Garckz}]
	Let $ f : \mathbb{R} \rightarrow \mathbb{R}$ be such that \[ f(x) + f\left(\dfrac{x+1003}{x-1}\right) = 4024-x\] for all real $x$. Find the value $ f(2009)$.
	\flushright \href{https://artofproblemsolving.com/community/c6h297738}{(Link to AoPS)}
\end{problem}



\begin{solution}[by \href{https://artofproblemsolving.com/community/user/29428}{pco}]
	\begin{tcolorbox}Let $ f : \mathbb{R} \rightarrow \mathbb{R}$ be such that $ f(x) + f(\dfrac{x + 1003}{x - 1}) = 4024 - x$. Find the value $ f(2009)$.\end{tcolorbox}

Let $ P(x)$ be the assertion $ f(x)+f(\frac{x+1003}{x-1})=4024-x$

I suppose that the given functional equation is true $ \forall x\neq 1$

Then no such function exists : 

Let $ g(x)=\frac{x+1003}{x-1}$. Notice that we have $ g(g(x))=x$

Then $ P(x)$ is  $ f(x)+f(g(x))=4024-x$, true $ \forall x\neq 1$
Then $ P(g(x))$ $ \implies$ $ f(g(x))+f(x)=4024-g(x)$ true $ \forall x\neq 1$ too

And so $ g(x)=x$ $ \forall x\neq 1$, which is obviously wrong
\end{solution}



\begin{solution}[by \href{https://artofproblemsolving.com/community/user/54134}{Lousin Garckz}]
	thanks pco!!.... but i'm really surprised because this was a problem in a contest and it didn't say determine if there exists a function, the problem was to assume that exist and find the value, oh too bad!! :rotfl:

p\/d: pco, can you solve the problem that i posted about the number of points in $ \mathbb{Z}_{p}^{2}$ such that $ x^{2}+y^{2} = a, a \in \mathbb{Z}_{p}$, please? you are great!
\end{solution}



\begin{solution}[by \href{https://artofproblemsolving.com/community/user/29428}{pco}]
	\begin{tcolorbox}thanks pco!!.... but i'm really surprised because this was a problem in a contest and it didn't say determine if there exists a function, the problem was to assume that exist and find the value, oh too bad!! :rotfl:

p\/d: pco, can you solve the problem that i posted about the number of points in $ \mathbb{Z}_{p}^{2}$ such that $ x^{2} + y^{2} = a, a \in \mathbb{Z}_{p}$, please? you are great!\end{tcolorbox}

If you want to show the contradiction in a very short way :

$ x=2$ $ \implies$ $ f(2)+f(1005)=4022$
$ x=1005$ $ \implies$ $ f(1005)+f(2)=3019$
\end{solution}
*******************************************************************************
-------------------------------------------------------------------------------

\begin{problem}[Posted by \href{https://artofproblemsolving.com/community/user/67317}{lgmco}]
	Find all functions $ f: \mathbb Z\to \mathbb Z$ such that
\[ f(x-y+f(y))=f(x)+f(y),\] for all $x, y \in \mathbb Z$.
	\flushright \href{https://artofproblemsolving.com/community/c6h298172}{(Link to AoPS)}
\end{problem}



\begin{solution}[by \href{https://artofproblemsolving.com/community/user/45922}{Kirill}]
	After $ x = y$ we will have $ f(f(x)) = 2f(x)$.
After $ y\rightarrow f(y)$ we will have $ f(x + f(y)) = f(x) + 2f(y)$.
For $ x = 0$: $ f(f(y)) = f(0) + 2f(y)$, but $ f(f(y)) = 2f(y)$, so $ f(0) = 0$.

Now, we have two cases:
1. Suppose, there are exist $ a\neq b\in Z$ such that $ f(a) = f(b)$. So, we have:
$ f(x - a + f(a)) = f(x - b + f(b))$
So, $ f$ is periodic. $ f: Z\rightarrow Z$, and $ f$ - periodic, so there exist $ f_{max}$ and $ f_{min}$.
By taking $ x = y = u$ with $ f_{max} = f(u)$, we will have:
$ f(f(u)) = 2f_{max}$, so $ f_{max}\leq 0$.
By taking $ x = y = v$ with $ f_{min} = f(v)$, we will have:
$ f(f(v)) = 2f_{min}$, so $ f_{min}\geq 0$.
From $ f_{max}\leq 0$ and $ f_{min}\geq 0$ we have $ f(x) = 0$ for all $ x$ - it's solution.

2. From $ f(a) = f(b)$ we will have $ a = b$.
Put $ x = 0:$ $ f(f(y) - y) = f(y)$, so $ f(y) - y = y$, and $ f(x) = 2x$ is solution - it's easy to check.
\end{solution}



\begin{solution}[by \href{https://artofproblemsolving.com/community/user/67317}{lgmco}]
	thanks for your fast reply  :)
\end{solution}



\begin{solution}[by \href{https://artofproblemsolving.com/community/user/45922}{Kirill}]
	lgmco, thanks to you. It's very nice Functional Equation.
Is there any mistakes in my solution (I hope, no)?
\end{solution}



\begin{solution}[by \href{https://artofproblemsolving.com/community/user/67317}{lgmco}]
	I don't think that  
\end{solution}



\begin{solution}[by \href{https://artofproblemsolving.com/community/user/29428}{pco}]
	\begin{tcolorbox}lgmco, thanks to you. It's very nice Functional Equation.
Is there any mistakes in my solution (I hope, no)?\end{tcolorbox}
Quite OK and nice solution!  :) 
Congrats.
\end{solution}



\begin{solution}[by \href{https://artofproblemsolving.com/community/user/40922}{mehdi cherif}]
	by simmetry : $ f(x-y+f(y))=f(y-x+f(x))$

$ y=0$ $ \implies$ $ f(x)=f(f(x)-x)$

so $ x,y\rightarrow  f(y)-y$ $ \implies$ $ f(x-f(y)+y+f(y))=f(x)+f(f(y)-y)=f(x)+f(y)$

so $ f(x+y)=f(x)+f(y)$

and it is cauchy equation $ \forall x,y \in \mathbb{Z}$ so $ f(x)=ax$

replace in the original equation :

$ a(x-y+ay)=a(x+y)\implies a=2$

 :)
\end{solution}
*******************************************************************************
-------------------------------------------------------------------------------

\begin{problem}[Posted by \href{https://artofproblemsolving.com/community/user/54046}{SUPERMAN2}]
	Find all functions $f: \mathbb R \to \mathbb R$ satisfying \[ f(x^2+f(y)-y)=(f(x))^2-f(y)\] for all $ x,y \in \mathbb R$.
	\flushright \href{https://artofproblemsolving.com/community/c6h298413}{(Link to AoPS)}
\end{problem}



\begin{solution}[by \href{https://artofproblemsolving.com/community/user/29428}{pco}]
	\begin{tcolorbox}Find all function $ f: R \to R$ satisfying  $ f(x^2 + f(y) - y) = (f(x))^2 - f(y)$ $ \forall$ $ x,y \in R$\end{tcolorbox}

Let $ P(x,y)$ be the assertion $ f(x^2+f(y)-y)=f(x)^2-f(y)$

1) Suppose $ \exists u$ such that $ f(u)=0$
$ P(u,u)$ $ \implies$ $ f(u^2-u)=0$ and $ P(-u,u)$ $ \implies$ $ f(u^2-u)=f(-u)^2$ and so $ f(-u)=0$
$ P(0,u)$ $ \implies$ $ f(-u)=f(0)^2$ and so $ f(0)=0$
$ P(x,0)$ $ \implies$ $ f(x^2)=f(x)^2$ and so $ f(x)\geq 0$ $ \forall x\geq 0$
Comparing $ P(x,y)$ and $ P(-x,y)$, we get that $ f(-x)=\pm  f(x)$

Then $ P(0,x)$ $ \implies$ $ f(f(x)-x))=-f(x)$. So $ x\geq 0$ would imply $ -f(x)\leq 0$ and so $ f(x)-x\leq 0$ and so $ x-f(x)\geq 0$ and so $ f(x-f(x))=\pm f(f(x)-x)=\pm f(x)=f(x)$ (since both $ f(x-f(x))$ and $ f(x)$ are $ \geq 0$)
So we get $ x\geq 0$ $ \implies$ $ x-f(x)\geq 0$ and $ f(x-f(x))=f(x)$
Applying this to $ x-f(x)$, we get $ x-f(x)-f(x-f(x))=x-2f(x)\geq 0$ and $ f(x-2f(x))=f(x-f(x)-f(x-f(x)))=f(x-f(x))=f(x)$ and so $ x\geq 0$ $ \implies$ $ x-2f(x)\geq 0$ and $ f(x-2f(x))=f(x)$
An immediate induction shows that $ x\geq 0$ $ \implies$ $ x-nf(x)\geq 0$ and $ f(x-nf(x))=f(x)$ $ \forall n\in\mathbb N$
And $ x\geq nf(x)$ $ \forall n$ obviously implies $ f(x)=0$ $ \forall x\geq 0$ and so (since $ f(-x)=\pm f(x)$), $ f(x)=0$ $ \forall x$

And it is immediate to check back that this function indeed is a solution.

2) Suppose $ f(x)\neq 0$ $ \forall x$
If $ \exists u$ such that $ f(-u)=-f(u)$, then Wlog say $ f(u)\leq 0$ (else swap $ u$ and $ -u$). Then $ P(\sqrt{-f(u)},u)$ $ \implies$ $ f(-u)=f(\sqrt{-f(u)})^2-f(u)$ and so $ f(\sqrt{-f(u)})=0$, which is impossible in this paragraph 2.

So $ f(-x)=f(x)$ $ \forall x$
Suppose now $ \exists u$ such that $ f(u)<0$. Then $ P(\sqrt{-f(u)},u)$ $ \implies$ $ f(-u)=f(\sqrt{-f(u)})^2-f(u)$ and so $ 2f(u)=f(\sqrt{-f(u)})^2$, which is impossible, since $ LHS<0$ while $ RHS>0$
So $ f(x)>0$ $ \forall x$
As a consequence, looking at RHS of $ P(x,y)$, we get $ f(x)^2>f(y)$ $ \forall x,y$ and so $ f(x)$ is bounded and $ \exists M$ such that $ 0<f(x)<M$ $ \forall x$
Let then $ y>M$, and so $ y>f(y)$ and $ x\geq 0$ : $ x+y-f(y)\geq 0$ and so :
$ P(\sqrt{x+y-f(y)},y)$ $ \implies$ $ f(x)=f(\sqrt{x+y-f(y)})^2-f(y)$
$ P(\sqrt{x+y-f(y)},-y)$ $ \implies$ $ f(x+2y)=f(\sqrt{x+y-f(y)})^2-f(y)$
And so $ f(x+2y)=f(x)$ $ \forall x\geq 0$, $ \forall y>M$

And so $ f(x)=a$ is a constant.
Plugging back in the original equation, we get $ a=a^2-a$ and so $ a=2$ (since $ f(x)\neq 0$ in this paragraph).

3) synthesis of solutions :
We got only two solutions :
$ f(x)=0$ $ \forall x$
$ f(x)=2$ $ \forall x$
\end{solution}



\begin{solution}[by \href{https://artofproblemsolving.com/community/user/43536}{nguyenvuthanhha}]
	\begin{italicized}Nice , actually , you have forgotten that you did that problem recently 

 And you have solved one hard problem in two distinct ways \end{italicized} 
\end{solution}



\begin{solution}[by \href{https://artofproblemsolving.com/community/user/29428}{pco}]
	\begin{tcolorbox}\begin{italicized}Nice , actually , you have forgotten that you did that problem recently 

 And you have solved one hard problem in two distinct ways \end{italicized} \end{tcolorbox}

Ohhh, you're right! you submitted this problem [url=http://www.mathlinks.ro/Forum/viewtopic.php?t=286774]Here[\/url].

I forgot   

And, btw, the two solutions I gave are not very different.
\end{solution}
*******************************************************************************
-------------------------------------------------------------------------------

\begin{problem}[Posted by \href{https://artofproblemsolving.com/community/user/43536}{nguyenvuthanhha}]
	Find all functions $ f : \mathbb{R^{+}} \ \to \ \mathbb{R^{+}}$ satisfies two conditions :

1 \/ If $ 0 < x < y$ then $ 0< f(y) \ \leq \ f(x)$

 2 \/ $ f(xy) f \left( \frac{f(y)}{x} \right) \ = \ 1 \ \forall \ x;y \ \in \ \mathbb{R^{+}}$
	\flushright \href{https://artofproblemsolving.com/community/c6h298438}{(Link to AoPS)}
\end{problem}



\begin{solution}[by \href{https://artofproblemsolving.com/community/user/9911}{Albanian Eagle}]
	[hide="edit"] there was some nonsense here, i had misread the problem. [\/hide]
\end{solution}



\begin{solution}[by \href{https://artofproblemsolving.com/community/user/29428}{pco}]
	\begin{tcolorbox}Find all functions $ f : \mathbb{R^{ + }} \ \to \ \mathbb{R^{ + }}$ satisfies two conditions :

1 \/ If $ 0 < x < y$ then $ 0 < f(y) \ \leq \ f(x)$

 2 \/ $ f(xy) f \left( \frac {f(y)}{x} \right) \ = \ 1 \ \forall \ x;y \ \in \ \mathbb{R^{ + }}$\end{tcolorbox}

Let $ P(x,y)$ be the assertion $ f(xy)f(\frac{f(y)}x)=1$

$ P(\sqrt{f(1)},1)$ $ \implies$ $ f(\sqrt{f(1)})=1$ and then $ P(\frac 1{\sqrt{f(1)}},\sqrt{f(1)})$ $ \implies$ $ f(1)=1$

Now two possibilities :

1) $ \exists u\neq 1$ such that $ f(u)=1$. Then :
$ P(u,1)$ $ \implies$ $ f(\frac 1u)=1$
$ P(u,u)$ $ \implies$ $ f(u^2)=1$
And so $ f(u^{2^n})=f(u^{-2^n})=1$ and so $ f(x)=1$ $ \forall x$ since $ f(x)$ is non increasing.
And it is easy to check that this indeed is a solution.

2) $ f(u)=1$ $ \iff$ $ u=1$. Then :
$ P(\sqrt{\frac{f(x)}x},x)$ $ \implies$ $ f(\sqrt{xf(x)})=1$ $ \implies$ $ \sqrt{xf(x)}=1$ $ \implies$ $ f(x)=\frac 1x$
And it is easy to check that this indeed is a solution.

Hence the two solutions :
$ f(x)=1$ $ \forall x$

$ f(x)=\frac 1x$ $ \forall x$
\end{solution}
*******************************************************************************
-------------------------------------------------------------------------------

\begin{problem}[Posted by \href{https://artofproblemsolving.com/community/user/46787}{moldovan}]
	Prove that there are functions $ f: \mathbb{Z} \rightarrow \mathbb{Z}$ so that $ f(f(x))+f(x)+x=0$ for any $ x \in \mathbb{Z}$.
	\flushright \href{https://artofproblemsolving.com/community/c6h298482}{(Link to AoPS)}
\end{problem}



\begin{solution}[by \href{https://artofproblemsolving.com/community/user/29428}{pco}]
	\begin{tcolorbox}Prove that there are functions $ f: \mathbb{Z} \rightarrow \mathbb{Z}$ so that $ f(f(x)) + f(x) + x = 0$ for any $ x \in \mathbb{Z}$.\end{tcolorbox}

Here is a simple way to create one :

Construct the following sequences of natural numbers $ a_n,b_n,c_n$ ($ n > 0$)

$ a_1 = 1$
$ b_1 = 2$
$ c_1 = 3$

For $ n > 1$ : Let $ A_n = \bigcup_{1\leq k < n}\{a_k,b_k,c_k\}$. Then :
Let $ a_n =$ fewest positive integer not in $ A_n$
Let $ b_n =$ fewest positive integer such that $ b_n\notin A_n\bigcup\{a_n\}$ and $ (a_n + b_n)\notin A_n\bigcup\{a_n\}$
Let $ c_n = a_n + b_n$

[hide="What these sequences look like"]
This will construct the following $ (a_n,b_n,c_n)$ : 
$ (1,2,3),(4,5,9),(6,7,13),(8,10,18),(11,12,23),$
$ (13,14,27),(15,16,31),(17,19,36),(20,21,41),$
$ (22,24,46) ...$
[\/hide]

It's immediate to see that all $ a_i,b_j,c_k$ are pairwise different and that $ \bigcup_k\{a_k,b_k,c_k\} = \mathbb N$

Then define $ f(x)$ as :
$ f(0) = 0$
For any $ n\neq 0$, $ |n|\in\mathbb N$ and $ \exists! k > 0$ such that $ |n|\in\{a_k,b_k,c_k\}$. Then :
If  $ a_k = |n|$, then $ f(n) = sign(n)b_k$
If  $ b_k = |n|$, then $ f(n) = - sign(n)c_k$
If  $ c_k = |n|$, then $ f(n) = - sign(n)a_k$

And it is easy to see that $ (x,f(x),f(f(x))$ is :
either $ (0,0,0)$
either $ (a_k,b_k, - c_k)$
either $ (b_k, - c_k,a_k)$
either $ (c_k, - a_k, - b_k)$
either $ ( - a_k, - b_k,c_k)$
either $ ( - b_k,c_k, - a_k)$
either $ ( - c_k,a_k,b_k)$
And in all cases $ x + f(x) + f(f(x)) = 0$
\end{solution}
*******************************************************************************
-------------------------------------------------------------------------------

\begin{problem}[Posted by \href{https://artofproblemsolving.com/community/user/66953}{Elementaryyy}]
	Let $ f(x)$ be a continuous functions on $ [0,1]$ and:
1. $ f(0) = 0$ and $f(1) = 1$, and
2. When $ a\in (0,1), x,y\in [0,1] , x\leq y$, \[ f\left(\frac {x + y}{2}\right) = (1 - a)f(x) + af(y).\]
Find $ f\left(\frac {1}{7}\right)$.
	\flushright \href{https://artofproblemsolving.com/community/c6h298763}{(Link to AoPS)}
\end{problem}



\begin{solution}[by \href{https://artofproblemsolving.com/community/user/29428}{pco}]
	\begin{tcolorbox}$ f(x)$ is a continuous functions on $ [0,1]$ and:
1. $ f(0) = 0 , f(1) = 1$
2. $ f(\frac {x + y}{2}) = (1 - a)f(x) + af(y)$ when $ a\in (0,1), x,y\in (0,1) , x\leq y$

Find $ f(\frac {1}{7})$

I want a simple solution.\end{tcolorbox}

Set $ x\to 0$ in 2. Continuity gives $ f(\frac y2) = af(y)$

Set $ y\to 0$ in 2. Continuity gives $ f(\frac x2) = (1 - a)f(x)$

So $ a = \frac 12$ (since $ f(1) = 1$ and so $ f(x)$ is not all zero) and $ f(\frac {x + y}2) = \frac {f(x) + f(y)}2$

This is a very classical equation whose immediate unique solution (since continuous and $ f(0) = 0$ and $ f(1) = 1$) is $ f(x) = x$ 

[hide="why ?"]
Let $ 0<a<b<1$. It's immediate to show thru induction that $ f(\frac p{2^q}a+(1-\frac p{2^q})b)=\frac p{2^q}f(a)+(1-\frac p{2^q})f(b)$ $ \forall q\in\mathbb N$, $ \forall p\in\mathbb N_0\cap[0,2^q]$

Then setting $ q\to +\infty$ and using continuity : $ f(x)=\frac{f(b)-f(a)}{b-a}(x-a)+f(a)$ $ \forall x\in[a,b]$

Setting then $ a\to 0$ and $ b\to 1$ and using continuity again, we get $ f(x)=x$
[\/hide]

So $ f(\frac 17) = \frac 17$
\end{solution}



\begin{solution}[by \href{https://artofproblemsolving.com/community/user/66953}{Elementaryyy}]
	Because $ x\leq y$ then if $ y=0$ then $ x=0$ so we can not have 
$ f(\frac{x}{2}) = (1-a)f(x)$
\end{solution}



\begin{solution}[by \href{https://artofproblemsolving.com/community/user/29428}{pco}]
	\begin{tcolorbox}Because $ x\leq y$ then if $ y = 0$ then $ x = 0$ so we can not have 
$ f(\frac {x}{2}) = (1 - a)f(x)$\end{tcolorbox}

You're right  :oops:
\end{solution}



\begin{solution}[by \href{https://artofproblemsolving.com/community/user/29428}{pco}]
	\begin{tcolorbox}$ f(x)$ is a continuous functions on $ [0,1]$ and:
1. $ f(0) = 0 , f(1) = 1$
2. $ f(\frac {x + y}{2}) = (1 - a)f(x) + af(y)$ when $ a\in (0,1), x,y\in [0,1] , x\leq y$

Find $ f(\frac {1}{7})$

I want a simple solution.\end{tcolorbox}

Next trial :)

Let $ P(x,y)$ be the assertion $ f(\frac{x+y}2)=(1-a)f(x)+af(y)$

$ P(0,1)$ $ \implies$ ${ f(\frac 12})=a$

$ P(0,\frac 12)$ $ \implies$ $ f(\frac 14)=a^2$

$ P(\frac 12,1)$ $ \implies$ $ f(\frac 34)=2a-a^2$

$ P(\frac 14,\frac 34)$ $ \implies$ $ f(\frac 12)=(1-a)a^2+a(2a-a^2)=3a^2-2a^3$

And so $ a=3a^2-2a^3$ and so $ a(a-1)(2a-1)=0$ and so $ a=\frac 12$ (since $ a\in(0,1)$

and so $ f(\frac {x + y}2) = \frac {f(x) + f(y)}2$

This is a very classical equation whose immediate unique solution (since continuous and $ f(0) = 0$ and $ f(1) = 1$) is $ f(x) = x$ 

[hide="why ?"]
Let $ 0 < a < b < 1$. It's immediate to show thru induction that $ f(\frac p{2^q}a + (1 - \frac p{2^q})b) = \frac p{2^q}f(a) + (1 - \frac p{2^q})f(b)$ $ \forall q\in\mathbb N$, $ \forall p\in\mathbb N_0\cap[0,2^q]$

Then setting $ q\to + \infty$ and using continuity : $ f(x) = \frac {f(b) - f(a)}{b - a}(x - a) + f(a)$ $ \forall x\in[a,b]$

Setting then $ a\to 0$ and $ b\to 1$ and using continuity again, we get $ f(x) = x$
[\/hide]

So $ f(\frac 17) = \frac 17$
\end{solution}



\begin{solution}[by \href{https://artofproblemsolving.com/community/user/66953}{Elementaryyy}]
	I think it is true.
My solution are too long.
\end{solution}



\begin{solution}[by \href{https://artofproblemsolving.com/community/user/46638}{thanhtra_dhsp}]
	My solution:
Let $ P(x,y)$ be the assertion $ f(\frac {x + y}2) = (1 - a)f(x) + af(y)$
$ P(0, 1)\Rightarrow f(\frac {1}{2}) = a$
$ P(x, 0)\Rightarrow f(\frac {x}{2}) = (1 - a)f(x)$
$ P(7x, x)\Rightarrow f(4x) = (1 - a)f(7x) + af(x)\Rightarrow \frac {1}{(1 - a)^2}f(x) = (1 - a)f(7x) + af(x)$
$ x = \frac{1}{7}$, we have $ f(\frac {1}{7})$

P\s: Elementaryyy is ThieuSax
\end{solution}



\begin{solution}[by \href{https://artofproblemsolving.com/community/user/29428}{pco}]
	\begin{tcolorbox}My solution:
Let $ P(x,y)$ be the assertion $ f(\frac {x + y}2) = (1 - a)f(x) + af(y)$
$ P(0, 1)\Rightarrow f(\frac {1}{2}) = a$
$ P(x, 0)\Rightarrow f(\frac {x}{2}) = (1 - a)f(x)$
$ P(7x, x)\Rightarrow f(4x) = (1 - a)f(7x) + af(x)\Rightarrow \frac {1}{(1 - a)^2}f(x) = (1 - a)f(7x) + af(x)$
$ x = \frac {1}{7}$, we have $ f(\frac {1}{7})$
\end{tcolorbox}

$ P(x,y)$ is true only for $ x\leq y$, so $ P(x,0)$ or $ P(7x,x)$ are true only for $ x=0$
\end{solution}



\begin{solution}[by \href{https://artofproblemsolving.com/community/user/66953}{Elementaryyy}]
	\begin{tcolorbox}My solution:
Let $ P(x,y)$ be the assertion $ f(\frac {x + y}2) = (1 - a)f(x) + af(y)$
$ P(0, 1)\Rightarrow f(\frac {1}{2}) = a$
$ P(x, 0)\Rightarrow f(\frac {x}{2}) = (1 - a)f(x)$
$ P(7x, x)\Rightarrow f(4x) = (1 - a)f(7x) + af(x)\Rightarrow \frac {1}{(1 - a)^2}f(x) = (1 - a)f(7x) + af(x)$
$ x = \frac {1}{7}$, we have $ f(\frac {1}{7})$
\end{tcolorbox}
Do you read  $ x \leq y$

P\/s: Thanhtra_dhsp : Anh che'e'e'e'e'm
\end{solution}
*******************************************************************************
-------------------------------------------------------------------------------

\begin{problem}[Posted by \href{https://artofproblemsolving.com/community/user/16129}{posabogdan}]
	Let $ f: \mathbb N \to \mathbb Z$ be a function such that for all positive integers $n$, \[(f(n+1)-f(n))(f(n+1)+f(n)+4)\leq0.\] Prove that $f$ is not injective.
	\flushright \href{https://artofproblemsolving.com/community/c6h298954}{(Link to AoPS)}
\end{problem}



\begin{solution}[by \href{https://artofproblemsolving.com/community/user/29428}{pco}]
	\begin{tcolorbox}Let $ f: N \to Z$ s.t $ (f(n + 1) - f(n))(f(n + 1) + f(n) + 4)\leq0,\forall n\in N$.Prove that f is not injective.\end{tcolorbox}

Suppose $ f(x)$ is injective. So $ f(n+1)\neq f(n)$. Then : 

$ f(n+1)>f(n)$ $ \implies$ $ f(n+1)+f(n)\leq -4$ $ \implies$ $ -4-f(n)\geq f(n+1)>f(n)$ $ \implies$ $ -2-f(n)\geq 2+f(n+1)>2+f(n)$ $ \implies$ $ |f(n+1)+2|\leq |f(n)+2|$

$ f(n+1)<f(n)$ $ \implies$ $ f(n+1)+f(n)\geq -4$ $ \implies$ $ -4-f(n)\leq f(n+1)<f(n)$ $ \implies$ $ -2-f(n)\leq 2+f(n+1)<2+f(n)$ $ \implies$ $ |f(n+1)+2|\leq |f(n)+2|$

And so the function $ |2+f(n)|$ is a non negative non increasing function and so has a limit
So for $ n$ great enough, $ |f(n+1)+2|=|f(n)+2|$ and, since $ f(n+1)\neq f(n)$ : $ f(n+1)+2=-2-f(n)$ for all $ n$ great enough.

But this implies $ f(n+1)=-4-f(n)$ and so $ f(n+2)=-4-f(n+1)=f(n)$, so contradiction.

And no injective function fits the requirement.
Q.E.D.
\end{solution}
*******************************************************************************
-------------------------------------------------------------------------------

\begin{problem}[Posted by \href{https://artofproblemsolving.com/community/user/66394}{reason}]
	Find all $ f\in{C}^{2}(\mathbb{R},\mathbb{R})$ such that
\[f(x+y)f(x-y)=(f(x))^{2}+(f(y))^{2}-1\]
for all reals $x$ and $y$.
	\flushright \href{https://artofproblemsolving.com/community/c6h298996}{(Link to AoPS)}
\end{problem}



\begin{solution}[by \href{https://artofproblemsolving.com/community/user/29428}{pco}]
	\begin{tcolorbox}find all $ f\in{C}^{2}(\mathbb{R},\mathbb{R})$ such that:

$ f(x + y)f(x - y) = (f(x))^{2} + (f(y))^{2} - 1$

thanx.\end{tcolorbox}
Let $ P(x,y)$ be the assertion $ f(x+y)f(x-y)=f(x)^2+f(y)^2-1$

$ f(x)=+1$ $ \forall x$ and $ f(x)=-1$ $ \forall x$ are solutions
So we'll now consider that $ \exists u$ such that $ f(u)\neq 1$

$ P(x,0)$ $ \implies$ $ f(0)^2=1$ and, since $ f(x)$ solution implies $ -f(x)$ solution, wlog consider $ f(0)=1$

1) $ f(u)>1$
$ P(x,u)$ $ \implies$ $ f(x+u)f(x-u)>0$ and so $ f(x)\neq0$ $ \forall x$ and so $ f(x)>0$ $ \forall x$ since $ f(x)$ is $ C^2$, so continuous.
$ P(x,x)$ $ \implies$ $ f(2x)=2f(x)^2-1$ and it is to see that if $ f(x)\in(0,1)$, then $ f(2^nx)<0$ sor some $ n$, which is impossible.
So $ f(x)\geq 1$ $ \forall x$
Let then $ f(a)=\cosh(b)$, $ b\geq 0$. From the above equation, we get $ f(\frac a{2^n})=\cosh(\frac b{2^n})$. 
Setting $ n\to+\infty$, and using the fact that $ f(x)$ is $ C^2$, we get $ f'(0)=0$ and $ f''(0)=\frac{b^2}{a^2}$ and so $ \frac{b^2}{a^2}=$ constant and so $ f(x)=\cosh(cx)$
And it is easy to see that this function indeed is a solution.

2) $ f(u)<1$
If $ f(u)\leq 0$, we get that $ \exists v$ such that $ f(v)=0$
If $ f(u)>0$, $ P(x,x)$ $ \implies$ $ f(2u)=2f(u)^2-1$ and it is easy too to see that $ f(2^nu)<0$ sor some $ n$ and so  $ \exists v$ such that $ f(v)=0$.
Then let $ A=\{x>0$ such that $ f(x)=0$ and let $ w=\inf(A)$. Continuity gives that $ f(w)=0$ and $ f(x)>0$ $ \forall x\in[0,w)$
Let then $ a\in[0,w]$, so that $ f(a)\in[0,1]$ and let $ f(a)=\cos(b)$ with $ b\in[0,\frac{\pi}2]$
From $ f(2x)=2f(x)^2-1$ and $ f(x)\in(0,1]$ $ \forall x\in[0,w)$, we get that $ f(\frac a{2^n})=\cos(\frac b{2^n})$.
Setting $ n\to+\infty$, and using the fact that $ f(x)$ is $ C^2$, we get $ f'(0)=0$ and $ f''(0)=-\frac{b^2}{a^2}$ and so $ \frac{b^2}{a^2}=$ constant and so $ f(x)=\cos(cx)=\cos(\frac{\pi}{2w}x)$ $ \forall x\in[0,w]$

Then $ P(x,-x)$ $ \implies$ $ f(-x)=\pm f(x)$ and so, with continuity, $ f(x)=\cos(\frac{\pi}{2w}x)$ $ \forall x\in[-w,w]$

Last, $ P(x+w,w)$ $ \implies$ $ f(x+2w)f(x)=f(x+w)^2-1$ and so knowledge of $ f(x)$ on $ [-w,+w]$ gives knowledge on $ [w,2w]$, then on $ [2w,3w]$, ... and a simple induction gives $ f(x)=\cos(\frac{\pi}{2w}x)$ $ \forall x$

And it is easy to see that this function indeed is a solution.

3) synthesis of solutions :
$ f(x)=-1$
$ f(x)=1$
$ f(x)=\cosh(ax)$
$ f(x)=-\cosh(ax)$
$ f(x)=\cos(ax)$
$ f(x)=-\cos(ax)$
\end{solution}



\begin{solution}[by \href{https://artofproblemsolving.com/community/user/43536}{nguyenvuthanhha}]
	\begin{italicized}Pco , The Lord of functional equations   \end{italicized}
\end{solution}
*******************************************************************************
-------------------------------------------------------------------------------

\begin{problem}[Posted by \href{https://artofproblemsolving.com/community/user/63660}{Victory.US}]
	Find all continuous functions $f: \mathbb R \to \mathbb R$ such that \[ f\left(\frac {x + y}{2}\right) + f\left( \frac {2xy}{x + y}\right) = f(x) + f(y)\]
holds for all $ x,y \ge 0$.
	\flushright \href{https://artofproblemsolving.com/community/c6h299576}{(Link to AoPS)}
\end{problem}



\begin{solution}[by \href{https://artofproblemsolving.com/community/user/29428}{pco}]
	\begin{tcolorbox}find all continuous function $ f: R\to R$ such that : $ f(\frac {x + y}{2}) + f( \frac {2xy}{x + y}) = f(x) + f(y)$ , $ \forall x,y \ge 0$\end{tcolorbox}

Setting $ y=0$ in the equation, we get $ f(x)=f(\frac x2)$ $ \forall x\geq 0$ and so $ f(x)=f(\frac x{2^n})$ and so, setting $ n\to +\infty$ and using continuity $ f(x)=f(0)$ $ \forall x\geq 0$

Hence the answer :
$ f(x)=a$ $ \forall x\geq 0$
$ f(x)=g(x)$ $ \forall x<0$ where $ g(x)$ is any continuous fonction such that $ g(0)=a$

And it is easy to check that this endeed is a solution.
\end{solution}



\begin{solution}[by \href{https://artofproblemsolving.com/community/user/54772}{enndb0x}]
	How you got this  $ f(x)=f\left(\frac{x}{2^n}\right)$   :oops:
\end{solution}



\begin{solution}[by \href{https://artofproblemsolving.com/community/user/45167}{Bugi}]
	$ f(x)=f(x\/2)=f(x\/4)=...=f(x\/2^n)$
\end{solution}



\begin{solution}[by \href{https://artofproblemsolving.com/community/user/63660}{Victory.US}]
	How can i show that $ f(\sqrt {xy} ) = \frac{{f(x) + f(y)}}{2}$
 it is true   :oops:
\end{solution}



\begin{solution}[by \href{https://artofproblemsolving.com/community/user/29428}{pco}]
	\begin{tcolorbox}How can i show that $ f(\sqrt {xy} ) = \frac {{f(x) + f(y)}}{2}$
 it is true   :oops:\end{tcolorbox}

Could you be a little bit clearer ?

Are you speaking about another functional equation ?

If you are speaking about the original equation, I made the demo that the solution is $ f(x)=a$ $ \forall x\geq 0$ and so obviously $ f(\sqrt {xy} ) = a = \frac{a+a}2$ $ =\frac {{f(x) + f(y)}}{2}$ $ \forall x,y\geq 0$
\end{solution}



\begin{solution}[by \href{https://artofproblemsolving.com/community/user/63660}{Victory.US}]
	\begin{tcolorbox}find all continuous function $ f: R\to R$ such that : $ f(\frac {x + y}{2}) + f( \frac {2xy}{x + y}) = f(x) + f(y)$ , $ \forall x,y \ge 0$\end{tcolorbox} 
 $ f$ can be rewrote that $ f(\sqrt {xy} ) = \frac {{f(x) + f(y)}}{2}$. but i can't show it  :oops:
\end{solution}



\begin{solution}[by \href{https://artofproblemsolving.com/community/user/29428}{pco}]
	\begin{tcolorbox}\begin{tcolorbox}find all continuous function $ f: R\to R$ such that : $ f(\frac {x + y}{2}) + f( \frac {2xy}{x + y}) = f(x) + f(y)$ , $ \forall x,y \ge 0$\end{tcolorbox} 
 $ f$ can be rewrote that $ f(\sqrt {xy} ) = \frac {{f(x) + f(y)}}{2}$. but i can't show it  :oops:\end{tcolorbox}

In one line, we showed that $ f(x)$ is a constant function (for $ x\geq 0$). So, in the next line, we can write a lot of trivial things as : 

$ f(\sqrt {xy} ) = \frac {{f(x) + f(y)}}{2}$

$ f(\sin^2(x + y) ) = \frac {f(\cos^2(x)) + f(\cos^2(y)) + f(e^{x + y})}{3}$

$ f(x^2 + y^2) = f(|x^4 - xy + y^4|)$

$ f(\sqrt {x + y} ) = \frac {{f(x^2) + f(y^2)}}{2}$

....

Now, if you want a more complex method to show your assertion (instead of the very simple two lines method I gave), you can study the sequence $ (u_n,v_n)$ defined as : $ (u_1,v_1)=(x,y)$ and $ (u_{n+1},v_{n+1})=(\frac{u_n+v_n}2,\frac{2u_nv_n}{u_n+v_n})$
You demonstate then that $ f(u_n)+f(v_n)=f(u_1)+f(v_1)$
You demonstate then that $ \lim_{n\to +\infty}u_n=\lim_{n\to +\infty}v_n=\sqrt{xy}$
You then use the fact that $ f(x)$ is a continuous function and you get your result.
But this is a crazy complex long method to show something which can be shown in a trivial two lines demo.
IMHO
\end{solution}
*******************************************************************************
-------------------------------------------------------------------------------

\begin{problem}[Posted by \href{https://artofproblemsolving.com/community/user/63660}{Victory.US}]
	Let $f: \mathbb R \to \mathbb R$ be a continuous function such that there exist numbers $a$ and $b$ with $ 0<a,b<\frac{1}{2}$ and 
\[ f(f(x))=af(x)+bx, \quad \forall x \in \mathbb R.\]
Show that there exist $ k$ for which $ f(x)=kx$ for all $x \in \mathbb R$.
	\flushright \href{https://artofproblemsolving.com/community/c6h300002}{(Link to AoPS)}
\end{problem}



\begin{solution}[by \href{https://artofproblemsolving.com/community/user/29428}{pco}]
	\begin{tcolorbox}let $ f: R \to R$ be continuous function such that: there exist $ 0 < a,b < \frac {1}{2}$ and 
$ f(f(x)) = af(x) + bx$ ,$ \forall x \in R$

Show that there exist $ k$ what satisfy $ f(x) = kx, \forall k \in R$\end{tcolorbox}

1) $ f(x)$ is bijective
Since $ b\neq 0$, we can write $ x = \frac 1b(f(f(x)) - af(x))$

A first consequence of this equation is that $ f(x)$ is injective, and so monotonous, since continuous.
A second consequence is that $ f(x)$ cant have a finite limit when $ x\to \pm\infty$ (else RHS would be finite and LHS not). So $ f(x)$, since continuous monotonous, is surjective and so bijective.
Q.E.D

2) Some infos on the roots of  $ x^2 - ax - b = 0$
Let then $ r_1 < 0 < r_2$ the two real roots of $ x^2 - ax - b = 0$ (remember $ a > 0$ and $ b > 0$)
$ r_2\geq 1$ would imply $ r_1\leq - \frac 12$ (since $ r_1 + r_2 = a < \frac 12$ and then $ - r_1r_2\geq \frac 12$, which is wrong since $ - r_1r_2 = b < \frac 12$. So $ r2 < 1$

$ r_1\leq - 1$ would imply $ r_2\geq\frac 12$ (since $ r_1 + r_2 = a < \frac 12$ and then $ - r_1r_2\geq \frac 12$, which is wrong since $ - r_1r_2 = b < \frac 12$. So $ r1 > - 1$

Last, since $ r_1 + r_2 = a > 0$, we get that $ |r_2| > |r_1|$
So : $ - 1 < r_1 < 0 < r_2 < 1$ and $ 1 > |r_2| > |r_1|$

3) main demo :
$ f(x)$ bijective implies that $ g(x) = f^{ - 1}(x)$ exists, is defined over $ \mathbb R$ and we have $ x = ag(x) + bg(g(x))$

Let then $ d = f(c)$ for some $ c\in\mathbb R$ and the sequences  $ u_n$ and $ v_n$ defined as :
$ u_0 = c$
$ u_1 = d$
$ u_{n + 2} = au_{n + 1} + bu_n$
$ v_0 = d$
$ v_1 = c$
$ v_{n + 2} = - \frac abv_{n + 1} + \frac 1b$

We clearly, with immediate induction, have $ u_{n + 1} = f(u_n)$ and $ v_{n + 1} = g(v_n)$ or again $ v_n = f(v_{n + 1})$

We also have $ u_n = \frac {(d - r_2c)r_1^n - (d - r_1c)r_2^n}{r_1 - r_2}$ and $ v_n = \frac {(d - r_2c)r_1^{1 - n} - (d - r_1c)r_2^{1 - n}}{r_1 - r_2}$

Since $ 1 > |r_2| > |r_1|$, we get that $ \lim_{n\to + \infty}u_n = 0$ and, since $ u_{n + 1} = f(u_n)$ and $ f(x)$ continuous, $ f(0) = 0$
So, since $ f(x)$ is a continuous monotonous bijection, $ \frac {f(x)}x$ has a constant sign over $ \mathbb R^*$ (positive if $ f(x)$ is increasing, negative if $ f(x)$ is decreasing).

But, since $ 1 > |r_2| > |r_1|$ :
If $ d - r_1c\neq 0$ $ \lim_{n\to + \infty}\frac {u_{n + 1}}{u_n} = r_2 > 0$ 

If $ d - r_2c\neq 0$ $ \lim_{n\to + \infty}\frac {v_{n + 1}}{v_n} = \frac 1{r_1} < 0$

And so, since the signs of ${ \frac {u_{n + 1}}{u_n} = \frac {f(u_n)}{u_n}}$ and $ \frac {v_{n + 1}}{v_n} = \frac {v_{n + 1}}{f(v_n)}$ must be the same, we conclude that :
Either $ d = r_1c$, either $ d = r_2c$

And so $ \forall x$, either $ f(x) = r_1x$, either $ f(x) = r_2x$

And, since $ f(x)$ is continuous and bijective :

Either $ f(x) = r_1x$ $ \forall x$
Either $ f(x) = r_2x$ $ \forall x$

Q.E.D.
\end{solution}



\begin{solution}[by \href{https://artofproblemsolving.com/community/user/66674}{thuyanh158}]
	thank you very much, pco for the last post   :)  :)
\end{solution}
*******************************************************************************
-------------------------------------------------------------------------------

\begin{problem}[Posted by \href{https://artofproblemsolving.com/community/user/31379}{\u0393\u03b9\u03ce\u03c1\u03b3\u03bf\u03c2}]
	If $ f(x)f\left( - \frac {1}{x}\right) = - 1$ for $ x\in \mathbb{R^*}$ and $ f$ is continuous, find $ f(0)$.
	\flushright \href{https://artofproblemsolving.com/community/c6h300187}{(Link to AoPS)}
\end{problem}



\begin{solution}[by \href{https://artofproblemsolving.com/community/user/32886}{dgreenb801}]
	I think $ f(0)$ can be any value. For any other real value $ k$, we can pair it up with $ - \frac {1}{k}$, thus pairing up all the nonzero reals since $ - \frac {1}{k}$ is the inverse of itself. Then we can make $ f(k)$ and $ f( - \frac {1}{k})$ whatever nonzero numbers we want as long as they multiply to $ - 1$.
\end{solution}



\begin{solution}[by \href{https://artofproblemsolving.com/community/user/31379}{\u0393\u03b9\u03ce\u03c1\u03b3\u03bf\u03c2}]
	\begin{tcolorbox}I think $ f(0)$ can be any value. For any other real value $ k$, we can pair it up with $ - \frac {1}{k}$, thus pairing up all the reals since $ - \frac {1}{k}$ is the inverse of itself. Then we can make $ f(k)$ and $ f( - \frac {1}{k})$ whatever we want as long as they multiply to $ - 1$.\end{tcolorbox}

I can't say that i have fully understood your explanation for this problem. Just work with Bolzano's Theorem for a little while to see what's happening...
\end{solution}



\begin{solution}[by \href{https://artofproblemsolving.com/community/user/32886}{dgreenb801}]
	Okay, suppose I have a function like this:

$ f(0) = 75894$

$ f(3) = 7$, $ f( - \frac {1}{3}) = - \frac {1}{7}$
$ f(342) = 5$, $ f( - \frac {1}{342}) = - \frac {1}{5}$
$ f(11) = 976$, $ f( - \frac {1}{11}) = - \frac {1}{976}$
.....
This could go on forever randomly and the function would work.

If you want a concrete example of a function, the following works:
$ f(0) = 3234$, $ f(x) = 1$ if $ x > 0$, $ - 1$ if $ x < 0$.

The general solution is:
$ f(0)=$ any number, $ f(x)=$ any number for $ x>0$, $ f(x)=-\frac{1}{f(-\frac{1}{x})}$ for $ x<0$.
\end{solution}



\begin{solution}[by \href{https://artofproblemsolving.com/community/user/31379}{\u0393\u03b9\u03ce\u03c1\u03b3\u03bf\u03c2}]
	\begin{tcolorbox}Okay, suppose I have a function like this:

$ f(0) = 75894$

$ f(3) = 7$, $ f( - \frac {1}{3}) = - \frac {1}{7}$
$ f(342) = 5$, $ f( - \frac {1}{342}) = - \frac {1}{5}$
$ f(11) = 976$, $ f( - \frac {1}{11}) = - \frac {1}{976}$
.....
This could go on forever randomly and the function would work.

If you want a concrete example of a function, the following works:
$ f(0) = 3234$, $ f(x) = 1$ if $ x > 0$, $ - 1$ if $ x < 0$.

The general solution is:
$ f(0) =$ any number, $ f(x) =$ any number for $ x > 0$, $ f(x) = - \frac {1}{f( - \frac {1}{x})}$ for $ x < 0$.\end{tcolorbox}


Now things are clear. So, you mean that $ f(0)$ isn't dependent from $ f(x)f(-1\/x)=-1$ (1).
But let me show you where i get confused.
(1) tells us that there exists a root $ x_0$ that belongs to the space $ (-1\/x,x)$ such that $ f(x_0)=0$ (2).
If we set $ x=x_0$ to (1) we get $ 0=-1$ thus a contradiction, so it must be $ x_0=0$. So, from (2) we have $ f(0)=0$ right :?:
\end{solution}



\begin{solution}[by \href{https://artofproblemsolving.com/community/user/32886}{dgreenb801}]
	You can only apply the intermediate value theorem (Bolzano's theorem) when the function is continuous. If it is given that the function is continuous then I think your solution is right.
\end{solution}



\begin{solution}[by \href{https://artofproblemsolving.com/community/user/31379}{\u0393\u03b9\u03ce\u03c1\u03b3\u03bf\u03c2}]
	\begin{tcolorbox}You can only apply the intermediate value theorem (Bolzano's theorem) when the function is continuous. If it is given that the function is continuous then I think your solution is right.\end{tcolorbox}

Oh my God!!! I forgot to say that $ f$ is continuous. But, even so, what about your solution? I can't find anything wrong on it
\end{solution}



\begin{solution}[by \href{https://artofproblemsolving.com/community/user/32886}{dgreenb801}]
	If you try to make a continuous function by doing it my way, you will find it will have to end up passing through (0,0) if it is to be continuous from what you said. For example, suppose f(x)=x for all x not equal to 0, then we can't just make f(0) whatever we want or it wouldn't be continuous, it would have to be 0.
The crazy function I made up probably wouldn't have been continuous.
\end{solution}



\begin{solution}[by \href{https://artofproblemsolving.com/community/user/31379}{\u0393\u03b9\u03ce\u03c1\u03b3\u03bf\u03c2}]
	\begin{tcolorbox}If you try to make a continuous function by doing it my way, you will find it will have to end up passing through (0,0) if it is to be continuous from what you said. For example, suppose f(x)=x for all x not equal to 0, then we can't just make f(0) whatever we want or it wouldn't be continuous, it would have to be 0.
The crazy function I made up probably wouldn't have been continuous.\end{tcolorbox}

Sorry for my idiot mistake :mad: . Anyway did you find any other way to solve it? Because that is the reason for posting this problem :roll:
\end{solution}



\begin{solution}[by \href{https://artofproblemsolving.com/community/user/29428}{pco}]
	\begin{tcolorbox}It's not hard but i'm looking for other ways to prove it rather than this of my teacher(he made the problem). 

If $ f(x)f\left( - \frac {1}{x}\right) = - 1$ with $ x\in \mathbb{R^*}$ and $ f$ is continuous, find $ f(0)$.\end{tcolorbox}
$ f(x)=x$ is a solution and so $ f(0)=0$ is possible.

Suppose now $ f(0)=a\neq 0$. Then $ \lim_{x\to +\infty}f(x)=-\frac 1a$ is of opposite sign than $ f(0)$ and so, since $ f(x)$ is continuous, $ \exists u>0$ such that $ f(u)=0$, but this is impossible since $ f(u)f(-\frac 1u)=-1$

So $ f(0)=0$
\end{solution}
*******************************************************************************
-------------------------------------------------------------------------------

\begin{problem}[Posted by \href{https://artofproblemsolving.com/community/user/50028}{hophinhan}]
	Find all functions $ f : \mathbb R^* \to \mathbb R^*$ such that \[ f(x^2 + y) = f^2(x) + \frac {f(xy)}{f(x)}\] holds for all $x, y \in R^*$ with $y \neq -x^2$.
	\flushright \href{https://artofproblemsolving.com/community/c6h300486}{(Link to AoPS)}
\end{problem}



\begin{solution}[by \href{https://artofproblemsolving.com/community/user/43536}{nguyenvuthanhha}]
	\begin{tcolorbox}Let $ \mathbb{R^{*}}$ be the set of non-zero real numbers. Find all functions $ f : R* - > R*$ such that : $ f(x^2 + y) = f^2(x) + \frac {f(xy)}{f(x)} (1)$
for all $ x, y \in \mathbb{R^{*}}, y \neq - x^{2}$\end{tcolorbox}

 \begin{italicized}  I think i have a solution :  

Change $ y = 1$ into $ (1)$ , we have : $ f(x^2 + 1 ) = (f(x))^2 + 1 (2)$

 Change $ x = 1$ into $ (1)$ : $ f(y + 1) = \frac {f(y)}{f(1)} + (f(1))^2 = \frac {f(y)}{a} + a^2 \ \forall \ y > 0 \ \ ( f(1) = a)$

$ \longrightarrow f(2) = a^2 + 1 ; f(3) = \frac {f(2)}{a} + a^2 = \frac {a^3 + a^2 + 1}{a}$

  Continue that work , we can calculate $ f(5)$ as a function of $ a$

   And , we can calculate $ f(5)$ in the other way by using  $ (2)$ ( $ f(5) = (f(2))^2 + 1$ )

  Thus , we will have a equation of $ a$ , this equation seems to be hard but actually very easy 
     
 $ \iff (a - 1) \left( a^4 (a^2 + a + 1) + (a + 1)^2 (a^2 - a + 1) + 2a^2 \right) = 0$  

The unique root of it is $ a = 1$

   So $ f(y + 1) = f(y) + 1 \ \forall \ y > 0$

  $ \longrightarrow f(y + n) = f(y) + n \forall \ n \in \ \mathbb{N} , \forall \ y > 0 (3)$

  Put $ y = \frac {1}{n} ; x = n \in \ \mathbb{N}$ into $ (1)$ , we have : $ f(1\/n) = 1\/n$

Put $ y = \frac {1}{n} ; x = m \in \ \mathbb{N}$ into $ (1) \longrightarrow f( m\/n) = m\/n$

    So $ f(x) = x \forall x \ \in \ \mathbb{Q^{ + }}$

   By $ (2) ; (3) : f(x^2) = (f(x))^2 \forall \ x > 0 \longrightarrow f(x) > 0 \forall \ x > 0$

   $ f(x^2 + y) > (f(x))^2 \forall \ x;y > 0$ , change $ x$ to $ \sqrt {x}$  we can prove $ f(x) < f(x + y) \forall \ x;y > 0$
 
$ \longrightarrow f$ is stricly increasing in $ (0; + \infty)$ ; $ f(x) = x \forall x \ \in \ \mathbb{Q^{ + }}$
          
For each  fixed positive real number $ x$ .Choose $ 2$ sequences of rational numbers $ (u_n) ; (v_n)$ such that :
               $ \lim u_n = \lim v_n \ = \ x ; u_n \ \leq x \ \leq \ v_n \forall \ n \in \ \mathbb{N}$
 
$ \longrightarrow f(u_n) \ \leq \ f(x) \ \leq f(v_n) \forall \ n \in \ \mathbb{N} \longrightarrow u_n \ \leq \ f(x) \ \leq v_n \forall \ n \in \ \mathbb{N}$

Setting $ n \to \infty$ , we can find : $ f(x) = x \ \forall \ x > 0$
         With $ y < 0$ , choose a positive real number $ x$ such that : $ x^2 + y > 0$
From $ (1)$ , we have $ f(xy) = xy$ . Note that , $ xy$ can take all value in $ ( - \infty ; 0)$
so   $ f(x) = x \ \forall \ \ x < 0$   And $ f \ \equiv \ x \ \forall \ x \ \in \ \mathbb{R^{*}}$ is the unique solution   


     \end{italicized}
\end{solution}



\begin{solution}[by \href{https://artofproblemsolving.com/community/user/29428}{pco}]
	\begin{tcolorbox}Let $ R*$ be the set of non-zero real numbers. Find all functions $ f : R* - > R*$ such that : $ f(x^2 + y) = f^2(x) + \frac {f(xy)}{f(x)}$
for all x, y ∈ R*, y ≠ −x^2\end{tcolorbox}
Same demo as nguyenvuthanhha's one (congrats!  :) ), a litlle bit quicker in the beginning :

Let $ P(x,y)$ the assertion $ f(x^2+y)=f^2(x)+\frac{f(xy)}{f(x)}$
Let $ a=f(1)$

$ P(x,y)$ $ \implies$ $ f(x^2+1)=f^2(x)+1$ and so $ f(x)=f(-x)$. As a consequence, $ f(-1)=\epsilon a$ where $ \epsilon\in\{-1,+1\}$

Comparing then $ P(1,x)$ and $ P(-1,x)$, we get $ f(-x)=\epsilon f(x)$ $ \forall x\neq 0$
$ P(1,1)$ $ \implies$ $ f(2)=a^2+1$
$ P(1,-2)$ $ \implies$ $ f(-2)=\epsilon a^2-a^3$

And since $ f(-2)=\epsilon f(2)$, we get $ a=-\epsilon$ and so $ f(-1)=-1$

Then $ P(1,x^2)$ $ \implies$ $ f(x^2+1)=1-\epsilon f(x^2)$ and $ P(x,1)$ $ \implies$ $ f(x^2+1)=f^2(x)+1$ and so $ f(x^2)=-\epsilon f^2(x)$

$ P(1,1)$ $ \implies$ $ f(2)=2$
$ P(1,2)$ $ \implies$ $ f(3)=1-2\epsilon$
$ P(1,3)$ $ \implies$ $ f(4)=3-\epsilon$ But $ f(4)=f(2^2)=-\epsilon f^2(2)=-4\epsilon$ and so $ \epsilon=-1$

So $ f(1)=1$ and $ f(-x)=-f(x)$ and $ f(x+1)=f(x)+1$ and $ f(x^2)=f^2(x)$ and so $ f(x)=x$ $ \forall x\in\mathbb Z^*$
$ P(n,x)$ $ \implies$ $ f(nx)=nf(x)$ $ \forall n\in\mathbb Z^*,x\in\mathbb R^*$ and so $ f(x)=x$ $ \forall x\in\mathbb Q$

From $ f(x^2)=f^2(x)$ and $ f(-x)=-f(x)$, we get that $ f(x)>0$ $ \forall x>0$ and $ f(x)<0$ $ \forall x<0$
Then $ P(x,y)$ with $ y>0$ shows that $ f(x^2+y)>f(x^2)$ and so $ f(x)$ is increasing for $ x>0$, so $ f(x)=x$ $ \forall x>0$
And, since $ f(-x)=-f(x)$ : $ \boxed{f(x)=x\text{   }\forall x\in\mathbb R^*}$
\end{solution}



\begin{solution}[by \href{https://artofproblemsolving.com/community/user/66394}{reason}]
	hi!
see here: [url]http://www.mathlinks.ro/viewtopic.php?p=1599126#1599126[\/url]
\end{solution}



\begin{solution}[by \href{https://artofproblemsolving.com/community/user/46039}{ll931110}]
	\begin{tcolorbox}[quote="hophinhan"]Let $ R*$ be the set of non-zero real numbers. Find all functions $ f : R* - > R*$ such that : $ f(x^2 + y) = f^2(x) + \frac {f(xy)}{f(x)}$
for all x, y ∈ R*, y ≠ −x^2\end{tcolorbox}
Same demo as nguyenvuthanhha's one (congrats!  :) ), a litlle bit quicker in the beginning :

Let $ P(x,y)$ the assertion $ f(x^2 + y) = f^2(x) + \frac {f(xy)}{f(x)}$
Let $ a = f(1)$

$ P(x,y)$ $ \implies$ $ f(x^2 + 1) = f^2(x) + 1$ and so $ f(x) = f( - x)$. As a consequence, $ f( - 1) = \epsilon a$ where $ \epsilon\in\{ - 1, + 1\}$

Comparing then $ P(1,x)$ and $ P( - 1,x)$, we get $ f( - x) = \epsilon f(x)$ $ \forall x\neq 0$
$ P(1,1)$ $ \implies$ $ f(2) = a^2 + 1$
$ P(1, - 2)$ $ \implies$ $ f( - 2) = \epsilon a^2 - a^3$

And since $ f( - 2) = \epsilon f(2)$, we get $ a = - \epsilon$ and so $ f( - 1) = - 1$

Then $ P(1,x^2)$ $ \implies$ $ f(x^2 + 1) = 1 - \epsilon f(x^2)$ and $ P(x,1)$ $ \implies$ $ f(x^2 + 1) = f^2(x) + 1$ and so $ f(x^2) = - \epsilon f^2(x)$

$ P(1,1)$ $ \implies$ $ f(2) = 2$
$ P(1,2)$ $ \implies$ $ f(3) = 1 - 2\epsilon$
$ P(1,3)$ $ \implies$ $ f(4) = 3 - \epsilon$ But $ f(4) = f(2^2) = - \epsilon f^2(2) = - 4\epsilon$ and so $ \epsilon = - 1$

So $ f(1) = 1$ and $ f( - x) = - f(x)$ and $ f(x + 1) = f(x) + 1$ and $ f(x^2) = f^2(x)$ and so $ f(x) = x$ $ \forall x\in\mathbb Z^*$
$ P(n,x)$ $ \implies$ $ f(nx) = nf(x)$ $ \forall n\in\mathbb Z^*,x\in\mathbb R^*$ and so $ f(x) = x$ $ \forall x\in\mathbb Q$

From $ f(x^2) = f^2(x)$ and $ f( - x) = - f(x)$, we get that $ f(x) > 0$ $ \forall x > 0$ and $ f(x) < 0$ $ \forall x < 0$
Then $ P(x,y)$ with $ y > 0$ shows that $ f(x^2 + y) > f(x^2)$ and so $ f(x)$ is increasing for $ x > 0$, so $ f(x) = x$ $ \forall x > 0$
And, since $ f( - x) = - f(x)$ : $ \boxed{f(x) = x\text{ }\forall x\in\mathbb R^*}$\end{tcolorbox}

I think you forgot the condition $ f : R+ \rightarrow R+$  :)
\end{solution}



\begin{solution}[by \href{https://artofproblemsolving.com/community/user/29428}{pco}]
	\begin{tcolorbox} I think you forgot the condition $ f : R + \rightarrow R +$  :)\end{tcolorbox}

I think you misread the problem : $ f : \mathbb R^*\to\mathbb R^*$ and not $ \mathbb R + \to\mathbb R +$ :)
\end{solution}



\begin{solution}[by \href{https://artofproblemsolving.com/community/user/46039}{ll931110}]
	\begin{tcolorbox}Same demo as nguyenvuthanhha's one (congrats!  :) ), a litlle bit quicker in the beginning :

Let $ P(x,y)$ the assertion $ f(x^2 + y) = f^2(x) + \frac {f(xy)}{f(x)}$
Let $ a = f(1)$

$ P(x,y)$ $ \implies$ $ f(x^2 + 1) = f^2(x) + 1$ and so $ f(x) = f( - x)$. As a consequence, $ f( - 1) = \epsilon a$ where $ \epsilon\in\{ - 1, + 1\}$

Comparing then $ P(1,x)$ and $ P( - 1,x)$, we get $ f( - x) = \epsilon f(x)$ $ \forall x\neq 0$
\end{tcolorbox}

pco, I don't understand why you can put $ P(-1,x)$ while $ f: R* \rightarrow R*$, and from the problem description, $ R*$ is the set of non-negative real numbers  :?:
\end{solution}



\begin{solution}[by \href{https://artofproblemsolving.com/community/user/29428}{pco}]
	\begin{tcolorbox} pco, I don't understand why you can put $ P( - 1,x)$ while $ f: R* \rightarrow R*$, and from the problem description, $ R*$ is the set of non-negative real numbers  :?:\end{tcolorbox}

I'm sorry, but from the problem description, $ R*$ is the set of non-\begin{bolded}zero\end{bolded}\end{underlined}, (and not non-negative) real numbers  (and it is a rather classical notation)
\end{solution}



\begin{solution}[by \href{https://artofproblemsolving.com/community/user/68025}{Pirkuliyev Rovsen}]
	This example team selection test for 46.IMO(2005) has been given 
\end{solution}
*******************************************************************************
-------------------------------------------------------------------------------

\begin{problem}[Posted by \href{https://artofproblemsolving.com/community/user/43536}{nguyenvuthanhha}]
	Find all functions $ f : \mathbb{R} \ \to \ \mathbb{R}$ such that $0< f(0)$ and \[ f(x+y) \geq f(x) + yf(f(x)) , \quad \forall x, y \in \mathbb{R}.\]
	\flushright \href{https://artofproblemsolving.com/community/c6h300639}{(Link to AoPS)}
\end{problem}



\begin{solution}[by \href{https://artofproblemsolving.com/community/user/29428}{pco}]
	\begin{tcolorbox}\begin{italicized}Find all function $ f : \mathbb{R} \ \to \ \mathbb{R}$  such that  :

   $ 0 < f(0)$ and $ f(x + y) \ \geq \ f(x) + yf(f(x)) \ \ \forall x; y \ \in \ \mathbb{R}$\end{italicized}\end{tcolorbox}
[hide="Intermediate result"]

Let $ P(x,y)$ be the assertion $ f(x+y)\geq f(x)+yf(f(x))$

$ P(x+y,-y)$ $ \implies$ $ f(x)\geq f(x+y)-yf(f(x+y))$ and so $ f(x+y)-yf(f(x))\geq f(x)\geq f(x+y)-yf(f(x+y))$

And so $ yf(f(x+y))\geq f(x+y)-f(x)\geq yf(f(x))$

A first conclusion is that $ yf(f(x+y))\geq yf(f(x))$ and so that $ f(f(x))$ is a non decreasing function.

A second conclusion is that $ f(x)$ is a continuous function (just set $ y\to 0$ and remember that $ f(f(x))$ is non decreasing)

A third conclusion is then that, for $ y>0$ : $ f(f(x+y))\geq \frac{f(x+y)-f(x)}y\geq f(f(x))$ and so $ f'(x)$ exists and $ f'(x)=f(f(x))$. And since $ f(f(x))$ is non decreasing, it means that $ f(x)$ is a convex function.

So $ f(x)$ solution $ \implies$ $ f(x)$ is a convex function such that $ f(0)>0$ and $ f'(x)=f(f(x))$

And it is easy to see that these necessary conditions are enough and so :

$ f(x)$ solution $ \iff$ $ f(x)$ is a convex function such that $ f(0)>0$ and $ f'(x)=f(f(x))$
[\/hide]
\end{solution}



\begin{solution}[by \href{https://artofproblemsolving.com/community/user/29428}{pco}]
	\begin{tcolorbox}\begin{italicized}Find all function $ f : \mathbb{R} \ \to \ \mathbb{R}$  such that  :

   $ 0 < f(0)$ and $ f(x + y) \ \geq \ f(x) + yf(f(x)) \ \ \forall x; y \ \in \ \mathbb{R}$\end{italicized}\end{tcolorbox}
There are no solution to this equation
Let $ P(x,y)$ be the assertion $ f(x+y)\geq f(x)+yf(f(x))$

1) \begin{bolded}$ f(x)$ is $ C_1$, convex and such that $ f'(x)=f(f(x))$\end{underlined}\end{bolded}

$ P(x+y,-y)$ $ \implies$ $ f(x)\geq f(x+y)-yf(f(x+y))$ and so $ f(x+y)-yf(f(x))\geq f(x)\geq f(x+y)-yf(f(x+y))$

And so $ yf(f(x+y))\geq f(x+y)-f(x)\geq yf(f(x))$

A first conclusion is that $ yf(f(x+y))\geq yf(f(x))$ and so that $ f(f(x))$ is a non decreasing function.

A second conclusion is that $ f(x)$ is a continuous function (just set $ y\to 0$ and remember that $ f(f(x))$ is monotonous)

A third conclusion is then that, for $ y>0$ : $ f(f(x+y))\geq \frac{f(x+y)-f(x)}y\geq f(f(x))$ and so $ f'(x)$ exists and $ f'(x)=f(f(x))$. And since $ f(f(x))$ is non decreasing, it means that $ f(x)$ is a convex function.

So $ f(x)$ solution $ \implies$ $ f(x)$ is a convex function such that $ f(0)>0$ and $ f'(x)=f(f(x))$

Q.E.D

2)\begin{bolded} $ \forall x>0$ : $ f(x)>0$ and $ f(f(x))>0$\end{underlined}\end{bolded}
If $ f(u)<0$ for some $ u>0$, and since $ f(0)>0$ and $ f(x)$ continuous, $ \exists v>0$ such that $ f(v)=0$
Let then $ w=\inf\{v>0$ such that $ f(v)=0\}$. Continuity implies that $ f(w)=0$ and (definition of $ w$) $ f(x)>0$ $ \forall x\in[0,w)$
But then $ f'(w)=f(f(w))=f(0)>0$, which is in contradiction with $ f(x)>0$ $ \forall x\in[0,w)$.
So $ f(x)>0$ $ \forall x\geq 0$
Then $ f'(0)=f(f(0))>0$ since $ f(0)>0$ and, since $ f(x)$ is convex, $ f(f(x))=f'(x)\geq f'(0)>0$ $ \forall x>0$
Q.E.D.

3) \begin{bolded}The contradiction\end{underlined}\end{bolded}
Let $ x>0$ : $ P(x,f(x)-x)$ $ \implies$ $ f(f(x))\geq f(x)+(f(x)-x)f(f(x))$ and, since $ f(f(x))>0$ : $ f(x)-x\leq 1-\frac{f(x)}{f(f(x))}$
And so, since $ f(x)>0$ and $ f(f(x))>0$ : $ f(x)<x+1$ $ \forall x>0$
And, since $ f(x)$ is convex, this implies $ f'(x)\leq 1$ $ \forall x$ and so $ f(f(x))\leq 1$ $ \forall x$
But $ f(x)$ convex and $ f'(0)>0$ imply $ \lim_{x\to +\infty}f(x)=+\infty$
So contradiction.

Hence the conclusion : no such function exists.
\end{solution}



\begin{solution}[by \href{https://artofproblemsolving.com/community/user/43536}{nguyenvuthanhha}]
	\begin{italicized}Oh , Pco , you are really patient . I don't have time to read your solution But it seems to take many time  

   Here is the solution :

  Assume $ f$ satisfies all conditions of the problem

    If $ f(f(x)) \ \leq \ 0 \ \forall \ x \in \ \mathbb{R} \ \ (1)$ then with all real number $ y \ \leq \ 0$ , we have : $ yf(f(x))   \ \geq \ 0\ \forall \ x \in \ \mathbb{R}$

  $ \rightarrow f(x+y) \ \geq \ f(x) + yf(f(x))   \ \geq \ f(x)  \ \forall \ x \in \ \mathbb{R}$

  $ f$ is non-increasing function .
  $ \rightarrow f(0) > 0 \ \geq \  f(f(x)) \ \forall \ x \in \ \mathbb{R}  \rightarrow f(x) >0 \ \forall \ x \in \ \mathbb{R}$ , contradict with $ (1)$

    $ \rightarrow$ there is $ z \ \in \  \mathbb{R} : f(f(z)) >0$

  Due to the fact that : $ f(x+z) \ \geq \ f(z) + xf(f(z))   \ \forall \ x \in \ \mathbb{R}$

   We can easily prove that $ \lim_{x \ \to \ + \infty} f(x) = \ \ +  \infty  \rightarrow \lim_{x \ \to \ + \infty} f(f(x)) = \ \ + \infty$

  So we can choose positive real numbers $ x ; y$ such that 

$ f(x) \ \geq \ 0 ;  f(f(x)) > 1 ; y \ \geq \ \frac{x+1}{f(f(x)) -1 } ; f(f(x+y+1)) \ \geq \ 0$

 We have : $ f(x+y) \ \geq \ f(x) + yf(f(x))   \ \geq \   yf(f(x)) = y(f(f(x)) -1) + y \ \geq \ x+y+1 \ \ (2)$
   Using $ (2)$ :
  $ f(f(x+y)) = f(  (x+y+1) + (f(x+y) - ( x+y+1) ) )$

$ \ \geq \   f(x+y+1) +   (f(x+y) - ( x+y+1) ) f(f(x+y+1)) \ \geq f(x+y+1)$

$ \geq \ \ f(x+y) + f(f(x+y))  \geq \ \ f(x) + yf(f(x) + f(f(x+y)) > f(f(x+y))$ , contradiction

   And our initial assumation is wrong .So , there is no function satisfies the condition   \end{italicized}
\end{solution}
*******************************************************************************
-------------------------------------------------------------------------------

\begin{problem}[Posted by \href{https://artofproblemsolving.com/community/user/66674}{thuyanh158}]
	Determine all continuous functions $f: \mathbb R \to \mathbb R$ such that
\[ f(x^2f(x) + f(y)) = (f(x))^3 + y\]
holds for all $x,y \in \mathbb R$.
	\flushright \href{https://artofproblemsolving.com/community/c6h300783}{(Link to AoPS)}
\end{problem}



\begin{solution}[by \href{https://artofproblemsolving.com/community/user/33334}{Brut3Forc3}]
	Plugging in $ x=0$, we have $ f(f(y))=f(0)^3+y$, so $ f$ is bijective. Let $ r$ be the real number such that $ f(r)=0$. $ x=y=r$ gives $ f(f(r))=r$. But we also have $ f(f(r))=f(0)^3+r$, so $ f(0)=0$. Thus, we have $ f(f(y))=y$.
Now let $ x=1$, giving $ f(f(y)+1)=f(1)^3+y$. Plugging in $ y=0$ gives $ f(1)=f(1)^3$, or $ f(1)=0,1$. We cannot have $ f(1)=0$, so $ f(1)=1$.
Since $ f$ is continuous and injective, it must be monotonous, and since $ f(0)=0, f(1)=1$, $ f$ must be monotonically increasing.
Suppose there were numbers $ a\neq b$, with $ f(a)=b,f(b)=a$. (WLOG, $ a<b$.) Then we would have $ f(a)>f(b)$, contradiction.
Thus, we must have $ f(x)=x$ for all $ x$.
\end{solution}



\begin{solution}[by \href{https://artofproblemsolving.com/community/user/29428}{pco}]
	\begin{tcolorbox}Plugging in $ x = 0$, we have $ f(f(y)) = f(0)^3 + y$, so $ f$ is bijective. Let $ r$ be the real number such that $ f(r) = 0$. $ x = y = r$ gives $ f(f(r)) = r$. But we also have $ f(f(r)) = f(0)^3 + r$, so $ f(0) = 0$. Thus, we have $ f(f(y)) = y$.
Now let $ x = 1$, giving $ f(f(y) + 1) = f(1)^3 + y$. Plugging in $ y = 0$ gives $ f(1) = f(1)^3$, or $ f(1) = 0,1$. We cannot have $ f(1) = 0$, so $ f(1) = 1$.
Since $ f$ is continuous and injective, it must be monotonous, and since $ f(0) = 0, f(1) = 1$, $ f$ must be monotonically increasing.
Suppose there were numbers $ a\neq b$, with $ f(a) = b,f(b) = a$. (WLOG, $ a < b$.) Then we would have $ f(a) > f(b)$, contradiction.
Thus, we must have $ f(x) = x$ for all $ x$.\end{tcolorbox}

Nice demo, with a little mistake on the third line : Setting $ x=1$ gives $ f(f(1)+f(y))=f(1)^3+y$ and not $ f(f(y) + 1) = f(1)^3 + y$. 

But this mistake as no consequence : setting then $ y=0$ as you suggested gives $ f(f(1))=f(1)^3$ (and not $ f(1) = f(1)^3$) and since $ f(f(1))=1$ you get immediately the same conclusion $ f(1)=1$

Congrats.
\end{solution}



\begin{solution}[by \href{https://artofproblemsolving.com/community/user/48364}{cnyd}]
	$ f(x^{2}.f(x) + f(y)) = (f(x))^{3} + y$.
if  $ f(y_{1}) = f(y_{2})$  $ \implies$  $ y_{1} = y_{2}$  $ \implies$ $ f$  is injective.
$ y = k - (f(x))^{3}$ $ \implies$  $ f$ is surjective.
$ f(u) = 0$  $ \implies$  $ f(f(y)) = y$,
$ f((f(y)) = f(0)^{3} + y = y$  $ \implies$  $ f(0) = 0$ $ \implies$  $ f(x^{2}.f(x)) = f(x)^{3}$
$ f(f(1)) = 1$ and $ f(f(1)) = f(1)^{3}$  $ \implies$  $ f(1) = 1$.
$ f$  is continous and ,$ f$ is injective $ \implies$  $ f$ is increasing function.$ (f(1)>f(0))$
$ f(f(x)) = x$  $ \implies$  $ f(x) = a$  $ \implies$ $ f(a) = x$
if $ a > x$ $ \implies$  $ f(x) > f(a)$ contradiction!
if $ a < x$  $ \implies$  $ f(x) < f(a)$  contradiction!
$ \implies$  only $ a = x$  $ \implies$  $ f(x) = x$ $ \Box$
\end{solution}
*******************************************************************************
-------------------------------------------------------------------------------

\begin{problem}[Posted by \href{https://artofproblemsolving.com/community/user/63946}{rajelaz}]
	Find all functions $ f:\mathbb{R} \rightarrow \mathbb{R}$ that satisfy \[ f(f(x)+y) = f(x^2 - y) + 4f(x)y\] for all $ x,y \in \mathbb{R}$.
	\flushright \href{https://artofproblemsolving.com/community/c6h301125}{(Link to AoPS)}
\end{problem}



\begin{solution}[by \href{https://artofproblemsolving.com/community/user/29428}{pco}]
	\begin{tcolorbox}1. Find all function $ f(x) \mathbb{R} \rightarrow \mathbb{R}$ that satisfy  $ f(f(x) + y) = f(x^2 - y) + 4f(x)y$  for $ \forall x,y \in \mathbb{R}$  \end{tcolorbox}
Set $ y=\frac{x^2-f(x)}2$ and we get $ f(x)(f(x)-x^2)=0$ and so : $ \forall x\in\mathbb R$, either $ f(x)=0$, either $ f(x)=x^2$

Suppose now $ \exists u\neq 0$ such that $ f(u)=0$ and $ v\neq 0$ such that $ f(v)=v^2$
Using $ x=u$ and $ y=v$, we get $ v^2=f(u^2-v)$ and so $ f(u^2-v)=(u^2-v)^2$ and so $ v=\frac{u^2}2$ and so contradiction : just choose another $ x\notin\{0,-u,u,v,\frac {u^2}2\}$ and either $ f(x)=0$ and we get $ v=\frac{x^2}2=\frac {u^2}2$, either $ f(x)\neq 0$ and we get $ x=\frac{u^2}2$

So two possibilities :
Either $ f(x)=0$ $ \forall x$
Either $ f(x)=x^2$ $ \forall x$

And it is easy to check that there two functions indeed are solutions.
\end{solution}
*******************************************************************************
-------------------------------------------------------------------------------

\begin{problem}[Posted by \href{https://artofproblemsolving.com/community/user/43536}{nguyenvuthanhha}]
	Find all function $ f : \mathbb{R^{*}} \to \mathbb{R}$ such that
\[g(xy) \ = \ \sqrt[2009]{y} g(x) +\sqrt[2009]{x} g(y), \quad \forall  x,y  \in    \mathbb{R^{*}},\]
and $ g(x)$ is continuous at $ x = - 2008$.
	\flushright \href{https://artofproblemsolving.com/community/c6h301286}{(Link to AoPS)}
\end{problem}



\begin{solution}[by \href{https://artofproblemsolving.com/community/user/29428}{pco}]
	\begin{tcolorbox}\begin{italicized}Find all function $ f : \mathbb{R^{*}} \to \mathbb{R}$ such that :

    $ g(xy) \ = \ \sqrt [2009]{y} g(x) + \sqrt [2009]{x} g(y) \ \forall \ x;y \ \in \ \mathbb{R^{*}}$

  and $ g(x)$ is continuous at $ x = - 2008$\end{italicized}\end{tcolorbox}

Let $ h(x)=g(x)x^{-\frac 1{2009}}$. The equation becomes $ h(xy)=h(x)+h(y)$ with $ h(x)$ continuous at $ x = - 2008$, so continuous all over $ \mathbb R^*$

And so $ h(x)=a\cdot\ln(|x|)$

and so $ g(x)=a\sqrt [2009]{x}\ln(|x|)$
\end{solution}
*******************************************************************************
-------------------------------------------------------------------------------

\begin{problem}[Posted by \href{https://artofproblemsolving.com/community/user/64682}{KDS}]
	1. Determine all functions $f: \mathbb R \to \mathbb R$ such that \[ f(x+yf(x))=f(x)+xf(y)\] for all reals $x$ and $y$.
2. Determine all functions $ f: \mathbb R^* \to \mathbb R^*$ which satisfy \[ f(x^2+y)=f(f(x))+\frac{f(xy)}{f(x)}\] for $x$ and $y$ such that $x,y,x^2+y \in \mathbb R^*$.
	\flushright \href{https://artofproblemsolving.com/community/c6h301766}{(Link to AoPS)}
\end{problem}



\begin{solution}[by \href{https://artofproblemsolving.com/community/user/29428}{pco}]
	\begin{tcolorbox}\begin{bolded}Problem 1\end{bolded}:Determine all functions $ f: R \to R$ such that $ f(x + yf(x)) = f(x) + xf(y)$\end{tcolorbox}

Let $ P(x,y)$ be the assertion $ f(x + yf(x)) = f(x) + xf(y)$

$ P(1,0)$ $ \implies$ $ f(0) = 0$

If $ f(1)\neq 1$, then $ P(1,\frac 1{1 - f(1)})$ $ \implies$ $ f(\frac 1{1 - f(1)}) = f(1) + f(\frac 1{1 - f(1)})$ and so $ f(1) = 0$. Then $ P(1,x)$ $ \implies$ $ f(x) = 0$ $ \forall x$ and we got one solution.

Consider now $ f(1) = 1$

$ P(1,x)$ $ \implies$ $ f(x + 1) = f(x) + 1$
Suppose now $ f(a) = 0$. Then  $ P(a,1)$ $ \implies$ $ a = 0$ and so $ f(x) = 0\iff x = 0$

Let then $ y\neq 0$ : $ f(y)\neq 0$ and 

$ P(y,\frac {x - y}{f(y)})$ $ \implies$ $ f(x) = f(y) + yf(\frac {x - y}{f(y)})$

$ P(y,\frac {x - y}{f(y)} + 1)$ $ \implies$ $ f(x + f(y)) = f(y) + yf(\frac {x - u}{f(u)} + 1)$ $ = f(y) + yf(\frac {x - u}{f(u)}) + y$

Comparing these two lines, we get $ f(x + f(y)) = f(x) + y$ $ \forall x,\forall y\neq 0$ and, since this equality is true also for $ y = 0$ :

New assertion $ Q(x,y)$ : $ f(x + f(y)) = f(x) + y$ $ \forall x,y$

$ Q(0,x)$ $ \implies$ $ f(f(x)) = x$ and so $ f(x)$ is a bijection.
Then $ Q(x,f(y))$ $ \implies$ $ f(x + y) = f(x) + f(y)$
Then $ P(x,y)$ becomes $ f(x) + f(yf(x)) = f(x) + xf(y)$ and so $ f(yf(x)) = xf(y)$ and so $ f(xy) = f(x)f(y)$

So we have the classical equation :
$ f(1) = 1$
$ f(x + y) = f(x) + f(y)$
$ f(xy) = f(x)f(y)$

Whose unique solution is $ f(x) = x$
[hide="See quick demo"]
$ f(x + y) = f(x) + f(y)$ and $ f(1) = 1$ imply $ f(x) = x$ $ \forall x\in\mathbb Q$
$ f(xy) = f(x)f(y)$ implies $ f(x^2)\geq 0$ and so $ f(x)\geq 0$ $ \forall x > 0$ and so, using $ f(x + y) = f(x) + f(y)$, $ f(x)$ is monotonous.
$ f(x) = x$ $ \forall x\in\mathbb Q$ and $ f(x)$  monotonous imply $ f(x) = x$ $ \forall x\in\mathbb R$
[\/hide]

And so the two solutions for this equation :
$ f(x) = 0$ $ \forall x$
$ f(x) = x$ $ \forall x$
\end{solution}



\begin{solution}[by \href{https://artofproblemsolving.com/community/user/29428}{pco}]
	\begin{tcolorbox}\begin{bolded}Problem 2\end{bolded}:Determine all functions $ f: R^* \to R^*$ such that $ f(x^2 + y) = f(f(x)) + \frac {f(xy)}{f(x)}$   $ \forall x,y,x^2 + y \in R^*$

$ R^*$ is the set of nonzero real numbers.\end{tcolorbox}

Let $ P(x,y)$ be the assertion $ f(x^2+y)=f(f(x))+\frac{f(xy)}{f(x)}$

Suppose $ \exists a\in\mathbb R^*$ such that $ f(a)\neq a^2$. Then :

$ P(a,f(a)-a^2)$ $ \implies$ $ f(f(a))=f(f(a))+\frac{f(a(f(a)-a^2))}{f(a)}$ and so $ f(a(f(a)-a^2))=0$ which is impossible.

So $ f(x)=x^2$ $ \forall x\in\mathbb R^*$

But then $ f(x^2+y)=x^4+y^2+2x^2y$ while $ f(f(x))+\frac{f(xy)}{f(x)}=x^4+y^2$ and so $ P(x,y)$ is wrong.

So no solution.
\end{solution}



\begin{solution}[by \href{https://artofproblemsolving.com/community/user/90621}{Love_Math1994}]
	Dear pco,i think KDS give us a fail question.It must be:
$ f(x^{2}+y)=f^2(x)+\frac{f(xy)}{f(x)} $
It a problem from Bulgaria
I have a similar problem from my school test selection team to VMO 2009
#2 $ f(x^{3}+y)=f^3(x)+\frac{f(xy)}{f(x)} $
The difficult is finding  $f(1)$...After have $f(1)=1$ we have new problems:
$f^n(x)=f(x^n)$ and 
$f(x+n)=f(x)+n$
$\to$ become trivial
[hide]Find the fomula for $f$ in $\mathbb Q$ and prove that $f$ is inscreasing so take the limit to get fomula of $f$  over $\mathbb R$[\/hide]
The step find $f(1)$ not very difficult,we can solve with a equation degree 9 for #2 of $f(1)$ and the only root is $\boxed1$  Try it.
[hide]Brute force :P [\/hide]
\end{solution}
*******************************************************************************
-------------------------------------------------------------------------------

\begin{problem}[Posted by \href{https://artofproblemsolving.com/community/user/63946}{rajelaz}]
	Find all functions  $ f : \mathbb{Z} \rightarrow \mathbb{Z}$ such that \[ f(x+y) + f(xy) = f(x) - f(y) +1,\] for all $x,y \in \mathbb{Z}$.
	\flushright \href{https://artofproblemsolving.com/community/c6h301788}{(Link to AoPS)}
\end{problem}



\begin{solution}[by \href{https://artofproblemsolving.com/community/user/54134}{Lousin Garckz}]
	there have to be some mistake here... I mean, plug in it $ x = y = 0$ and you get $ f(0+0)+f(0.0) = f(0)-f(0)+1 = 1$, that is $ 2f(0) = 1$ and $ f(0) = \dfrac{1}{2}$ which is not in $ \mathbb{Z}$, contradiction!.
\end{solution}



\begin{solution}[by \href{https://artofproblemsolving.com/community/user/63946}{rajelaz}]
	\begin{tcolorbox}there have to be some mistake here... I mean, plug in it $ x = y = 0$ and you get $ f(0 + 0) + f(0.0) = f(0) - f(0) + 1 = 1$, that is $ 2f(0) = 1$ and $ f(0) = \dfrac{1}{2}$ which is not in $ \mathbb{Z}$, contradiction!.\end{tcolorbox}

Sorry, this the correct one 
$ f(x+y) + f(xy) = f(x)f(y) +1$, $ f : \mathbb{Z} \rightarrow \mathbb{Z}$
\end{solution}



\begin{solution}[by \href{https://artofproblemsolving.com/community/user/29428}{pco}]
	\begin{tcolorbox} Sorry, this the correct one 
$ f(x + y) + f(xy) = f(x)f(y) + 1$, $ f : \mathbb{Z} \rightarrow \mathbb{Z}$\end{tcolorbox}

Let $ P(x,y)$ be the assertion $ f(x + y) + f(xy) = f(x)f(y) + 1$
Let $ f(1)=a$

If $ a=1$, $ P(x-1,1)$ $ \implies$ $ f(x)+f(x-1)=f(x-1)+1$ and so $ f(x)=1$ $ \forall x$

Consider now $ a\neq 1$
$ P(0,0)$ $ \implies$ $ (f(0)-1)^2=0$ and so $ f(0)=1$
$ P(-1,1)$ $ \implies$ $ (a-1)f(-1)=0$ and so $ f(-1)=0$
$ P(-2,1)$ $ \implies$ $ 1 = (1-a)f(-2)$ and so $ 1-a|1$ and so $ a\in\{0,2\}$

If $ a=2$ $ P(x,1)$ $ \implies$ $ f(x+1)=f(x)+1$ and we got the solution $ f(x)=x+1$ $ \forall x$

If $ a=0$ $ P(x,1)$ $ \implies$ $ f(x+1)=1-f(x)$ and so $ f(x+2)=f(x)$ and we got the solution $ f(x)=0$ if $ x$ odd and $ f(x)=1$ if $ x$ even.

So three solutions :

$ f(x)=x+1$ $ \forall x$
$ f(x)=1$ $ \forall x$
$ f(x)=0$ if $ x$ odd and $ f(x)=1$ if $ x$ even
\end{solution}
*******************************************************************************
-------------------------------------------------------------------------------

\begin{problem}[Posted by \href{https://artofproblemsolving.com/community/user/63946}{rajelaz}]
	Given the function $ f : \mathbb{R} \rightarrow \mathbb{R}$ such that $ f(x^3 + y^3) = (x+y)(f(x)^2 - f(x)f(y) + f(y)^2)$, $ \forall x,y \in \mathbb{R}$. Prove that  $ f(2010x)=2010f(x)$.
	\flushright \href{https://artofproblemsolving.com/community/c6h301934}{(Link to AoPS)}
\end{problem}



\begin{solution}[by \href{https://artofproblemsolving.com/community/user/29428}{pco}]
	\begin{tcolorbox}There's function $ f(x) \mathbb{R} \rightarrow \mathbb{R}$ such that $ f(x^3 + y^3) = (x + y)(f(x)^2 - f(x)f(y) + f(y)^2)$ , $ \forall x,y \in \mathbb{R}$ .. prove that  $ f(2010x) = 2010f(x)$\end{tcolorbox}
Let $ P(x,y)$ be the assertion $ f(x^3+y^3)=(x+y)(f(x)^2-f(x)f(y)+f(y)^2)$
Let $ \mathbb S=\{x\in\mathbb R$ such that $ f(xy)=xf(y)$ $ \forall y\in\mathbb R\}$

$ 1\in\mathbb S$
$ P(0,0)$ $ \implies$ $ f(0)=0$ and $ 0\in\mathbb S$
$ P(\sqrt[3]x,0)$ $ \implies$ $ f(x)=\sqrt[3]xf(\sqrt[3]x)^2$ and so $ f(x)$ and $ x$ have same signs.
If $ p,q\in\mathbb S$, then $ P(p\sqrt[3]x,q\sqrt[3]x)$ $ \implies$ $ p^3+q^3\in\mathbb S$ (1)

If $ p\in\mathbb S$, then :
$ P(x,0)$ $ \implies$ $ f(x^3)=xf(x)^2$
$ P(\sqrt[3]px,0)$ $ \implies$ $ pf(x^3)=f(px^3)=\sqrt[3]pxf(\sqrt[3]px)^2$
Comparing these two lines, we get, for $ x\neq 0$, and using the fact that $ f(x)$ and $ x$ have same signs  : $ f(\sqrt[3]px)=\sqrt[3]pf(x)$ and so $ \sqrt[3]p\in\mathbb S$ (2)

Now :
$ p\in\mathbb S$ $ \implies$ $ \sqrt[3]p\in\mathbb S$ (using (2))
$ \sqrt[3]p\in\mathbb S$ and $ 1\in\mathbb S$ $ \implies$ $ p+1\in\mathbb S$ (using (1))

So $ p\in\mathbb S$ $ \implies$ $ p+1\in\mathbb S$, and, since $ 1\in\mathbb S$, $ \mathbb N\subseteq\mathbb S$ and so $ 2010\in\mathbb S$
Q.E.D.
\end{solution}
*******************************************************************************
-------------------------------------------------------------------------------

\begin{problem}[Posted by \href{https://artofproblemsolving.com/community/user/66674}{thuyanh158}]
	Determine all continuous functions $ f: [-1;1] \to \mathbb R$ such that \[ f(2x^2-1)=2x \cdot f(x), \quad \forall x \in [-1,1].\]
	\flushright \href{https://artofproblemsolving.com/community/c6h302132}{(Link to AoPS)}
\end{problem}



\begin{solution}[by \href{https://artofproblemsolving.com/community/user/29428}{pco}]
	\begin{tcolorbox}determine all continuous function :$ [ - 1;1] \to R$ such that : $ f(2x^2 - 1) = 2x.f(x)$.  $ \forall x \in [ - 1,1]$
 :roll:\end{tcolorbox}

Let $ P(x)$ be the assertion $ f(2x^2-1)=2xf(x)$
Let $ Q(t)$ be the assertion $ P(\cos(t))$ : $ f(\cos(2t))=2\cos(t)f(\cos(t))$

Comparing $ P(x)$ and $ P(-x)$, we get that $ f(-x)=-f(x)$ $ \forall x\neq 0$ and so, with continuity, $ f(0)=0$

So $ f(\cos(\frac{(2k+1)\pi}2))=0$

Then $ Q(\frac{(2k+1)\pi}4)$ $ \implies$ $ f(\cos(\frac{(2k+1)\pi}2))=2\cos(\frac{(2k+1)\pi}4)f(\cos(\frac{(2k+1)\pi}4))$ and so $ f(\cos(\frac{(2k+1)\pi}4))=0$

And an immediate induction gives $ f(\cos(\frac{(2k+1)\pi}{2^n}))=0$ $ \forall k,n\in\mathbb N$

And continuity implies then $ f(x)=0$ $ \forall x$
\end{solution}



\begin{solution}[by \href{https://artofproblemsolving.com/community/user/66674}{thuyanh158}]
	thank you very much for your useful post. it's very kind of you  :roll:
\end{solution}
*******************************************************************************
-------------------------------------------------------------------------------

\begin{problem}[Posted by \href{https://artofproblemsolving.com/community/user/25184}{jedaihan}]
	Find all $ f: \mathbb{R}\to\mathbb{R}$ such that
\[ f(6x(1-f(x)))+f(6x^2)=f(3xf(y))+(2-y)f(3x)\] for all $ x,y\in\mathbb{R}$.
	\flushright \href{https://artofproblemsolving.com/community/c6h302258}{(Link to AoPS)}
\end{problem}



\begin{solution}[by \href{https://artofproblemsolving.com/community/user/29428}{pco}]
	\begin{tcolorbox}Find all $ f: \mathbb{R}\to\mathbb{R}$ such that
$ f(6x(1 - f(x))) + f(6x^2) = f(3xf(y)) + (2 - y)f(3x)$ for all $ x,y\in\mathbb{R}$


own\end{tcolorbox}

Let $ P(x,y)$ be the assertion $ f(6x(1-f(x)))+f(6x^2)=f(3xf(y))+(2-y)f(3x)$

$ f(x)=0$ $ \forall x$ is a solution. We'll from now consider $ \exists c$ such that $ f(c)\neq 0$

Consider then $ f(a)=f(b)$. Comparing $ P(\frac c3,a)$ and $ P(\frac c3,b)$, we get $ a=b$ and $ f(x)$ is injective.
$ P(0,0)$ $ \implies$ $ f(0)=0$

$ P(\frac 13,x)$ $ \implies$ $ f(f(x))=xf(1)+f(2(1-f(\frac 13)))+f(\frac 23)-2f(1)$ $ =xf(1)+u$
Setting $ x=0$ in the above equation gives $ u=0$ and so $ f(f(x))=xf(1)$
Setting $ x=1$ in the above equation and using injectivity gives $ f(1)=1$ and so $ f(f(x))=x$ and so $ f(x)$ is a bijection.

Subtracting $ P(\frac x3,f(y))$ from $ P(\frac x3,0)$ gives $ f(xy)=f(x)f(y)$

Using this property for $ x\neq 0$, $ P(x,y)$ may be simplified as $ f(1-f(x))+f(x)=\frac 2{f(2)}$ 
Using $ x=0$ in the above equation gives $ f(2)=2$ and so $ f(1-f(x))=1-f(x)$

And so, since $ f(x)$ is bijective (so $ 1-f(x)$ is surjective) : $ f(x)=x$ $ \forall x$ and it is easy to verify that this indeed is a solution

So two solutions :
$ f(x)=0$ $ \forall x$
$ f(x)=x$ $ \forall x$
\end{solution}
*******************************************************************************
-------------------------------------------------------------------------------

\begin{problem}[Posted by \href{https://artofproblemsolving.com/community/user/63946}{rajelaz}]
	Find all functions $ f: \mathbb{N} \rightarrow \mathbb{N} - \{1\}$  such that for all $n \in \mathbb{N}$, \[ f(n) + f(n+1) = f(n+2)f(n+3) - 168.\]
	\flushright \href{https://artofproblemsolving.com/community/c6h303012}{(Link to AoPS)}
\end{problem}



\begin{solution}[by \href{https://artofproblemsolving.com/community/user/29428}{pco}]
	\begin{tcolorbox}Find all function $ f: \mathbb{N} \rightarrow \mathbb{N} - \{1\}$  then $ \forall n \in \mathbb{N}$ satisfy $ f(n) + f(n + 1) = f(n + 2)f(n + 3) - 168$\end{tcolorbox}

Let $ P(n)$ be the assertion $ f(n)+f(n+1)=f(n+2)f(n+3)-168$

Subtracting $ P(n)$ from $ P(n+1)$ implies $ f(n+2)-f(n)=f(n+3)(f(n+4)-f(n+2))$ and, since $ f(n+3)\geq 2$, $ |f(n+4)-f(n+2)|\leq|\frac{f(n+2)-f(n)}2|$ 

So, for $ n$ great enough, $ f(n+2)=f(n)$ and the sequence is $ a,b,a,b,a,b,...$ with $ a+b=ab-168$ $ \iff$ $ (a-1)(b-1)=169=13^2$

So the sequence ends with either $ 14,14,14,14, ...$, either with $ 2,170,2,170,2,170, ...$

And, since $ f(n)=f(n+2)f(n+3)-f(n+1)-168$, we have only three solutions :

$ 2,170,2,170,2,170, ....$
$ 170,2,170,2,170,2, ....$
$ 14,14,14,14,14,14,...$
\end{solution}
*******************************************************************************
-------------------------------------------------------------------------------

\begin{problem}[Posted by \href{https://artofproblemsolving.com/community/user/46039}{ll931110}]
	Find all functions $ f: \mathbb R \rightarrow \mathbb R$ satisfying
\[ f(f(x) + y) = 2x + f(f(y) - x)\] for all $x,y \in \mathbb R$.
	\flushright \href{https://artofproblemsolving.com/community/c6h303332}{(Link to AoPS)}
\end{problem}



\begin{solution}[by \href{https://artofproblemsolving.com/community/user/29428}{pco}]
	\begin{tcolorbox}I've just solved this problem, but my solution is not very natural. So I would like an alternate one

\begin{italicized}Find all functions $ f: R \rightarrow R$ satisfying
$ f(f(x) + y) = 2x + f(f(y) - x)$     $ \forall x,y \in R$\end{italicized}\end{tcolorbox}

Let $ P(x,y)$ be the assertion $ f(f(x)+y)=2x+f(f(y)-x)$
Let $ a=f(0)$

$ P(\frac {a-x}2,-f(\frac {a-x}2))$ $ \implies$ $ x=f(f(-f(\frac {a-x}2))-\frac {a-x}2)$ and so $ f(x)$ is a surjection.

If $ f(y_1)=f(y_2)$, and comparing $ P(x,y_1)$ and $ P(x,y_2)$, we get $ f(f(x)+y_1)=f(f(x)+y_2)$ and, since $ f(x)$ is surjective, $ f(y_1+u)=f(y_2+u)$ $ \forall u\in\mathbb R$ and so $ f(x)$ is periodic (and one period is $ T=y_1-y_2$)
But then, $ P(x+T,y)$ $ \implies$ $ f(f(x)+y)=2x+2T+f(f(y)-x)$ and so $ T=0$
So $ f(x)$ is injective.

Then $ P(0,x)$ $ \implies$ $ f(x+a)=f(f(x))$ and so, since $ f(x)$ is injective, $ f(x)=x+a$

And it is immediate to check that this indeed is a solution.

Hence the result : $ \boxed{f(x)=x+a}$
\end{solution}



\begin{solution}[by \href{https://artofproblemsolving.com/community/user/46039}{ll931110}]
	Thanks a lot, pco. Because of your solution, I realized the mistaken in mine.\/.  :)
\end{solution}
*******************************************************************************
-------------------------------------------------------------------------------

\begin{problem}[Posted by \href{https://artofproblemsolving.com/community/user/51029}{mathVNpro}]
	Determine all continuous functions $ f: \mathbb {R}\to\mathbb {R}$ such that: 
\[ f'\left(\frac {x+y}{2}\right) =\frac {f(x)-f(y)}{x-y}\]
for all reals $x$ and $y$.
	\flushright \href{https://artofproblemsolving.com/community/c6h303362}{(Link to AoPS)}
\end{problem}



\begin{solution}[by \href{https://artofproblemsolving.com/community/user/67201}{PhilG}]
	$ f'(\frac {x + y}{2})$ $ = \frac {f(x) - f(y)}{x - y}$

$ x = 1 + z, y = 1 - z \implies 2zf'(1) = f(1 + z) - f(1 - z)$
$ x = 1 + z, y = z - 1 \implies 2f'(z) = f(1 + z) - f(z - 1)$
$ x = 1 - z, y = z - 1 \implies 2(1 - z)f'(0) = f(1 - z) - f(z - 1)$

$ \therefore f'(z) = zf'(1) + (1-z)f'(0)  \implies f'(z) = 2az + b$ for constants $ 2a = f'(1) -f'(0), b = f'(0)$
$ \implies \boxed{f(z) = az^2 + bz + c}$

It can be checked by substitution that this form satisfies the equation for any constants $ a$, $ b$ and $ c$
\end{solution}



\begin{solution}[by \href{https://artofproblemsolving.com/community/user/29428}{pco}]
	\begin{tcolorbox}$ f'(\frac {x + y}{2})$ $ = \frac {f(x) - f(y)}{x - y}$

$ x = 1 + z, y = 1 - z \implies 2zf'(1) = f(1 + z) - f(1 - z)$
$ x = 1 + z, y = z - 1 \implies 2f'(z) = f(1 + z) - f(z - 1)$
$ x = 1 - z, y = z - 1 \implies 2(1 - z)f'(0) = f(1 - z) - f(z - 1)$

$ \therefore f'(z) = zf'(1) + (1 - z)f'(0) \implies f'(z) = 2az + b$ for constants $ 2a = f'(1) - f'(0), b = f'(0)$
$ \implies \boxed{f(z) = az^2 + bz + c}$

It can be checked by substitution that this form satisfies the equation for any constants $ a$, $ b$ and $ c$\end{tcolorbox}

Very short and nice, congrats !
\end{solution}
*******************************************************************************
-------------------------------------------------------------------------------

\begin{problem}[Posted by \href{https://artofproblemsolving.com/community/user/45762}{FelixD}]
	Find all functions $ f: \mathbb{R} \to \mathbb{R}$, such that
\[ f(xf(y)) + f(f(x) + f(y)) = yf(x) + f(x + f(y))\]
holds for all $ x$, $ y \in \mathbb{R}$, where $ \mathbb{R}$ denotes the set of real numbers.
	\flushright \href{https://artofproblemsolving.com/community/c6h303622}{(Link to AoPS)}
\end{problem}



\begin{solution}[by \href{https://artofproblemsolving.com/community/user/29428}{pco}]
	\begin{tcolorbox}Find all functions $ f: \mathbb{R} \to \mathbb{R}$, such that
\[ f(xf(y)) + f(f(x) + f(y)) = yf(x) + f(x + f(y))\]
holds for all $ x$, $ y \in \mathbb{R}$, where $ \mathbb{R}$ denotes the set of real numbers.\end{tcolorbox}

Let $ P(x,y)$ be the assertion $ f(xf(y))+f(f(x)+f(y))=yf(x)+f(x+f(y))$
Let $ f(0)=a$

$ f(x)=0$ $ \forall x$ is a solution. Let us from now consider $ \exists c$ such that $ f(c)\neq 0$

If $ f(y_1)=f(y_2)$, subtracting $ P(c,y_1)$ from $ P(c,y_2)$ implies $ 0=f(c)(y_2-y_1)$ and so $ y_1=y_2$ and $ f(x)$ is injective.

$ P(0,1)$ $ \implies$ $ a+f(a+f(1))=a+f(f(1))$, so $ f(a+f(1))=f(f(1))$ and so, since $ f(x)$ is injective, $ a=0$

Then $ P(x,0)$ $ \implies$ $ f(f(x))=f(x)$ and, since $ f(x)$ is injective, $ f(x)=x$

Hence two solutions :
$ f(x)=0$ $ \forall x$
$ f(x)=x$ $ \forall x$
\end{solution}



\begin{solution}[by \href{https://artofproblemsolving.com/community/user/109774}{littletush}]
	let us suppose that f is non-constant($f(x)\equiv 0 $is a trivial solution)let $x=0$,$f(0)+f(f(y)+f(0))=yf(0)+f(f(y))$
let $y=0,f(xf(0))+f(f(x)+f(0))=f(x+f(0))$
it's trivial that f is injective.
let $y=1$,then $f(f(1)+f(0))=f(f(1))$,hence $f(0)=0$
so $f(f(x))=f(x)$ hence $f(x)=x$.
\end{solution}



\begin{solution}[by \href{https://artofproblemsolving.com/community/user/29428}{pco}]
	\begin{tcolorbox}let us suppose that f is non-constant($f(x)\equiv 0 $is a trivial solution)let $x=0$,$f(0)+f(f(y)+f(0))=yf(0)+f(f(y))$
let $y=0,f(xf(0))+f(f(x)+f(0))=f(x+f(0))$
it's trivial that f is injective.
let $y=1$,then $f(f(1)+f(0))=f(f(1))$,hence $f(0)=0$
so $f(f(x))=f(x)$ hence $f(x)=x$.\end{tcolorbox}
Very interesting brand new solution !
Congrats !
\end{solution}



\begin{solution}[by \href{https://artofproblemsolving.com/community/user/234290}{getrektm9}]
	is this solution ok?
Let $P(x,y)$ be assertion of given equation.We can easily prove that the function is an injection(like in the post before).
$P(x,x)$ ->$f(xf(x))+f(2f(x))=xf(x)+f(x+f(x))$
$P(f(x),x)$ ->$f(f(x)x)+f(2x)=f(x)x+f(f(x)+x)=f(xf(x))+f(2f(x))$ which implies $f(2x)=f(2f(x))$ so since the function is an injection we get $f(x)=x$ which is solution.
Hence the only solutions are $f(x)=0,f(x)=x$
\end{solution}



\begin{solution}[by \href{https://artofproblemsolving.com/community/user/360187}{jikonmlime}]
	\begin{tcolorbox}let us suppose that f is non-constant($f(x)\equiv 0 $is a trivial solution)let $x=0$,$f(0)+f(f(y)+f(0))=yf(0)+f(f(y))$
let $y=0,f(xf(0))+f(f(x)+f(0))=f(x+f(0))$
it's trivial that f is injective.
let $y=1$,then $f(f(1)+f(0))=f(f(1))$,hence $f(0)=0$
so $f(f(x))=f(x)$ hence $f(x)=x$.\end{tcolorbox}

How did u get that $f$ is injective?
\end{solution}
*******************************************************************************
-------------------------------------------------------------------------------

\begin{problem}[Posted by \href{https://artofproblemsolving.com/community/user/63660}{Victory.US}]
	Let $ f: [0,1] \to \mathbb R$ be a function such that $ f'$ exists on $ [0,1]$ and $ f(0) = f'(0) = f'(1) = 0$. Show that there exists $ c \in (0,1)$ for which $ f(c) = c\cdot f'(c)$.
	\flushright \href{https://artofproblemsolving.com/community/c6h303639}{(Link to AoPS)}
\end{problem}



\begin{solution}[by \href{https://artofproblemsolving.com/community/user/29428}{pco}]
	\begin{tcolorbox}let $ f: [0,1] \to R$ such that there exist  $ f'$ on $ [0,1]$ and $ f(0) = f'(0) = f'(1) = 0$

show that :$ \exists c \in [0;1]$ and $ f(c) = c.f'(c)$\end{tcolorbox}

$ c=0$
\end{solution}



\begin{solution}[by \href{https://artofproblemsolving.com/community/user/63660}{Victory.US}]
	\begin{tcolorbox}[quote="Victory.US"]let $ f: [0,1] \to R$ such that there exist  $ f'$ on $ [0,1]$ and $ f(0) = f'(0) = f'(1) = 0$

show that :$ \exists c \in [0;1]$ and $ f(c) = c.f'(c)$\end{tcolorbox}

$ c = 0$\end{tcolorbox}

i checked my místake
\end{solution}



\begin{solution}[by \href{https://artofproblemsolving.com/community/user/64716}{mavropnevma}]
	Define $ g : (0, 1] \to \mathbb{R}$ by $ g(x) = \frac {f(x)} {x}$. Since $ \lim_{x \to 0} \frac {f(x)} {x} = \lim_{x \to 0} \frac {f(x) - f(0)} {x - 0} = f'(0) = 0$, it means we can prolong $ g$ to be continuous on $ [0, 1]$, by taking $ g(0) = 0$. Now, on $ (0, 1]$ we have $ g'(x) = \frac {xf'(x) - f(x)} {x^2}$, so $ g'(1) = -f(1) = -g(1)$.
If $ g(1) = 0$, as $ g(0) = 0$, by Rolle's theorem, there must exist $ c \in (0, 1)$ such that $ \frac {cf'(c) - f(c)} {c^2} = g'(c) = 0$, whence $ f(c) = cf'(c)$.
Assume therefore $ g(1) \neq 0$. 
If $ g'(x) > 0$ on $ (0, 1]$; then $ g$ is increasing and $ g'(1) = -g(1) < -g(0) = 0$, absurd.
Similarly, if $ g'(x) < 0$ on $ (0, 1]$; then $ g$ is decreasing and $ g'(1) = -g(1) > -g(0) = 0$, absurd.
Therefore there must exist $ c \in (0, 1]$ such that $ \frac {cf'(c) - f(c)} {c^2} = g'(c) = 0$, whence $ f(c) = cf'(c)$, but we work under assumption  $ g'(1) = -g(1) \neq 0$, so in fact $ c \in (0, 1)$.
\end{solution}



\begin{solution}[by \href{https://artofproblemsolving.com/community/user/18812}{pohoatza}]
	A slightly more general result holds: Let $ f\ : \ [a,b]\ \rightarrow \mathbb{R}$ be a differentiable function such that $ f'(a)=f'(b)$. Then, there is $ c \in (a,b)$ such that $ \frac{f(c)-f(a)}{c-a}=f'(c)$. This is known as Flett's Mean-Value Theorem (see for example here: http://www.math.sc.edu\/~girardi\/m555\/current\/hw\/MVT-Flett.pdf).

However, mavropnevma's proof can be adapted.
\end{solution}



\begin{solution}[by \href{https://artofproblemsolving.com/community/user/63660}{Victory.US}]
	\begin{tcolorbox}A slightly more general result holds: Let $ f\ : \ [a,b]\ \rightarrow \mathbb{R}$ be a differentiable function such that $ f'(a) = f'(b)$. Then, there is $ c \in (a,b)$ such that $ \frac {f(c) - f(a)}{c - a} = f'(c)$. This is known as Flett's Mean-Value Theorem (see for example here: http://www.math.sc.edu\/~girardi\/m555\/current\/hw\/MVT-Flett.pdf).

However, mavropnevma's proof can be adapted.\end{tcolorbox}

Could you tell me how to prove Flett's Mean-Value Theorem  :?: thanks
\end{solution}
*******************************************************************************
-------------------------------------------------------------------------------

\begin{problem}[Posted by \href{https://artofproblemsolving.com/community/user/56873}{duythuc_lqd}]
	Find all functions $f: \mathbb R \to \mathbb R$ which satisfy \[ f(x+y)=f(x) \cdot e^{f(y)-1}\] for all $x,y \in \mathbb R$.
	\flushright \href{https://artofproblemsolving.com/community/c6h304546}{(Link to AoPS)}
\end{problem}



\begin{solution}[by \href{https://artofproblemsolving.com/community/user/29428}{pco}]
	\begin{tcolorbox}Find all function $ f: R\to R$ satisfy $ f(x + y) = f(x).e^{f(y) - 1}$\end{tcolorbox}

Let $ P(x,y)$ be the assertion $ f(x+y)=f(x)\cdot e^{f(y)-1}$

If $ \exists a$ such that $ f(a)=0$, then $ P(a,x-a)$ $ \implies$ $ f(x)=0$ $ \forall x$

Consider now $ f(x)\neq 0$ $ \forall x$. Comparing $ P(x,y)$ and $ P(y,x)$, we get $ \frac{e^{f(x)-1}}{f(x)}=\frac{e^{f(y)-1}}{f(y)}$ and so $ e^{f(x)-1}=c\cdot f(x)$ $ \forall x$

But, for a given $ c$, the equation $ e^{x-1}=x$ has at most two solutions $ a_c$ and $ b_c$ and so  $ f(x)\in\{a_c,b_c\}$

Considering now two values $ x_1$ and $ x_2$ such that $ f(x_1)=f(x_2)$, $ P(x_1,x_2-x_1)$ implies $ f(x_2-x_1)=1$ and so $ c=1$ and $ f(x)=1$ $ \forall x$

Hence the two solutions :
$ f(x)=0$ $ \forall x$
$ f(x)=1$ $ \forall x$
\end{solution}



\begin{solution}[by \href{https://artofproblemsolving.com/community/user/69324}{greenhand}]
	\begin{tcolorbox}[quote="duythuc_lqd"]Find all function $ f: R\to R$ satisfy $ f(x + y) = f(x).e^{f(y) - 1}$\end{tcolorbox}

Let $ P(x,y)$ be the assertion $ f(x + y) = f(x)\cdot e^{f(y) - 1}$

If $ \exists a$ such that $ f(a) = 0$, then $ P(a,x - a)$ $ \implies$ $ f(x) = 0$ $ \forall x$

Consider now $ f(x)\neq 0$ $ \forall x$. Comparing $ P(x,y)$ and $ P(y,x)$, we get $ \frac {e^{f(x) - 1}}{f(x)} = \frac {e^{f(y) - 1}}{f(y)}$ and so $ e^{f(x) - 1} = c\cdot f(x)$ $ \forall x$

But, for a given $ c$, the equation $ e^{x - 1} = x$ has at most two solutions $ a_c$ and $ b_c$ and so  $ f(x)\in\{a_c,b_c\}$

Considering now two values $ x_1$ and $ x_2$ such that $ f(x_1) = f(x_2)$, $ P(x_1,x_2 - x_1)$ implies $ f(x_2 - x_1) = 1$ and so $ c = 1$ and $ f(x) = 1$ $ \forall x$

Hence the two solutions :
$ f(x) = 0$ $ \forall x$
$ f(x) = 1$ $ \forall x$\end{tcolorbox}

in fact there are no solutions because $ f: R\to R$
\end{solution}



\begin{solution}[by \href{https://artofproblemsolving.com/community/user/1276}{Nukular}]
	$ f : \mathbb{R} \to \mathbb{R}$ doesn't mean its surjective, just that it's a real-valued function on the reals. $ f(x) = 0$, and $ f(x) = 1$ are real-valued functions on the reals.
\end{solution}



\begin{solution}[by \href{https://artofproblemsolving.com/community/user/43536}{nguyenvuthanhha}]
	\begin{tcolorbox}Find all function $ f: R\to R$ satisfy $ f(x + y) = f(x).e^{f(y) - 1} (1)$\end{tcolorbox}
        \begin{italicized}  Oh , easy problem 

 :)   $ f \equiv 0$ is a root  . We will consider the case $ f \not \equiv 0$

      So , there is a real number $ x_0$ such that : $ f(x_0) \neq 0$

   Put $ x = x_0 \ ; y = 0$ into $ (1)$ : we have : $ f(x_0) = f(x_0)e^{f(0) - 1} \rightarrow e^{f(0) - 1} = 1$

  $ \boxed{\rightarrow f(0) = 1}$

  Put $ x = 0$ into $ (1)$ we have : $ f(y) = f(0)e^{f(y) - 1} = e^{f(y) - 1} \forall y \in \mathbb{R} (2)$

   Consider the function $ f(x) = e^{x - 1} - x$ , we can prove that :

   $ f(x) \ \geq \ 0 \ \ \forall x \in \mathbb{R}$ and  $ f(x) = 0 \iff x = 1$( Using derivative and variation board )

   So from $ (2)$ , we deduce that : $ f(y) \equiv 1 \rightarrow f(x) = 1 \forall x \in \mathbb{R}$ 

  By a direct check , we can see that $ f(x) = 1 \forall x \in \mathbb{R}$ is a root of our initial functional equation

    Done   \end{italicized}
\end{solution}
*******************************************************************************
-------------------------------------------------------------------------------

\begin{problem}[Posted by \href{https://artofproblemsolving.com/community/user/32725}{SAPOSTO}]
	Find all functions $ f: \mathbb{Q}^{+}\to \mathbb{R}$ such that $ f\left(x\right)=f\left(\frac{1}{x}\right)$ and $ xf\left(x\right)=\left(x+1\right)f\left(x-1\right)$ for all positive rational numbers $ x$ and $ f\left(1\right)=1$.
	\flushright \href{https://artofproblemsolving.com/community/c6h305160}{(Link to AoPS)}
\end{problem}



\begin{solution}[by \href{https://artofproblemsolving.com/community/user/32234}{Mashimaru}]
	We will state a way to define $ f(\frac {a}{b})$ where $ a,b$ are positive integers relatively prime to each other.
By the hypothesis, we deduce that $ \frac {f(x)}{f(x - 1)} = \frac {x + 1}{x}$. Therefore:
$ \frac {f(x)}{f(x - 1)}\cdot\frac {f(x - 1)}{f(x - 2)}\cdots\frac {f(1 + \{x\})}{f(\{x\})} = \frac {x + 1}{x}\cdot\frac {x}{x - 1}\cdots\frac {\{x\} + 2}{\{x\} + 1}$
Where $ \{x\} = x - \lfloor x \rfloor$ is the fractional part of $ x$.
This yields: $ f(x) = f(\{x\})\cdot\frac {\{x\} + 2}{\{x\} + 1}$
Since $ f(\frac {a}{b}) = f(\frac {b}{a})$, we can assume that $ a\geq b$. Consider the Euclidean division:
$ a = q_0b + r_1 \\
b = q_1r_1 + r_2 \\
r_1 = q_2r_2 + r_3 \\
\cdots \\
r_{n - 1} = q_{n}r_{n}$
where $ b: = r_0 > r_1 > r_2 > \cdots > r_{n - 1} > r_n = 1$.

Since $ (a,b) = 1$, the last $ r_n$ must be $ 1$. We have:
$ f\left(\frac {a}{b}\right)$
$ = f\left(\frac {q_1b + r_1}{b_1}\right)$
$ = f\left(\frac {r_1}{b}\right)\cdot\frac {2 + \frac {r_1}{b}}{1 + \frac {r_1}{b}}$
$ = f\left(\frac {b}{r_1}\right)\cdot\frac {2b + r_1}{b + r_1}$
$ = f\left(\frac {q_2r_1 + r_2}{r_1}\right)\cdot\frac {2b + r_1}{b + r_1}$
$ = f\left(\frac {r_1}{r_2}\right)\cdot\frac {2r_0 + r_1}{r_0 + r_1}\cdot\frac {2r_1 + r_2}{r_1 + r_2}$
$ = ... = f(1)\prod_{k = 0}^{n - 1}\frac {2r_k + r_{k + 1}}{r_k + r_{k + 1}}$

But $ f(1) = 1$ so we have $ f\left(\frac {a}{b}\right) = \prod_{k = 0}^{n - 1}\frac {2r_k + r_{k + 1}}{r_k + r_{k + 1}}$

It is easy to check that function $ f$ defined like above is the unique solution of our problem.
\end{solution}



\begin{solution}[by \href{https://artofproblemsolving.com/community/user/32725}{SAPOSTO}]
	So how much is for example $ f\left(\frac{12}{5}\right)$.
I will try to evaluate it using your method.
So we have $ 12=2\cdot5+2$ therefore $ r_{0}=5$, $ r_{1}=2$
$ 5=2\cdot2+1$ therefore $ r_{2}=1$
$ 2=1\cdot1+1$ therefore $ r_{3}=1$
$ f\left(\frac{12}{5}\right)=\frac{2\cdot5+2}{5+2}\cdot\frac{2\cdot2+1}{2+1}\cdot\frac{2\cdot1+1}{1+1}=\frac{30}{7}$. Are my calculations right? If so there must be a mistake in your solution.
\end{solution}



\begin{solution}[by \href{https://artofproblemsolving.com/community/user/32234}{Mashimaru}]
	\begin{tcolorbox}So how much is for example $ f\left(\frac {12}{5}\right)$.
I will try to evaluate it using your method.
So we have $ 12 = 2\cdot5 + 2$ therefore $ r_{0} = 5$, $ r_{1} = 2$
$ 5 = 2\cdot2 + 1$ therefore $ r_{2} = 1$
$ 2 = 1\cdot1 + 1$ therefore $ r_{3} = 1$
$ f\left(\frac {12}{5}\right) = \frac {2\cdot5 + 2}{5 + 2}\cdot\frac {2\cdot2 + 1}{2 + 1}\cdot\frac {2\cdot1 + 1}{1 + 1} = \frac {30}{7}$. Are my calculations right? If so there must be a mistake in your solution.\end{tcolorbox}

Why must there be a mistake with $ f \left( \frac{12}{5} \right) = \frac{30}{7}$ :maybe:
\end{solution}



\begin{solution}[by \href{https://artofproblemsolving.com/community/user/29428}{pco}]
	\begin{tcolorbox} Why must there be a mistake with $ f \left( \frac {12}{5} \right) = \frac {30}{7}$ :maybe:\end{tcolorbox}

Because $ f(\frac pq) = \frac {p + q}2$ $ \forall p,q\in\mathbb N$ such that $ \gcd(p,q) = 1$

or again $ f(\frac pq) = \frac {p + q}{2\gcd(p,q)}$ $ \forall p,q\in\mathbb N$
\end{solution}



\begin{solution}[by \href{https://artofproblemsolving.com/community/user/32234}{Mashimaru}]
	\begin{tcolorbox}[quote="Mashimaru"] Why must there be a mistake with $ f \left( \frac {12}{5} \right) = \frac {30}{7}$ :maybe:\end{tcolorbox}

Because $ f(\frac pq) = \frac {p + q}2$ $ \forall p,q\in\mathbb N$ such that $ \gcd(p,q) = 1$

or again $ f(\frac pq) = \frac {p + q}{2\gcd(p,q)}$ $ \forall p,q\in\mathbb N$\end{tcolorbox}

With the known result, it is easy to make an induction.
Since $ f(1) = 1$ and $ \frac {f(x)}{f(x - 1)} = \frac {x + 1}{x}$ we deduce that $ f(n) = \frac {n + 1}{2},\forall n\in\mathbb{N}*$.

Suppose that $ f(\frac {p}{q}) = \frac {p + q}{2}$ for every $ p\in\mathbb{N}^*$ for such $ (p,q) = 1$ and $ q\leq q_0$. Consider $ q_0 + 1$.

If $ p\leq q_0 + 1$ then by the inductive hypothesis we have: $ f(\frac {p}{q_0 + 1}) = f(\frac {q_0 + 1}{p}) = \frac {p + q_0 + 1}{2}$.

Else if $ p > q_0 + 1$, let $ p = k(q_0 + 1) + r$ where $ 0\leq r \leq q_0$, we have:
$ \frac {f(x)}{f(x - 1)}\cdot\frac {f(x - 1)}{f(x - 2)}\cdots\frac {f(1 + \{x\})}{f(\{x\})} = \frac {x + 1}{x}\cdot\frac {x}{x - 1}\cdots\frac {\{x\} + 2}{\{x\} + 1} = \frac {x + 1}{\{x\} + 1}$

Since then, by the inductive hypothesis:
$ f(\frac {p}{q_0 + 1}) = f(\frac {k(q_0 + 1) + r}{q_0 + 1}) = \frac {\frac {p}{q_0 + 1} + 1}{\frac {r}{q_0 + 1} + 1}\cdot f(\frac {q_0 + 1}{r}) = \frac {p + q_0 + 2}{r + q_0 + 1}\cdot\frac {r + q_0 + 1}{2} = \frac {p + (q_0 + 1) + 1}{2}$

So $ f(\frac {p}{q}) = \frac {p + q}{2}$ for every $ p,q\in\mathbb{N}^*$ as pco mentioned.

Sorry for my "solution" some posts before :blush:
\end{solution}
*******************************************************************************
-------------------------------------------------------------------------------

\begin{problem}[Posted by \href{https://artofproblemsolving.com/community/user/44674}{Allnames}]
	Find all functions  $ f: \mathbb R \to \mathbb R$ such that
i) $ f(2)=2$, and
ii) $ f(xy)=yf(x)+f(xy-xf(y)), \forall x;y \in  \mathbb R$.
	\flushright \href{https://artofproblemsolving.com/community/c6h305813}{(Link to AoPS)}
\end{problem}



\begin{solution}[by \href{https://artofproblemsolving.com/community/user/48278}{Dimitris X}]
	I'm not sure for my answer....

Setting $ z=x-f(x)$ the equation becomes:
$ f(xy)=yf(x)+f(xz)$

Setting $ z=y$ thw equation becomes:
$ zf(x)=0$.
But $ f(x) \not =0$ so $ z=0 \Longleftrightarrow f(x)=x$,which true by checking.
\end{solution}



\begin{solution}[by \href{https://artofproblemsolving.com/community/user/29428}{pco}]
	\begin{tcolorbox}I'm not sure for my answer....

Setting $ z = x - f(x)$ the equation becomes:
$ f(xy) = yf(x) + f(xz)$

Setting $ z = y$ thw equation becomes:
$ zf(x) = 0$.
But $ f(x) \not = 0$ so $ z = 0 \Longleftrightarrow f(x) = x$,which true by checking.\end{tcolorbox}

You cant write once $ z=y-f(y)$ and two lines later $ z=y$ !!

So your solution is wrong.

And, btw, $ f(x)=x$ is not the unique solution of this equation.
\end{solution}



\begin{solution}[by \href{https://artofproblemsolving.com/community/user/44674}{Allnames}]
	\begin{tcolorbox}

And, btw, $ f(x) = x$ is not the unique solution of this equation.\end{tcolorbox}
Yes pco   .The solution is $ f(x) = \frac {x}{1 + cx}$ with $ c$ is a real number.
And dear pco; I guess that  you had a nice proof for it  :)
\end{solution}



\begin{solution}[by \href{https://artofproblemsolving.com/community/user/29428}{pco}]
	\begin{tcolorbox}[quote="pco"]

And, btw, $ f(x) = x$ is not the unique solution of this equation.\end{tcolorbox}
Yes pco   .The solution is $ f(x) = \frac {x}{1 + cx}$ with $ c$ is a real number.
And dear pco; I guess that  you had a nice proof for it  :)\end{tcolorbox}

This is surely not the solution : any simple test prove this is wrong.

For example $ f(2)\neq 2$ if $ c\neq 0$

For example too, it's immediate to show that $ f(1) = 1$, so $ f(x) = \frac x{1 + cx}$ doest not fit the equation if $ c\neq 0$
\end{solution}



\begin{solution}[by \href{https://artofproblemsolving.com/community/user/44674}{Allnames}]
	:blush:  :blush:  
My god! Today; I am very tired so I made many many mistakes.Indeed, That function should be solution of the equation in the topic  http://www.mathlinks.ro/viewtopic.php?t=305815 .And that is my confusion! :oops: 
So dear pco, Can you show us your idea!
.
\end{solution}



\begin{solution}[by \href{https://artofproblemsolving.com/community/user/29428}{pco}]
	\begin{tcolorbox}Find all functions  $ f: \mathbb R \to \mathbb R$ such that: 
i) $ f(2) = 2$
ii) $ f(xy) = yf(x) + f(xy - xf(y)) \forall x;y \in \mathbb R$\end{tcolorbox}

Let $ P(x,y)$ be the assertion $ f(xy)=yf(x)+f(xy-xf(y))$

$ P(0,1)$ $ \implies$ $ f(0)=0$. Then, if $ f(u)=0$, then $ P(2,u)$ $ \implies$ $ f(2u)=2u+f(2u)$ and so $ u=0$ and so $ \boxed{f(x)=0\iff x=0}$

$ P(1,1)$ $ \implies$ $ f(1)=f(1)+f(1-f(1))$ and so $ f(1-f(1))=0$ and so $ \boxed{f(1)=1}$

$ P(x,2)$ $ \implies$ $ f(2x)=2f(x)$

Suppose $ \exists u$ such that $ f(u)=-u$. Then $ P(x,u)$ $ \implies$ $ f(ux)=uf(x)+f(2ux)$ and so $ f(ux)=-uf(x)$
Then $ P(1,y)$ $ \implies$ $ f(y)=y+f(y-f(y))$, so $ f(y-f(y))=-(y-f(y))$ and so, using the line above, $ f((y-f(y))x)=-(y-f(y))f(x)$

But $ f(xy-xf(y))=f(xy)-yf(x)$ and so $ f(xy)-yf(x)=-yf(x)+f(x)f(y)$ and so $ \boxed{f(xy)=f(x)f(y)}$
Setting $ x=y=-1$ in this equation, we get $ f(-1)^2=1$ and so two cases :

1) $ f(-1)=-1$
Then $ P(x,-1)$ $ \implies$ $ f(-x)=-f(x)$
And since $ f(xy)=f(x)f(y)$ implies that $ f(x)>0$ $ \forall x>0$, we get $ f(x)$ has the same sign as $ x$ $ \forall x\neq 0$
But $ P(1,x)$ $ \implies$ $ f(x-f(x))=-(x-f(x))$ and so $ f(x)=x$ $ \forall x$ else we would have $ f(u)=-u$ and a contradiction with the previous line.

So $ f(x)=x$ $ \forall x$ and it is easy to check that this is indeed a solution

2) $ f(-1)=1$
Then $ P(x,-1)$ $ \implies$ $ f(-x)=f(x)$
And since $ f(xy)=f(x)f(y)$ implies that $ f(x)>0$ $ \forall x>0$, we get $ f(x)>0$ $ \forall x\neq 0$
But $ P(1,x)$ $ \implies$ $ f(x-f(x))=-(x-f(x))$ and so $ f(x)\geq x$ $ \forall x$

Let then $ x>0$. If $ f(x)>x$, and since $ f(\frac 1x)\geq \frac 1x$, we get $ f(x)f(\frac 1x)>1$, which is impossible since $ f(x)f(\frac 1x)=f(x\frac 1x)=f(1)=1$

So $ f(x)=x$ $ \forall x>0$
And so, since $ f(-x)=f(x)$ :  $ f(x)=|x|$ $ \forall x$ and it is easy to check that this is indeed a solution

3) Synthesis of solutions :
We got two solutions :
$ f(x)=x$ $ \forall x$
$ f(x)=|x|$ $ \forall x$
\end{solution}



\begin{solution}[by \href{https://artofproblemsolving.com/community/user/43536}{nguyenvuthanhha}]
	\begin{italicized}Nice solution   \end{italicized}
\end{solution}



\begin{solution}[by \href{https://artofproblemsolving.com/community/user/44674}{Allnames}]
	\begin{tcolorbox}

Suppose $ \exists u$ such that $ f(u) = - u$.\end{tcolorbox}
Here is the unique sentence that I don't understand in your nice proof dear pco.Can you explain more !
I also proved $ f(xy) = f(x)f(y)$ but by the other way!.Maybe it is a bit shorter !
We have $ P(x;y - f(y))$ implies 
$ f(x(y - f(y)) = [y - f(y)]f(x) + f([y - f(y)]x - x[f(y) - y])$
$ = yf(x) - f(x)f(y) + f(2x(y - f(y))$
But note $ f(2x) = 2(x)$ thus $ f(xy - xf(y)) = f(x)f(y) - yf(x) = f(xy) - yf(x)$.Then $ f(x)f(y) = f(xy)$.
You finished this problem too well so I don't post my last parts!
Thank you!
\end{solution}



\begin{solution}[by \href{https://artofproblemsolving.com/community/user/29428}{pco}]
	\begin{tcolorbox}[quote="pco"]

Suppose $ \exists u$ such that $ f(u) = - u$.\end{tcolorbox}
Here is the unique sentence that I don't understand in your nice proof dear pco.Can you explain more !
\end{tcolorbox}

I dont understand what you dont understand :)

I demonstrated that if  $ \exists u$ such that $ f(u) = - u$, then $ f(ux)=-uf(x)$ for all $ x$

Then i demonstrated that $ f(y-f(y))=-(y-f(y))$ and so, (considering $ u=y-f(y)$), we get $ f((y-f(y))x)=-(y-f(y)f(x)$

What is the part not clear for you ?
\end{solution}



\begin{solution}[by \href{https://artofproblemsolving.com/community/user/44674}{Allnames}]
	\begin{tcolorbox}
I demonstrated that if  $ \exists u$ such that $ f(u) = - u$, then $ f(ux) = - uf(x)$ for all $ x$

Then i demonstrated that $ f(y - f(y)) = - (y - f(y))$ and so, (considering $ u = y - f(y)$), we get $ f((y - f(y))x) = - (y - f(y)f(x)$

\end{tcolorbox}
 :blush:  :blush:  :blush: .
Yes pco, Sorry for my silly question above. Now, I really realized that your proof was very nice and skillful.
Thank you!
How about my problem I linked at #6?
\end{solution}
*******************************************************************************
-------------------------------------------------------------------------------

\begin{problem}[Posted by \href{https://artofproblemsolving.com/community/user/56873}{duythuc_lqd}]
	Find all continuous functions $ f: \mathbb R \to \mathbb R$ such that $ f(x)=0$ if and only if $x=0$, and also 
\[f(x)f(y)f(x+y)=f(xy)(f(x)+f(y)),\]
holds for all reals $x$ and $y$.
	\flushright \href{https://artofproblemsolving.com/community/c6h305836}{(Link to AoPS)}
\end{problem}



\begin{solution}[by \href{https://artofproblemsolving.com/community/user/29428}{pco}]
	\begin{tcolorbox}Find all f continuous on R.$ f: R\to R$ and $ f(x) = 0\Leftrightarrow x = 0$. And satisfy $ f(x)f(y)f(x + y) = f(xy)(f(x) + f(y))$\end{tcolorbox}

Let $ P(x,y)$ be the assertion $ f(x)f(y)f(x+y)=f(xy)(f(x)+f(y))$

For $ x\neq 0$ : $ P(x,1)$ $ \implies$ $ f(1)f(x+1)=f(x)+f(1)$ and then, setting $ x\to 0$ and using continuity, $ f(1)=1$

Then $ f(x+1)=f(x)+1$ and so $ f(x+n)=f(x)+n$ and $ f(n)=n$ $ \forall x,\forall n\in\mathbb Z$

Then $ P(x,n)$ $ \implies$ $ nf(x)(f(x)+n)=f(nx)(f(x)+n)$ and so $ f(nx)=nf(x)$ $ \forall x\neq -n$ and so, with continuity, $ f(nx)=nf(x)$ $ \forall x$

So $ f(x)=x$ $ \forall x\in\mathbb Q$

So, with continuity, $ f(x)=x$ $ \forall x\in\mathbb R$
\end{solution}
*******************************************************************************
-------------------------------------------------------------------------------

\begin{problem}[Posted by \href{https://artofproblemsolving.com/community/user/43536}{nguyenvuthanhha}]
	Find all functions $ f : \mathbb{R} \to \mathbb{R}$ such that
\[f( x^2 + y + f(y)) \ = \ (f(x))^2, \quad \forall x , y  \in \mathbb{R}.\]
	\flushright \href{https://artofproblemsolving.com/community/c6h306007}{(Link to AoPS)}
\end{problem}



\begin{solution}[by \href{https://artofproblemsolving.com/community/user/42371}{mai quoc thang}]
	Pco , where are you ? :D
\end{solution}



\begin{solution}[by \href{https://artofproblemsolving.com/community/user/29428}{pco}]
	\begin{tcolorbox}Pco , where are you ? :D\end{tcolorbox}
I have an ugly solution and I'm writing it. I'll soon post.
\end{solution}



\begin{solution}[by \href{https://artofproblemsolving.com/community/user/29428}{pco}]
	\begin{tcolorbox}\begin{italicized}Find all function $ f : \mathbb{R} \to \mathbb{R} \#$ such that :

   $ f( x^2 + y + f(y)) \ = \ (f(x))^2 \ \forall \ x ;y \ \in \ \mathbb{R}$\end{italicized}\end{tcolorbox}

I really think there is a shorter demo, but here is the best I can do up to now :

Let $ P(x,y)$ be the assertion $ f(x^2+y+f(y))=f(x)^2$
Let $ U=\{x+f(x),\forall x\in\mathbb R\}$
Let $ Q(x,u)$ be the assertion $ f(x^2+u)=f(x)^2$ where $ x\in\mathbb R$ and $ u\in U$

1) \begin{bolded}$ f(-x)=f(x)\geq 0$ $ \forall x$\end{bolded}
=====================
Suppose $ \exists x$ such that $ f(x)<0$. Then $ P(\sqrt{-f(x)},x)$ $ \implies$ $ f(x)=f(\sqrt{-f(x)})^2\geq 0$ and so contradiction, and so $ f(x)\geq 0$ $ \forall x$

Comparing then $ P(x,y)$ and $ P(-x,y)$ and using the fact that $ f(x)\geq 0$, we get $ f(-x)=f(x)$ $ \forall x$

2) $ \exists z$ as great as we want such that either $ f(z)=0$, either $ f(z)=1$
======================================================
Let then $ u,v\in U$ and $ w=\max(u+v,u^2+v)+f(\max(u+v,u^2+v))$ such that $ w\in U$ and $ w>u+v$ and $ w>u^2+v$

$ Q(x+u,v)$ $ \implies$ $ f((x+u)^2+v)=f(x+u)^2$
If $ x\geq 0$ and comparing $ Q(\sqrt x,u)$ and $ Q(\sqrt x,w)$ : $ f(x+u)=f(x+w)$ and so $ f((x+u)^2+v)=f(x+w)^2$

Consider then the quadratic $ (x+u)^2+v=x+w$ $ \iff$ $ x^2+(2u-1)x+u^2+v-w$ whose discriminant is $ \Delta=1-4u-4v+4w$
Since $ w>u+v$, $ \Delta>0$ and this quadratic has two real roots $ a\geq b$
Since $ w>u^2+v$, $ ab=u^2+v-w<0$ and so $ a>0$

So, since $ a>0$, we have $ f((a+u)^2+v)=f(a+w)^2$ and, since $ (a+u)^2+v=a+w$, we get $ f(a+w)=f(a+w)^2$ and so $ f(a+w)
\in\{0,1\}$
And since $ a>0$ and $ w$ as great as we want, $ \exists z$ as great as we want such that either $ f(z)=0$, either $ f(z)=1$
Q.E.D.

3) The only two solutions are $ f(x)=0$ $ \forall x$ and $ f(x)=1$ $ \forall x$
=================================================

3.1) Suppose $ \exists z$ such that $ f(z)=0$
$ P(z,y)$ $ \implies$ $ f(z^2+y+f(y))=0$ and so we can find a $ z$ as great as we want (just choosing $ y$ great enough).
Let then $ x\in \mathbb R$ and $ z>u=x+f(x)$ such that $ f(z)=0$ and so $ f(-z)=0$ too.

$ P(\sqrt{z-x-f(x)},-z)$ $ \implies$ $ f(-x-f(x))=f(\sqrt{z-x-f(x)})^2$
$ P(\sqrt{z-x-f(x)},x)$ $ \implies$ $ 0=f(z)=f(\sqrt{z-x-f(x)})^2$
So $ f(-x-f(x))=0$ $ \forall x$ and so $ f(x+f(x))=0$ $ \forall u\in U$. Then : 

$ P(0,x)$ $ \implies$ $ f(0)=0$

$ P(\sqrt{|x+f(x)|},0)$ $ \implies$ $ f(\sqrt{|x+f(x)|})=0$

$ P(\sqrt{|x+f(x)|},y+f(y))$ $ \implies$ $ f(|x+f(x)|+y+f(y))=0$
$ P(\sqrt{|x+f(x)|},-y-f(y))$ $ \implies$ $ f(|x+f(x)|-y-f(y))=0$
So $ f(x+f(x)-y-f(y))=0$

And now, just set $ x=\frac a2$ and $ y=-\frac a2$ and we get $ f(a)=0$ $ \forall a$

3.2) Suppose $ \not\exists z$ such that $ f(z)=0$
Using paragraph 2 result, we know that $ \exists z$ such that $ f(z)=1$

$ P(z,y)$ $ \implies$ $ f(z^2+y+f(y))=1$ and so we can find a $ z$ as great as we want (just choosing $ y$ great enough).
Let then $ x\in \mathbb R$ and $ z>u=x+f(x)$ such that $ f(z)=1$ and so $ f(-z)=1$ too.

$ P(\sqrt{z-x-f(x)},-z)$ $ \implies$ $ f(1-x-f(x))=f(\sqrt{z-x-f(x)})^2$
$ P(\sqrt{z-x-f(x)},x)$ $ \implies$ $ 1=f(z)=f(\sqrt{z-x-f(x)})^2$
So $ f(1-u)=1$ $ \forall u\in U$ and so $ 2-u\in U$ $ \forall u\in U$ (since $ 2-u=1-u+f(1-u)$)

Then, comparing $ Q(\sqrt x,2-u)$ and $ Q(\sqrt x, 2-v)$, we get $ f(x+2-u)=f(x+2-v)$ $ \forall x\geq 0$ $ \forall u,v\in U$

So $ f(-x-2+u)=f(-x-2+v)$ $ \forall x\geq 0$ $ \forall u,v\in U$

So $ f(x+u)=f(x+v)$ $ \forall x\leq -2$ $ \forall u,v\in U$

So $ f(x)=f(x+u-v)$ $ \forall x\leq v-2$ $ \forall u,v\in U$

Let then $ u=a+f(a)$ and $ v=-a+f(-a)$ : $ f(x)=f(x+2a)$ $ \forall x\leq -a+f(-a)-2$ $ \forall x, a\in \mathbb R$

So $ f(x)=f(x+2a)$ $ \forall a$ $ \forall x\leq -a-2$ 

Let then $ a=-\frac{x+y}2$ : $ f(x)=f(-y)$ $ \forall y$ $ \forall x\leq \frac{x+y}2-2$ 

So $ f(x)=f(y)$ $ \forall y$ $ \forall x\leq y-4$

And so $ f(x)=$ constant $ =1$
Q.E.D.
\end{solution}



\begin{solution}[by \href{https://artofproblemsolving.com/community/user/43536}{nguyenvuthanhha}]
	Here is a short solution of my friend :

Step 1: $ f(x) \geq$ 0 for all $ x$.
Proof: Suppose $ f(y) = - t^2$ with positive $ t$.
We have $ f( t ^ 2 + y + f (y ) ) = f(y) = - t^2 < 0$, false.

From step 1 ,since $ f^ 2 ( x ) = f^2 ( - x )$ we have $ f(x) = f( - x)$.


Step 2: For all $ a \geq f(y) - y$ we have $ f(a) = f(a + 2y)$

Proof: For all $ x,y$ :
$ f ( x^ 2 + y + f ( y ) ) = f ( x^ 2 - y + f ( y ) )$ , so step 2 follows.

Step 3: For all $ b > a > f(1) - 1, f(a) = f(b)$

Proof: Since $ b > a > f(1) - 1$, from step 2 we have $ f(a + 2n) = f(a)$ and $ f(b + 2n) = f(b)$ for all natural $ n$. 

Choose n sufficently large so that $ a + 2n > f( \frac {b - a} {2} ) - \frac {b - a} {2}$ , with this n we have $ f(a + 2n) = f(a + 2n + b - a) = f(b + 2n)$, so 
$ f(a) = f(b)$. 

Note that in this step we can replace 1 by an abritary positive real $ r$ and we still have the result.

Step 4: $ f(x) = 0$ for all $ x$ or $ f(x) = 1$ for all $ x$.

Proof: From step 3 and our initial equation, clearly $ f ^ 2 ( a ) = f(a)$ for all 

$ a > f(1) - 1$, so $ f(a) = 1$ or $ 0$. But from step 3, it follows that $ f(a) = 0$ for all $ a > f(1) - 1$ or $ f(a) = 1$ for all $ a > f(1) - 1.$

Choose sufficently large a such that $ f(a) - a < 0$.

Now we repeat the reasoning of step 3 to get $ f(x) = f(y)$ for all $ x,y > f(a) - a$

In particular, $ f$ is constant for all non - negative reals. But since $ f$ is an even function, $ f$ must be constant and we obtain the desired result.
\end{solution}
*******************************************************************************
-------------------------------------------------------------------------------

\begin{problem}[Posted by \href{https://artofproblemsolving.com/community/user/32234}{Mashimaru}]
	Find all functions $ f: \mathbb{R}\to\mathbb{R}$ which are continuous on $ \mathbb{R}$ and satisfy $ 6f(f(x)) = 2f(x) + x$ for every $ x\in \mathbb{R}$.
	\flushright \href{https://artofproblemsolving.com/community/c6h306191}{(Link to AoPS)}
\end{problem}



\begin{solution}[by \href{https://artofproblemsolving.com/community/user/29428}{pco}]
	\begin{tcolorbox}Find every function $ f: \mathbb{R}\mapsto \mathbb{R}$ satisfies: $ 6f(f(x)) = 2f(x) + x$ for every $ x\in \mathbb{R}$.\end{tcolorbox}

Hemmm, beside the two trivial solutions $ f(x)=\frac{1\pm\sqrt 7}6x$, are you sure we can determine the other or even show that there are no other without any complementary condition (continuity ?, monotonous ?, ...) ?

Since you posted in "proposed and own", it means you have the solution.
\end{solution}



\begin{solution}[by \href{https://artofproblemsolving.com/community/user/32234}{Mashimaru}]
	I am extremely sorry, Mr.\begin{bolded}pco\end{bolded}, the problem I posted above should have been modified the condition that $ f$ is continuous on $ \mathbb{R}$  :blush:

My outline for the solution is first, we prove that $ f(0) = 0$, since $ f$ is continuous and injective (from the assertion $ 6f(f(x)) = 2f(x) + x$) then $ f$ is monotone. Suppose that $ f$ is increasing, we will first prove that $ f(x) = \frac{1+\sqrt{7}}{6}x$ for every $ x>0$ and then for every $ x\in\mathbb{R}$. For this, we will assume in contrast that there exist $ x_0$ such that $ f(x_0) < ax_0$ and point out the contradiction.
\end{solution}



\begin{solution}[by \href{https://artofproblemsolving.com/community/user/69640}{k.l.l4ever}]
	Can you explain your idea in a clear way,dear Mashimaru?I didn't understand what is your  \begin{bolded}a\end{bolded}?
\end{solution}
*******************************************************************************
-------------------------------------------------------------------------------

\begin{problem}[Posted by \href{https://artofproblemsolving.com/community/user/66215}{kenan aze}]
	Find all functions $f: \mathbb R \to \mathbb R$ which satisfy \[f( x + f ( xy ) )= f( x ) + x f( y ) \] for all reals $x$ and $y$.
	\flushright \href{https://artofproblemsolving.com/community/c6h306696}{(Link to AoPS)}
\end{problem}



\begin{solution}[by \href{https://artofproblemsolving.com/community/user/29428}{pco}]
	\begin{tcolorbox}f:R - R\/      x, y are real numbers, Find all functions such that 
f[ x + f ( xy ) ] = f( x ) + x f( y ) :)\end{tcolorbox}

I found this problem rather hard. Thank you for giving us a simpler solution if you got one.

Let $ P(x,y)$ be the assertion $ f(x+f(xy))=f(x)+xf(y)$

$ f(x)=0$ $ \forall x$ is a trivial solution. So we'll now consider non all-zero solutions.

1) $ f(x)=0$ $ \iff$ $ x=0$
========================

Suppose that $ \exists u\neq 0$ such that $ f(u)=0$. then, for $ x\neq 0$, $ P(\frac ux,x)$ $ \implies$ $ \frac uxf(x)=0$ and so :

$ f(x)=0$ $ \forall x\neq 0$ and then $ P(0,0)$ $ \implies$ $ f(f(0))=f(0)$ and so $ f(0)=0$ (else we would have a non zero real $ f(0)$ whose image would be non zero.

So $ f(x)=0$ $ \implies$ $ x=0$.
Then $ P(-1,-1)$ $ \implies$ $ f(f(1)-1)=0$ and so $ f(1)=1$ and $ f(0)=0$
Q.E.D.

2) $ f(n)=n$ $ \forall n\in\mathbb Z$
==================================

$ P(1,x)$ $ \implies$ $ f(f(x)+1)=f(x)+1$
$ P(1,f(x)+1)$ $ \implies$ $ f(f(x)+2)=f(x)+2$
An immediate induction gives $ f(f(x)+n)=f(x)+n$ $ \forall x$, $ \forall n>0\in\mathbb N$

Then $ P(-n,-1)$ $ \implies$ $ f(-n)=nf(-1)$

Then $ P(-1,-2)$ $ \implies$ $ f(1)=f(-1)-f(-2)=-f(-1)$ and so $ f(-1)=-1$ and $ f(n)=n$ $ \forall n\in\mathbb Z$
Q.E.D.

3) $ f(x)=1$ $ \iff$ $ x=1$
=========================

We already know that $ f(1)=1$. Suppose now $ f(a)=1$, then : 

$ P(\frac 1x,x)$ $ \implies$ $ f(\frac 1x+1)=f(\frac 1x)+\frac{f(x)}x$

$ P(\frac 1x,ax)$ $ \implies$ $ f(\frac 1x+1)=f(\frac 1x)+\frac{f(ax)}x$

And so $ f(ax)=f(x)$ $ \forall x\neq 0$ and so $ f(ax)=f(x)$ $ \forall x$

So $ f(-a)=-1$ and then $ P(a,-1)$ $ \implies$ $ f(a-1)=1-a$
So $ f(a(a-1))=f(a-1)=1-a$ and then $ P(a,a-1)$ $ \implies$ $ f(a+f(a(a-1)))=f(a)+af(a-1)$ and so $ 1=1+a(1-a)$ and so $ a=1$
Q.E.D

4) $ f(x)$ is injective
=======================

Consider now $ f(y_1)=f(y_2)$ with $ y_1\neq y_2$. Obviously $ y_1,y_2\neq 0$ (since $ f(x)=0$ implies $ x=0$) and let then $ a=\frac{y_1}{y_2}$ :
Comparing  $ P(y_2,a)$ and $ P(y_2,1)$, we get $ f(a)=1$, so $ a=1$ (point 3 above) and $ y_1=y_2$ and $ f(x)$ is injective.
Q.E.D.

5) $ f(x)=x$ $ \forall x$
=======================

Let $ A=f(\mathbb R)$

$ P(1,x)$ $ \implies$ $ f(f(x)+1)=f(x)+1$ and so $ x\in A$ $ \implies$ $ x+1\in A$
$ P(-1,x)$ $ \implies$ $ f(f(-x)-1)=-1-f(x)$ and so $ x\in A$ $ \implies$ $ -1-x\in A$ and so, using line above, $ -x\in A$
So, since $ x\in A$ implies $ x+1\in A$ and $ -x\in A$ we get $ -x+1\in A$ and $ x-1\in A$

Then for a given $ x$, and since $ f(x)-1\in A$, let $ z$ such that $ f(z)=f(x)-1$ :
$ P(1,z)$ $ \implies$ $ f(f(z)+1)=f(z)+1$ and so $ f(f(x))=f(x)$ $ \forall x$ and, since $ f(x)$ is injective, $ f(x)=x$ $ \forall 
x$

And it's immediate to check back that this indeed is a solution.

6) synthesis of solutions
==========================

Hence the two solutions of this equation :
$ f(x)=0$ $ \forall x$
$ f(x)=x$ $ \forall x$
\end{solution}



\begin{solution}[by \href{https://artofproblemsolving.com/community/user/66215}{kenan aze}]
	congrats you. i think your solution is very good. but  i prove it by finding f(1) and f(-1). i have no time to write solution. i will send my solution in a week when school holiday will begin. again thank you  for your solution. :)
\end{solution}



\begin{solution}[by \href{https://artofproblemsolving.com/community/user/66215}{kenan aze}]
	f(x+f(xy))= f(x) + xf(y) 

1) let find f(0)=0
let f(0) gets other value than 0. x=y=o  ff(0)=f(0)
x=f(0),y=1 then x=f(0),y=0.  then we get f(0)=f(1)
x=y=1 then x=1, y= 1 + f(0) then x= 1 + f(1) , y =0    we will get f(0)f(1)=0 . which means f(0)=0

2)let find f(1)=1, 
if f(1) + 1 = A = 0  then  x = y = 1 then we get f(1) = 0 condraction A is not zero
x= 1 => f( 1 + f ( y )) = f(1) + f ( y ) is B
at B y= 1 + f( y ) => f( A + f(y) ) = f( A ) + f (y)
x=A y= y\/A  f ( A + f(y) ) = f( A ) + A f ( y\/A )
then we get f(y) = A f ( y \/ A )
at B y=A  then we get f(1) ( f(1) - 1 ) = 0 
if f(1) = 0 x= 1\/m y= m then we get  f(m) = 0 and f(0) = 0 which means f is obviously zero
then f(1) = 1 

3)let prove that if n is N then f(yn)=nf(y)
it is easy to prove by helping of B by induction if n is NATURAL then  C=> f( n + f(y) ) = n + f(y)  
at C y = 1 then xis natural f( x ) = x
at C y = yn f( n + f( yn ) ) = n + f( yn ) 
x= n, n is N, f ( n + f( yn ) ) = f (n) + nf(y) = n + f (yn)  we get if n is N then f(yn)=nf(y)
n=2 y= -1\/2 then f( -1\/2 ) = - 1\/2

4)let find f( -1 ) = -1
if f(k) = 0 x=k , y= 1 then we get k = 0
x= -1 y= 1 then f( -1 ) = 1 + f ( -1 + f( -1 )) then ff( -1 ) = f(1 + f ( -1 + f( -1 ))) = 1 + f( -1 + f( -1 )) = 1 + f ( -1) -1 we get ff(-1) = f ( -1)
x=1 y= -1 then f(-1) = -1 + f ( 1 + f ( -1 )) then ff( -1 ) = f ( -1 + f ( 1 + f ( -1 ))) =  f(-1) + f(-1 - f( -1 )) then f(-1 - f( -1 ))= 0 => f(-1) = -1

5) let finish problem
at B y = -1 + f( -y)    f( 1 + f( -1 + f(-y)))= 1 + f ( -1 + f(-y) ) = -f(y) then we get D=>f( - f(y)) = -f(y)
at D let give y = -f(y) we get ff(y) = f(y)
y= -1    f(x - f(-x)) = f(x) - x then   f( f(x - f(-x)))= f(f(x) - x) = f ( x + f( -x)) = x + f( -x) = f(x) - x then we get 2x + f(-x) = f(x)
then we get 2x + f(-x) = f(x) , then f(2x + f(-x)) = ff(x) = f(x) = f(2x) + 2xf( -1\/2 ) = 2f(x) - x  which means f(x) = x

6) last part is checking
............................................ is obviously true :D 


=========================================================================================================================
the land of fire AZERBAIJAN
\end{solution}



\begin{solution}[by \href{https://artofproblemsolving.com/community/user/29428}{pco}]
	\begin{tcolorbox}f(x+f(xy))= f(x) + xf(y) 

1) let find f(0)=0
let f(0) gets other value than 0. x=y=o  ff(0)=f(0)
x=f(0),y=1 then x=f(0),y=0.  then we gets f(0)=f(1)
x=y=1 then x=1, y= 1 + f(0) then x= 1 + f(1) , y =0    we will get f(0)f(1)=0 . which means f(0)=0
\end{tcolorbox}

$ x = y = 1$ $ \implies$ $ f(1 + f(1)) = f(1) + f(1)$
$ x = 1, y = 1 + f(0)$ $ \implies$ $ f(1 + f(1 + f(0))) = f(1) + f(1 + f(0))$

And I dont understand how you conclude from these two lines that $ f(0)f(1) = 0$
 :blush:
\end{solution}



\begin{solution}[by \href{https://artofproblemsolving.com/community/user/66215}{kenan aze}]
	f(x+f(xy))= f(x) + xf(y) 

1) let find f(0)=0 
let f(0) gets other value than 0. x=y=o ff(0)=f(0) 
x=f(0),y=1 then x=f(0),y=0. then we gets f(0)=f(1) 
x=y=1                   f(1+f(1))=2f(1)
x=1, y= 1 + f(0)     f(1 + f(1+f(0)))=f(1) + f( 1 + f(0))= f(1) + f(1 + f(1)) = f(1) + 2f(1)= 3 f(1)
x= 1 + f(1) , y =0   f(1 + f(1) + f(0))= f( 1 + f(1)) + (1 + f(1))f(0) = 2f(1) + f(0) + f(1)f(0) = 3f(1) + f(1)f(0)
from last two equations we get that  f(0)f(1)=0 . which means f(0)=0

i think there is no dificculty to calculate some simple operations.... :D 

====================================================================================================================================
LAND OF FIRE AZERBAIJAN
\end{solution}



\begin{solution}[by \href{https://artofproblemsolving.com/community/user/29428}{pco}]
	\begin{tcolorbox} x= 1 + f(1) , y =0   f(1 + f(1) + f(0))= f( 1 + f(1)) + (1 + f(1))f(0) = 2f(1) + f(0) + f(1)f(0) = 3f(1) + f(1)f(0)

i think there is no dificculty to calculate some simple operations.... :D 
\end{tcolorbox}

Sure, but you did not give the last step and since you gave the two first, I thought there was no missing step.

And I dont worry finding $ f(0)$. I gave an earlier proof of this problem and was just reading yours in order to help you to improve.

I'm sorry for your reaction and will no longer read your proofs.

Enjoy mathlinks and maths.
\end{solution}



\begin{solution}[by \href{https://artofproblemsolving.com/community/user/66215}{kenan aze}]
	i am sorry if you feel worse yourself for my reply. i use it randomly.  big percent you are elder than me. i dont want to make you unhappy.  maybe i make somethink wrongly
\end{solution}
*******************************************************************************
-------------------------------------------------------------------------------

\begin{problem}[Posted by \href{https://artofproblemsolving.com/community/user/46039}{ll931110}]
	Find all functions $f: \mathbb R \to \mathbb R$ satisfying
\[ f(x^2) = f^2(x)\] and \[ f(x + 1) = f(x) + 1\] for all real numbers $x$.
	\flushright \href{https://artofproblemsolving.com/community/c6h306697}{(Link to AoPS)}
\end{problem}



\begin{solution}[by \href{https://artofproblemsolving.com/community/user/29428}{pco}]
	\begin{tcolorbox}Find all functions $ f: R \rightarrow R$ satisfying
$ f(x^2) = f^2(x)$ and $ f(x + 1) = f(x) + 1$ for all real number x.\end{tcolorbox}

1) $ f(-x)=-f(x)$ $ \forall x$
============================
$ f(x)^2=f(x^2)=f((-x)^2)=f(-x)^2$ and so, $ \forall x$, either $ f(-x)=f(x)$, either $ f(-x)=-f(x)$

Suppose now $ \exists u$ such that $ f(-u)=f(u)$. Then $ f(1-u)=1+f(-u)=1+f(u)$ and $ f(u-1)=f(u)-1$
But, since either $ f(1-u)=f(u-1)$, either $ f(1-u)=-f(u-1)$, we get that either $ f(u)+1=f(u)-1$, either $ f(u)+1=1-f(u)$ and so $ f(u)=0$ and so $ f(-u)=f(u)=0=-f(u)$

So $ f(-x)=-f(x)$ $ \forall x$
Q.E.D.

2) $ f(x)$ has same sign as $ x$
==============================
$ \forall x\geq 0$ : $ f(x)=f((\sqrt x)^2)=f(\sqrt x)^2\geq 0$
$ \forall x\leq 0$ : $ f(x)=-f(-x)=-f((\sqrt{-x})^2)=-f(\sqrt{-x})^2\leq 0$
Q.E.D.

3) $ f(x)=x$ $ \forall x$
=======================
3.1) $ f(x)\leq x$ $ \forall x$
Suppose $ \exists x>1$ and $ u>0$ such that $ f(x)=x+u>x$. Then $ f(x^{2^n})=(x+u)^{2^n}$ $ \forall n$
But $ \exists n$ great enough so that $ (x+u)^{2^n}>x^{2^n}+1$ and so $ f(x^{2^n})>x^{2^n}+1$
So $ f(x^{2^n}-[x^{2^n}]-1)>x^{2^n}+1-[x^{2^n}]>0$ and so we have a negative number ($ x^{2^n}-[x^{2^n}]-1$) whose image by $ f(x)$ would be positive, which is impossible, according to point 2) above.
So $ f(x)\leq x$ $ \forall x>1$

And since $ f(x+n)=f(x)+n$ $ \forall n\in\mathbb Z$, we get that $ f(x+n)\leq x+n$ $ \forall x>1$ and so $ f(x)\leq x$ $ \forall x$

3.2) $ f(x)=x$ $ \forall x$
We got $ f(x)\leq x$ $ \forall x$
So $ f(-x)\leq -x$ $ \forall x$ and, since $ f(-x)=-f(x)$, we get :  $ f(x)\geq x$ $ \forall x$
And so $ f(x)=x$ $ \forall x$
Q.E.D.
\end{solution}
*******************************************************************************
-------------------------------------------------------------------------------

\begin{problem}[Posted by \href{https://artofproblemsolving.com/community/user/51029}{mathVNpro}]
	1. Find all $ f: \mathbb {R}\to\mathbb {R}$ such that
\[ f(x^2 + y) + f(f(x) - y) = 2f(f(x)) + 2y^2,\]
happens for all $ x,y\in \mathbb {R}$.

2. Find all $ f: \mathbb {R}\to\mathbb {R}$ such that for all $ x,y\in \mathbb {R}$, we have
\[f(f(x) + y) = 2x + f(f(y) - x).\]
	\flushright \href{https://artofproblemsolving.com/community/c6h307117}{(Link to AoPS)}
\end{problem}



\begin{solution}[by \href{https://artofproblemsolving.com/community/user/29428}{pco}]
	\begin{tcolorbox}Problem 1:\end{underlined} Let $ f: \mathbb {R}\longrightarrow \mathbb {R}$. Find all $ f$ such that:
$ f(x^2 + y) + f(f(x) - y) = 2f(f(x)) + 2y^2$, $ \forall$ $ x,y\in \mathbb {R}$.\end{tcolorbox}

Let $ P(x,y)$ be the assertion $ f(x^2 + y) + f(f(x) - y) = 2f(f(x)) + 2y^2$

$ P(x,0)$ $ \implies$ $ f(f(x))=f(x^2)$

$ P(x,f(x)-x^2)$ $ \implies$ $ f(f(x))+f(x^2)=2f(f(x))+2(f(x)-x^2)^2$ and so, since $ f(f(x))=f(x^2)$ : $ 2(f(x)-x^2)^2=0$ and so $ \boxed{f(x)=x^2}$

And it's easy to check that this function indeed is a solution.
\end{solution}



\begin{solution}[by \href{https://artofproblemsolving.com/community/user/40922}{mehdi cherif}]
	Probleme 1:

let $ P(x,y)$ be the assertioon : $ f(x^2+y)+f(f(x)-y)=2f(f(x))+2y^2$

denote $ a=f(0)$

$ P(0,0)\implies f(a)=a$

$ P(0,a) \implies a+a=2a+2a^2 \implies a=0$

so $ f(0)=0$

$ P(x,f(x))\implies f(x^2+f(x))=2f(f(x))+2f^2(x)$ (1)

$ P(x,-x^2)\implies f(f(x)+x^2)=2f(f(x))+2x^4$  (2)

$ P(0,x)\implies f(x)+f(-x)=2x^2$  (3)

from (1) ,(2) and (3) we get $ f(x)=x^2 ,\forall x$

Mehdi
\end{solution}



\begin{solution}[by \href{https://artofproblemsolving.com/community/user/29428}{pco}]
	\begin{tcolorbox}Probleme 1:

let $ P(x,y)$ be the assertioon : $ f(x^2 + y) + f(f(x) - y) = 2f(f(x)) + 2y^2$

denote $ a = f(0)$

$ P(0,0)\implies f(a) = a$

$ P(0,a) \implies a + a = 2a + 2a^2 \implies a = 0$

so $ f(0) = 0$

$ P(x,f(x))\implies f(x^2 + f(x)) = 2f(f(x)) + 2f^2(x)$ (1)

$ P(x, - x^2)\implies f(f(x) + x^2) = 2f(f(x)) + 2x^4$  (2)

$ P(0,x)\implies f(x) + f( - x) = 2x^2$  (3)

from (1) ,(2) and (3) we get $ f(x) = x^2 ,\forall x$

Mehdi\end{tcolorbox}

Hello Mehdi !

Quite nice :) (I appreciated the usage of assertion 3 for solving $ f(x)^2 = x^4$)

Congrats.
(and sorry to have posted some seconds before you) :)
\end{solution}



\begin{solution}[by \href{https://artofproblemsolving.com/community/user/40922}{mehdi cherif}]
	\begin{tcolorbox}[quote="mehdi cherif"]Probleme 1:

let $ P(x,y)$ be the assertioon : $ f(x^2 + y) + f(f(x) - y) = 2f(f(x)) + 2y^2$

denote $ a = f(0)$

$ P(0,0)\implies f(a) = a$

$ P(0,a) \implies a + a = 2a + 2a^2 \implies a = 0$

so $ f(0) = 0$

$ P(x,f(x))\implies f(x^2 + f(x)) = 2f(f(x)) + 2f^2(x)$ (1)

$ P(x, - x^2)\implies f(f(x) + x^2) = 2f(f(x)) + 2x^4$  (2)

$ P(0,x)\implies f(x) + f( - x) = 2x^2$  (3)

from (1) ,(2) and (3) we get $ f(x) = x^2 ,\forall x$

Mehdi\end{tcolorbox}

Hello Mehdi !

Quite nice :) (I appreciated the usage of assertion 3 for solving $ f(x)^2 = x^4$)

Congrats.
(and sorry to have posted some seconds before you) :)\end{tcolorbox}

thank's mr.Patrick  :)
\end{solution}



\begin{solution}[by \href{https://artofproblemsolving.com/community/user/29428}{pco}]
	\begin{tcolorbox}Problem 2:\end{underlined} Let $ f: \mathbb {R}\longrightarrow \mathbb {R}$. Find all $ f$ such that:
$ f(f(x) + y) = 2x + f(f(y) - x)$, $ \forall$ $ x,y\in \mathbb {R}$\end{tcolorbox}

Let $ P(x,y)$ be the assertion $ f(f(x)+y)=2x+f(f(y)-x)$

$ P(\frac{f(v)}2,u-f(\frac{f(v)}2))$ $ \implies$ $ f(u)=f(v)+f(\text{something})$ and so $ f(u)-f(v)\in f(\mathbb R)$ $ \forall u,v$

So $ f(f(x)+y)-f(f(y)-x)=2x\in f(\mathbb R)$ and so $ f(x)$ is surjective.

If $ f(y_1)=f(y_2)$, then comparing $ P(x,y_1)$ and $ P(x,y_2)$, we get $ f(y_1+f(x))=f(y_2+f(x))$ and, since $ f(x)$ is surjective, $ f(x+T)=f(x)$ $ \forall x$ with $ T=y_1-y_2$

Comparing then $ P(x,y)$ and $ P(x+T,y)$, we get $ T=0$ and so $ f(x)$ is injective.

Then $ P(0,x)$ $ \implies$ $ f(x+f(0))=f(f(x))$ and, since $ f(x)$ is injective, $ f(x)=x+f(0)$

And it's immediate to check back that this indeed is a solution

Hence the result : $ \boxed{f(x)=x+a}$ $ \forall x$
\end{solution}



\begin{solution}[by \href{https://artofproblemsolving.com/community/user/37447}{mr.danh}]
	\begin{tcolorbox}
Problem 2:\end{underlined} Let $ f: \mathbb {R}\longrightarrow \mathbb {R}$. Find all $ f$ such that:
$ f(f(x) + y) = 2x + f(f(y) - x)$, $ \forall$ $ x,y\in \mathbb {R}$.

Best regard,
\begin{bolded}mathVNpro\end{bolded}\end{tcolorbox}
$ P(x,-f(x))\rightarrow f(0)=2x+f(f(-f(x))-x)$, so $ f$ is surjective.
Let $ a\in \mathbb R$ such that $ f(a)=0$.
$ P(a,y)\rightarrow f(y)=2a+f(f(y)-a)\Rightarrow f(y)-a=a+f(f(y)-a)$
Since $ f$ is surjective, for each real $ x$, there exist a real $ y$ for $ x=f(y)-a$. Then, $ x=a+f(x)$ for every $ x\in \mathbb R$.
The result  $ f(x)=x+c$, $ c\in \mathbb R$
\end{solution}



\begin{solution}[by \href{https://artofproblemsolving.com/community/user/29428}{pco}]
	\begin{tcolorbox}[quote="mathVNpro"]
Problem 2:\end{underlined} Let $ f: \mathbb {R}\longrightarrow \mathbb {R}$. Find all $ f$ such that:
$ f(f(x) + y) = 2x + f(f(y) - x)$, $ \forall$ $ x,y\in \mathbb {R}$.

Best regard,
\begin{bolded}mathVNpro\end{bolded}\end{tcolorbox}
$ P(x, - f(x))\rightarrow f(0) = 2x + f(f( - f(x)) - x)$, so $ f$ is surjective.
Let $ a\in \mathbb R$ such that $ f(a) = 0$.
$ P(a,y)\rightarrow f(y) = 2a + f(f(y) - a)\Rightarrow f(y) - a = a + f(f(y) - a)$
Since $ f$ is surjective, for each real $ x$, there exist a real $ y$ for $ x = f(y) - a$. Then, $ x = a + f(x)$ for every $ x\in \mathbb R$.
The result  $ f(x) = x + c$, $ c\in \mathbb R$\end{tcolorbox}

Quite OK. Nice and quick , and better than mine !

Congrats!  :)
\end{solution}



\begin{solution}[by \href{https://artofproblemsolving.com/community/user/70520}{hvaz}]
	How can I edit my text thus he saty like yours( R -> R) hehe
Thanks
\end{solution}
*******************************************************************************
-------------------------------------------------------------------------------

\begin{problem}[Posted by \href{https://artofproblemsolving.com/community/user/27740}{MeKnowsNothing}]
	1. Is there a function $ f: \mathbb{R}\to \mathbb{R}$ satisfying $ f(f(x)) = - x$?
2. Is there a function $ f: \mathbb{Z}\to \mathbb{Z}$ satisfying $ f(f(x)) = - x$?
	\flushright \href{https://artofproblemsolving.com/community/c6h307131}{(Link to AoPS)}
\end{problem}



\begin{solution}[by \href{https://artofproblemsolving.com/community/user/29428}{pco}]
	\begin{tcolorbox}1. Is there a function $ f: \mathbb{R}\to \mathbb{R}$ satisfying $ f(f(x)) = - x$?
1. Is there a function $ f: \mathbb{Z}\to \mathbb{Z}$ satisfying $ f(f(x)) = - x$?\end{tcolorbox}

Yes and yes, infinitely many in both cases :

For 1. 
Split $ \mathbb R^ +$ in two equinumerous sets $ A$ and $ B$ and let $ h(x)$ a bijection from $ A\to B$. Then :

$ f(0) = 0$
$ \forall x\in A$ : $ f(x) = h(x)$ and $ f( - x) = - h(x)$
$ \forall x\in B$ : $ f(x) = - h^{[ - 1]}(x)$ and $ f( - x) = h^{[ - 1]}(x)$



For 2.
Split $ \mathbb N$ in two equinumerous sets $ A$ and $ B$ and let $ h(x)$ a bijection from $ A\to B$. Then :

$ f(0) = 0$
$ \forall x\in A$ : $ f(x) = h(x)$ and $ f( - x) = - h(x)$
$ \forall x\in B$ : $ f(x) = - h^{[ - 1]}(x)$ and $ f( - x) = h^{[ - 1]}(x)$

Example : $ A = \{2p,p\in\mathbb N\}$, $ B = \{2p - 1,p\in\mathbb N\}$ and $ h(x) = x - 1$. So :
$ f(0) = 0$
$ f(2n) = 2n - 1$ $ \forall n > 0$
$ f(2n - 1) = - 2n$ $ \forall n > 0$
$ f( - 2n) = - 2n + 1$ $ \forall n > 0$
$ f( - 2n + 1) = 2n$ $ \forall n > 0$
\end{solution}
*******************************************************************************
-------------------------------------------------------------------------------

\begin{problem}[Posted by \href{https://artofproblemsolving.com/community/user/32234}{Mashimaru}]
	Prove that there exists one and only one function $ f: \mathbb{Q}^ + \to \mathbb{Q}^ +$ which satisfies the following conditions:
i. For $ 0 < q < \frac {1}{2}$, \[ f(q) = 1 + f(\frac {q}{1 - 2q}).\]
ii. For $ 1 < q < 2$, \[ f(q) = 1 + f(q - 1).\]
iii. For every $ q\in\mathbb{Q}^ +$, 
\[ f(q)f\left(\frac {1}{q}\right) = 1\].
	\flushright \href{https://artofproblemsolving.com/community/c6h307262}{(Link to AoPS)}
\end{problem}



\begin{solution}[by \href{https://artofproblemsolving.com/community/user/29428}{pco}]
	\begin{tcolorbox}Let $ \mathbb{Q}^ +$ denoted the set of positive rational numbers. Prove that there exists one and only one function $ f: \mathbb{Q}^ + \to \mathbb{Q}^ +$ satisfies the following condition:
i. For $ 0 < q < \frac {1}{2}$ then $ f(q) = 1 + f(\frac {q}{1 - 2q})$.
ii. For $ 1 < q < 2$ then $ f(q) = 1 + f(q - 1)$.
iii For every $ q\in\mathbb{Q}^ +$, $ f(q)f(\frac {1}{q}) = 1$.\end{tcolorbox}

I think the result is wrong.

Let $ q=\frac ab$ with $ \gcd(a,b)=1$

If $ 0<q<\frac 12$, we get $ 2a<b$ and  $ f(\frac ab)=1+f(\frac a{b-2a})$
If $ \frac 12<q<1$, we get $ 1<\frac 1q<2$ and so $ f(\frac ab)=\frac 1{1+f(\frac {b-a}a)}$
If $ 1<q<2$, we get $ f(\frac ab)=1+f(\frac{a-b}b)$
If $ 2<q$, we get $ 0<\frac 1q<\frac 12$ and so $ f(\frac ab)=\frac 1{1+f(\frac b{a-2b})}$

So, for any $ q=\frac ab>0\notin\{\frac 12,1,2\}$, we can calculate in a unique manner from another rational $ \frac cd$ such that $ c+d<a+b$

And so, any $ f(\frac ab)$ is uniquely determined by the knowledge of $ f(1),f(\frac 12)$ and $ f(2)$

We obviously have $ f(1)=1$ but we can freely choose $ f(2)=u$ and $ f(\frac 12)=\frac 1u$ so that all rules are always fullfilled.

And, in my opinion, there are as many solutions as we have different values $ u=f(2)$
\end{solution}
*******************************************************************************
-------------------------------------------------------------------------------

\begin{problem}[Posted by \href{https://artofproblemsolving.com/community/user/46787}{moldovan}]
	Determine all monotonous functions $ f: \mathbb{R} \rightarrow \mathbb{R}$ which satisfy \[ f^3(x)-3f^2(x)+6f(x)=4x+3,\] for all $x \in \mathbb{R}$, where $ f^3=f \circ f \circ f$ and $ f^2=f \circ f$.
	\flushright \href{https://artofproblemsolving.com/community/c6h307281}{(Link to AoPS)}
\end{problem}



\begin{solution}[by \href{https://artofproblemsolving.com/community/user/29428}{pco}]
	\begin{tcolorbox}Determine all monotonous functions $ f: \mathbb{R} \rightarrow \mathbb{R}$ which satisfy: $ f^3(x) - 3f^2(x) + 6f(x) = 4x + 3, (\forall) x \in \mathbb{R}$, where $ f^3 = f \circ f \circ f$ and $ f^2 = f \circ f.$\end{tcolorbox}

Let $ u\in\mathbb R$. Consider the sequence $ a_n=f^{[n]}(u)$

We get $ a_n=3a_{n-1}-6a_{n-2}+4a_{n-3}+3$ with $ a_0=u,a_1=f(u)$, and $ a_2=f(f(u))$

The equation $ x^3-3x^2+6x-4$ has 3 roots $ 1,2e^{\frac{i\pi}3}$ and $ 2e^{-\frac{i\pi}3}$ and so we get :

$ a_n=u+n+r2^n\cos(\frac{n\pi}3+\theta)-r\cos(\theta)$ for some $ r>0,\theta$ which may be be computed from $ a_1$ and $ a_2$

Since $ f(x)$ is monotonous, we need to have $ u_n=\frac{a_{n+2}-a_{n+1}}{a_{n+1}-a_n}$ $ =\frac{f(a_{n+1})-f(a_n)}{a_{n+1}-a_n}$ with a constant sign.

But $ u_n=\frac{1-r\sqrt 32^{n+1}\sin(\frac{(n+1)\pi}3+\theta)}{1-r\sqrt 32^{n}\sin(\frac{n\pi}3+\theta)}$ $ =\frac{1-r\sqrt 32^{n+1}v_{n+1}}{1-r\sqrt 32^{n}v_n}$ where $ v_n=\sin(\frac{n\pi}3+\theta)$

And, obviously, when $ n\to +\infty$, we always have some $ n$ where $ v_{n+1}$ and $ v_n$ have the same sign and some $ n$ where $ v_{n+1}$ and $ v_n$ have opposite sign.

So $ u_n=\frac{1-r\sqrt 32^{n+1}v_{n+1}}{1-r\sqrt 32^{n}v_n}$ can have a constant sign when $ n\to +\infty$ only if $ r=0$

So $ a_n=u+n$ and $ f(u)=u+1$ which indeed is a solution.

So the unique monotonous solution to this equation is $ \boxed{f(x)=x+1}$
\end{solution}
*******************************************************************************
-------------------------------------------------------------------------------

\begin{problem}[Posted by \href{https://artofproblemsolving.com/community/user/46787}{moldovan}]
	Determine all continous functions $ f: \mathbb{R} \rightarrow \mathbb{R}$ which satisfy the relation:
\[2f(x+y)+f(2x-y)+f(2y-x)=9f(x)+9f(y),\]
for all $x,y \in \mathbb{R}$.
	\flushright \href{https://artofproblemsolving.com/community/c6h307282}{(Link to AoPS)}
\end{problem}



\begin{solution}[by \href{https://artofproblemsolving.com/community/user/29428}{pco}]
	\begin{tcolorbox}Determine all continous functions $ f: \mathbb{R} \rightarrow \mathbb{R}$ which satisfy the relation:

$ 2f(x + y) + f(2x - y) + f(2y - x) = 9f(x) + 9f(y), (\forall) x,y \in \mathbb{R}.$\end{tcolorbox}

Let $ P(x,y)$ be the assertion $ 2f(x + y) + f(2x - y) + f(2y - x) = 9f(x) + 9f(y)$

$ P(0,0)$ $ \implies$ $ f(0) = 0$
$ P(x,x)$ $ \implies$ $ f(2x) = 8f(x)$
$ P(x,-x)$ $ \implies$ $ f(-x)=-f(x)$
$ P(x,2x)$ $ \implies$ $ f(3x) = 27f(x)$

It's very easy to check that $ f(x) = x^3$ is a solution.

Let then $ g(x)$ any solution. $ h(x) = g(x) - g(1)x^3$ is also a solution such that $ h(1) = h(0) = 0$
Then $ h(2) = 8h(1) = 0$ and $ h(3) = 27h(1) = 0$ and $ h(4) = 8h(2) = 0$ and so $ h( - 4) = h( - 3) = h( - 2) = h( - 1)$ $ = h(0) = h(1) = h(2) = h(3) = h(4) = 0$

$ P(\frac x2,1)$ may be written $ 2f(\frac x2 + 1) + f(x - 1) + f(2 - \frac x2) = 9f(\frac x2) + 9f(1)$
$ \iff$ $ 2f(x + 2) + 8f(x - 1) + f(4 - x) = 9f(x) + 72f(1)$
$ \iff$ $ f(x) = \frac 12(9f(x - 2) - 8f(x - 3) + f(x - 7) + 72f(1))$

Applying this to $ h(x)$, we get $ h(n) = 0$ $ \forall n\in\mathbb Z$ and so $ h(n2^p) = 0$ $ \forall n,p\in\mathbb Z$ and so $ h(x) = 0$ $ \forall x\in\mathbb R$ (continuity and considering that $ \{n2^p,n,p\in\mathbb Z\}$ is dense in $ \mathbb R$).

So $ g(x) = g(1)x^3$ and it's easy to check back that this indeed is a solution.

And the only solutions are $ \boxed{f(x) = ax^3}$
\end{solution}
*******************************************************************************
-------------------------------------------------------------------------------

\begin{problem}[Posted by \href{https://artofproblemsolving.com/community/user/36491}{Adriana N.}]
	Determine all real numbers $a$ such that the function \[ f(x) =\{ ax + \sin x\}\] is periodic. Here, $\{y\}$ is the fractional part of $y$.
	\flushright \href{https://artofproblemsolving.com/community/c6h307480}{(Link to AoPS)}
\end{problem}



\begin{solution}[by \href{https://artofproblemsolving.com/community/user/29428}{pco}]
	\begin{tcolorbox}Determine all real numbers $ a$ such that the function $ f(x) =${$ ax + sinx$}  is periodic.Here {$ y$} is the fractional part of $ y$.
P.S. Please explain,don't write only the value\/values of $ a$. Thank!\end{tcolorbox}

Let $ T$ be a period. Then $ f(x+T)=f(x)$ $ \forall x$

$ \iff$ $ a(x+T)+\sin(x+T)-ax-\sin(x)\in\mathbb Z$

$ \iff$ $ aT+2\sin(\frac T2)\sin(x+\frac T2)\in\mathbb Z$

Since $ aT+2\sin(\frac T2)\sin(x+\frac T2)$ is continuous, this means $ \sin(\frac T2)=0$ and $ aT\in\mathbb Z$

So $ T=2k\pi$ and $ a=\frac{n}{2k\pi}$

Hence the answer : $ \boxed{a\in\{\frac p{\pi},p\in\mathbb Q\}}$
\end{solution}



\begin{solution}[by \href{https://artofproblemsolving.com/community/user/36491}{Adriana N.}]
	Thank a lot!  :P
\end{solution}



\begin{solution}[by \href{https://artofproblemsolving.com/community/user/68920}{prester}]
	\begin{tcolorbox}Determine all real numbers $ a$ such that the function $ f(x) =${$ ax + sinx$}  is periodic.Here {$ y$} is the fractional part of $ y$.
P.S. Please explain,don't write only the value\/values of $ a$. Thank!\end{tcolorbox}

Using the definition of $ \{x\}$ we may write $ f(x) = ax + sinx - [ax + sinx]$
But since $ -1 \le \sin x \le 1$ we may finally write $ f(x) = \{ax\} +  \{\sin x\}$

Since $ \{\sin x\}$ is periodic with period $ 2\pi$ the function $ f(x)$ is periodic when $ a=0$

Consider now $ a \neq 0$. It is also $ \left\{a\left(x + \frac{k}{a}\right)\right\} = \{ax + k\} = \{ax\}\;\forall k \in \mathbb{Z}$
So the function $ \{ax\}$ is also periodic with period $ \;\frac1a$

Since $ f(x)$ is the sum of two periodic functions, in order to be periodic the ratio of two periods must be a rational number
So it means that   $ 2 \pi a$ must be a rational number

In conclusion, the infinite values of $ a$ are members of the real subset $ A=\{\frac{q}{\pi}, q \in \mathbb{Q} \}$
\end{solution}



\begin{solution}[by \href{https://artofproblemsolving.com/community/user/29428}{pco}]
	\begin{tcolorbox} Using the definition of $ \{x\}$ we may write $ f(x) = ax + sinx - [ax + sinx]$
But since $ - 1 \le \sin x \le 1$ we may finally write $ f(x) = \{ax\} + \{\sin x\}$...\end{tcolorbox}

Unfortunately, this is wrong. You cant write $ f(x) =\{ax + \sin x\}= \{ax\} + \{\sin x\}$ 

Choose for example $ a=\frac 3{\pi}$ and $ x= \frac{\pi}6$ : $ \{ax + \sin x\}= 0$ while $ \{ax\} + \{\sin x\}=1$
\end{solution}



\begin{solution}[by \href{https://artofproblemsolving.com/community/user/68920}{prester}]
	\begin{tcolorbox}[quote="prester"] Using the definition of $ \{x\}$ we may write $ f(x) = ax + sinx - [ax + sinx]$
But since $ - 1 \le \sin x \le 1$ we may finally write $ f(x) = \{ax\} + \{\sin x\}$...\end{tcolorbox}

Unfortunately, this is wrong. You cant write $ f(x) = \{ax + \sin x\} = \{ax\} + \{\sin x\}$ 

Choose for example $ a = \frac 3{\pi}$ and $ x = \frac {\pi}6$ : $ \{ax + \sin x\} = 0$ while $ \{ax\} + \{\sin x\} = 1$\end{tcolorbox}

Thank you  pco. You're obviosuly right. Sorry for mistake  :(
\end{solution}



\begin{solution}[by \href{https://artofproblemsolving.com/community/user/44887}{Mathias_DK}]
	\begin{tcolorbox}Determine all real numbers $ a$ such that the function $ f(x) =${$ ax + sinx$}  is periodic.Here {$ y$} is the fractional part of $ y$.
P.S. Please explain,don't write only the value\/values of $ a$. Thank!\end{tcolorbox}
If it is periodic then there exists a $ h \in \mathbb{R}$ s.t. $ f(x+h) = f(x) \iff$
$ \{ax+\sin x\} = \{ax+ha+\sin(h+x) \}$
$ \sin (h+x) = \cos x \sin h + \sin x \cos h$.
So: $ \{ax+\sin x\} = \{ax+ha+\sin(h+x) \} \iff ax+\sin x - (ax+ha+\sin(h+x)) \in \mathbb{Z} \iff$
$ p(x) = \sin x - \sin x \cos h + \cos x \sin h  - ha \in \mathbb{Z}$
Since $ p$ is continious we see that $ p$ is constant. 
So $ p'(x) = 0 \iff \cos x ( 1 - \cos h) - \sin x \sin h = 0 \forall x$
Assume that $ \cos h \neq 1$. Then $ p'(x) = 0 \iff \cot x = \frac{\sin h}{1-\cos h} \forall x$ which gives a contradiction. So $ \cos h = 1 \iff h = 2n\pi, n \in \mathbb{Z}$.
Then $ p(x) = -ha = -2n\pi a \in \mathbb{Z}$ so $ a = \frac{t}{\pi}$ for some $ t \in \mathbb{Q}$.
If $ a = \frac{p}{q\pi}, p,q \in \mathbb{Z}$ then $ f(x) = f(x+2q\pi)$. So they fit. Hence the answer is $ a = \frac{t}{\pi},t \in \mathbb{Q}$
\end{solution}
*******************************************************************************
-------------------------------------------------------------------------------

\begin{problem}[Posted by \href{https://artofproblemsolving.com/community/user/70793}{moriarti}]
	Let $n \geq 2$ be a positive integer. Find all functions $f: \mathbb R \to \mathbb R$ such that
\[f(x^n + f(y)) = f^n(x) + y\]
for any real $x$ and $y$.
	\flushright \href{https://artofproblemsolving.com/community/c6h307854}{(Link to AoPS)}
\end{problem}



\begin{solution}[by \href{https://artofproblemsolving.com/community/user/48552}{ocha}]
	$ f(f(0))=f^n(0)$ for all $ n\ge 2$ and since the LHS is constant we have $ f(0) = 1$ or $ 0 \qquad(1)$

Similarly, $ f(1+f(0))=f^n(1) \Rightarrow f(1) = 1$ or $ 0 \qquad(2)$

Also $ f(f(1))=f^n(0) + 1$ so from (1) and (2) it follows that $ f(0)=0, f(1)=1$
=========================================================

Let $ y=0$, $ f(x^n)=f^n(x)$, let $ y=1$, $ f(x^n+1) = f^n(x) + 1$

Subtracting the two equations gives $ f(x^n+1) = f(x^n)+1$

so let $ x^n=a$ and define $ g$ such that $ f(x)=x-g(x)$ 

Plugging into the equation gives $ f(a+1)=f(a)+1$ and $ g(a+1)=g(a) (*)$
=========================================================

Now let $ x=0$ and we find $ f(f(y))=y$

From (*) $ y-g(y) - g(y-g(y)) = y \Longrightarrow g(y) = -g(y-g(y)) \Longrightarrow 0=g(0)=g(-g(y))$ 

so $ g(y)=0$

Hence $ f(x)=x-g(x) = x$
\end{solution}



\begin{solution}[by \href{https://artofproblemsolving.com/community/user/29428}{pco}]
	\begin{tcolorbox}$ f(f(0)) = f^n(0)$ for all $ n\ge 2$ and ...\end{tcolorbox}

I think you misread the statement : $ n$ is fixed and considered as a parameter. So you cant write $ f(f(0)) = f^n(0)$ \begin{bolded}for all \end{underlined}\end{bolded}$ n\ge 2$
\end{solution}



\begin{solution}[by \href{https://artofproblemsolving.com/community/user/29428}{pco}]
	\begin{tcolorbox}Let $ 2 \le n \in N$. Find all functions f:R -> R such that

$ f(x^n + f(y)) = f^n(x) + y$

for any x, y.\end{tcolorbox}
Let $ P(x,y)$ be the assertion $ f(x^n+f(y))=f(x)^n+y$

If $ f(y_1)=f(y_2)$, and comparing $ P(x,y_1)$ and $ P(x,y_2)$ we get $ y_1=y_2$ and $ f(x)$ is injective.
$ P(x,u-f(x)^n)$ $ \implies$ $ f(\text{something})=u$ and $ f(x)$ is surjective, so is a bijection.

let then $ f(0)=a$ and $ b$ such $ f(b)=0$ and $ c$ such that $ f(c)=b$

$ P(0,x^n)$ $ \implies$ $ f(f(x^n))=x^n+a^n$
$ P(x,b)$ $ \implies$ $ f(x^n)=f(x)^n+b$ and so, using line above : $ f(f(x)^n+b)=x^n+a^n$
$ P(f(x),c)$ $ \implies$ $ f(f(x)^n+b)=f(f(x))^n+c$ and so, using line above : $ f(f(x))^n+c=x^n+a^n$
$ P(0,x)$ $ \implies$ $ f(f(x))=x+a^n$ and so, using line above : $ (x+a^n)^n+c=x^n+a^n$ $ \forall x$ and so $ a=b=c=0$

$ P(0,x)$ $ \implies$ $ f(f(x))=x$
$ P(x,0)$ $ \implies$ $ f(x^n)=f(x)^n$
$ P(x,f(y))$ $ \implies$ $ f(x^n+y)=f(x^n)+f(y)$

1) $ n$ is even
==============
$ f(x^n+y)=f(x^n)+f(y)$ $ \implies$ $ f(x+y)=f(x)+f(y)$ $ \forall x\geq 0$, $ \forall y$
Setting $ y=-x$ in the above equation, we get $ f(-x)=-f(x)$ $ \forall x\ge 0$ and so $ f(x+y)=f(x)+f(y)$ $ \forall x,y$
This is a Cauchy equation and we also have, using $ f(x^n)=f(x)^n$ and $ n$ even : $ f(x)\ge 0$ $ \forall x\ge 0$ 
So $ f(x)$ has a lower bound on $ [0,+\infty)$ and so $ f(x)=ax$ for some real $ a$
Plugging this back in the original equation, we get $ a=1$

And the solution $ f(x)=x$

2) $ n$ is odd
=============
$ f(x^n+y)=f(x^n)+f(y)$ $ \implies$ $ f(x+y)=f(x)+f(y)$ $ \forall x,y$
$ P(1,0)$ $ \implies$  $ f(1)=f(1)^n$ and, since $ f(1)\ne 0$ ($ f(0)=0$ and $ f(x)$ is injective) and $ n$ odd : $ f(1)=1$

So we get $ f(p)=p$ and $ f(px)=pf(x)$  $ \forall p\in\mathbb Z$ 

Consider then the $ n-1$ quantities $ y_k=f(x^k)$, $ k\in[2,n]$
$ f((x+p)^n)=f(x+p)^n$ is a linear equation of $ y_k$ 
Using as many different values of $ p$ as we need, it's easy to show that $ y_k=f(x)^k$ $ \forall k\in[2,n]$

So $ f(x^2)=f(x)^2$ and we can then use the demonstration in $ 1)$ above to conclude $ f(x)=x$


Hence the unique solution $ \boxed{f(x)=x}$
\end{solution}



\begin{solution}[by \href{https://artofproblemsolving.com/community/user/48552}{ocha}]
	oh sorry, how about this

First show that $ f$ is a bijection
$ f(a) = f(b) \Rightarrow f(a^n + f(a)) = f(a^n + f(b)) \Rightarrow f^n(a) + a = f^n(a) + b \Rightarrow a = b$, so $ f$ is injective

Also for any real number $ c$, we can fix $ x$ and let $ y = c - f^n(x)$, which gives $ f(x^n + f(c - f^n(x))) = c$ so $ f$ is surjective.
=========================

Now show that $ f(0) = 0$
let $ x = 0 \Longrightarrow f(f(y)) = y + f^n(0)$ Using this result

$ f(x^n + f(f(y))) = f^n(x) + f(y)$
$ \therefore f(x^n + y + f^n(0)) = f^n(x) + f(y)$
$ \therefore f(f(x^n + y + f^n(0)) = f(f^n(x) + f(y))$
$ \therefore x^n + y + 2f^n(0) = f^n(f(x)) + y = x + f^n(0) + y$
$ \therefore f^n(0) = 0 \Longrightarrow f(0) = 0$
==========================

Now we have $ f(f(x)) = x$ and $ f(x^n + f( - f^n(x))) = 0 \Longrightarrow x^n + f(f^n( - x)) = 0$ 
$ \therefore f( - x^n) = - f(x^n)$

\begin{bolded}For $ n$ even:\end{bolded}
If $ a > b$ then there exists an $ x > 0$ such that $ a = x^n + b$
$ \therefore f(a) = f(x^n + b) = f(x^n + f(f(b))) = f^n(x) + f(b) > f(b)$
So $ f$ is increasing, but since $ f(f(x)) = x$ we must have $ f(x) = x$


\begin{bolded}For $ n$ odd:\end{bolded}
$ f(x - f(x)) = f((x^\frac {1}{n})^n + f(-x)) = f(x) - x$

However if we have an $ a$ such that $ f(a) = - a$ then
$ f(a^n - a) = f(a^n + f(a)) = a - a = 0$
So $ a^n - a = 0$ and $ a = 0$ or $ 1$, so $ f(x) = x$ or $ x + 1$, the later leading to an obvious contradiction

So we have $ f(x) = x$ as the only solution
\end{solution}



\begin{solution}[by \href{https://artofproblemsolving.com/community/user/43923}{turbo taskytojas}]
	But when $ n$ is odd, function $ f(x)=-x$ also satisfies original equation. I think you missed something.
\end{solution}



\begin{solution}[by \href{https://artofproblemsolving.com/community/user/29428}{pco}]
	\begin{tcolorbox}But when $ n$ is odd, function $ f(x) = - x$ also satisfies original equation. I think you missed something.\end{tcolorbox}

Ooops,  :blush: , you're obvously right.

Here is the error in my own proof :
\begin{tcolorbox}2) $ n$ is odd
=============
$ f(x^n + y) = f(x^n) + f(y)$ $ \implies$ $ f(x + y) = f(x) + f(y)$ $ \forall x,y$
$ P(1,0)$ $ \implies$  $ f(1) = f(1)^n$ and, since $ f(1)\ne 0$ ($ f(0) = 0$ and $ f(x)$ is injective) and $ n$ odd : $ f(1) = 1$\end{tcolorbox}

$ f(1) = f(1)^n$ and $ f(1)\ne 0$ implies $ f(1)^{n-1}=1$ and since $ n$ odd, $ n-1$ is \begin{bolded}even \end{bolded}\end{underlined}and then $ f(1)=1$ or $ f(1)=-1$ (I missed this).

Thanks for your remark.
\end{solution}



\begin{solution}[by \href{https://artofproblemsolving.com/community/user/43923}{turbo taskytojas}]
	Still I am not sure I completely understood the way you showed $ y_{k}=f(x)^{k}$ in your solution. Changing $ p$ we can get a system of $ n-1$ nonequivalent linear equations in $ n-1$ variables $ y_{k}$. Am I right, claiming that this system has at most one unique solution and then just checking and confirming that solution $ y_{k}=f(x)^{k}$ satisfies equations?
\end{solution}



\begin{solution}[by \href{https://artofproblemsolving.com/community/user/62475}{hqthao}]
	@pco: thanks to give me this link, your solution is so nice, if compare with my solution I had read (it used more in analysis). :oops:
and I think you should put $ y_k=f(x^k)-f(x)^k$ so we have a linear system equation with variable is $ y_k$, so $ y_k=0$, it will be clearer. 
\end{solution}
*******************************************************************************
-------------------------------------------------------------------------------

\begin{problem}[Posted by \href{https://artofproblemsolving.com/community/user/70793}{moriarti}]
	Find all functions $f: \mathbb R^{+} \to \mathbb R^{+}$ such that
\[f(xf(x) + f(y)) = f^2(x) + y\]
for any $x, y>0$.
	\flushright \href{https://artofproblemsolving.com/community/c6h307855}{(Link to AoPS)}
\end{problem}



\begin{solution}[by \href{https://artofproblemsolving.com/community/user/29428}{pco}]
	\begin{tcolorbox}Find all functions $ f: R^ + - > R^ +$ such that

$ f(xf(x) + f(y)) = f^2(x) + y$

for any x, y>0.\end{tcolorbox}
Let $ P(x,y)$ be the assertion $ f(xf(x)+f(y))=f(x)^2+y$
Let $ m=\inf f(\mathbb R^+)$

$ f(y_1)=f(y_2)$ $ \implies$ (comparing $ P(x,y_1)$ and $ P(x,y_2)$) $ y_1=y_2$ and $ f(x)$ is injective.

$ \forall u>m^2$, $ \exists x$ such that $ m\leq f(x)<\sqrt u$. Then $ u-f(x)^2>0$ and $ P(x,u-f(x)^2)$ $ \implies$ $ f(\text{something})=u$ and so $ (m^2,+\infty)\subseteq f(\mathbb R^+)$

$ P(x,f(y)^2)$ $ \implies$ $ f(xf(x)+f(f(y)^2))=f(x)^2+f(y)^2$
$ P(y,f(x)^2)$ $ \implies$ $ f(yf(y)+f(f(x)^2))=f(y)^2+f(x)^2$
So $ f(xf(x)+f(f(y)^2))=f(yf(y)+f(f(x)^2))$ and, since $ f(x)$ is injective, $ xf(x)+f(f(y)^2)=yf(y)+f(f(x)^2)$
So $ xf(x)-f(f(x)^2)=yf(y)-f(f(y)^2)$
So $ xf(x)-f(f(x)^2)=a$ for some real $ a$

Then $ P(x,y)$ may be written $ f(f(f(x)^2)+f(y)+a)=f(x)^2+y$ and, since $ (m^2,+\infty)\subseteq f(\mathbb R^+)$ :

New assertion $ Q(x,y)$ : $ f(f(x)+f(y)+a)=x+y$ $ \forall x>m^4$,  $ \forall y>0$

Comparing then $ Q(x+z,y)$ and $ Q(x,y+z)$, we get $ f(f(x+z)+f(y)+a)=f(f(x)+f(y+z)+a)$ $ \forall x>m^4,y>0,z>0$ and, since $ f(x)$ is injective :

$ f(x+z)+f(y)=f(x)+f(y+z)$ $ \forall x>m^4,y>0,z>0$

So $ f(y+z)-f(y)=h(z)$ for some function $ h(z)$ independant of $ y$

But $ f(y+z)=f(y)+h(z)$ $ \forall y,z>0$ $ \implies$ $ f(y)+h(z)=f(z)+h(y)$ and so $ f(y)-h(y)=b$ for some real $ b$

So $ h(x)=f(x)-b$ and we got $ f(x+y)=f(x)+f(y)-b$ $ \forall x,y>0$

Let then $ g(x)=f(x)-b$ from $ \mathbb R+\to(-b,+\infty)$ : $ g(x+y)=g(x)+g(y)$
This is a Cauchy equation for a function which has a lower bound and so $ g(x)=cx$ (non continuous solutions have neither upper, neither lower bounds).

So $ f(x)=cx+b$ and, plugging this back in the original equation, we get $ c=1$ and $ b=0$

Hence the unique solution $ \boxed{f(x)=x}$
\end{solution}
*******************************************************************************
-------------------------------------------------------------------------------

\begin{problem}[Posted by \href{https://artofproblemsolving.com/community/user/70793}{moriarti}]
	Find all functions $f: \mathbb R^{\geq 0} \to \mathbb R^{\geq 0}$ such that
\[f(y)f(xf(y))=f(x+y)\]
for any non-negative real choice of $x$ and $y$.
	\flushright \href{https://artofproblemsolving.com/community/c6h307856}{(Link to AoPS)}
\end{problem}



\begin{solution}[by \href{https://artofproblemsolving.com/community/user/29428}{pco}]
	\begin{tcolorbox}Find all functions $ f: R^ + _0 - > R^ + _0$ such that

$ f(y)f(xf(y)) = f(x + y)$

for any x, y.\end{tcolorbox}

Let $ P(x,y)$ be the assertion $ f(y)f(xf(y))=f(x+y)$

$ f(x)=0$ $ \forall x$ and $ f(x)=1$ $ \forall x$ are obviously the only two constant solutions. Let's now consider non constant solutions.

1) Assume $ f(x)$ is a non constant solution and $ \exists u>0$ such that $ f(u)=0$
================================================================================
Since $ f(x)$ is non constant, $ \exists c\geq 0$ such that $ f(c)\ne 0$. Then $ P(0,c)$ $ \implies$ $ f(c)f(0)=f(c)$ and so $ f(0)=1$
Let $ x\ge u$ : $ P(x-u,u)$ $ \implies$ $ f(x)=0$ $ \forall x\ge u$

Let then $ a=\inf\{x>0$ such that $ f(x)=0\}$. We got $ f(x)=0$ $ \forall x>a$
If $ a=0$, we got a solution : $ f(0)=1$ and $ f(x)=0$ $ \forall x>0$ (easy to check back that this is a solution)
Consider now $ a>0$ (and so $ f(x)\ne 0$ $ \forall x\in[0,a)$
If $ f(a)\ne 0$, $ P(\frac a{f(a)},a)$ $ \implies$ $ f(a)^2=f(a+\frac a{f(a)})$ and we got a contradiction ($ LHS\ne 0$ while $ RHS=0$) so $ f(a)=0$.

Suppose now $ \exists y_1<y_2<a$ such that $ f(y_1)=f(y_2)$. Then :
Comparing $ P(x,y_1)$ and $ P(x,y_2)$, we get $ f(x+y_1)=f(x+y_2)$
Setting $ x=a-\frac{y_1+y_2}2$ in this equation implies $ f(a-\frac{y_2-y_1 }2)=f(a+\frac{y_2-y_1}2)$ but then $ RHS=0$ while $ LHS\ne 0$ and so $ f(x)$ is injective over $ [0,a)$

Let then $ x,y<a$ :
$ P(\frac x{f(y)},y$ $ \implies$ $ f(x)f(y)=f(\frac x{f(y)}+y)$ and so $ \frac x{f(y)}+y<a$ since $ LHS\ne 0$
$ P(\frac y{f(x)},x$ $ \implies$ $ f(y)f(x)=f(\frac y{f(x)}+x)$ and so $ \frac y{f(x)}+x<a$ since $ LHS\ne 0$
So $ f(\frac x{f(y)}+y)=f(\frac y{f(x)}+x)$ with both args $ <a$ and so (since $ f(x)$ is injective :
$ \frac x{f(y)}+y=\frac y{f(x)}+x$
$ \iff$ $ xf(x)+yf(x)f(y)=yf(y)+xf(x)f(y)$
$ \iff$ $ xf(x)(f(y)-1)=yf(y)(f(x)-1$
$ \iff$ $ xf(x)=c(f(x)-1)$ for some real $ c\ne 0$
$ \iff$ $ f(x)=\frac c{c-x}$ $ \forall x\in[0,a)$

Obviously, either $ c<0$, either $ c\ge a$ (else $ f(x)<0$ for some $ x\in[0,a)$

Plugging this back in initial equation and remembering that $ f(x)=0$ $ \forall x\ge a$ :
consider $ 0\le y< a\le x+y$, we need $ xf(y)\ge a$ (else $ LHS\ne 0$ while $ RHS= 0$) so $ \frac {cx}{c-y}\ge a$ $ \forall 0\le y< a\le x+y$ and so $ 0<c\le a$, so $ c=a$

And it's easy then to check that we got a solution.

2) Assume $ f(x)$ is a non constant solution and $ f(x)>0$ $ \forall x$
===================================================================
Since $ f(x)$ is non constant, $ \exists c\geq 0$ such that $ f(c)\ne 0$. Then $ P(0,c)$ $ \implies$ $ f(c)f(0)=f(c)$ and so $ f(0)=1$
Assume $ \exists u$ such that $ f(u)>1$
$ P(\frac u{f(u)-1},u)$ $ \implies$ $ f(u)f(\frac {uf(u)}{f(u)-1})=f(\frac {uf(u)}{f(u)-1})$ and so $ f(\frac {uf(u)}{f(u)-1})=0$, which is impossible since we assumed $ f(x)>0$ $ \forall x$, so $ f(x)\le 1$ $ \forall x$

Let then $ x<y$. Then $ P(y-x,x)$ $ \implies$ $ f(x)f((y-x)f(x))=f(y)$ and, since $ f((y-x)f(x))\le 1$, we get $ f(y)\le f(x)$ and $ f(x)$ is a non increasing function.

If $ \exists u>0$ such that $ f(u)=1$, then $ P(u,u)$ $ \implies$ $ f(2u)=1$ and so $ f(2^nu)=1$ and, since $ f(x)$ is non increasing and $ f(0)=1$ : $ f(x)=1$ $ \forall x$, which is impossible (since we assume $ f(x)$ is a non constant solution).

So $ f(x)<1$ $ \forall x>0$ and $ f(x)$ is a decreasing function and, so, is injective.
Then :
$ P(\frac x{f(y)},y$ $ \implies$ $ f(x)f(y)=f(\frac x{f(y)}+y)$ 
$ P(\frac y{f(x)},x$ $ \implies$ $ f(y)f(x)=f(\frac y{f(x)}+x)$ 
So $ f(\frac x{f(y)}+y)=f(\frac y{f(x)}+x)$ and so (since $ f(x)$ is injective) :
$ \frac x{f(y)}+y=\frac y{f(x)}+x$
$ \iff$ $ xf(x)+yf(x)f(y)=yf(y)+xf(x)f(y)$
$ \iff$ $ xf(x)(f(y)-1)=yf(y)(f(x)-1$
$ \iff$ $ xf(x)=c(f(x)-1)$ for some real $ c\ne 0$
$ \iff$ $ f(x)=\frac c{c-x}$ $ \forall x$

In order to have $ f(x)\ge 0$ $ \forall x$, we need $ c<0$ and(setting $ b=-c>0$) : $ f(x)=\frac b{x+b}$ $ \forall x$
Plugging this back in the initial equation, we check that this indeed a solution.

3) synthesis of solutions :
===========================

$ f(x)=0$ $ \forall x$

$ f(x)=1$ $ \forall x$

$ f(0)=1$ and $ f(x)=0$ $ \forall x>0$

$ f(x)=\frac a{a-x}$ $ \forall x\in[0,a)$ and $ f(x)=0$ $ \forall x\ge a$ for some real $ a>0$

$ f(x)=\frac a{x+a}$ $ \forall x$ for any real $ a>0$
\end{solution}
*******************************************************************************
-------------------------------------------------------------------------------

\begin{problem}[Posted by \href{https://artofproblemsolving.com/community/user/70793}{moriarti}]
	Find all functions $f: \mathbb N \to \mathbb N$ such that
\[ 2n \le f(f(n)) + f(n) \le 2n + 1\]
for any $n \in \mathbb N$.
	\flushright \href{https://artofproblemsolving.com/community/c6h307859}{(Link to AoPS)}
\end{problem}



\begin{solution}[by \href{https://artofproblemsolving.com/community/user/29428}{pco}]
	\begin{tcolorbox}Find all functions $ f: N - > N$ such that

$ 2n \le f(f(n)) + f(n) \le 2n + 1$

for any n.\end{tcolorbox}

First, we can see that $ f(n)$ is injective :
Let $ n_1<n_2$ such that $ f(n_1)=f(n_2)$. Then $ f(f(n_1))+f(n_1) \le 2n_1+1$ $ < 2n_2\le f(f(n_2))+f(n_2)$ which is impossible since $ f(f(n_1))+f(n_1) = f(f(n_2))+f(n_2)$. Hence $ f(n)$ is injective.

With $ n=1$, we get $ 2\le f(f(1))+f(1) \le 3$
$ f(f(1))>0$ and $ f(1)>0$, so $ 1\le f(1)\le 2$
If $ f(1)=2$, then $ 2\le f(2)+2 \le 3$ and so $ f(2)=1$

But then, setting $ n=2$ in the original inequation, we get $ 4\le f(f(2))+f(2)\le 5$ $ \iff$ $ 4\le f(1)+1=3 \le 5$, which is impossible, so $ f(1)=1$

We can extend this demo thru induction :
Suppose $ f(p)=p$ $ \forall p<n$. Then :
Since $ f(n)$ is injective, $ f(x)>n-1$ $ \forall x\ge n$ (and so $ f(n)\ge n$ and $ f(n+1)\ge n$)
We know that $ 2n\le f(f(n))+f(n)\le 2n+1$
If $ f(n)>n+1$, we get $ f(f(n))<n$, which is impossible since $ f(n)>n+1$ implies $ f(f(n))>n-1$
If $ f(n)=n+1$, then $ 2n\le f(f(n))+f(n)\le 2n+1$ implies $ n-1\le f(n+1)\le n$ and so $ f(n+1)=n$

But then, since $ 2n+2\le f(f(n+1))+f(n+1)\le 2n+3$, we get $ 2n+2\le (n+1)+n\le 2n+3$, which is wrong.
So $ f(n)=n$

And, since $ f(1)=1$, the induction is terminated.

Hence the unique solution $ \boxed{f(n)=n}$ $ \forall n$
\end{solution}
*******************************************************************************
-------------------------------------------------------------------------------

\begin{problem}[Posted by \href{https://artofproblemsolving.com/community/user/70793}{moriarti}]
	Find all functions $f: \mathbb R \to \mathbb R$ such that
\[ f(x) \ge x + 1\] and \[ f(x+y) \ge f(x)f(y)\] for all reals $x$ and $y$.
	\flushright \href{https://artofproblemsolving.com/community/c6h307861}{(Link to AoPS)}
\end{problem}



\begin{solution}[by \href{https://artofproblemsolving.com/community/user/29428}{pco}]
	\begin{tcolorbox}Find all functions $ f: R - > R$ such that

$ f(x) \ge x + 1$ and $ f(x + y) \ge f(x)f(y)$ for any x, y.\end{tcolorbox}

$ f(\frac x2+\frac x2)\ge f(\frac x2)^2\ge 0$ and so $ f(x)\ge 0$

If $ f(u)=0$, then $ f(-\frac 12)f(u+\frac 12)\le f(u)=0$ and so $ f(-\frac 12)f(u+\frac 12)=0$. But $ f(-\frac 12)\ge -\frac 12+1>0$ and so $ f(u+\frac 12)=0$
So $ f(u+\frac n2)=0$ and this is impossible since $ f(x)\ge x+1>0$ $ \forall x>-1$

So $ f(x)>0$ $ \forall x$

$ f(0+0)\ge f(0)^2$ and so $ f(0)\le 1$ but $ f(0)\ge 0+1$ and so $ f(0)=1$
$ \forall y>0$, $ f(y)\ge y+1>1$ and so $ f(x+y)\ge f(x)f(y)>f(x)$ (since $ f(x)>0$ and $ f(y)>1$ and so $ f(x)$ is increasing.

Let $ x>0$ : $ f(x+y)\ge f(x)f(y)\ge f(y)(x+1)$ and so $ \frac{f(x+y)-f(y)}x\ge f(y)$
Let $ x>0$ : $ f(y)\ge f(-x)f(x+y)\ge f(x+y)(-x+1)$ and so $ \frac{f(x+y)-f(y)}{x}\le f(x+y)$

So $ f(x+y)\ge \frac{f(x+y)-f(y)}x\ge f(y)$

Setting $ x\to 0$ in this inequality, and remembering that $ f(x)$ is increasing, we get that $ \lim_{x\to 0}f(x+y)=f(y)$ and so $ f(x)$ is continuous.

More, since $ f(x)$ is continous, we see that $ \lim_{x\to 0}\frac{f(x+y)-f(y)}x=f(y)$ and so $ f(x)$ is $ C_1$ and $ f'(x)=f(x)$
And, since $ f(0)=1$, we get $ f(x)=e^x$

And it is easy to check that this mandatory condition is enough and that this indeed is a solution.

Hence the unique solution : $ \boxed{f(x)=e^x}$ $ \forall x$
\end{solution}
*******************************************************************************
-------------------------------------------------------------------------------

\begin{problem}[Posted by \href{https://artofproblemsolving.com/community/user/70793}{moriarti}]
	Find all functions $f: \mathbb R \to \mathbb R$ such that $ f(0) \ge 0$ and
\[ f(x+y) \ge f(x) + yf(f(x))\]
for any reals $x$ and $y$.
	\flushright \href{https://artofproblemsolving.com/community/c6h307863}{(Link to AoPS)}
\end{problem}



\begin{solution}[by \href{https://artofproblemsolving.com/community/user/29428}{pco}]
	\begin{tcolorbox}Find all functions $ f: R - > R$ such that $ f(0) \ge 0$ and
$ f(x + y) \ge f(x) + yf(f(x))$
for any x, y.\end{tcolorbox}
I hope somebody got a simpler solution than mine ...  :blush: 

Let $ P(x,y)$ be the assertion $ f(x+y)\geq f(x)+yf(f(x))$

1) $ f(x)$ is differentiable, and $ f'(x)=f(f(x))$ is non decreasing
==================================================================
$ P(x,y)$ $ \implies$ $ f(x+y)\geq f(x)+yf(f(x))$ $ \implies$ $ f(x+y)-f(x)\geq yf(f(x))$
$ P(x+y,-y)$ $ \implies$ $ f(x)\geq f(x+y)-yf(f(x+y))$ $ \implies$ $ yf(f(x+y))\geq f(x+y)-f(x)$

And so a new assertion $ Q(x,y)$ : $ yf(f(x+y))\geq f(x+y)-f(x)\geq yf(f(x))$

From this, we can first conclude that $ f(f(x+y))\geq f(f(x))$ $ \forall y>0$ and so $ f(f(x))$ is a non decreasing function
Then we can write :
$ yf(f(x+1))\geq yf(f(x+y))\geq f(x+y)-f(x)\geq yf(f(x))$ $ \forall y\in[0,1]$
$ yf(f(x-1))\geq yf(f(x+y))\geq f(x+y)-f(x)\geq yf(f(x))$ $ \forall y\in[-1,0]$
And so $ \lim_{y\to 0}(f(x+y)-f(x))=0$ and $ f(x)$ is continuous
And so $ \lim_{y\to 0}\frac{f(x+y)-f(x)}y=f(f(x))$ and so $ f(x)$ is $ C_1$ and $ f'(x)=f(f(x))$
Q.E.D.

2) $ f(x)$ is monotonous
=======================
Since $ f'(x)=f(f(x))$ is non decreasing :
either $ f'(x)=f(f(x))\geq 0$ $ \forall x$ and $ f(x)$ is monotonous non decreasing
either $ f'(x)=f(f(x))\leq 0$ $ \forall x$ and $ f(x)$ is monotonous non increasing
either $ f'(x)=f(f(x))<0$ for some values and $ f'(x)=f(f(x))>0$ for some other values. In such a case :
$ f(x)$ is decreasing then increasing. But then we can find infinitely many couples $ (a,b),a\ne b$ such that $ f(a)=f(b)$

Then $ Q(a,b-a)$ $ \implies$ $ (b-a)f(f(b))\geq f(b)-f(a)\geq (b-a)f(f(a))$ and, since $ f(a)=f(b)$ : $ f(f(a))=f(f(b))=0$
Then $ P(a,x-a)$ $ \implies$ $ f(x)\geq f(a)$ $ \forall x$ and so $ f(a)=f(b)$ is the absolute minimum of $ f(x)$

But, if $ f(x)$ is decreasing then increasing, we can find two couples $ (a,b),a\ne b$ and $ (c,d),c\ne d$ such that $ f(a)=f(b)\ne f(c)=f(d)$ Hence contradiction.

So $ f(x)$ is monotonous
Q.E.D

3) $ f(x)$ is non decreasing
===========================
Suppose now that $ f(x)$ is a non increasing function which is somewhere decreasing.
So $ f'(x)=f(f(x))$ is somewhere $ <0$, so $ \exists u=f(v)$ such that $ f(f(v))<0$. 
Since $ f(0)\geq 0$  and $ f(u)<0$, $ \exists a$ such that $ f(a)=0$
Since $ f(x)$ is non increasing, $ f'(a)=f(f(a))\leq 0$ and so $ f(0)\leq 0$ and so $ f(0)=0$
But then $ P(0,x)$ $ \implies$ $ f(x)\geq 0$ $ \forall x$, hence a contradiction with $ f(f(v))<0$
So $ f(x)$ is non decreasing.
Q.E.D.

4) $ f(x)=0$ $ \forall x$
=======================
Since $ f(x)$ is non decreasing, either $ \lim_{x\to +\infty}f(x)=+\infty$, either $ \lim_{x\to +\infty}f(x)=c$
If $ \lim_{x\to +\infty}f(x)=+\infty$, then $ \lim_{x\to +\infty}f(f(x))=+\infty$ and so, $ f(f(x))>0$ $ \forall x>A$
But then $ P(x,f(x)-x)$ $ \implies$ $ f(f(x))\ge f(x)+(f(x)-x)f(f(x))$ $ \ge (f(x)-x)f(f(x))$ for $ x$ great enough and so :
$ f(x)\le x+1$ for $ x$ great enough but this is a contradiction with $ \lim_{x\to +\infty}f'(x)=+\infty$ (remember $ f'(x)=f(f(x))$)

So $ \lim_{x\to +\infty}f(x)=c$ and $ \lim_{x\to +\infty}f'(x)=0$ and so $ f(c)=0$
But then $ c\le 0$ (since $ f(0)\ge 0$ and $ f(x)$ non decreasing) and the only possibility is $ c=0$ and $ f(x)=0$ $ \forall x\ge 0$

But we saw that $ f(a)=f(b)$ implies that this value is the absolute minimum of $ f(x)$
So $ f(x)\ge 0$ $ \forall x$
So $ f(x)=0$ $ \forall x$ 
Q.E.D.
\end{solution}



\begin{solution}[by \href{https://artofproblemsolving.com/community/user/44674}{Allnames}]
	Case 1: $ f(0) > 0$
If $ f(f(x)) \le 0$ for all $ x$.Then for all $ y\le 0$ we have $ yf(f(x))\ge 0$ implies $ f(x + y)\ge f(x) + yf(f(x))$ for all $ x$.Thus $ f(x)$ is decreasing.
Because $ f(0) > 0 > f(f(x))$ so $ 0 < f(x)$ for all $ x$.Contracts to  $ f(f(x)) < 0$
Then there exists $ z$ such that $ f(f(z)) > 0$.
From our inequality $ f(x + z)\ge f(z) + xf(f(x))\rightarrow lim_{x\to \infty} f(x) = + \infty$
Then of course $ lim_{x\to \infty}f(f(x)) = + \infty$
Now we can choose $ x; y > 0$ satisfying $ f(x)\ge 0; f(f(x)) > 1; y\ge \frac {x + 1}{f(f(x)) - 1}$ and $ f(f(x + y + a)) \ge 0$
From it , we obtain $ f(x + y)\ge f(x) + yf(f(x))\ge x + y + 1$.
Repeat it, 
$ f(f(x + y))\ge f(x + y + 1) + [f(x + y) - (x + y + 1)](f(f(x + y + 1)) \ge f(x + y + 1)\ge f(x + y) + f(f(x + y))\ge f(x) + y(ff(x)) + f(f(x + y)) > f(f(x + y))$
Hence contradiction.
Case 2: $ f(0) = 0$
I am trying.But I think it is also hard!
Edited: 
Yeah! Case $ f(0)=0$ is rather easy.
When $ x=0$ then $ f(y)\ge 0$ for all $ y\in \mathbb R$
For all $ y\ge 0$ then choose $ x=-y$ we obtain $ f(0)=0\ge f(-y)+ yf(f(-y))$ which holds if only if $ f(x)=0 \forall x$.
In conclusion. There is unique function $ f(x)=0\forall x$
\end{solution}



\begin{solution}[by \href{https://artofproblemsolving.com/community/user/29428}{pco}]
	\begin{tcolorbox}Case 1: $ f(0) > 0$
If $ f(f(x)) \le 0$ for all $ x$.Then for all $ y\le 0$ we have $ yf(f(x))\ge 0$ implies $ f(x + y)\ge f(x) + yf(f(x))$ for all $ x$.Thus $ f(x)$ is decreasing.
Because $ f(0) > 0 > f(f(x))$ so $ 0 < f(x)$ for all $ x$.Contracts to  $ f(f(x)) < 0$
Then there exists $ z$ such that $ f(f(z)) > 0$.
From our inequality $ f(x + z)\ge f(z) + xf(f(x))\rightarrow lim_{x\to \infty} f(x) = + \infty$
Then of course $ lim_{x\to \infty}f(f(x)) = + \infty$
Now we can choose $ x; y > 0$ satisfying $ f(x)\ge 0; f(f(x)) > 1; y\ge \frac {x + 1}{f(f(x)) - 1}$ and $ f(f(x + y + a)) \ge 0$
From it , we obtain $ f(x + y)\ge f(x) + yf(f(x))\ge x + y + 1$.
Repeat it, 
$ f(f(x + y))\ge f(x + y + 1) + [f(x + y) - (x + y + 1)](f(f(x + y + 1)) \ge f(x + y + 1)\ge f(x + y) + f(f(x + y))\ge f(x) + y(ff(x)) + f(f(x + y)) > f(f(x + y))$
Hence contradiction.
Case 2: $ f(0) = 0$
I am trying.But I think it is also hard!
Edited: 
Yeah! Case $ f(0) = 0$ is rather easy.
When $ x = 0$ then $ f(y)\ge 0$ for all $ y\in \mathbb R$
For all $ y\ge 0$ then choose $ x = - y$ we obtain $ f(0) = 0\ge f( - y) + yf(f( - y))$ which holds if only if $ f(x) = 0 \forall x$.
In conclusion. There is unique function $ f(x) = 0\forall x$\end{tcolorbox}

Quite OK for me (just replace in the middle $ f(f(x + y + a)) \ge 0$ with $ f(f(x + y + 1)) \ge 0$)
Quite nice !
Congrats  :)
\end{solution}
*******************************************************************************
-------------------------------------------------------------------------------

\begin{problem}[Posted by \href{https://artofproblemsolving.com/community/user/64682}{KDS}]
	Let $ k$ be a positive integer. Find all functions $ f: \mathbb  N \to \mathbb N$ satisfying 
\[f(f(n))+f(n)=2n+3k\] for all positive integers $n$.
	\flushright \href{https://artofproblemsolving.com/community/c6h308067}{(Link to AoPS)}
\end{problem}



\begin{solution}[by \href{https://artofproblemsolving.com/community/user/29428}{pco}]
	\begin{tcolorbox}Let $ k$ be a positive integer.Find all functions $ f: N \to N$ satisfying $ f(f(n)) + f(n) = 2n + 3k$ $ \forall n \in N$.\end{tcolorbox}

Let $ f^{[p]}(n) = a_pf(n) + b_pn + c_pk$ $ \forall n > 0$
We get $ f^{[p + 1]}(n) = f^{[p]}(f(n)) = a_pf(f(n)) + b_pf(n) + c_pk$ $ = a_p( - f(n) + 2n + 3k) + b_pf(n) + c_pk$ $ = (b_p - a_p)f(n) + 2a_pn + (c_p + 3a_p)k$
 hence the equations :
$ a_{p + 1} = b_p - a_p$
$ b_{p + 1} = 2a_p$
$ c_{p + 1} = c_p + 3a_p$

And the result : $ f^{[p]}(n) = \frac {1 - ( - 2)^p}3f(n) + \frac {2 + ( - 2)^p}3n + (p - \frac {1 - ( - 2)^p}3)k$ $ \forall n > 0$, $ \forall p\geq 0$

Setting $ p = 2q$ and writing $ f^{[p]}(n) > 0$, we get $ f(n) < \frac {4^q + 2}{4^q - 1}n + (1 + \frac {6q}{4^q - 1})k$
Using $ q\to + \infty$ in this line, we get $ f(n)\leq n + k$

Setting $ p = 2q + 1$ and writing $ f^{[p]}(n) > 0$, we get $ f(n) > \frac {2\cdot 4^q - 2}{2\cdot 4^q + 1}n + (1 - \frac {6q + 3}{2\cdot 4^q + 1})k$
Using $ q\to + \infty$ in this line, we get $ f(n)\geq n + k$

So $ f(n) = n + k$ which, indeed, is a solution.

So the unique solution : $ \boxed{f(n) = n + k}$
\end{solution}
*******************************************************************************
-------------------------------------------------------------------------------

\begin{problem}[Posted by \href{https://artofproblemsolving.com/community/user/70887}{rogueknight}]
	Find all functions $f: \mathbb R \to \mathbb R$ which satisfies
\[f(f(x+y))=f(x+y)+f(x) \cdot f(y)-xy\]
for all reals $x$ and $y$.
	\flushright \href{https://artofproblemsolving.com/community/c6h308263}{(Link to AoPS)}
\end{problem}



\begin{solution}[by \href{https://artofproblemsolving.com/community/user/42763}{bpgbcg}]
	Switching x and y in the equation, $ f(f(x+y))=f(x+y)+f(x)f(y)-y$, which in combination with the original equation yields x=y, which is clearly not true for all real x and y.
\end{solution}



\begin{solution}[by \href{https://artofproblemsolving.com/community/user/70887}{rogueknight}]
	I'm sorry. I have fixed the problem  :blush:
\end{solution}



\begin{solution}[by \href{https://artofproblemsolving.com/community/user/29428}{pco}]
	\begin{tcolorbox}find all f that: f R->R:
$ f(f(x + y)) = f(x + y) + f(x)f(y) - xy$\end{tcolorbox}

Let $ P(x,y)$ be the assertion $ f(f(x + y)) = f(x + y) + f(x)f(y) - xy$

$ P(x + y,0)$ $ \implies$ $ f(f(x + y)) = f(x + y) + f(x + y)f(0)$ and so (subtracting $ P(x,y)$) we get the new assertion $ Q(x,y)$ : $ f(x + y)f(0) = f(x)f(y) - xy$

$ Q(f(0), - f(0))$ $ \implies$ $ f(f(0))f( - f(0)) = 0$ and so $ \exists u$ such that $ f(u) = 0$ and then $ Q(x - u,u)$ $ \implies$  $ f(x)f(0) = u(u - x)$

If $ f(0)\neq 0$ $ f(x) = ax + b$ and plugging this in original equation, we get $ f(x) = x$, which is wrong since then $ f(0) = 0$ but we supposed $ f(0)\ne 0$ in this line

So $ f(0) = 0$ and $ Q(x,y)$ becomes $ f(x)f(y) = xy$ and so either $ f(x) = x$, either $ f(x) = - x$ and plugging this in original equation, we get $ f(x) = x$

hence the unique solution $ \boxed{f(x) = x}$ $ \forall x$
\end{solution}
*******************************************************************************
-------------------------------------------------------------------------------

\begin{problem}[Posted by \href{https://artofproblemsolving.com/community/user/68025}{Pirkuliyev Rovsen}]
	Does there exist a function $f$ such that for all $x$ the following equality holds?  
\[ f(f(x)) = \begin{cases}
\sqrt{2009}, & \mbox{if } x\in \mathbb Q,\\
 2009, &\mbox{if } x\in\mathbb{R} \setminus \mathbb Q.
\end{cases}
\]
	\flushright \href{https://artofproblemsolving.com/community/c6h308559}{(Link to AoPS)}
\end{problem}



\begin{solution}[by \href{https://artofproblemsolving.com/community/user/29428}{pco}]
	\begin{tcolorbox}Does there exist a function f such that for all x the following equality holds  $ f(f(x)) = \left\{\begin{array}{l} \sqrt {2009} ,x\in Q \\
2009,x\in\mathbb{R} \end{array}\right.$  :?:\end{tcolorbox}

I suppose you mean $ f(x)=2009$ $ \forall x\in\mathbb R\backslash \mathbb Q$ (and not $ \mathbb R$). If so, no such function exists :

Let $ g(x)$ be the function defined thru $ f(\mathbb Q)=\{\sqrt{2009}\}$ and $ f(\mathbb R\backslash \mathbb Q)=\{2009\}$ 
We have $ f(f(x))=g(x)$ and so $ f(g(x))=g(f(x))$ and so $ f(2009)\in\{\sqrt{2009},2009\}$ and $ f(\sqrt{2009})\in\{\sqrt{2009},2009\}$

Obviously $ f(x)\neq x$ $ \forall x$ (since $ g(x)\neq x$ $ \forall x$) and so $ f(2009)\in\{\sqrt{2009},2009\}$ and $ f(\sqrt{2009})\in\{\sqrt{2009},2009\}$ implies :
$ f(2009)=\sqrt{2009}$ and $ f(\sqrt{2009})=2009$ but then $ f(f(2009))=2009\ne g(2009)=\sqrt{2009}$

So no such function.
\end{solution}
*******************************************************************************
-------------------------------------------------------------------------------

\begin{problem}[Posted by \href{https://artofproblemsolving.com/community/user/50645}{stvs_f}]
	Find all functions $ f: \mathbb N \to \mathbb R$ such that $ f(x)=f(x+22)$ for all $x \in \mathbb N$ and
\[f(x^2y)=(f(x))^2 \cdot f(y)\]
for all positive integers $x$ and $y$.
	\flushright \href{https://artofproblemsolving.com/community/c6h309049}{(Link to AoPS)}
\end{problem}



\begin{solution}[by \href{https://artofproblemsolving.com/community/user/29428}{pco}]
	\begin{tcolorbox}find all $ f: N \to R$ such that :
$ f(x) = f(x + 22)$
$ \forall x,y \in N: f(x^2y) = (f(x))^2.f(y)$\end{tcolorbox}

Let $ P(x,y)$ be the assertion $ f(x^2y)=f(x)^2f(y)$

$ P(x,x)$ $ \implies$ $ f(x^3)=f(x)^3$ and, since $ f(x)$ is periodic, $ f(\mathbb N)$ may contain at most $ 22$ different values, so $ f(x)\in\{-1,0,1\}$ $ \forall x$

If $ f(1)=0$ $ P(1,x)$ $ \implies$ $ f(x)=0$ $ \forall x$, which indeed is a solution :
$ S_1$ : $ f(\{1...22\} )=\{0,0,0,0,0,0,0,0,0,0,0,$ $ 0,0,0,0,0,0,0,0,0,0,0\}$

If $ f(x)$ is a solution, $ -f(x)$ is also a solution, so WLOG consider from now $ f(1)=1$

If $ f(11)\neq 0$, then $ f(11)\in\{-1,1\}$ and $ f(11)^2=1$. then $ P(11,x)$ $ \implies$ $ f(121x)=f(x)$ and $ P(11,x+2)$ $ \implies$ $ f(121x+242)=f(x+2)$ and so $ f(x)=f(x+2)$ and so $ f(x)=1$ $ \forall$ odd $ x$ and we got two other solutions :
$ S_2$ : $ f(\{1...22\})=\{1,1,1,1,1,1,1,1,1,1,1,$ $ 1,1,1,1,1,1,1,1,1,1,1\}$
$ S_3$ : $ f(\{1...22\})=\{1,0,1,0,1,0,1,0,1,0,1,$ $ 0,1,0,1,0,1,0,1,0,1,0\}$
And the solutions $ -f(x)$ :
$ S_3$ : $ f(\{1...22\})=\{-1,-1,-1,-1,-1,-1,-1,-1,-1,-1,-1,$ $ -1,-1,-1,-1,-1,-1,-1,-1,-1,-1,-1\}$
$ S_4$ : $ f(\{1...22\})=\{-1,0,-1,0,-1,0,-1,0,-1,0,-1,$ $ 0,-1,0,-1,0,-1,0,-1,0,-1,0\}$

Then $ P(x,1)$ $ \implies$ $ f(x^2)=f(x)^2$ and so $ f(x^2)\in\{0,1\}$
If $ x$ is invertible in $ \mathbb Z\/22\mathbb Z$, then $ P(x,\frac 1{x^2})$ $ \implies$ $ 1=f(x^2)f(\frac 1{x^2})$ and so $ f(x^2)\neq 0$ and so $ f(x^2)=1$
So $ f(1)=f(3)=f(5)=f(9)=f(15)=1$

If $ f(2)\neq 0$, then $ f(2)\in\{-1,1\}$ and $ f(2)^2=1$. then $ P(2,x)$ $ \implies$ $ f(4x)=f(x)$ and $ P(4,x+11)$ $ \implies$ $ f(4x+44)=f(x+11)$ and so $ f(x)=f(x+11)$ 
So $ f(4x)=f(x)$ $ \implies$ $ f(4)=f(12)=f(20)=f(14)=f(16)=1$
And $ f(x+11)=f(x)$ $ \implies$ $ f(12)=f(14)=f(17)=f(21)=f(4)=1$
$ P(x,2)$ $ \implies$ $ f(2x^2)=f(2)f(x)^2$ and so $ f(2)=f(6)=f(10)=f(18)=f(8)$ and $ f(x+11)=f(x)$ $ \implies$ $ f(2)=f(13)=f(17)=f(21)=f(7)=f(15)$ and so the solutions :
$ S_5$ : $ f(\{1...22\})=\{1,1,1,1,1,1,1,1,1,1,0,$ $ 1,1,1,1,1,1,1,1,1,1,0\}$
$ S_6$ : $ f(\{1...22\})=\{1,-1,1,1,1,-1,-1,-1,1,-1,0,$ $ 1,-1,1,1,1,-1,-1,-1,1,-1,0\}$

If $ f(2)=0$, we get $ f(4x)=0$ and so $ f(x)=0$ $ \forall$ even $ x$
Since we got $ f(1)=f(3)=f(5)=f(9)=f(15)=1$, it's easy to show $ f(21)=f(19)=f(17)=f(13)=f(7)$ and the solutions :
$ S_7$ : $ f(\{1...22\})=\{1,0,1,0,1,0,1,0,1,0,0,$ $ 0,1,0,1,0,1,0,1,0,1,0\}$
$ S_8$ : $ f(\{1...22\})=\{1,0,1,0,1,0,-1,0,1,0,0,$ $ 0,-1,0,1,0,-1,0,-1,0,-1,0\}$
$ S_9$ : $ f(\{1...22\})=\{-1,0,-1,0,-1,0,-1,0,-1,0,0,$ $ 0,-1,0,-1,0,-1,0,-1,0,-1,0\}$
$ S_{10}$ : $ f(\{1...22\})=\{-1,0,-1,0,-1,0,1,0,-1,0,0,$ $ 0,1,0,-1,0,1,0,1,0,1,0\}$

Hence ten solutions.
\end{solution}
*******************************************************************************
-------------------------------------------------------------------------------

\begin{problem}[Posted by \href{https://artofproblemsolving.com/community/user/33535}{wangsacl}]
	Find all functions $ f: \mathbb{R}\rightarrow\mathbb{R}$ such that
\[ f(x^3+y^3)=xf(x^2)+yf(y^2)\]
for all $ x,y\in\mathbb{R}$.
	\flushright \href{https://artofproblemsolving.com/community/c6h309219}{(Link to AoPS)}
\end{problem}



\begin{solution}[by \href{https://artofproblemsolving.com/community/user/29428}{pco}]
	\begin{tcolorbox}Find all functions $ f: \mathbb{R}\rightarrow\mathbb{R}$ such that
\[ f(x^3 + y^3) = xf(x^2) + yf(y^2)\]
for all $ x,y\in\mathbb{R}$\end{tcolorbox}

Let $ P(x,y)$ be the assertion $ f(x^3+y^3)=xf(x^2)+yf(y^2)$
Let $ f(1)=a$

$ P(x,0)$ $ \implies$ $ f(x^3)=xf(x^2)$ and so $ f(x^3+y^3)=f(x^3)+f(y^3)$ and so $ f(x+y)=f(x)+f(y)$ and so $ f(px)=pf(x)$ $ \forall p\in\mathbb Q$

$ P(x+1,0)$ $ \implies$ $ f(x^3+3x^2+3x+1)=(x+1)f(x^2+2x+1)$ and so new assertion $ Q(x)$ : $ 2f(x^2)=(2x-1)f(x)+ax$

$ Q(x+1)$ $ \implies$ $ 2f(x^2+2x+1)=(2x+1)f(x+1)+a(x+1)$ and so $ 2f(x^2)=(2x-3)f(x)+3ax$

Subtracting $ Q(x)$ from $ Q(x+1)$, we get $ 0=-2f(x)+2ax$ and so $ f(x)=ax$ and it's immediate to check back that this indeed is a solution.

Hence the unique solution $ \boxed{f(x)=ax}$
\end{solution}



\begin{solution}[by \href{https://artofproblemsolving.com/community/user/33535}{wangsacl}]
	\begin{tcolorbox}

$ P(x + 1,0)$ $ \implies$ $ f(x^3 + 3x^2 + 3x + 1) = (x + 1)f(x^2 + 2x + 1)$ and so new assertion $ Q(x)$ : $ 2f(x^2) = (2x - 1)f(x) + ax$
\end{tcolorbox}
I don't get this part  :blush:
\end{solution}



\begin{solution}[by \href{https://artofproblemsolving.com/community/user/29428}{pco}]
	\begin{tcolorbox}[quote="pco"]

$ P(x + 1,0)$ $ \implies$ $ f(x^3 + 3x^2 + 3x + 1) = (x + 1)f(x^2 + 2x + 1)$ and so new assertion $ Q(x)$ : $ 2f(x^2) = (2x - 1)f(x) + ax$
\end{tcolorbox}
I don't get this part  :blush:\end{tcolorbox}

$ P(x + 1,0)$ $ \implies$ $ f(x^3 + 3x^2 + 3x + 1) = (x + 1)f(x^2 + 2x + 1)$ 

But we previously got that $ f(x+y)=f(x)+f(y)$ and $ f(px)=pf(x)$ with $ p\in\mathbb Q$, so :

$ f(x^3 + 3x^2 + 3x + 1)=f(x^3)+3f(x^2)+3f(x)+f(1)=xf(x^2)+3f(x^2)+3f(x)+a$
$ (x + 1)f(x^2 + 2x + 1)=(x+1)(f(x^2)+2f(x)+a)$ $ =xf(x^2)+2xf(x)+ax+f(x^2)+2f(x)+a$

So $ xf(x^2)+3f(x^2)+3f(x)+a=xf(x^2)+2xf(x)+ax+f(x^2)+2f(x)+a$
So $ 3f(x^2)+3f(x)=2xf(x)+ax+f(x^2)+2f(x)$
So $ 2f(x^2)=2xf(x)+ax-f(x)$
So $ 2f(x^2)=(2x-1)f(x)+ax$
\end{solution}
*******************************************************************************
-------------------------------------------------------------------------------

\begin{problem}[Posted by \href{https://artofproblemsolving.com/community/user/68025}{Pirkuliyev Rovsen}]
	Find all functions $ f:\mathbb{R}\to\mathbb{R}$ such that 
\[ f(x+y+z)=f(x)f(1-y)+f(y)f(1-z)+f(z)f(1-x)\] for all $ x,y,z\in\mathbb{R}$.
	\flushright \href{https://artofproblemsolving.com/community/c6h309237}{(Link to AoPS)}
\end{problem}



\begin{solution}[by \href{https://artofproblemsolving.com/community/user/29428}{pco}]
	\begin{tcolorbox}Find all functions $ f: \mathbb{R}\to\mathbb{R}$ such that $ f(x + y + z) = f(x)f(1 - y) + f(y)f(1 - z) + f(z)f(1 - x)$ for all $ x,y,z\in\mathbb{R}$\end{tcolorbox}

Let $ P(x,y,z)$ be the assertion $ f(x + y + z) = f(x)f(1 - y) + f(y)f(1 - z) + f(z)f(1 - x)$


$ P(x,1 - y,y)$ $ \implies$ $ f(x + 1) = (f(x) + f(1 - x))f(y) + f(1 - y)^2$
$ P(x,1 - z,z)$ $ \implies$ $ f(x + 1) = (f(x) + f(1 - x))f(z) + f(1 - z)^2$

Subtracting these two lines, we get : $ (f(x) + f(1 - x))(f(y) - f(z)) = f(1 - z)^2 - f(1 - y)^2$ and so :

If $ f(x)$ is not constant ($ \exists f(y)\ne f(z)$), then $ f(x) + f(1 - x) = a$ but then $ P(x,1 - y,y)$ becomes $ f(x + 1) = af(y) + f(1 - y)^2$ constant hence contradiction.
So $ f(x) = c$ constant. Plugging back in the original equation, we get $ 3c^2 = c$ and $ c\in\{0,\frac 13\}$

Hence the only two solutions :
$ f(x) = 0$ $ \forall x$

$ f(x) = \frac 13$ $ \forall x$
\end{solution}
*******************************************************************************
-------------------------------------------------------------------------------

\begin{problem}[Posted by \href{https://artofproblemsolving.com/community/user/22187}{Jumbler}]
	Let $ a_1,a_2,\ldots,a_n$ be real constants and 
\[ f(x)=\cos(a_1+x)+\frac{\cos(a_2+x)}{2}+\frac{\cos(a_2+x)}{2^2}+\cdots +\frac{\cos(a_n+x)}{2^{n-1}}\]
If $ x_1$ and $x_2$ are real numbers and $ f(x_1)=f(x_2)=0$, prove that $ x_1-x_2=m\pi$ for some integer $ m$.
	\flushright \href{https://artofproblemsolving.com/community/c6h309421}{(Link to AoPS)}
\end{problem}



\begin{solution}[by \href{https://artofproblemsolving.com/community/user/22187}{Jumbler}]
	Can anybody solve this?
\end{solution}



\begin{solution}[by \href{https://artofproblemsolving.com/community/user/29428}{pco}]
	\begin{tcolorbox}Let $ a_1,a_2,...,a_n$ be real constants and
\[ f(x) = \cos(a_1 + x) + \frac {\cos(a_2 + x)}{2} + \frac {\cos(a_2 + x)}{2^2} + ... + \frac {\cos(a_n + x)}{2^{n - 1}}\]
If $ x_1,x_2$ are real numbers and $ f(x_1) = f(x_2) = 0$, prove that $ x_1 - x_2 = m\pi$ for some integer $ m$.\end{tcolorbox}

$ f(x)=\cos(x)\sum_{k=1}^n\frac{\cos(a_k)}{2^{k-1}}-\sin(x)\sum_{k=1}^n\frac{\sin(a_k)}{2^{k-1}}$

$ f(x)=0$ $ \iff$ $ \cos(x)\sum_{k=1}^n\frac{\cos(a_k)}{2^{k-1}}=\sin(x)\sum_{k=1}^n\frac{\sin(a_k)}{2^{k-1}}$

$ \iff$ $ u\cos(x)=v\sin(x)$ with $ u=\sum_{k=1}^n\frac{\cos(a_k)}{2^{k-1}}$ and $ v=\sum_{k=1}^n\frac{\sin(a_k)}{2^{k-1}}$

If $ u=v=0$, we have $ f(x)=0$ $ \forall x$ and then 
If $ n=1$, this would mean $ \cos(a_1+x)=0$ $ \forall x$, which is impossible.
If $ n>1$, using $ x=-a_1$, we get : $ f(-a_1)=0=1+\sum_{k=2}^n\frac{\cos(a_k-a_1)}{2^{k-1}}$ and so $ \sum_{k=2}^n\frac{\cos(a_k-a_1)}{2^{k-1}}=-1$

But this is still impossible since $ |\sum_{k=2}^n\frac{\cos(a_k-a_1)}{2^{k-1}}|\le$ $ \sum_{k=2}^n\frac{1}{2^{k-1}}$ $ =1-\frac 1{2^{n-1}}<1$

So we cant have $ u=v=0$ and :

either $ u\ne 0$, and $ f(x)=0$ $ \iff$ $ \cot(x)=\frac vu$ and so $ x=\text{arccot}(\frac vu)+n\pi$. Q.E.D.

either $ v\ne 0$, and $ f(x)=0$ $ \iff$ $ \tan(x)=\frac uv$ and so $ x=\arctan(\frac uv)+n\pi$. Q.E.D.
\end{solution}
*******************************************************************************
-------------------------------------------------------------------------------

\begin{problem}[Posted by \href{https://artofproblemsolving.com/community/user/44083}{jgnr}]
	1. Find all functions $f: \mathbb R \to \mathbb R$ such that \[ f(x^2-y^2)=(x-y)(f(x)+f(y))\] holds for all real values of $x$ and $y$.

2. Find all functions $f: \mathbb R \to \mathbb R$ such that \[ xf(x)-yf(y)=(x-y)f(x+y)\] for all $x,y \in \mathbb R$.
	\flushright \href{https://artofproblemsolving.com/community/c6h309538}{(Link to AoPS)}
\end{problem}



\begin{solution}[by \href{https://artofproblemsolving.com/community/user/46039}{ll931110}]
	\begin{tcolorbox}
2. Find all functions on real such that $ xf(x) - yf(y) = (x - y)f(x + y)$.\end{tcolorbox}

Denote $ g: R \rightarrow R$ satisying $ g(x) = f(x) - f(0)$. Thus, $ g(0) = 0$ and
$ xg(x) - yg(y) = (x - y)g(x + y) (1)$

In (1), putting $ y = - x$ yields $ x(g(x) + g( - x)) = 0 \rightarrow - g(x) = g( - x)$
In (1), putting $ y : = - y$ yields $ xg(x) - ( - y)g( - y) = (x + y)g(x - y)$, implying $ xg(x) - yg(y) = (x + y)g(x - y) (2)$

Combining (1) and (2), we obtain $ (x - y)g(x + y) = (x + y)g(x - y)$, or $ \frac {g(x + y)}{x + y} = \frac {g(x - y)}{x - y}$

Setting $ x = \frac {u + v}{2}, v = \frac {u - v}{2}$, we obtain $ \frac {g(u)}{u} = \frac {g(v)}{v}$ for all non-zero real number u,v. So $ g(x) = ax$ for some constant a.

Hence the result is $ f(x) = ax + b$
\end{solution}



\begin{solution}[by \href{https://artofproblemsolving.com/community/user/29428}{pco}]
	\begin{tcolorbox}1. Find all functions on real such that $ f(x^2 - y^2) = (x - y)(f(x) + f(y))$.\end{tcolorbox}
The solution to the first is the same than ll931110 gave for the second :

Let $ P(x,y)$ be the assertion $ f(x^2 - y^2) = (x - y)(f(x) + f(y))$

$ P(x,x)$ $ \implies$ $ f(0)=0$
$ P(x,0)$ $ \implies$ $ f(x^2)=xf(x)$
$ P(-x,0)$ $ \implies$ $ f(x^2)=-xf(-x)$ and so $ f(-x)=-f(x)$ $ \forall x$
$ P(x,-y)$ $ \implies$ $ f(x^2-y^2)=(x+y)(f(x)-f(y))$

and so $ (x - y)(f(x) + f(y))=(x+y)(f(x)-f(y))$
and so $ 2xf(y)=2yf(x)$ 

And so the solution $ \boxed{f(x)=ax}$ (after easily checking that this is indeed a solution)
\end{solution}
*******************************************************************************
-------------------------------------------------------------------------------

\begin{problem}[Posted by \href{https://artofproblemsolving.com/community/user/52090}{Dumel}]
	We are given real numbers $a$ and $b$ which satisfy $ 0<a,b< \frac{1}{2}$. Find all continuous functions $ \mathbb R \to \mathbb R$ such that $ f(f(x))=af(x)+bx$ for all real $x$.
	\flushright \href{https://artofproblemsolving.com/community/c6h309540}{(Link to AoPS)}
\end{problem}



\begin{solution}[by \href{https://artofproblemsolving.com/community/user/29428}{pco}]
	\begin{tcolorbox}reals a,b satisfy $ 0 < a,b < \frac {1}{2}$. find all continuous functions $ R \to R$ such that $ f(f(x)) = af(x) + bx$\end{tcolorbox}

Since $ b\ne 0$, $ f(u)=f(v)$ implies $ f(f(u))-af(u)=f(f(v))-af(v)$ and so $ bu=bv$ and so $ f(x)$ is injective, and so monotonous (since continuous).
If $ \lim_{x\to\infty}f(x)=l$, then the equality $ f(f(x))-ax=bx$ is wrong when $ x\to\infty$ since $ LHS\to f(l)-l$ while $ RHS\to\infty$. So $ f(x)$ is surjective and so is a bijection from $ \mathbb R\to\mathbb R$

Let $ h(x)=f(x)-x$ : $ h(f(x))+(1-a)h(x)=(a+b-1)x$ and, since $ a+b-1\ne 0$ and $ 1-a>0$, $ h(x)$ cant have a constant sign over $ \mathbb R$ (since then LHS would have a constant sign while RHS would not). So $ \exists u$ such that $ h(u)=0$ and so $ f(u)=u$ and so $ f(f(u))=af(u)+u$ implies $ (a+b-1)u=0$ and so $ u=0$ since $ a+b<1$ and so $ f(0)=0$


Then $ g(x)=f^{[-1]}(x)$ exists, is monotonous and respects the equation $ x=ag(x)+bg(g(x))$

Let then $ c\in\mathbb R$ and $ f(c)=d$ and consider the two sequences :
$ u_0=c$, $ u_1=d$ and $ u_{n+2}=au_{n+1}+bu_n$ such that $ u_{n+1}=f(u_n)$
$ v_0=d$, $ v_1=c$ and $ v_{n+2}=-\frac abv_{n+1}+\frac 1bu_n$ such that $ v_{n+1}=g(v_n)$

Let the quadratic $ x^2-ax-b=0$. It has two real roots $ r_1<0<-r_1<r_2$ (since $ r_1+r_2=a>0$ and $ r_1r_2=-b<0$

We get $ u_n=-\frac{d-r_2c}{r_2-r_1}r_1^n+\frac{d-r_1c}{r_2-r_1}r_2^n$ and $ v_n=-\frac{d-r_2c}{r_2-r_1}r_1^{1-n}+\frac{d-r_1c}{r_2-r_1}r_2^{1-n}$


Consider then $ r_n=\frac{u_{n+1}}{u_n}=\frac{f(u_{n})}{u_n}$ and $ s_n=\frac{v_{n+1}}{v_n}=\frac{g(v_{n})}{v_n}$

Since $ f(x)$ is monotonous and $ f(0)=0$, we get that $ r_n$ must always have the same sign $ \forall n$ 
Since $ g(x)$ is monotonous and $ g(0)=0$, we get that $ s_n$ must always have the same sign $ \forall n$ 

$ r_n=\frac{-(d-r_2c)r_1^{n+1}+(d-r_1c)r_2^{n+1}}{-(d-r_2c)r_1^{n}+(d-r_1c)r_2^{n}}$ $ =\frac{-(d-r_2c)r_1+(d-r_1c)r_2(\frac{r_2}{r_1})^{n}}{-(d-r_2c)+(d-r_1c)(\frac{r_2}{r_1})^{n}}$

So, if $ d\ne r_1c$, $ \lim_{n\to+\infty}r_n=r_2>0$ and $ f(x)$ is increasing and so the only decreasing solution would be such that $ d=r_1c$ and so $ f(x)=r_1x$

$ s_n=\frac{-(d-r_2c)r_1^{-n}+(d-r_1c)r_2^{-n}}{-(d-r_2c)r_1^{1-n}+(d-r_1c)r_2^{1-n}}$ $ =\frac{-(d-r_2c)(\frac{r_2}{r_1})^n+(d-r_1c)}{-(d-r_2c)r_1(\frac{r_2}{r_1})^n+(d-r_1c)}$

So, if $ d\ne r_2c$, $ \lim_{n\to+\infty}s_n=\frac 1{r_1}<0$ and $ g(x)$ is decreasing so $ f(x)$ is decreasing and so the only increasing solution would be such that $ d=r_2c$ and so $ f(x)=r_2x$

It's then easy to check that $ r_1x$ and $ r_2x$ indeed are solutions.

And so the only two solutions of this functional equation are $ f(x)=r_1x$ and $ f(x)=r_2x$
\end{solution}



\begin{solution}[by \href{https://artofproblemsolving.com/community/user/40922}{mehdi cherif}]
	excellent  
\end{solution}
*******************************************************************************
-------------------------------------------------------------------------------

\begin{problem}[Posted by \href{https://artofproblemsolving.com/community/user/43536}{nguyenvuthanhha}]
	Find all continuous functions $ f : \mathbb{R} \to \mathbb{R}$ such that
\[f(x+y) = \frac{ f(x) + f(y) - f(x)f(y) }{1 + 2f(x)f(y) },\quad \forall x, y  \in \mathbb{R}.\]
	\flushright \href{https://artofproblemsolving.com/community/c6h309557}{(Link to AoPS)}
\end{problem}



\begin{solution}[by \href{https://artofproblemsolving.com/community/user/29428}{pco}]
	\begin{tcolorbox}\begin{italicized}Find all continuous  function $ f \ : \ \mathbb{R} \mapsto \mathbb{R}$ such that :

        $ f(x + y) \ = \ \frac { f(x) + f(y) - f(x)f(y) }{1 + 2f(x)f(y) } \ \forall \ x; y \ \in \ \mathbb{R}$\end{italicized}\end{tcolorbox}

Let $ P(x,y)$ be the assertion $ f(x+y)=\frac{f(x)+f(y)-f(x)f(y)}{1+2f(x)f(y)}$
Let $ f(0)=a$

1) the only constant solutions are $ f(x)=c$ with $ c\in\{-1,0,\frac 12\}$
==============================================

Constant solutions $ f(x)=a$ imply $ a=\frac{2a-a^2}{1+2a^2}$ and so $ a\in\{-1, 0,\frac 12\}$ and it is easy to check that these 3 values are indeed solutions.
Q.E.D.

Consider from now that $ f(x)$ is not a constant function.

2) Non constant solutions are such that $ f(0)=0$ and $ f(x)\in(-1,\frac 12)$ $ \forall x$
=======================================================
$ P(x,0)$ $ \implies$ $ a(f(x)+1)(2f(x)-1)=0$
If $ a\ne 0$, this implies either $ f(x)$ constant and so $ f(0)=a=0$
Suppose now $ \exists u$ such that $ f(u)=-1$. $ P(x-u,u)$ $ \implies$ $ f(x)=-1$ and $ f(x)$ would be constant
Suppose now $ \exists u$ such that $ f(u)=\frac 12$. $ P(x-u,u)$ $ \implies$ $ f(x)=\frac 12$ and $ f(x)$ would be constant
So $ f(x)=0$ and continuity implies $ f(x)\in(-1,\frac 12)$
Q.E.D.

3) non constant solutions are injective
=========================

If $ f(x)$ is continuous but not monotonous, we can always find $ u,v$ with $ |v-u|>0$ as little as we want such that $ f(u)=f(v)$

But, if $ f(u)=f(v)$, with $ u\ne v$ then $ P(u,v-u)$ $ \implies$ $ f(u)=f(v)=\frac{f(u)+f(v-u)-f(u)f(v-u)}{1+2f(u)f(v-u)}$ and so $ f(v-u)(f(u)+1)(2f(u)-1)=0$ and so $ f(v-u)=0$ and then $ P(x,v-u)$ $ \implies$ $ f(x+v-u)=f(x)$ and $ f(x)$ is periodic (and one period is $ v-u$)

So $ f(x)$ is periodic and there exist periods as small as we want. Then, since $ f(x)$ is continuous, $ f(x)$ is constant, which is impossible (we supposed $ f(x)$ is not a constant function)

So $ f(x)$ is injective and strictly monotonous.
Q.E.D

4) non constant solutions are $ f(x)=\frac{1-e^{ax}}{2+e^{ax}}$
============================================
Let $ g(x)=\frac{1-e^{x}}{2+e^{x}}$. $ g(x)$ is a bijection from $ (-\infty,+\infty)\to(-1,\frac 12)$.

[hide="How did I think to this?"]
================  hidden part =====================

I supposed $ f(x)$ differentiable and I wrote $ f(x+y)-f(x)=-f(y)\frac{2f(x)^2+f(x)-1}{1+2f(x)f(y)}$

So $ \frac{f(x+y)-f(x)}y=-\frac{f(y)}y\frac{2f(x)^2+f(x)-1}{1+2f(x)f(y)}$

Setting $ y\to 0$, I got $ \frac{f'(x)}{{2f(x)^2+f(x)-1}}=c$ which is easy to solve with solution $ f(x)=\frac{1-e^{ax+b}}{2+e^{ax+b}}$. Then $ f(0)=0$ gives $ b=0$

================ end of hidden part ===================
[\/hide]

Let then $ h(x)=g^{[-1]}(f(x))$ and $ h(x)$ is continuous.

It's [rather] easy to verify that $ g(x)$ is a solution of the required equation and so that $ g(x+y)=\frac{g(x)+g(y)-g(x)g
(y)}{1+2g(x)g(y)}$

Then $ g(h(x)+h(y))=\frac{g(h(x))+g(h(y))-g(h(x))g(h(y))}{1+2g(h(x))g(h(y))}$ $ =\frac{f(x)+f(y)-f(x)f(y)}{1+2f(x)f(y)}$ $ =f(x+y)=g(h(x+y))$

So $ g(h(x)+h(y))=g(h(x+y)$ and, since $ g(x)$ is injective, $ h(x+y)=h(x)+h(y)$ and so $ h(x)=ax$ (remember $ h(x)$ is continuous)
Q.E.D

5) synthesis of solutions
================

$ f(x)=-1$ $ \forall x$

$ f(x)=0$ $ \forall x$

$ f(x)=\frac 12$ $ \forall x$

$ f(x)=\frac{1-e^{ax}}{2+e^{ax}}$
\end{solution}



\begin{solution}[by \href{https://artofproblemsolving.com/community/user/43536}{nguyenvuthanhha}]
	\begin{italicized}Here is the official solution

   Although it was written in Vietnamese 

   I think you can understand it , Pco  \end{italicized}
\end{solution}
*******************************************************************************
-------------------------------------------------------------------------------

\begin{problem}[Posted by \href{https://artofproblemsolving.com/community/user/64682}{KDS}]
	Find all non-decreasing functions $f: \mathbb R \to \mathbb R$ satisfying \[ f(x+f(y))=f(f(x))+f(y),\quad \forall x,y \in \mathbb R.\]
	\flushright \href{https://artofproblemsolving.com/community/c6h309564}{(Link to AoPS)}
\end{problem}



\begin{solution}[by \href{https://artofproblemsolving.com/community/user/16261}{Rust}]
	Let $ a=f(f(0))$. Then $ x=0$ give $ f(f(y))=f(y)+a$, $ y=0$ give $ f(0)=0\to a=0$. Therefore $ f(f(x))=f(x),f(0)=0$. It mean, that $ f$ is proektor $ f: R\to D$
$ x=-f(y)$ give $ f(-f(y))=-f(y)$. It mean, that $ D$ is symmetric.
Because $ f$ is non decrease, exist $ b=f(0^+)$. If $ b>0$, then for $ 0<x\le b$ we get $ f(b)\le f(f(x))=f(x)\le b$
and for $ -b<x<0$ we get $ f(-b)\ge f(f(x))=f(x)\ge -b$. Together it give $ b=0$ and $ f$ is continiously in 0.
It give $ f(x)=0$ or $ f(x)=x$.
\end{solution}



\begin{solution}[by \href{https://artofproblemsolving.com/community/user/29428}{pco}]
	\begin{tcolorbox}Find all non decreasing functions $ f: R\to R$ satisfying $ f(x + f(y)) = f(f(x)) + f(y)$ $ \forall x,y \in R$\end{tcolorbox}

Let $ P(x,y)$ be the assertion $ f(x + f(y)) = f(f(x)) + f(y)$

$ P(0,0)$ $ \implies$ $ f(0) = 0$
$ P(0,x)$ $ \implies$ $ f(f(x)) = f(x)$ and so $ P(x,y)$ may be written $ Q(x,y)$ : $ f(x + f(y)) = f(x) + f(y)$
$ Q( - f(x),x$ $ \implies$ $ f( - f(x)) = - f(x)$

So $ f(\mathbb R)$ is an additive subgroup of $ \mathbb R$ and $ f(x) = x$ $ \forall x\in f(\mathbb R)$. Then :

$ f(x) = 0$ $ \forall x$ is a solution.
If $ f(x)$ is not the zero function, $ f(\mathbb R)\cap\mathbb R^ + _*\ne \emptyset$ and let $ a = \inf (f(\mathbb R)\cap \mathbb R^ + _*)$

1) $ a = 0$
So $ f(\mathbb R)$ is dense in $ \mathbb R$ and so $ f(x) = x$ $ \forall x$ since $ f(x)$ is non decreasinf and since $ f(x) = x$ for all $ x$ in a set dense in $ \mathbb R$

2) $ a > 0$
Then $ f(\mathbb R) = a\mathbb Z$
$ Q(x,a)$ $ \implies$ $ f(x + a) = f(x) + a$ and so $ f(x)$ is defined exaclty on $ \mathbb R$ when it's defined on $ [0,a)$
And since $ f(0) = 0$, $ f(a) = a$ and $ f(x)$ non decreasing, 
either $ \exists b\in(0,a]$ such that $ f(x) = 0$ $ \forall x\in[0,b)$ and $ f(x) = a$ $ \forall x\in[b,a)$ and so $ f(x) = a[\frac {x + a - b}a]$

either $ \exists b\in[0,a)$ such that $ f(x) = 0$ $ \forall x\in[0,b]$ and $ f(x) = a$ $ \forall x\in(b,a)$ and so $ f(x) = -a[\frac { - x + b}a]$

And it is easy to check that these two functions indeed are solutions.

3) synthesis of solutions

$ f(x) = 0$ $ \forall x$

$ f(x) = x$ $ \forall x$

$ f(x) = a[\frac {x + b}a]$ with $ a > b \ge 0$

$ f(x) = -a[\frac { - x + b}a]$ with $ a > b \ge 0$
\end{solution}
*******************************************************************************
-------------------------------------------------------------------------------

\begin{problem}[Posted by \href{https://artofproblemsolving.com/community/user/45084}{\u222bFaiLurE\u222e}]
	Find all functions $f: \mathbb R^{+} \to \mathbb R^{+}$ satisfying 
\[ (1+yf(x))(1-yf(x+y))=1, \quad \forall x,y \in \mathbb R^+.\]
	\flushright \href{https://artofproblemsolving.com/community/c6h310386}{(Link to AoPS)}
\end{problem}



\begin{solution}[by \href{https://artofproblemsolving.com/community/user/16261}{Rust}]
	Let $ g(x)=xf(x)$, then we get $ g(x+y)=\frac{(x+y)g(x)}{x+yg(x)}$.
If $ y=zx$, then $ g(x(1+z))=\frac{(1+z)g(x)}{1+zg(x)}$ or \[ g(t)=\frac{t}{t+g_1(x)}, g_1(x)=x\frac{1-g(x)}{g(x)},t=x(1+z).\]
It mean $ g_1(x)=const=a$. It give $ g(x)=\frac{x}{x+a},f(x)=\frac{1}{x+a}.$
\end{solution}



\begin{solution}[by \href{https://artofproblemsolving.com/community/user/45762}{FelixD}]
	\begin{tcolorbox}Let $ R^ +$ be the set of all positive real numbers. Find all functions $ f: R^ + \to R^ +$ satisfying 
$ (1 + yf(x))(1 - yf(x + y)) = 1$ ,$ \forall x,y \in R^ +$\end{tcolorbox}

Set $ g(x) = \frac {1}{f(x)} - x$. The given equation is equivalent to $ f(x) = f(x + y) + yf(x)f(x + y)$ or $ g(x + y) = g(x)$ for all $ x,y \in \mathbb{R_ + }$. It follows that $ g$ is constant, hence $ f(x) = \frac {1}{x + k}$ for some arbitrarily chosen $ k \ge 0$.
\end{solution}



\begin{solution}[by \href{https://artofproblemsolving.com/community/user/29428}{pco}]
	\begin{tcolorbox} ... hence $ f(x) = \frac {1}{x + k}$ for some arbitrarily chosen $ k > 0$.\end{tcolorbox}

To be precise, $ k=0$ fits too.  :)
\end{solution}



\begin{solution}[by \href{https://artofproblemsolving.com/community/user/45762}{FelixD}]
	\begin{tcolorbox}[quote="FelixD"] ... hence $ f(x) = \frac {1}{x + k}$ for some arbitrarily chosen $ k > 0$.\end{tcolorbox}

To be precise, $ k = 0$ fits too.  :)\end{tcolorbox}

Of course it does^^... Thanks, pco.
\end{solution}
*******************************************************************************
-------------------------------------------------------------------------------

\begin{problem}[Posted by \href{https://artofproblemsolving.com/community/user/48278}{Dimitris X}]
	Let $ f: \mathbb {R} \to \mathbb{R}$ be a function such that for all real numbers $x$ and $y$ with $x \neq y$, we have 
\[ f \left(\frac{x+y}{x-y} \right)=\frac{f(x)+f(y)}{f(x)-f(y)}.\]
Prove that $ f(x)=x$ for all $x \in \mathbb{R}$.
	\flushright \href{https://artofproblemsolving.com/community/c6h310591}{(Link to AoPS)}
\end{problem}



\begin{solution}[by \href{https://artofproblemsolving.com/community/user/48364}{cnyd}]
	$ y = 0$  $ \implies$  $ f(1) = \frac {f(x) + f(0)}{f(x) - f(0)}$

$ \implies$  $ f(x)f(1) - f(1)f(0) = f(x) + f(0)$  $ \implies$  $ f(x)[f(1) - 1] = f(0)[f(1) + 1]$

if $ f(0)\not = 0,f(1)\not = 1$  $ \implies$  $ f(x) = c$ , $ c\in\mathbb{R}$

if $ f$ is not constant function and $ f(0) = 0$ $ \iff$   $ f(1) = 1$

$ f(\frac {x + y}{x - y}) = \frac {f(x) + f(y)}{f(x) - f(y)}$

Let $ x = yk(k\not = 1)$  $ \implies$  $ f(\frac {k + 1}{k - 1}) = \frac {f(yk) + f(y)}{f(yk) - f(y)} = \frac {f(k) + 1}{f(k) - 1}$

$ \implies$  $ (f(k) + 1)(f(yk) - f(y)) = (f(k) - 1)(f(yk) + f(y))$

$ \implies$  $ f(yk) = f(k)f(y)$

$ \implies$  $ [f(x + y) + f(x - y)]f(y) = [f(x + y) - f(x - y)]f(x)$

and $ f(1) = 1$  $ \implies$  by the induction

$ f(x + 1) + x - 1 = [f(x + 1) - x + 1]x$  $ \implies$  $ f(x) = x$  for $ x\in \mathbb{N}$

$ f(\frac {x + y}{x - y}) = \frac {f(x) + f(y)}{f(x) - f(y)}$

Let $ x = - y$  $ \implies$  $ f(x) = - f( - x)$  $ \implies$ $ f(x) = x$  for $ x\in\mathbb{Z}$

and $ f(xy ) = f(x)f(y )$  $ \implies$  $ f( x ) = x$  for $ x\in\mathbb{Q}$

$ f(x)$ is an increasing function

$ \exists$ $ u\in\mathbb{R}$  $ f(u) > u$

We can $ u < r = f(r) < f(u)$ and $ r\in\mathbb{Q}$

Contradiction!  Similarly $ f(u) < u$

$ \implies$  $ f(x) = x$
\end{solution}



\begin{solution}[by \href{https://artofproblemsolving.com/community/user/29428}{pco}]
	\begin{tcolorbox}Let $ f: \mathbb {R} \to \mathbb{R}$ is a function such that $ f \left(\frac {x + y}{x - y} \right) = \frac {f(x) + f(y)}{f(x) - f(y)},\forall x \not = y$.Prove that $ f(x) = x,\forall x \in \mathbb{R}$.\end{tcolorbox}
Let $ P(x,y)$ be the assertion $ f\left(\frac {x + y}{x - y}\right) = \frac {f(x) + f(y)}{f(x) - f(y)}$

$ P(0,x)$ $ \implies$ $ f( - 1) = \frac {f(0) + f(x)}{f(0) - f(x)}$ and so $ f(x)(1 + f( - 1)) = f(0)(f( - 1) - 1)$. So, if $ f( - 1)\ne - 1$, $ f(x)$ is constant, which is impossible.

So $ f( - 1) = - 1$ and $ f(0) = 0$

Then $ P(x, - x)$ $ \implies$ $ f( - x) = - f(x)$ and so $ f(1) = 1$.

Then, comparing $ P(x,1)$ and $ P(xy,y)$, we get $ \frac {f(x) + 1}{f(x) - 1} = \frac {f(xy) + f(y)}{f(xy) - f(y)}$ and so $ f(xy) = f(x)f(y)$

From this, it's easy to conclude :
1) $ f(x) > 0$ $ \forall x > 0$ and $ f(x) < 0$ $ \forall x < 0$
2) Using then $ P(x,y)$ : $ x > y > 0$ $ \implies$ $ f(x) > f(y) > 0$ and so $ f(x)$ is an increasing function.
3) $ f(x) = x^a$ $ \forall x > 0$ such that $ \ln(x)\in\mathbb Q$

From 3) and 2), and since $ \{x > 0$ such that $ \ln(x)\in\mathbb Q\}$ is dense in $ \mathbb R^ +$, we get that $ f(x) = x^a$ $ \forall x\in\mathbb R^ +$

Then $ P(2,1)$ $ \implies$ $ 3^a = \frac {2^a + 1}{2^a - 1}$ whose unique solution is $ a = 1$ and so $ f(x) = x$ $ \forall x > 0$

And so $ f(x) = x$ $ \forall x$


\begin{bolded}For cnyd :\end{bolded}\end{underlined}
\begin{tcolorbox}...$ \implies$  $ [f(x + y) + f(x - y)]f(y) = [f(x + y) - f(x - y)]f(x)$

and $ f(1) = 1$  $ \implies$  by the induction

$ f(x + 1) + x - 1 = [f(x + 1) - x + 1]x$  $ \implies$  $ f(x) = x$  for $ x\in \mathbb{N}$\end{tcolorbox}

You cant use the induction simply here since you need at least $ 2$ initial values different and you just have one $ f(1)=1$. You should show first that $ f(2)=2$

\begin{tcolorbox}...   $ f( x ) = x$  for $ x\in\mathbb{Q}$\end{tcolorbox}

The request is about all $ \mathbb R$
\end{solution}



\begin{solution}[by \href{https://artofproblemsolving.com/community/user/48364}{cnyd}]
	$ f(3)+1=f(2)[f(3)-1]$

$ f(5)[f(3)-f(2)]=f(3)+f(2)$

$ f(5)[f(2)^{2}-1]=f(3)[f(2)^{2}+1]$

$ \implies$  $ f(2)=2$ 
\end{solution}



\begin{solution}[by \href{https://artofproblemsolving.com/community/user/29428}{pco}]
	\begin{tcolorbox}$ f(3) + 1 = f(2)[f(3) - 1]$

$ f(5)[f(3) - f(2)] = f(3) + f(2)$

$ f(5)[f(2)^{2} - 1] = f(3)[f(2)^{2} + 1]$

$ \implies$  $ f(2) = 2$ \end{tcolorbox}

Just for fun ;) This 3-equations system has 3 solutions :

$ (f(2),f(3),f(5))\in\{(0,-1,1),(-1,0,-1),(2,3,5)\}$ :)
\end{solution}



\begin{solution}[by \href{https://artofproblemsolving.com/community/user/48278}{Dimitris X}]
	$ f(2) = 2$ is really difficult to be proved i think!!!!
\end{solution}



\begin{solution}[by \href{https://artofproblemsolving.com/community/user/48364}{cnyd}]
	But $ \forall x>0$  $ \implies$  $ f(x)>0$
\end{solution}



\begin{solution}[by \href{https://artofproblemsolving.com/community/user/29428}{pco}]
	\begin{tcolorbox}But $ \forall x > 0$  $ \implies$  $ f(x) > 0$\end{tcolorbox}

You're right. So you got it.
I just wanted to say that some steps were missing in your demo. :)
\end{solution}
*******************************************************************************
-------------------------------------------------------------------------------

\begin{problem}[Posted by \href{https://artofproblemsolving.com/community/user/48278}{Dimitris X}]
	Find all functions $ f: \mathbb{R^*} \to \mathbb{R^*}$ satisfying the functional relation
\[ f(f(x)-x)=2x, \quad \forall x\in \mathbb{R^*}.\]
	\flushright \href{https://artofproblemsolving.com/community/c6h310761}{(Link to AoPS)}
\end{problem}



\begin{solution}[by \href{https://artofproblemsolving.com/community/user/66394}{reason}]
	hi!
First we consider the function g such that: $ g(x)=f(x)-x$ so the f.e become:

$ g(g(x))+g(x)=2x$ $ \forall x\in{\mathbb{R^{*}}}$

then we consider the sequence $ (x_{n})$ define by:$ x_{0}=x$ and $ g(x_{n})=x_{n+1}$ so if we solve this equation: $ x^{2}+x-2=0$ we find $ 1$ and $ -2$ ==>

 $ x_{n}=\alpha+(-2)^{n}\beta  
 ;(\alpha,\beta)\in{\mathbb{R}}$ with $ x=\alpha+\beta$ ,then for $ n=1$ we find: $ g(x)=\alpha-
2\beta$ ==> $ f(x)=x+\alpha-2\beta$ $ (*)$.

We have: $ f(f(x)-x)=2x$ so by using $ (*)$ we find: $ \beta=0$ ==> $ f(x)=2x \forall x\in{\mathbb{R^{*}}}$
\end{solution}



\begin{solution}[by \href{https://artofproblemsolving.com/community/user/64716}{mavropnevma}]
	I fail to see why from the star (*) relation follows $ \beta = 0$ (other than it's convenient to be so).
\end{solution}



\begin{solution}[by \href{https://artofproblemsolving.com/community/user/66394}{reason}]
	we have $ f(f(x)-x)=2x$ and $ f(x)=x+\alpha-2\beta$ so:
$ f(x)-x=x+\alpha-2\beta-x=\alpha-2\beta$ and $ f(\alpha-2\beta)=2(\alpha-2\beta)$,we have $ x=\alpha+\beta$ too so we find : $ 2(\alpha-2\beta)=2(\alpha+\beta)$ ==> $ \beta=0$..
\end{solution}



\begin{solution}[by \href{https://artofproblemsolving.com/community/user/64716}{mavropnevma}]
	I still don't see why $ f(\alpha-2\beta)=2(\alpha-2\beta)$.
\end{solution}



\begin{solution}[by \href{https://artofproblemsolving.com/community/user/66394}{reason}]
	we have : $ f(x)=x+\alpha-2\beta$ so $ f(\alpha-2\beta)=\alpha-2\beta+\alpha-2\beta$ ==> $ f(\alpha-2\beta)=2(\alpha-2\beta)$
\end{solution}



\begin{solution}[by \href{https://artofproblemsolving.com/community/user/64716}{mavropnevma}]
	I don't think you can do that. Let's go back to where you started solving the recurrence relatiion: you started with $ x_0 = x$, etc. Then the coefficients $ \alpha$ and $ \beta$ depend on the value of the initial term, so in fact you have $ \alpha_x$ and $ \beta_x$.

Now, you got $ f(x) = x + \alpha_x - 2\beta_x$, and you're not allowed to replace $ x$ with any value you want, respectively $ \alpha_x - 2\beta_x$, because for this value you may have different coefficients $ \alpha$ and $ \beta$, in fact $ \alpha_{\alpha_x - 2\beta_x}$ and $ \beta_{\alpha_x - 2\beta_x}$.

My point is that coefficients $ \alpha$ and $ \beta$ are not "universal", over all $ x$.
\end{solution}



\begin{solution}[by \href{https://artofproblemsolving.com/community/user/66394}{reason}]
	yes,I undrestand your point, but I think that can correct my solution:
we have: 

$ g(g(x)) + g(x) = 2x$

$ g(x) = x - 3\beta_{x}$

$ g(g(x)) = g(x) - 3\beta_{g(x)}$

then we conclude that: $ - 3(2\beta_{x} - \beta_{g(x)}) = 0$ ==> $ 2\beta_{x} = \beta_{g(x)}$, then by using this result we can find that: $ g(g(x)) + g(x) = 2x$<==>$ 2x - 12\beta_{x} = 2x$ ==> $ \beta_{x} = 0$ so $ g(x) = x$==>$ f(x) = 2x \forall x\in{\mathbb{R^{*}}}$

I hope that will be correct.
\end{solution}



\begin{solution}[by \href{https://artofproblemsolving.com/community/user/66698}{Vincent Gilbert}]
	First ,$ f(x)-x \in R^* \forall x \in R^*$
$ g(x)=f(x)-x \Rightarrow g: R^* \rightarrow R^*  , g(g(x))+g(x)=2x$
LEt's consider the sequence :  $ u_0=x_0; u_{n+1}=g(x_n)$
$ \Rightarrow u_n>0 (*) \forall n \in N ;  u_(n+1)+u_n=2u_(n-1)$
$ \Rightarrow u_n=\alpha+(-2)^n .\beta$
* $ \beta <0 \Rightarrow u_{2n}<0$ when $ n$ is large enough (Contradition (*))
*$ \beta>0\Rightarrow u_{2n+1}<0$ when $ n$ is large enough (Contradition (*))
Hence :$ \beta=0$
$ \Rightarrow u_1=u_0=x_0$
$ \Rightarrow g(x_0)=x_0$
$ \Rightarrow f(x)=2x \forall x \in R^*$
\end{solution}



\begin{solution}[by \href{https://artofproblemsolving.com/community/user/32234}{Mashimaru}]
	\begin{tcolorbox}First ,$ f(x) - x \in R^* \forall x \in R^*$
$ g(x) = f(x) - x \Rightarrow g: R^* \rightarrow R^* , g(g(x)) + g(x) = 2x$
LEt's consider the sequence :  $ u_0 = x_0; u_{n + 1} = g(x_n)$
$ \Rightarrow u_n > 0 (*) \forall n \in N ; u_(n + 1) + u_n = 2u_(n - 1)$
$ \Rightarrow u_n = \alpha + ( - 2)^n .\beta$
* $ \beta < 0 \Rightarrow u_{2n} < 0$ when $ n$ is large enough (Contradition (*))
*$ \beta > 0\Rightarrow u_{2n + 1} < 0$ when $ n$ is large enough (Contradition (*))
Hence :$ \beta = 0$
$ \Rightarrow u_1 = u_0 = x_0$
$ \Rightarrow g(x_0) = x_0$
$ \Rightarrow f(x) = 2x \forall x \in R^*$\end{tcolorbox}

What is the contradiction when ${ u_{2n}<0}$ or $ u_{2n+1}<0$? I think $ \mathbb{R}^*$ is the set of real numbers other than $ 0$, not the set of positive real numbers.
\end{solution}



\begin{solution}[by \href{https://artofproblemsolving.com/community/user/66698}{Vincent Gilbert}]
	You're right
\end{solution}



\begin{solution}[by \href{https://artofproblemsolving.com/community/user/32234}{Mashimaru}]
	I think there should be some additional condition to this functional equation because of the following reasons:

1. It is clearly that we just need to solve an equivalent problem:
Find every function $ g: \mathbb{R}^*\to\mathbb{R}^*$ satisfies: $ g(g(x)) + g(x) = 2x,\forall x\in\mathbb{R}$.

2. By considering the sequence $ \{u_n\}_{n=0}^{\infty}$ defined by $ u_0 = x, u_{n+1} =g(u_n)$, it is known that $ u_n = \lambda_1 + \lambda_2 (-2)^n$, thus if $ \lambda_2\neq 0$, the sequence $ \{u_n\}_{n=0}^{\infty}$ takes both negative and positive values. Thus if we add the \begin{italicized}continuity\end{italicized} of $ g(x)$, we can deduce that $ \exists x: g(x)=0$, contradiction. Moreover, if $ g: \mathbb{R}^+\to\mathbb{R}^+$ or $ g: \mathbb{R}^- \to \mathbb{R}^-$, we can also find $ g$.

3. Still consider the sequence $ \{u_n\}$ above, we see that the definition of $ g$ is just in $ \{u_n\}$, therefore if we can point out a partion of $ \mathbb{R}^*$ into such disjoint sequences like $ \{u_n\}$, i.e. sequences satisfies $ u_{n+1} + u_n - 2u_{n-1} = 0$ then there is a function $ g$. This is just my sense that we can construct infinitely many functions $ g$ satisfies the problem in this way. :maybe:
\end{solution}



\begin{solution}[by \href{https://artofproblemsolving.com/community/user/48278}{Dimitris X}]
	\begin{italicized}solution\end{italicized} (Abhay Kumar Jha).

We begin with observation that $ f(x)\ge 0,\forall x\in R^*$.Define a new function $ g$ on $ R^*$ by setting $ g(x) = f(x) - x$ .Since $ f(x)\ge x$,we see that $ g(g(x))$ is meaningful.A simple computation gives $ g(g(x)) = f(g(x)) - g(x) = f(f(x) - x) - (f(x) - x) = 3x - f(x) = 2x - g(x).$

Thus we have to find $ g: R^* \to R^*$ satisfying

$ g(g(x)) + g(x) = 2x.$

Fix some $ a$ in $ R^*$ and define $ u_n = g^n(a)$ where $ g^n(x) = g(g^{n - 1}(x)).$Then $ < u_n >$ satysfies the recurrence relation $ u_{n + 2} + u_{n + 1} + - 2u_n = 0$ for all $ n \ge 0$,where $ u_o = a$.This diference equation for $ u_n$ has auxiliary equation $ x^2 + x - 2 = 0$ which has solutions $ x = 1$ and $ x = - 2$.Using the theory of difference equations,the general solution of the difference equation is given by

$ u_n = A(1)^n - B( - 2)^n$

for some constants $ A$ and $ B$.However, we see that $ g(x)\ge 0$ for all $ x\in R^*$,and hence $ u_n = g^n(a) \ge 0$.Since $ ( - 2)^n$ alters sign as $ n$ runs through natural numbers,we conclude that $ B = 0$.This forces $ u_n = A$ for all $ n$ and using initial condition,we can see that $ 2A = u_1 + u_2 = 2a$.We thus obtain $ A = a$ and hence $ u_n = a$ for all $ n$.But then $ g(a) = u_1 = a$ and we conclude that $ f(a) = 2a$.Since this is true for every $ a\in R^*$ we arrive at the solution $ f(x) = 2x$.

P.s.1
I cant understand the solution very well because i'm not familiar with sequences and thats why i post this problem on unsolved,hoping for an elementary solution....

P.s.2
The book uses the notation $ Ro$ and i guess he mean $ R^*$......
\end{solution}



\begin{solution}[by \href{https://artofproblemsolving.com/community/user/44887}{Mathias_DK}]
	\begin{tcolorbox}\begin{italicized}solution\end{italicized} (Abhay Kumar Jha).

We begin with observation that $ f(x)\ge 0,\forall x\in R^*$.\end{tcolorbox}
$ f(x) = -x$ is a solution too, so i don't get the point  :roll:
\end{solution}



\begin{solution}[by \href{https://artofproblemsolving.com/community/user/32234}{Mashimaru}]
	\begin{tcolorbox}
We begin with observation that $ f(x)\ge 0,\forall x\in R^*$\end{tcolorbox}

This is non-sense. If $ f(x) \geq 0,\forall x\in\mathbb{R}^*$ then we directly get the contradiction from the assertion $ f(f(x) - x) = 2x$ (just let $ x < 0$).
\end{solution}



\begin{solution}[by \href{https://artofproblemsolving.com/community/user/45762}{FelixD}]
	Maybe $ f: \mathbb{R_+} \to \mathbb{R_+}$^^
\end{solution}



\begin{solution}[by \href{https://artofproblemsolving.com/community/user/32234}{Mashimaru}]
	\begin{tcolorbox}Maybe $ f: \mathbb{R_ + } \to \mathbb{R_ + }$^^\end{tcolorbox}

I think so, either. But could anyone prove that if $ f: \mathbb{R}^*\to \mathbb{R}^*$ then there are infinitely many functions $ f$ satisfies the problem? I have an idea in post #12 but I did not have time to work on it :(
\end{solution}



\begin{solution}[by \href{https://artofproblemsolving.com/community/user/29428}{pco}]
	\begin{tcolorbox}[quote="FelixD"]Maybe $ f: \mathbb{R_ + } \to \mathbb{R_ + }$^^\end{tcolorbox}

I think so, either. But could anyone prove that if $ f: \mathbb{R}^*\to \mathbb{R}^*$ then there are infinitely many functions $ f$ satisfies the problem? I have an idea in post #12 but I did not have time to work on it :(\end{tcolorbox}

Sure, take any Hamel basis of the $ \mathbb Q$-vectorspace $ \mathbb R$ and randomly choose either $ f(b_i)=2b_i$, either $ f(b_i)=-b_i$ for the elements of the basis and you find infinitely many solutions.
\end{solution}



\begin{solution}[by \href{https://artofproblemsolving.com/community/user/32234}{Mashimaru}]
	\begin{tcolorbox}[quote="Mashimaru"][quote="FelixD"]Maybe $ f: \mathbb{R_ + } \to \mathbb{R_ + }$^^\end{tcolorbox}

I think so, either. But could anyone prove that if $ f: \mathbb{R}^*\to \mathbb{R}^*$ then there are infinitely many functions $ f$ satisfies the problem? I have an idea in post #12 but I did not have time to work on it :(\end{tcolorbox}

Sure, take any Hamel basis of the $ \mathbb Q$-vectorspace $ \mathbb R$ and randomly choose either $ f(b_i) = 2b_i$, either $ f(b_i) = - b_i$ for the elements of the basis and you find infinitely many solutions.\end{tcolorbox}

Once you told me about the Hamel basis, Mr.\begin{bolded}pco\end{bolded}. But I still did not get how can we generate infinitely many functions satisfy the problem in this way. Could you make it more clearly, please? Thank you in advanced.
\end{solution}



\begin{solution}[by \href{https://artofproblemsolving.com/community/user/29428}{pco}]
	Just take one basis $ \{b_i\}$ of the $ \mathbb Q$-vector space $ \mathbb R$

Then define :
$ f_1(\sum q_ib_i)=\sum q_if(b_i)$ with $ f(b_1)=2b_1$ and $ f(b_i)=-b_i$ for all other elements of the basis. You got a solution.

$ f_2(\sum q_ib_i)=\sum q_if(b_i)$ with $ f(b_2)=2b_2$ and $ f(b_i)=-b_i$ for all other elements of the basis. You got another solution.

$ f_3(\sum q_ib_i)=\sum q_if(b_i)$ with $ f(b_3)=2b_3$ and $ f(b_i)=-b_i$ for all other elements of the basis. You got another solution.

And so on. Choosing randomly $ f(b_i)$ as either $ 2b_i$, either $ -b_i$, you build infinitely many different solutions to $ f(f(x)-x)=2x$
\end{solution}
*******************************************************************************
-------------------------------------------------------------------------------

\begin{problem}[Posted by \href{https://artofproblemsolving.com/community/user/112}{Diogene}]
	Find all functions $f,g: \mathbb R^{+} \to \mathbb R^{+}$ such that
\[ f(x)=\frac 1x + \frac 1{f(g(x))},\ g(x)=\frac 1x + \frac 1{g(f(x))},\]
for all positive reals $x$ and $y$.
	\flushright \href{https://artofproblemsolving.com/community/c6h310814}{(Link to AoPS)}
\end{problem}



\begin{solution}[by \href{https://artofproblemsolving.com/community/user/29428}{pco}]
	\begin{tcolorbox}Find the functions $ f,g: R^ + \rightarrow R^ +$( positive real numbers) such that :
$ f(x) = \frac 1x + \frac 1{f(g(x))},\ g(x) = \frac 1x + \frac 1{g(f(x))}$
 :cool:\end{tcolorbox}

I think this problem is very difficult.
Beside the trivial solution $ f(x) = g(x) = \frac 2x$, there exist infinitely many solutions.

==========================  example of solutions =====================
For example, considering the cases where $ f(x) = g(x)$ and $ f(x) = \frac 1x + \frac 1{f(f(x))}$, we have solutions like :

Let $ p > 1\in\mathbb N$
Let $ A = \{x = 2^u(p + 1)^{\frac v2}p^{\frac w2}$, $ \forall u,v,w\in\mathbb Z\}$
Let $ B = \mathbb R^ + \backslash A$

Consider the relation $ \sim$ defined in $ B\times B$ by $ x\sim y$ $ \iff$ either $ xy = (p + 1)p^n$, either $ \frac xy = p^n$ for some $ n\in\mathbb Z$
This clearly is an equivalence relation.
Let then $ h(x): B\to B$ any choice function associating to $ x\in B$ a representant for it's equivalence class.

Let $ x\in B$. Since $ x\sim h(x)$ :
either $ \exists n\in\mathbb Z$ such that $ xh(x) = (p + 1)p^n$
either $ \exists n\in\mathbb Z$ such that $ \frac {h(x)}x = p^n$
and , for a given $ x$, it's impossible to have two different integers $ n_1$ and $ n_2$ verifying such equalities :
$ xh(x) = (p + 1)p^{n_1} = (p + 1)p^{n_2}$ would imply $ n_1 = n_2$

$ \frac {h(x)}x = p^{n_1} = p^{n_2}$ would imply $ n_1 = n_2$

$ xh(x) = (p + 1)p^{n_1}$ and $ \frac {h(x)}x = p^{n_2}$ would imply $ x^2 = (p + 1)p^{n_1 - n_2}$ and so $ x\in A$ and so $ x\notin B$

So there is a well defined function $ n(x): B\to\mathbb Z$ such that either $ xh(x) = (p + 1)p^{n(x)}$, either $ \frac {h(x)}x = p^{n(x)}$

Then we can define $ f(x) = g(x)$ as :

$ \forall x\in A$ : $ f(x) = \frac 2x$

$ \forall x\in B$ such that $ xh(x) = (p + 1)p^{n(x)}$ : $ f(x) = h(x)p^{ - n(x) - 1}$ $ =\frac{p+1}{px}$

$ \forall x\in B$ such that $ \frac {h(x)}x = p^{n(x)}$ : $ f(x) = \frac {(p + 1)p^{n(x)}}{h(x)}$ $ =\frac{p+1}x$


[hide="verification that this function fits the requirements"]
1) $ x\in A$
Then $ f(x) = \frac 2x$ and $ f(x)\in A$. So $ f(f(x)) = \frac 2{f(x)} = x$ and $ \frac 1x + \frac 1{f(f(x))} =$ $ \frac 1x + \frac 1x = f(x)$
And so $ f(x) = \frac 1x + \frac 1{f(f(x))}$

2) $ x\in B$ such that $ xh(x) = (p + 1)p^{n(x)}$
Then $ f(x)=h(x)p^{ - n(x) - 1}$ and so $ f(x)\in B$ and $ f(x)\sim h(x)$ and so $ h(f(x)) = h(x)$

So $ f(x) = h(f(x))p^{ - n(x) - 1}$ and $ \frac {h(f(x))}{f(x)} = p^{n(x) + 1}$ and so $ n(f(x)) = n(x) + 1$

So $ \frac {h(f(x))}{f(x)} = p^{n(f(x))}$ and $ f(f(x)) = \frac {(p + 1)p^{n(f(x))}}{h(f(x))}$ ${ = \frac {(p + 1)p^{n(x) + 1}}h(x)}$

Then $ \frac 1x + \frac 1{f(f(x))}$ $ = \frac {h(x)}{(p + 1)p^{n(x)}} + \frac {h(x)}{(p + 1)p^{n(x) + 1}}$ $ = \frac {(p + 1)h(x)}{(p + 1)p^{n(x) + 1}}$ $ = \frac {h(x)}{p^{n(x) + 1}}$ $ = f(x)$

And so $ f(x) = \frac 1x + \frac 1{f(f(x))}$

3) $ x\in B$ such that $ \frac {h(x)}x = p^{n(x)}$
Then $ f(x) = \frac {(p + 1)p^{n(x)}}{h(x)}$ and so $ f(x)\in B$ and $ f(x)h(x) = (p + 1)p^{n(x)}$ so $ f(x)\sim h(x)$ and so $ h(f(x)) = h(x)$

So $ f(x)h(f(x)) = (p + 1)p^{n(x)}$ and $ n(f(x)) = n(x)$

Then $ f(x)h(f(x)) = (p + 1)p^{n(f(x))}$ and so $ f(f(x)) = h(f(x))p^{ - n(f(x)) - 1}$ $ = h(x)p^{ - n(x) - 1}$

So $ \frac 1x + \frac 1{f(f(x))}$ $ = \frac {p^{n(x)}}{h(x)} + \frac {p^{n(x) + 1}}{h(x)}$ $ = \frac {(p + 1)p^{n(x)}}{h(x)}$ $ = f(x)$

And so $ f(x) = \frac 1x + \frac 1{f(f(x))}$


Q.E.D

[\/hide]



========================== end of example ================================

I would be very interested, Diogene, on informations about the origin of the problem (it's a crazy problem for olympiad, or olympiad preparation, in my opinion).

And I would be very interested in your own general solution.

Thanks in advance.
\end{solution}



\begin{solution}[by \href{https://artofproblemsolving.com/community/user/112}{Diogene}]
	Hello pco,
 I was inspired  by http://www.mathlinks.ro/viewtopic.php?t=307928 
The functional system $ f(x) = \frac ax + \frac 1{f(g(x))},\ g(x) = \frac ax + \frac 1{g(f(x))}$ has only the solution $ f(x) = g(x) = \frac {a + 1}x$ if $ a > 1$
If $ a \leq 1$ the probleme is very hard, and I am not much more advanced than you .

 :cool:
\end{solution}
*******************************************************************************
-------------------------------------------------------------------------------

\begin{problem}[Posted by \href{https://artofproblemsolving.com/community/user/40136}{bjh7790}]
	find all function $ f: \mathbb R \to \mathbb R$ that satisfy the following condition for all reals $x$ and $y$:
\[f(x + yf(x)) = f(x) + xf(y).\]
	\flushright \href{https://artofproblemsolving.com/community/c6h311094}{(Link to AoPS)}
\end{problem}



\begin{solution}[by \href{https://artofproblemsolving.com/community/user/29428}{pco}]
	\begin{tcolorbox}find all function $ f: \mathbb R \to \mathbb R$ that satisfy the following condition.

$ f(x + yf(x)) = f(x) + xf(y)$\end{tcolorbox}

This problem has already been posted and I answered with a different demo (but same result :) )

Let $ P(x,y)$ be the assertion $ f(x+yf(x))=f(x)+xf(y)$

(i) $ f(x)=0$ $ \forall x$ is a trivial solution. So we'll consider from now that $ \exists c$ such that $ f(c)\ne 0$

(ii) $ P(1,0)$ $ \implies$ $ f(1)=f(1)+f(0)$ and so $ f(0)=0$

(iii) If $ f(a)=0$ then $ P(a,c)$ $ \implies$ $ 0=af(c)$ and so $ a=0$

(iv) $ P(-1,-1)$ $ \implies$ $ f(-1-f(-1))=0$ $ \implies$ (from iii) $ f(-1)=-1$

If $ f(1)\ne 1$, $ P(1,\frac 1{1-f(1)}$ $ \implies$ $ f(\frac 1{1-f(1)})=f(1)+f(\frac 1{1-f(1)})$ and so $ f(1)=0$ which is impossible (see iii). So $ f(1)=1$

(v) $ P(1,x)$ $ \implies$ $ f(x+1)=f(x)+1$ and so $ f(x)=f(x+1)-1$ and so $ f(x-1)=f(x)-1$ and so $ f(n)=n$ $ \forall n\in\mathbb Z$

(vi) Let $ n\in\mathbb Z$ : $ P(n,x-1)$ $ \implies$ $ f(nx)=n+nf(x-1)=n(f(x-1)+1)=nf(x)$ and so $ f(nx)=nf(x)$ and so $ f(x)=x$ $ \forall x\in\mathbb Q$

(vii) If $ f(a)=a$ for $ a\ne 0$, then $ P(a,x-1)$ $ \implies$ $ f(ax)=a+af(x-1)=af(x)$ and then $ P(a,y)$ may be written $ f(a+ya)=a+af(y)=a+f(ay)$; Setting then $ y=\frac xa$, we get $ f(a+x)=a+f(x)$, still true for $ a=0$

(viii) $ P(x,1)$ $ \implies$ $ f(x+f(x))=x+f(x)$. Using then $ a=x+f(x)$ in (vii), we get the new assertion $ Q(x,y)$ : $ f(x+y+f(x))=x+f(x)+f(y)$

(ix) $ Q(x,-x)$ $ \implies$ $ f(f(x))=x$

(x) For $ x\ne 0$, $ P(x,1+\frac y{f(x)})$ $ \implies$ $ f(x+y + f(x))=f(x)+xf(1+\frac y{f(x)})$ $ =x+f(x)+xf(\frac y{f(x)})$
Comparing this with $ Q(x,y)$, we get $ x+f(x)+f(y)=$ $ x+f(x)+xf(\frac y{f(x)})$ and so $ f(y)=xf(\frac y{f(x)})$
Using $ y=uv$ and $ x=f(u)$, we get then $ f(uv)=f(u)f(v)$ $ \forall uv$

(xi) from (x) we get that $ f(u^2)=f(u)^2$ and so $ f(x)>0$ $ \forall x>0$ and $ f(x)<0$ $ \forall x<0$
So, for $ x\ne 0$ $ P(x,\frac y{f(x)})$ $ \implies$ $ f(x+y)=f(x)+xf(\frac y{f(x)})$ and $ xf(\frac y{f(x)}$ has same sign as $ y$. So $ f(x)$ is increasing.
And since $ f(x)=x$ $ \forall x\in\mathbb Q$, we get that $ f(x)=x$ $ \forall x\in\mathbb R$

Hence the two solutions :
1) $ f(x)=0$ $ \forall x$
2) $ f(x)=x$ $ \forall x$
\end{solution}
*******************************************************************************
-------------------------------------------------------------------------------

\begin{problem}[Posted by \href{https://artofproblemsolving.com/community/user/66394}{reason}]
	Find all $ f: \mathbb{R}\to\mathbb{R}$ such that for all $x,y \in \mathbb R$,
\[ f(x+y) = f(x)f(y) + \gamma\sin{(x)}\sin{(y)}\]
where $ \gamma$ is a real number.
	\flushright \href{https://artofproblemsolving.com/community/c6h311129}{(Link to AoPS)}
\end{problem}



\begin{solution}[by \href{https://artofproblemsolving.com/community/user/29428}{pco}]
	\begin{tcolorbox}hi!!
find all $ f: \mathbb{R}\to\mathbb{R}$ such that:

   $ f(x + y) = f(x)f(y) + \gamma\sin{x}\sin{y}$    

  with $ \gamma\in{\mathbb{R}}$.

thanx.\end{tcolorbox}

Let $ P(x,y)$ be the assertion $ f(x + y) = f(x)f(y) + a\sin(x)\sin(y)$

Let $ f(\frac {\pi}2) = b$

1) $ a = 0$
========
We get $ f(x + y) = f(x)f(y)$, classical equation with two solutions :
1.1) $ f(x) = 0$ $ \forall x$
1.2) $ f(x) = e^{h(x)}$ where $ h(x)$ is any solution of Cauchy equation $ h(x + y) = h(x) + h(y)$

2) $ a\ne 0$
===========
Then $ f(x)$ is not equal to zero for every $ x$ and so exists $ w$ such that $ f(w)\ne 0$
$ P(w,0)$ $ \implies$ $ f(w) = f(w)f(0)$ and so $ f(0) = 1$

$ P(\frac {\pi}2,\frac {\pi}2)$ $ \implies$ $ f(\pi) = b^2 + a$

$ P(\pi,\frac {\pi}2)$ $ \implies$ $ f(\frac {3\pi}2) = b(b^2 + a)$

$ P(\frac {3\pi}2,\frac {\pi}2)$ $ \implies$ $ f(2\pi) = b^2(b^2 + a) - a$

But $ P(\pi,\pi)$ $ \implies$ $ f(2\pi) = f(\pi)^2$ and so $ b^2(b^2 + a) - a = (b^2 + a)^2$ and so $ a(a + b^2 + 1) = 0$ and so $ a + b^2 + 1 = 0$

This implies $ a\le - 1$ and so $ 1\ge - \frac 1a > 0$ and $ \exists x_0$ such that $ \sin(x_0)^2 = - \frac 1a$

$ P(x_0, - x_0)$ $ \implies$ $ 1 = f(x_0)f( - x_0) - a\sin(x_0)^2$ and so $ f(x_0)f( - x_0) = 0$ and so the equation $ f(x) = 0$ has real roots.

Let then $ u$ any root of $ f(x) = 0$ :

$ P(u, - u)$ $ \implies$ $ 1 = - a\sin(u)^2$
$ P(x - u,u)$ $ \implies$ $ f(x) = a\sin(x - u)\sin(u) = - \frac {\sin(x - u)}{\sin(u)}$

And it's easy to check back that this indeed is a solution.

3) Synthesis of solutions :
==========================
If $ a = 0$ : $ f(x) = 0$ $ \forall x$ or $ f(x) = e^{h(x)}$ where $ h(x)$ is any solution of Cauchy equation $ h(x + y) = h(x) + h(y)$

If $ a > - 1$ and $ a\ne 0$ : No solution

If $ a\le - 1$ : For each $ u$ such that $ \sin(u)^2 = - \frac 1a$, we get a solution $ f(x) = - \frac {\sin(x - u)}{\sin(u)}$
\end{solution}
*******************************************************************************
-------------------------------------------------------------------------------

\begin{problem}[Posted by \href{https://artofproblemsolving.com/community/user/8994}{Number1}]
	Find all bijections $ f: \mathbb{N}\rightarrow \mathbb{N}$ such that if $n|f(m)$, then $ f(n)|m$.
	\flushright \href{https://artofproblemsolving.com/community/c6h311304}{(Link to AoPS)}
\end{problem}



\begin{solution}[by \href{https://artofproblemsolving.com/community/user/29428}{pco}]
	\begin{tcolorbox}Find all bijection $ f: \mathbb{N}\rightarrow \mathbb{N}$ such that:

if $ n|f(m)$   then   $ f(n)|m$.\end{tcolorbox}

$ f(m)|f(m)$ and so $ f(f(m))|f(m)$ and so $ f(f(x))\le x$ $ \forall x$. So, since $ f(f(x))$ is a bijection : $ f(f(x)) = x$ $ \forall x$

If $ m|n$, then $ m|f(f(n))$, then $ f(m)|f(n)$ and, since $ f(f(x)) = x$, we get $ a|b$ $ \iff$ $ f(a)|f(b)$

It's then immediate to find a general solution :
Let $ A$ and $ B$ a split of the set of all prime numbers in two equinumerous subsets and $ h(x)$ a bijection from $ A\to B$.

$ f(x)$ is defined as :

$ f(1) = 1$
$ \forall$ prime $ p\in A$ $ f(p) = h(p)$
$ \forall$ prime $ p\in B$ $ f(p) = h^{[ - 1]}(p)$
$ \forall$ composite number $ > 1$ : $ f(\prod p_i^{n_i}) = \prod f(p_i)^{n_i}$
\end{solution}



\begin{solution}[by \href{https://artofproblemsolving.com/community/user/29428}{pco}]
	\begin{tcolorbox}[quote="Number1"]Find all bijection $ f: \mathbb{N}\rightarrow \mathbb{N}$ such that:

if $ n|f(m)$   then   $ f(n)|m$.\end{tcolorbox}

$ f(m)|f(m)$ and so $ f(f(m))|f(m)$ and so $ f(f(x))\le x$ $ \forall x$. So, since $ f(f(x))$ is a bijection : $ f(f(x)) = x$ $ \forall x$

If $ m|n$, then $ m|f(f(n))$, then $ f(m)|f(n)$ and, since $ f(f(x)) = x$, we get $ a|b$ $ \iff$ $ f(a)|f(b)$

It's then immediate to find a general solution :
Let $ A$ and $ B$ a split of the set of all prime numbers in two equinumerous subsets and $ h(x)$ a bijection from $ A\to B$.

$ f(x)$ is defined as :

$ f(1) = 1$
$ \forall$ prime $ p\in A$ $ f(p) = h(p)$
$ \forall$ prime $ p\in B$ $ f(p) = h^{[ - 1]}(p)$
$ \forall$ composite number $ > 1$ : $ f(\prod p_i^{n_i}) = \prod f(p_i)^{n_i}$\end{tcolorbox}

Sorry, there is a mistake here. I missed some solutions :

The problem is equivalent to identify all bijections $ h(x)$ from the set $ \mathbb P$ of prime numbers into itself such that $ h(h(x)) = x$ and then we have  $ f(\prod p_i^{n_i}) = \prod h(p_i)^{n_i}$

And, to identify all $ h(x)$, we must split $ \mathbb P$ in \begin{bolded}three \end{bolded}\end{underlined}subsets $ A,B,C$ with $ A$ and $ B$ equinumerous and $ b(x)$ a bijection from $ A\to B$. Then :
$ \forall p\in A$ : $ h(p) = b(p)$
$ \forall p\in B$ : $ h(p) = b^{[ - 1]}(p)$
$ \forall p\in C$ : $ h(p) = p$  (I forgot this)
\end{solution}
*******************************************************************************
-------------------------------------------------------------------------------

\begin{problem}[Posted by \href{https://artofproblemsolving.com/community/user/63660}{Victory.US}]
	Let $f: \mathbb R \to \mathbb R$ be a continuous function such that \[ f(x + f(x)) = f(x), \quad \forall x \in \mathbb R.\]
Prove that $ f$ is a constant function.
	\flushright \href{https://artofproblemsolving.com/community/c6h311574}{(Link to AoPS)}
\end{problem}



\begin{solution}[by \href{https://artofproblemsolving.com/community/user/29428}{pco}]
	\begin{tcolorbox}Let$ f: R \to R$ be continuous function such that $ f(x + f(x)) = f(x)$, $ \forall x \in R$
Prove that $ f$ is a const function\end{tcolorbox}

Constant functions are obviously solutions. So let us consider from now that $ f(x)$ is non constant.

From $ f(x+f(x))=f(x)$, we immediately get $ f(x+nf(x))=f(x)$ $ \forall x,\forall n\in\mathbb N_0$

Let then $ x\ne y$
Since $ f(x)$ is continuous and non constant, we can find $ u$ such that $ f(u)=\frac pq\in\mathbb Q$ (with $ q\in\mathbb N$)and $ f(u)$ as near as I want from $ f(x)$
For same reasons, I can find $ v$ such that $ f(v)\notin\mathbb Q$ and $ f(v)$ as near as I want from $ f(y)$

We know that $ f(u+nq)=f(u)$ $ \forall n\in\mathbb N_0$
We also know that $ f(v+mqf(v))=f(v)$ $ \forall m\in\mathbb N_0$

Since $ f(v)$ is irrational, it's also well known that $ \{kf(v)\}$ is dense in $ [0,1]$ and so we can find $ m$ such that $ \{mf(v)\}$ is as near as we want from $ \{\frac{u-v}q\}$ and so $ n$ and $ m$ such that $ u+nq$ can be as near as we want from  $ v+mqf(v)$.

Then, continuity implies $ f(u)=f(v)$ and so $ f(x)=f(y)$
Hence the result.
\end{solution}
*******************************************************************************
-------------------------------------------------------------------------------

\begin{problem}[Posted by \href{https://artofproblemsolving.com/community/user/50645}{stvs_f}]
	Find all $f: \mathbb N \to \mathbb N \setminus \{1\}$ such that
\[f(n)+f(n+1)=f(n+2)f(n+3)-168\]
for all positive integers $n$.
	\flushright \href{https://artofproblemsolving.com/community/c6h311579}{(Link to AoPS)}
\end{problem}



\begin{solution}[by \href{https://artofproblemsolving.com/community/user/29428}{pco}]
	\begin{tcolorbox}find all $ f: N \to N - \{1\}$ such that :
$ f(n) + f(n + 1) = f(n + 2)f(n + 3) - 168$\end{tcolorbox}

Let $ P(n)$ be the assertion $ f(n) + f(n + 1) = f(n + 2)f(n + 3) - 168$

Subtracting $ P(n)$ from $ P(n+1)$, we get $ f(n+2)-f(n)=f(n+3)(f(n+4)-f(n+2))$

And so : $ f(n+2)-f(n)=\left(\prod_{i=1}^pf(n+1+2i)\right)(f(n+2p+2)-f(n+2p))$ $ \forall n,p$
And since $ f(x)\ne 1$ $ \forall x$, the product $ \prod_{i=1}^pf(n+1+2i)\to+\infty$ and so $ f(n+2)=f(n)$

So the sequence is $ (a,b,a,b,a,b,...)$ with $ a,b$ such that $ a+b=ab-168$ $ \iff$ $ (a-1)(b-1)=169$ hence the three solutions :

$ 14,14,14,14,14,14,14,....$
$ 170,2,170,2,170,2,170,2,...$
$ 2,170,2,170,2,170,2,170,...$
\end{solution}
*******************************************************************************
-------------------------------------------------------------------------------

\begin{problem}[Posted by \href{https://artofproblemsolving.com/community/user/50645}{stvs_f}]
	Find all $f: \mathbb N \to \mathbb N$ such that
\[f(n+1)+f(n+3)=f(n+5)f(n+7)-1375\]
for all positive integers $n$.
	\flushright \href{https://artofproblemsolving.com/community/c6h311580}{(Link to AoPS)}
\end{problem}



\begin{solution}[by \href{https://artofproblemsolving.com/community/user/29428}{pco}]
	\begin{tcolorbox}find all $ f: N \to N$ that:
$ f(n + 1) + f(n + 3) = f(n + 5)f(n + 7) - 1375$\end{tcolorbox}

Let $ P(n)$ be the assertion $ f(n+1)+f(n+3)=f(n+5)f(n+7)-1375$

First notice that we can study solutions $ f(n)$ for $ n$ odd and translate them immediately for $ n$ even.

Subtracting $ P(n)$ from $ P(n+2)$, we get new assertion $ Q(n)$ : $ f(n+5)-f(n+1)=f(n+7)(f(n+9)-f(n+5)$

$ Q(1)$ $ \implies$ $ f(6)-f(2)=f(8)(f(10)-f(6)$ and so $ f(6)-f(2)=f(8)f(12)f(16)(f(18)-f(14)$ ... and so on. So :

either $ f(6+4n+4)=f(6+4n)$ for some $ n$ and so $ f(6)=f(2)$ and so $ f(4n+6)=f(4n+2)$ $ \forall n\ge 0$
either $ f(4n+6)\ne f(4n+2)$ $ \forall n\ge 0$ and so $ f(4n)=1$ $ \forall n$ great enough. But then :
$ Q(3)$ $ \implies$ $ f(8)-f(4)=f(10)(f(12)-f(8)$ and so $ f(4n)=1$ $ \forall n>0$
Then $ P(4n+1)$ $ \implies$ $ f(4n+2)+1=f(4n+6)-1375$ and so $ f(4n+6)=f(4n+2)+1376$

As a first conclusion, we got :
Either $ f(4n+2)=f(2)$ $ \forall n$
Either $ f(4n)=1$ $ \forall n$ and $ f(4n+2)=1376n+f(2)$ $ \forall n$

Using the same approach starting from $ Q(3)$ instead of $ Q(1)$, we get 
Either $ f(4n)=f(4)$ $ \forall n$
Either $ f(4n+2)=1$ $ \forall n$ and $ f(4n+4)=1376n+f(4)$ $ \forall n$

Combining these two conclusions, we get three possibilities :
1) $ f(4n)=a$ and $ f(4n+2)=b$ $ \forall n$ and checking back in the equation, this gives $ a+b=ab-1375$ $ \iff$ $ (a-1)(b-1)=1376=2^543$

2) $ f(4n+2)=1$ $ \forall n$ and $ f(4n+4)=1376n+f(4)$ $ \forall n$ and it is easy to check that this indeed is a solution

3) $ f(4n)=1$ $ \forall n$ and $ f(4n+2)=1376n+f(2)$ $ \forall n$ and it is easy to check that this indeed is a solution

Hence the $ 14$ solutions for even arguments :
$ 2,1377,2,1377,2,1377, ....$
$ 3,689,3,689,3,689,....$
$ 5,345,5,345,5,345,....$
$ 9,173,9,173,9,173,...$
$ 17,87,17,87,17,87,....$
$ 33,44,33,44,33,44,....$
$ 1377,2,1377,2,1377,2 ...$
$ 689,3,689,3,689,3....$
$ 345,5,345,5,345,5....$
$ 173,9,173,9,173,9...$
$ 87,17,87,17,87,17....$
$ 44,33,44,33,44,33....$
$ 1,u,1,u+1376,1,u+2\cdot 1376, ...$
$ u,1,u+1376,1,u+2\cdot 1376,1 ...$

And we obviously have the $ 14$ same solutions for odd arguments, leading to $ 196$ solutions.
For example :

$ 9,17,173,87,9,17,173,87,9,17,173,87,...$
$ u,2,1,1377,u+1376,2,1,1377,u+2\cdot 1376, ...$
$ u,1,1,v,u+1376,1,1,v+1376, u+2\cdot 1376,1,1  ...$
...
\end{solution}
*******************************************************************************
-------------------------------------------------------------------------------

\begin{problem}[Posted by \href{https://artofproblemsolving.com/community/user/29191}{zaya_yc}]
	Find all $ f: \mathbb{R}\to\mathbb{R}$ such that for all $x,y \in \mathbb R$,
\[f(x+y)f(x-y)=(f(x)f(y))^{2}.\]
	\flushright \href{https://artofproblemsolving.com/community/c6h311747}{(Link to AoPS)}
\end{problem}



\begin{solution}[by \href{https://artofproblemsolving.com/community/user/37364}{kihe_freety5}]
	I think that we need the condition $ f$ is continuous
\end{solution}



\begin{solution}[by \href{https://artofproblemsolving.com/community/user/29428}{pco}]
	\begin{tcolorbox}$ \forall x,y\in R$ 

$ f(x + y)f(x - y) = (f(x)f(y))^{2}$ . 

Find all $ f(x)$\end{tcolorbox}

I think that this problem is rather hard without any supplementary condition (continuity for example).
Here is a partial solution :

Let $ P(x,y)$ be the assertion $ f(x + y)f(x - y) = (f(x)f(y))^2$

1) Let us first have a look on the case where $ \exists u$ such that $ f(u) = 0$
================================================
Let $ A = \{x$ such that $ f(x)\ne 0\}$
If $ A = \emptyset$, then $ f(x) = 0$ $ \forall x$ and we got a solution. We'll from now consider $ A\ne\emptyset$

$ P(x,y)$ shows that $ x,y\in A$ $ \implies$ $ x + y,x - y\in A$, so $ 0\in A$
For $ x\in A$, $ P(x,0)$ $ \implies$ $ f(0)^2 = 1$. 
If $ f(x)$ is a solution, then $ - f(x)$ is a solution too. So wlog say from now $ f(0) = + 1$
$ P(\frac x2,\frac x2)$ $ \implies$ $ f(x) = f(\frac x2)^4$ and so $ f(x)\ge 0$ $ \forall x$

$ \forall x\in A$, $ P(x,x)$ $ \implies$ $ f(2x) = f(x)^4$
$ \forall x\in A$, $ P(2x,x)$ $ \implies$ $ f(3x) = f(x)^9$
And an immediate induction shows that $ \forall x\in A$, $ f(nx) = f(x)^{n^2}$ $ \forall n\in\mathbb Z$
From there we immediately get $ \forall x\in A$, $ f(px) = f(x)^{p^2}$ $ \forall p\in\mathbb Q$

So $ x\in A$ $ \implies$ $ px\in A$ $ \forall p\in\mathbb Q$
So $ A$ is a $ \mathbb Q$-vector space.

The reciprocal assertion may easily be established : if we have a solution $ f(x)$ we can find a new solution $ g(x)$ by taking $ g(x)$ as the restriction of $ f(x)$ on any $ \mathbb Q$-subvector space of $ \mathbb R$ and $ g(x) = 0$ anywhere else.

2) Let then $ A$ any $ \mathbb Q$-subvector space of $ \mathbb R$ over which $ f(x)\ne 0$ and let us have a look on $ f(x)$ over A.
====================================================================================
From the above elements, we can say :
$ f(x) > 0$ $ \forall x\in A$ (we considered the solutions where $ f(0) = + 1$)
$ f(px) = f(x)^{p^2}$ $ \forall p\in\mathbb Q$

Since $ f(x) > 0$ and $ f( - x) = f(x)$, we can set $ f(x) = e^{h(x^2)}$ for some $ h(x)$ and then :

$ h(x^2 + 2xy + y^2) + h(x^2 - 2xy + y^2) = 2h(x^2) + 2h(y^2)$

And we have at least all solutions of Cauchy's equation $ h(x + y) = h(x) + h(y)$

\begin{italicized} And here, we should study if it exists some other solution .....\end{italicized}\end{underlined}

3) Synthesis of solutions we found up to now :
==============================
The solutions we found up to now are :

3.1) : $ f(x) = 0$ $ \forall x$

3.2) : Let $ A$ any $ \mathbb Q$-subvector space of $ \mathbb R$ and $ h(x)$ any solution of Cauchy's equation $ h(x + y) = h(x) + h(y)$
$ f(x) = e^{h(x^2)}$ $ \forall x\in A$
$ f(x) = 0$ $ \forall x\notin A$
example : $ f(x)=1$ $ \forall x\in\mathbb Q$ and $ f(x)=0$ $ \forall x\notin \mathbb Q$


3.3) : Let $ A$ any $ \mathbb Q$-subvector space of $ \mathbb R$ and $ h(x)$ any solution of Cauchy's equation $ h(x + y) = h(x) + h(y)$
$ f(x) = - e^{h(x^2)}$ $ \forall x\in A$
$ f(x) = 0$ $ \forall x\notin A$

It would we rather easy to show that the only continuous solutions are $ f(x) = 0$ and $ f(x) = e^{ax^2}$ and $ f(x) = - e^{ax^2}$ but unfortunately, continuity is not a constraint.

Please, zaya_yc, could you kindly tell us where is this strange problem coming from ?
\end{solution}
*******************************************************************************
-------------------------------------------------------------------------------

\begin{problem}[Posted by \href{https://artofproblemsolving.com/community/user/46488}{Raja Oktovin}]
	Find all functions $ f: \mathbb{R} \rightarrow \mathbb{R}$ satisfying
\[ f(f(x + y)) = f(x + y) + f(x)f(y) - xy\]
for all real numbers $x$ and $y$.
	\flushright \href{https://artofproblemsolving.com/community/c6h312067}{(Link to AoPS)}
\end{problem}



\begin{solution}[by \href{https://artofproblemsolving.com/community/user/29428}{pco}]
	\begin{tcolorbox}Find all functions $ f: \mathbb{R} \rightarrow \mathbb{R}$ satisfying
\[ f(f(x + y)) = f(x + y) + f(x)f(y) - xy\]
for all real numbers $ x$ dan $ y$.\end{tcolorbox}
Let $ P(x,y)$ be the assertion $ f(f(x+y))=f(x+y)+f(x)f(y)-xy$
Let $ f(0)=a$

It's elementary first (just plug in the original equation) to see that the only solution of the type $ f(x)=bx+a$ is $ f(x)=x$

Subtracting $ P(x,y)$ from $ P(x+y,0)$, we get the new assertion $ Q(x,y)$ : $ af(x+y)=f(x)f(y)-xy$

Then $ Q(a,-a)$ $ \implies$ $ f(a)f(-a)=0$ and so $ \exists u\in\{f(-a),f(a)\}$ such that $ f(u)=0$

Then $ Q(x-u,u)$ $ \implies$ $ af(x)=-u(x-u)$

If $ a=0$, $ Q(x,y)$ implies $ f(x)f(y)=xy$ and so either $ f(x)=x$ $ \forall x$, which indeed is a solution, either $ f(x)=-x$ $ \forall x$, which is not a solution.

If $ a\ne 0$, this implies $ f(x)=-\frac uax +\frac{u^2}a=bx+a$ for some $ b$ and so (see beginning lines), $ f(x)=x$, which is impossible since $ f(0)\ne 0$ in this subcase.

Hence the unique solution $ \boxed{f(x)=x}$ $ \forall x$
\end{solution}



\begin{solution}[by \href{https://artofproblemsolving.com/community/user/32234}{Mashimaru}]
	I have another solution with much more calculation the the one of Mr.\begin{bolded}pco\end{bolded}.

Let $ P(x,y)$ be the assertion $ f(f(x+y)) = f(x+y) + f(x)f(y) - xy$ and let $ a: =f(0)$.

We have $ P(x,0) \Rightarrow f(f(x)) = (a+1)f(x),\forall x\in\mathbb{R}$. Thus, let $ x=0$ we have $ f(a) = a(a+1)$. Moreover, the assertion $ P(x,y)$ becomes: $ af(x+y) = f(x)f(y) - xy,\forall x,y,\in\mathbb{R}$.

Suppose that $ a\neq 0$. We have:

$ P(a,a) \Rightarrow af(2a) = a^2(a+1)^2 - a^2$, yielding $ f(2a) = a^2(a+2)$.

$ P(2a,a) \Rightarrow af(3a) = a(a+1)a^2(a+2) - 2a^2$, yielding $ f(3a) = a^2(a+1)(a+2) - 2a = a(a^3 + 3a^2 + 2a -2)$.

$ P(3a,a) \Rightarrow af(4a) = a(a+1)a(a^3 + 3a^2 + 2a -2) - 3a^2$

$ P(2a,2a) \Rightarrow af(4a) = a^4(a+2)^2 - 4a^2$.

This means that $ a(a+1)a(a^3 + 3a^2 + 2a -2) - 3a^2 = a^4(a+2)^2 - 4a^2\Leftrightarrow a^2(a-3)(a+3) = 0$

But we have supposed that $ a\neq 0$ then $ a=3$ or $ a=-3$.

1. If $ a=3$ then $ P(x,y)$ becomes $ 3f(x+y) = f(x)f(y) - xy$ and hence $ f(3) = 3\cdot (3+1) = 12$.

$ P(x,1) \Rightarrow 3f(x+1) = f(1)f(x) - x$.

$ P(x+1,1) \Rightarrow 3f(x+2) = f(1)\frac{f(1)f(x) - x}{3} - (x+1)$

$ P(x+2,1) \Rightarrow 3f(x+3) = f(1)\frac{f(1)\frac{f(1)f(x) - x}{3} - (x+1)}{3} - (x+2)$.

$ P(x,3) \Rightarrow 3f(x+3) = f(x)f(3) - 3x = 12f(x) - 3x$.

Therefore, $ 12f(x) - 3x = f(1)\frac{f(1)\frac{f(1)f(x) - x}{3} - (x+1)}{3} - (x+2),\forall x\in\mathbb{R}$, contradiction.

2. If $ a= -3$, the same argument yields another contradiction.

Thus $ a=0$ and we have $ f(x)f(y) = xy,\forall x,y\in\mathbb{R}$. Put $ y=1$ then $ f(x) = kx,\forall x\in\mathbb{R}$, where $ k=\frac{1}{f(1)}$. Now the assertion $ P(x,y)$ yields that $ k=1$.

Therefore, the only solution is $ f(x) = x$.
\end{solution}



\begin{solution}[by \href{https://artofproblemsolving.com/community/user/73926}{RobRoobiks}]
	Hello!!!!!!
[hide=" my much easier solution which i am doubtful of"]
Let $ y=0$
$ f(f(x))=f(x)+f(0)f(x)$
$ f(f(x))=(f(0)+1)f(x)$
thus its easy to see
$ f(x)=(f(0)+1)x$

Now to find f(0)
Let $ x=0$ as well
$ f(0)=(f(0)+1)*0$
$ f(0)=0$ and $ f(x)=x$

Is that valid?
[\/hide]
Roobiks
[\/hide]
\end{solution}



\begin{solution}[by \href{https://artofproblemsolving.com/community/user/29428}{pco}]
	\begin{tcolorbox}Hello!!!!!!
Let $ y = 0$
$ f(f(x)) = f(x) + f(0)f(x)$
$ f(f(x)) = (f(0) + 1)f(x)$
thus its easy to see
$ f(x) = (f(0) + 1)x$
\end{tcolorbox}

Yes, easy to see, but wrong. You got $ f(x) = (f(0) + 1)x$  $ \forall x\in f(\mathbb R)$ and not $ \forall x\in\mathbb R$ (you did not show that $ f(x)$ is a surjection).
\end{solution}



\begin{solution}[by \href{https://artofproblemsolving.com/community/user/44887}{Mathias_DK}]
	\begin{tcolorbox}Therefore, $ 12f(x) - 3x = f(1)\frac {f(1)\frac {f(1)f(x) - x}{3} - (x + 1)}{3} - (x + 2),\forall x\in\mathbb{R}$, contradiction.\end{tcolorbox}
How do you reach the contradiction? For example if $ f(1) = 0$ then it is equivalent to $ f(x) = \frac {x - 1}{6}$. If $ f(1) \neq \sqrt [3]{108}$ then it is equivalent to $ f(x) = ax + b$ for some $ a,b \in \mathbb{R}$. If $ f(1) = \sqrt [3]{108}$ then you can reach the contradiction at once. So you can only use it to see that $ f$ is a linear function. (Sorry if this was obvious)
\end{solution}



\begin{solution}[by \href{https://artofproblemsolving.com/community/user/30264}{lasha}]
	$ f(f(x + y)) = f(x + y) + f(x)f(y) - xy$     (1)
     Denote $ f(0) = c$.
     Take $ y = 0$ to get    $ f(f(x)) = f(x)(1 + c)$.    (2)
     Set $ y = - x$ in (1):    $ f(f(0)) - f(0) = x^{2} - f(x)f( - x)$, but  by (2),  $ f(f(0)) - f(0) = c^{2}$.  
Hence, $ f( - x)f(x) = c^{2} - x^{2}$   (3).   Take $ x = c$  in (3) to conclude there is a real number $ u$ satisfying $ f(u) = 0$. Take $ x = u$ in (2):   $ f(0) = f(f(u)) = f(u)(1 + c) = 0$. It means $ f(f(x)) = f(x)$, as $ c = 0$, but than from (1),  $ f(x)f(y) = xy$  for all $ x,y$.
     Easy to verify $ f(x) = 0$, for all $ x$ is not a solution. Take $ t$ such that $ f(t) = w$ and $ w$ is nonzero.  
     $ wf(x) = xt$, which can be rewritten as $ f(x) = kx$, where $ k$ is a fixed real number. Plugging it in (1),  
     $ (x + y)(k^{2} - k) = xy(k^{2} - 1)$, for all $ x,y$. If $ x = - y$ and they are both nonzero, the last equation would give $ k^{2} - 1 = 0$. Assume $ k = - 1$;  Than, for all pairs $ (x,y)$, $ 2(x + y) = 0$, which is clearly false. It means that $ k = 1$ and $ f(x) = x$, which obviously satisfies the given conditions.
\end{solution}
*******************************************************************************
-------------------------------------------------------------------------------

\begin{problem}[Posted by \href{https://artofproblemsolving.com/community/user/46488}{Raja Oktovin}]
	Find all pairs of function $ f: \mathbb{N} \rightarrow \mathbb{N}$ and polynomial with integer coefficients $ p$ such that:
(i) $ p(mn) = p(m)p(n)$ for all positive integers $ m,n > 1$ with $ \gcd(m,n) = 1$, and
(ii) $ \sum_{d|n}f(d) = p(n)$ for all positive integers $ n$.
	\flushright \href{https://artofproblemsolving.com/community/c6h312088}{(Link to AoPS)}
\end{problem}



\begin{solution}[by \href{https://artofproblemsolving.com/community/user/29428}{pco}]
	\begin{tcolorbox}Find all pairs of function $ f: \mathbb{N} \rightarrow \mathbb{N}$ and polynomial with integer coefficients $ p$ such that:
(i). $ p(mn) = p(m)p(n)$ for all positive integers $ m,n > 1$ with $ \gcd(m,n) = 1$, and
(ii). $ \sum_{d|n}f(d) = p(n)$ for all positive integers $ n$.\end{tcolorbox}

$ P(2n)=P(2)P(n)$ $ \forall$ odd $ n>1$ and so $ P(2x)=aP(x)$ $ \forall x$ (else non zero polynomial $ P(2x)-P(2)P(x)$ would have infinitely many roots).

As a consequence, we get $ P(x)=0$ or $ P(x)=x^k$ for some $ k$ but, since $ f(d)>0$ ($ 0\notin\mathbb N$), we get only $ P(x)=x^k$ for some $ k$ 

Using then the second formula for $ n=1$, we get $ f(1)=1$
Using then the second formula for $ n=p$ prime, we get $ f(1)+f(p)=p^k$ and so $ f(p)=p^k-1$
Using then the second formula for $ n=p^2$ with $ p$ prime, we get $ f(1)+f(p)+f(p^2)=p^{2k}$ and so $ f(p^2)=p^{2k}-p^k$

An immediate induction gives $ f(p^n)=p^{nk}-p^{(n-1)k}$ for any prime $ p$ and positive integer $ n$.

Using then the second formula for $ n=pq$ with $ p,q$ prime, we get $ f(1)+f(p)+f(q)+f(pq)=(pq)^k$ and so $ f(pq)=p^kq^k-p^k-q^k+1$ and so $ f(pq)=(p^k-1)(q^k-1)$

And so, with many successive inductions, it's rather easy to show that $ f(\prod p_i^{n_i})=\prod(p_i^{n_ik}-p_i^{(n_i-1)k})$

Hence the result :

$ P(x)=x^k$ and $ f(\prod p_i^{n_i})=\prod(p_i^{n_ik}-p_i^{(n_i-1)k})$
\end{solution}
*******************************************************************************
-------------------------------------------------------------------------------

\begin{problem}[Posted by \href{https://artofproblemsolving.com/community/user/64868}{mahanmath}]
	Find all functions $ f : \mathbb{R}\mapsto\mathbb{R}$ such that \[ f((x - y)^2)=(f(x))^2 -2xf(y)+y^2\] holds for all $x,y \in \mathbb R$.
	\flushright \href{https://artofproblemsolving.com/community/c6h312141}{(Link to AoPS)}
\end{problem}



\begin{solution}[by \href{https://artofproblemsolving.com/community/user/46488}{Raja Oktovin}]
	\begin{tcolorbox}Hi ! Find all function $ f : \mathbb{R}\mapsto\mathbb{R}$ such that $ f((x - y)^2) = (f(x))^2 - 2xf(y) + y^2$\end{tcolorbox}

I have a lengthy solution, please check if i made an error in it. Thank you.

Let $ P(x,y)$ be the assertion for $ f((x - y)^2) = (f(x))^2 - 2xf(y) + y^2$.

$ P(0,0)$, then $ f(0) = f(0)^2$ and thus $ f(0) = 0$ or $ f(0) = 1$.

\begin{italicized}Case 1.\end{italicized} $ f(0) = 0$.

$ P(0,t)$, then $ f(t^2) = t^2$, and thus $ f(t) = t$ for all $ t \ge 0$.

Thus $ (x - y)^2 = (f(x))^2 - 2xf(y) + y^2$ for all real numbers $ x$ and $ y$ and thus $ x^2 - 2xy = (f(x))^2 - 2xf(y)$ for all real numbers $ x$ and $ y$. Here, if we put $ x = y = t$, then $ - t^2 = (f(t))^2 - 2tf(t)$ and thus $ (f(t) - t)^2 = 0$ and thus $ f(t) = t$ for all real numbers $ t$. This is function obviously satisfies the condition.

\begin{italicized}Case 2.\end{italicized} $ f(0) = 1$.

$ P(t,0)$, then $ f(t^2) = (f(t))^2 - 2t$. (*)

$ P(0,t)$, then $ f(t^2) = 1 + t^2$ and thus $ f(t) = 1 + t$ for all $ t \ge 0$. (**)

Now, from (*), we have that $ (f(t))^2 - 2t = 1 + t^2$ so $ (f(t))^2 = (t + 1)^2$ for all real numbers $ t$.

The first function we get is $ f(t) = t + 1$. Check that in this case, $ f((x - y)^2) = x^2 - 2xy + y^2 + 1$ and
\[ (f(x))^2 - 2xf(y) + y^2 = (x + 1)^2 - 2x(y + 1) + y^2 = x^2 + 2x + 1 - 2xy - 2x + y^2 = x^2 - 2xy + y^2 + 1\]
, so this function satisfies the condition.

The second is $ f(x) = - (x + 1)$ for all real numbers $ x$. But $ f(0) = 1$, contradiction.

Now, assume that $ \exists a,b \in \mathbb{R}$ such that $ f(a) = a + 1$ and $ f(b) = - (b + 1)$. Here, we may assume that $ a,b \ne - 1$.
By assertion $ P(a,b)$, then
\[ f((a - b)^2) = (f(a))^2 - 2af(b) + b^2\]

\[ f((a - b)^2) = a^2 + 2a + 1 + 2a(b + 1) + b^2 = a^2 + b^2 + 2ab + 1 + 4a.\]
By assertion $ P(b,a)$, then
\[ f((a - b)^2) = (f(b))^2 - 2bf(a) + a^2 = b^2 + 2b + 1 - 2b(a + 1) + a^2 = a^2 + b^2 - 2ab + 1.\]
Now,
\[ a^2 + b^2 + 2ab + 1 + 4a = a^2 + b^2 - 2ab + 1\]

\[ 4ab + 4a = 0\]
but since $ b \ne 1$, then $ a = 0$.
Now, we obtain that $ f(x) = 1$ for $ x = 0$ and $ f(x) = - (x + 1)$ for $ x \ne 0$. Thus $ f(1) = - 2$ but by (**), $ f(1) = 1 + 1 = 2$, contradiction.

Thus, he solutions are: (1) $ f(x)=x$ for all real numbers $ x$, and (2) $ f(x)=x+1$ for all real numbers $ x$.
\end{solution}



\begin{solution}[by \href{https://artofproblemsolving.com/community/user/29428}{pco}]
	\begin{tcolorbox}Hi ! Find all function $ f : \mathbb{R}\mapsto\mathbb{R}$ such that $ f((x - y)^2) = (f(x))^2 - 2xf(y) + y^2$\end{tcolorbox}

$ f((x - y)^2) = (f(x))^2 - 2xf(y) + y^2$
$ f((y - x)^2) = (f(y))^2 - 2yf(x) + x^2$

Subtracting : $ 0=f(x)^2+2yf(x)+y^2-f(y)^2-2xf(y)-x^2$

$ \iff$ $ (f(x)+y)^2=(f(y)+x)^2$ $ \forall x,y$

Setting $ y=0$ in this equality and naming $ f(0)=a$, we get $ f(x)^2=(x+a)^2$

And so, $ \forall x$, either $ f(x)=x+a$, either $ f(x)=-x-a$

1) Suppose $ \exists u\ne 0$ such that $ f(u)=u+a$
Suppose then $ \exists y$ such that $ f(y)=-y-a$ :
$ (f(u)+y)^2=(f(y)+u)^2$ $ \implies$ $ (u+a+y)^2=(-y-a+u)^2$ $ \implies$ $ y=-a$ and so $ f(x)=x+a$ $ \forall x$
Putting this in the original equation, we get $ a=0$ or $ a=1$ and the solutions $ f(x)=x$ and $ f(x)=x+1$

2) Suppose $ \exists u\ne -a$ such that $ f(u)=-x-a$
Suppose then $ \exists y$ such that $ f(y)=y+a$ :
$ (f(u)+y)^2=(f(y)+u)^2$ $ \implies$ $ (-u-a+y)^2=(y+a+u)^2$ $ \implies$ $ y=0$ and so $ f(x)=-x-a$ $ \forall x\ne 0$
Using this in the original equation with $ y=0$, we see that this is not a solution.


Hence the two solutions $ \boxed{f(x)=x}$ $ \forall x$ and $ \boxed{f(x)=x+1}$ $ \forall x$
\end{solution}



\begin{solution}[by \href{https://artofproblemsolving.com/community/user/32886}{dgreenb801}]
	Let $ y=x$, then we have
$ f(0)=(f(x)-x)^2$
Let c be the positive square root of $ f(0)$, then $ f(x)=x \pm c$, where it is plus for some values and minus for others. Substituting into the original equation, we have
$ (x-y)^2 \pm c = (x \pm c)^2 - 2x(y \pm c)+y^2$
$ \pm c=c^2 \pm 2xc \mp 2xc$
We see here that $ c=0$ is possible, and $ f(x)=x$ is a solution. Now assume $ c$ is not $ 0$.
$ \pm 1=c \pm2x \mp 2x$
The only way this can be true for all $ x$ is if $ c=1$, since $ c$ was the positive square root of $ f(0)$ and the $ 2x$'s always have opposite signs and cancel. So we try $ f(x)=x+1$, and it works. 
So the only two solutions are $ f(x)=x$ and $ f(x)=x+1$
\end{solution}



\begin{solution}[by \href{https://artofproblemsolving.com/community/user/29428}{pco}]
	Nice solution, dgreenb801.

Here is a mixed form between our two solutions  :

Let $ P(x,y)$ be the equation $ f((x - y)^2) = (f(x))^2 - 2xf(y) + y^2$

$ P(0,0)$ $ \implies$ $ f(0) = f(0)^2$ and so either $ f(0) = 0$, either $ f(0) = 1$

1) If $ f(0) = 0$, $ P(x,x)$ $ \implies$ $ 0 = (f(x) - x)^2$ and so $ f(x) = x$

2) If $ f(0) = 1$, subtracting then $ P(0,x)$ from $ P(x,0)$, we get $ f(x)^2 = (x + 1)^2$ and so : $ \forall x$ : either $ f(x) = x + 1$, either $ f(x) = - x - 1$
$ P(x,x)$ $ \implies$ $ 1 = (f(x) - x)^2$ and so : $ \forall x$ : either $ f(x) = x + 1$, either $ f(x) = x - 1$
And so $ f(x) = x + 1$ (else we would have for some $ x$ : $ - x - 1 = x - 1$ and so $ x = 0$ and $ f(0) = - 1\ne + 1$)

And it's easy to check back that these two funtions indeed are solutions.
\end{solution}
*******************************************************************************
-------------------------------------------------------------------------------

\begin{problem}[Posted by \href{https://artofproblemsolving.com/community/user/60875}{Kono}]
	Let $ f(n)$ be the integer closest to $ n^{\frac{1}{4}}$. Find \[ \sum_{k=1}^{1995} \frac{1}{f(k)}.\]
	\flushright \href{https://artofproblemsolving.com/community/c6h312166}{(Link to AoPS)}
\end{problem}



\begin{solution}[by \href{https://artofproblemsolving.com/community/user/29428}{pco}]
	\begin{tcolorbox}Let $ f(n)$ be the integer closest to $ n^{\frac {1}{4}}$. Find $ \sum_{k = 1}^{1995} \frac {1}{f(k)}$.

Can anyone explain how to do this step by step? Please put in all necessary steps and all workings because I'm not too good at this.

Thanks!\end{tcolorbox}

[hide="Basic calculus"]
First notice that $ n^{\frac 14}$ is never of the form ${ p + \frac 12}$, so "closest" integer is defined.

$ f(n) = 1$ $ \iff$ $ 1 - \frac 12 < n^{\frac 14} < 1 + \frac 12$ $ \iff$ $ \frac 1{2^4} < n < \frac {3^4}{2^4}$ $ \iff$ $ n\in[1,5]$

$ f(n) = 2$ $ \iff$ $ 2 - \frac 12 < n^{\frac 14} < 2 + \frac 12$ $ \iff$ $ \frac {3^4}{2^4} < n < \frac {5^4}{2^4}$ $ \iff$ $ n\in[6,39]$

$ f(n) = 3$ $ \iff$ $ 3 - \frac 12 < n^{\frac 14} < 3 + \frac 12$ $ \iff$ $ \frac {5^4}{2^4} < n < \frac {7^4}{2^4}$ $ \iff$ $ n\in[40,150]$

$ f(n) = 4$ $ \iff$ $ 4 - \frac 12 < n^{\frac 14} < 4 + \frac 12$ $ \iff$ $ \frac {7^4}{2^4} < n < \frac {9^4}{2^4}$ $ \iff$ $ n\in[151,410]$

$ f(n) = 5$ $ \iff$ $ 5 - \frac 12 < n^{\frac 14} < 5 + \frac 12$ $ \iff$ $ \frac {9^4}{2^4} < n < \frac {11^4}{2^4}$ $ \iff$ $ n\in[411,915]$

$ f(n) = 6$ $ \iff$ $ 6 - \frac 12 < n^{\frac 14} < 6 + \frac 12$ $ \iff$ $ \frac {11^4}{2^4} < n < \frac {13^4}{2^4}$ $ \iff$ $ n\in[916,1785]$

$ f(n) = 7$ $ \iff$ $ 7 - \frac 12 < n^{\frac 14} < 7 + \frac 12$ $ \iff$ $ \frac {13^4}{2^4} < n < \frac {15^4}{2^4}$ $ \iff$ $ n\in[1786,3164]$

So $ \sum_{k = 1}^{1995}\frac 1{f(k)}$ $ = \sum_{k = 1}^{5}\frac 1{f(k)}$ $ + \sum_{k = 6}^{39}\frac 1{f(k)}$ $ + \sum_{k = 40}^{150}\frac 1{f(k)}$ $ + \sum_{k = 151}^{410}\frac 1{f(k)}$ $ + \sum_{k = 411}^{915}\frac 1{f(k)}$ $ + \sum_{k = 916}^{1785}\frac 1{f(k)}$ $ + \sum_{k = 1786}^{1995}\frac 1{f(k)}$

So $ \sum_{k = 1}^{1995}\frac 1{f(k)}$ $ = 5\frac 1{1}$ $ + 34\frac 1{2}$ $ + 111\frac 1{3}$ $ + 260\frac 1{4}$ $ + 505\frac 1{5}$ $ + 870\frac 1{6}$ $ + 210\frac 1{7}$

So $ \sum_{k = 1}^{1995}\frac 1{f(k)} = 400$
[\/hide]

[hide="More general computation"]
$ f(n) = k$ $ \iff$ $ k - \frac 12 < n^{\frac 14} < k + \frac 12$ $ \iff$ ${ (k - \frac 12)^4 < n < (k + \frac 12)^4}$ 
$ \iff$ $ k^4 - 2k^3 + \frac 32k^2 - \frac 12k + \frac 1{16}$ $ < n <$ $ k^4 + 2k^3 + \frac 32k^2 + \frac 12k + \frac 1{16}$

$ \iff$ $ k^4 - 2k^3 + \frac 32k^2 - \frac 12k + 1$ $ \le n \le$ $ k^4 + 2k^3 + \frac 32k^2 + \frac 12k$

So $ \sum_{n = k^4 - 2k^3 + \frac 32k^2 - \frac 12k + 1}^{k^4 + 2k^3 + \frac 32k^2 + \frac 12k}$ $ \frac 1{f(n)}$ $ = (4k^3 + k)\frac 1k$ $ = 4k^2 + 1$

$ \sum_{n = 1}^{k^4 + 2k^3 + \frac 32k^2 + \frac 12k}$ $ \frac 1{f(n)}$ $ = k + 4\sum_{n = 1}^kn^2$ $ = k + \frac 23k(k + 1)(2k + 1)$

In our case, since $ 7^4 - 2\cdot 7^3 + \frac 327^2 - \frac 127 + 1 = 1786$ $ \le 1995 \le$ $ 7^4 + 2\cdot 7^3 + \frac 327^2 + \frac 127 = 3164$ :

$ \sum_{n = 1}^{6^4 + 2\cdot 6^3 + \frac 326^2 + \frac 126 = 1785}$ $ \frac 1{f(n)}$ $ = 6 + \frac 236\cdot 7\cdot 13$ $ = 370$

$ \sum_{n = 7^4 - 2\cdot 7^3 + \frac 327^2 - \frac 127 + 1 = 1786}^{n = 1995}$ $ \frac 1{f(n)}$ $ = (1995 - 1785)\frac 17 = 30$

Hence the result $ 400$
[\/hide]

[hide="General closed formula"]
Using all the above computations, we get the general formula :

$ \sum_{n=1}^x$ $ \frac 1{f(n)}$ $ =[\sqrt[4]x-\frac 12]$ $ +\frac 23[\sqrt[4]x-\frac 12]([\sqrt[4]x-\frac 12]+1)(2[\sqrt[4]x-\frac 12]+1)$ $ +\frac{x-([\sqrt[4]x+\frac 12]-\frac 12)^4+\frac 1{16}}{[\sqrt[4]x+\frac 12]}$
[\/hide]
\end{solution}
*******************************************************************************

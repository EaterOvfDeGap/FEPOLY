-------------------------------------------------------------------------------

\begin{problem}[Posted by \href{https://artofproblemsolving.com/community/user/103227}{shohvanilu}]
	Does existes a function $f:N->N$ and for all positeve integer n
$f(f(n)+2011)=f(n)+f(f(n))$
	\flushright \href{https://artofproblemsolving.com/community/c6h439124}{(Link to AoPS)}
\end{problem}



\begin{solution}[by \href{https://artofproblemsolving.com/community/user/29428}{pco}]
	\begin{tcolorbox}Does existes a function $f:N->N$ and for all positeve integer n
$f(f(n)+2011)=f(n)+f(f(n))$\end{tcolorbox}
Yes. Choose for example :

Let $A=\{2011k$  $\forall$ positive integer $k\ge 3\}$

$\forall x\in A$ : $f(x)=\frac {x(x-2011)}{4022}$
$\forall x\in\mathbb N\setminus A$ : $f(x)=6033$
\end{solution}



\begin{solution}[by \href{https://artofproblemsolving.com/community/user/53051}{vinhhop}]
	Dear Mr pco, could you explain why could you choose the such exemple?
\end{solution}



\begin{solution}[by \href{https://artofproblemsolving.com/community/user/167328}{nam}]
	Dear \begin{bolded}Mr Patrick\end{bolded},How did you get that ? Maybe it's your experience or a miracle  ? :roll: 
Can you post a full solution or your approach about it ?
Thank alot in advance !
\end{solution}



\begin{solution}[by \href{https://artofproblemsolving.com/community/user/29428}{pco}]
	\begin{tcolorbox}Dear \begin{bolded}Mr Patrick\end{bolded},How did you get that ? Maybe it's your experience or a miracle  ? :roll: 
Can you post a full solution or your approach about it ?
Thank alot in advance !\end{tcolorbox}
No miracle.

The question is to find one solution, not all.

We have $f(x+2011)=x+f(x)$ $\forall x\in f(\mathbb N)$

With such an equation, the first trial is to look for a polynomial expression and then it's simple to think to something like $f(x)=ax^2+bx+c$ which gives $a=\frac 1{4022}$ and $b=-\frac 12$ and any $c$

And so for example $f(x)=\frac{x(x-2011)}{4022}$. In this attempt, we need $2011|x$ and $x>2011$ and so $x=2011k$ with $k\ge 2$; Note that this imply $2011\notin f(\mathbb N)$

But $x=2011\times 2$ would imply $f(x)=2011$ and so does not fit;

If we decide that $k\ge 3$, then $f(2011\times 3)=2011\times 3$, which fits.

So we can decide that $f(\mathbb N)\subseteq A=\{2011k$ $\forall$ positive integer $k\ge 3\}$ and $f(x)=\frac{x(x-2011)}{4022}$ $\forall x\in A$

for $x\notin A$, we can choose for $f(x)$ any value in $A$, so for example $f(x)=6033$

That's the method I used to get this example.

Obviously, there are a lot of other solutions. But the problem was just to find one.
\end{solution}
*******************************************************************************
-------------------------------------------------------------------------------

\begin{problem}[Posted by \href{https://artofproblemsolving.com/community/user/68025}{Pirkuliyev Rovsen}]
	Determine all function $f: \mathbb[0;+\infty)\to\mathbb{R}$ such that

1)$\lim_{x\to+\infty}f(x)=0$    2)$f(x)+3f(x^2)=5f(x^3+1)$ for all $x\in{R}$


____________________________________
Azerbaijan Land of the Fire 
	\flushright \href{https://artofproblemsolving.com/community/c6h447138}{(Link to AoPS)}
\end{problem}



\begin{solution}[by \href{https://artofproblemsolving.com/community/user/167328}{nam}]
	Is there any solution?  :roll: 

Dear \begin{bolded}pco\end{bolded}, where are you? I sure that you will have a perfect solution. 
\end{solution}



\begin{solution}[by \href{https://artofproblemsolving.com/community/user/177508}{mathuz}]
	\begin{tcolorbox}Is there any solution?  :roll: 

Dear \begin{bolded}pco\end{bolded}, where are you? I sure that you will have a perfect solution. \end{tcolorbox}
for example:  $f(x)=0$  for all $x \in [0; \infty ] $. 


\end{solution}



\begin{solution}[by \href{https://artofproblemsolving.com/community/user/29428}{pco}]
	\begin{tcolorbox}Determine all function $f: \mathbb[0;+\infty)\to\mathbb{R}$ such that

1)$\lim_{x\to+\infty}f(x)=0$    2)$f(x)+3f(x^2)=5f(x^3+1)$ for all $x\in{R}$\end{tcolorbox}
This is certainly not a real olympiad exercise.
Besides the solution $f(x)=0$ $\forall x$ discovered by Mathuz, infinitely many solutions exist and I dont think that there is a general form for all of them.

Hereunder is one family of infinitely many continuous solutions :

Let $a>0$
Let $h(x)$ any continuous decreasing function from $[0,1]\to [a,\frac{4a}5]$

Let $\{u_n\}_{n\ge 1}$ the sequence defined as : $u_1=2$ and $u_{n+1}=u_n^{\frac 32}+1$

We can build a function $f(x)$ piece per piece in the following manner :

$\forall x\in[0,1)$ : $f(x)=h(x)$. 
Up to now, the function is defined over $[0,1)$, is positive continuous and decreasing and the functional equation is verified nowhere.

$\forall x\in[1,u_1=2)$ : $f(x)=\frac{f(\sqrt[3]{x-1})+3f(\sqrt[3]{(x-1)^2})}5$
For $x\in[1,u_1=2)$, both $\sqrt[3]{x-1}$ and $\sqrt[3]{(x-1)^2}$ are in $[0,1)$ and so this definition is meaningful.
Up to now, the function is defined over $[0,u_1)$, is positive continuous and decreasing and the functional equation is verified over $[0,1)$
Continuity in $1$ is because $h(1)=\frac 45h(0)$; Other properties are trivial.

$\forall x\in[u_1,u_2)$ : $f(x)=\frac{f(\sqrt[3]{x-1})+3f(\sqrt[3]{(x-1)^2})}5$
For $x\in[u_1,u_2)$, both $\sqrt[3]{x-1}$ and $\sqrt[3]{(x-1)^2}$ are in $[0,u_1)$ and so this definition is meaningful.
Up to now, the function is defined over $[0,u_2)$, is positive continuous and decreasing and the functional equation is verified over $[0,\sqrt{u_1})$

...

$\forall x\in[u_n,u_{n+1})$ : $f(x)=\frac{f(\sqrt[3]{x-1})+3f(\sqrt[3]{(x-1)^2})}5$
For $x\in[u_n,u_{n+1})$, both $\sqrt[3]{x-1}$ and $\sqrt[3]{(x-1)^2}$ are in $[0,u_n)$ and so this definition is meaningful.
Up to now, the function is defined over $[0,u_{n+1})$, is positive continuous and decreasing and the functional equation is verified over $[0,\sqrt{u_n})$

...

And so we built over $\mathbb R$ a positive continuous decreasing function matching the functional équation over $\mathbb R$
It remains to see that, as continuous, positive and decreasing, the function has a limit $l\ge 0$ when $x\to +\infty$ and it's easy to show that $l=0$

So we got a family of infinitely many continuous solutions.

Obviously, there exists a lot of other solutions, maybe non contibuous, maybe not decreasing in $[0,1)$, ...
\end{solution}
*******************************************************************************
-------------------------------------------------------------------------------

\begin{problem}[Posted by \href{https://artofproblemsolving.com/community/user/142737}{fara}]
	Find all $f:\mathbb{R}\rightarrow\mathbb{R}$ satisfy:
 $x\ne{0},x\ne{1}$ and $f(x)+f(\frac{1}{1-x})=x+1-\frac{1}{x}$
	\flushright \href{https://artofproblemsolving.com/community/c6h516363}{(Link to AoPS)}
\end{problem}



\begin{solution}[by \href{https://artofproblemsolving.com/community/user/29428}{pco}]
	\begin{tcolorbox}Find all $f:\mathbb{R}\rightarrow\mathbb{R}$ satisfy:
 $x\ne{0},x\ne{1}$ and $f(x)+f(\frac{1}{1-x})=x+1-\frac{1}{x}$\end{tcolorbox}
Let $P(x)$ be the assertion $f(x)+f(\frac 1{1-x})=x+1-\frac 1x$, true $\forall x\notin\{0,1\}$

Let $x\notin\{0,1\}$ so that $\frac 1{1-x}\notin\{0,1\}$ and $1-\frac 1x\notin\{0,1\}$ :

(a) : $P(x)$ $\implies$ $f(x)+f(\frac 1{1-x})=x+1-\frac 1x$

(b) : $P(\frac 1{1-x})$ $\implies$ $f(\frac 1{1-x})+f(1-\frac 1x)=\frac 1{1-x}+x$

(c) : $P(1-\frac 1x)$ $\implies$ $f(1-\frac 1x)+f(x)=1-\frac 1x-\frac 1{x-1}$

(a)-(b)+(c) : $f(x)=1-\frac 1x$ which indeed fits the requirements

Hence the solution :
$f(x)=1-\frac 1x$ $\forall x\notin\{0,1\}$
$f(0)$ and $f(1)$ can be any real values we want.
\end{solution}



\begin{solution}[by \href{https://artofproblemsolving.com/community/user/68025}{Pirkuliyev Rovsen}]
	Hello Patrik.I'm glad to see you here 
\end{solution}
*******************************************************************************
-------------------------------------------------------------------------------

\begin{problem}[Posted by \href{https://artofproblemsolving.com/community/user/169515}{abl}]
	1,find all functions $f:\mathbb{R}\to\mathbb{R}$ such that \[f(x^2 f(y)+f(x+y))=xyf(x)+x+f(y),\forall x,y\in\mathbb{R}\]
2,find all continuous  functions $f:\mathbb{R}\to\mathbb{R}$ such that \[f(( f(y))^2+f(x+f(y)))=y^2+y+f(x),\forall x,y\in\mathbb{R}\]
	\flushright \href{https://artofproblemsolving.com/community/c6h525743}{(Link to AoPS)}
\end{problem}



\begin{solution}[by \href{https://artofproblemsolving.com/community/user/152203}{borntobeweild}]
	[hide="Problem 1"]This is much less scary than it looks... :)

Let $P$ be the given assertion. We get:

$P(x,0): f(x^2f(0)+f(x))=x+f(0) \implies f$ is surjective, so $\exists c$ such that $f(c)=0$

$P(0,c): f(0)=0$

$P(0,x): f(f(x)=f(x)$

$P(x,0): f(f(x))=x$

Combining the above equations, we get $f(x)=x$, which works.[\/hide]
\end{solution}



\begin{solution}[by \href{https://artofproblemsolving.com/community/user/169515}{abl}]
	Could anybody solve problem 2?
\end{solution}



\begin{solution}[by \href{https://artofproblemsolving.com/community/user/29428}{pco}]
	\begin{tcolorbox}2,find all continuous  functions $f:\mathbb{R}\to\mathbb{R}$ such that \[f(( f(y))^2+f(x+f(y)))=y^2+y+f(x),\forall x,y\in\mathbb{R}\]\end{tcolorbox}
Let $P(x,y)$ be the assertion $f((f(y))^2+f(x+f(y)))=y^2+y+f(x)$

Let $v\in f(\mathbb R)$ and $u$ such that $f(u)=v$
Let $w\ge v-\frac 14$. The equation $x^2+x+v=w$ has at least a real solution $t$ and so $P(u,t)$ $\implies$ $f((f(t))^2+f(u+f(t)))=w$ and so $w\in f(\mathbb R)$

So, we got that $f(x)$ is continuous and that $v\in f(\mathbb R)\implies \left(v-\frac 14,+\infty\right)\subseteq f(\mathbb R)$ and so $f(\mathbb R)=\mathbb R$ and $f(x)$ is surjective.

So $0\in f(\mathbb R)$ and let $z$ such that $f(z)=0$. $P(x,z)$ $\implies$ $f(f(x))=f(x)+z^2+z$ and so, since $f(x)$ is surjective, $f(x)=x+c$ for some real $c$

Plugging back in original equation, we easily get $c=0$ and so $\boxed{f(x)=x}$ which indeed is a solution.
\end{solution}



\begin{solution}[by \href{https://artofproblemsolving.com/community/user/152203}{borntobeweild}]
	I may be misunderstanding something here, but why do we need continuity?
\end{solution}



\begin{solution}[by \href{https://artofproblemsolving.com/community/user/29428}{pco}]
	\begin{tcolorbox}I may be misunderstanding something here, but why do we need continuity?\end{tcolorbox}
You're right. We dont need continuity with this proof.
Thanks for your remark.
\end{solution}
*******************************************************************************
-------------------------------------------------------------------------------

\begin{problem}[Posted by \href{https://artofproblemsolving.com/community/user/165281}{NLT}]
	Fine all function $f:\mathbb{Z}^{+} \to \mathbb{Z}^{+}$ such that: \[ f(f(n)+m)=n+f(m+2012), \forall m,n \in \mathbb{Z}^{+}\]
	\flushright \href{https://artofproblemsolving.com/community/c6h526789}{(Link to AoPS)}
\end{problem}



\begin{solution}[by \href{https://artofproblemsolving.com/community/user/29428}{pco}]
	\begin{tcolorbox}Fine all function $f:\mathbb{Z}^{+} \to \mathbb{Z}^{+}$ such that: \[ f(f(n)+m)=n+f(m+2012), \forall m,n \in \mathbb{Z}^{+}\]\end{tcolorbox}
Let $P(m,n)$ be the assertion $f(f(n)+m)=n+f(m+2012)$

if $f(a)=f(b)$, then comparing $P(1,a)$ and $P(1,b)$, we get $a=b$ and so $f(n)$ is injective.

(a) : $P(2012,1)$ $\implies$ $f(f(1)+2012)=1+f(4024)$
(b) : $P(2012,2)$ $\implies$ $f(f(2)+2012)=2+f(4024)$
(c) : $P(f(1),n+1)$ $\implies$ $f(f(n+1)+f(1))=n+1+f(f(1)+2012)$
(d) ; $P(f(2),n)$ $\implies$ $f(f(n)+f(2))=n+f(f(2)+2012)$
(a)-(b)+(c)-(d) : $f(f(n+1)+f(1))=f(f(n)+f(2))$

And, since $f(n)$ is injective : $f(n+1)+f(1)=f(n)+f(2)$ and so $f(n+1)=f(n)+a$ for some constant $a$

So $f(n)=an+b$ for some constants $a,b$
Plugging this back in original equation, we get $a=1$ (remember $f(n)>0$) and $b=2012$

Hence the unique solution $\boxed{f(n)=n+2012}$
\end{solution}



\begin{solution}[by \href{https://artofproblemsolving.com/community/user/167328}{nam}]
	Great! 
\begin{bolded}Pco\end{bolded}, Welcome you come back again! 
you alway is a genius in my eyes ! :)
\end{solution}



\begin{solution}[by \href{https://artofproblemsolving.com/community/user/165281}{NLT}]
	I'm very happy when Pco come back.
And tks u so much about this solution !
\end{solution}



\begin{solution}[by \href{https://artofproblemsolving.com/community/user/148207}{Particle}]
	\begin{tcolorbox}Fine all function $f:\mathbb{Z}^{+} \to \mathbb{Z}^{+}$ such that: \[ f(f(n)+m)=n+f(m+2012), \forall m,n \in \mathbb{Z}^{+}\]\end{tcolorbox}
Let $P(m,n)\implies f(f(n)+m)=n+f(m+2012)$. Suppose that $z$ is any positive integer.
So 
\begin{align*}&f(f(n)+m)=n+f(m+2012)\\
&\Longleftrightarrow f(f(f(n)+m)+z)=f((n+z)+f(m+2012))\\
&\Longleftrightarrow (f(n)+m)+f(z+2012)=(m+2012)+f(n+z+2012)\\
&\Longleftrightarrow f(n)+f(z+2012)=2012+f(n+z+2012)\forall n,z\in \mathbb N,\quad \quad (1)\\
\end{align*}
Take $z\ge 2$.
Now define a function $g(n)=f(n)-2012$. Now (1) becomes
\begin{align*}&g(n)+g(z+2012)=g(z+n+2012)\\
&\Longleftrightarrow g(n)+g((z-1)+1+2012)=g((n+z-1)+1+2012)\\
&\Longleftrightarrow g(n)+g(z-1)+g(2013)=g(n+z-1)+g(2013)\\
&\Longleftrightarrow g(n)+g(z-1)=g(n+z-1)\end{align*}
Since we have taken $z\ge 2,$ both $z-1$ and $n$ range through all the positive integers. So $g(n)$, because of satisfying cauchy, becomes a linear function; i.e. $g(n)=ng(1)\implies f(n)=ng(1)+2012$. Now pluging this value in the main equation we see $g(1)=1$, hence only function satisfying this equation is $f(n)=n+2012$.
[Q.E.D.]
\end{solution}
*******************************************************************************
-------------------------------------------------------------------------------

\begin{problem}[Posted by \href{https://artofproblemsolving.com/community/user/122641}{TrungK40PBC}]
	Problem. Find all the functions $f:\mathbb{R}\rightarrow \mathbb{R}$ which satisfy, for any $x,y\in \mathbb{R}$, \[f\left ( x^{2}+f(y)-y \right )=f(x)^{2}-2013f(y)\]
	\flushright \href{https://artofproblemsolving.com/community/c6h526796}{(Link to AoPS)}
\end{problem}



\begin{solution}[by \href{https://artofproblemsolving.com/community/user/29428}{pco}]
	\begin{tcolorbox}Problem. Find all the functions $f:\mathbb{R}\rightarrow \mathbb{R}$ which satisfy, for any $x,y\in \mathbb{R}$, \[f\left ( x^{2}+f(y)-y \right )=f(x)^{2}-2013f(y)\]\end{tcolorbox}

Let $P(x,y)$ be the assertion $f(x^2+f(y)-y)=f(x)^2-2013f(y)$


1) there are no injective solutions
=======================
Let us suppose that $f(x)$ is an injective solution.
Comparing $P(x,y)$ with $P(-x,y)$, we get $f(x)^2=f(-x)^2$ and so, since injective, $f(-x)=-f(x)$ $\forall x\ne 0$

So we can find $a\ne 0$ such that $f(a)<0$  and then :

$P(\sqrt{-f(a)},a)$ $\implies$ $f(-a)=f(\sqrt{-f(a)})^2-2013f(a)$ and so $2012f(a)=f(\sqrt{-f(a)})^2$ and so $f(a)\ge 0$ and so contradiction

So no such injective solution

2) the only non injective solutions are $f(x)=0$ $\forall x$ and $f(x)=2014$ $\forall x$
=====================================================
Let us suppose that $f(x)$ is a non injective solution.
So we can find $a$ and $\Delta>0$ such that $f(a+\Delta)=f(a)$
Let $m=f(a)-a-\Delta$

2.1) $f(x+\Delta)=f(x)$ $\forall x\ge m$
----------------------------------------
Let $x\ge m$ and $u=\sqrt{x-m}$
$P(u,a+\Delta)$ $\implies$ $f(u^2+f(a+\Delta)-a-\Delta)=f(u)^2-2013f(a+\Delta)$ and so $f(x)=f(u)^2-2013f(a)$
$P(u,a)$ $\implies$ $f(u^2+f(a)-a)=f(u)^2-2013f(a)$ and so $f(x+\Delta)=f(u)^2-2013f(a)$
Q.E.D.

2.2) $f(x)=f(y)$ $\forall x>y\ge m$
----------------------------------------
If $x>y\ge m$, previous paragraph gives $f(x+\Delta)=f(x)$ and $f(y+\Delta)=f(y)$
And, more generally $f(x+p\Delta)=f(y+q\Delta)$ $\forall p,q\in\mathbb Z^+$

Let $k\in\mathbb Z^+$ great enough such that $u=\frac {x-y+k\Delta-\Delta^2}{2\Delta}\ge m$ so that $f(u+\Delta)=f(u)$

$P(u,0)$ $\implies$ $f(u^2+f(0))=f(u)^2-2013f(0)$
$P(u+\Delta,0)$ $\implies$ $f((u+\Delta)^2+f(0))=f(u+\Delta)^2-2013f(0)$ $=f(u)^2-2013f(0)$ 

So $f(u^2+f(0))=f((u+\Delta)^2+f(0))$ $=f(u^2+2\Delta u+\Delta^2+f(0)$ $=f(u^2+x-y+k\Delta+f(0))$

Writing $b=u^2+f(0)$ and $\Delta'=x-y+k\Delta$, this becomes $f(b+\Delta')=f(b)$

The same process as the one used in 2.1 above implies $f(z+\Delta')=f(z)$ $\forall z\ge M=f(b)-b-\Delta'$
Choosing $n\in\mathbb Z^+$ great enough such that $y+n\Delta \ge M$, we get $f(y+n\Delta+\Delta')=f(y+n\Delta)$

$\iff$ $f(x+(n+k)\Delta)=f(y+n\Delta)$ and since $f(x+(n+k)\Delta)=f(x)$ and $f(y+n\Delta)=f(y)$, we got $f(x)=f(y)$
Q.E.D

2.3) $f(x)$ is a constant function
----------------------------------------
From 2.2, we got that $f(x)=c$ $\forall x\ge m$ and for some real $c$
Choosing $v\ge m$ and $u$ such that $u\ge m$ and $u^2+f(v)-v\ge m$, $P(u,v)$ $\implies$ $c=c^2-2013c$ (and so either $c=0$, either $c=2014$)

Let then any $z\in\mathbb R$. Its always possible to choose $x\in\mathbb R$ such that both $x\ge m$ and $x^2+f(z)-z\ge m$
Then $P(x,z)$ $\implies$ $c=c^2-2013f(z)$ and so, since $c=c^2-2013c$, we get $f(z)=c$ $\forall z$
Q.E.D.

2.4 The end
---------------
It's immediate to find that the only constant solutions are :
$f(x)=0$ $\forall x$
$f(x)=2014$ $\forall x$
\end{solution}



\begin{solution}[by \href{https://artofproblemsolving.com/community/user/49556}{xxp2000}]
	Obviously, we have $f(-x)=-f(x)$ or $f(-x)=f(x)$ by comparing $P(x,y)$ and $P(-x,y)$.
1) $f(y)\geq0,\forall y$.
Suppose $f(y)<0$ for some $y$. Let $x^2=-f(y)$, we have $2013f(y)+f(-y)=f(x)^2\geq0$. However LHS is either $2014f(y)<0$ or $2012f(y)<0$. Absurd!

2) If $f(0)=0$, then $f(y)=0,\forall y$.
$P(0,y)$ yields $f(f(y)-y)=-2013f(y)$. To make sure both $f(y)\geq0$ and $-2013f(y)\geq0$, $f(y)=0$ for all $y$

3) If $f(0)\neq0$, then $f(y)=2014,\forall y$. 
Let $a=f(0)$ and $y$ be fixed, we iteratively apply $P(0,y): f(f(y)-y)=a^2-2013f(y)$. We see a series $\{(-2013)^n(f(y)-c)+c,n=1,2,3,...\}$ is in the range of $f$, where $c=\frac{a^2}{2014}$. To make sure there is no negative number in this series, $f(y)=c,\forall y$. It is easy to get $c=2014$.
\end{solution}
*******************************************************************************
-------------------------------------------------------------------------------

\begin{problem}[Posted by \href{https://artofproblemsolving.com/community/user/169515}{abl}]
	1,find all contnious functions $f:\mathbb{R}\to\mathbb{R}$ such that \[f\left(2x-\alpha f(x)\right)=\frac{x}{\alpha},\forall x\in\mathbb{R}\] where $\alpha\neq 0,\alpha\in\mathbb{R}$ 
2,find all continuous  functions $f:\mathbb{R}\to\mathbb{R}$ such that \[f(x^2+ f(x)+y)=f(x+y)+x^2,\forall x,y\in\mathbb{R}\]
	\flushright \href{https://artofproblemsolving.com/community/c6h527008}{(Link to AoPS)}
\end{problem}



\begin{solution}[by \href{https://artofproblemsolving.com/community/user/29428}{pco}]
	\begin{tcolorbox}1,find all contnious functions $f:\mathbb{R}\to\mathbb{R}$ such that \[f\left(2x-\alpha f(x)\right)=\frac{x}{\alpha},\forall x\in\mathbb{R}\] where $\alpha\neq 0,\alpha\in\mathbb{R}$ \end{tcolorbox}
Let $g(x)=x-\alpha f(x)$ and the equation becomes $g(x+g(x))=g(x)$

$g(x)=0$ $\forall x$ is a solution and let us from now look for non all zero solutions.
If $g(x)$ is solution, then $-g(-x)$ is solution too and so Wlog say $g(p)=q>0$ for some $p$

Let $A=\{x\ge p$ such that $g(x)=g(p)=q\}$

From $g(x+g(x))=g(x)$, we get $g(x+ng(x))=g(x)$ and so $p+nq\in A$ $\forall n\in\mathbb N\cup\{0\}$

If $A$ is not dense in $[p,+\infty)$, let then $a,b\in A$ such that $p\le a<b$ and $(a,b)\cap A=\emptyset$. (existence of $a,b$ needs continuity of $g(x)$)

Let then $y\in(a,b)$. So $g(y)\ne q$ 
Consider then $y-a+n(g(y)-q)$ for $n\in\mathbb N$
Since $g(y)\ne q$, this quantity, for $n$ great enough is out of $[-q,+q]$ and so let $m>0$ such that $y-a+m(g(y)-q)\notin[-q,+q]$ and so such that $y+mg(y)\notin[a+(m-1)q,a+(m+1)q]$

Looking at the continuous function $h(x)=x+mg(x)$, we get :
$h(a)=a+mq\in(a+(m-1)q,a+(m+1)q)$
$h(y)=y+mg(y)\notin[a+(m-1)q,a+(m+1)q]$

So (using continuity of $h(x)$), $\exists z\in(a,y)$ such that $h(z)=a+(m-1)q$ or $h(z)=a+(m+1)q$
But then $g(h(z))=q$ and so $g(z+mg(z))=g(z)=q$, impossible since $z\in(a,b)$ and $(a,b)\cap A=\emptyset$.

So $A$ is dense in $[p,+\infty)$

Then continuity of $g(x)$ implies $g(x)=q$ $\forall x\ge p$.
Let then any $w<p$ : If $g(w)>0$, then $\exists n\in\mathbb N$ such that $w+ng(w)>p$ and so $g(w)=q$. So $\forall x<p$ : either $g(x)=q$, either $g(x)\le 0$ and continuity gives the conclusion $g(x)=q$ $\forall x$

So $g(x)=$ constant function and $\boxed{f(x)=\frac{x}{\alpha}+c}$ which indeed is a solution.

(Nota : this proof is copied from my post in http://www.artofproblemsolving.com/Forum/viewtopic.php?f=36&t=485367 subcase 7.3.2)
\end{solution}



\begin{solution}[by \href{https://artofproblemsolving.com/community/user/169515}{abl}]
	Could Pco solve problem 2 ? I am not sure the problem is true.Could you change it  to become a true problem ?
\end{solution}



\begin{solution}[by \href{https://artofproblemsolving.com/community/user/29428}{pco}]
	\begin{tcolorbox}2,find all continuous  functions $f:\mathbb{R}\to\mathbb{R}$ such that \[f(x^2+ f(x)+y)=f(x+y)+x^2,\forall x,y\in\mathbb{R}\]\end{tcolorbox}
Let $P(x,y)$ be the assertion $f(x^2+f(x)+y)=f(x+y)+x^2$
Let $\mathbb A=\{x$ such that $f(x)=x\}$

$P(x,-x)$ $\implies$ $f(x^2-x+f(x))=x^2+f(0)$ and so $f(\mathbb R)$ is not upper bounded. 
As a consequence (since continuous), $f(x)$ cannot be periodic.

$P(0,x)$ $\implies$ $f(x+f(0))=f(x)$ and so, since $f(x)$ is not periodic, $f(0)=0$

Let $g(x)=x^2+f(x)$ : $g(x)$ is continuous, and since $g(0)=0$ and $f(x)$ not upperbounded, we get $[0+\infty)\subseteq g(\mathbb R)$

$P(x,0)$ $\implies$ $f(g(x))=g(x)$ and so $g(\mathbb R)\subseteq \mathbb A$ and so $[0+\infty)\subseteq \mathbb A$

Let $x\in \mathbb A$ : $P(x,-x^2)$ $\implies$ $f(x-x^2)=x-x^2$ and so $x-x^2\in\mathbb A$
Then, since any real $\ge 1$ is in $\mathbb A$, we get that any real $\le 0$ (applying previous line) is in $\mathbb A$ too.

So $\mathbb A=\mathbb R$ and so $\boxed{f(x)=x\text{   }\forall x}$ which indeed is a solution.
\end{solution}
*******************************************************************************
-------------------------------------------------------------------------------

\begin{problem}[Posted by \href{https://artofproblemsolving.com/community/user/157239}{school5}]
	Find all functions $f:N \to N_0$ such that following conditions are satisfied:

$f(mn)=f(m)+f(n)$ for all  $m,n\in N$
$f(10m+3)=0$ for all $m\in N_0$
$f(10)=0$
	\flushright \href{https://artofproblemsolving.com/community/c6h527022}{(Link to AoPS)}
\end{problem}



\begin{solution}[by \href{https://artofproblemsolving.com/community/user/29428}{pco}]
	\begin{tcolorbox}Find all functions $f:N \to N_0$ such that following conditions are satisfied:

$f(mn)=f(m)+f(n)$ for all  $m,n\in N$
$f(10m+3)=0$ for all $m\in N_0$
$f(10)=0$\end{tcolorbox}
$0=f(10)=f(2)+f(5)$ and so $f(2)=f(5)=0$

$f(2n)=f(2)+f(n)=f(n)$
$f(5n)=f(5)=f(n)=f(n)$

Let $n\in \mathbb N$. We can write $n=2^u5^va$ where neither $2|a$, neither $5|a$ so that we can find $b\in\mathbb N$ such that $ab\equiv 3\pmod{10}$
We know that $f(n)=f(a)$
Then $f(ab)=0=f(a)+f(b)$ and so $\boxed{f(n)=0\text{    }\forall n\in\mathbb N}$ which indeed is a solution
\end{solution}
*******************************************************************************
-------------------------------------------------------------------------------

\begin{problem}[Posted by \href{https://artofproblemsolving.com/community/user/169515}{abl}]
	Problem
 find all continuous  functions  $f:[0;1]\to\mathbb{R}$ such that $f(x_1)+f(x_2)+f(x_3)=1,
\forall x_1,x_2,x_{3}\in [0;1],x_1+x_2+x_3=1$
	\flushright \href{https://artofproblemsolving.com/community/c6h527098}{(Link to AoPS)}
\end{problem}



\begin{solution}[by \href{https://artofproblemsolving.com/community/user/29428}{pco}]
	\begin{tcolorbox}Problem
 find all continuous  functions  $f:[0;1]\to\mathbb{R}$ such that $f(x_1)+f(x_2)+f(x_3)=1,
\forall x_1,x_2,x_{3}\in [0;1],x_1+x_2+x_3=1$\end{tcolorbox}
Let $P(x,y,z)$ be the assertion $f(x)+f(y)+f(z)=1$ true whenever $x,y,z\in[0,1]$ and $x+y+z=1$

Let $x,y$ such that $x,y,x+y\in[0,1]$

$P(x+y,1-x-y,0)$ $\implies$ $f(x+y)+f(1-x-y)+f(0)=1$
$P(x,y,1-x-y)$ $\implies$ $f(x)+f(y)+f(1-x-y)=1$
Subtracting these two lines, we get $f(x+y)=f(x)+f(y)-f(0)$

Setting $g(x)=f(x)-f(0)$, we get $g(x+y)=g(x)+g(y)$ $\forall x,y,x+y\in[0,1]$

from there, it's immediate (Cauchy equation solution approach) to get $g(\frac 1q)=\frac 1qg(1)$ $\forall q\in\mathbb Z^+$ and then  $g(\frac pq)=\frac pqg(1)$ $\forall p\le q\in\mathbb Z^+$
And continuity gives $g(x)=xg(1)$ $\forall x\in[0,1]$
And so $f(x)=ax+b$ $\forall x\in[0,1]$

Plugging this back in original equation, we get $a+3b=1$ and so the general solution $\boxed{f(x)=ax+\frac{1-a}3}$ where $a$ is any real
\end{solution}
*******************************************************************************
-------------------------------------------------------------------------------

\begin{problem}[Posted by \href{https://artofproblemsolving.com/community/user/68025}{Pirkuliyev Rovsen}]
	Determine all function ${{f: \mathbb[0;+\infty)}\to\mathbb[0;+\infty)}$  such that $f(x)+\sqrt[n]{f^n([ x ])+f^n(\{ x \})}=x$
for all $x{\ge}0, n{\in}N$, when $[ \; ]$ and  $\{ \; \}$ denote the integer, respective the fractional part.

_________________________________________
Azerbaijan Land of the Fire 
	\flushright \href{https://artofproblemsolving.com/community/c6h527145}{(Link to AoPS)}
\end{problem}



\begin{solution}[by \href{https://artofproblemsolving.com/community/user/68025}{Pirkuliyev Rovsen}]
	Patrick hard with these problems?
\end{solution}



\begin{solution}[by \href{https://artofproblemsolving.com/community/user/29428}{pco}]
	In fact, I'm not sure these are real problems. So i'm not really interested;

In this problem, for example, problem statement is not quite clear : is $n$ a parameter and we must find some solutions $f_n$ , or must the solutions match the functional equation for any $n$ ? (if so we miss a $\forall$ just before the $n$, according to me).

So ... not_so_clear not_so_real problems;
So not_so_great_effort.
So...rry.
\end{solution}



\begin{solution}[by \href{https://artofproblemsolving.com/community/user/68025}{Pirkuliyev Rovsen}]
	Again from the Journal Octogon Mathematical Magazine.author Mihály Bencze .I also wonder how he invents such a problem.
\end{solution}



\begin{solution}[by \href{https://artofproblemsolving.com/community/user/29428}{pco}]
	Maybe you could stop trying to solve problems from this journal
\end{solution}



\begin{solution}[by \href{https://artofproblemsolving.com/community/user/164660}{abitofmath}]
	\begin{tcolorbox}Determine all function ${{f: \mathbb[0;+\infty)}\to\mathbb[0;+\infty)}$  such that $f(x)+\sqrt[n]{f^n([ x ])+f^n(\{ x \})}=x$
for all $x{\ge}0, n{\in}N$, when $[ \; ]$ and  $\{ \; \}$ denote the integer, respective the fractional part.

_________________________________________
Azerbaijan Land of the Fire \end{tcolorbox}
i have found the only solution is

$f(x)=x-\frac 1 2 \sqrt[n]{([ x ])^n+(\{ x \})^n}$
\end{solution}



\begin{solution}[by \href{https://artofproblemsolving.com/community/user/68025}{Pirkuliyev Rovsen}]
	How to find abitofmath ?
\end{solution}



\begin{solution}[by \href{https://artofproblemsolving.com/community/user/164660}{abitofmath}]
	\begin{tcolorbox}How to find abitofmath ?\end{tcolorbox}
easily $f(0)=0$ for $n\in N,f(n)=\frac n 2$,for $x \in (0,1),f(x)=\frac x 2$ so the claim follows
\end{solution}
*******************************************************************************
-------------------------------------------------------------------------------

\begin{problem}[Posted by \href{https://artofproblemsolving.com/community/user/169515}{abl}]
	Problem :find all  continuous  functions $f:\mathbb{R}\to\mathbb{R}$ such that \[ f((f(x))^2+2y +f(y))=x^2+2y+f(y),\forall x,y\in \mathbb{R}\]
	\flushright \href{https://artofproblemsolving.com/community/c6h527375}{(Link to AoPS)}
\end{problem}



\begin{solution}[by \href{https://artofproblemsolving.com/community/user/29428}{pco}]
	\begin{tcolorbox}Problem :find all  continuous  functions $f:\mathbb{R}\to\mathbb{R}$ such that \[ f((f(x))^2+2y +f(y))=x^2+2y+f(y),\forall x,y\in \mathbb{R}\]\end{tcolorbox}
Let $P(x,y)$ be the assertion $f(f(x)^2+2y+f(y))=x^2+2y+f(y)$

If $f(a)=f(b)$, comparing $P(a,0)$ and $P(b,0)$, we get $|a|=|b|$

$P(0,0)$ $\implies$ $f(f(0)^2+f(0))=f(0)$ and so $f(0)^2+f(0)=0$ and so $f(0)\in\{-1,0\}$

1) If $f(0)=0$ :
Let $g(x)=2x+f(x)$
$P(x,0)$ $\implies$ $f(f(x)^2)=x^2$ and so $g(f(x)^2)=2f(x)^2+f(f(x)^2)=x^2+2f(x)^2$
So $g(0)=0$ and $\lim_{x\to+\infty}g(f(x)^2)=+\infty$ and continuity implies $[0,+\infty)\in g(\mathbb R)$

$P(0,x)$ $\implies$ $f(g(x))=g(x)$ and so, since $[0,+\infty)\in g(\mathbb R)$ : $f(x)=x$ $\forall x\ge 0$
Comparing $P(x,0)$ and $P(-x,0)$ we get $f(-x)=\pm f(x)$ and so (continuity), only two possible solutions :
$f(x)=x$ $\forall x$ which indeed is a solution
$f(x)=|x|$ $\forall x$ which is not a solution (for example $P(0,-1)$ is false).

2) If $f(0)=-1$
Let $g(x)=2x+f(x)+1$
$P(x,0)$ $\implies$ $f(f(x)^2-1)=x^2-1$ and so $g(f(x)^2-1)=2f(x)^2-2+f(f(x)^2-1)+1=2f(x)^2+x^2-2$
So $g(0)=0$ and $\lim_{x\to+\infty}g(f(x)^2-1)=+\infty$ and continuity implies $[0,+\infty)\in g(\mathbb R)$
$P(0,x)$ $\implies$ $f(g(x))=g(x)-1$ and so, since $[0,+\infty)\in g(\mathbb R)$ : $f(x)=x-1$ $\forall x\ge 0$
But then $P(1,1)$ is false.


Hence the unique solution : $\boxed{f(x)=x}$ $\forall x$
\end{solution}
*******************************************************************************
-------------------------------------------------------------------------------

\begin{problem}[Posted by \href{https://artofproblemsolving.com/community/user/172442}{Drmat28}]
	Find all function $f: \mathbb{N} \rightarrow \mathbb{N}$ such that $f( f( f(n)))+f(f(n))+f(n)=3n$.
	\flushright \href{https://artofproblemsolving.com/community/c6h527513}{(Link to AoPS)}
\end{problem}



\begin{solution}[by \href{https://artofproblemsolving.com/community/user/29428}{pco}]
	\begin{tcolorbox}Find all function $f: \mathbb{N} \rightarrow \mathbb{N}$ such that $f( f( f(n)))+f(f(n))+f(n)=3n$.\end{tcolorbox}
Suppose that $\exists m$ such that $f(m)\ne m$ and let then $n_0=\min\{n$ such that $f(n)\ne n\}$
$f(n)$ is injective;

We get :
$f(n)=n$ $\forall n<n_0$ (definition of $n_0$)
$f(n_0)> n_0$ (definition of $n_0$ and $f(n)$ injective)
$f(n)\ge n_0$ $\forall n> n_0$ ($f(n)$ injective)

So $f(f(n_0))\ge n_0$ (since $f(n_0)>n_0$)
So $f(f(f(n_0)))\ge n_0$
So $f(f(f(n_0)))+f(f(n_0))+f(n_0)>3n_0$, impossible
So no such $n_0$

So $\boxed{f(n)=n}$ $\forall n$
\end{solution}
*******************************************************************************
-------------------------------------------------------------------------------

\begin{problem}[Posted by \href{https://artofproblemsolving.com/community/user/169515}{abl}]
	1,Let $\alpha >0,\alpha \in\mathbb{R}$,find all functions $f:\mathbb{R}\to\mathbb{R}$ such that \[ 2 f(x+f(y))=f(x+y)+f(x-y)+\alpha f(y),\forall x,y\in \mathbb{R}\] 
2,find all functions $f:\mathbb{R}\to\mathbb{R^{*}}$ such that \[\frac{(f(x)+f(y))f(x+y)}{f(xy)}+3xy(x+y)f(x+y)=1,\forall x,y\in\mathbb{R}\]
	\flushright \href{https://artofproblemsolving.com/community/c6h527619}{(Link to AoPS)}
\end{problem}



\begin{solution}[by \href{https://artofproblemsolving.com/community/user/29428}{pco}]
	\begin{tcolorbox}2,find all functions $f:\mathbb{R}\to\mathbb{R^{*}}$ such that \[\frac{(f(x)+f(y))f(x+y)}{f(xy)}+3xy(x+y)f(x+y)=1,\forall x,y\in\mathbb{R}\]\end{tcolorbox}
Let $P(x,y)$ be the assertion $\frac{f(x)+f(y)}{f(xy)}f(x+y)+3xy(x+y)f(x+y)=1$

$P(0,0)$ $\implies$ $f(0)=\frac 12$

$P(x,0)$ $\implies$ $(2f(x)+1)f(x)=1$ and so $\forall x$ either $f(x)=\frac 12$, either $f(x)=-1$

But then, rewriting $P(x,y)$ as $3xy(x+y)=\frac 1{f(x+y)}-\frac{f(x)+f(y)}{f(xy)}$, we see that RHS may at most take $16$ different values (since each of $f(x), f(y), f(x+y), f(xy)$ may take at most two values) while LHS may take infinitely many values.

so no solution for this functional equation.
\end{solution}



\begin{solution}[by \href{https://artofproblemsolving.com/community/user/169515}{abl}]
	Could Pco solve  problem 1?
\end{solution}



\begin{solution}[by \href{https://artofproblemsolving.com/community/user/29428}{pco}]
	\begin{tcolorbox}1,Let $\alpha >0,\alpha \in\mathbb{R}$,find all functions $f:\mathbb{R}\to\mathbb{R}$ such that \[ 2 f(x+f(y))=f(x+y)+f(x-y)+\alpha f(y),\forall x,y\in \mathbb{R}\] \end{tcolorbox}
Is it a real exercise ? From where did you get it ?

The trivial solution is $f(x)=\frac{\alpha}2x+b$ but there are infinitely many strange solutions.

For example, choose any non surjective solution $g(x)$ of additive Cauchy equation such that $g(x)=\frac{\alpha}2x$ $\forall x\in g(\mathbb R)$ (not $\mathbb R$) and you can get a solution with $f(x)=g(x)+g(c)$, whatever is $c$.
\end{solution}



\begin{solution}[by \href{https://artofproblemsolving.com/community/user/169515}{abl}]
	I get it from a Mathematical Magazine,but I think the problem should change :
Let $\alpha >0,\alpha \in\mathbb{R}$,find all  continuous functions $f:\mathbb{R}\to\mathbb{R}$ such that \[ 2 f(x+f(y))=f(x+y)+f(x-y)+\alpha f(y),\forall x,y\in \mathbb{R}\] 
Could Pco solve this problem ?If you couldn't solve,I think the problem is not true
\end{solution}



\begin{solution}[by \href{https://artofproblemsolving.com/community/user/29428}{pco}]
	\begin{tcolorbox}Let $\alpha >0,\alpha \in\mathbb{R}$,find all  continuous functions $f:\mathbb{R}\to\mathbb{R}$ such that \[ 2 f(x+f(y))=f(x+y)+f(x-y)+\alpha f(y),\forall x,y\in \mathbb{R}\] \end{tcolorbox}
$\boxed{f(x)=0}$ $\forall x$ is a solution.
Let us from now look only for non allzero solutions

Let $P(x,y)$ be the assertion $2f(x+f(y))=f(x+y)+f(x-y)+\alpha f(y)$
Let $a=f(0)$

(1) : $P(x-a,y)$ $\implies$ $2f(x+f(y)-a)=f(x+y-a)+f(x-y-a)+\alpha f(y)$
(2) : $P(x+y-a,0)$ $\implies$ $f(x+y)=f(x+y-a)+\frac{\alpha a}2$
(3) : $P(x-y-a,0)$ $\implies$ $f(x-y)=f(x-y-a)+\frac{\alpha a}2$
(1)-(2)-(3) : $2f(x+f(y)-a)=f(x+y)+f(x-y)-\alpha a+\alpha f(y)$
which may be written $2(f(x+(f(y)-a))-a)=(f(x+y)-a)+(f(x-y)-a)+\alpha(f(y)-a)$
and so $f(x)$ solution implies $f(x)-f(0)$ solution too and we can look only for solutions such that $f(0)=0$

Let us from now look only for solutions where $f(0)=0$

Let then $g(x)=f(x)-\frac{\alpha x}2$. 
$g(0)=0$
$P(x,y)$ may be written as new assertion $Q(x,y)$ : $2g(x+g(y)+\frac{\alpha y}2)=g(x+y)+g(x-y)$

$Q(x-g(-y)+\frac{\alpha y}2,y)$ $\implies$ $2g(x-g(-y)+g(y)+\alpha y)$ $=g(x-g(-y)+\frac{\alpha y}2+y)$ $+g(x-g(-y)+\frac{\alpha y}2-y)$
$Q(x-g(-y)+\frac{\alpha y}2,-y)$ $\implies$ $2g(x)=g(x-g(-y)+\frac{\alpha y}2-y)$ $+g(x-g(-y)+\frac{\alpha y}2+y)$
Subtracting, we get $g(x-g(-y)+g(y)+\alpha y)=g(x)$ $\forall x,y$

Let $h(y)=-g(-y)+g(y)+\alpha y$, continuous, and we got $g(x+h(y))=g(x)$ $\forall x,y$

We get two cases :

\begin{bolded}Case 1\end{bolded}: the image $h(\mathbb R)$ is not a single point and so $\exists u<v$ such that $g(x+t)=g(x)$ $\forall x\in\mathbb R$, $\forall t\in(u,v)$
It's easy to conclude that $g(x)$ is constant and so $f(x)=\frac{\alpha x}2$ which indeed is a solution
Checking then for suitable other values for $f(0)$, we get that any value is possible, hence the general solution $\boxed{f(x)=\frac{\alpha x}2+c}$ $\forall x$, and whatever is the real $c$

\begin{bolded}Case 2\end{bolded}: the image $h(\mathbb R)$ is a single point and so $h(x)=h(0)=0$ and so :
$-g(-x)+g(x)+\alpha x=0$
$Q(0,x)$ $\implies$ $2g(g(x)+\frac{\alpha x}2)=g(x)+g(-x)$
Subtracting, we get $g(g(x)+\frac{\alpha x}2)=g(x)+\frac{\alpha x}2$
Which is also $f(f(x))=\frac{\alpha+2}2f(x)$

Since $\alpha >0$ and $f(x)$ is not the allzero function, this implies that either $(-\infty,0]\subseteq f(\mathbb R)$, either $[0,+\infty)\subseteq f(\mathbb R)$ (either both)

So $f(x)=\frac{\alpha+2}2x$ $\forall x\in (-\infty,0]$ or $\forall x\in [0,+\infty)$
And it's easy to check that no such solution fits the original equation.

Hence the only two solutions $f(x)=0$ and $f(x)=\frac{\alpha x}2+c$
\end{solution}
*******************************************************************************
-------------------------------------------------------------------------------

\begin{problem}[Posted by \href{https://artofproblemsolving.com/community/user/68025}{Pirkuliyev Rovsen}]
	Find all function ${{f: \mathbb{N}_0}\to\mathbb{N}_0}$ such that $f(x^2+y^2)=xf(x)+yf(y)$
where $N_0$ set of nonnegative integers numbers.

_____________________________________
Azerbaijan Land of the Fire 
	\flushright \href{https://artofproblemsolving.com/community/c6h527781}{(Link to AoPS)}
\end{problem}



\begin{solution}[by \href{https://artofproblemsolving.com/community/user/29428}{pco}]
	\begin{tcolorbox}Find all function ${{f: \mathbb{N}_0}\to\mathbb{N}_0}$ such that $f(x^2+y^2)=xf(x)+yf(y)$
where $N_0$ set of nonnegative integers numbers.\end{tcolorbox}
Let $P(x,y)$ be the assertion $f(x^2+y^2)=xf(x)+yf(y)$
Let $a=f(1)$

Let $p>q>0$ two positive integers. Let $x\ge \frac pq$
$P(px+q,qx-p)$ $\implies$ $f(p^2x^2+q^2x^2+p^2+q^2)=(px+q)f(px+q)+(qx-p)f(qx-p)$
$P(px-q,qx+p)$ $\implies$ $f(p^2x^2+q^2x^2+p^2+q^2)=(px-q)f(px-q)+(qx+p)f(qx+p)$
Subtracting, we get new assertions :
$Q_p^q(x)$ : $f(px+q)=\frac{px-q}{px+q}f(px-q)$ $+\frac{qx+p}{px+q}f(qx+p)$ $-\frac{qx-p}{px+q}f(qx-p)$ $\forall x\ge\frac pq>1$

$P(0,0)$ $\implies$ $f(0)=0$
$P(1,1)$ $\implies$ $f(2)=2a$
$P(2,2)$ $\implies$ $f(4)=4a$
$P(1,2)$ $\implies$ $f(5)=5a$
$Q_2^1(2)$ $\implies$ $f(3)=3a$
$Q_2^1(3)$ $\implies$ $f(7)=7a$
$P(2,2)$ $\implies$ $f(8)=8a$
$P(3,0)$ $\implies$ $f(9)=9a$
$Q_3^2(3)$ $\implies$ $f(11)=11a$
$Q_4^3(2)$ $\implies$ $f(10)=10a$
$Q_4^2(2)$ $\implies$ $f(6)=6a$
$Q_5^3(2)$ $\implies$ $f(13)=13a$
$Q_5^4(2)$ $\implies$ $f(14)=14a$
$Q_6^4(2)$ $\implies$ $f(16)=16a$
$Q_6^5(2)$ $\implies$ $f(17)=17a$
$Q_4^3(3)$ $\implies$ $f(15)=15a$
$Q_6^3(2)$ $\implies$ $f(12)=12a$
$Q_7^4(2)$ $\implies$ $f(18)=18a$
$Q_7^5(2)$ $\implies$ $f(19)=19a$
$Q_7^6(2)$ $\implies$ $f(20)=20a$
$Q_8^5(2)$ $\implies$ $f(21)=21a$
$Q_8^6(2)$ $\implies$ $f(22)=22a$
$Q_8^7(2)$ $\implies$ $f(23)=23a$
$Q_9^6(2)$ $\implies$ $f(24)=24a$

And so $f(x)=ax$ $\forall x\in[0,24]$

Then :
$P(3x,4x)$ $\implies$ $f(25x^2)=3xf(3x)+4xf(4x)$ and so $f(5x)=\frac 35f(3x)+\frac 45f(4x)$

$Q_5^1(x)$ $\implies$ $f(5x+1)=\frac{5x-1}{5x+1}f(5x-1)$ $+\frac{x+5}{5x+1}f(x+5)$ $-\frac{x-5}{5x+1}f(x-5)$ $\forall x\ge 5$

$Q_5^2(x)$ $\implies$ $f(5x+2)=\frac{5x-2}{5x+2}f(5x-2)$ $+\frac{2x+5}{5x+2}f(2x+5)$ $-\frac{2x-5}{5x+2}f(2x-5)$ $\forall x\ge 5$

$Q_5^3(x)$ $\implies$ $f(5x+3)=\frac{5x-3}{5x+3}f(5x-3)$ $+\frac{3x+5}{5x+3}f(3x+5)$ $-\frac{3x-5}{5x+3}f(3x-5)$ $\forall x\ge 5$

$Q_5^4(x)$ $\implies$ $f(5x+4)=\frac{5x-4}{5x+4}f(5x-4)$ $+\frac{4x+5}{5x+4}f(4x+5)$ $-\frac{4x-5}{5x+4}f(4x-5)$ $\forall x\ge 5$

Using these five formulas, we get thru induction $\boxed{f(x)=ax}$ $\forall x\in\mathbb N_0$ which indeed is a solution, whatever is the nonnegative integer $a$
\end{solution}
*******************************************************************************
-------------------------------------------------------------------------------

\begin{problem}[Posted by \href{https://artofproblemsolving.com/community/user/169515}{abl}]
	problem : find all differentiable functions $f:\mathbb{R}\to\mathbb{R}$ such that 
$a\frac{f(x)-f(y)}{x-y}+b\frac{f(y)-f(z)}{y-z}+c\frac{f(z)-f(x)}{z-x}\geq (a+b+c)max\{f'(x);f'(y);f'(z)\}$
$\forall x,y,z\in\mathbb{R},\forall a,b,c>0$
	\flushright \href{https://artofproblemsolving.com/community/c6h527788}{(Link to AoPS)}
\end{problem}



\begin{solution}[by \href{https://artofproblemsolving.com/community/user/29428}{pco}]
	\begin{tcolorbox}problem : find all differentiable functions $f:\mathbb{R}\to\mathbb{R}$ such that 
$a\frac{f(x)-f(y)}{x-y}+b\frac{f(y)-f(z)}{y-z}+c\frac{f(z)-f(x)}{z-x}\geq (a+b+c)max\{f'(x);f'(y);f'(z)\}$
$\forall x,y,z\in\mathbb{R},\forall a,b,c>0$\end{tcolorbox}
Maybe not a full solution:

Setting $z\to x$ and writing that $RHS\ge (a+b+c)f'(x)$, we get $\frac{f(x)-f(y)}{x-y}\ge f'(x)$

so $f(x)$ crosses its tangent at each point and so $\boxed{f(x)=ux+v}$, which indeed is a solution (not quite sure of the conclusion : maybe we need $f(x)$ twice differentiable for such a conclusion).
\end{solution}
*******************************************************************************
-------------------------------------------------------------------------------

\begin{problem}[Posted by \href{https://artofproblemsolving.com/community/user/169515}{abl}]
	find all functions $f:\mathbb{R}\to\mathbb{R}$ such that 
$2f(x+y+z)+f(x)f(y)+f(y)f(z)+f(z)f(x)=1+x^2y^2+y^2z^2+z^2x^2+
                                                                       +4(xy+yz+zx)$
$\forall x,y,z\in\mathbb{R}$
	\flushright \href{https://artofproblemsolving.com/community/c6h527792}{(Link to AoPS)}
\end{problem}



\begin{solution}[by \href{https://artofproblemsolving.com/community/user/29428}{pco}]
	\begin{tcolorbox}find all functions $f:\mathbb{R}\to\mathbb{R}$ such that 
$2f(x+y+z)+f(x)f(y)+f(y)f(z)+f(z)f(x)=1+x^2y^2+y^2z^2+z^2x^2+
                                                                       +4(xy+yz+zx)$
$\forall x,y,z\in\mathbb{R}$\end{tcolorbox}
Let $P(x,y,z)$ be the assertion $2f(x+y+z)+f(x)f(y)+f(y)f(z)+f(z)f(x)$ $=1+x^2y^2+y^2z^2+z^2x^2+4(xy+yz+zx)$

$P(x,1,-1)$ $\implies$ $f(x)(2+f(1)+f(-1))=2x^2-2-f(1)f(-1)$

If $2+f(1)+f(-1)=0$, then LHS is constant while RHS is not, which is impossible. So $2+f(1)+f(-1)\ne 0$ and we get $f(x)=ax^2+b$ for some $a,b$

Plugging this back in original equation, we get the unique solution $\boxed{f(x)=x^2-1}$
\end{solution}
*******************************************************************************
-------------------------------------------------------------------------------

\begin{problem}[Posted by \href{https://artofproblemsolving.com/community/user/68025}{Pirkuliyev Rovsen}]
	Determine all $a,b>0$ for which the function $f(x)=\log_{x+a}(x+b)$ is increasing and convex, for all $x>1$.

________________________________________
Azerbaijan Land of the Fire 
	\flushright \href{https://artofproblemsolving.com/community/c6h527821}{(Link to AoPS)}
\end{problem}



\begin{solution}[by \href{https://artofproblemsolving.com/community/user/29428}{pco}]
	\begin{tcolorbox}Determine all $a,b>0$ for which the function $f(x)=\log_{x+a}(x+b)$ is increasing and convex, for all $x>1$.\end{tcolorbox}
No such values for $a,b$ since $\lim_{x\to+\infty}f(x)=1$ and so $f(x)$ can not be both increasing and convex when $x\to+\infty$

(Note : if you replace increasing by non decreasing, the only solution is $a=b$)
\end{solution}
*******************************************************************************
-------------------------------------------------------------------------------

\begin{problem}[Posted by \href{https://artofproblemsolving.com/community/user/169515}{abl}]
	Problem find all continuous functions $f:\mathbb{R}\to\mathbb{R}$ such that 
\[f(x^2+f(y+y^2))=(f(x))^2+f(y^2)+y,\forall x,y\in\mathbb{R}\]
	\flushright \href{https://artofproblemsolving.com/community/c6h528113}{(Link to AoPS)}
\end{problem}



\begin{solution}[by \href{https://artofproblemsolving.com/community/user/29428}{pco}]
	\begin{tcolorbox}Problem find all continuous functions $f:\mathbb{R}\to\mathbb{R}$ such that 
\[f(x^2+f(y+y^2))=(f(x))^2+f(y^2)+y,\forall x,y\in\mathbb{R}\]\end{tcolorbox}
Let $P(x,y)$ be the assertion $f(x^2+f(y+y^2))=f(x)^2+f(y^2)+y$
Let $a=f(0)$

(1) : $P(x,y)$ $\implies$ $f(x^2+f(y+y^2))=f(x)^2+f(y^2)+y$
(2) : $P(0,y)$ $\implies$ $f(f(y+y^2))=a^2+f(y^2)+y$
(3) : $P(x,0)$ $\implies$ $f(x^2+a)=f(x)^2+a$
(1)-(2)-(3) $\implies$ (4) : $f(x^2+f(y+y^2))=f(x^2+a)+f(f(y+y^2))-a^2-a$

Comparing $P(0,x)$ with $P(0,-x-1)$, we get $f((x+1)^2)=f(x^2)+2x+1$ and simple induction gives $f(n^2)=n^2+a$ $\forall$ integer $n\ge 0$
Continuity implies then $\forall v\ge a$ $\exists u\ge 0$ such that $f(u)=v$
And so $\forall v\ge a$ $\exists w\ge 0$ such that $f(w^2+w)=v$

so assertion (4) above implies new assertion $Q(x,y)$ : $f(x+y)=f(x+a)+f(y)-a^2-a$ true $\forall x\ge 0$, $\forall y\ge a$

Let then $x\ge \max(0,a)$ subtracting $Q(x,|a|)$ from $Q(|a|,x)$, we get $f(x+a)+f(|a|)=f(|a|+a)+f(x)$ and so $f(x+a)=f(x)+b$ for some $b$

And so new assertion $R(x,y)$ : $f(x+y)=f(x)+f(y)+c$ true $\forall x\ge \max(0,a)$, $\forall y\ge a$, where $c$ is some real

Using similar techniques than for Cauchy plus continuity, it's rather easy to get $f(x)=\alpha x+c$ $\forall x$ great enough
And since $f(n^2)=n^2+a$ $\forall$ integer $n\ge 0$, we get $f(x)=x+a$ $\forall x$ great enough

Let then $y\in \mathbb R$. Choosing $x$ great enough, $P(x,y)$ becomes $x^2+f(y+y^2)+a=(x+a)^2+f(y^2)+y$ and so $a=0$ and :
$f(x^2+x)=f(x^2)+x$ $\forall x$
Let then $M=\inf\{x\ge 0$ such that $f(t)=t$ $\forall t\ge x\}$
If $M>0$, choosing $x>0$ such that $x^2+x>M>x^2$, the above equation leads to contradiction on definition of $M$ and so $f(x)=x$ $\forall x\ge 0$

Comparing $P(x,0)$ and $P(-x,0)$ we get $f(-x)=\pm f(x)$ and continuity implies only two possibilities :

$\boxed{f(x)=x}$ $\forall x$ which indeed is a solution

$f(x)=|x|$ $\forall x$ which is not.
\end{solution}
*******************************************************************************
-------------------------------------------------------------------------------

\begin{problem}[Posted by \href{https://artofproblemsolving.com/community/user/169515}{abl}]
	problem find all functions continuous $f:\mathbb{R}\to\mathbb{R}$ such that 
\[f(x^2 +xy+(f(y))^2)=xy+f(x^2+y^2),\forall x,y\in\mathbb{R}\]
	\flushright \href{https://artofproblemsolving.com/community/c6h528128}{(Link to AoPS)}
\end{problem}



\begin{solution}[by \href{https://artofproblemsolving.com/community/user/29428}{pco}]
	\begin{tcolorbox}problem find all functions continuous $f:\mathbb{R}\to\mathbb{R}$ such that 
\[f(x^2 +xy+(f(y))^2)=xy+f(x^2+y^2),\forall x,y\in\mathbb{R}\]\end{tcolorbox}
Let $P(x,y)$ be the assertion $f(x^2+xy+f(y)^2)=xy+f(x^2+y^2)$

Let $x\ne 0$ : $P(x-\frac{f(x)^2}x,x)$ $\implies$ $x^2-f(x)^2=0$ and so, using continuity, $f(0)=0$ and only four possibilities :

1 : $\boxed{f(x)=x}$ $\forall x$, which indeed is a solution

2 : $\boxed{f(x)=|x|}$ $\forall x$, which indeed is a solution

3 : $f(x)=-x$ $\forall x$, which is not

4 : $f(x)=-|x|$ $\forall x$, which is not
\end{solution}
*******************************************************************************
-------------------------------------------------------------------------------

\begin{problem}[Posted by \href{https://artofproblemsolving.com/community/user/68025}{Pirkuliyev Rovsen}]
	Function  $f: \mathbb{N}\to\mathbb{N}$  satisfies  $f(f(n+1)+f(n+f(n)))=n+2$ and $f(1)=1$.Find the value of $f(2^2+4^2+8^2+64^2)$.

____________________________________
Azerbaijan Land of the Fire 
	\flushright \href{https://artofproblemsolving.com/community/c6h528325}{(Link to AoPS)}
\end{problem}



\begin{solution}[by \href{https://artofproblemsolving.com/community/user/29428}{pco}]
	\begin{tcolorbox}Function  $f: \mathbb{N}\to\mathbb{N}$  satisfies  $f(f(n+1)+f(n+f(n)))=n+2$ and $f(1)=1$.Find the value of $f(2^2+4^2+8^2+64^2)$.\end{tcolorbox}
Finding such a function and the required value is rather simple : choose for example $f(x)=1+\left\lfloor\frac{-1+\sqrt 5}2x\right\rfloor$ and we get $\boxed{f(4180)=2584}$

The real problem (and I'm, as usual, not sure the problem's creator understood it) is to show that either there is a unique solution (given above), either that there are many solutions, all giving the same value, either that there are many solutons giving different values and then to give them.
\end{solution}
*******************************************************************************
-------------------------------------------------------------------------------

\begin{problem}[Posted by \href{https://artofproblemsolving.com/community/user/148207}{Particle}]
	Find all functions $f:\mathbb N\to \mathbb N$ and $g:\mathbb N\to \mathbb N$ such that for any two positive integers $m,n,$ \[1.\quad f(m)-f(n)=(m-n)(g(m)-g(n))\]
\[2.\quad f(m)-f(n)=(m-n)(g(m)+g(n))\]
	\flushright \href{https://artofproblemsolving.com/community/c6h528722}{(Link to AoPS)}
\end{problem}



\begin{solution}[by \href{https://artofproblemsolving.com/community/user/29428}{pco}]
	\begin{tcolorbox}Find all functions $f:\mathbb N\to \mathbb N$ and $g:\mathbb N\to \mathbb N$ such that for any two positive integers $m,n,$ \[f(m)-f(n)=(m-n)(g(m)-g(n))\]\end{tcolorbox}
Let $P(m,n)$ be the assertion $f(m)-f(n)=(m-n)(g(m)-g(n))$

$P(n+1,n)$ $\implies$ $f(n+1)-g(n+1)=f(n)-g(n)$ and so $g(n)=f(n)+c$ where $c$ is some integer constant. Then :

$P(n,1)$ $\implies$ $(f(n)-f(1))(n-2)=0$ and so $f(n)=f(1)$ $\forall n>2$
$P(4,2)$ $\implies$ $f(2)=f(4)$

Hence the solutions :
$f(n)=a$ $\forall n>0$ where $a\in\mathbb N$
$g(n)=b$ $\forall n>0$ where $b\in\mathbb N$
\end{solution}



\begin{solution}[by \href{https://artofproblemsolving.com/community/user/148207}{Particle}]
	Oh, edited, I did a typo.
\end{solution}



\begin{solution}[by \href{https://artofproblemsolving.com/community/user/142879}{ionbursuc}]
	\begin{tcolorbox}Find all functions $f:\mathbb N\to \mathbb N$ and $g:\mathbb N\to \mathbb N$ such that for any two positive integers $m,n,$ \[1.\quad f(m)-f(n)=(m-n)(g(m)-g(n))\]
\[2.\quad f(m)-f(n)=(m-n)(g(m)+g(n))\]\end{tcolorbox}
1.$f(m)-f(n)=(m-n)(g(m)-g(n))\ \ \ \ (1)$
$F\left( x \right)=f\left( x \right)-f\left( 1 \right),G\left( x \right)=g\left( x \right)-g\left( 1 \right)$
$(1)\Leftrightarrow F(m)-F(n)=(m-n)(G(m)-G(n))\ \ \ (2)$
$n=1\Rightarrow F(m)=(m-1)G(m),\forall m\in \mathbb{N}\ $
$\ (2)\Leftrightarrow (m-1)G(m)-(n-1)G(n)=(m-n)(G(m)-G(n))\ \ \ (3)$
$\Leftrightarrow (n-1)G(m)=(2n-m-1)G(n)$
$m=1\Rightarrow G(m)=\frac{1}{2}G\left( 1 \right)=0\Rightarrow g\left( x \right)={{c}_{1}},f\left( x \right)={{c}_{2}}$
\end{solution}



\begin{solution}[by \href{https://artofproblemsolving.com/community/user/142879}{ionbursuc}]
	2.
$f(m)-f(n)=(m-n)(g(m)+g(n))\ \ \ \ (1)$
$F\left( x \right)=f\left( x \right)-f\left( 1 \right),G\left( x \right)=g\left( x \right)-g\left( 1 \right),{{c}_{1}}=g\left( 1 \right),{{c}_{0}}=f\left( 1 \right)$
$(1)\Leftrightarrow F(m)-F(n)=(m-n)(G(m)+G(n)+2c)\ \ \ (2)$
$n=1\Rightarrow F(m)=(m-1)\left( G\left( m \right)+2c \right),\forall m\in \mathbb{N}$
$\ (2)\Leftrightarrow (m-1)G(m)-(n-1)G(n)=(m-n)(G(m)+G(n))\ \ \ (3)$
$\Leftrightarrow (n-1)G(m)=(m-1)G(n)\Rightarrow G\left( x \right)=\left( x-1 \right){{c}_{2}},\forall x\in \mathbb{N}$
$\Rightarrow g\left( x \right)={{c}_{2}}\left( x-1 \right)+{{c}_{1}},f\left( x \right)={{c}_{2}}{{\left( x-1 \right)}^{2}}+2{{c}_{1}}\left( x-1 \right)+{{c}_{0}}$
\end{solution}



\begin{solution}[by \href{https://artofproblemsolving.com/community/user/170079}{MMEEvN}]
	I don't know whether this solution is correct since it looks surprisingly  simple for the first one just replace $n$ by $m$ and $m$ by $n$ to get $f(n)-f(m)=(n-m)(g(n)-g(m)) \Rightarrow f(m)-f(n)=(m-n)(g(n)-g(m)) \Rightarrow g(n)=g(m)$ .Hence function $g$ is a constant $\Rightarrow f$ is a constant as well
\end{solution}



\begin{solution}[by \href{https://artofproblemsolving.com/community/user/148207}{Particle}]
	\begin{tcolorbox}I don't know whether this solution is correct since it looks surprisingly  simple for the first one just replace $n$ by $m$ and $m$ by $n$ to get $f(n)-f(m)=(n-m)(g(n)-g(m)) \Rightarrow f(m)-f(n)=(m-n)(g(n)-g(m)) \Rightarrow g(n)=g(m)$ .Hence function $g$ is a constant $\Rightarrow f$ is a constant as well\end{tcolorbox}
Well the first equation is too damn easy. In fact it didn't appear in the tst, the second one did. I intended to post that one. Unfortunately I did a typo and posted the first equation. After pco's solution I noticed that error and fixed it.
\end{solution}
*******************************************************************************
-------------------------------------------------------------------------------

\begin{problem}[Posted by \href{https://artofproblemsolving.com/community/user/68025}{Pirkuliyev Rovsen}]
	Determine all $f: \mathbb{R}\to\mathbb{R}$ for which $f(f(x+2y)+f(2x+y))=f(f(x+y)+2(x+y))$ for all $x,y{\in}R$.


_____________________________________________
Azerbaijan Land of the Fire 
	\flushright \href{https://artofproblemsolving.com/community/c6h528759}{(Link to AoPS)}
\end{problem}



\begin{solution}[by \href{https://artofproblemsolving.com/community/user/29428}{pco}]
	\begin{tcolorbox}Determine all $f: \mathbb{R}\to\mathbb{R}$ for which $f(f(x+2y)+f(2x+y))=f(f(x+y)+2(x+y))$ for all $x,y{\in}R$.\end{tcolorbox}
Once again certainly not a real problem.

Infinitly many solutions, and I have serious doubts about the possibility of finding a general form for them. Here are some examples :

Example 1\end{underlined} : $f(x)=c$ $\forall x$, where $c$ is any real.

Example 2\end{underlined} : Let $A,B$  be two supplementary sub vectorspaces of the $\mathbb Q$-vectorspace $\mathbb R$
Let $a(x)$ and $b(x)$ the two projections of $x$ in $A$ and $B$ such that $x=a(x)+b(x)$
Choose then $f(x)=a(x)$


Example 3\end{underlined}: $f(x)=-2+1_{Q^+}(x)\sqrt 2$ (where $1_A(x)$ is the indicator function of the set $A$)
\end{solution}
*******************************************************************************
-------------------------------------------------------------------------------

\begin{problem}[Posted by \href{https://artofproblemsolving.com/community/user/145173}{Aiscrim}]
	Find all the functions $f:\Bbb{N}^\star\/\{1\}\rightarrow (0,\infty)$ for which $f(a)\cdot f(b)+1=f^2\left( \sqrt{ab+1} \right ) $
	\flushright \href{https://artofproblemsolving.com/community/c6h528894}{(Link to AoPS)}
\end{problem}



\begin{solution}[by \href{https://artofproblemsolving.com/community/user/29428}{pco}]
	\begin{tcolorbox}Find all the functions $f:\Bbb{N}^\star\/\{1\}\rightarrow (0,\infty)$ for which $f(a)\cdot f(b)+1=f^2\left( \sqrt{ab+1} \right ) $\end{tcolorbox}
1) Must we understand that domain of function is the set of all integers $>1$ ?. If so, RHS is not defined when $a=b=2$
2) if this is an "own problem" and you dont have the solution, and so dont know if a solution exists, better, according to me, to post in "open" category.
\end{solution}



\begin{solution}[by \href{https://artofproblemsolving.com/community/user/145173}{Aiscrim}]
	Then $f:(0,\infty)\rightarrow (0,\infty)$. How can I move the topic to open questions?
\end{solution}



\begin{solution}[by \href{https://artofproblemsolving.com/community/user/29428}{pco}]
	\begin{tcolorbox}Find all the functions $f:(0,+\infty)\to(0,+\infty)$ for which $f(a)\cdot f(b)+1=f^2\left( \sqrt{ab+1} \right ) \forall a,b>0$\end{tcolorbox}
I suppose that $f^2(x)$ means $f(x)f(x)$ and not $f(f(x))$

Let $P(x,y)$ be the assertion $f(x)f(y)+1=f(\sqrt{xy+1})^2$
Let $a=f(1)$

Comparing $P(x,y)$ and $P(xy,1)$, we get new assertion $Q(x,y)$ : $af(xy)=f(x)f(y)$

$P(x,1)$ $\implies$ $af(x)+1=f(\sqrt{x+1})^2$
$Q(\sqrt{x+1},\sqrt{x+1})$ $\implies$ $af(x+1)=f(\sqrt{x+1})^2$
and so $f(x+1)=f(x)+\frac 1a$ and $f(x+n)=f(x)+\frac na$

So $f(n)=a+\frac{n-1}a$ and $a=1$ in order to have $af(mn)=f(m)f(n)$

so we got :
$f(xy)=f(x)f(y)$
$f(x+n)=f(x)+n$

Since $f(x)>0$, the second equation implies $f(x)>\lfloor x\rfloor$ $\forall x>1$

Writing then $f(x)=e^{g(\ln x)}$ where $g(x)$ is a function from $\mathbb R\to\mathbb R$, we get :
$g(x+y)=g(x)+g(y)$ and $g(x)\ge \ln(\lfloor e^x\rfloor)$ $\forall x>0$

So we are looking for a solution of additive Cauchy equation which is locally lower bounded on a non empty open interval, and so continuous, and so $g(x)=cx$ and $f(x)=x^c$

Since $f(n)=n$, we get $c=1$ and the only possibility $\boxed{f(x)=x}$ which indeed is a solution.
\end{solution}
*******************************************************************************
-------------------------------------------------------------------------------

\begin{problem}[Posted by \href{https://artofproblemsolving.com/community/user/68025}{Pirkuliyev Rovsen}]
	Find all functions $f: \mathbb{R}\to\mathbb{R}$  such that $f(x+f(y))=f(y+f(x))$.

__________________________________
Azerbaijan Land of the Fire 
	\flushright \href{https://artofproblemsolving.com/community/c6h529209}{(Link to AoPS)}
\end{problem}



\begin{solution}[by \href{https://artofproblemsolving.com/community/user/29428}{pco}]
	\begin{tcolorbox}Find all functions $f: \mathbb{R}\to\mathbb{R}$  such that $f(x+f(y))=f(y+f(x))$\end{tcolorbox}
As usual certainly not a real problem.

Infinitly many solutions, and I have serious doubts about the possibility of finding a general form for them. Here are some examples :

Example 1\end{underlined} : $f(x)=c$ $\forall x$, where $c$ is any real.

Example 2\end{underlined} : Let $A,B$  be two supplementary sub vectorspaces of the $\mathbb Q$-vectorspace $\mathbb R$
Let $a(x)$ and $b(x)$ the two projections of $x$ in $A$ and $B$ such that $x=a(x)+b(x)$
Choose then $f(x)=a(x)$


Example 3\end{underlined}: $f(x)=\frac{\lfloor 2\{x\}\rfloor}2$

And so on ...

Note that this problem could be solved by adding the continuity constraint (but your book \/ famous magazine \/ ... did not give it :( ) giving $f(x)=c$ or $f(x)=x+c$
\end{solution}
*******************************************************************************
-------------------------------------------------------------------------------

\begin{problem}[Posted by \href{https://artofproblemsolving.com/community/user/149155}{msop}]
	\[f:\mathbb{N} \mapsto \mathbb{N}-{1} which:
\forall n
\in \mathbb {N}-{1}:f(n+1)+f(n+3)=f(n+5).f(n+7)-1375\]
	\flushright \href{https://artofproblemsolving.com/community/c6h529247}{(Link to AoPS)}
\end{problem}



\begin{solution}[by \href{https://artofproblemsolving.com/community/user/16261}{Rust}]
	\begin{tcolorbox}\[f:\mathbb{N} \mapsto \mathbb{N}-{1} which:
\forall n
\in \mathbb {N}-{1}:f(n+1)+f(n+3)=f(n+5).f(n+7)-1375\]\end{tcolorbox}
There are many solytions.
For example periodik solutions with period 4:
 Let $x=f(n+1)=f(n+5), y=f(n+3)=f(n+7)$, then $x+y=xy-1375,$ or $(x-1)(y-1)=1376$ or x=d+1, y=\frac{1376}{d}+1$, were $d$ is any divisor of 1376.
\end{solution}



\begin{solution}[by \href{https://artofproblemsolving.com/community/user/29428}{pco}]
	\begin{tcolorbox}\[f:\mathbb{N} \mapsto \mathbb{N}-{1} which:
\forall n
\in \mathbb {N}-{1}:f(n+1)+f(n+3)=f(n+5).f(n+7)-1375\]\end{tcolorbox}
Let $P(n)$ be the assertion $f(n+1)+f(n+3)=f(n+5)f(n+7)-1375$ true $\forall n\ge 2$

Subtracting $P(n)$ from $P(n+2)$, we get $f(n+5)-f(n+1)=f(n+7)(f(n+9)-f(n+5))$ $\forall n\ge 2$

And so $f(n+5)-f(n+1)=f(n+7)f(n+11)(f(n+13)-f(n+9))$ $\forall n\ge 2$
and so $\prod_{i=1}^kf(n+3+4i)|f(n+5)-f(n+1)$ $\forall n\ge 2$ and $\forall k\in\mathbb N$
since $f(n+3+4i)>1$, the only possiblity is $f(n+5)=f(n+1)$ $\forall n\ge 2$

So sequence of $f(i)$ is $a,b,c,d,e,f,c,d,e,f,c,d,e,f...$ and we must just check :

$P(2)$ : $f(3)+f(5)=f(7)f(9)-1375$ $\iff$ $c+e=ce-1375$ $\iff$ $(c-1)(e-1)=1376$
$P(3)$ : $f(4)+f(6)=f(8)f(10)-1375$ $\iff$ $d+f=df-1375$ $\iff$ $(d-1)(f-1)=1376$
All the remaining cases are the same.

So \begin{bolded}all the solutions are\end{underlined}\end{bolded} :
$f(1)=a$ where $a$ is any integer $\ge 2$
$f(2)=b$ where $b$ is any integer $\ge 2$
$f(3)=c$ where $c$ is any integer $\in\{2,3,5,9,17,33,44,87,173,345,689,1377\}$
$f(4)=d$ where $d$ is any integer $\in\{2,3,5,9,17,33,44,87,173,345,689,1377\}$
$f(5)=e=\frac{1376}{c-1}+1$
$f(6)=f=\frac{1376}{d-1}+1$
$f(n)=f(n-4)$ $\forall n\ge 7$
\end{solution}
*******************************************************************************
-------------------------------------------------------------------------------

\begin{problem}[Posted by \href{https://artofproblemsolving.com/community/user/127783}{Sayan}]
	Find all functions $f: \mathbb{R} \to \mathbb{R}$ satisfying
\[f(x+f(y))-f(x)=(x+f(y))^4-x^4\]
for all $x,y \in \mathbb{R}$.
	\flushright \href{https://artofproblemsolving.com/community/c6h529313}{(Link to AoPS)}
\end{problem}



\begin{solution}[by \href{https://artofproblemsolving.com/community/user/149155}{msop}]
	x=-f(y) :arrow:  we have f(0)=f(x)-x^4 so we have f(x)=x^4+f(0).(f(0)=c)so f(x)=x^4+c.and if you test this answer you can see that it is true so the answer is :arrow:  f(x)=(x^4)+c which c is a rational number
\end{solution}



\begin{solution}[by \href{https://artofproblemsolving.com/community/user/29428}{pco}]
	\begin{tcolorbox}Find all functions $f: \mathbb{R} \to \mathbb{R}$ satisfying
\[f(x+f(y))-f(x)=(x+f(y))^4-x^4\]
for all $x,y \in \mathbb{R}$\end{tcolorbox}
$f(x)=0$ $\forall x$ is a solution. So let us from now look only for non all zero solutions.

Let $P(x,y)$ be the assertion $f(x+f(y))-f(x)=(x+f(y))^4-x^4$
Let $a=f(0)$
Let $u$ such that $f(u)\ne 0$ and let then $v=f(u)\ne 0$

The equation $(x+v)^4-x^4=t$ is a polynomial of degree $3$ (since $v\ne 0$) and so always has at least one real solution.
Then $P(x,u)$ $\implies$ $f(x+v)-f(x)=t$ and so any real $t$ may be written as $f(r)-f(s)$ for some other real numbers $r,s$

$P(-f(y),y)$ $\implies$ $f(-f(y))=f(y)^4+a$
$P(-f(y),x)$ $\implies$ $f(f(x)-f(y))-f(-f(y))=(f(x)-f(y))^4-f(y)^4$
 
Adding, we get $f(f(x)-f(y))=(f(x)-f(y))^4+a$
An since any real may be written as $f(x)-f(y)$ for some real numbers $x,y$, we got $f(x)=x^4+a$ $\forall x\in\mathbb R$ which indeed is a solution, whatever is $a$

\begin{bolded}Hence the solutions\end{underlined}\end{bolded} :
$f(x)=0$ $\forall x$
$f(x)=x^4+a$ $\forall x$ whatever is $a\in\mathbb R$
\end{solution}



\begin{solution}[by \href{https://artofproblemsolving.com/community/user/149155}{msop}]
	thanks patrick iwas wrong we can't given that f(y)=-x when f(x) is constant so we must first given that f(x)=a which a is constant  and test it then we can given that f(y)=-x and we must prove that this is surjective.
\end{solution}



\begin{solution}[by \href{https://artofproblemsolving.com/community/user/29428}{pco}]
	\begin{tcolorbox}...and we must prove that this is surjective.\end{tcolorbox}
Which unfortunately is impossible since none of the solutions we found is surjective ...
\end{solution}
*******************************************************************************
-------------------------------------------------------------------------------

\begin{problem}[Posted by \href{https://artofproblemsolving.com/community/user/68025}{Pirkuliyev Rovsen}]
	Find all function $f: \mathbb{R}\to\mathbb{R}$ such that $f([ x ])+f(\{\ x\})=2f(x)-x^2$ for all $x{\in}R$.

______________________________________
Azerbaijan Land of the Fire 
	\flushright \href{https://artofproblemsolving.com/community/c6h529335}{(Link to AoPS)}
\end{problem}



\begin{solution}[by \href{https://artofproblemsolving.com/community/user/29428}{pco}]
	\begin{tcolorbox}Find all function $f: \mathbb{R}\to\mathbb{R}$ such that $f([ x ])+f(\{\ x\})=2f(x)-x^2$ for all $x{\in}R$.\end{tcolorbox}
Let $P(x)$ be the assertion $f(\lfloor x\rfloor)+f(\{x\})=2f(x)-x^2$
Let $a=f(0)$

$P(\lfloor x\rfloor)$ $\implies$ $f(\lfloor x\rfloor)=\lfloor x\rfloor^2+a$

$P(\{x\})$ $\implies$ $f(\{x\})=\{x\}^2+a$

Plugging this back in original equation, we get $\boxed{f(x)=\frac{\lfloor x\rfloor^2+\{x\}^2+x^2}2+a}$ $\forall x$, which indeed is a solution, whatever is $a\in\mathbb R$
\end{solution}
*******************************************************************************
-------------------------------------------------------------------------------

\begin{problem}[Posted by \href{https://artofproblemsolving.com/community/user/68025}{Pirkuliyev Rovsen}]
	The function$ f $ is monotone and satisfies $(f(x)-f(y))(yf(x)-xf(y)){\leq}0$ for all $x,y{\in}(0,+\infty)$.
Prove that the function $f$  is continuous on $(0,+\infty)$.

______________________________________________
Azerbaijan Land of the Fire 
	\flushright \href{https://artofproblemsolving.com/community/c6h529472}{(Link to AoPS)}
\end{problem}



\begin{solution}[by \href{https://artofproblemsolving.com/community/user/29428}{pco}]
	\begin{tcolorbox}The function$ f $ is monotone and satisfies $(f(x)-f(y))(yf(x)-xf(y)){\leq}0$ for all $x,y{\in}(0,+\infty)$.
Prove that the function $f$  is continuous on $(0,+\infty)$.\end{tcolorbox}
Since monotonic, $f(x)$ has left limit and right limit at each point of $(0,+\infty)$
Suppose now that at some point $a>0$, we have $\lim_{x\to a^-}f(x)=b$ and $\lim_{x\to a+}f(x)=b+c$ for some $c>0$

Setting $x\to a^+$ and $y\to a^-$ in the functional equation, we get $ac^2\le 0$, impossible;

So $\forall a>0$ : $\lim_{x\to a^-}f(x)=\lim_{x\to a+}f(x)$ and so $f(x)$ is continuous at $a$
Q.E.D.
\end{solution}



\begin{solution}[by \href{https://artofproblemsolving.com/community/user/144631}{n0de}]
	\begin{tcolorbox}[quote="Pirkuliyev Rovsen"]
Suppose now that at some point $a>0$, we have $\lim_{x\to a^-}f(x)=b$ and $\lim_{x\to a+}f(x)=b+c$ for some $c>0$
\end{tcolorbox}\end{tcolorbox}

Since we don't know whether $f$ increasing or decreasing, why should $c$ be positive?
\end{solution}



\begin{solution}[by \href{https://artofproblemsolving.com/community/user/29428}{pco}]
	\begin{tcolorbox}[quote="pco"][quote="Pirkuliyev Rovsen"]
Suppose now that at some point $a>0$, we have $\lim_{x\to a^-}f(x)=b$ and $\lim_{x\to a+}f(x)=b+c$ for some $c>0$
\end{tcolorbox}\end{tcolorbox}

Since we don't know whether $f$ increasing or decreasing, why should $c$ be positive?\end{tcolorbox}
English is not my natural language. So I looked for precise definition of "monotonic function" on Web and it seems that monotonic means "non decreasing". See for example http://en.wikipedia.org\/wiki\/Monotonic_function:
 "a function  $f$ defined on a subset of the real numbers with real values is called monotonic (also monotonically increasing, increasing or non-decreasing), if for all  $x$ and  $y$ such that $x\le y$ one has $f(x)\le f(y)$, so  $f$ preserves the order"
\end{solution}



\begin{solution}[by \href{https://artofproblemsolving.com/community/user/164292}{babylon5}]
	We can write the given inequality as $\left(f(x)-\frac{(x+y)f(y)}{2y}\right)^2\leq \frac{(x-y)^2f^2(y)}{4y^2}$ and from that we take 
$\frac{(x+y)f(y)}{2y}-\frac{|(x-y)f(y)|}{2y}\leq f(x)\leq \frac{(x+y)f(y)}{2y}+\frac{|(x-y)f(y)|}{2y}$.
Taking the limits for $x\rightarrow y$ and using the squeeze theorem  we have $f(x)\rightarrow f(y)$.
\end{solution}
*******************************************************************************
-------------------------------------------------------------------------------

\begin{problem}[Posted by \href{https://artofproblemsolving.com/community/user/169515}{abl}]
	Problem find all continuous functions $f:\mathbb{R}\to\mathbb{R}$ such that 
\[f(f(x^2)+y+f(y))=x^2+2f(y),\forall x,y\in\mathbb{R}\]
	\flushright \href{https://artofproblemsolving.com/community/c6h529496}{(Link to AoPS)}
\end{problem}



\begin{solution}[by \href{https://artofproblemsolving.com/community/user/29428}{pco}]
	\begin{tcolorbox}Problem find all continuous functions $f:\mathbb{R}\to\mathbb{R}$ such that 
\[f(f(x^2)+y+f(y))=x^2+2f(y),\forall x,y\in\mathbb{R}\]\end{tcolorbox}
Let $P(x,y)$ be the assertion $f(f(x^2)+y+f(y))=x^2+2f(y)$
Let $a=f(0)$

$P(0,x)$ $\implies$ $f(x+f(x)+a)=2f(x)$ and so $x+f(x)$ is injective
If $\lim_{x\to+\infty}f(x)=l$, then setting $y\to+\infty$ in $P(0,y)$, we get $\lim_{y\to+\infty}f(y)=\frac 12f(l)$ and so contradiction
If $\lim_{x\to-\infty}f(x)=l$, then setting $y\to-\infty$ in $P(0,y)$, we get $\lim_{y\to-\infty}f(y)=\frac 12f(l)$ and so contradiction
So $g(x)=x+f(x)$ is a continuous bijection from $\mathbb R\to\mathbb R$

$P(x,0)$ $\implies$ $f(f(x^2)+a)=x^2+2a$ $\implies$ $[2a,+\infty)\subseteq f(\mathbb R)$
$P(x,g^{-1}(-f(x^2)))$ $\implies$ $f(g^{-1}(-f(x^2)))=\frac{a-x^2}2$ $\implies$ $(-\infty,\frac a2]\subseteq f(\mathbb R)$
So, since continuous, $f(\mathbb R)=\mathbb R$

Let then $b$ such that $f(b)=-a$. $P(0,b)$ $\implies$ $f(b)=0$ and so $a=0$

Then : 
$P(x,0)$ $\implies$ $f(f(x^2))=x^2$ and so $f(f(x))=x$ $\forall x\ge 0$
Let $x\ge 0$ : 
$P(0,f(x))$ $\implies$ $f(f(x)+x)=2x$
$P(0,x)$ $\implies$ $f(x+f(x))=2f(x)$
And so $f(x)=x$ $\forall x\ge 0$

Let then $y<0$. Let $x$ such that $x^2+y+f(y)>0$
$P(x,y)$ $\implies$ $f(x^2+y+f(y))=x^2+2f(y)$ and so $x^2+y+f(y)=x^2+2f(y)$ and $f(y)=y$

hence the unique possibility : $\boxed{f(x)=x}$ $\forall x$ which indeed is a solution.
\end{solution}



\begin{solution}[by \href{https://artofproblemsolving.com/community/user/169515}{abl}]
	\begin{tcolorbox}
If $\lim_{x\to+\infty}f(x)=l$, then setting $y\to+\infty$ in $P(0,y)$, we get $\lim_{y\to+\infty}f(y)=\frac 12f(l)$ and so contradiction
If $\lim_{x\to-\infty}f(x)=l$, then setting $y\to-\infty$ in $P(0,y)$, we get $\lim_{y\to-\infty}f(y)=\frac 12f(l)$ and so contradiction

\end{tcolorbox}
Could you explain the lines ?
\end{solution}



\begin{solution}[by \href{https://artofproblemsolving.com/community/user/29428}{pco}]
	\begin{tcolorbox}[quote="pco"]
If $\lim_{x\to+\infty}f(x)=l$, then setting $y\to+\infty$ in $P(0,y)$, we get $\lim_{y\to+\infty}f(y)=\frac 12f(l)$ and so contradiction
If $\lim_{x\to-\infty}f(x)=l$, then setting $y\to-\infty$ in $P(0,y)$, we get $\lim_{y\to-\infty}f(y)=\frac 12f(l)$ and so contradiction

\end{tcolorbox}
Could you explain the lines ?\end{tcolorbox}
I cant, because there is a typo. Hereunder is the good typo-free part :

The goal is to prove that $g(\mathbb R)=\mathbb R$ where $g(x)=x+f(x)$

We know that $g(x)$ is continuous and injective. So $\lim_{x\to+\infty}g(x)$ is either $-\infty$, either $+\infty$, either a finite real $l$
(the typo is that I wrote $f(x)$ instead of $g(x)$ or $x+f(x)$)

If $\lim_{x\to+\infty}g(x)=l$, then, from $f(g(x)+a)=2f(x)$ we get :
$\lim_{x\to+\infty}f(g(x)+a)=2\lim_{x\to+\infty}f(x)$

And so : $f(l+a)=2\lim_{x\to+\infty}f(x)$

And so : $\lim_{x\to+\infty}f(x)=\frac 12f(l+a)$

But this is a contradiction because we cant have at the same time :
$\lim_{x\to+\infty}(x+f(x))=l$
$\lim_{x\to+\infty}f(x)=\frac 12f(l+a)$
(just take the difference in order to see the contradiction)

So $\lim_{x\to+\infty}g(x)$ is either $-\infty$, either $+\infty$

Same thing with $x\to-\infty$ gives that $\lim_{x\to-\infty}g(x)$ is either $-\infty$, either $+\infty$

And continuity + injectivity gives $g(\mathbb R)=\mathbb R$


Thanks for reading me :)
\end{solution}



\begin{solution}[by \href{https://artofproblemsolving.com/community/user/105169}{Nikpour}]
	Let $x=\sqrt{a},y=a$. Its follows from assumptation that $f(\underbrace{f(a)+a+f(a)}_{t})=\underbrace{a+2f(a)}_{t}\Rightarrow f(t)=t$
\end{solution}



\begin{solution}[by \href{https://artofproblemsolving.com/community/user/29428}{pco}]
	\begin{tcolorbox}Let $x=\sqrt{a},y=a$. Its follows from assumptation that $f(\underbrace{f(a)+a+f(a)}_{t})=\underbrace{a+2f(a)}_{t}\Rightarrow f(t)=t$\end{tcolorbox}
So you proved that $f(x)=x$ for any $x$ which can be written under the form $x=2f(a)+a$ for some $a\ge 0$

It just remains to prove that any $x\in\mathbb R$ may be written in such a form. :)
\end{solution}



\begin{solution}[by \href{https://artofproblemsolving.com/community/user/49556}{xxp2000}]
	Here is a solution without continuity assumption.
We let $a=f(0)$.

1) $a=0$.
Suppose $a>0$. 
$P(0,0):f(2a)=2a$, 
$P(\sqrt{2a},0):f(3a)=4a$, 
$P(0,2a):f(5a)=4a$, 
$P(\sqrt{3a},0):f(5a)=5a$. Absurd!

Suppose $a<0$. We notice $f$ can take any value no less than $2f(0)=2a<0$. We can find $b$ such that $f(b)=-a>0$. $P(0,b):-a=0$. Absurd!

2) $f(x)=x,\forall x\geq0$.
$P(x,0)$ implies $f(f(x))=x,\forall x\geq0$
If we let $x\geq0$ and compare $P(0,x)$ and $P(0,f(x))$, we get $f(x)=x,\forall x\geq0$

3) $f(x)=x,\forall x$
For any $y$, we can find $x$ such that $x^2+y+f(y)>0$. Then $P(x,y)$ implies $f(y)=y$.
\end{solution}
*******************************************************************************
-------------------------------------------------------------------------------

\begin{problem}[Posted by \href{https://artofproblemsolving.com/community/user/169515}{abl}]
	Problem find all continuous functions $f:\mathbb{R}\to\mathbb{R}$ such that
    \[f(x+y+f(2xy))=2xy+f(x+y),\forall x,y\in\mathbb{R}\]
	\flushright \href{https://artofproblemsolving.com/community/c6h529497}{(Link to AoPS)}
\end{problem}



\begin{solution}[by \href{https://artofproblemsolving.com/community/user/29428}{pco}]
	\begin{tcolorbox}Problem find all continuous functions $f:\mathbb{R}\to\mathbb{R}$ such that
    \[f(x+y+f(2xy))=2xy+f(x+y),\forall x,y\in\mathbb{R}\]\end{tcolorbox}
The system of equations $x+y=u$ and $2xy=v$ has a solution whenever $u^2-2v\ge 0$ and so the fonctional equation may be written as :
Assertion $P(x,y)$ : $f(x+f(y))=f(x)+y$ $\forall x,y$ such that $x^2\ge 2y$

If $f(u)=f(v)$, and choosing $x$ such that $x^2\ge\max(2u,2v)$, comparaison of $P(x,u)$ and $P(x,v)$ implies $u=v$ and so $f(x)$ is injective.

$P(x,0)$ $\implies$ $f(x+f(0))=f(x)$ and so, since injective, $f(0)=0$

Since injective and continuous, $f(x)$ is either increasing, either decreasing.

1) The only increasing solution is $f(x)=x$ $\forall x$
================================
Let $x\le 0$ : $P(0,x)$ $\implies$ $f(f(x))=x$ $\forall x\le 0$ and so, since increasing,  $f(x)=x$ $\forall x\le 0$

Let then $y\in\mathbb R$
We can choose negative $x$ such that $x+f(y)\le 0$ and $x^2\ge 2y$
Then $f(x)=x$ and $f(x+f(y))=x+f(y)$ and $P(x,y)$ becomes $x+f(y)=f(x)+y=x+y$ and so $f(y)=y$ $\forall y$
It remains to easily check that $f(x)=x$ $\forall x$ indeed is a solution to get the result.

2) The only decreasing solution is $f(x)=-x$ $\forall x$
=================================
Let $g(x)=x+f(x)$ continuous.
Let $x\le 0$  $P(x,x)$ $\implies$ $f(g(x))=g(x)$ and so $f(x)=x$ $\forall x\in g((-\infty,0])$
But we know that $f(x)$ is decreasing and so $g((-\infty,0])=\{c\}$ for some real $c$ ($g((-\infty,0])$ can contain no non empty open interval) and :
$f(x)=c-x$ $\forall x\in (-\infty,0]$ and so (looking at $0$) : $f(x)=-x$ $\forall x\le 0$

Let then $y\in\mathbb R$
We can choose negative $x$ such that $x+f(y)\le 0$ and $x^2\ge 2y$
Then $f(x)=-x$ and $f(x+f(y))=-(x+f(y))$ and $P(x,y)$ becomes $-x-f(y)=f(x)+y=-x+y$ and so $f(y)=-y$ $\forall y$
It remains to easily check that $f(x)=-x$ $\forall x$ indeed is a solution to get the result.

\begin{bolded}Hence the only two solutions\end{underlined}\end{bolded} :
$f(x)=x$ $\forall x$
$f(x)=-x$ $\forall x$
\end{solution}
*******************************************************************************
-------------------------------------------------------------------------------

\begin{problem}[Posted by \href{https://artofproblemsolving.com/community/user/154337}{supama97}]
	Find all function $ f:\mathbb{Q}_{+}\rightarrow\mathbb{Q}_{+} $ that satisfy:
$ f(f(x)+x+y) = f(x)+x+f(y)$
	\flushright \href{https://artofproblemsolving.com/community/c6h529498}{(Link to AoPS)}
\end{problem}



\begin{solution}[by \href{https://artofproblemsolving.com/community/user/29428}{pco}]
	\begin{tcolorbox}Find all function $ f:\mathbb{Q}_{+}\rightarrow\mathbb{Q}_{+} $ that satisfy:
$ f(f(x)+x+y) = f(x)+x+f(y)$\end{tcolorbox}
Strange problem for an olympiad exam \/ test \/ ... . In what olympiad  did you get it ?
We have infinitely many very different solutions; For example :
$f(x)=x+1$ 

$f(x)=3+\lfloor x\rfloor-\{x\}$

$f(x)=x+2+\lfloor 10^{1+\{x\}}\rfloor-2\{x\}$

$f(x)=x+2\lfloor3+50\sin^2 \{\frac x2\}\pi \rfloor-4\{\frac x2\}$

1) General solution :
==============
Let $\mathbb A$ any additive subgroup of $\mathbb Q$ different from $\{0\}$ (Notice that $\mathbb A$ is unbound)
Let $\sim$ the equivalence relation in $\mathbb Q^+$ defined as $x\sim y$ $\iff$ $x-y\in\mathbb A$
Let $r(x)$ be any function associating to each positive rational a representent (unique per class) of its equivalence class.
Let $a(x)$ any function from $r(\mathbb Q^+)\to \mathbb A$ such that $a(r(x))>2r(x)-x$ $\forall x\in\mathbb Q^+$ (such functions always exist : remember that $\mathbb A$ is unbound and so you can choose for example any $a(x)>2x$)
You can define $f(x)$ as $\boxed{f(x)=x+a(r(x))-2r(x)}$ $\forall x\in\mathbb Q^+$

2) Proof that this indeed is a solution
==========================
$f(x)$ is a function from $\mathbb Q^+\to\mathbb Q^+$ 
If $y\in\mathbb Q^+$ and $u\in A^+$, then $y+u\sim y$ and so $r(y+u)=r(y)$ and so :
$f(y+u)=y+u+a(r(y+u))-2r(y+u)$ $=y+u+a(r(y))-2r(y)$ $=f(y)+u$

$f(x)+x=a(r(x))+2(x-r(x))$. Since both $a(r(x))$ and $x-r(x)$ a$\in\mathbb A$, we get $f(x)+x\in\mathbb A$
And so (see previous line) : $f(y+x+f(x))=f(y)+x+f(x)$
Q.E.D.

3) Proof that any solution may be put in this form (and so that it is indeed a general solution) :
==================================================================
Let $f(x)$ solution of the functional equation.
Let $\mathbb A=\{y\in\mathbb Q$ such that $f(x+y)=f(x)+y$ $\forall$ positive rational number ${x>-y}$
It's easy to show that $\mathbb A$ is an additive subgroup of $\mathbb Q$ different from $\{0\}$
Let $\sim$ the equivalence relation in $\mathbb Q^+$ defined as $x\sim y$ $\iff$ $x-y\in\mathbb A$
Let $r(x)$ be any function associating to each positive rational a representent (unique per class) of its equivalence class.

let $x>y$ such that $r(x)=r(y)$ : $f(x)=f(y+x-y)=f(y)+x-y$ (since $x-y\in\mathbb A$) and so :
$f(x)+2r(x)-x=f(y)+x-y+2r(x)-x$ $=f(y)+2r(y)-y$
So we can define a function $a(r(x))=f(x)+2r(x)-x$
$a(r(x))=f(x)+x+2(r(x)-x)$. $f(x)+x\in\mathbb A$ and $r(x)-x\in\mathbb A$ and so $a(r(x))\in\mathbb A$
So $a(x)$ is a function from $r(\mathbb Q^+)\to \mathbb A$ and such that $a(r(x))>2r(x)-x$ $\forall x\in\mathbb Q^+$ 

And $f(x)=x+a(r(x))-2r(x)$ $\forall x\in\mathbb Q^+$
Q.E.D

4) Some examples
=============
4.1 $\mathbb A=\mathbb Q$
---------------
Then we have a unique equivalence class and $r(x)=c$ constant and $a(r(x))=d$ constant such that $d>2c-x$ and so $d\ge 2c$
Then $f(x)=x+d-2c$ and we got the trivial solution $\boxed{f(x)=x+a}$ $\forall x\in\mathbb Q^+$ and for any $a\in\mathbb Q^+\cup\{0\}$

4.2 $\mathbb A=\mathbb Z$
------------------------
Choose then for example $r(x)=1+\{x\}$
$a(x)$ must be a function from $[1,2)\to\mathbb Z$ such that $a(1+\{x\})>2(1+\{x\})-x$ and so we can choose for example any function $\ge 
4$

Example 4.2.1 : $a(x)=5$ and so $\boxed{f(x)=3+\lfloor x\rfloor-\{x\}}$

Example 4.2.2 : $a(x)=4+\lfloor 10^x\rfloor$ and so $\boxed{f(x)=x+2+\lfloor 10^{1+\{x\}}\rfloor-2\{x\}}$

And a lot of others

4.3 $\mathbb A=q\mathbb Z$ with $q\in\mathbb Q^+$
-----------------------------------------------
Choose then for example $r(x)=q(1+\{\frac xq\})$

Example 4.3.1 with $q=2$ : $\boxed{f(x)=x+2\lfloor3+50\sin^2 \{\frac x2\}\pi \rfloor-4\{\frac x2\}}$

4.3 Some other $\mathbb A$
-------------------------
some other $\mathbb A$ exist, giving very strange functions. For example : $\mathbb A=\{(2m-1)2^n, m\in\mathbb N,n\in\mathbb Z\}$

...
\end{solution}
*******************************************************************************
-------------------------------------------------------------------------------

\begin{problem}[Posted by \href{https://artofproblemsolving.com/community/user/50065}{SherlockBond}]
	Prove that: f(x)=cos(x^2) is not a cyclic function
	\flushright \href{https://artofproblemsolving.com/community/c6h529727}{(Link to AoPS)}
\end{problem}



\begin{solution}[by \href{https://artofproblemsolving.com/community/user/29428}{pco}]
	\begin{tcolorbox}Prove that: f(x)=cos(x^2) is not a cyclic function\end{tcolorbox}
I suppose that "cyclic" means "periodic";

\begin{bolded}General claim\end{underlined}\end{bolded} :
Let $f(x)$ be a continuous function from $\mathbb R\to\mathbb R$ such that :
$f(x)$ is periodic and one period is $T_1>0$
$f(x^2)$ is periodic and one period is $T_2>0$
Then $f(x)=c$ is the constant function.

\begin{bolded}Proof \end{underlined}\end{bolded}:
Let the set $A=\{\sqrt{mT_1}-nT_2$ $\forall m\in\mathbb N_0,\forall n\in\mathbb Z\}$

1) $f(x^2)=f(0)$ $\forall x\in\mathbb A$ :
Let $x=\sqrt{mT_1}-nT_2$ : $f(x^2)=f((x+nT_2)^2)=f(mT_1)=f(0)$

2) $\mathbb A$ is dense in $\mathbb R$
Let $\epsilon>0$.
Let $p>\frac{T_1}{4\epsilon^2}$ So that $\sqrt{(m+1)T_1}-\sqrt{mT_1}<\epsilon$ $\forall m\ge p$
So the sequence $a_n=\sqrt{(p+n)T_1}$ is an increasing sequence of elements of $\mathbb A$ such that $\epsilon >a_{n+1}-a_n>0$ $\forall n$
Let then $x\in\mathbb R$
Let $k\in\mathbb N$ great enough such that $x+kT_2>a_1$ so that we can find $n$ such that $a_n\le x+kT_2\le a_{n+1}$

So $a_n-kT_2\le x\le a_{n+1}-kT_2$ and since $a_n-kT_2\in\mathbb A$ and $a_{n+1}-kT_2\in\mathbb A$ and $|(a_{n+1}-kT_2)-(a_{n}-kT_2)|<\epsilon$, we got the result.

3) $f(x)=f(0)$ $\forall x$
The points 1) and 2) above + continuity of $f(x)$ imply $f(x)=f(0)$ $\forall x\ge 0$
$f(x)$ is periodic and so $f(x)=f(0)$ $\forall x$


Hence the result.
\end{solution}
*******************************************************************************
-------------------------------------------------------------------------------

\begin{problem}[Posted by \href{https://artofproblemsolving.com/community/user/122641}{TrungK40PBC}]
	\begin{bolded}Problem.\end{bolded} Find the functions $f:\mathbb{N}\rightarrow \mathbb{N}$ with satisfy:
(i), $f(f(n))=n$ 
(ii), $n|f(1)+f(2)+...+f(n)$
for every $n\in \mathbb{N}$ ($\mathbb{N}$ is the set of all positive integers)
	\flushright \href{https://artofproblemsolving.com/community/c6h529756}{(Link to AoPS)}
\end{problem}



\begin{solution}[by \href{https://artofproblemsolving.com/community/user/29428}{pco}]
	\begin{tcolorbox}\begin{bolded}Problem.\end{bolded} Find the functions $f:\mathbb{N}\rightarrow \mathbb{N}$ with satisfy:
(i), $f(f(n))=n$ 
(ii), $n|f(1)+f(2)+...+f(n)$
for every $n\in \mathbb{N}$ ($\mathbb{N}$ is the set of all positive integers)\end{tcolorbox}
Is it a real Olympiad exercise ?
In what olympiad did you get it ?

I dont think it's possible to get a general solution for such an equation.
We can build infinitely many such solutions. hereunder are an algorithm to build infinetely many solutions and some examples.

We'll build a sequence of sets $\{A_n\}_{n\in\mathbb N}$ such that the seven following assertions all are true :
P1(n) : $[1,n]\subseteq A_n$
P2(n) : $f(x)$ is defined $\forall x\in A_n$
P3(n) : $f(A_n)=A_n$
P4(n) : $f(f(k))=k$ $\forall k\in A_n$
P5(n) : $k|\sum_{i=1}^kf(i)$ $\forall k\in[1,n]$
P6(n) : If $n+1\in A_n$, then $n+1|\sum_{i=1}^{n+1}f(i)$
P7(n) : If $k\in A_n$ with $k>n$, then $k+1\notin A_n$

This will be enough to completely define a solution $f(x)$ to your problem


Let $A_1=\{1,a\}$ with $a>2$ and $f(1)=a$ and $f(a)=1$
The seven assertions are obviously true for $n=1$

Consider now that $A_n$ is defined with all properties. We can build $A_{n+1}$ in such way :
1) If $n+1\in A_n$, then $A_{n+1}=A_n$
$n+1\in A_n$ and $P1(n)$ $\implies$ $P1(n+1)$
$A_{n+1}=A_n$ and $P2(n)$ $\implies$ $P2(n+1)$
$A_{n+1}=A_n$ and $P3(n)$ $\implies$ $P3(n+1)$
$A_{n+1}=A_n$ and $P4(n)$ $\implies$ $P4(n+1)$
$P5(n)$ and $P6(n)$ $\implies$ $P5(n+1)$
$P7(n)$ $\implies$ $n+2\notin A_{n+1}$ and so $P6(n+1)$
$A_{n+1}=A_n$ and $P7(n)$ $\implies$ $P7(n+1)$

2) If $n+1\notin A_n$ and $n+2\notin A_n$
Let $a\in[0,n]$ such that $a\equiv -\sum_{i=1}^nf(i)\pmod{n+1}$
Choose then any integer $m$ such that $a+m(n+1)>\max(\max(A_n)+1,n+2)$
Notice that neither $n+1$, neither $a+m(n+1)$ are in $A_n$ and that $n+1\ne a+m(n+1)$ 

Set $A_{n+1}=A_n\cup\{n+1,a+m(n+1)\}$ and define $f(n+1)=a+m(n+1)$ and $f(a+m(n+1))=n+1$

$P1(n+1)$ is true since we added $n+1$ to $A_n$
$P2(n+1)$ is true since we defined $f(n+1)$ and $f(a+m(n+1))$ for the two new distinct elements added
$P3(n+1)$ is true since $P3(n)$ is true and $f(n+1)=a+m(n+1)$ and $f(a+m(n+1))=n+1$
$P4(n+1)$ is true since $P4(n)$ is true and $f(n+1)=a+m(n+1)$ and $f(a+m(n+1))=n+1$
$P5(n+1)$ is true since $P5(n)$ is true and $f(n+1)\equiv -\sum_{i=1}^nf(i)\pmod{n+1}$
$P6(n+1)$ is true since $n+2\notin A_n$ and $a+m(n+1)>n+2$ and so $n+2\notin A_{n+1}$
$P7(n+1)$ is true since $P7(n)$ is true and $a+m(n+1)>\max(A_n)+1$

3) If $n+1\notin A_n$ and $n+2\in A_n$
Let $a\in[0,n]$ such that $a\equiv -\sum_{i=1}^nf(i)\pmod{n+1}$
Let $b\in[0,n+1]$ such that $b\equiv a+f(n+2)+\sum_{i=1}^nf(i)\pmod{n+2}$
Choose then any integer $m$ such that $a+b(n+1)+m(n+1)(n+2)>\max(\max(A_n)+1,n+2)$

Set $A_{n+1}=A_n\cup\{n+1,a+b(n+1)+m(n+1)(n+2)\}$ and define $f(n+1)=a+b(n+1)+m(n+1)(n+2)$ and $f(a+b(n+1)+m(n+1)(n+2))=n+1$

$P1(n+1)$ is true since we added $n+1$ to $A_n$
$P2(n+1)$ is true since we defined $f(n+1)$ and $f(a+b(n+1)+m(n+1)(n+2))$ for the two new distinct elements added
$P3(n+1)$ is true since $P3(n)$ is true and $f(n+1)=a+b(n+1)+m(n+1)(n+2)$ and $f(a+b(n+1)+m(n+1)(n+2))=n+1$
$P4(n+1)$ is true since $P4(n)$ is true and $f(n+1)=a+b(n+1)+m(n+1)(n+2)$ and $f(a+b(n+1)+m(n+1)(n+2))=n+1$
$P5(n+1)$ is true since $P5(n)$ is true and $f(n+1)\equiv -\sum_{i=1}^nf(i)\pmod{n+1}$
$P6(n+1)$ is true since $n+2\in A_{n+1}$ and $f(n+1)\equiv a-b\equiv -f(n+2)-\sum_{i=1}^nf(i)\pmod{n+2}$
$P7(n+1)$ is true since $P7(n)$ is true and $a+b(n+1)+m(n+1)(n+2)>\max(A_n)+1$

So we can build as many such functions as we want.

Some examples :

E1): applying the algorithm with $f(1)=3$ and always minimum $m$ in steps 2) and 3) :
$f(1)=3$ and $f(3)=1$
$f(2)=5$ and $f(5)=2$
$f(3)=1$ and $f(1)=3$
$f(4)=19$ and $f(19)=4$
$f(5)=2$ and $f(2)=5$
$f(6)=24$ and $f(24)=6$
$f(7)=30$ and $f(30)=7$
$f(8)=36$ and $f(36)=8$
$f(9)=42$ and $f(42)=9$
$f(10)=48$ and $f(48)=10$
$f(11)=54$ and $f(54)=11$
$f(12)=60$ and $f(60)=12$
$f(13)=66$ and $f(66)=13$
$f(14)=72$ and $f(72)=14$
$f(15)=78$ and $f(78)=15$
$f(16)=84$ and $f(84)=16$
$f(17)=90$ and $f(90)=17$
$f(18)=384$ and $f(384)=18$
$f(19)=4$ and $f(4)=19$
$f(20)=398$ and $f(398)=20$
$f(21)=411$ and $f(411)=21$
$f(22)=421$ and $f(421)=22$
$f(23)=566$ and $f(566)=23$
$f(24)=6$ and $f(6)=24$
$f(25)=571$ and $f(571)=25$
$f(26)=581$ and $f(581)=26$
$f(27)=588$ and $f(588)=27$
$f(28)=592$ and $f(592)=28$
$f(29)=1057$ and $f(1057)=29$
$f(30)=7$ and $f(7)=30$
$f(31)=1078$ and $f(1078)=31$
$f(32)=1102$ and $f(1102)=32$
$f(33)=1123$ and $f(1123)=33$
$f(34)=1141$ and $f(1141)=34$
$f(35)=2136$ and $f(2136)=35$
$f(36)=8$ and $f(8)=36$
$f(37)=2171$ and $f(2171)=37$
$f(38)=2193$ and $f(2193)=38$
$f(39)=2209$ and $f(2209)=39$
$f(40)=2219$ and $f(2219)=40$
$f(41)=2797$ and $f(2797)=41$
$f(42)=9$ and $f(9)=42$
$f(43)=2819$ and $f(2819)=43$
$f(44)=2835$ and $f(2835)=44$

E2): applying the algorithm with $f(1)=4$ and always minimum $m$ in steps 2) and 3) :
$f(1)=4$ and $f(4)=1$
$f(2)=6$ and $f(6)=2$
$f(3)=17$ and $f(17)=3$
$f(4)=1$ and $f(1)=4$
$f(5)=42$ and $f(42)=5$
$f(6)=2$ and $f(2)=6$
$f(7)=47$ and $f(47)=7$
$f(8)=49$ and $f(49)=8$
$f(9)=57$ and $f(57)=9$
$f(10)=65$ and $f(65)=10$
$f(11)=73$ and $f(73)=11$
$f(12)=81$ and $f(81)=12$
$f(13)=89$ and $f(89)=13$
$f(14)=97$ and $f(97)=14$
$f(15)=105$ and $f(105)=15$
$f(16)=129$ and $f(129)=16$
$f(17)=3$ and $f(3)=17$
$f(18)=141$ and $f(141)=18$
$f(19)=151$ and $f(151)=19$
$f(20)=161$ and $f(161)=20$
$f(21)=171$ and $f(171)=21$
$f(22)=181$ and $f(181)=22$
$f(23)=191$ and $f(191)=23$
$f(24)=201$ and $f(201)=24$
$f(25)=211$ and $f(211)=25$
$f(26)=221$ and $f(221)=26$
$f(27)=231$ and $f(231)=27$
$f(28)=241$ and $f(241)=28$
$f(29)=251$ and $f(251)=29$
$f(30)=261$ and $f(261)=30$
$f(31)=271$ and $f(271)=31$
$f(32)=281$ and $f(281)=32$
$f(33)=291$ and $f(291)=33$
$f(34)=301$ and $f(301)=34$
$f(35)=311$ and $f(311)=35$
$f(36)=321$ and $f(321)=36$
$f(37)=331$ and $f(331)=37$
$f(38)=341$ and $f(341)=38$
$f(39)=351$ and $f(351)=39$
$f(40)=361$ and $f(361)=40$
$f(41)=453$ and $f(453)=41$
$f(42)=5$ and $f(5)=42$
$f(43)=470$ and $f(470)=43$
$f(44)=484$ and $f(484)=44$

E3): applying the algorithm with $f(1)=3$ and not always minimum $m$ in steps 2) and 3) (choosing with alea $m,m+1,m+2$)
$f(1)=3$ and $f(3)=1$
$f(2)=5$ and $f(5)=2$
$f(3)=1$ and $f(1)=3$
$f(4)=39$ and $f(39)=4$
$f(5)=2$ and $f(2)=5$
$f(6)=52$ and $f(52)=6$
$f(7)=59$ and $f(59)=7$
$f(8)=63$ and $f(63)=8$
$f(9)=73$ and $f(73)=9$
$f(10)=93$ and $f(93)=10$
$f(11)=105$ and $f(105)=11$
$f(12)=117$ and $f(117)=12$
$f(13)=129$ and $f(129)=13$
$f(14)=169$ and $f(169)=14$
$f(15)=200$ and $f(200)=15$
$f(16)=234$ and $f(234)=16$
$f(17)=254$ and $f(254)=17$
$f(18)=274$ and $f(274)=18$
$f(19)=332$ and $f(332)=19$
$f(20)=336$ and $f(336)=20$
$f(21)=379$ and $f(379)=21$
$f(22)=381$ and $f(381)=22$
$f(23)=449$ and $f(449)=23$
$f(24)=499$ and $f(499)=24$
$f(25)=502$ and $f(502)=25$
$f(26)=528$ and $f(528)=26$
$f(27)=554$ and $f(554)=27$
$f(28)=636$ and $f(636)=28$
$f(29)=695$ and $f(695)=29$
$f(30)=757$ and $f(757)=30$
$f(31)=760$ and $f(760)=31$
$f(32)=824$ and $f(824)=32$
$f(33)=924$ and $f(924)=33$
$f(34)=928$ and $f(928)=34$
$f(35)=999$ and $f(999)=35$
$f(36)=1073$ and $f(1073)=36$
$f(37)=1076$ and $f(1076)=37$
$f(38)=3432$ and $f(3432)=38$
$f(39)=4$ and $f(4)=39$
$f(40)=3460$ and $f(3460)=40$
$f(41)=3528$ and $f(3528)=41$
$f(42)=3590$ and $f(3590)=42$
$f(43)=3646$ and $f(3646)=43$
$f(44)=3740$ and $f(3740)=44$
\end{solution}



\begin{solution}[by \href{https://artofproblemsolving.com/community/user/86115}{Mahi}]
	\begin{tcolorbox}

Some examples :
[hide]
E1): applying the algorithm with $f(1)=3$ and always minimum $m$ in steps 2) and 3) :
$f(1)=3$ and $f(3)=1$
$f(2)=5$ and $f(5)=2$
$f(3)=1$ and $f(1)=3$
$f(4)=19$ and $f(19)=4$
$f(5)=2$ and $f(2)=5$
$f(6)=24$ and $f(24)=6$
$f(7)=30$ and $f(30)=7$
$f(8)=36$ and $f(36)=8$
$f(9)=42$ and $f(42)=9$
$f(10)=48$ and $f(48)=10$
$f(11)=54$ and $f(54)=11$
$f(12)=60$ and $f(60)=12$
$f(13)=66$ and $f(66)=13$
$f(14)=72$ and $f(72)=14$
$f(15)=78$ and $f(78)=15$
$f(16)=84$ and $f(84)=16$
$f(17)=90$ and $f(90)=17$
$f(18)=384$ and $f(384)=18$
$f(19)=4$ and $f(4)=19$
$f(20)=398$ and $f(398)=20$
$f(21)=411$ and $f(411)=21$
$f(22)=421$ and $f(421)=22$
$f(23)=566$ and $f(566)=23$
$f(24)=6$ and $f(6)=24$
$f(25)=571$ and $f(571)=25$
$f(26)=581$ and $f(581)=26$
$f(27)=588$ and $f(588)=27$
$f(28)=592$ and $f(592)=28$
$f(29)=1057$ and $f(1057)=29$
$f(30)=7$ and $f(7)=30$
$f(31)=1078$ and $f(1078)=31$
$f(32)=1102$ and $f(1102)=32$
$f(33)=1123$ and $f(1123)=33$
$f(34)=1141$ and $f(1141)=34$
$f(35)=2136$ and $f(2136)=35$
$f(36)=8$ and $f(8)=36$
$f(37)=2171$ and $f(2171)=37$
$f(38)=2193$ and $f(2193)=38$
$f(39)=2209$ and $f(2209)=39$
$f(40)=2219$ and $f(2219)=40$
$f(41)=2797$ and $f(2797)=41$
$f(42)=9$ and $f(9)=42$
$f(43)=2819$ and $f(2819)=43$
$f(44)=2835$ and $f(2835)=44$

E2): applying the algorithm with $f(1)=4$ and always minimum $m$ in steps 2) and 3) :
$f(1)=4$ and $f(4)=1$
$f(2)=6$ and $f(6)=2$
$f(3)=17$ and $f(17)=3$
$f(4)=1$ and $f(1)=4$
$f(5)=42$ and $f(42)=5$
$f(6)=2$ and $f(2)=6$
$f(7)=47$ and $f(47)=7$
$f(8)=49$ and $f(49)=8$
$f(9)=57$ and $f(57)=9$
$f(10)=65$ and $f(65)=10$
$f(11)=73$ and $f(73)=11$
$f(12)=81$ and $f(81)=12$
$f(13)=89$ and $f(89)=13$
$f(14)=97$ and $f(97)=14$
$f(15)=105$ and $f(105)=15$
$f(16)=129$ and $f(129)=16$
$f(17)=3$ and $f(3)=17$
$f(18)=141$ and $f(141)=18$
$f(19)=151$ and $f(151)=19$
$f(20)=161$ and $f(161)=20$
$f(21)=171$ and $f(171)=21$
$f(22)=181$ and $f(181)=22$
$f(23)=191$ and $f(191)=23$
$f(24)=201$ and $f(201)=24$
$f(25)=211$ and $f(211)=25$
$f(26)=221$ and $f(221)=26$
$f(27)=231$ and $f(231)=27$
$f(28)=241$ and $f(241)=28$
$f(29)=251$ and $f(251)=29$
$f(30)=261$ and $f(261)=30$
$f(31)=271$ and $f(271)=31$
$f(32)=281$ and $f(281)=32$
$f(33)=291$ and $f(291)=33$
$f(34)=301$ and $f(301)=34$
$f(35)=311$ and $f(311)=35$
$f(36)=321$ and $f(321)=36$
$f(37)=331$ and $f(331)=37$
$f(38)=341$ and $f(341)=38$
$f(39)=351$ and $f(351)=39$
$f(40)=361$ and $f(361)=40$
$f(41)=453$ and $f(453)=41$
$f(42)=5$ and $f(5)=42$
$f(43)=470$ and $f(470)=43$
$f(44)=484$ and $f(484)=44$

E3): applying the algorithm with $f(1)=3$ and not always minimum $m$ in steps 2) and 3) (choosing with alea $m,m+1,m+2$)
$f(1)=3$ and $f(3)=1$
$f(2)=5$ and $f(5)=2$
$f(3)=1$ and $f(1)=3$
$f(4)=39$ and $f(39)=4$
$f(5)=2$ and $f(2)=5$
$f(6)=52$ and $f(52)=6$
$f(7)=59$ and $f(59)=7$
$f(8)=63$ and $f(63)=8$
$f(9)=73$ and $f(73)=9$
$f(10)=93$ and $f(93)=10$
$f(11)=105$ and $f(105)=11$
$f(12)=117$ and $f(117)=12$
$f(13)=129$ and $f(129)=13$
$f(14)=169$ and $f(169)=14$
$f(15)=200$ and $f(200)=15$
$f(16)=234$ and $f(234)=16$
$f(17)=254$ and $f(254)=17$
$f(18)=274$ and $f(274)=18$
$f(19)=332$ and $f(332)=19$
$f(20)=336$ and $f(336)=20$
$f(21)=379$ and $f(379)=21$
$f(22)=381$ and $f(381)=22$
$f(23)=449$ and $f(449)=23$
$f(24)=499$ and $f(499)=24$
$f(25)=502$ and $f(502)=25$
$f(26)=528$ and $f(528)=26$
$f(27)=554$ and $f(554)=27$
$f(28)=636$ and $f(636)=28$
$f(29)=695$ and $f(695)=29$
$f(30)=757$ and $f(757)=30$
$f(31)=760$ and $f(760)=31$
$f(32)=824$ and $f(824)=32$
$f(33)=924$ and $f(924)=33$
$f(34)=928$ and $f(928)=34$
$f(35)=999$ and $f(999)=35$
$f(36)=1073$ and $f(1073)=36$
$f(37)=1076$ and $f(1076)=37$
$f(38)=3432$ and $f(3432)=38$
$f(39)=4$ and $f(4)=39$
$f(40)=3460$ and $f(3460)=40$
$f(41)=3528$ and $f(3528)=41$
$f(42)=3590$ and $f(3590)=42$
$f(43)=3646$ and $f(3646)=43$
$f(44)=3740$ and $f(3740)=44$[\/hide]
\end{tcolorbox}

A little question @pco: did you compute these values by hand or did you write a code? :|
\end{solution}



\begin{solution}[by \href{https://artofproblemsolving.com/community/user/29428}{pco}]
	\begin{tcolorbox}A little question @pco: did you compute these values by hand or did you write a code? :|\end{tcolorbox}
I used standard spreadsheet (EXC**) to generate these examples.
\end{solution}



\begin{solution}[by \href{https://artofproblemsolving.com/community/user/29428}{pco}]
	Reposted in two parts. Part 1\/2
\begin{tcolorbox}\begin{bolded}Problem.\end{bolded} Find the functions $f:\mathbb{N}\rightarrow \mathbb{N}$ with satisfy:
(i), $f(f(n))=n$ 
(ii), $n|f(1)+f(2)+...+f(n)$
for every $n\in \mathbb{N}$ ($\mathbb{N}$ is the set of all positive integers)\end{tcolorbox}
Is it a real Olympiad exercise ?
In what olympiad did you get it ?

I dont think it's possible to get a general solution for such an equation.
We can build infinitely many such solutions. hereunder are an algorithm to build infinetely many solutions and some examples.

We'll build a sequence of sets $\{A_n\}_{n\in\mathbb N}$ such that the seven following assertions all are true :
P1(n) : $[1,n]\subseteq A_n$
P2(n) : $f(x)$ is defined $\forall x\in A_n$
P3(n) : $f(A_n)=A_n$
P4(n) : $f(f(k))=k$ $\forall k\in A_n$
P5(n) : $k|\sum_{i=1}^kf(i)$ $\forall k\in[1,n]$
P6(n) : If $n+1\in A_n$, then $n+1|\sum_{i=1}^{n+1}f(i)$
P7(n) : If $k\in A_n$ with $k>n$, then $k+1\notin A_n$

This will be enough to completely define a solution $f(x)$ to your problem


Let $A_1=\{1,a\}$ with $a>2$ and $f(1)=a$ and $f(a)=1$
The seven assertions are obviously true for $n=1$

Consider now that $A_n$ is defined with all properties. We can build $A_{n+1}$ in such way :
1) If $n+1\in A_n$, then $A_{n+1}=A_n$
$n+1\in A_n$ and $P1(n)$ $\implies$ $P1(n+1)$
$A_{n+1}=A_n$ and $P2(n)$ $\implies$ $P2(n+1)$
$A_{n+1}=A_n$ and $P3(n)$ $\implies$ $P3(n+1)$
$A_{n+1}=A_n$ and $P4(n)$ $\implies$ $P4(n+1)$
$P5(n)$ and $P6(n)$ $\implies$ $P5(n+1)$
$P7(n)$ $\implies$ $n+2\notin A_{n+1}$ and so $P6(n+1)$
$A_{n+1}=A_n$ and $P7(n)$ $\implies$ $P7(n+1)$

2) If $n+1\notin A_n$ and $n+2\notin A_n$
Let $a\in[0,n]$ such that $a\equiv -\sum_{i=1}^nf(i)\pmod{n+1}$
Choose then any integer $m$ such that $a+m(n+1)>\max(\max(A_n)+1,n+2)$
Notice that neither $n+1$, neither $a+m(n+1)$ are in $A_n$ and that $n+1\ne a+m(n+1)$ 

Set $A_{n+1}=A_n\cup\{n+1,a+m(n+1)\}$ and define $f(n+1)=a+m(n+1)$ and $f(a+m(n+1))=n+1$

$P1(n+1)$ is true since we added $n+1$ to $A_n$
$P2(n+1)$ is true since we defined $f(n+1)$ and $f(a+m(n+1))$ for the two new distinct elements added
$P3(n+1)$ is true since $P3(n)$ is true and $f(n+1)=a+m(n+1)$ and $f(a+m(n+1))=n+1$
$P4(n+1)$ is true since $P4(n)$ is true and $f(n+1)=a+m(n+1)$ and $f(a+m(n+1))=n+1$
$P5(n+1)$ is true since $P5(n)$ is true and $f(n+1)\equiv -\sum_{i=1}^nf(i)\pmod{n+1}$
$P6(n+1)$ is true since $n+2\notin A_n$ and $a+m(n+1)>n+2$ and so $n+2\notin A_{n+1}$
$P7(n+1)$ is true since $P7(n)$ is true and $a+m(n+1)>\max(A_n)+1$

3) If $n+1\notin A_n$ and $n+2\in A_n$
Let $a\in[0,n]$ such that $a\equiv -\sum_{i=1}^nf(i)\pmod{n+1}$
Let $b\in[0,n+1]$ such that $b\equiv a+f(n+2)+\sum_{i=1}^nf(i)\pmod{n+2}$
Choose then any integer $m$ such that $a+b(n+1)+m(n+1)(n+2)>\max(\max(A_n)+1,n+2)$

Set $A_{n+1}=A_n\cup\{n+1,a+b(n+1)+m(n+1)(n+2)\}$ and define $f(n+1)=a+b(n+1)+m(n+1)(n+2)$ and $f(a+b(n+1)+m(n+1)(n+2))=n+1$

$P1(n+1)$ is true since we added $n+1$ to $A_n$
$P2(n+1)$ is true since we defined $f(n+1)$ and $f(a+b(n+1)+m(n+1)(n+2))$ for the two new distinct elements added
$P3(n+1)$ is true since $P3(n)$ is true and $f(n+1)=a+b(n+1)+m(n+1)(n+2)$ and $f(a+b(n+1)+m(n+1)(n+2))=n+1$
$P4(n+1)$ is true since $P4(n)$ is true and $f(n+1)=a+b(n+1)+m(n+1)(n+2)$ and $f(a+b(n+1)+m(n+1)(n+2))=n+1$
$P5(n+1)$ is true since $P5(n)$ is true and $f(n+1)\equiv -\sum_{i=1}^nf(i)\pmod{n+1}$
$P6(n+1)$ is true since $n+2\in A_{n+1}$ and $f(n+1)\equiv a-b\equiv -f(n+2)-\sum_{i=1}^nf(i)\pmod{n+2}$
$P7(n+1)$ is true since $P7(n)$ is true and $a+b(n+1)+m(n+1)(n+2)>\max(A_n)+1$

So we can build as many such functions as we want.


\end{solution}



\begin{solution}[by \href{https://artofproblemsolving.com/community/user/29428}{pco}]
	Part 2\/2

Some examples :

E1): applying the algorithm with $f(1)=3$ and always minimum $m$ in steps 2) and 3) :
$f(1)=3$ and $f(3)=1$
$f(2)=5$ and $f(5)=2$
$f(3)=1$ and $f(1)=3$
$f(4)=19$ and $f(19)=4$
$f(5)=2$ and $f(2)=5$
$f(6)=24$ and $f(24)=6$
$f(7)=30$ and $f(30)=7$
$f(8)=36$ and $f(36)=8$
$f(9)=42$ and $f(42)=9$
$f(10)=48$ and $f(48)=10$
$f(11)=54$ and $f(54)=11$
$f(12)=60$ and $f(60)=12$
$f(13)=66$ and $f(66)=13$
$f(14)=72$ and $f(72)=14$
$f(15)=78$ and $f(78)=15$
$f(16)=84$ and $f(84)=16$
$f(17)=90$ and $f(90)=17$
$f(18)=384$ and $f(384)=18$
$f(19)=4$ and $f(4)=19$
$f(20)=398$ and $f(398)=20$
$f(21)=411$ and $f(411)=21$
$f(22)=421$ and $f(421)=22$
$f(23)=566$ and $f(566)=23$
$f(24)=6$ and $f(6)=24$
$f(25)=571$ and $f(571)=25$
$f(26)=581$ and $f(581)=26$
$f(27)=588$ and $f(588)=27$
$f(28)=592$ and $f(592)=28$
$f(29)=1057$ and $f(1057)=29$
$f(30)=7$ and $f(7)=30$
$f(31)=1078$ and $f(1078)=31$
$f(32)=1102$ and $f(1102)=32$
$f(33)=1123$ and $f(1123)=33$
$f(34)=1141$ and $f(1141)=34$
$f(35)=2136$ and $f(2136)=35$
$f(36)=8$ and $f(8)=36$
$f(37)=2171$ and $f(2171)=37$
$f(38)=2193$ and $f(2193)=38$
$f(39)=2209$ and $f(2209)=39$
$f(40)=2219$ and $f(2219)=40$
$f(41)=2797$ and $f(2797)=41$
$f(42)=9$ and $f(9)=42$
$f(43)=2819$ and $f(2819)=43$
$f(44)=2835$ and $f(2835)=44$

E2): applying the algorithm with $f(1)=4$ and always minimum $m$ in steps 2) and 3) :
$f(1)=4$ and $f(4)=1$
$f(2)=6$ and $f(6)=2$
$f(3)=17$ and $f(17)=3$
$f(4)=1$ and $f(1)=4$
$f(5)=42$ and $f(42)=5$
$f(6)=2$ and $f(2)=6$
$f(7)=47$ and $f(47)=7$
$f(8)=49$ and $f(49)=8$
$f(9)=57$ and $f(57)=9$
$f(10)=65$ and $f(65)=10$
$f(11)=73$ and $f(73)=11$
$f(12)=81$ and $f(81)=12$
$f(13)=89$ and $f(89)=13$
$f(14)=97$ and $f(97)=14$
$f(15)=105$ and $f(105)=15$
$f(16)=129$ and $f(129)=16$
$f(17)=3$ and $f(3)=17$
$f(18)=141$ and $f(141)=18$
$f(19)=151$ and $f(151)=19$
$f(20)=161$ and $f(161)=20$
$f(21)=171$ and $f(171)=21$
$f(22)=181$ and $f(181)=22$
$f(23)=191$ and $f(191)=23$
$f(24)=201$ and $f(201)=24$
$f(25)=211$ and $f(211)=25$
$f(26)=221$ and $f(221)=26$
$f(27)=231$ and $f(231)=27$
$f(28)=241$ and $f(241)=28$
$f(29)=251$ and $f(251)=29$
$f(30)=261$ and $f(261)=30$
$f(31)=271$ and $f(271)=31$
$f(32)=281$ and $f(281)=32$
$f(33)=291$ and $f(291)=33$
$f(34)=301$ and $f(301)=34$
$f(35)=311$ and $f(311)=35$
$f(36)=321$ and $f(321)=36$
$f(37)=331$ and $f(331)=37$
$f(38)=341$ and $f(341)=38$
$f(39)=351$ and $f(351)=39$
$f(40)=361$ and $f(361)=40$
$f(41)=453$ and $f(453)=41$
$f(42)=5$ and $f(5)=42$
$f(43)=470$ and $f(470)=43$
$f(44)=484$ and $f(484)=44$

E3): applying the algorithm with $f(1)=3$ and not always minimum $m$ in steps 2) and 3) (choosing with alea $m,m+1,m+2$)
$f(1)=3$ and $f(3)=1$
$f(2)=5$ and $f(5)=2$
$f(3)=1$ and $f(1)=3$
$f(4)=39$ and $f(39)=4$
$f(5)=2$ and $f(2)=5$
$f(6)=52$ and $f(52)=6$
$f(7)=59$ and $f(59)=7$
$f(8)=63$ and $f(63)=8$
$f(9)=73$ and $f(73)=9$
$f(10)=93$ and $f(93)=10$
$f(11)=105$ and $f(105)=11$
$f(12)=117$ and $f(117)=12$
$f(13)=129$ and $f(129)=13$
$f(14)=169$ and $f(169)=14$
$f(15)=200$ and $f(200)=15$
$f(16)=234$ and $f(234)=16$
$f(17)=254$ and $f(254)=17$
$f(18)=274$ and $f(274)=18$
$f(19)=332$ and $f(332)=19$
$f(20)=336$ and $f(336)=20$
$f(21)=379$ and $f(379)=21$
$f(22)=381$ and $f(381)=22$
$f(23)=449$ and $f(449)=23$
$f(24)=499$ and $f(499)=24$
$f(25)=502$ and $f(502)=25$
$f(26)=528$ and $f(528)=26$
$f(27)=554$ and $f(554)=27$
$f(28)=636$ and $f(636)=28$
$f(29)=695$ and $f(695)=29$
$f(30)=757$ and $f(757)=30$
$f(31)=760$ and $f(760)=31$
$f(32)=824$ and $f(824)=32$
$f(33)=924$ and $f(924)=33$
$f(34)=928$ and $f(928)=34$
$f(35)=999$ and $f(999)=35$
$f(36)=1073$ and $f(1073)=36$
$f(37)=1076$ and $f(1076)=37$
$f(38)=3432$ and $f(3432)=38$
$f(39)=4$ and $f(4)=39$
$f(40)=3460$ and $f(3460)=40$
$f(41)=3528$ and $f(3528)=41$
$f(42)=3590$ and $f(3590)=42$
$f(43)=3646$ and $f(3646)=43$
$f(44)=3740$ and $f(3740)=44$
\end{solution}
*******************************************************************************
-------------------------------------------------------------------------------

\begin{problem}[Posted by \href{https://artofproblemsolving.com/community/user/102873}{Dynamite127}]
	Find all continuous functions f and g that map reals to reals such that $f(x-y)=f(x)f(y)+g(x)g(y)$ for all x,y.
	\flushright \href{https://artofproblemsolving.com/community/c6h529829}{(Link to AoPS)}
\end{problem}



\begin{solution}[by \href{https://artofproblemsolving.com/community/user/29428}{pco}]
	\begin{tcolorbox}Find all continuous functions f and g that map reals to reals such that $f(x-y)=f(x)f(y)+g(x)g(y)$ for all x,y.\end{tcolorbox}
Look the Web for "d'Alembert functional equations"
If $f(x)=c$ is constant, then $g(x)^2=c-c^2$ and so we need $c\in[0,1]$ and either $g^2(x)=c-c^2$ and so, since $g(x)$ is continuous :
either $g(x)=\sqrt{c-c^2}$ $\forall x$
either $g(x)=-\sqrt{c-c^2}$ $\forall x$

Let us from now look only for solutions where $f(x)$ is non constant.
Let $P(x,y)$ be the assertion $f(x-y)=f(x)f(y)+g(x)g(y)$

1) $f(0)=1$
=========
If $f(0)\ne 1$, let $a=\frac{g(0)}{1-f(0)}$. Then $P(x,0)$ $\implies$ $f(x)=a.g(x)$
Plugging this in $P(x,x)$, we get $g(x)^2=\frac a{a^2+1}g(0)$
So, since continuous, $g(x)$ is constant and so is $f(x)$, which is impossible in this part of the solution.
Q.E.D

2) $f(x)$ is an even function and $g(x)$ is an odd function
=======================================
Comparing $P(x,0)$ and $P(0,x)$ immediately gives $f(-x)=f(x)$
$P(x,x)$ $\implies$ $f(x)^2+g(x)^2=1$ and so $g(-x)^2=g(x)^2$ $\forall x$

Suppose now that $g(-x)=g(x)$ on some non empty interval $(u,v)$ with $u<v$
Let then $y\in(u,v)$ :
$P(x+y,y)$ $\implies$ $f(x)=f(x+y)f(y)+g(x+y)g(y)$
$P(x+y,-y)$ $\implies$ $f(x+2y)=f(x+y)f(-y)+g(x+y)g(-y)$ $=f(x+y)f(y)+g(x+y)g(y)$ $=f(x)$
so $f(x+2y)=f(x)$ $\forall x$ and $\forall y\in(u,v)$ and it's quite easy to conclude, using continuity, that $f(x)$ is constant, impossible in this part of the solution;

So $g(-x)=g(x)$ is only possible in a separated space and continuity implies $g(-x)=-g(x)$ $\forall x$
Q.E.D.

3) $f(x)=\cos (ux)$ for some $u>0$
=========================
3.1) $\exists a>0$ such that $f(a)=0$ and $f(x)\in(0,1)$ $\forall x\in (0,a)$
--------------------------------------------------------------------
$P(x,x)$ $\implies$ $f(x)^2+g(x)^2=1$ and so $|f(x)|\le 1$ $\forall x$
$P(x,-x)$ $\implies$ $f(2x)=f(x)^2-g(x)^2$ and so $f(2x)=2f(x)^2-1$
Since $f(x)$ is not constant and $f(0)=1$, $\exists c>0$ such that $f(c)\in (-1,1)$ (we can choose $c>0$ since $f(x)$ is even
It's then easy, applying $f(2x)=2f(x)^2-1$ to show that some of $f(c), f(2c),f(4c),f(8c),...$ is $\le 0$
So, using $f(0)>0$ and continuity, $\exists x_0>0$ such that $f(x_0)=0$ 

Let then $A=\{x>0$ such that $f(x)=0\}$
$A$ is non empty and so $\exists a=\inf(A)$. Continuity implies $f(a)=0$ and so $a>0$
So $f(0)=1$ and $f(a)=0$ and $f(x)>0$ $\forall x\in[0,a)$

If $f(t)=1$ for some $t\in(0,a)$, then $g(t)=0$ and $P(\frac t2,-\frac t2)$ $\implies$ $f(\frac t2)^2=1$ and so $f(\frac t2)=1$
So $f(t2^{-n})=1$ $\forall n\in\mathbb N$
Using then induction with $P(kt2^{-n},-t2^{-n})$, we get $f(kt2^{-n})=1$ $\forall k,n\in\mathbb N$ and continuity implies $f(x)=1$ $\forall x$, impossible.

So $f(x)\in(0,1)$ $\forall x\in(0,a)$
Q.E.D

3.2) $f(x)=\cos ux$ $\forall x\in [0,a]$ with $u=\frac{\pi}{2a}$
--------------------------------------------------------------
$f(a)=0$ implies $g(a)^2=1$
if $(f,g)$ is a solution, $(f,-g)$ is also a solution, so Wlog consider $g(a)=+1$ and so $g(x)\in(0,1)$ $\forall x\in(0,a)$

Let $u=\frac{\pi}{2a}$
let $x\in [0,a]$ : $P(\frac x2,-\frac x2)$ $\implies$ $f(x)=2f(\frac x2)^2-1$ and so $f(\frac x2)=\sqrt{\frac{1+f(x)}2}$
(we can choose the positive root instead of the negative one since we are in $[0,a]$ where $f(x)\ge 0$)

We have $f(a)=\cos ua$ and so $f(a2^{-n})=\cos (ua2^{-n})$ $\forall n\in\mathbb N$
As a consequence, $g(a2^{-n})=\sin (ua2^{-n})$ $\forall n\in\mathbb N$

Using then induction with $P(ka2^{-n},-a2^{-n})$, we get $f(ka2^{-n})=\cos kua2^{-n} $ $\forall k,n\in\mathbb N$ such that $k2^{-n}\in(0,a)$ and continuity implies $f(x)=\cos ux$ $\forall x\in [0,a]$
Q.E.D.

3.3) $f(x)=\cos ux$ $\forall x$
----------------------------
Since $f(x)$ is even, we got $f(x)=\cos ux$ $\forall x\in[-a,+a]$
$P(a,-a)$ $\implies$ $f(2a)=-1$ and so $g(2a)=0$
$P(x+2a,2a)$ $\implies$ $f(x+2a)=-f(x)$
And so $f(x)=\cos ux$ $\forall x$

Then $P(x,a)$ $\implies$ $\cos u(x-a)=g(x)$ and so $g(x)=\sin ux$
And it's easy to check that these functions indeed are solutions.

4) Synthesis of solutions
=================
We got :

S1 : $f(x)=c$ $\forall x$ and $g(x)=\sqrt{c-c^2}$ $\forall x$ where $c$ is any real $\in[0,1]$

S2 : $f(x)=c$ $\forall x$ and $g(x)=-\sqrt{c-c^2}$ $\forall x$ where $c$ is any real $\in[0,1]$

S3 : $f(x)=\cos ux$ $\forall x$ and $g(x)=\sin ux$ $\forall x$ where $u$ is any real.
(Note that we proved this result with $u>0$ but we wrote that $(f,g)$ solution implies $(f,-g)$ solution and allowing $u<0$ includes this possibility)
\end{solution}
*******************************************************************************
-------------------------------------------------------------------------------

\begin{problem}[Posted by \href{https://artofproblemsolving.com/community/user/122611}{oty}]
	Find all continious function $f:\mathbb{R}\to \mathbb{R}$ such that for all $x$ in $\mathbb{R}$ : 
\[\sum_{k=0}^{n}\binom{n}{k}f(x^{2^{k}})=0\]
	\flushright \href{https://artofproblemsolving.com/community/c6h530057}{(Link to AoPS)}
\end{problem}



\begin{solution}[by \href{https://artofproblemsolving.com/community/user/29428}{pco}]
	\begin{tcolorbox}Find all continious function $f:\mathbb{R}\to \mathbb{R}$ such that for all $x$ in $\mathbb{R}$ : 
\[\sum_{k=0}^{n}\binom{n}{k}f(x^{2^{k}})=0\]\end{tcolorbox}
Let $n=1$ and the equation $f(x)+f(x^2)=0$
$f(0)=f(1)=0$ and $f(-x)=f(x)$
$f(x^4)=f(x)$ and so $f(x)=f(x^{4^{-k}})$ $\forall x>0$ $\forall k\in\mathbb N$
Setting then $k\to+\infty$ and using continuity, we get $f(x)=f(1)=0$ $\forall x>0$ and so $\boxed{f(x)=0}$ $\forall x$

For $n>1$, let $f(x)$ continuous and such that $\sum_{k=0}^n\binom nkf(x^{2^k})=0$
Let $g(x)=\sum_{k=0}^{n-1}\binom{n-1}kf(x^{2^k})$ : the equation becomes $g(x)+g(x^2)=0$ with $g(x)$ continuous and so $g(x)=0$

And simple induction implies then that the only solution, whatever is $n$, is $f(x)=0$ $\forall x$
\end{solution}
*******************************************************************************
-------------------------------------------------------------------------------

\begin{problem}[Posted by \href{https://artofproblemsolving.com/community/user/177724}{vanu1996}]
	Find all solutions f:R->R such that f(f(x)(f(x)+f(y)))=2y(f(x)+1)+3 for all x and y.
	\flushright \href{https://artofproblemsolving.com/community/c6h530193}{(Link to AoPS)}
\end{problem}



\begin{solution}[by \href{https://artofproblemsolving.com/community/user/29428}{pco}]
	\begin{tcolorbox}Find all solutions f:R->R such that f(f(x)(f(x)+f(y)))=2y(f(x)+1)+3\end{tcolorbox}
Let $P(x,y)$ be the assertion $f(f(x)(f(x)+f(y)))=2y(f(x)+1)+3$

$f(x)=-1$ $\forall x$ is not a solution. Let then $a$ such that $f(a)\ne -1$ and let $b=\frac{-3}{2(f(a)+1)}$ and $c=f(a)(f(a)+f(b))$

$P(a,b)$ $\implies$ $f(c)=0$
$P(c,x)$ $\implies$ $f(0)=2x+3$ $\forall x$, which is impossible

So no solution for this functional equation.
\end{solution}
*******************************************************************************
-------------------------------------------------------------------------------

\begin{problem}[Posted by \href{https://artofproblemsolving.com/community/user/68025}{Pirkuliyev Rovsen}]
	Determine all function $f: \mathbb{R}\to\mathbb{R}$ such that $f(xf(y))+f(f(x)-f(y))=yf(x)+f(f(y))-f(f(x))$
for all $x,y{\in}R$.

__________________________________
Azerbaijan Land of the Fire 
	\flushright \href{https://artofproblemsolving.com/community/c6h530203}{(Link to AoPS)}
\end{problem}



\begin{solution}[by \href{https://artofproblemsolving.com/community/user/29428}{pco}]
	\begin{tcolorbox}Determine all function $f: \mathbb{R}\to\mathbb{R}$ such that $f(xf(y))+f(f(x)-f(y))=yf(x)+f(f(y))-f(f(x))$
for all $x,y{\in}R$.\end{tcolorbox}
Let $P(x,y)$ be the assertion $f(xf(y))+f(f(x)-f(y))=yf(x)+f(f(y))-f(f(x))$

$P(0,0)$ $\implies$ $f(0)=0$
$P(x,0)$ $\implies$ $f(f(x))=0$
$P(x,f(x))$ $\implies$ $f(x)^2=0$

Hence the unique possibility : $\boxed{f(x)=0}$ $\forall x$ which indeed is a solution
\end{solution}
*******************************************************************************
-------------------------------------------------------------------------------

\begin{problem}[Posted by \href{https://artofproblemsolving.com/community/user/10045}{socrates}]
	Determine all functions $f:\Bbb{R}\to\Bbb{R}$ such that \[  f(x^2 + f(y)) = (f(x) + y^2)^ 2 \] , for all $x,y\in \Bbb{R}.$
	\flushright \href{https://artofproblemsolving.com/community/c6h530289}{(Link to AoPS)}
\end{problem}



\begin{solution}[by \href{https://artofproblemsolving.com/community/user/29428}{pco}]
	\begin{tcolorbox}Determine all functions $f:\Bbb{R}\to\Bbb{R}$ such that \[  f(x^2 + f(y)) = (f(x) + y^2)^ 2 \] , for all $x,y\in \Bbb{R}.$\end{tcolorbox}
Let $P(x,y)$ be the assertion $f(x^2+f(y))=(f(x)+y^2)^2$
Let $a=f(0)$

If $a<0$, then $P(0,\sqrt{-a})$ $\implies$ $f(f(\sqrt{-a}))=0$ and $P(0,f(\sqrt{-a}))$ $\implies$ $a=(a+f(\sqrt{-a})^2)^2>0$ and so contradiction.
So $a\ge 0$

If $f(u)=f(v)$ subtracting $P(0,v)$ from $P(0,u)$ gives $(u^2-v^2)(u^2+v^2+2a)=0$ and so $|u|=|v|$ (since $a\ge 0$)

Comparing $P(x,0)$ and $P(-x,0)$, we get $f(-x)=\pm f(x)$ $\forall x$

if $f(-x)=-f(x)$ for some $x$, then comparing $P(1,x)$ and $P(1,-x)$, we get $f(1+f(x))=f(1-f(x))$ and so $|1+f(x)|=|1-f(x)|$ and so $f(x)=0$
So $f(-x)=f(x)$ $\forall x$

If $x\ge f(y)$ for some $x,y$, $P(\sqrt{x-f(y)},y)$ $\implies$ $f(x)\ge 0$
So, if $f(x)<0$ for some $x$, $x<f(y)$ $\forall y$ and so $x<f(x)<0$ which is impossible since $f(x)=f(-x)$ and both $x,-x$ cant be negative;
So $f(x)\ge 0$ $\forall x$

$P(0,0)$ $\implies$ $f(a)=a^2$
$P(\sqrt{f(x)},a)$ $\implies$ $f(f(x)+a^2)=(f(\sqrt{f(x)})+a^2)^2$
$P(a,x)$ $\implies$ $f(a^2+f(x))=(a^2+x^2)^2$
and so $f(\sqrt{f(x)})=x^2$ 

As a consequence, $f(x)$ is a bijection from $[0,+\infty)\to[0,+\infty)$ and $\exists u\ge 0$ such that $f(f(u))=0$
$P(0,u)$ $\implies$ $0=(u^2+a)^2$ and so $u=a=0$ 

Let then $g(x)=\sqrt{f(x)}$. We got that $g(x)$ is a bijective function from $[0,+\infty)\to[0,+\infty)$ with $g(0)=0$
$P(x,y)$ may be written $Q(x,y)$ : $g(x^2+g(y)^2)=g(x)^2+y^2$
$Q(x,0)$ $\implies$ $g(x^2)=g(x)^2$ and $Q(x,y)$ may be written $R(x,y)$ : $g(x+g(y))=g(x)+y$ $\forall x,y\ge 0$
This implies $g(g(y))=y$ and $g(x+y)=g(x)+g(y)$
And so $g(x)$ is an increasing solution (since $\ge 0$) of additive Cauchy equation and we get $g(x)=g(1)x$
Plugging this in $g(x^2)=g(x)^2$, we get $g(1)=1$ and $g(x)=x$

Hence the result : $\boxed{f(x)=x^2}$ $\forall x$ which indeed is a solution.
\end{solution}



\begin{solution}[by \href{https://artofproblemsolving.com/community/user/49556}{xxp2000}]
	1) $f(x)\geq0,\forall x\geq0$.
If there exists $f(y)<0$, $P(x,y)$ implies when $x\geq 0>f(y)$, we have $f(x)\geq0$.

2) If $f(d)=d^2$ for some $d\geq0$, $f(x)\geq d,\forall x\geq d$.
$P(x,d)$  for $x\geq0$ implies $f(x)\geq d^4,\forall x\geq d^2$.
Now $P(x,0)$ for $x\geq d$ implies $f^2(x)=f(x^2+f(0))\geq d^4$ or $f(x)\geq d^2,\forall x\geq d$.

3) $f$ is increasing when $x\geq0$.
Suppose $f(x)<f(y)$ for some $x>y\geq0$.
We have $a=f(y)+x^2>d=f(x)+x^2>b=f(x)+y^2\geq0$.
Notice $f(a)=b^2<d^2=f(d)$, contradiction with 2)

4) $f(x)=x^2,\forall x\geq0$
Fix $x>y\geq0$. Let $a=f(y)+x^2$ and $b=f(x)+y^2$.
We have $f(a)=b^2$ and $f(b)=a^2$. 3) implies $a=b$. So $f(x)-x^2$ is constant when $x\geq0$. Looking at $P(x,y)$ for $x,y\geq0$ we see that constant has to be zero.

5) $f(x)=x^2$
For any $y$, we can pick large $x>0$ such that $x^2+f(y)>0$. Then 4) implies $f(y)=y^2$.
\end{solution}
*******************************************************************************
-------------------------------------------------------------------------------

\begin{problem}[Posted by \href{https://artofproblemsolving.com/community/user/177722}{matematikolimpiyati}]
	What is the least positive integer $n$ such that $\overbrace{f(f(\dots f}^{21 \text{ times}}(n)))=2013$ where $f(x)=x+1+\lfloor \sqrt x \rfloor$? ($\lfloor a \rfloor$ denotes the greatest integer not exceeding the real number $a$.)

$ 
\textbf{(A)}\ 1214
\qquad\textbf{(B)}\ 1202
\qquad\textbf{(C)}\ 1186
\qquad\textbf{(D)}\ 1178
\qquad\textbf{(E)}\ \text{None of above}
$
	\flushright \href{https://artofproblemsolving.com/community/c6h530386}{(Link to AoPS)}
\end{problem}



\begin{solution}[by \href{https://artofproblemsolving.com/community/user/29428}{pco}]
	\begin{tcolorbox}What is the least positive integer $n$ such that $\overbrace{f(f(\dots f}^{21 \text{ times}}(n)))=2013$ where $f(x)=x+1+\lfloor \sqrt x \rfloor$? ($\lfloor a \rfloor$ denotes the greatest integer not exceeding the real number $a$.)

$ 
\textbf{(A)}\ 1214
\qquad\textbf{(B)}\ 1202
\qquad\textbf{(C)}\ 1186
\qquad\textbf{(D)}\ 1178
\qquad\textbf{(E)}\ \text{None of above}
$\end{tcolorbox}
According to me, the simplest way is just to compute $f^{[21]}(x)$ for the four given values.

Another more general way is to compute $f^{-1}(x)$ $21$ times :

$f(x)$ is an increasing function.

Writing $f(x)=a$ and $x=n^2+p$ with $p\in[0,2n]$, we get $n^2+n+1+p=a$ and so $n^2+n\le a-1\le n^2+3n$

So $\frac{-3+\sqrt{4a+5}}2\le n\le\frac{-1+\sqrt{4a-3}}2$ and since this interval is less than $1$, we get a method for computing $f^{-1}(a)$

Let $a=2013$ and we get $n=44$ and so $p=32$ and $f^{-1}(2013)=1968$
Let $a=1968$ and we get $n=43$ and so $p=75$ and $f^{-1}(1968)=1924$
Let $a=1924$ and we get $n=43$ and so $p=31$ and $f^{-1}(1924)=1880$
Let $a=1880$ and we get $n=42$ and so $p=73$ and $f^{-1}(1880)=1837$
Let $a=1837$ and we get $n=42$ and so $p=30$ and $f^{-1}(1837)=1794$
Let $a=1794$ and we get $n=41$ and so $p=71$ and $f^{-1}(1794)=1752$
Let $a=1752$ and we get $n=41$ and so $p=29$ and $f^{-1}(1752)=1710$
Let $a=1710$ and we get $n=40$ and so $p=69$ and $f^{-1}(1710)=1669$
Let $a=1669$ and we get $n=40$ and so $p=28$ and $f^{-1}(1669)=1628$
Let $a=1628$ and we get $n=39$ and so $p=67$ and $f^{-1}(1628)=1588$
Let $a=1588$ and we get $n=39$ and so $p=27$ and $f^{-1}(1588)=1548$
Let $a=1548$ and we get $n=38$ and so $p=65$ and $f^{-1}(1548)=1509$
Let $a=1509$ and we get $n=38$ and so $p=26$ and $f^{-1}(1509)=1470$
Let $a=1470$ and we get $n=37$ and so $p=63$ and $f^{-1}(1470)=1432$
Let $a=1432$ and we get $n=37$ and so $p=25$ and $f^{-1}(1432)=1394$
Let $a=1394$ and we get $n=36$ and so $p=61$ and $f^{-1}(1394)=1357$
Let $a=1357$ and we get $n=36$ and so $p=24$ and $f^{-1}(1357)=1320$
Let $a=1320$ and we get $n=35$ and so $p=59$ and $f^{-1}(1320)=1284$
Let $a=1284$ and we get $n=35$ and so $p=23$ and $f^{-1}(1284)=1248$
Let $a=1248$ and we get $n=34$ and so $p=57$ and $f^{-1}(1248)=1213$
Let $a=1213$ and we get $n=34$ and so $p=22$ and $f^{-1}(1213)=1178$

Hence the answer : $\boxed{1178}$
\end{solution}



\begin{solution}[by \href{https://artofproblemsolving.com/community/user/9551}{xeroxia}]
	\begin{bolded}Lemma 1\end{bolded}:
$\boxed{\left \lfloor \sqrt {f^{n}(x)} \right \rfloor \neq \left \lfloor \sqrt {f^{n+1}(x)} \right \rfloor \Longrightarrow \left \lfloor \sqrt {f^{n+1}(x)} \right \rfloor = \left \lfloor \sqrt {f^{n+2}(x)} \right \rfloor}$

Let $\left \lfloor \sqrt {f^{n}(x)} \right \rfloor  = a$. 

$a^2 \leq f^n(x) \leq a^2 + 2a$ and $a^2+2a+1\leq f^{n+1}(x)$.

$a^2+2a+1\leq f^{n+1}(x) \leq a^2 + 3a+1 = a^2 + 3a + 1 < (a+2)^2$

$\Longrightarrow \left \lfloor \sqrt {f^{n}(x)} \right \rfloor = a + 1$.

Calculating $f^{n+2}(x)$ for the largest value of $f^{n+1}(x)$:

$f^{n+2}(x)\leq a^2 + 3a+ 1 + 1 + a + 1 = a^2 + 4a+ 3 < (a+2)^2$

So $\left \lfloor \sqrt {f^{n+1}(x)} \right \rfloor = \left \lfloor \sqrt {f^{n+2}(x)} \right \rfloor = a+1$. $\blacksquare$


\begin{bolded}Lemma 2\end{bolded}:
$\boxed{\left \lfloor \sqrt {f^{n}(x)} \right \rfloor = \left \lfloor \sqrt {f^{n+1}(x)} \right \rfloor = a \Longrightarrow \left \lfloor \sqrt {f^{n+2}(x)} \right \rfloor = \left \lfloor \sqrt {f^{n+3}(x)} \right \rfloor = a+1 }$

$a^2 \leq f^n(x) < f^{n+1}(x) \leq a^2 + 2a$. 

Calculating $f^{n+2}(x)$ for the smallest value of $f^n(x)$:

$a^2+a+1 \leq f^{n+1}(x)$ and $f^{n+2}(x) \geq a^2 + a + 1+ 1+a = a^2 +2a+2 > (a+1)^2$.

Calculating $f^{n+2}(x)$ for the largest value of $f^{n+1}(x)$:

$f^{n+2}(x) \leq a^2 + 2a + 1+ a = a^2 +3a+1 < (a+2)^2$.

$(a+1)^2 < f^{n+2}<(a+2)^2 \Rightarrow \left \lfloor \sqrt {f^{n+2}(x)} \right \rfloor = a+1 $

From Lemma 1, $\left \lfloor \sqrt {f^{n+1}(x)} \right \rfloor \neq \left \lfloor \sqrt {f^{n+2}(x)} \right \rfloor \Longrightarrow  \left \lfloor \sqrt {f^{n+2}(x)} \right \rfloor = \left \lfloor \sqrt {f^{n+3}(x)} \right \rfloor$. $\blacksquare$


Since $f^{21}(x) = 2013$ and $44^2<2013<45^2$, $f^{22}(x) = 2013+1+44 = 2058 \geq 45^2$.
If we use the above lemmas repeatedly, we get 

$\left \lfloor \sqrt {x} \right \rfloor  = \left \lfloor \sqrt {f(x)} \right \rfloor  = a$

$\left \lfloor \sqrt {f^2(x)} \right \rfloor  =  \left \lfloor \sqrt {f^{3}(x)} \right \rfloor  = a+1$

$...$

$\left \lfloor \sqrt {f^{20}(x)} \right \rfloor  =  \left \lfloor \sqrt {f^{21}(x)} \right \rfloor  = a+10 = 44$

So $a=34$ and $34^2\leq x < 35^2$.

Repeatedly applting $f$ gives $f^{21}(x) = x + 35 + 36 + 36 + 37 + 37 + \dots + 43 + 44 + 45 $.
Also $f^{21}(x) = 2013 = x + 35 + 80\cdot 10  \Rightarrow x = 1178$.
So the answer is $\boxed{D}$.
\end{solution}
*******************************************************************************
-------------------------------------------------------------------------------

\begin{problem}[Posted by \href{https://artofproblemsolving.com/community/user/169515}{abl}]
	problem find all  contions functions $f:\mathbb{R}\to\mathbb{R}$ such that 
\[f(x+f(y^2)+(f(y))^2)=f(x)+2y^2,\forall x,y\in\mathbb{R}\]
	\flushright \href{https://artofproblemsolving.com/community/c6h530569}{(Link to AoPS)}
\end{problem}



\begin{solution}[by \href{https://artofproblemsolving.com/community/user/29428}{pco}]
	\begin{tcolorbox}problem find all  contions functions $f:\mathbb{R}\to\mathbb{R}$ such that 
\[f(x+f(y^2)+(f(y))^2)=f(x)+2y^2,\forall x,y\in\mathbb{R}\]\end{tcolorbox}
Let $P(x,y)$ be the assertion $f(x+f(y^2)+f(y)^2)=f(x)+2y^2$
Let $a=f(0)$
Let $g(x)=f(x^2)+f(x)^2$

The equation is $f(x+g(y))=f(x)+2y^2$ and since, setting $x=0$, we get $f(g(y))=a+2y^2$, we get  :
$f(x+y)=f(x)+f(y)-a$ $\forall x\in\mathbb R$, $\forall y\in g(\mathbb R)$

$g(x)$ is a continuous function and is non constant (since $f(g(x))=2x^2+a$) and so $\exists [u,v]\subseteq g(\mathbb R)$ with $u<v$

Let $h(x)=f(x)-a$ and we get :
$h(x+y)=h(x)+h(y)$ $\forall x\in\mathbb R$ and $\forall y\in[u,v]$

It's then easy to get $h(x)=cx$ $\forall x$ and $f(x)=cx+a$ and plugging this back in original equation, we get only one solution :

$\boxed{f(x)=x}$ $\forall x$
\end{solution}
*******************************************************************************
-------------------------------------------------------------------------------

\begin{problem}[Posted by \href{https://artofproblemsolving.com/community/user/169515}{abl}]
	problem find all  contionuous functions ${f:\mathbb{R}}\to\mathbb{R}$ such that
\[f(y)+f(x+f(y))=2y+f(f(x)+f(y)),\forall x,y\in\mathbb{R}\]
Continuous function is necessary or not ?
Generality:Solve:Let $\alpha\in\mathbb{R}$ find all   functions ${f:\mathbb{R}}\to\mathbb{R}$ such that
\[f(y)+f(x+f(y))=\alpha y+f(f(x)+f(y)),\forall x,y\in\mathbb{R}\]
	\flushright \href{https://artofproblemsolving.com/community/c6h530729}{(Link to AoPS)}
\end{problem}



\begin{solution}[by \href{https://artofproblemsolving.com/community/user/29428}{pco}]
	\begin{tcolorbox}problem find all  contionuous functions ${f:\mathbb{R}}\to\mathbb{R}$ such that
\[f(y)+f(x+f(y))=2y+f(f(x)+f(y)),\forall x,y\in\mathbb{R}\]\end{tcolorbox}
Let $P(x,y)$ be the assertion $f(y)+f(x+f(y))=2y+f(f(x)+f(y))$
Let $a=f(0)$

If $f(u)=f(v)$, then comparaison of $P(0,u)$ and $P(0,v)$ implies $u=v$ and $f(x)$ is injective and so, since continuous, monotonous.
Looking at $P(x,0)$ : $a+f(x+a)=f(f(x)+a)$, we see that $f(x)$ cant be decreasing (LHS would be decreasing while RHS would be increasing).
So $f(x)$ is continuous and increasing.

If $\lim_{x\to+\infty}f(x)=M$ finite, then setting $x\to+\infty$ in $P(x,x)$ would imply contradiction (finite LHS, unbounded RHS)
So $\lim_{x\to+\infty}f(x)=+\infty$

If $\lim_{x\to-\infty}f(x)=m$ finite, then setting $x\to-\infty$ in $P(x,x)$ would imply contradiction (finite LHS, unbounded RHS)
So $\lim_{x\to-\infty}f(x)=-\infty$

So $f(x)$ is a continuous increasing bijection from $\mathbb R\to\mathbb R$

Let then $u$ such that $f(u)=0$ : $P(x,u)$ $\implies$ $f(x)=2u+f(f(x))$ and so, since bijective, $f(x)=x-2u$ which is never a solution, whatever is $u$

So \begin{bolded}no solution for this functional equation\end{underlined}\end{bolded}.
\end{solution}
*******************************************************************************
-------------------------------------------------------------------------------

\begin{problem}[Posted by \href{https://artofproblemsolving.com/community/user/151349}{andrejilievski}]
	Find all functions f:Z\{0}-->N0 such that for every a,b which are elements of Z\{0} and a+b is an element of Z\{0}, f(a+b)>=min{f(a),f(b)} and f(ab)=f(a)+f(b).
	\flushright \href{https://artofproblemsolving.com/community/c6h530731}{(Link to AoPS)}
\end{problem}



\begin{solution}[by \href{https://artofproblemsolving.com/community/user/29428}{pco}]
	\begin{tcolorbox}Find all functions f:Z\{0}-->N0 such that for every a,b which are elements of Z\{0} and a+b is an element of Z\{0}, f(a+b)>=min{f(a),f(b)} and f(ab)=f(a)+f(b).\end{tcolorbox}
Let $P(x,y)$ be the assertion $f(x+y)\ge\min(f(x),f(y))$ true $\forall x,y\in\mathbb z$ such that $x,y,x+y\ne 0$
Let $Q(x,y)$ be the assertion $f(xy)=f(x)+f(y)$ true $\forall x,y\in\mathbb z$ such that $x,y,x+y\ne 0$

$Q(1,1)$ $\implies$ $f(1)=0$
$Q(-1,-1)$ $\implies$ $f(-1)=0$
$Q(x,-1)$ $\implies$ $f(-x)=f(x)$ $\forall x\ne 0$ (the case $1$ is already true)


$P(x,1-x)$ $\implies$ $0\ge\min(f(x),f(1-x))$ $\forall x\notin\{0,1\}$
And so, since $f(x)\ge 0$ $\forall x\ne 0$ : $\forall x\notin\{0,1\}$ : either $f(x)=0$, either $f(1-x)=0$
and so either $f(x)=0$, either $f(x-1)=0$

Let $A=\{0\}\cup\{x\in\mathbb Z^*$ such that $f(x)\ne 0\}$
Since $f(-x)=f(x)$, we get that $x\in A\implies -x\in A$
Let $u,v\in A$ :
If either $u=0$, either $v=0$, then $u+v\in A$
If $u+v=0$, then $u+v\in A$
If $u,v,u+v\ne 0$, then $P(u,v)$ $\implies$ $u+v\in A$
So $A$ is an additive subgroup of $\mathbb Z$ and so is $a\mathbb Z$ for some $a\ge 0$

If $a=0$, we get  the solution $f(x)=0$ $\forall x\ne 0$
If $a=1$, then $f(x)>0$ $\forall x\ne 0$, in contradiction with the previous sentence "either $f(x)=0$, either $f(x-1)=0$"
if $a>1$ is not prime, let $a=uv$ with $u,v>1$ 
$f(ua)=f(u)+f(a)=f(a)$ (since $u\notin a\mathbb Z)$
$f(va)=f(v)+f(a)=f(a)$ (since $v\notin a\mathbb Z)$
$f((ua)(va))=f(a^3)=3f(a)$
$f(ua)+f(va)=2f(a)$ and so $Q(ua, va)$ is wrong

If $a>1$ is prime, then we get the function :

$\boxed{f(x)=cv_p(|x|)}$ $\forall x\in\mathbb Z^*$, and whatever are "$p$" prime and $c\in\mathbb N\cup\{0\}$ (case $c=0$ is the previous solution)
and this indeed is a solution.
\end{solution}
*******************************************************************************
-------------------------------------------------------------------------------

\begin{problem}[Posted by \href{https://artofproblemsolving.com/community/user/68025}{Pirkuliyev Rovsen}]
	Find all real numbers $a$ for which the function $f(x)=\{ ax+sinx\}$ is periodic.


___________________________________
Azerbaijan Land of the Fire 
	\flushright \href{https://artofproblemsolving.com/community/c6h530753}{(Link to AoPS)}
\end{problem}



\begin{solution}[by \href{https://artofproblemsolving.com/community/user/29428}{pco}]
	\begin{tcolorbox}Find all real numbers $a$ for which the function $f(x)=\{ ax+sinx\}$ is periodic.\end{tcolorbox}
Let $T>0$ a period. We get $f(x+T)=f(x)$ $\forall x$ and so $(\sin(x+T)+a(x+T))-(\sin(x)+ax)\in\mathbb Z$ $\forall x$

And since $LHS$ is continuous, this means $(\sin(x+T)+a(x+T))-(\sin(x)+ax)=n\in\mathbb Z$ $\forall x$

Taking the derivative, we get $\cos(x+T)=\cos(x)$ $\forall x$ and so $T=2k\pi$ for some $k\in\mathbb N$ and so $2k\pi a=n$

Hence the answer :$\boxed{a=\frac{q}{2\pi}}$ for any ${q\in\mathbb Q}$
\end{solution}
*******************************************************************************
-------------------------------------------------------------------------------

\begin{problem}[Posted by \href{https://artofproblemsolving.com/community/user/75382}{djb86}]
	A function $f : \mathbb{Z}_+ \to \mathbb{Z}_+$, where $\mathbb{Z}_+$ is the set of positive integers, is non-decreasing and satisfies $f(mn) = f(m)f(n)$ for all relatively prime positive integers $m$ and $n$. Prove that $f(8)f(13) \ge (f(10))^2$.
	\flushright \href{https://artofproblemsolving.com/community/c6h530761}{(Link to AoPS)}
\end{problem}



\begin{solution}[by \href{https://artofproblemsolving.com/community/user/51248}{professordad}]
	[hide]
\begin{align*}f(26) &\geq f(25)\\ f(2) \cdot f(13) &\geq f(5) \cdot f(5)\\ \left(f(2)\right)^3 \cdot f(13) &\geq [f(2) \cdot f(5)]^2 \end{align*}
The result follows.
[\/hide]
\end{solution}



\begin{solution}[by \href{https://artofproblemsolving.com/community/user/89198}{chaotic_iak}]
	But $f(2)^3$ is not necessarily equal to $f(8)$.
\end{solution}



\begin{solution}[by \href{https://artofproblemsolving.com/community/user/29428}{pco}]
	\begin{tcolorbox}But $f(2)^3$ is not necessarily equal to $f(8)$.\end{tcolorbox}
and $f(25)$ is not necessarily equal to $f(5)f(5)$
\end{solution}



\begin{solution}[by \href{https://artofproblemsolving.com/community/user/127783}{Sayan}]
	$f(27)f(8) \ge f(21)f(10)$
$f(21)f(13) \ge f(27)f(10)$
\end{solution}



\begin{solution}[by \href{https://artofproblemsolving.com/community/user/89198}{chaotic_iak}]
	How do you get $f(27)f(8) \ge f(21)f(10)$ (and the second inequality too)?
\end{solution}



\begin{solution}[by \href{https://artofproblemsolving.com/community/user/51248}{professordad}]
	Oh whoops forgot "relatively prime"
Anyways, aren't both of the inequalities true by the condition "$f(mn) = f(m)f(n)$"?  216>210 and 273>270, and our function is "non-decreasing".  Also, all of the pairs are "relatively prime".
\end{solution}



\begin{solution}[by \href{https://artofproblemsolving.com/community/user/151586}{KOSNITA}]
	Also $f(8).f(19) \ge f(15).f(10)$
and $f(15).f(13)\ge f(19).f(10) $.
\end{solution}



\begin{solution}[by \href{https://artofproblemsolving.com/community/user/151586}{KOSNITA}]
	Sorry i didn't see that$15$and $10$ are coprime. Though i have found a solution.
$f(9).f(8) \ge f(10).f(7) $ and $f(13).f(7) \ge f(10).f(9) $
\end{solution}



\begin{solution}[by \href{https://artofproblemsolving.com/community/user/20689}{BOGTRO}]
	We want to have 2 inequalities of the form 

$f(8)f(x) \geq f(y)f(10)$
$f(y)f(13) \geq f(x)f(10)$

Which, when multiplied, gives the desired answer. 

Thus we require $8x>10y$ and $13y>10x \implies 26y>20x>25y$. Additionally, $(x,8), (x,10), (y,10), (y,13)$ are all relatively prime. These suffice as $f$ is a non-decreasing function. Therefore $x, y$ are odd. 

Let $n$ be the remainder when $25y$ is divided by $20$. Then, $n+y>20$ for an $x$ to exist. Note $25y \equiv n \pmod {20}$ means that $n=5,15$.

Case 1: $n=5 \implies y \equiv 1 \pmod 4 \implies y \geq 17$.
Case 2: $n=15 \implies y \equiv 3 \pmod 4 \implies y \geq 7$. 

7 satisfies the conditions, so we use

$f(8)f(9) \geq f(7)f(10)$
$f(7)f(13) \geq f(9)f(10)$

giving the desired answer.

Note that, of course, there are an infinite number of $y$ such that this works. But 7 is the smallest such $y$. Indeed, 21 is the smallest working $y$ for Case 1, as in Sayan's answer.
\end{solution}
*******************************************************************************
-------------------------------------------------------------------------------

\begin{problem}[Posted by \href{https://artofproblemsolving.com/community/user/68025}{Pirkuliyev Rovsen}]
	Find all functions $f: \mathbb{R_+}\to\mathbb{R_+}$ such that for all $x,y{\in}R_+$   $f(xf(y))=f(xy)+x$.


____________________________________
Azerbaijan Land of the Fire 
	\flushright \href{https://artofproblemsolving.com/community/c6h530776}{(Link to AoPS)}
\end{problem}



\begin{solution}[by \href{https://artofproblemsolving.com/community/user/29428}{pco}]
	\begin{tcolorbox}Find all functions $f: \mathbb{R_+}\to\mathbb{R_+}$ such that for all $x,y{\in}R_+$   $f(xf(y))=f(xy)+x$.\end{tcolorbox}
Let $P(x,y)$ be the assertion $f(xf(y))=f(xy)+x$
Let $a=f(1)$

(a) : $P(x,1)$ $\implies$ $f(ax)=f(x)+x$
(b) : $P(1,x)$ $\implies$ $f(f(x))=f(x)+1$
(c) : $P(f(x),1)$ $\implies$ $f(af(x))=f(f(x))+f(x)$
(d) : $P(a,x)$ $\implies$ $f(af(x))=f(ax)+a$
(a)-(b)-(c)+(d) : $f(x)=x+a-1$

Plugging this back in original equation, we get $a=2$ and the unique solution $\boxed{f(x)=x+1}$ $\forall x>0$
\end{solution}



\begin{solution}[by \href{https://artofproblemsolving.com/community/user/152460}{rahman}]
	i have a question!. before we take f(1)=a 
should we prove that the function is surjective .if so how 
I'll be thankfull to you
\end{solution}



\begin{solution}[by \href{https://artofproblemsolving.com/community/user/29428}{pco}]
	\begin{tcolorbox}i have a question!. before we take f(1)=a 
should we prove that the function is surjective .if so how 
I'll be thankfull to yuo\end{tcolorbox}
I just renamed $f(1)$ as "$a$" in order to get easier writing (I'm quite lazy). Nothing more.
\end{solution}



\begin{solution}[by \href{https://artofproblemsolving.com/community/user/148207}{Particle}]
	$(x,y)\rightarrow (f(y),x)\implies f(f(y)f(x))=f(f(y)x)+f(y)=f(xy)+x+f(y)$

$(x,y)\rightarrow (f(x),y)\implies f(f(x)f(y))=f(yx)+y+f(x)$. 

Hence $y+f(x)=x+f(y)$. Substituting $y=1$ yields $f(x)=x+f(1)-1$. So $f$ is a linear function and by checking, we find $f(x)=x+1$.
[Q.E.D.]
\end{solution}
*******************************************************************************
-------------------------------------------------------------------------------

\begin{problem}[Posted by \href{https://artofproblemsolving.com/community/user/68025}{Pirkuliyev Rovsen}]
	Find all functions $f: \mathbb{R_+}\to\mathbb{R_+}$ such that for any  $x,y>0$ , $x^2(f(x)+f(y))=(x+y)f(f(x)y)$.


____________________________________
Azerbaijan Land of the Fire 
	\flushright \href{https://artofproblemsolving.com/community/c6h530779}{(Link to AoPS)}
\end{problem}



\begin{solution}[by \href{https://artofproblemsolving.com/community/user/152460}{rahman}]
	if we put $x=y$ we get  $xf(x)=f(xf(x))$
easy we show that the function is injective than the unicue solution is $f(x)=x$
\end{solution}



\begin{solution}[by \href{https://artofproblemsolving.com/community/user/29428}{pco}]
	\begin{tcolorbox}Find all functions $f: \mathbb{R_+}\to\mathbb{R_+}$ such that for any  $x,y>0$ , $x^2(f(x)+f(y))=(x+y)f(f(x)y)$.\end{tcolorbox}
Let $P(x,y)$ be the assertion $x^2(f(x)+f(y))=(x+y)f(f(x)y)$
Let $a=f(1)$

$P(x,x)$ $\implies$ $f(xf(x))=xf(x)$ (note that setting here $x=1$ we get $f(a)=a$)

$P(xf(x),a)$ $\implies$ $x^2f(x)^2=f(axf(x))$
$p(a,xf(x))$ $\implies$ $a^2=f(axf(x))$

And so $x^2f(x)^2=a^2$ and $f(x)=\frac ax$ and then $f(a)=a$ implies $a=1$

hence the solution $\boxed{f(x)=\frac 1x}$ $\forall x>0$ which indeed is a solution.
\end{solution}



\begin{solution}[by \href{https://artofproblemsolving.com/community/user/148207}{Particle}]
	A bit different solution:
Like pco, I proved $f(1)=a=f(a)$.

$P(a,1)\implies 2a^3=(a+1)a\implies a(a-1)(2a+1)=0$. Since $a>0$, so $a=1$.

Now $P(1,y)\implies 1+f(y)=(1+y)f(y)\implies f(y)=\frac 1 y\; \forall y\in \mathbb R_+$
\end{solution}
*******************************************************************************
-------------------------------------------------------------------------------

\begin{problem}[Posted by \href{https://artofproblemsolving.com/community/user/75382}{djb86}]
	Find all functions $f$ such that
\[f(f(x) + y) = f(x^2-y) + 4yf(x)\]
for all real numbers $x$ and $y$.
	\flushright \href{https://artofproblemsolving.com/community/c6h530789}{(Link to AoPS)}
\end{problem}



\begin{solution}[by \href{https://artofproblemsolving.com/community/user/29428}{pco}]
	\begin{tcolorbox}Find all functions $f$ such that
\[f(f(x) + y) = f(x^2-y) + 4yf(x)\]
for all real numbers $x$ and $y$.\end{tcolorbox}
Let $P(x,y)$ be the assertion $f(f(x)+y)=f(x^2-y)+4yf(x)$

$P(x,\frac{x^2-f(x)}2)$ $\implies$ $f(x)(f(x)-x^2)=0$ $\forall x$ and so : $\forall x$, either $f(x)=0$, either $f(x)=x^2$

Suppose now that $\exists a,b\ne 0$ such that $f(a)=0$ and $f(b)=b^2$ :
$P(a,b)$ $\implies$ $b^2=f(a^2-b)$ and since $f(a^2-b)$ is either $0$, either $(a^2-b)^2$, we get $b^2=(a^2-b)^2$ and so $a^2=2b$

let then $c\notin\{0,a,b,\frac{a^2}2,\pm\sqrt{|2b|}\}$ : 
If $f(c)=0$, previous path using $c$ instead of $a$ implies $c^2=2b$, impossible
If $f(c)=c^2$, previous path using $c$ instead of $b$ implies $a^2=2c$, impossible

So no such $a,b$ exist and we got two solutions :

$f(x)=0$ $\forall x$, which indeed is a solution
$f(x)=x^2$ $\forall x$, which indeed is a solution
\end{solution}



\begin{solution}[by \href{https://artofproblemsolving.com/community/user/64716}{mavropnevma}]
	Is the similarity with [url]http://www.artofproblemsolving.com/Forum/viewtopic.php?p=827178&sid=87879fa80ebb0714ee55f93e2120b749#p827178[\/url] accidental?
\end{solution}
*******************************************************************************
-------------------------------------------------------------------------------

\begin{problem}[Posted by \href{https://artofproblemsolving.com/community/user/92485}{elegant}]
	Find all function $ f:\mathbb{R}\rightarrow\mathbb{R} $ such that :
$ f(xf(x)+f(y)) = (f(x))^2+y $ for all $ x, y\in\mathbb{R} $
	\flushright \href{https://artofproblemsolving.com/community/c6h531042}{(Link to AoPS)}
\end{problem}



\begin{solution}[by \href{https://artofproblemsolving.com/community/user/29428}{pco}]
	\begin{tcolorbox}Find all function $ f:\mathbb{R}\rightarrow\mathbb{R} $ such that :
$ f(xf(x)+f(y)) = (f(x))^2+y $ for all $ x, y\in\mathbb{R} $\end{tcolorbox}
Let $P(x,y)$ be the assertion $f(xf(x)+f(y))=f(x)^2+y$

$P(0,x)$ $\implies$ $f(f(x))=x+f(0)^2$ and so $f(x)$ is bijective. Let then $u=f(-f(0)^2)$ such that $f(u)=0$
$P(u,x)$ $\implies$ $f(f(x))=x$
So $f(0)=0$ (just compare the two values of $f(f(x))$ in the two previous lines)

$P(x,0)$ $\implies$ $f(xf(x))=f(x)^2$
$P(f(x),0)$ $\implies$ $f(f(x)x)=x^2$ 
And so $\forall x$, either $f(x)=x$, either $f(x)=-x$

Suppose now that $\exists a,b\ne 0$ such that $f(a)=a$ and $f(b)=-b$ : $P(a,b)$ $\implies$ $f(a^2-b)=a^2+b$ and so :
either $f(a^2-b)=a^2-b$ and we get $b=0$, impossible.
either $f(a^2-b)=b-a^2$ and we get $a=0$, impossible.
and so no such $a,b$

So :
Either $f(x)=x$ $\forall x$ which indeed is a solution.
Either $f(x)=-x$ $\forall x$, which indeed is a solution.
\end{solution}



\begin{solution}[by \href{https://artofproblemsolving.com/community/user/162032}{MrRTI}]
	\begin{tcolorbox}
$P(0,x)$ $\implies$ $f(f(x))=x+f(0)^2$ and so $f(x)$ is bijective.$
.\end{tcolorbox}

Could you explain me about determining the injectivity, surjectivity, or bijectivity of a function, sir ?
\end{solution}



\begin{solution}[by \href{https://artofproblemsolving.com/community/user/29428}{pco}]
	\begin{tcolorbox}Could you explain me about determining the injectivity, surjectivity, or bijectivity of a function, sir ?\end{tcolorbox}
Surjection :
From $f(f(x))=x+f(0)^2$, you see that $RHS$ may take any value we want and so $f(something)$ may take any value we want. Hence surjection.
On a more formal way, we can write here that $\forall x,\exists u$ such that $f(u)=x$ : just choose $u=f(x-f(0)^2)$

Injection :
if $f(a)=f(b)$, then $f(f(a))=f(f(b))$ and so $a+f(0)^2=b+f(0)^2$ and so $a=b$. Hence injection.

Notice that in this problem, I just needed surjectivity (in fact I just needed that $0\in f(\mathbb R)$ which is a far weaker need.
\end{solution}
*******************************************************************************
-------------------------------------------------------------------------------

\begin{problem}[Posted by \href{https://artofproblemsolving.com/community/user/68025}{Pirkuliyev Rovsen}]
	Find all function $f: \mathbb{R}\to\mathbb{R}$ such that $f(x^n-y^n)=xf^{n-1}(y)-yf^{n-1}(x)$ for all $x,y{\in}R$  and $n{\in}N$.


_______________________________________
Azerbaijan Land of the Fire 
	\flushright \href{https://artofproblemsolving.com/community/c6h531174}{(Link to AoPS)}
\end{problem}



\begin{solution}[by \href{https://artofproblemsolving.com/community/user/29428}{pco}]
	\begin{tcolorbox}Find all function $f: \mathbb{R}\to\mathbb{R}$ such that $f(x^n-y^n)=xf^{n-1}(y)-yf^{n-1}(x)$ for all $x,y{\in}R$  and $n{\in}N$.\end{tcolorbox}
Since this is true $\forall n\in\mathbb N$, this is true for $n=1$ and so $f(x-y)=x-y$ and so $\boxed{f(x)=x}$ which indeed is a solution.
\end{solution}
*******************************************************************************
-------------------------------------------------------------------------------

\begin{problem}[Posted by \href{https://artofproblemsolving.com/community/user/172163}{joybangla}]
	Does there exist a function $f$ defined on $\mathbb{N}$ such that $f(f(n))=n^2-19n+99$ ?
	\flushright \href{https://artofproblemsolving.com/community/c6h531177}{(Link to AoPS)}
\end{problem}



\begin{solution}[by \href{https://artofproblemsolving.com/community/user/29428}{pco}]
	\begin{tcolorbox}Does there exist a function $f$ defined on $\mathbb{N}$ such that $f(f(n))=n^2-19n+99$ ?\end{tcolorbox}
Yes, infinitely many such functions exist. Here is a method in order to build some :
Let $g(x)=x^2-19x+99$ from $\mathbb [12,+\infty)\to\mathbb N$ : $g(x)$ is increasing and injective.

Let $A=g([12,+\infty)$ so that $g(x)$ is a bijection from $[12,+\infty)\to A$
Let $B=[12,+\infty)\setminus A$
$B$ is an infinite set and so we can split it in two equinumerous subsets $B_1$ and $B_2$ and let $h(x)$ any bijection from $B_1\to B_2$

If $x\in A$ : $g^{-1}(x)$ exists and is such that $x>g^{-1}(x)\ge 12$ and so is either in $A$, either in $B$. We can then define in an unique manner :
$n(x)$ function from $A\to\mathbb N$
$b(x)$ function from $A\to B$
Such that $x=g^{n(x)}(x)(b(x))$ $\forall x\in A$

we can then define a solution $f(x)$ as :
If $x\ge 12$ :
  If $x\in A$ and $b(x)\in B_1$ : $f(x)=g^{n(x)}(h(b(x)))$
  If $x\in A$ and $b(x)\in B_2$ : $f(x)=g^{1+n(x)}(h^{-1}(b(x))$
  If $x\in B_1$ : $f(x)=h(x)$
  If $x\in B_2$ : $f(x)=g(h^{-1}(x))$
If $12>x>9$ :
  $f(11)=11$
  $f(10)=9$
If $x\le 9$ :
  $f(x)=f(19-x)$

And it's easy to check that this function indeed fits.
\end{solution}



\begin{solution}[by \href{https://artofproblemsolving.com/community/user/177726}{manuel153}]
	Very nice solution!

I think that also the starting points at $11$ and $9$ are not uniquely determined, and could
for instance be chosen as $f(9)=f(10)=11$ and $f(8)=f(11)=9$.
\end{solution}
*******************************************************************************
-------------------------------------------------------------------------------

\begin{problem}[Posted by \href{https://artofproblemsolving.com/community/user/68025}{Pirkuliyev Rovsen}]
	Consider all functions $f: \mathbb{N}\to\mathbb{N}$ satisfying $f(xf(y))=yf(x)$ for any $x,y{\in}N$.Find the least possible value of $f(2007)$.


________________________________________
Azerbaijan Land of the Fire 
	\flushright \href{https://artofproblemsolving.com/community/c6h531213}{(Link to AoPS)}
\end{problem}



\begin{solution}[by \href{https://artofproblemsolving.com/community/user/29428}{pco}]
	\begin{tcolorbox}Consider all functions $f: \mathbb{N}\to\mathbb{N}$ satisfying $f(xf(y))=yf(x)$ for any $x,y{\in}N$.Find the least possible value of $f(2007)$.\end{tcolorbox}
Let $P(x,y)$ be the assertion $f(xf(y))=yf(x)$

If $f(a)=f(b)$, then comparaison of $P(1,a)$ and $P(1,b)$ implies $a=b$ and so $f(x)$ is injective
$P(1,1)$ $\implies$ $f(f(1))=f(1)$ and so, since injective, $f(1)=1$

$P(1,x)$ $\implies$ $f(f(x))=x$
$P(1,f(y))$ $\implies$ $f(xy)=f(x)f(y)$

So $f(p)$ is prime for any prime and the solutions are :

Let $A,B,C$ a split of prime numbers where $B,C$ are equinumerous.
Let $h(x)$ any bijection from $B\to C$;
$f(x)$ is defined as :
$f(1)=1$
$\forall p\in A$ : $f(p)=p$
$\forall p\in B$ : $f(p)=h(p)$
$\forall p\in C$ : $f(p)=h^{-1}(p)$
$f(\prod p_i^{n_i})=\prod f(p_i)^{n_i}$

So $f(2007)=f(3)^2f(223)$ and the smallest value is when $f(3)=3$ and $f(223)=2$ and so is $\boxed{18}$
(the other case to look at is $f(3)=2$ and $f(223)=5$ which gives $20$)
\end{solution}



\begin{solution}[by \href{https://artofproblemsolving.com/community/user/177508}{mathuz}]
	it's old problem and  nice.
 First, $ f $ is injective.  From $y=1,$ we have $ f(1)=1.$
Then  \[ f(f(y))=y, \]  so  $f$ is surjective. Hence, $f$ - bijective.
From,  $ f(f(y))=y $  then  \[ f(xf(y))=f(f(y))f(x) \]  and  \[ f(zt)=f(z)f(t) \]  for any positive integers $z,t.$ 
Because, $f$ - bijective.
So,  $f$ is multiplicative. Hence,   $f$  from $P$ to $P$,
$P$ is set prime numbers.
$ 2007=3^2 *223, $ minimum $ f(2007)=3^3 *2=18. $
\end{solution}



\begin{solution}[by \href{https://artofproblemsolving.com/community/user/177726}{manuel153}]
	\begin{tcolorbox}it's old problem and  nice.\end{tcolorbox}
So how old is it, and what's its origin?

The problem is very similar in spirit to IMO problem 1998-6:
      http: \/\/www.artofproblemsolving.com/Forum/viewtopic.php?p=124426
\end{solution}
*******************************************************************************
-------------------------------------------------------------------------------

\begin{problem}[Posted by \href{https://artofproblemsolving.com/community/user/68025}{Pirkuliyev Rovsen}]
	A real number $\alpha$ is given.Find all functions $f: \mathbb{R_+}\to\mathbb{R_+}$ satisfying ${\alpha}x^2f(\frac{1}{x})+f(x)=\frac{x}{x+1}$ for all $x>0$.

____________________________________________
Azerbaijan Land of the Fire 
	\flushright \href{https://artofproblemsolving.com/community/c6h531223}{(Link to AoPS)}
\end{problem}



\begin{solution}[by \href{https://artofproblemsolving.com/community/user/29428}{pco}]
	\begin{tcolorbox}A real number $\alpha$ is given.Find all functions $f: \mathbb{R_+}\to\mathbb{R_+}$ satisfying ${\alpha}x^2f(\frac{1}{x})+f(x)=\frac{x}{x+1}$ for all $x>0$.\end{tcolorbox}
Let $P(x)$ be the assertion $\alpha x^2f(\frac 1x)+f(x)=\frac x{x+1}$

$P(\frac 1x)$ $\implies$ $x^2f(\frac 1x)+\alpha f(x)=\frac {x^2}{x+1}$
Multiplying by $\alpha$ and subtracting from $P(x)$, we get :

$(\alpha^2-1) f(x)=\frac {\alpha x^2-x}{x+1}$

If $\alpha=\pm 1$ : no solution

If $\alpha\ne \pm 1$ : $\boxed{f(x)=\frac {x(\alpha x-1)}{(\alpha^2-1) (x+1)}\text{  which indeed is a solution whenever }\alpha\in(-1,0]}$
\end{solution}
*******************************************************************************
-------------------------------------------------------------------------------

\begin{problem}[Posted by \href{https://artofproblemsolving.com/community/user/68025}{Pirkuliyev Rovsen}]
	Find all function $f: \mathbb{R}\to\mathbb{R}$ such that $a)(f(x))^{1992}=x^{1992}$, $b)I{\subset}R,I$ interval ${\Rightarrow}f(I)$ interval.

_______________________________________
Azerbaijan Land of the Fire 
	\flushright \href{https://artofproblemsolving.com/community/c6h531246}{(Link to AoPS)}
\end{problem}



\begin{solution}[by \href{https://artofproblemsolving.com/community/user/177508}{mathuz}]
	i don't understand to $b).$
What is it? 
$I$ interval $ \rightarrow $  $f(I)$ interval.
This is mean  $I$ - exist? 
Or,  $I$ is arbitrary interval?
\end{solution}



\begin{solution}[by \href{https://artofproblemsolving.com/community/user/29428}{pco}]
	\begin{tcolorbox}Find all function $f: \mathbb{R}\to\mathbb{R}$ such that $a)(f(x))^{1992}=x^{1992}$, $b)I{\subset}R,I$ interval ${\Rightarrow}f(I)$ interval.\end{tcolorbox}
So $f(x)=\pm x$ and $f(x)$ is continuous. Hence the four solutions :
$f(x)=x$ $\forall x$
$f(x)=-x$ $\forall x$
$f(x)=|x|$ $\forall x$
$f(x)=-|x|$ $\forall x$
\end{solution}
*******************************************************************************
-------------------------------------------------------------------------------

\begin{problem}[Posted by \href{https://artofproblemsolving.com/community/user/68025}{Pirkuliyev Rovsen}]
	Determine all function $f: \mathbb{R}\to\mathbb{R}$ for which $f((x-y)^3)=f^3(x)-3f^2(x)y+3f(x)y^2-y^3$ for all $x,y{\in}R$. Answer :$f(x)=x?$


____________________________________
Azerbaijan Land of the Fire 
	\flushright \href{https://artofproblemsolving.com/community/c6h531680}{(Link to AoPS)}
\end{problem}



\begin{solution}[by \href{https://artofproblemsolving.com/community/user/29428}{pco}]
	\begin{tcolorbox}Determine all function $f: \mathbb{R}\to\mathbb{R}$ for which $f((x-y)^3)=f^3(x)-3f^2(x)y+3f(x)y^2-y^3$ for all $x,y{\in}R$. Answer :$f(x)=x?$\end{tcolorbox}
I suppose that $f^n(x)$ here means $\left(f(x)\right)^n$
Let $P(x,y)$ be the assertion $f((x-y)^3)=f(x)^3-3f(x)^2y+3f(x)y^2-y^3$

$P(x,x)$ $\implies$ $f(x)=x+\sqrt[3]{f(0)}$
$P(0,-\sqrt[3]x)$ $\implies$ $f(x)=(\sqrt[3]x+f(0))^3$ 
and so $f(0)=0$ and $\boxed{f(x)=x}$ which indeed is a solution;
\end{solution}



\begin{solution}[by \href{https://artofproblemsolving.com/community/user/68025}{Pirkuliyev Rovsen}]
	Patrick you're a genius.Thanks you  
\end{solution}
*******************************************************************************
-------------------------------------------------------------------------------

\begin{problem}[Posted by \href{https://artofproblemsolving.com/community/user/68025}{Pirkuliyev Rovsen}]
	Determine all function $f: \mathbb{R}\to\mathbb{R}$ for which $f(x+y){\cdot}(f(x^2)-f(xy)+f^2(y))=x^3-f^3(y)$ for all $x,y{\in}R$ .


____________________________________
Azerbaijan Land of the Fire 
	\flushright \href{https://artofproblemsolving.com/community/c6h531681}{(Link to AoPS)}
\end{problem}



\begin{solution}[by \href{https://artofproblemsolving.com/community/user/29428}{pco}]
	\begin{tcolorbox}Determine all function $f: \mathbb{R}\to\mathbb{R}$ for which $f(x+y){\cdot}(f(x^2)-f(xy)+f^2(y))=x^3-f^3(y)$ for all $x,y{\in}R$ \end{tcolorbox}
I suppose that $f^n(x)$ here means $\left(f(x)\right)^n$
Let $P(x,y)$ be the assertion $f(x+y)(f(x^2)-f(xy)+f(y)^2)=x^3-f(y)^3$

$P(0,0)$ $\implies$ $f(0)=0$
$P(-x,x)$ $\implies$ $f(x)=-x$ which unfortunately is not a solution.

So no solution for this functional equation.
\end{solution}
*******************************************************************************
-------------------------------------------------------------------------------

\begin{problem}[Posted by \href{https://artofproblemsolving.com/community/user/68025}{Pirkuliyev Rovsen}]
	Determine all continuous functions $f: \mathbb{R_+}\to\mathbb{R}$ for which ${{{{f(x+\frac{1}{y^2}})+f(y+\frac{1}{x^2}})=f(x+\frac{1}{x^2}})+f(y+\frac{1}{y^2}})$  for which $x,y{\in}R_+$


____________________________________
Azerbaijan Land of the Fire 
	\flushright \href{https://artofproblemsolving.com/community/c6h531704}{(Link to AoPS)}
\end{problem}



\begin{solution}[by \href{https://artofproblemsolving.com/community/user/29428}{pco}]
	\begin{tcolorbox}Determine all continuous functions $f: \mathbb{R_+}\to\mathbb{R}$ for which ${{{{f(x+\frac{1}{y^2}})+f(y+\frac{1}{x^2}})=f(x+\frac{1}{x^2}})+f(y+\frac{1}{y^2}})$  for which $x,y{\in}R_+$\end{tcolorbox}
Let $P(x,y)$ be the assertion $f(x+\frac 1{y^2})+f(y+\frac 1{x^2})=f(x+\frac 1{x^2})+f(y+\frac 1{y^2})$

Let $2<a<b$
Let $u(x)=\frac 1{\sqrt{a-x}}$ and $v(x)=b-\frac 1{x^2}$ two continuous functions defined over $(0,a)$

$u(x)$ is convex while $v(x)$ is concave
when $x=\frac 1{\sqrt b}$ : $u(x)>v(x)$
When $x=1>\frac 1{\sqrt b}$ : $u(1)=\frac 1{\sqrt{a-1}}<1<b-1=v(1)$ 
So the equation $u(x)=v(x)$ has exactly one root $r$ in $(\frac 1{\sqrt b},1)$ and a second one in $(1,a)$
Let $s=u(r)=v(r)$
$s=u(r)$ $\implies$ $s=\frac 1{\sqrt{a-r}}$ $\implies$ $r+\frac 1{s^2}=a$
$s=v(r)$ $\implies$ $s=b-\frac 1{r^2}$ $\implies$ $s+\frac 1{r^2}=b$

Since $a<b$, we get $r^3-ar^2+1>r^3-br^2+1$ and then :
If $r^3-br^2+1<0$, then $(b-r)r^2>1$ and so $s=v(r)=b-\frac 1{r^2}>r$
If $r^3-br^2+1\ge 0$, then $r^3-ar^2+1>0$ and so $1>r^2(a-r)>0$ so $s=u(r)=\frac 1{\sqrt{a-r}}>r$
So $s>r$

Let $a'=r+\frac 1{r^2}$ and $b'=s+\frac 1{s^2}$
$P(r,s)$ $\implies$ $f(a)+f(b)=f(a')+f(b')$
Since $r<s$, we get $s+\frac 1{r^2}>r+\frac 1{r^2}>r+\frac 1{s^2}$ and so $b>a'>a$ and since $a'+b'=a+b$, we also got $b>b'>a$
Let then $a_1=\min(a',b')$ and $b_1=\max(a',b')$ :

If $a_1=b_1$, we have $a_1=b_1=\frac{a+b}2$ and we got $f(a)+f(b)=2f(\frac{a+b}2)$
If $a_1<b_1$, we started from $2<a< b$ and we got $2<a<a_1< b_1<b$ with $a+b=a_1+b_1$ and $f(a)+f(b)=f(a_1)+f(b_1)$ and we can continue the same process, starting from $(a_1,b_1)$ and getting $(a_2,b_2)$

As a consequence :
either we have $f(a)+f(b)=2f(\frac{a+b}2)$, either we built an increasing sequence $a_n$ and a decreasing sequence $b_n$ such that $a_n<b_n$, $a+b=a_n+b_n$ and $f(a)+f(b)=f(a_n)+f(b_n)$

These two sequences both have a limit $l_a\le l_b$.
If $l_a<l_b$, continuity implies that the iterative process described above transforms $(l_a,l_b)\to(l_a,l_b)$ and so :
either $r+\frac 1{s^2}=r+\frac 1{r^2}$, either $r+\frac 1{s^2}=s+\frac 1{s^2}$, and so $r=s$, impossible since $l_a<l_b$

So $l_a=l_b=\frac{a+b}2$

And so $f(x)+f(y)=2f(\frac{x+y}2)$ $\forall x,y>2$
This is a classical equation which easily implies (since continuous) $f(x)=cx+d$ $\forall x>2$ and so (continuity) $\forall x\ge 2$

let now $\frac 32\sqrt[3]2>t>\sqrt[3]2$ such that $t+\frac 1{t^2}=2$

Choosing $x\ge t$ and $y$ great enough, we get $y+\frac 1{x^2}, x+\frac 1{x^2}, y+\frac 1{y^2}\ge 2$ and so $P(x,y)$ implies $f(x+\frac 1{y^2})=c(x+\frac 1{y^2})+d$
Setting $y\to+\infty$ and using continuity, we get $f(x)=cx+d$ $\forall x\ge t$

let then any $x>0$ : $x+\frac 1{x^2}\ge \frac 32\sqrt[3]2>t$ and so, choosing $y$ great enough, we get $y+\frac 1{x^2}, x+\frac 1{x^2}, y+\frac 1{y^2}> t$ and so $P(x,y)$ implies $f(x+\frac 1{y^2})=c(x+\frac 1{y^2})+d$
Setting $y\to+\infty$ and using continuity, we get $\boxed{f(x)=cx+d}$ $\forall x>0$ which indeed is a solution.
\end{solution}
*******************************************************************************
-------------------------------------------------------------------------------

\begin{problem}[Posted by \href{https://artofproblemsolving.com/community/user/68025}{Pirkuliyev Rovsen}]
	Determine all continuous function ${{f: \mathbb(0;+\infty)}\to\mathbb(0;+\infty)}$ satisfying $f(x^3)+f(y^3)+f(z^3)=f(xyz)f(\frac{x}{y})f(\frac{y}{z})f(\frac{z}{x})$.


____________________________________
Azerbaijan Land of the Fire 
	\flushright \href{https://artofproblemsolving.com/community/c6h531784}{(Link to AoPS)}
\end{problem}



\begin{solution}[by \href{https://artofproblemsolving.com/community/user/29428}{pco}]
	\begin{tcolorbox}Determine all continuous function ${{f: \mathbb(0;+\infty)}\to\mathbb(0;+\infty)}$ satisfying $f(x^3)+f(y^3)+f(z^3)=f(xyz)f(\frac{x}{y})f(\frac{y}{z})f(\frac{z}{x})$.\end{tcolorbox}
Let $P(x,y,z)$ be the assertion $f(x^3)+f(y^3)+f(z^3)=f(xyz)f(\frac xy)f(\frac yz)f(\frac zx)$

$P(1,1)$ $\implies$ $f(1)=\sqrt[3]3$

$P(xy,y,1)$ $\implies$ $f(x^3y^3)+f(y^3)+f(1)=f(xy^2)f(x)f(y)f(\frac 1{xy})$
$P(y,xy,1)$ $\implies$ $f(y^3)+f(x^3y^3)+f(1)=f(xy^2)f(\frac 1x)f(xy)f(\frac 1y)$
and so $f(x)f(y)f(\frac 1{xy})=f(\frac 1x)f(xy)f(\frac 1y)$

Let $g(x)=\frac{f(\frac 1x)}{f(x)}$ : $g(x)$ is positive, continuous and previous equation becomes $g(xy)=g(x)g(y)$ and so $g(x)=x^a$ for some $a\in\mathbb R$

So $f(\frac 1x)=f(x)x^a$

$P(x,1,1)$ $\implies$ $f(x^3)+2\sqrt[3]3=f(x)^2f(\frac 1x)\sqrt[3]3$ $=f(x)^3x^a\sqrt[3]3$
So $P(\frac 1x,1,1)$ $\implies$ $f((\frac 1x)^3)+2\sqrt[3]3=f(\frac 1x)^3x^{-a}\sqrt[3]3$ and so $f(x^3)x^{3a}+2\sqrt[3]3=f(x)^3x^{2a}\sqrt[3]3$

Eliminating $f(x)^3$ between the two equations, we get $f(x^3)=\frac{2\sqrt[3]3}{x^a(x^a+1)}$ and so $f(x)=\frac{2\sqrt[3]3}{x^{\frac a3}(x^{\frac a3}+1)}$

Plugging this back in for example $P(x,1,1)$, we easily get (for example setting $x\to +\infty$) $a=0$

Hence the solution $\boxed{f(x)=\sqrt[3]3}$ $\forall x$ which indeed is a solution.
\end{solution}
*******************************************************************************
-------------------------------------------------------------------------------

\begin{problem}[Posted by \href{https://artofproblemsolving.com/community/user/153261}{hoangpham}]
	Find all function $f:\mathbb{R}\rightarrow \mathbb{R}$ such that :
\[f\left ( x^2-y \right )=xf(x)-f(y), \forall x,y \in \mathbb{R}\]
	\flushright \href{https://artofproblemsolving.com/community/c6h531794}{(Link to AoPS)}
\end{problem}



\begin{solution}[by \href{https://artofproblemsolving.com/community/user/29428}{pco}]
	\begin{tcolorbox}Find all function $f:\mathbb{R}\rightarrow \mathbb{R}$ such that :
\[f\left ( x^2-y \right )=xf(x)-f(y), \forall x,y \in \mathbb{R}\]\end{tcolorbox}
Let $P(x,y)$ be the assertion $f(x^2-y)=xf(x)-f(y)$

$P(0,0)$ $\implies$ $f(0)=0$
$P(0,x)$ $\implies$ $f(-x)=-f(x)$
$P(x,0)$ $\implies$ $f(x^2)=xf(x)$

So $f(x^2-y)=f(x^2)+f(-y)$ and so $f(x+y)=f(x)+f(y)$ $\forall x\ge 0,\forall y$ and, since odd, $f(x+y)=f(x)+f(y)$ $\forall x,y$

$P(x+1,0)$ $\implies$ $f(x^2+x+x+1)=(x+1)f(x+1)$ and so $f(x^2)+2f(x)+f(1)=xf(x)+xf(1)+f(x)+f(1)$ and so $f(x)=xf(1)$

And so $\boxed{f(x)=cx}$ $\forall x$ which indeed is a solution, whatever is $c\in\mathbb R$
\end{solution}



\begin{solution}[by \href{https://artofproblemsolving.com/community/user/141397}{subham1729}]
	Let $P(x,y):f(x^2-y)=xf(x)-f(y)$.Now $P(0,y)\implies f(-x)=-f(x)\implies f(0)=0$.Also $P(x,0)\implies f(x^2)=xf(x)$.And now so $f(x+y)=f(x)+f(y)$.Also from $f((x+y)^2)$ we get $f(2xy)=xf(y)+yf(x)=2f(xy)$ and so $2f(xy)+2f(x)=xf(y)+xf(1)+yf(x)+f(x)\implies f(x)=xf(1)$.
\end{solution}
*******************************************************************************
-------------------------------------------------------------------------------

\begin{problem}[Posted by \href{https://artofproblemsolving.com/community/user/75382}{djb86}]
	Find all functions $f : \mathbb{R} \to \mathbb{R}$ that satisfy the condition
\[f(f(x) + y) = 2x + f(f(y) - x)\quad \text{ for all } x, y \in\mathbb{R}.\]
	\flushright \href{https://artofproblemsolving.com/community/c6h531830}{(Link to AoPS)}
\end{problem}



\begin{solution}[by \href{https://artofproblemsolving.com/community/user/29428}{pco}]
	\begin{tcolorbox}...
Write $x+f(0)=A$ and $f(0)-x=B$ and we get : $f(A) - f(B) = A - B$ for all $A, B \in \mathbb{R}$.
...\end{tcolorbox}
Unfortunately, this is wrong : you got $f(A)-f(B)=A-B$ only for those $A,B$ such that $A+B=2f(0)$, and not $\forall A,B$
\end{solution}



\begin{solution}[by \href{https://artofproblemsolving.com/community/user/109781}{math_mat}]
	\begin{tcolorbox}[quote="math_mat"]...
Write $x+f(0)=A$ and $f(0)-x=B$ and we get : $f(A) - f(B) = A - B$ for all $A, B \in \mathbb{R}$.
...\end{tcolorbox}
Unfortunately, this is wrong : you got $f(A)-f(B)=A-B$ only for those $A,B$ such that $A+B=2f(0)$, and not $\forall A,B$\end{tcolorbox}

Yes, I also saw my mistake and I was trying to fix it, without success apparently. I deleted the previous post.
\end{solution}



\begin{solution}[by \href{https://artofproblemsolving.com/community/user/177508}{mathuz}]
	it is problem A:1 IMO shortlist-2002. You can see IMO page.
\end{solution}



\begin{solution}[by \href{https://artofproblemsolving.com/community/user/29428}{pco}]
	\begin{tcolorbox}Find all functions $f : \mathbb{R} \to \mathbb{R}$ that satisfy the condition
\[f(f(x) + y) = 2x + f(f(y) - x)\quad \text{ for all } x, y \in\mathbb{R}.\]\end{tcolorbox}
Let $P(x,y)$ be the assertion $f(f(x)+y)=2x+f(f(y)-x)$
Let $a=f(0)$

$P(\frac{a-x}2,-f(\frac{a-x}2))$ $\implies$ $x=f(f(-f(\frac{a-x}2))-\frac{a-x}2)$ and so $f(x)$ is surjective.

$P(x,y)$ $\implies$ $f(f(x)+y)=2x+f(f(y)-x)$
$P(y,-x)$ $\implies$ $f(f(y)-x)=2y+f(f(-x)-y)$
And so, adding these two lines : $f(f(x)+y)=2x+2y+f(f(-x)-y)$
setting $y=\frac{f(-x)-f(x)}2$ in this last equation, we get new assertion $Q(x)$ : $f(-x)=f(x)-2x$ $\forall x$

$P(x+a,0)$ $\implies$ $f(f(x+a))=2x+2a+f(-x)$
$P(0,x+a)$ $\implies$ $f(x+2a)=f(f(x+a))$
$Q(x)$ $\implies$ $f(-x)=f(x)-2x$
Adding these three lines, we get $f(x+2a)=f(x)+2a$ 

$P(0,x)$ $\implies$ $f(f(x))=f(x+a)$ and so $f(f(f(x)))=f(f(x+a))=f(x+2a)=f(x)+2a$
$P(0,f(x))$ $\implies$ $f(f(f(x)))=f(f(x)+a)$
So $f(f(x)+a)=f(x)+2a$

And surjectivity implies then $\boxed{f(x)=x+a}$ $\forall x$ which indeed is a solution
\end{solution}



\begin{solution}[by \href{https://artofproblemsolving.com/community/user/177508}{mathuz}]
	Let $P(x,y)$ be the assertion $ f(f(x)+y)=2x+f(f(y)-x). $ 
$P(x,f(x))$ $ \Rightarrow $ $f(0)-2x=f(f(-f(x))-x) $ and so $f(x)$ is surjective. 
Let  for some $c$, such that $f(c)=0$.
From $x=c$  we have $ f(y)=2c+f(f(y)-c) $ $ \Rightarrow $ 
$f(f(y)-c)=(f(y)-c)-c$.  
Hence,  $f$ is surjective then $f(x)=x+a$  for all real $x$.
\end{solution}



\begin{solution}[by \href{https://artofproblemsolving.com/community/user/141397}{subham1729}]
	Let $P(x,y)$ be that assertion. Now $P(x,-f(x))$ implies $f$ is onto so there exist $a$ such that $f(a)=0$ and now $P(a,x)$ implies $f(f(x)-a)=f(x)-2a$ also as $f$ is onto so we get $f(x)=x+a$.
\end{solution}
*******************************************************************************
-------------------------------------------------------------------------------

\begin{problem}[Posted by \href{https://artofproblemsolving.com/community/user/68025}{Pirkuliyev Rovsen}]
	Find all $f: \mathbb{R}\to\mathbb{R}$ such that 
$(f(x)+f(y))(f^2(y)-f(y)f(z)+f^2(z))=xf(x^2+z^2)-(y-z)f(yz)+y^3-f(x)f(y)f(z)$ for all   $x,y,z{\in}R$.

_____________________________________
Azerbaijan Land of the Fire 
	\flushright \href{https://artofproblemsolving.com/community/c6h531982}{(Link to AoPS)}
\end{problem}



\begin{solution}[by \href{https://artofproblemsolving.com/community/user/84959}{TheBottle}]
	First let's set $y=z$, so we obtain $f^2 (y)( f(x)+f(y)=xf(x^2+y^2)+y^3-f^2(y)f(x)$. Now let $x=0$ and $f(0)=a$ and the equation becomes $f^3(y)+2af^2(y)=y^3$. Setting $y=0$ we get $a=0$, so we finally obtain $f^3(y)=y^3$. Which hopefully has the only solution $f(x)=x$. I'm writing in a hurry from a mobile, so there could be a plenty of mistakes.
\end{solution}



\begin{solution}[by \href{https://artofproblemsolving.com/community/user/29428}{pco}]
	\begin{tcolorbox}Find all $f: \mathbb{R}\to\mathbb{R}$ such that 
$(f(x)+f(y))(f^2(y)-f(y)f(z)+f^2(z))=xf(x^2+z^2)-(y-z)f(yz)+y^3-f(x)f(y)f(z)$ for all   $x,y,z{\in}R$.\end{tcolorbox}
\begin{tcolorbox}First let's set $y=z$, so we obtain $f^2 (y)( f(x)+f(y)=xf(x^2+y^2)+y^3-f^2(y)f(x)$. Now let $x=0$ and $f(0)=a$ and the equation becomes $f^3(y)+2af^2(y)=y^3$. Setting $y=0$ we get $a=0$, so we finally obtain $f^3(y)=y^3$. Which hopefully has the only solution $f(x)=x$. I'm writing in a hurry from a mobile, so there could be a plenty of mistakes.\end{tcolorbox}
The steps are OK but you need, at the end, to check that the mandatory form you got ($f(x)=x$) indeed is a solution.

And, here, unfortunately, it is not.

So no solution for this functional equation.
\end{solution}
*******************************************************************************
-------------------------------------------------------------------------------

\begin{problem}[Posted by \href{https://artofproblemsolving.com/community/user/75382}{djb86}]
	Find all functions $f: \mathbb{R} \to \mathbb{R}$ such that
\[f(x + y) + y \le f(f(f(x)))\]
holds for all $x, y \in \mathbb{R}$.
	\flushright \href{https://artofproblemsolving.com/community/c6h532050}{(Link to AoPS)}
\end{problem}



\begin{solution}[by \href{https://artofproblemsolving.com/community/user/29428}{pco}]
	\begin{tcolorbox}Find all functions $f: \mathbb{R} \to \mathbb{R}$ such that
\[f(x + y) + y \le f(f(f(x)))\]
holds for all $x, y \in \mathbb{R}$.\end{tcolorbox}
Let $P(x,y)$ be the assertion $f(x+y)+y\le f(f(f(x)))$

$P(x,f(f(x))-x)$ $\implies$ $f(f(x))\le x$ and so $P(x,y)$ implies $Q(x,y)$ : $f(x+y)+y\le f(x)$

$Q(x,y-x)$ $\implies$ $f(y)+y\le f(x)+x$ and so $\boxed{f(x)=a-x}$ which indeed is a solution whatever is $a\in\mathbb R$
\end{solution}



\begin{solution}[by \href{https://artofproblemsolving.com/community/user/149023}{Nguyenhuyhoang}]
	\begin{tcolorbox}$Q(x,y-x)$ $\implies$ $f(y)+y\le f(x)+x$ and so $\boxed{f(x)=a-x}$ which indeed is a solution whatever is $a\in\mathbb R$\end{tcolorbox}
That solution is very elegant, but can you explain this line? How can you conclude that fact?
\end{solution}



\begin{solution}[by \href{https://artofproblemsolving.com/community/user/29428}{pco}]
	\begin{tcolorbox}[quote="pco"]$Q(x,y-x)$ $\implies$ $f(y)+y\le f(x)+x$ and so $\boxed{f(x)=a-x}$ which indeed is a solution whatever is $a\in\mathbb R$\end{tcolorbox}
That solution is very elegant, but can you explain this line? How can you conclude that fact?\end{tcolorbox}
$f(y)+y\le f(x)+x$
Swapping $x,y$, we get $f(x)+x\le f(y)+y$

And so $f(x)+x=f(y)+y$ constant $=a$ for some real $a$.

And so $f(x)=a-x$
\end{solution}



\begin{solution}[by \href{https://artofproblemsolving.com/community/user/177508}{mathuz}]
	\begin{tcolorbox}Find all functions $f: \mathbb{R} \to \mathbb{R}$ such that
\[f(x + y) + y \le f(f(f(x)))\]
holds for all $x, y \in \mathbb{R}$.\end{tcolorbox}  
(1) $f(f(f(x)))=f(x)$:  
  because, $P(x,0)$ $ \Rightarrow $ $f(x)\le f(f(f(x)))$ and 
$P(f(x), f(f(f(x)))-f(x))$ $ \Rightarrow $ $ f(f(f(x)))\le f(x) $. 
hence,  $f(x)\le f(f(f(x)))\le f(x) $ and \[ f(f(f(x)))=f(x).  \]  
(2) $f(x)=c-x$: ($c=f(0)$)
   because,  from (1)  we have $f(x+y)+y\le f(x) $. 
$P(0,x)$ $ \Rightarrow $  $ f(x)+x\le f(0)$  and  $P(x, -x)$ $ \Rightarrow $ $ f(0)\le f(x)+x $. 
hence, $ f(x)+x=f(0) $  and  \[ f(x)=c-x  .\] 
Answer:\end{underlined} $ f(x)=c-x $.
\end{solution}



\begin{solution}[by \href{https://artofproblemsolving.com/community/user/213306}{saturzo}]
	Easy and nice.   

[hide="My Solution (almost same as others):"]$P(x, f(x)-x) \Rightarrow f(f(x)) \leq x$
and $P(x, 0) \Rightarrow f(x) \leq f(f(f(x)))$
These imply $f(f(f(x))) = f(x)$
So the main equation becomes $f(x+y) + y \leq f(x)$
Now $P(x+y, -y) \Rightarrow f(x+y) + y \geq f(x)$
So, $f(x+y) + y = f(x)$ and putting $x=0$ implies $f(y)= \alpha - y , \forall y \in \mathbb{R} [$ where $\alpha = f(0)$ is a constant $]$ - which is indeed the solution[\/hide]
\end{solution}



\begin{solution}[by \href{https://artofproblemsolving.com/community/user/225003}{Jeje}]
	I am sorry Bumping old post 

f(f(x)) =< x imply that f(f(f(x))) =< f(x)

This mean it is increasing function isn't it????

Can anyone explain me why it was increasing function???
\end{solution}



\begin{solution}[by \href{https://artofproblemsolving.com/community/user/35881}{Ronald Widjojo}]
	No, in their proof, it is stated that $f(f(x)) \le x$ is true for every $x \in \mathbb{R}$, so we can change $x$ by $f(x)$.
\end{solution}



\begin{solution}[by \href{https://artofproblemsolving.com/community/user/225003}{Jeje}]
	Oh yeahhhh i am sorry too stupid argumentt
\end{solution}



\begin{solution}[by \href{https://artofproblemsolving.com/community/user/159507}{MSTang}]
	[s]Take $y=0$ to get $f(x) \le f(f(f(x)))$ for all $x$.[\/s] Take $y = f(f(x)) - x$ to get $f(f(f(x))) + f(f(x)) - x \le f(f(f(x)))$, or $f(f(x)) \le x$ for all $x$. Replacing $x$ with $f(x)$ in this inequality, we get $f(f(f(x))) \le f(x)$ for all $x$. [s]Thus, $f(x) = f(f(f(x)))$ for all $x$.[\/s] Then the given inequality becomes $f(x+y) + y \le f(x)$ for all $x$, $y$.

Take $x=0$ to get $f(y) + y \le f(0)$ for all $y$, and take $y=-x$ to get $f(0) - x \le f(x) \implies f(0) \le f(x) + x$ for all $x$. Thus, $f(0) = f(x) + x$ for all $x$, so $f(x) = a-x$ where $a=f(0)$ is a constant. It is easy to check that all functions of this form satisfy the original functional inequality, so the solutions are $f(x) = a-x$ for any constant $a$.

(Strikethrough edits made after reading first solution in this thread :blush:)
\end{solution}



\begin{solution}[by \href{https://artofproblemsolving.com/community/user/146669}{trumpeter}]
	[hide=Solution]
The answer is $f\left(x\right)=c-x$ for some real constant $c$.

Letting $y=f\left(f\left(x\right)\right)-x$, we get that \[f\left(f\left(f\left(x\right)\right)\right)+f\left(f\left(x\right)\right)-x\leq f\left(f\left(f\left(x\right)\right)\right),\] so $f\left(f\left(x\right)\right)\leq x$, so $f\left(f\left(f\left(x\right)\right)\right)\leq f\left(x\right)$, so \[f\left(x+y\right)+y\leq f\left(x\right).\] Applying this inequality with $x=a+b$, $y=-b$ gives that \[f\left(a\right)-b\leq f\left(a+b\right)\leq f\left(a\right)-b,\] so equality holds in all cases. Thus, \[f\left(x+y\right)+y=f\left(x\right).\] letting $f\left(0\right)=c$ and substitutin $x=0$, we get that the only solution is $f\left(x\right)=c-x$ for some real constant $c$. It is easy to see that this works.

Q.E.D.
[\/hide]
\end{solution}



\begin{solution}[by \href{https://artofproblemsolving.com/community/user/168680}{Tommy2000}]
	$(x, f(f(x))-x)$ yields $f(f(x)) \le x$. $(x, 0)$ yields $f(f(f(x))) \ge f(x)$. Combining the two gives $f(f(f(x))) = f(x)$. Thus, $f(x + y) + y \le f(x)$. Now, "swap" the roles of $x+y$ and $x$ to get $f(x) - y \le f(x + y)$. Thus, $f(x + y) = f(x) - y$. Letting $x = 0$, $f(y) = f(0) - y$. Indeed all such functions work. Yay. 
\end{solution}
*******************************************************************************
-------------------------------------------------------------------------------

\begin{problem}[Posted by \href{https://artofproblemsolving.com/community/user/75382}{djb86}]
	Find all functions $f : \mathbb{R} \to \mathbb{R}$ that satisfy
\[f(x^2 + y) + f(f(x) - y) = 2f(f(x)) + 2y^2\quad\text{ for all }x, y \in \mathbb{R}.\]
	\flushright \href{https://artofproblemsolving.com/community/c6h532134}{(Link to AoPS)}
\end{problem}



\begin{solution}[by \href{https://artofproblemsolving.com/community/user/29428}{pco}]
	\begin{tcolorbox}Find all functions $f : \mathbb{R} \to \mathbb{R}$ that satisfy
\[f(x^2 + y) + f(f(x) - y) = 2f(f(x)) + 2y^2\quad\text{ for all }x, y \in \mathbb{R}.\]\end{tcolorbox}
Let $P(x,y)$ be the assertion $f(x^2+y)+f(f(x)-y)=2f(f(x))+2y^2$

$P(x,0)$ $\implies$ $f(x^2)=f(f(x))$
$P(x,f(x)-x^2)$ $\implies$ $f(x^2)=f(f(x))+2(f(x)-x^2)^2$

Subtracting, we get $\boxed{f(x)=x^2}$ $\forall x$ which indeed is a solution.
\end{solution}



\begin{solution}[by \href{https://artofproblemsolving.com/community/user/177508}{mathuz}]
	Let $P(x,y)$ be the assertion 
\[ f(x^2+y)+f(f(x)-y)=2f(f(x))+2y^2 .\]
$P(x,-x^2)$ $ \Rightarrow $ $f(0)+f(f(x)+x^2)=2f(f(x))+x^4$ and  $P(x,f(x))$ $ \Rightarrow $  $ f(x^2+f(x))+f(0)=2f(f(x))+f(x)^2 $.
So,  we get that  $ f(x)^2=x^4 $,  $ \forall x .$
Answer: $f(x)=x^2$.  :D:
\end{solution}



\begin{solution}[by \href{https://artofproblemsolving.com/community/user/29428}{pco}]
	\begin{tcolorbox}Let $P(x,y)$ be the assertion 
\[ f(x^2+y)+f(f(x)-y)=2f(f(x))+2y^2 .\]
$P(x,-x^2)$ $ \Rightarrow $ $f(0)+f(f(x)+x^2)=2f(f(x))+x^4$ and  $P(x,f(x))$ $ \Rightarrow $  $ f(x^2+f(x))+f(0)=2f(f(x))+f(x)^2 $.
So,  we get that  $ f(x)^2=x^4 $,  $ \forall x .$
Answer: $f(x)=x^2$.  :D:\end{tcolorbox}

1) two typos errors : 
==============
$f(0)+f(f(x)+x^2)=2f(f(x))+2x^4$  and not $f(0)+f(f(x)+x^2)=2f(f(x))+x^4$ 
$ f(x^2+f(x))+f(0)=2f(f(x))+2f(x)^2 $ and not $ f(x^2+f(x))+f(0)=2f(f(x))+f(x)^2 $

2) a missing step :
============
You cant immediately conclude from $f(x)^2=x^4$ that $f(x)=x^2$. A lot of other functions are such that $f(x)^2=x^4$ :
$f(x)=-x^2$

$f(x)=x^2$ when $x\in\mathbb Q$ and $f(x)=-x^2$ when $x\notin\mathbb Q$

$f(x)=\lim_{t\to x^+}\frac{|t^3-t|}{t-1}$

...
\end{solution}
*******************************************************************************
-------------------------------------------------------------------------------

\begin{problem}[Posted by \href{https://artofproblemsolving.com/community/user/68025}{Pirkuliyev Rovsen}]
	Find all functions $f: \mathbb{R}\to\mathbb{R}$ such that $f(xyz+f(x)+f(y)+f(z))=f(x)f(y)f(z)+\frac{1}{3}(f(x)+f(y)+f(z))$ and $f(1)=1$ for all $x,y,z{\in}R$.

_____________________________________
Azerbaijan Land of the Fire 
	\flushright \href{https://artofproblemsolving.com/community/c6h532137}{(Link to AoPS)}
\end{problem}



\begin{solution}[by \href{https://artofproblemsolving.com/community/user/29428}{pco}]
	\begin{tcolorbox}Find all functions $f: \mathbb{R}\to\mathbb{R}$ such that $f(xyz+f(x)+f(y)+f(z))=f(x)f(y)f(z)+\frac{1}{3}(f(x)+f(y)+f(z))$ and $f(1)=1$ for all $x,y,z{\in}R$.\end{tcolorbox}
Let $P(x,y,z)$ be the assertion $f(xyz+f(x)+f(y)+f(z))=f(x)f(y)f(z)+\frac 13(f(x)+f(y)+f(z))$

$P(1,1,1)$ $\implies$ $f(4)=2$

$P(1,1,4)$ $\implies$ $f(8)=\frac{10}3$

$P(1,4,4)$ $\implies$ $f(21)=\frac{17}3$

$P(1,1,8)$ $\implies$ $f(\frac{40}3)=\frac{46}9$

$P(4,8,8)$ $\implies$ $f(\frac{794}3)=\frac{226}9$

From there, we get a contradiction :

$P(1,1,\frac{794}3)$ $\implies$ $f(\frac{2626}9)=\frac{922}{27}$

$P(1,21,\frac{40}3)$ $\implies$ $f(\frac{2626}9)=\frac{296}9$

And so no solution for this functional equation.
\end{solution}
*******************************************************************************
-------------------------------------------------------------------------------

\begin{problem}[Posted by \href{https://artofproblemsolving.com/community/user/68025}{Pirkuliyev Rovsen}]
	Find all continuous functions ${{f: \mathbb(0;+\infty)}\to\mathbb(0;+\infty)}$ such that $f(x)f(y)f(z)=f(xyz)+f(\frac{x}{y})+f(\frac{y}{z})+f(\frac{z}{x})$ for all $x,y,z>0$.


_____________________________________
Azerbaijan Land of the Fire 
	\flushright \href{https://artofproblemsolving.com/community/c6h532138}{(Link to AoPS)}
\end{problem}



\begin{solution}[by \href{https://artofproblemsolving.com/community/user/29428}{pco}]
	\begin{tcolorbox}Find all continuous functions ${{f: \mathbb(0;+\infty)}\to\mathbb(0;+\infty)}$ such that $f(x)f(y)f(z)=f(xyz)+f(\frac{x}{y})+f(\frac{y}{z})+f(\frac{z}{x})$ for all $x,y,z>0$.\end{tcolorbox}
Let $P(x,y,z)$ be the assertion $f(x)f(y)f(z)=f(xyz)+f(\frac xy)+f(\frac yz)+f(\frac zx)$

$P(1,1,1)$ $\implies$ $f(1)^3=4f(1)$ and so $f(1)=2$
$P(x,1,1)$ $\implies$ $2f(x)=f(\frac 1x)+2$
$P(\frac 1x,1,1)$ $\implies$ $2f(\frac 1x)=f(x)+2$
Eliminating $f(\frac 1x)$ between these two lines, we get $\boxed{f(x)=2}$ $\forall x$ which indeed is a solution.

And, btw, no need for continuity.
\end{solution}
*******************************************************************************
-------------------------------------------------------------------------------

\begin{problem}[Posted by \href{https://artofproblemsolving.com/community/user/169515}{abl}]
	problem find all continuous functions $f:\mathbb{R}\to\mathbb{R}$ such that
\[f(x^2)+f(y+(f(y))^2)=y+(f(x))^2+(f(y))^2,\forall x.y\in\mathbb{R}\]
I added  ''continuous functions ''
Could anybody solve the problem ?
	\flushright \href{https://artofproblemsolving.com/community/c6h532369}{(Link to AoPS)}
\end{problem}



\begin{solution}[by \href{https://artofproblemsolving.com/community/user/29428}{pco}]
	\begin{tcolorbox}problem find all continuous functions $f:\mathbb{R}\to\mathbb{R}$ such that
\[f(x^2)+f(y+(f(y))^2)=y+(f(x))^2+(f(y))^2,\forall x.y\in\mathbb{R}\]
I added  ''continuous functions ''
Could anybody solve the problem ?\end{tcolorbox}
I dont really understand why did you add "continuity". What exactly is the problem you got in your olympiad training or contest : with or without continuity ?

with continuity, the problem is far simpler :
Let $P(x,y)$ be the assertion $f(x^2)+f(y+f(y)^2)=y+f(x)^2+f(y)^2$
Let $a=f(0)$

$P(0,0)$ $\implies$ $a=2a^2-f(a^2)$
$P(x,0)$ $\implies$ $f(x^2)=f(x)^2+a^2-f(a^2)$
Subtracting, we get new assertion $Q(x)$ : $f(x^2)=f(x)^2+a-a^2$
$P(0,x)$ $\implies$ new assertion $R(x)$ : $f(g(x))=g(x)-a+a^2$ where $g(x)=x+f(x)^2$

$P(x)$ $\implies$ $f(-x)=\pm f(x)$
$g(x)$ is continuous and $g(0)=a^2$ and $\lim_{x\to+\infty}g(x)=+\infty$ and so $[a^2,+\infty)\in g(\mathbb R)$ and so $f(x)=x-a+a^2$ $\forall x\ge a^2$

Let $x\ge\max(1,a^2)$ so that $x^2\ge x\ge a^2$. $Q(x)$ becomes $(x^2-a+a^2)=(x-a+a^2)^2+a-a^2$ and so $a\in\{0,1\}$

Synthesis at this step :
$f(0)=a\in\{0,1\}$
$Q(x)$ : $f(x^2)=f(x)^2$ $\forall x$
$R(x)$ : $f(g(x))=g(x)$ where $g(x)=x+f(x)^2$ and so $f(x)=x$ $\forall x\ge a^2$

If $a=0$, we get $f(x)=x$ $\forall x\ge 0$ and, since $f(-x)=\pm f(x)$ and $f(x)$ is continuous :
either $f(x)=x$ $\forall x$ which indeed is a solution.
either $f(x)=|x|$ $\forall x$, which is not a solution (for example $P(0,-\frac 12)$ would be wrong).

If $a=1$ :
$P(0,0)$ $\implies$ $f(1)=1$ and $f(x)=x$ $\forall x\ge 1$
If $f(u)=0$ for some $u$, then $R(u)$ $\implies$ $u=0$, impossible, and so $f(x)>0$ $\forall x$
If $f(u)\ne 1$ for some $u\in(0,1)$, then repeated applications of $f(x^2)=f(x)^2$ and continuity imply $f(0)\ne 1$;
So $f(x)=1$ $\forall x\in[0,1]$
Since $f(-x)=\pm f(x)$ and $f(x)$ continuous we get $f(x)=\max(1,|x|)$ $\forall x$ which is not a solution (for example $P(0,-\frac 12)$ would be wrong).

Hence the only solution : $\boxed{f(x)=x}$ $\forall x$
\end{solution}



\begin{solution}[by \href{https://artofproblemsolving.com/community/user/169515}{abl}]
	\begin{tcolorbox}
I dont really understand why did you add "continuity". What exactly is the problem you got in your olympiad training or contest : with or without continuity ?
\end{tcolorbox}
Yes,I added,because I thinked that it was a unsolved problem some days ago.Now I believe  it can be solved.
And if without continuity,could you solve the problem ?
\end{solution}
*******************************************************************************
-------------------------------------------------------------------------------

\begin{problem}[Posted by \href{https://artofproblemsolving.com/community/user/173452}{Jayjayniboon}]
	1.find all functions$f(mn+3m+5n)=f(mn)+3f(m)+5f(n),f:\mathbf{Z}\rightarrow \mathbf{Z}$
2.find all functions $ f(f(m)+n)=m+f(n+2012),f:\mathbf{N}\rightarrow \mathbf{N}$
	\flushright \href{https://artofproblemsolving.com/community/c6h532620}{(Link to AoPS)}
\end{problem}



\begin{solution}[by \href{https://artofproblemsolving.com/community/user/29428}{pco}]
	\begin{tcolorbox}1.find all functions$f(mn+3m+5n)=f(mn)+3f(m)+5f(n),f:\mathbf{Z}\rightarrow \mathbf{Z}$\end{tcolorbox}
Let $P(x,y)$ be the assertion $f(xy+3x+5y)=f(xy)+3f(x)+5f(y)$
Let $a=f(1)$

$P(0,0)$ $\implies$ $f(0)=0$
$P(x,0)$ $\implies$ $f(3x)=3f(x)$
$P(0,x)$ $\implies$ $f(5x)=5f(x)$
$P(5x,3y)$ $\implies$ new assertion $Q(x,y)$ : $f(xy+x+y)=f(xy)+f(x)+f(y)$

$Q(x,-1)$ $\implies$ $f(-x)=-f(x)$
$Q(1,1)$ $\implies$ $f(3)=3a$
$Q(1,-2)$ $\implies$ $f(2)=2a$
$Q(-2,-2)$ $\implies$ $f(4)=4a$

$Q(n-1,1)$ $\implies$ $f(2n-1)=2f(n-1)+a$
$Q(n,-2)$ $\implies$ $f(2n)=f(n+2)+f(n)-2a$
Application of these two formulas starting with $n=3$ gives with very easy induction $f(n)=na$ $\forall n\ge 0$

Hence the solution : $\boxed{f(n)=an}$ $\forall n\in\mathbb Z$, whatever is $a\in\mathbb Z$, which indeed is a solution.
\end{solution}



\begin{solution}[by \href{https://artofproblemsolving.com/community/user/29428}{pco}]
	\begin{tcolorbox}2.find all functions $ f(f(m)+n)=m+f(n+2012),f:\mathbf{N}\rightarrow \mathbf{N}$\end{tcolorbox}
Let $P(x,y)$ be the assertion $f(f(x)+y)=x+f(y+2012)$

If $f(a)=2012$ for some $a\in\mathbb N$, then $P(a,1)$ $\implies$ $a=0$, impossible.
If $f(a)<2012$ for some $a\in\mathbb N$, then $P(a,x-f(a))$ $\implies$ $f(x)=a+f(x+2012-f(a))$ $\forall x\ge 2012$
From there, we easily get $f(x)=ka+f(x+k(2012-f(a)))$ $\forall x\ge 2012$ and for any non negative integer $k$, which is impossible
So $f(x)>2012$ $\forall x$

Then $P(y,x-2012)$ $\implies$ $f(x+(f(y)-2012))=f(x)+y$ $\forall x>2012$
So $f(x+k(f(y)-2012))=f(x)+ky$ $\forall x>2012$ and for any non negative integer $k$
Setting $k=f(z)-2012$, we get $f(x+(f(z)-2012)(f(y)-2012))=f(x)+(f(z)-2012)y$ $\forall x>2012$
Swapping $z,y$, we get $f(x+(f(y)-2012)(f(z)-2012))=f(x)+(f(y)-2012)z$ $\forall x>2012$
subtracting, we get $(f(z)-2012)y=(f(y)-2012)z$ $\forall y,z$

Setting $z=1$, this implies $f(y)=(f(1)-2012)y+2012$ and so $f(x)=ax+2012$ $\forall x$.

Plugging this back in original equation, we get $a=1$ and so the unique solution $\boxed{f(n)=n+2012}$
\end{solution}



\begin{solution}[by \href{https://artofproblemsolving.com/community/user/173452}{Jayjayniboon}]
	3.Find the function that$f:\mathbf{Q}\rightarrow \mathbf{Q},f(f(x)+yz)=x+f(y)f(z)$
\end{solution}



\begin{solution}[by \href{https://artofproblemsolving.com/community/user/89198}{chaotic_iak}]
	Let $P(x,y,z)$ be the preposition $f(f(x)+yz) = x + f(y)f(z)$.

$P(x,0,z) \implies f(f(x)) = x + f(0)f(z)$. So $f(0)f(x) = f(0)f(y)$ for all $x,y$, which means either $f(0) = 0$ or $f$ is constant. The latter is impossible, so $f(0) = 0$.
$P(x,0,z) \implies f(f(x)) = x$.
$P(0,y,1) \implies f(y) = f(y)f(1)$.
$P(f(x),y,1) \implies f(f(f(x))+y) = f(x) + f(y)f(1) \implies f(x+y) = f(x) + f(y)$. Since the domain is $\mathbb{Q}$, the answer is $f(x) = cx$ for all $x$, for some $c$.

$P(0,y,1) \implies f(y) = f(y)f(1) \implies cy = cy \cdot c$, so either $c = 0$ or $c = 1$. But the former has already been ruled impossible (it gives $f(x) = 0$ which is constant), so $f(x) = x$. This can be verified to satisfy $P$.
\end{solution}
*******************************************************************************
-------------------------------------------------------------------------------

\begin{problem}[Posted by \href{https://artofproblemsolving.com/community/user/163554}{ablbabybb}]
	problem:find all functions $f:\mathbb{R^+}\to\mathbb{R^+}$($\mathbb{R^+}=\{x\in\mathbb{R}:x>0\})$ such that
\[f(xf(x)+f(y))=(f(x))^2+y,\forall x,y\in\mathbb{R^+}\]
Could anybody solve the problem ?
	\flushright \href{https://artofproblemsolving.com/community/c6h532651}{(Link to AoPS)}
\end{problem}



\begin{solution}[by \href{https://artofproblemsolving.com/community/user/177508}{mathuz}]
	\begin{tcolorbox}problem:find all functions $f:\mathbb{R^+}\to\mathbb{R^+}$($\mathbb{R^+}=\{x\in\mathbb{R}:x>0\})$ such that
\[f(xf(x)+f(y))=(f(x))^2+y,\forall x,y\in\mathbb{R^+}\]
Could anybody solve the problem ?\end{tcolorbox}
it's old and easy.
  First,  $f$ - bijective.
It's not difficult.

So, \[ f(f(y))=y \] and \[ f(xf(x))=f(x)^2 \]
because,  $f(0)=0$.
Hence, it's solution $f(x)=x
$ and $f(x)=-x$.
\end{solution}



\begin{solution}[by \href{https://artofproblemsolving.com/community/user/29428}{pco}]
	\begin{tcolorbox}
  First,  $f$ - bijective.
It's not difficult.\end{tcolorbox}
Could you show me precisely surjectivity ?
I'm trying since few days (since first post)
Thanks a lot.

\begin{tcolorbox}
because,  $f(0)=0$.
.\end{tcolorbox}
$f(x)$ is not defined at $0$ and $0$ does not belong to domain.
So no.
\end{solution}



\begin{solution}[by \href{https://artofproblemsolving.com/community/user/177508}{mathuz}]
	\begin{tcolorbox}[quote="mathuz"]
  First,  $f$ - bijective.
It's not difficult.\end{tcolorbox}
Could you show me precisely surjectivity ?
I'm trying since few days (since first post)
Thanks a lot.

\begin{tcolorbox}
because,  $f(0)=0$.
.\end{tcolorbox}
$f(x)$ is not defined at $0$ and $0$ does not belong to domain.
So no.\end{tcolorbox}
sorry, i am wrong.
You are right.

I speak  without a moment's thought! :(
\end{solution}



\begin{solution}[by \href{https://artofproblemsolving.com/community/user/177508}{mathuz}]
	is it true?  There is  real number  $a$, such that  $f(a)\le 1$.
Please, help me!
\end{solution}



\begin{solution}[by \href{https://artofproblemsolving.com/community/user/177508}{mathuz}]
	yea, i have  some idea!

First,  $f$ - injective (*).
$ P(x, f(y)^2) $ $ \Longrightarrow $  $ f(xf(x)+f(f(y)^2))=f(x)^2+f(y)^2 $ and
from symmetry and  (*) we get  \[ xf(x)+f(f(y)^2)=yf(y)+f(f(x)^2)  \Rightarrow \]
\[ f(f(x)^2)-xf(x)=c (constant) (**).\]
  
          Two useful facts: 
  (1) if  for some real number $x$ such that $ f(f(x)^2) > f(x)^2 $,  then  $P(x, f(f(x)^2)-f(x)^2)$ $ \Rightarrow $ $ xf(x)+f(f(f(x)^2)-f(x)^2)=f(x)^2$ (because injectivity(*)) and \[ f(x)>x .\] 
  (2) if for some  real number  $x$ such that  $f(x^2)>f(x)^2$,  then  $P(x, f(x^2)-f(x)^2)$ $ \Rightarrow $  $ xf(x)+f(f(x^2)-f(x)^2)=x^2 $ (because injectivity(*) )and  \[ x>f(x) .\]
 
$f$ - surjective and $c=0$:

Suppose that  $c>0$ $ \Rightarrow $ \[ f(f(x)^2)-f(x)^2+f(x)(f(x)-x) = c >0 (***).\] 
If for some real number $x$ such that $f(x)<x$  then from (***) we have $f(f(x)^2)-f(x)^2>0$ and from (1) $ \Rightarrow $   $f(x)>x$ and contradiction.  Hence,  $f(x)\ge x$ for any positive reals  $x$.  From this,  $c\ge f(x)(f(x)-x) $. Suppose that $f(x)-x=g(x)$ and $g:R_{+}\rightarrow R_{\ge 0}$.
$c\ge g(x)(g(x)+x) $ $ \Rightarrow $ $c>xg(x)$ and \[ x\rightarrow \infty then  g(x)\rightarrow  0 .\]
Hence, $ x\rightarrow \infty $  we have $f(x)=x$ and $c=0$.
I think  if $c\le 0$ then $f(f(x)^2)-f(x)^2 +f(x)(f(x)-x)=c\le 0$ $ \Rightarrow $ $f(x)\le x$ and $f$-surjective .
Because, $f(x)< x$ $ \Rightarrow $  $P(x,x^2-f(x)^2)$ $ \Rightarrow $ $f(u)=x^2$ and surjective...
\end{solution}



\begin{solution}[by \href{https://artofproblemsolving.com/community/user/49556}{xxp2000}]
	1) $f(x)^2-xf(x)=c$ for some constant $c$.
$P(x,xf(x)+f(y))$ yields $f(y+p)=p+f(y)$, where $p=xf(x)+f(x)^2$. This implies $f(y)>mp,\forall y>mp$, where $m$ can be any positive integer.

$P(x,zf(z)+f(y))$ yields $f(y+a)=b+f(y)$, where $a=xf(x)+f(z)^2$ and $b=zf(z)+f(x)^2$.

Suppose $a>b$, we can find a large $n$ such that $y+na>f(y)+nb+3p$. Then we let $m=[\frac{y+na}p]-1$, we see 
$y+na>mp>f(y)+nb=f(y+na)$. Absurd!

So we have $a\leq b$ or  $xf(x)+f(z)^2\leq zf(z)+f(x)^2$. We can switch $x$ and $z$ and we get 
$xf(x)+f(z)^2= zf(z)+f(x)^2$. So $f(x)^2-xf(x)$ is constant

2) $f(x)=x$
We know $f(x+p)=f(x)+p$. With $f(x)^2-xf(x)=c$, we have
both $(f(x)+p)(f(x)-x)=c$ and $f(x)(f(x)-x)=c$. Obviously $f(x)=x$.
\end{solution}
*******************************************************************************
-------------------------------------------------------------------------------

\begin{problem}[Posted by \href{https://artofproblemsolving.com/community/user/177508}{mathuz}]
	Find all continuous function $f:R\rightarrow R$  such that   \[ f(x+y)+f(y+z)+f(z+x)=f(x)+f(y)+f(z)+f(x+y+z) .\]
	\flushright \href{https://artofproblemsolving.com/community/c6h532771}{(Link to AoPS)}
\end{problem}



\begin{solution}[by \href{https://artofproblemsolving.com/community/user/91617}{Lyub4o}]
	Easy to see that $f(0)=0$.Then use that $f(x+y+z)-f(x+y)+f(y)-f(0)=f(x+y)-f(x)+f(z+y)-f(z)$ and after taking $g(x)=f(x+y)-f(x) $ for fixed $y$ we get that for every $x$ $f(x+y)=f(x)+f(y)+a.x$ for some constant $a$,which may be different for the different values of $y$.After taking $x$ as fixed we get $a=y.c$ which leads to $f(x+y)=f(x)+f(y)+c.x.y$.It is not hard then to reach the additive Cauchy equation.
Answer:$f(x)=k.x$ where $k$ is a fixed real number.
\end{solution}



\begin{solution}[by \href{https://artofproblemsolving.com/community/user/29428}{pco}]
	\begin{tcolorbox}Find all continuous function $f:R\rightarrow R$  such that   \[ f(x+y)+f(y+z)+f(z+x)=f(x)+f(y)+f(z)+f(x+y+z) .\]\end{tcolorbox}
Let $P(x,y,z)$ be the assertion $f(x+y)+f(y+z)+f(z+x)=f(x)+f(y)+f(z)+f(x+y+z)$

$P(0,0,0)$ $\implies$ $f(0)=0$
$P(x,y,-x-y)$ $\implies$ $f(x+y)-f(-x-y)=f(x)-f(-x)+f(y)-f(-y)$ and so $g(x+y)=g(x)+g(y)$ where $g(x)=f(x)-f(-x)$ is continuous
Fo $f(x)-f(-x)=cx$ and so $f(-x)=f(x)-cx$

$P((n+1)x,x,-x)$ $\implies$ $f((n+2)x)=2f((n+1)x)-f(nx)+(2f(x)-cx)$
Considering this as a sequence $a_{n+2}=2a_{n+1}-a_n+b$, we easily get $f(px)=p^2f(x)-cx\frac {p(p-1)}2$

So $f(x)=q^2f(\frac xq)-cx\frac {(q-1)}2$

And so $f(\frac pqx)=\frac{p^2}{q^2}f(x)-\frac 12 cx\frac pq(\frac pq-1)$

And so $f(x)=x^2f(1)-\frac 12 cx(x-1)$ $\forall x\in\mathbb Q^+$ and the equation $f(-x)=f(x)-cx$ shows that this must be true $\forall x\in\mathbb Q$

Continuity implies then $\boxed{f(x)=ax^2+bx}$ $\forall x$ which indeed is a solution
\end{solution}



\begin{solution}[by \href{https://artofproblemsolving.com/community/user/177508}{mathuz}]
	\begin{tcolorbox}Easy to see that $f(0)=0$.Then use that $f(x+y+z)-f(x+y)+f(y)-f(0)=f(x+y)-f(x)+f(z+y)-f(z)$ and after taking $g(x)=f(x+y)-f(x) $ for fixed $y$ we get that for every $x$ $f(x+y)=f(x)+f(y)+a.x$ for some constant $a$,which may be different for the different values of $y$.After taking $x$ as fixed we get $a=y.c$ which leads to $f(x+y)=f(x)+f(y)+c.x.y$.It is not hard then to reach the additive Cauchy equation.
Answer:$f(x)=k.x$ where $k$ is a fixed real number.\end{tcolorbox}
You are wrong!  Answer isn't true.
$y$ - const or variable.
\end{solution}



\begin{solution}[by \href{https://artofproblemsolving.com/community/user/141397}{subham1729}]
	Suppose $P(x,y,z): f(x+y)+f(y+z)+f(x+z)=f(x)+f(y)+f(z)+f(x+y+z)$. 
Now $P(0,0,0)\implies f(0)=0$ also $P(x,y,-y)\implies f(x+y)+f(x-y)+f(0)=2f(x)+f(y)+f(-y)$ now just by induction we've $f((n+1)x)+f((n-1)x)=2f(nx)+f(x)+f(-x)$  now so taking $f(nx)=a_n$ we get $a_{n+2}+a_{n}=2a_{n+1}+k$ from here solving we get $f(nx)=an^2+bn+c$ for fixed $a,b,c$ depending on $x$ and for all $n\in\mathbb N$ now take a fixed very small $x$ say $x=\omega$. Now so at points $\omega,2\omega,3\omega........$ our $f$ behaves like a parabola now due to continuity all points between two consecutive $\omega$ must lie on the same parabola since their distance can be make as small as possible. Now so we conclude all points must line on the same parabola and so for all $x>0$ we get $f$ represents a parabola whose equation is $f(x)=ax^2+bx+c$ now putting that into original equation taking $y,z>0$ and $x<0,x+y>0,x+z>0,x+y+z>0$ we get $f(x)=ax^2+bx+c$ for all $x\in\mathbb R$.
\end{solution}



\begin{solution}[by \href{https://artofproblemsolving.com/community/user/177508}{mathuz}]
	Generalization:  
Find all continuous functions $f,g,h :R \rightarrow R $ \[ f(x+y)+f(y+z)+f(z+x)=g(x)+g(y)+g(z)+h(x+y+z) \]
\end{solution}
*******************************************************************************
-------------------------------------------------------------------------------

\begin{problem}[Posted by \href{https://artofproblemsolving.com/community/user/68025}{Pirkuliyev Rovsen}]
	Let ${{f: \mathbb{N}_{0}}\to\mathbb{N}_{0}}$ be a function that satisfies $1)x-f(x)=19[ \frac{x}{19}]-90[ \frac{f(x)}{90}]$ , $2)1900<f(1990)<2000$.Find all possible values that $f(1990)$ can take.


___________________________________
Azerbaijan Land of the Fire
	\flushright \href{https://artofproblemsolving.com/community/c6h532797}{(Link to AoPS)}
\end{problem}



\begin{solution}[by \href{https://artofproblemsolving.com/community/user/29428}{pco}]
	\begin{tcolorbox}Let ${{f: \mathbb{N}_{0}}\to\mathbb{N}_{0}}$ be a function that satisfies $1)x-f(x)=19[ \frac{x}{19}]-90[ \frac{f(x)}{90}]$ , $2)1900<f(1990)<2000$.Find all possible values that $f(1990)$ can take.\end{tcolorbox}
Setting $x=1990$ in the equation, we get $f(1990)\equiv 14\pmod{90}$ and so $\boxed{f(1990)\in\{1904,1994\}}$
\end{solution}



\begin{solution}[by \href{https://artofproblemsolving.com/community/user/139996}{Faustus}]
	\begin{tcolorbox}$ x-f(x)=19[\frac x{19}]-90[\frac{f(x)}{90}]$ $ =19(\frac x{19}-\{\frac x{19}\})$ $ -90(\frac{f(x)}{90}-\{\frac{f(x)}{90}\})$ $ \iff$ $ 19\{\frac x{19}\}=90\{\frac{f(x)}{90}\})$ $ \iff$ $ \{\frac{f(x)}{90}\}=\frac{19}{90}\{\frac x{19}\}$

$ \iff$ $ \frac{f(x)}{90}=h(x)+\frac{19}{90}\{\frac x{19}\}$ where $ h(x)=[\frac{f(x)}{90}]\in\mathbb N_0$ and so $ f(x)=90h(x)+19\{\frac x{19}\}$

so $ f(1990)=90n+19\{\frac {1990}{19}\}=90n+14$ and, since $ 1900<f(1990)<2000$, $ f(1990)\in\{1904,1994\}$\end{tcolorbox}
\end{solution}
*******************************************************************************
-------------------------------------------------------------------------------

\begin{problem}[Posted by \href{https://artofproblemsolving.com/community/user/177508}{mathuz}]
	Find all continuous function $f:R\rightarrow R $  such that \[ f(x+y)+f(x)f(y)=f(xy+1) .\]
	\flushright \href{https://artofproblemsolving.com/community/c6h532905}{(Link to AoPS)}
\end{problem}



\begin{solution}[by \href{https://artofproblemsolving.com/community/user/29428}{pco}]
	\begin{tcolorbox}Find all continuous function $f:R\rightarrow R $  such that \[ f(x+y)+f(x)f(y)=f(xy+1) .\]\end{tcolorbox}
Let $P(x,y)$ be the assertion $f(x+y)+f(x)f(y)=f(xy+1)$
The only constant solution is $f(x)=0$ $\forall x$ and let us from now look only for nonconstant solutions.

$P(x,0)$ $\implies$ $f(x)(1+f(0))=f(1)$ and so $f(0)=-1$ and $f(1)=0$ else $f(x)$ would be constant.

Notice that if $f(u)=0$ for some $u\notin\{-1,0,1\}$ : $P(x+1,u)$ $\implies$ $f(x+u+1)=f(ux+u+1)$ and so $f(x+u+1)=f(u^nx+u+1)$
Setting then $n\to\pm\infty$ in order to have $u^n\to 0$ and using continuity, we get $f(x+u+1)=f(u+1)=$ constant, impossible.
So $f(x)\ne 0$ $\forall x\notin\{-1,+1\}$

$P(x+2,-1)$ $\implies$ $f(x+1)+f(x+2)f(-1)=f(-x-1)$
$p(-x,-1)$ $\implies$ $f(-x-1)+f(-x)f(-1)=f(x+1)$
adding these two lines, we get $f(-1)(f(x+2)+f(-x))=0$ and so either $f(-1)=0$, either $f(-x)=-f(x+2)$ $\forall x$

1) If $f(-1)=0$
===========
Let $a=f(2)$
$P(x-1,-1)$ $\implies$ $f(-x)=f(x)$
$P(x,2)$ $\implies$ $f(x+2)+af(x)=f(2x+1)$
$P(x+1,-2)$ $\implies$ $f(x-1)+af(x+1)=f(-2x-1)=f(2x+1)$
So $f(x+2)=af(x+1)-af(x)+f(x-1)$
From there, it's easy to get :
$f(3)=a^2-1$
$f(4)=a^3-a^2-a$
$f(5)=a^4-2a^3-a^2+2a$
$P(2,2)$ $\implies$ $a^4-3a^3-a^2+3a=0$ $\iff$ $a(a-3)(a^2-1)=0$ and so $a=3$ ($a=\pm 1$ would imply $f(3)=0$, impossible, and $a=0$ would imply $f(2)=0$, impossible.

So $f(x+2)=3f(x+1)-3f(x)+f(x-1)$ gives thru induction : $f(x+n)=f(x+1)\frac{n^2+n}2-f(x)(n^2-1)+f(x-1)\frac{n^2-n}2$
Note that this implies, setting $x=0$ : $f(n)=n^2-1$ $\forall n\in\mathbb Z$

$P(x,n)$ $\implies$ $f(nx+1)=f(x+n)+f(x)(n^2-1)$
Substituing there $f(x+n)$ by the expression we got two lines above, we get :

New assertion $Q(x,n)$ : $f(nx+1)=f(x+1)\frac {n^2+n}2+f(x-1)\frac{n^2-n}2$ $\forall x\in\mathbb R,\forall n\in\mathbb Z$

$Q(\frac pq-1,q)$ $\implies$ $f(\frac pq)\frac {q^2+q}2+f(\frac pq-2)\frac{q^2-q}2=f(p-q+1)=(p-q+1)^2-1$

$Q(\frac pq-1,2q)$ $\implies$ $f(\frac pq)(2q^2+q)+f(\frac pq-2)(2n^2-n)=f(2p-2q+1)=(2p-2q+1)^2-1$

Eliminating $f(\frac pq-2)$ between these two lines, we get $f(\frac pq)=\left(\frac pq\right)^2-1$ and so $f(x)=x^2-1$ $\forall x\in\mathbb Q$

And continuity gives $f(x)=x^2-1$ $\forall x\in\mathbb R$ which indeed is a solution 

2) If $f(-1)\ne 0$ 
============
So $f(-x)=-f(x+2)$ $\forall x$
Setting $x=0$ there, we get $f(2)=1$
Let $a=f(3)=f(-1)\ne 0$ note that $a>0$ else $f(2)=1$ and $f(3)=a$ plus continuity would imply $f(u)=0$ for some $u\in(1,3]$, impossible
$P(1-x,-1)$ $\implies$ $f(x+2)=af(x+1)-f(x)$
From there, it's easy to get :
$f(-1)=-a$
$f(0)=-1$
$f(1)=0$
$f(2)=1$
$f(3)=a$
$f(4)=a^2-1$
$f(5)=a^3-2a$
$P(2,2)$ $\implies$  $a(a+1)(a-2)=0$ and so $a=2$ (since $>0$)

So $f(x+2)=2f(x+1)-f(x)$ which gives thru induction : $f(x+n)=f(x+1)n-f(x)(n-1)$
Note that this implies, setting $x=0$ : $f(n)=n-1$ $\forall n\in\mathbb Z$

$P(x,n)$ $\implies$ $f(nx+1)=f(x+n)+f(x)(n-1)$
Substituing there $f(x+n)$ by the expression we got two lines above, we get :
New assertion $Q(x,n)$ : $f(nx+1)=f(x+1)n$ $\forall x\in\mathbb R,\forall n\in\mathbb Z$


$Q(\frac pq-1,q)$ $\implies$ $f(\frac pq)q=f(p-q+1)=p-q$ and so $f(\frac pq)=\frac pq-1$ and so $f(x)=x-1$ $\forall x\in\mathbb Q$

And continuity gives $f(x)=x-1$ $\forall x\in\mathbb R$ which indeed is a solution 

3) Synthesis of solutions
=================
We got three solutions :
$f(x)=0$ $\forall x$
$f(x)=x^2-1$ $\forall x$
$f(x)=x-1$ $\forall x$
\end{solution}



\begin{solution}[by \href{https://artofproblemsolving.com/community/user/177508}{mathuz}]
	thank you, $pco$.
Answer is complete.

Really,  this  equation  is very hard.
\end{solution}



\begin{solution}[by \href{https://artofproblemsolving.com/community/user/49556}{xxp2000}]
	Continuity assumption seems unnecessary.
\end{solution}



\begin{solution}[by \href{https://artofproblemsolving.com/community/user/29428}{pco}]
	\begin{tcolorbox}Continuity assumption seems unnecessary.\end{tcolorbox}
My proof demanded continuity.
Have you a proof without continuity ?
\end{solution}



\begin{solution}[by \href{https://artofproblemsolving.com/community/user/49556}{xxp2000}]
	Here is the proof without continuity. It partially follows pco's footsteps.
We will define $F(x)=f(x)+1$.

Case 1) $f(-1)=0$, $f(2)=3$. $f(x+2)=3f(x+1)-3f(x)+f(x-1)$, $f(-x)=f(x)$.

1.1) $F(rx)=r^2F(x),\forall x$, where $r$ is any rational.
Now $P(x,\frac{y-1}x+2)-3P(x,\frac{y-1}x+1)+3P(x,\frac{y-1}x)-P(x,\frac{y-1}x-1)$  yields
$f(y+2x)=3f(y+x)-3f(y)+f(y-x),\forall y,x$.
In particular, we have $f(2x)+1=4(f(x)+1)$, or $F(2x)=4F(x)$.
We can show by induction $F(nx)=n^2F(x)$ thus $F(rx)=r^2F(x)$ for rational $r$.

1.2) $F(x)\geq x^2,\forall x$.
$P(x,-x): f(x^2-1)=-1+f(x)^2$ implies $F(x)\geq0,\forall x\geq-1$. $F$ is even, so $F(x)\geq0,\forall x$.
$P(x,x):f(1+x^2)=f(x)^2+f(2x)=(2+f(x))^2-1$ or $F(1+x^2)=(1+F(x))^2$. So $F(x)\geq1,\forall x\geq1$. 
Using $F(rx)=r^2F(x)$, we have $F(x)\geq r^2,\forall x\geq r$ where $r$ is positive rational. 
For any $x>0$, we can find rationals $\{r_n\}$ increasingly converging to $x$. So $F(x)\geq r_n^2$ implies $F(x)\geq x^2,\forall x\geq0$. Since $F$ is even, we proved 1.2

1.3) $f(x)=x^2-1,\forall x$
$P(x,\frac2x):f(x+\frac 2x)=8-f(x)f(\frac2x)$
Let $x\in(1,2)$, we have $f(x)\geq x^2-1>0$ and $f(\frac2x)\geq \frac4{x^2}-1>0$.
So $f(x+\frac2x)\leq (x+\frac2x)^2-1$. Now 1.2) implies $F(x+\frac2x)=(x+\frac2x)^2,\forall x\in(1,2)$. Or $f(x)=x^2-1, \forall 2\sqrt2<x<3$. Now we get 1.3) by using 1.1)

Case 2) $f(-1)=-2$, $f(x+2)=2f(x+1)-f(x)$

$P(x,\frac{y-1}x+2)-2P(x,\frac{y-1}x+1)+P(x,\frac{y-1}x)$ yields
$f(y+2x)=2f(y+x)-f(y)$. In particular, $f(2x)=2f(x)+1$. 
So $f(x)+f(y)=f(x+y)-1$ or $F(x+y)=F(x)+F(y)$
$P(x,x): f(1+x^2)=f(x)^2+2f(x)+1=f(x^2)+1$. $F(x^2)=F(x)^2\geq0$. Now Cauchy equation implies $F(x)=x$ or $f(x)=x-1$.
\end{solution}
*******************************************************************************
-------------------------------------------------------------------------------

\begin{problem}[Posted by \href{https://artofproblemsolving.com/community/user/68025}{Pirkuliyev Rovsen}]
	Find all functions $f: \mathbb{Q}\to\mathbb{R}$ with the property  that $f(-8)=0$ and $f(x+y)=f(x)+f(y)+3xy(x+y+6)-8$ for all $x,y{\in}Q$


______________________________________
Azerbaijan Land of the Fire 
	\flushright \href{https://artofproblemsolving.com/community/c6h533165}{(Link to AoPS)}
\end{problem}



\begin{solution}[by \href{https://artofproblemsolving.com/community/user/29428}{pco}]
	\begin{tcolorbox}Find all functions $f: \mathbb{Q}\to\mathbb{R}$ with the property  that $f(-8)=0$ and $f(x+y)=f(x)+f(y)+3xy(x+y+6)-8$ for all $x,y{\in}Q$\end{tcolorbox}
Let $f(x)=g(x)+x^3+9x^2+8$; The problem becomes $g(-8)=-72$ and $g(x+y)=g(x)+g(y)$ and so $g(x)=9x$

Hence the answer $\boxed{f(x)=x^3+9x^2+9x+8}$
\end{solution}
*******************************************************************************
-------------------------------------------------------------------------------

\begin{problem}[Posted by \href{https://artofproblemsolving.com/community/user/68025}{Pirkuliyev Rovsen}]
	Find all functions $f: \mathbb{R}\to\mathbb{R}$ that satisfy  $f(x+y)-f(x-y)=2y(3x^2+y^2)$ for all $x,y{\in}R$


______________________________________
Azerbaijan Land of the Fire 
	\flushright \href{https://artofproblemsolving.com/community/c6h533167}{(Link to AoPS)}
\end{problem}



\begin{solution}[by \href{https://artofproblemsolving.com/community/user/29428}{pco}]
	\begin{tcolorbox}Find all functions $f: \mathbb{R}\to\mathbb{R}$ that satisfy  $f(x+y)-f(x-y)=2y(3x^2+y^2)$ for all $x,y{\in}R$\end{tcolorbox}
Let $P(x,y)$ be the assertion $f(x+y)-f(x-y)=2y(3x^2+y^2)$
Let $a=f(0)$

$P(\frac x2,\frac x2)$ $\implies$ $\boxed{f(x)=x^3+a}$ which indeed is a solution, whatever is $a\in\mathbb R$
\end{solution}
*******************************************************************************
-------------------------------------------------------------------------------

\begin{problem}[Posted by \href{https://artofproblemsolving.com/community/user/68025}{Pirkuliyev Rovsen}]
	Find all functions ${{f: \mathbb[0;+\infty)}\to\mathbb[0;+\infty)}$ such that  $f(x+y-z)+f(2\sqrt{xz})+f(2\sqrt{yz})=f(x+y+z)$ for all real $x,y,z{\ge}0$ such that $x+y{\ge}z$


______________________________________
Azerbaijan Land of the Fire 
	\flushright \href{https://artofproblemsolving.com/community/c6h533451}{(Link to AoPS)}
\end{problem}



\begin{solution}[by \href{https://artofproblemsolving.com/community/user/29428}{pco}]
	\begin{tcolorbox}Find all functions ${{f: \mathbb[0;+\infty)}\to\mathbb[0;+\infty)}$ such that  $f(x+y-z)+f(2\sqrt{xz})+f(2\sqrt{yz})=f(x+y+z)$ for all real $x,y,z{\ge}0$ such that $x+y{\ge}z$\end{tcolorbox}
Let $P(x,y,z)$ be the assertion $f(x+y-z)+f(2\sqrt{xz})+f(2\sqrt{yz})=f(x+y+z)$

$P(0,0,0)$ $\implies$ $f(0)=0$

$P(x+\frac y2,0,\frac y2)$ $\implies$ $f(x+y)=f(x)+f(\sqrt{2xy+y^2})$ 

$P(x,\frac y2,\frac y2)$ $\implies$ $f(x+y)=f(x)+f(y)+f(\sqrt{2xy})$
(note that both lines imply $f(x+y)\ge f(x)$ and so $f(x)$ is non decreasing)

Subtracting the two lines above, we get new assertion $Q(x,y)$ : $f(\sqrt{2xy+y^2})=f(y)+f(\sqrt{2xy})$

$Q(\frac{x}{2\sqrt y},\sqrt y)$ $\implies$ $f(\sqrt{x+y})=f(\sqrt{x})+f(\sqrt{y})$ $\forall x$, $\forall y\ne 0$, still true when $y=0$

So the function $g(x)=f(\sqrt x)$ is non decreasing and such that $g(x+y)=g(x)+g(y)$ and so is $ax$ for some $a\ge 0$

So $\boxed{f(x)=ax^2}$ which indeed is a solution, whatever is $a\ge 0$
\end{solution}
*******************************************************************************
-------------------------------------------------------------------------------

\begin{problem}[Posted by \href{https://artofproblemsolving.com/community/user/154880}{a00012025}]
	Find all function $f:R^+\rightarrow R^+$  such that
$f(f(x)+y)=xf(1+xy),\forall x,y\in R^+$
	\flushright \href{https://artofproblemsolving.com/community/c6h533460}{(Link to AoPS)}
\end{problem}



\begin{solution}[by \href{https://artofproblemsolving.com/community/user/29428}{pco}]
	\begin{tcolorbox}Find all function $f:R^+\rightarrow R^+$  such that
$f(f(x)+y)=xf(1+xy),\forall x,y\in R^+$\end{tcolorbox}
Let $P(x,y)$ be the assertion $f(f(x)+y)=xf(1+xy)$

1) $f(x)$ is bijective and $f(1)=1$ and $f(x)<1$ $\forall x>1$ and $f(x)>1$ $\forall x<1$
=======================================================
$P(\frac x{f(2)},\frac {f(2)}x)$ $\implies$ $f(\text{something})=x$ and $f(x)$ is surjective.

If $f(x)>1$ for some $x>1$, then $P(x,\frac{f(x)-1}{x-1})$ $\implies$ $x=1$, impossible. So $f(x)\le 1$ $\forall x>1$

If $f(x)<1$ for some $x<1$, then $P(x,\frac{f(x)-1}{x-1})$ $\implies$ $x=1$, impossible. So $f(x)\ge 1$ $\forall x<1$

If $f(a)=f(b)$ for some $a>b$. Surjection implies $\exists t\ne 1$ such that $1>f(t)>\frac ba$ and previous property implies $t>1$
then comparaison of $P(a,\frac{t-1}a)$ and $P(b,\frac{t-1}a)$ gives $f(1+b\frac{t-1}a)=\frac abf(t)>1$, impossible since $1+b\frac{t-1}a>1$. So $f(x)$ is injective.

$P(1,x)$ $\implies$ $f(x+f(1))=f(x+1)$ and, since injective, $f(1)=1$
Since injective, the property $f(x)\le 1$ $\forall x>1$ becomes then $f(x)<1$ $\forall x>1$
Since injective, the property $f(x)\ge 1$ $\forall x<1$ becomes then $f(x)>1$ $\forall x<1$
Q.E.D.

2) $f(x)$ is continuous and involutive
=========================
if $f(f(x))<x$ for some $x$, then $P(f(x),x-f(f(x)))$ $\implies$ $f(1+f(x)(x-f(f(x))))=1$, impossible; So $f(f(x))\ge x$ $\forall x$

$1+xy>1$ and so $f(1+xy)<1$ and so $P(x,y)$ $\implies$ $f(f(x)+y)<x\le f(f(x))$ and so, since surjective $f(x+y)<f(x)$ and $f(x)$ is decreasing
$f(x)$, as a decreasing bijection, is continuous.
$1+xy>1$ and so $f(1+xy)<1$ and so $P(x,y)$ $\implies$ $f(f(x)+y)<x$. Setting $y\to 0+$ and using continuity, we get $f(f(x))\le x$
So $f(f(x))=x$ $\forall x$ (remember we previously got $f(f(x))\ge x$
Q.E.D.

3) $f(x)=\frac 1x$ $\forall x$
=================
Let $x,y>1$
$P(f(x),\frac{y-1}{f(x)})$ $\implies$ $f(x+\frac{y-1}{f(x)})=f(x)f(y)$
$P(f(y),\frac{x-1}{f(y)})$ $\implies$ $f(y+\frac{x-1}{f(y)})=f(y)f(x)$

So $x+\frac{y-1}{f(x)}=y+\frac{x-1}{f(y)}$ $\forall x,y>1$. Setting $y\to 1^+$ and using continuity, we get $f(x)=\frac 1x$ $\forall x>1$

let $x<1$ : $f(x)>1$ and so $x=f(f(x))=\frac 1{f(x)}$ and so $\boxed{f(x)=\frac 1x}$ $\forall x$ which indeed a solution.
Q.E.D.
\end{solution}



\begin{solution}[by \href{https://artofproblemsolving.com/community/user/49556}{xxp2000}]
	1) $f(x)\leq1,\forall x>1$ and $f(x)\geq1,\forall x<1$
Suppose it is false for $f(a)=b$ $P(a,\frac{b-1}{a-1})$ leads $f(b+\frac{b-1}{a-1})=0$. Absurd!


2) $f(x)=\frac1x,\forall x>1$
Let $f(a)=b$ for $a>1$.
$P(a,1-\frac1a): f(1+\frac{ab-1}a)=ab$.
$ab\neq1$ always leads a contradiction with 1). 
So $ab=1$ or $f(x)=\frac1x,\forall x>1$.

3) $f(x)=\frac1x$.
With 2), $P(x,1)$ implies $f(x)=\frac1x,\forall x$.
\end{solution}



\begin{solution}[by \href{https://artofproblemsolving.com/community/user/29428}{pco}]
	\begin{tcolorbox}...
$P(a,1-\frac1a): f(1+\frac{ab-1}a)=ab$.
$ab\neq1$ always leads a contradiction with 1). 
...\end{tcolorbox}
Quite quite nice and direct.
I sometimes choose long sinuous paths instead of so direct ways  :blush: 

Congrats !
\end{solution}



\begin{solution}[by \href{https://artofproblemsolving.com/community/user/184652}{CanVQ}]
	\begin{tcolorbox}Find all function $f:R^+\rightarrow R^+$  such that
$f(f(x)+y)=xf(1+xy),\forall x,y\in R^+\quad (1)$\end{tcolorbox}
From the given hypothesis, we have \[\frac{f\big(f(x)+yz\big)}{x}=f(1+xyz),\quad \forall x,\,y,\,z \in \mathbb R^+.\quad (2)\] Changing the position of $x$ and $y,$ we get \[\frac{f\big(f(x)+yz\big)}{x}=\frac{f\big(f(y)+xz\big)}{y},\quad \forall x,\,y,\,z \in \mathbb R^+. \quad (3)\] If there exist $x>y>0$ such that $f(x)>f(y),$ then we may choose $z=\frac{f(x)-f(y)}{x-y}>0.$ In this case, we have $f(x)+yz=f(y)+xz$ and hence, it follows that $x=y,$ a contradiction. So we must have \[f(x) \le f(y), \quad \forall x>y>0. \quad (4)\] Now, assume that there are $0<a<b$ such that $f(a)=f(b).$ From this, we have \[f\big(f(a)+y\big)=f\big(f(b)+y\big),\] or \[a\cdot f(1+ay)=b\cdot f(1+by),\quad \quad \forall y\in \mathbb R^+. \quad (5)\] Setting $k=\frac{b}{a}>1$ and replacing $y$ by $\frac{y}{a},$ we get \[f(1+y)=k\cdot f(1+ky),\quad \forall y\in\mathbb R^+. \quad (6)\] From $(6),$ using induction, we get \[f(1+y)=k^n\cdot f(1+k^ny),\quad \forall y\in \mathbb R^+,\, n \in \mathbb N^*.\quad (7)\] Replacing $y=\frac{1}{k^n}$ in $(7),$ we get \[f(1) \ge f\left( 1+\frac{1}{k^n}\right) =k^n\cdot f(2),\quad \forall n \in \mathbb N^*.\] Taking $n \to +\infty,$ we get a contradiction. And hence, $f$ is injective.

Now, replacing $y=1$ in $(1),$ we get \[f\big(f(x)+1\big)=x\cdot f(1+x),\quad \forall x\in \mathbb R^+. \quad (8)\] Replacing $y=\frac{f(t)}{x}$ in $(1)$ and using $(8),$ we get \[f\left(f(x)+\frac{f(t)}{x}\right)=x\cdot f\big(1+f(t)\big)=xt\cdot f(1+t),\quad \forall x,\, t \in \mathbb R^+. \quad (9)\] Replacing $x$ by $\frac{x}{t}$ and using $(1)$ again, we get \[\begin{aligned} f\left(f\left(\frac{x}{t}\right)+\frac{t\cdot f(t)}{x}\right) &=x\cdot f(1+t)=x\cdot f\left(1+x\cdot \frac{t}{x}\right) \\ &=f\left(f(x)+\frac{t}{x}\right),\quad \forall x,\, t \in \mathbb R^+.\end{aligned}\] Since $f$ is injective, we have \[f\left(\frac{x}{t}\right)+\frac{t\cdot f(t)}{x}=f(x)+\frac{t}{x},\quad \forall x,\,t \in \mathbb R^+. \quad (10)\] Replacing $x$ by $xt$ in $(10),$ we get \[f(x)+\frac{f(t)}{x}-\frac{1}{x}=f(xt),\quad \forall x,\,t \in \mathbb R^+.\quad (11)\] Changing the position of $x$ and $t$ in $(11),$ we obtain \[f(x)+\frac{f(t)-1}{x}=f(t)+\frac{f(x)-1}{t},\quad \forall x,\,t \in \mathbb R^+. \quad (12)\] Replacing $t=\frac{1}{2}$ in $(12),$ we get \[f(x)+\frac{f\left( \frac{1}{2}\right)-1}{x}=f\left(\frac{1}{2}\right)+2\big[ f(x)-1\big],\] or \[f(x)=\frac{f\left(\frac{1}{2}\right)-1}{x}+2-f\left( \frac{1}{2}\right),\quad \forall x \in \mathbb R^+. \quad (13)\] So we have $f(x)=\frac{u}{x}+1-u,\, \forall x \in \mathbb R^+.$ Plugging this into $(1),$ we can easily find that $u=1.$ This means that \[f(x)=\frac{1}{x},\quad \forall x \in \mathbb R^+.\] This function satisfies our condition.

There is a shorter solution using estimation but I like this solution much more than that one. :)
\end{solution}



\begin{solution}[by \href{https://artofproblemsolving.com/community/user/184652}{CanVQ}]
	My second solution:

After showing that $f $ is decreasing and injective, we can easily calculate $f(1)=1$ (just take $x=1$). Now, for $x>1,$ replacing $y=\frac{x-1}{x}$ in $(1),$ we get \[f\left(f(x)+\frac{x-1}{x}\right)=x\cdot f(x),\quad \forall x >1.\] If $f(x)>\frac{1}{x},$ then $f(x)+\frac{x-1}{x}>1$ and hence, it follows that \[x\cdot f(x)=f\left(f(x)+\frac{x-1}{x}\right)<f(1)=1,\] or $f(x)<\frac{1}{x},$ a contradiction. Similarly, we cannot have $f(x)<\frac{1}{x}.$ So, we must have \[f(x)=\frac{1}{x},\quad \forall x>1.\] Now, setting $y=1$ in $(1)$ and using the above result, we can easily deduce that \[f(x)=\frac{1}{x},\quad \forall x \in \mathbb R^+.\]
\end{solution}
*******************************************************************************
-------------------------------------------------------------------------------

\begin{problem}[Posted by \href{https://artofproblemsolving.com/community/user/156523}{dizzy}]
	Find all functions from the set of integers into itself such that $ f(m+n)+f(mn-1)=f(m)f(n) $,for all integers $m,n$.
	\flushright \href{https://artofproblemsolving.com/community/c6h534092}{(Link to AoPS)}
\end{problem}



\begin{solution}[by \href{https://artofproblemsolving.com/community/user/29428}{pco}]
	\begin{tcolorbox}Find all functions from the set of integers into itself such that $ f(m+n)+f(mn-1)=f(m)f(n) $,for all integers $m,n$.\end{tcolorbox}
The only constant solutions are $f(x)=0$ $\forall x$ and $f(x)=2$ $\forall x$.
So let us from now look only for non constant solutions.
Let $P(x,y)$ be the assertion $f(x+y)+f(xy-1)=f(x)f(y)$
Let $a=f(1)$

$P(x,0)$ $\implies$ $f(x)(f(0)-1)=f(-1)$ and so $f(0)=1$ and $f(-1)=0$ since $f(x)$ is non constant;
$P(x,1)$ $\implies$ $f(x+1)=af(x)-f(x-1)$ and so :

$f(-1)=0$
$f(0)=1$
$f(1)=a$
$f(2)=a^2-1$
$f(3)=a^3-2a$
$f(4)=a^4-3a^2+1$

$P(2,2)$ $\implies$ $a(a+1)(a-2)=0$ and so $a\in\{-1,0,2\}$

$a=-1$ implies $f(x+1)=-f(x)-f(x-1)$ which gives immediately $f(3x)=1$, $f(3x+1)=-1$ and $f(3x+2)=0$ which indeed is a solution.

$a=0$ implies $f(x+1)=-f(x-1)$ which gives immediately $f(4x)=1$ and $f(4x+2)=-1$ and $f(2x+1)=0$ which indeed is a solution.

$a=2$ implies $f(x+1)=2f(x)-f(x-1)$ which gives immediately $f(x)=x+1$ which indeed is a solution.

\begin{bolded}Hence the five solutions\end{underlined}\end{bolded} :
$f(x)=x+1$ $\forall x$
$f(x)=0$ $\forall x$
$f(x)=2$ $\forall x$
$f(3x)=1$, $f(3x+1)=-1$ and $f(3x+2)=0$ $\forall x$
$f(4x)=1$ and $f(4x+2)=-1$ and $f(2x+1)=0$ $\forall x$
\end{solution}



\begin{solution}[by \href{https://artofproblemsolving.com/community/user/177508}{mathuz}]
	Let  $ f(x)=g(x)+1 $ and  $g:Z \rightarrow Z$.
 \[ g(m+n)+g(mn-1)+1=g(m)g(n)+g(m)+g(n) .\]
If $g$ is constant, then there are two solutions $g(m)=1 $ and $g(m)=-1 $. Assume that  $g$ is not constant. 
Not difficult : $g(0)=0$ and  $g(-1)=-1$.
We have  that at $P(m,1)$, \[ g(m+1)+g(m-1)+1=g(m)g(1)+g(m)+g(1) .\]  
So, $g(-2)=-2$ and  $g(1)=-2,-1,1$.
If $g(1)=-2$, then  $g(m+1)+g(m-1)+g(m)=-3$ and  $g(3k)=0$,  $g(3k+1)=-2$ and  $g(3k+2)=-1 $.

If $g(1)=-1$, then  $g(m+1)+g(m-1)=-2$ and  $g(4k+2)=-2$,  $g(2k+1)=-1$ and $g(4k)=0$.

If $g(1)=1$, then \[ g(m+1)+g(m-1) =2g(m) .\] 
So,   \[ g(m+n)+g(m-n)=2g(m) \] for any  $m,n \in Z$.
It's Jensen's equation.
Hence, $g(m)=m$ unique solution.
 
ANSWER:  
(1) $f(x)=0$;
(2) $f(x)=2$;
(3) $f(x)=x+1$;
(4) $f(4k)=1$, $f(2k+1)=0$ and  $f(4k+2)=-1$;
(5) $f(3k)=1$, $f(3k+1)=-1$ and $f(3k+2)=0$;
  
\end{solution}



\begin{solution}[by \href{https://artofproblemsolving.com/community/user/156523}{dizzy}]
	thank you very much pco and mathuz!
\end{solution}
*******************************************************************************
-------------------------------------------------------------------------------

\begin{problem}[Posted by \href{https://artofproblemsolving.com/community/user/156523}{dizzy}]
	Let $ k $ be an even positive integer. Find the number of all functions  $ f: \mathbb{N}_0 \rightarrow \mathbb{N}_0 $ such that $ f(f(n))=n+k $, for any $ n\in\mathbb{N}_0 $.
	\flushright \href{https://artofproblemsolving.com/community/c6h534201}{(Link to AoPS)}
\end{problem}



\begin{solution}[by \href{https://artofproblemsolving.com/community/user/29428}{pco}]
	\begin{tcolorbox}Let $ k $ be an even positive integer. Find the number of all functions  $ f: \mathbb{N}_0 \rightarrow \mathbb{N}_0 $ such that $ f(f(n)=n+k $, for any $ n\in\mathbb{N}_0 $\end{tcolorbox}
It's easy and quite classical to show that the general solution of this functional equation is :

Let $A,B$ any split of $\{0,1,...,k-1\}$ in two equinumerous subsets (that's why the problem told $k$ even) and $h(x)$ any bijection from $A\to B$
Let $r\in A$ : $f(kn+r)=kn+h(r)$
Let $r\in B$ : $f(kn+r)=(k+1)n+h^{-1}(r)$

So : $\binom k{\frac k2}$ choices for $A$ and $\left(\frac k2\right)!$ bijections $h(x)$ and so $\boxed{\frac{k!}{\left(\frac k2\right)!}}$ such functions
\end{solution}



\begin{solution}[by \href{https://artofproblemsolving.com/community/user/177508}{mathuz}]
	if  $k$  is  odd,   then  $f$ - no exists.
\end{solution}



\begin{solution}[by \href{https://artofproblemsolving.com/community/user/156523}{dizzy}]
	thank you!
\end{solution}
*******************************************************************************
-------------------------------------------------------------------------------

\begin{problem}[Posted by \href{https://artofproblemsolving.com/community/user/74657}{ArefS}]
	function $f:\mathbb R\rightarrow \mathbb Z$ satisfies:
$f(x)+f(y)+f(f(x^2+y^2))=1$
for all reals $x,y$.
we are given that there are two reals $a,b$ such that $f(a)-f(b)=3$.
prove that there exist reals $c,d$ such that $f(c)-f(d)=1$

[hide="Comment"]This was given in the exam. It's not hard to show that hypothesis cannot hold. the main problem was $f:\mathbb Z\rightarrow \mathbb Z$.[\/hide]
	\flushright \href{https://artofproblemsolving.com/community/c6h534217}{(Link to AoPS)}
\end{problem}



\begin{solution}[by \href{https://artofproblemsolving.com/community/user/29428}{pco}]
	\begin{tcolorbox}function $f:\mathbb R\rightarrow \mathbb Z$ satisfies:
$f(x)+f(y)+f(f(x^2+y^2))=1$
for all reals $x,y$.
we are given that there are two reals $a,b$ such that $f(a)-f(b)=3$.
prove that there exist reals $c,d$ such that $f(c)-f(d)=1$

[hide="Comment"]This was given in the exam. It's not hard to show that hypothesis cannot hold. the main problem was $f:\mathbb Z\rightarrow \mathbb Z$.[\/hide]\end{tcolorbox}
I dont understand your problem.

It's indeed easy to show that $f(x)$ must be an odd integer $\forall x$ and so it can't exist $a,b$ such that $f(a)-f(b)=3$
\end{solution}



\begin{solution}[by \href{https://artofproblemsolving.com/community/user/74657}{ArefS}]
	\begin{tcolorbox}[quote="ArefS"]function $f:\mathbb R\rightarrow \mathbb Z$ satisfies:
$f(x)+f(y)+f(f(x^2+y^2))=1$
for all reals $x,y$.
we are given that there are two reals $a,b$ such that $f(a)-f(b)=3$.
prove that there exist reals $c,d$ such that $f(c)-f(d)=1$

[hide="Comment"]This was given in the exam. It's not hard to show that hypothesis cannot hold. the main problem was $f:\mathbb Z\rightarrow \mathbb Z$.[\/hide]\end{tcolorbox}
I dont understand your problem.

It's indeed easy to show that $f(x)$ must be an odd integer $\forall x$ and so it can't exist $a,b$ such that $f(a)-f(b)=3$\end{tcolorbox}
as I said, prove the problem for $f:\mathbb Z \rightarrow \mathbb Z$
\end{solution}



\begin{solution}[by \href{https://artofproblemsolving.com/community/user/29428}{pco}]
	Here is the modified problem as I understand it :\begin{tcolorbox}function $f:\mathbb Z\rightarrow \mathbb Z$ satisfies:
$f(x)+f(y)+f(f(x^2+y^2))=1$
for all integers $x,y$.
we are given that there are two integers $a,b$ such that $f(a)-f(b)=3$.
prove that there exist integers $c,d$ such that $f(c)-f(d)=1$
\end{tcolorbox}
That's a fake problem even with $\mathbb Z\to\mathbb Z$

Claim : there are no functions from $\mathbb Z\to\mathbb Z$ such that $f(x)+f(y)+f(f(x^2+y^2))=1$

1) Let us first solve a more general problem :
===============================
Find all functions $f(x)$ from $\mathbb Z\to\mathbb R$ such that $f(x)+f(y)=h(x^2+y^2)$ $\forall x,y\in\mathbb Z$ and for some $h(x)$ from $\mathbb Z\to\mathbb R$

We quickly get :
The set of solutions is a vectorspace
$f(-x)=f(x)$
$f(2x+1)+f(x-2)=f(2x-1)+f(x+2)$
$f(2x)+f(x-5)=f(2x-4)+f(x+3)$

From there, we get that the knowledge of $f(0),f(1),f(2),f(3),f(4),f(6)$ gives the full knowledge of the solution. So the dimension of the vectorspace is at most $6$

Let then the six functions :
$f_1(x)=1$
$f_2(x)=x^2$
$f_3(2x)=1$ and $f_3(2x+1)=0$
$f_4(3x)=1$ and $f_4(3x+1)=f_4(3x+2)=0$
$f_5(4x)=1$ and $f_5(4x+1)=f_5(4x+2)=f_5(4x+3)=0$
$f_6(5x)=0$ and $f_6(5x+1)=f_6(5x+4)=-1$ and $f_6(5x+2)=f_6(5x=3)=1$

It's easy to check that these six functions are independant solutions of the general equation and so dimension of vectorspace is exactly $6$ and we got a basis and so all the solutions.

2) back to our problem.
================
All our solutions are solution of general problem and so are $ax^2+$ a bounded function with $a\in\mathbb R$
Obviously $a=0$ else $LHS$ can not be bounded;
So $f(x)$ is bounded. Let $m$ its min value and $M$ its max value

But then $f(f(x^2+y^2))=1-f(x)-f(y)$ implies $m\le 1-2M$ and $M\ge 1-2m$ and so $M=m$ and the function is constant, which is impossible.

So no solution.
Hence the claim.
\end{solution}
*******************************************************************************
-------------------------------------------------------------------------------

\begin{problem}[Posted by \href{https://artofproblemsolving.com/community/user/173452}{Jayjayniboon}]
	For real number. $ x,y$
	\flushright \href{https://artofproblemsolving.com/community/c6h534230}{(Link to AoPS)}
\end{problem}



\begin{solution}[by \href{https://artofproblemsolving.com/community/user/176542}{ArT_Ty}]
	These problem come from day 1 of 2013 Thailand Mathematical Olympiad .
\end{solution}



\begin{solution}[by \href{https://artofproblemsolving.com/community/user/29428}{pco}]
	\begin{tcolorbox}Determine all functions $f$ : $\mathbb R\to\mathbb R$ satisfying $(x^2+y^2)f(xy)=f(x)f(y)f(x^2+y^2)$ $\forall x,y\in\mathbb R$\end{tcolorbox}
Let $P(x,y)$ be the assertion $(x^2+y^2)f(xy)=f(x)f(y)f(x^2+y^2)$

$P(0,0)$ $\implies$ $f(0)=0$
If $f(a)=0$ for some $a\ne 0$, then $P(\frac xa,a)$ $\implies$ $f(x)=0$ $\forall x$ which indeed is a solution.
Let us from now look only for solutions such that $f(x)\ne 0$ $\forall x\ne 0$

Let then $x\ne 0$ $P(x,1)$ $\implies$ $(x^2+1)f(x)=f(x)f(1)f(x^2+1)$ and so $f(x^2+1)=\frac{x^2+1}{f(1)}$ since $f(1)\ne 0$ and $f(x)\ne 0$

So $f(x)=ax$ $\forall x>1$ where $a=\frac 1{f(1)}$
$P(2,2)$ $\implies$ $a=1$ and so $f(x)=x$ $\forall x\ge 1$
Let $0<x<1$ so that $\frac 1x>1$ and $x^2+\frac 1{x^2}>1$ : $P(x,\frac 1x)$ $\implies$ $(x^2+\frac 1{x^2})f(1)=f(x)f(\frac 1x)f(x^2+\frac 1{x^2})$ $\implies$ $(x^2+\frac 1{x^2})=f(x)\frac 1x(x^2+\frac 1{x^2})$ $\implies$ $f(x)=x$ $\forall x\in(0,1)$

so $f(x)=x$ $\forall x\ge 0$

$P(-1,-1)$ $\implies$ $f(-1)\in\{-1,+1\}$
Let $x<0$ : $P(x,-1)$ $\implies$ $-x=f(x)f(-1)$ and so :
either $f(x)=x$ $\forall x$ which indeed is a solution
either $f(x)=|x|$ $\forall x$ which indeed is a solution

\begin{bolded}Hence the solutions\end{underlined}\end{bolded} :
$f(x)=0$ $\forall x$

$f(x)=x$ $\forall x$

$f(x)=|x|$ $\forall x$
\end{solution}



\begin{solution}[by \href{https://artofproblemsolving.com/community/user/125513}{hal9v4ik}]
	I think that $f(x)=-x$ is also solution
\end{solution}



\begin{solution}[by \href{https://artofproblemsolving.com/community/user/125513}{hal9v4ik}]
	I think here is mistake:
\begin{tcolorbox} $P(2,2)\implies a=1$\end{tcolorbox}
it implies $a^2=1$
\end{solution}



\begin{solution}[by \href{https://artofproblemsolving.com/community/user/29428}{pco}]
	You're right, thanks.

So five solutions :
$f(x)=0$
$f(x)=x$
$f(x)=-x$
$f(x)=|x|$
$f(x)=-|x|$
\end{solution}
*******************************************************************************
-------------------------------------------------------------------------------

\begin{problem}[Posted by \href{https://artofproblemsolving.com/community/user/141397}{subham1729}]
	Find all functions $f:R\to R$ such that $f(x+y)(x-y)=(x+y)(f(x)-f(y))$
	\flushright \href{https://artofproblemsolving.com/community/c6h534246}{(Link to AoPS)}
\end{problem}



\begin{solution}[by \href{https://artofproblemsolving.com/community/user/29428}{pco}]
	\begin{tcolorbox}Find all functions $f:R\to R$ such that $f(x+y)(x-y)=(x+y)(f(x)-f(y))$\end{tcolorbox}
Let $P(x,y)$ be the assertion $f(x+y)(x-y)=(x+y)(f(x)-f(y))$

$P(x,1)$ $\implies$ $f(x+1)(x-1)=(x+1)(f(x)-f(1))$ $\implies$ (multiplying by $x$) : $f(x+1)(x^2-x)=(x^2+x)f(x)-(x^2+x)f(1)$
$P(x+1,-1)$ $\implies$ $f(x)(x+2)=x(f(x+1)-f(-1))$ $\implies$ (multiplying by $x-1$) : $f(x)(x^2+x-2)=(x^2-x)f(x+1)-(x^2-x)f(-1)$

Adding these two lines, we get $f(x)=\frac{(x^2+x)f(1)+(x^2-x)f(-1)}2$ and so $\boxed{f(x)=ax^2+bx}$ which indeed is a solution, whatever are $a,b\in\mathbb R$
\end{solution}
*******************************************************************************
-------------------------------------------------------------------------------

\begin{problem}[Posted by \href{https://artofproblemsolving.com/community/user/118004}{soad}]
	Find all functions $f:\mathbb{R} \to \mathbb{R}$ that satisfy

$f(x^2+f(x)+y) = f(x-y)+f(x^2)+2y$
	\flushright \href{https://artofproblemsolving.com/community/c6h534271}{(Link to AoPS)}
\end{problem}



\begin{solution}[by \href{https://artofproblemsolving.com/community/user/29428}{pco}]
	\begin{tcolorbox}Find all functions $f:\mathbb{R} \to \mathbb{R}$ that satisfy

$f(x^2+f(x)+y) = f(x-y)+f(x^2)+2y$\end{tcolorbox}
Let $g(x)=f(x)-x$ so that the fonctional equation becomes assertion $P(x,y)$ : $g(x^2+x+g(x)+y)=g(x-y)+g(x^2)-g(x)$

$P(x,-\frac{g(x)+x^2}2)$ $\implies$ $g(x^2)=g(x)$ and so $g(-x)=g(x)$ and new assertion $Q(x,y)$ : $g(x^2+x+g(x)+y)=g(x-y)$

$Q(\frac{x}4,\frac{3x}4-\frac{x^2}{16}-g(\frac{x}4))$ $\implies$ $g(x)=g(-\frac{x}2+\frac{x^2}{16}+g(\frac{x}4))$

$Q(\frac{-x}4,\frac{x}4-\frac{x^2}{16}-g(\frac{x}4))$ $\implies$ $g(0)=g(-\frac{x}2+\frac{x^2}{16}+g(\frac{x}4))$

And so $g(x)=g(0)$ which indeed is a solution.

Hence the answer : $\boxed{f(x)=x+c}$ $\forall x$ and whatever is $c\in\mathbb R$
\end{solution}
*******************************************************************************
-------------------------------------------------------------------------------

\begin{problem}[Posted by \href{https://artofproblemsolving.com/community/user/177508}{mathuz}]
	Find  all contiouous  function $f:R \rightarrow R$  such that \[ f(x_{999})\ge \frac{1}{1996}(f(x_1)+f(x_2)+...+f(x_{998})+f(x_{1000}) +...+f(x_{1997}))  \]
for each set of $1997$ numbers $ x_1< ...< x_{1997} $.
	\flushright \href{https://artofproblemsolving.com/community/c6h534324}{(Link to AoPS)}
\end{problem}



\begin{solution}[by \href{https://artofproblemsolving.com/community/user/29428}{pco}]
	\begin{tcolorbox}Find  all contiouous  function $f:R \rightarrow R$  such that \[ f(x_{999})\ge \frac{1}{1996}(f(x_1)+f(x_2)+...+f(x_{998})+f(x_{1000}) +...+f(x_{1997}))  \]
for each set of $1997$ numbers $ x_1< ...< x_{1997} $.\end{tcolorbox}
Let $a<b$. Choose $a<x_i<b$ and set $x_i\to a$ for $i=1,...,998$ and $x_i\to b$ for $i=1000,...,1997$

Given property plus continuity imply $f(x)\ge \frac{f(a)+f(b)}2$ $\forall x\in[a,b]$ and so ${\min(f(a),f(b))\ge \frac{f(a)+f(b)}2}$ and so $f(a)=f(b)$

Hence the answer : $\boxed{f(x)=c}$ $\forall x$ and whatever is $c\in\mathbb R$, which indeed is a solution.
\end{solution}



\begin{solution}[by \href{https://artofproblemsolving.com/community/user/177508}{mathuz}]
	thank you, very much!
\end{solution}
*******************************************************************************
-------------------------------------------------------------------------------

\begin{problem}[Posted by \href{https://artofproblemsolving.com/community/user/172163}{joybangla}]
	Suppose $f:\mathbb{Z}^{2}\rightarrow\mathbb{Z}$ is $f(x,y)=\lfloor{x}\sqrt{2}\rfloor+\lfloor{y}\sqrt{3}\rfloor$.Is $f$ a bijection?
	\flushright \href{https://artofproblemsolving.com/community/c6h534507}{(Link to AoPS)}
\end{problem}



\begin{solution}[by \href{https://artofproblemsolving.com/community/user/29428}{pco}]
	\begin{tcolorbox}Suppose $f:\mathbb{Z}^{2}\rightarrow\mathbb{Z}$ is $f(x,y)=\lfloor{x}\sqrt{2}\rfloor+\lfloor{y}\sqrt{3}\rfloor$.Is $f$ a bijection?\end{tcolorbox}
$f(1,0)=f(0,1)$, so no ... .
\end{solution}



\begin{solution}[by \href{https://artofproblemsolving.com/community/user/172163}{joybangla}]
	Sorry!I meant $f:\mathbb{Z}^{2}\rightarrow\mathbb{Z}$ and if $f$ is onto. :oops_sign:
\end{solution}



\begin{solution}[by \href{https://artofproblemsolving.com/community/user/158287}{dibyo_99}]
	No, it isn't since if $x>y, f(x,a) \ge f(y,a)$ and $f(1,1)=2$ which means the value $1$ is never atttained.
\end{solution}



\begin{solution}[by \href{https://artofproblemsolving.com/community/user/29428}{pco}]
	\begin{tcolorbox}Suppose $f:\mathbb{Z}^{2}\rightarrow\mathbb{Z}$ is $f(x,y)=\lfloor{x}\sqrt{2}\rfloor+\lfloor{y}\sqrt{3}\rfloor$.Is $f$ a surjection ?\end{tcolorbox}
Ohhhhh ! :( You silently edited your last post after \begin{bolded}dibyo_99\end{bolded}'s correct answer, transforming his answer in a wrong one .....
(\begin{italicized}this is the reason for which I always quote the problem I answer\end{italicized})

At the time when dibyo_99 answered, the question was $f:\mathbb N^2\to\mathbb N$

Please, when you change the problem after somebody answered, post some explicit message, or at least indicate in the edited message what was the previous problem and what you changed ... .

About your new problem, just consider $g(n)=f(-n,n)=\lfloor n\sqrt 3\rfloor+\lfloor -n\sqrt 2\rfloor$. Then :
$g(0)=0$
$\lim_{n\to+\infty}g(n)=+\infty$
$\lim_{n\to-\infty}g(n)=-\infty$
$g(n+1)-g(n)\in\{-1,0,1\}$ 

And so $g(\mathbb Z)=\mathbb Z$ and $f(x,y)$ is surjective.
\end{solution}



\begin{solution}[by \href{https://artofproblemsolving.com/community/user/172163}{joybangla}]
	Sorry for that,it was wrong what I did. :oops:
\end{solution}
*******************************************************************************
-------------------------------------------------------------------------------

\begin{problem}[Posted by \href{https://artofproblemsolving.com/community/user/125513}{hal9v4ik}]
	Find all $f:N \to N$ which satisfies
$(i)$ $f(2n)=f(n)+1$
$(ii)$ $(f(2n+1)-f(n))(f(2n)-f(n))=f(n)$
for all $n\in N$
	\flushright \href{https://artofproblemsolving.com/community/c6h534510}{(Link to AoPS)}
\end{problem}



\begin{solution}[by \href{https://artofproblemsolving.com/community/user/29428}{pco}]
	\begin{tcolorbox}Find all $f:N \to N$ which satisfies
$(i)$ $f(2n)=f(n)+1$
$(ii)$ $(f(2n+1)-f(n))(f(2n)-f(n))=f(n)$
for all $n\in N$\end{tcolorbox}
Let $u=f(1)$

We have $f(2n)=f(n)+1$ and $f(2n+1)=2f(n)$

let then $a_1>a_2>...>a_p\ge 0$ all the binary positions of $1$ in binary representation of $n$ so that $n=\sum 2^{a_k}$

A simple induction gives $\boxed{f(\sum 2^{a_k})=2+2^{p-1}(2a_1+u-2)-\sum_{i=1}^{p} 2^{p-i}a_i}$

And I dont know if there is a smarter form ... :?:
\end{solution}
*******************************************************************************
-------------------------------------------------------------------------------

\begin{problem}[Posted by \href{https://artofproblemsolving.com/community/user/125513}{hal9v4ik}]
	Find all surjective  functions $f: N \to N$ such that
$(i)$ $f(4n+2)=f(4n+1)$
$(ii)$ $(f(2n+1)-f(n))(f(2n)-f(n))=f(n)$
	\flushright \href{https://artofproblemsolving.com/community/c6h534518}{(Link to AoPS)}
\end{problem}



\begin{solution}[by \href{https://artofproblemsolving.com/community/user/29428}{pco}]
	\begin{tcolorbox}Find all surjective  functions $f: N \to N$ such that
$(i)$ $f(4n+2)=f(4n+1)$
$(ii)$ $(f(2n+1)-f(n))(f(2n)-f(n))=f(n)$\end{tcolorbox}
My conclusion hereunder is that infinitely many solutions exist and I'm unable to find  any general form for these solutions, although I'm able to build as many solutions as I want.

Since you created the problem, could you kindly give us your solution, please ( I spent a lot of time on it and am very interested in your general form for solutions).
Thanks in advance.

1) preliminary definitions : nice pairs
=========================
Let us call $(x,y)$ a "nice" pair of positive integers if $\exists u,v,a,b\in\mathbb Z$ such that :
$x=uv$, $y=ab$, and $ab+a=uv+v$
Note that not all pairs are nice. For example $(2,5)$ is not nice.
Note that decomposition of nice pairs may be multiple : $(24,26)$ is a nice pair and we have $26\times 1+1=3\times 8+3$ but also $13\times 2+2=6\times 4+4$

2) building infinitely many non surjective solutions
===================================
2.1) choosing $f(1),f(2),f(3)$
-----------------------------
From $(f(3)-f(1))(f(2)-f(1))=f(1)$, we get :
$f(1)=xy$ and $f(2)-f(1)=x$ and $f(3)-f(1)=y$ and so :
$f(1)=xy$
$f(2)=xy+x$
$f(3)=xy+y$
Note that $(f(2),f(3))$ is a nice pair since $(xy+x)+(y+1)=(xy+y)+(x+1)$

2.2) choosing $f(4n),f(4n+1),f(4n+2),f(4n+3)$ from $f(2n),f(2n+1)$
----------------------------------------------------------------------------------

(ii) with $2n$ implies $f(4n)-f(2n)|f(2n)$ and so $f(2n)=uv$ and $f(4n)=uv+u$ for some $u,v\ge 1$ or $u\le -1$ and $v\le -2$
(ii) with $2n$ implies then $f(4n+1)=uv+v$
(i) implies then $f(4n+2)=uv+v$
(ii) with $2n+1$ implies $f(4n+1)-f(2n+1)|f(2n+1)$ and so $f(2n+1)=ab$ and $f(4n+1)=ab+a=uv+v$ for some $a,b\ge 1$ or $a\le -1$ and $b\le -2$
(ii) with $2n+1$ implies then $f(4n+3)=ab+b$

So we got :
$(f(2n),f(2n+1))$ is a nice pair : $f(2n)=uv$ and $f(2n+1)=ab$ with $ab+a=uv+v$ and then :
$f(4n)=uv+u$
$f(4n+1)=uv+v$
$f(4n+2)=uv+v=ab+a$
$f(4n+3)=ab+b$

Note that $(f(4n),f(4n+1))$ is a nice pair : $u(v+1)+(v+1)=v(u+1)+(u+1)$
Note that $(f(4n+2),f(4n+3))$ is a nice pair : $a(b+1)+(b+1)=b(a+1)+(a+1)$

2.3) Building infinitely many solutions
---------------------------------------------
So, from a nice pair $(f(2n),f(2n+1))=(uv,ab)$ with $uv+v=ab+a$, we build two new nice pairs :

$(f(4n),f(4n+1))=(uv+u,uv+v)$ with $(uv+u)+(v+1)=(uv+v)+(u+1)$
$(f(4n+2),f(4n+3)=(ab+a,ab+b)$ with $(ab+a)+(b+1)=(ab+b)+(a+1)$

so :
$f(1)=xy$
$(f(2),f(3))$ as nice pair $(xy+x,xy+y)$
Starting from nice pair $(f(2),f(3))$, we can build nice pairs $(f(4),f(5))$ and $(f(6),f(7))$
Starting from nice pair $(f(4),f(5))$, we can build nice pairs $(f(8),f(9))$ and $(f(10),f(11))$
Starting from nice pair $(f(6),f(7))$, we can build nice pairs $(f(12),f(13))$ and $(f(14),f(15))$
Starting from nice pair $(f(8),f(9))$, we can build nice pairs $(f(16),f(17))$ and $(f(18),f(19))$
...

And so we can build step by step a solution.
Choosing fifferent values for $f(1)$ gives infinitely many solutions.
At each step, if the nice pair $(f(2n),f(2n+1))$ has different decompositions, we can also build many different families of solution.

3) What about surjective solutions ?
=========================
3.1) $f(1)=1$ and $f(2)=f(3)=2$
-----------------------------
the second equation may be written $(2f(n)-(f(2n)+f(2n+1)+1))^2$ $=f(2n)^2+f(2n+1)^2$ $+2f(2n)+2f(2n+1)-2f(2n)f(2n+1)+1$

If $f(2n)=1$ or $f(2n+1)=1$ for some $n$, $RHS$ is $u^2+4$ (where $u$ is the other quantity). But $u^2+4$ can never be a square when $u>0$
So, since surjective, $f(1)=1$ 

Then, setting $n=1$ in (ii) :  $(f(3)-1)(f(2)-1)=1$ and $f(2)=f(3)=2$
Q.E.D.

3.2) surjective solutions do exist
----------------------------------------
It's possible to build a surjective function with for example : $f(1)=1$ and $(f(2^n),f(2^n+1))=(n+1,2n)$ $\forall n\ge 1$

3.3) Infinitely many surjective solutions exist
------------------------------------------------------
In the construction proposed in 3.2) above, we can build the other nice pairs without any consideration about surjectivity, and so create as many different functions as we want.
Note also that it is possible to create cycles. For example :
$f(1)=1$
$(f(2),f(3))=(2,2)$
From $(f(2),f(3))=(2,2)$ : $(f(4),f(5))=(3,4)$ and $(f(6),f(7))=(4,3)$
From $(f(6),f(7))=(4,3)$ : $(f(12),f(13))=(6,6)$ and $(f(14),f(15))=(6,4)$
From $(f(12),f(13))=(6,6)$ : $(f(24),f(25))=(4,3)$ and $(f(26),f(27))=(3,4)$
And so cycles around $(3,4)$ (and we can break cycles when we want)

\begin{bolded}Conclusion \end{bolded}:
Infinitely many solutions exist and there is likely not any general form for these solutions.

Since you created the problem, could you kindly give us your solution, please ?
\end{solution}



\begin{solution}[by \href{https://artofproblemsolving.com/community/user/125513}{hal9v4ik}]
	Sorry \begin{bolded}pco\end{bolded}. But actually I created problem but not its solution(till now). I thought that statements $(i)$ and $(ii)$ with surjectivity will imply uniqueness of function.For example my answer:
Let there will be $a$ $0$'s and $b$ $1$'s in binary representation of $n$.Then function $f(n)=ab$ satisfies statement of problem.
\begin{bolded}Surjectivity\end{underlined}\end{bolded} Just take any $n$ with $k$ $0$'s and 1 $1$'s.
$f(4n+2)=f(4n+1)$
  It is obvious that in the decimal represantation of $4n+2$ and $4n+1$ there are equal number of $1$'s  and $0$'s.
$(f(2n+1)-f(n))(f(2n)-f(n))=f(n)$
Similarly suppose $n$ has $a$ $1$'s and $b$ $0$'s.We get $f(2n+1)=ab+b$ and  $f(2n)=ab+a$ which imples $(ii)$.
\end{solution}



\begin{solution}[by \href{https://artofproblemsolving.com/community/user/125513}{hal9v4ik}]
	Ups! In statement of problem there should be $f:\mathbb{N} \to \mathbb{N}_0$
\end{solution}



\begin{solution}[by \href{https://artofproblemsolving.com/community/user/29428}{pco}]
	\begin{tcolorbox}Sorry \begin{bolded}pco\end{bolded}. But actually I created problem but not its solution(till now). I\end{tcolorbox}
If you dont have the solution and are not sure such a solution exists, better to post in "open" category.
\end{solution}
*******************************************************************************
-------------------------------------------------------------------------------

\begin{problem}[Posted by \href{https://artofproblemsolving.com/community/user/68025}{Pirkuliyev Rovsen}]
	Does there exist a function $f: \mathbb{R}\to\mathbb{R}$ satisfying the conditions
1)$ f(x+y+z){\leq}3(xy+yz+zx)$ for all real $x,y,z$ and 2)there is a function $g: \mathbb{R}\to\mathbb{R}$ and a natural number $n$
such that $g(g(x))=x^{2n+1}$ and $f(g(x))=g^2(x)$ for every real $x$?

________________________________________
Azerbaijan Land of the Fire 
	\flushright \href{https://artofproblemsolving.com/community/c6h534741}{(Link to AoPS)}
\end{problem}



\begin{solution}[by \href{https://artofproblemsolving.com/community/user/29428}{pco}]
	\begin{tcolorbox}Does there exist a function $f: \mathbb{R}\to\mathbb{R}$ satisfying the conditions
1)$ f(x+y+z){\leq}3(xy+yz+zx)$ for all real $x,y,z$ and 2)there is a function $g: \mathbb{R}\to\mathbb{R}$ and a natural number $n$
such that $g(g(x))=x^{2n+1}$ and $f(g(x))=g^2(x)$ for every real $x$?\end{tcolorbox}
$g(g(x))=x^{2n+1}$ $\implies$ $g(\mathbb R)=\mathbb R$
$f(g(x))=(g(x))^2$ $\implies$ $f(x)=x^2$ $\forall x\in g(\mathbb R)$ and so $f(x)=x^2$ $\forall x\in R$
But then, choosing for example $x=1$ and $y=z=0$, the condition $1)$ becomes $1\le 0$, wrong;

So no such function.
\end{solution}
*******************************************************************************
-------------------------------------------------------------------------------

\begin{problem}[Posted by \href{https://artofproblemsolving.com/community/user/177335}{rizkydarmawan}]
	Determine all function that satisfy $ f(1) = 1 $ and $ f(m+n) = f(m) + f(n) + mn $ where m,n are integers

Am I true that $ f(x) = \frac{x+1}{2} $ ? 
I found it with subtitute m,n with number 1, 2, 3, ... etc, then i got the pattern, so i guess like that..

How is the true solution ?
	\flushright \href{https://artofproblemsolving.com/community/c6h534875}{(Link to AoPS)}
\end{problem}



\begin{solution}[by \href{https://artofproblemsolving.com/community/user/177508}{mathuz}]
	is $f:Z\rightarrow Z$!??
\end{solution}



\begin{solution}[by \href{https://artofproblemsolving.com/community/user/177508}{mathuz}]
	Sorry, original version:
if  $f:Z\rightarrow (Q or R)$,  we have  function  $g:Z\rightarrow  (Q or R) $  and \[ g(m)=f(m)-\frac{m^2}{2} \] and \[ g(m+n)=g(m)+g(n) .\]
It's Cauchy equation, thus  $g(m)=cm$, here $c=g(1)$ - constant.  
Hence, \[ f(m)=cm+\frac{m^2}{2} \] for all  $m\in Z.$
\end{solution}



\begin{solution}[by \href{https://artofproblemsolving.com/community/user/177335}{rizkydarmawan}]
	Yes... $ f : Z -> Z$
\end{solution}



\begin{solution}[by \href{https://artofproblemsolving.com/community/user/177508}{mathuz}]
	answer:  $f(m)=cm+\frac{m(m-1)}{2}$.
\end{solution}



\begin{solution}[by \href{https://artofproblemsolving.com/community/user/177335}{rizkydarmawan}]
	Everyone, Could you give me the clear solution, please? 
because Mr Mathuz's solution isn't clear.. :(
\end{solution}



\begin{solution}[by \href{https://artofproblemsolving.com/community/user/29428}{pco}]
	\begin{tcolorbox}Determine all function that satisfy $ f(1) = 1 $ and $ f(m+n) = f(m) + f(n) + mn $ where m,n are integers\end{tcolorbox}
Write $f(x)=g(x)+\frac{x(x+1)}2$ and the equation becomes $g(1)=0$ and $g(m+n)=g(m)+g(n)$ and so $g(n)=0$ $\forall n$ and so $\boxed{f(n)=\frac{n(n+1)}2}$
\end{solution}



\begin{solution}[by \href{https://artofproblemsolving.com/community/user/177335}{rizkydarmawan}]
	\begin{tcolorbox}[quote="rizkydarmawan"]Determine all function that satisfy $ f(1) = 1 $ and $ f(m+n) = f(m) + f(n) + mn $ where m,n are integers\end{tcolorbox}
Write $f(x)=g(x)+\frac{x(x+1)}2$ and the equation becomes $g(1)=0$ and $g(m+n)=g(m)+g(n)$ and so $g(n)=0$ $\forall n$ and so $\boxed{f(n)=\frac{n(n+1)}2}$\end{tcolorbox}

why do you immediately let $ f(x) = g(x) + \frac{x(x+1)}{2} $. Where does it come from?
\end{solution}



\begin{solution}[by \href{https://artofproblemsolving.com/community/user/29428}{pco}]
	You look for a pecular solution in order to subtract it from your equation. 

Basic trials with polynomials quickly show that $\frac 12x^2+ax$ are such solutions. 

Choosing $a=\frac 12$ allows to keep the "integer" property. You should also choose $\frac 32,\frac 52, -\frac 12, ...$ but $\frac 12$ gives $g(1)=0$ which is simpler for the end part :)
\end{solution}
*******************************************************************************
-------------------------------------------------------------------------------

\begin{problem}[Posted by \href{https://artofproblemsolving.com/community/user/177508}{mathuz}]
	Find all  function  \[ f^{(19)}(n)+97f(n)=98n+232 \]
for all  $ n\in N .$
	\flushright \href{https://artofproblemsolving.com/community/c6h534882}{(Link to AoPS)}
\end{problem}



\begin{solution}[by \href{https://artofproblemsolving.com/community/user/172163}{joybangla}]
	$f(n)=n+2$ is an obvious solution.Here $f^{(19)}$ means the 19th composite of $f$ right?
\end{solution}



\begin{solution}[by \href{https://artofproblemsolving.com/community/user/86443}{roza2010}]
	What's meaning of $f^{(19)}(n)$ ???
\end{solution}



\begin{solution}[by \href{https://artofproblemsolving.com/community/user/177508}{mathuz}]
	\begin{tcolorbox}What's meaning of $f^{(19)}(n)$ ???\end{tcolorbox}  
this is mean number of radicals:
 
$f{(19)}(n)=f(f(..f(n)..))$,  19 times.   
\begin{tcolorbox}$f(n)=n+2$ is an obvious solution.\end{tcolorbox}
Write your solution?  Please .
\end{solution}



\begin{solution}[by \href{https://artofproblemsolving.com/community/user/172163}{joybangla}]
	We show $f(n)=n+2\forall{n}$.Put $f(n)=n+2+a_n\forall{n}$. $98n+232>97f(n)$,implies $f(n)\le n+3$ for $n\le156$.Now $f^{(19)}(n)\le f^{(18)}(n+3)\le........\le n+57 \forall{n}\le102$ .Now $97f(n)<f^{(19)}(n)+97f(n)\le n+57+97f(n) \Rightarrow 97(n+2+a_n)< 98n+232 \le n+57+97(n+2+a_n) \Rightarrow 97a_n<38,97a_n+251\ge 232$.1st ineq gives $a_n\le0$and 2nd ineq gives $a_n\ge0$ so $a_n=0$,for bound on $n$ we consider $n\le36$.So $f(n)=n+2$ for $n\le36$. Suppose for $k>36$,we have $f(n)=n+2$,for all $n<k$.$f^{(18)}(k-36)=k$,so $f{(19)}(k-36)=f(k)=98(k-36)+232-97f(k-36)=98k-98*36+2*97+2*19-97k+97*34=k+2$ and we are done!!!(Finally!)Pls check if this is right.
\end{solution}



\begin{solution}[by \href{https://artofproblemsolving.com/community/user/29428}{pco}]
	\begin{tcolorbox}Find all  function  \[ f^{(19)}(n)+97f(n)=98n+232 \]
for all  $ n\in N .$\end{tcolorbox}
What are domains and codomains of required functions ?
\end{solution}



\begin{solution}[by \href{https://artofproblemsolving.com/community/user/167328}{nam}]
	Hello\begin{bolded} pco\end{bolded}, i think mathuz's problem is

Find all  function $f:\mathbb{N}\to\mathbb{N}$ such that \[ f^{(19)}(n)+97f(n)=98n+232  \quad \forall n\in \mathbb{N}\]
Where $ f^{(19)}(n)= f(f(..f(n)..))$  $19$ times 

Dear\begin{bolded} Pco\end{bolded},  Can you give us a solution ?  
\end{solution}



\begin{solution}[by \href{https://artofproblemsolving.com/community/user/29428}{pco}]
	\begin{tcolorbox}Hello\begin{bolded} pco\end{bolded}, i think mathuz's problem is

Find all  function $f:\mathbb{N}\to\mathbb{N}$ such that \[ f^{(19)}(n)+97f(n)=98n+232  \quad \forall n\in \mathbb{N}\]
Where $ f^{(19)}(n)= f(f(..f(n)..))$  $19$ times 
\end{tcolorbox}
Let the sequences :
$a_1=\frac{98}{97}$ and $a_{n+1}=\frac{98-a_n^{19}}{97}$
$b_1=\frac{232}{97}$ and $b_{n+1}=\frac{232}{97}-\frac {b_n(a_n^{19}-1)}{97(a_n-1)}$
It's not very difficult to show that :
$a_{2n-1}$ is a decreasing sequence converging towards $1$
$a_{2n}$ is an increasing sequence converging towards $1$
$b_{2n-1}$ is a decreasing sequence converging towards $2$
$b_{2n}$ is an increasing sequence converging towards $2$

If $f(x)<a_nx+b_n$ $\forall x$, then $f^{(19)}(x)<a_n^{19}x+\frac {b_n(a_n^{19}-1)}{a_n-1}$

And so (using fonctional equation) :  $f(x)>\frac{98-a_n^{19}}{97}+\frac{232}{97}-\frac {b_n(a_n^{19}-1)}{97(a_n-1)}$
And so $f(x)>a_{n+1}x+b_{n+1}$

Same, If $f(x)>a_nx+b_n$ $\forall x$, $f(x)<a_{n+1}x+b_{n+1}$

And since we obviously have $f(x)<a_1x+b_1$, we get $a_{2n}x+b_{2n}<f(x)<a_{2n-1}x+b_{2n-1}$

Setting then $n\to+\infty$ in this inequality, we get $\boxed{f(n)=n+2}$ $\forall n$ which indeed is a solution.
\end{solution}
*******************************************************************************
-------------------------------------------------------------------------------

\begin{problem}[Posted by \href{https://artofproblemsolving.com/community/user/177508}{mathuz}]
	Find all function $f:N\rightarrow N$  such that  

$ f(f(n))+f(n)=2n +2013$  or  $2n +2014$,  $n\in N.$
	\flushright \href{https://artofproblemsolving.com/community/c6h534883}{(Link to AoPS)}
\end{problem}



\begin{solution}[by \href{https://artofproblemsolving.com/community/user/86443}{roza2010}]
	$f(n)=n+671$ , or $f(n)=n+\frac{2014}{3}$

being that $f(n)\in N$ , we get $\boxed{f(n)=n+671}$
\end{solution}



\begin{solution}[by \href{https://artofproblemsolving.com/community/user/29428}{pco}]
	\begin{tcolorbox}Find all function $f:N\rightarrow N$  such that  

$ f(f(n))+f(n)=2n +2013$  or  $2n +2014$,  $n\in N.$\end{tcolorbox}
I'm sorry but I understood nothing to short proof by roza2010 :oops:.

so here is mine (longer :blush: ) :

Equation implies $0*n+1\le f(n)\le 2n+2013$

Suppose then $an+b\le f(n)\le cn+d$ $\forall n\in\mathbb N$ and for some $a,b\ge 0$ and $c,d\in[0,2013]$.

This implies $af(n)+b\le f(f(n))\le cf(n)+d$ and so $(a+1)f(n)+b\le f(f(n))+f(n)\le (c+1)f(n)+d$ and so :

$2n+2013\le(c+1)f(n)+d$ and $(a+1)f(n)+b\le 2n+2014$

and so $\frac 2{c+1}n+\frac{2013-d}{c+1}\le f(n)\le \frac 2{a+1}n+ \frac{2014-b}{a+1}$

So we can build a sequence of quadruples $(x_k,y_k,z_k, t_k)$ :
$(x_1,y_1,z_1,t_1)=(0,1,2,2013)$

$(x_{k+1},y_{k+1},z_{k+1}, t_{k+1})=(\frac 2{z_k+1},\frac{2013-t_k}{z_k+1},\frac 2{x_k+1},\frac{2014-y_k}{x_k+1})$

Such that $x_kn+y_k\le f(n)\le z_kn+t_k$ $\forall n,k$

It's then easy to show that the sequence is convergent towards $(1,\frac{2012}3,1,\frac{2015}3)$ and so $n+670+\frac 23\le f(n)\le n+671+\frac 23$

Hence the solution $\boxed{f(x)=n+671}$ which indeed is a solution
\end{solution}



\begin{solution}[by \href{https://artofproblemsolving.com/community/user/13}{enescu}]
	See also [url]http://www.artofproblemsolving.com/Forum/viewtopic.php?t=76767[\/url]

(Balkan Math Olympiad 2002)
\end{solution}



\begin{solution}[by \href{https://artofproblemsolving.com/community/user/177508}{mathuz}]
	very nice  $pco$!

very beautiful!
:)

I am sorry,  convergent true?
  I think  it's true.
But,  where is it come?
\end{solution}
*******************************************************************************
-------------------------------------------------------------------------------

\begin{problem}[Posted by \href{https://artofproblemsolving.com/community/user/176600}{balthazar}]
	Find min of this function: $f(x)=|x+a|+|x+b|+|x+c|+ . . . + |x+n|$
	\flushright \href{https://artofproblemsolving.com/community/c6h534900}{(Link to AoPS)}
\end{problem}



\begin{solution}[by \href{https://artofproblemsolving.com/community/user/139996}{Faustus}]
	See [url=http://www.artofproblemsolving.com/Forum/viewtopic.php?f=36&t=526376&p=2998169#p2998169]here(p1)[\/url].
\end{solution}



\begin{solution}[by \href{https://artofproblemsolving.com/community/user/29428}{pco}]
	\begin{tcolorbox}Find min of this function: $f(x)=|x+a|+|x+b|+|x+c|+ . . . + |x+n|$\end{tcolorbox}
Sort $a,b,c,...,n$ and call them $-a_1,-a_2,...,-a_p$ with $a_1\le a_2\le ... \le a_p$ so that $f(x)=\sum_{k=1}^p|x-a_k|$

when $x\in(-\infty, a_1]$ : $f(x)=-px+\sum_{k=1}^p a_k$
When $x\in[a_1,a_2]$ : $f(x)=-(p-2)x-a_1+\sum_{k=2}^p a_k$
...
When $x\in[a_{p-1},a_p]$ : $f(x)=(p-2)x+a_p-\sum_{k=1}^{p-1} a_k$
when $x\in[a_p,+\infty)$ : $f(x)=px-\sum_{k=1}^p a_k$

So $f(x)$ is decreasing over $(-\infty,a_{\lfloor\frac p2\rfloor}]$ constant over $[a_{\lfloor\frac p2\rfloor},a_{\lceil\frac p2\rceil}]$ and increasing over $[a_{\lceil\frac p2\rceil},+\infty)$

and minimum is $f(a_{\lfloor\frac p2\rfloor})=f(a_{\lceil\frac p2\rceil})=\boxed{-(p-2\lfloor\frac p2\rfloor)a_{\lfloor\frac p2\rfloor}-\sum_{k=1}^{\lfloor\frac p2\rfloor}a_k+\sum_{k=\lfloor\frac p2\rfloor+1}^pa_k}$
\end{solution}



\begin{solution}[by \href{https://artofproblemsolving.com/community/user/105169}{Nikpour}]
	http://www.artofproblemsolving.com/Forum/viewtopic.php?f=77&t=533857
\end{solution}



\begin{solution}[by \href{https://artofproblemsolving.com/community/user/179739}{Hello_Kitty}]
	Pco, allow me a proof that uses $|x|+|y|\ge|x+y|$.
With $\varphi(x)=\sum_{i=1}^n |x-a_i|, \quad a_1\le a_2\le \cdots a_n$.

\begin{bolded}$\ast$   Case $n=2p$\end{bolded}
$\varphi(x)=\sum_{i=1}^p \big(|a_{i+p}-x|+|x-a_i|\big) \ge \sum_{i=1}^p |\underbrace{a_{i+p}-a_i}_{\ge 0}|$
$= \sum_{i=1}^p \big((a_{i+p}-a_p)+(a_p-a_i) \big)=\varphi(a_p)$.

\begin{bolded}$\ast$ Case $n=2p+1$\end{bolded}
$\varphi(x)=|x-a_p|+\sum_{i=1}^p \big(|x-a_i|+|a_{i+p}-x|\big)\ge \sum_{i=1}^p |a_{i+p}-a_i|= \varphi(a_p)$

Answer is $\varphi(a_p)$.
\end{solution}



\begin{solution}[by \href{https://artofproblemsolving.com/community/user/180976}{tensor}]
	i think a little of statistical knowledge will work here....we know that mean deviation is least when it is measured about median....
\end{solution}
*******************************************************************************
-------------------------------------------------------------------------------

\begin{problem}[Posted by \href{https://artofproblemsolving.com/community/user/177508}{mathuz}]
	Find all functions  $f: \mathbb{N} \to \mathbb{N}$, such that (here $f^k$ is the $k$-th iterate of $f$)
\[ f^{2012}(n)=2n,    n\in \mathbb{N}.  \]
	\flushright \href{https://artofproblemsolving.com/community/c6h534916}{(Link to AoPS)}
\end{problem}



\begin{solution}[by \href{https://artofproblemsolving.com/community/user/29428}{pco}]
	\begin{tcolorbox}Find all functions  $f: \mathbb{N} \to \mathbb{N}$, such that (here $f^k$ is the $k$-th iterate of $f$)
\[ f^{2012}(n)=2n,    n\in \mathbb{N}.  \]\end{tcolorbox}
Let $A_1,A_2,...,A_{2012}$ any split of the set of odd positive integers in $2012$ equinumerous subsets.
For any $k\in\{1,2,...,2011\}$, let $h_k(x)$ any bijection from $A_k\to A_{k+1}$

We can define $f(x)$ as :

Any natural number may be written $x=2^ny$ with odd $y$. Then :
If $y\in A_k$ for any $k\in\{1,2,...,2011\}$ : $f(x)=2^nh_k(y)$
If $y\in A_{2012}$ : $f(x)=2^{n+1}h_{1}^{-1}(h_2^{-1}(...(h_{2011}^{-1}(y))...))$
\end{solution}
*******************************************************************************
-------------------------------------------------------------------------------

\begin{problem}[Posted by \href{https://artofproblemsolving.com/community/user/167328}{nam}]
	1) Find all continuous function $f:[0;1]\to\mathbb{R}$ so that 
$f(x)=\frac{1}{2009}\left(f(\frac{1}{2009})+f(\frac{x+1}{2009})+f(\frac{x+2}{2009})+...+f(\frac{x+2008}{2009})\right) $  for all $x\in [0;1]$

2) Find all continuous function $f:\mathbb{R}\to\mathbb{R}$ so that \[f(2x-\frac{f(x)}{3})=3x,\forall x\in\mathbb{R}\]
	\flushright \href{https://artofproblemsolving.com/community/c6h535045}{(Link to AoPS)}
\end{problem}



\begin{solution}[by \href{https://artofproblemsolving.com/community/user/29428}{pco}]
	\begin{tcolorbox}2) Find all continuous function $f:\mathbb{R}\to\mathbb{R}$ so that \[f(2x-\frac{f(x)}{3})=3x,\forall x\in\mathbb{R}\]\end{tcolorbox}
Let $g(x)=x-\frac{f(x)}3$ so that functional equation becomes $g(x+g(x))=g(x)$ (already posted somewhere in this forum)

Induction gives immediately $g(x+ng(x))=g(x)$ $\forall x,\forall n\in \mathbb N\cup\{0\}$

$g(x)=0$ is a solution.
We'll from now consider non all-zero solutions.

Let $b\in g(\mathbb R)$ such that $b\ne 0$. Since $g(x)$ solution implies $-g(-x)$ solution too, wlog say $b>0$

Let then $a<c$ such that $g(a)=g(c)=b$ ( such $a,c$ exist : consider for example $t$ such that $g(t)=b$ and choose $a=t$ and $c=a+g(a)=a+b$)

1) We'll show that $\exists u\in(a,c)$ such that $g(u)=b$
========================================
If $c>a+g(a)$, just choose $u=a+g(a)$
If $c\le a+g(a)$ : let then $d\in (a,c)$

1.1) If $g(d)=b$, just choose $u=d$ and we got our result

1.2) If $g(d)>b$ , then $\exists n\in\mathbb N$ such that $n>\frac{a+g(a)-d}{g(d)-b}$ and so $d+ng(d)>a+(n+1)g(a)$

So $a+ng(a)<a+(n+1)g(a)$ and $d+ng(d)>a+(n+1)g(a)$ and so continuity of $x+ng(x)$ implies $\exists u\in(a,d)$ such that $u+ng(u)=a+(n+1)g(a)$

And so $\exists u\in(a,d)$ such that $g(u+ng(u))=g(a+(n+1)g(a))$ and so $g(u)=g(a)$.

1.3) If $g(d)<b$, then $\exists n\in\mathbb N$ such that $n>\frac{d-a}{g(a)-g(d)}$ and so $d+ng(d)< a+ng(a)$

So $d+ng(d)< a+ng(a)$ and $c+ng(c)=c+ng(a)>a+ng(a)$ and so continuity of $x+ng(x)$ implies $\exists u\in(d,c)$ such that $u+ng(u)=a+ng(a)$

And so $\exists u\in (d,c)$ such that $g(u+ng(u))=f(a+ng(a))$ and so $g(u)=g(a)$.

2) the set $g^{-1}(\{g(a)\})$ is dense in $[a,+\infty)$
====================================
Suppose the contrary and let $a\le u<v$ such that $g(x)\ne g(a)$ $\forall x\in (u,v)$

Let then $E=\{x\le u$ such that $g(x)=g(a)\}$ and $F=\{x\ge v$ such that $g(x)=g(a)\}$
E and F are non empty (consider $g(a+ng(a))$)

So, let $w=\sup (E)$ and $z=\inf(F)$ Continuity implies $g(w)=g(z)=g(a)$

And previous paragraph implies $\exists t\in(w,z)$ such that $g(t)=g(a)$ and so contradiction

3) $g(x)=g(a)$ $\forall x$
==========================
Previous paragraph shows that $g(x)=g(a)$ $\forall x\ge a$ (continuity)
It's then immediate to conclude $g(x)=c$ is constant

And it's obvious that these functions indeed are solutions.

Hence the answer : $\boxed{f(x)=3x+c}$
\end{solution}



\begin{solution}[by \href{https://artofproblemsolving.com/community/user/167328}{nam}]
	Great, your solution is very nice ! 
Dear \begin{bolded}Mr Patrick\end{bolded}, Please try to solve problem 1? :)
Thank you so much !
\end{solution}



\begin{solution}[by \href{https://artofproblemsolving.com/community/user/29428}{pco}]
	\begin{tcolorbox}1) Find all continuous function $f:[0;1]\to\mathbb{R}$ so that 
$f(x)=\frac{1}{2009}\left(f(\frac{1}{2009})+f(\frac{x+1}{2009})+f(\frac{x+2}{2009})+...+f(\frac{x+2008}{2009})\right) $  for all $x\in [0;1]$\end{tcolorbox}
Since $f(x)$ is continuous over $[0,1]$, then it is bounded and so $\exists a,b\in[0,1]$ such that $f(b)\le f(x)\le f(a)$ $\forall x\in[0,1]$

Setting $x=a$ in functional equation, we get $f(\frac 1{2009})$ $=f(\frac{a+1}{2009})$ $=f(\frac{a+2}{2009})$ ... $=f(\frac{a+2008}{2009})$ $=f(a)$ (else $LHS>RHS$)

Setting $x=b$ in functional equation, we get $f(\frac 1{2009})$ $=f(\frac{b+1}{2009})$ $=f(\frac{b+2}{2009})$ ... $=f(\frac{b+2008}{2009})$ $=f(b)$ (else $LHS < RHS$)

So $f(a)=f(\frac 1{2009})=f(b)$ and so $\boxed{f(x)=c}$ which indeed is a solution whatever is the constant $c\in\mathbb R$
\end{solution}
*******************************************************************************
-------------------------------------------------------------------------------

\begin{problem}[Posted by \href{https://artofproblemsolving.com/community/user/179538}{NHTuprince}]
	1.Find all $f:\mathbb{R}\rightarrow \mathbb{R}$ such that:
$xf(y)+yf(x)=(x+y)f(x)f(y)$
2.Find function $\mathbb{Z}\rightarrow \mathbb{Z}$ such that:
$\left\{\begin{matrix}
 f(1)=1 \\ 
 f(x+y)(f(x)-f(y))=f(x-y)(f(x)+f(y))
\end{matrix}\right.$
3.$f:\mathbb{R}\rightarrow \mathbb{R}$ such that:
$f(x+y)-f(x-y)=f(x)f(y)$
	\flushright \href{https://artofproblemsolving.com/community/c6h535058}{(Link to AoPS)}
\end{problem}



\begin{solution}[by \href{https://artofproblemsolving.com/community/user/29428}{pco}]
	\begin{tcolorbox}1.Find all $f:\mathbb{R}\rightarrow \mathbb{R}$ such that:
$xf(y)+yf(x)=(x+y)f(x)f(y)$\end{tcolorbox}
Let $P(x,y)$ be the assertion $xf(y)+yf(x)=(x+y)f(x)f(y)$

If $f(u)=0$ for some $u\ne 0$, then $P(x,u)$ $\implies$ $f(x)=0$ $\forall x$ which indeed is a solution;
If $f(x)\ne 0$ $\forall x\ne 0$, then $P(\frac x{f(x)}-x,x)$ $\implies$ $f(x)=1$ $\forall x\ne 0$

Hence the solutions :
$f(x)=0$ $\forall x$
$f(0)=a$ and $f(x)=1$ $\forall x\ne 0$, whatever is $a\in\mathbb R$
\end{solution}



\begin{solution}[by \href{https://artofproblemsolving.com/community/user/61542}{AwesomeToad}]
	[hide="first one"]
Substitute $x=0$ to get $yf(0)=yf(0)f(y)$; thus either $f(0)=0$ or $f(y)=1$ for all $y\neq 0$. If the latter, any $f$ with $f(x)=1,x\neq 0$ suffices. If $f(0)=0$, Substitute $x=y=t$ to get $(f(t))^2=f(t)$ so for each $t$, $f(t)=1$ or $f(t)=0$.

The functions $f(x)=1$ and $f(x)=0$ both suffice. If for two real numbers $x,y$, $f(x)=0$ and $f(y) \neq 0$, we have $xf(y)=0$, so either $x=0$ (in which case $f(0)=0$ and $f(y)=1$ for all $y\neq 0$) or $f(y)=0$ as well, implying $f(x)=0 \forall x$.

Thus $f(x)=0$ and all $f$ with $f(x)=1 \forall x\neq 0$ suffice.
[\/hide]
\end{solution}



\begin{solution}[by \href{https://artofproblemsolving.com/community/user/29428}{pco}]
	\begin{tcolorbox}3.$f:\mathbb{R}\rightarrow \mathbb{R}$ such that:
$f(x+y)-f(x-y)=f(x)f(y)$\end{tcolorbox}
Let $P(x,y)$ be the assertion $f(x+y)-f(x-y)=f(x)f(y)$

$P(0,0)$ $\implies$ $f(0)=0$
$P(0,x)$ $\implies$ $f(-x)=f(x)$

$P(x,x)$ $\implies$ $f(2x)=f(x)^2$
$P(x,-x)$ $\implies$ $-f(2x)=f(x)f(-x)=f(x)^2$
Subtracting, we get $f(2x)=0$ $\forall x$

Hence the unique solution : $\boxed{f(x)=0}$ $\forall x$ which indeed is a solution.
\end{solution}



\begin{solution}[by \href{https://artofproblemsolving.com/community/user/49556}{xxp2000}]
	Problem 2

$P(1,1):f(0)=0$. $P(0,y):-f(y)^2=f(-y)f(y)$. So $f(-y)=-f(y)$ when $f(y)\neq0$. If $f(y)=0$ and $f(-y)\neq0$ then the latter implies $f(y)=-f(-y)=0$. Absurd!
Hence $f$ is odd function.

$P(2,1)$ implies $f(2)\neq f(1)$ and $f(3)=\frac{f(2)+1}{f(2)-1}$.Since $f(3)$ is integer, we only have $f(2)=-1,0,2,3$

Case 1) $f(2)=3$.
$P(x,1)$ gives $f(x+1)=\frac{f(x)+1}{f(x)-1}f(x-1)$ when $f(x)\neq 1$. We will get $f(3)=2,f(4)=9,f(5)=\frac52$. No solution!

Case 2) $f(2)=2$.
With $f(x+1)=\frac{f(x)+1}{f(x)-1}f(x-1)$, we can show $f(x)=x,x>0$ by induction. Since $f$ is odd, $f(x)=x$ is the first solution.

Case 3) $f(2)=0$
We get $f(3)=-1,f(4)=0$. $P(x,2)$ implies $f(x+2)=f(x-2)$ when $f(x)\neq0$.So $f(4k+1)=1$ and $f(4k-1)=-1$ by induction. 
$P(4k+1,-1)$ implies $f(4k)=0$ and $P(4k+2,1)$ implies $f(4k+2)=0$. 
The second solution is $f(x)=0$ when $x$ is even; $f(x)=(-1)^{\frac {x-1}2}$ when $x$ is odd.

Case 4) $f(2)=-1$.
We get $f(3)=0,f(4)=1$. $P(x,3)$ implies $f(x+3)=f(x-3)$ when $f(x)\neq0$. So $f(5)=-1$. Also $f(3k+1)=1,f(3k+2)=-1$ by induction. $P(3k,1)$ implies $f(3k)=0$.
The third solution is $f(x)=\left(\frac x3\right)$, where we use Legendre symbol.

To sum up, we have three solutions.
\end{solution}
*******************************************************************************
-------------------------------------------------------------------------------

\begin{problem}[Posted by \href{https://artofproblemsolving.com/community/user/177508}{mathuz}]
	Find all surjective functions $f:N\rightarrow N$  such that  \[ f(n)\ge n+(-1)^n \] for all $n \in N$.
	\flushright \href{https://artofproblemsolving.com/community/c6h535209}{(Link to AoPS)}
\end{problem}



\begin{solution}[by \href{https://artofproblemsolving.com/community/user/86443}{roza2010}]
	$f(n)=k^n\ ,\ k\in N\setminus\{1\}$
\end{solution}



\begin{solution}[by \href{https://artofproblemsolving.com/community/user/177508}{mathuz}]
	please, write your solution?
\end{solution}



\begin{solution}[by \href{https://artofproblemsolving.com/community/user/29428}{pco}]
	\begin{tcolorbox}$f(n)=k^n\ ,\ k\in N\setminus\{1\}$\end{tcolorbox}
Huhh ? The only - little - difficulty of this exercise is surjectivity. and you claim a non surjective result. :?:
\end{solution}



\begin{solution}[by \href{https://artofproblemsolving.com/community/user/125513}{hal9v4ik}]
	Answer\end{underlined}:$f(2k+1)=2k$ and $f(2k)=2k+1$, with $f(1)=1$.
We proceed by induction see that we get from condition that $f(2k+1)\ge 2k$ and $f(2k)\ge 2k+1$.
And so there is for any $u\ge 2$ $f(u)\ge 2$ so  $ f(1)=1$.Similarly since $f(4),f(5)\ge 4$ values of $f(3)=2$ and $f(2)=3$.
Now suppose that we have proven the result for all $n=1,2,3 \cdots 2k+1$.
Since $f(2k+4),f(2k+5)\cdots >2k+3$ we get by surjectivity $f(2k+2)=2k+3$ which implies $f(2k+3)=2k+2$.So induction completed.
\end{solution}
*******************************************************************************
-------------------------------------------------------------------------------

\begin{problem}[Posted by \href{https://artofproblemsolving.com/community/user/169700}{s372102}]
	Let $f:\Bbb R\to\Bbb R$ be a non-injective additive function. Show that $\forall r\in\Bbb R$, the set
\[\{x\in\Bbb R|f(x)=f(r)\}\]
is dense in $\Bbb R$
	\flushright \href{https://artofproblemsolving.com/community/c6h535258}{(Link to AoPS)}
\end{problem}



\begin{solution}[by \href{https://artofproblemsolving.com/community/user/29428}{pco}]
	\begin{tcolorbox}Let $f:\Bbb R\to\Bbb R$ be a non-injective additive function. Show that $\forall r\in\Bbb R$, the set
\[\{x\in\Bbb R|f(x)=f(r)\}\]
is dense in $\Bbb R$\end{tcolorbox}
Since non injective, $\exists a\ne b$ such that $f(a)=f(b)$ and so, since additive, $f(a-b)=0$ and so $f(x(a-b))=0$ $\forall x\in\mathbb Q$

So $f(r+x(a-b))=f(r)$ $\forall r\in\mathbb R,\forall x\in\mathbb Q$

So, $\forall r\in\mathbb R$ :  $\{r+x(a-b),\forall x\in\mathbb Q\}\subseteq f^{-1}(\{f(r)\})$

And since $\{r+x(a-b),\forall x\in\mathbb Q\}$ is dense in $\mathbb R$, we got the required result.
\end{solution}
*******************************************************************************
-------------------------------------------------------------------------------

\begin{problem}[Posted by \href{https://artofproblemsolving.com/community/user/80321}{ahaanomegas}]
	Is there an infinite power tower \[{ \left( f(x) \right)^{\left( f(x) \right)^{\left( f(x) \right)\cdots}}} \]such that it equals the inverse of $f(x)$?
	\flushright \href{https://artofproblemsolving.com/community/c6h535504}{(Link to AoPS)}
\end{problem}



\begin{solution}[by \href{https://artofproblemsolving.com/community/user/29428}{pco}]
	\begin{tcolorbox}Is there an infinite power tower \[{ \left( f(x) \right)^{\left( f(x) \right)^{\left( f(x) \right)\cdots}}} \]such that it equals the inverse of $f(x)$?\end{tcolorbox}
You should define domains and codomains of function.

In absence of such informations, we must suppose that $f(x)$ is $\mathbb R\to\mathbb R$ and since you speak of inverse, we know that $f(x)$ is bijective.

Let then $a$ such that $f(a)=2$ and your equation is $2^{2^{2^{...}}}=f^{-1}(a)$ which is wrong.

Without any precison more, the answer is $\boxed{\text{No}}$

Note that with some pecular other domains and codomains, such functions may exist.
\end{solution}
*******************************************************************************
-------------------------------------------------------------------------------

\begin{problem}[Posted by \href{https://artofproblemsolving.com/community/user/91617}{Lyub4o}]
	Find all functions $f:\mathbb R \longrightarrow \mathbb R$ so that 
$f(x)+f(y+xy)=f(y)+f(x+xy)$
	\flushright \href{https://artofproblemsolving.com/community/c6h535577}{(Link to AoPS)}
\end{problem}



\begin{solution}[by \href{https://artofproblemsolving.com/community/user/164531}{pvskand}]
	$ f(x)+f(y+xy)=f(y)+f(x+xy) $
Now let $y=k$ Where $k$ is some arbitary constant...
Then the equation becomes:
$ f(x)+f(k+xk)=f(k)+f(x+xk) $
Now differentiate with respect to $x$ on both sides...
$f'(x)+kf'(k+xk)=kf'(x+xk)$
Now substitute $k=0$
We get, $f'(x)=0$
Which implies that 
$f(x)=t$ Where $t$ is any constant..... i.e the function is a constant function...(which is obvious )


But my doubt is that how can we get the other functions.... For example in the above question $f(x)=x$ is a solution to the functional equation....
Please can anybody HELP....
\end{solution}



\begin{solution}[by \href{https://artofproblemsolving.com/community/user/91617}{Lyub4o}]
	You don't know if the function is differentiable or not.That is why you cannot differentiate in this case.
\end{solution}



\begin{solution}[by \href{https://artofproblemsolving.com/community/user/49556}{xxp2000}]
	\begin{tcolorbox}Find all functions $f:\mathbb R \longrightarrow \mathbb R$ so that 
$f(x)+f(y+xy)=f(y)+f(x+xy)$\end{tcolorbox}

Cauchy equation is sufficient for this f.e. to hold. Without additional conditions, we will have infinitely many solutions.
\end{solution}



\begin{solution}[by \href{https://artofproblemsolving.com/community/user/29428}{pco}]
	\begin{tcolorbox}Cauchy equation is sufficient for this f.e. to hold. Without additional conditions, we will have infinitely many solutions.\end{tcolorbox}
Yes, but the problem is to show if additive condition is mandatory.
\end{solution}



\begin{solution}[by \href{https://artofproblemsolving.com/community/user/49556}{xxp2000}]
	\begin{tcolorbox}[quote="xxp2000"]Cauchy equation is sufficient for this f.e. to hold. Without additional conditions, we will have infinitely many solutions.\end{tcolorbox}
Yes, but the problem is to show if additive condition is mandatory.\end{tcolorbox}

WLOG, we assume $f(0)=0$. Or we can replace $f$ with $f(x)-f(0)$.
$P(1,y):f(2y)+f(1)=f(y+1)+f(y)$
$P(x,y+1):f(xy+y+x+1)+f(x)=f(xy+2x)+f(y+1)$. 
Switching $x,y$,
$f(xy+2x)+f(y+1)+f(y)=f(xy+2y)+f(x+1)+f(x)$, or
$f(xy+2x)+f(2y)=f(xy+2y)+f(2x)$. 
Comparing with $P(2x,\frac y2)$,
$f(2y)-f(\frac y2)=f(xy+2y)-f(xy+\frac y2)$.
Notice LHS does not depend on $x$. Let $x=-\frac12$ above,
$f(xy+2y)-f(xy+\frac y2)=f(\frac32 y)$.
Hence $f$ is additive.
\end{solution}
*******************************************************************************
-------------------------------------------------------------------------------

\begin{problem}[Posted by \href{https://artofproblemsolving.com/community/user/90757}{Marcinek665}]
	Find all continuous functions $f : \mathbb{R}_+ \rightarrow \mathbb{R}_+$ such that $f(f(x))=xf(x).$
	\flushright \href{https://artofproblemsolving.com/community/c6h535710}{(Link to AoPS)}
\end{problem}



\begin{solution}[by \href{https://artofproblemsolving.com/community/user/29428}{pco}]
	\begin{tcolorbox}Find all continuous functions $f : \mathbb{R}_+ \rightarrow \mathbb{R}_+$ such that $f(f(x))=xf(x).$\end{tcolorbox}
Let $g(x)=\ln(f(e^x))$ : $g(x)$ is a continuous function from $\mathbb R\to\mathbb R$ and functional equation is $g(g(x))=g(x)+x$

This equation is very classical (see for example http://www.artofproblemsolving.com/Forum/viewtopic.php?p=2721097#p2721097  subcase 7.1.1)

Its only continuous solutions are $g(x)=\frac{1+\sqrt 5}2x$ and  $g(x)=\frac{1-\sqrt 5}2x$

Hence the solutions of original equation : $\boxed{f_1(x)=x^{\frac{1+\sqrt 5}2}}$ and $\boxed{f_2(x)=x^{\frac{1-\sqrt 5}2}}$
\end{solution}
*******************************************************************************
-------------------------------------------------------------------------------

\begin{problem}[Posted by \href{https://artofproblemsolving.com/community/user/177508}{mathuz}]
	Let $0<a<1$ be a real number and $f$ continuous function on [0,1] which satisfies $f(0)=0,$ $f(1)=1$ and  \[ f(\frac{x+y}{2})=(1-a)f(x)+af(y) \] for every two real numbers $x,y\in [0,1]$ such that $x\le y$. Determine $f(\frac{1}{7}).$
	\flushright \href{https://artofproblemsolving.com/community/c6h536136}{(Link to AoPS)}
\end{problem}



\begin{solution}[by \href{https://artofproblemsolving.com/community/user/31919}{tenniskidperson3}]
	Let $x=0$ and $y=1$, so that $f\left(\frac{1}{2}\right)=a$.

Let $x=0$ and $y=\frac{1}{2}$ so that $f\left(\frac{1}{4}\right)=a^2$.

Let $x=\frac{1}{2}$ and $y=1$ so that $f\left(\frac{3}{4}\right)=2a-a^2$.

Let $x=\frac{1}{4}$ and $y=\frac{3}{4}$ so that $f\left(\frac{1}{2}\right)=3a^2-2a^3$.

So we require $a=3a^2-2a^3$, or $a(a-1)(2a-1)=0$.  So $a=\frac{1}{2}$.

Then $x=0, y=\frac{2}{7}$; $x=\frac{1}{7}, y=\frac{3}{7}$; $x=\frac{2}{7}, y=\frac{4}{7}$; $x=\frac{3}{7}, y=\frac{5}{7}$; $x=\frac{4}{7}, y=\frac{6}{7}$; $x=\frac{5}{7}, y=1$, give

$f\left(\frac{1}{7}\right)=\frac{f(0)+f\left(\frac{2}{7}\right)}{2}$
$f\left(\frac{2}{7}\right)=\frac{f\left(\frac{1}{7}\right)+f\left(\frac{3}{7}\right)}{2}$
$f\left(\frac{3}{7}\right)=\frac{f\left(\frac{2}{7}\right)+f\left(\frac{4}{7}\right)}{2}$
$f\left(\frac{4}{7}\right)=\frac{f\left(\frac{3}{7}\right)+f\left(\frac{5}{7}\right)}{2}$
$f\left(\frac{5}{7}\right)=\frac{f\left(\frac{4}{7}\right)+f\left(\frac{6}{7}\right)}{2}$
$f\left(\frac{6}{7}\right)=\frac{f\left(\frac{5}{7}\right)+f(1)}{2}$

Multiplying the first equation by 12, the second by 10, the third by 8, the fourth by 6, the fifth by 4, and the sixth by 2, and adding them together, we get

$12f\left(\frac{1}{7}\right)+10f\left(\frac{2}{7}\right)+8f\left(\frac{3}{7}\right)+6f\left(\frac{4}{7}\right)+4f\left(\frac{5}{7}\right)+2f\left(\frac{6}{7}\right)$

$=6f(0)+5f\left(\frac{1}{7}\right)+10f\left(\frac{2}{7}\right)+8f\left(\frac{3}{7}\right)+6f\left(\frac{4}{7}\right)+4f\left(\frac{5}{7}\right)+2f\left(\frac{6}{7}\right)+f(1)$

or

$f\left(\frac{1}{7}\right)=\frac{6f(0)+f(1)}{7}=\frac{1}{7}$.
\end{solution}



\begin{solution}[by \href{https://artofproblemsolving.com/community/user/172163}{joybangla}]
	The functional eqn is equality case of Jensen's ineq so $f$ must be linear.Let $f(x)=ax+b$ we get $f(x)=x$.So $f(\frac{1}{7})=\frac {1}{7}$
\end{solution}



\begin{solution}[by \href{https://artofproblemsolving.com/community/user/177508}{mathuz}]
	\begin{tcolorbox}The functional eqn is equality case of Jensen's ineq so $f$ must be linear.Let $f(x)=ax+b$ we get $f(x)=x$.So $f(\frac{1}{7})=\frac {1}{7}$\end{tcolorbox}

Is Jensen's inequality  $(1-a)f(x)+af(y)\ge f((1-a)x+ay)???$
:maybe:
\end{solution}



\begin{solution}[by \href{https://artofproblemsolving.com/community/user/29428}{pco}]
	\begin{tcolorbox}...

So we require $a=3a^2-2a^3$, or $a(a-1)(2a-1)=0$.  So $a=\frac{1}{2}$.

...\end{tcolorbox}
Just note that from there, functional equation becomes $f(\frac{x+y}2)=\frac{f(x)+f(y)}2$ and so (very classical) $f(x)=x$ $\forall x$
Hence $f(\frac 17)=\frac 17$
\end{solution}



\begin{solution}[by \href{https://artofproblemsolving.com/community/user/16261}{Rust}]
	If $f(cx+(1-c)y)=af(x)+(1-a)f(y)$, then $f(x)$ continiosly
if and only if $f(x)=const$ or $a=c$. For case $a=c$ function $f(x)$ is linear.
\end{solution}



\begin{solution}[by \href{https://artofproblemsolving.com/community/user/31919}{tenniskidperson3}]
	\begin{tcolorbox}[quote="tenniskidperson3"]...

So we require $a=3a^2-2a^3$, or $a(a-1)(2a-1)=0$.  So $a=\frac{1}{2}$.

...\end{tcolorbox}
Just note that from there, functional equation becomes $f(\frac{x+y}2)=\frac{f(x)+f(y)}2$ and so (very classical) $f(x)=x$ $\forall x$
Hence $f(\frac 17)=\frac 17$\end{tcolorbox}

You're right, of course, but I was just showing how you could arrive at $f\left(\frac{1}{7}\right)$ without going through the trouble to prove (or cite) Cauchy's equation, since that's all that was asked.
\end{solution}
*******************************************************************************
-------------------------------------------------------------------------------

\begin{problem}[Posted by \href{https://artofproblemsolving.com/community/user/41546}{maff}]
	Find all real functions $f$ such that \[f(xy)=f(f(x)+f(y))\] for all real $x,y$.
	\flushright \href{https://artofproblemsolving.com/community/c6h536371}{(Link to AoPS)}
\end{problem}



\begin{solution}[by \href{https://artofproblemsolving.com/community/user/29428}{pco}]
	\begin{tcolorbox}Find all real functions $f$ such that \[f(xy)=f(f(x)+f(y))\] for all real $x,y$.\end{tcolorbox}
If $f(a)\ne f(b)$ for some $a\ne b$, one of $f(a),f(b)$ is different from $-f(0)$. Wlog say $f(b)+f(0)=u\ne 0$

$f(a)=f(\frac au(f(b)+f(0)))$ $=f(f(\frac au)+f(f(b)+f(0)))=f(f(\frac au)+f(0))$ $=f(0)$

$f(b)=f(\frac bu(f(b)+f(0)))$ $=f(f(\frac bu)+f(f(b)+f(0)))=f(f(\frac bu)+f(0))$ $=f(0)=f(a)$ and so contradiction

Hence the answer : $f(a)=f(b)$ $\forall a,b$ and so  $\boxed{f(x)=c}$ $\forall x$, whatever is constant $c\in\mathbb R$, which indeed is a solution.
\end{solution}



\begin{solution}[by \href{https://artofproblemsolving.com/community/user/170079}{MMEEvN}]
	First if $f(a)=-f(0)$ for all $a$ we get $f(0)=0$ and hence the function is constant.
If not there exist at least one $l$ such that $f(l)\neq -f(0) \Rightarrow f(l)+f(0) \neq 0$
Let $P(x,y)$ be the assertion.
$P(0,y) ,f(f(y)+f(0))=f(0),, P(f(l)+f(0),\frac{y}{f(l)+f(0)}) ,f(y)=f(f(f(l)+f(0))+f(\frac{y}{f(l)+f(0)})=f(f(0)+f(\frac{y}{f(l)+f(0)})=f(0)$ 
Hence the function is constant
\end{solution}
*******************************************************************************
-------------------------------------------------------------------------------

\begin{problem}[Posted by \href{https://artofproblemsolving.com/community/user/68025}{Pirkuliyev Rovsen}]
	Find all functions $f(x)$ defined in the range $(-\frac{\pi}{2};\frac{\pi}{2})$ they can be differentiated for $x=0$ and satisfy the condition:
$f(x)=\frac{1}{2}(1+\frac{1}{cosx})f(\frac{x}{2})$ for every $x$ in the range $(-\frac{\pi}{2};\frac{\pi}{2})$.


_______________________________________
Azerbaijan Land of the Fire 
	\flushright \href{https://artofproblemsolving.com/community/c6h536448}{(Link to AoPS)}
\end{problem}



\begin{solution}[by \href{https://artofproblemsolving.com/community/user/29428}{pco}]
	\begin{tcolorbox}Find all functions $f(x)$ defined in the range $(-\frac{\pi}{2};\frac{\pi}{2})$ they can be differentiated for $x=0$ and satisfy the condition:
$f(x)=\frac{1}{2}(1+\frac{1}{cosx})f(\frac{x}{2})$ for every $x$ in the range $(-\frac{\pi}{2};\frac{\pi}{2})$.\end{tcolorbox}
Let $x\ne 0$ : $f(x)=\frac{\cos^2\frac x2}{\cos x}f(\frac x2)$

So $x\frac{f(x)}{\tan x}=\frac x2\frac{f(\frac x2)}{\tan \frac x2}$

So $x\frac{f(x)}{\tan x}=\frac x{2^n}\frac{f(\frac x{2^n})}{\tan \frac x{2^n}}$

Setting $n\to+\infty$ and using continuity at $0$, we get then $x\frac{f(x)}{\tan x}=f(0)$

And so $\boxed{f(0)=a\text{ and }f(x)=a\frac{\tan x}x}$ $\forall x\ne 0$ which indeed is a solution, whatever is $a\in\mathbb R$
\begin{italicized}Note that we just needed continuity at $0$, not differentiability\end{italicized}.
\end{solution}



\begin{solution}[by \href{https://artofproblemsolving.com/community/user/91617}{Lyub4o}]
	Maybe I cannot see something obvious,but don't we need $\lim_{n \to \infty}\frac {x}{2^n.tan(\frac {x}{2^n})} =1$
\end{solution}



\begin{solution}[by \href{https://artofproblemsolving.com/community/user/115063}{PhantomR}]
	$\lim_{n\to\infty}\frac{x}{2^n \cdot \tan \left( \frac{x}{2^n} \right ) } = \lim_{n\to\infty} \frac{x}{2^n \cdot \frac{x}{2^n} \cdot \frac{\tan \frac{x}{2^n}}{\frac{x}{2^n}}}=\lim_{n\to\infty} \frac{1}{\frac{\tan \frac{x}{2^n}}{\frac{x}{2^n}}}\stackrel{\frac{x}{2^n}=t\to 0}=$ $\lim_{t \to 0} \frac{1}{\frac{\tan t}{t}}=\frac{1}{1}=1$.
\end{solution}



\begin{solution}[by \href{https://artofproblemsolving.com/community/user/29428}{pco}]
	\begin{tcolorbox}Maybe I cannot see something obvious,but don't we need $\lim_{n \to \infty}\frac {x}{2^n.tan(\frac {x}{2^n})} =1$\end{tcolorbox}
Yes, but I supposed well known the fact that $\lim_{x\to 0}\frac {\tan x}x=1$
\end{solution}
*******************************************************************************
-------------------------------------------------------------------------------

\begin{problem}[Posted by \href{https://artofproblemsolving.com/community/user/125553}{lehungvietbao}]
	Find all functions  $f: \mathbb R\to\mathbb R$ which satisfy the following conditions
a) $f(x+1)=f(x)+1$
b) $f\left(\frac{1}{f(x)}\right)=\frac{1}{x}$
	\flushright \href{https://artofproblemsolving.com/community/c6h567600}{(Link to AoPS)}
\end{problem}



\begin{solution}[by \href{https://artofproblemsolving.com/community/user/29428}{pco}]
	\begin{tcolorbox}Find all functions  $f: \mathbb R\to\mathbb R$ which satisfy the following conditions
a) $f(x+1)=f(x)+1$
b) $f\left(\frac{1}{f(x)}\right)=\frac{1}{x}$\end{tcolorbox}
I note that this real olympiad exercise does not demand continuity. So there are obvioulsly infinity many solutions as strange as we want.
I note that this real olympiad exercise forgot to indicate that condition b) can not be true when $x=0$

Let $P(x)$ be the assertion $f(x+1)=f(x)+1$, true $\forall x$
Let $Q(x)$ be the assertion $f(\frac 1{f(x)})=\frac 1x$, true $\forall x\ne 0$

If $f(x)=0$ for some $x\ne 0$, then $Q(x)$ is wrong. So $f(x)=0$ $\implies$ $x=0$

$Q(1)$ $\implies$ $f(\frac 1{f(1)})=1$
$P(\frac 1{f(1)}-1)$ $\implies$ $f(\frac 1{f(1)}-1)=0$ and so $\frac 1{f(1)}-1=0$ and so $f(1)=1$

Note that $f(x)=x$ $\implies$ $f(x+n)=f(x+n)$ and $f(\frac 1x)=\frac 1x$

Using continued fractions representations, we get then $\boxed{f(x)=x\text{   }\forall x\in\mathbb Q}$

And extension to $\mathbb R$ is obviously impossible without additional conditions (which are not present in this real olympiad exercise).
It's then quite easy to build infinitely many solutions.
Example :

$f(p\sqrt 2+q)=p\sqrt 3+q$ $\forall p,q\in\mathbb Q$
$f(p\sqrt 3+q)=p\sqrt 2+q$ $\forall p,q\in\mathbb Q$
$f(x)=x$ $\forall$ other $x\in\mathbb R$

And a lot of other solutions ...
\end{solution}
*******************************************************************************
-------------------------------------------------------------------------------

\begin{problem}[Posted by \href{https://artofproblemsolving.com/community/user/125553}{lehungvietbao}]
	Find all functions$ f:\mathbb R\to\mathbb R$ such that
\[f((x - y)^2) = x^2 - 2yf(x) + (f(y))^2 \quad \forall x,y\in\mathbb R\]
	\flushright \href{https://artofproblemsolving.com/community/c6h567602}{(Link to AoPS)}
\end{problem}



\begin{solution}[by \href{https://artofproblemsolving.com/community/user/89198}{chaotic_iak}]
	Let $P(x,y)$ be the statement $f((x-y)^2) = x^2 - 2yf(x) + f(y)^2$.

$P(x,x) \implies f(0) = x^2 - 2xf(x) + f(x)^2 = (f(x) - x)^2$. In particular, $P(0,0) \implies f(0) = f(0)^2$, so $f(0) = 0, 1$.

When $f(0) = 0$, we have $(f(x)-x)^2 = 0$, so $f(x) = x \forall x$ which is indeed a solution. Otherwise $f(0) = 1$, and so $(f(x)-x)^2 = 1$ for all $x$. So for all $x$, either $f(x) = x+1$ or $f(x) = x-1$.

$P(x,0) \implies f(x^2) = x^2 + 1$
$P(0,y) \implies f(y^2) = -2y + f(y)^2$
Combining, we get $x^2 + 1 = -2x + f(x)^2$, or $f(x)^2 = (x + 1)^2$. So for all $x$, either $f(x) = x+1$ or $f(x) = -x-1$.

So for all $x$, either $f(x) = x+1$ or both of $f(x) = x-1$ and $f(x) = -x-1$ are true. The latter implies $x = 0$, so $f(0) = -1$ which contradicts our assumption $f(0) = 0$ above, so $f(x) = x+1 \forall x$ which is indeed a solution.

So the solutions are $f(x) = x \forall x$ and $f(x) = x+1 \forall x$.
\end{solution}



\begin{solution}[by \href{https://artofproblemsolving.com/community/user/29428}{pco}]
	\begin{tcolorbox}Find all functions$ f:\mathbb R\to\mathbb R$ such that
\[f((x - y)^2) = x^2 - 2yf(x) + (f(y))^2 \quad \forall x,y\in\mathbb R\]\end{tcolorbox}
Let $P(x,y)$ be the assertion $f((x-y)^2)=x^2-2yf(x)+f^2(y)$
Let $a=f(0)$

$P(0,0)$ $\implies$ $a=a^2$ and so $a\in\{0,1\}$

Let $x\ge 0$ : $P(\sqrt x,0)$ $\implies$ $f(x)=x+a$ (remember $a^2=a$)

$P(x,1)$ $\implies$ $(x-1)^2+a=x^2-2f(x)+(1+a)^2$ and so $f(x)=x+a$ $\forall x$

Hence the two solutions :
$\boxed{f(x)=x}$ $\forall x$, which indeed is a solution

$\boxed{f(x)=x+1}$ $\forall x$, which indeed is a solution

\begin{bolded}Edit\end{underlined} \end{bolded}: too late :)
\end{solution}
*******************************************************************************
-------------------------------------------------------------------------------

\begin{problem}[Posted by \href{https://artofproblemsolving.com/community/user/125553}{lehungvietbao}]
	Find all functions $f: \mathbb R\to  \mathbb R$ for which $x (f (x +1) - f (x)) = f (x)$ for all $x \in\mathbb R$ and $\left | f (x) - f (y) \right| \leq  \left | xy \right | \quad \forall x, y \in\mathbb R$.
	\flushright \href{https://artofproblemsolving.com/community/c6h567604}{(Link to AoPS)}
\end{problem}



\begin{solution}[by \href{https://artofproblemsolving.com/community/user/29428}{pco}]
	\begin{tcolorbox}Find all functions $f: \mathbb R\to  \mathbb R$ for which $x (f (x +1) - f (x)) = f (x)$ for all $x \in\mathbb R$ and $\left | f (x) - f (y) \right| \leq  \left | xy \right | \quad \forall x, y \in\mathbb R$.\end{tcolorbox}
Setting $x=0$ il left equation, we get $f(0)=0$
Setting $y=0$ in right inequation, we get $f(x)=f(0)$

Hence the answer : $\boxed{f(x)=0}$ $\forall x$, which indeed is a solution.
\end{solution}
*******************************************************************************
-------------------------------------------------------------------------------

\begin{problem}[Posted by \href{https://artofproblemsolving.com/community/user/198285}{ilovemath121}]
	Find all functions $f: \mathbb{R}^{+} \to \mathbb{R}^{+}$ such that $ \displaystyle f\left (\frac{f(x)}{yf(x)+1} \right )=\frac{x}{xf(y)+1} $ for all $x,y\in \mathbb{R}^{+}$.
	\flushright \href{https://artofproblemsolving.com/community/c6h567755}{(Link to AoPS)}
\end{problem}



\begin{solution}[by \href{https://artofproblemsolving.com/community/user/29428}{pco}]
	\begin{tcolorbox}Find all functions $f: \mathbb{R}^{+} \to \mathbb{R}^{+}$ such that $ \displaystyle f\left (\frac{f(x)}{yf(x)+1} \right )=\frac{x}{xf(y)+1} $ for all $x,y\in \mathbb{R}^{+}$.\end{tcolorbox}
Let $g(x)=\frac 1{f(\frac 1x)}$

Functional equation becomes assertion $P(x,y)$ : $g\left(g(x)+\frac 1y\right)=x+\frac 1{g(y)}$

$P(x,\frac 1{g(1)})$ $\implies$  $g\left(g(x)+g(1)\right)=x+\frac 1{g(\frac 1{g(1)})}$

$P(1,\frac 1{g(x)})$ $\implies$  $g\left(g(1)+g(x)\right)=1+\frac 1{g(\frac 1{g(x)})}$

So $\frac 1{g(\frac 1{g(x)})}=x+\alpha$ where $\alpha=\frac 1{g(\frac 1{g(1)})}-1$

$P(x,\frac 1{g(y)})$ $\implies$  $g\left(g(x)+g(y)\right)=x+y+\alpha$

Setting there $x=g(a)+g(b)$ and $y=g(c)+g(d)$, we get $g(a+b+c+d+2\alpha)=g(a)+g(b)+g(c)+g(d)+\alpha$

So $g(x+y)=g(x)+g(y)+\beta$ $\forall x>0$, $\forall y>2\alpha$

So $g(x+y)=g(x)+g(y)+\beta$ $\forall x>0$, $\forall y>0$

And since $g(x)$ is lower bounded, it is continuous and we get $g(x)=ax+b$ for some $a,b$

Plugging back in original equation, we get $a=1$ and $b=0$ and $g(x)=x$

And so $\boxed{f(x)=x}$ $\forall x$
\end{solution}
*******************************************************************************
-------------------------------------------------------------------------------

\begin{problem}[Posted by \href{https://artofproblemsolving.com/community/user/93274}{khaitang}]
	Find all functions $f: \mathbb{R}\rightarrow \mathbb{R}$ for which :
$f(f(x)+y)=-2y+f(f(y)-x)$
holds for all real numbers $x$ and $y$.
	\flushright \href{https://artofproblemsolving.com/community/c6h567869}{(Link to AoPS)}
\end{problem}



\begin{solution}[by \href{https://artofproblemsolving.com/community/user/29428}{pco}]
	\begin{tcolorbox}Find all functions $f: \mathbb{R}\rightarrow \mathbb{R}$ for which :
$f(f(x)+y)=-2y+f((y)-x)$
holds for all real numbers $x$ and $y$.\end{tcolorbox}
The expression $f((y)-x)$ is not clear. Is it $f(f(y)-x)$ of $f(y-x)$ ?

1) case where we must understand $f((y)-x)$ as $f(f(y)-x)$
=======================================
Let $P(x,y)$ be the assertion $f(f(x)+y)=-2y+f(f(y)-x)$
Let $a=f(0)$

$P(f(\frac{a-x}2),\frac{a-x}2)$ $\implies$ $f(f(f(\frac{a-x}2))+\frac{a-x}2)=x$ and so $f(x)$ is surjective

If $f(u)=f(v)$ for some $u,v$, Let $x\in\mathbb R$ and $y$ such that $f(y)=x+a$ (which exists since $f(x)$ is surjective) :
Comparaison of $P(u,y)$ with $P(v,y)$ implies $f(x)=f(x+u-v)$ $\forall x$
Then comparaison of $P(0,0)$ and $P(0,u-v)$ implies $u=v$ and so $f(x)$ is injective.

$P(x,0)$ $\implies$ $f(f(x))=f(a-x)$ and so, since injective, $\boxed{f(x)=a-x}$ $\forall x$ which indeed is a solution, whatever is $a\in\mathbb R$

2) case where we must understand $f((y)-x)$ as $f(y-x)$
======================================
Let $P(x,y)$ be the assertion $f(f(x)+y)=-2y+f(y-x)$

If $f(u)=f(v)$ for some $u,v$, then comparaison of $P(u,x+u)$ with $P(v,x+v)$ implies $f(x)=f(x+u-v)$ $\forall x$
Then comparaison of $P(0,0)$ and $P(0,u-v)$ implies $u=v$ and so $f(x)$ is injective.

$P(x,0)$ $\implies$ $f(f(x))=f(-x)$ and so, since injective, $f(x)=-x$ $\forall x$ which is not a solution.

\begin{bolded}So no solution in this case\end{underlined}\end{bolded}.
\end{solution}



\begin{solution}[by \href{https://artofproblemsolving.com/community/user/190093}{KamalDoni}]
	Then comparaison of $P(0,0)$ and $P(0,u-v)$ implies $u=v$ and so $f(x)$ is injective.

but $P(0,0)$ gives nothing usefull , so I think that there must be $P(v-u,0)$ and $P(0,u-v)$
\end{solution}



\begin{solution}[by \href{https://artofproblemsolving.com/community/user/29428}{pco}]
	\begin{tcolorbox}Then comparaison of $P(0,0)$ and $P(0,u-v)$ implies $u=v$ and so $f(x)$ is injective.

but $P(0,0)$ gives nothing usefull , so I think that there must be $P(v-u,0)$ and $P(0,u-v)$\end{tcolorbox}
No, just $P(0,u-v)$ gives the result
(in fact, in my first redaction, I wrote "comparaison of $P(x,y)$ with $P(x,y+u-v)$" and I simplified later)
\end{solution}



\begin{solution}[by \href{https://artofproblemsolving.com/community/user/190093}{KamalDoni}]
	I got it) thank you for explanation , your way is easier than mine )
\end{solution}
*******************************************************************************
-------------------------------------------------------------------------------

\begin{problem}[Posted by \href{https://artofproblemsolving.com/community/user/125553}{lehungvietbao}]
	Find all functions $f:\mathbb{R} \to \mathbb{R}$ such that \[x^2+y^2+2f(xy)=f(x+y)(f(x)+f(y))\]
	\flushright \href{https://artofproblemsolving.com/community/c6h568166}{(Link to AoPS)}
\end{problem}



\begin{solution}[by \href{https://artofproblemsolving.com/community/user/29428}{pco}]
	\begin{tcolorbox}Find all functions $f:\mathbb{R} \to \mathbb{R}$ such that \[x^2+y^2+2f(xy)=f(x+y)(f(x)+f(y))\]\end{tcolorbox}
Let $P(x,y)$ be the assertion $x^2+y^2+2f(xy)=f(x+y)(f(x)+f(y))$

$P(0,0)$ $\implies$ $2f(0)=2f(0)^2$ and so $f(0)\in\{0,1\}$

1) If $f(0)=0$
============
$P(x,-x)$ $\implies$ $f(-x^2)=-x^2$ and so $f(x)=x$ $\forall x\le 0$

If $x>0$, let $y<-x<0$ : $P(x,y)$ $\implies$ $f(x)=x$

And so $\boxed{f(x)=x}$ $\forall x$, which indeed is a solution.

2) If $f(0)=1$
============
$P(1,0)$ $\implies$ $f(1)^2+f(1)-3=0$ and so $f(1)\in\{\frac{-1-\sqrt{13}}2,\frac{-1+\sqrt{13}}2\}$
$P(-1,0)$ $\implies$ $f(-1)^2+f(-1)-3=0$ and so $f(-1)\in\{\frac{-1-\sqrt{13}}2,\frac{-1+\sqrt{13}}2\}$

So $f(1)-f(-1)\in\{-\sqrt{13},0,\sqrt{13}\}$

But $P(1,-1)$ $\implies$ $f(1)-f(-1)=2$ and so contradiction.
So no solution in this case.
\end{solution}
*******************************************************************************
-------------------------------------------------------------------------------

\begin{problem}[Posted by \href{https://artofproblemsolving.com/community/user/125553}{lehungvietbao}]
	Let $a,b,c,d$ be positive real numbers. Find all increasing functions $f:\mathbb{R}\to\mathbb{R}$ such that
\[f(af(y)+bx)=cx+dy+1 \quad \forall x,y\in\mathbb{R}\]
	\flushright \href{https://artofproblemsolving.com/community/c6h568168}{(Link to AoPS)}
\end{problem}



\begin{solution}[by \href{https://artofproblemsolving.com/community/user/29428}{pco}]
	\begin{tcolorbox}Let $a,b,c,d$ be positive real numbers. Find all increasing functions $f:\mathbb{R}\to\mathbb{R}$ such that
\[f(af(y)+bx)=cx+dy+1 \quad \forall x,y\in\mathbb{R}\]\end{tcolorbox}
Let $P(x,y)$ be the assertion $f(af(y)+bx)=cx+dy+1$

$P(\frac abf(x),0)$ $\implies$ $f(af(0)+af(x))=\frac{ac}bf(x)+1$

$P(\frac abf(0),x)$ $\implies$ $f(af(x)+af(0))=\frac{ac}bf(0)+dx+1$

Subracting, we get $f(x)=\frac{bd}{ac}x+f(0)$

Plugging back in equation, we get the necessary condition $\boxed{b^2d=ac^2}$ and the solution $\boxed{f(x)=\frac{bd}{ac}x+\frac c{bd+c}}$ $\forall x$

And, btw, no need for "increasing" property.
\end{solution}
*******************************************************************************
-------------------------------------------------------------------------------

\begin{problem}[Posted by \href{https://artofproblemsolving.com/community/user/190536}{DonaldLove}]
	$a,b \in R$ such that $1<a^3<b^3$. find $f:(a,b) \to R$ such that $f(xyz)=f(x)+f(y)+f(z) \forall x,y,z \in (a,b) ,xyz \in (a,b)$
	\flushright \href{https://artofproblemsolving.com/community/c6h568194}{(Link to AoPS)}
\end{problem}



\begin{solution}[by \href{https://artofproblemsolving.com/community/user/29428}{pco}]
	\begin{tcolorbox}$a,b \in R$ such that $1<a^3<b^3$. find $f:(a,b) \to R$ such that $f(xyz)=f(x)+f(y)+f(z) \forall x,y,z \in (a,b) ,xyz \in (a,b)$\end{tcolorbox}
Note that if $a^3\ge b$, any function is a solution.
If $a^3<b$, then :
Let $g(x)=f(e^x)$ so that $g(x+y+z)=g(x)+g(y)+g(z)$ $\forall x,y,z$ such that $x,y,z,x+y+z\in (\ln a,\ln b)$

Let $a'\in(\frac b{a^2},b)$
Let $x<y\in(\lna,\ln b)$ such that $x+y<\ln b - \ln a'$ :
$g(x)+g(y)+g(a')=g(x+y+a')=g(\frac {x+y}2+\frac{x+y}2+g(a')=2g(\frac{x+y}2)+g(a')$ and so $g(\frac{x+y}2)=\frac{g(x)+g(y)}2$

With additional constraint of continuity or bounds, we get $g(x)=ux$ $\forall x\in(\ln a,\ln b-2\ln a)$ and $g(x)$ any value in $[\ln b-2\ln a,\ln b)$

And so : 
$f(x)=x^c$ $\forall x\in(a,\frac{b}{a^2})$ and $f(x)$ is any value $\forall x\in[\frac b{a^2},b)$

Without additional constraint of continuity or bounds, we'll get additive functions.
\end{solution}
*******************************************************************************
-------------------------------------------------------------------------------

\begin{problem}[Posted by \href{https://artofproblemsolving.com/community/user/201392}{nima-amini}]
	is there exit two function
 f,g :R --->R
for all x not equal to y
|f(x)-f(y)|+|g(x)-g(y)|>1
	\flushright \href{https://artofproblemsolving.com/community/c6h568215}{(Link to AoPS)}
\end{problem}



\begin{solution}[by \href{https://artofproblemsolving.com/community/user/29428}{pco}]
	\begin{tcolorbox}is there exit two function
 f,g :R --->R
for all x not equal to y
|f(x)-f(y)|+|g(x)-g(y)|>1\end{tcolorbox}
For any $k\in\mathbb Z$, let $A_k=\left[\frac k2,\frac {k+1}2\right)$ and $B_k=f^{-1}(A_k)$

Since $\bigcup B_k=\mathbb R$ is uncountable, at least one of $B_k$, say $B_n$ is uncountable.

Let $h(x)$ be the function from $B_n\to\mathbb Z$ such that $g(x)\in A_{h(x)}$

Since $B_n$ is uncountable, $h(x)$ can not be an injection and so $\exists x,y\in B_n$ such that $h(x)=h(y)$

Since $x,y\in B_n$, we get $f(x)\in A_n$ and $f(y)\in A_n$ and so $|f(x)-f(y)|<\frac 12$

Since  $h(x)=h(y)=m$, we get $g(x)\in A_m$ and $g(y)\in A_m$ and so $|g(x)-g(y)|<\frac 12$

And so $|f(x)-f(y)|+|g(x)-g(y)|<1$

So no such functions $f,g$
\end{solution}



\begin{solution}[by \href{https://artofproblemsolving.com/community/user/201392}{nima-amini}]
	can you explain more?
\end{solution}



\begin{solution}[by \href{https://artofproblemsolving.com/community/user/29428}{pco}]
	\begin{tcolorbox}can you explain more?\end{tcolorbox}
What part dont you understand ?

Maybe this more direct form will be easier to understand :

Let $h(x)$ from $\mathbb R\to\mathbb Z^2$ defined as $h(x)=(\lfloor 2f(x)\rfloor,\lfloor 2g(x)\rfloor)$

Since $\mathbb R$ is uncountable while $\mathbb Z^2$ is countable, $h(x)$ can not be injective and so $\exists x,y$ such that $h(x)=h(y)$ and so :

$\lfloor 2f(x)\rfloor=\lfloor 2f(y)\rfloor$ and so $|f(x)-f(y)|<\frac 12$
$\lfloor 2g(x)\rfloor=\lfloor 2g(y)\rfloor$ and so $|g(x)-g(y)|<\frac 12$

And so $|f(x)-f(y)|+|g(x)-g(y)|<1$
\end{solution}
*******************************************************************************
-------------------------------------------------------------------------------

\begin{problem}[Posted by \href{https://artofproblemsolving.com/community/user/125553}{lehungvietbao}]
	Let $f:\mathbb{R}\to \mathbb{R}$ be a continuous function. Prove that exist at least $a\in \mathbb{R}$ such that 
\[\left| {a - f\left( a \right)} \right| \le \left| {x - f\left( x \right)} \right| + \left| {a - x} \right| \quad \forall x\in \mathbb{R}\]
	\flushright \href{https://artofproblemsolving.com/community/c6h568380}{(Link to AoPS)}
\end{problem}



\begin{solution}[by \href{https://artofproblemsolving.com/community/user/29428}{pco}]
	\begin{tcolorbox}Let $f:\mathbb{R}\to \mathbb{R}$ be a continuous function. Prove that exist at least $a\in \mathbb{R}$ such that 
\[\left| {a - f\left( a \right)} \right| \le \left| {x - f\left( x \right)} \right| + \left| {a - x} \right| \quad \forall x\in \mathbb{R}\]\end{tcolorbox}
Let $g(x)$ any continuous function from $\mathbb R\to[0,+\infty)$
Let $h(x)=g(x)+|x|$

Since $h(x)$ is continuous, $\exists a\in[-h(0),+h(0)]$ such that $h(a)=\min_{x\in[-h(0),+h(0)]}h(x)$
So $h(x)\ge h(a)$ $\forall x\in[-h(0),+h(0)]$
$\forall x>h(0)$ : $h(x)\ge |x|>h(0)\ge h(a)$
$\forall x<-h(0)$ : $h(x)\ge|x|>h(0)\ge h(a)$
So $h(x)\ge h(a)$ $\forall x$
So $g(x)+|a-x|=h(x)+|a-x|-|x|\ge h(a)+|a-x|-|x|$ $=g(a)+|a|+|x-a|-|x|\ge g(a)$

Applying this to $g(x)=|x-f(x)|$, we get $|x-f(x)|+|a-x|\ge |a-f(a)|$ $\forall x$
Q.E.D.
\end{solution}
*******************************************************************************
-------------------------------------------------------------------------------

\begin{problem}[Posted by \href{https://artofproblemsolving.com/community/user/125553}{lehungvietbao}]
	Find all functions $f: \mathbb{R}\to \mathbb{R}$ such that \[f(x+f(y))=f(x)+\frac{2x}{9}f(3y)+f(f(y))\quad \forall x,y\in\mathbb R \]
	\flushright \href{https://artofproblemsolving.com/community/c6h568382}{(Link to AoPS)}
\end{problem}



\begin{solution}[by \href{https://artofproblemsolving.com/community/user/29428}{pco}]
	\begin{tcolorbox}Find all functions $f: \mathbb{R}\to \mathbb{R}$ such that \[f(x+f(y))=f(x)+\frac{2x}{9}f(3y)+f(f(y))\quad \forall x,y\in\mathbb R \]\end{tcolorbox}
$\boxed{f(x)=0}$ $\forall x$ is a solution. So let us from now look only for non allzero solutions.

Let $P(x,y)$ be the assertion $f(x+f(y))=f(x)+\frac{2x}9f(3y)+f(f(y))$
Let $u$ such that $f(u)=v\ne 0$

Subtracting $P(f(x),u)$ from $P(f(u),x)$, we get $f(3x)=kf(x)$ where $k=\frac{f(3u)}{f(u)}$
Note that $v=f(u)=kf(\frac u3)$ implies $k\ne 0$
$P(x,y)$ may then be written $f(x+f(y))=f(x)+\frac{2x}9kf(y)+f(f(y))$

$P(f(x),x)$ $\implies$ $f(2f(x))=2f(f(x))+\frac{2}9kf(x)^2$
$P(2f(x),x)$ $\implies$ $f(3f(x))=3f(f(x))+\frac{2}3kf(x)^2$
Then, since $f(3f(x)))=kf(f(x))$, we get $(k-3)f(f(x))=\frac{2}3kf(x)^2$

If $k=3$, this implies $f(x)=0$ $\forall x$, impossible. So $k\ne 3$ and  $f(f(x))=\frac{2k}{3(k-3)}f(x)^2$ $\forall x$

$P(-f(x),x)$ $\implies$ $f(-f(x))=-\frac{2k(6-k)}{9(k-3)}f(x)^2$
$P(-2f(x),x)$ $\implies$ $f(-2f(x))=\frac{2k(k-5)}{3(k-3)}f(x)^2$
$P(-3f(x),x)$ $\implies$ $f(-3f(x))=\frac{2k(2k-9)}{3(k-3)}f(x)^2$
Then, since $f(-3f(x))=kf(-f(x))$, we got $\frac{2k(2k-9)}{3(k-3)}f(x)^2=-\frac{2k^2(6-k)}{9(k-3)}f(x)^2$
And so $k^2-12k+27=0$ and, since $k\ne 3$ : $k=9$ 

And $P(x,y)$ becomes $f(x+f(y))=f(x)+2xf(y)+f(y)^2$


$P(\frac{x-f(v)}{2v},u)$ $\implies$ $f(\frac{x-f(v)}{2v}+v)-f(\frac{x-f(v)}{2v})=x$ and so any real may be written as $x=f(a)-f(b)$ for some $a,b$

$P(-f(b),b)$ $\implies$ $f(-f(b))=f(b)^2$

$P(-f(b),a)$ $\implies$ $f(f(a)-f(b))=f(b)^2-2f(b)f(a)+f(a)^2$ $=(f(a)-f(b))^2$

And so $\boxed{f(x)=x^2}$ $\forall x$, which indeed is a solution.
\end{solution}
*******************************************************************************
-------------------------------------------------------------------------------

\begin{problem}[Posted by \href{https://artofproblemsolving.com/community/user/185327}{shadow10}]
	Find all continuous function $f,g,h:\mathbb{R} \to \mathbb{R}$ which satisfy the following equation,
           $f(x+y)-f(x-y)=2 g(x)h(y)$
	\flushright \href{https://artofproblemsolving.com/community/c6h568387}{(Link to AoPS)}
\end{problem}



\begin{solution}[by \href{https://artofproblemsolving.com/community/user/29428}{pco}]
	\begin{tcolorbox}Find all continuous function $f,g,h:\mathbb{R} \to \mathbb{R}$ which satisfy the following equation,
           $f(x+y)-f(x-y)=2 g(x)h(y)$\end{tcolorbox}
Let $P(x,y)$ be the assertion $f(x+y)-f(x-y)=2g(x)h(y)$

If $g(x)=0$ $\forall x$, then we get the solution $f(x)=c$, $g(x)=0$, and $h(x)$ is any continuous function.
If $h(x)=0$ $\forall x$, then we get the solution $f(x)=c$, $g(x)$ is any continuous function. and $h(x)=0$ 
So let us consider from now that $g(x),h(x)$ are not the allzero function.

1) If $g(t)=0$ for some $t$ 
=================
$(f(x),g(x),h(x))$ solution implies $(f(x+t),g(x+t),h(x))$ solution. So WLOG $t=0$
Let $u,v\ne 0$ such that $g(u)=v\ne 0$

$P(0,x-u)$ $\implies$ $f(x-u)=f(u-x)$
$P(x,u)$ $\implies$ $f(x+u)-f(x-u)=2g(x)h(u)$
$P(u,x)$ $\implies$ $f(x+u)-f(u-x)=2vh(x)$
And so $h(x)=\alpha g(x)$ where $\alpha=\frac{h(u)}v$. Note that $\alpha\ne 0$ since $h(x)$ is not the allzero function.

Equation becomes $f(x+y)-f(x-y)=2\alpha g(x)g(y)$ and so  $f(2x)=f(0)+2\alpha g(x)^2$
Equation becomes then $g(\frac{x+y}2)^2-g(\frac{x-y}2)^2=g(x)g(y)$ or also $g(x)^2-g(y)^2=g(x+y)g(x-y)$

This equation has already be solved in this forum (see for example http://www.artofproblemsolving.com/Forum/viewtopic.php?f=36&p=3321414#p3321414) and a rather long proof gives the three solutions :
$g(x)=cx$
$g(x)=c\sin ax$
$g(x)=c\sinh ax$

And so Solutions of our problem :
$f(x)=\frac{c}2(ax+b)^2+k$ and $g(x)=ax+b$ and $h(x)=acx$
$f(x)=k-c^2d\cos(ax+b)$ and $g(x)=c\sin(ax+b)$ and $h(x)=cd\sin ax$
$f(x)=c^2d\cosh(ax+b)+k$ and $g(x)=c\sinh(ax+b)$ and $h(x)=cd\sinh ax$


2) If $g(x)\ne 0$ $\forall x$
=================
$(f(x),g(x),h(x))$ solution implies $(f(x),\frac{g(x)}{2g(0)},2g(0)h(x))$ solution too. So WLOG $g(0)=\frac 12$ and $g(x)>0$ $\forall x$

$P(0,x)$ $\implies$ $h(x)=f(x)-f(-x)$ and so $h(0)=0$ and $h(x)$ is an odd function.

2.1) $h(x)\ne 0$ $\forall x\ne 0$ and WLOG $h(x)>0$ $\forall x>0$
-------------------------------------------------------------------------
If $h(u)=0$ for some $u\ne 0$, then :
$P(x+u,u)$ $\implies$ $f(x+2u)=f(x)$. So $f(u+2ku)=f(u)$ and $f(u-2ku)=f(u)$ and then 
$P(u,2ku)$ $\implies$ $h(2ku)=0$ $\forall k\in\mathbb Z$

Let $A=\{x>0$ such that $h(x)=0\}$ and $a=\inf(A)$.
If $a=0$, then continuity and property $h(2ku)=0$ $\forall k\in\mathbb Z$ imply $h(x)=0$ $\forall x$, impossible.
So $a>0$
$h(x)$ odd and $h(x+2a)=h(x)$ imply then that $h(x)$ and $h(2x)$ are non zero real numbers with opposite signs $\forall x\in(0,a)$
Let then $t\in(0,a)$

$P(t,-t)$ $\implies$ $f(0)-f(2t)=-2g(t)h(t)$
$P(-t,t)$ $\implies$ $f(-2t)-f(0)=-2g(-t)h(t)$
$P(0,2t)$ $\implies$ $f(2t)-f(-2t)=h(2t)$
Adding, we get $h(2t)=2h(t)(g(t)+g(2t))$ and so contradiction since $h(t)$ and $h(2t$ have opposite signs while $g(t)+g(-t)>0$

So $h(x)\ne 0$ $\forall x\ne 0$
$(f(x),g(x),h(x))$ solution implies $(-f(x),g(x),-h(x))$ solution too. So WLOG $h(x)>0$ $\forall x>0$
Q.E.D.

2.2) $\exists u(x)>0$ continuous even function such that $g(x+y)+g(x-y)=g(x)u(y)$ $\forall x,y$
-------------------------------------------------------------------------------------------------------
1) : $P(x+y,y)$ $\implies$ $f(x+2y)-f(x)=2g(x+y)h(y)$
2) : $P(x-y,y)$ $\implies$ $f(x)-f(x-2y)=2g(x-y)h(y)$
3) : $P(x,2y)$ $\implies$ $f(x+2y)-f(x-2y)=2g(x)h(2y)$
1)+2)-3) $\implies$ $g(x+y)+g(x-y)=g(x)\frac{h(2y)}{h(y)}$ $\forall y\ne 0$
Note that this implies $\lim_{y\to 0}\frac{h(2y)}{h(y)}=2$

Setting $u(y)=\frac{h(2y)}{h(y)}$ $\forall y\ne 0$ and $u(0)=2$, we get the required result.
Q.E.D.

2.3) $u(x)\ge 2$ $\forall x$
-------------------------
If $u(y)< 2$ for some $y$, let $u(y)=2\cos \alpha$ (where $\alpha\in(0,\frac{\pi}2)$ is a function of $y$)

Equation $g(x+y)+g(x-y)=g(x)u(y)$ implies $g((n+2)y)=2 g((n+1)y)\cos\alpha-g(ny)$

And so $g(ny)=a\cos(n\alpha+\beta)$ for some $a,\beta$, which is impossible since we need $g(ny)>0$ $\forall n$
So $u(x)\ge 2$ $\forall x$
Q.E.D.

2.4) If $u(t)>2$ for some $t$, then $g(x)$ is unbounded
----------------------------------------------------------------
Consider the quadratic $x^2-u(t)x+1=0$. It has two real roots $r>1$ and $\frac 1r$

Equation $g(x+t)+g(x-t)=g(x)u(t)$ implies $g((n+2)t)= g((n+1)y)g(t)-g(nt)$ and so $g(nt)=ar^n+br^{-n}$ for some $a,b\ge 0$ not both zero.
Setting either $n\to+\infty$, either $n\to -\infty$, we get the result.
Q.E.D.


2.5) If $u(t)=2$ for some $t\ne 0$, then $g(x)=\frac 12$ $\forall x$
--------------------------------------------------------------------------
Equation $g(x+t)+g(x-t)=2g(x)$ implies $g(x+nt)=g(x)+n(g(x+t)-g(x))$ and so, since $>0$ $\forall n$, $g(x+t)=g(x)$
So $g(x)$ is periodic and so is bounded (since continuous).
So $u(y)=2$ $\forall y$, according to 2.4) above.
So $g(x+y)=g(x)$ $\forall x,y$ 
Hence the result (remember we said WLOG $g(0)=\frac 12$)

Back to original equation :
$P(x+\frac y2,\frac y2)$ $\implies$ $f(x+y)-f(x)=h(\frac y2)$
$P(\frac y2,\frac y2)$ $\implies$ $f(y)-f(0)=h(\frac y2)$
So $f(x+y)=f(x)+f(y)-f(0)$ and so, since continuous, $f(x)=ux+v$

Hence the solution : $f(x)=\frac {ab}2x+c$ and $g(x)=a$ and $h(x)=bx$

2.6) If $u(t)>2$ for some $t$ then $g(x)=\frac 1{2\cosh b}\cosh(ax+b)$ and $u(x)=2\cosh(ax)$
----------------------------------------------------------------------------------------------------------
Looking at 2.4) and 2.5) above, we get $u(x=>2$ $\forall x$ and $g(x)$ is unbounded.
Let then $t_n=\frac 1{2^n}$.

We got $g(kt_n)=a_nr_n^k+b_nr_n^{-k}$ for some $a_n,b_n$ and $r_n>1$ root of $x^2-u(t_n)x+1=0$

Setting $k=0$, we get $a_n+b_n=\frac 12$ and so $g(kt_n)=a_nr_n^k+(\frac 12-a_n)r_n^{-k}$

Setting $k=2^n$, we get $g(1)=a_nr_n^{2^n}+(\frac 12-a_n)r_n^{-2^n}$
Setting $k=2^{n+1}$, we get $g(2)=a_nr_n^{2^{n+1}}+(\frac 12-a_n)r_n^{-2^{n+1}}$

Writing $r_n^{2^n}=u_n$, this becomes :
$g(1)=a_nu_n+(\frac 12-a_n)\frac 1{u_n}$
$g(2)=a_nu_n^2+(\frac 12-a_n)\frac 1{u_n^2}$
This implies that $a_n$ and $u_n$ do not depend on $n$ and we get $a_n=a$ and $u_n=u$ and so :

$g(\frac k{2^n})=au^{\frac k{2^n}}+(\frac 12-a)u^{-\frac k{2^n}}$
Setting $n\to+\infty$ and using continuity, we get $g(x)=au^x+(\frac 12-a)u^{-x}$ with $a,2-a\ge 0$

And so the solution : $g(x)=\frac 1{2\cosh b}\cosh(ax+b)$ and $u(x)=2\cosh(ax)$

2.7) Last step : final solution of original equation when $u(t)>2$
---------------------------------------------------------------------
Original equation is $f(x+y)-f(x-y)=\frac 1{\cosh b}\cosh(ax+b)h(y)$

Setting $x=0$, we get $f(y)-f(-y)=h(y)$ and so $P(x,y)$ becomes $f(x+y)-f(x-y)=\frac 1{\cosh b}\cosh(ax+b)(f(y)-f(-y))$

If $\exists t\ne 0$ such that $f(t)=f(0)$, $P(\frac t2,\frac t2)$ $\implies$ $f(\frac t2)-f(-\frac t2)=0$

And so $P(x+\frac t2,\frac t2)$ $\implies$ $f(x+t)=f(x)$ 

Comparing then $P(x+t,y)$ and $P(x,y)$, we get $f(x)=c$ constant

Note then that $f(x)$ solution implies $f_1(x)=f(x)+\alpha \sinh (ax+b)$ is solution too.

Choosing then $\alpha=\frac{f(0)-f(1)}{\sinh (a+b)-\sinh(b)}$, we get $f_1(1)=f_1(0)$ and so $f_1(x)=$ constant.

So $f(x)=c\sinh(ax+b)+d$ and $h(x)=2c\cosh b\sinh ax$

Hence (getting rid of all WLOGs) :
$f(x)=c_1c_2\sinh(ax+b)+d$ and $g(x)=c_1\cosh(ax+b)$ and $h(x)=c_2\sinh ax$

3) Synthesis of results :
================
We got seven families of solutions :

$f(x)=c$, $g(x)=0$, and $h(x)$ is any continuous function.
$f(x)=c$, $g(x)$ is any continuous function. and $h(x)=0$
$f(x)=\frac{c}2(ax+b)^2+k$ and $g(x)=ax+b$ and $h(x)=acx$
$f(x)=k-c^2d\cos(ax+b)$ and $g(x)=c\sin(ax+b)$ and $h(x)=cd\sin ax$
$f(x)=c^2d\cosh(ax+b)+k$ and $g(x)=c\sinh(ax+b)$ and $h(x)=cd\sinh ax$
$f(x)=\frac {ab}2x+c$ and $g(x)=a$ and $h(x)=bx$
$f(x)=c_1c_2\sinh(ax+b)+d$ and $g(x)=c_1\cosh(ax+b)$ and $h(x)=c_2\sinh ax$
\end{solution}
*******************************************************************************
-------------------------------------------------------------------------------

\begin{problem}[Posted by \href{https://artofproblemsolving.com/community/user/68025}{Pirkuliyev Rovsen}]
	Suppose $f: \mathbb{R}\to\mathbb{R}$ is a decreasing, continious function.Find all such functions for which $f(x+y)+f(f(x^2)+f(y^2))=f(f(x)+f(y)+f(y^2+f(x^2))$.
	\flushright \href{https://artofproblemsolving.com/community/c6h568502}{(Link to AoPS)}
\end{problem}



\begin{solution}[by \href{https://artofproblemsolving.com/community/user/29428}{pco}]
	\begin{tcolorbox}Suppose $f: \mathbb{R}\to\mathbb{R}$ is a decreasing, continious function.Find all such functions for which $f(x+y)+f(f(x^2)+f(y^2))=f(f(x)+f(y)+f(y^2+f(x^2))$.\end{tcolorbox}
Missing parenthesis in RHS.
\end{solution}



\begin{solution}[by \href{https://artofproblemsolving.com/community/user/68025}{Pirkuliyev Rovsen}]
	Sorry  $f(x+y)+f(f(x^2)+f(y^2))=f(f(x)+f(y)+f(y^2+f(x^2)))$
\end{solution}



\begin{solution}[by \href{https://artofproblemsolving.com/community/user/29428}{pco}]
	\begin{tcolorbox}Suppose $f: \mathbb{R}\to\mathbb{R}$ is a decreasing, continious function.Find all such functions for which $f(x+y)+f(f(x^2)+f(y^2))=f(f(x)+f(y)+f(y^2+f(x^2)))$.\end{tcolorbox}
I understand "decreasing" as $x\le y$ $\implies$ $f(x)\ge f(y)$
Let $P(x,y)$ be the assertion $f(x+y)+f(f(x^2)+f(y^2))=f(f(x)+f(y)+f(y^2+f(x^2)))$

Subtracting $P(x,-x)$ from $P(x,x)$, we get $f(2x)-f(0)=f(f(x)+f(x)+f(x^2+f(x^2)))-f(f(x)+f(-x)+f(x^2+f(x^2)))$

If $x\ge 0$ :
$2x\ge 0$ $\implies$ $LHS \le  0$
$x\ge -x$ $\implies$ $f(x)\le f(-x)$ $\implies$ $f(x)+f(x)+f(x^2+f(x^2))\le f(x)+f(-x)+f(x^2+f(x^2))$ $\implies$ $RHS\ge 0$
So $LHS=RHS=0$ and so $f(x)=f(0)$ $\forall x\ge 0$

If $x\le 0$ :
$2x\le 0$ $\implies$ $LHS \ge  0$
$x\le -x$ $\implies$ $f(x)\ge f(-x)$ $\implies$ $f(x)+f(x)+f(x^2+f(x^2))\ge f(x)+f(-x)+f(x^2+f(x^2))$ $\implies$ $RHS\le 0$
So $LHS=RHS=0$ and so $f(x)=f(0)$ $\forall x\le 0$

So $f(x)$ is constant and plugging this in original equation gives the unique solution $\boxed{f(x)=0}$ $\forall x$
\end{solution}
*******************************************************************************
-------------------------------------------------------------------------------

\begin{problem}[Posted by \href{https://artofproblemsolving.com/community/user/192463}{arkanm}]
	Find all functions $f:\mathbb{R}\to \mathbb{R}$  such that : \[f(xy)+f(x)+f(y)=f(x)f(y)+x^2f(y)+y^2f(x)+f(x+y) \quad \forall x,y\in\mathbb R\]
	\flushright \href{https://artofproblemsolving.com/community/c6h568563}{(Link to AoPS)}
\end{problem}



\begin{solution}[by \href{https://artofproblemsolving.com/community/user/29428}{pco}]
	\begin{tcolorbox}Find all functions $f:\mathbb{R}\to \mathbb{R}$  such that : \[f(xy)+f(x)+f(y)=f(x)f(y)+x^2f(y)+y^2f(x)+f(x+y) \quad \forall x,y\in\mathbb R\]\end{tcolorbox}
Let $P(x,y)$ be the assertion $f(xy)+f(x)+f(y)=f(x)f(y)+x^2f(y)+y^2f(x)+f(x+y)$

$P(x,0)$ $\implies$ $f(0)(f(x)-2+x^2)=0$
If $f(0)\ne 0$, this gives $f(x)=2-x^2$ $\forall x$, which unfortunately is not a solution. So $f(0)=0$.

Let $a=f(1)$
$P(x,1)$ $\implies$ $f(x+1)=f(x)(1-a)+a-ax^2$

If $a=1$ this gives $\boxed{f(x)=2x-x^2}$ $\forall x$, which indeed is a solution.
So let us from look for solutions where $a\ne 1$

$P(1,1)$ $\implies$ $f(2)=a-a^2$
$P(2,2)$ $\implies$ $a(a+2)(a-3)=0$ and so $a\in\{-2, 0, 3\}$

1) If $a=0$, we get the solution $f(x)=0$ $\forall x$
===============================
Simple inductions give :
$f(x+n)=f(x)$ $\forall x$, $\forall n\in\mathbb Z$
$f(n)=0$ $\forall n\in\mathbb Z$
$f(nx)=n^2f(x)$ $\forall x$, $\forall n\in\mathbb Z$

Then $P(x,x)$ $\implies$ $f(x^2)=f(x)^2+2(x^2+1)f(x)$
And $P(2x,2x)$ $\implies$ $f(4x^2)=16f(x)^2+8(4x^2+1)f(x)$
But $f(4x^2)=16f(x^2)$ and so $16f(x)^2+8(4x^2+1)f(x)=16(f(x)^2+2(x^2+1)f(x))$ and so $\boxed{f(x)=0}$ $\forall x$, which indeed is a solution.
Q.E.D.


2) If $a=3$, we get no solution
=================
If $a=3$, we get $f(2)=-6$ and :
$P(2,1)$ $\implies$ $f(3)=3$
$P(3,1)$ $\implies$ $f(4)=-30$
$P(4,1)$ $\implies$ $f(5)=15$
$P(5,1)$ $\implies$ $f(6)=-102$
And then $P(2,3)$ is false. 
Q.E.D.

3) If $a=-2$, we get the soluton $f(x)=-x^2-x$
==========================
If $a=-2$
We easily get now with induction on $P(x,1)$ that :
$f(x+n)+(x+n)+(x+n)^2=3^n (f(x)+x^2+x)$ with special case when $x=0$ : $f(n)=-n-n^2$ $\forall n\in\mathbb Z$
Then, $P(x,n)$ $\implies$ $f(nx)+n^2x^2+nx=(3^n-n-1)(f(x)+x^2+x)$

Writing $g(x)=f(x)+x+x^2$, this becomes :
$g(n)=0$ $\forall n\in\mathbb Z$
$g(x+n)=3^ng(x)$
$g(nx)=(3^n-n-1)g(x)$

So $g(2x)=6g(x)$ and $g(3x)=23g(x)$ and so $g(6x)=138g(x)$
But direct application of $g(nx)$ for $n=6$ gives $g(6x)=722 g(x)$

So $g(x)=0$ $\forall x$
And $\boxed{f(x)=-x^2-x}$ $\forall x$, which indeed is a solution.
Q.E.D.
\end{solution}
*******************************************************************************
-------------------------------------------------------------------------------

\begin{problem}[Posted by \href{https://artofproblemsolving.com/community/user/125553}{lehungvietbao}]
	Determine all functions $f: \mathbb R\to\mathbb R$ differentiable at $0$ satisfying \[f (2x) = \frac{2f(x)}{(f (x))^2 +1} \quad \forall x\in\mathbb R\]
	\flushright \href{https://artofproblemsolving.com/community/c6h568650}{(Link to AoPS)}
\end{problem}



\begin{solution}[by \href{https://artofproblemsolving.com/community/user/29428}{pco}]
	\begin{tcolorbox}Determine all functions $f: \mathbb R\to\mathbb R$ differentiable at $0$ satisfying \[f (2x) = \frac{2f(x)}{(f (x))^2 +1} \quad \forall x\in\mathbb R\]\end{tcolorbox}
Let $P(x)$ be the assertion $f(2x)=\frac{2f(x)}{f(x)^2+1}$

If $f(u)=1$ for some $u$, then $P(\frac u2)$ $\implies$ $f(\frac u2)=1$ and so $f(\frac u{2^n})=1$ and so $f(0)=1$
If $f(v)=-1$ for some $v$, then $P(\frac v2)$ $\implies$ $f(\frac v2)=1$ and so $f(\frac v{2^n})=1$ and so $f(0)=-1$
$P(0)$ $\implies$ $f(0)\in\{-1,0,+1\}$

1) If $f(0)=-1$, then $f(x)=-1$ $\forall x$
=========================
Then $f(x)\ne 1$ $\forall x$ and let $g(x)=\frac 2{1-f(x)}-1$ and so $g(0)=0$ and equation becomes : $g(2x)=g(x)^2$
So $g(x)^{2^{-n}}=g(\frac x{2^n})$

If $g(x)\ne 0$ for some $x$, setting $n\to +\infty$ in this last line implies contradiction ($LHS\to 1$ while $RHS\to 0$)
So $g(x)=0$ $\forall x$
So $\boxed{f(x)=-1}$ $\forall x$, which indeed is a solution.
Q.E.D.

2) If $f(0)=0$, then $f(x)=\tanh x$ $\forall x$
=============================
Then $f(x)\ne 1$ $\forall x$ and let $g(x)=\frac 2{1-f(x)}-1$ and so $g(0)=1$ and equation becomes : $g(2x)=g(x)^2\ge 0$

If $g(x)=0$ for some $x$, then $g(\frac x2)=0$ and so, $g(0)=0$, impossible. So $g(x)>0$ $\forall x$

Writing then $g(x)=e^{h(x)}$ with $h(0)=0$ and $h(x)$ differentiable at $0$, we get $h(2x)=2h(x)$ and so $\frac{h(x)}x=\frac{h(\frac x{2^n})}{\frac x{2^n}}$

And so $h(x)=xh'(0)$ (here we use the fact that $h(x)$ is differentiable at $0$)
And so $\boxed{f(x)=\tanh ax}$ $\forall x$, which indeed is a solution.
Q.E.D.

3) If $f(0)=1$, then $f(x)=1$ $\forall x$
=========================
So $f(x)\ne -1$ $\forall x$
If $f(u)\ne 1$ for some $u$, then $f(2^nu)\ne 1$ $\forall n\in\mathbb Z$ and we can then write $f(2^nu)=\frac{u_n+1}{u_n-1}$ for some $u_n\notin\{0,1\}$

Equation implies $u_{n+1}=u_n^2$ $\forall n\in\mathbb Z$ and so $u_n>0$ $\forall n\in\mathbb Z$

So $u_{-n}=u_0^{2^{-n}}$ and so $\lim_{n\to -\infty}u_n=1$ and so $\lim_{n\to-\infty}|f(2^nu)|=+\infty$, in contradiction with the fact that $f(x)$ is continuous at $0$ and that $f(0)=1$
So no such $u$  and so $\boxed{f(x)=1}$ $\forall x$, which indeed is a solution.
Q.E.D.
\end{solution}
*******************************************************************************
-------------------------------------------------------------------------------

\begin{problem}[Posted by \href{https://artofproblemsolving.com/community/user/125553}{lehungvietbao}]
	Find all real-valued functions defined for integers and such that:
\[f (x + y) + f (1) + f (xy) = f (x) + f (y) + f (1 + xy)\quad \forall x,y\in\mathbb{Z}\]
	\flushright \href{https://artofproblemsolving.com/community/c6h568652}{(Link to AoPS)}
\end{problem}



\begin{solution}[by \href{https://artofproblemsolving.com/community/user/29428}{pco}]
	\begin{tcolorbox}Find all real-valued functions defined for integers and such that:
\[f (x + y) + f (1) + f (xy) = f (x) + f (y) + f (1 + xy)\quad \forall x,y\in\mathbb{R}\]\end{tcolorbox}
There is a contradiction between : "defined for integers"  and "$\forall x,y\in\mathbb R$"
Please check.
\end{solution}



\begin{solution}[by \href{https://artofproblemsolving.com/community/user/29428}{pco}]
	\begin{tcolorbox}Find all real-valued functions defined for integers and such that:
\[f (x + y) + f (1) + f (xy) = f (x) + f (y) + f (1 + xy)\quad \forall x,y\in\mathbb{Z}\]\end{tcolorbox}
I modified the domain of functional equation from $\mathbb R\to\mathbb Z$ in order the statement remains consistent.

Let $P(x,y)$ be the assertion $f(x+y)+f(1)+f(xy)=f(x)+f(y)+f(1+xy)$

1) $f(x)$ is completely known if we know $f(0),f(1),f(2),f(3),f(4),f(6)$
======================================================
Let $x\ge 2$ :
$P(x,2)$ $\implies$ $f(2x+1)=f(2x)+f(x+2)-f(x)+f(1)-f(2)$ and so any positive odd integer is known from the given numbers

Let $x\ge 4$ :
Subtracting $P(x-1,4)$ from $P(2x-2,2)$, we get : $f(2x)=f(2x-2)+f(x+3)-f(x-1)+f(2)-f(4)$ and so any positive even integer is known from the given numbers

Let $h(x)=f(x)-f(-x)$
Subtracting $P(-x,-y)$ from $P(x,y)$, we get $h(x+y)=h(x)+h(y)$ and so $h(x)=xh(1)$
So $f(-x)=f(x)-x(f(1)-f(-1))$ and so any negative integer may be known from $f(-1)$ and the given numbers

$P(-2,2)$ $\implies$ $f(0)+f(1)+f(-4)=f-2)+f(2)+f(-3)$
$P(-1,4)$ $\implies$ $f(3)+f(1)+f(-4)=f(-1)+f(4)+f(-3)$
Subtracting, we get : $-f(-1)+f(-2)+f(3)+f(2)-f(4)-f(0)=0$
And since $f(-2)=f(2)-2(f(1)-f(-1))$, we get $f(-1)=f(0)-2f(2)+2f(1)-f(3)+f(4)$
And so $f(-1)$ may be known from the given numbers.

Q.E.D.

2) $f(x)$ is completely known if we know $f(0),f(1),f(2),f(3),f(4)$
==================================================
$f(-6)=f(6)-6(f(1)-f(-1))$
$f(-5)=f(5)-5(f(1)-f(-1))$
$f(-2)=f(2)-2(f(1)-f(-1))$
$f(-1)=f(0)-2f(2)+2f(1)-f(3)+f(4)$
$P(2,2)$ $\implies$ $f(5)=f(1)-2f(2)+2f(4)$
$P(-2,3)$ $\implies$ $2f(1)+f(-6)=f(-2)+f(3)+f(-5)$ and so 

$f(6)=f(0)-3f(2)+3f(4)$

And so $f(6)$ is completely known if we know $f(0),f(1),f(2),f(3),f(4)$
Q.E.D.


3) The set of solutions is a $\mathbb R$-vectorspace with dimension $5$
==========================================
$f(x)$ solution and $g(x)$ solution imply $af(x)+bg(x)$ solution. So the set of solution is a $\mathbb R$-vectorspace.

Point 2) above implies that dimension is at most $5$

Hereunder are 5 independant solutions :
$b_1(n)=1$ $\forall n$
$b_2(n)=1$ if $n\equiv 0\pmod 2$ and $b_2(n)=0$ otherwise
$b_3(n)=1$ if $n\equiv 0\pmod 3$ and $b_3(n)=0$ otherwise
$b_4(n)=n$
$b_5(n)=n^2$

So we got a basis.
Q.E.D

4) General solution
==============
General solution is $\boxed{f(n)=an^2+bn+c+\alpha b_2(n)+\beta b_3(n)}$ where :
$b_2(n)=1$ if $n\equiv 0\pmod 2$ and $b_2(n)=0$ otherwise
$b_3(n)=1$ if $n\equiv 0\pmod 3$ and $b_3(n)=0$ otherwise
\end{solution}
*******************************************************************************
-------------------------------------------------------------------------------

\begin{problem}[Posted by \href{https://artofproblemsolving.com/community/user/198085}{nik9796}]
	Find all $f:R\rightarrow R$ such that $(f(x)+f(y))(f(z)+f(t))=f(xz-yt)+f(xt+yz)$ where $x,y,z,t \in R$
	\flushright \href{https://artofproblemsolving.com/community/c6h568693}{(Link to AoPS)}
\end{problem}



\begin{solution}[by \href{https://artofproblemsolving.com/community/user/165750}{mathdebam}]
	looks like a solution is $f(x)=x^2$
\end{solution}



\begin{solution}[by \href{https://artofproblemsolving.com/community/user/198085}{nik9796}]
	I got three solutions.
$f(x)=0$,$f(x)=\frac{1}{2}$ and $f(x)=x^2$
My solution was quite lenghty so I'm looking for a more elegant solution
\end{solution}



\begin{solution}[by \href{https://artofproblemsolving.com/community/user/172163}{joybangla}]
	http://www.artofproblemsolving.com/Forum/viewtopic.php?p=118703&sid=6cf4a69d4cf861fda0f5f3bd773f4d67#p118703   an IMO problem.Should have checked the contests section before posting.
\end{solution}



\begin{solution}[by \href{https://artofproblemsolving.com/community/user/29428}{pco}]
	\begin{tcolorbox}Find all $f:R\rightarrow R$ such that $(f(x)+f(y))(f(z)+f(t))=f(xz-yt)+f(xt+yz)$ where $x,y,z,t \in R$\end{tcolorbox}
Let $P(x,y,z,t)$ be the assertopn $(f(x)+f(y))(f(z)+f(t))=f(xz-yt)+f(xt+yz)$

$P(0,0,0,0)$ $\implies$ $f(0)(f(0)-\frac 12)=0$ and so $f(0)\in\{0,\frac 12\}$

If $f(0)=\frac 12$, $P(x,0,0,0)$ $\implies$ $\boxed{f(x)=\frac 12}$ $\forall x$, which indeed is a solution
Let us from now consider $f(0)=0$

$P(1,0,1,0)$ $\implies$ $f(1)(f(1)-1)=0$ and so $f(1)\in\{0,1\}$

If $f(1)=0$, $P(x,0,1,0)$ $\implies$ $\boxed{f(x)=0}$ $\forall x$, which indeed is a solution
Let us from now consider $f(1)=1$

$P(1,1,1,1)$ $\implies$ $f(2)=4$
$P(x+1,1,1,1)$ $\implies$ $f(x+2)=2f(x+1)-f(x)+2$ and quick induction gives $f(n)=n^2$ $\forall n\in\mathbb Z$

$P(x,0,y,0)$ $\implies$ $f(xy)=f(x)f(y)$ and so :
$P(x,y,z,t)$ may be written $f(xz)+f(xt)+f(yz)+f(yt)=f(xz-yt)+f(xt+yz)$
$f(nx)=n^2f(x)$ $\forall n\in\mathbb Z$
$f(px)=p^2f(x)$ $\forall n\in\mathbb Q$
$f(-x)=f(x)$
$f(x)=x^2$ $\forall x\in\mathbb Q$
$\forall x\ge 0$ : $f(x)=f(\sqrt x)^2\ge 0$

Let $u,v>0$ :
$P(\sqrt{uv},v,1,\sqrt{\frac uv})$ $\implies$ $f(u+v)=f(u)+f(v)+2f(\sqrt{uv})\ge f(u)$ and so $f(x)$ is non decreasing over $\mathbb R^+$

So $f(x)=x^2$ $\forall x\ge 0$
So, since even,  $\boxed{f(x)=x^2}$ $\forall x$, which indeed is a solution.
\end{solution}
*******************************************************************************
-------------------------------------------------------------------------------

\begin{problem}[Posted by \href{https://artofproblemsolving.com/community/user/125553}{lehungvietbao}]
	Fimd all strictly increasing  functions $f: \mathbb{N^*} \setminus \{1 \} \to \mathbb{R}$ such that $f(3)=9$ and 
\[\left\{\begin{matrix}f(xy-x-y+2)-f(x)-f(y)+8=0 \\ 2^{f(x)-8} \in \mathbb{N^*} \end{matrix}\right.\quad \forall x,y \in \mathbb{N^*} \setminus \{1 \}\]
Where $ \mathbb{N^*} =\{1,2,..,\}$ is set of positive integers.
	\flushright \href{https://artofproblemsolving.com/community/c6h568772}{(Link to AoPS)}
\end{problem}



\begin{solution}[by \href{https://artofproblemsolving.com/community/user/29428}{pco}]
	\begin{tcolorbox}Fimd all strictly increasing  functions $f: \mathbb{N^*} \setminus \{1 \} \to \mathbb{R}$ such that $f(3)=9$ and 
\[\left\{\begin{matrix}f(xy-x-y+2)-f(x)-f(y)+8=0 \\ 2^{f(x)-8} \in \mathbb{N^*} \end{matrix}\right.\quad \forall x,y \in \mathbb{N^*} \setminus \{1 \}\]
Where $ \mathbb{N^*} =\{1,2,..,\}$ is set of positive integers.\end{tcolorbox}
Let $g(x)$, strictly increasing from the set of positive integers to $\mathbb R$ defined as $g(x)=f(x+1)-8$
Equation is :

$g(2)=1$
$g(xy)=g(x)+g(y)$ $\forall x,y$ positive integers
$2^{g(x)}$ positive integer $\forall x$ positive integer.

Note that :
Last condition implies $g(x)\ge 0$ $\forall x$
Second condition implies $g(1)=0$
"strictly increasing" condition implies $g(x)>0$ $\forall x>1$ positive integer.

Let $p,q$ positive integers $>1$ (so that $g(p), g(q)>0$)

If $\frac {\ln p}{\ln q}>\frac mn$ for some positive integers $m,n$, then $p^n>q^m$, then $g(p^n)>g(q^m)$ and so $\frac {g(p)}{g(q)}>\frac mn$
So $\frac {g(p)}{g(q)}\ge\frac {\ln p}{\ln q}$

If $\frac {\ln p}{\ln q}<\frac mn$ for some positive integers $m,n$, then $p^n<q^m$, then $g(p^n)<g(q^m)$ and so $\frac {g(p)}{g(q)}<\frac mn$
So $\frac {g(p)}{g(q)}\le\frac {\ln p}{\ln q}$

So $\frac {g(p)}{g(q)}=\frac {\ln p}{\ln q}$ and $g(p)=\ln p^c$ for some constant $c$.
$g(2)=1$ implies $g(p)=\log_2 p$

Hence the solution $\boxed{f(x)=8+\log_2(x-1)}$ $\forall x\ge 2$, which indeed is a solution.
\end{solution}
*******************************************************************************
-------------------------------------------------------------------------------

\begin{problem}[Posted by \href{https://artofproblemsolving.com/community/user/125553}{lehungvietbao}]
	Find all continous functions $f:[0;+\infty )\to [0;+\infty )$ such that
\[2f(x)=f(\frac{x}{x^{2}+x+1})+f(\frac{x+1}{2})\quad\forall x\in [0;+\infty )\]
	\flushright \href{https://artofproblemsolving.com/community/c6h568796}{(Link to AoPS)}
\end{problem}



\begin{solution}[by \href{https://artofproblemsolving.com/community/user/29428}{pco}]
	\begin{tcolorbox}Find all continous functions $f:[0;+\infty )\to [0;+\infty )$ such that
\[2f(x)=f(\frac{x}{x^{2}+x+1})+f(\frac{x+1}{2})\quad\forall x\in [0;+\infty )\]\end{tcolorbox}
$f(x)$ is continuous over $[0,1]$ and so is bounded.
Let $M=\sup_{x\in[0,1]}f(x)=\max_{x\in[0,1]}f(x)=f(u)$ for some $u\in[0,1]$
Let $m=\inf_{x\in[0,1]}f(x)=\min_{x\in[0,1]}f(x)=f(v)$ for some $v\in[0,1]$

Applying functional equation with $x=u$, we get that $f(\frac{u+1}2)=M$ and successive applications plus continuity imply $f(1)=M$
Applying functional equation with $x=v$, we get that $f(\frac{v+1}2)=m$ and successive applications plus continuity imply $f(1)=m$

So $f(x)=M=m=c$ $\forall x\in[0,1]$

If $f(x)>c$ for some $x>1$, then, from $2f(x)=c+f(\frac {x+1}2)$, we get $f(\frac {x+1}2)>c$ and successive applications plus continuity imply $f(1)>c$, impossible.
So $f(x)\le c$ $\forall x>1$

If $f(x)<c$ for some $x>1$, then, from $2f(x)=c+f(\frac {x+1}2)$, we get $f(\frac {x+1}2)<c$ and successive applications plus continuity imply $f(1)<c$, impossible.
So $f(x)\ge c$ $\forall x>1$

Hence the unique solution $\boxed{f(x)=c}$ constant $\forall x$, which indeed is a solution.
\end{solution}
*******************************************************************************
-------------------------------------------------------------------------------

\begin{problem}[Posted by \href{https://artofproblemsolving.com/community/user/168537}{vutuanhien}]
	Find all increase function $f:\mathbb{R}\rightarrow \mathbb{R}$ such that:
\[f(x-y+f(y))=f(x+y)+2014\]
	\flushright \href{https://artofproblemsolving.com/community/c6h568855}{(Link to AoPS)}
\end{problem}



\begin{solution}[by \href{https://artofproblemsolving.com/community/user/29428}{pco}]
	\begin{tcolorbox}Find all increase function $f:\mathbb{R}\rightarrow \mathbb{R}$ such that:
\[f(x-y+f(y))=f(x+y)+2014\]\end{tcolorbox}
Let $P(x,y)$ be the assertion $f(x-y+f(y))=f(x+y)+2014$

Let $A=\{f(x)-2x, x\in\mathbb R\}$

If $|A|=1$, then $f(x)-2x=c$ $\forall x$ and, plugging this in orignal equation, we get the solution $\boxed{f(x)=2x+1007}$ $\forall x$

If $|A|>1$, let $f(u)-2u=a<f(v)-2v=b\in A$

Let $x<y$ and $n>\frac{y-x}{b-a}$ so that $x+na<y+na<x+nb$
$P(x-u,u)$ $\implies$ $f(x+a)=f(x)+2014$ and so $f(x+na)=f(x)+2014n$
$P(y-u,u)$ $\implies$ $f(y+a)=f(y)+2014$ and so $f(y+na)=f(y)+2014n$
$P(x-v,v)$ $\implies$ $f(x+b)=f(x)+2014$ and so $f(x+nb)=f(x)+2014n$

So $f(x+na)=f(x+nb)$ an so, since increasing, $f(t)=f(x+na)$ $\forall t\in[x+na, x+nb]$
So $f(y+na)=f(x+na)$ and so $f(x)=f(y)$ and $f(x)$ is constant, which is not a solution.

So no other solution.
\end{solution}
*******************************************************************************
-------------------------------------------------------------------------------

\begin{problem}[Posted by \href{https://artofproblemsolving.com/community/user/119826}{seby97}]
	Let $f: \mathbb{N}\to \mathbb{Z}$ such that $(f(n+1)-f(n))(f(n+1)+f(n)+4) \le 0$. Prove that $f$ is not injective.
	\flushright \href{https://artofproblemsolving.com/community/c6h568869}{(Link to AoPS)}
\end{problem}



\begin{solution}[by \href{https://artofproblemsolving.com/community/user/29428}{pco}]
	\begin{tcolorbox}Let $f: \mathbb{N}\to \mathbb{Z}$ such that $(f(n+1)-f(n))(f(n+1)+f(n)+4) \le 0$. Prove that $f$ is not injective.\end{tcolorbox}
Let $P(n)$ be the assertion $(f(n+1)-f(n))(f(n+1)+f(n)+4)\le 0$
Suppose $f(x)$ is injective.

If $f(n)=-2$ for some $n$, $P(n)$ $\implies$ $f(n+1)=-2=f(n)$, impossible since injective.

If $f(n)>-2$ $\forall n$, then $P(n)$ $\implies$ $-2< f(n+1)\le f(n)$ $\forall n$, impossible since injective.

If $f(n)<-2$ $\forall n$, then $P(n)$ $\implies$ $-2> f(n+1)\ge f(n)$ $\forall n$, impossible since injective.

If $f(n)>-2$ for some $n$ and $f(n)<-2$ for some other $n$ :
Let $a=\min(f(\mathbb N)\cap(-2,+\infty))$ and $u$ such that $f(u)=a$
Let $b=\max(f(\mathbb N)\cap(-\infty,-2))$ and $v$ such that $f(v)=b$

$P(u)$ $\implies$ $-f(u)-4\le f(u+1)<f(u)$ and so $-a-4\le f(u+1)\le b$ and so $-4\le a+b$ 
$P(v)$ $\implies$ $-f(v)-4\ge f(v+1)>f(v)$ and so $-b-4\ge f(v+1)\ge a$ and so $-4\ge a+b$
So $a+b=-4$ and $f(u+1)=b=f(v)$ and so $u+1=v$ and $f(v+1)=a=f(u)$ and so $v+1=u$
Impossible.

So no injective solution.
\end{solution}
*******************************************************************************
-------------------------------------------------------------------------------

\begin{problem}[Posted by \href{https://artofproblemsolving.com/community/user/123851}{ctumeo}]
	Let $ \mathbb{N}^{+}$ be the set of the positive integers. Consider all the functions $f: \mathbb{N}^{+} \rightarrow \mathbb{N}^{+}$ such that 
\[f(m+n) \geq f(m) +f(f(n)) -1\]
for each positive integers $m$, $n$.
Determine all the possible values of $f(2008)$.
Thanks
	\flushright \href{https://artofproblemsolving.com/community/c6h568881}{(Link to AoPS)}
\end{problem}



\begin{solution}[by \href{https://artofproblemsolving.com/community/user/29428}{pco}]
	\begin{tcolorbox}Let $ \mathbb{N}^{+}$ be the set of the positive integers. Consider all the functions $f: \mathbb{N}^{+} \rightarrow \mathbb{N}^{+}$ such that 
\[f(m+n) \geq f(m) +f(f(n)) -1\]
for each positive integers $m$, $n$.
Determine all the possible values of $f(2008)$.
Thanks\end{tcolorbox}
Let $P(x,y)$ be the assertion $f(x+y)\ge f(x)+f(f(y))-1$

$P(x,1)$ $\implies$ $f(x+1)\ge f(x)+f(f(1))-1\ge f(x)$ and $f(x)$ is non decreasing.

1) $f(2008)\le 2009$ 
===============
If $f(2008)\ge 2010$, then $f(x)=1$ $\implies$ $x<2008$

If $f(n)\ge n+2$ for some $n$, then $f(f(n))\ge f(n+2)\ge f(n)\ge n+2$ and $P(n,n)$ $\implies$ $f(2n)-2n\ge f(n)-n+1$ and so $f(x)-x$ is not upper bounded.
If $f(n)>n$ for some $n$, then $P(f(n)-n,n)$ $\implies$ $f(f(n)-n)=1$ and so $f(n)-n<2008$ is upper bounded.
So contradiction.
Q.E.D.

2) $f(2008)$ can take any integer value $k\in[1,2009]$
======================================
For $k\in[1,2007]$, choose $f_k(x)=1$ $\forall x\le 2007$ and $f_k(x)=k$ $\forall x\ge 2008$
For $k=2008$, choose $f_{2008}(x)=x$
For $k=2009$, choose $f_{2009}(x)=2\left\lfloor\frac x2\right\rfloor+1$

Q.E.D.
\end{solution}
*******************************************************************************
-------------------------------------------------------------------------------

\begin{problem}[Posted by \href{https://artofproblemsolving.com/community/user/192463}{arkanm}]
	A function $f$ is defined on the positive integers by $f(1)=1$ and, for $n>1$, \[f(n)=f\left(\left\lfloor \frac{2n-1}{3}\right\rfloor\right)+f\left(\left\lfloor \frac{2n}{3}\right\rfloor\right).\] Is it true that $f(n)-f(n-1)\le n\ \forall n>1?$
	\flushright \href{https://artofproblemsolving.com/community/c6h569021}{(Link to AoPS)}
\end{problem}



\begin{solution}[by \href{https://artofproblemsolving.com/community/user/29428}{pco}]
	\begin{tcolorbox}A function $f$ is defined on the positive integers by $f(1)=1$ and, for $n>1$, \[f(n)=f\left(\left\lfloor \frac{2n-1}{3}\right\rfloor\right)+f\left(\left\lfloor \frac{2n}{3}\right\rfloor\right).\] Is it true that $f(n)-f(n-1)\le n\ \forall n>1?$\end{tcolorbox}
$f(242)-f(241)=2(f(161)-f(160))$
$=4(f(107)-f(106))$
$=8(f(71)-f(70))$
$=16(f(47)-f(46))$
$=32(f(31)-f(30))$
$=32(f(20)-f(19))$
$=64(f(13)-f(12))$
$=64(f(8)-f(7))$
$=128(f(5)-f(4))$
$=256(f(3)-f(2))$
$=256(f(2)-f(1))$
$=256>242$

So no.

See also http://www.artofproblemsolving.com/Forum/viewtopic.php?f=56&t=564317
\end{solution}
*******************************************************************************
-------------------------------------------------------------------------------

\begin{problem}[Posted by \href{https://artofproblemsolving.com/community/user/150671}{mathisfun7}]
	Let $\mathbb{R}$ denote the set of real numbers. Find all functions $f:\mathbb{R}\rightarrow\mathbb{R}$ such that 
\[f(xf(y)+y)+f(-f(x))=f(yf(x)-y)+y\]
for all $x,y\in\mathbb{R}$
	\flushright \href{https://artofproblemsolving.com/community/c6h569071}{(Link to AoPS)}
\end{problem}



\begin{solution}[by \href{https://artofproblemsolving.com/community/user/29428}{pco}]
	\begin{tcolorbox}Let $\mathbb{R}$ denote the set of real numbers. Find all functions $f:\mathbb{R}\rightarrow\mathbb{R}$ such that 
\[f(xf(y)+y)+f(-f(x))=f(yf(x)-y)+y\]
for all $x,y\in\mathbb{R}$\end{tcolorbox}
Let $P(x,y)$ be the assertion $f(xf(y)+y)+f(-f(x))=f(yf(x)-y)+y$
Let $a=f(0)$

1) If $a=0$ : no solution 
================
Suppose $a=0$
$P(x,0)$ $\implies$ $f(-f(x))=0$
$P(0,f(x))$ $\implies$ $f(f(x))=f(x)$
$P(-1,f(x))$ $\implies$ $f(x)=-f(f(x)(f(-1)-1))$ and so $f(LHS)=f(RHS)$ and so $f(x)=0$ $\forall x$, which is not a solution.
Q.E.D.

2) If $a\ne -1$ : no solution
===================
Suppose $a\ne 1$

2.1) If $f(a-x)\ne a$, then $x\in f(\mathbb R)$
-------------------------------------------
If $f(a-x)\ne a$, then :
$P(\frac{a-x}{a-f(a-x)},0)$ $\implies$ $f(\frac{a(a-x)}{a-f(a-x)})+f(-f(\frac{a-x}{a-f(a-x)}))=a$
$P(\frac{a-x}{a-f(a-x)},a-x)$ $\implies$ $f(\frac{a(a-x)}{a-f(a-x)})+f(-f(\frac{a-x}{a-f(a-x)}))$ $=f(\text{something})+a-x$
Subtracting, we get $f(\text{something})=x$
Q.E.D.

2.2) $f(\frac 1{1-a})=a$
-----------------------
If $f(\frac 1{1-a})\ne a$ then (using 2.1) $\frac a{a-1}\in f(\mathbb R)$ and so $\exists t$ such that : $f(t)=\frac a{a-1}$

$P(0,0)$ $\implies$ $f(-a)=0$
$P(t,-a)$ $\implies$ $a=0$, impossible
Q.E.D.

2.3) No such solution
--------------------
$f(\frac 1{1-a})=a$
$P(0,0)$ $\implies$ $f(-a)=0$
$P(\frac 1{1-a},0)$ $\implies$  $f(\frac a{1-a})=a$
$P(0,\frac a{1-a})$ $\implies$  $f(\frac a{1-a})=\frac a{1-a}$
And so $a=0$, impossible
Q.E.D.

3) $f(x)=x+1$ $\forall x$
===============
We previously got $a=1$ and $f(-a)=f(-1)=0$
$P(0,x)$ $\implies$ $\boxed{f(x)=x+1}$ $\forall x$, which indeed is a solution
Q.E.D.
\end{solution}



\begin{solution}[by \href{https://artofproblemsolving.com/community/user/115808}{KMO1}]
	If $f(\frac 1{1-a})\ne a$ then (using 2.1) $\frac a{a-1}\in f(\mathbb R)$ and so $\exists t$ such that : $f(t)=\frac a{a-1}$


in this part, something mistaken... 
$\frac 1{1-a}=\ a-x$ didn't mean $\frac a{a-1}\in f(\mathbb R)$
\end{solution}



\begin{solution}[by \href{https://artofproblemsolving.com/community/user/254374}{Blacklord}]
	is there any better suloution??
\end{solution}



\begin{solution}[by \href{https://artofproblemsolving.com/community/user/198450}{wu2481632}]
	Is there any other solution?
\end{solution}



\begin{solution}[by \href{https://artofproblemsolving.com/community/user/334227}{reveryu}]
	.........
\end{solution}
*******************************************************************************
-------------------------------------------------------------------------------

\begin{problem}[Posted by \href{https://artofproblemsolving.com/community/user/125553}{lehungvietbao}]
	Let be given two functions $p(x),q(x)$  defined on $ \mathbb R$. Find all functions   $f: \mathbb{ R}^*\to\mathbb R$ such that 
\[p(x)f\left(\frac{-1}{x}\right)+f(x)=q(x)\quad \forall x\neq 0 \]
	\flushright \href{https://artofproblemsolving.com/community/c6h569097}{(Link to AoPS)}
\end{problem}



\begin{solution}[by \href{https://artofproblemsolving.com/community/user/29428}{pco}]
	\begin{tcolorbox}Let be given two functions $p(x),q(x)$  defined on $ \mathbb R$. Find all functions   $f: \mathbb{ R}^*\to\mathbb R$ such that 
\[p(x)f\left(\frac{-1}{x}\right)+f(x)=q(x)\quad \forall x\neq 0 \]\end{tcolorbox}
From $p(x)f(-\frac 1x)+f(x)=q(x)$ and $p(-\frac 1x)f(x)+f(-\frac 1x)=q(-\frac 1x)$, we get $(p(x)p(-\frac 1x)-1)f(x)=p(x)q(-\frac 1x)-q(x)$ and so :

If $\exists x\ne 0$ such that $p(x)p(-\frac 1x)= 1$ and $p(x)q(-\frac 1x)-q(x)\ne 0$, then no solution.

If $p(x)p(-\frac 1x)= 1$ implies $p(x)q(-\frac 1x)=q(x)$, then we get the solution :

$\forall x\ne 0$ such that $p(x)p(-\frac 1x)\ne 1$ : $f(x)=\frac{p(x)q(-\frac 1x)-q(x)}{p(x)p(-\frac 1x)-1}$
$\forall x\ne 0$ such that $p(x)p(-\frac 1x)\ne 1$ : $f(-\frac 1x)=a(x)$ and $f(x)=q(x)-a(x)p(x)$

Which indeed is a solution, whatever is $a(x)$ from $\mathbb R\to\mathbb R$
\end{solution}
*******************************************************************************
-------------------------------------------------------------------------------

\begin{problem}[Posted by \href{https://artofproblemsolving.com/community/user/125553}{lehungvietbao}]
	Let be given a function $h(x),x\in \mathbb {R}^*$. Find all functions  such that 
\[f\left(\frac{1}{x}\right)=xf(x)+h(x)\quad \forall x\neq 0 \]
	\flushright \href{https://artofproblemsolving.com/community/c6h569098}{(Link to AoPS)}
\end{problem}



\begin{solution}[by \href{https://artofproblemsolving.com/community/user/29428}{pco}]
	\begin{tcolorbox}Let be given a function $h(x),x\in \mathbb {R}^*$. Find all functions  such that 
\[f\left(\frac{1}{x}\right)=xf(x)+h(x)\quad \forall x\neq 0 \]\end{tcolorbox}
From $f(\frac 1x)=xf(x)+h(x)$ and $f(x)=\frac 1xf(\frac 1x)+h(\frac 1x)$, we get :

If $\exists x\ne 0$ such that $-h(x)\ne xh(\frac1x)$, then no solution.

If $-h(x)= xh(\frac1x)$ $\forall x\ne 0$, then solution is :

$\forall x\in(-1,0)\cup(0,1]$ $f(x)=a(x)$
$\forall x\in(-\infty,-1)\cup(1,+\infty)$ : $f(x)=\frac{a(\frac 1x)}x+h(\frac 1x)$
$f(-1)=\frac{h(-1)}2$

Which indeed is a solution, whatever is $a(x)$ from $(-1,1]\to\mathbb R$
\end{solution}
*******************************************************************************
-------------------------------------------------------------------------------

\begin{problem}[Posted by \href{https://artofproblemsolving.com/community/user/125553}{lehungvietbao}]
	Let be given a function $\omega (x)=\frac{x-1}{x}$  . Find all continuous functions  defined on  $ \mathbb{ R}^*$ such that
\[f(x)f(\omega (x))f(\omega (\omega (x))=1\quad \forall x\neq 0, x\neq 1\]
	\flushright \href{https://artofproblemsolving.com/community/c6h569100}{(Link to AoPS)}
\end{problem}



\begin{solution}[by \href{https://artofproblemsolving.com/community/user/29428}{pco}]
	\begin{tcolorbox}Let be given a function $\omega (x)=\frac{x-1}{x}$  . Find all continuous functions  defined on  $ \mathbb{ R}^*$ such that
\[f(x)f(\omega (x))f(\omega (\omega (x))=1\quad \forall x\neq 0, x\neq 1\]\end{tcolorbox}
Notice that :
$\omega((0,1))=(-\infty,0)$ and $\omega((-\infty,0)=(1,+\infty)$ and $\omega((1,+\infty))=(0,1)$
$\omega(\omega(\omega(x)))=x$

General solution is then easy to build ;
Let $a(x)$ any continuous function from $[0,1]\to \mathbb R^*$
Let $b(x)$ any continuous function from $(-\infty,0)\to\mathbb R^*$ such that $\lim_{x\to-\infty}a(1)a(\omega(\omega(x)))b(x)=1$

$\forall x\in(0,1]$ : $f(x)=a(x)$
$\forall x\in(-\infty,0)$ : $f(x)=b(x)$
$\forall x\in(1,+\infty)$ : $f(x)=\frac 1{a(\omega(x))b(\omega(\omega(x))}$
\end{solution}
*******************************************************************************
-------------------------------------------------------------------------------

\begin{problem}[Posted by \href{https://artofproblemsolving.com/community/user/125553}{lehungvietbao}]
	Let be given a function $h(x)$ defined on  $ \mathbb{ R}$ and $b= \pm1$ . Find all  functions $f(x)$ such that \[\begin{cases}f(x^4)=f(x)\\f(x^2)+bf(x)=h(x) \end{cases}\quad \forall x\in\mathbb R\]
	\flushright \href{https://artofproblemsolving.com/community/c6h569101}{(Link to AoPS)}
\end{problem}



\begin{solution}[by \href{https://artofproblemsolving.com/community/user/29428}{pco}]
	\begin{tcolorbox}Let be given a function $h(x)$ defined on  $ \mathbb{ R}$ and $b= \pm1$ . Find all  functions $f(x)$ such that \[\begin{cases}f(x^4)=f(x)\\f(x^2)+bf(x)=h(x) \end{cases}\quad \forall x\in\mathbb R\]\end{tcolorbox}
If $h(x^2)\ne bh(x)$ for some $x$ : no solution.

If $h(x^2)=bh(x)$ $\forall x$, then general solution is easy to build :

1) Definition of $f(x)$ over $(1,+\infty)$
=========================
Let $u(x)=\left\{\frac{\ln |\ln x|}{\ln 4}\right\}$ from $(1,+\infty)\to[0,1)$
Let $a(x)$ any function from $[0,\frac 12)\to\mathbb R$

If $u(x)\in[0,\frac 12)$ : $f(x)=a(u(x))$
If $u(x)\in[\frac 12,1)$ : $f(x)=h(\sqrt x)-ba(u(x)-\frac 12)$

2) Definition of $f(1)$
====================
If $b=-1$, $f(1)$ can take any value we want
If $b=+1$, $f(1)=\frac{h(1)}2$

3) Definition of $f(x)$ over $(0,1)$
======================
Let $u(x)=\left\{\frac{\ln |\ln x|}{\ln 4}\right\}$ from $(1,+\infty)\to[0,1)$
Let $b(x)$ any function from $[0,\frac 12)\to\mathbb R$

If $u(x)\in[0,\frac 12)$ : $f(x)=b(u(x))$
If $u(x)\in[\frac 12,1)$ : $f(x)=h(\sqrt x)-b\times b(u(x)-\frac 12)$

4) Definition of $f(0)$
=============
If $b=-1$, $f(0)$ can take any value we want
If $b=+1$, $f(0)=\frac{h(0)}2$

5) Definition of $f(x)$ over $(-\infty,0)$
========================
$f(x)=f(-x)$
\end{solution}
*******************************************************************************
-------------------------------------------------------------------------------

\begin{problem}[Posted by \href{https://artofproblemsolving.com/community/user/68025}{Pirkuliyev Rovsen}]
	Find all the functions $f: \mathbb{R}\to\mathbb{R}$ which satisfy the relation $xf(x)=[ x ]f(\{ x \})+\{ x \}f([ x])$, where $[ . ]$ and $\{ . \}$ denote the integral part and fractional part functions, respectively.
	\flushright \href{https://artofproblemsolving.com/community/q2h564541}{(Link to AoPS)}
\end{problem}



\begin{solution}[by \href{https://artofproblemsolving.com/community/user/29428}{pco}]
	\begin{tcolorbox}Find all the functions $f: \mathbb{R}\to\mathbb{R}$ which satisfy the relation $xf(x)=[ x ]f(\{ x \})+\{ x \}f([ x])$, where $[ . ]$ and $\{ . \}$ denote the integral part and fractional part functions, respectively.\end{tcolorbox}
Let $P(x)$ be the assertion $xf(x)=\lfloor x\rfloor f(\{x\})+\{x\}f(\lfloor x\rfloor)$

If $x\in[0,1)$ : $P(x)$ $\implies$ $xf(x)=xf(0)$ and so $f(x)=f(0)$. So $f(\{x\})=f(0)$ $\forall x$

If $x\in\mathbb Z$ : $P(x)$ $\implies$ $xf(x)=xf(0)$ and so $f(x)=f(0)$. So $f(\lfloor x\rfloor)=f(0)$ $\forall x$

Then $P(x)$ becomes $xf(x)=(\lfloor x\rfloor+\{x\})f(0)=xf(0)$ and so $f(x)=f(0)$ $\forall x$

So $\boxed{f(x)=c}$ $\forall x$, which indeed is a solution, whatever is $c\in\mathbb R$
\end{solution}
*******************************************************************************
-------------------------------------------------------------------------------

\begin{problem}[Posted by \href{https://artofproblemsolving.com/community/user/68025}{Pirkuliyev Rovsen}]
	Determine all functions  ${f: \mathbb{R}\to\mathbb(0; +\infty)}$  such that 
$3f(x+y+z)-f(-x+y+z)-f(x-y+z)-f(x+y-z)=$ $8(\sqrt{f(x)f(y)}+\sqrt{f(y)f(z)}+\sqrt{f(z)f(x)})$,  for all $x,y,z{\in}R$.


_______________________________________
Azerbaijan Land of the Fire 
	\flushright \href{https://artofproblemsolving.com/community/c6h536452}{(Link to AoPS)}
\end{problem}



\begin{solution}[by \href{https://artofproblemsolving.com/community/user/181692}{magiccarl2}]
	I think it's none solution
\end{solution}



\begin{solution}[by \href{https://artofproblemsolving.com/community/user/29428}{pco}]
	\begin{tcolorbox}Determine all functions  ${f: \mathbb{R}\to\mathbb(0; +\infty)}$  such that 
$3f(x+y+z)-f(-x+y+z)-f(x-y+z)-f(x+y-z)$ $=8(\sqrt{f(x)f(y)}+\sqrt{f(y)f(z)}$ $+\sqrt{f(z)f(x)})$  for all $x,y,z{\in}R$\end{tcolorbox}
Set $x=y=z=0$ and you get $f(0)=0$ which is impossible since $0\notin(0,+\infty)$

So no solution, as \begin{bolded}magiccarl2 \end{bolded}suggested.
\end{solution}



\begin{solution}[by \href{https://artofproblemsolving.com/community/user/177508}{mathuz}]
	i don't understand!
\end{solution}



\begin{solution}[by \href{https://artofproblemsolving.com/community/user/64716}{mavropnevma}]
	What is it not to understand? We cannot have $f(0)=0$, since $0$ is outside the range of $f$.
\end{solution}



\begin{solution}[by \href{https://artofproblemsolving.com/community/user/177508}{mathuz}]
	thank you, very much.
@ mavropnevma.
\end{solution}



\begin{solution}[by \href{https://artofproblemsolving.com/community/user/177508}{mathuz}]
	Similar problem, but it's old:

Find all functions $f:R^+ \rightarrow R^+$ that satisfy the following conditions:
$(i)$ $f(xyz)+f(x)+f(y)+f(z)=f(\sqrt{xy})f(\sqrt{yz})f(\sqrt{zx});$
$(ii)$  $f(x)<f(y)$ for all  $1\le x<y.$

\end{solution}



\begin{solution}[by \href{https://artofproblemsolving.com/community/user/181692}{magiccarl2}]
	If ${ f:\mathbb{R}\to\mathbb[0;+\infty)} $ what happen ?
\end{solution}



\begin{solution}[by \href{https://artofproblemsolving.com/community/user/181692}{magiccarl2}]
	I can substitute if ${ f:\mathbb{R}\to\mathbb[0;+\infty)} $ It's make $f(x)=x^2$ 
substitute $x=y=z=0$
$3f(0)-f(0)-f(0)-f(0)=8(f(0)+f(0)+f(0))$ then f(0)=0 
substitute $y=z=0$
$3f(x)-f(x)-f(x)-f(-x)=f(0)$ so we have f(x)=f(-x)
substitute $-x=x$
$ 3f(-x+y+z)-f(x+y+z)-f(-x-y+z)-f(-x+y-z) $ = $ 8(\sqrt{f(-x)f(y)}+\sqrt{f(y)f(z)}+\sqrt{f(-x)f(z)}) $ 
$ 3f(-x+y+z)-f(-x+y+z)-f(x+y-z)-f(x-y+z) $ = $ 8(\sqrt{f(x)f(y)}+\sqrt{f(y)f(z)}+\sqrt{f(x)f(z)}) $ = $3f(x+y+z)-f(-x+y+z)-f(x-y+z)-f(x+y-z) $
then $ f(-x+y+z)$ = $f(x+y+z)$
so we substitute $y+z=x$
$ 3f(0)$ = $f(2x)=0$
so $f(x)=0$ but it have $f(x)=x^2$
where is my mistake ?
\end{solution}



\begin{solution}[by \href{https://artofproblemsolving.com/community/user/29428}{pco}]
	\begin{tcolorbox}I can substitute if ${ f:\mathbb{R}\to\mathbb[0;+\infty)} $ It's make $f(x)=x^2$ 
substitute $x=y=z=0$
$3f(0)-f(0)-f(0)-f(0)=8(f(0)+f(0)+f(0))$ then f(0)=0 
substitute $y=z=0$
$3f(x)-f(x)-f(x)-f(-x)=f(0)$ so we have f(x)=f(-x)
substitute $-x=x$
$ 3f(-x+y+z)-f(x+y+z)-f(-x-y+z)-f(-x+y-z) $ = $ 8(\sqrt{f(-x)f(y)}+\sqrt{f(y)f(z)}+\sqrt{f(-x)f(z)}) $ 
$ 3f(-x+y+z)-f(-x+y+z)-f(x+y-z)-f(x-y+z) $ = $ 8(\sqrt{f(x)f(y)}+\sqrt{f(y)f(z)}+\sqrt{f(x)f(z)}) $ = $3f(x+y+z)-f(-x+y+z)-f(x-y+z)-f(x+y-z) $
then $ f(-x+y+z)$ = $f(x+y+z)$
so we substitute $y+z=x$
$ 3f(0)$ = $f(2x)=0$
so $f(x)=0$ but it have $f(x)=x^2$
where is my mistake ?\end{tcolorbox}
If $f(x)$ is from $\mathbb R\to[0,+\infty)$, the proof is quite simple :
$x=y=z=0$ implies $f(0)=0$
$y=z=0$ implies $f(x)=f(-x)$
$x=\frac t2,y=-\frac t2,z=0$ implies $LHS=-f(t)$ while $RHS\ge 0$ and so $f(t)\le 0$ and so $\boxed{f(x)=0}$ $\forall x$  which indeed is a solution

And $f(x)=x^2$ is not a solution because $\sqrt{f(x)f(y)}$ is then $|xy|$ and not $xy$ :)
\end{solution}



\begin{solution}[by \href{https://artofproblemsolving.com/community/user/91617}{Lyub4o}]
	Hello,
  can somebody post a link for $mathuz's$ problem because I cannot find it.Thanks!
\end{solution}
*******************************************************************************
-------------------------------------------------------------------------------

\begin{problem}[Posted by \href{https://artofproblemsolving.com/community/user/91617}{Lyub4o}]
	Find all functions $f:\mathbb R^{+} \longrightarrow \mathbb R^{+}$ so that 

$f(xy + f(x^y)) = x^y + xf(y)$ for all positive reals $x,y$.
	\flushright \href{https://artofproblemsolving.com/community/c6h536585}{(Link to AoPS)}
\end{problem}



\begin{solution}[by \href{https://artofproblemsolving.com/community/user/29428}{pco}]
	\begin{tcolorbox}Find all functions $f:\mathbb R^{+} \longrightarrow \mathbb R^{+}$ so that 

$f(xy + f(x^y)) = x^y + xf(y)$ for all positive reals $x,y$.\end{tcolorbox}
Let $P(x,y)$ be the assertion $f(xy+f(x^y))=x^y+xf(y)$

Let $x\ne 1$ :
$P(x^{\frac 1{x-1}},x)$ $\implies$ $f(x^{\frac x{x-1}}+f(x^{\frac x{x-1}}))$ $=x^{\frac x{x-1}}+x^{\frac 1{x-1}}f(x)$

$P(x^{\frac x{x-1}},1)$ $\implies$ $f(x^{\frac x{x-1}}+f(x^{\frac x{x-1}}))=x^{\frac x{x-1}}+x^{\frac x{x-1}}f(1)$

Subtracting, we get $x^{\frac 1{x-1}}f(x)=x^{\frac x{x-1}}f(1)$ and so $f(x)=xf(1)$ $\forall x\ne 1$, still true when $x=1$

Plugging back $f(x)=ax$ in original equation, we get $a^2=1$ and so $a=1$ since $f(x)>0$ $\forall x$

Hence the unique solution : $\boxed{f(x)=x}$ $\forall x$
\end{solution}
*******************************************************************************
-------------------------------------------------------------------------------

\begin{problem}[Posted by \href{https://artofproblemsolving.com/community/user/177184}{math1200}]
	Find $f : \mathbb{R} \to \mathbb{R}$, $f$ continuous, with $f(1)=a\neq 1$, given by  $f(f(x))=xf(x)+1$ for $x\in \mathbb{R}$.

if $x\in N^{+}$,then  this problem  equivalent
\[a_{n+1}=na_{n}+1,a_{1}=1\]

we have $a_{n}=[e(n-1)!]$

But for this $x\in R$, I can't find and prove it.Thank you everyone
	\flushright \href{https://artofproblemsolving.com/community/c6h536597}{(Link to AoPS)}
\end{problem}



\begin{solution}[by \href{https://artofproblemsolving.com/community/user/148207}{Particle}]
	\begin{tcolorbox}Find $f : \mathbb{R} \to \mathbb{R}$, $f$ continuous,and $f(1)=1$ given by  $f(f(x))=xf(x)+1$ for $x\in \mathbb{R}$.

if $x\in N^{+}$,then  this problem  equivalent
\[a_{n+1}=na_{n}+1,a_{1}=1\]

we have $a_{n}=[e(n-1)!]$

But for this $x\in R$, I can't find and prove it.Thank you everyone\end{tcolorbox}
Wait a second. $f(1)=1$ is not true. Because substituting $x=1$ yields $1=f(f(1))=1\cdot f(1)+1$. Impossible.

We claim $\exists c\in R:f(c)=0$. Because if such $c$ doesn't exist, then $f$ is injective on its whole domain and since continuous, so bijective. Now indeed we are getting a real $c$ such that $f(c)=0$.

Substitute $x=c$. We get $0=f(f(c))=c\cdot f(c)+1$. Again a contradiction.
\end{solution}



\begin{solution}[by \href{https://artofproblemsolving.com/community/user/177184}{math1200}]
	\begin{tcolorbox}[quote="math1200"]Find $f : \mathbb{R} \to \mathbb{R}$, $f$ continuous,and $f(1)=1$ given by  $f(f(x))=xf(x)+1$ for $x\in \mathbb{R}$.

if $x\in N^{+}$,then  this problem  equivalent
\[a_{n+1}=na_{n}+1,a_{1}=1\]

we have $a_{n}=[e(n-1)!]$

But for this $x\in R$, I can't find and prove it.Thank you everyone\end{tcolorbox}
Wait a second. We can prove $f(c)=1\implies c=1$. Substitute $x=0$. So $f(f(0))=1\implies f(0)=1\implies 0=1$.\end{tcolorbox}


Thank you, But Now I let $f(1)=a$,and this problem have wrong?
\end{solution}



\begin{solution}[by \href{https://artofproblemsolving.com/community/user/91617}{Lyub4o}]
	\begin{tcolorbox}Because if such $c$ doesn't exist, then $f$ is injective on its whole domain and since continuous, so bijective.\end{tcolorbox}
You can't say that.
For example $f$ may be decreasing on $\mathbb R$ and converge to a given positive constant.However,there are no negative numbers that are functional values,so the surjectivity fails.
*Edit
$c$,as you can see,is such a value in $Particle's$ solution that $f(c)=0$,not $1$.
\end{solution}



\begin{solution}[by \href{https://artofproblemsolving.com/community/user/64716}{mavropnevma}]
	We have $f(f(0)) = 1$, so there is your $c = f(0)$, such that $f(c) = 1$. Assume there exists $v$ such that $f(v) = 0$. Then $f(0) = f(f(v)) = vf(v) + 1 = 1$, meaning $c=1$. Now, $f(0) = f(1) = 1$, so $1 = f(f(0)) = f(f(1)) = f(1)+1 = 2$, absurd.

So $f(x) \neq 0$ for all $x$, thus $f$ takes constant sign, being continuous. Assume now $f(x) = f(y) = t \neq 0$, so  $xt+1 = f(f(x)) = f(f(y))= yt + 1$, whence $(x-y)t=0$, thus $x=y$. This means $f$ is injective, therefore monotonous, being continuous. Moreover, assume $x=f(x)$, so $x = f(x) = f(f(x)) = xf(x) + 1 = x^2+1$, thus $x^2-x+1 = 0$, but this has no real roots, so $f(x) \neq x$ for all $x$. Then either $f(x) > x$ for all $x$, or $f(x) < x$ for all $x$, since $f(x)-x$ is continuous.
\end{solution}



\begin{solution}[by \href{https://artofproblemsolving.com/community/user/177184}{math1200}]
	\begin{tcolorbox}We have $f(f(0)) = 1$, so there is your $c = f(0)$, such that $f(c) = 1$. Assume there exists $v$ such that $f(v) = 0$. Then $f(0) = f(f(v)) = vf(v) + 1 = 1$, meaning $c=1$. Now, $f(0) = f(1) = 1$, so $1 = f(f(0)) = f(f(1)) = f(1)+1 = 2$, absurd.

So $f(x) \neq 0$ for all $x$, thus $f$ takes constant sign, being continuous. Assume now $f(x) = f(y) = t \neq 0$, so  $xt+1 = f(f(x)) = f(f(y))= yt + 1$, whence $(x-y)t=0$, thus $x=y$. This means $f$ is injective, therefore monotonous, being continuous. Moreover, assume $x=f(x)$, so $x = f(x) = f(f(x)) = xf(x) + 1 = x^2+1$, thus $x^2-x+1 = 0$, but this has no real roots, so $f(x) \neq x$ for all $x$. Then either $f(x) > x$ for all $x$, or $f(x) < x$ for all $x$, since $f(x)-x$ is continuous.\end{tcolorbox}

Thank you  my frend. I think you mean $c=0$,and But $f(1)=a\neq 1$.
\end{solution}



\begin{solution}[by \href{https://artofproblemsolving.com/community/user/64716}{mavropnevma}]
	No, I don't mean $c=0$. I denoted $c=f(0)$ and I proved that assuming $f(v) = 0$ for some $v$, that leads to $c=1$, and then a contradiction. Therefore $f(x)\neq x$ for all $x$, thus of course $c\neq 0$.
\end{solution}



\begin{solution}[by \href{https://artofproblemsolving.com/community/user/177184}{math1200}]
	\begin{tcolorbox}No, I don't mean $c=0$. I denoted $c=f(0)$ and I proved that assuming $f(v) = 0$ for some $v$, that leads to $c=1$, and then a contradiction. Therefore $f(x)\neq x$ for all $x$, thus of course $c\neq 0$.\end{tcolorbox}


Oh, Thank you,but I don't understand:either $f(x)-x<0$,for all $x$,or $f(x)-x>0$ for all $x$,since $f(x)-x$ is continuous.

you mean $f$is not exist?
\end{solution}



\begin{solution}[by \href{https://artofproblemsolving.com/community/user/64716}{mavropnevma}]
	No, I don't mean such $f$ does not exist (by the way, your condition $f(1) \neq 1$ is superfluous, since if $f(1) = 1$ we get $1 = f(1) = f(f(1)) = f(1)+1 = 2$, absurd). All I did was derive a few conclusions, with no final answer ...
\end{solution}



\begin{solution}[by \href{https://artofproblemsolving.com/community/user/177184}{math1200}]
	\begin{tcolorbox}No, I don't mean such $f$ does not exist (by the way, your condition $f(1) \neq 1$ is superfluous, since if $f(1) = 1$ we get $1 = f(1) = f(f(1)) = f(1)+1 = 2$, absurd). All I did was derive a few conclusions, with no final answer ...\end{tcolorbox}

oh, thanks,
\end{solution}



\begin{solution}[by \href{https://artofproblemsolving.com/community/user/29428}{pco}]
	\begin{tcolorbox}Find $f : \mathbb{R} \to \mathbb{R}$, $f$ continuous, with $f(1)=a\neq 1$, given by  $f(f(x))=xf(x)+1$ for $x\in \mathbb{R}$.\end{tcolorbox}
Let us solve the problem : find all continuous functions $f(x)$ from $\mathbb R\to\mathbb R$ such that $f(f(x))=xf(x)+1$ $\forall x$
(This is certainly not a real olympiad exercise :( )
We have infinitely many such fonctions which may be built piece per piece :

1) Claim about the form of the general solution
=================================
Let $u\in(0,1)$
Let $\{a_n\}_{n\ge 0}$ the sequence $a_0=0$, $a_1=u$, $a_n=1+a_{n-1}a_{n-2}$ $\forall n>1$
(note that $a_n$ is an increasing sequence whose limit is $+\infty$)
Let $g(x)$ any continuous increasing bijection from $[0,+\infty)\to[0,1-a_1)$ such that :
$g(x)$ is diffentiable at $0+$ and $g'(0)=a_1$
$\frac{g(x)}x$ is decreasing over $\mathbb R^+$
(it is easy to build such functions, choosing for example concave $g(x)$)

Let $h_0(x)$ from $(-\infty,0]\to (0,a_1]$ defined as $h_0(0)=a_1$ and $h_0(x)=-\frac{g(-x)}x$ $\forall x<0$
From constraints about $g(x)$, it is easy to get :
$h_0(x)$ is a continuous increasing bijection from $(-\infty,0]\to(0,a_1]$
$xh_0(x)+1$ is a continuous increasing bijection from $(-\infty,0]\to(a_1,1]$

Let $h_1(x)$ from $[0,a_1]\to [a_1,a_2]$ defined as $h_1(0)=a_1$ and $h_1(x)=xh_0^{[-1]}(x)+1$ $\forall x\in(0,a_1]$
Since $xh_0(x)+1$ is a continuous increasing bijection from $(-\infty,0]\to(a_1,1]$, we get :
$h_1(x)$ is a continuous increasing bijection from $[0,a_1]\to[a_1,a_2]$

Suppose now that $h_n(x),n\ge 1$ is a continuous increasing bijection from $[a_{n-1},a_n]\to[a_n,a_{n+1}]$
Then $h_n^{[-1]}(x)$ is a continuous increasing bijection from $[a_n,a_{n+1}]\to[a_{n-1},a_n]$
And $h_{n+1}(x)=xh_n^{[-1]}(x)+1$ is a continuous increasing bijection from $[a_n,a_{n+1}]\to[a_{n+1},a_{n+2}]$
And so it is possible to build an infnite sequence of continuous increasing bijections $h_n(x)$

Define then $f(x)$ as :
$f(x)=h_0(x)$ $\forall x<0$
For any $n\in\mathbb N$ : $f(x)=h_n(x)$ $\forall x\in(a_{n-1},a_n]$

2) Any function in the form defined in 1) above is indeed a solution
=============================================
$f(x)$ is indeed a continuous function (the only point to check is continuity at $a_k$ but this is quite simple.

For $x\in(-\infty,0]$, $f(x)=h_0(x)\in(0,a_1]$ and so $f(f(x))=h_1(h_0(x))=xh_0(x)+1=xf(x)+1$
Let $n\in\mathbb N$. For $x\in[a_{n-1},a_n]$, $f(x)=h_n(x)\in[a_n,a_{n+1}]$ and so $f(f(x)=h_{n+1}(h_n(x))=xh_n(x)+1=xf(x)+1$

And so $f(f(x))=xf(x)+1$ $\forall x\in\mathbb R$

And so any function in the form defined in 1) above indeed is a solution of the functional equation.

3) any solution of the functional equation may be written in the form 1) (and so we indeed got a general solution)
===============================================================================

Let $P(x)$ be the assertion $f(f(x))=xf(x)+1$
Let $a_1=f(0)$

3.1) $f(x)>0$ $\forall x$
-----------------------
$P(0)$ $\implies$ $f(f(0))=1$
If $f(u)=0$ for some $u$, then $P(u)$ $\implies$ $f(0)=1$ and then $P(0)$ $\implies$ $f(1)=1$ and then $P(1)$ is wrong
So $f(x)\ne 0$ and since $f(f(0))>0$, we get $f(x)>0$ $\forall x$
Q.E.D.

3.2) $f(x)$ is an increasing bijection from $\mathbb R\to\mathbb R^+$ and $\lim_{x\to-\infty}xf(x)=a_1-1$
-----------------------------------------------------------------------------------------------------
If $f(a)=f(b)=c$ for some $a,b$, then $P(a)$ $\implies$ $f(c)=ac+1$ and $P(b)$ $\implies$ $f(c)=bc+1$ and since $c\ne 0$ (from 1) 

above), we get $a=b$ and $f(x)$ is injective, and so, since continuous, monotonous.
$f(f(x))>0$ $\implies$ $xf(x)+1>0$ and so $f(x)<-\frac 1x$ $\forall x<0$ and so $f(x)$ is increasing

If $\lim_{x\to-\infty}f(x)=l>0$, then setting $x\to-\infty$ in $P(x)$ implies $f(l)=-\infty$, impossible. So  $\lim_{x\to-\infty}f(x)
=0$

Setting then $x\to-\infty$ in $P(x)$, we get  $\lim_{x\to-\infty}xf(x)=f(0)-1$

If $\lim_{x\to+\infty}f(x)=L$, then setting $x\to+\infty$ in $P(x)$ implies $f(L)=+\infty$, impossible. So  $\lim_{x\to+\infty}f(x)=+
\infty$

Q.E.D.

3.3) The function $h(x)=xf(x)$ from $(-\infty,0]\to (a_1-1,0]$ is an increasing continuous bijection and $a_1\in(0,1)$
-------------------------------------------------------------------------------------------------
This is a natural consequence of $f(x)$ increasing and $xf(x)=f(f(x))-1$
As a consequence $f(0)-1<0$ and so $a_1\in(0,1)$
Q.E.D.

3.4) any solution may be put in form 1)
--------------------------------------
Set $u=a_1=f(0)\in(0,1)$
Let $\{a_n\}_{n\ge 0}$ the sequence $a_0=0$, $a_1=u$, $a_n=1+a_{n-1}a_{n-2}$ $\forall n>1$
(note that $a_n$ is an increasing sequence whose limit is $+\infty$)

Let $g(x)$ from $[0,+\infty)\to[0,1-a_1)$ defined as $g(x)=xf(-x)$ $\forall x\ge 0$
Using 3.3, we get that $g(x)$ is a continuous increasing bijection from $[0,+\infty)\to[0,1-a_1)$
$g(0)=0$ and $\frac{g(x)}x=f(-x)$ has a limit $f(0)=a_1$ when $x\to 0+$ and so $g(x)$ is diffentiable at $0+$ and $g'(0)=a_1$
Using 3.2, we get that $\frac{g(x)}x=f(-x)$ is decreasing over $\mathbb R^+$

From there, it is immediate to see that the construction proposed in 1) indeed build $f(x)$.
Q.E.D.
\end{solution}



\begin{solution}[by \href{https://artofproblemsolving.com/community/user/177184}{math1200}]
	Thank you $pco$, this solution is true?
As we  have shown, $f$ is monotonous, injective, and does not cross $y=x$. This means it's inverse exists, and it's inverse is exactly the mirrored image around $y=x$, so that $f^{-1}$ has the same properties. Since we know $f(f(0)) = 1$: $f(0) = f^{-1}(1)$. If $f(x) < x$, then $0 > f(0) = f^{-1}(1)> 1$, a contradiction. 

Thus we know that $f(x)>x$. Now since $f^{-1}$ is also increasing, \[f(x) = f^{-1}(xf(x)+1) > f^{-1}(x^2+1)\] Or that:
\[x > x^2+1\]
A contradiction. Thus no such function exists.
\end{solution}



\begin{solution}[by \href{https://artofproblemsolving.com/community/user/29428}{pco}]
	As I proved, infinitely many such functions exist. Dont hesitate to quote mistakes in my proof.

On the other hand, I dont understand your "contradiction" : why $f^{-1}(xf(x)+1)>f^{-1}(x^2+1)$ ?

Since $f^{-1}$ is increasing, this would be true if $xf(x)+1>x^2+1$ which is trivially false when $x<0$
\end{solution}



\begin{solution}[by \href{https://artofproblemsolving.com/community/user/29428}{pco}]
	About the discussion about $f(1)$ : we get $f(1)=1+f(0)$ and since we proved $f(0)\in(0,1)$, the only constraint about $f(1)$ is $\boxed{f(1)\in(1,2)}$
\end{solution}
*******************************************************************************
-------------------------------------------------------------------------------

\begin{problem}[Posted by \href{https://artofproblemsolving.com/community/user/178156}{War-Hammer}]
	Find all function $f : \mathbb{R^+} \to \mathbb{R}$ satisfy :

1) $ f(x)+f(y) \leq \frac {f(x+y)}{2} $
2) $ \frac {f(x)}{x} + \frac {f(y)}{y} \ge \frac {f(x+y)}{x+y} $
	\flushright \href{https://artofproblemsolving.com/community/c6h536599}{(Link to AoPS)}
\end{problem}



\begin{solution}[by \href{https://artofproblemsolving.com/community/user/29428}{pco}]
	\begin{tcolorbox}Find all function $f : \mathbb{R^+} \to \mathbb{R}$ satisfy :

1) $ f(x)+f(y) \leq \frac {f(x+y)}{2} $
2) $ \frac {f(x)}{x} + \frac {f(y)}{y} \ge \frac {f(x+y)}{x+y} $\end{tcolorbox}
So $\frac{x+y}xf(x)+\frac{x+y}yf(y)\ge f(x+y)\ge 2f(x)+2f(y)$ (E1)

(E1) implies $\frac{x+y}xf(x)+\frac{x+y}yf(y)\ge 2f(x)+2f(y)$ and so $(y-x)(\frac{f(x)}x-\frac{f(y)}y)\ge 0$ and so $\frac{f(x)}x$ is non increasing.
Setting $y=x$ in (E1), we get $f(2x)=4f(x)$

So $\frac{f(2x)}{2x}=2\frac{f(x)}x$ and since $\frac{f(x)}x$ is non increasing, this implies $f(x)\le 0$ $\forall x>0$

For easier writing, let then $f(x)=-xg(x)$
$g(x)$ is a non decreasing non negative function such that $g(2x)=2g(x)$ and 2) implies $g(x+y)\ge g(x)+g(y)$

This implies $g(nx)\ge ng(x)$ $\forall x>0$ and $\forall n\in\mathbb N$

If $g(mu)>mg(u)$ for some $m\in\mathbb N$ and $u\in\mathbb R^+$, then $g(mu+ku)\ge g(mu)+g(ku)>(m+k)g(u)$

Choosing then $k$ such that $m+k=2^t$, we get $g(2^tu)>2^tg(u)$ in contradiction with $g(2x)=2g(x)$

So $g(nx)=ng(x)$ $\forall n\in\mathbb N$, $\forall x\in\mathbb R^+$

So $g(x)=xg(1)$ $\forall x\in\mathbb Q$ and since $g(x)$ is non decreasing, we get $g(x)=ax$ $\forall x\in\mathbb R^+$ and whatever is $a\ge 0$

Hence the solution $\boxed{f(x)=-ax^2}$ $\forall x\in\mathbb R^+$, and whatever is constant $a\ge 0$
And it's immediate to check that this indeed is a solution.
\end{solution}
*******************************************************************************
-------------------------------------------------------------------------------

\begin{problem}[Posted by \href{https://artofproblemsolving.com/community/user/148207}{Particle}]
	Find all continuous functions $f:\mathbb R_+\to\mathbb R$ such that for all positive real $x,y$
\[f(x)f(y)=f(xy)+f\left (\frac x y\right )\]
	\flushright \href{https://artofproblemsolving.com/community/c6h536676}{(Link to AoPS)}
\end{problem}



\begin{solution}[by \href{https://artofproblemsolving.com/community/user/29428}{pco}]
	\begin{tcolorbox}Find all continuous functions $f:\mathbb R_+\to\mathbb R$ such that for all positive real $x,y$
\[f(x)f(y)=f(xy)+f\left (\frac x y\right )\]\end{tcolorbox}
Setting $g(x)=\frac 12f(e^x)$, we get an equivalent problem :

Find all continuous functions $g(x)$ from $\mathbb R\to\mathbb R$ such that $g(x+y)+g(x-y)=2g(x)g(y)$ $\forall x,y\in\mathbb R$

This equation is a classical d'Alembert equation whose solutions are $0,\cos ax, \cosh ax$

Hence the solutions of equation :

$f(x)=0$ $\forall x$
$f(x)=2\cos(a\ln x)$
$f(x)=2\cosh(a\ln x)=x^a+x^{-a}$
\end{solution}
*******************************************************************************
-------------------------------------------------------------------------------

\begin{problem}[Posted by \href{https://artofproblemsolving.com/community/user/156523}{dizzy}]
	Find all functions $ f: \mathbb{Q}^+\rightarrow\mathbb{Q}^+ $ that satisfy
$ f(x)+f(\frac{1}{x})=1 $ and $ f(1+2x)=\frac{f(x)}{2} $ for all $ x\in\mathbb{Q}^+ $.
	\flushright \href{https://artofproblemsolving.com/community/c6h536877}{(Link to AoPS)}
\end{problem}



\begin{solution}[by \href{https://artofproblemsolving.com/community/user/29428}{pco}]
	\begin{tcolorbox}Find all functions $ f: \mathbb{Q}^+\rightarrow\mathbb{Q}^+ $ that satisfy
$ f(x)+f(\frac{1}{x})=1 $ and $ f(1+2x)=\frac{f(x)}{2} $ for all $ x\in\mathbb{Q}^+ $.\end{tcolorbox}
Setting $x=1$ in the first equality, we get $f(1)=\frac 12$

Let $p\ne q\in\mathbb N$ and the sequences $p_n,q_n$ defined as :

$p_1=p$ and $q_1=q$
If $p_n=q_n$, then $p_{n+1}=p_n$ and $q_{n+1}=q_n$.
If $p_n>q_n$, then $p_{n+1}=p_n-q_n$ and $q_{n+1}=2q_n$.
If $p_n<q_n$, then $p_{n+1}=q_n-p_n$ and $q_{n+1}=2p_n$.

At each step such that $p_n\ne q_n$, we get $f(\frac{p_{n+1}}{q_{n+1}})=2f(\frac{p_n}{q_n})$ if $p_n>q_n$ or $f(\frac{p_{n+1}}{q_{n+1}})=2-2f(\frac{p_n}{q_n})$ if$p_n<q_n$
So $f(\frac{p_{n+1}}{q_{n+1}})=c_n\pm 2^{n}f(\frac pq)$

Note that $p_n+q_n=p+q$ and so, whatever are $p,q$ :
either the sequence becomes constant ($p_n=q_n$) from a given point.
either the sequence enter a non constant loop. $p_n=p_{n+k}$ and $q_n=q_{n+k}$ for some $k>1$ from a given point

In the first case, let $n>1$ the first indice such that $p_n=q_n$. We get $\frac 12=f(\frac{p_n}{q_n})=c_{n-1}\pm 2^{n-1}f(\frac pq)$ and so $f(\frac pq)$ is uniquely found.

In the second case, we get  $f(\frac{p_n}{q_n})=f(\frac{p_{n+k}}{q_{n+k}})$ and so $c_{n-1}\pm 2^{n-1}f(\frac pq)=c_{n+k-1}\pm 2^{n+k-1}f(\frac pq)$ and again $f(\frac pq)$ is uniquely found.

So, if such a function exists, it  must be unique.

And since $\boxed{f(x)=\frac 1{x+1}}$ $\forall x$ is such a function, this is the unique solution.
\end{solution}
*******************************************************************************
-------------------------------------------------------------------------------

\begin{problem}[Posted by \href{https://artofproblemsolving.com/community/user/156523}{dizzy}]
	Find all functions $ f: \mathbb{N}\rightarrow\mathbb{N} $ satisfying $ f(m+f(n))=n+f(m+k) $ for $ m,n \in\mathbb{N} $ where $ k\in\mathbb{N} $ is fixed. 
Here $ \mathbb{N} $ is the set of positive integers.
	\flushright \href{https://artofproblemsolving.com/community/c6h536879}{(Link to AoPS)}
\end{problem}



\begin{solution}[by \href{https://artofproblemsolving.com/community/user/100909}{lovermath}]
	You can add arbitrary natural number $a$ and give $f()$ signs to both sides
\end{solution}



\begin{solution}[by \href{https://artofproblemsolving.com/community/user/29428}{pco}]
	\begin{tcolorbox}Find all functions $ f: \mathbb{N}\rightarrow\mathbb{N} $ satisfying $ f(m+f(n))=n+f(m+k) $ for $ m,n \in\mathbb{N} $ where $ k\in\mathbb{N} $ is fixed. 
Here $ \mathbb{N} $ is the set of positive integers.\end{tcolorbox}
Let $P(x,y)$ be the assertion $f(x+f(y))=y+f(x+k)$

If $x>f(k)$, then $P(x-f(k),k)$ $\implies$ $f(x)>k$

Let then $x>k$ and $y>f(k)$  : $P(x-k,y)$ $\implies$ $f(x+(f(y)-k))=f(x)+y$
So $f(x+n(f(y)-k))=f(x)+ny$

Let then $z>f(k)$. Setting $n=(f(z)-k)$ in last equation, we get $f(x+(f(z)-k)(f(y)-k))=f(x)+(f(z)-k)y$
Swapping $z,y$, this gives also $f(x+(f(y)-k)(f(y)-k))=f(x)+(f(y)-k)z$

And so $(f(z)-k)y=(f(y)-k)z$ $\forall y,z>f(k)$

So $\frac{f(x)-k}x$ is constant over $(f(k),+\infty)$ and $f(x)=ax+k$ $\forall x>f(k)$ and for some $a$

Let $x,y>f(k)$ : $P(x,y)$ $\implies$ $a^2=1$ and so $a=1$ (else $f(x)<0$ for some $x$) and $f(x)=x+k$ $\forall x>f(k)$

Let $x>f(k)$ and any $y\in\mathbb N$. $x+f(y)>f(k)$ and $x+k>f(k)$ and so : $P(x,y)$ $\implies$ $f(y)=y+k$

And so $\boxed{f(x)=x+k}$ $\forall x$, which indeed is a solution.
\end{solution}
*******************************************************************************
-------------------------------------------------------------------------------

\begin{problem}[Posted by \href{https://artofproblemsolving.com/community/user/156523}{dizzy}]
	Let $ k\in\mathbb{Z} $. Find all functions $ f: \mathbb{Z}\rightarrow\mathbb{Z} $ that satisfy $ f(m+n)+f(mn-1)=f(m)f(n)+k $
	\flushright \href{https://artofproblemsolving.com/community/c6h536881}{(Link to AoPS)}
\end{problem}



\begin{solution}[by \href{https://artofproblemsolving.com/community/user/178156}{War-Hammer}]
	You can post all these function problems in one post :)
\end{solution}



\begin{solution}[by \href{https://artofproblemsolving.com/community/user/64716}{mavropnevma}]
	NO. No multiple problems in a same post.
\end{solution}



\begin{solution}[by \href{https://artofproblemsolving.com/community/user/29428}{pco}]
	\begin{tcolorbox}Let $ k\in\mathbb{Z} $. Find all functions $ f: \mathbb{Z}\rightarrow\mathbb{Z} $ that satisfy $ f(m+n)+f(mn-1)=f(m)f(n)+k $\end{tcolorbox}
I'm quite surprised that you got such an exercise in an olympiad contest :?:

Let $P(x,y)$ be the assertion $f(x+y)+f(xy-1)=f(x)f(y)+k$

1) Case $f(0)\ne 1$
==================
$P(x,0)$ $\implies$ $f(x)+f(-1)=f(x)f(0)+k$ and so $f(x)=\frac{f(-1)-k}{f(0)-1}=c$ is constant.
We need then $2c=c^2+k$ and $c\ne 1$ (since $f(0)\ne 1$) and so :

If $k=2c-c^2$ for some integer $c\ne 1$, we get the solution $f(x)=c$

2) Case $f(0)=1$
===============
Let $a=f(1)$
$P(x,0)$ $\implies$ $f(-1)=k$
$P(-1,-1)$ $\implies$  $f(-2)+1=k^2+k$
$P(1,-1)$ $\implies$ $1+f(-2)=ka+k$
Subtracting, we get $k(a-k)=0$

2.1 subcase $k=0$
----------------
So $f(-2)=-1$ and $f(-1)=0$ and $f(0)=1$ and $f(1)=a$
$P(x+1,1)$ $\implies$ $f(x+2)=af(x+1)-f(x)$ which gives :

$f(2)=a^2-1$
$f(3)=a^3-2a$
$f(4)=a^4-3a^2+1$

$P(2,2)$ $\implies$ $f(4)+f(3)=f(2)^2$ and so $a(a+1)(a-2)=0$

2.1.1 subcase $a=-1$
-------------------
Application of $f(x+2)=af(x+1)-f(x)$ gives 
$f(3x)=1$ and $f(3x+1)=-1$ and $f(3x+2)=0$ which indeed is a solution

2.1.2 subcase $a=0$
-------------------
Application of $f(x+2)=af(x+1)-f(x)$ gives 
$f(4x)=1$ and $f(4x+2)=-1$ and $f(2x+1)=0$ which indeed is a solution

2.1.3 subcase $a=2$
-------------------
Application of $f(x+2)=af(x+1)-f(x)$ gives 
$f(x)=x+1$ which indeed is a solution.

2.2 subcase $k\ne 0$
----------------
So $a=k$ and so $f(-1)=k$ and $f(0)=1$ and $f(1)=k$
$P(x+1,1)$ $\implies$ $f(x+2)=kf(x+1)-f(x)+k$ and so :

$f(2)=k^2+k-1$
$f(3)=k^3+k^2-k$
$f(4)=k^4+k^3-2k^2+1$
$f(5)=k^5+k^4-3k^3-k^2+3k$

$P(3,2)$ $\implies$ $2f(5)=f(3)f(2)+k$ and so $k(k^2-1)(k^2-4)=0$ and $k\in\{-2,-1,1,2\}$

2.2.1 subcase $k=-2$
--------------------
$f(-1)=-2$ and $f(0)=1$ and $f(1)=-2$ and $f(x+2)=-2f(x+1)-f(x)-2$ which gives :
$f(2x)=1$ and $f(2x+1)=-2$
Which indeed is a solution.

2.2.2 subcase $k=-1$
--------------------
$f(-1)=-1$ and $f(0)=1$ and $f(1)=-1$ and $f(x+2)=-2f(x+1)-f(x)-1$ which gives :
$f(3x)=1$ and $f(3x+1)=-1$ and $f(3x+2)=-1$
Which indeed is a solution.

2.2.3 subcase $k=1$
--------------------
$f(-1)=1$ and $f(0)=1$ and $f(1)=1$ and $f(x+2)=-2f(x+1)-f(x)+1$ which gives :
$f(x)=1$
Which indeed is a solution.

2.2.4 subcase $k=2$
--------------------
$f(-1)=2$ and $f(0)=1$ and $f(1)=2$ and $f(x+2)=-2f(x+1)-f(x)+2$ which gives :
$f(x)=x^2+1$
Which indeed is a solution.

3. Synthesis of solutions
========================
The functional equation has solutions only for some values of $k$

If $k=2c-c^2$ for any integer $c\notin\{0,1,2\}$, we get the solution $f(x)=c$

If $k=-2$, we get one solution :
$f(2x)=1$ and $f(2x+1)=-2$

If $k=-1$, we get one solution :
$f(3x)=1$ and $f(3x+1)=-1$ and $f(3x+2)=-1$

If $k=0$, we get 5 solutions :
$f(3x)=1$ and $f(3x+1)=-1$ and $f(3x+2)=0$ 
$f(4x)=1$ and $f(4x+2)=-1$ and $f(2x+1)=0$ 
$f(x)=x+1$ 
$f(x)=0$
$f(x)=2$

If $k=1$, we get one solution :
$f(x)=1$

If $k=2$, we get one solution :
$f(x)=x^2+1$
\end{solution}



\begin{solution}[by \href{https://artofproblemsolving.com/community/user/156523}{dizzy}]
	Dear pco. Thank you very much for posting the solution! This problem is from one book, the solution there I didn't understand. And why are you surprised? Is this problem too easy or non-standard?
\end{solution}



\begin{solution}[by \href{https://artofproblemsolving.com/community/user/29428}{pco}]
	\begin{tcolorbox}Dear pco. Thank you very much for posting the solution! This problem is from one book, the solution there I didn't understand. And why are you surprised? Is this problem too easy or non-standard?\end{tcolorbox}
According to me, too much subcases and too much solutions
\end{solution}
*******************************************************************************
-------------------------------------------------------------------------------

\begin{problem}[Posted by \href{https://artofproblemsolving.com/community/user/177508}{mathuz}]
	Find all function  $f:R\rightarrow Z$  which satisfy the conditions:
$ f(x+y)\le f(x)+f(y) $ and
$ f((x))=\lfloor x \rfloor +2$.
	\flushright \href{https://artofproblemsolving.com/community/c6h537104}{(Link to AoPS)}
\end{problem}



\begin{solution}[by \href{https://artofproblemsolving.com/community/user/29428}{pco}]
	\begin{tcolorbox}Find all function  $f:R\rightarrow Z$  which satisfy the conditions:
$ f(x+y)\le f(x)+f(y) $ and
$ f((x))=\lfloor x \rfloor +2$.\end{tcolorbox}
What's the meaning of $ f((x))$ ?
\end{solution}



\begin{solution}[by \href{https://artofproblemsolving.com/community/user/177508}{mathuz}]
	oh, sorry:
Find all function  $f:R\rightarrow Z$  which satisfy the conditions:
$ f(x+y)\le f(x)+f(y) $ and
$ f(f(x))=\lfloor x \rfloor +2$.
\end{solution}



\begin{solution}[by \href{https://artofproblemsolving.com/community/user/177508}{mathuz}]
	if,  $f:R\rightarrow R$?
\end{solution}



\begin{solution}[by \href{https://artofproblemsolving.com/community/user/29428}{pco}]
	\begin{tcolorbox}Find all function  $f:R\rightarrow Z$  which satisfy the conditions:
$ f(x+y)\le f(x)+f(y) $ and
$ f(f(x))=\lfloor x \rfloor +2$.\end{tcolorbox}
Let $a=f(0)$
Let $b=f(1)$
Second equation implies $f(\lfloor x\rfloor+2)=\lfloor f(x)\rfloor+2=f(x)+2$ (one consequence is that $f(x)=f(\lfloor x\rfloor)$

Let $n\in\mathbb Z$ : We got $f(n+2)=f(n)+2$ and so :
$f(n)=n+a$ $\forall $ even $n$
$f(n)=n+b-1$ $\forall $ odd $n$

If $a$ is even, then $2=f(f(0))=f(a)=2a$ and so $a=1$, contradiction. So $a$ is odd. Then $2=f(f(0))=f(a)=a+b-1$ and so $a+b=3$

$f(0+0)\le f(0)+f(0)$ $\implies$ $a\ge 0$ and so, since odd, $a\ge 1$

$f(\frac 12+\frac 32)\le f(\frac 12)+f(\frac 32)$ $\implies$ $f(2)\le f(0)+f(1)$ $\implies$ $b\ge 2$

But $a\ge 1$ and $b\ge 2$ and $a+b=3$ imply $a=1$ and $b=2$ and $\boxed{f(x)=\lfloor x\rfloor+1}$ $\forall x$, which indeed is a solution.
\end{solution}
*******************************************************************************
-------------------------------------------------------------------------------

\begin{problem}[Posted by \href{https://artofproblemsolving.com/community/user/177508}{mathuz}]
	Find all functions  $f:N\rightarrow N$ such that  \[ f(n)+\sqrt{f(n)+n^2}=3n+2 \] for all $n\in N$.
	\flushright \href{https://artofproblemsolving.com/community/c6h537477}{(Link to AoPS)}
\end{problem}



\begin{solution}[by \href{https://artofproblemsolving.com/community/user/86443}{roza2010}]
	$f(n)=2n+1$
\end{solution}



\begin{solution}[by \href{https://artofproblemsolving.com/community/user/29428}{pco}]
	\begin{tcolorbox}Find all functions  $f:N\rightarrow N$ such that  \[ f(n)+\sqrt{f(n)+n^2}=3n+2 \] for all $n\in N$.\end{tcolorbox}
$\iff$ $(f(n)+n^2)+\sqrt{f(n)+n^2}+\frac 14=n^2+3n+\frac 94$

$\iff$ $(\sqrt{f(n)+n^2}+\frac 12)^2=(n+\frac 32)^2$

and since both quantities squared are $\ge 0$ : $\iff$ $\sqrt{f(n)+n^2}+\frac 12=n+\frac 32$

$\iff$ $\sqrt{f(n)+n^2}=n+1$

$\iff$ $f(n)=(n+1)^2-n^2=2n+1$
\end{solution}
*******************************************************************************
-------------------------------------------------------------------------------

\begin{problem}[Posted by \href{https://artofproblemsolving.com/community/user/68025}{Pirkuliyev Rovsen}]
	There is a strictly increasing function $f: \mathbb{R}\to\mathbb{R}$ , all of whose values are irrational?


________________________________________
Azerbaijan Land of the Fire 
	\flushright \href{https://artofproblemsolving.com/community/c6h537521}{(Link to AoPS)}
\end{problem}



\begin{solution}[by \href{https://artofproblemsolving.com/community/user/105169}{Nikpour}]
	I think there is no such a function. The function   $f(x)=e\pi {{x}^{3}}\ln 2$ has no the required property. For example$f(\frac{1}{\sqrt[3]{e\pi \ln 2}})=e\pi {{(\frac{1}{\sqrt[3]{e\pi \ln 2}})}^{3}}\ln 2=1\in \mathbb{Q}$
such a functhin mus have a discontinuity at any interval.
\end{solution}



\begin{solution}[by \href{https://artofproblemsolving.com/community/user/146093}{SZRoberson}]
	Nope! Let's think of the intermediate value theorem.

How do we get to, say, from $\sqrt{2}$ to $\sqrt{5}$? We must pass through $\sqrt{4}=2$, a rational number!

As Nikpour said above, there must be discontinuity.
\end{solution}



\begin{solution}[by \href{https://artofproblemsolving.com/community/user/70567}{Maxima}]
	We employ the [url=https:\/\/en.wikipedia.org\/wiki\/Archimedean_property]Archimedean property[\/url], of $\mathbb{R}$ which for our purposes states:

If $a > 0$ and $b>0$, then for some positive integer $n$, we have $na > b$.  

This may seem trivial, but there exist number systems for which this isn't true, so we employ it explicitly.

Claim:
If $c, d \in \mathbb{R}$ and $c<d$, then there is a rational $r \in \mathbb{Q}$ such that $c < r < d$.

We claim that for some integers $m$ and $n$, we have $c < \frac{m}{n} < d$, which is equivalent to 
\[
cn < m < dn
\]

$d-c$ is positive, so by the Archimedean property for $a = d-c$ and $b = 1$, there is some positive integer $n$ such that $n(d-c) > 1$.  

This implies that $dn - cn > 1$, which intuitively makes it clear that there is some integer in between the bounds.  We now make this explicit.

By the Archimedean property, for $a = 1$ and $b = \max\{|cn|, |dn|\}$, there exists an integer $x >\max\{|cn|, |dn|\}$, which yields
\[
-x < cn < dn < x
\]

Now consider the set $S$ of integers $y$ such that $-x < y \leq x$ and $cn < y$.

Set $m = \min S$.  This would yield $cn < m$ and $m-1 \leq cn$.  We argue as follows:
\[
m = (m-1) + 1 \leq cn + 1 < cn + (dn - cn) = dn
\]

as claimed.  Thus such an $m$ exists by construction.

This result is known as the denseness of $\mathbb{Q}$, which you can read about in a real analysis book.
\end{solution}



\begin{solution}[by \href{https://artofproblemsolving.com/community/user/89198}{chaotic_iak}]
	The problem with all the above solutions is that $f$ is not necessarily continuous.
\end{solution}



\begin{solution}[by \href{https://artofproblemsolving.com/community/user/64716}{mavropnevma}]
	Let us build an example. Consider $r_1,r_2, \ldots,r_n,\ldots$ an enumeration of $\mathbb{Q}$. Define $f: \mathbb{R} \to \mathbb{R}$ by $\displaystyle f(x) = \sum_{n \in Q_x} \dfrac {1} {n!}$, where $Q_x = \{n \mid r_n \leq x\}$ (clearly an infinite set for any $x$). Now, the series $\displaystyle \sum_{n \in Q_x} \dfrac {1} {n!}$ is convergent, so $f$ is well-defined, and $f$ is strictly increasing, since for any $x<y$ there exists a rational $r$ with $x<r<y$. It remains to show that $f(x)$ is irrational for any $x$. Assume $f(x) = \dfrac {p} {q}$ for some $x$. Take $N$ large enough so that $q\mid N!$. Then $N!f(x) = \dfrac {p} {q} N! \in \mathbb{N}$, but $\displaystyle N!f(x) = N!\sum_{n \in Q_x, n\leq N} \dfrac {1} {n!} + N!\sum_{n \in Q_x, n > N} \dfrac {1} {n!}$, with $\displaystyle N!\sum_{n \in Q_x, n\leq N} \dfrac {1} {n!} \in \mathbb{N}$ and $\displaystyle 0< N!\sum_{n \in Q_x, n > N} \dfrac {1} {n!} $ $\leq N!\sum_{n > N} \dfrac {1} {n!} <$ $ \dfrac {1} {N+1} + \sum_{k \geq 1} \dfrac {1} {(N+k)(N+k+1)} =$ $ \dfrac {2} {N+1} < 1$, contradiction.
Of course $f$ is discontinuous, precisely at all rational points $r$ (for $r=r_n$, it will have a jump of $\dfrac {1} {n!}$).
\end{solution}



\begin{solution}[by \href{https://artofproblemsolving.com/community/user/29428}{pco}]
	Soooooo nice !
Congrats \begin{bolded}mavropnevma\end{bolded} !

Sorry for the spam : it's too nice to just a "thanks" button :)
\end{solution}
*******************************************************************************
-------------------------------------------------------------------------------

\begin{problem}[Posted by \href{https://artofproblemsolving.com/community/user/177508}{mathuz}]
	A fuction $f$ from the pairs of nonnegative integers to the  real numbers satisfies the following  conditions:
$f(0,0)=0,$   $f(2x,2y)=f(2x+1,2y+1)=f(x,y),$  $f(2x+1,2y)=f(2x,2y+1)=f(x,y)+1$ 
for all nonnegative  integers $x,y$.  Let  $a,b$ and $n$ be nonnegative integers  such that  $f(a,b)=n$.  Find the number of the integral  solutions $x$ of the  equation  $f(a,x)+f(b,x)=n.$
	\flushright \href{https://artofproblemsolving.com/community/c6h537660}{(Link to AoPS)}
\end{problem}



\begin{solution}[by \href{https://artofproblemsolving.com/community/user/29428}{pco}]
	\begin{tcolorbox}A fuction $f$ from the pairs of nonnegative integers to the  real numbers satisfies the following  conditions:
$f(0,0)=0,$   $f(2x,2y)=f(2x+1,2y+1)=f(x,y),$  $f(2x+1,2y)=f(2x,2y+1)=f(x,y)+1$ 
for all nonnegative  integers $x,y$.  Let  $a,b$ and $n$ be nonnegative integers  such that  $f(a,b)=n$.  Find the number of the integral  solutions $x$ of the  equation  $f(a,x)+f(b,x)=n.$\end{tcolorbox}
$f(x,y)$ is the number of binary positions where binary representations of $x$ and $y$ are different.

If $a,b$ are different at some position, then contribution of this position in $f(a,x)$ plus contribution of this position in $f(b,x)$ is $1$, exactly the contribution of this position in $f(a,b)$, whatever is $x$ at this position.

If $a,b$ are equal at some position and $x$ is different, then contribution of this position in $f(a,x)$ plus contribution of this position in $f(b,x)$ is $2$, while the contribution of this position in $f(a,b)$ is $0$

If $a,b$ are equal at some position and $x$ is equal then contribution of this position in $f(a,x)$ plus contribution of this position in $f(b,x)$ is $0$, exactly the contribution of this position in $f(a,b)$

So $f(a,x)+f(b,x)=f(a,b)$ means that binary digits of $x$ must be the same than binary digits of $a,b$ when these are equal, and can be any in other positions.

Hence the answer : $\boxed{2^n}$
\end{solution}
*******************************************************************************
-------------------------------------------------------------------------------

\begin{problem}[Posted by \href{https://artofproblemsolving.com/community/user/172163}{joybangla}]
	$f:\mathbb{R}^+ \mapsto\mathbb{R}^+$ such that $f(x+f(y))=f(x)+\pi^2 f(y)$.Find $f$.
	\flushright \href{https://artofproblemsolving.com/community/c6h537941}{(Link to AoPS)}
\end{problem}



\begin{solution}[by \href{https://artofproblemsolving.com/community/user/105169}{Nikpour}]
	$f(x)={{\pi }^{2}}x$
\end{solution}



\begin{solution}[by \href{https://artofproblemsolving.com/community/user/29428}{pco}]
	\begin{tcolorbox}$f(x)={{\pi }^{2}}x$\end{tcolorbox}
That's just one of the infinitely many solutions.
\end{solution}



\begin{solution}[by \href{https://artofproblemsolving.com/community/user/29428}{pco}]
	\begin{tcolorbox}$f:\mathbb{R}^+ \mapsto\mathbb{R}^+$ such that $f(x+f(y))=f(x)+\pi^2 f(y)$.Find $f$.\end{tcolorbox}
Questions :
1) in what olympiad contest did you get this exercise ?
2) could you kindly give us your own solution, please ?

Let $P(x,y)$ be the assertion $f(x+f(y))=f(x)+\pi^2f(y)$

1) Claim about a general form for solution
=============================
Let $A\ne\{0\}$ any additive subgroup of $\mathbb R$ such that $a\in A\implies \pi^2a\in A$ and $A^+\ne\emptyset$ the subset of its positive elements.
The relation $x\sim y$ defined as $(x-y)\in A$ is an equivalence relation in $\mathbb R$
Let $r(x)$ any function from $\mathbb R\to\mathbb R^+$ which associates to a real a representant (unique per class) of its equivalence class (since $A\ne\{0\}$, such a positive representant always exists).
Let $g(x)$ any function from $\mathbb R^+\to A^+$ such that $g(r(x))\ge\pi^2r(x)$  (always exists since $A^+$ is not upperbounded)

Then we can define $f(x)$ as $\boxed{f(x)=\pi^2x+g(r(x))-\pi^2r(x)}$

2) Proof that any function in the form defined in 1) is indeed a solution
================================================
First, note that $f(x)$ is indeed a function from $\mathbb R^+\to\mathbb R^+$ since $g(r(x))\ge\pi^2r(x)$

$g(r(x))\in A$
$x\sim r(x)$ and so $x-r(x)\in A$ and so $\pi^2(x-r(x))\in A$
So ${f(x)=\pi^2x+g(r(x))-\pi^2r(x)}$ $=\pi^2(x-r(x))+g(r(x))$ is sum of two elements of $A$ and so $\in A$

So $x+f(y)\sim x$ and $r(x+f(y))=r(x)$ and so :
$f(x+f(y))=\pi^2(x+f(y))+g(r(x+f(y)))-\pi^2r(x+f(y))$ $=\pi^2(x+f(y))+g(r(x))-\pi^2r(x)$ $=(\pi^2x+g(r(x))-\pi^2r(x))+\pi^2f(y)$ $=f(x)+\pi^2f(y)$
Q.E.D.

3) Proof that any solution may be put in the form given in 1) and so it is indeed a general solution
===================================================================
Let $f(x)$ a solution of the functional equation.
Let $A=\{y\in\mathbb R$ such that $f(x+y)=f(x)+\pi^2y$ $\forall x\in\mathbb R$ such that $x>0$ and $x+y>0\}$
Let $A^+=A\cap\mathbb R^+$
Note that $f(\mathbb R^+)\subseteq A^+$ and so $A$ contains at least one positive element and so $A\ne\{0\}$

3.1) $A$ is an additive subgroup of $\mathbb R$
-----------------------------------------------------------
Obviously $0\in A$

If $a\in A$ and $x>0$ and $x-a>0$, then $f((x-a)+a)=f(x-a)+\pi^2a$ and so $f(x-a)=f(x)+\pi^2(-a)$ and so $-a\in A$

If $a,b\in A$ and $x>0$ and $x+a+b>0$, then $2x+a+b>0$ and so $(x+a)+(x+b)>0$ and so either $x+a>0$, either $x+b>0$. Wlog say $x+a>0$
$x+a>0$ and $x+a+b>0$ and $b\in A$ $\implies$ $f(x+a+b)=f(x+a)+\pi^2b$
$x>0$ and $x+a>0$ and $a\in A$ $\implies$ $f(x+a)=f(x)+\pi^2a$
And so $f(x+a+b)=f(x)+\pi^2(a+b)$ $\forall x$ such that $x>0$ and $x+a+b>0$
So $a+b\in A$
Q.E.D.

3.2) $a\in A$ $\implies$ $\pi^2a\in A$
----------------------------------------
Let $a\in A$ and $x\in\mathbb R$ such that $x>0$ and $x+a>0$
$f(x+a)=f(x)+\pi^2a$ and so $\pi^2a=f(x+a)-f(x)$
And since $f(\mathbb R^+)\subseteq A$ and $A$ is an additive group, then  $\pi^2a\in A$
Q.E.D.

3.3) $A$ is dense in $\mathbb R$
--------------------------------
Note that if $a\in A^+$, then $\pi^2a\in A$ and $-9a\in A$ and so $(\pi^2-9)a\in A$ and $a>(\pi^2-9)a>0$
So $A^+$ has no min element.
And since $A$ is an additive group, we got the result.

3.4) $f(x)$ may be put in the form described in 1)
---------------------------------------------------------
Since $A$ is an additive group, than the relation $x\sim y$ $\iff$ $x-y\in A$ is an equivalence relation in $\mathbb R$
Let $r(x)$ any function from $\mathbb R\to\mathbb R^+$ which associates to a real a representant (unique per class) of its equivalence class (since $A\ne\{0\}$, such a positive representant always exists).

If $f(r(x))<\pi^2r(x)$ for some $x>0$ and since $A$ is dense in $\mathbb R$, $\exists a\in A$ such that $0<x+a<\frac{\pi^2r(x)-f(r(x))}{\pi^2}$
Then $f(x+a)=f(r(x)+x-r(x)+a)$ and since $x-r(x)\in A$, we get $x-r(x)+a\in A$ and $f(r(x)+x-r(x)+a)=f(r(x))+\pi^2(x-r(x)+a)<0$, impossible.
So $f(r(x))\ge \pi^2r(x)$ $\forall x$

Let then $g(x)$ from $\mathbb R^+\to A^+$ defined as $g(x)=f(x)$ (remember that $f(\mathbb R^+)\subseteq A^+$)
We get from the line above $g(r(x))=f(r(x))\ge \pi^2r(x)$

Then $f(x)=f(r(x)+x-r(x))$ and since $r(x)>0$ and $x-r(x)\in A$, we have $f(r(x)+x-r(x))=f(r(x))+\pi^2(x-r(x))$ ${=\pi^2x+g(r(x))-\pi^2r(x)}$
Q.E.D

And so we had a general solution.

4) some examples of solutions :
======================
4.1) The trivial one : $f(x)=\pi^2x+c$ whatever is $c\ge 0$
--------------------------------------------------------------
Just choose $A=\mathbb R$ and $r(x)=1$ and $g(x)=\pi^2x+c$ whatever is $c\ge 0$ and we get $f(x)=\pi^2x+c$

4.2) Many other solutions
------------------------------
Choose for example $A=\{h(\pi^2)$ $\forall$ polynomial $h(x)\in\mathbb Z[X]\}$
This is an additive subgroup of $\mathbb R$, stable thru multplication by $\pi^2$, and different from $\mathbb R$ (since countable)
Choosing any suitable $r(x)$ and $g(x)$ gives as many solutions we want.

And obviously, we can choose infinitely many other subgroups $A$
\end{solution}



\begin{solution}[by \href{https://artofproblemsolving.com/community/user/172163}{joybangla}]
	sorry pco I am not giving my sol because since you pointed out my sol is incorrect.So I wouldn't post it.sorry :blush:  :oops:
\end{solution}
*******************************************************************************
-------------------------------------------------------------------------------

\begin{problem}[Posted by \href{https://artofproblemsolving.com/community/user/173452}{Jayjayniboon}]
	Find all functions $f:\mathbf{R} \rightarrow \mathbf{R}$ such that $f\left(x-1\right)+f\left(1-\frac{4}{x}\right)=\ln x^2$.
	\flushright \href{https://artofproblemsolving.com/community/c6h537944}{(Link to AoPS)}
\end{problem}



\begin{solution}[by \href{https://artofproblemsolving.com/community/user/29428}{pco}]
	\begin{tcolorbox}$f:\mathbf{R} \rightarrow \mathbf{R}$ find all functions that $f(x-1)+f(1-\frac{4}{x})=ln x^2$\end{tcolorbox}
I suppose that you forgot to mention that domain of functional equation is $R^*$, else obviously no solution. :(

Let $g(x)$ from $\mathbb R\setminus\{-1\}\to\mathbb R\setminus\{+1\}$ defined as $g(x)=\frac{x-3}{x+1}$
Note that $\forall x\notin\{-1,+1\}$, $g(x)\ne-1$ and $g(g(x))\ne -1$ and $g(g(g(x)))=x$

Let $P(x)$ be the assertion $f(x)+f(g(x))=\ln (x+1)^2$ true $\forall x\ne -1$

Let $x\notin\{-1,+1\}$ :
$x\ne -1$ and so $P(x)$ $\implies$ $f(x)+f(g(x))=\ln (x+1)^2$
$g(x)\ne -1$ and so $P(g(x)$ $\implies$ $f(g(x))+f(g(g(x)))=\ln (g(x)+1)^2$
$g(g(x))\ne -1$ and so $P(g(g(x))$ $\implies$ $f(g(g(x)))+f(x)=\ln (g(g(x))+1)^2$

Adding the first and the third and subtracting the second, we get $f(x)=\frac 12(\ln (x+1)^2+\ln (g(g(x))+1)^2-\ln (g(x)+1)^2)$

So $f(x)=\frac 12\ln\left(\frac{(x+1)(g(g(x))+1)}{g(x)+1}\right)^2$ $=\ln\frac{2(x+1)^2}{(x-1)^2}$ $\forall x\notin\{-1,+1\}$ 

$P(1)$ $\implies$ $f(1)+f(-1)=\ln 4$

\begin{bolded}Hence the answer\end{underlined}\end{bolded} :
$f(1)=a$ whetever is $a\in\mathbb R$
$f(-1)=-a+\ln 4$
$\forall x\notin\{-1,+1\}$ : $f(x)=\ln\frac{2(x+1)^2}{(x-1)^2}$

And it is easy to check back that this indeed is a solution
\end{solution}



\begin{solution}[by \href{https://artofproblemsolving.com/community/user/160887}{aditi1012207}]
	plug in x= y+1 in given eqn to get: 
f(y) + f((y-3)\/(y+1)) = ln(y+1)^2......(1)
plug y= (z-3)\/(z+1) in (1) to get:

f((z-3)\/(z+1)) + f((z+3)\/(1-z)) = ln( [2(x-1)\/(x+!)] ^2).......(2)
plug y= (z+3)\/(z-1) in (1) to get:

f((z+3)\/(z-1)) + f(z) = ln( [4\/(1-x)]^2 ).....(3)
subtract (2) from (3) to get:

f(z) - f((z-3)\/(z+1)) = ln( 4(x+1)^2\/(1-x)^4 ).............(4)
plug yz in (1), then add to (4) to finally get, 

f(z)= ln( 2(x+1)^2\/(x-1)^2 )
\end{solution}
*******************************************************************************
-------------------------------------------------------------------------------

\begin{problem}[Posted by \href{https://artofproblemsolving.com/community/user/175698}{emptyhands7697}]
	Find $f: \mathbb{N} \rightarrow  \mathbb{N}$ satisfying, in turn
a) $f(f(n)) = n + 2012$;
b) $f(f(n)) = 3n$;
c) $f(f(n)) = pn$ ($p$ being a prime).
	\flushright \href{https://artofproblemsolving.com/community/c6h538089}{(Link to AoPS)}
\end{problem}



\begin{solution}[by \href{https://artofproblemsolving.com/community/user/29428}{pco}]
	\begin{tcolorbox}Find $f: \mathbb{N} \rightarrow  \mathbb{N}$ satisfying, in turn
a) $f(f(n)) = n + 2012$;\end{tcolorbox}
General solution :
Let $A,B$ any split of $\{1,2,3,...,2012\}$ in two equinumerous subsets of $1006$ elements each.
Let $h(x)$ any bijection from $A\to B$

Let $x=2012k+r$ with $r\in\{1,2,...,2011,2012\}$
If $r\in A$, then $f(x)=2012k+h(r)$
If $r\in B$, then $f(x)=2012(k+1)+h^{-1}(r)$
\end{solution}



\begin{solution}[by \href{https://artofproblemsolving.com/community/user/29428}{pco}]
	\begin{tcolorbox}Find $f: \mathbb{N} \rightarrow  \mathbb{N}$ satisfying, in turn
b) $f(f(n)) = 3n$;\end{tcolorbox}
General solution :
Let $A,B$ any split of $E=\{n\in\mathbb N$ such that $n\not\equiv 0\pmod 3\}$ in two equinumerous subsets.
Let $h(x)$ any bijection from $A\to B$

Let $r(n)=n3^{-v_3(n)}$ function from $\mathbb N\to E$

If $r(n)\in A$, then $f(n)=3^{v_3(n)}h(r(n))$
If $r(n)\in B$, then $f(n)=3^{1+v_3(n)}h^{-1}(r(n))$
\end{solution}



\begin{solution}[by \href{https://artofproblemsolving.com/community/user/29428}{pco}]
	\begin{tcolorbox}Find $f: \mathbb{N} \rightarrow  \mathbb{N}$ satisfying, in turn
c) $f(f(n)) = pn$ ($p$ being a prime).\end{tcolorbox}
General solution :
Let $A,B$ any split of $E=\{n\in\mathbb N$ such that $n\not\equiv 0\pmod p\}$ in two equinumerous subsets.
Let $h(x)$ any bijection from $A\to B$

Let $r(n)=np^{-v_p(n)}$ function from $\mathbb N\to E$

If $r(n)\in A$, then $f(n)=p^{v_p(n)}h(r(n))$
If $r(n)\in B$, then $f(n)=p^{1+v_p(n)}h^{-1}(r(n))$
\end{solution}



\begin{solution}[by \href{https://artofproblemsolving.com/community/user/175698}{emptyhands7697}]
	Sorry but what does $v_{3}$  and  $v_{p}$ mean? Can you explain clearly for me?
\end{solution}



\begin{solution}[by \href{https://artofproblemsolving.com/community/user/183430}{lafius}]
	I think that $ v_{3}(n) $ and $ v_{p}(n) $ denote respectively the greatest power of 3 and p (a prime number) that divides n
\end{solution}
*******************************************************************************
-------------------------------------------------------------------------------

\begin{problem}[Posted by \href{https://artofproblemsolving.com/community/user/157508}{hctb00}]
	1:$f:R->R,f(x+y+z)+f(x)+f(y)+f(z)=f(x+y)+f(y+z)+f(z+x),$find$ f$
2:$f:R->R,f(x+f(y))+f(y+f(z))+f(z+f(x))=0,$find $f$
	\flushright \href{https://artofproblemsolving.com/community/c6h542527}{(Link to AoPS)}
\end{problem}



\begin{solution}[by \href{https://artofproblemsolving.com/community/user/29428}{pco}]
	\begin{tcolorbox}2:$f:R->R,f(x+f(y))+f(y+f(z))+f(z+f(x))=0,$find $f$\end{tcolorbox}
Let $P(x,y,z)$ be the assertion $f(x+f(y))+f(y+f(z))+f(z+f(x))=0$
Let $a=f(0)$

1) functional equation is equivalent to $f(x+f(y))=f(x)-f(y)$ $\forall x,y$
===============================================
$P(0,0,0)$ $\implies$ $f(a)=0$
$P(x,0,0)$ $\implies$ $f(x+a)+f(f(x))=0$
$P(x,a,0)$ $\implies$ $f(x)+f(2a)+f(f(x))=0$
$P(x,y,0)$ $\implies$ $f(x+f(y))+f(y+a)+f(f(x))=0$
And so new assertion $Q(x,y)$ : $f(x+f(y))=f(x)-f(y)$
And it's obvious to see that this necessary condition is also sufficient

2) General solution of functional equation $f(x+f(y))=f(x)-f(y)$ $\forall x,y$
=================================================
2.1 Claim about general solution
-------------------------------
Let $A$ any additive subgroup of $\mathbb R$
Let $\sim$ the equivalence relation in $\mathbb R$ defined as $x\sim y$ $\iff$ $x-y\in A$
Let $r(x)$ any function which associates to a real $x$ a representent (unique per class) of its equivalence class.
Let $g(x)$ any function from $\mathbb R\to A$

Then we can write $f(x)=g(r(x))+r(x)-x$

2.2 Proof that any function of the form defined in 2.1 indeed is a solution
-----------------------------------------------------------------------------
$x\sim r(x)$ and so $x-r(x)\in A$
$g(r(x))\in A$ and $x-r(x)\in A$ and $A$ additive subgroup $\implies$ $f(x)\in A$
$f(y)\in A$ $\implies$ $x+f(y)\sim x$ and so $r(x+f(y))=r(x)$
So $f(x+f(y))=g(r(x+f(y)))+r(x+f(y))-(x+f(y))$ $=g(r(x))+r(x)-x-f(y)$ $=f(x)-f(y)$
Q.E.D.

2.3 Proof that any solution may be written in the form 2.1, and so we indeed got a general form for solutions
--------------------------------------------------------------------------------------------------------
Let $f(x)$ any solution of the functional equation $f(x+f(y))=f(x)-f(y)$
Let $A=\{u\in\mathbb R$ such that $f(x+u)=f(x)-u$ $\forall x\in\mathbb R\}$
$0\in A$
$u\in A$ $\implies$ $f((x-u)+u)=f(x-u)-u$ $f(x-u)=f(x)+u$ $\implies$ $-u\in A$
$u,v\in A$ $\implies$ $f(x+u+v)=f(x+u)-v=f(x)-u-v$ $\implies$ $u+v\in A$
So $A$ is an additive subgroup
Note that $f(\mathbb R)\subseteq A$

Let $\sim$ the equivalence relation in $\mathbb R$ defined as $x\sim y$ $\iff$ $x-y\in A$
Let $r(x)$ any function which associates to a real $x$ a representent (unique per class) of its equivalence class.
Let $g(x)=f(x)$

$f(x)=f(r(x)+(x-r(x)))$ and since $x-r(x)\in A$, we get $f(x)=f(r(x))-(x-r(x))$

So $f(x)=g(r(x))+r(x)-x$
Q.E.D.


3) Some examples of solutions
======================
3.1) $A=\mathbb R$
------------------
So $r(x)=c$ constant
So $g(r(x))=d$ constant
So $f(x)=d+c-x$ and we got the solutions $\boxed{f(x)=a-x}$ $\forall x$ and for any real $a$

3.2) $A=\{0\}$
------------
So $r(x)=x$
And $g(r(x))=0$ (since $g(x)$ is from $\mathbb R\ro A=\{0\}$
So $f(x)=0+x-x$ and we got the solution $\boxed{f(x)=0}$ $\forall x$

3.3 $A=\mathbb Z$
----------------
Choose for example $r(x)=x-\lfloor x\rfloor$ anf $g(x)=\lfloor 10\sin 2\pi x\rfloor$

Then $\boxed{f(x)=\left\lfloor 10\sin 2\pi x\right\rfloor-\lfloor x\rfloor}$
(this is ust one example amongst infinitely many)

3.4 $A=\mathbb Q$
----------------
Let $Q,B$ be two supplementary vector subspaces of the $\mathbb Q$-vectorspace $\mathbb R$
Let $a(x)$ from $\mathbb R\to \mathbb Q$ and $b(x)$ from $\mathbb R\to B$ the projections of $x$ in $\mathbb Q,B$ so that $x=a(x)+b(x)$
Let $h(x)$ any function from $\mathbb R\to\mathbb R$

Choose $r(x)=b(x)$ and $g(x)=a(h(x))$

Then $\boxed{f(x)=a(h(b(x)))-a(x)}$


And infinitely many other solutions.
\end{solution}
*******************************************************************************
-------------------------------------------------------------------------------

\begin{problem}[Posted by \href{https://artofproblemsolving.com/community/user/180422}{pmtrig}]
	Find all functions $f: \mathbb{R} \rightarrow \mathbb{R}$ such that
                                                                                                              $f(x+f(y))=f(x+y)+f(y)$  $\forall x,y \in \mathbb{R}$
	\flushright \href{https://artofproblemsolving.com/community/c6h543699}{(Link to AoPS)}
\end{problem}



\begin{solution}[by \href{https://artofproblemsolving.com/community/user/29428}{pco}]
	\begin{tcolorbox}Find all functions $f: \mathbb{R} \rightarrow \mathbb{R}$ such that
                                                                                                              $f(x+f(y))=f(x+y)+f(y)$  $\forall x,y \in \mathbb{R}$\end{tcolorbox}
Let $P(x,y)$ be the assertion $f(x+f(y))=f(x+y)+f(y)$

$P(x-y,y)$ $\implies$ $f(x+f(y)-y)=f(x)+f(y)$ $\implies$ $f(x+f(y)-y)-(x+f(y)-y)=(f(x)-x)+y$

Setting $g(x)=f(x)-x$, this becomes new assertion $Q(x,y)$ : $g(x+g(y))=g(x)+y$

$g(x)$ is bijective and so (since injective) $Q(x,0)$ $\implies$ $g(0)=0$ and $Q(0,x)$ $\implies$ $g(g(x))=x$

So $Q(x,g(y))$ $\implies$ $g(x+y)=g(x)+g(y)$ and we easily get that $g(x)$ is any involutive additive function.

\begin{bolded}Hence the answer\end{underlined}\end{bolded} : $f(x)=x+g(x)$ where $g(x)$ is any involutive additive function.

Nota \end{underlined}: a general form for involutive additive functions is easy :
Choose any pair $A,B$ of supplementary vector subspaces of the $\mathbb Q$-vectorspace $\mathbb R$
Let $a(x)$ from $\mathbb R\to A$ and $b(x)$ from $\mathbb R\to B$ the projections of $x$ in $A$ and $B$ so that $x=a(x)+b(x)$
Then $g(x)=a(x)-b(x)$

\begin{bolded}Examples \end{underlined}\end{bolded}:
$(A,B)=(\mathbb R,\{0\})$ $\implies$ $a(x)=x$ and $b(x)=0$ and so $g(x)=x$ and so $\boxed{f(x)=2x}$

$(A,B)=(\{0\},\mathbb R)$ $\implies$ $a(x)=0$ and $b(x)=x$ and so $g(x)=-x$ and so $\boxed{f(x)=0}$

And infinitely many non continuous other solutions
\end{solution}



\begin{solution}[by \href{https://artofproblemsolving.com/community/user/180422}{pmtrig}]
	Very nice solution Patrick. Thank you very much..
\end{solution}



\begin{solution}[by \href{https://artofproblemsolving.com/community/user/185157}{octogon}]
	Similar problem but no one post full solution:
 http://www.artofproblemsolving.com/Forum/viewtopic.php?t=543976&p3143141#p3143141
\end{solution}
*******************************************************************************
-------------------------------------------------------------------------------

\begin{problem}[Posted by \href{https://artofproblemsolving.com/community/user/185157}{octogon}]
	Find all functions $f:\mathbb{R} \rightarrow \mathbb{R}$ such that 
                                                         $f(x+f(y))=f(x+y)+2y$              $\forall x,y \in \mathbb{R}$
	\flushright \href{https://artofproblemsolving.com/community/c6h543976}{(Link to AoPS)}
\end{problem}



\begin{solution}[by \href{https://artofproblemsolving.com/community/user/180422}{pmtrig}]
	Assume $f(0)=a$ and let $x=y=0$ so we get that $f(a)=a$. Now let $y=a$ $\Rightarrow$ $f(x+a)=f(x+a)+2a$ so $f(0)=0$.
$(i)$..... $f(b)=0$ $\Leftrightarrow$ $b=0$...
assume $f(b)=0$ and in the equation let $x=0$ and  $y=b \Rightarrow f(f(b))=f(b)+2b$ since $f(b)=f(0)=f(f(b))=0$ we get $b=o$.
$(ii)... f(b)=b \Leftrightarrow b=0$.....
assume $f(b)=b$... in the equation let $y=b \Rightarrow f(x+b)=f(x+b)+2b \Rightarrow b=0$.
Now let $x \to f(x)$ then $f(f(x)+f(y))=f(y+f(x))+2y=f(x+y)+2(x+y)$...............(1)
and in (1) let $x=-y \Rightarrow f(f(y)+f(-y))=0 \Rightarrow f(y)+f(-y)=0$ so $f$ is an odd function.
In the original equation let $x=-y \Rightarrow f(f(y)-y)=2y$ then let $y \to f(y)-y$ and $x \to -f(y) 
 \Rightarrow f(2y-f(y))=f(-y)+2y=2y-f(y)$ so from $(ii)$ we get that $2y-f(y)=0$ and $f(y)=2y \forall y \in \mathbb{R}$. Q.E.D
\end{solution}



\begin{solution}[by \href{https://artofproblemsolving.com/community/user/156523}{dizzy}]
	\begin{tcolorbox}
 then let $y \to f(y)-y$ and $x \to -f(y) 
 \Rightarrow f(2y-f(y))=f(-y)+2y=2y-f(y)$ \end{tcolorbox}
It must be $ f(2y-f(y))=f(-y)+2(f(y)-y) $. The conclusion is not true, because $ f(x)=-x $, for all real $ x $ is also a solution
\end{solution}



\begin{solution}[by \href{https://artofproblemsolving.com/community/user/180422}{pmtrig}]
	\begin{tcolorbox}[quote="pmtrig"]
 then let $y \to f(y)-y$ and $x \to -f(y) 
 \Rightarrow f(2y-f(y))=f(-y)+2y=2y-f(y)$ \end{tcolorbox}
It must be $ f(2y-f(y))=f(-y)+2(f(y)-y) $. The conclusion is not true, because $ f(x)=-x $, for all real $ x $ is also a solution\end{tcolorbox}
you are right "dizzy".!
\end{solution}



\begin{solution}[by \href{https://artofproblemsolving.com/community/user/185304}{aymas}]
	[hide="hint"]
Let $P(x,y)$ be the assertion of the functional equation $f(x+f(y))=f(x+y)+2y$
$P(0,x)$ $(1)-f(f(x))=f(x)+2x$ so $f$ is injective .
$P(\frac{-x+f(0)}{2},\frac{x-f(0)}{2})$ : $ f(something)=x$ so $f$ is surjective .
Hence $f$ is bijective .
$P(f(x),y)$ $f(f(x)+f(y))=f(f(x)+y)+2y=f(x+y)+2(x+y)$
But $f(f(x+y))=f(x+y)+2(x+y)$
So $f(f(x+y))=f(f(x)+f(y))$ by injectivity $(2)f(x+y)=f(x)+f(y)$
So $f$ is additive.
$P(x,y)$ become $f(x)+f(f(y))=f(x+y)+2y$
Which's always true for all function that satifie relation $(1)$ and $(2)$ .
So the question becomes :
find all aditive functions such that :
$f(f(x))=f(x)+2x$ for all $x$ 
[\/hide]
\end{solution}



\begin{solution}[by \href{https://artofproblemsolving.com/community/user/180422}{pmtrig}]
	\begin{tcolorbox}[hide="hint"]
Let $P(x,y)$ be the assertion of the functional equation $f(x+f(y))=f(x+y)+2y$
$P(0,x)$ $(1)-f(f(x))=f(x)+2x$ so $f$ is injective .
$P(\frac{-x+f(0)}{2},\frac{x-f(0)}{2})$ : $ f(something)=x$ so $f$ is surjective .
Hence $f$ is bijective .
$P(f(x),y)$ $f(f(x)+f(y))=f(f(x)+y)+2y=f(x+y)+2(x+y)$
But $f(f(x+y))=f(x+y)+2(x+y)$
So $f(f(x+y))=f(f(x)+f(y))$ by injectivity $(2)f(x+y)=f(x)+f(y)$
So $f$ is additive.
$P(x,y)$ become $f(x)+f(f(y))=f(x+y)+2y$
Which's always true for all function that satifie relation $(1)$ and $(2)$ .
So the question becomes :
find all aditive functions such that :
$f(f(x))=f(x)+2x$ for all $x$ 
[\/hide]\end{tcolorbox}
I have also found these. What should we do next?
\end{solution}



\begin{solution}[by \href{https://artofproblemsolving.com/community/user/67223}{Amir Hossein}]
	Apparently still unsolved (this post will be deleted as soon as someone posts a solution).
\end{solution}
*******************************************************************************
-------------------------------------------------------------------------------

\begin{problem}[Posted by \href{https://artofproblemsolving.com/community/user/68025}{Pirkuliyev Rovsen}]
	Find all functions $f: \mathbb{Z}\to\mathbb{Z}$ such that $f(-1)=f(1)$ and $f(x)+f(y)=f(x+2xy)+f(y-2xy)$ for all $x,y{\in}Z$.

___________________________________
Azerbaijan Land of the Fire 
	\flushright \href{https://artofproblemsolving.com/community/c6h544699}{(Link to AoPS)}
\end{problem}



\begin{solution}[by \href{https://artofproblemsolving.com/community/user/29428}{pco}]
	\begin{tcolorbox}Find all functions $f: \mathbb{Z}\to\mathbb{Z}$ such that $f(-1)=f(1)$ and $f(x)+f(y)=f(x+2xy)+f(y-2xy)$ for all $x,y{\in}Z$.\end{tcolorbox}
Let $P(x,y)$ be the assertion $f(x)+f(y)=f(x+2xy)+f(y-2xy)$
Let $a=f(1)=f(-1)$

Note that $P(x,0)$ shows that $f(0)$ can be any value (since $x+2xy\ne 0$ and $y-2xy\ne 0$ whenever $x,y\ne 0)$

$P(1,x)$ $\implies$ $a+f(x)=f(2x+1)+f(-x)$
$P(x,-1)$ $\implies$ $f(x)+a=f(-x)+f(2x-1)$
And so $f(2x+1)=f(2x-1)$ $\forall x$ and so $f(2x+1)=a$ $\forall x\in\mathbb Z$

$P(1,x)$ $\implies$ $f(-x)=f(x)$ $\forall x\in\mathbb Z$

$P(x,2n+1)$ $\implies$ $f(x)=f((4n+3)x)$
$P(2n+1,x)$ $\implies$ $f(x)=f((4n+1)x)$
And so $f((2n+1)x)=f(x)$

And obviously this mandatory property is sufficient.

\begin{bolded}Hence the answer\end{underlined}\end{bolded} : 

$f(0)$ can be any value
When $x\ne 0$, $f(x)=g(v_2(|x|))$ where $g(x)$ is any function from $\mathbb N\cup\{0\}\to\mathbb Z$
\end{solution}
*******************************************************************************
-------------------------------------------------------------------------------

\begin{problem}[Posted by \href{https://artofproblemsolving.com/community/user/178156}{War-Hammer}]
	Find all functions $f: \mathbb{N}\to\mathbb{N}$ such that :

\[ 2f(n+3)f(n+2)=f(n+1)+f(n)+1 \]
For all $n \in \mathbb {N}$.
	\flushright \href{https://artofproblemsolving.com/community/c6h544722}{(Link to AoPS)}
\end{problem}



\begin{solution}[by \href{https://artofproblemsolving.com/community/user/153026}{msaeids}]
	I think it can be done using recursive functions.
\end{solution}



\begin{solution}[by \href{https://artofproblemsolving.com/community/user/178156}{War-Hammer}]
	I don't think so , if you can do it.
\end{solution}



\begin{solution}[by \href{https://artofproblemsolving.com/community/user/29428}{pco}]
	\begin{tcolorbox}Find all functions $f: \mathbb{N}\to\mathbb{N}$ such that :

\[ 2f(n+3)f(n+2)=f(n+1)+f(n)+1 \]
For all $n \in \mathbb {N}$.\end{tcolorbox}
Let $P(n)$ be the assertion $2f(n+3)f(n+2)=f(n+1)+f(n)+1$

Subtracting $P(n)$ from $P(n+1)$, we get $f(n+2)-f(n)=2f(n+3)(f(n+4)-f(n+2))$

So $f(n+2)-f(n)=2^k\prod_{i=1}^{k}\left(f(n+2+i)(f(n+3+i)-f(n+1+i))\right)$

Setting $k\to+\infty$, we get $f(n+2)=f(n)$ and so $f(n)$ is $a,b,a,b,a,b,a,b,....$ with $2ab=a+b+1$

And so $(a,b)\in\{(1,2),(2,1)\}$

\begin{bolded}And the two solutions\end{underlined}\end{bolded} :

$f(2n)=1$ and $f(2n+1)=2$

$f(2n)=2$ and $f(2n+1)=1$
\end{solution}



\begin{solution}[by \href{https://artofproblemsolving.com/community/user/185567}{dionescu}]
	Or observe that:
1) f is strictly positive for n greater than 1
2) f is bounded
3) Take the equality given at a max point M and show that the points surrounding the max have to be 1, and also the previous two have to be M-1 and and M. Therefore M=2 and the conclusion follows. (you need a bit of care here but it should work)
\end{solution}
*******************************************************************************
-------------------------------------------------------------------------------

\begin{problem}[Posted by \href{https://artofproblemsolving.com/community/user/185316}{truonghung}]
	Find all $f: \mathbb{R} \to \mathbb{R}$ such that:

\[f(f(x+y)) = f(x) + f(y) +f(x)f(y) - xy\] $\forall x;y \in \mathbb{R}$

Thanks! :)
	\flushright \href{https://artofproblemsolving.com/community/c6h544799}{(Link to AoPS)}
\end{problem}



\begin{solution}[by \href{https://artofproblemsolving.com/community/user/29428}{pco}]
	\begin{tcolorbox}Find all $f: \mathbb{R} \to \mathbb{R}$ such that:

\[f(f(x+y)) = f(x) + f(y) +f(x)f(y) - xy\] $\forall x;y \in \mathbb{R}$

Thanks! :)\end{tcolorbox}
Let $P(x,y)$ be the assertion $f(f(x+y))=f(x)+f(y)+f(x)f(y)-xy$

Let $g(x)=1+f(x)$. Subtracting $P(x+y,0)$ from $P(x,y)$, we get new assertion $Q(x,y)$ : $g(0)g(x+y))=g(x)g(y)-xy$

1) If $g(0)=0$
We get $g(x)g(y)=xy$ and so $g(1)=\pm 1$ and $g(x)=\frac 1{g(1)}x$ and so $g(x)=x$ or $g(x)=-x$ and so two possibilities :
$f(x)=x-1$ which is not a solution of original equation
$f(x)=-x-1$ which is not a solution of original equation

2) If $g(0)\ne 0$
Let then $h(x)=\frac{g(xg(0))}{g(0)}$ and $Q(x,y)$ becomes new assertion $R(x,y)$ : $h(x+y)=h(x)h(y)-xy$ with $h(0)=1$

$R(y,1)$ $\implies$ $h(y+1)=h(y)h(1)-y$
$R(x,y+1)$ $\implies$ $h(x+y+1)=h(x)h(y+1)-xy-x$ $=h(x)h(y)h(1)-h(x)y-xy-x$
Swapping $x,y$, we get $h(x+y+1)=h(x)h(y)h(1)-h(y)x-xy-y$
Subtracting, we get $h(x)y+x=h(y)x+y$
Setting $y=1$, we get $h(x)=(h(1)-1)x+1$

So $f(x)=ux+v$ for some $u,v$

Plugging this in original equation, we get two solutions : $\boxed{f(x)=x}$ and $\boxed{f(x)=-x-2}$
\end{solution}
*******************************************************************************
-------------------------------------------------------------------------------

\begin{problem}[Posted by \href{https://artofproblemsolving.com/community/user/149744}{ryuzaki}]
	Find all functions  $f: \mathbb{R}\to \mathbb{R}$, such that $f(f(x)+y) = f(x)+f(y)$ for all $x,y \in \mathbb{R}$.
	\flushright \href{https://artofproblemsolving.com/community/c6h544838}{(Link to AoPS)}
\end{problem}



\begin{solution}[by \href{https://artofproblemsolving.com/community/user/29428}{pco}]
	\begin{tcolorbox}Find all functions  $f: \mathbb{R}\to \mathbb{R}$, such that $f(f(x)+y) = f(x)+f(y)$ for all $x,y \in \mathbb{R}$.\end{tcolorbox}
1) Claim about general solution
=====================
Let $A$ any additive subgroup of $\mathbb R$
Let $\sim$ the equivalence relation in $\mathbb R$ such that $x\sim y$ $\iff$ $x-y\in A$
Let $r(x)$ from $\mathbb R\to\mathbb R$ a function which associates to a real a representant (unique per class) of its equivalence class.
Let $g(x)$ any function from $\mathbb R\to A$

$f(x)=x+g(r(x))-r(x)$

2) Proof that any function in the form defined in 1) above indeed is a solution.
======================================================
$g(r(x))\in A$ and $x-r(x)\in A$ and $A$ is an additive group, so $f(x)=g(r(x))+(x-r(x))\in A$
So $f(x)+y\sim y$ and so $r(f(x)+y)=r(y)$

Then $f(f(x)+y)=f(x)+y+g(r(f(x)+y))-r(f(x)+y)$ $=f(x)+y+g(r(y))-r(y)$ $=f(x)+f(y)$
Q.E.D.

3) Proof that any solution may be written in the form defined in 1), so that we indeed got a general solution
==========================================================================
Let $f(x)$ from $\mathbb R\to\mathbb R$ such that $f(f(x)+y)=f(x)+f(y)$ $\forall x,y\in\mathbb R$

Let $A=\{a\in\mathbb R$ such that $f(x+a)=f(x)+a$ $\forall x\in\mathbb R\}$
Note that $f(\mathbb R)\subseteq A$

$f(x+0)=f(x)+0$ $\forall x$ $\implies$ $0\in A$
$a\in A$ $\implies$ $f(x+a)=f(x)+a$ $\forall x$ $\implies$ $f(x)=f(x-a)+a$ $\forall x$ $\implies$ $f(x-a)=f(x)-a$ $\forall x$ $\implies$ $-a\in A$
$a,b\in A$ $\implies$ $f(x+a+b)=f((x+a)+b)=f(x+a)+b=f(x)+a+b$ $\forall x$ $\implies$ $a+b\in A$

So $A$ is an additive subgroup of $\mathbb R$
So $\sim$ defined as $x\sim y$ $\iff$ $x-y\in A$ is an equivalence relation
Let then $r(x)$ from $\mathbb R\to\mathbb R$ any function which associates to a real a representant (unique per class) of its equivalence class.
Let $g(x)=f(x)$ : $g(x)$ is a function from $\mathbb R\to A$

$f(x)=f(r(x)+(x-r(x))$ and, since $x-r(x)\in A$, we get :
$f(x)=f(r(x)+(x-r(x))$ $=f(r(x))+x-r(x)$ $=g(r(x))+x-r(x)$
Q.E.D.

4) Some examples
=============
4.1) $A=\{0\}$
--------------
So $r(x)=x$ and $g(x)=0$ and so $\boxed{f(x)=0}$ $\forall x$

4.2) $A=\mathbb R$
-----------------
So $r(x)=c$ and $g(r(x))=d$ and $\boxed{f(x)=x+a}$ $\forall x$

4.3) $A=\mathbb Z$
------------------
Choose for example $r(x)=x-\lfloor x\rfloor$
Choose for example $g(x)=\left\lfloor 100\sin(2\pi x)\right\rfloor$
Then $\boxed{f(x)=\left\lfloor 100\sin(2\pi x)\right\rfloor+\lfloor x\rfloor}$

And infinitely many other solutions (choosing other sets $A$ and varying $r(x)$ and $g(x)$)
\end{solution}



\begin{solution}[by \href{https://artofproblemsolving.com/community/user/10045}{socrates}]
	Does the above analysis work when $f:\Bbb{R}^+\to \Bbb{R}^+$?
\end{solution}



\begin{solution}[by \href{https://artofproblemsolving.com/community/user/31919}{tenniskidperson3}]
	There's no additive subgroup of $\mathbb{R}^{+}$ because you always need $0$ to be in it.  So no, there must be a different analysis.
\end{solution}



\begin{solution}[by \href{https://artofproblemsolving.com/community/user/29428}{pco}]
	\begin{tcolorbox}Does the above analysis work when $f:\Bbb{R}^+\to \Bbb{R}^+$?\end{tcolorbox}
If the problem is transformed in "find all functions from $\mathbb R^+\to\mathbb R^+$ such that $f(f(x)+y)=f(x)+f(y)$ $\forall x,y>0$", then the solution is quite similar :
1) Claim about general solution
=====================
Let $A$ any additive subgroup of $\mathbb R$ different from$\{0\}$
Let $\sim$ the equivalence relation in $\mathbb R^+$ such that $x\sim y$ $\iff$ $x-y\in A$
Let $r(x)$ from $\mathbb R^+\to\mathbb R^+$ a function which associates to a real a representant (unique per class) of its equivalence class. (such a function exists since $A\ne\{0\}$ and so is infinite and contains positive elements as great as we want).
Let $g(x)$ any function from $\mathbb R^+\to A\setminus(-\infty,x)$ so that $g(x)-x\ge 0$ (one again, such a function exists since $A\ne\{0\}$ and so is infinite and contains positive elements as great as we want).

$f(x)=x+g(r(x))-r(x)$

2) Proof that any function in the form defined in 1) above indeed is a solution.
======================================================
$g(r(x))\ge r(x)$ and so $f(x)>0$, as required.
$g(r(x))\in A$ and $x-r(x)\in A$ and $A$ is an additive group, so $f(x)=g(r(x))+(x-r(x))\in A$
So $f(x)+y\sim y$ and so $r(f(x)+y)=r(y)$

Then $f(f(x)+y)=f(x)+y+g(r(f(x)+y))-r(f(x)+y)$ $=f(x)+y+g(r(y))-r(y)$ $=f(x)+f(y)$
Q.E.D.

3) Proof that any solution may be written in the form defined in 1), so that we indeed got a general solution
==========================================================================
Let $f(x)$ from $\mathbb R^+\to\mathbb R^+$ such that $f(f(x)+y)=f(x)+f(y)$ $\forall x,y>0$

Let $A=\{a\in\mathbb R$ such that $f(x+|a|)=f(x)+|a|$ $\forall x\in\mathbb R^+\}$
Note that $f(\mathbb R^+)\subseteq A$

It's not very difficult to show that $A$ is an additive subgroup of $\mathbb R$
And since $f(1)\ne 0$ and $f(1)\in A$, we get that $A\ne \{0\}$

So $\sim$ defined as $x\sim y$ $\iff$ $x-y\in A$ is an equivalence relation
Let then $r(x)$ from $\mathbb R^+\to\mathbb R^+$ any function which associates to a real a representant (unique per class) of its equivalence class.(such a function exists since $A\ne\{0\}$ and so is infinite and contains positive elements as great as we want).

Let $g(x)=f(x)\in A$
Let $a\in A$ :
If $x>0\ge a$, then $f(x)>0>a$
If $x>a>0$, then $f((x-a)+a)=f(x-a)+a>a$ and so $f(x)>a$
So $g(x)$ is function from $\mathbb R^+\to A\setminus(-\infty,x)$ so that $g(x)-x\ge 0$

If $x\ge r(x)$ : $f(x)=f(r(x)+|x-r(x)|)$ and, since $x-r(x)\in A$, we get $f(x)=f(r(x))+|x-r(x)|$ $=g(r(x))+x-r(x)$
If $x<r(x)$ : $g(r(x))=f(r(x))=f(x+|x-r(x)|)=f(x)+|x-r(x)|=f(x)-x+r(x)$ and so $f(x)=g(r(x))+x-r(x)$
Q.E.D.

4) Some examples
=============
($A=\{0\}$ is no longer possible)

4.1) $A=\mathbb R$
-----------------
So $r(x)=c>0$ and $g(r(x))=d\ge c$ and $\boxed{f(x)=x+a}$ $\forall x$ and whatever is $a\ge 0$

4.2) $A=\mathbb Z$
------------------
Choose for example $r(x)=1+x-\lfloor x\rfloor\in[1,2)$
Choose for example $g(x)=\left\lfloor x+1+100\sin^2(2\pi x)\right\rfloor$
Then $\boxed{f(x)=\left\lfloor x+1+ 100\sin^2(2\pi x)\right\rfloor}$

And infinitely many other solutions (choosing other sets $A$ and varying $r(x)$ and $g(x)$)
\end{solution}
*******************************************************************************
-------------------------------------------------------------------------------

\begin{problem}[Posted by \href{https://artofproblemsolving.com/community/user/68025}{Pirkuliyev Rovsen}]
	Find all functions ${{f,g: \mathbb(0;+\infty)}\to\mathbb(0;+\infty)}$ such that  $f(x+g(y))=f(x+y)+g(y)$ and $g(x+f(y))=g(x+y)+f(y)$ for all $x,y{\in}R$.
	\flushright \href{https://artofproblemsolving.com/community/c6h545187}{(Link to AoPS)}
\end{problem}



\begin{solution}[by \href{https://artofproblemsolving.com/community/user/29428}{pco}]
	\begin{tcolorbox}... and $g(x+f(y))=g(x+y)+f(y)$ for all $x,y{\in}R$.\end{tcolorbox}
I suppose you mean $\forall x,y>0$ and not $\forall x,y\in\mathbb R$ :?:
\end{solution}



\begin{solution}[by \href{https://artofproblemsolving.com/community/user/68025}{Pirkuliyev Rovsen}]
	Yes, must be  $x,y>0$.Thanks you.
\end{solution}



\begin{solution}[by \href{https://artofproblemsolving.com/community/user/29428}{pco}]
	\begin{tcolorbox}Find all functions ${{f,g: \mathbb(0;+\infty)}\to\mathbb(0;+\infty)}$ such that  $f(x+g(y))=f(x+y)+g(y)$ and $g(x+f(y))=g(x+y)+f(y)$ for all $x,y{\in}R$.\end{tcolorbox}
\begin{bolded}Some examples\end{bolded} :

1) $f(x)=ax$ and $g(x)=\frac a{a-1}x$ where $a>1$

2) $f(x)=x+a(x)$ and $g(x)=x+a^{[-1]}(x)$ where $a(x)$ is any additive bijective function from $\mathbb R^+\to\mathbb R^+$

\begin{bolded}And a question\end{underlined} \end{bolded}: is it a real exercise you got in an olympiad contest ?
\end{solution}



\begin{solution}[by \href{https://artofproblemsolving.com/community/user/49556}{xxp2000}]
	1) $F(x)=f(x)-x>0$ and $G(x)=g(x)-x>0$.
$g(a)=a$ implies $g(a)=0$. Absurd!
Suppose $g(a)=b<a$. $f(x+b)=f(x+a)+b$ implies $f(x+n(a-b)+b)=f(x+b)-nb$ for any integer $n>0$. We can fix $x$ and find large $n$ to make $f(x+b)<nb$. Absurd! So $G(x)>0$. Similarly we can show $F(x)>0$.

2) $F(x+G(y))=F(x)+y$ and $G(x+F(y))=G(x)+y$.
With 1), we can rewrite the f.e. as $F(x+y+G(y))=F(x+y)+y$ or $F(x+G(y))=F(x)+y,x>y>0$. 
When $0<x_0<y_1$, we can pick $0<y_2<x_0$. Let $G(y_1)=G_1$ and $G(y_2)=G_2$.
Pick $n$ such that $x_0+nG_2>y_1$, then we have
$F(x_0+nG_2)=F(x_0)+ny_2$, 
$F(x_0+nG_2+G_1)=F(x_0+G_1)+ny_2$, and
$F(x_0+nG_2+G_1)=F(x_0+nG_2)+y_1$.
So $F(x_0+G_1)=F(x_0)+y_1$ and $F(x+G(y))=F(x)+y,x,y>0$. Similarly we can show the other half.

3) $F$ and $G$ are linear
Swap $x,y$ in $G(F(x)+F(y))=G(F(x))+y$, we get $G(F(x))=x+c_1$.
Then $G(x+F(y))=G(x)+y=G(x)+G(F(y))-c_1$ 
Since $F(1+G(y))=F(1)+y$ implies $F$ can take any value greater than $c_2=F(1)$, the equation above implies
$G(x+y+c_2)=G(x)+G(y+c_2)-c_1$.
Swap $x,y$, we get $G(y+c_2)=G(y)+c_3$. Then equation above becomes $G(x+y)=G(x)+G(y)-c_1$.
Since $G(x)-c_1$ satisfies Cauchy and $G$ is bounded from below, we can show $G$ is linear. So is $F$.

4) $f(x)=ax$ and $g(x)=\frac{a}{a-1}x$ for some $a>1$
Let $F(x)=bx+c$ and $G(x)=dx+e$, it is easy to get $bd=1$ and $c=e=0$.
Q.E.D.
\end{solution}
*******************************************************************************
-------------------------------------------------------------------------------

\begin{problem}[Posted by \href{https://artofproblemsolving.com/community/user/151516}{neerajbhauryal}]
	Consider $f(x)=[x]+(x-[x])^2$ for $x\in \mathbb{R}$ ,where $[x]$ denotes largest integer not exceeding $x$.
What will be the range of $f(x)$
	\flushright \href{https://artofproblemsolving.com/community/c6h546808}{(Link to AoPS)}
\end{problem}



\begin{solution}[by \href{https://artofproblemsolving.com/community/user/29428}{pco}]
	\begin{tcolorbox}Consider $f(x)=[x]+(x-[x])^2$ for $x\in \mathbb{R}$ ,where $[x]$ denotes largest integer not exceeding $x$.
What will be the range of $f(x)$\end{tcolorbox}

Let $y=n+u$ with $n\in\mathbb Z$ and $u\in[0,1)$ : $f(n+\sqrt u)=y$ and so $f(x)$ is surjective and $\boxed{f(\mathbb R)=\mathbb R}$
\end{solution}
*******************************************************************************
-------------------------------------------------------------------------------

\begin{problem}[Posted by \href{https://artofproblemsolving.com/community/user/187640}{ams1215}]
	find $f:(1,+\infty )\rightarrow R$ such that:
$f(x)-f(y)=(x-y)f(xy)$
	\flushright \href{https://artofproblemsolving.com/community/c6h548037}{(Link to AoPS)}
\end{problem}



\begin{solution}[by \href{https://artofproblemsolving.com/community/user/29428}{pco}]
	\begin{tcolorbox}find $f:(1,+\infty )\rightarrow R$ such that:
$f(x)-f(y)=(x-y)f(xy)$\end{tcolorbox}
Let $P(x,y)$ be the assertion $f(x)-f(y)=(x-y)f(xy)$, true $\forall x,y>1$

Let $a\ge 1$
Let $x>1$ such that $\sqrt x\in(a,2a)$ so that $a\sqrt x>1$ and $\frac{\sqrt x}a>1$ and $\frac{2a}{\sqrt x}>1$

$P(a\sqrt x,\frac 1a\sqrt x)$ $\implies$ $f(a\sqrt x)-f(\frac 1a\sqrt x)=(a\sqrt x-\frac 1a\sqrt x)f(x)$

$P(\frac 1a\sqrt x,\frac{2a}{\sqrt x})$ $\implies$ $f(\frac 1a\sqrt x)-f(\frac{2a}{\sqrt x})=(\frac 1a\sqrt x-\frac{2a}{\sqrt x})f(2)$

$P(\frac{2a}{\sqrt x},a\sqrt x)$ $\implies$ $f(\frac{2a}{\sqrt x})-f(a\sqrt x)=(\frac{2a}{\sqrt x}-a\sqrt x)f(2a^2)$

Adding and dividing by $\sqrt x$, we get $(a-\frac 1a)f(x)=af(2a^2)-\frac 1af(2)+\frac{2a}{x}(f(2)-f(2a^2))$

So $\forall x\in(a^2,4a^2)$ : $f(x)=\alpha(a)+\frac{\beta(a)}x$

Choosing for example successive $a$ starting with $a_0=1$ and $a_{n+1}=\sqrt 2a_n$, we see that $(a_i^2,4a_i^2)$ overlap and so $\alpha(a_i)$ are all equal and $\beta(a_i)$ also all are equal.

And since $a_n\to+\infty$, we get $f(x)=u+\frac vx$ $\forall x>1$

Plugging this back in original equation, we get $u=v=0$ and the only solution $\boxed{f(x)=0}$ $\forall x$
\end{solution}



\begin{solution}[by \href{https://artofproblemsolving.com/community/user/187640}{ams1215}]
	Thanks for your solution. But I think you make a mistake in the end as $f(x)=\frac{v}{x}$ is a solution?
\end{solution}



\begin{solution}[by \href{https://artofproblemsolving.com/community/user/187640}{ams1215}]
	Oh nvm it is $(x-y)f(xy)$ not $(y-x)f(xy)$
\end{solution}
*******************************************************************************
-------------------------------------------------------------------------------

\begin{problem}[Posted by \href{https://artofproblemsolving.com/community/user/169700}{s372102}]
	$f(x)=\sum\limits_{i=1}^{2013}\left[\dfrac{x}{i!}\right]$. A integer $n$ is called \begin{italicized}good\end{italicized} if $f(x)=n$ has real root. How many good numbers are  in $\{1,3,5,\dotsc,2013\}$?
	\flushright \href{https://artofproblemsolving.com/community/c6h548151}{(Link to AoPS)}
\end{problem}



\begin{solution}[by \href{https://artofproblemsolving.com/community/user/181762}{jred}]
	\begin{tcolorbox}$f(x)=\sum\limits_{i=1}^{2013}\left[\dfrac{x}{i!}\right]$. A integer $n$ is called \begin{italicized}good\end{italicized} if $f(x)=n$ has real root. How many good numbers are  in $\{1,3,5,\dotsc,2013\}$?\end{tcolorbox}
there are some minor errors in the statement.
Here an improved version: Let $f(x)=\sum\limits_{i=1}^{2013}\left[\dfrac{x}{i!}\right]$. Integer $n$ is called a "\begin{italicized}good number\end{italicized}" if equation $f(x)=n$ has at least a real solution. How many "\begin{italicized}good numbers\end{italicized}" are there in the set $\{1,3,5,\dotsc,2013\}$?
\end{solution}



\begin{solution}[by \href{https://artofproblemsolving.com/community/user/29428}{pco}]
	\begin{tcolorbox}$f(x)=\sum\limits_{i=1}^{2013}\left[\dfrac{x}{i!}\right]$. A integer $n$ is called \begin{italicized}good\end{italicized} if $f(x)=n$ has real root. How many good numbers are  in $\{1,3,5,\dotsc,2013\}$?\end{tcolorbox}
$f(x)$ is increasing and $\forall x\in [1,7!)$ : 
$f(x)=x+\left\lfloor\frac x2\right\rfloor$ $+\left\lfloor\frac x6\right\rfloor$ $+\left\lfloor\frac x{24}\right\rfloor$ $+\left\lfloor\frac x{120}\right\rfloor$ $+\left\lfloor\frac x{720}\right\rfloor$ 

$1+\frac 12+\frac 16+\frac 1{24}$ $+\frac 1{120}+\frac 1{720}$ $=\frac {1237}{720}$

So $\forall x\in[1,5040)$ : $5+\frac {1237}{720}x\ge f(x)> \frac {1237}{720}x-5$ and $\forall x\ge 5040$ $f(x)>x\ge 5040$

So $f(1171)<2013$ and $f(1175)>2013$ and very easy quick calculus gives $f(1172)=2011$ and $f(1173)=2012$ and $f(1174)=2014$

So we have exactly $1173$ good numbers in $\{1,2,3,4,...,2012,2013\}$

It remains to deal with odd and even such numbers.

$\forall x\in[1,720]$ : $f(x+720)=f(x)+1237$ and so we have exactly $720$ odd numbers and $720$ even numbers in $f([1,1440])$
$f(1440)=2474$ is even

$\forall x\in[1416,1439]$ : $f(x)-24=f(x)-41$ and so we have exacty $24$ odd numbers and $24$ even numbers in $f([1392,1439])$
$\forall x\in[1368,1391]$ : $f(x)-24=f(x)-41$ and so we have exacty $24$ odd numbers and $24$ even numbers in $f([1344,1391])$

$f(1338)=2297$
$f(1339)=2298$
$f(1340)=2300$
$f(1341)=2301$
$f(1342)=2303$
$f(1343)=2304$
And so we have exactly $3$ odd numbers and $3$ even numbers in $f([1338,1343])$

$\forall x\in[1338,1343]$ : $f(x)-6=f(x)-10$ and so we have exacty $3$ odd numbers and $3$ even numbers in $f([1332,1337])$
$\forall x\in[1332,1337]$ : $f(x)-6=f(x)-10$ and so we have exacty $3$ odd numbers and $3$ even numbers in $f([1326,1331])$

$f(1320)=2267$
$f(1321)=2268$
$f(1322)=2270$
$f(1323)=2271$
$f(1324)=2273$
$f(1325)=2274$
And so we have exactly $3$ odd numbers and $3$ even numbers in $f([1320,1325])$

$\forall x\in[1296,1319]$ : $f(x)-24=f(x)-41$ and so we have exacty $24$ odd numbers and $24$ even numbers in $f([1272,1319])$
$\forall x\in[1248,1271]$ : $f(x)-24=f(x)-41$ and so we have exacty $24$ odd numbers and $24$ even numbers in $f([1224,1271])$

Using previous lines, we have exactly $12$ odd numbers and $12$ even numbers in $f([1320,1343])$
$\forall x\in [1320,1343]$ : $f(x-120)=f(x)-206$ and so we have exactly $12$ odd numbers and $12$ even numbers in $f([1200,1223])$

So we have exactly $600$ odd numbers and $599$ even numbers in $f([1,1199])$

$f(1194)=2049$
$f(1195)=2050$
$f(1196)=2052$
$f(1197)=2053$
$f(1198)=2055$
$f(1199)=2056$
And so we have exactly $3$ odd numbers and $3$ even numbers in $f([1194,1199])$

$\forall x\in[1194,1199]$ : $f(x)-6=f(x)-10$ and so we have exacty $3$ odd numbers and $3$ even numbers in $f([1188,1193])$
$\forall x\in[1188,1193]$ : $f(x)-6=f(x)-10$ and so we have exacty $3$ odd numbers and $3$ even numbers in $f([1182,1187])$
$\forall x\in[1182,1187]$ : $f(x)-6=f(x)-10$ and so we have exacty $3$ odd numbers and $3$ even numbers in $f([1176,1181])$

$f(1174)=2014$
$f(1175)=2015$

So we have exactly $13$ odd numbers and $13$ even numbers in $f([1174,1199])$

So we have exactly $587$ odd numbers and $586$ even numbers in $f([1,1173])$


Hence the answer : $\boxed{587}$
\end{solution}
*******************************************************************************
-------------------------------------------------------------------------------

\begin{problem}[Posted by \href{https://artofproblemsolving.com/community/user/68025}{Pirkuliyev Rovsen}]
	Determine all functions $f: \mathbb{R}\to\mathbb{R}$ such that $f(x[y]\{z\})=f(x){\cdot}[f(y)]{\cdot}\{f(z)\}$ for all $x,y,z{\in}R$,where $[\ ]$ and $\{\ \}$ denote the integer and fractional part.
	\flushright \href{https://artofproblemsolving.com/community/c6h548203}{(Link to AoPS)}
\end{problem}



\begin{solution}[by \href{https://artofproblemsolving.com/community/user/29428}{pco}]
	\begin{tcolorbox}Determine all functions $f: \mathbb{R}\to\mathbb{R}$ such that $f(x[y]\{z\})=f(x){\cdot}[f(y)]{\cdot}\{f(z)\}$ for all $x,y,z{\in}R$,where $[\ ]$ and $\{\ \}$ denote the integer and fractional part.\end{tcolorbox}
Let us first look for constant solutions $f(x)=t$ : the equation is $t=t[t]\{t\}$
First trivial soluton is $t=0$
If $t\ne 0$, we get $1=[t]\{t\}$ and the solutions $t=n+\frac 1n$ where $n$ is any positive integer.

So let us from now look only for non constant solutions.

Let $P(x,y,z)$ be the assertion $f(x[y]\{z\})=f(x)[f(y)]\{f(z)\}$
Let $a$ such that $f(a)=b\ne 0$ (such $a,b$ exist since $(x)$ is non constant, and so non allzero).

If $[f(1)]=0$ or $\{f(\frac 12)\}=0$ then $P(2a,1,\frac 12)$ $\implies$ $f(a)=0$, impossible and so $[f(1)]\ne 0$ and $\{f(\frac 12)\}\ne 0$

$P(x,\frac 12,\frac 12)$ $\implies$ $f(0)=f(x)[f(\frac 12)]\{f(\frac 12)\}$ and so, since $f(x)$ is non constant : $[f(\frac 12)]=f(0)=0$

Let $x\in[0,1)$ : $P(a,x,\frac 12)$ $\implies$ $[f(x)]=0$ and so $\{f(x)\}=f(x)$
Let $x\in\mathbb Z$ : $P(a,1,x)$ $\implies$ $\{f(x)\}=0$ and so $[f(n)]=f(n)$

Let $x,y\in(0,1)$ : $P(x,1,y)$ $\implies$ $f(xy)=f(x)f(y)$ and so $f(x)=x^c$ for some $c>0$ $\forall x\in(0,1)$
(remember that $f(x)=\{f(x)\}$ and so $f(x)\in(0,1)$ $\forall x\in(0,1)$ an so we have a Cauchy equation with a bounded solution)

Let $x>1$ : $P(x,1,\frac 1x)$ $\implies$ $f(x)=x^c$ and so $f(x)=x^c$ $\forall x\in(0,1)\cup(1,+\infty)$
So $f(n)=n^c$ $\forall$ integer $n>1$ and so $c\in\mathbb N$ since we need $n^c\in\mathbb Z$ $\forall$ integer $n>1$

Let $x\in(0,1)$ : $P(1,1,x)$ $\implies$ $x^c=f(1)^2x^c$ and so $f(1)^2=1$
Let $x>1$ : $P(1,1,x)$ $\implies$ $\{x\}^c=\{x^c\}$ and so $c=1$
$P(2,2,\frac 14)$ $\implies$ $f(1)=1$

So we got $f(x)=x$ $\forall x\ge 0$

$P(x,-2,\frac 12)$ $\implies$ $f(-x)=f(x)f(-2)\frac 12$
$P(-x,-2,\frac 12)$ $\implies$ $f(x)=f(-x)f(-2)\frac 12$
So $f(-2)\in\{-2,2\}$

If $f(-2)=-2$, we get $f(x)=x$ which indeed is a solution
If $f(-2)=2$, we get $f(x)=|x|$ which is not a solution.

\begin{bolded}Hence the solutions\end{underlined}\end{bolded} :
$f(x)=0$ $\forall x$
$f(x)=n+\frac 1n$ $\forall x$ and whatever is positive integer $n$
$f(x)=x$ $\forall x$
\end{solution}



\begin{solution}[by \href{https://artofproblemsolving.com/community/user/177726}{manuel153}]
	Nice solution!
A tiny correction: $f(x)=n+\frac1n$ is a solution for any positive integer $n\ge2$.
\end{solution}
*******************************************************************************
-------------------------------------------------------------------------------

\begin{problem}[Posted by \href{https://artofproblemsolving.com/community/user/99931}{jaspion}]
	Consider two real functions $f(x)$ and $g(x)$, $f(x)=3x-1+|2x+1|$, $g(x)=\frac{3x+5-|2x+5|}{5}$. Prove:
$1) g(f(x))=f(g(x))$; 
$2) (f(f(x)))^{-1}=g(g(x))$.
	\flushright \href{https://artofproblemsolving.com/community/c6h548597}{(Link to AoPS)}
\end{problem}



\begin{solution}[by \href{https://artofproblemsolving.com/community/user/29428}{pco}]
	\begin{tcolorbox}Consider two real functions $f(x)$ and $g(x)$, $f(x)=3x-1+|2x+1|$, $g(x)=\frac{3x+5-|2x+5|}{5}$. Prove:
$1) g(f(x))=f(g(x))$; 
$2) (f(f(x)))^{-1}=g(g(x))$.\end{tcolorbox}
If $x\le -\frac 12$, $f(x)=x-2$
If $x\ge -\frac 12$, $f(x)=5x$

If $x\le -\frac 52$, $g(x)=x+2$
If $x\ge -\frac 52$, $g(x)=\frac x5$

Note that both $f(x)$ and $g(x)$ are bijection

If $x\le -\frac 12$, then $f(x)=x-2\le -\frac 52$ and so $g(f(x))=f(x)+2=x$
If $x\ge -\frac 12$, then $f(x)=5x\ge-\frac 52$ and so $g(f(x))=\frac{f(x)}5=x$
So $g(f(x))=x$ and so $g(x)=f^{[-1]}(x)$

So $f(g(x))=g(f(x))=x$

And $f(f(x))^{[-1]}=f^{[-1]}(f^{[-1]}(x))=g(g(x))$
\end{solution}
*******************************************************************************
-------------------------------------------------------------------------------

\begin{problem}[Posted by \href{https://artofproblemsolving.com/community/user/154928}{hieu1411997}]
	Prove that if $f$ continuous and $f(f(f(x)))=x$ on $R$ then $f(x)=x$
	\flushright \href{https://artofproblemsolving.com/community/c6h549090}{(Link to AoPS)}
\end{problem}



\begin{solution}[by \href{https://artofproblemsolving.com/community/user/29428}{pco}]
	\begin{tcolorbox}Prove that if $f$ continuous and $f(f(f(x)))=x$ on $R$ then $f(x)=x$\end{tcolorbox}

$f(x)$ is injective and so, since continuous, monotonous, and so increasing.

If $f(a)>a$ for some $a$, then $f(f(a))>f(a)>a$ and $f(f(f(a)))>f(f(a))>a$, impossible
If $f(a)<a$ for some $a$, then $f(f(a))<f(a)<a$ and $f(f(f(a)))<f(f(a))<a$, impossible

So $\boxed{f(x)=x}$ $\forall x$, which indeed is a solution.
\end{solution}
*******************************************************************************
-------------------------------------------------------------------------------

\begin{problem}[Posted by \href{https://artofproblemsolving.com/community/user/179739}{Hello_Kitty}]
	[color=#000080]\begin{bolded}1)\end{bolded}  Does there exist real functions $u, v$ such that ... 
$\forall x, \; u\circ v(x)=x^2$ and $v\circ u(x)=x^3$ ? 
\begin{bolded}2)\end{bolded}  Same question with  $\forall x, \; u\circ v(x)=x^2$ and $v\circ u(x)=x^4$ ? [\/color]
	\flushright \href{https://artofproblemsolving.com/community/c6h549246}{(Link to AoPS)}
\end{problem}



\begin{solution}[by \href{https://artofproblemsolving.com/community/user/29428}{pco}]
	\begin{tcolorbox}[color=#000080]\begin{bolded}1)\end{bolded}  Does there exist real functions $u, v$ such that ... 
$\forall x, \; u\circ v(x)=x^2$ and $v\circ u(x)=x^3$ ? [\/color]\end{tcolorbox}
We get immediately $u(x)$ injective and $v(x)$ surjective.

Since $v(x)$ is surjective :
Let $a$ such that $v(a)=-1$ and so $u(-1)=u(v(a))=a^2$
Let $b$ such that $v(b)=0$ and so $u(0)=u(v(b))=b^2$
Let $c$ such that $v(c)=1$ and so $u(1)=u(v(c))=c^2$
Since $u$ is injective, this means that $a^2,b^2,c^2$ are pairwise different

$v(a^2)=v(u(v(a)))=v(a)^3=-1$ and so $u(-1)=u(v(a^2))=a^4$ and so $a^4=a^2$ and so $a^2\in\{0,1\}$
$v(b^2)=v(u(v(b)))=v(b)^3=0$ and so $u(0)=u(v(b^2))=b^4$ and so $b^4=b^2$ and so $b^2\in\{0,1\}$
$v(c^2)=v(u(v(c)))=v(c)^3=1$ and so $u(1)=u(v(c^2))=c^4$ and so $c^4=c^2$ and so $c^2\in\{0,1\}$

But $a^2,b^2,c^2\in\{0,1\}$ and $a^2,b^2,c^2$ pairwise different is a contradiction.

So no such function.
\end{solution}



\begin{solution}[by \href{https://artofproblemsolving.com/community/user/29428}{pco}]
	\begin{tcolorbox}[color=#000080]\begin{bolded}2)\end{bolded}  Same question with  $\forall x, \; u\circ v(x)=x^2$ and $v\circ u(x)=x^4$ ? [\/color]\end{tcolorbox}
Yes such functions exist :

Let $k$ a positive integer $>1$
Let $\sim_k$ the equivalence relation in $\mathbb R^+$ defined as $x\sim_k y$ $\iff$ $\exists n\in\mathbb Z$ such that ${x=y^{k^n}}$
Let $r_k(x)$ a function associating to any positive real a representant, unique per class, of its equivalence class.
Note that Class of $1$ is $\{1\}$ and so $r_k(1)=1$
Let $n_k(x)$ the function which associates to any positive real different from $1$, the unique integer such that $x=r_k(x)^{2^{n_k(x)}}$

Obviously $r_2(%Error. "mathBB" is a bad command.
R^+)$ and $r_4(%Error. "mathBB" is a bad command.
R^+)$ both are infinite equinumerous sets and let $g(x)$ any bijection from $r_2(%Error. "mathBB" is a bad command.
R^+)\to r_4(%Error. "mathBB" is a bad command.
R^+)$ such that $g(1)=1$

Let then $u(x)$ defined as :
$u(0)=0$
$u(1)=u(-1)=1$
$\forall x\notin\{-1,0,1\}$ : $u(x)=(g^{-1}(r_4(|x|)))^{2^{n_4(|x|)}}$

And $v(x)$ defined as :
$v(0)=0$
$v(1)=v(-1)=1$
$\forall x\notin\{-1,0,1\}$ : $v(x)=(g(r_2(|x|)))^{4^{1+n_2(|x|)}}$

For $x\in\{-1,0,1\}$, the equalities $u(v(x))=x^2$ and $v(u(x))=x^4$ are trivially true.

For $x\notin\{-1,0,1\}$ :
$v(x)=(g(r_2(|x|)))^{4^{1+n_2(|x|)}}$
So $r_4(|v(x)|)=g(r_2(|x|))$ and $n_4(|v(x)|)=1+n_2(x)$
So $g^{-1}(r_4(|v(x)|)=r_2(|x|)$
And so $u(v(x))=(r_2(|x|))^{2^{1+n_2(x)}}$ $=x^2$

For $x\notin\{-1,0,1\}$ :
$u(x)=(g^{-1}(r_4(|x|)))^{2^{n_4(|x|)}}$
So $r_2(|u(x)|)=g^{-1}(r_4(|x|)$ and $n_2(|v(x)|)=n_4(x)$
So $g(r_2(|u(x)|)=r_4(|x|)$
And so $v(u(x))=(r_4(|x|))^{4^{1+n_4(x)}}$ $=x^4$

Q.E.D.
\end{solution}



\begin{solution}[by \href{https://artofproblemsolving.com/community/user/49556}{xxp2000}]
	\begin{tcolorbox}[color=#000080]
\begin{bolded}2)\end{bolded}  Same question with  $\forall x, \; u\circ v(x)=x^2$ and $v\circ u(x)=x^4$ ? [\/color]\end{tcolorbox}
Here is another pair.
Let $u,v$ be even function with $u(0)=v(0)=0$.
$u(x)=e^{2\sqrt{\log(x)}},x\ge1$ and $u(x)=e^{-2\sqrt{-\log(x)}},x\in(0,1)$.
$v(x)=e^{(\log(x))^2},x\ge1$ and $v(x)=e^{-(\log(x))^2},x\in(0,1)$.
\end{solution}
*******************************************************************************
-------------------------------------------------------------------------------

\begin{problem}[Posted by \href{https://artofproblemsolving.com/community/user/134330}{Arman}]
	Find all functions $f$ such that, for any numbers $x$ and $y$  holds equality $f(x+y) =f(x)\cos{y}+f(y)\cos{x}$.
	\flushright \href{https://artofproblemsolving.com/community/c6h551168}{(Link to AoPS)}
\end{problem}



\begin{solution}[by \href{https://artofproblemsolving.com/community/user/29428}{pco}]
	\begin{tcolorbox}Find all functions $f$ such that, for any numbers $x$ and $y$  holds equality $f(x+y) =f(x)\cos{y}+f(y)\cos{x}$.\end{tcolorbox}
Let $P(x,y)$ be the assertion $f(x+y)=f(x)\cos y+f(y)\cos x$

$P(x-\frac{\pi}2,\frac{\pi}2)$ $\implies$  $f(x)=f(\frac{\pi}2)\sin x$

And so $\boxed{f(x)=a\sin x}$ $\forall x$, whetever is $a\in\mathbb R$, which indeed is a solution
\end{solution}
*******************************************************************************
-------------------------------------------------------------------------------

\begin{problem}[Posted by \href{https://artofproblemsolving.com/community/user/68025}{Pirkuliyev Rovsen}]
	Let $f: \mathbb{R}\to\mathbb{R}$ be a strictly increasing and bijective function. Find all functions $g: \mathbb{R}\to\mathbb{R}$ such that $(f{\circ}g)(x){\leq}x{\leq}(g{\circ}f)(x)$ for all real $x$.
	\flushright \href{https://artofproblemsolving.com/community/c6h552307}{(Link to AoPS)}
\end{problem}



\begin{solution}[by \href{https://artofproblemsolving.com/community/user/29428}{pco}]
	\begin{tcolorbox}Let $f: \mathbb{R}\to\mathbb{R}$ be a strictly increasing and bijective function. Find all functions $g: \mathbb{R}\to\mathbb{R}$ such that $(f{\circ}g)(x){\leq}x{\leq}(g{\circ}f)(x)$ for all real $x$.\end{tcolorbox}
Since $f(x)$ is a strictly increasing bijective function, $f^{[-1]}(x)$ exists and is also a strictly increasing bijective function.

$f(g(x))\le x$ and $f^{[-1]}(x)$ increasing imply $g(x)\le f^{[-1]}(x)$
Setting $x\to f^{[-1]}(x)$ in $x\le g(f(x))$, we get $f^{[-1]}(x)\le g(x)$

And so $\boxed{g(x)=f^{[-1]}(x)}$ which indeed is a solution.
\end{solution}
*******************************************************************************
-------------------------------------------------------------------------------

\begin{problem}[Posted by \href{https://artofproblemsolving.com/community/user/186084}{acupofmath}]
	Is there infinity functions $ f:\mathbb{R} \rightarrow \mathbb{R} $ such that $ f(x+y+xy)=f(x)+f(y)+xf(y)+yf(x) \quad \forall x,y \in \mathbb{R} $ ?
	\flushright \href{https://artofproblemsolving.com/community/c6h554630}{(Link to AoPS)}
\end{problem}



\begin{solution}[by \href{https://artofproblemsolving.com/community/user/29428}{pco}]
	\begin{tcolorbox}Is there infinity functions $ f:\mathbb{R} \rightarrow \mathbb{R} $ such that $ f(x+y+xy)=f(x)+f(y)+xf(y)+yf(x) \quad \forall x,y \in \mathbb{R} $ ?\end{tcolorbox}
Yes. For example :

Let $a(x)$ any additive function ($a(x+y)=a(x)+a(y)$)

Choose $f(x)$ such that :
$f(-1)=0$
$f(x)=(x+1)a(\ln|x+1|)$ $\forall x\ne -1$
\end{solution}
*******************************************************************************
-------------------------------------------------------------------------------

\begin{problem}[Posted by \href{https://artofproblemsolving.com/community/user/123816}{pqhoai}]
	Find all functions  $ f: \mathbb{R}\rightarrow \mathbb{R} $ such that  $f(x+f(y))=f(x)+\frac{2x}{9}f(3y)+f(f(y)) $
	\flushright \href{https://artofproblemsolving.com/community/c6h557461}{(Link to AoPS)}
\end{problem}



\begin{solution}[by \href{https://artofproblemsolving.com/community/user/29428}{pco}]
	\begin{tcolorbox}Find all functions  $ f: \mathbb{R}\rightarrow \mathbb{R} $ such that  $f(x+f(y))=f(x)+\frac{2x}{9}f(3y)+f(f(y)) $\end{tcolorbox}
$\boxed{f(x)=0}$ $\forall x$ is a solution.
Let us from now look only for non allzero solutions.
Notice that the only constant solution is the allzero one.

Let $P(x,y)$ be the assertion $f(x+f(y))=f(x)+\frac 29xf(3y)+f(f(y))$
Let $a$ such that $f(a)\ne 0$ and let $b=\frac{f(3a)}{f(a)}$
Let $c=\frac 29b$

$P(0,y)$ $\implies$ $f(0)=0$

$P(f(x),a)$ $\implies$ $f(f(x)+f(a))=f(f(x))+\frac 29bf(x)f(a)+f(f(a))$
$P(f(a),x)$ $\implies$ $f(f(a)+f(x))=f(f(a))+\frac 29f(a)f(3x)+f(f(x))$
Subtracting and using the fact that $f(a)\ne 0$, we get $f(3x)=bf(x)$ $\forall x$
As a consequence, $b\ne 0$

$P(x,y)$ becomes then new assertion $Q(x,y)$ : $f(x+f(y))=f(x)+cxf(y)+f(f(y))$

$Q(f(x),x)$ $\implies$ $f(2f(x))=2f(f(x))+cf(x)^2$
$Q(2f(x),x)$ $\implies$ $f(3f(x))=3f(f(x))+3cf(x)^2$
And, since $f(3f(x))=\frac 92cf(f(x))$, this last line becomes $(\frac 32c-1)f(f(x))=cf(x)^2$

$c\ne \frac 23$, else we would have $f(x)=0$ $\forall x$. So $f(f(x))=\frac{2c}{3c-2}f(x)^2$

$Q(x,y)$ becomes then new assertion $R(x,y)$ : $f(x+f(y))=f(x)+cxf(y)+\frac{2c}{3c-2}f(y)^2$

Notice that, choosing $y=a$, and since $c\ne 0$, the expression $cxf(y)+\frac{2c}{3c-2}f(y)^2$ can take any value we want.
So any real may be written as $f(u)-f(v)$ for some $u,v$

$R(-f(x),x)$ $\implies$ $f(-f(x))=\frac{c(3c-4)}{3c-2}f(x)^2$

$R(-f(y),x)$ $\implies$ $f(f(x)-f(y))=\frac{c(3c-4)}{3c-2}f(x)^2-cf(x)f(y)+\frac{2c}{3c-2}f(y)^2$

$R(f(x)-f(y),y)$ $\implies$ $f(f(x)-f(y))=\frac{2c}{3c-2}f(x)^2-cf(x)f(y)+c\frac{3c-4}{3c-2}f(y)^2$

Subtracting, we get $c\frac{3c-6}{3c-2}(f(x)^2-f(y)^2)=0$ and so $c=2$ (since $f(x)$ is not constant).

So $f(f(x)-f(y))=f(x)^2-2f(x)f(y)+f(y)^2$ $=(f(x)-f(y))^2$ and since any real may be written as $f(u)-f(v)$ for some $u,v$, we get :

$\boxed{f(x)=x^2}$ $\forall x$ which indeed is a solution
\end{solution}



\begin{solution}[by \href{https://artofproblemsolving.com/community/user/68025}{Pirkuliyev Rovsen}]
	Excellent Patrick
\end{solution}
*******************************************************************************
-------------------------------------------------------------------------------

\begin{problem}[Posted by \href{https://artofproblemsolving.com/community/user/175698}{emptyhands7697}]
	Find all function f : $(0,+\infty ) \mapsto (0,+\infty)$ such as :
$f(x^{3}+y)=f^{3}(x)+\frac{f(xy)}{f(x)}$ with all x,y > 0
	\flushright \href{https://artofproblemsolving.com/community/c6h557587}{(Link to AoPS)}
\end{problem}



\begin{solution}[by \href{https://artofproblemsolving.com/community/user/29428}{pco}]
	\begin{tcolorbox}Find all function f : $(0,+\infty ) \mapsto (0,+\infty)$ such as :
$f(x^{3}+y)=f^{3}(x)+\frac{f(xy)}{f(x)}$ with all x,y > 0\end{tcolorbox}
Let $P(x,y)$ be the assertion $f(x^3+y)=f(x)^3+\frac{f(xy)}{f(x)}$
Let $a=f(1)$


1) If $a>1$
===========
$P(1,x)$ $\implies$ $f(x+1)=a^3+\frac{f(x)}a$ and quick induction gives $f(x+n)=\frac{a^{n}-1}{a-1}a^{4-n}+f(x)a^{-n}$

This implies $\lim_{n\in\mathbb Z\to+\infty}f(n)=a^3$

$P(x,1)$ $\implies$ $f(x^3+1)=f(x)^3+1$
$P(1,x^3)$ $\implies$ $f(x^3+1)=a^3+\frac{f(x^3)}a$
And so $f(x^3)=af(x)^3+a-a^4$

Setting then $x\in\mathbb Z\to+\infty$ in $f(x^3)=af(x)^3+a-a^4$, we get  $a^{10}-a^4-a^3+a=0$ which has \begin{bolded}no solution\end{underlined}\end{bolded} $>1$

2) If $a=1$
==========
$P(1,x)$ $\implies$ $f(x+1)=f(x)+1$
Then $P(x,y+1)$ $\implies$ $f(x^3+y)+1=f(x)^3+\frac{f(xy+x)}{f(x)}$ and so $f(xy+x)=f(xy)+f(x)$

So $f(x+y)=f(x)+f(y)$ $\forall x,y>0$ and $f(x)>0$ $\forall x>0$ (so lower bounded) and $f(1)=1$

So $\boxed{f(x)=x}$ $\forall x$, which indeed is a solution

3) If $a<1$
==========
$P(1,x)$ $\implies$ $f(x+1)=a^3+\frac{f(x)}a$ and simple induction gives $f(n)=\frac{a^{n-1}-1}{a-1}a^{5-n}+a^{2-n}$ $\forall n\in\mathbb N$

Plugging this in $P(m,n)$ with $m,n\in\mathbb N$, we get \begin{bolded}no solution\end{underlined}\end{bolded}
(in order to quickly get this result, consider $f(n)\sim a^{2-n}$ when $n\to +\infty$ and set $x,y\in\mathbb N\to+\infty$ in $P(x,y)$)
\end{solution}
*******************************************************************************
-------------------------------------------------------------------------------

\begin{problem}[Posted by \href{https://artofproblemsolving.com/community/user/175803}{tweener}]
	Find all functions $f:\mathbb{Q}^+\to\mathbb{Q}^+$ such that $f(x)=yf(xf(y))$ for all $x,y\in\mathbb{Q}^+$
	\flushright \href{https://artofproblemsolving.com/community/c6h557601}{(Link to AoPS)}
\end{problem}



\begin{solution}[by \href{https://artofproblemsolving.com/community/user/29428}{pco}]
	\begin{tcolorbox}Find all functions $f:\mathbb{Q}^+\to\mathbb{Q}^+$ such that $f(x)=yf(xf(y))$  for all $x,y\in\mathbb{Q}^+$\end{tcolorbox}
Let $P(x,y)$ be the assertion $f(x)=yf(xf(y))$

If $f(a)=f(b)$, comparaison of $P(1,a)$ and $P(1,b)$ implies $a=b$ and so $f(x)$ is injective
$P(x,1)$ $\implies$ $f(x)=f(xf(1))$ and so, since injective, $f(1)=1$

$P(1,x)$ $\implies$ $f(f(x))=\frac 1x$ (notice that this implies that $f(x)$ is a bijection)

$P(1,f(x))$ $\implies$ $f(\frac 1x)=\frac 1{f(x)}$

$P(x,f(\frac 1y))$ $\implies$ $f(xy)=f(x)f(y)$

So the \begin{bolded}problem is equivalent to\end{underlined}\end{bolded} $f(xy)=f(x)f(y)$ and $f(f(x))=\frac 1x$

Since multiplicative, we only need to define $f(x)$ for prime numbers with respect of second equation.

And there, I  stop since obviously infinitely many solutions exist : for example, split prime numbers in set or couples $(p,q)$ of distinct prime numbers.
For each pair define either :
$f(p)=q$ and $f(q)=\frac 1p$
or $f(p)=pq$ and $f(q)=\frac 1{p^2q}$

And obviously infinitely other solutions are possible (in each pair or thru grouping prime in bigger subsets) ...

So I wonder what could be the solution requested in your exam .
\end{solution}
*******************************************************************************
-------------------------------------------------------------------------------

\begin{problem}[Posted by \href{https://artofproblemsolving.com/community/user/68025}{Pirkuliyev Rovsen}]
	Determine all functions $f: \mathbb{R}\to\mathbb{R}$ such that $(x-2)f(y)+f(y+2f(x))=f(x+yf(x))$.
	\flushright \href{https://artofproblemsolving.com/community/c6h557981}{(Link to AoPS)}
\end{problem}



\begin{solution}[by \href{https://artofproblemsolving.com/community/user/29428}{pco}]
	\begin{tcolorbox}Determine all functions $f: \mathbb{R}\to\mathbb{R}$ such that $(x-2)f(y)+f(y+2f(x))=f(x+yf(x))$.\end{tcolorbox}
$\boxed{f(x)=0}$ $\forall x$ is a solution.
Let us from now look only for non allzero solutions.

Let $P(x,y)$ be the assertion $(x-2)f(y)+f(y+2f(x))=f(x+yf(x))$
Let $a$ such that $f(a)\ne 0$

If $f(u)=0$ for some $u$, then $P(u,a)$ $\implies$ $(u-1)f(a)=0$ and so $u=1$
So $f(0)\ne 0$

If $f(x_1)=f(x_2)$ for some $x_1,x_2$, then comparaison of $P(x_1,0)$ and $P(x_2,0)$ implies $x_1=x_2$ and $f(x)$ is injective.

$P(2,1)$ $\implies$ $f(1+2f(2))=f(2+f(2))$ and so, since injective, $f(2)=1$

Let $x\ne 2$ so that $f(x)\ne 1$ (since injective) :
$P(x,\frac{2f(x)-x}{f(x)-1})$ $\implies$ $(x-2)f(\frac{2f(x)-x}{f(x)-1})+f(\frac{2f(x)^2-x}{f(x)-1})$ $=f(\frac{2f(x)^2-x}{f(x)-1})$
So $f(\frac{2f(x)-x}{f(x)-1})=0$

And so $\frac{2f(x)-x}{f(x)-1}=1$ and $f(x)=x-1$ $\forall x\ne 2$ and since $f(2)=1$, we get :

$\boxed{f(x)=x-1}$ $\forall x$, which indeed is a solution.
\end{solution}
*******************************************************************************
-------------------------------------------------------------------------------

\begin{problem}[Posted by \href{https://artofproblemsolving.com/community/user/68025}{Pirkuliyev Rovsen}]
	Determine all functions $f: \mathbb{R}\to\mathbb{R}$ such that $f(x)f(yf(x)-1)=x^2f(y)-f(x)$.
	\flushright \href{https://artofproblemsolving.com/community/c6h557982}{(Link to AoPS)}
\end{problem}



\begin{solution}[by \href{https://artofproblemsolving.com/community/user/29428}{pco}]
	\begin{tcolorbox}Determine all functions $f: \mathbb{R}\to\mathbb{R}$ such that $f(x)f(yf(x)-1)=x^2f(y)-f(x)$.\end{tcolorbox}
$\boxed{f(x)=0}$ $\forall x$ is a solution.
Let us from now look only for non allzero solutions.

Let  $P(x,y)$ be the assertion $f(x)f(yf(x)-1)=x^2f(y)-f(x)$
Let $a$ such that $f(a)\ne 0$

If $f(u)=0$ for some $u$, then $P(u,a)$ $\implies$ $u=0$
$P(1,1)$ $\implies$ $f(1)f(f(1)-1)=0$ and so $f(1)=1$ and $f(0)=0$
$P(1,x)$ $\implies$ $f(x-1)=f(x)-1$ (and so $f(x)=x$ $\forall x\in\mathbb Z$)

As a consequence, $f(yf(x)-1)=f(yf(x))-1$ and $P(x,y)$ implies new assertion $Q(x,y)$ : $f(x)f(yf(x))=x^2f(y)$
$Q(x,1)$ $\implies$ $f(x)f(f(x))=x^2$.
Pluging this value of $x^2$ in $Q(x,y)$, we get $f(x)f(yf(x))=f(x)f(f(x))f(y)$ and so :
$f(yf(x))=f(f(x))f(y)$ $\forall x\ne 0$, still true when $x=0$

Let $x\ne 0$ : $Q(x,x)$ $\implies$ $f(xf(x))=x^2$, still true when $x=0$ and so $[0,+\infty)\subseteq f(\mathbb R)$

So $f(yf(x))=f(f(x))f(y)$ may be written $f(xy)=f(x)f(y)$ $\forall x\ge 0$,$\forall y$
Since $f(-1)=-1$, this equality implies $f(-x)=-f(x)$ $\forall x\ge 0$ and so $f(xy)=f(x)f(y)$ $\forall x,y$

So $f(xy+x)=f(x(y+1))=f(x)f(y+1)=f(x)f(y)+f(x)=f(xy)+f(x)$ and so $f(x+y)=f(x)+f(y)$ $\forall x,y$

$f(x+y)=f(x)+f(y)$ and $f(xy)=f(x)f(y)$ and $f(x)$ not allzero imply $\boxed{f(x)=x}$ $\forall x$ which indeed is a solution.
\end{solution}



\begin{solution}[by \href{https://artofproblemsolving.com/community/user/93837}{jjax}]
	Haha I started off the same way as pco, but I found a cute ending (:

As above, until the step $f(x)f(f(x))=x^2$.
Then, setting $a+1=x$ we have, after repeated applications of $f(x+1)=f(x)+1$,
$(f(a)+1)(f(f(a))+1) = (a+1)^2$, and noting that $f(a)f(f(a))=a^2$, we get $f(a)+f(f(a))=2a$.
And so $(f(a)-f(f(a)))^2 = (2a)^2 - 4a^2 = 0$, so $2a=f(a)+f(f(a))=2f(a)$.
\end{solution}
*******************************************************************************
-------------------------------------------------------------------------------

\begin{problem}[Posted by \href{https://artofproblemsolving.com/community/user/10045}{socrates}]
	Determine all functions $f,g : \mathbb{R}^+ \to \mathbb{R}^+$ such that \[ f(2x-f(x)+g(y)+y)+f(y)=2x-f(x)+2g(y)+y ,\]
       
for all $x,y \in \mathbb{R}^+$
	\flushright \href{https://artofproblemsolving.com/community/c6h558086}{(Link to AoPS)}
\end{problem}



\begin{solution}[by \href{https://artofproblemsolving.com/community/user/29428}{pco}]
	\begin{tcolorbox}Determine all functions $f,g : \mathbb{R}^+ \to \mathbb{R}^+$ such that \[ f(2x-f(x)+g(y)+y)+f(y)=2x-f(x)+2g(y)+y ,\]
       
for all $x,y \in \mathbb{R}^+$\end{tcolorbox}
Let $P(x,y)$ be the assertion $f(2x-f(x)+g(y)+y)+f(y)=2x-f(x)+2g(y)+y$

Let $u(x)=f(x)-x$. $P(x,y)$ may be written $u(2x-f(x)+g(y)+y)=g(y)-u(y)-y$

$P(x,x)$ may be written $2(3x-f(x)+g(x))-f(3x-f(x)+g(x))=3x$ and so $2x-f(x)$ can take any positive value we want.
So, using previous equation, we get two conclusions :
1) : $g(y)-u(y)-y=c$ is constant
2) : $u(x)=c$ is constant from a given point (namely $x>\inf(g(y)+y))$)

So the problem is equivalent to :
a) $u(x)+x>0$ $\forall x>0$ ($f(x)$ must be positive)
b) $u(x)+x+c>0$ $\forall x>0$ ($g(x)$ must be positive)
c) $u(y)+2y+c>u(x)-x$ $\forall x,y>0$ (in order LHS be defined)
d) $u(x-u(x)+u(y)+2y+c)=c$ $\forall x,y>0$

Setting $y=x$ in c), we get $3x+c>0$ $\forall x>0$ and so $c\ge 0$ and so a) implies b).
Setting $y=x$ in d), we get $u(3x+c)=c$ $\forall x>0$ and so $u(x)=c$ $\forall x>c$

From there, it's possible to find infiniely many such $u(x)$ and I dont know if it is possibl to show a gneral form for all the solutions. 

Here are some solutons :

Example 1 : $u(x)=\min(x,c)$ where $c\ge 0$ is some real. And so 
$f(x)=\min(2x,x+c)$
$g(x)=f(x)+c$

Example 2 : 
$f(x)=x(1+\frac 12\sin(x))$ $\forall x\in(0,c)$
$f(x)=x+c$ $\forall x\ge c$
$g(x)=f(x)+c$

and infinitely many other examples.

Is it really an exercise given in a book ? and what does the book tell about solution ?
\end{solution}
*******************************************************************************
-------------------------------------------------------------------------------

\begin{problem}[Posted by \href{https://artofproblemsolving.com/community/user/194905}{questionbot}]
	Find all polynomials $f(n), g(n)$ such that \[f(n)(n^2 + n + 1) + g(n)(n^2 + 1) = 1\] for all integer $n.$
	\flushright \href{https://artofproblemsolving.com/community/c6h558187}{(Link to AoPS)}
\end{problem}



\begin{solution}[by \href{https://artofproblemsolving.com/community/user/29428}{pco}]
	\begin{tcolorbox}Find all polynomials $f(n), g(n)$ such that \[f(n)(n^2 + n + 1) + g(n)(n^2 + 1) = 1\] for all integer $n.$\end{tcolorbox}
Writing $f(n)=u(n)(n^2+1)+an+b$ and $g(n)=v(n)(n^2+n+1)+cn+d$, we get :

$(u(n)+v(n))(n^2+1)(n^2+n+1)+(a+c)n^3+(a+b+d)n^2+(a+b+c)n+(b+d-1)=0$

So $v(n)=-u(n)$ and $a+c=0$ and $a+b+d=0$ and $a+b+c=0$ and $b+d-1=0$

And so $(a,b,c,d)=(-1,0,1,1)$ and :

$f(n)=u(n)(n^2+1)-n$ and $g(n)=-u(n)(n^2+n+1)+n+1$, whatever is $u(x)\in\mathbb R[X]$
\end{solution}
*******************************************************************************
-------------------------------------------------------------------------------

\begin{problem}[Posted by \href{https://artofproblemsolving.com/community/user/120756}{qua96}]
	Let´s $f: \mathbb{R} \to\mathbb{R}^{*} $ such that
\[f(x+y)+f(z)+f(t)=f(x+y)+f(z+t) \] 
For all $x,y,z,t$ satisfy $xz+yt=xy$, here $\mathbb{R}^{*}$ is the set of nonegative real number. Prove that. $f$ is continuos and find $f$.
	\flushright \href{https://artofproblemsolving.com/community/c6h558318}{(Link to AoPS)}
\end{problem}



\begin{solution}[by \href{https://artofproblemsolving.com/community/user/29428}{pco}]
	\begin{tcolorbox}Let´s $f: \mathbb{R} \to\mathbb{R}^{*} $ such that
\[f(x+y)+f(z)+f(t)=f(x+y)+f(z+t) \] 
For all $x,y,z,t$ satisfy $xz+yt=xy$, here $\mathbb{R}^{*}$ is the set of nonegative real number. Prove that. $f$ is continuos and find $f$.\end{tcolorbox}
Choosing $x=y=0$, we get $f(z+t)=f(z)+f(t)$ $\forall z,t\in\mathbb R$

So $f(x)$ is an additive lower bounded (since non negative) function, and so is continuous and is $f(x)=cx$

In order to have $f(x)\ge 0$ $\forall x$, we get $c=0$ and the unique solution $\boxed{f(x)=0}$ $\forall x$, which indeed is a solution.
\end{solution}



\begin{solution}[by \href{https://artofproblemsolving.com/community/user/120756}{qua96}]
	$f(x)= a.x^2$ is a solution??
\end{solution}



\begin{solution}[by \href{https://artofproblemsolving.com/community/user/29428}{pco}]
	\begin{tcolorbox}$f(x)= a.x^2$ is a solution??\end{tcolorbox}
Yes, if $a=0$
No, if $a\ne 0$
\end{solution}



\begin{solution}[by \href{https://artofproblemsolving.com/community/user/120756}{qua96}]
	Sorry, i make a mistake, the conditional must be
$f(x+y)+f(z)+f(t)=f(x+z)+f(y+t)$
\end{solution}
*******************************************************************************
-------------------------------------------------------------------------------

\begin{problem}[Posted by \href{https://artofproblemsolving.com/community/user/153761}{caubemetoan96}]
	Determine all functions $f : \mathbb{Q} \to \mathbb{Q}$ such that $f(x+y+f(x))= x+ f(x) +f(y))$
	\flushright \href{https://artofproblemsolving.com/community/c6h558485}{(Link to AoPS)}
\end{problem}



\begin{solution}[by \href{https://artofproblemsolving.com/community/user/29428}{pco}]
	\begin{tcolorbox}Determine all functions $f : \mathbb{Q} \to \mathbb{Q}$ such that $f(x+y+f(x))= x+ f(x) +f(y)$\end{tcolorbox}
Let $A=\{a\in\mathbb Q$ such that $f(x+a)=f(x)+a$  $\forall x\in\mathbb Q\}$
Obviously, $A$ is an additive subgroup of $\mathbb Q$

Let $\sim$ the equivalence relation defined in $\mathbb Q$ as $x\sim y$ $\iff$ $x-y\in A$
Let $r(x)$ from $\mathbb Q\to\mathbb Q$ a function which associates to any rational a representant (unique per class) of its class.

$x+f(x)\in A$ $\forall x$ and so $f(r(x))=a(r(x))-r(x)$ for some function $a(x)$ from $\mathbb Q\to A$

So $f(x)=f(x-r(x)+r(x))=f(r(x))+x-r(x)=a(r(x))+x-2r(x)$

And it's immediate to show that this necessary form is sufficient.

\begin{bolded}Hence the general solution\end{underlined}\end{bolded} :

Let $A$ any additive subgroup of $\mathbb Q$
Let $a(x)$ any function from $\mathbb Q\to A$
Let $\sim$ the equivalence relation defined in $\mathbb Q$ as $x\sim y$ $\iff$ $x-y\in A$ and $r(x)$ from $\mathbb Q\to\mathbb Q$ a function which associates to any rational a representant (unique per class) of its class.

Then $f(x)=a(r(x))+x-2r(x)$

\begin{bolded}Some examples\end{underlined}\end{bolded} :

1) $A=\{0\}$ and so $r(x)=x$ and $a(x)=0$ and so $\boxed{f(x)=-x}$

2) $A=\mathbb Q$ and so $r(x)=c$ and $a(x)$ any function and $\boxed{f(x)=x+b}$

3) $A=\mathbb Z$ and $r(x)=\{x\}$ and $a(x)=\lfloor 10\sin 2\pi x\rfloor$ and $\boxed{f(x)=\lfloor 10\sin 2\pi x)\rfloor+x-2\{x\}}$

And infinitely many other solutions.
\end{solution}
*******************************************************************************
-------------------------------------------------------------------------------

\begin{problem}[Posted by \href{https://artofproblemsolving.com/community/user/68025}{Pirkuliyev Rovsen}]
	Given a function $f: \mathbb{N}\to\mathbb{R}$,$f(n)=f(\frac{n}{p})-f(p)$, where $n{\in}N$ and $p$ prime number.It is known that $f(2^{2007})+f(3^{2008})+f(5^{2009})=2006$ Calculate: $f(2007^2)+f(2008^3)+f(2009^5)$.
	\flushright \href{https://artofproblemsolving.com/community/c6h558506}{(Link to AoPS)}
\end{problem}



\begin{solution}[by \href{https://artofproblemsolving.com/community/user/29428}{pco}]
	\begin{tcolorbox}Given a function $f: \mathbb{N}\to\mathbb{R}$,$f(n)=f(\frac{n}{p})-f(p)$, where $n{\in}N$ and $p$ prime number.It is known that $f(2^{2007})+f(3^{2008})+f(5^{2009})=2006$ Calculate: $f(2007^2)+f(2008^3)+f(2009^5)$.\end{tcolorbox}
Equation gives $f(\prod p_i^{n_i})=f(1)-\sum n_if(p_i)$
But $f(p)=f(1)-f(p)$ and so $f(p)=\frac 12f(1)$

So $f(\prod p_i^{n_i})=f(1)(1-\frac 12\sum n_i)$ and so :

$f(2^{2007})=-\frac{2005}2f(1)$ 

$f(3^{2008})=-\frac{2006}2f(1)$ 

$f(5^{2009})=-\frac{2007}2f(1)$ 

So $f(1)=-\frac 23$ and then :

$f(2007^2)=f(3^4223^2)=f(1)(1-\frac{4+2}2)=\frac 43$

$f(2008^3)=f(2^9251^3)=f(1)(1-\frac{9+3}2)=\frac{10}3$

$f(2009^5)=f(7^{10}41^5)=f(1)(1-\frac{10+5}2)=\frac{13}3$

And so $\boxed{f(2007^2)+f(2008^3)+f(2009^5)=9}$
\end{solution}
*******************************************************************************
-------------------------------------------------------------------------------

\begin{problem}[Posted by \href{https://artofproblemsolving.com/community/user/169515}{abl}]
	Problem 
a)find all continuous functions $f:\mathbb{R}\to\mathbb{R}$ such that
\[f(f(x+y))=f(x)+f(y)+f(xy)-xy\;\forall x,y\in\mathbb{R}\]
b)find all continuous functions $f:\mathbb{R}\to\mathbb{R}$ such that
\[f(x^2 +y^3)= x^2 - 3y(y+1) + (f(y))^3,\forall x,y\in\mathbb{R}\]
	\flushright \href{https://artofproblemsolving.com/community/c6h558732}{(Link to AoPS)}
\end{problem}



\begin{solution}[by \href{https://artofproblemsolving.com/community/user/29428}{pco}]
	\begin{tcolorbox}Problem 
a)find all continuous functions $f:\mathbb{R}\to\mathbb{R}$ such that
\[f(f(x+y))=f(x)+f(y)+f(xy)-xy\;\forall x,y\in\mathbb{R}\]
\end{tcolorbox}
Let $P(x,y)$ be the assertion $f(f(x+y))=f(x)+f(y)+f(xy)-xy$
Let $a=f(0)$

$P(x+y,0)$ $\implies$ $f(f(x+y))=f(x+y)+2a$ and so new assertion $Q(x,y)$ : $f(x+y)=f(x)+f(y)+f(xy)-xy-2a$

So $f(x+y+z)=f(x+(y+z))=f(x)+f(y+z)+f(xy+xz)-xy-xz-2a$ $=f(x)+f(y)+f(z)+f(yz)-yz+f(xy+xz)-xy-xz-4a$
Swapping $x$ and $y$ and subtracting, we get : $f(xy+xz)+f(yz)=f(xy+yz)+f(xz)$

So $f(x+y)+f(z)=f(x+z)+f(y)$ $\forall x,y,z$ such that $xyz>0$

So $f(x+y)-f(y)=f(x+z)-f(z)$ and so $f(x+y)=g(x)+f(y)$ for some $g(x)$ $\forall x,y\ne 0$

But we also have $f(x+y)=f(x)+g(y)$ and so $g(x)=f(x)+c$ and $f(x+y)=f(x)+f(y)+c$ $\forall x,y\ne 0$

Plugging this in $Q(x,y)$, we get $f(xy)=xy+2a+c$ $\forall x,y\ne 0$

So $f(x)=x+2a+c$ $\forall x\ne 0$

Plugging this in $Q(1,1)$, we get $c=-a$ and so $f(x)=x+a$ $\forall x\ne 0$, still true when $x=0$

Plugging this in $P(x,y)$, we get $a=0$ and so the unique solution $\boxed{f(x)=x}$ $\forall x$

And, btw, no need for continuity
\end{solution}



\begin{solution}[by \href{https://artofproblemsolving.com/community/user/29428}{pco}]
	\begin{tcolorbox}Problem 
b)find all continuous functions $f:\mathbb{R}\to\mathbb{R}$ such that
\[f(x^2 +y^3)= x^2 - 3y(y+1) + (f(y))^3,\forall x,y\in\mathbb{R}\]\end{tcolorbox}
Let $P(x,y)$ be the assertion $f(x^2+y^3)=x^2-3y(y+1)+f(y)^3$
Let $a=f(0)$

$P(x,0)$ $\implies$ $f(x^2)=x^2+a^3$ and so $f(x)=x+a^3$ $\forall x\ge 0$

Let $x\ge 0$ : $P(0,x)$ becomes $x^3+a^3=-3x(x+1)+(x+a^3)^3$ $\forall x\ge 0$ and so $a=1$ and $f(x)=x+1$ $\forall x\ge 0$

Let $y<0$. Choosing $x$ such that $x^2+y^3>0$, $P(x,y)$ becomes $f(y)^3=(y+1)^3$

And so $\boxed{f(x)=x+1}$ $\forall x$, which indeed is a solution.

And, btw, no need for continuity.
\end{solution}



\begin{solution}[by \href{https://artofproblemsolving.com/community/user/10045}{socrates}]
	\begin{tcolorbox}Problem 
a)find all continuous functions $f:\mathbb{R}\to\mathbb{R}$ such that
\[f(f(x+y))=f(x)+f(y)+f(xy)-xy\;\forall x,y\in\mathbb{R}\]\end{tcolorbox}

$y=0\implies f(f(x))=f(x)+2f(0)$ so the given equation rewrites $f(x+y)+2f(0)=f(x)+f(y)+f(xy)-xy \ (*).$

So $f(x+1)=2f(x)-x+c_1,$ where $c_1$ is constant. 

So $x:=x-1$ into  the last equation gives $f(x)=2f(x-1)-x+c_2, $ where $c_2$ is constant. 

Putting $y:=-1$ into $(*)$ we get $f(x-1)=f(x)+f(-x)+x+c_3,$ where $c_3$ is constant. 

Combining last two facts we have $f(x)+2f(-x)+x+l=0, $ where $l$ is constant.  

Hence we also have $f(-x)+2f(x)-x+l=0.$

Solving for $f(x)$ we get $f(x)=x+a$ and pluging this into the original equation we find $a=0.$
\end{solution}
*******************************************************************************
-------------------------------------------------------------------------------

\begin{problem}[Posted by \href{https://artofproblemsolving.com/community/user/68025}{Pirkuliyev Rovsen}]
	Given a function ${{f: \mathbb(0;+\infty)}\to\mathbb(0;+\infty)}$.Prove that there exist $x,y{\in}R$ , $x,y>0$ such that $f(x+y)<yf(f(x))$.
	\flushright \href{https://artofproblemsolving.com/community/c6h558931}{(Link to AoPS)}
\end{problem}



\begin{solution}[by \href{https://artofproblemsolving.com/community/user/29428}{pco}]
	\begin{tcolorbox}Given a function ${{f: \mathbb(0;+\infty)}\to\mathbb(0;+\infty)}$.Prove that there exist $x,y{\in}R$ , $x,y>0$ such that $f(x+y)<yf(f(x))$.\end{tcolorbox}
Suppose on the contrary that $f(x+y)\ge yf(f(x))$ $\forall x,y>0$
Let $P(x,y)$ be the assertion  $f(x+y)\ge yf(f(x))$

Let $x>0$.
If $f(x)\le x$, then $f(x)\le x+1$
If $f(x)>x$, then $P(x,f(x)-x)$ $\implies$ $f(f(x))\ge (f(x)-x)f(f(x))$ and so $f(x)\le x+1$
So $f(x)\le x+1$ $\forall x>0$

$P(x,y)$ implies then $x+y+1\ge f(x+y)\ge yf(f(x))$ and so $f(f(x))\le 1+\frac{x+1}y$
Setting $y\to+\infty$ in this inequality, we get $f(f(x))\le 1$ $\forall x>0$

Let $x>0$. If $f(x)>1$, then $P(1,f(x)-1)$ $\implies$ $1\ge f(f(x))\ge (f(x)-1)f(f(1))$ and so $f(x)\le 1+\frac 1{f(f(1))}$

So $f(x)<M$ $\forall x$ for some $M>0$ and $P(x,y)$ implies $f(f(x))<\frac My$
Setting $y\to +\infty$ in this inequality, we get $f(f(x))\le 0$, which is impossible.

Hence the result.
\end{solution}
*******************************************************************************
-------------------------------------------------------------------------------

\begin{problem}[Posted by \href{https://artofproblemsolving.com/community/user/142747}{yt12}]
	Let$ f:\mathbb{R}\to\mathbb{R}$ such that $f(xy)+f(y-x) \ge f(y+x) ,\forall x,y\in\mathbb{R}$.
If $f(x) \ge 0 $, find $x$?
	\flushright \href{https://artofproblemsolving.com/community/c6h559010}{(Link to AoPS)}
\end{problem}



\begin{solution}[by \href{https://artofproblemsolving.com/community/user/29428}{pco}]
	\begin{tcolorbox}Let$ f:\mathbb{R}\to\mathbb{R}$ such that $f(xy)+f(y-x) \ge f(y+x) ,\forall x,y\in\mathbb{R}$.
If $f(x) \ge 0 $, find $x$?\end{tcolorbox}
Let $P(x,y)$ be the assertion $f(xy)+f(y-x)\ge f(y+x)$

Let $u\in\mathbb R$. The quadratic $x^2+(u-2)x-u=0$ always has at least one real root $r\ne 1$.
Note then that $2r-r^2=u(r-1)$ and so $\frac{2r-r^2}{r-1}=u$

Then $P(r,\frac r{r-1})$ $\implies$ $f(\frac{2r-r^2}{r-1})\ge 0$ and so $\boxed{f(u)\ge 0\text{    }\forall u\in\mathbb R}$
\end{solution}
*******************************************************************************
-------------------------------------------------------------------------------

\begin{problem}[Posted by \href{https://artofproblemsolving.com/community/user/64716}{mavropnevma}]
	Let $\mathcal{F}$ be the family of bijective increasing functions $f: [0,1] \to [0,1]$, and let $a \in (0,1)$. Determine the best constants $m_a$ and $M_a$, such that for all $f \in \mathcal{F}$ we have
\[m_a \leq f(a) + f^{-1}(a) \leq M_a.\]

\begin{italicized}(Dan Schwarz)\end{italicized}
	\flushright \href{https://artofproblemsolving.com/community/c6h559021}{(Link to AoPS)}
\end{problem}



\begin{solution}[by \href{https://artofproblemsolving.com/community/user/29428}{pco}]
	\begin{tcolorbox}Let $\mathcal{F}$ be the family of bijective increasing functions $f: [0,1] \to [0,1]$, and let $a \in (0,1)$. Determine the best constants $m_a$ and $M_a$, such that for all $f \in \mathcal{F}$ we have
\[m_a \leq f(a) + f^{-1}(a) \leq M_a.\]

\begin{italicized}(Dan Schwarz)\end{italicized}\end{tcolorbox}

1) $M_0=0$ and $M_1=2$ and $M_a=a+1$ $\forall a\in(0,1)$
=============================
If $a=0$, then $f(0)=0$ $\forall f$ and so $M_0=0$
If $a=1$, then $f(1)=1$ $\forall f$ and so $M_1=2$
If $1>a>0$ :
If $f(a)=a$, then $f(a)+f^{-1}(a)=2a\le a+1$
If $f(a)<a$, then $f(a)+f^{-1}(a)<a+1$
If $1\ge f(a)>a$, then $f^{-1}(a)<a$ and so $f(a)+f^{-1}(a)<a+1$
So $M_a\le a+1$

Choosing $f_n(x)$ (with $n$ great enough) as the three segments curve defined by the four points $(0,0)$, $(a-\frac 1n,a)$, $(a,1-\frac 1n)$ and $(1,1)$,
We get $f_n(a)+f_n^{-1}(a)=1+a-\frac 2n$ and so $f(a)+f^{-1}(a)$ may be as near as we want of $a+1$
Q.E.D.

2) $m_0=0$ and $m_1=2$ and $m_a=a$ $\forall a\in(0,1)$
=============================
If $a=0$, then $f(0)=0$ $\forall f$ and so $m_0=02$
If $a=1$, then $f(1)=1$ $\forall f$ and so $m_1=2$
If $0<a<1$ :
If $f(a)=a$, then $f(a)+f^{-1}(a)=2a\ge a$
If $f(a)<a$, then $f^{-1}(a)>a$ and so $f(a)+f^{-1}(a)>a$
If $f(a)>a$, then $f(a)+f^{-1}(a)>a$
So $m_a\ge a$

Choosing $f_n(x)$ (with $n$ great enough) as the three segments curve defined by the four points $(0,0)$, $(a,\frac 1n)$, $(a+\frac 1n,a)$ and $(1,1)$,
We get $f_n(a)+f_n^{-1}(a)=a+\frac 2n$ and so $f(a)+f^{-1}(a)$ may be as near as we want of $a$
Q.E.D.
\end{solution}
*******************************************************************************
-------------------------------------------------------------------------------

\begin{problem}[Posted by \href{https://artofproblemsolving.com/community/user/68025}{Pirkuliyev Rovsen}]
	Let $S$ be the set of all reals greater than one.Find all the functions $ f:S->R$ such that $ f(x)-f(y)=(y-x)f(xy)$ for all $x,y>1$
	\flushright \href{https://artofproblemsolving.com/community/c6h559052}{(Link to AoPS)}
\end{problem}



\begin{solution}[by \href{https://artofproblemsolving.com/community/user/29428}{pco}]
	\begin{tcolorbox}Let $S$ be the set of all reals greater than one.Find all the functions $ f:S->R$ such that $ f(x)-f(y)=(y-x)f(xy)$ for all $x,y>1$\end{tcolorbox}
Let $P(x,y)$ be the assertion $f(x)-f(y)=(y-x)f(xy)$

Let $a>b>1$
Let $x>1$ such that $x\notin\{ab,b,\frac ab\}$

$P(b,x)$ $\implies$ $f(b)-f(x)=(x-b)f(bx)$ and so $f(bx)=\frac{f(b)-f(x)}{x-b}$

$P(a,bx)$ $\implies$ $f(a)-f(bx)=(bx-a)f(abx)$ and so $f(abx)=\frac{f(a)-f(bx)}{bx-a}$ $=\frac{f(a)}{bx-a}-\frac{f(b)-f(x)}{(x-b)(bx-a)}$

$P(ab,x)$ $\implies$ $f(ab)-f(x)=(x-ab)f(abx)$ and so $f(abx)=\frac{f(ab)-f(x)}{x-ab}$

And so $\frac{f(a)}{bx-a}-\frac{f(b)-f(x)}{(x-b)(bx-a)}$ $=\frac{f(ab)-f(x)}{x-ab}$

Which may be written $f(x)=\frac{x^2(bf(ab)-f(a))+ux+v}{bx^2-wx}$ $\forall x>1\notin\{ab,b,\frac ab,\frac wb\}$ for some real numbers $u,v,w$

If $bf(ab)-f(a)\ne 0$, then $\lim_{x\to+\infty}f(x)=f(ab)-\frac{f(a)}b\ne 0$ which is a contradiction with original equation.

So $bf(ab)=f(a)$ $\forall a>b>1$ and so $\boxed{f(x)=\frac cx}$ $\forall x$ and whatever is $c\in\mathbb R$, which indeed is a solution.
\end{solution}
*******************************************************************************
-------------------------------------------------------------------------------

\begin{problem}[Posted by \href{https://artofproblemsolving.com/community/user/122611}{oty}]
	Find all function $f:\mathbb{R} \to \mathbb{R}$ such that : 

$f(x+f(y))=f(x)+y , \forall x,y \in \mathbb{R}$
	\flushright \href{https://artofproblemsolving.com/community/c6h559211}{(Link to AoPS)}
\end{problem}



\begin{solution}[by \href{https://artofproblemsolving.com/community/user/186557}{odnerpmocon}]
	$x=y=0 \implies f(f(0))=f(0) \implies f(f(f(0)))=f(0)$
$y=f(0), x=0 \implies f(f(f(0)))=2f(0) \implies f(0)=0$
$x=0 \implies f(f(y))=y$
$y=f(x) \implies f(2x)=2f(x)$
$x=2k,y=f(k) \implies f(3k)=3f(k)$, and similarly $ f(nk)=nf(k) \implies f(n)=nf(1), \forall n \in \mathbb N$. 
You can extend this to $\mathbb Z$ and $\mathbb Q$, but if $f$ is not continuous, I'm not sure what to do with $\mathbb R$.
\end{solution}



\begin{solution}[by \href{https://artofproblemsolving.com/community/user/126756}{panamath}]
	$x=0 \implies f(f(y))=f(0)+y \implies f$ bijective.
$x=y=0 \implies f(f(0))=f(0)$ , but since $f$ is bijective it implies $f(0)=0 \implies f(f(y))=y$.

Now we have, $f(x+f(y))=f(x)+f(f(y))$ wich is a Cauchy type functional equation, so we just need to prove $f$ is monotonic, which it is indeed since $f$ is bijective.

Finally $f(x)=cx  \implies f(f(x))=c^2 x = x \implies c=1, -1 \implies f(x)=x$
\end{solution}



\begin{solution}[by \href{https://artofproblemsolving.com/community/user/75382}{djb86}]
	\begin{tcolorbox}$\ldots$ so we just need to prove $f$ is monotonic, which it is indeed since $f$ is bijective.\end{tcolorbox}
This is not true. A function can be bijective, but not monotone. For example, divide the real numbers into pairs $(a,b)$ and define $f(a)=b$ and $f(b)=a$. Then $f$ is bijective, but not in general monotone.

For this problem in particular, once we have surjectivity, we may replace $f(y)$ with $z$, so $f(x+f(y))=f(x)+f(f(y))$ becomes $f(x+z)=f(x)+f(z)$. Thus the solutions are all those functions $f$ satisfying the Cauchy equation, which [s]are also bijective[\/s] also satisfy $f(f(x))=x$. 

[s][In more advanced terms, in stead of all $\mathbb{Q}$-linear homomorphisms from $\mathbb{R}$ to $\mathbb{R}$ we need all $\mathbb{Q}$-linear automorphisms of $\mathbb{R}$.][\/s] This is not quite true. See pco's solution below for a complete answer.
\end{solution}



\begin{solution}[by \href{https://artofproblemsolving.com/community/user/123120}{Jordan2386}]
	Actually, I know this problem (I've made it up (a year ago) while solving an already existing one) and I've tried to solve it for several days with no great result. This problem is extremely difficult and may be put as a real open question. I'm familiar with the results of djb86 but I think there is no trivial way of describing such functions (Cauchy+bijection).
\end{solution}



\begin{solution}[by \href{https://artofproblemsolving.com/community/user/29428}{pco}]
	\begin{tcolorbox}Find all function $f:\mathbb{R} \to \mathbb{R}$ such that : 

$f(x+f(y))=f(x)+y , \forall x,y \in \mathbb{R}$\end{tcolorbox}
Previous posts quicky gave that solutions are Cauchy-functions such that $f(f(x))=x$.

1) General solution of such equation 
=========================

Let $A,B$ two supplementary subvectorspaces of the $\mathbb Q$-vectorspace $\mathbb R$
Let $a(x)$ from $\mathbb R\to A$ and $b(x)$ from $\mathbb R\to B$ the projections of $x$ in $A,B$ so that $x=a(x)+b(x)$

Then $f(x)=a(x)-b(x)$ 

2) Proof that any function in the form of $1)$ is indeed a solution
=============================================
$a(x)-b(x)$ is linear.
$a(a(x)-b(x))=a(x)$ and $b(a(x)-b(x))=-b(x)$ and so $f(f(x))=a(x)+b(x)=x$
Q.E.D.

3) proof that any solution may be written in the form of $1)$ and so we indeed got a general solution
=======================================================================
Let $f(x)$ any function such that $f(x+y)=f(x)+f(y)$ $\forall x,y$ and $f(f(x))=x$ $\forall x$

Let $A=\{x\in\mathbb R$ such that $f(x)=x\}$
Let $B=\{x\in\mathbb R$ such that $f(x)=-x\}$
Obviously, $A,B$ both are subvectorspaces of the $\mathbb Q$-vectorspace $\mathbb R$
If $x\in A\cap B$, then $f(x)=x$ and $f(x)=-x$ and so $x=0$ and $A\cap B=\{0\}$

$f(f(x)+x)=f(f(x))+f(x)=f(x)+x$ and so $f(x)+x\in A$ $\forall x$
$f(f(x)-x)=f(f(x))-f(x)=x-f(x)$ and so $f(x)-x\in B$ $\forall x$

Let $a(x)=\frac{f(x)+x}2$ so that $a(x)\in A$
Let $b(x)=\frac{x-f(x)}2$ so that $b(x)\in B$

$a(x)+b(x)=x$ and so $A,B$ are supplementary subvectorspaces.

And $a(x)-b(x)=f(x)$
Q.E.D.

4) some examples
=============
4.1 : $A=\mathbb R$ and $B=\{0\}$ and so $f(x)=x$
4.2 : $A=\{0\}$ and $B=\mathbb R$ and so $f(x)=-x$

and infinitely many other solutions with axiom of choice.
\end{solution}
*******************************************************************************
-------------------------------------------------------------------------------

\begin{problem}[Posted by \href{https://artofproblemsolving.com/community/user/68025}{Pirkuliyev Rovsen}]
	The function ${f: \mathbb{R}\to\mathbb[0;1]}$ satisfies the condition $f^2(x)+f^2(x+2a)+f^2(x+3a)+f^2(x+5a)=1$, where $a{\in}R$.Prove that the function $f$ of a periodic.
	\flushright \href{https://artofproblemsolving.com/community/c6h559309}{(Link to AoPS)}
\end{problem}



\begin{solution}[by \href{https://artofproblemsolving.com/community/user/186353}{randomusername}]
	Try $a=0$. 
For instance, $f(x)=-1\/2$ for $x\neq0$ and $f(x)=1\/2$ for $x=0$ is non-periodic and satisfies the equation. ??

[Edit.] Ignore this post, I misunderstood the problem. Firstly, $f$ is from $R$ to $[0,1]$, and secondly, It wasn't clear that $a$ is a given constant.
\end{solution}



\begin{solution}[by \href{https://artofproblemsolving.com/community/user/29428}{Try $a=0$. \nFor instance, $f(x)=-1\/2$ for $x\\neq0$ and $f(x)=1\/2$ for $x=0$ is non-periodic and satisfies the equation. ??\end{tcolorbox}\n$f(x)=-\\frac 12$ is impossible since $f(x)$ is from $\\mathbb R\\to [0,1]$","username":"pco}]
	[quote="ran
\end{solution}



\begin{solution}[by \href{https://artofproblemsolving.com/community/user/150054}{wer}]
	For x←x+2a→f^2 (x+2a)+f^2 (x+4a)+f^2 (x+5a)+f^2 (x+7a)=1,  result f^2 (x)+f^2 (x+3a)=f^2 (x+4a)+f^2 (x+7a). (1) For x←x+3a→f^2 (x+3a)+f^2 (x+5a)+f^2 (x+6a)+f^2 (x+8a)=1, result f^2 (x)+f^2 (x+2a)=f^2 (x+6a)+f^2 (x+8a).We can write f^2 (x+5a)-f^2 (x+9a)=f^2 (x+6a)-f^2 (x+2a)=f^2 (x)-f^2 (x+8a)→f^2 (x)+f^2 (x+9a)=f^2 (x+5a)+f^2 (x+8a). From (1) result f^2 (x+5a)+f^2 (x+8a)=f^2 (x+9a)+f^2 (x+12a)→f^2 (x)+f^2 (x+9a)=f^2 (x+9a)+f^2 (x+12a)→f(x)=f(x+12a).f is periodic, T=12a
\end{solution}



\begin{solution}[by \href{https://artofproblemsolving.com/community/user/188367}{the_creater}]
	LATEXED

For $x=x+2a, f^2 (x+2a)+f^2 (x+4a)+f^2 (x+5a)+f^2 (x+7a)=1,$
 result $f^2 (x)+f^2 (x+3a)=f^2 (x+4a)+f^2 (x+7a).$ $(1)$
 For $x=x+3a, f^2 (x+3a)+f^2 (x+5a)+f^2 (x+6a)+f^2 (x+8a)=1,$ 
result $f^2 (x)+f^2 (x+2a)=f^2 (x+6a)+f^2 (x+8a).$
We can write $f^2 (x+5a)-f^2 (x+9a)=f^2 (x+6a)-f^2 (x+2a)=f^2 (x)-f^2 (x+8a)$ 
so $f^2 (x)+f^2 (x+9a)=f^2 (x+5a)+f^2 (x+8a).$ 
From (1) result $f^2 (x+5a)+f^2 (x+8a)=f^2 (x+9a)+f^2 (x+12a),$ 
so $f^2 (x)+f^2 (x+9a)=f^2 (x+9a)+f^2 (x+12a),$ 
so $f(x)=f(x+12a).
$f is periodic, $T=12a$
\end{solution}



\begin{solution}[by \href{https://artofproblemsolving.com/community/user/64716}{mavropnevma}]
	If you call that "latexed", then pigs will fly; it's worse than the original. Just putting dollar signs around expressions containing non-$\LaTeX$ symbols won't do any good.
\end{solution}



\begin{solution}[by \href{https://artofproblemsolving.com/community/user/150054}{wer}]
	Thanks Creater.
\end{solution}



\begin{solution}[by \href{https://artofproblemsolving.com/community/user/188367}{the_creater}]
	i have editted the LATEX(OR NON-LATEX)
\end{solution}
*******************************************************************************
-------------------------------------------------------------------------------

\begin{problem}[Posted by \href{https://artofproblemsolving.com/community/user/142747}{yt12}]
	Let $f(x^3+y^3)=(x+y)[f^2(x)-f(x)f(y)+f^2(y)]$. Prove that $f(2013x)=2013f(x),\forall x\in\mathbb{R} $
	\flushright \href{https://artofproblemsolving.com/community/c6h559440}{(Link to AoPS)}
\end{problem}



\begin{solution}[by \href{https://artofproblemsolving.com/community/user/29428}{pco}]
	\begin{tcolorbox}Let $f(x^3+y^3)=(x+y)[f^2(x)-f(x)f(y)+f^2(y)]$. Prove that $f(2013x)=2013f(x),\forall x\in\mathbb{R} $\end{tcolorbox}
Let $P(x,y)$ be the assertion $f(x^3+y^3)=(x+y)(f(x)^2-f(x)f(y)+f(y)^2)$
Let $A=\{a\in\mathbb R$ such that $f(ax)=af(x)$ $\forall x\in\mathbb R\}$
$1\in A$

$P(0,0)$ $\implies$ $f(0)=0$ and so $0\in A$
$P(x,0)$ $\implies$ $f(x^3)=xf(x)^2$
Notice that previous line implies that either $f(x)=0$, either $f(x)$ and $x$ both have same signs.

Let $a\in A\setminus\{0\}$ and $x\ne 0$ : $axf(x)^2=af(x^3)=f(ax^3)$ $=\sqrt[3]axf(\sqrt[3]ax)^2$ and so $(\sqrt[3]af(x))^2=f(\sqrt[3]ax)^2$
Since we know that $f(x)$ and $x$ have same signs, we get $\sqrt[3]af(x)=f(\sqrt[3]ax)$ and so $\sqrt[3]a\in A$

Let $a\in A$ : $P(ax,x)$ $\implies$ $f((a^3+1)x^3)=(ax+x)(a^2f(x)^2-af(x)^2+f(x)^2)$ $=(a^3+1)xf(x)^2=(a^3+1)f(x^3)$
And so $a^3+1\in A$

So $a\in A$ $\implies$ $\sqrt[3]a\in A$ $\implies$ $\sqrt[3]a^3+1\in A$ $\implies$ $a+1\in A$

And since $1\in A$, we get $2013\in A$

Q.E.D.
\end{solution}
*******************************************************************************
-------------------------------------------------------------------------------

\begin{problem}[Posted by \href{https://artofproblemsolving.com/community/user/190536}{DonaldLove}]
	$p \in P$ is odd. Find all $f:\mathbb{Z} \to \mathbb{Z}$ such that $\forall m,n \in \mathbb{Z}$
1, if $m -n \vdots p$ then $f(m)=f(n)$
2, $f(mn)=f(m)f(n)$
	\flushright \href{https://artofproblemsolving.com/community/c6h559451}{(Link to AoPS)}
\end{problem}



\begin{solution}[by \href{https://artofproblemsolving.com/community/user/29428}{pco}]
	\begin{tcolorbox}$p \in P$ is odd. Find all $f:\mathbb{Z} \to \mathbb{Z}$ such that $\forall m,n \in \mathbb{Z}$
1, if $m -n \vdots p$ then $f(m)=f(n)$
2, $f(mn)=f(m)f(n)$\end{tcolorbox}
1) If $f(1)=0$
==============
$f(n\times 1)=f(n)f(1)=0$ 
And so we get solution S1 : $\boxed{f(n)=0}$ $\forall n\in\mathbb Z$, which indeed is a solution

2) If $f(1)\ne 0$ and $f(p)\ne 0$
================================
Let $n\in\mathbb Z$ : $p|p^2-pn$ and so $f(p)(f(p)-f(n))=0$ and so $f(n)=f(p)$ $\forall n$
Plugging this in $f(mn)=f(m)f(n)$, we get solution S2 : $\boxed{f(n)=1}$ $\forall n\in\mathbb Z$, which indeed is a solution

3) $f(1)\ne 0$ and $f(p)=0$
==========================
$f(1)=f(1\times 1)=f(1)^2$ and so $f(1)=1$

Let $n\not\equiv 0\pmod p$ 
Let $m\equiv \frac 1n\pmod p$
$1=f(mn)=f(m)f(n)$ and so $f(n)=\pm 1$ $\forall n\not\equiv 0\pmod p$

Let $n\not\equiv 0\pmod p$. If $n\equiv m^2\pmod p$ is quadratic residue $\pmod p$, then $f(n)=f(m)^2=(\pm 1)^2=1$

And so it's immediate to get the two solutions :

Solution S3 : $\boxed{f(n)=0\text{   }\forall n\equiv 0\pmod p\text{ and  }f(n)=1\text{   }\forall n\not\equiv 0\pmod p}$

Solution S4 : $\boxed{f(n)=0\text{   }\forall n\equiv 0\pmod p\text{ and  }f(n)=\left(\frac np\right)\text{   }\forall n\not\equiv 0\pmod p}$
\end{solution}
*******************************************************************************
-------------------------------------------------------------------------------

\begin{problem}[Posted by \href{https://artofproblemsolving.com/community/user/112643}{proglote}]
	Find all injective functions $f: \mathbb{R}^* \to \mathbb{R}^* $ from the non-zero reals to the non-zero reals, such that \[f(x+y) \left(f(x) + f(y)\right) = f(xy)\] for all non-zero reals $x, y$ such that $x+y \neq 0$.
	\flushright \href{https://artofproblemsolving.com/community/c6h559598}{(Link to AoPS)}
\end{problem}



\begin{solution}[by \href{https://artofproblemsolving.com/community/user/29428}{pco}]
	\begin{tcolorbox}Find all injective functions $f: \mathbb{R}^* \to \mathbb{R}^* $ from the non-zero reals to the non-zero reals, such that \[f(x+y) \left(f(x) + f(y)\right) = f(xy)\] for all non-zero reals $x, y$ such that $x+y \neq 0$.\end{tcolorbox}
Let $P(x,y)$ be the assertion $f(x+y)(f(x)+f(y))=f(xy)$ $\forall x,y,x+y\ne 0$
Let $a=f(1)\ne 0$

1) $a=1$ and $f(nx)=\frac {f(x)}n$ $\forall x\ne 0$ $\forall n\in\mathbb Z\setminus\{0\}$
======================================
Let $x>0$ : 
$P(x,1)$ $\implies$ $f(x+1)=\frac{f(x)}{f(x)+a}$ and simple induction gives $f(x+n)=\frac 1{\frac{a^n}{f(x)}+\sum_{k=0}^{n-1}a^k}$
So $f(n+1)=\frac 1{a^{n-1}+\sum_{k=0}^{n-1}a^k}$
1.1) $|a|>1$ is impossible
------------------------------
If $|a|>1$, then $f(n)\sim \frac 12a^{1-n}$ when $n\to+\infty$
Plugging this equivalence in $P(n,n)\iff f(n^2)=2f(n)f(2n)$ implies contradiction
Q.E.D.

1.2 $|a|<1$ is impossible
------------------------------
If $|a|<1$, then $\lim_{n\to+\infty}f(n)=1-a$
Plugging this value in $P(n,n)\iff f(n^2)=2f(n)f(2n)$ implies $a=\frac 12$
But then $f(n)=\frac 12$ $\forall n\in\mathbb N$, impossible since $f(x)$ is injective
Q.E.D

1.3 $a\ne -1$
----------------
If $a=-1$, then $f(x+2)=f(x)$, impossible since $f(x)$ is injective
Q.E.D.

1.4 $a=1$ and $f(nx)=\frac {f(x)}n$ $\forall x\ne 0$ $\forall n\in\mathbb Z\setminus\{0\}$
-------------------------------------------------------------------
$a=1$ is an immediate consequence of the three previous assertions.
So $f(n+1)=\frac 1{a^{n-1}+\sum_{k=0}^{n-1}a^k}$ becomes $f(n+1)=\frac 1{n+1}$ $\forall n\in\mathbb N$
$P(x,n)$ $\implies$  $f(nx)=\frac {f(x)}n$ $\forall x\ne 0$ $\forall n\in\mathbb N$
$P(x,x)$ $\implies$ $f(x^2)=f(x)^2$ and so, since injective, $f(-x)=-f(x)$ $\forall x\ne 0$
Q.E.D.


2) $f(xy)=f(x)f(y)$ $\forall x,y\ne 0$
========================
Let $x,y>0$ and $n\in\mathbb N$

$P(x,ny)$ $\implies$ $f(x+ny)=\frac{f(xy)}{nf(x)+f(y)}$

$P(2nxy,n^2y^2)$ $\implies$ $f(2nxy+n^2y^2)=\frac{f(2xy^3)}{n^2f(2xy)+nf(y^2)}$

$P(2nx^3y,n^2x^2y^2)$ $\implies$ $f(2nx^3y+n^2x^2y^2)=\frac{f(2x^5y^3)}{n^2f(2x^3y)+nf(x^2y^2)}$

$P(x^2,2nxy+n^2y^2)$ $\implies$ $f(x^2+2nxy+n^2y^2)=\frac{f(2nx^3y+n^2x^2y^2)}{f(x^2)+f(2nxy+n^2y^2)}$

And so :
$f((x+ny)^2)=\frac{f(2x^5y^3)(nf(2xy)+f(y^2))}{(nf(2x^3y)+f(x^2y^2))(n^2f(x^2)f(2xy)+nf(x^2)f(y^2)+f(2xy^3))}$

$P(x,x)$ $\implies$ $f(x^2)=f(x)^2$ and so $f((x+ny)^2)=f(x+ny)^2=\left(\frac{f(xy)}{nf(x)+f(y)}\right)^2$

So $\frac{f(2x^5y^3)(nf(2xy)+f(y^2))}{(nf(2x^3y)+f(x^2y^2))(n^2f(x^2)f(2xy)+nf(x^2)f(y^2)+f(2xy^3))}$ $=\frac{f(xy)^2}{n^2f(x)^2+2nf(x)f(y)+f(y)^2}$

Reducing, we get a cubic in $n$, true for any natural $n$. So it must be the zero polynomial and all its coefficients must be zero.

Equating the $n^3$ coefficient to zero, we get $f(2x^5y^3)f(2xy)f(x)^2=f(xy)^2f(2x^3y)f(x^2)f(2xy)$ and so (remembering that $f(u^2)=f(u)^2$) :

$f(x^5y^3)=f(x^2y^2)f(x^3y)$ $\forall x,y>0$

So $f(xy)=f(x)f(y)$ $\forall x,y>0$
And since $f(-x)=-f(x)$, we get $f(xy)=f(x)f(y)$ $\forall x,y\ne 0$
Q.E.D.

3) $f(x)=\frac 1x$ $\forall x\ne 0$
================
Plugging $f(xy)=f(x)f(y)$ in $P(x,y)$, we get $\frac 1{f(x+y)}=\frac 1{f(x)}+\frac 1{f(y)}$ and $\frac 1{f(xy)}=\frac 1{f(x)}\frac 1{f(y)}$

And this is a very classical problem whose unique solution is $\frac 1{f(x)}=x$
Q.E.D.

Hence the unique injective solution $\boxed{f(x)=\frac 1x}$ $\forall x\ne 0$, whih indeed is a solution
(note that at least one other non injective solution exists : $f(x)=\frac 12$ $\forall x\ne 0$)
\end{solution}



\begin{solution}[by \href{https://artofproblemsolving.com/community/user/184652}{CanVQ}]
	\begin{tcolorbox}Find all injective functions $f: \mathbb{R}^* \to \mathbb{R}^* $ from the non-zero reals to the non-zero reals, such that \[f(x+y) \left(f(x) + f(y)\right) = f(xy) \] for all non-zero reals $x, y$ such that $x+y \neq 0$.\end{tcolorbox}
Setting $g(x)=\frac{1}{f(x)},$ we get \[g(x+y)\cdot g(x)\cdot g(y)=g(xy)\cdot \big[g(x)+g(y)\big],\quad \forall x ,\,y \in \mathbb R^*,\, x+y \ne 0. \quad (1)\] Replacing $x=y=2$ in $(1),$ we get $g(2)=2.$ Now, let $a=g(1)$ and $b=g({-1}).$ Replacing $x=2,\,y=-1$ in $(1),$ we get \[g({-2})=\frac{2ab}{2+b}.\quad (2)\] On the other hand, replacing $x=-2,\,y=1$ in $(1),$ we also have \[g({-2})=ab-a.\quad (3) \] Combining $(2)$ and $(3),$ we have $ab-a=\frac{2ab}{2+b},$ and then it follows that $b=2$ or $b=-1.$ But $g$ is injective (since $f$ is injective), so we must have $b=-1$ or $g({-1})=-1.$ It follows that $g({-2})=-2a.$

Replacing $y=1$ in $(1),$ we get \[g(x+1)=\frac{g(x)}{a}+1,\quad \forall x \in \mathbb R^*,\, x \ne -1.\quad (4)\] It follows that $g(3)=\frac{2}{a}+1$ and \[g(x-1)=a\big[g(x)-1\big],\, \forall x \in \mathbb R^*, x\ne 1.\quad (5)\] Replacing $y=-1$ in $(1)$ and using $(5),$ we get \[g({-x})\cdot \big[ g(x)-1\big] =-g(x)\cdot g(x-1)=-g(x)\cdot a \cdot \big[ g(x)-1\big],\] or \[g({-x})=-a\cdot g(x),\quad \forall x \in \mathbb R^*,\, x \ne 1.\quad (6)\] It follows that $g({-3})=-a\cdot g(3)=-a\left(\frac{2}{a}+1\right)=-a-2.$ Now, replacing $y=-2$ in $(1),$ we get \[g({-2x})\cdot \big[g(x)-2a\big]=-2a\cdot g(x)\cdot g(x-2),\quad \forall x \in \mathbb R^*, x\ne 2.\quad (7)\] Replacing $x=-1$ in $(7),$ we have \[g({-3})=-\frac{1+2a}{a}.\] So $a+2=\frac{2a+1}{a}$ and thus $a=1$ or $a=-1.$ Since $g$ is injective, we have $a=1$ or $g(1)=1.$ It follows that \[g(x+1)=g(x)+1,\quad \forall x \in \mathbb R^*, x \ne -1\quad (8)\]
and
\[g(x+2)=g(x+1)+1=g(x)+2,\quad \forall x \in \mathbb R^*\setminus \{{-1},\, {-2}\}. \quad (9)\] Now, replacing $y=2$ in $(1)$ and using $(9),$ we get \[g(2x)\cdot \big[g(x)+2\big]=2\cdot g(x)\cdot g(x+2)=2\cdot g(x)\cdot \big[ g(x)+2\big],\] therefore \[g(2x)=2\cdot g(x),\quad \forall x \in \mathbb R^*\setminus \{{-1},{-2}\}.\quad (10)\] Since $g({-x})=-a\cdot g(x)=-g(x),\, \forall x \in \mathbb R^*\setminus \{1\},$ we have \[g({-2})=-g(2)=-2\cdot g(1)=-2g\big({-({-1})}\big)=2\cdot g({-1})\] and \[g({-4})=-g(4)=-2\cdot g(2)=-2\cdot g\big({-({-2})}\big)=2\cdot g({-2}).\] This shows that the equality $(10)$ still holds for $x=-1$ and $x=-2,$ from which we conclude that \[g(2x)=2\cdot g(x),\quad \forall x \in \mathbb R^*.\quad (11)\] Now, replacing $y$ by $2y$ in $(1)$ and using $(11),$ we get \[g(xy)\cdot \big[g(x)+2\cdot g(y)\big]=g(x)\cdot g(y)\cdot g(x+2y),\quad \forall x ,\, y \in \mathbb R^*,\, x+2y \ne 0.\quad (12)\] Dividing $(1)$ for $(12),$ side by side, we get \[\frac{g(x)+g(y)}{g(x)+2\cdot g(y)}=\frac{g(x+y)}{g(x+2y)},\quad \forall x ,\,y \in \mathbb R^*,\, x +y \ne 0,\, x+2y \ne 0.\quad (13)\] Replacing $x$ by $x-y$ in $(13),$ we get \[\frac{g(x-y)+g(y)}{g(x)}=\frac{g(x-y)+2\cdot g(y)}{g(x+y)},\quad \forall x ,\,y \in \mathbb R^*,\, x^2-y^2 \ne 0.\quad (14)\] Now, replacing $y$ by $-y$ in $(14)$ with notice that $g({-y})=-g(y),\, \forall y \in \mathbb R^*$ (it is easy to show this), we have \[\frac{g(x+y)-g(y)}{g(x)}=\frac{g(x+y)-2\cdot g(y)}{g(x-y)},\quad \forall x,\,y \in \mathbb R^*,\, x^2-y^2 \ne 0. \quad (15)\] Replacing $y$ by $-y$ in $(1),$ we get \[g(xy)\cdot \big[g(x)-g(y)\big] =g(x)\cdot g(y)\cdot g(x-y),\quad \forall x ,\,y \in \mathbb R^*,\, x -y \ne 0. \quad (16)\] Dividing $(1)$ for $(16),$ side by side, we get \[\frac{g(x)+g(y)}{g(x)-g(y)}=\frac{g(x+y)}{g(x-y)},\quad \forall x,\,y \in \mathbb R^*,\, x^2-y^2 \ne 0.\quad (17)\] Setting $A=g(x),\,B=g(y), \, C=g(x+y)$ and $D=g(x-y).$ From $(14),\, (15)$ and $(17),$ we have \[\left\{\begin{aligned} &\displaystyle \frac{D+B}{A}=\frac{D+2B}{C} &&(1') \\ &\displaystyle \frac{C-B}{A}=\frac{C-2B}{D} &&(2') \\ &\displaystyle \frac{A+B}{A-B}=\frac{C}{D} &&(3')\end{aligned}\right.\] From $(3'),$ we have $D=\frac{C(A-B)}{A+B}.$ Plugging this into $(1')$ and $(2'),$ we get \[\frac{\frac{C(A-B)}{A+B}+B}{A}=\frac{\frac{C(A-B)}{A+B}+2B}{C} \Leftrightarrow (C-A-B)(2AB+AC-BC)=0\] and \[\frac{C-B}{A}=\frac{C-2B}{\frac{C(A-B)}{A+B}}\Leftrightarrow (C-A-B)(2AB+BC-AC)=0.\] It follows that \[\begin{aligned} 4AB(C-A-B)&=(C-A-B)(2AB+AC-BC)+(C-A-B)(2AB+BC-AC)\\ &=0,\end{aligned}\] and hence \[C=A+B.\] In the other word, we have \[g(x+y)=g(x)+g(y),\quad \forall x,\,y \in \mathbb R^*,\, x^2-y^2 \ne 0. \quad (18)\] Combining this result with $(11),$ we get \[g(x+y)=g(x)+g(y),\quad \forall x ,\,y \in \mathbb R^*,\, x+y \ne 0. \quad (19)\] Plugging this result into $(1),$ we also have \[g(xy)=g(x)\cdot g(y),\quad \forall x ,\,y \in \mathbb R^*,\, x+y \ne 0. \quad (20)\] From now, we can easily show that $g(x)=x,\, \forall x \in \mathbb R^*$ and hence $f(x)=\frac{1}{x},\, \forall x \in \mathbb R^*.$ This function satisfies the given condition.
\end{solution}



\begin{solution}[by \href{https://artofproblemsolving.com/community/user/213306}{saturzo}]
	$P(1, 1) : f(2) = \frac{1}{2}$.

Again, using this, $P(-1, -1) : 2(-1)f(-2)=f(1)$ and $P(-1, 2) : f(1) \{ f(-1)+ \frac{1}{2} \} = f(-2)$.

Multiplying, we get $2f(-1)^2 + f(-1) - 1 = 0 \implies f(-1) = -1$ or $f(-1)=\frac{1}{2}$. but as $f$ is injective, we get $f(-1)=-1$.

Now, define the function $g : \mathbb{R}^* \to \mathbb{R}^*$ as $g(x) = \frac{1}{f(x)}$.

So the the original problem becomes finding all injective $g : \mathbb{R}^* \to \mathbb{R}^*$ , s.t. , $g(xy)\{g(x)+g(y)\}=g(x+y)g(x)g(y), \forall x, y \in \mathbb{R}^*$ with $x+y \neq 0$. And also, $g(-1) =-1, g(2) = 2$.

$P(x, 1) : g(x+1) = \frac{g(x)}{g(1)}+1$.

Using all these, $P(x, -1) : g(-x) = -g(-1)g(x), \forall x \in \mathbb{R}^*, x \neq 1$.
So, $g(x)=-g(-1)g(x)$ and multiplying, $g(1)^2=1 \implies g(1) \in \{1, -1 \}$. But as $g$ injective and $g(-1)=-1$, we must have $g(1)=1$.

$\therefore g(-x)=-g(x)$ & $g(x+1) = g(x)+1, [x\neq -1] \forall x \in \mathbb{R}^*$.

Using this, $g(1)=1$ and $g(-1)=-1$, we get $g(n)=n, \forall n \in \mathbb{Z}^*$.

And if $x \in \mathbb{R}^*$ is not an integer, then we get (repeatedly using $g(x+1) = g(x)+1, [x\neq -1] \forall x \in \mathbb{R}^*$) $g(x+n)=g(x)+n$. [and this is obviously also true for non-zero integer $x$ as $g(n)=n, \forall n \in \mathbb{Z}^*$]

Now let $q = \frac{a}{b} \in \mathbb{Q}^* [b \neq 1]$ with $a, b \in \mathbb{Z}^*$. $\therefore bq = a$.

Now, $P(b, q) : g(q) = q$. $\therefore g(q) = q, \forall q \in \mathbb{Q}^*$.

Again, $P(x, n) : g(nx) = ng(x), \forall n \in \mathbb{Z}^*, x \in \mathbb{R}^*, x \neq - n$. So, also $g(qx) = qg(x), \forall q \in \mathbb{Q}^*, x \in \mathbb{R}^*$.

$P(x, q) : g(x+q) = g(x)+q, \forall q \in \mathbb{Q}^*, x \in \mathbb{R}^*, x \neq -q$.

Again, $P(x, x) : g(x^2) = g(x)^2 > 0 [\because g(2x)=2g(x)] \implies g(x) > 0 \forall x \in \mathbb{R}^+$.

So, we must have  $g(x)=x, i.e. , \boxed{f(x) =\frac{1}{x}} \forall x \in \mathbb{R}^*$.
\end{solution}
*******************************************************************************
-------------------------------------------------------------------------------

\begin{problem}[Posted by \href{https://artofproblemsolving.com/community/user/125553}{lehungvietbao}]
	Find all functions $f:\mathbb{R}\to\mathbb{R}$ such that \[f(x+y)=x^2f(\frac{1}{x})+y^2f(\frac{1}{y}) \quad \forall x,y\in \mathbb{R}^*\]
	\flushright \href{https://artofproblemsolving.com/community/c6h559684}{(Link to AoPS)}
\end{problem}



\begin{solution}[by \href{https://artofproblemsolving.com/community/user/29428}{pco}]
	\begin{tcolorbox}Find all functions $f:\mathbb{R}\to\mathbb{R}$ such that \[f(x+y)=x^2f(\frac{1}{x})+y^2f(\frac{1}{y}) \quad \forall x,y\in \mathbb{R}^*\]\end{tcolorbox}
Let $P(x,y)$ be the assertion $f(x+y)=x^2f(\frac 1x)+y^2f(\frac 1y)$ $\forall x,y\ne 0$

$P(x,x)$ $\implies$ $f(2x)=2x^2f(\frac 1x)$ and so new asserion $Q(x,y)$ : $f(x+y)=\frac 12f(2x)+\frac 12f(2y)$ $\forall x,y\ne 0$

Let $x,y,z\ne 0$ : 
$Q(x,y+z)$ $\implies$ $f(x+y+z)=\frac 12f(2x)+\frac 12f(2y+2z)$ $=\frac 12f(2x)+\frac 14f(4y)+\frac 14f(4z)$
$Q(y,x+z)$ $\implies$ $f(x+y+z)=\frac 12f(2y)+\frac 14f(4x)+\frac 14f(4z)$

Subtracting, we get $f(4x)-2f(2x)=f(4y)-2f(2y)$ and so $f(2x)=2f(x)+c$ $\forall x\ne 0$ and for some $c\in\mathbb R$

$Q(x,y)$ becomes then new assertion $R(x,y)$ : $f(x+y)=f(x)+f(y)+c$ $\forall x,y\ne 0$

$P(1,1)$ $\implies$ $f(2)=2f(1)$
$R(1,1)$ $\implies$ $f(2)=2f(1)+c$ and so $c=0$ and $R(x,y)$ becomes $S(x,y)$ : $f(x+y)=f(x)+f(y)$ $\forall x,y\ne 0$
Plugging this in $P(x,y)$, we get $f(x)=x^2f(\frac 1x)+u$ and setting $x=1$, we get $u=0$

So, system is equivalent to :
$f(x+y)=f(x)+f(y)$ $\forall x,y\ne 0$ and $f(x)=x^2f(\frac 1x)$ $\forall x\ne 0$
Then :
$f(x^2)+f(x)=f(x(x+1))=x^2(x+1)^2f(\frac 1{x(x+1)})$ $=x^2(x+1)^2f(\frac 1x-\frac 1{x+1})$ $=x^2(x+1)^2f(\frac 1x)-x^2(x+1)^2f(\frac 1{x+1})$
$=(x+1)^2f(x)-x^2f(x+1)$ $=(x+1)^2f(x)-x^2f(x)-x^2f(1)$

And so new assertion $T(x)$ : $f(x^2)=2xf(x)-f(1)x^2$ $\forall x\ne 0$
Subtracting $T(x)+T(y)$ from $T(x+y)$, we get $f(xy)=yf(x)+xf(y)-xyf(1)$ $\forall x,y,x+y\ne 0$

Seting $y=\frac 1x$, this becomes $f(x)=f(1)x$

And so $\boxed{f(x)=cx}$ $\forall x\ne 0$, and whatever is $c\in\mathbb R^*$, which indeed is a solution.
\end{solution}



\begin{solution}[by \href{https://artofproblemsolving.com/community/user/125553}{lehungvietbao}]
	Dear Mr Pco. Thank you  very much ! 

Your solutions  are always perfect  :first: . I  always appreciate your solutions. I hope you can post more and more  your solutions in this forum.

Once again , thank you so much 
\end{solution}
*******************************************************************************
-------------------------------------------------------------------------------

\begin{problem}[Posted by \href{https://artofproblemsolving.com/community/user/68025}{Pirkuliyev Rovsen}]
	Find all the pairs of functions $f,g: \mathbb{R}\to\mathbb{R}$ such that $g(f(x)-y)=f(g(y))+x$.
	\flushright \href{https://artofproblemsolving.com/community/c6h559736}{(Link to AoPS)}
\end{problem}



\begin{solution}[by \href{https://artofproblemsolving.com/community/user/29428}{pco}]
	\begin{tcolorbox}Find all the pairs of functions $f,g: \mathbb{R}\to\mathbb{R}$ such that $g(f(x)-y)=f(g(y))+x$.\end{tcolorbox}
We immediately get that $g(x)$ is surjective.
$P(0,a-x)$ $\implies$ $f(g(a-x))=g(x)$ and so $f(x)$ is surjective too

a) : $P(x,-y)$ $\implies$ $g(f(x)+y)=f(g(-y))+x$
b) : $P(0,-y)$ $\implies$ $g(a+y)=f(g(-y))$
c) ; $P(x,0)$ $\implies$ $g(f(x))=f(g(0))+x$
a-b-c) : $g(f(x)+y)=g(a+y)+g(f(x))-f(g(0))$

And, since $f(x)$ is surjective : $g(x+y)=g(x)+g(y+a)-b$ where $b=f(g(0))$

So $g(x)=h(x)+b-h(a)$ where $h(x)$ is some additive Cauchy function.

$P(0,x)$ $\implies$ $g(a-x)=f(g(x))$ and so $f(g(x))=b-h(x)=2b-h(a)-g(x)$
And, since $g(x)$ is surjective : $f(x)=c-x$ where $c=2b-h(a)$

Plugging these values in $P(x,y)$, we get : $h(x)=h(c)-h(a)-x$ and so $g(x)=d-x$ for some $d$

Plugging these values in $P(x,y)$, we get : $c=d$

Hence the solutions $\boxed{(f,g)=(a-x,a-x)}$
\end{solution}
*******************************************************************************
-------------------------------------------------------------------------------

\begin{problem}[Posted by \href{https://artofproblemsolving.com/community/user/29428}{pco}]
	\begin{tcolorbox}Find all functions $f$ from the reals to the reals that satisfy \[f(x^3)+f(y^3)=(x+y)(f(x^2)+f(y^2)-f(xy))\] for all real numbers $x,y$.\end{tcolorbox}
Let $P(x,y)$ be the assertion $f(x^3)+f(y^3)=(x+y)(f(x^2)+f(y^2)-f(xy))$
Let $a=f(1)$

$P(0,0)$ $\implies$ $f(0)=0$
$P(x,0)$ $\implies$ $f(x^3)=xf(x^2)$ and so $f(-x)=-f(x)$

$P(x,1)$ $\implies$ $f(x^3)+a=(x+1)(f(x^2)+a-f(x))$
$P(x,0)$ $\implies$ $f(x^3)=xf(x^2)$
Subtracting, we get $0=f(x^2)+ax-(x+1)f(x)$
Setting $x\to -x$ in this last line, we get $0=f(x^2)-ax+(-x+1)f(x)$
Subtracting then these two last equalities, we get $\boxed{f(x)=ax}$ $\forall x$ and whatever is $a\in\mathbb R$, which indeed is a solution
	\flushright \href{https://artofproblemsolving.com/community/c6h559755}{(Link to AoPS)}
\end{problem}



\begin{solution}[by \href{https://artofproblemsolving.com/community/user/29428}{pco}]
	\begin{tcolorbox}... 
so that the last equation becomes $g(x^3)=g(x^2)$ for all real $x$. Hence, $g(x^3)=g(x^2)=c$ for some constant $c$. 
...
\end{tcolorbox}
$g(x^3)=g(x^2)$, without any continuity assumption, does not imply $g(x)=c$
\end{solution}



\begin{solution}[by \href{https://artofproblemsolving.com/community/user/29428}{pco}]
	\begin{tcolorbox}[quote="pco"]\begin{tcolorbox}... 
so that the last equation becomes $g(x^3)=g(x^2)$ for all real $x$. Hence, $g(x^3)=g(x^2)=c$ for some constant $c$. 
...
\end{tcolorbox}
$g(x^3)=g(x^2)$, without any continuity assumption, does not imply $g(x)=c$\end{tcolorbox}
:oops: please explain 'continuity assumption'?\end{tcolorbox}

Choose for example :
$g(0)=g(-1)=g(1)=0$

$\forall x\notin\{-1,0,1\}$ : $g(x)=\left\{\frac{\ln(|\ln(|x|)|)}{\ln 3 - \ln 2}\right\}$

This function is not constant and such that $g(x^3)=g(x^2)$ $\forall x$

You need to add the continuity property to conclude from $g(x^3)=g(x^2)$ that $g(x)$ is constant.
\end{solution}



\begin{solution}[by \href{https://artofproblemsolving.com/community/user/29428}{pco}]
	\begin{tcolorbox}[quote="pco"]You need to add the continuity property\end{tcolorbox}
Sorry if this is a stupid question, but how should I do this?\end{tcolorbox}
I dont understand your question.

You cant conclude from $g(x^3)=g(x^2)$ the fact that $g(x)=c$ (I gave you such a non contant example).

It would be possible to get this conclusion if the problem statement says that $g(x)$ is continuous (just write $g(x^{\left(\frac 32\right)^n})=g(x)$ and take the limit when $\n\to\pm \infty$)

But this can not be done in the current problem.

So this path can not lead to a solution of the current problem.
\end{solution}
*******************************************************************************
-------------------------------------------------------------------------------

\begin{problem}[Posted by \href{https://artofproblemsolving.com/community/user/196684}{cause_im_batman}]
	For positive integers n , define f(n)
$f(n)=\sum_{i=1}^{n}i^{n+1-i}$
What is the minimum value of f(n+1)\/f(n)
	\flushright \href{https://artofproblemsolving.com/community/c6h559844}{(Link to AoPS)}
\end{problem}



\begin{solution}[by \href{https://artofproblemsolving.com/community/user/29428}{pco}]
	\begin{tcolorbox}For positive integers n , define f(n)
$f(n)=\sum_{i=1}^{n}i^{n+1-i}$
What is the minimum value of f(n+1)\/f(n)\end{tcolorbox}
$f(1)=1$

$f(2)=3$ and so $\frac{f(2)}{f(1)}=3>\frac 83$

$f(3)=8$ and so $\frac{f(3)}{f(2)}=\frac 83$

$f(4)=22$ and so $\frac{f(4)}{f(3)}=\frac{11}4>\frac 83$

$f(5)=65$ and so $\frac{f(5)}{f(4)}=\frac{65}{22}>\frac 83$

$f(6)=209$ and so $\frac{f(6)}{f(5)}=\frac{209}{65}>\frac 83$

If $n\ge 6$ : $f(n+1)-3f(n)\ge (1+2^n+4^{n-2})-3(1+2^{n-1}+4^{n-3})=4^{n-3}-2^{n-1}-2\ge 0$ and so $\frac{f(n+1)}{f(n)}\ge 3>\frac 83$

Hence the result $\boxed{\min_{n\in\mathbb N}\frac{f(n+1)}{f(n)}=\frac{f(3)}{f(2)}=\frac 83}$
\end{solution}
*******************************************************************************
-------------------------------------------------------------------------------

\begin{problem}[Posted by \href{https://artofproblemsolving.com/community/user/68025}{Pirkuliyev Rovsen}]
	Find all the function $f(x)$, continuous on the whole real axis, such that for every real $x$,  $f(3x-2){\leq}f(x){\leq}f(2x-1)$.
	\flushright \href{https://artofproblemsolving.com/community/c6h560148}{(Link to AoPS)}
\end{problem}



\begin{solution}[by \href{https://artofproblemsolving.com/community/user/29428}{pco}]
	\begin{tcolorbox}Find all the function $f(x)$, continuous on the whole real axis, such that for every real $x$,  $f(3x-2){\leq}f(x){\leq}f(2x-1)$.\end{tcolorbox}
Let $g(x)=f(x+1)$ and the inequation becomes $g(3x)\le g(x)\le g(2x)$

And so $g(\frac x2)\le g(x)\le g(\frac x3)$

And so $g(\frac x{2^n})\le g(x)\le g(\frac x{3^n})$

Setting $n\to+\infty$ and using continuity, we get $g(x)=g(0)$ constant and so $\boxed{f(x)=c}$ constant, which indeed is a solution
\end{solution}
*******************************************************************************
-------------------------------------------------------------------------------

\begin{problem}[Posted by \href{https://artofproblemsolving.com/community/user/68025}{Pirkuliyev Rovsen}]
	Find all  functions $f: \mathbb{R}\to\mathbb{R}$ such that $f(ab)=f(a+b)$ for all irrational $a$ and $b$.
	\flushright \href{https://artofproblemsolving.com/community/c6h560278}{(Link to AoPS)}
\end{problem}



\begin{solution}[by \href{https://artofproblemsolving.com/community/user/29428}{pco}]
	\begin{tcolorbox}Find all  functions $f: \mathbb{R}\to\mathbb{R}$ such that $f(ab)=f(a+b)$ for all irrational $a$ and $b$.\end{tcolorbox}
Let $I=\mathbb R\setminus\mathbb Q$
Let $P(x,y)$ be the assertion $f(xy)=f(x+y)$, true $\forall x,y\in I$

Let $x\in I$ : $x,-4x,2x,-2x\in I$ and so :
$P(x,-4x)$ $\implies$ $f(-3x)=f(-4x^2)$
$P(2x,-2x)$ $\implies$ $f(0)=f(-4x^2)$
So $f(-3x)=f(0)$ $\forall x\in I$
And so $f(x)=f(0)$ $\forall x\in I$

Let $x\in\mathbb Q$ : $\sqrt 2,x-\sqrt 2,x\sqrt 2-2\in I$ and so :
$P(\sqrt 2,x-\sqrt 2)$ $\implies$ $f(x)=f(x\sqrt 2 - 2)=f(0)$

And so $\boxed{f(x)=c}$ $\forall x\in\mathbb R$, and whatever is $c\in\mathbb R$, which indeed is a solution
\end{solution}
*******************************************************************************
-------------------------------------------------------------------------------

\begin{problem}[Posted by \href{https://artofproblemsolving.com/community/user/196684}{cause_im_batman}]
	Find al functions f: N → N with the property that
f(f(m) + f(n)) = m + n, for all m and n

I tried finding t with t=f(t) but couldnt get close.
	\flushright \href{https://artofproblemsolving.com/community/c6h560671}{(Link to AoPS)}
\end{problem}



\begin{solution}[by \href{https://artofproblemsolving.com/community/user/89198}{chaotic_iak}]
	Does $\mathbb{N}$ include $0$?
\end{solution}



\begin{solution}[by \href{https://artofproblemsolving.com/community/user/150054}{wer}]
	f is injective, f (f (n)) + f (n) = 2n., f (0) = 0, f (1) = 1, and by induction, we assume that f (n) = n.How
f (n +1)> = n +1 and f (f (n +1))> = n +1, the f (f (n +1)) + f (n +1) = 2n +2, result to f (n +1) = n + 1
\end{solution}



\begin{solution}[by \href{https://artofproblemsolving.com/community/user/29428}{pco}]
	\begin{tcolorbox}Find al functions $f: \mathbb{N} \to \mathbb{N}$ with the property that
\[f(f(m) + f(n)) = m + n \quad \forall m ,n\]

I tried finding t with t=f(t) but couldnt get close.\end{tcolorbox}
@wer : it seems you misread parenthesis
@chaotic_jak ; I'll consider $0\notin \mathbb N$

Let $P(x,y)$ be the assertion $f(f(x)+f(y))=x+y$
Note that $\mathbb N\setminus\{1\}\subseteq \mathbb N$

$P(1,1)$ $\implies$ $f(2f(1))=2$
$P(f(x)+f(1),2f(1))$ $\implies$ $f(x+3)=f(x)+3f(1)$

If $f(1)>1$, density of elements of $f(\mathbb N)$ in $[1,p]$ has limit $\frac 1{f(1)}<1$ when $p\to+\infty$, in contradiction with the fact that $\mathbb N\setminus\{1\}\subseteq \mathbb N$

So $f(1)=1$ and $f(x+3)=f(x)+3$
$P((f(x)+f(1),1)$ $\implies$ $f(x+2)=f(x)+2$
So $f(x+1)=f(x)+1$ $\forall x\ge 3$
So $f(x+1)=f(x)+1$ $\forall x\in\mathbb N$

Hence the unique solution $\boxed{f(x)=x}$ $\forall x\in\mathbb N$, which indeed is a solution.
\end{solution}



\begin{solution}[by \href{https://artofproblemsolving.com/community/user/196684}{cause_im_batman}]
	I coudnt quite catch your argument for why f(1)=1.Can you please elaborate?
\end{solution}



\begin{solution}[by \href{https://artofproblemsolving.com/community/user/35881}{Ronald Widjojo}]
	1. $f$ is injective (easy)
2. $f(f(n)+f(n+2)=f(2(f(n+1))) \rightarrow f(n+2)-f(n+1)=f(n+1)-f(n)$
3. $f$ is linear, we can easily check that $f(n)=n$
\end{solution}



\begin{solution}[by \href{https://artofproblemsolving.com/community/user/29428}{pco}]
	\begin{tcolorbox}I coudnt quite catch your argument for why f(1)=1.Can you please elaborate?\end{tcolorbox}
$f(x+3)=f(x)+3f(1)$ and so :
$f(3n+1)=f(1)+3nf(1)$ and so at most $\frac N{3f(1)}$ such numbers in $[1,N]$
$f(3n+2)=f(2)+3nf(1)$ and so at most $\frac N{3f(1)}$ such numbers in $[1,N]$
$f(3n+3)=f(3)+3nf(1)$ and so at most $\frac N{3f(1)}$ such numbers in $[1,N]$

And so at most $\frac N{f(1)}$ elements of $f(\mathbb N)$ in $[1,N]$ (what I called "density $\le \frac 1{f(1)}$")

But we know that we have at least $N-1$ elements of $f(\mathbb N)$ in $[1,N]$ (what I called "density is $1$")

So $N-1\le \frac N{f(1)}$ $\forall N$ and so $f(1)=1$
\end{solution}



\begin{solution}[by \href{https://artofproblemsolving.com/community/user/196684}{cause_im_batman}]
	\begin{tcolorbox}
$P(f(x)+f(1),2f(1))$ $\implies$ $f(x+3)=f(x)+3f(1)$
\end{tcolorbox}
Im getting only  $f(f(x)+f(1)+2)=f(x)+3f(1)$
Sorry if im making a dumb mistake
\end{solution}



\begin{solution}[by \href{https://artofproblemsolving.com/community/user/29428}{pco}]
	\begin{tcolorbox}[quote="pco"]
$P(f(x)+f(1),2f(1))$ $\implies$ $f(x+3)=f(x)+3f(1)$
\end{tcolorbox}
Im getting only  $f(f(x)+f(1)+2)=f(x)+3f(1)$
Sorry if im making a dumb mistake\end{tcolorbox}
$P(f(x)+f(1),2f(1))$ $\implies$ $f(f(f(x)+f(1))+f(2f(1)))=f(x)+f(1)+2f(1)$

But $P(x,1)$ $\implies$ $f(f(x)+f(1))=x+1$

And so $f(x+1+2)=f(x)+f(1)+2f(1)$

And $f(x+3)=f(x)+3f(1)$
\end{solution}



\begin{solution}[by \href{https://artofproblemsolving.com/community/user/179506}{IvL}]
	My solution:

$f(f(m)+f(n))=m+n$

Function is injective:
$f(m)=f(n)$
$2f(m)=2f(n)$
$f(2f(m))=f(2f(n))$
$m=n$

Function is semi-surjective:
$m=1$
$f(f(1)+f(n))=n+1$
Therefore, for all $n\geqslant 2$ there is $x$ such that $f(x)=n$

$a=f(1), b=f(2)$

$f(a+f(2n))=2n+1$

$(m=n+1)$
$f(f(n)+f(n+1))=2n+1$

$a+f(2n)=f(n)+f(n+1)$
$f(2n)=f(n)+f(n+1)-a$

$(m=n)$
$f(2f(n))=2n=f(a+f(2n-1))$
$f(2n-1)=2f(n)-a$


$f(3)=f(4-1)=2f(2)-a=2b-a$
$f(4)=f(2)+f(3)-a=3b-2a$
$f(5)=f(6-1)=2f(3)-a=4b-3a$

Conclusion: $f(n)=(n-1)b-(n-2)a$ for all $n\geqslant 2$
Proof by induction:
$f(2n)=f(n)+f(n+1)-a=(n-1)b-(n-2)a+nb-(n-1)a-a=(2n-1)b-(2n-2)a$
$f(2n-1)=2f(n)-a=2(n-1)b-2(n-2)a-a=(2n-2)b-(2n-3)a$

$f(n)=(n-1)b-(n-2)a=n(b-a)+(2a-b)$
$f(n)$ is an arithmetic sequence, and from semi-surjectivity we conclude that $f(n)$ must grow by 1
$b-a=1$
$b=1+a$

$f(n)=n+(2a-b)=n+(2a-1-a)=n+a-1$


$f(f(n)+f(m))=m+n$
$f(n+m+2a-2)=m+n$
$n+m+2a-2+a-1=m+n$
$3a=3$
$a=1$

$f(n)=n$
\end{solution}
*******************************************************************************
-------------------------------------------------------------------------------

\begin{problem}[Posted by \href{https://artofproblemsolving.com/community/user/167245}{TheChainheartMachine}]
	Let $f: \mathbb{R} \cup \{\infty\} \to \mathbb{R} \cup \{\infty\}$ be defined by
\[ f(x) = \begin{cases} \frac{ax+b}{cx + d} & \mathrm{if} \quad x \in \mathbb{R} \setminus \left\{-\frac{d}{c}\right\} \\ \infty & \mathrm{if} \quad x = -\frac{d}{c} \\
\frac{a}{c} & \mathrm{if} \quad x = \infty \end{cases} \]
for some $a, b, c, d \in \mathbb{R}$, satisfying $(a - d)^2 < -4bc$ and $ad - bc > 0$.
Prove that there exists $n \in \mathbb{N}$ such that $f^{n}(x) := (\underbrace{f \circ f \circ \dots \circ f}_{n \, \mathrm{times}})(x) = x$ if and only if $\frac{1}{\pi}\cos^{-1}\left(\frac{a+d}{2\sqrt{ad - bc}}\right) \in \mathbb{Q}$.

Edit: Thanks for the catch, pco.
	\flushright \href{https://artofproblemsolving.com/community/c6h560674}{(Link to AoPS)}
\end{problem}



\begin{solution}[by \href{https://artofproblemsolving.com/community/user/29428}{pco}]
	\begin{tcolorbox}Let $f: \mathbb{R} \setminus \left\{-\frac{d}{c}\right\} \to \mathbb{R}$ be defined by $f(x) = \frac{ax + b}{cx + d}$ for some $a, b, c, d \in \mathbb{R}$, satisfying $(a - d)^2 < -4bc$ and $ad - bc > 0$.
Prove that there exists $n \in \mathbb{N}$ such that $f^{n}(x) := (\underbrace{f \circ f \circ \dots \circ f}_{n \, \mathrm{times}})(x) = x$ if and only if $\frac{1}{\pi}\cos^{-1}\left(\frac{a+d}{2\sqrt{ad - bc}}\right) \in \mathbb{Q}$.\end{tcolorbox}
In order to answer the question, we need a full better definition of $f^n(x)$

For example, what is the value of $f^2(x)$ when $x=-\frac{bc+d^2}{c(a+d)}$ since then $f(x)=-\frac dc$ and so your definition is no longer available.
Knowledge of the choices you give in these situations may change the value of $f^n(x)$ and so the answer to the question.
Thanks to be precise (it's always the same problem : one forget definitions of domain and codomain of all functions involved :( )
\end{solution}



\begin{solution}[by \href{https://artofproblemsolving.com/community/user/167245}{TheChainheartMachine}]
	Throwing $\infty$ into the mix may help things in this case--take, then, $f\left(-\frac{d}{c}\right) = \infty$ and $f(\infty) = \frac{a}{c}$. In this case then we are heading into the real projective line.
(This is the most natural extension that I know. I might have wanted to add the caveat "except on a finite set," I guess, but I'm not certain that's strong enough to restrict our $f$.)
\end{solution}



\begin{solution}[by \href{https://artofproblemsolving.com/community/user/29428}{pco}]
	\begin{tcolorbox}Let $f: \mathbb{R} \cup \{\infty\} \to \mathbb{R} \cup \{\infty\}$ be defined by
\[ f(x) = \begin{cases} \frac{ax+b}{cx + d} & \mathrm{if} \quad x \in \mathbb{R} \setminus \left\{-\frac{d}{c}\right\} \\ \infty & \mathrm{if} \quad x = -\frac{d}{c} \\
\frac{a}{c} & \mathrm{if} \quad x = \infty \end{cases} \]
for some $a, b, c, d \in \mathbb{R}$, satisfying $(a - d)^2 < -4bc$ and $ad - bc > 0$.
Prove that there exists $n \in \mathbb{N}$ such that $f^{n}(x) := (\underbrace{f \circ f \circ \dots \circ f}_{n \, \mathrm{times}})(x) = x$ if and only if $\frac{1}{\pi}\cos^{-1}\left(\frac{a+d}{2\sqrt{ad - bc}}\right) \in \mathbb{Q}$.

Edit: Thanks for the catch, pco.\end{tcolorbox}
\begin{bolded}Note \end{underlined}\end{bolded}: following quick proof need to be more rigorous about $\infty$ usage (for example case $a+d=0$)

$(a-d)^2<-4bc$ $\implies$ $c\ne 0$

Let then $u=\frac{\sqrt{-(a-d)^2-4bc}}{2c}$ and $v=\frac{a-d}{2c}$

Some simple calculus imply $f(u\tan x +v)=u\tan (x+\theta) +v$ where $\tan\theta=-\frac{\sqrt{-(a-d)^2-4bc}}{a+d}$

And so $f^n(u\tan x+v)=u\tan (x+n\theta) +v$

So the required property is $\frac{\theta}{\pi}\in\mathbb Q$

And so $\frac 1{\pi}\tan^{-1}\left(-\frac{\sqrt{-(a-d)^2-4bc}}{a+d}\right)\in\mathbb Q$

which is easily transformed in the required expression with $\cos^{-1}$
\end{solution}



\begin{solution}[by \href{https://artofproblemsolving.com/community/user/167245}{TheChainheartMachine}]
	Nicely done! My original solution to this problem was probably out of the bounds of olympiad mathematics. Glad to see something somewhat more elementary. Here's what I had:

Remember that the group of Mobius transformations $f(z) = \frac{az + b}{cz + d}$ under composition is isomorphic to the projective linear group which is actually the quotient group $GL(2, \mathbb{C})\/\{ cI_2 \}$. Since the identity corresponds to the matrices $\{cI_2\}$ we want $A = \left(\begin{array}{cc}a & b \\ c & d\end{array}\right)^n = cI_2$ for some $c$. If we permit complex numbers outside $\mathbb{R}$, and there should be no reason for us not to, we can write $M = P^{-1}DP$ and so $M^n = P^{-1}D^nP$ where $D$ is a diagonal matrix consisting of the eigenvalues $\lambda_1$ and $\lambda_2$ of $M$. $M^n = cI_2$ iff $D^n = cI_2$, meaning that we want $\lambda_1^n = \lambda_2^n$ for some $n$. Now, $\lambda_1 = \frac{(a+d) + \sqrt{(a-d)^2 + 4bc}}{2}$ and $\lambda_2 = \frac{(a+d) - \sqrt{(a-d)^2 + 4bc}}{2}$. By the first given inequality, these are conjugate complex eigenvalues which can be written in the form $\lambda_1 = R\operatorname{cis}\theta$, $\lambda_2 = R\operatorname{cis}(-\theta)$. We in fact get $R = \frac{1}{2}\sqrt{(a+d)^2 - (a-d)^2 - 4bc} = \sqrt{ad - bc}$ and so $\cos{\theta} = \frac{a+d}{2\sqrt{ad-bc}}$. We then want $\cos{n\theta} = 1$ for some $n$, and that will happen if and only if $\frac{\theta}{\pi} \in \mathbb{Q}$.
\end{solution}
*******************************************************************************
-------------------------------------------------------------------------------

\begin{problem}[Posted by \href{https://artofproblemsolving.com/community/user/119826}{seby97}]
	Find the functions $f:[1,\infty) \rightarrow [1,\infty)$ such that:
1)$f(x)=\sqrt{xf(x+1)+1}[\/tex]$for every $x \in [1, \infty)$ 
2)$f(x) \le 2x+2$ for every $x \ge 1$
	\flushright \href{https://artofproblemsolving.com/community/c6h560714}{(Link to AoPS)}
\end{problem}



\begin{solution}[by \href{https://artofproblemsolving.com/community/user/29428}{pco}]
	\begin{tcolorbox}Find the functions $f:[1,\infty) \rightarrow [1,\infty)$ such that:
1)$f(x)=\sqrt{xf(x+1)+1}$ for every $x \in [1, \infty)$ 
2)$f(x) \le 2x+2$ for every $x \ge 1$\end{tcolorbox}
1) $f(x)\le x+1$ $\forall x\ge 1$
==================
Suppose we have $f(x)\le a_n(x+1)$ $\forall x\ge 1$ for some $a_n\ge 1$

Then $f(x)\le\sqrt{xa_n(x+2)+1}$ $\le \sqrt{xa_n(x+2)+a_n}=\sqrt{a_n}(x+1)$

Setting $a_1=2$ and $a_{n+1}=\sqrt{a_n}$ and since $\lim_{n\to+\infty}a_n=1$, we get $f(x)\le x+1$

2) $f(x)\ge x$ $\forall x\ge 1$
===============
Since $f(x+1)\ge 1$, we get $f(x)\ge\sqrt{x+1}$

Suppose we have $f(x)\ge \sqrt{x^{a_n}+1}$ $\forall x\ge 1$ for some $a_n\ge 1$

Then $f(x)\ge\sqrt{x\sqrt{(x+1)^{a_n}+1}+1}\ge \sqrt{x^{1+\frac{a_n}2}+1}$

Setting $a_1=1$ and $a_{n+1}=1+\frac{a_n}2$ and since $\lim_{n\to+\infty}a_n=2$, we get $f(x)\ge \sqrt {x^2+1}>x$

3) $f(x)\ge x+1$ $\forall x\ge 1$
==================
Suppose we have $f(x)\ge x+a_n$ $\forall x\ge 1$ for some $a_n\in[0,1]$

Then $f(x)\ge\sqrt{x(x+a_n+1)+1}$ $\ge \sqrt{x(x+a_n+1)+\left(\frac{a_n+1}2\right)^2}$ $=x+\frac{a_n+1}2$

Setting $a_1=0$ and $a_{n+1}=\frac{a_n+1}2$ and since $\lim_{n\to+\infty}a_n=1$, we get $f(x)\ge x+1$


And so $\boxed{f(x)=x+1}$ $\forall x$ which indeed is a solution.
\end{solution}
*******************************************************************************
-------------------------------------------------------------------------------

\begin{problem}[Posted by \href{https://artofproblemsolving.com/community/user/125553}{lehungvietbao}]
	Let $f$ be a function on natural numbers $f:\mathbb{N}\to\mathbb{N}$ with the following properties 
\[\begin{cases}\left ( f(2n)+f(2n+1)+1 \right )\left ( f(2n+1)-f(2n)-1 \right )=3\left ( 1+2f(n) \right )\\f(2n)\geq f(n)\end{cases}\] for all natural numbers $n$. Determine all values of $n$ such that $f(n)\leq 2009$.
	\flushright \href{https://artofproblemsolving.com/community/c6h560850}{(Link to AoPS)}
\end{problem}



\begin{solution}[by \href{https://artofproblemsolving.com/community/user/29428}{pco}]
	\begin{tcolorbox}Let $f$ be a function on natural numbers $f:\mathbb{N}\to\mathbb{N}$ with the following properties 
\[\begin{cases}\left ( f(2n)+f(2n+1)+1 \right )\left ( f(2n+1)-f(2n)-1 \right )=3\left ( 1+2f(n) \right )\\f(2n)\geq f(n)\end{cases}\] for all natural numbers $n$. Determine all values of $n$ such that $f(n)\leq 2009$.\end{tcolorbox}
This problem can not be solved without a supplementary information. For example $f(1)$ :

First equation may be written $f(2n+1)^2-(f(2n)+1)^2=6f(n)+3$ and so $f(2n+1)>f(2n)+1$ and $f(2n+1)$ and $f(2n)$ both are odd or both are even.

$f(2n)\ge f(n)$ implies then $f(2n+1)^2-(f(2n)+1)^2\le 6f(2n)+3$
and so $f(2n+1)^2-(f(2n)+4)^2\le -12$ and so $f(2n+1)< f(2n)+4$

So $f(2n+1)=f(2n)+2$ (the case $f(2n+1)=f(2n)+3$ would imply one is odd and the other even)

Plugging this in first equation, we get $f(2n)=3f(n)$ and so we have :

$f(2n)=3f(n)$
$f(2n+1)=3f(n)+2$

Choose then any positive value you want for $f(1)$ and let $\overline a$ be the representation of $f(1)$ in base $3$
Write $n$ in binary form.
Replace all the ones by $2$
Replace the leading $2$ by $\overline a$
You got the base $3$ representation of $f(n)$

And obviously the values of $n$ such that $f(n)\le 2009$ can easily be computed as soon as we know $f(1)$
But maybe your olympiad problem ask for the required number as a function of $f(1)$  ???
\end{solution}



\begin{solution}[by \href{https://artofproblemsolving.com/community/user/125553}{lehungvietbao}]
	Dear Mr.Patrick 
Maybe you misunderstood our problem :( 
The answer is $\boxed{ f(n)\le 2009 \iff n\in \{0,1,2,...,107\}}$
\end{solution}



\begin{solution}[by \href{https://artofproblemsolving.com/community/user/29428}{pco}]
	\begin{tcolorbox}Dear Mr.Patrick 
Maybe you misunderstood our problem :( 
The answer is $\boxed{ f(n)\le 2009 \iff n\in \{0,1,2,...,107\}}$\end{tcolorbox}

Maybe you misunderstood my answer :(

If $f(1)=3000$, for example, answer is $f(n)>2009$  $\forall n$

I gave you my proof that problem can not be solved without a supplementary information.
Dont hesitate to show where is my error and to give us your own proof for your own answer.

And, btw, $0\notin\mathbb N$ (at least in this forum) and so your answer is wrong.
\end{solution}



\begin{solution}[by \href{https://artofproblemsolving.com/community/user/125553}{lehungvietbao}]
	Dear Mr.Patrick
Sorry for late responding :) . 
Here is my effort  to solve this problem ( Note that $0\in\mathbb{N}$ and $0\notin\mathbb{N}^*$ ) 


Since $3\left (1+2f(n)\right )$ is odd. So $f(2n+1)-f(2n)-1$ is odd and so $f(2n+1)\geq f(2n)+2>f(2n)\geq f(n) \quad \forall n\in\mathbb{N}$

Therefore \[f(2n+1)+f(2n)+1\geq 2f(2n)+3>1+2f(n)\quad \forall n\in\mathbb{N}\] And so
\[ \begin{cases} f(2n+1)-f(2n)-1=1\\f(2n+1)+f(2n)+1=3\left ( 1+2f(n)\right )\quad \forall n\in\mathbb{N}\end{cases} \]

So $f(2n+1)=f(2n)+2;f(2n)=3f(n) \quad  \forall n\in\mathbb{N} \quad (1)$

Now by mathematical induction, we will show that $f(n)<f(n+1)\quad  \forall n\in\mathbb{N} \quad (2)$

Since $(1)$ we have $f(1)=f(0)+2>f(0) \quad (f(0)=3f(0)\implies f(0)=0)$

Assume that $f(0)<f(1)<...<f(k), \quad k\in\mathbb{N}^*$

If $k$ is even ,$k=2m \quad m\in\mathbb{N}$  then $f(k+1)=f(2m+1)=f(2m)+2>f(2m)=f(k)$

If $k$ is odd , $k=2m+1 \quad m\in\mathbb{N}$ then 
$f(k+1)=f(2m+2)=3f(m+1)\geq 3(f(m)+1) $ $>3f(m)+2=f(2m)+2=f(2m+1)=f(k)$ 
(Note that $k=2m+1\implies m+1\leq k\implies f(m+1)< f(m) \implies f(m)+1\leq f(m+1)$ because all numbers are integer in this case)
So we get $f(k+1)>f(k)$ ( So the statement $(2)$ is true )

Since $(1)$ we get $f(0)=0,f(1)=2$
So
$f(2)=3f(1)=6$

$f(3)=f(2)+2=8$

$f(13)=f(12)+2=f(2^2\cdot 3)+2=3^2f(3)+2=74$

 $f(27)=f(2\cdot 13+1)=3f(13)+2=224$

$f(53)=f(2^2\cdot 13+1)=3^2f(13)+2=668$

$f(108)=f(2^2\cdot 27)=3^2f(27)=2016$

$f(107)=f(2\cdot 53+1)=3f(53)+2=2006$

Therefore $f(107)<2009<f(108)$
Since $(2)$ we get the conclusion  $ \boxed{ f(n)\le 2009\iff n\in\{0,1,2,...,107\}} $.





I have a small suggestion for you: Why don't you solve some problems in [url=http://www.artofproblemsolving.com/Forum\/viewforum.php?f=151]High School Pre-Olympiad (Ages 14+)[\/url] ?
 (There are many hard problem as  Olympiad-level  problems and I bet that  they are your strong points , such as: functional equation, number theory, equations,...  )
I always appreciate  all your solutions !
\end{solution}



\begin{solution}[by \href{https://artofproblemsolving.com/community/user/29428}{pco}]
	\begin{tcolorbox}...
( Note that $0\in\mathbb{N}$ and $0\notin\mathbb{N}^*$ ) 
...
\end{tcolorbox}
Unfortunately, this is a local convention not widely used (although it's used in France too :)).
On this forum, it is considered without precision that $0\notin \mathbb N$
And it is also considered as a wise attitude to precise the exact definition (positive integers or nonnegative integers) when the problem consider a different definition.
\end{solution}
*******************************************************************************
-------------------------------------------------------------------------------

\begin{problem}[Posted by \href{https://artofproblemsolving.com/community/user/180320}{Torus121}]
	A real- valued function $f$ satisfies the relation 

$ f(x^2+x) + 2f(x^2-3x+2) = 9x^2-15x$ for all real values of $x$. Find $f(2013)$.
	\flushright \href{https://artofproblemsolving.com/community/c6h560865}{(Link to AoPS)}
\end{problem}



\begin{solution}[by \href{https://artofproblemsolving.com/community/user/29428}{pco}]
	\begin{tcolorbox}A real- valued function $f$ satisfies the relation 

$ f(x^2+x) + 2f(x^2-3x+2) = 9x^2-15x$ for all real values of $x$. Find $f(2013)$.\end{tcolorbox}
Let $P(x)$ be the assertion $f(x^2+x)+2f(x^2-3x+2)=9x^2-15x$

a) : $P(\frac 32-x)$ $\implies$ $f(x^2-4x+\frac{15}4)+2f(x^2-\frac 14)=9x^2-12x-\frac 94$

b) : $P(x-\frac 12)$ $\implies$ $f(x^2-\frac 14)+2f(x^2-4x+\frac{15}4)=9x^2-24x+\frac{39}4$

2a-b : $f(x^2-\frac 14)=3x^2-\frac{19}4$

And so $f(x)=3x-4$ $\forall x\ge-\frac 14$

And so $\boxed{f(2013)=6035}$
\end{solution}



\begin{solution}[by \href{https://artofproblemsolving.com/community/user/197138}{MathBM}]
	Substituting ([code]1-x[\/code]) in place of  ([code]x[\/code])  we get - 
[code]f(x^2-3x+2) + 2f(x^2+x) = 9x^2 - 3x-6[\/code] and
[code]f(x^2+x)+2f(x^2-3x+2) = 9x^2-15x[\/code]
On solving we get
[code]f(x^2+x)=3(x^2+x) + 1[\/code] OR
[code]f(t)=3t+1[\/code]
So[code]f(2013)=3.2013 + 1 = 6040[\/code]
\end{solution}



\begin{solution}[by \href{https://artofproblemsolving.com/community/user/29428}{pco}]
	\begin{tcolorbox}...
$f(t)=3t+1$
...\end{tcolorbox}
Unfortunately, plugging back this expression in original equation, we see that it does not fit.
\end{solution}
*******************************************************************************
-------------------------------------------------------------------------------

\begin{problem}[Posted by \href{https://artofproblemsolving.com/community/user/68025}{Pirkuliyev Rovsen}]
	Find all  functions $f: \mathbb{R}\to\mathbb{R}$ such that $f(f(x)+y^2)=x+y^2$.
	\flushright \href{https://artofproblemsolving.com/community/c6h561100}{(Link to AoPS)}
\end{problem}



\begin{solution}[by \href{https://artofproblemsolving.com/community/user/29428}{pco}]
	\begin{tcolorbox}Find all  functions $f: \mathbb{R}\to\mathbb{R}$ such that $f(f(x)+y^2)=x+y^2$.\end{tcolorbox}
Let $P(x,y)$ be the assertion $f(f(x)+y^2)=x+y^2$
Let $a=f(0)$

$P(0,0)$ $\implies$ $f(a)=0$
$P(a,x)$ $\implies$ $f(x^2)=x^2+a$ and so $f(x)=x+a$ $\forall x\ge 0$

Let $x\in \mathbb R$ and $y$ such that $f(x)+y^2\ge 0$ : $P(x,y)$ $\implies$ $f(x)+y^2+a=x+y^2$ and so $f(x)=x-a$ $\forall x$

So $a=0$ and $\boxed{f(x)=x}$ $\forall x$, which indeed is a solution.
\end{solution}
*******************************************************************************
-------------------------------------------------------------------------------

\begin{problem}[Posted by \href{https://artofproblemsolving.com/community/user/68025}{Pirkuliyev Rovsen}]
	Find all  functions $f: \mathbb{R}\to\mathbb{R}$ such that $f(f(f(x))+y)=x+y$.
	\flushright \href{https://artofproblemsolving.com/community/c6h561101}{(Link to AoPS)}
\end{problem}



\begin{solution}[by \href{https://artofproblemsolving.com/community/user/29428}{pco}]
	\begin{tcolorbox}Find all  functions $f: \mathbb{R}\to\mathbb{R}$ such that $f(f(f(x))+y)=x+y$.\end{tcolorbox}
Let $P(x,y)$ be the assertion $f(f(f(x))+y)=x+y$
Let $a=f(0)$

$P(0,0)$ $\implies$ $f(f(a))=0$
$P(a,x)$ $\implies$ $f(x)=x+a$  $\forall x$

Plugging this back in original equation, we get $a=0$ and so $\boxed{f(x)=x}$ $\forall x$, which indeed is a solution.
\end{solution}
*******************************************************************************
-------------------------------------------------------------------------------

\begin{problem}[Posted by \href{https://artofproblemsolving.com/community/user/177525}{reny}]
	We have $f(x)+f\left(\dfrac{x+2016}{x+1}\right)=x$,find $f(x).$
	\flushright \href{https://artofproblemsolving.com/community/c6h561270}{(Link to AoPS)}
\end{problem}



\begin{solution}[by \href{https://artofproblemsolving.com/community/user/29428}{pco}]
	\begin{tcolorbox}We have $f(x)+f\left(\dfrac{x+2016}{x+1}\right)=x$,find $f(x).$\end{tcolorbox}
Is it a real exercise you got in a real contest ?

There are infinitely many solutions which can be built point per point and no closed form is available for all these solutions.
\end{solution}



\begin{solution}[by \href{https://artofproblemsolving.com/community/user/141363}{alibez}]
	\begin{tcolorbox}We have $f(x)+f\left(\dfrac{x+2016}{x+1}\right)=x$,find $f(x).$\end{tcolorbox}

find $f(x)$ or find all $f(x)$ ?
\end{solution}



\begin{solution}[by \href{https://artofproblemsolving.com/community/user/177525}{reny}]
	\begin{tcolorbox}Is it a real exercise you got in a real contest ?
There are infinitely many solutions which can be built point per point and no closed form is available for all these solutions.\end{tcolorbox}
Can you add a conditon to obtain only $f(x)$?
I want to know the way to solve these types of problem. Thanks!
\end{solution}



\begin{solution}[by \href{https://artofproblemsolving.com/community/user/29428}{pco}]
	\begin{tcolorbox}[quote="pco"]Is it a real exercise you got in a real contest ?
There are infinitely many solutions which can be built point per point and no closed form is available for all these solutions.\end{tcolorbox}
Can you add a conditon to obtain only $f(x)$?
I want to know the way to solve these types of problem. Thanks!\end{tcolorbox}
Please give us the exact statement of the problem you got in your contest \/ exam \/ course \/ book.
\end{solution}
*******************************************************************************
-------------------------------------------------------------------------------

\begin{problem}[Posted by \href{https://artofproblemsolving.com/community/user/196684}{cause_im_batman}]
	Let f:R->R be a function such that f(x+y+xy)=f(x)+f(y)+f(xy) for all x,y.Prove that f satisifies f(x+y)=f(x)+f(y) for all x,y in R. R is the set of all real nos.
	\flushright \href{https://artofproblemsolving.com/community/c6h561279}{(Link to AoPS)}
\end{problem}



\begin{solution}[by \href{https://artofproblemsolving.com/community/user/29428}{pco}]
	\begin{tcolorbox}Let $f:\mathbb{R}\to\mathbb{R}$ be a function such that $f(x+y+xy)=f(x)+f(y)+f(xy)\quad \forall x,y$. Prove that $f$ satisfies $f(x+y)=f(x)+f(y)\quad \forall x,y \in \mathbb{R}$. Where $\mathbb{R}$ is the set of all real numbers.\end{tcolorbox}
Let $P(x,y)$ be the assertion $f(x+y+xy)=f(x)+f(y)+f(xy)$

$P(0,0)$ $\implies$ $f(0)=0$
$P(0,-1)$ $\implies$ $f(-x)=-f(x)$

Let $x\ne -1$
$P(x,\frac y{x+1})$ $\implies$ $f(x+y)=f(x)+f(\frac y{x+1})+f(\frac{xy}{x+1})$

$P(x,-\frac y{x+1})$ $\implies$ $f(x-y)=f(x)-f(\frac y{x+1})-f(\frac{xy}{x+1})$

Adding, we get $f(x+y)+f(x-y)=2f(x)$ $\forall x\ne -1$, $\forall y$
Setting $x=1$, we get $f(1+y)+f(1-y)=2f(1)$ and so (multiplying by $-1$ and since $f(-x)=-f(x)$) :
$f(-1-y)+f(-1+y)=2f(-1)$ and so assertion is true for $x=-1$ too.

Hence new assertion $Q(x,y)$ : $f(x+y)+f(x-y)=2f(x)$ $\forall x,y$

$Q(\frac{x+y}2,\frac{x+y}2)$ $\implies$ $f(x+y)=2f(\frac{x+y}2)$

$Q(\frac{x+y}2,\frac{x-y}2)$ $\implies$ $f(x)+f(y)=2f(\frac{x+y}2)$

subtracting, we get $f(x+y)=f(x)+f(y)$
Q.E.D.
\end{solution}



\begin{solution}[by \href{https://artofproblemsolving.com/community/user/93837}{jjax}]
	Let $P(x,y)$ denote the proposition that $f(xy+x+y)=f(xy)+f(x)+f(y)$.

$P(0,0)$ givs $f(0)=0$.

Consider $a>0$.
Comparing $P(x,a)$ and $P(xa, \frac{1}{a})$ gives $f(xa+x+a) - f(xa+x+ \frac{1}{a}) = f(a) - f( \frac{1}{a})$.
Varying $x$, we see that $f(z+a- \frac{1}{a}) - f(z)$ is constant for all real $z$.

Note that as $a$ varies over all positive reals, $a - \frac{1}{a}$ takes all real values. 
Thus, for any real $y$, we see that $f(z+y)-f(z)$ is constant as $z$ varies.
As a result, $f(x+y)-f(x)=f(y)-f(0)$ for all $x,y$. As $f(0)=0$ we have $f(x+y)=f(x)+f(y)$.
\end{solution}
*******************************************************************************
-------------------------------------------------------------------------------

\begin{problem}[Posted by \href{https://artofproblemsolving.com/community/user/125553}{lehungvietbao}]
	Find all functions $f:\mathbb{N}^*\to\mathbb{N}^*$ which satisfy the following the conditions :
a) $f$ is strictly increasing 
b) $f(f(n))=4n+9 \quad \forall n\in\mathbb{N}^*$
c) $f(f(n)-n)=2n+9 \quad \forall n\in\mathbb{N}^*$

Note that $0\notin \mathbb{N}^*$
	\flushright \href{https://artofproblemsolving.com/community/c6h561366}{(Link to AoPS)}
\end{problem}



\begin{solution}[by \href{https://artofproblemsolving.com/community/user/29428}{pco}]
	\begin{tcolorbox}Find all functions $f:\mathbb{N}^*\to\mathbb{N}^*$ which satisfy the following the conditions :
a) $f$ is strictly increasing 
b) $f(f(n))=4n+9 \quad \forall n\in\mathbb{N}^*$
c) $f(f(n)-n)=2n+9 \quad \forall n\in\mathbb{N}^*$

Note that $0\notin \mathbb{N}^*$\end{tcolorbox}
Let $P(n)$ be the assertion $f(f(n))=4n+9$ which implies that $f(x)$ is injective.
Let $Q(n)$ be the assertion $f(f(n)-n)=2n+9$
Let $a=f(1)$

Comparing $P(n)$ and $Q(2n)$ and using injectivity, we get $f(2n)=f(n)+2n$ and so $f(2^n)=2^{n+1}+a-2$

Comparing $Q(n+1)$ and $Q(n)$ and since $f(x)$ is strictly increasing, we get $f(n+1)-(n+1)>f(n)-n$ and so $f(n+1)\ge f(n)+2$
So $f(n)\ge 2n+a-2$

But $f(2^n)=2(2^n)+a-2$ and so $f(n+1)=f(n)+2$ $\forall n$ and so $f(n)=2n+a-2$

Plugging this back in original equation, we get $a=5$ and so $\boxed{f(n)=2n+3}$ $\forall n$, which indeed is a solution.
\end{solution}
*******************************************************************************
-------------------------------------------------------------------------------

\begin{problem}[Posted by \href{https://artofproblemsolving.com/community/user/172182}{Arangeh}]
	Find functions $f:R \rightarrow R$ such that 
$f(x)f(yf(x)-1)=x^2f(y)-f(x)$
(PS:I have proved that for every $x \in Q$ ,$f(x)=x$ but I can not find a method that completes the argument over R.I think my idea was wrong and the correct procedure is different!)
	\flushright \href{https://artofproblemsolving.com/community/c6h561433}{(Link to AoPS)}
\end{problem}



\begin{solution}[by \href{https://artofproblemsolving.com/community/user/29428}{pco}]
	\begin{tcolorbox}Find functions $f:R \rightarrow R$ such that 
$f(x)f(yf(x)-1)=x^2f(y)-f(x)$
(PS:I have proved that for every $x \in Q$ ,$f(x)=x$ but I can not find a method that completes the argument over R.I think my idea was wrong and the correct procedure is different!)\end{tcolorbox}
$\boxed{f(x)=0}$ $\forall x$ is a solution. So let us from now look only for non allzero solutions.

Let $P(x,y)$ be the assertion $f(x)f(yf(x)-1)=x^2f(y)-f(x)$
Let $u$ such that $f(u)\ne 0$

If $f(0)\ne 0$, then $P(0,\frac {x+1}{f(0)})$ $\implies$ $f(x)=-1$ which is not a solution. So $f(0)=0$

If $f(x)=0$, then $P(x,b)$ $\implies$ $x=0$
$P(1,1)$ $\implies$ $f(1)f(f(1)-1)=0$ and so $f(1)=1$

$P(1,x+1)$ $\implies$ $f(x+1)=f(x)+1$

$P(x,1)$ $\implies$ $f(x)f(f(x)-1)=x^2-f(x)$
$P(1,f(x))$ $\implies$ $f(f(x)-1)=f(f(x))-1$ and so $f(x)f(f(x)-1)=f(x)f(f(x))-f(x)$
And so $x^2=f(f(x))f(x)$ $\forall x$

So $(x+1)^2=f(f(x+1))f(x+1)$ $=(f(f(x))+1)(f(x)+1)$ $=x^2+f(f(x))+f(x)+1$ and so $f(f(x))+f(x)=2x$

From $f(f(x))f(x)=x^2$ and $f(f(x))+f(x)=2x$, we get $f(f(x))=\boxed{f(x)=x}$ $\forall x$, which indeed is a solution
\end{solution}
*******************************************************************************
-------------------------------------------------------------------------------

\begin{problem}[Posted by \href{https://artofproblemsolving.com/community/user/196603}{nbh}]
	problem :
1)find all continuous functions $f:\mathbb{R}\to\mathbb{R}$ such that 
\[f(x+y)+f(x-y)=k(f(x)+f(y)),\forall x,y\in\mathbb{R}\]where $k> 1$
2)find all  functions $f:\mathbb{R}\to\mathbb{R}$ such that 
\[f(x^2+f(xy)+f(y))=f(x^2)+f^2(y) +y\forall x,y\in\mathbb{R}\]
	\flushright \href{https://artofproblemsolving.com/community/c6h561500}{(Link to AoPS)}
\end{problem}



\begin{solution}[by \href{https://artofproblemsolving.com/community/user/29428}{pco}]
	\begin{tcolorbox}problem :
1)find all continuous functions $f:\mathbb{R}\to\mathbb{R}$ such that 
\[f(x+y)+f(x-y)=k(f(x)+f(y)),\forall x,y\in\mathbb{R}\]where $k> 1$\end{tcolorbox}
$\boxed{f(x)=0}$ $\forall x$ is a solution. So let us from now look only for non allzero solutions.

Let $P(x,y)$ be the assertion $f(x+y)+f(x-y)=k(f(x)+f(y))$
Let $a=f(1)$
Let $u$ such that $f(u)\ne 0$

$P(0,0)$ $\implies$ $2(k-1)f(0)=0$ and so, since $k>1$ : $f(0)=0$
$P(u,0)$ $\implies$ $(k-2)f(u)=0$ and so $k=2$, else no solution.
Let us from now consider $k=2$

$P((n+1)x,x)$ $\implies$ $f((n+2)x)=2f((n+1)x)-f(nx)+2f(x)$

And so $f(nx)=n^2f(x)$ and so $f(\frac pqx)=\left(\frac pq\right)^2f(x)$ and so $f(x)=ax^2$ $\forall x\in \mathbb Q$

And continuity allows conclusion $\boxed{f(x)=ax^2}$ $\forall x$, which indeed is a solution when $k=2$
\end{solution}



\begin{solution}[by \href{https://artofproblemsolving.com/community/user/29428}{pco}]
	\begin{tcolorbox}problem :
2)find all  functions $f:\mathbb{R}\to\mathbb{R}$ such that 
\[f(x^2+f(xy)+f(y))=f(x^2)+f^2(y) +y\forall x,y\in\mathbb{R}\]\end{tcolorbox}
Let $P(x,y)$ be the assertion $f(x^2+f(xy)+f(y))=f(x^2)+f(y)^2+y$
Let $a=f(0)$

$P(0,x)$ $\implies$ $f(f(x)+a)=f(x)^2+x+a$ and so $f(x)$ is injective

$P(x,1)$ $\implies$ $f(x^2+f(x)+f(1))=f(x^2)+f(1)^2+1$
$P(-x,1)$ $\implies$ $f(x^2+f(-x)+f(1))=f(x^2)+f(1)^2+1$
So $f(x^2+f(x)+f(1))=f(x^2+f(-x)+f(1))$ and so, since injective, $f(x)=f(-x)$ $\forall x$, impossible since injective.

And so no solution.
\end{solution}



\begin{solution}[by \href{https://artofproblemsolving.com/community/user/196603}{nbh}]
	Problem :find all function $f:\mathbb{R}\to\mathbb{R}$ such that 
\[f( (f(x))^2+f(y) + x^2)=2(f(y))^2+x,\forall x,y\in\mathbb{R}\]
\end{solution}



\begin{solution}[by \href{https://artofproblemsolving.com/community/user/29428}{pco}]
	You are welcome. Glad to have helped you.

\begin{tcolorbox}Problem :find all function $f:\mathbb{R}\to\mathbb{R}$ such that 
\[f( (f(x))^2+f(y) + x^2)=2(f(y))^2+x,\forall x,y\in\mathbb{R}\]\end{tcolorbox}

$f(x)$ is surjective and so equation is equivalent to assertion $P(x,y)$ : $f(f(x)^2+y+x^2)=2y^2+x$
Let $a=f(0)$

$P(a+1,-(a+1)^2-f(a+1)^2)$ $\implies$ $2((a+1)^2+f(a+1)^2)^2=-1$ : impossible.

So no solution.

And I'm quite sure these are not real exercises you got in real contests \/ exams.  :(
\end{solution}
*******************************************************************************
-------------------------------------------------------------------------------

\begin{problem}[Posted by \href{https://artofproblemsolving.com/community/user/122641}{TrungK40PBC}]
	\begin{bolded}Problem\end{bolded}. Let $n$ be an integer great than $1$. Find all funcitons $f:\mathbb{R}\rightarrow \mathbb{R}$ such that: \[f(x^{2n}+2f(y))=f(x)^{2n}+y+f(y)\] for all $x,y\in \mathbb{R}$
	\flushright \href{https://artofproblemsolving.com/community/c6h561529}{(Link to AoPS)}
\end{problem}



\begin{solution}[by \href{https://artofproblemsolving.com/community/user/29428}{pco}]
	\begin{tcolorbox}\begin{bolded}Problem\end{bolded}. Let $n$ be an integer great than $1$. Find all funcitons $f:\mathbb{R}\rightarrow \mathbb{R}$ such that: \[f(x^{2n}+2f(y))=f(x)^{2n}+y+f(y)\] for all $x,y\in \mathbb{R}$\end{tcolorbox}
Let $P(x,y)$ be the assertion $f(x^{2n}+2f(y))=f(x)^{2n}+y+f(y)$
Let $a=f(0)$

$f(x)$ is injective
Comparing $P(x,y)$ with $P(-x,y)$ and using injectivity, we get $f(-x)=-f(x)$ $\forall x\ne 0$

Let $x\ne 0$ such that $f(x)\ne 0$ (such $x$ exists since $f(x)$ is injective :
$P(0,x)$ $\implies$ $f(2f(x))=a^{2n}+x+f(x)$
$P(0,-x)$ $\implies$ $f(2f(-x))=a^{2n}-x+f(-x)$ and so (since $x\ne 0$ and $f(x)\ne 0$) : $-f(2f(x))=a^{2n}-x-f(x)$
Adding, we get $a=0$ (and so $f(-x)=-f(x)$ also when $x=0$

$P(x,0)$ $\implies$ $f(x^{2n})=f(x)^{2n}$ and so $P(x,y)$ becomes $f(x^{2n}+2f(y))=f(x^{2n})+y+f(y)$ and so $f(x+2f(y))=f(x)+y+f(y)$ $\forall x\ge 0$
But this is still true for $x<0$ since $f(x)$ is an odd function.

So problem is equivalent to :
$f(x)$ is an odd function such that :
$f(x^{2n})=f(x)^{2n}$ . Notice that this implies$ f(x)\ge 0$ $\forall x\ge 0$
new assertion $Q(x,y)$ : $f(x+2f(y))=f(x)+y+f(y)$ $\forall x,y$

$Q(-x,x)$ $\implies$ $f(-x+2f(x))=x$ and so $f(x)$ is surjective and so bijective.
Subtracting $Q(0,y)$ from $Q(x,y)$, we get  $f(x+2f(y))=f(x)+f(2f(y))$ and so, since surjective, $f(x+y)=f(x)+f(y)$

And since we already got $f(x)\ge 0$ $\forall x\ge 0$, we get $f(x)=xf(1)$ $\forall x$

Plugging this back in original equation, we get $a=1$ and the solution $\boxed{f(x)=x}$ $\forall x$
\end{solution}
*******************************************************************************
-------------------------------------------------------------------------------

\begin{problem}[Posted by \href{https://artofproblemsolving.com/community/user/152100}{Data-SM}]
	Find all functions $ \mathbb{R}^*_+ \to  \mathbb{R}^*_+ $ that satisfy  : 

$f(x)f(y)=2f(x+yf(x))$ , $\forall (x,y) \in \mathbb{R}^*_+ ^2$
	\flushright \href{https://artofproblemsolving.com/community/c6h561575}{(Link to AoPS)}
\end{problem}



\begin{solution}[by \href{https://artofproblemsolving.com/community/user/29428}{pco}]
	\begin{tcolorbox}Find all functions $ \mathbb{R}^*_+ \to  \mathbb{R}^*_+ $ that satisfy  : 

$f(x)f(y)=2f(x+yf(x))$ , $\forall (x,y) \in \mathbb{R}^*_+ ^2$\end{tcolorbox}
Is $\mathbb R^*_+$ the set of positive reals ?
\end{solution}



\begin{solution}[by \href{https://artofproblemsolving.com/community/user/64716}{mavropnevma}]
	Yes. $\mathbb{R}_+ = [0, +\infty)$ and $\mathbb{R}_+^* = (0, +\infty)$. (Similarly for $\mathbb{Z}$ and $\mathbb{Q}$. Also, in Romania (and probably France) $\mathbb{N} = \mathbb{Z}_+$).
\end{solution}



\begin{solution}[by \href{https://artofproblemsolving.com/community/user/29428}{pco}]
	\begin{tcolorbox}Find all functions $ \mathbb{R}^*_+ \to  \mathbb{R}^*_+ $ that satisfy  : 

$f(x)f(y)=2f(x+yf(x))$ , $\forall (x,y) \in \mathbb{R}^*_+ ^2$\end{tcolorbox}
Let $P(x,y)$ be the assertion $f(x)f(y)=2f(x+yf(x))$

1) If $f(x)$ is injective
=============
$P(1,x)$ $\implies$ $f(1)f(x)=2f(1+xf(1))$
$P(x,1)$ $\implies$ $f(x)f(1)=2f(x+f(x))$
$\implies$ $f(x+f(x))=f(1+xf(1))$ $\implies$ $x+f(x)=1+xf(1)$ $\implies$ $f(x)=1+x(f(1)-1)$

Plugging this back in original equation, we get that no injective solution exists.

2) If $f(x)$ is not injective
=================
So $\exists a>b$ such that $f(a)=f(b)$
Let $u=\frac {b-a}{f(a)}>0$ : $P(a,u)$ $\implies$ $f(u)=2$

2.1) $f(x)\ge 2$ $\forall x$
----------------------------
Suppose that exists $x$ such that $f(x)<1$, then $P(x,\frac x{1-f(x)})$ $\implies$ $f(x)=2$, contradiction
So $f(x)\ge 1$ $\forall x$

Suppose now that $f(x)\ge c\ge 1$ $\forall x$ : $P(x,x)$ $\implies$ $f(x)^2\ge 2c$ and so $f(x)\ge \sqrt{2c}$ $\forall x$

Setting then $a_0=1$ and $a_{n+1}=\sqrt{2a_n}$, we get $f(x)\ge a_n$ $\forall x$
And since $\lim_{n\to+\infty}a_n=2$ we get the required result.

2.2) $f(x)\le 2$ $\forall x$
----------------------------
$P(u,x)$ $\implies$ $f(x)=f(2x+u)$ and so $f(x)=f(2^nx+(2^n-1)u)$ $\forall x$

So $2=f(u)=f(2^nu+(2^n-1)u)$ and we can find numbers $u_n$ as great as we want with $f(u_n)=2$
Let $x>0$ and $u_k>x$ : $P(x,\frac{u_k-x}{f(x)})$ $\implies$ $f(x)f(\frac{u_k-x}{f(x)})=4$
And since $f(\frac{u_k-x}{f(x)})\ge 2$, we get $f(x)\le 2$
Q.E.D.

Hence $\boxed{f(x)=2}$ $\forall x$ which indeed is a solution
\end{solution}



\begin{solution}[by \href{https://artofproblemsolving.com/community/user/173631}{pouyanss26}]
	$ p(x,0) => f(x)f(0)=2f(x) => f(x)=0$ for all x or $ f(0)=2 $
if $f(0)=2$ then let $ p(0,y) => f(y)=f(2y) => p(2x,y) => f(2x+yf(x))=f(x+yf(x))$ so $ f(x)=c $ and we knew that $ f(0)=2 $ so $f(x)=2$
\end{solution}



\begin{solution}[by \href{https://artofproblemsolving.com/community/user/64716}{mavropnevma}]
	And why from $f(2x+yf(x))=f(x+yf(x))$ do you infer $ f(x)=c $? ... not to mention you are \begin{bolded}not allowed \end{bolded}to use the argument $0$, since $f$ is only defined on $(0,\infty)$.
\end{solution}



\begin{solution}[by \href{https://artofproblemsolving.com/community/user/199494}{IMI-Mathboy}]
	For given y we consider $g(x)=x+yf(x)$ and this function is injective:  $g(x_1)=(x_2)  f(x_1)f(y)=2f(g(x_1))=2f(g(x_2))=f(x_2)f(y)$ so $f(x_1)=f(x_2)$ hence $x_1=x_2$.if $x_1>x_2$ and $f(x_1)<f(x_2)$ we have $g(x_1)=g(x_2)$ for$y=\frac{x_1-x_2}{f(x_2)-f(x_1)}>0$ which is impossible,hence $f$ is nondecreasing $\to$ $f(x)f(y)=2f(x+yf(x))>=2f(x)$ $\to$ $f(y)>=2$ for $y>0$. Therefore $2f(x+yf(x))=f(x)f(y)=2f(y+xf(y))>=2f(2x)$ $\to$ $f(x+yf(x)>=f(2x)$ since this inequality holds arbitrary small $y$,which means $f$ is constant so $f(x)=2$
\end{solution}
*******************************************************************************
-------------------------------------------------------------------------------

\begin{problem}[Posted by \href{https://artofproblemsolving.com/community/user/68025}{Pirkuliyev Rovsen}]
	Find all continuous functions $f:R \rightarrow R$  such that $f(\sin{\pi}x)=f(x)\cos{\pi}x$. I know the author's solution. Is there any other solution?
	\flushright \href{https://artofproblemsolving.com/community/c6h561668}{(Link to AoPS)}
\end{problem}



\begin{solution}[by \href{https://artofproblemsolving.com/community/user/29428}{pco}]
	\begin{tcolorbox}Find all continuous functions $f:R \rightarrow R$  such that $f(\sin{\pi}x)=f(x)\cos{\pi}x$. I know the author's solution. Is there any other solution?\end{tcolorbox}
$\boxed{f(x)=0}$ $\forall x$ is a solution.
$g(x)$ solution implies $c|g(x)|$ with $c>0$ is continuous non negative solution of the - different - assertion $P(x)$ : $f(\sin \pi x)=f(x)|\cos \pi x|$
So let us from now look only for non all zero non negative continuous solutions of $P(x)$

$P(\frac 12)$ $\implies$ $f(1)=0$
$P(1)$ $\implies$ $f(0)=0$

Subtracting $P(x)$ from $P(x+2)$, we get $(f(x+2)-f(x))|\cos \pi x|=0$ and so $f(x+2)=f(x)$ $\forall x\ne n+\frac 12$
Continuity implies then $f(x+2)=f(x)$ $\forall x$
So $f(x)$ is bounded and we can choose $c$ such that upper bound is $1$

Let then $t\in[-1,1)$ such that $f(t)=1$ (remember that $2$ is a period of $f(x)$)
Let $u\in[0,2)$ such that $\sin \pi u=t$
$P(u)$ $\implies$ $1=f(u)|\cos \pi u|$ and so $u\in\{0,1\}$, and so $t=0$ in contradiction with $f(0)=0$

Hence no non allzero solution for the required equation.
\end{solution}
*******************************************************************************
-------------------------------------------------------------------------------

\begin{problem}[Posted by \href{https://artofproblemsolving.com/community/user/68025}{Pirkuliyev Rovsen}]
	Determine all ${\lambda}>0$ for which the function ${f: \mathbb{N}_0}\to\mathbb{N}$, where $f(n)=[{\lambda}n(n+1)...(n+k)]$ is injective, where $k{\in}N_0$
is given and $[.]$ denote the integer part.
	\flushright \href{https://artofproblemsolving.com/community/c6h561671}{(Link to AoPS)}
\end{problem}



\begin{solution}[by \href{https://artofproblemsolving.com/community/user/29428}{pco}]
	\begin{tcolorbox}Determine all ${\lambda}>0$ for which the function ${f: \mathbb{N}_0}\to\mathbb{N}$, where $f(n)=[{\lambda}n(n+1)...(n+k)]$ is injective, where $k{\in}N_0$
is given and $[.]$ denote the integer part.\end{tcolorbox}
$f(0)=0$ and $f(1)=\lfloor\lambda (k+1)!\rfloor$ and so a necessary condition is $\lambda\ge \frac 1{(k+1)!}$

If $k>0$ and $\lambda\ge \frac 1{(k+1)!}$ :
$f(n+1)>\lambda(n+1)...(n+k)(n+k+1)-1$ $=\lambda n(n+1)...(n+k)+\lambda(k+1)(n+1)...(n+k)-1\ge f(n)+1-1$
So $f(n+1)>f(n)$ and the condition is sufficient.

If $k=0$ and $\lambda\ge \frac 1{(k+1)!}=1$ :
$f(n+1)>\lambda(n+1)-1\ge \lfloor\lambda n\rfloor= f(n)$ and the condition is sufficient.

Hence the answer : $\boxed{\lambda\ge \frac 1{(k+1)!}}$
\end{solution}
*******************************************************************************
-------------------------------------------------------------------------------

\begin{problem}[Posted by \href{https://artofproblemsolving.com/community/user/125553}{lehungvietbao}]
	1) Find all functions $f:\mathbb{R}\to\mathbb{R}$ such that
\[f(xf(y)+y)+f(xy+x)=f(x+y)+2xy \quad \forall x,y\in \mathbb{R}\]

2) Find all continuous functions $f:\mathbb{R}\to\mathbb{R}$ such that
\[f(xy)+f(x+y)=f(xy+x)+f(y) \quad \forall x,y\in \mathbb{R}\]
	\flushright \href{https://artofproblemsolving.com/community/c6h561816}{(Link to AoPS)}
\end{problem}



\begin{solution}[by \href{https://artofproblemsolving.com/community/user/29428}{pco}]
	\begin{tcolorbox}1) Find all functions $f:\mathbb{R}\to\mathbb{R}$ such that
\[f(xf(y)+y)+f(xy+x)=f(x+y)+2xy \quad \forall x,y\in \mathbb{R}\]\end{tcolorbox}
$f(x)=0$ $\forall x$ is not a solution.
Let $P(x,y)$ be the assertion $f(xf(y)+y)+f(xy+x)=f(x+y)+2xy$
Let $a$ such that $f(a)\ne 0$

If $f(0)\ne 0$, then $P(\frac a{f(0)},0)$ $\implies$ $f(a)=0$, impossible. So $f(0)=0$

Let $u=f(\frac 12)+\frac 12$ : $P(1,\frac 12)$ $\implies$ $f(u)=1$.

If $u=-1$, then $P(1,u)$ $\implies$ $f(0)=-2$, impossible. So $u\ne -1$

$P(\frac x{u+1},u)$ $\implies$ $f(x)=\frac{2u}{u+1}x$

Plugging back this in original equation, we get $u=1$ or $u=-\frac 12$ and so

two solutions $\boxed{f(x)=x}$ $\forall x$ and $\boxed{f(x)=-2x}$ $\forall x$
\end{solution}



\begin{solution}[by \href{https://artofproblemsolving.com/community/user/29428}{pco}]
	\begin{tcolorbox}2) Find all continuous functions $f:\mathbb{R}\to\mathbb{R}$ such that
\[f(xy)+f(x+y)=f(xy+x)+f(y) \quad \forall x,y\in \mathbb{R}\]\end{tcolorbox}
$f(x)$ solution implies $f(x)+c$ solution too. So WLOG consider $f(0)=0$
Let $P(x,y)$ be the assertion $f(xy)+f(x+y)=f(xy+x)+f(y)$

1) $f(x+y)=f(x)+f(y)$ $\forall x,y\ge 0$
=========================
Let $S>x>0$
$P(\frac x{S-x+1},S-x)$ $\implies$ $f(\frac{x(S-x)}{S-x+1})+f(\frac x{S-x+1}+S-x)=f(x)+f(S-x)$

$P(S-x,\frac x{S-x+1})$ $\implies$ $f(\frac{x(S-x)}{S-x+1})+f(\frac x{S-x+1}+S-x)=f(\frac x{S-x+1})+f(S-\frac x{S-x+1})$

So $f(x)+f(S-x)=f(\frac x{S-x+1})+f(S-\frac x{S-x+1})$

Considering the sequence $u_0=x\in(0,S)$ and $u_{n+1}=\frac {u_n}{S-u_n+1}$, we get $f(x)+f(S-x)=f(u_n)+f(S-u_n)$

Setting $n\to +\infty$, using continuity, and the fact that $\lim_{n\to+\infty}u_n=0$, we get $f(x)+f(S-x)=f(S)$

And so $f(x+y)=f(x)+f(y)$ $\forall x,y>0$ 

2) $f(-x)=-f(x)$ $\forall x$
=================
Let $x>0$
$P(x,-\frac x{x+1})$ $\implies$ $f(-\frac {x^2}{x+1})+f(\frac{x^2}{x+1})$ $=f(\frac x{x+1})+f(-\frac x{x+1})$

$P(-\frac x{x+1},x)$ $\implies$ $f(-\frac {x^2}{x+1})+f(\frac{x^2}{x+1})$ $=f(x)+f(-x)$

So $f(x)+f(-x)=f(\frac x{x+1})+f(-\frac x{x+1})$ $\forall x>0$

Considering the sequence $u_0=x>0$ and $u_{n+1}=\frac{u_n}{u_n+1}$, we get $f(x)+f(-x)=f(u_n)+f(-u_n)$

Setting $n\to +\infty$, using continuity, and the fact that $\lim_{n\to+\infty}u_n=0$, we get $f(x)+f(-x)=0$
Q.E.D.

3) Solution
=======
From 1) and 2), we easily get $f(x+y)=f(x)+f(y)$ $\forall x,y$ and continuity implies then $f(x)=ax$

And so the solution $\boxed{f(x)=ax+b}$ $\forall x$ and whatever are $a,b\in\mathbb R$, which indeed is a solution.
\end{solution}



\begin{solution}[by \href{https://artofproblemsolving.com/community/user/90995}{hakarimian}]
	\begin{tcolorbox}1) Find all functions $f:\mathbb{R}\to\mathbb{R}$ such that
\[f(xf(y)+y)+f(xy+x)=f(x+y)+2xy \quad \forall x,y\in \mathbb{R}\]\end{tcolorbox}

$P(x,y)$ is the assertion $f(xf(y)+y)+f(xy+x)=f(x+y)+2xy$
$P(1,y)$ $\implies$ $f(f(y)+y)=2y$
Then f is surjective and thus there is b such that $f(b)=1$
$P(x,b)$ $\implies$ $f(xf(b)+b)+f(xb+x)=f(x+b)+2bx$ $\implies$ $f(bx+x)=2bx$
$x=\frac b{b+1}$ then:
$f(b\frac b{b+1}+\frac b{b+1})=2b\frac b{b+1}$ $\implies$ $f(b)=1=\frac {2b^2}{b+1}$ $\implies$ $b=1$ or $b=-1\/2$ 
if $b=1$ $\implies$ $f(2x)=2x$ $\implies$  $\boxed{f(x)=x}$
if $b=-1\/2$ $\implies$ $f(x\/2)=-2.x\/2$ $\implies$ $\boxed{f(x)=-2x}$
\end{solution}
*******************************************************************************
-------------------------------------------------------------------------------

\begin{problem}[Posted by \href{https://artofproblemsolving.com/community/user/153386}{MichiPanaitescu}]
	Let $a$ be a real number. Find all functions $f:\Bbb{R}\rightarrow\Bbb{R}$ which satisfy the condition $f(x)+f(y)=(x+y+a)f(x)f(y).$
	\flushright \href{https://artofproblemsolving.com/community/c6h561820}{(Link to AoPS)}
\end{problem}



\begin{solution}[by \href{https://artofproblemsolving.com/community/user/29428}{pco}]
	\begin{tcolorbox}Let $a$ be a real number. Find all functions $f:\Bbb{R}\rightarrow\Bbb{R}$ which satisfy the condition $f(x)+f(y)=(x+y+a)f(x)f(y).$\end{tcolorbox}
Let $P(x,y)$ be the assertion $f(x)+f(y)=(x+y+a)f(x)f(y)$

If $f(u)\ne 0$ for some $u$, then $P(\frac 1{f(u)}-u-a,u)$ $\implies$ $f(u)=0$ and so contradiction.

Hence the unique solution : $\boxed{f(x)=0}$ $\forall x$
\end{solution}



\begin{solution}[by \href{https://artofproblemsolving.com/community/user/146965}{shivangjindal}]
	Putting , $x,y=0$ gives $2f(0)=af(0)^2$ 
suppose that $f(0)=0$ the putting $y=0$ gives $f(x)=0 \forall x$ 
and if $f(0) \neq 0 \implies f(0)=\frac{2}{a}$ But putting $x=-a\/2$ and $y=0$ again gives contradiction since , $f(\frac{-a}{2})+f(0)=(\frac{a}{2})f(\frac{-a}{2})f(0) \implies \frac{2}{a}=0$ which is contradiction . 
Thus $f(x)=0$
\end{solution}
*******************************************************************************
-------------------------------------------------------------------------------

\begin{problem}[Posted by \href{https://artofproblemsolving.com/community/user/196603}{nbh}]
	1)let $k>0$ be a real number,find all functions $f:\mathbb{R^+}\to\mathbb{R^+}$ such that 
\[f(x)+f(y)=(x+y+k)f(x)f(y)f(x+y),\forall x,y\in\mathbb{R^+}\]
2)find all continuous functions $f:\mathbb{R^+}\to\mathbb{R^+}$ such that 
\[f(2x-y)+f(2x+y)=8xf(x)+2yf(y),\forall 2x>y \;\in\mathbb{R^+}\]
	\flushright \href{https://artofproblemsolving.com/community/c6h561900}{(Link to AoPS)}
\end{problem}



\begin{solution}[by \href{https://artofproblemsolving.com/community/user/29428}{pco}]
	\begin{tcolorbox}1)let $k>0$ be a real number,find all functions $f:\mathbb{R^+}\to\mathbb{R^+}$ such that 
\[f(x)+f(y)=(x+y+k)f(x)f(y)f(x+y),\forall x,y\in\mathbb{R^+}\]\end{tcolorbox}
Let $P(x,y)$ be the assertion $f(x)+f(y)=(x+y+k)f(x)f(y)f(x+y)$

$P(x,x)$ $\implies$ $f(2x)=\frac{2}{(2x+k)f(x)}$ and so $f(4x)=\frac{2x+k}{4x+k}f(x)$

$P(4x,4y)$ $\implies$ $\frac{2x+k}{4x+k}f(x)+\frac{2y+k}{4y+k}f(y)$ $=(4x+4y+k)\frac{2x+k}{4x+k}$ $\frac{2y+k}{4y+k}$ $\frac{2x+2y+k}{4x+4y+k}$ $f(x)f(y)f(x+y)$

And so $\frac{2x+k}{4x+k}f(x)+\frac{2y+k}{4y+k}f(y)$ $=\frac{2x+k}{4x+k}$ $\frac{2y+k}{4y+k}$ $(2x+2y+k)\frac{f(x)+f(y)}{x+y+k}$

And so $\left(\frac{4y+k}{2y+k}-\frac{2x+2y+k}{x+y+k}\right)f(x)$ $+\left(\frac{4x+k}{2x+k}-\frac{2x+2y+k}{x+y+k}\right)f(y)=0$ 

Setting $y=1$ and simplifying, we get $f(x)=\frac{k+2}{2x+k}f(1)$ which unfortunately is not a solution.

\begin{bolded}So no solution for this equation.\end{underlined}\end{bolded}
\end{solution}



\begin{solution}[by \href{https://artofproblemsolving.com/community/user/29428}{pco}]
	\begin{tcolorbox}2)find all continuous functions $f:\mathbb{R^+}\to\mathbb{R^+}$ such that 
\[f(2x-y)+f(2x+y)=8xf(x)+2yf(y),\forall 2x>y \;\in\mathbb{R^+}\]\end{tcolorbox}
Let $P(x,y)$ be the assertion $f(2x-y)+f(2x+y)=8xf(x)+2yf(y)$

$P(\frac 1{10},\frac 1{10})$ $\implies$ $f(\frac 3{10})=0$, impossible.

\begin{bolded}So no solution for this functional equation\end{underlined}\end{bolded}.
\end{solution}
*******************************************************************************
-------------------------------------------------------------------------------

\begin{problem}[Posted by \href{https://artofproblemsolving.com/community/user/125553}{lehungvietbao}]
	1) Find all functions $f:\mathbb{N}\to\mathbb{N}$ such that
\[f(f(n))=2n \quad \forall n\in \mathbb{N}\]

2) Let $p$ be a prime number. Find all functions $f:\mathbb{N}\to\mathbb{N}$ such that
\[f(f(n))=pn \quad  \forall n\in \mathbb{N}\]

Note that $0\in \mathbb{N}$
	\flushright \href{https://artofproblemsolving.com/community/c6h561941}{(Link to AoPS)}
\end{problem}



\begin{solution}[by \href{https://artofproblemsolving.com/community/user/29428}{pco}]
	\begin{tcolorbox}1) Find all functions $f:\mathbb{N}\to\mathbb{N}$ such that
\[f(f(n))=2n \quad \forall n\in \mathbb{N}\]

2) Let $p$ be a prime number. Find all functions $f:\mathbb{N}\to\mathbb{N}$ such that
\[f(f(n))=pn \quad  \forall n\in \mathbb{N}\]

Note that $0\in \mathbb{N}$\end{tcolorbox}
This is a very classical problem whose general solution is :

Let $A=\{$ positive integers $x$ such that $x\not\equiv 0\pmod p\}$
Let $A_1,A_2$ a split of $A$ in two equinumerous sets (such a split exists since $|A|=+\infty$)

Let $g(x)$ any bijection from $A_1\to A_2$
Define  then $f(x)$ as :

$f(0)=0$
$\forall x\in A_1$, $\forall $non negative integer $k$ : $f(p^kx)=p^kg(x)$ 
$\forall x\in A_2$, $\forall $non negative integer $k$ : $f(p^kx)=p^{k+1}g^{[-1]}(x)$
\end{solution}
*******************************************************************************
-------------------------------------------------------------------------------

\begin{problem}[Posted by \href{https://artofproblemsolving.com/community/user/125553}{lehungvietbao}]
	1) Find all functions $f:\mathbb{R}^+\to\mathbb{R}^+$ such that
\[f\left ( \frac{f(x)}{y} \right )=yf(y)f(f(x))\quad \forall x,y\in \mathbb{R}^+\]

2) Find all strictly increasing functions $f:\mathbb{R}^+\to\mathbb{R}^+$ such that
\[f\left ( \frac{x^2}{f(x)} \right )=x \quad \forall x\in \mathbb{R}^+\]
	\flushright \href{https://artofproblemsolving.com/community/c6h561947}{(Link to AoPS)}
\end{problem}



\begin{solution}[by \href{https://artofproblemsolving.com/community/user/29428}{pco}]
	\begin{tcolorbox}1) Find all functions $f:\mathbb{R}^+\to\mathbb{R}^+$ such that
\[f\left ( \frac{f(x)}{y} \right )=yf(y)f(f(x))\quad \forall x,y\in \mathbb{R}^+\]\end{tcolorbox}
Let $P(x,y)$ be the assertion $f\left(\frac {f(x)}y\right)=yf(y)f(f(x))$

$P(1,\frac u{f(f(1))})$ $\implies$ $\frac{f\left(\frac {f(1)}{\frac u{f(f(1))}}\right)}{f(\frac u{f(f(1))})}=u$ 
and so any positive real $x$ may be written as $x=\frac{f(a)}{f(b)}$ for some positive real numbers $a,b$

$P(x,\sqrt{f(x)})$ $\implies$ $f(f(x))=\frac 1{\sqrt{f(x)}}$

$P(a,f(b))$ $\implies$ $f\left(\frac{f(a)}{f(b)}\right)=f(b)f(f(b))f(f(a))$ $=\sqrt{\frac{f(b)}{f(a)}}$

And so $\boxed{f(x)=\frac 1{\sqrt x}}$ $\forall x$ which indeed is a solution
\end{solution}



\begin{solution}[by \href{https://artofproblemsolving.com/community/user/29428}{pco}]
	\begin{tcolorbox}2) Find all strictly functions $f:\mathbb{R}^+\to\mathbb{R}^+$ such that
\[f\left ( \frac{x^2}{f(x)} \right )=x \quad \forall x\in \mathbb{R}^+\]\end{tcolorbox}
Strictly what ?
\end{solution}



\begin{solution}[by \href{https://artofproblemsolving.com/community/user/125553}{lehungvietbao}]
	Very sorry Mr. Patrick. :) 
It's  ''strictly increasing'' . I edited

2) Find all strictly increasing functions $f:\mathbb{R}^+\to\mathbb{R}^+$ such that
\[f\left ( \frac{x^2}{f(x)} \right )=x \quad \forall x\in \mathbb{R}^+\]
\end{solution}



\begin{solution}[by \href{https://artofproblemsolving.com/community/user/29428}{pco}]
	\begin{tcolorbox}2) Find all strictly increasing functions $f:\mathbb{R}^+\to\mathbb{R}^+$ such that
\[f ( \frac{x^2}{f(x)}  )=x \quad \forall x\in \mathbb{R}^+\]\end{tcolorbox}
$f(x)$ is injective and surjective and so is an increasing bijection from $\mathbb R^+\to\mathbb R^+$ and so is continuous.

$f(x)=x$ $\forall x$ is a solution. So let us from now look for non identical solutions.
Let $u\ne v$ and $a,b\ne 1$ such that $f(u)=au$ and $f(v)=bv$
If $b\ne a$, wlog $a>b$, continuity allows us to find $v,b$ such that $a>b$ and $\frac{\log b}{\log a}\notin\mathbb Q$

Then $\{\{m\frac{\log b}{\log a}\}\}_{m\in\mathbb Z}$ is dense in $[0,1)$

So $\{m\frac{\log b}{\log a}-n\}_{m,n\in\mathbb Z}$ is dense in $\mathbb R$

So ${\{m\log b}-n\log a\}_{m,n\in\mathbb Z}$ is dense in $\mathbb R$

So $\{\frac{b^m}{a^n}\}_{m,n\in\mathbb Z}$ is dense in $\mathbb R^+$

So we can find $m,n\in\mathbb Z$ such that $\frac ab>\frac {vb^m}{ua^n}>1$

It's easy to establish that $f(ua^n)=aua^n$  and $f(vb^m)=bvb^m$ $\forall m,n\in\mathbb Z$

And so we got $vb^m>ua^n$ and $f(ua^n)>f(vb^m)$, in contradiction with the fact that $f(x)$ is increasing.

So $a=b$

And the solution $\boxed{f(x)=ax}$ $\forall x$ which indeed is a solution, whatever is $a>0$
\end{solution}
*******************************************************************************
-------------------------------------------------------------------------------

\begin{problem}[Posted by \href{https://artofproblemsolving.com/community/user/68025}{Pirkuliyev Rovsen}]
	Determine all functions $ f,g:\mathbb{N}\to\mathbb{N} $ such that $ \underbrace{f{\circ}f{\circ}\cdots{\circ}f}_{g(n)\textrm{ times}}(m)=f(n)+g(m) $ and  $ \underbrace{g{\circ}g{\circ}\cdots{\circ}g}_{f(n)\textrm{ times}}(m)=g(n)+f(m) $
	\flushright \href{https://artofproblemsolving.com/community/c6h562146}{(Link to AoPS)}
\end{problem}



\begin{solution}[by \href{https://artofproblemsolving.com/community/user/29428}{pco}]
	\begin{tcolorbox}Determine all functions $ f,g:\mathbb{N}\to\mathbb{N} $ such that $ \underbrace{f{\circ}f{\circ}\cdots{\circ}f}_{g(n)\textrm{ times}}(m)=f(n)+g(m) $ and  $ \underbrace{g{\circ}g{\circ}\cdots{\circ}g}_{f(n)\textrm{ times}}(m)=g(n)+f(m) $\end{tcolorbox}
\begin{bolded}Claim : No solution for this functional equation\end{underlined}.\end{bolded}

Let $P(x,y)$ be the assertion $f^{g(x)}(y)=f(x)+g(y)$
Let $Q(x,y)$ be the assertion $g^{f(x)}(y)=g(x)+f(y)$
Note that $f(x)=c$ constant is not a solution since $P(x,y)$ would imply $g(y)=0\notin \mathbb N$
Note that $g(x)=c$ constant is not a solution since $Q(x,y)$ would imply $f(y)=0\notin \mathbb N$
Let $u=\min(f(\mathbb N))$
Let $c=\min(\{f(x)-f(y)$ $\forall x,y\in\mathbb N$ such that $f(x)-f(y)>0\}$ which exists since $f(x)$ is not constant
Let $v=\min(g(\mathbb N))$
Let $d=\min(\{g(x)-g(y)$ $\forall x,y\in\mathbb N$ such that $g(x)-g(y)>0\}$ which exists since $g(x)$ is not constant

1) $f(a)=f(b)$ $\iff$ $g(a)=g(b)$
===================
If $f(a)=f(b)$, then comparaison of $P(1,a)$ and $P(1,b)$ implies $g(a)=g(b)$
If $g(a)=g(b)$, then comparaison of $Q(1,a)$ and $Q(1,b)$ implies $f(a)=f(b)$
Q.E.D.

2) $f(a)>f(b)$ $\iff$ $g(a)>g(b)$
=====================
If $f(a)>f(b)$ : comparaison of $Q(a,x)$ and $Q(b,g^{f(a)-f(b)}(x))$ implies $f(g^{f(a)-f(b)}(x))=f(x)+(g(a)-g(b))$
If $g(a)<g(b)$, then setting $f(x)=u$ would imply $f(g^{f(a)-f(b)}(x))<u$, impossible.
So $g(a)>g(b)$ (since $g(a)=g(b)$ would imply, according to 1) above, $f(a)=f(b)$)

If $g(a)>g(b)$ : comparaison of $P(a,x)$ and $P(b,f^{g(a)-g(b)}(x))$ implies $g(f^{g(a)-g(b)}(x))=g(x)+(f(a)-f(b))$
If $f(a)<f(b)$, then setting $g(x)=v$ would imply $g(f^{g(a)-g(b)}(x))<v$, impossible.
So $f(a)>f(b)$ (since $f(a)=f(b)$ would imply, according to 1) above, $g(a)=g(b)$)
Q.E.D.

3) $f(a)<f(b)$ $\iff$ $g(a)<g(b)$
=====================
Direct consequence of $1)$ and $2)$ above

4) $\exists p,q,c\in\mathbb N$ such that $f(\mathbb N)=\{kc$ $\forall k\ge p\}$ and $g(\mathbb N)=\{kc$ $\forall k\ge q\}$
============================================================
Let $a,b$ such that $g(a)-g(b)=d$ : 
$g(a)-g(b)>0$ $\implies$ $f(a)>f(b)$
Then comparaison of $Q(a,x)$ and $Q(b,g^{f(a)-f(b)}(x))$ implies $f(g^{f(a)-f(b)}(x))=f(x)+d$
So $u+kd\in f(\mathbb N)$ $\forall k\ge 0$ and $d\ge c$

Let $a,b$ such that $f(a)-f(b)=c$ : 
$f(a)-f(b)>0$ $\implies$ $g(a)>g(b)$
Then comparaison of $P(a,x)$ and $P(b,f^{g(a)-g(b)}(x))$ implies $g(f^{g(a)-g(b)}(x))=g(x)+c$
So $v+kc\in g(\mathbb N)$ $\forall k\ge 0$ and $c\ge d$

So $c=d$ and $f(\mathbb N)=\{u+kc$ $\forall k\ge 0\}$ and $g(\mathbb N)=\{v+kc$ $\forall k\ge 0\}$ 

Setting then $x,y$ such that $f(x)=u$ and $g(y)=v$ :
$P(x,y)$ $\implies$ $f^{g(x)}(y)=u+v$ and so $c|v$
$Q(y,x)$ $\implies$ $g^{f(y)}(x)=u+v$ and so $c|u$
Q.E.D.

5) $g(x)=f(x)+(q-p)c$ $\forall x$
===================
This is an immediate consequence of 4) and (1)+2)+3)).

6) No solution for this functional equation
=============================
Let $\Delta=(q-p)c$
$P(x,y)$ $\implies$ (taking $f(.)$ of both sides) : $f(f(x)+g(y))=f(x)+g(f(y))$ $\implies$ $f(f(x)+f(y)+\Delta)=f(x)+f(f(y))+\Delta$

Swapping $x,y$, we get $f(f(y)+f(x)+\Delta)=f(y)+f(f(x))+\Delta$ and so $f(f(x))=f(x)+t$ for some $t\in\mathbb Z$

So $f^n(x)=f(x)+(n-1)t$
Plugging this in $P(x,y)$, we get $f(y)+(g(x)-1)t=f(x)+g(y)$
$\implies$ $f(y)+(f(x)+\Delta-1)t=f(x)+f(y)+\Delta$
$\implies$ $f(x)(t-1)=t-\Delta(t-1)$

But $t=1$ is impossible in this last equality and so we get $f(x)=$constant, which is not a solution
Q.E.D
\end{solution}



\begin{solution}[by \href{https://artofproblemsolving.com/community/user/68025}{Pirkuliyev Rovsen}]
	You're a genius Patrick. Thanks you 
\end{solution}
*******************************************************************************
-------------------------------------------------------------------------------

\begin{problem}[Posted by \href{https://artofproblemsolving.com/community/user/68025}{Pirkuliyev Rovsen}]
	If $a_0{\in}R$ and $a_{i+1}=f(a_i)(a_i-f(a_i))$ for all $i{\ge}0$, then determine all functions $f: \mathbb{Z}\to\mathbb{Z}$ such that $a_i=a_{i+2}$ for sufficiently large $i$.
	\flushright \href{https://artofproblemsolving.com/community/c6h562294}{(Link to AoPS)}
\end{problem}



\begin{solution}[by \href{https://artofproblemsolving.com/community/user/31919}{tenniskidperson3}]
	What if $a_0=.5$?  Then $f(a_0)$ is undefined.  Either $a_0$ must be in $\mathbb{Z}$ or $f:\mathbb{R}\rightarrow\mathbb{R}$.
\end{solution}



\begin{solution}[by \href{https://artofproblemsolving.com/community/user/29428}{pco}]
	\begin{tcolorbox}If $a_0{\in}R$ and $a_{i+1}=f(a_i)(a_i-f(a_i))$ for all $i{\ge}0$, then determine all functions $f: \mathbb{Z}\to\mathbb{Z}$ such that $a_i=a_{i+2}$ for sufficiently large $i$.\end{tcolorbox}
I suppose $a_0 \in\mathbb Z$

There are obviously infinitely many very different solutions ,and I dont think we can find a general closed form for all these solutions.

Hereunder is a method to build some of them (when $a_0\ne 0$):

Let $g(n)$ from $\mathbb Z\to \mathbb Z$ any function such that :
$g(n)+n$ is even
$g(n)^2>n^2+4|n|$

Let the sequence $b_n$ defined as :
$b_0=a_0$
$b_{n+1}=\frac{b_n^2-g(b_n)^2}4$
Note that $|b_n|$ is an increasing positive sequence

You can then define $f(n)$ as :
Let $m\in\mathbb N$
$f(0)=0$
$\forall k\in[0,m-1]$ : $f(b_k)=\frac{b_k+g(b_k)}2$
$f(b_m)=0$
$f(x)$ takes any value you want when $x\notin\{0,b_0,b_1, ..., b_m\}$
With these $f(x)$, sequence $a_n$ ends with $0,0,0,0,...$


But there are infinitely many other possibilities, ending for example with $4,4,4,4,...$ (if $a_0\notin\{0,-4,4\}$), or ending with $5,6,5,6,...$
\end{solution}
*******************************************************************************
-------------------------------------------------------------------------------

\begin{problem}[Posted by \href{https://artofproblemsolving.com/community/user/125553}{lehungvietbao}]
	Let $a,b,\alpha, \beta \in \mathbb{R}; a\neq 0;\alpha^2> 4\beta ;\alpha+ \beta \neq 1$. Find all functions $f:\mathbb{R}\to\mathbb{R}$ such that 
\[f(x+a)=\alpha f(x)+\beta f(x-a)+b\]
	\flushright \href{https://artofproblemsolving.com/community/c6h562410}{(Link to AoPS)}
\end{problem}



\begin{solution}[by \href{https://artofproblemsolving.com/community/user/29428}{pco}]
	\begin{tcolorbox}Let $a,b,\alpha, \beta \in \mathbb{R}; a\neq 0;\alpha^2> 4\beta ;\alpha+ \beta \neq 1$. Find all functions $f:\mathbb{R}\to\mathbb{R}$ such that 
\[f(x+a)=\alpha f(x)+\beta f(x-a)+b\]\end{tcolorbox}
Are you sure about $\alpha^2>4\beta$. It should be, according to me $\alpha^2>-4\beta$
\end{solution}



\begin{solution}[by \href{https://artofproblemsolving.com/community/user/125553}{lehungvietbao}]
	Dear Mr.Patrick. I'm very sorry, please for give me, i'm wrong :( . I checked $\alpha^2>-4\beta$

Let $a,b,\alpha, \beta \in \mathbb{R}; a\neq 0;\alpha^2>- 4\beta ;\alpha+ \beta \neq 1$. Find all functions $f:\mathbb{R}\to\mathbb{R}$ such that 
\[f(x+a)=\alpha f(x)+\beta f(x-a)+b\]
\end{solution}



\begin{solution}[by \href{https://artofproblemsolving.com/community/user/29428}{pco}]
	\begin{tcolorbox}Dear Mr.Patrick. I'm very sorry, please for give me, i'm wrong :( . I checked $\alpha^2>-4\beta$

Let $a,b,\alpha, \beta \in \mathbb{R}; a\neq 0;\alpha^2>- 4\beta ;\alpha+ \beta \neq 1$. Find all functions $f:\mathbb{R}\to\mathbb{R}$ such that 
\[f(x+a)=\alpha f(x)+\beta f(x-a)+b\]\end{tcolorbox}
Let $f(x)=g(x)-\frac b{\alpha+\beta-1}$ and equation becomes $g(x+a)=\alpha g(x)+\beta g(x-a)$

Let $r_1,r_2$ be the two distinct real roots of the quadratic $x^2-\alpha x-\beta=0$

$g(x+na)=g(x+a)\frac{r_1^n-r_2^n}{r_1-r_2}+g(x)\frac{r_1r_2^n-r_2r_1^n}{r_1-r_2}$

Hence the general solution : 
Define $g(x)$ as you want in $[0,2a)$ (adapt if $a<0$) and extend elsewhere thru previous formula
Define then $f(x)=g(x)-\frac b{\alpha+\beta-1}$

If  you want a closed form :

$f(x)=h(a\left\{\frac xa\right\}+a)\frac{r_1^{\left\lfloor\frac xa\right\rfloor}-r_2^{\left\lfloor\frac xa\right\rfloor}}{r_1-r_2}+h(a\left\{\frac xa\right\})\frac{r_1r_2^{\left\lfloor\frac xa\right\rfloor}-r_2r_1^{\left\lfloor\frac xa\right\rfloor}}{r_1-r_2}-\frac b{\alpha+\beta-1}$

Where $h(x)$ is any function you want.
\end{solution}
*******************************************************************************
-------------------------------------------------------------------------------

\begin{problem}[Posted by \href{https://artofproblemsolving.com/community/user/125553}{lehungvietbao}]
	1) Find all functions $f:\mathbb{R}\to\mathbb{R}$ such that 

\[f(xy-uv)=f(x)f(y)-f(u)f(v) \quad \forall x,y,u,v\in\mathbb{R}\]

2) Find all functions $f:\mathbb{R}\to\mathbb{R}$ such that 
\[f(x+y)+f(y+z)+f(z+x)\geq 3f(x+2y+3z)  \quad \forall x,y,z \in\mathbb{R}\]
	\flushright \href{https://artofproblemsolving.com/community/c6h562411}{(Link to AoPS)}
\end{problem}



\begin{solution}[by \href{https://artofproblemsolving.com/community/user/29428}{pco}]
	\begin{tcolorbox}1) Find all functions $f:\mathbb{R}\to\mathbb{R}$ such that 

\[f(xy-uv)=f(x)f(y)-f(u)f(v) \quad \forall x,y,u,v\in\mathbb{R}\]\end{tcolorbox}
Let $P(x,y,u,v)$ be the assertion $f(xy-uv)=f(x)f(y)-f(u)f(v)$

$P(0,0,0,0)$ $\implies$ $f(0)=0$
$P(x,y,0,0)$ $\implies$ $f(xy)=f(x)f(y)$
$P(x+y,1,y,1)$ $\implies$ $f(x)=f(x+y)f(1)-f(y)f(1)=f(x+y)-f(y)$

And so $f(x+y)=f(x)+f(y)$ and $f(xy)=f(x)f(y)$ which is a very classical problem whose solutions are :
$\boxed{f(x)=0}$ $\forall x$ which indeed is a solution.
$\boxed{f(x)=x}$ $\forall x$ which indeed is a solution.
\end{solution}



\begin{solution}[by \href{https://artofproblemsolving.com/community/user/29428}{pco}]
	\begin{tcolorbox}2) Find all functions $f:\mathbb{R}\to\mathbb{R}$ such that 
\[f(x+y)+f(y+z)+f(z+x)\geq 3f(x+2y+3z)  \quad \forall x,y,z \in\mathbb{R}\]\end{tcolorbox}
Let $P(x,y,z)$ be the assertion $f(x+y)+f(y+z)+f(z+x)\ge 3f(x+2y+3z)$

$P(x,0,0)$ $\implies$ $f(0)\ge f(x)$

$P(\frac x2,\frac x2,-\frac x2)$ $\implies$ $f(x)\ge f(0)$

And so $\boxed{f(x)=c}$ constant $\forall x$, which indeed is a solution, whatever is $c\in\mathbb R$
\end{solution}
*******************************************************************************
-------------------------------------------------------------------------------

\begin{problem}[Posted by \href{https://artofproblemsolving.com/community/user/185304}{aymas}]
	Determine all continous functions from $R$ to $R$ such that : $f(x-y)+f(x+y)=f(x)f(y)$
	\flushright \href{https://artofproblemsolving.com/community/c6h562475}{(Link to AoPS)}
\end{problem}



\begin{solution}[by \href{https://artofproblemsolving.com/community/user/29428}{pco}]
	\begin{tcolorbox}Determine all continous functions from $R$ to $R$ such that : $f(x-y)+f(x+y)=f(x)f(y)$\end{tcolorbox}
Let $P(x,y)$ be the assertion $f(x+y)+f(x-y)=f(x)f(y)$
Note that, setting $f(x)=2g(x)$, we find a very well known equation ("D'Alembert functional equation").

$P(0,0)$ $\implies$ $f(0)(f(0)-2)=0$ 

If $f(0)=0$, $P(\frac x2,\frac x2)$ $\implies$ $\boxed{f(x)=0}$ $\forall x$ which indeed is a solution

Let us from now consider that $f(0)=2$
$P(0,x)$ $\implies$ $f(-x)=f(x)$ and $f(x)$ is even.
$P(x,x)$ $\implies$ $f(2x)=f(x)^2-2$

1) Case 1 : If $\exists u$ such that $f(u)<2$, then $f(x)=2\cos cx$
==========================================
Suppose $\exists u$ (Wlog $u>0$) such that $a_0=f(u)<2$
Setting $f(2^{n+1}u)=a_{n+1}=a_n^2-2$, we get that $a_k<0$ for some $k\ge 0$.
$f(0)>0$ and $f(2^ku)<0$ $\implies$ $\exists t>0$ such that $f(t)=0$

Let then $A=\{x>0$ such that $f(x)=0\}$
$A$ is non empty and we can define $a=\inf(A)$
Continuity implies $f(a)=0$ (and so $a\ne 0$) and we get $f(x)>0$ $\forall x\in (0,a)$

$P(x+a,a)$ $\implies$ $f(x+2a)=-f(x)$ and so $f(x+4a)=f(x)$ and $f(x)$ is periodic and so, since continuous, is bounded.

If $f(x)>2$ for some $x\in(0,a)$, then, using $f(2x)=f(x)^2-2$, we get that $f(2^nx)$ is unbounded, and so contradiction

So $f(x)\in (0,2]$ $\forall x\in [0,a]$

Choose then $x\in(0,a)$ : $f(x)\in (0,2]$ and so $\exists t\in[0,\frac{\pi}2)$ such that $f(x)=2\cos t$
Successive applications of $f(\frac x2)^2=f(x)+2$ plus the fact that $f(x)\in (0,2]$ imply $f(\frac x{2^n})=2\cos \frac t{2^n}$

Successive applications of $P(y+\frac x{2^n},\frac x{2^n})$ starting from $y=\frac x{2^n}$ imply $f(k\frac x{2^n})=2\cos k\frac t{2^n}$ $\forall k,n$ such that $k\frac x{2^n}\in (0,a)$

Continuity implies than $f(x)=2\cos \frac{\pi}{2a}x$ $\forall x\in[0,a]$

$f(x)$ even $\implies$ $f(x)=2\cos \frac{\pi}{2a}x$ $\forall x\in[-a,a]$

$P(x+a,a)$ $\implies$ $f(x+2a)=-f(x)$ $\implies$ $f(x)=2\cos \frac{\pi}{2a}x$ $\forall x\in[-a,3a]$

$f(x+4a)=f(x)$ $\implies$ $f(x)=2\cos \frac{\pi}{2a}x$ $\forall x$

And so $\boxed{f(x)=2\cos cx}$ $\forall x$ which indeed is a solution, whatever is $c\in\mathbb R^*$
Q.E.D.

2) Case 2 : $f(x)\ge 2$ $\forall x$
====================
2.1 subcase : If  $\exists u$ such that $f(u)>2$, then $f(x)=2\cosh cx$
--------------------------------------------------------------------
Wlog $u>0$
Since $f(u)>2$, $\exists t>0$ such that $f(u)=2\cosh t$
Successive applications of $f(\frac x2)^2=f(x)+2$ plus the fact that $f(x)\ge 2$ imply $f(\frac u{2^n})=2\cong \frac t{2^n}$

Successive applications of $P(x+\frac u{2^n},\frac u{2^n})$ starting from $x=\frac u{2^n}$ imply $f(k\frac u{2^n})=2\cosh k\frac t{2^n}$ $\forall k,n\ge 0$

Continuity implies than $f(x)=2\cosh \frac tu x$ $\forall x\ge 0$

$f(x)$ even $\implies$ $f(x)=2\cosh  \frac tu x$ $\forall x$
And so $\boxed{f(x)=2\cosh cx}$ $\forall x$, which indeed is a solution, whatever is $c\in\mathbb R^*$
Q.E.D.

2.2 subcase : If  $\not \exists u$ such that $f(u)>2$, then $f(x)=2$ $\forall x$
--------------------------------------------------------------------
Case 2 is $f(x)\ge 2$
Subcase 2.2 is $f(x)\le 2$
Hence $\boxed{f(x)=2}$ $\forall x$ which indeed is a solution.

3) Synthesis of solutions
=================
We got three solutions :
$f(x)=0$ $\forall x$
$f(x)=2\cos cx$ $\forall x$ and whatever is $c\in\mathbb R$ (note that $c=0$ is subcase 2.2 : $f(x)=2$ $\forall x$)
$f(x)=2\cosh cx$ $\forall x$ and whatever is $c\in\mathbb R$ (note that $c=0$ is subcase 2.2 : $f(x)=2$ $\forall x$)
\end{solution}
*******************************************************************************
-------------------------------------------------------------------------------

\begin{problem}[Posted by \href{https://artofproblemsolving.com/community/user/125553}{lehungvietbao}]
	1) Find all   functions $f:\mathbb{R}^+\to\mathbb{R}^+$ such that 
\[f(xf(y))f(y)=f(x+y) \quad\forall x,y\in\mathbb{R}^+\]

2) Find all functions $f:\mathbb{R}\to\mathbb{R}$ such that 
\[\left (  f(x+y)\right )^2=f(x)f(x+2y)+yf(y) \quad\forall x,y\in\mathbb{R}\]
	\flushright \href{https://artofproblemsolving.com/community/c6h562685}{(Link to AoPS)}
\end{problem}



\begin{solution}[by \href{https://artofproblemsolving.com/community/user/29428}{pco}]
	\begin{tcolorbox}1) Find all   functions $f:\mathbb{R}^+\to\mathbb{R}^+$ such that 
\[f(xf(y))f(y)=f(x+y) \quad\forall x,y\in\mathbb{R}^+\]\end{tcolorbox}
See http://www.artofproblemsolving.com/Forum/viewtopic.php?f=36&t=562698
\end{solution}



\begin{solution}[by \href{https://artofproblemsolving.com/community/user/29428}{pco}]
	\begin{tcolorbox}2) Find all functions $f:\mathbb{R}\to\mathbb{R}$ such that 
\[\left (  f(x+y)\right )^2=f(x)f(x+2y)+yf(y) \quad\forall x,y\in\mathbb{R}\]\end{tcolorbox}
Let $P(x,y)$ be the assertion $f(x+y)^2=f(x)f(x+2y)+yf(y)$
Let $a=f(0)$

$P(x,-x)$ $\implies$ $a^2=f(x)f(-x)-xf(-x)$
$P(-x,x)$ $\implies$ $a^2=f(-x)f(x)+xf(x)$
Subtracting, we get $f(-x)=-f(x)$ $\forall x\ne 0$

Suppose now $a\ne 0$ so that $f(-a)=-f(a)$ :
$P(a,-a)$ $\implies$ $f(a)^2-af(a)+a^2=0$ which is impossible since $LHS >0$
So $a=0$ and $f(-x)=-f(x)$ $\forall x$

$P(-x,x)$ $\implies$ $f(x)(f(x)-x)=0$ and so $\forall x$ either $f(x)=0$, either $f(x)=x$

Suppose now $\exists u,v\ne 0$ such that $f(u)=0$ and $f(v)=v$ : 
$P(u,v)$ $\implies$ $f(u+v)^2=v^2\ne 0$ and so $f(u+v)=u+v$ and $u=-2v$
Let then $w\notin \{0,-2v,-\frac u2\}$ :
If $f(w)=0$, then $P(w,v)$ $\implies$ $f(w+v)^2=v^2\ne 0$ and so $f(w+v)=w+v$ and $w=-2v$, impossible
If $f(w)=w$, then $P(u,w)$ $\implies$ $f(u+w)^2=w^2\ne 0$ and so $f(u+w)=u+w$ and $u=-2w$, impossible
So no such $u,v$

So :
either $\boxed{f(x)=0}$ $\forall x$, which indeed is a solution
either $\boxed{f(x)=x}$ $\forall x$, which indeed is a solution
\end{solution}
*******************************************************************************
-------------------------------------------------------------------------------

\begin{problem}[Posted by \href{https://artofproblemsolving.com/community/user/125553}{lehungvietbao}]
	1) Find all  functions $f:[1;+\infty)\to[1;+\infty)$ such that 
\[f(xf(y))=yf(x) \quad\forall x,y\in [1;+\infty)\]

2) Find all functions $f:\mathbb{R}\to\mathbb{R}$ such that 
\[ \left | f(x)-f(q) \right |\leq 5(x-q) \quad  \forall x\in\mathbb{R}, \forall q\in\mathbb{Q} \]
	\flushright \href{https://artofproblemsolving.com/community/c6h562688}{(Link to AoPS)}
\end{problem}



\begin{solution}[by \href{https://artofproblemsolving.com/community/user/89198}{chaotic_iak}]
	2: Taking $x = 0, q = 1$ gives $|f(0) - f(1)| \le -5$, impossible.

Do you mean $|f(x) - f(q)| \le 5|x-q|$?
\end{solution}



\begin{solution}[by \href{https://artofproblemsolving.com/community/user/29428}{pco}]
	\begin{tcolorbox}1) Find all  functions $f:[1;+\infty)\to[1;+\infty)$ such that 
\[f(xf(y))=yf(x) \quad\forall x,y\in [1;+\infty)\]\end{tcolorbox}
Let $P(x,y)$ be the assertion $f(xf(y))=yf(x)$

If $f(a)=f(b)$, comparaison of $P(1,a)$ and $P(1,b)$ implies $a=b$ and so $f(x)$ is injective.

$P(x,1)$ $\implies$ $f(xf(1))=f(x)$ and so, since injective, $f(1)=1$

$P(1,x)$ $\implies$ $f(f(x))=x$
$P(x,f(y))$ $\implies$ $f(xy)=f(x)f(y)$

Let then $g(x)$ from $[0,+\infty)\to [0,+\infty)$ defined as $g(x)=\ln(f(e^x))$ 
$f(xy)=f(x)f(y)$ becomes $g(x+y)=g(x)+g(y)$ $\forall x,y\ge 0$
And since $g(x)\ge 0$, we classicaly get $g(x)=ax$ for some $a\ge 0$ and so $f(x)=x^a$

Plugging this back in original equation, we get $a=1$ and so $\boxed{f(x)=x}$ $\forall x$, which indeed is a solution.
\end{solution}



\begin{solution}[by \href{https://artofproblemsolving.com/community/user/125553}{lehungvietbao}]
	2) Find all functions $f:\mathbb{R}\to\mathbb{R}$ such that 
\[ \left | f(x)-f(q) \right |\leq 5(x-q)^2 \quad  \forall x\in\mathbb{R}, \forall q\in\mathbb{Q} \]

Sorry for my mistake . I fixed my mistake
\end{solution}



\begin{solution}[by \href{https://artofproblemsolving.com/community/user/29428}{pco}]
	\begin{tcolorbox}2) Find all functions $f:\mathbb{R}\to\mathbb{R}$ such that 
\[ \left | f(x)-f(q) \right |\leq 5(x-q)^2 \quad  \forall x\in\mathbb{R}, \forall q\in\mathbb{Q} \]\end{tcolorbox}
Let $n\in\mathbb N$
Let $a,b\in\mathbb Q$ and numbers $a_k=a+k\frac{b-a}n\in\mathbb Q$

$f(b)-f(a)=\sum_{k=0}^{n-1}(f(a_{k+1})-f(a_k))$

So $|f(b)-f(a)|\le \sum_{k=0}^{n-1}|f(a_{k+1})-f(a_k)|$ $\le 5\frac{(b-a)^2}n$

Setting $n\to+\infty$, we get $f(q)=c$ constant $\forall q\in\mathbb Q$

And so $|f(x)-c|\le 5(x-q)^2$ $\forall x\in\mathbb R,\forall q\in\mathbb Q$
Setting in above equation $q$ as near of $x$ as we want, we get $\boxed{f(x)=c}$ constant $\forall x$
\end{solution}
*******************************************************************************
-------------------------------------------------------------------------------

\begin{problem}[Posted by \href{https://artofproblemsolving.com/community/user/43631}{mathwizarddude}]
	Find all functions $f:\mathbb{R}^+\to \mathbb{R}^+$ such that for all $x,y\in\mathbb{R}^+$\[f(x)f(yf(x))=f(x+y)\]

A start: set y=0 to get $f(x)f(0)=f(x)$. So $f(0)=1$ unless $f$ is identically zero.
	\flushright \href{https://artofproblemsolving.com/community/c6h562698}{(Link to AoPS)}
\end{problem}



\begin{solution}[by \href{https://artofproblemsolving.com/community/user/29428}{pco}]
	\begin{tcolorbox}Find all functions $f:\mathbb{R}^+\to \mathbb{R}^+$ such that for all $x,y\in\mathbb{R}^+$\[f(x)f(yf(x))=f(x+y)\]

A start: set y=0 to get $f(x)f(0)=f(x)$. So $f(0)=1$ unless $f$ is identically zero.\end{tcolorbox}
You cant set $y=0$ since we are speaking about positive reals (and so not including $0$)
\end{solution}



\begin{solution}[by \href{https://artofproblemsolving.com/community/user/29428}{pco}]
	\begin{tcolorbox}Find all functions $f:\mathbb{R}^+\to \mathbb{R}^+$ such that for all $x,y\in\mathbb{R}^+$\[f(x)f(yf(x))=f(x+y)\]\end{tcolorbox}
Let $P(x,y)$ be the assertion $f(x)f(yf(x))=f(x+y)$ true $\forall x,y>0$
Let $a=f(1)$

If $f(u)>1$ for some $u>0$, then $P(u,\frac u{f(u)-1})$ $\implies$ $f(u)=1$ and so contradiction.

So $f(x)\le 1$ $\forall x$ and so $P(x,y)$ implies $f(x+y)\le f(x)$ and $f(x)$ is non increasing.

If $f(u)=1$ for some $u>0$, then $P(u,x)$ $\implies$ $f(x)=f(x+u)$ and so, since non increasing, $\boxed{f(x)=1}$ constant which indeed is a solution.

So, from there, let us consider $f(x)<1$ $\forall x$ and $f(x)$decreasing (and so injective).

$P(1,\frac xa)$ $\implies$ $af(x)=f(\frac xa+1)$ 
$P(x,\frac 1{f(x)})$ $\implies$ $af(x)=f(\frac 1{f(x)}+x)$ 

So, since injective, $\frac xa+1=\frac 1{f(x)}+x$ and so $f(x)=\frac 1{1+(\frac 1a-1)x}$

So $\boxed{f(x)=\frac 1{cx+1}}$ $\forall x$, which indeed is a solution, whatever is $c\ge 0$
\end{solution}
*******************************************************************************
-------------------------------------------------------------------------------

\begin{problem}[Posted by \href{https://artofproblemsolving.com/community/user/68025}{Pirkuliyev Rovsen}]
	Find all  functions $f:Z \rightarrow Z$  such that $(f(a+b))^3-(f(a))^3-(f(b))^3=3f(a)f(b)f(a+b)$.
	\flushright \href{https://artofproblemsolving.com/community/c6h562704}{(Link to AoPS)}
\end{problem}



\begin{solution}[by \href{https://artofproblemsolving.com/community/user/29428}{pco}]
	\begin{tcolorbox}Find all  functions $f:Z \rightarrow Z$  such that $(f(a+b))^3-(f(a))^3-(f(b))^3=3f(a)f(b)f(a+b)$.\end{tcolorbox}
Let $P(x,y)$ be the assertion $f(x+y)^3=f(x)^3+f(y)^3+3f(x)f(y)f(x+y)$
Let $a=f(1)$

$P(0,0)$ $\implies$ $f(0)=0$
$P(x,-x)$ $\implies$ $f(-x)=-f(x)$

If $a=0$, simple induction using $P(n,1)$ implies $f(n)=0$ $\forall n\ge 0$ and so $\boxed{f(x)=0}$ $\forall x\in\mathbb Z$, which indeed is a solution.

So let us from now consider $a\ne 0$
Since then $f(x)$ solution implies $\frac 1af(x)$ solution too, Wlog consider $a=1$

$P(1,1)$ $\implies$ $(f(2)-2)(f(2)+1)^2=0$ and so $f(2)\in\{-1,2\}$

1) If $f(2)=2$
==========
Let $n\ge 2$. Suppose $f(k)=k$ $\forall k\in[0,n]$
$P(n,1)$ $\implies$ $(f(n+1)-(n+1))(f(n+1)^2+(n+1)f(n+1)+(n^2-n+1))=0$
If $n>1$ the quadratic $x^2+(n+1)x+(n^2-n+1)$ is $>0$ and so $f(n+1)=n+1$

So $f(n)=n$ $\forall n\ge 0$ and so $f(n)=n$ $\forall n$
Hence the solution $\boxed{f(x)=ax}$ $\forall x$, which indeed is a solution, whatever is $a\in\mathbb Z$

2) If $f(2)=-1$
===========
$P(2,1)$ $\implies$ $f(3)=0$
$P(n,3)$ $\implies$ $f(n+3)=f(n)$ and so :

$\boxed{f(3n)=0\text{ and }f(3n+1)=a\text{ and }f(3n+2)=-a}$ $\forall n$ which indeed is a solution, whatever is $a\in\mathbb Z$
\end{solution}
*******************************************************************************
-------------------------------------------------------------------------------

\begin{problem}[Posted by \href{https://artofproblemsolving.com/community/user/125553}{lehungvietbao}]
	1) Generalize for IMO 1986
Find all functions $f:\mathbb{R}_{\geq 0}\to\mathbb{R}{\geq 0}$ such that 
\[f(xf(y))f(y)=f(x+y) \quad  \forall x,y\geq 0\]
	\flushright \href{https://artofproblemsolving.com/community/c6h562718}{(Link to AoPS)}
\end{problem}



\begin{solution}[by \href{https://artofproblemsolving.com/community/user/29428}{pco}]
	\begin{tcolorbox}1) Generalize for IMO 1986
Find all functions $f:\mathbb{R}_{\geq 0}\to\mathbb{R}{\geq 0}$ such that 
\[f(xf(y)f(y)=f(x+y) \quad  \forall x,y\geq 0\]\end{tcolorbox}
Missing parenthesis. I suppose we must read $f(xf(y))f(y)=f(x+y)$

$\boxed {f(x)=0}$ $\forall x$ is a solution. So let us from now look only for non allzero solutions.

Let $P(x,y)$ be the assertion $f(y)f(xf(y))=f(x+y)$ true $\forall x,y\ge 0$
Let $a=f(1)$
Let $t$ such that $f(t)\ne 0$
$P(0,t)$ $\implies$ $f(0)=1$

1) If $\exists v$ such that $f(v)=0$
=======================
$P(x,v)$ $\implies$ $f(x+v)=0$ and so $f(x)=0$ $\forall x\ge v$
Let then $u=\inf f^{-1}(\{0\})$ so that $f(x)>0$ $\forall x<u$ and $f(x)=0$ $\forall x>u$

1.1) If $u=0$
-------------
We got the solution $\boxed{f(0)=1\text{  and  }f(x)=0\text{   }\forall x>0}$ which indeed is a solution

1.2) If $u>0$
------------
Let $0<x<u$ : $P(u-x+\frac 1n,x)$ $\implies$ $f((u-x+\frac 1n)f(x))=0$ and so $f(x)\ge \frac u{u-x+\frac 1n}$
Setting $n\to+\infty$, we get $f(x)\ge\frac u{u-x}>1$ $\forall x<u$

Then $P(\frac x{f(x)-1},x)$ $\implies$ $f(\frac {xf(x)}{f(x)-1})=0$ and so $\frac {xf(x)}{f(x)-1}\ge u$ and so $f(x)\le \frac u{u-x}$

And so $f(x)=\frac u{u-x}>1$ $\forall x<u$

$P(\frac u2,\frac u2)$ $\implies$ $f(u)=0$

Hence the solution $\boxed{f(x)=\frac u{u-x}\text{   }\forall x\in[0,u)\text{  and  }f(x)=0\text{  }\forall x\ge u}$ which indeed is a solution.

2) If $f(x)\ne 0$ $\forall x$
================
If $f(u)>1$ for some $u>0$, then $P(\frac u{f(u)-1},u)$ $\implies$ $f(u)=1$ and so contradiction.

So $f(x)\le 1$ $\forall x$ and so $P(x,y)$ implies $f(x+y)\le f(y)$ and $f(x)$ is non increasing.

If $f(u)=1$ for some $u>0$, then $P(x,u)$ $\implies$ $f(x)=f(x+u)$ and so, since non increasing, $\boxed{f(x)=1}$ constant which indeed is a solution.

So, from there, let us consider $0<f(x)<1$ $\forall x$ and $f(x)$decreasing (and so injective).

$P(\frac xa,1)$ $\implies$ $af(x)=f(\frac xa+1)$ 
$P(\frac 1{f(x)},x)$ $\implies$ $af(x)=f(\frac 1{f(x)}+x)$ 

So, since injective, $\frac xa+1=\frac 1{f(x)}+x$ and so $f(x)=\frac 1{1+(\frac 1a-1)x}$

So $\boxed{f(x)=\frac 1{cx+1}}$ $\forall x$, which indeed is a solution, whatever is $c\ge 0$
\end{solution}
*******************************************************************************
-------------------------------------------------------------------------------

\begin{problem}[Posted by \href{https://artofproblemsolving.com/community/user/125553}{lehungvietbao}]
	1) Find all continus functions $f:[0;1]\to\mathbb{R}$ such that
\[f(x)\geq 2xf(x^2)\quad \forall x\in [0;1]\]

2) Find all continus functions $f:\mathbb{R}\to\mathbb{R}$ such that
\[f(x)+f(x^2)=2 \quad \forall x\in\mathbb{R}\]
	\flushright \href{https://artofproblemsolving.com/community/c6h562724}{(Link to AoPS)}
\end{problem}



\begin{solution}[by \href{https://artofproblemsolving.com/community/user/29428}{pco}]
	\begin{tcolorbox}2) Find all continus functions $f:\mathbb{R}\to\mathbb{R}$ such that
\[f(x)+f(x^2)=2 \quad \forall x\in\mathbb{R}\]\end{tcolorbox}
Let $g(x)=f(x)-1$ and equation is $g(x)=-g(x^2)$ and so $g(x)=g(x^4)$ 
Note that $g(x)$ is even and that $g(x)=-g(x^2)$ implies $g(0)=g(1)=0$

Let then $x\in (0,1)$ : Setting $n\to+\infty$ in $g(x)=g(x^{4^n})$ and using continuity, we get $g(x)=g(0)=0$
Let then $x>1$:  Setting $n\to+\infty$ in $g(x)=g(x^{4^{-n}})$ and using continuity, we get $g(x)=g(1)=0$
So $g(x)=0$ $\forall x\ge 0$
So, since even, $g(x)=0$ $\forall x$

And so $\boxed{f(x)=1}$ $\forall x$, which indeed is a solution.
\end{solution}



\begin{solution}[by \href{https://artofproblemsolving.com/community/user/167245}{TheChainheartMachine}]
	My solution to the first uses a bit of calculus.
Continuity on a compact interval implies integrability, so we can integrate both sides over $[0, 1]$. We then get $\int_0^1 \! f(x) \, dx \ge \int_0^1 \! 2xf(x^2) \, dx$. However, making the sub $u = x^2$ turns the right-hand side into $\int_0^1 \! f(u) \, du = \int_0^1 \! f(x) \, dx$. We are then forced to have that $f(x) = 2xf(x^2)$ on $[0, 1]$. Now we can finish up. Solving for the endpoints is easy; we have $f(0) = 0f(0) = 0$ and $f(1) = 2f(1)$ which gives $f(1) = 0$ as well. For the interior, we have $f(x) = 2xf(x^2) = 4x^3f(x^4) = \dots = 2^nx^{2^n-1}f(x^{2^n}) \dots$. Taking the limit of this sequence, together with continuity at $x = 0$ (though now that I think about it we only need that $f$ is bounded) ensures in fact that $f(x) = 0$.

Thus our only solution is $f(x) \equiv 0$ on $[0,1]$.
\end{solution}



\begin{solution}[by \href{https://artofproblemsolving.com/community/user/168716}{smy2012}]
	\begin{tcolorbox}[quote="lehungvietbao"]
Let then $x>1$:  Setting $n\to+\infty$ in $g(x)=g(x^{4^{-n}})$\end{tcolorbox}\end{tcolorbox}

Typo: $n\to+\infty$ in $g(x)=g(x^{-4^{n}})$
\end{solution}



\begin{solution}[by \href{https://artofproblemsolving.com/community/user/29428}{pco}]
	\begin{tcolorbox}[quote="pco"][quote="lehungvietbao"]
Let then $x>1$:  Setting $n\to+\infty$ in $g(x)=g(x^{4^{-n}})$\end{tcolorbox}\end{tcolorbox}

Typo: $n\to+\infty$ in $g(x)=g(x^{-4^{n}})$\end{tcolorbox}
No : $n\to+\infty$ in $g(x)=g(x^{4^{-n}})$
\end{solution}
*******************************************************************************
-------------------------------------------------------------------------------

\begin{problem}[Posted by \href{https://artofproblemsolving.com/community/user/125553}{lehungvietbao}]
	1) Find all  functions $f:\mathbb{R}\to\mathbb{R}$ such that
\[f(f(x+y))=f(x+y)+f(x)f(y)-xy  \quad \forall x,y\in\mathbb{R}\]

2) Find all functions $f:\mathbb{R}\to\mathbb{R}$ such that
\[f(f(x-y))=f(x)-f(y)+f(x)f(y)-xy  \quad \forall x,y\in\mathbb{R}\]

3) Find all functions $f:\mathbb{R}\to\mathbb{R}$ such that
\[f(f(x)-y)=f(x)-f(y)+f(x)f(y)-xy  \quad \forall x,y\in\mathbb{R}\]
	\flushright \href{https://artofproblemsolving.com/community/c6h562841}{(Link to AoPS)}
\end{problem}



\begin{solution}[by \href{https://artofproblemsolving.com/community/user/29428}{pco}]
	\begin{tcolorbox}1) Find all  functions $f:\mathbb{R}\to\mathbb{R}$ such that
\[f(f(x+y))=f(x+y)+f(x)f(y)-xy  \quad \forall x,y\in\mathbb{R}\]\end{tcolorbox}
Let $P(x,y)$ be the assertion $f(f(x+y))=f(x+y)+f(x)f(y)-xy$
Let $a=f(0)$
$f(x)=0$ $\forall x$ is not a solution. So let $t$ such that $f(t)\ne 0$

Subtracting $P(x+y,0)$ from $P(x,y)$, we get new assertion $Q(x,y)$ : $af(x+y)=f(x)f(y)-xy$

If $a=0$, $Q(x,t)$ $\implies$ $f(x)=x\frac t{f(t)}$
Plugging this back in original equation, we get the solution $\boxed{f(x)=x}$ $\forall x$

If $a\ne 0$ :
$Q(1,1)$ $\implies$ $af(2)=f(1)^2-1$
$Q(x,2)$ $\implies$ $af(x+2)=f(x)f(2)-2x$ $=\frac 1af(x)f(1)^2-\frac 1af(x)-2x$

$Q(x,1)$ : $af(x+1)=f(x)f(1)-x$
$Q(x+1,1)$ $\implies$ $af(x+2)=f(x+1)f(1)-x-1$ $=\frac 1af(x)f(y)f(1)-\frac 1axf(1)-x-1$

Equating the two expressions of $af(x+2)$, we get $f(x)=x(f(1)-a)+a$ with $a\ne 0$
Plugging this back in original equation we find no new solution.
\end{solution}



\begin{solution}[by \href{https://artofproblemsolving.com/community/user/29428}{pco}]
	\begin{tcolorbox}2) Find all functions $f:\mathbb{R}\to\mathbb{R}$ such that
\[f(f(x-y))=f(x)-f(y)+f(x)f(y)-xy  \quad \forall x,y\in\mathbb{R}\]\end{tcolorbox}
Let $P(x,y)$ be the assertion $f(f(x-y))=f(x)-f(y)+f(x)f(y)-xy$
Let $a=f(0)$

$P(x,x)$ $\implies$ $f(x)^2=x^2+f(a)$ and so $f(x)=e(x)\sqrt{x^2+f(a)}$ where $e(x)=\pm 1$
Plugging this in $P(x,0)$, we get $e(f(x))\sqrt{x^2+2f(a)}=(a+1)e(x)\sqrt{x^2+f(a)}-a$ and so $f(a)=a=0$

And $\forall x$, either $f(x)=x$, either $f(x)=-x$

Suppose now that $f(u)=-u$ for some $u$
$P(u,0)$ $\implies$ $f(-u)=-u$
$P(0,u)$ $\implies$ $u=0$

And so $\boxed{f(x)=x}$ $\forall x$ which indeed is a solution.
\end{solution}



\begin{solution}[by \href{https://artofproblemsolving.com/community/user/29428}{pco}]
	\begin{tcolorbox}3) Find all functions $f:\mathbb{R}\to\mathbb{R}$ such that
\[f(f(x)-y)=f(x)-f(y)+f(x)f(y)-xy  \quad \forall x,y\in\mathbb{R}\]\end{tcolorbox}
Let $P(x,y)$ be the assertion $f(f(x)-y)=f(x)-f(y)+f(x)f(y)-xy$
Let $a=f(0)$

1) If $f(x)\ne 0$ $\forall x$
$P(0,0)$ $\implies$ $f(a)=a^2$
$P(0,a)$ $\implies$ $a^2(a-1)=0$ and so $a=1$ (since $a=f(0)\ne 0$)
$P(-2,0)$ $\implies$ $f(f(-2))=2f(-2)-1$
$P(-2,f(-2))$ $\implies$ $f(-2)=0$, contradiction

2) If $f(u)=0$ for some $u$
$P(u,0)$ $\implies$ $a=0$
$P(x,0)$ $\implies$ $f(f(x))=f(x)$
$P(x,f(x))$ $\implies$ $f(x)(f(x)-x)=0$ and so $\forall x$ either $f(x)=0$, either $f(x)=x$

Suppose now that $\exists u,v\ne 0$ such that $f(u)=u$ and $f(v)=0$
$P(v,u)$ $\implies$ $f(-u)=-u-uv$ and so, since $uv\ne 0$, $v=-1$ and $f(x)=x$ $\forall x\ne -1$
$P(2,-1)$ $\implies$ $f(3)=4$, contradiction

So no such $u,v$ and  :
Either $\boxed{f(x)=x}$ $\forall x$, which indeed is a solution
Either $f(x)=0$ $\forall x$, which is not  solution.
\end{solution}



\begin{solution}[by \href{https://artofproblemsolving.com/community/user/179088}{Panoz93}]
	\begin{tcolorbox}1) Find all  functions $f:\mathbb{R}\to\mathbb{R}$ such that
\[f(f(x+y))=f(x+y)+f(x)f(y)-xy  \quad \forall x,y\in\mathbb{R}\]

\end{tcolorbox}
Let $f(0)=a$ then for $x=y=0$ we have $f(a)=a+a^{2}$
Also set $x=a,y=-a$ to get $f(a)f(-a)=0\Leftrightarrow (a+a^{2})f(-a)=0$
We consider the cases
$i)$------- $f(-a)=0$
For $x=0,y=-a$ in the initial equation we conclude $f(0)=0$
 
$ii)$-------$a=-1$
Setting  $x=0$ in the given equation we get $f(f(y))=f(y)-f(y)=0$ hence $f$ is constant , $f(x)=0$ which is not a solution 

So we must have $f(0)=0$
Then for $y=0$ in the initial we get $f(f(x))=f(x)$
The given equation reduces to  $f(x)f(y)=xy$ so  $f(x)=\frac{x}{f(1)}$
Substituting this value of $f$ in the initial we have $f(1)=1$ .. and therefore $f(x)=x$
\end{solution}



\begin{solution}[by \href{https://artofproblemsolving.com/community/user/29428}{pco}]
	\begin{tcolorbox}...
Setting  $x=0$ in the given equation we get $f(f(y))=f(y)-f(y)=0$ hence $f$ is constant
...
\end{tcolorbox}
Error : $f(f(x))=0$ does not imply $f(x)$ is constant. Look for example at $f(x)=|x|-x$
\end{solution}



\begin{solution}[by \href{https://artofproblemsolving.com/community/user/179088}{Panoz93}]
	:oops: .. yes you 're right . I'll try to correct it ;)
\end{solution}



\begin{solution}[by \href{https://artofproblemsolving.com/community/user/179088}{Panoz93}]
	Ok I think I found it ..
In the second case we have 
$i)$------- a=-1
Setting $x=y=0$ in the initial we get  $f(-1)=f(0) +f(0)^2 =-1+1=0$
so we also set $x=-1,y=0$ to get $f(f(-1))=f(-1)+f(-1)f(0)-0\Leftrightarrow f(0)=0 $
A contradiction !
\end{solution}



\begin{solution}[by \href{https://artofproblemsolving.com/community/user/179088}{Panoz93}]
	\begin{tcolorbox}

3) Find all functions $f:\mathbb{R}\to\mathbb{R}$ such that
\[f(f(x)-y)=f(x)-f(y)+f(x)f(y)-xy  \quad \forall x,y\in\mathbb{R}\]\end{tcolorbox}
We will first prove $f(0)=0$
Let $f(0)=a$
Setting $x=y=0$ gives $f(a)=a^{2}$ $(1)$
Also $x=a,y=0$ gives $f(a^{2})=a^{2}-a+a^{3}$
Now we set $x=0 ,y=a$ which enables us to conclude $a=a-a^{2}+a^{3} $ (using $(1)$)
So $f(0)\in {1,-1,0}$
$i)$Suppose first that $f(0)=1$
then for $x=y=1$ using $(1)$ (f(1)=1) we get  $1=0$ , contradiction .
$ii)$Now assume $f(0)=-1$ 
Setting $x=-1,y=1$ in the initial equation gives ( using $(1)$ and $(2)$ we get $f(-1)=f(1)=1$)
$-1=2 $ , a contradiction again 
Therefore, we conlcude $f(0)=0$
$x=0$ gives $f(-y)=-f(y)$------ $f$ is odd 
We also set $x=y$ and get $f( f(x)-x)=f(x)^{2}-x^{2}$
Now set $x=-x$ in this equation to get (since odd)  $f(f(-x)+x)=f(x)^{2}-x^{2}$
Hence we get $f(f(x)-x)=f(-f(x)+x)$ which reduces (since injective ) to $2f(x)=2x\Leftrightarrow f(x)=x$
this equation indeed satisfies the initial
\end{solution}
*******************************************************************************
-------------------------------------------------------------------------------

\begin{problem}[Posted by \href{https://artofproblemsolving.com/community/user/125553}{lehungvietbao}]
	1) Find all functions $f:\mathbb{R}\to\mathbb{R}$ such that
\[f(x+y)=f(x)e^{f(y)-1}   \quad \forall x,y\in\mathbb{R}\]

2) Find all functions $f:\mathbb{R}\to\mathbb{R}$ which satisfy the following the conditions :
a) $f(f(x)+y)=xf(y)+f\left(f(x)+f(y)\right)  \quad \forall x,y\in\mathbb{R}$
b) $f$ has a fixed point  ( $x_0$ is called fixed point if $f(x_{0})=x_{0}$ )
	\flushright \href{https://artofproblemsolving.com/community/c6h562844}{(Link to AoPS)}
\end{problem}



\begin{solution}[by \href{https://artofproblemsolving.com/community/user/89198}{chaotic_iak}]
	Problem 1

Let $P(x,y)$ be the statement $f(x+y) = f(x) e^{f(y)-1}$.

$P(x,0) \implies f(x) = f(x)e^{f(0)-1}$. Either $f(x) = 0$ for all $x$ which is a solution, or $f(0) = 1$.
$P(0,x) \implies f(x) = e^{f(x)-1}$, or $f(x) - \ln f(x) = 1$. Also note that $f(x) > 0$ for all $x$. Since $h(x) = x - \ln x$ is decreasing in $(0,1]$ and increasing in $[1, \infty)$, and $h(1) = 1$, then the minimum value of $h$ is $1$ only achieved at $x=1$, and so since minimum is achieved, we must have $f(x) = 1$ for all $x$.

So $f(x) = 0$ for all $x$ or $f(x) = 1$ for all $x$ are all the solutions, which indeed fit the statement.
\end{solution}



\begin{solution}[by \href{https://artofproblemsolving.com/community/user/29428}{pco}]
	\begin{tcolorbox}2) Find all functions $f:\mathbb{R}\to\mathbb{R}$ which satisfy the following the conditions :
a) $f(f(x)+y)=xf(y)+f\left(f(x)+f(y)\right)  \quad \forall x,y\in\mathbb{R}$
b) $f$ has a fixed point  ( $x_0$ is called fixed point if $f(x_{0})=x_{0}$ )\end{tcolorbox}
$\boxed{f(x)=0}$ $\forall x$ is a solution. So let us from now look only for non allzero solutions.

Let $P(x,y)$ be the assertion $f(f(x)+y)=xf(y)+f(f(x)+f(y))$
Let $x_0$ such that $f(x_0)=x_0$
Let $t$ such that $f(t)\ne 0$

If $f(a)=f(b)$, then comparaison of $P(a,t)$ and $P(b,t)$ implies $a=b$ and so $f(x)$ is injective.
$P(1,x_0)$ $\implies$ $x_0=0$
$P(0,x)$ $\implies$ $f(x)=f(f(x))$ and so, since injective, $f(x)=x$ $\forall x$, which is not a solution.

Hence the unique solution $f(x)=0$ $\forall x$ previously given.
\end{solution}
*******************************************************************************
-------------------------------------------------------------------------------

\begin{problem}[Posted by \href{https://artofproblemsolving.com/community/user/125553}{lehungvietbao}]
	1) Find all functions $f:\mathbb{R}\to\mathbb{R}^+$ such that
\[(f(x))^2=1+xf(x+1) \quad \forall x\in\mathbb{R}\]

2) Find all functions $f:(0;1)\to\mathbb{R}$ such that
\[f(xy)=xf(y)+yf(x) \quad \forall x,y\in (0;1)\]
	\flushright \href{https://artofproblemsolving.com/community/c6h562846}{(Link to AoPS)}
\end{problem}



\begin{solution}[by \href{https://artofproblemsolving.com/community/user/29428}{pco}]
	\begin{tcolorbox}2) Find all functions $f:(0;1)\to\mathbb{R}$ such that
\[f(xy)=xf(y)+yf(x) \quad \forall x,y\in (0;1)\]\end{tcolorbox}
Let $g(x)$ from $(-\infty,0)\to\mathbb R$ defined as $g(x)=e^{-x}f(e^x)$
Equation becomes $g(x+y)=g(x)+g(y)$ $\forall x,y<0$

Hence the solutions $\boxed{f(x)=x g(\ln x)}$ $\forall x\in(0,1)$ which indeed is a solution whatever is the fonction $g(x)$, solution of Cauchy equation.
\end{solution}



\begin{solution}[by \href{https://artofproblemsolving.com/community/user/29428}{pco}]
	\begin{tcolorbox}1) Find all functions $f:\mathbb{R}\to\mathbb{R}^+$ such that
\[(f(x))^2=1+xf(x+1) \quad \forall x\in\mathbb{R}\]\end{tcolorbox}
There is a lack of precision about the notation $\mathbb R^+$ (which should be avoided, according to me,  with preference for $\mathbb R_{>0}$ or $\mathbb R_{\ge 0}$ or a full sentence)

If $0\notin\mathbb R^+$ :
$x=0$ $\implies$ $f(0)=1$
$x=-1$ $\implies$ $f(-1)=0$
So no solution.


If $0\in\mathbb R^+$ :
$f(-\frac 52)\ge 0$

$f(-\frac 32)=-\frac 25(f(-\frac 52)^2-1)\le \frac 25$

$f(-\frac 12)=-\frac 23(f(-\frac 32)^2-1)\ge \frac{42}{75}$

$f(\frac 12)=-2(f(-\frac 12)^2-1)\le  1.3728<1.4$

$f(\frac 32)=2(f(\frac 12)^2-1)< 1.92$

$f(\frac 52)=\frac 23(f(\frac 32)^2-1)< \sim 1.7909333 < 1.8$

$f(\frac 72)=\frac 25(f(\frac 52)^2-1)< 0.896< 1$

$f(\frac 92)=\frac 27(f(\frac 72)^2-1)<0$

So no solution.

Note that without the constraint $f(x)>0$ or $f(x)\ge 0$, we have the trivial solution $f(x)=x+1$
\end{solution}
*******************************************************************************
-------------------------------------------------------------------------------

\begin{problem}[Posted by \href{https://artofproblemsolving.com/community/user/125553}{lehungvietbao}]
	1) Find all functions $f:\mathbb{R}\to\mathbb{R}$ which satisfiy the following the conditions

a) $f(x)+f(y) \neq 0 \quad \forall x,y \in \mathbb{R}$

b) $\frac{f(x)-f(x-y)}{f(x)+f(x+y)}+\frac{f(x)-f(x+y)}{f(x)+f(x-y)}=0 \quad \forall x,y \in \mathbb{R}$


2) Find all functions $f:\mathbb{R}^+\to\mathbb{R}^+$  such that
\[f(f(x))=6x-f(x) \quad \forall x,y \in \mathbb{R}^+\]
	\flushright \href{https://artofproblemsolving.com/community/c6h563026}{(Link to AoPS)}
\end{problem}



\begin{solution}[by \href{https://artofproblemsolving.com/community/user/141363}{alibez}]
	see here : 

http://www.artofproblemsolving.com/Forum/viewtopic.php?t=336063
\end{solution}



\begin{solution}[by \href{https://artofproblemsolving.com/community/user/29428}{pco}]
	\begin{tcolorbox}1) Find all functions $f:\mathbb{R}\to\mathbb{R}$ which satisfiy the following the conditions

a) $f(x)+f(y) \neq 0 \quad \forall x,y \in \mathbb{R}$

b) $\frac{f(x)-f(x-y)}{f(x)+f(x+y)}+\frac{f(x)-f(x+y)}{f(x)+f(x-y)}=0 \quad \forall x,y \in \mathbb{R}$
\end{tcolorbox}
$b)$ $\implies$ $f(x+y)^2+f(x-y)^2=2f(x)^2$ $\implies$ $f(x+ny)^2=(f(x+y)^2-f(x)^2)n+f(x)^2$ $\forall x,y\in\mathbb R$, $\forall n\in\mathbb Z$

So $f(x+y)^2=f(x)^2$ (else, choosing $n\to\infty$ with appropriate sign would imply $f(x+ny)^2<0$)

$a)$ implies then $\boxed{f(x)=c}$ $\forall x$, which indeed is a solution, whatever is $c\ne 0$
\end{solution}



\begin{solution}[by \href{https://artofproblemsolving.com/community/user/192463}{arkanm}]
	\begin{tcolorbox}2) Find all functions $f:\mathbb{R}^+\to\mathbb{R}^+$  such that
\[f(f(x))=6x-f(x) \quad \forall x,y \in \mathbb{R}^+\]\end{tcolorbox}
By iterating the function we find that
$f^2(x)=6x-f(x)\ge 0\Leftrightarrow f(x)\le 6x$
$f^3(x)=7f(x)-6x\ge 0\Leftrightarrow f(x)\ge \tfrac{6}{7}x$
$f^4(x)=42x-13f(x)\ge 0\Leftrightarrow f(x)\le \tfrac{42}{13}x$
$f^5(x)=55f(x)-78x\ge 0\Leftrightarrow f(x)\ge \tfrac{78}{55}x$
$f^6(x)=330x-133f(x)\ge 0\Leftrightarrow f(x)\le \tfrac{330}{133}x$
$f^7(x)=463f(x)-798x\ge 0\Leftrightarrow f(x)\ge \tfrac{798}{463}x$
$f^8(x)=2778x-1261f(x)\ge 0\Leftrightarrow f(x)\le \tfrac{2778}{1261}x$

The sequence of absolute values of the coefficients of $f(x)$ in these iterated functions is $1, 7, 13, 55, 133, 463, 1261, 4039, 11605, 35839,..., F_n,...$, where $F_n$ is the $n$th generalized Fibonacci number. On the other hand, the absolute values of the coefficients of $x$ are $6, 6, 42, 78, 330, 798, 2778, 7566, 24234, 69630,...$, which satisfy the recurrence relation $a_n = a_{n-1} + 6a_{n-2}$, defined by $a_0=1$ and $a_1=0$. Hence,

$f^n(x)= \begin{cases}F^nf(x)-a_nx\quad \text{if n is odd}\\ \\a_nx-F^nf(x)\quad \text{if n is even}\end{cases}$

or

$f(x) \begin{cases}\ge \frac{a_n}{F_n}x\quad \text{if n is odd}\\ \\\le \frac{a_n}{F_n}x\quad \text{if n is even}\end{cases}$

We claim that $\lim_{n\rightarrow \infty}\left(\frac{a_n}{F_n}\right)=2$.

Withholding the proof of the claim, we see that $f(x)=2x$. Plugging this into the original equation we find that it works. Hence, the unique solution is $\boxed{f(x)=2x}$ $\forall x\in \mathbb{R}$.
\end{solution}



\begin{solution}[by \href{https://artofproblemsolving.com/community/user/89198}{chaotic_iak}]
	Problem 2

Define a sequence $a_0 = x_0, a_{n+1} = f(a_n)$ for some $x_0$. Then plugging $x = a_n$ to the equation for some $x_0, n$ gives $a_{n+2} + a_{n+1} - 6a_n = 0$, giving the closed formula $a_n = A2^n + B(-3)^n$ for some $A,B$.

Note that $A2^n + B(-3)^n > 0$, or $\left( \dfrac{-3}{2} \right)^n > \dfrac{-A}{B}$ if $B \neq 0$. But by taking odd and sufficiently large $n$ such that $\left( \dfrac{3}{2} \right)^n > \dfrac{A}{B}$, we have $\left( \dfrac{-3}{2} \right)^n = - \left( \dfrac{3}{2} \right)^n < \dfrac{-A}{B}$, contradiction. So $B = 0$, so $a_n = A2^n$. So $a_0 = x_0 = A, a_1 = 2A$. In other words, $f(x_0) = a_1 = 2A = 2x_0$, or $f(x) = 2x$ for all $x$, which satisfies the condition.
\end{solution}
*******************************************************************************
-------------------------------------------------------------------------------

\begin{problem}[Posted by \href{https://artofproblemsolving.com/community/user/125553}{lehungvietbao}]
	1) Find all functions $f:\mathbb{R}^+\to\mathbb{R}^+$  such that
\[xf(xf(y))=f(f(y)) \quad \forall x,y \in \mathbb{R}^+\]

2) Find all functions $f:\mathbb{R}^+\to\mathbb{R}^+$  such that
\[f(x+y)+f(x)f(y)=f(xy)+f(x)+f(y) \quad \forall x,y \in \mathbb{R}^+\]
	\flushright \href{https://artofproblemsolving.com/community/c6h563027}{(Link to AoPS)}
\end{problem}



\begin{solution}[by \href{https://artofproblemsolving.com/community/user/29428}{pco}]
	\begin{tcolorbox}1) Find all functions $f:\mathbb{R}^+\to\mathbb{R}^+$  such that
\[xf(xf(y))=f(f(y)) \quad \forall x,y \in \mathbb{R}^+\]\end{tcolorbox}
Let $P(x,y)$ be the assertion $xf(xf(y))=f(f(y))$, true $\forall x,y>0$
Let $a=f(1)>0$

$P(\frac xa,1)$ $\implies$ $f(x)=\frac {af(a)}x$

Hence the answer : $\boxed{f(x)=\frac cx}$ $\forall x>0$, which indeed is a solution, whatever is $c>0$
\end{solution}



\begin{solution}[by \href{https://artofproblemsolving.com/community/user/29428}{pco}]
	\begin{tcolorbox}2) Find all functions $f:\mathbb{R}^+\to\mathbb{R}^+$  such that
\[f(x+y)+f(x)f(y)=f(xy)+f(x)+f(y) \quad \forall x,y \in \mathbb{R}^+\]\end{tcolorbox}
Let $P(x,y)$ be the assertion $f(x+y)+f(x)f(y)=f(xy)+f(x)+f(y)$, true $\forall x,y>0$
Let $a=f(1)$

If $a=2$, then :
$P(x,1)$ $\implies$ $f(x+1)=2$ and so $f(x)=2$ $\forall x\ge 1$
$P(x,2)$ $\implies$ $f(x)=f(2x)$ and so $f(x)=f(2^nx)$ and so $\boxed{f(x)=2}$ $\forall x$, which indeed is a solution

If $a\ne 2$, then :
$P(x,1)$ $\implies$ $f(x+1)=(2-a)f(x)+a$
$P(x,y+1)$ $\implies$ $(2-a)f(x+y)+(2-a)f(x)f(y)=f(xy+x)+(1-a)f(x)+(2-a)f(y)$
$P(x,y)$ $\implies$ $(2-a)f(x+y)+(2-a)f(x)f(y)=(2-a)f(xy)+(2-a)f(x)+(2-a)f(y)$
And so $f(xy+x)=(2-a)f(xy)+f(x)$

Which may be written (replacing $y\to \frac yx$) : $f(x+y)=f(x)+(2-a)f(y)$ $\forall x,y>0$

Swapping $x,y$, we get $f(x)+(2-a)f(y)=(2-a)f(x)+f(y)$ and so $(a-1)(f(x)-f(y))=0$ and so :
Either $f(x)$ is constant, which  gives $f(x)=2$ $\forall x$, impossible in this part of the proof ($a\ne 2$)
Either $a=1$ and so $f(x+y)=f(x)+f(y)$ (and so is increasing)
So $f(x)=x$ $\forall x\in\mathbb Q^+$ and, since increasing, $\boxed{f(x)=x}$ $\forall x$, which indeed is a solution.
\end{solution}
*******************************************************************************
-------------------------------------------------------------------------------

\begin{problem}[Posted by \href{https://artofproblemsolving.com/community/user/141363}{alibez}]
	find all function such that $f:\mathbb{Q}^{+}\rightarrow \mathbb{Q}^{+}$ and :

$f(x)+f(\frac{1}{x})=1 \: \: \: \: \: \: \forall x\in \mathbb{Q}^{+} $ 

$f(2x+1)=\frac{f(x)}{2}  \: \: \: \: \: \: \forall x\in \mathbb{Q}^{+}$
	\flushright \href{https://artofproblemsolving.com/community/c6h563035}{(Link to AoPS)}
\end{problem}



\begin{solution}[by \href{https://artofproblemsolving.com/community/user/29428}{pco}]
	\begin{tcolorbox}find all function such that $f:\mathbb{Q}^{+}\rightarrow \mathbb{Q}^{+}$ and :

$f(x)+f(\frac{1}{x})=1 \: \: \: \: \: \: \forall x\in \mathbb{Q}^{+} $ 

$f(2x+1)=\frac{f(x)}{2}  \: \: \: \: \: \: \forall x\in \mathbb{Q}^{+}$\end{tcolorbox}
1) One solution exists : $f(x)=\frac 1{x+1}$
Trivial check

2) If a solution exists, it must be unique.
Setting $x=1$ in the first equation gives $f(1)=\frac 12$

Let $p\ne q\in\mathbb N$. Consider the following sequence :
$p_0=p$, $q_0=q$, $a_0=1$, $b_0=0$
If $p_n>q_n$ : $p_{n+1}=p_n-q_n$ and $q_{n+1}=2q_n$ and $a_{n+1}=2a_n$ and $b_{n+1}=2b_n$
If $p_n=q_n$ : $p_{n+1}=p_n$ and $q_{n+1}=q_n$ and $a_{n+1}=a_n$ and $b_{n+1}=b_n$
If $p_n<q_n$ : $p_{n+1}=q_n-p_n$ and $q_{n+1}=2p_n$ and $a_{n+1}=-2a_n$ and $b_{n+1}=2-2b_n$

Note that $f(\frac{p_n}{q_n})=a_nf(\frac pq)+b_n$

$p_{n+1}+q_{n+1}=p_n+q_n$ and $p_n,q_n>0$ and so :
Either the sequence ends in $p_n=q_n$
Either the sequence enters a loop.

If the sequence ends in $p_n=q_n$, we get $\frac 12=f(\frac{p_n}{q_n})=a_nf(\frac pq)+b_n$ and so $f(\frac pq)$ has at most one possible value.

If the sequence enters a loop $p_{m}=p_n$ and $q_m=q_n$, with $p_n\ne q_n$, we get $a_nf(\frac pq)+b_n=a_mf(\frac pq)+b_m$
And since each step where $p_n\ne q_n$ doubles $|a_n|$, we get $a_n\ne a_m$ and so $f(\frac pq)$ has at most one possible value.
Q.E.D.

3) hence the unique solution : $\boxed{f(x)=\frac 1{x+1}}$ $\forall x$
\end{solution}
*******************************************************************************
-------------------------------------------------------------------------------

\begin{problem}[Posted by \href{https://artofproblemsolving.com/community/user/68025}{Pirkuliyev Rovsen}]
	Find all function $f: \mathbb{R}\to\mathbb{R}$ such that $f(f(x)+f(y))=x+f(y)+2013$.
	\flushright \href{https://artofproblemsolving.com/community/c6h563060}{(Link to AoPS)}
\end{problem}



\begin{solution}[by \href{https://artofproblemsolving.com/community/user/29428}{pco}]
	\begin{tcolorbox}Find all function $f: \mathbb{R}\to\mathbb{R}$ such that $f(f(x)+f(y))=x+f(y)+2013$.\end{tcolorbox}
Let $P(x,y)$ be the assertion $f(f(x)+f(y))=x+f(y)+2013$

$f(x)$ is bijective.
Let then $u$ such that $f(u)=0$ : 
$P(x,u)$ $\implies$ $f(f(x))=x+2013$ $\implies$ $f(f(x)+f(y))=f(f(x))+f(y)$ $\implies$ $f(x+y)=f(x)+y$

Swapping then $x,y$ and subtracting, we get $f(x)-x=f(y)-y=c$

Plugging $f(x)=x+c$ i,n original equation, we get $\boxed{f(x)=x+\frac{2013}2}$ $\forall x$
\end{solution}
*******************************************************************************
-------------------------------------------------------------------------------

\begin{problem}[Posted by \href{https://artofproblemsolving.com/community/user/125553}{lehungvietbao}]
	1) Find all functions $f:\mathbb{R}\to\mathbb{R}$ which satisfiy the following the conditions
a) $f(x^2-y)=xf(x)-f(y) \quad \forall x,y \in \mathbb{R}$
b) $xf(x)>0 \quad \forall x\neq 0$

2)  Find all functions $f:\mathbb{R}\to\mathbb{R}$ which satisfiy the following the conditions
a) $f$ is injective
b) $f\left(\frac{x+y}{x-y} \right) =\frac{f(x)+f(y)}{f(x)-f(y)} \quad \forall x\neq y$
	\flushright \href{https://artofproblemsolving.com/community/c6h563121}{(Link to AoPS)}
\end{problem}



\begin{solution}[by \href{https://artofproblemsolving.com/community/user/29428}{pco}]
	\begin{tcolorbox}1) Find all functions $f:\mathbb{R}\to\mathbb{R}$ which satisfiy the following the conditions
a) $f(x^2-y)=xf(x)-f(y) \quad \forall x,y \in \mathbb{R}$
b) $xf(x)>0 \quad \forall x\neq 0$\end{tcolorbox}
Let $P(x,y)$ be the assertion $f(x^2-y)=xf(x)-f(y)$

$P(0,0)$ $\implies$ $f(0)=0$
$P(x,0)$ $\implies$ $f(x^2)=xf(x)$
$P(0,x)$ $\implies$ $f(-x)=-f(x)$

So $P(x,y)$ implies $f(x+y)=f(x)+f(y)$ $\forall x\ge 0$, $\forall y$
And, since odd : $f(x+y)=f(x)+f(y)$ $\forall x,y$ and so $f(x)=xf(1)$ $\forall x\in\mathbb Q$

Since $f(x)>0$ $\forall x>0$, $f(x+y)=f(x)+f(y)$ implies that $f(x)$ is increasing

So $\boxed{f(x)=ax}$ $\forall x$, which indeed is a solution, whatever is $a>0$
\end{solution}



\begin{solution}[by \href{https://artofproblemsolving.com/community/user/29428}{pco}]
	\begin{tcolorbox}2)  Find all functions $f:\mathbb{R}\to\mathbb{R}$ which satisfiy the following the conditions
a) $f$ is injective
b) $f\left(\frac{x+y}{x-y} \right) =\frac{f(x)+f(y)}{f(x)-f(y)} \quad \forall x\neq y$\end{tcolorbox}
Let $P(x,y)$ be the assertion $f\left(\frac{x+y}{x-y}\right)=\frac{f(x)+f(y)}{f(x)-f(y)}$ true $\forall x\ne y$

If $f(1)\ne 1$, then $P(x,0)$ $\implies$ $f(x)=f(0)\frac{f(1)+1}{f(1)-1}$ is not injective $\implies$ $f(1)=1$

Let $x\ne 0$ : subtracting $P(0,1)$ from $P(0,x)$, we get $f(0)(f(x)-1)=0$ and so, since injective, $f(0)=0$

Let $x\ne 1,y\ne 0$ : Subtracting $P(x,1)$ from $P(xy,y)$, we get new assertion $Q(x,y)$ : $f(xy)=f(x)f(y)$, still true when $x=1$ or $y=0$
$Q(-1,-1)$ $\implies$ $f(-1)^2=1$ and so;, since injective, $f(-1)=-1$
$Q(x,-1)$ $\implies$ $f(-x)=-f(x)$ and so $f(x)$ is odd.

Note then that $P(x,y)$ may be written $f(x+y)=f(x-y)\frac{f(x)+f(y)}{f(x)-f(y)}$

Let $a=f(2)$. $a\notin\{0,1\}$ since $f(x)$ is injective and $f(0)=0$ and $f(1)=1$
$P(2,1)$ $\implies$ $f(3)=\frac{a+1}{a-1}$
$P(3,1)$ $\implies$ $f(4)=a^2$
$P(4,1)$ $\implies$ $f(5)=\frac{a^2+1}{(a-1)^2}$
$P(5,1)$ $\implies$ $f(6)=a^3-a^2+a$
And since $f(6)=f(2)f(3)$, we get :
$a^3-a^2+a=\frac{a^2+a}{a-1}$ and so $a^3-2a^2+a-2=0$ $\iff$ $(a-2)(a^2+1)=0$ and so $f(2)=2$

It's then easy to establish with induction that $f(n)=n$, then  $f(nx)=nf(x)$ and then $f(x)=x$ $\forall x\in\mathbb Q$

$Q(x,x)$ $\implies$  $f(x)>0$ $\forall x>0$
$P(x,1)$ $\implies$ $f(x+1)(f(x)-1)=f(x-1)(f(x)+1)$ and so $f(x)>1$ $\forall x>1$
So $f(xy)=f(x)f(y)$ implies then (choose $y>1$) that $f(x)$ is increasing.

So $\boxed{f(x)=x}$ $\forall x\in\mathbb R$, which indeed is a solution
\end{solution}
*******************************************************************************
-------------------------------------------------------------------------------

\begin{problem}[Posted by \href{https://artofproblemsolving.com/community/user/125553}{lehungvietbao}]
	1) Find all functions $f:[0;1]\to [0;1]$  which satisfiy the following the conditions 
a) $f(x_{1})\neq f(x_{2}) \quad \forall x_{1}\neq x_{2}$
b) $2x-f(x)\in [0;1] \quad \forall x\in [0;1]$
c) $f(2x-f(x))=x \quad \forall x\in [0;1]$

2) Find all functions $f:\mathbb{R}\to\mathbb{R}$ such that
\[f((x+1)f(y))=yf(f(x)+1) \quad \forall x,y \in \mathbb{R}\]
	\flushright \href{https://artofproblemsolving.com/community/c6h563122}{(Link to AoPS)}
\end{problem}



\begin{solution}[by \href{https://artofproblemsolving.com/community/user/29428}{pco}]
	\begin{tcolorbox}1) Find all functions $f:[0;1]\to [0;1]$  which satisfiy the following the conditions 
a) $f(x_{1})\neq f(x_{2}) \quad \forall x_{1}\neq x_{2}$
b) $2x-f(x)\in [0;1] \quad \forall x\in [0;1]$
c) $f(2x-f(x))=x \quad \forall x\in [0;1]$\end{tcolorbox}
From b), we get $2x-1\le f(x)\le 2x$

Let $a>1,b>0$ such that $ax-b\le f(x)\le ax$ $\forall x\in[0,1]$

c) implies then $a(2x-f(x))-b\le x\le a(2x-f(x))$ and so $\frac{2a-1}ax-\frac ba\le f(x) \le \frac{2a-1}ax$
Note that $\frac ba>0$ and $\frac {2a-1}a>1$

Considering the sequence $a_0=2$, $b_0=1$, $a_{n+1}=\frac{2a_n-1}{a_n}$, $b_{n+1}=\frac{b_n}{a_n}$ we get :

$a_nx-b_n\le f(x)\le a_nx$ $\forall x\in[0,1]$, $\forall n\in\mathbb N\cup\{0\}$

Setting $n\to+\infty$ and considering that $\lim_{n\to+\infty}a_n=1$ and $\lim_{n\to+\infty}b_n=0$, we get :
$x\le f(x)\le x$

And so $\boxed{f(x)=x}$ $\forall x$, which indeed is a solution
(Note that we did not use the point a, injectivity)
\end{solution}



\begin{solution}[by \href{https://artofproblemsolving.com/community/user/29428}{pco}]
	\begin{tcolorbox}2) Find all functions $f:\mathbb{R}\to\mathbb{R}$ such that
\[f((x+1)f(y))=yf(f(x)+1) \quad \forall x,y \in \mathbb{R}\]\end{tcolorbox}
$\boxed{f(x)=0}$ $\forall x$ is a solution. So let us from now look only for non allzero solutions.

Let $P(x,y)$ be the assertion $f((x+1)f(y))=yf(f(x)+1)$
Let $t$ such that $f(t)\ne 0$
Let $a=f(1)$

If $f(0)\ne 0$, then $P(\frac t{f(0)}-1,0)$ $\implies$ $f(t)=0$, impossible. So $f(0)=0$

If $a=0$, then $P(x,1)$ $\implies$ $f(f(x)+1)=0$ $\forall x$ and so $P(\frac t{f(t)}-1,t)$ $\implies$ $f(t)=0$, impossible. So $a\ne 0$

$P(0,x)$ $\implies$ $f(f(x))=ax$ and, since $a\ne 0$, $f(x)$ is bijective.
$P(x,1)$ $\implies$ $f(ax+a)=f(f(x)+1)$ and, since injective, $f(x)=ax+a-1$

Plugging this back in original equation, we get $a=1$ and $\boxed{f(x)=x}$ $\forall x$
\end{solution}
*******************************************************************************
-------------------------------------------------------------------------------

\begin{problem}[Posted by \href{https://artofproblemsolving.com/community/user/192463}{arkanm}]
	Find all functions from the reals to the reals satisfying \[f((x-y)(x+y))=xf(x)-yf(y).\]
	\flushright \href{https://artofproblemsolving.com/community/c6h563126}{(Link to AoPS)}
\end{problem}



\begin{solution}[by \href{https://artofproblemsolving.com/community/user/29428}{pco}]
	\begin{tcolorbox}Find all functions from the reals to the reals satisfying \[f((x-y)(x+y))=xf(x)-yf(y).\]\end{tcolorbox}
Let $P(x,y)$ be the assertion $f((x+y)(x-y))=xf(x)-yf(y)$

$P(0,0)$ $\implies$ $f(0)=0$
Comparing $P(x,0)$ with $P(-x,0)$, we get $f(-x)=-f(x)$ $\forall x$ and $f(x)$ is odd.

$P(x,0)$ $\implies$ $f(x^2)=xf(x)$ and so $P(x,y)$ implies $f(x-y)=f(x)-f(y)$ $\forall x,y\ge 0$
And, since $f(x)$ is odd, it's easy to extend to $f(x+y)=f(x)+f(y)$ $\forall x,y$

$P(x+1,0)$ $\implies$  $f(x^2+2x+1)=(x+1)f(x+1)$ $\implies$ $f(x^2)+2f(x)+f(1)=xf(x)+xf(1)+f(x)+f(1)$ $\implies$ $f(x)=xf(1)$ $\forall x$

And so $\boxed{f(x)=ax}$ $\forall x$, which indeed is a solution, whatever is $a\in\mathbb R$
\end{solution}
*******************************************************************************
-------------------------------------------------------------------------------

\begin{problem}[Posted by \href{https://artofproblemsolving.com/community/user/198285}{ilovemath121}]
	Find all function $\mathbb{R} \rightarrow \mathbb{R}: f(x+yf(x)) = f(x) +x f(y)$
	\flushright \href{https://artofproblemsolving.com/community/c6h563153}{(Link to AoPS)}
\end{problem}



\begin{solution}[by \href{https://artofproblemsolving.com/community/user/29428}{pco}]
	\begin{tcolorbox}Find all function $\mathbb{R} \rightarrow \mathbb{R}: f(x+yf(x)) = f(x) +x f(y)$\end{tcolorbox}
$\boxed{f(x)=0}$ $\forall x$ is a solution. So let us from now look only for non allzero solutions.
$\boxed{f(x)=x}$ $\forall x$ is a solution. So let us from now look only for non allzero, non identical solutions.

Let $P(x,y)$ be the assertion $f(x+yf(x))=f(x)+xf(y)$
Let $t$ such that $f(t)\ne 0$
Let $a$ such that $b=a-f(a)\ne 0$

$P(1,0)$ $\implies$ $f(0)=0$. If $f(u)=0$, then $P(u,t)$ $\implies$ $u=0$. So $f(x)=0$ $\iff$ $x=0$

$P(-1,-1)$ $\implies$ $f(-1-f(-1))=0$ and so $f(-1)=-1$
$P(a,-1)$ $\implies$ $f(b)=-b$
$P(b,1)$ $\implies$ $b(f(1)-1)=0$ and so, since $b\ne 0$ : $f(1)=1$
$P(1,x)$ $\implies$ $f(x+1)=f(x)+1$

Let $x\ne 0$ : $P(x,\frac{y-x}{f(x)}+1)$ $\implies$ $f(y+f(x))=f(x)+xf(\frac{y-x}{f(x)})+x$  $=f(y)+x$, still true when $x=0$

Let $Q(x,y)$ be the new assertion $f(f(x)+y)=x+f(y)$
$Q(x,0)$ $\implies$ $f(f(x))=x$
$Q(f(x),y)$ $\implies$ $f(x+y)=f(x)+f(y)$
$P(f(x),y)$ $\implies$ $f(xy)=f(x)f(y)$

And this is a classical simple problem whose solutions are $f(x)=0$ $\forall x$ and $f(x)=x$ $\forall x$, none of which being acceptable in this part of the proof.
So no other solutions.
\end{solution}



\begin{solution}[by \href{https://artofproblemsolving.com/community/user/141363}{alibez}]
	the another solution

Suppose that $\exists\:   x :\:  \: f(x)\neq 0$

Let $P(x,y)$ be the assertion $f(x+yf(x))=f(x)+xf(y)$

it is easy to see that : $f(u)=0\rightarrow u=0$  

so we have :

$P(x , \frac{f(x)-x}{f(x)})\Rightarrow f(x)=x \: \: \: \forall x\in \mathbb{R}$
\end{solution}



\begin{solution}[by \href{https://artofproblemsolving.com/community/user/29428}{pco}]
	\begin{tcolorbox}...$P(x , \frac{f(x)-x}{f(x)})\Rightarrow f(x)=x \: \: \: \forall x\in \mathbb{R}$\end{tcolorbox}
Uhhhh ?

According to me, $P(x , \frac{f(x)-x}{f(x)})$ $\implies$ $f(f(x))=f(x)+xf(\frac{f(x)-x}{f(x)})$ 
And I dont see how you conclude $f(x)=x$ :?: (maybe you made a confusion between $f(f(x))$ and $f(x)$ ?)
\end{solution}
*******************************************************************************
-------------------------------------------------------------------------------

\begin{problem}[Posted by \href{https://artofproblemsolving.com/community/user/68025}{Pirkuliyev Rovsen}]
	Find all functions $f$ continuous on $R$ such that $f(f(f(x)))+f(x)=2x$, for all $x{\in}R$.
	\flushright \href{https://artofproblemsolving.com/community/c6h563254}{(Link to AoPS)}
\end{problem}



\begin{solution}[by \href{https://artofproblemsolving.com/community/user/29428}{pco}]
	\begin{tcolorbox}Find all functions $f$ continuous on $R$ such that $f(f(f(x)))+f(x)=2x$, for all $x{\in}R$.\end{tcolorbox}
$f(x)$ is injective and so, since continuous, monotonous.
If $f(x)$ is decreasing, then $LHS$ is decreasing while RHS is increasing, so $f(x)$ is increasing.

If $f(u)>u$ for some $u$, then, since increasing : $f(f(u))>f(u)>u$ and $f(f(f(u)))>f(u)>u$ and $LHS > 2u=RHS$, impossible
If $f(u)<u$ for some $u$, then, since increasing : $f(f(u))<f(u)<u$ and $f(f(f(u)))<f(u)<u$ and $LHS < 2u=RHS$, impossible

Hence $\boxed{f(x)=x}$ $\forall x$, which indeed is a solution.
\end{solution}
*******************************************************************************
-------------------------------------------------------------------------------

\begin{problem}[Posted by \href{https://artofproblemsolving.com/community/user/68025}{Pirkuliyev Rovsen}]
	Find all continuous functions $f: \mathbb{R}\to\mathbb{R}$ such that  $9f(8x)-9f(4x)+2f(2x)=100x$.
	\flushright \href{https://artofproblemsolving.com/community/c6h563255}{(Link to AoPS)}
\end{problem}



\begin{solution}[by \href{https://artofproblemsolving.com/community/user/29428}{pco}]
	\begin{tcolorbox}Find all continuous functions $f: \mathbb{R}\to\mathbb{R}$ such that  $9f(8x)-9f(4x)+2f(2x)=100x$.\end{tcolorbox}
Equation may be written $3(3f(8x)-f(4x)-50x)-2(3f(4x)-f(2x)-25x)=0$

Writing $g(x)=3f(4x)-f(2x)-25x$, this is $3g(2x)-2g(x)=0$

$g(x)=\frac 23g(\frac x2)$ $\implies$ $g(x)=\left(\frac 23\right)^ng(\frac x{2^n})$ and so, setting $n\to+\infty$ and using continuity : $g(x)=0$ $\forall x$

So $3f(4x)-f(2x)-25x=0$ $\iff$ $3(f(4x)-10x)=f(2x)-5x$
Writing $h(x)=f(2x)-5x$, this is $3h(2x)=h(x)$

$h(x)=\frac 13h(\frac x2)$ $\implies$ $h(x)=\left(\frac 13\right)^nh(\frac x{2^n})$ and so, setting $n\to+\infty$ and using continuity : $h(x)=0$ $\forall x$

And so $\boxed{f(x)=\frac 52x}$ $\forall x$ which indeed is a solution
\end{solution}
*******************************************************************************
-------------------------------------------------------------------------------

\begin{problem}[Posted by \href{https://artofproblemsolving.com/community/user/198285}{ilovemath121}]
	find all function $ f:\mathbb{R}^{+}\to\mathbb{R}^{+} $.satisfying $ f\left(x+f\left(y\right)\right) = f\left(x+y\right)+f\left(y\right) $  for all pairs of positive reals x and y.
	\flushright \href{https://artofproblemsolving.com/community/c6h563277}{(Link to AoPS)}
\end{problem}



\begin{solution}[by \href{https://artofproblemsolving.com/community/user/29428}{pco}]
	\begin{tcolorbox}find all function $ f:\mathbb{R}^{+}\to\mathbb{R}^{+} $.satisfying $ f\left(x+f\left(y\right)\right) = f\left(x+y\right)+f\left(y\right) $  for all pairs of positive reals x and y.\end{tcolorbox}
I considered that $\mathbb R^+=\mathbb R_{>0}$
Let $P(x,y)$ be the assertion $f(x+f(y))=f(x+y)+f(y)$

If $f(u)<u$ for some $u>0$, then $P(u-f(u),u)$ $\implies$ $f(2u-f(u))=0$, impossible.
If $f(u)=u$ for some $u>0$, then $P(1,u)$ $\implies$ $u=0$, impossible.
So $f(x)>x$ $\forall x$

Let then $g(x)$ from $\mathbb R^+\to\mathbb R^+$ defined as $g(x)=f(x)-x$
$P(x,y)$ becomes new assertion $Q(x,y)$ : $g(x+y+g(y))=g(x+y)+y$
If $g(a)=g(b)$, comparaison of $Q(a,b)$ and $Q(b,a)$ implies $a=b$ and so $g(x)$ is injective.

$Q(x+t+g(t)+z+g(z),y)$ $\implies$ $g(x+y+z+t+g(y)+g(z)+g(t))=g(x+y+z+t+g(z)+g(t))+y$
$Q(x+y+t+g(t),z)$ $\implies$ $g(x+y+z+t+g(z)+g(t))=g(x+y+z+t+g(t))+z$
$Q(x+y+z,t)$ $\implies$ $g(x+y+z+t+g(t))=g(x+y+z+t)+t$
And so $g(x+y+z+t+g(y)+g(z)+g(t))=g(x+y+z+t)+y+z+t$
And so injectivity implies that $g(a)+g(b)+g(c)=g(u)+g(v)+g(w)$ $\forall a,b,c,u,v,w$ such that $a+b+c=u+v+w$

So $g(x+y)+g(u)+g(v)=g(x)+g(y)+g(u+v)$
So $g(x+y)=g(x)+g(y)+c$ $\forall x,y>0$ and so $g(x)+c$ is additive and lower bounded and so is $ax$ for some $a>0$

So $f(x)=(a+1)x-c$

Plugging this back in original equation, we get the unique solution $\boxed{f(x)=2x}$ $\forall x$
\end{solution}



\begin{solution}[by \href{https://artofproblemsolving.com/community/user/177508}{mathuz}]
	similar:
Find all continuous functions $f:R \rightarrow R$ such that \[ f(x-f(y))=f(x-y)-f(y). \]
\end{solution}



\begin{solution}[by \href{https://artofproblemsolving.com/community/user/29428}{pco}]
	\begin{tcolorbox}similar:
Find all continuous functions $f:R \rightarrow R$ such that \[ f(x-f(y))=f(x-y)-f(y). \]\end{tcolorbox}
This one is quite simpler, according to me.
Let $g(x)=x-f(x)$ so that equation is assertion $P(x,y)$ : $g(x+g(y))=g(x)+y$
Let $a=g(0)$

$P(0,x)$ $\implies$ $g(g(x))=x+a$
$P(x-a,g(y-a))$ $\implies$ $g(x+y-a)=g(x-a)+g(y-a)$

Let $h(x)=g(x-a)$ and we get $h(x+y)=h(x)+h(y)$ and so $h(x)=cx$, since continuous
So $g(x)=cx+ca$
So $f(x)=(1-c)x-ca$

Plugging this back in original equation, we get two solutions :
$f(x)=0$ $\forall x$
$f(x)=2x$ $\forall x$
\end{solution}
*******************************************************************************
-------------------------------------------------------------------------------

\begin{problem}[Posted by \href{https://artofproblemsolving.com/community/user/125553}{lehungvietbao}]
	1) Find all  functions $f:[0;+\infty) \to [0;+\infty)$   such that
\[f(f(x))+f(x)=12x \quad \forall x\in [0;1)\]

2) Find all  functions $f,g:\mathbb{R}\to\mathbb{R}$   such that
\[f \left ( x^{3}+2y \right )+f(x+y)=g(x+2y) \quad \forall x,y \in \mathbb{R}\]
	\flushright \href{https://artofproblemsolving.com/community/c6h563285}{(Link to AoPS)}
\end{problem}



\begin{solution}[by \href{https://artofproblemsolving.com/community/user/192463}{arkanm}]
	\begin{tcolorbox}1) Find all  functions $f:[0;+\infty) \to [0;+\infty)$   such that
\[f(f(x))+f(x)=12x \quad \forall x\in [0;1)\]\end{tcolorbox}
$\boxed{f(x)=3x}$ $\forall x\in \mathbb{R^+}$. Similar to #2 here: http://www.artofproblemsolving.com/Forum/viewtopic.php?f=36&t=563026
\end{solution}



\begin{solution}[by \href{https://artofproblemsolving.com/community/user/29428}{pco}]
	\begin{tcolorbox}2) Find all  functions $f,g:\mathbb{R}\to\mathbb{R}$   such that
\[f \left ( x^{3}+2y \right )+f(x+y)=g(x+2y) \quad \forall x,y \in \mathbb{R}\]\end{tcolorbox}
Let $P(x,y)$ be the assertion $f(x^3+2y)+f(x+y)=g(x+2y)$

$P(0,\frac x2+y)$ $\implies$ $g(x+2y)=f(x+2y)+f(\frac x2+y)$ and so 

New assertion $Q(x,y)$ : $f(x^3+2y)+f(x+y)=f(x+2y)+f(\frac x2+y)$

$Q(1,x-\frac 12)$ $\implies$ $f(x+\frac 12)=f(x)$

Subtracting then $Q(y,\frac{x-y^3}2)$ from $Q(y+1,\frac{x-y^3}2)$, we get $f(x+(3y^2+3y+1))=f(x)$

And so $f(x+\Delta)=f(x)$ $\forall x$ and $\forall \Delta\ge \frac 14$

\begin{bolded}And so the solution\end{underlined}\end{bolded} :
$f(x)=a$ constant $\forall x$
$g(x)=2a$ constant $\forall x$
\end{solution}
*******************************************************************************
-------------------------------------------------------------------------------

\begin{problem}[Posted by \href{https://artofproblemsolving.com/community/user/125553}{lehungvietbao}]
	1) Find all  functions $f,g:\mathbb{R}\to\mathbb{R}$   such that
\[f \left ( x^{3}+4y \right )+f(x+y)=g(x+y) \quad \forall x,y \in \mathbb{R}\]

2) Find all  functions $f,g,h:\mathbb{R}\to\mathbb{R}$   such that
\[f\left ( x+y^{3} \right )+g \left ( x^{3}+y \right )=h(xy) \quad \forall x,y \in \mathbb{R}\]
	\flushright \href{https://artofproblemsolving.com/community/c6h563286}{(Link to AoPS)}
\end{problem}



\begin{solution}[by \href{https://artofproblemsolving.com/community/user/29428}{pco}]
	\begin{tcolorbox}1) Find all  functions $f,g:\mathbb{R}\to\mathbb{R}$   such that
\[f \left ( x^{3}+4y \right )+f(x+y)=g(x+y) \quad \forall x,y \in \mathbb{R}\]\end{tcolorbox}
Let $P(x,y)$ be the assertion $f(x^3+4y)+f(x+y)=g(x+y)$

Let $u\in \mathbb R$

Let $x$ be a real root of equation $x^3-4x-u=0$ wich exists whatever is $u$

$P(x,-x)$ $\implies$ $f(u)=g(0)-f(0)$ constant

Hence the solution : $\boxed{f(x)=c\text{    }\forall x\text{ and }g(x)=2c\text{    }\forall x}$ which indeed is a solution, whatever is $c\in\mathbb R$
\end{solution}



\begin{solution}[by \href{https://artofproblemsolving.com/community/user/29428}{pco}]
	\begin{tcolorbox}2) Find all  functions $f,g,h:\mathbb{R}\to\mathbb{R}$   such that
\[f\left ( x+y^{3} \right )+g \left ( x^{3}+y \right )=h(xy) \quad \forall x,y \in \mathbb{R}\]\end{tcolorbox}
Let $P(x,y)$ be the assertion $f(x+y^3)+g(x^3+y)=h(xy)$
Note that $(f,g,h)$ solution implies $(f-f(0),g+f(0)-h(0), h-h(0))$ solution too. So WLOG consider $f(0)=h(0)=0$

$P(0,x)$ $\implies$ $g(x)=-f(x^3)$
$P(x,0)$ $\implies$ $f(x)=-g(x^3)$ and so $f(x)=f(x^9)$

$P(x,y)$ implies then new assertion $Q(x,y)$ : $f(x+y^3)-f((x^3+y)^3)=h(xy)$

Let $a<0$ and $u(x)=(x^4+a^3)^9-x^{30}-ax^{26}$ : $u(x)$ is continuous, $u(0)<0$ and $\lim_{x\to+\infty}u(x)=+\infty$
So $\exists t>0$ such that $u(t)=0$ and so $(t^4+a^3)^9=t^{30}+at^{26}$
$\implies$ $(t+\frac{a^3}{t^3})^9=t^{3}+\frac at$
$\implies$ $(t+\frac{a^3}{t^3})^{27}=(t^{3}+\frac at)^3$
$\implies$ $f(t+\frac{a^3}{t^3})=f((t+\frac{a^3}{t^3})^{27})=f((t^{3}+\frac at)^3)$

$Q(t,\frac at)$ $\implies$ $h(a)=0$ $\forall a\le 0$

$Q(x,-x^3)$ $\implies$ $f(x-x^9)=h(-x^4)=0$ and since $x-x^9$ is surjective, we get $f(x)=0$ $\forall x$

\begin{bolded}Hence the solutions\end{underlined}\end{bolded} :
$f(x)=a$ $\forall x$
$g(x)=b$ $\forall x$
$h(x)=a+b$ $\forall x$
\end{solution}
*******************************************************************************
-------------------------------------------------------------------------------

\begin{problem}[Posted by \href{https://artofproblemsolving.com/community/user/125553}{lehungvietbao}]
	1) Find all  functions $f:\mathbb{R}^+\to\mathbb{R}^+$  which satisfy the following the conditions
a) If $0<x<y$ then $0<f(x)\leq f(y)$
b) $f(xy)f\left ( \frac{f(y)}{x} \right )=1 \quad \forall x,y \in \mathbb{R}^+$



2) Find all  functions $f:\mathbb{R}\to\mathbb{R}$  which satisfy the following the conditions
a) $f$ is increasing
b) $f(f(x)+y)=f(x+y)+f(0)  \quad \forall x,y \in \mathbb{R}$
	\flushright \href{https://artofproblemsolving.com/community/c6h563288}{(Link to AoPS)}
\end{problem}



\begin{solution}[by \href{https://artofproblemsolving.com/community/user/29428}{pco}]
	\begin{tcolorbox}2) Find all  functions $f:\mathbb{R}\to\mathbb{R}$  which satisfy the following the conditions
a) $f$ is increasing
b) $f(f(x)+y)=f(x+y)+f(0)  \quad \forall x,y \in \mathbb{R}$\end{tcolorbox}
Let $P(x,y)$ be the assertion $f(f(x)+y))=f(x+y)+f(0)$

$P(x,-f(x))$ $\implies$ $f(x-f(x))=0$ and so, since increasing, $x-f(x)=$ constant and so $\boxed{f(x)=x+a}$ $\forall x$ which indeed is a solution, whatever is $a\in\mathbb R$
\end{solution}



\begin{solution}[by \href{https://artofproblemsolving.com/community/user/29428}{pco}]
	\begin{tcolorbox}1) Find all  functions $f:\mathbb{R}^+\to\mathbb{R}^+$  which satisfy the following the conditions
a) If $0<x<y$ then $0<f(x)\leq f(y)$
b) $f(xy)f\left ( \frac{f(y)}{x} \right )=1 \quad \forall x,y \in \mathbb{R}^+$\end{tcolorbox}
Let $P(x,y)$ be the assertion $f(xy)f(\frac{f(y)}x)=1$

$P(\sqrt{\frac{f(x)}x},x)$ $\implies$ $f(\sqrt{xf(x)})=1$

Then, since $f(x)$ is non decreasing, we get $\lim_{x\to 0}\sqrt{xf(x)}=0$ and $\lim_{x\to +\infty}\sqrt{xf(x)}=+\infty$ 
So $\boxed{f(x)=1}$ $\forall x$
\end{solution}
*******************************************************************************
-------------------------------------------------------------------------------

\begin{problem}[Posted by \href{https://artofproblemsolving.com/community/user/119826}{seby97}]
	Let $w\in \mathbb{C}\setminus \mathbb{R}$, $|w|\neq 1$. Prove that $f: \mathbb{C} \to \mathbb{C}$, given by $f(z)= z+w\overline{z}$, is a bijection, and find its inverse.
	\flushright \href{https://artofproblemsolving.com/community/c6h563297}{(Link to AoPS)}
\end{problem}



\begin{solution}[by \href{https://artofproblemsolving.com/community/user/29428}{pco}]
	\begin{tcolorbox}Let $w\in C\R,|w|$different from 1.Prove that f:C-C ,$f(z)=z+w\overline{z}$ is a bijection and find it's inverse\end{tcolorbox}
$f(z)=z+w\overline z$
$\overline{f(z)}=\overline z+\overline wz$

So $w\overline{f(z)}-f(z)=(|w|^2-1)z$

And, since $|w|\ne 1$ : $\boxed{z=\frac{w\overline{f(z)}-f(z)}{|w|^2-1}}$
\end{solution}
*******************************************************************************
-------------------------------------------------------------------------------

\begin{problem}[Posted by \href{https://artofproblemsolving.com/community/user/68025}{Pirkuliyev Rovsen}]
	Find all  functions $f: \mathbb{R}\to\mathbb{R}$ such that $f(x^2+f(y))=y+xf(x)$ for all $x,y{\in}R$.
	\flushright \href{https://artofproblemsolving.com/community/c6h563304}{(Link to AoPS)}
\end{problem}



\begin{solution}[by \href{https://artofproblemsolving.com/community/user/141363}{alibez}]
	\begin{tcolorbox}Find all  functions $f: \mathbb{R}\to\mathbb{R}$ such that $f(x^2+f(y))=y+xf(x)$ for all $x,y{\in}R$.\end{tcolorbox}


[hide="hint"]it is easy to see that : $f(u)=0\Leftrightarrow u=0$ so we have : $f(f(x))=x\: \: ,\: \: f(x^{2})=xf(x)\: \: \forall x\in \mathbb{R}$

$f$  is injective ...

now $P(f(x),y)-P(x,y)$ kills it ... :!: 

[\/hide]
\end{solution}



\begin{solution}[by \href{https://artofproblemsolving.com/community/user/29428}{pco}]
	\begin{tcolorbox}Find all  functions $f: \mathbb{R}\to\mathbb{R}$ such that $f(x^2+f(y))=y+xf(x)$ for all $x,y{\in}R$.\end{tcolorbox}
Let $P(x,y)$ be the assertion $f(x^2+f(y))=y+xf(x)$
Let $a=f(0)$

$P(0,x)$ $\implies$ $f(f(x))=x$
$P(x,a)$ $\implies$ $f(x^2)=xf(x)+a$
$P(-x,a)$ $\implies$ $f(x^2)=-xf(-x)+a$ and so $f(-x)=-f(x)$ $\forall x\ne 0$
$P(x,f(y))$ $\implies$ $f(x^2+y)=xf(x)+f(y)$ $=f(x^2)+f(y)-a$

So new assertion $Q(x,y)$ : $f(x+y)=f(x)+f(y)-a$ $\forall x\ge 0$ $\forall y$
$Q(2,-1)$ $\implies$ $f(1)=f(2)+f(-1)-a$ and so $a=f(2)-2f(1)$
$Q(1,-2)$ $\implies$ $f(-1)=f(1)+f(-2)-a$ and so $a=2f(1)-f(2)$
So $a=0$ and $f(x)$ is odd and $f(x+y)=f(x)+f(y)$ $\forall x,y$ (even when both negative)

$P(x+1,0)$ $\implies$ $f(x^2+2x+1)=(x+1)f(x+1)$ and so $f(x^2)+2f(x)+f(1)=xf(x)+xf(1)+f(x)+f(1)$ $\implies$ $f(x)=xf(1)$

Pluging this back in orginal equation, we get $f(1)\in\{-1,1\}$ and so the two solutions :

$\boxed{f(x)=x}$ $\forall x$ and $\boxed{f(x)=-x}$ $\forall x$
\end{solution}



\begin{solution}[by \href{https://artofproblemsolving.com/community/user/29428}{pco}]
	\begin{tcolorbox}[quote="Pirkuliyev Rovsen"]Find all  functions $f: \mathbb{R}\to\mathbb{R}$ such that $f(x^2+f(y))=y+xf(x)$ for all $x,y{\in}R$.\end{tcolorbox}
...
it is easy to see that : $f(u)=0\Leftrightarrow u=0$
... \end{tcolorbox}
How :?:
\end{solution}



\begin{solution}[by \href{https://artofproblemsolving.com/community/user/83393}{fractals}]
	Similar to \begin{bolded}alibez\end{bolded}'s idea.

Let $P(x, y)$ denote the assertion. So $P(0, x)$ gives $f(f(x)) = x$ so it is injective and $P(f(x), y), P(x, y)$ give $f(f(x)^2 + f(y)) = y + f(x)f(f(x)) = y + xf(x) = f(x^2 + f(y))$, so by injectivity $f(x)^2 + f(y) = x^2 + f(y)$ so $f(x) = x, -x$, where each choice is individual for each $x$. Clearly then $f(0) = 0$. Now suppose there existed $a, b \not= 0$ with $f(a) = a, f(b) = -b$. Then $P(a, b)$ gives $f(a^2 - b) = b + a^2$. But now $f(a^2 - b) = a^2 - b, b - a^2$, depending. And $b - a^2 = b + a^2$ gives $a = 0$, bad, and $a^2 - b = b + a^2$ gives $b = 0$, bad. So they don't exist. So $f(x) = x$ for all $x$ or $f(x) = -x$ for all $x$, and both of those solutions clearly work.
\end{solution}



\begin{solution}[by \href{https://artofproblemsolving.com/community/user/141363}{alibez}]
	\begin{tcolorbox}[quote="alibez"][quote="Pirkuliyev Rovsen"]Find all  functions $f: \mathbb{R}\to\mathbb{R}$ such that $f(x^2+f(y))=y+xf(x)$ for all $x,y{\in}R$.\end{tcolorbox}
...
it is easy to see that : $f(u)=0\Leftrightarrow u=0$
... \end{tcolorbox}
How :?:\end{tcolorbox}

$f$ is injective and we know : $f(f(x))=x$

$P(u,y)-P(0,y)\Rightarrow .....$
\end{solution}
*******************************************************************************
-------------------------------------------------------------------------------

\begin{problem}[Posted by \href{https://artofproblemsolving.com/community/user/68025}{Pirkuliyev Rovsen}]
	Determine all functions $f$ defined on the interval $(0;+\infty)$ which have a derivative at $x=1$ and that satisfy $f(xy)=\sqrt{x}f(y)+\sqrt{y}f(x)$ for all positive real numbers $x,y$.
	\flushright \href{https://artofproblemsolving.com/community/c6h563323}{(Link to AoPS)}
\end{problem}



\begin{solution}[by \href{https://artofproblemsolving.com/community/user/29428}{pco}]
	\begin{tcolorbox}Determine all functions $f$ defined on the interval $(0;+\infty)$ which have a derivative at $x=1$ and that satisfy $f(xy)=\sqrt{x}f(y)+\sqrt{y}f(x)$ for all positive real numbers $x,y$.\end{tcolorbox}
Let $g(x)=f(e^x)$ function from $\mathbb R\to\mathbb R$ with a derivative at $x=0$

Let $P(x,y)$ be the assertion $g(x+y)=e^{\frac x2}g(y)+e^{\frac y2}g(x)$

$P(0,0)$ $\implies$ $g(0)=0$

$P(\frac x2,\frac x2)$ $\implies$ $g(x)=2e^{\frac x4}g(\frac x2)$ $\implies$ $g(x)=2^ne^{\frac x2(1-\frac 1{2^n})}g(\frac x{2^n})$

Setting $n\to+\infty$, we get $g(x)=g'(0)xe^{\frac x2}$

And so $\boxed{f(x)=a\sqrt x\ln x}$ $\forall x$, which indeed is a solution, whatever is $a\in\mathbb R$
\end{solution}
*******************************************************************************
-------------------------------------------------------------------------------

\begin{problem}[Posted by \href{https://artofproblemsolving.com/community/user/193279}{a2413097drdrbcom}]
	Find all functions $f:\mathbb{R}\to\mathbb{R}$ such, that for every pair $(x,y)\in\mathbb{R}^2$ the equation
$(x+y)\big(f(x)-f(y)\big)=(x-y)f(x+y)$
holds.
	\flushright \href{https://artofproblemsolving.com/community/c6h563332}{(Link to AoPS)}
\end{problem}



\begin{solution}[by \href{https://artofproblemsolving.com/community/user/29428}{pco}]
	\begin{tcolorbox}Find all functions $f:\mathbb{R}\to\mathbb{R}$ such, that for every pair $(x,y)\in\mathbb{R}^2$ the equation
$(x+y)\big(f(x)-f(y)\big)=(x-y)f(x+y)$
holds.\end{tcolorbox}
Let $P(x,y)$ be the assertion $(x+y)(f(x)-f(y))=(x-y)f(x+y)$

$P(1,-1)$ $\implies$ $f(0)=0$

Let $x\ne 0$
$P(\frac{x+1}2,\frac{x-1}2)$ $\implies$ $f(\frac{x+1}2)-f(\frac{x-1}2)=\frac{f(x)}x$

$P(\frac{x-1}2,\frac{3-x}2)$ $\implies$ $f(\frac{x-1}2)-f(\frac{3-x}2)=(x-2)f(1)$

$P(\frac{3-x}2,\frac{x+1}2)$ $\implies$ $f(\frac{3-x}2)-f(\frac{x+1}2)=\frac{1-x}2f(2)$

Adding these three lines, we get $\frac{f(x)}x+(x-2)f(1)+\frac{1-x}2f(2)=0$

And so $f(x)=x^2(\frac{f(2)}2-f(1))+x(2f(1)-\frac{f(2)}2)$ $\forall x\ne 0$, still true when $x=0$

Hence $\boxed{f(x)=ax^2+bx}$ $\forall x$, which indeed is a solution, whatever are $a,b\in\mathbb R$
\end{solution}
*******************************************************************************
-------------------------------------------------------------------------------

\begin{problem}[Posted by \href{https://artofproblemsolving.com/community/user/125553}{lehungvietbao}]
	1) Find all functions $f:\mathbb{N}\to\mathbb{Z}$ which satisfy the following the condition 
a) $f(n+11)=f(n) \quad \forall n \in \mathbb{N}$
b) $f(mn)=f(m)f(n) \quad \forall m,n \in \mathbb{N}$
Note that $0 \in \mathbb{N}$

2) Find all functions $f:\mathbb{Q}\to\mathbb{Q}$  such that
\[f(f(x)+x+y)=x+f(x)+f(y) \quad \forall x,y \in \mathbb{Q}\]
	\flushright \href{https://artofproblemsolving.com/community/c6h563453}{(Link to AoPS)}
\end{problem}



\begin{solution}[by \href{https://artofproblemsolving.com/community/user/29428}{pco}]
	\begin{tcolorbox}2) Find all functions $f:\mathbb{Q}\to\mathbb{Q}$  such that
\[f(f(x)+x+y)=x+f(x)+f(y) \quad \forall x,y \in \mathbb{Q}\]\end{tcolorbox}
Let $P(x,y)$ be the assertion $f(f(x)+x+y)=x+f(x)+f(y)$

1) General solution 
==============
Let $A$ any additive subgroup of $\mathbb Q$
Let $\sim$ the equivalence relation in $\mathbb Q$ defined as $x\sim y$ $\iff$ $x-y\in A$
Let $r(x)$ from $\mathbb Q\to\mathbb Q$ any choice function which associates to a rational $x$ a representant (unique per class) of its class.
Let $a(x)$ any function from $\mathbb Q\to A$

Then $f(x)=a(r(x))+x-2r(x)$

2) Proof that any function in the form of 1) indeed is a solution
=============================================
$f(x)+x+y=y+2(x-r(x))$ and, since $x-r(x)\in A$, we get $f(x)+x+y\sim y$ and so $r(f(x)+x+y)=r(y)$

So $f(f(x)+x+y)=a(r(f(x)+x+y))+f(x)+x+y-2r(f(x)+x+y)$ $=a(r(y))+f(x)+x+y-2r(y)$ $=x+f(x)+f(y)$
Q.E.D.

3) Proof that any solution may be written in the form of 1), so that we indeed got a general solution
======================================================================
Let $f(x)$ from $\mathbb Q\to\mathbb Q$ such that $f(f(x)+x+y)=x+f(x)+f(y)$ $\forall x,y\in\mathbb Q$
Let $A=\{a\in\mathbb Q$ such that $f(x+a)=f(x)$ $\forall x\in\mathbb Q\}$
It's immediate to see that $A$ is an additive subgroup of $\mathbb Q$
Let then $\sim$ the equivalence relation in $\mathbb Q$ defined as $x\sim y$ $\iff$ $x-y\in A$
Let $r(x)$ from $\mathbb Q\to\mathbb Q$ any choice function which associates to a rational $x$ a representant (unique per class) of its class.

$P(x,y)$ $\implies$ $x+f(x)\in A$
Let then $a(x)=f(x)+x$ from $\mathbb Q\to A$

$f(x)=f(r(x)+x-r(x))$ and, since $x-r(x)\in A$ : $f(x)=f(r(x))+x-r(x)$ $=a(r(x))+x-2r(x)$
Q.E.D.

4) Some examples of solutions
=====================
4.1) $A=\{0\}$
------------
$r(x)=x$ and $a(x)=0$ and so $\boxed{f(x)=-x}$

4.2) $A=\mathbb Q$
------------
$r(x)=c$ and $a(c)=d$ and so $\boxed{f(x)=x+a}$ where $a=d-2c$

4.3) $A=\mathbb Z$
----------------
$r(x)=\{x\}$
$a(x)=\left\lfloor 100\sin 2\pi x\right\rfloor$

$\boxed{f(x)=\left\lfloor 100\sin 2\pi x\right\rfloor+\lfloor x\rfloor -\{x\}}$

And infinitely many other solutions, varying for example $A$
...
\end{solution}



\begin{solution}[by \href{https://artofproblemsolving.com/community/user/29428}{pco}]
	\begin{tcolorbox}1) Find all functions $f:\mathbb{N}\to\mathbb{Z}$ which satisfy the following the condition 
a) $f(n+11)=f(n) \quad \forall n \in \mathbb{N}$
b) $f(mn)=f(m)f(n) \quad \forall m,n \in \mathbb{N}$
Note that $0 \in \mathbb{N}$\end{tcolorbox}
The only constant solutions are $f(n)=0$ $\forall n$ and $f(n)=1$ $\forall n$
Let us from now look only for non constant solutions

From b) we get $f(0)=0$ and $f(1)=1$ (since $f(x)$ is non constant)

a) implies $f(11n)=f(0)=0$
Let prime $p\ne 11$. $\exists k\in\mathbb N$ such that $1+11k=pm$. Then $1=f(1)=f(1+11k)=f(p)f(m)$ and so $f(p)=\pm 1$

If $\left(\frac n{11}\right)=1$, then $n\equiv m^2\pmod{11}$ and so $f(n)=f(m)^2=1$

If $\left(\frac n{11}\right)=\left(\frac m{11}\right)=-1$, then $\exists q$ such that $\left(\frac q{11}\right)=1$ and $m\equiv qn\pmod {11}$ and so $f(n)=f(m)$

\begin{bolded}Hence the solutions\end{underlined}\end{bolded} :
S1 : $f(n)=0$ $\forall n$

S2 : $f(n)=1$ $\forall n$

S3 : $f(11n)=0$ $\forall n$ and $f(m)=1$ $\forall m\not\equiv 0\pmod{11}$

S4 : $f(11n)=0$ $\forall n$ and $f(m)=\left(\frac m{11}\right)$  $\forall m\not\equiv 0\pmod{11}$
\end{solution}



\begin{solution}[by \href{https://artofproblemsolving.com/community/user/174583}{lagger11}]
	Sorry, how do you get the fourth solution ? Thanks :)
\end{solution}



\begin{solution}[by \href{https://artofproblemsolving.com/community/user/29428}{pco}]
	\begin{tcolorbox}Sorry, how do you get the fourth solution ? Thanks :)\end{tcolorbox}
When $m,n\not\equiv 0\pmod{11}$, we got :

If $\left(\frac n{11}\right)=\left(\frac m{11}\right)=1$, then $f(n)=f(m)=1$

If $\left(\frac n{11}\right)=\left(\frac m{11}\right)=-1$, then $f(n)=f(m)=\pm 1$

And so :
Either $f(n)=1$ $\forall n\not\equiv 0\pmod{11}$

Either $f(n)=\left(\frac n{11}\right)$ $\forall n\not\equiv 0\pmod{11}$

And it is easy to check that both indeed are solutions.
\end{solution}



\begin{solution}[by \href{https://artofproblemsolving.com/community/user/174583}{lagger11}]
	Is (n\/11) is n mod 11 ? Thanks
\end{solution}



\begin{solution}[by \href{https://artofproblemsolving.com/community/user/29428}{pco}]
	\begin{tcolorbox}Is (n\/11) is n mod 11 ? Thanks\end{tcolorbox}
No, $\left(\frac n{11}\right)$ is Legendre symbol : see http://en.wikipedia.org\/wiki\/Legendre_symbol
\end{solution}
*******************************************************************************
-------------------------------------------------------------------------------

\begin{problem}[Posted by \href{https://artofproblemsolving.com/community/user/125553}{lehungvietbao}]
	1) Find all functions $f:\mathbb{R}^+\to\mathbb{R}^+$  such that
\[f\left(\sqrt{x^{2}+y^{2} \right)=f(x)f(y) \quad \forall x,y \in \mathbb{R}^+}\]

2) Find all continuous functions $f:\mathbb{R}\to\mathbb{R}$  such that
\[f\left(\sqrt{x^{2}+y^{2} \right)=f(x)+f(y) \quad \forall x,y \in \mathbb{R}}\]
	\flushright \href{https://artofproblemsolving.com/community/c6h563454}{(Link to AoPS)}
\end{problem}



\begin{solution}[by \href{https://artofproblemsolving.com/community/user/29428}{pco}]
	\begin{tcolorbox}1) Find all functions $f:\mathbb{R}^+\to\mathbb{R}^+$  such that
\[f\left(\sqrt{x^{2}+y^{2} \right)=f(x)f(y) \quad \forall x,y \in \mathbb{R}^+}\]\end{tcolorbox}
Writing $h(x)=\ln f(\sqrt x)$, from $\mathbb R^+\to \mathbb R$, equation becomes $h(x+y)=h(x)+h(y)$

And so solution is ${\boxed{f(x)=e^{h(x^2)}}}$ $\forall x$, which indeed is a solution, whatever if $h(x)$ solution of additive Cauchy equation.
\end{solution}



\begin{solution}[by \href{https://artofproblemsolving.com/community/user/29428}{pco}]
	\begin{tcolorbox}2) Find all continuous functions $f:\mathbb{R}\to\mathbb{R}$  such that
\[f\left(\sqrt{x^{2}+y^{2} \right)=f(x)+f(y) \quad \forall x,y \in \mathbb{R}}\]\end{tcolorbox}
Moving $x\to -x$, we immediately get that $f(x)$ is even.
Writing $h(x)=f(\sqrt x)$, from $[0,+\infty)\to \mathbb R$, equation becomes $h(x+y)=h(x)+h(y)$ $\forall x,y\ge 0$ and so $h(x)=cx$, since continuous.

So $f(x)=cx^2$ $\forall x\ge 0$, and so, since even, $\boxed{f(x)=cx^2}$ $\forall x$, which indeed is a solution, whatever is $c\in\mathbb R$
\end{solution}
*******************************************************************************
-------------------------------------------------------------------------------

\begin{problem}[Posted by \href{https://artofproblemsolving.com/community/user/68025}{Pirkuliyev Rovsen}]
	Find all functions $f: \mathbb{N}\to\mathbb{N}$ such that $2(f(m^2+n^2))^3=f^2(m)f(n)+f^2(n)f(m)$ for distinct $m$ and $n$.
	\flushright \href{https://artofproblemsolving.com/community/c6h563459}{(Link to AoPS)}
\end{problem}



\begin{solution}[by \href{https://artofproblemsolving.com/community/user/29428}{pco}]
	\begin{tcolorbox}Find all functions $f: \mathbb{N}\to\mathbb{N}$ such that $2(f(m^2+n^2))^3=f^2(m)f(n)+f^2(n)f(m)$ for distinct $m$ and $n$.\end{tcolorbox}
I considered $0\notin\mathbb N$
Let $c=\gcd\left(\{f(n)\}_{n\in\mathbb N}\right)$ : $f(x)$ solution implies $\frac 1cf(x)$ solution. So WLOG consider $c=1$

Let $p$ be an odd prime and $n$ such that $p|f(n)$
Let $k=v_p(f(n))$. then $p^k|LHS$ 
If $k\not\equiv 0\pmod 3$, then $p^{k+1}|LHS$ and so $p|f(m)$ $\forall m$, impossible. So $3|v_p(f(n))$
But then $3|v_p(f(m^2+n^2))$ and so the power of $p$ dividing $LHS$ is a multiple of $9$
So $9|v_p(f(n))$ .... and so on
So no such $p$ and $f(n)$ is a power of two.

If $4|f(n)$, then $2|f(m^2+n^2)$ and so $6|LHS$ and so $2|f(m)$  $\forall m$ and so $p|c=1$ and so no such $p$ and $f(n)\in\{1,2\}$.

$f(n)=1$ $\forall n$ is a solution
$f(n)=2$ $\forall n$ is a not solution here since it should imply $2|c=1$
If $f(m)=1$ and $f(n)=2$ for some $m,n$, then $f(m^2+n^2)^2=3$, impossible.

Hence the only solution : $\boxed{f(x)=c}$ $\forall x$ which indeed is a solution, whatever is $c\in\mathbb N$
\end{solution}
*******************************************************************************
-------------------------------------------------------------------------------

\begin{problem}[Posted by \href{https://artofproblemsolving.com/community/user/125553}{lehungvietbao}]
	1) Find all continuous functions $f:\mathbb{R}^+\to\mathbb{R}^+$  such that
\[f\left( \frac{x+y}{2}\right)+f\left( \frac{2xy}{x+y}\right)=f(x)+f(y) \quad \forall x,y \in \mathbb{R}^+\]


2) Find all continuous functions $f:\mathbb{R}\to\mathbb{R}$  such that
\[f(x^{2}f(x)+f(y))=(f(x))^3+y \quad \forall x,y \in \mathbb{R}\]
	\flushright \href{https://artofproblemsolving.com/community/c6h563602}{(Link to AoPS)}
\end{problem}



\begin{solution}[by \href{https://artofproblemsolving.com/community/user/29428}{pco}]
	\begin{tcolorbox}1) Find all continuous functions $f:\mathbb{R}^+\to\mathbb{R}^+$  such that
\[f\left( \frac{x+y}{2}\right)+f\left( \frac{2xy}{x+y}\right)=f(x)+f(y) \quad \forall x,y \in \mathbb{R}^+\]\end{tcolorbox}
Let $P(x,y)$ be the assertion $f(\frac{x+y}2)+f(\frac{2xy}{x+y})=f(x)+f(y)$

Let $a>0$ and $x>\sqrt a$ and the sequence $u_0=x$ and $u_{n+1}=\frac{u_n+\frac a{u_n}}2$
It's easy to show that $u_n$ is a decreasing sequence converging towards $\sqrt a$
$P(u_n,\frac a{u_n})$ $\implies$ $f(u_{n+1})+f(\frac a{u_{n+1}})=f(u_n)+f(\frac a{u_n})$
And so $f(u_n)+f(\frac a{u_n})=f(u_0)+f(\frac a{u_0})=f(x)+f(\frac ax)$

Setting $n\to +\infty$, we get $f(x)+f(\frac ax)=2f(\sqrt a)$ and so $f(x)+f(y)=2f(\sqrt{xy})$

Setting $f(x)=g(\ln x)$ where $g(x)$ is a continuous function from $\mathbb R\to\mathbb R^+$, we get $g(\frac{x+y}2)=\frac{g(x)+g(y)}2$
So, using continuity, $g(x)=ax+b$ but $g(x)>0$ $\forall x$ implies $a=0$ and so :

$\boxed{f(x)=c}$ constant $\forall x$, which indeed is a solution, whatever is $c>0$
\end{solution}



\begin{solution}[by \href{https://artofproblemsolving.com/community/user/29428}{pco}]
	\begin{tcolorbox}2) Find all continuous functions $f:\mathbb{R}\to\mathbb{R}$  such that
\[f(x^{2}f(x)+f(y))=(f(x))^3+y \quad \forall x,y \in \mathbb{R}\]\end{tcolorbox}
Let $P(x,y)$ be the assertion $f(x^2f(x)+f(y))=f(x)^3+y$
$f(x)$ is bijective and so, since continuous, is monotone.
Since bijective, let $u$ such that $f(u)=0$

$P(u,x)$ $\implies$ $f(f(x))=x$

$P(f(1),f(x))$ $\implies$ $f(x+f(1)^2)=f(x)+1$ and so $f(x)$ can not be decreasing, and so, since monotone, is increasing

And it's a very classical result that the only increasing fonction such that $f(f(x))=x$ is $\boxed{f(x)=x}$ $\forall x$ which indeed is a solution.
\end{solution}
*******************************************************************************
-------------------------------------------------------------------------------

\begin{problem}[Posted by \href{https://artofproblemsolving.com/community/user/125553}{lehungvietbao}]
	1) Find all functions $f:\mathbb{R}\to\mathbb{R}$  such that 
\[f(xy)+f(x-y)+f(x+y+1)=xy+2x+1  \quad \forall x,y \in \mathbb{R}\]

2)  Find all functions $f:\mathbb{R}\to\mathbb{R}$  which satisfy the following the conditions 
a) $f(x)\neq f(y) \quad \forall x\neq y$
b) $f(xf(y)-y)=f(yf(x)-x)+f(x-y) \quad \forall x,y \in \mathbb{R}$
	\flushright \href{https://artofproblemsolving.com/community/c6h563603}{(Link to AoPS)}
\end{problem}



\begin{solution}[by \href{https://artofproblemsolving.com/community/user/29428}{pco}]
	\begin{tcolorbox}1) Find all functions $f:\mathbb{R}\to\mathbb{R}$  such that 
\[f(xy)+f(x-y)+f(x+y+1)=xy+2x+1  \quad \forall x,y \in \mathbb{R}\]\end{tcolorbox}
Let $g(x)=f(x)-x$ and the equation becomes assertion $P(x,y)$ : $g(xy)+g(x-y)+g(x+y+1)=0$

a) : $P(x+1,y)$ $\implies$ $g(xy+y)+g(x-y+1)+g(x+y+2)=0$
b) : $P(x,y)$ $\implies$  $g(xy)+g(x-y)+g(x+y+1)=0$
c) : $P(x-y,0)$ $\implies$ $g(0)+g(x-y)+g(x-y+1)=0$
d) : $P(x+y+1,0)$ $\implies$ $g(0)+g(x+y+1)+g(x+y+2)=0$
a)+b)-c)-d) : $g(xy+y)+g(xy)=2g(0)$

Let $u\ne v$. Setting in above line $x=\frac v{u-v}$ and $y=u-v$, we get $g(u)+g(v)=2g(0)$ and so $g(x)=c$ $\forall x$

Plugging $f(x)=x+c$ in original equation, we get $c=0$ and so $\boxed{f(x)=x}$ $\forall x$
\end{solution}



\begin{solution}[by \href{https://artofproblemsolving.com/community/user/29428}{pco}]
	\begin{tcolorbox}2)  Find all functions $f:\mathbb{R}\to\mathbb{R}$  which satisfy the following the conditions 
a) $f(x)\neq f(y) \quad \forall x\neq y$
b) $f(xf(y)-y)=f(yf(x)-x)+f(x-y) \quad \forall x,y \in \mathbb{R}$\end{tcolorbox}
Let $P(x,y)$ be the assertion $f(xf(y)-y)=f(yf(x)-x)+f(x-y)$

$P(0,0)$ $\implies$ $f(0)=0$ and so, since injective, $f(1)\ne 0$

Let $u=\frac 1{f(1)}\ne 0$

$P(1,u)$ $\implies$ $f(f(u)-u)=f(1-u)$ and so, since injective, $f(u)=1$

$P(x,u)$ $\implies$ $f(uf(x)-x)=0$ and so, since injective, $f(x)=\frac xu$

And so $\boxed{f(x)=ax}$ $\forall x$, which indeed is a solution, whatever is $a\ne 0$
\end{solution}
*******************************************************************************
-------------------------------------------------------------------------------

\begin{problem}[Posted by \href{https://artofproblemsolving.com/community/user/125553}{lehungvietbao}]
	1) Find all continuous functions $f:\mathbb{R}^+\to\mathbb{R}$  such that
\[f\left(\frac{1}{f(xy)}\right) =f(x)f(y) \quad \forall x,y \in \mathbb{R}^+\]

2) Find all continuous functions $f:[0;1]\to\mathbb{R}$  which satisfy the following the conditions
a) $f(0)=f(1)=0$
b) $f\left(\frac{x+y}{2} \right) \leq f(x)+f(y) \quad \forall x,y \in [0;1]$
	\flushright \href{https://artofproblemsolving.com/community/c6h563604}{(Link to AoPS)}
\end{problem}



\begin{solution}[by \href{https://artofproblemsolving.com/community/user/29428}{pco}]
	\begin{tcolorbox}1) Find all continuous functions $f:\mathbb{R}^+\to\mathbb{R}$  such that
\[f\left(\frac{1}{f(xy)}\right) =f(x)f(y) \quad \forall x,y \in \mathbb{R}^+\]\end{tcolorbox}
Let $P(x,y)$ be the assertion $f(\frac 1{f(xy)})=f(x)f(y)$
In order functional equation be defined, we need $f(xy)> 0$ $\forall x,y$ and so $f(x)$ is from $\mathbb R^+\to\mathbb R^+$

$P(x,y)$ $\implies$ $f(\frac 1{f(xy)})=f(x)f(y)$
$P(xy,1)$ $\implies$ $f(\frac 1{f(xy)})=f(xy)f(1)$
And so $\frac{f(xy)}{f(1)}=\frac{f(x)}{f(1)}\frac{f(y)}{f(1)}$

And so, since continuous, $f(x)=f(1)x^c$ for some $c\in\mathbb R$

Plugging this back in original equation, we get the two solutions :
$\boxed{f(x)=1}$ $\forall x$

$\boxed{f(x)=\frac ax}$ $\forall x$ and whatever is $a>0$
\end{solution}



\begin{solution}[by \href{https://artofproblemsolving.com/community/user/29428}{pco}]
	\begin{tcolorbox}2) Find all continuous functions $f:[0;1]\to\mathbb{R}$  which satisfy the following the conditions
a) $f(0)=f(1)=0$
b) $f\left(\frac{x+y}{2} \right) \leq f(x)+f(y) \quad \forall x,y \in [0;1]$\end{tcolorbox}
Let $P(x,y)$ be the assertion $f(\frac {x+y}2)\le f(x)+f(y)$

$P(0,1)$ $\implies$ $f(\frac 12)\le 0$
$P(1,\frac 12)$ and $P(\frac 12,1)$ $\implies$ $f(\frac 14)\le 0$ and $f(\frac 34)\le 0$
Simple induction and continuity imply then $f(x)\le 0$ $\forall x\in[0,1]$

$P(x,x)$ $\implies$  $f(x)\ge 0$ $\forall x\in[0,1]$

And so $\boxed{f(x)=0}$ $\forall x\in[0,1]$ which indeed is a solution
\end{solution}
*******************************************************************************
-------------------------------------------------------------------------------

\begin{problem}[Posted by \href{https://artofproblemsolving.com/community/user/68025}{Pirkuliyev Rovsen}]
	Find all functions $f: \mathbb{Q}\to\mathbb{Q}$ such that $f(f(x)+y)=x+f(y)$ for all $x,y{\in}Q$.
	\flushright \href{https://artofproblemsolving.com/community/c6h563635}{(Link to AoPS)}
\end{problem}



\begin{solution}[by \href{https://artofproblemsolving.com/community/user/29428}{pco}]
	\begin{tcolorbox}Find all functions $f: \mathbb{Q}\to\mathbb{Q}$ such that $f(f(x)+y)=x+f(y)$ for all $x,y\in \mathbb{Q}$.\end{tcolorbox}
Let $P(x,y)$ be the assertion $f(f(x)+y)=x+f(y)$
Let $a=f(0)$

$P(x-a,0)$ $\implies$ $f(f(x-a))=x$
$P(f(x-a),y-a)$ $\implies$ $f(x+y-a)=f(x-a)+f(y-a)$

Let $g(x)=f(x-a)$ and equation is $g(x+y)=g(x)+g(y)$ and so, since in $\mathbb Q$, $g(x)=cx$ and $f(x)=cx+ca$

Plugging this back in original equation, we get two solutions :
$\boxed{f(x)=x}$ $\forall x$

$\boxed{f(x)=-x}$ $\forall x$
\end{solution}



\begin{solution}[by \href{https://artofproblemsolving.com/community/user/177508}{mathuz}]
	we get that $f$ is bijective, so $f(f(x)+y)=f(f(x))+f(y)-f(0)$ and $f(x)=ax+c$.
\end{solution}



\begin{solution}[by \href{https://artofproblemsolving.com/community/user/141363}{alibez}]
	another solution: 

we know $f$ is injective so $f(0)=0$ so we have : $f(f(x))=x$ so $f$ is surjective . 

so : $f(x)+f(y)=f(x+y)$ .....
\end{solution}
*******************************************************************************
-------------------------------------------------------------------------------

\begin{problem}[Posted by \href{https://artofproblemsolving.com/community/user/125553}{lehungvietbao}]
	1) $f:\mathbb{Q}^+\to \mathbb{R}$ satisfies the following the condition : 
\[\left | f(x+y)-f(x) \right | \leq \frac{y}{x} \quad \forall x,y \in \mathbb{Q}^+ \]
Show that  $\sum_{i=1}^{n}\left | f(2^{n})-f(2^{i}) \right |\leq\frac{n(n-1)}{2} \quad \forall n \in \mathbb{N}$
Note that $0 \in \mathbb{N}$

2) Find all continuous functions $f:\mathbb{R}\to\mathbb{R}$  such that
\[xf(x)-yf(y)=(x-y)f(x+y) \quad \forall x,y \in \mathbb{R}\]
	\flushright \href{https://artofproblemsolving.com/community/c6h563765}{(Link to AoPS)}
\end{problem}



\begin{solution}[by \href{https://artofproblemsolving.com/community/user/29428}{pco}]
	\begin{tcolorbox}2) Find all continuous functions $f:\mathbb{R}\to\mathbb{R}$  such that
\[xf(x)-yf(y)=(x-y)f(x+y) \quad \forall x,y \in \mathbb{R}\]\end{tcolorbox}
Let $P(x,y)$ be the assertion $xf(x)-yf(y)=(x-y)f(x+y)$

$P(\frac{x+1}2,\frac{x-1}2)$ $\implies$ $\frac{x+1}2f(\frac{x+1}2)-\frac{x-1}2f(\frac{x-1}2)=f(x)$

$P(\frac{x-1}2,\frac{1-x}2)$ $\implies$ $\frac{x-1}2f(\frac{x-1}2)-\frac{1-x}2f(\frac{1-x}2)=(x-1)f(0)$

$P(\frac{1-x}2,\frac{x+1}2)$ $\implies$ $\frac{1-x}2f(\frac{1-x}2)-\frac{x+1}2f(\frac{x+1}2)=-xf(1)$

Adding, we get $f(x)=x(f(1)-f(0))+f(0)$

And so $\boxed{f(x)=ax+b}$ $\forall x$ which indeed is a solutoin, whatever are $a,b\in\mathbb R$

And, btw, no need for continuity
\end{solution}
*******************************************************************************
-------------------------------------------------------------------------------

\begin{problem}[Posted by \href{https://artofproblemsolving.com/community/user/125553}{lehungvietbao}]
	1) Find all  functions $f:\mathbb{R}\to\mathbb{R}$   such that
\[f(xy)=f(x)f(y)-f(x+y)+1 \quad \forall x,y \in \mathbb{R}\]

2) Find all  functions $f:\mathbb{R}\to\mathbb{R}$   such that
\[f(x)+f^{-1}(x)=2x \quad \forall x\in\mathbb{R}\] 
Where $f^{-1}$ is inverse function of $f$
	\flushright \href{https://artofproblemsolving.com/community/c6h563766}{(Link to AoPS)}
\end{problem}



\begin{solution}[by \href{https://artofproblemsolving.com/community/user/29428}{pco}]
	\begin{tcolorbox}1) Find all  functions $f:\mathbb{R}\to\mathbb{R}$   such that
\[f(xy)=f(x)f(y)-f(x+y)+1 \quad \forall x,y \in \mathbb{R}\]\end{tcolorbox}
Let $P(x,y)$ be the assertion $f(xy)=f(x)f(y)-f(x+y)+1$
Let $a=f(1)$

$P(0,0)$ $\implies$ $f(0)=1$
$P(1,1)$ $\implies$ $f(2)=a^2-a+1$
$P(2,1)$ $\implies$ $f(3)=a^3-2a^2+2a$
$P(3,1)$ $\implies$ $2f(4)=2a^4-6a^3+8a^2-4a+2$
$P(2,2)$ $\implies$ $2f(4)=a^4-2a^3+3a^2-2a+2$
And so $2a^4-6a^3+8a^2-4a+2=a^4-2a^3+3a^2-2a+2$ $\iff$ $a(a-1)^2(a-2)=0$ And so $a\in\{0,1,2\}$

1) $a=0$
=======
$P(x,1)$ $\implies$ $f(x+1)=1-f(x)$ and so $f(x+2)=f(x)$ and so :
$f(2)=f(0)=1$
$f(\frac 52)=f(\frac 12)$
But $P(2,\frac 12)$ $\implies$ $f(\frac 52)=f(\frac 12)+1$ and so contradiction

2) $a=1$
========
$P(x-1,1)$ $\implies$  $\boxed{f(x)=1}$ $\forall x$, which indeed is a solution

3) $a=2$
=======
$P(x,1)$ $\implies$ $f(x+1)=f(x)+1$
Let $y\ne 0$ : 
$P(\frac xy+1,y)$ $\implies$ $f(x+y)=(f(\frac xy)+1)f(y)-f(\frac xy+y)$
$P(\frac xy,y)$ $\implies$ $f(x)=f(\frac xy)f(y)-f(\frac xy+y)+1$
Subtracting, we get $f(x+y)=f(x)+f(y)-1$ $\forall x$, $\forall y\ne 0$, still true when $y=0$

And $P(x,y)$ becomes $f(xy)=f(x)f(y)-f(x)-f(y)+2$

Setting $f(x)=g(x)+1$, these two lines become $g(x+y)=g(x)+g(y)$ and $g(xy)=g(x)g(y)$ which is a very classical equation whose only solutions are $g(x)=0$, which does not fit the property $a=2$ and $g(x)=x$

And so $\boxed{f(x)=x+1}$ $\forall x$, which indeed is a solution
\end{solution}



\begin{solution}[by \href{https://artofproblemsolving.com/community/user/29428}{pco}]
	\begin{tcolorbox}2) Find all  functions $f:\mathbb{R}\to\mathbb{R}$   such that
\[f(x)+f^{-1}(x)=2x \quad \forall x\in\mathbb{R}\] 
Where $f^{-1}$ is inverse function of $f$\end{tcolorbox}
I dont think that this is a real olympiad exercise :( .
The classical problem contains the continuity constraint (but you dont have) and gives the solution $f(x)=x+a$

Without continuity, we have infinitely many strange solutions.
For example (but these are not the unique solutions) :

Let $A,B$ two supplementary $\mathbb Q$-subvectorspaces of the $\mathbb Q$-vectorspace $\mathbb R$
Let $a(x)$ from $\mathbb R\to A$ and $b(x)$ from $\mathbb R\to B$ the projections of $x$ so that $x=a(x)+b(x)$ in a unique manner.
Let $u(x)$ any linear fonction from $\mathbb R\to A$

Choose then $f(x)=x+u(b(x))$
\end{solution}
*******************************************************************************
-------------------------------------------------------------------------------

\begin{problem}[Posted by \href{https://artofproblemsolving.com/community/user/125553}{lehungvietbao}]
	1) Find all  functions $f:\mathbb{R}^+\to\mathbb{R}^+$   such that
\[f(xy)f(x+y)=1 \quad \forall x,y>0\]

2) Find all  functions $f:\mathbb{R}\to\mathbb{R}$   such that
\[f\left ( x^{3}+2y \right )=f(x+y)+f(3x+y) +1 \quad \forall x,y \in \mathbb{R}\]
	\flushright \href{https://artofproblemsolving.com/community/c6h563767}{(Link to AoPS)}
\end{problem}



\begin{solution}[by \href{https://artofproblemsolving.com/community/user/29428}{pco}]
	\begin{tcolorbox}1) Find all  functions $f:\mathbb{R}^+\to\mathbb{R}^+$   such that
\[f(xy)f(x+y)=1 \quad \forall x,y>0\]\end{tcolorbox}
Let $P(x,y)$ be the assertion $f(xy)f(x+y)=1$

Let $a\ge 4$ and $x,y$ be the two positive roots of the quadratic $x^2-ax+a=0$ so that $xy=x+y=a$ : $P(x,y)$ $\implies$ $f(a)=1$ and so $f(x)=1$ $\forall x\in[4,+\infty)$

Let $x\in[3,4)$ $P(x,1)$ $\implies$ $f(x)=1$ and so $f(x)=1$ $\forall x\in[3,+\infty)$
Let $x\in[2,3)$ $P(x,1)$ $\implies$ $f(x)=1$ and so $f(x)=1$ $\forall x\in[2,+\infty)$
Let $x\in[1,2)$ $P(x,1)$ $\implies$ $f(x)=1$ and so $f(x)=1$ $\forall x\in[1,+\infty)$
Let $x\in(0,1)$ $P(x,1)$ $\implies$ $f(x)=1$ and so $\boxed{f(x)=1}$ $\forall x\in\mathbb R^+$, which indeed is a solution.
\end{solution}



\begin{solution}[by \href{https://artofproblemsolving.com/community/user/29428}{pco}]
	\begin{tcolorbox}2) Find all  functions $f:\mathbb{R}\to\mathbb{R}$   such that
\[f\left ( x^{3}+2y \right )=f(x+y)+f(3x+y) +1 \quad \forall x,y \in \mathbb{R}\]\end{tcolorbox}
Let $P(x,y)$ be the assertion $f(x^3+2y)=f(x+y)+f(3x+y)+1$

$P(0,x)$ $\implies$ $f(2x)=2f(x)+1$ and so $f(8x)=8f(x)+7$

$P(2x,0)$ $\implies$ $f(8x^3)=f(2x)+f(6x)+1$ $\implies$ $8f(x^3)+7=(2f(x)+1)+(2f(3x)+1)+1$ and so :
$4f(x^3)=f(x)+f(3x)-2$ 
$P(x,0)$ $\implies$ $f(x^3)=f(x)+f(3x)+1$ 
Subtracting these two last lines, we get $f(x^3)=-1$

And so $\boxed{f(x)=-1}$ $\forall x$, which indeed is a solution.
\end{solution}
*******************************************************************************
-------------------------------------------------------------------------------

\begin{problem}[Posted by \href{https://artofproblemsolving.com/community/user/125553}{lehungvietbao}]
	1) Find all continuous functions $f$ on $\left [ \frac{-1}{12};\frac{1}{6} \right ]$ such that 
\[1996f(x)-\frac{1997}{1998}f\left ( \frac{x}{1999} \right )=1996x^{2000} \quad \forall x \in  \left [ \frac{-1}{12};\frac{1}{6} \right ]\]


2) Find all continuous functions $f:\mathbb{R}\to\mathbb{R}$  such that
\[\left ( f\left ( \frac{x+y}{2} \right ) \right )^{2}=f(x)f(y) \quad \forall x,y \in \mathbb{R}\]
	\flushright \href{https://artofproblemsolving.com/community/c6h563768}{(Link to AoPS)}
\end{problem}



\begin{solution}[by \href{https://artofproblemsolving.com/community/user/29428}{pco}]
	\begin{tcolorbox}2) Find all continuous functions $f:\mathbb{R}\to\mathbb{R}$  such that
\[\left ( f\left ( \frac{x+y}{2} \right ) \right )^{2}=f(x)f(y) \quad \forall x,y \in \mathbb{R}\]\end{tcolorbox}
Let $P(x,y)$ be the assertion $f(\frac {x+y}2)^2=f(x)f(y)$

If $f(a)=0$ for some $a$, then $P(2x-a,a)$ $\implies$ $f(x)=0$ $\forall x$, which indeed is a solution
So let us consider from now solutions where $f(x)\ne 0$ $\forall x$, and so, since continuous, $f(x)$ has a constant sign.

$f(x)$ solution $\implies$ $-f(x)$ solution. So WLOG consider $f(x)>0$ $\forall x$
Let then $g(x)=\ln (f(x))$ and the equation becomes $g(\frac {x+y}2)=\frac{g(x)+g(y)}2$

And this is a classical equation which immediately gives (using continuity) $g(x)=bx+c$

Hence the answer ; $\boxed{f(x)=a e^{bx}}$ $\forall x$ which indeed is a solution, whatever are $a,b\in\mathbb R$

Note that $a=0$ gives the allzero solution we already found and that the transformation $e^c\to a$ allows to take in count the transformations $f\to -f$
\end{solution}



\begin{solution}[by \href{https://artofproblemsolving.com/community/user/29428}{pco}]
	\begin{tcolorbox}1) Find all continuous functions $f$ on $\left [ \frac{-1}{12};\frac{1}{6} \right ]$ such that 
\[1996f(x)-\frac{1997}{1998}f\left ( \frac{x}{1999} \right )=1996x^{2000} \quad \forall x \in  \left [ \frac{-1}{12};\frac{1}{6} \right ]\]\end{tcolorbox}
For easier writing, let $a=\frac{1997}{1996\times 1998}$ and $b=\frac 1{1999}$ so that equation is $f(x)-af(bx)=x^{2000}$

Writing ${f(x)=g(x)+\frac{x^{2000}}{1-ab^{2000}}}$, equation becomes $g(x)=ag(bx)$

So $g(x)=a^ng(b^nx)$ and, setting $n\to+\infty$ and using continuity, $g(x)=0$

And so $\boxed{f(x)=\frac{x^{2000}}{1-ab^{2000}}}$ $\forall x\in[-\frac 1{12},\frac 16]$, which indeed is a solution.
\end{solution}
*******************************************************************************
-------------------------------------------------------------------------------

\begin{problem}[Posted by \href{https://artofproblemsolving.com/community/user/125553}{lehungvietbao}]
	1) Suppose $b,c\in \mathbb{R}\setminus \{0\}$ and $d\in\mathbb{R}$. Find all functions $f:\mathbb{R}\to\mathbb{R}$ such that
\[f(x+b)=cf(x)+d  \quad \forall x\in\mathbb{R}\]

2) Suppose $a\in\mathbb{R}\setminus \{-1;0;1\} , b\in \mathbb{R}\setminus \{0\}$ and $c\in\mathbb{R}$ . Find all functions $f:\mathbb{R}\to\mathbb{R}$ such that
\[f(ax)=bf(x)+c  \quad \forall x\in\mathbb{R}\]
	\flushright \href{https://artofproblemsolving.com/community/c6h563895}{(Link to AoPS)}
\end{problem}



\begin{solution}[by \href{https://artofproblemsolving.com/community/user/29428}{pco}]
	\begin{tcolorbox}1) Suppose $b,c\in \mathbb{R}\setminus \{0\}$ and $d\in\mathbb{R}$. Find all functions $f:\mathbb{R}\to\mathbb{R}$ such that
\[f(x+b)=cf(x)+d  \quad \forall x\in\mathbb{R}\]\end{tcolorbox}
So $f(x+nb)=c^nf(x)+d\sum_{k=0}^{n-1}c^k$ and $f(x-nb)=c^{-n}f(x)-\frac dc\sum_{k=0}^{n-1}c^{-k}$

Wrting then $x=b\left\lfloor\frac xb\right\rfloor+b\left\{\frac xb\right\}$, we get the result :

Let $g(x)$ any function from $[0,b)\to\mathbb R$ (or from $(b,0]\to\mathbb R$ if $b<0$). Then :

If $\left\lfloor\frac xb\right\rfloor<0$, $f(x)=c^{\left\lfloor\frac xb\right\rfloor}g(b\left\{\frac xb\right\})-\frac dc\sum_{k=0}^{-\left\lfloor\frac xb\right\rfloor-1}c^{-k}$

If $\left\lfloor\frac xb\right\rfloor=0$, $f(x)=g(x)$

If $\left\lfloor\frac xb\right\rfloor>0$, $f(x)=c^{\left\lfloor\frac xb\right\rfloor}g(b\left\{\frac xb\right\})+d\sum_{k=0}^{\left\lfloor\frac xb\right\rfloor-1}c^k$
\end{solution}



\begin{solution}[by \href{https://artofproblemsolving.com/community/user/29428}{pco}]
	\begin{tcolorbox}2) Suppose $a\in\mathbb{R}\setminus \{-1;0;1\} , b\in \mathbb{R}\setminus \{0\}$ and $c\in\mathbb{R}$ . Find all functions $f:\mathbb{R}\to\mathbb{R}$ such that
\[f(ax)=bf(x)+c  \quad \forall x\in\mathbb{R}\]\end{tcolorbox}
Just boring and basic.

Choose in a free manner $f(x)$ over $[-1,-|a|)$ and $[1,|a|)$ and just apply formula to full definition of $f(x)$

You must pay little attention to $f(0)$ :
$f(0)=\frac c{1-b}$ if $b\ne 1$
$f(0)$ has any value if $b=1$ and $c=0$
No solution if $b=1$ and $c\ne 0$

Have you really got such a question in a national olympiad ?????? What a shame for your teacers !
\end{solution}



\begin{solution}[by \href{https://artofproblemsolving.com/community/user/125553}{lehungvietbao}]
	\begin{tcolorbox}

Just boring and basic.
.....
Have you really got such a question in a national olympiad ?????? What a shame for your teacers !\end{tcolorbox}

Dear Mr.Patrick 
Firstly , thanks for your kind help :)

All my problem are real Olympiad problems.  And maybe you are a master in this field (functional equation )  so you feel  boring when these problem are too easy for you ! :D
\end{solution}



\begin{solution}[by \href{https://artofproblemsolving.com/community/user/29428}{pco}]
	I mean that such general cases with parameters are stupid in an olympiad.
These are just useful as school cases.

But you claimed you got them in a real olympiad contest \/ training session and I just think that these are poor olympiad contests .
It reminds me of this post : http://www.artofproblemsolving.com/Forum/viewtopic.php?f=36&t=485367
\end{solution}
*******************************************************************************
-------------------------------------------------------------------------------

\begin{problem}[Posted by \href{https://artofproblemsolving.com/community/user/125553}{lehungvietbao}]
	1) Suppose $b\in\mathbb{R}\setminus \{0\}$ and $c\in\mathbb{R}$ . Find all functions $f:\mathbb{R}\to\mathbb{R}$ such that
\[f(-x)=bf(x)+c\quad \forall x\in\mathbb{R}\]
	\flushright \href{https://artofproblemsolving.com/community/c6h563896}{(Link to AoPS)}
\end{problem}



\begin{solution}[by \href{https://artofproblemsolving.com/community/user/29428}{pco}]
	\begin{tcolorbox}1) Suppose $b\in\mathbb{R}\setminus \{0\}$ and $c\in\mathbb{R}$ . Find all functions $f:\mathbb{R}\to\mathbb{R}$ such that
\[f(-x)=bf(x)+c\quad \forall x\in\mathbb{R}\]\end{tcolorbox}
1) If $|b|\ne 1$
=========
$f(-x)=bf(x)+c$
$f(x)=bf(-x)+c$

And so $f(x)=\frac c{1-b}$ constant

2) If $b=1$
=======
$f(-x)=f(x)+c$

If $c\ne 0$ : no solution (look at $x=0$)
If $c=0$ : Solutions are even functions

3) If $b=-1$
=======
$f(-x)=-f(x)+c$

If $c\ne 0$ : no solution (look at $x=0$)
If $c=0$ : Solutions are odd functions
\end{solution}
*******************************************************************************
-------------------------------------------------------------------------------

\begin{problem}[Posted by \href{https://artofproblemsolving.com/community/user/125553}{lehungvietbao}]
	1) Find all  functions $f:\mathbb{R}\to\mathbb{R}$   such that
\[f\left ( x+f(y)\right )=3f(x)+f(y) -2x \quad \forall x,y \in \mathbb{R}\]

2) Find all continuous functions on $\mathbb{R}$ such that
\[f(3x)=f(x)=f\left(\frac{x^2}{1+x^2}\right)  \quad \forall x\in\mathbb{R}\]
	\flushright \href{https://artofproblemsolving.com/community/c6h563897}{(Link to AoPS)}
\end{problem}



\begin{solution}[by \href{https://artofproblemsolving.com/community/user/29428}{pco}]
	\begin{tcolorbox}1) Find all  functions $f:\mathbb{R}\to\mathbb{R}$   such that
\[f\left ( x+f(y)\right )=3f(x)+f(y) -2x \quad \forall x,y \in \mathbb{R}\]\end{tcolorbox}
Let $P(x,y)$ be the assertion $f(x+f(y))=3f(x)+f(y)-2x$
Let $a=f(0)$

$P(\frac {a-x}2,0)$ $\implies$ $f(\frac {3a-x}2)-3f(\frac {a-x}2)=x$ and so any real may be written as $f(u)-3f(v)$ for some $u,v$

$P(-f(x),x)$ $\implies$ $f(-f(x))=-f(x)+\frac a3$

$P(-2f(x),x)$ $\implies$ $f(-2f(x))=-2f(x)+\frac a9$

$P(-3f(x),x)$ $\implies$ $f(-3f(x))=-3f(x)+\frac a{27}$

$P(-3f(v),u)$ $\implies$ $f(f(u)-3f(v))=f(u)-3f(v)+\frac a{9}$

And so $f(x)=x+\frac a9$ $\forall x$ and so $a=0$and $\boxed{f(x)=x}$ $\forall x$ which indeed is a solution.
\end{solution}



\begin{solution}[by \href{https://artofproblemsolving.com/community/user/29428}{pco}]
	\begin{tcolorbox}2) Find all continuous functions on $\mathbb{R}$ such that
\[f(3x)=f(x)=f\left(\frac{x^2}{1+x^2}\right)  \quad \forall x\in\mathbb{R}\]\end{tcolorbox}
$f(3x)=f(x)$ implies $f(x)=f(x3^{-n})$ and so continuity implies $f(x)=f(0)$ constant, and no need for third equality.
You should change your teachers ....
\end{solution}
*******************************************************************************
-------------------------------------------------------------------------------

\begin{problem}[Posted by \href{https://artofproblemsolving.com/community/user/119826}{seby97}]
	Let $n \in N*$ be a fixed number.Find all monotonic function ,f:R->R such that $f(x+f(y))=f(x)+y^n$,for all $x,y \in R$
	\flushright \href{https://artofproblemsolving.com/community/c6h563940}{(Link to AoPS)}
\end{problem}



\begin{solution}[by \href{https://artofproblemsolving.com/community/user/29428}{pco}]
	\begin{tcolorbox}Let $n \in \mathbb N*$ be a fixed number.Find all monotonic functions $f:\mathbb R \to \mathbb R$ such that $f(x+f(y))=f(x)+y^n$,for all $x,y \in \mathbb R$\end{tcolorbox}
I suppose that $N*$ is $\mathbb N$, the set of all positive integers.
Let $P(x,y)$ be the assertion $f(x+f(y))=f(x)+y^n$

$P(x,0)$ $\implies$ (since injective), $f(0)=0$
$P(0,y)$ $\implies$ $f(f(y))=y^n$
$P(0,f(y))$ $\implies$ $f(y^n)=f(y)^n$
$P(x,f(y))$ $\implies$ $f(x+y^n)=f(x)+f(y^n)$

So $f(x+y)=f(x)+f(y)$ $\forall x$, $\forall y\ge 0$

$P(-y^n,f(y))$ $\implies$ $0=f(-y^n)+f(y^n)$ and so $f(x)$ is odd

So $f(x+y)=f(x)+f(y)$ $\forall x,y$

So, since monotonous, $f(x)=cx$ $\forall x$

Plugging this back in original equation, we get  :
If $n\ne 1$ : no solution.
If $n=1$ : two solutions : $\boxed{f(x)=x}$ $\forall x$ and $\boxed{f(x)=-x}$ $\forall x$
\end{solution}



\begin{solution}[by \href{https://artofproblemsolving.com/community/user/169068}{hero12}]
	I think monotonic function is not enough, but strict monotony is required.
\end{solution}



\begin{solution}[by \href{https://artofproblemsolving.com/community/user/29428}{pco}]
	\begin{tcolorbox}I think monotonic function is not enough, but strict monotony is required.\end{tcolorbox}
Yes, I considered monotonous as stricly monotonous.

But it's very easy to adapt. I used it only twice :

1) for $f(0)=0$
Adaptation :If $f(0)=a\ne 0$, then $f(x+a)=f(x)$ and so $f(x)$ is periodic, and so, since monotonous, is constant, which is never a solution .

2) $f(x+y)=f(x)+f(y)$ implies $f(x)=cx$
simple monotony is enough there.
\end{solution}
*******************************************************************************
-------------------------------------------------------------------------------

\begin{problem}[Posted by \href{https://artofproblemsolving.com/community/user/68025}{Pirkuliyev Rovsen}]
	Find all  functions ${{f: \mathbb[1;+\infty)}\to\mathbb[1;+\infty)}$ such that $f(xf(y))=yf(x)$.
	\flushright \href{https://artofproblemsolving.com/community/c6h564046}{(Link to AoPS)}
\end{problem}



\begin{solution}[by \href{https://artofproblemsolving.com/community/user/29428}{pco}]
	\begin{tcolorbox}Find all  functions ${{f: \mathbb[1;+\infty)}\to\mathbb[1;+\infty)}$ such that $f(xf(y))=yf(x)$.\end{tcolorbox}
Let $P(x,y)$ be the assertion $f(xf(y))=yf(x)$

$P(x,1)$ $\implies$ $f(xf(1))=f(x)$ and, since $f(x)$ is obviously injective, $f(1)=1$
$P(1,y)$ $\implies$ $f(f(y))=y$
$P(x,f(y))$ $\implies$ $f(xy)=f(x)f(y)$

Let then $g(x)=\ln (f(e^x))$ function from $[0,+\infty)\to [0,+\infty)$ :
$f(xy)=f(x)f(y)$ implies $g(x+y)=g(x)+g(y)$ $\forall x,y\ge 0$
And since $g(x)\ge 0$ $\forall x$, we get $g(x)=ax$ and so $f(x)=x^a$ for some $a\ge 0$

Plugging back in original equation, we get $a=1$ and so $\boxed{f(x)=x}$ $\forall x$
\end{solution}
*******************************************************************************
-------------------------------------------------------------------------------

\begin{problem}[Posted by \href{https://artofproblemsolving.com/community/user/190536}{DonaldLove}]
	find $f:Z^{+} \to Z^{+}$ such that for all distinct positive integer a,b,c we have $f(a)+f(b)+f(c) \vdots a+b+c$ and there exists polynomial P(x) such that $f(a)<P(a) \forall a \in Z^{+}$
	\flushright \href{https://artofproblemsolving.com/community/c6h564055}{(Link to AoPS)}
\end{problem}



\begin{solution}[by \href{https://artofproblemsolving.com/community/user/29428}{pco}]
	\begin{tcolorbox}Find $f:\mathbb{Z}^{+} \to \mathbb{Z}^{+}$ such that for all distinct positive integer $a,b,c$ we have $f(a)+f(b)+f(c) \vdots a+b+c$ and there exists polynomial $P(x)$ such that $f(a)<P(a) \quad  \forall a \in \mathbb{Z}^{+}$\end{tcolorbox}
Let $P(x,y,z)$ b the assertion $x+y+z|f(x)+f(y)+f(z)$, true $\forall$ distinct positive integers $x,y,z$

Let $x> 2\in\mathbb Z$
Let $p$ any number $>2x+3$
$p-x-2>x>2$ and so $P(x,2,p-x-2)$ $\implies$ $p|f(x)+f(2)+f(p-x-2)$
$p-x-2>x+1>1$ and so $P(x+1,1,p-x-2)$ $\implies$ $p|f(x+1)+f(1)+f(p-x-2)$

So $p|f(x+1)-f(x)+f(1)-f(2)$ and so, since we have infinitely many such $p$ :$f(x+1)-f(x)=f(2)-f(1)$ $\forall x>2$

Let us deal with the case $x=2$ :
Let any number $p>12$ 
$P(4,3,p-7)$ $\implies$ $p|f(4)+f(3)+f(p-7)$
$P(5,2,p-7)$ $\implies$ $p|f(5)+f(2)+f(p-7)$

So $p|f(5)-f(4)+f(2)-f(3)$ and so, since we have infinitely many such $p$ :$f(2)-f(1)=f(5)-f(4)=f(3)-f(2)$ 

So $f(x+1)-f(x)=f(2)-f(1)$ $\forall x$ and $f(x)=ax+b$

Plugging this in required equation, we get $b=0$ and $\boxed{f(x)=ax}$ $\forall x\in\mathbb Z^+$, which indeed is a solution, whatever is $a\in\mathbb Z^+$

And I dont see what could be the usage of the strange condition about polynomial bound :?:
\end{solution}
*******************************************************************************
-------------------------------------------------------------------------------

\begin{problem}[Posted by \href{https://artofproblemsolving.com/community/user/68025}{Pirkuliyev Rovsen}]
	Given a function $f(x)=4x^3-3x$. Solve the equation: $ \underbrace{f{\circ}f{\circ}\cdots{\circ}f}_{2001\textrm{ times}}(x)=x $.
	\flushright \href{https://artofproblemsolving.com/community/c6h564061}{(Link to AoPS)}
\end{problem}



\begin{solution}[by \href{https://artofproblemsolving.com/community/user/29428}{pco}]
	\begin{tcolorbox}Given a function $f(x)=4x^3-3x$. Solve the equation: $ \underbrace{f{\circ}f{\circ}\cdots{\circ}f}_{2001\textrm{ times}}(x)=x $.\end{tcolorbox}
If $x>1$ then $f(x)>x>1$ and so no solution
If $x<-1$, then $f(x)<x<-1$ and so no solution

If $x\in[-1,+1]$. Let $x=\cos t$ so that $f^{[2001]}(x)=\cos 3^{2001}t$ and equation is $\cos 3^{2001}t=\cos t$

And so $3^{2001}t=\pm t+2k\pi$

Hence the solutions : $\boxed{\cos\frac{2k\pi}{3^{2001}\pm 1}}$ $\forall k\in\mathbb Z$
\end{solution}
*******************************************************************************
-------------------------------------------------------------------------------

\begin{problem}[Posted by \href{https://artofproblemsolving.com/community/user/125553}{lehungvietbao}]
	1) Find all continuous functions $f,g,h:\mathbb{R}\to\mathbb{R}$ such that
\[f(x^2)-f(y^2)=g(x+y)h(xy) \quad \forall x,y\in\mathbb{R}\]

2) Find all continuous functions $f,g,h:\mathbb{R}\to\mathbb{R}$ such that
\[f(x^2)+f(y^2)=g(x+y)\left (h(x)-h(y)  \right ) \quad \forall x,y\in\mathbb{R}\]
	\flushright \href{https://artofproblemsolving.com/community/c6h564189}{(Link to AoPS)}
\end{problem}



\begin{solution}[by \href{https://artofproblemsolving.com/community/user/29428}{pco}]
	\begin{tcolorbox}1) Find all continuous functions $f,g,h:\mathbb{R}\to\mathbb{R}$ such that
\[f(x^2)-f(y^2)=g(x+y)h(xy) \quad \forall x,y\in\mathbb{R}\]\end{tcolorbox}
Let $P(x,y)$ be the assertion $f(x^2)-f(y^2)=g(x+y)h(xy)$

Subtracting $P(y,x)$ from $P(x,y)$, we get $f(x^2)=f(y^2)$ and $g(x+y)h(xy)=0$ $\forall x$

So $f(x)=c$ $\forall x\ge 0$ and $f(x)$ is any value (but continuous) we want for $x<0$

If $g(x)=0$ $\forall x$, then $h(x)$ can be any value we want.
If $h(x)=0$ $\forall x$, then $g(x)$ can be any value we want.
If neither $g(x)$, neither $h(x)$ are allzero, then the equation is $g(u)h(v)=0$ $\forall u,v$ such that $u^2\ge 4v$

As a consequence $h(x)=0$ $\forall x\le 0$ and $\exists t=\inf\{x$ such that $h(x)\ne 0\}$ and $t>0$

And then we get $g(x)=0$ $\forall |x|\ge 2\sqrt t$ which is a sufficient condition.
Note that $t=0$ and continuity would imply $g(x)=0$ $\forall x$

\begin{bolded}Hence the answer\end{underlined}\end{bolded} :

About $f(x)$ : $f(x)$ is any continuous fonction constant over $[0,+\infty)$

About $g(x),h(x)$ :
- either $g(x)=0$ $\forall x$ and $h(x)$ is any continuous function we want.
- either $h(x)=0$ $\forall x$ and $g(x)$ is any continuous function we want.
- either $\exists t>0$ such that :;
... $g(x)$ is any continuous fonction such that $g(x)=0$ $\forall x$ such that $|x|\ge 2\sqrt t$
... $h(x)$ is any continuous function such that $h(x)=0$ $\forall x\le t$
\end{solution}



\begin{solution}[by \href{https://artofproblemsolving.com/community/user/29428}{pco}]
	\begin{tcolorbox}2) Find all continuous functions $f,g,h:\mathbb{R}\to\mathbb{R}$ such that
\[f(x^2)+f(y^2)=g(x+y)\left (h(x)-h(y)  \right ) \quad \forall x,y\in\mathbb{R}\]\end{tcolorbox}
Let $P(x,y)$ be the assertion $f(x^2)+f(y^2)=g(x+y)(h(x)-h(y))$

$P(x,x)$ $\implies$ $f(x^2)=0$ $\forall x$ and so $g(x+y)(h(x)-h(y))=0$ $\forall x,y$
If $g(x)=0$ $\forall x$, $h(x)$ can be any function we want.

If $g(x)$ is not allzero, and since continuous, $\exists a<b$ such that $g(x)\ne 0$ $\forall x\in (a,b)$

Let then $u\in (a,b)$ : $P(a-x,u+x-a)$ $\implies$ $h(a-x)=h(x+u-a)$ $\forall x$

Moving $u$ thru $(a,b)$, we get that $h(x)$ is constant over any inyerval $(x,x+b-a)$ and so is constant over $\mathbb R$

\begin{bolded}Hence the answer\end{underlined}\end{bolded} :

About $f(x)$ : $f(x)$ is any continuous function constant over $[0,+\infty)$

About $g(x),h(x)$ :
- Either $g(x)=0$ $\forall x$ and $h(x)$ can be any continuous function  we want
- Either $h(x)=c$ $\forall x$ and $g(x)$ can be any continuous function  we want
\end{solution}
*******************************************************************************
-------------------------------------------------------------------------------

\begin{problem}[Posted by \href{https://artofproblemsolving.com/community/user/125553}{lehungvietbao}]
	1) Find all functions $f:\mathbb R \rightarrow \mathbb R $ such that:
\[f(x^2(z^2+1)+f(y)(z+1))=1-f(z)(x^2+f(y))-z((1+z)x^2+2f(y)) \quad \forall x,y,z \in \mathbb R\]

2) Find all functions $f:\mathbb{R}\rightarrow \mathbb{R}$ such that 
\[f(y)+\sum_{i=1}^{2007}\left ( (-1)^kC_{2007}^{k}y^{2007-k}(f(x))^k\right  )=f(y-f(x)) \quad \forall x,y \in \mathbb R\]
	\flushright \href{https://artofproblemsolving.com/community/c6h564193}{(Link to AoPS)}
\end{problem}



\begin{solution}[by \href{https://artofproblemsolving.com/community/user/29428}{pco}]
	\begin{tcolorbox}1) Find all functions $f:\mathbb R \rightarrow \mathbb R $ such that:
\[f(x^2(z^2+1)+f(y)(z+1))=1-f(z)(x^2+f(y))-z((1+z)x^2+2f(y)) \quad \forall x,y,z \in \mathbb R\]\end{tcolorbox}
Let $P(x,y,z)$ be the assertion $f(x^2(z^2+1)+f(y)(z+1))=1-f(z)(x^2+f(y))-z((1+z)x^2+2f(y))$

$P(0,x,-1)$ $\implies$ $(2-f(1))f(x)=f(0)-1$

If $f(1)\ne 2$, then $f(x)=\frac{f(0)-1}{2-f(1)}$ is constant, which is never a solution. So $f(1)=2$ and $f(0)=1$

Then $P(y,0,0)$ $\implies$ $f(y^2+1)=-y^2$

And $P(x,y^2+1,0)$ $\implies$ $f(x^2-y^2)=1-(x^2-y^2)$

And so, since $x^2-y^2$ can take any value we want, $\boxed{f(x)=1-x}$ $\forall x$, which indeed is a solution.
\end{solution}



\begin{solution}[by \href{https://artofproblemsolving.com/community/user/29428}{pco}]
	\begin{tcolorbox}2) Find all functions $f:\mathbb{R}\rightarrow \mathbb{R}$ such that 
\[f(y)+\sum_{i=1}^{2007}\left ( (-1)^kC_{2007}^{k}y^{2007-k}(f(x))^k\right  )=f(y-f(x)) \quad \forall x,y \in \mathbb R\]\end{tcolorbox}
I suppose that index $i$ for the sum is in fact index $k$
$\boxed{f(x)=0}$ $\forall x$ is a solution. So let us from now look only for non allzero solutions.

Let $n=2007$
Let $P(x,y)$ be the assertion $f(y)+(y-f(x))^n-y^n=f(y-f(x))$ (simpler writing equivalent to the original one)
Let $a=f(0)$
Let $c=f(b)\ne 0$

$P(b,xf(b))$ $\implies$ $f(xc)+c^n((x-1)^n-x^n)=f(xc-c)$ and so :
$f(cx)-f(c(x-1))=c^n(x^n-(x-1)^n)$
$f(c(x-1))-f(cx)=-c^n(x^n-(x-1)^n)$
Since  $c\ne 0$ and  $x^n-(x-1)^n$ can take any value we want in $[2^{1-n},+\infty)$ (value $2^{1-n}$ is reached for $x=\frac 12$), we get that any number in $(-\infty,-2^{1-n}]\cup[2^{1-n},+\infty)$ may be wrtitten as $f(u)-f(v)$ for some $u,v$

$P(x,f(x))$ $\implies$ $f(f(x))-f(x)^n=a$
$P(y,f(x))$ $\implies$ $f(f(x)-f(y))=(f(x)-f(y))^n+(f(f(x))-f(x)^n)$ $=(f(x)-f(y))^n+a$

So $f(x)=x^n+a$ $\forall x\in(-\infty,-2^{1-n}]\cup[2^{1-n},+\infty)$

Let then $x\in\mathbb R$. It is always possible to choose some $y<-2^{1-n}$ small enough such that $f(y)=y^n+a<x-2^{1-n}$
Then $x-f(y)>2^{1-n}$ and so $f(x-f(y))=(x-f(y))^n+a$
Then $P(y,x)$ $\implies$ $f(x)=x^n+f(x-f(y))-(x-f(y))^n = x^n+a$

And so $\boxed{f(x)=x^{2007}+a}$ $\forall x$ which indeed is a solution
\end{solution}
*******************************************************************************
-------------------------------------------------------------------------------

\begin{problem}[Posted by \href{https://artofproblemsolving.com/community/user/68025}{Pirkuliyev Rovsen}]
	Find all the function $f: \mathbb{Z}\to\mathbb{Z}$ satisfying the following conditions
1) $f(f(m-n))=f(m^2)+f(n)-2nf(m)$ for all $m,n{\in}Z$. 2)$f(1)>0$.
	\flushright \href{https://artofproblemsolving.com/community/c6h564333}{(Link to AoPS)}
\end{problem}



\begin{solution}[by \href{https://artofproblemsolving.com/community/user/29428}{pco}]
	\begin{tcolorbox}Find all the function $f: \mathbb{Z}\to\mathbb{Z}$ satisfying the following conditions
1) $f(f(m-n))=f(m^2)+f(n)-2nf(m)$ for all $m,n{\in}Z$. 2)$f(1)>0$.\end{tcolorbox}
Let $P(x,y)$ be the assertion $f(f(x-y))=f(x^2)+f(y)-2yf(x)$
Let $a=f(1)>0$

$P(1,1)$ $\implies$ $f(f(0))=0$
$P(0,0)$ $\implies$ $f(f(0))=2f(0)$ and so $f(0)=0$

$P(-x,0)$ $\implies$ $f(f(-x))=f(x^2)$
$P(0,x)$ $\implies$ $f(f(-x))=f(x)$  and so $f(x)=f(x^2)$ and $f(-x)=f(x^2)=f(x)$

$P(x,y)$ may then be written as new assertion $Q(x,y)$ : $f(f(x-y))=f(x)+f(y)-2yf(x)$

$Q(x,1)$ $\implies$ $f(f(x-1))=a-f(x)$
$Q(1,x)$ $\implies$ $f(f(1-x))=a+f(x)-2ax$
And so, since $f(x-1)=f(1-x)$  : $a-f(x)=a+f(x)-2ax$ and so $f(x)=ax$ which is not a solution since $a\ne 0$

So no solution for this equation.
\end{solution}
*******************************************************************************
-------------------------------------------------------------------------------

\begin{problem}[Posted by \href{https://artofproblemsolving.com/community/user/125553}{lehungvietbao}]
	1) Find all functions $ f : ( 0 ; 1) \to ( 0 ; 1) $ such that : $ f \left( \dfrac{1}{2} \right) = \dfrac{1}{2}$  and $ (f(ab))^2 = ( af(b) + f(a)) ( f(b) + bf(a)) \ \ \forall a ; b \ \ \in ( 0 ; 1) $

2) Find all functions $f:\mathbb{R}\to \mathbb{R}$ such that $\frac{1+f(x)}{f^{2}(x)}=\sqrt[3]{f(x+1)}$
	\flushright \href{https://artofproblemsolving.com/community/c6h564343}{(Link to AoPS)}
\end{problem}



\begin{solution}[by \href{https://artofproblemsolving.com/community/user/29428}{pco}]
	\begin{tcolorbox}1) Find all functions $ f : ( 0 ; 1) \to ( 0 ; 1) $ such that : $ f \left( \dfrac{1}{2} \right) = \dfrac{1}{2}$  and $ (f(ab))^2 = ( af(b) + f(a)) ( f(b) + bf(a)) \ \ \forall a ; b \ \ \in ( 0 ; 1) $\end{tcolorbox}
Let $P(x,y)$ be the assertion $f(xy)^2=(xf(y)+f(x))(f(y)+yf(x))$

$P(x,x)$ $\implies$ $f(x^2)=(x+1)f(x)$

$f(x^2)<1$ $\implies$ $f(x)<\frac 1{1+x}$

$\implies$ $f(x^2)<\frac 1{1+x^2}$ $\implies$ $f(x)<\frac 1{(1+x)(1+x^2)}$

$\implies$ $f(x^2)<\frac 1{(1+x^2)(1+x^4)}$ $\implies$ $f(x)<\frac 1{(1+x)(1+x^2)(1+x^4)}$

And so $f(x)<\frac 1{\prod_{k=0}^n(1+x^{2^k})}$ $=\frac{1-x}{1-x^{2^{n+1}}}$

Setting $n\to +\infty$, we get $f(x)\le 1-x$

Suppose now $\exists u\in(0,1)$ such that $f(u)<1-u$ :

So $uf(\frac 1{2u})+f(u)<u(1-\frac 1{2u})+1-u=\frac 12$

And $f(\frac 1{2u})+\frac 1{2u}f(u)<1-\frac 1{2u}+\frac 1{2u}(1-u)=\frac 12$

Then $P(u,\frac 1{2u})$ $\implies$ $f(\frac 12)^2<\frac 14$, which is impossible

So $\boxed{f(x)=1-x}$ $\forall x\in(0,1)$, which indeed is a solution.
\end{solution}



\begin{solution}[by \href{https://artofproblemsolving.com/community/user/29428}{pco}]
	\begin{tcolorbox}2) Find all functions $f:\mathbb{R}\to \mathbb{R}$ such that $\frac{1+f(x)}{f^{2}(x)}=\sqrt[3]{f(x+1)}$\end{tcolorbox}
Certainly not a real olympiad exercise again :(

Infinitely many solutions which may be built piece per piece without global closed form. For example :

Define $f(x)>0$ as you want over $[0,1)$

Extend $f(x)$ for $x\ge 1$ thru formula $f(x+1)= \left(\frac{1+f(x)}{f(x)^2}\right)^3$

Extend $f(x)$ for $x<0$ thru formula $f(x-1)=\frac{1+\sqrt{4\sqrt[3]{f(x)}+1}}{2\sqrt[3]{f(x)}}$ (note that we always choose the positive root. Choosing the negative one may be possible in some cases and gives some other solutions)
\end{solution}
*******************************************************************************
-------------------------------------------------------------------------------

\begin{problem}[Posted by \href{https://artofproblemsolving.com/community/user/125553}{lehungvietbao}]
	1) Find all functions $f:\mathbb{N}^{*}\to \mathbb{N}^{*}$ such that :

\[a.f(a)+b.f(b)+2ab\,\,\text{is perfect square } \quad \forall  a,b\in \mathbb{N}^{*}\]


2)   Let $f:\mathbb{N}\rightarrow \mathbb{N}$ be a function satisfies the following condition
 \[f\left ( xy+1 \right )=xf\left ( y \right )+2012\quad \forall x,y\in \mathbb{N}\]
Prove that $\sum_{i=1}^{2012}f^{3}\left ( i \right )>\frac{1}{4}\left ( 2012 \right )^{7}$.
	\flushright \href{https://artofproblemsolving.com/community/c6h564345}{(Link to AoPS)}
\end{problem}



\begin{solution}[by \href{https://artofproblemsolving.com/community/user/29428}{pco}]
	What are $\mathbb N$ and $\mathbb N^*$ in these problems ?
\end{solution}



\begin{solution}[by \href{https://artofproblemsolving.com/community/user/125553}{lehungvietbao}]
	Dear Mr.Patrick
Sorry for my carelessness! 
Note that $\mathbb N=\{0,1,2,...\}$ or $ 0\in\mathbb N$ and $ \mathbb{N}^*=\{1,2,3...\} $or $ 0\notin \mathbb {N}^*$ in our problems.
\end{solution}



\begin{solution}[by \href{https://artofproblemsolving.com/community/user/119240}{Altricono}]
	1) Note, first, that $af(a)$ is a quadratic residue modulo $b$ for all natural numbers $b$. Hence, $af(a)$ must be a perfect square. Thus, in particular, if $a$ is a prime we may write: $f(a) = k^2a$.

Thus, for any prime $a$, there exists a constant $k$ such that  $k^2a^2 + f(1) + 2a$ is a perfect square. Since $k^2a^2 + f(1) + 2a > k^2a^2$, it follows that $k^2a^2 + f(1) + 2a \ge (ka +1)^2 \Rightarrow f(1) - 1 \ge 2a(k-1)$. Hence, for all sufficiently large primes $a$, $k = 1$ and $f(a) = a$.

Hence, for a fixed $b$, $a^2 + bf(b) + 2ab$ is a perfect square for infinitely many $a$. Thus, we have $(a + b)^2 + bf(b) - b^2 = r^2$ for infinitely many $a$. Thus, it follows that the equation $r^2 - l^2 = bf(b) - b^2$ has infinitely many solutions $(r, l)$. Hence, $bf(b) - b^2 = 0$ and it follows that $f(x) = x$.
\end{solution}



\begin{solution}[by \href{https://artofproblemsolving.com/community/user/29428}{pco}]
	\begin{tcolorbox}2)   Let $f:\mathbb{N}\rightarrow \mathbb{N}$ be a function satisfies the following condition
 \[f\left ( xy+1 \right )=xf\left ( y \right )+2012\quad \forall x,y\in \mathbb{N}\]
Prove that $\sum_{i=1}^{2012}f^{3}\left ( i \right )>\frac{1}{4}\left ( 2012 \right )^{7}$.\end{tcolorbox}
Let $a=2012$
Let $P(x,y)$ be the assertion $f(xy+1)=xf(y)+a$

$P(0,0)$ $\implies$ $f(1)=a$
$P(1,0)$ $\implies$ $f(0)=0$
$P(x,1)$ $\implies$ $f(x+1)=a(x+1)$ and so $f(x)=ax$ $\forall x\ge 1$, still true when $x=0$
So $f(x)=ax$ $\forall x$, which indeed is a solution.

So $\sum_{i=1}^af(i)^3$ $=a^3\sum_{i=1}^ai^3$ $=a^3\left(\frac{a(a+1)}2\right)^2$ $=\frac{a^5(a+1)^2}4$ $>\frac{a^7}4$

Q.E.D.
\end{solution}
*******************************************************************************
-------------------------------------------------------------------------------

\begin{problem}[Posted by \href{https://artofproblemsolving.com/community/user/190536}{DonaldLove}]
	find $f:N \to N$ such that $f(f(m^2)+2f(n^2))=m^2+2n^2 \forall m,n \in N$

N is including 0
	\flushright \href{https://artofproblemsolving.com/community/c6h564346}{(Link to AoPS)}
\end{problem}



\begin{solution}[by \href{https://artofproblemsolving.com/community/user/29428}{pco}]
	\begin{tcolorbox}find $f:N \to N$ such that $f(f(m^2)+2f(n^2))=m^2+2n^2 \forall m,n \in N$

N is including 0\end{tcolorbox}
Is it a real olympiad exercise ?
There are infinitely many solutions and I dont think there is a global common form for all of them.

Example 1 : trivial solution $f(n)=n$

Example 2 :
$f(n^2)=3^{2(n+1)}+1$
$f(3^{2(m+1)}+2\times 3^{2(n+1)}+3)=m^2+2n^2$
$f(n)=\lfloor n^3\sin^2(n)\rfloor$ for any other $n$
Note, in order to understand this example, that :
1) $g(m,n)=3^{2(m+1)}+2\times 3^{2(n+1)}+3$ is an injective function from $(\mathbb N\cup\{0\})^2\to\mathbb N$
2) $g(m,n)$ is never a square (look at modulo $9$)

And infinitely many examples with the same principles that example 2 may be built.
\end{solution}



\begin{solution}[by \href{https://artofproblemsolving.com/community/user/8994}{Number1}]
	\begin{tcolorbox}[quote="DonaldLove"]find $f:N \to N$ such that $f(f(m^2)+2f(n^2))=m^2+2n^2 \forall m,n \in N$

N is including 0\end{tcolorbox}
Is it a real olympiad exercise ?
There are infinitely many solutions and I dont think there is a global common form for all of them.



And infinitely many examples with the same principles that example 2 may be built.\end{tcolorbox}

''Very similary''

http://www.artofproblemsolving.com/Forum/viewtopic.php?p=1484888&sid=877aaae6f3a18eed0c55a8224142e5be#p1484888
\end{solution}



\begin{solution}[by \href{https://artofproblemsolving.com/community/user/29428}{pco}]
	\begin{tcolorbox}...
''Very similary''
...
\end{tcolorbox}
Ohhh no. The one you show is quite different since we can establish injectivity. In the current one, there is no injectivity.

You cant write that $f(x^2)$ is "very similar" to $f^2(x)$ :)
\end{solution}



\begin{solution}[by \href{https://artofproblemsolving.com/community/user/8994}{Number1}]
	\begin{tcolorbox}

You cant write that $f(x^2)$ is "very similar" to $f^2(x)$ :)\end{tcolorbox}

Of course, you are right. But at first I did not even see the difference.
\end{solution}
*******************************************************************************
-------------------------------------------------------------------------------

\begin{problem}[Posted by \href{https://artofproblemsolving.com/community/user/187039}{malilim}]
	find all the funtion $f:\mathbb{R} \to \mathbb{R^+}$ that 
i, $f(x^2)=(f(x))^2-2xf(x)$
ii, $f(-x)=f(x-1)$
iii, if $1<x<y$ then $f(x)<f(y)$
	\flushright \href{https://artofproblemsolving.com/community/c6h564355}{(Link to AoPS)}
\end{problem}



\begin{solution}[by \href{https://artofproblemsolving.com/community/user/29428}{pco}]
	\begin{tcolorbox}find all the funtion $f:\mathbb{R} \to \mathbb{R^+}$ that 
i, $f(x^2)=(f(x))^2-2xf(x)$
ii, $f(-x)=f(x-1)$
iii, if $1<x<y$ then $f(x)<f(y)$\end{tcolorbox}
Let $P(x)$ be the assertion $f(x^2)=f(x)^2-2xf(x)$
Let $Q(x)$ be the assertion $f(-x)=f(x-1)$

$P(x)$ $\implies$ $f(x^2)=f(x)^2-2xf(x)$
$P(-x)$ $\implies$ $f(x^2)=f(-x)^2+2xf(-x)$
Subtracting, we get $(f(x)-x)^2=(f(-x)+x)^2$ and so $f(-x)=f(x)-2x$ (the other case would be $f(-x)=-f(x)$, impossible since $f(x)>0$ $\forall x$)

So $f(x-1)=f(-x)=f(x)-2x$

So $f(x+1)=f(x)+2x+2$

So (simple induction) : $f(x+n)=f(x)+2nx+n^2+n$

$P(1)$ $\implies$ $f(1)=3$ (since $f(1)>0$) and so $f(n)=n^2+n+1$

Let $A=\{x\in\mathbb R$ such that $f(x)=x^2+x+1\}$
We easily get from previous calculus :
$\mathbb Z\subseteq A$
$x\in A$ $\implies$ $x+n\in A$
$x\in A$ $\implies$ $-x\in A$
$x\ge 0\in A$ $\implies$ $\sqrt x\in A$ (using $P(x)$ and the fact that $f(x)>0$)

From there, it's easy to show that $A$ is dense in $\mathbb R$

Then iii implies $[1,+\infty)\subseteq A$
And $x\in A$ $\implies$ $x+n\in A$ ends the proof that $A=\mathbb R$

Hence the result :$\boxed{f(x)=x^2+x+1}$ $\forall x$, which indeed is a solution.
\end{solution}
*******************************************************************************
-------------------------------------------------------------------------------

\begin{problem}[Posted by \href{https://artofproblemsolving.com/community/user/125553}{lehungvietbao}]
	1) Find all continuous functions $f,g,h:\mathbb{R}\to\mathbb{R}$ such that $f(0)\neq 0$ and 
\[f(xy)=g(x+y)h(x-y) \quad \forall x,y\in\mathbb{R}\]

2) Find all continuous functions $f,g,h:\mathbb{R}\to\mathbb{R}$ such that 
\[f\left ( x^{2} +y^{2}\right )=g\left (x+y  \right )h\left ( x-y \right ) \quad \forall x,y\in\mathbb{R}\]
	\flushright \href{https://artofproblemsolving.com/community/c6h564898}{(Link to AoPS)}
\end{problem}



\begin{solution}[by \href{https://artofproblemsolving.com/community/user/29428}{pco}]
	\begin{tcolorbox}1) Find all continuous functions $f,g,h:\mathbb{R}\to\mathbb{R}$ such that $f(0)\neq 0$ and 
\[f(xy)=g(x+y)h(x-y) \quad \forall x,y\in\mathbb{R}\]\end{tcolorbox}
Let $P(x,y)$ be the assertion $f(xy)=g(x+y)h(x-y)$
If $g(u)=0$, then $P(u,0)$ $\implies$ $f(0)=0$, impossible, so $g(x)$, since continuous, has a constant sign
$(f,g,h)$ solution implies $(abf,ag,bh)$ solution and so WLOG $f(0)=1$ and $g(x)>0$ $\forall x$

$P(x,0)$ $\implies$ $h(x)=\frac 1{g(x)}$ and $P(x,y)$ becomes $f(xy)=\frac{g(x+y)}{g(x-y)}$

Comparing then $P(x,y)$ with $P(xy,1)$ we get $\frac{g(x+y)}{g(x-y)}=\frac{g(xy+1)}{g(xy-1)}$

For easier writing, let $v(x)=\ln(g(x))$ and the problem becomes :
New assertion $Q(x,y)$ : $v(x+y)-v(x-y)=v(xy+1)-v(xy-1)$

Let $u(x)=v(\frac xp)$
Comparing $Q(\frac mp,\frac np)$ with $Q(\frac{mn}p,\frac 1p)$, we get $u(m+n)-u(m-n)=u(mn+1)-u(mn-1)$

So let us for a moment consider the problem :
-------------- begin of integer problem ---------------------
Find all functions $u(x)$ from $\mathbb Z\to \mathbb R$ such that :
$Q(m,n)$ : $u(m+n)-u(m-n)=u(mn+1)-u(mn-1)$ $\forall m,n\in\mathbb Z$
Swapping $m$ and $n$, we get that $u(-x)=u(x)$ and $u(x)$ is an even function.

Subtracting $Q(4,1)$ from $Q(2,2)$, we get $u(5)=u(4)+u(3)-u(0)$
Subtracting $Q(6,1)$ from $Q(3,2)$, we get $u(7)=2u(5)-u(1)$ $=2u(4)+2u(3)-u(1)-2u(0)$
Subtracting $Q(6,2)$ from $Q(4,3)$, we get $u(8)=u(7)+u(4)-u(1)$ $=3u(4)+2u(3)-2u(1)-2u(0)$
Subtracting $Q(8,1)$ from $Q(4,2)$, we get $u(9)=u(7)+u(6)-u(2)$ $=u(6)+2u(4)+2u(3)-u(2)-u(1)-2u(0)$
Subtracting $Q(9,1)$ from $Q(3,3)$, we get $u(10)=u(8)+u(6)-u(0)$ $=u(6)+3u(4)+2u(3)-2u(1)-3u(0)$
Subtracting $Q(10,1)$ from $Q(5,2)$, we get $u(11)=u(9)+u(7)-u(3)$ $=u(6)+4u(4)+3u(3)-u(2)-2u(1)-4u(0)$

From there :
Subtracting $Q(2m-2,2)$ from $Q(m-1,4)$, we get $u(2m)=u(2m-4)+u(m+3)-u(m-5)$
Subtracting $Q(2m,1)$ from $Q(m,2)$, we get $u(2m+1)=u(2m-1)+u(m+2)-u(m-2)$

And so knowledge of $u(0),u(1),u(2),u(3), u(4),u(6)$ is enough to know $u(n)$ $\forall n\in\mathbb Z$
So the set of solutions is a vectorspace whose dimension is at most $6$
And it is easy to check that the six following functions are independant solutions :
$u_1(n)=1$ $\forall n$
$u_2(n)=1$ if $n\equiv 0\pmod 2$ and $u_2(n)=0$ otherwise
$u_3(n)=1$ if $n\equiv 0\pmod 3$ and $u_3(n)=0$ otherwise
$u_4(n)=1$ if $n\equiv 0\pmod 4$ and $u_4(n)=0$ otherwise
$u_5(n)=1$ if $n\equiv 2\pmod 4$ and $u_5(n)=0$ otherwise
$u_6(n)=n^2$ $\forall n$
So dimension is $6$ and we got all the solutions of the simplified problem :

$u(x)=ax^2+b+c\times u_2(x)+d\times u_3(x)+e\times u_4(x)+f\times u_5(x)$
-------------- end of integer problem ---------------------

Back to previous problem, we get : $v(\frac xp)=a_px^2+b_p+c_pu_2(x)+d_pu_3(x)+e_pu_4(x)+f_pu_5(x)$ $\forall x\in\mathbb Z$

Choosing $x=kp$, we get $v(k)=a_pk^2p^2+b_p+c_pu_2(kp)+d_pu_3(kp)+e_pu_4(kp)+f_pu_5(kp)$ and so $a_pp^2=a$ $\forall p$

Note that $u_1,u_2,u_3,u_4,u_5$ all have a period $12$
Then $v(\frac {x+12k}p)-v(\frac xp)=a(\frac{x+12k}p)^2-a(\frac xp)^2$
So $v(\frac {x+12np}{12nq})-v(\frac x{12nq})=a(\frac{x+12np}{12nq})^2-a(\frac x{12nq})^2$

Setting $n\to+\infty$ and using continuity, we get $v(\frac pq)=a(\frac pq)^2+v(0)$

And, using continuity again, $v(x)=ax^2+b$ $\forall x\in\mathbb R$

So $g(x)=e^{ax^2+b}$

\begin{bolded}Hence the general result\end{underlined}\end{bolded} :
$f(x)=\alpha\beta e^{4ax}$
$g(x)=\alpha e^{ax^2}$
$h(x)=\beta e^{-ax^2}$
Which indeed is a solution, whatever are $\alpha,\beta\in\mathbb R\setminus\{0\}$
\end{solution}



\begin{solution}[by \href{https://artofproblemsolving.com/community/user/29428}{pco}]
	\begin{tcolorbox}2) Find all continuous functions $f,g,h:\mathbb{R}\to\mathbb{R}$ such that 
\[f\left ( x^{2} +y^{2}\right )=g\left (x+y  \right )h\left ( x-y \right ) \quad \forall x,y\in\mathbb{R}\]\end{tcolorbox}
Let $P(x,y)$ be the assertion $f(x^2+y^2)=g(x+y)h(x-y)$

1) If $g(0)=0$
$P(x,-x)$ $\implies^$ $f(2x^2)=0$ and so $f(x)=0$ $\forall x\ge 0$
Then, if $g(a)\ne 0$ and $h(b)\ne 0$ for some $a,b$, $P(\frac {a+b}2,\frac{a-b}2)$ $\implies$ $g(a)h(b)=0$, contradiction.
So :
Either $g(x)=0$ $\forall x$ \begin{bolded}and the solution\end{underlined}\end{bolded} $f\equiv 0$ over $\mathbb R_{\ge 0}$, $g\equiv 0$, any $h$
Either $h(x)=0$ $\forall x$ \begin{bolded}and the solution\end{underlined}\end{bolded} $f\equiv 0$ over $\mathbb R_{\ge 0}$, any g, $h\equiv 0$ including $g(0)\ne 0$

2) If $h(0)=0$
$P(x,x)$ $\implies^$ $f(2x^2)=0$ and so $f(x)=0$ $\forall x\ge 0$
Then, if $g(a)\ne 0$ and $h(b)\ne 0$ for some $a,b$, $P(\frac {a+b}2,\frac{a-b}2)$ $\implies$ $g(a)h(b)=0$, contradiction.
So :
Either $g(x)=0$ $\forall x$ \begin{bolded}and the solution\end{underlined}\end{bolded} $f\equiv 0$ over $\mathbb R_{\ge 0}$, $g\equiv 0$, any $h$ including $h(0)\ne 0$
Either $h(x)=0$ $\forall x$ \begin{bolded}and the solution\end{underlined}\end{bolded} $f\equiv 0$ over $\mathbb R_{\ge 0}$, any g, $h\equiv 0$ 

3) If $g(0),h(0)\ne 0$
$P(x,x)$ $\implies$ $f(2x^2)=g(2x)h(0)$
$P(x,-x)$ $\implies$ $f(2x^2)=g(0)h(2x)$
And so $h(x)=ag(x)$  with $a\ne 0$

Since $(f,g,h)$ solution implies $(af,ag,h)$ solution, WLOG $a=1$

$h(x)=g(x)$
$P(x,y)$ becomes $f(x^2+y^2)=g(x+y)g(x-y)$
$P(x,0)$ $\implies$ $f(x^2)=g(x)^2$ and so $f(x)\ge 0$ $\forall x$

$P(x,y)$ $\implies$ $g(\sqrt{x^2+y^2})^2=g(x+y)g(x-y)$

If $g(u)=0$ for some $u$, then $P(\frac u2,\frac u2)$ $\implies$ $g(\frac{|u|}{\sqrt 2})=0$  and so  $g(\frac{|u|}{(\sqrt 2})^n)=0$ 
Continuity implies then, setting $n\to +\infty$ that $g(0)=0$, impossible.

So $g(x)\ne 0$ $\forall x$ and since $g$ solution implies $-g$ solution, WLOG say $g(x)>0$ $\forall x$

Let then $u(x)=\ln(g(x))$ and assertion becomes $2u(\sqrt{x^2+y^2})=u(x+y)+u(x-y)$
Obviously, $u(x)$ is even and we can write $u(x)=v(x^2)$ for some $v(x)$ and so $2v(x^2+y^2)=v((x+y)^2)+v((x-y)^2)$

Setting $x=\frac{a+b}2$ and $y=\frac {a-b}2$, this implies $v(\frac{a^2+b^2}2)=\frac{v(a^2)+v(b^2)}2$

Continuity implies then $v(x)=ax+b$ $\forall x\ge 0$ and so $u(x)=ax^2+b$ with $a\ge 0$, $b>0$

\begin{bolded}Hence the solution\end{underlined} \end{bolded}: $f\equiv \alpha\beta e^{2ax}$ over $\mathbb R_{\ge 0}$, $g(x)=\alpha e^{ax^2}$ and $h(x)=\beta e^{ax^2}$ which indeed is a solution, whatever are $\alpha,\beta\ne 0$
\end{solution}
*******************************************************************************
-------------------------------------------------------------------------------

\begin{problem}[Posted by \href{https://artofproblemsolving.com/community/user/125553}{lehungvietbao}]
	1) Find all continuous functions $f,g,h:\mathbb{R}\to\mathbb{R}$ such that 
\[f(x+y)+g(x-y)=h(xy)\quad \forall x,y\in\mathbb{R}\]

2) Find all continuous functions $f,g:\mathbb{R}\to\mathbb{R}$ such that 
\[f(x^2)-f(y^2)=g(x-y)g(x+y) \quad \forall x,y\in\mathbb{R}\]
	\flushright \href{https://artofproblemsolving.com/community/c6h564899}{(Link to AoPS)}
\end{problem}



\begin{solution}[by \href{https://artofproblemsolving.com/community/user/29428}{pco}]
	\begin{tcolorbox}1) Find all continuous functions $f,g,h:\mathbb{R}\to\mathbb{R}$ such that 
\[f(x+y)+g(x-y)=h(xy)\quad \forall x,y\in\mathbb{R}\]\end{tcolorbox}
Let $P(x,y)$ be the assertion $f(x+y)+g(x-y)=h(xy)$

$P(x,0)$ $\implies$ $g(x)=h(0)-f(x)$ and $P(x,y)$ becomes $f(x+y)-f(x-y)=h(xy)-h(0)$

$P(xy,1)$ $\implies$ then $f(xy+1)-f(xy-1)=h(xy)-h(0)$

And so $f(x+y)-f(x-y)=f(xy+1)-f(xy-1)$

Looking at http://www.artofproblemsolving.com/Forum/viewtopic.php?f=36&t=564898 for the solution, we get $f(x)=ax^2+b$

\begin{bolded}Hence the solutions\end{underlined}\end{bolded} :
$f(x)=ax^2+b$ $\forall x$
$g(x)=-ax^2+c$$\forall x$
$h(x)=4ax+b+c$ $\forall x$
\end{solution}



\begin{solution}[by \href{https://artofproblemsolving.com/community/user/125553}{lehungvietbao}]
	Dear Mr.Patrick 
How about problem 2 ? :)
\end{solution}



\begin{solution}[by \href{https://artofproblemsolving.com/community/user/29428}{pco}]
	\begin{tcolorbox}2) Find all continuous functions $f,g:\mathbb{R}\to\mathbb{R}$ such that 
\[f(x^2)-f(y^2)=g(x-y)g(x+y) \quad \forall x,y\in\mathbb{R}\]\end{tcolorbox}
Setting $y=0$, we get $f(x^2)=f(0)+g(x)^2$ and so equation is $g(x)^2-g(y)^2=g(x+y)g(x-y)$
$g(x)=0$ $\forall x$ is a solution. So let us from now look only for non allzero solutions.

Let $P(x,y)$ be the assertion $g(x)^2-g(y)^2=g(x+y)g(x-y)$
Let $u$ such that $g(u)\ne 0$

$P(0,0)$ $\implies$ $g(0)=0$ (and so $u\ne 0$)

$P(0,x)$ $\implies$  $g(x)(g(x)+g(-x))=0$
$P(0,-x)$ $\implies$  $g(-x)(g(x)+g(-x))=0$
Adding, we get $g(-x)=-g(x)$ $\forall x$ and $g(x)$ is odd.

1) Preliminary results
==============
Let $y$ such that $g(y)\ne 0$

$P(\frac x2+y,\frac x2)$ $\implies$ $g(\frac x2+y)^2-g(\frac x2)^2=g(x+y)g(y)$
$P(\frac x2,\frac x2-y)$ $\implies$ $g(\frac x2)^2-g(\frac x2-y)^2=g(x-y)g(y)$
$P(\frac x2-y,\frac x2+y)$ $\implies$ $g(\frac x2-y)^2-g(\frac x2+y)^2=-g(x)g(2y)$
Adding, we get  $g(x+y)+g(x-y)=g(x)\frac{g(2y)}{g(y)}$

So (setting there $x=(n+1)y$) : $g((n+2)y)=g((n+1)y)\frac{g(2y)}{g(y)}-g(ny)$

1.1) If $g(y)\ne 0$ and $\left|\frac{g(2y)}{g(y)}\right|< 2$
-------------------------------------------------------------
Let then $g(2y)=2g(y)\cos\alpha$
Equation is $g((n+2)y)=2g((n+1)y)\cos \alpha-g(ny)$  and we get $g(ny)=g(y)\frac{\sin n\alpha}{\sin \alpha}$
Note then that $\{g(ny)\}$ does not have a constant signe and so continuity implies that $g(t)=0$ for some $t\ne 0$
Then $P(x+t,x)$ $\implies$ $g(x+t)^2=g(x)^2$ and so $g(x)^2$ is periodic, and so bounded. 
So $g(x)$ is bounded too and is $zero$ for some nonzero values of $x$.

1.2) If $g(y)\ne 0$ and $\frac{g(2y)}{g(y)}= 2$
----------------------------------------------
Equation is $g((n+2)y)=g((n+1)y)-g(ny)$We get $g(ny)=ng(y)$
Note then that $\lim_{n\to+\infty}|g(ny)|=+\infty$ and so $g(x)$ is not bounded.

1.3) If $g(y)\ne 0$ and $\frac{g(2y)}{g(y)}= -2$
----------------------------------------------
Equation is $g((n+2)y)=-g((n+1)y)-g(ny)$We get $g(ny)=(-1)^{n+1}ng(y)$
Note then that $\lim_{n\to+\infty}|g(ny)|=+\infty$ and so $g(x)$ is not bounded.
Note then that $\{g(ny)\}$ does not have a constant signe and so continuity implies that $g(t)=0$ for some $t\ne 0$
Then $P(x+t,x)$ $\implies$ $g(x+t)^2=g(x)^2$ and so $g(x)^2$ is periodic, and so bounded, and so contradiction.
So no solution in this case.

1.4) If $g(y)\ne 0$ and $\left|\frac{g(2y)}{g(y)}\right|> 2$
-------------------------------------------------------------
The quadratic $x^2-\frac{g(2y)}{g(y)}x+1=0$ has two real roots $r$ and $\frac 1r$, with $|r|>1$

And we get $g(ny)=\frac{rg(y)}{r^2-1}(r^n-r^{-n})$
Note then that $\lim_{n\to+\infty}|g(ny)|=+\infty$ and so $g(x)$ is not bounded.


2) If $\exists t\ne 0$ such that $g(t)=0$, then $g(x)=c\sin bx$ for some $b,c\ne 0$
=====================================================
Since odd, WLOG $t>0$
Let $A=\{x>0$ such that $g(x)=0\}$ and $a=\inf (A)$ (which exists since $A\ne\emptyset$)
let $x\in A$ : $P((n+1)x,nx)$ $\implies$ $g((n+1)x)^2=g(nx)^2$ and so $g(nx)=0$ $\forall n\in\mathbb N$. So $x\in A$ $\implies$ $nx\in A$
If $a=0$, then $A$ is dense in $\mathbb R^+$ and so, using continuity and the fact that $g(x)$ is odd, $g(x)=0$ $\forall x$
So $a>0$
$g(x)$ has constant sign over $(0,a)$ and since $g(x)$ solution implies $-g(x)$ solution, WLOG $g(x)>0$ $\forall x\in(0,a)$

Let $y=\frac a{2^n}$ with $n\ge 2$ so that $g(y)\ne 0$ and $g(2y)\ne 0$
$P(x+t,x)$ $\implies$ $g(x+t)^2=g(x)^2$ and so $g(x)^2$ is periodic, and so bounded. So, according to 1.2, 1.3, 1.4) above, we get that $\left|\frac{g(2y)}{g(y)}\right|< 2$ and so (see 1.1) :
$g(\frac{ka}{2^n})=g(\frac a{2^n})\frac{\sin k\alpha_n}{\sin \alpha_n}$ where $\alpha_n$ is such that $g(\frac a{2^{n-1}})=2g\frac a{2^n}y)\cos\alpha_n$ and $\alpha_n\in(0,\frac{\pi}2)$

Setting $k=2^n$, we get $\sin (2^n\alpha_n)=0$ and so $\alpha_n=m_n\frac{\pi}{2^n}$ with some $m_n\in\{1,2,...,2^{n-1}-1\}$
But, if $m_n>1$, we'll get some $g(\frac{ka}{2^n})\le 0$ with $k\in\{1,2,3,...,2^n-1\}$ and so $m_n=1$ and $\alpha_n=\frac{\pi}{2^n}$

So $g(\frac{ka}{2^n})=g(\frac a{2^n})\frac{\sin k\frac{\pi}{2^n}}{\sin \frac{\pi}{2^n}}$


Setting $k=2^{n-1}$, we get $g(\frac a2)=g(\frac a{2^n})\frac{1}{\sin \frac{\pi}{2^n}}$

And so $g(\frac{ka}{2^n})=g(\frac a2)\sin{k\frac{\pi}{2^n}}$ $\forall k\ge 0$

And continuity implies then $g(x)=g(\frac a2)\sin \frac{\pi}ax)$ $\forall x\ge 0$ and, since odd $g(x)=c\sin bx$ which indeed is a solution.
Q.E.D.

3) If $\not\exists t\ne 0$ such that $g(t)=0$
=============================
Since $g(x)$ solution implies $-g(x)$ solution, WLOG $g(x)>0$ $\forall x>0$

Let $a>0$ and $n\ge 2$

Let $c_n=\frac {g(\frac a{2^{n-1}})}{g(\frac a{2^n})}>0$ : We need, according to 1) : $c_n\ge 2$

3.1) If $c_n=2$, then $g(x)=cx$ $\forall x$ for some $c\ne 0$
-----------------------------------------------------------
Then, according to 1.2), $g(k\frac a{2^n})=kg(\frac a{2^n})$ $\forall k\ge 0$
Setting $k=2^n$, we get $g(a)=2^ng(\frac a{2^n})$ 

And so $g(k\frac a{2^n})=g(a)k\frac 1{2^n}$ $\forall k\ge 0$

And continuity implies $g(x)=\frac{g(a)}ax$ $\forall x\ge 0$ and so, since odd, $g(x)=cx$ $\forall x$, which indeed is a solution
Q.E.D.

3.2) If $c_n>2$, then $g(x)=c\sinh bx$ for some $b,c\ne 0$
---------------------------------------------------------
Applying 1.4) above, we get $g(k\frac a{2^n})=\frac{r_ng(\frac a{2^n})}{r_n^2-1}(r_n^k-r_n^{-k})$
Where $r_n$ is the root of quadratic $x^2-c_nx+1=0$ such that $|r_n|>1$

Setting $k=2^n$, we get $g(a)=\frac{r_ng(\frac a{2^n})}{r_n^2-1}(r_n{}^2{}^n-r_n^{-2^n})$

Setting $k=2^{n+1}$, we get $g(2a)=\frac{r_ng(\frac a{2^n})}{r_n^2-1}(r_n{}^{2{}^{n+1}-r_n^{-2^{n+1}})}$

So $\frac{g(2a)}{g(a)}=r_n{}^2{}^n+r_n^{-2^n}$ and so $r_n{}^2{}^n=r_0$ and $\frac{r_ng(\frac a{2^n})}{r_n^2-1}=\frac{g(a)}{r_0-r_0^{-1}}$

And so $g(k\frac a{2^n})=\frac{g(a)}{r_0-r_0^{-1}}(r_0^{\frac k{2^n}}-r_0^{-\frac k{2^n}})$

And continuity implies $g(x)=\frac{g(a)}{r_0-r_0^{-1}}(r_0^{\frac xa}-r_0^{-\frac xa})$ $\forall x\ge 0$

And so, since odd, $g(x)=c\sinh bx$ $\forall x$, which indeed is a solution, whatever are $b,c\ne 0$
Q.E.D;

4) Synthesis of solutions
=================
Let $a,c\in\mathbb R$ 
Let $h(x)$ any continuous function from $\mathbb R\to\mathbb R$,
We got three families of solution.

$f(x)=h(0)+c^2x$ $\forall x\ge 0$ and $f(x)=h(x)$ $\forall x\le 0$
$g(x)=cx$

$f(x)=h(0)+c^2\sin^2 a\sqrt x$ $\forall x\ge 0$ and $f(x)=h(x)$ $\forall x\le 0$
$g(x)=c\sin ax$

$f(x)=h(0)+c^2\sinh^2 a\sqrt x$ $\forall x\ge 0$ and $f(x)=h(x)$ $\forall x\le 0$
$g(x)=c\sinh ax$
\end{solution}
*******************************************************************************
-------------------------------------------------------------------------------

\begin{problem}[Posted by \href{https://artofproblemsolving.com/community/user/125553}{lehungvietbao}]
	1) Find all  functions $f:\mathbb{N^*}\rightarrow \mathbb{N^*}$ are bijective  such that 
\[ f(f(n)) \le \dfrac{n+f(n)}{2}, \forall n \in \mathbb{N^*}\]

2) Find all functions $f:\mathbb{N}^* \to \mathbb{Z}$ which satisfy the following conditions: 
a) If $a \vdots b$ then $f(a) \geq f(b)$
b) $f(ab)+f\left ( a^2+b^2 \right ) = f(a)+f(b)$
	\flushright \href{https://artofproblemsolving.com/community/c6h564926}{(Link to AoPS)}
\end{problem}



\begin{solution}[by \href{https://artofproblemsolving.com/community/user/29428}{pco}]
	\begin{tcolorbox}1) Find all  functions $f:\mathbb{N^*}\rightarrow \mathbb{N^*}$ are bijective  such that 
\[ f(f(n)) \le \dfrac{n+f(n)}{2}, \forall n \in \mathbb{N^*}\]\end{tcolorbox}
$f(f(f(n)))\le \frac{f(n)+f(f(n))}2\le \frac n4+\frac 34f(n)$

$f(f(f(f(n))))\le \frac {f(n)}4+\frac 34f(f(n))\le \frac 38n+\frac 58f(n)$

And easy induction gives assertion $P(k,n)$ : $f^{k+1}(n)\le \frac{1-(-2)^{-k}}3n+\frac{2+(-2)^{-k}}3f(n)$

Let $n\in\mathbb N$ :
If $f(n)\le n$, above formula implies $f^{k}(n)\le n$ $\forall k$ and so $\exists k_1>k_2$ such that $f^{k_1}(n)=f^{k_2}(n)$ 
So $\exists k_n>0$ such that $f^{k_n}(n)=n$

If $f(n)\ge n$, above formula implies $f^{k}(n)\le f(n)$ $\forall k$ and so $\exists k_1>k_2$ such that $f^{k_1}(n)=f^{k_2}(n)$ 
So $\exists k_n>0$ such that $f^{k_n}(n)=n$

$P(k_n-1,n)$ $\implies$ $n\le \frac{1-(-2)^{-k_n+1}}3n+\frac{2+(-2)^{-k_n+1}}3f(n)$

And so $f(n)\ge n$

And so, since bijective $\boxed{f(n)=n}$ $\forall n$
\end{solution}



\begin{solution}[by \href{https://artofproblemsolving.com/community/user/29428}{pco}]
	\begin{tcolorbox}2) Find all functions $f:\mathbb{N}^* \to \mathbb{Z}$ which satisfy the following conditions: 
a) If $a \vdots b$ then $f(a) \geq f(b)$
b) $f(ab)+f\left ( a^2+b^2 \right ) = f(a)+f(b)$\end{tcolorbox}
Let $P(x,y)$ be the assertion $f(xy)+f(x^2+y^2)=f(x)+f(y)$
Let $c=f(1)$
$1|n$ $\implies$ $f(n)\ge c$ $\forall n$

Let $A=\{x$ such that $f(x)=c\}$
$P(1,1)$ $\implies$ $2\in A$
If $a,b\in A$, then $P(a,b)$ $\implies$ $ab,a^2+b^2\in A$

Let prime $p\equiv 1\pmod 4$ So that $\exists x$ such that $p|x^2+1$
$P(x,1)$ $\implies$ $f(x^2+1)=c$
$p|x^2+1$ $\implies$ $c\le f(p)\le f(x^2+1)=c$ and so $f(p)=c$ and $p\in A$

So $A$ contains at least all the numbers whose prime divisors are $2$ or any odd prime $\equiv 1\pmod 4$

As a consequence $a^2+b^2\in A$ $\forall$ coprime $a,b$ and $P(x,y)$ may be written :
New assertion $Q(x,y)$ : $f(xy)+c=f(x)+f(y)$ true $\forall$ coprime $x,y$

And so $(f(xy)-c)=(f(x)-c)+(f(y)-c)$ $\forall$ coprime $x,y$

So $f(\prod p_i^{n_i})=c+\sum n_i(f(p_i)-c)$

Let then odd prime $p$ : $P(p,p)$ $\implies$ $f(p^2)+f(2p^2)=2f(p)$ and so $f(p)=c$ 

Hence the unique solution : $\boxed{f(n)=c}$ $\forall n$
\end{solution}
*******************************************************************************
-------------------------------------------------------------------------------

\begin{problem}[Posted by \href{https://artofproblemsolving.com/community/user/119826}{seby97}]
	Let $a,b \in R*,f:R->R$ such that $f(f(x)+y)=ax+f(f(y)+bx)$.Find f
	\flushright \href{https://artofproblemsolving.com/community/c6h565012}{(Link to AoPS)}
\end{problem}



\begin{solution}[by \href{https://artofproblemsolving.com/community/user/29428}{pco}]
	\begin{tcolorbox}Let $a,b \in R*,f:R->R$ such that $f(f(x)+y)=ax+f(f(y)+bx)$.Find f\end{tcolorbox}
Let $P(x,y$ be the assertion $f(f(x)+y)=ax+f(f(y)+bx)$
Let $c=f(0)$
If $b=1$, then $P(1,1)$ $\implies$ $a=0$, impossible, so $b\ne 1$

$P(\frac ca,-f(\frac ca))$ $\implies$ $f(f(-f(\frac ca))+\frac {bc}a)=0$ and so $\exists u$ such that $f(u)=0$

$P(-\frac xa, u-f(-\frac xa))$ $\implies$ $x=f(f(u-f(-\frac xa))-x\frac ba)$ and so $f(x)$ is surjective.

$P(y,bx)$ $\implies$ $f(f(y)+bx)=ay+f(f(bx)+by)$ and, adding to $P(x,y)$, we get
New assertion $Q(x,y)$ : $f(f(x)+y)=ax+ay+f(f(bx)+by)$

$Q(x,\frac{f(x)-f(bx)}{b-1})$ $\implies$ $\frac{f(x)-f(bx)}{b-1}+x=0$ and so $f(bx)=f(x)+(b-1)x$

$P(u,u)$ $\implies$ $0=au+f(bu)$ $=(a+b-1)u$  and so $(a+b-1)u=0$

1) If $a+b-1\ne 0$
==============
Then $u=0$ and so $c=u=f(0)=0$
$P(1,0)$ $\implies$ $f(f(1))=ax+f(bx)$ $=f(1)+a+b-1$
$P(0,1)$ $\implies$ $f(1)=f(f(1))$
And so $a+b-1=0$, contradiction and so no solution.

2) If $a+b-1=0$
=============
$P(x,u)$  $\implies$ $f(f(x)+u)=ax+f(bx)$ $=f(x)+(a+b-1)x=f(x)$
And so, since surjective, $f(x+u)=x$

And so $f(x)=x-u$ which indeed is a solution, whatever is $u$

3) \begin{bolded}Hence the result\end{underlined}\end{bolded} :
If $a+b\ne 1$ : no solution
If $a+b=1$ : $f(x)=x+c$ $\forall x$, and whatever is $c\in\mathbb R$
\end{solution}



\begin{solution}[by \href{https://artofproblemsolving.com/community/user/199494}{IMI-Mathboy}]
	I've solved it for $a\neq0$ : putting$P(x,-f(x))$ we know that $f$ can be described as all real numbers. $\to$ there exist d that $f(d)=0$.$P(0,x)$ we get $f(f(x))=f(x+c)$  After putting $P(x,y+c)$ and $P(x,f(y))$ we get: $f(f(x)+y+c)=f(f(x)+f(y))$  
Then by symmetry we have: $f(f(x)+y+c)=f(f(y)+x+c)$  $(*)$. In$(*)$ $Q(x,f(y)) \to$  $f(f(x)+f(y)+c)=f(f(y+c)+x+c)$ $(1)$
So interchanging x and y in $(1)$ we get $f(f(y+c)+x+c)=f(f(x+c)+y+c)$  means $f(f(y)+x)=f(f(x)+y)$  which makes original equation $f(f(y)+x)=ax+f(f(y)+bx)$ $(2)$. So in  $(2)$ R$(x,d)$ f(bx)=f(x)-ax.In original equation P$(x,d)$ we have $f(f(x)+d)=f(x) \to f(x)=x-d$ $(d-const.)$ for all real x.while checking we know $a+b=1$.
\end{solution}



\begin{solution}[by \href{https://artofproblemsolving.com/community/user/199494}{IMI-Mathboy}]
	can anyone tell me that how can i restore it.?
\end{solution}
*******************************************************************************
-------------------------------------------------------------------------------

\begin{problem}[Posted by \href{https://artofproblemsolving.com/community/user/59765}{hungkg}]
	Find all Functions $f:R \to R$ such that $f\left( {{f^2}\left( x \right) + f\left( y \right)} \right) = xf\left( x \right) + y,\forall x,y \in R.$
	\flushright \href{https://artofproblemsolving.com/community/c6h565030}{(Link to AoPS)}
\end{problem}



\begin{solution}[by \href{https://artofproblemsolving.com/community/user/29428}{pco}]
	\begin{tcolorbox}Find all Functions $f:\mathbb R \to \mathbb R$ such that $f\left( {{f^2}\left( x \right) + f\left( y \right)} \right) = xf\left( x \right) + y,\forall x,y \in \mathbb R.$\end{tcolorbox}
Let $P(x,y)$ be the assertion $f(f(x)^2+f(y))=xf(x)+y$
$f(x)$ is bijective.
Let then $u$ such that $f(u)=0$

$P(u,x)$ $\implies$ $f(f(x))=x$

$P(x,u)$ $\implies$ $f(f(x)^2)=xf(x)+u$
$P(f(x),u)$ $\implies$ $f(x^2)=xf(x)+u$
So $f(f(x)^2)=f(x^2)$ and so $\forall x$ either $f(x)=x$, either $f(x)=-x$

Suppose now that $\exists u,v\ne 0$ such that $f(u)=u$ and $f(v)=-v$
$P(u,v)$ $\implies$ $f(u^2-v)=u^2+v$ and so :
either $f(u^2-v)=u^2-v$ and so $v=0$, impossible
either $f(u^2-v)=-(u^2-v)$ and so $u=0$, impossible
So no such $u,v$ and :

Either $\boxed{f(x)=x}$ $\forall x$, which indeed is a solution

Either $\boxed{f(x)=-x}$ $\forall x$, which indeed is a solution
\end{solution}



\begin{solution}[by \href{https://artofproblemsolving.com/community/user/179088}{Panoz93}]
	Another approach ...
In the initial equation we set $x=y=0$ and we get $f(f(0)^{2}+f(0))=0$ 
Also , we set $x=y=f(0)^{2}+f(0)$ and we immidiately conclude that $f(0)^2=0\Leftrightarrow f(0)=0$

So , now we set $x=0$ to get $f(f(y))=y$        $(1)$
Using $(1)$ we can substitute $x=f(x), y=0$ to get $f(x^2)=xf(x)$
Therefore $f$ is odd 

The given equation reduces to 
$f(x^{2}+y)=f(x^{2})+f(y)\Leftrightarrow f(a+b)=f(a)+f(b) $ for a non-negative 
Since odd , though , we conclude that $f$ satisfies cauchy for any real numbers $a,b$

Combining $f((x+1)^2)=(x+1)f(x+1)$ with cauchy we get f(x)=xf(1)
and thus , substitution to the initial gives us two possible solutions :

$f(x)=x$
$f(x)=-x$
\end{solution}



\begin{solution}[by \href{https://artofproblemsolving.com/community/user/204311}{Onlygodcanjudgeme}]
	we know that function is bijective.
$  f(f(x^2) + f(y)) = f(x) \cdot x +y  $
$  (x,y) = (0,0) $ then  $ f(f(o)^2 + f(0)) =0 $
if $ f(k)= 0 $ , and $ f(k) = f(f(0)^2 + f(0)) $ we know that function is injective, so  $ k= f(0)^2 + f(0)  $
Put $ (x,y) = (k,k) $ then we have  $ f(0) = k $ 
 $ k= f(0)^2 + f(0)  $ and  $ f(0) =k $
$ f(0)= 0 $.So k is equal to 0 .
    Put $ (x,y) = (x,0) $ then we have 
$ f(f(x)^2)= f(x) \cdot x $  
    Put    $ (x,y) =(f(x) ,0) $ so we have $ f(x^2)= f(x) \cdot x $
in there $ f(f(x)^2) = f(x^2) $  then   $ f(x)^2 = x^2  $ 
$ f(x) = x $ or $ f(x) =-x $
\end{solution}



\begin{solution}[by \href{https://artofproblemsolving.com/community/user/29428}{pco}]
	\begin{tcolorbox}...
in there $ f(f(x)^2) = f(x^2) $  then   $ f(x)^2 = x^2  $ 
$ f(x) = x $ or $ f(x) =-x $\end{tcolorbox}
No, $f(x)^2=x^2$ does not imply : "either $f(x)=x$ $\forall x$, either $f(x)=-x$ $\forall x$"

It implies : "$\forall x$, either $f(x)=x$, either $f(x)=-x$"
Which is quite different. For example $f(x)=|x|$ is such that $f(x)^2=x^2$

So you miss some work again before conclusion.
\end{solution}
*******************************************************************************
-------------------------------------------------------------------------------

\begin{problem}[Posted by \href{https://artofproblemsolving.com/community/user/68025}{Pirkuliyev Rovsen}]
	Find all functions $f: \mathbb{R}\to\mathbb{R}$ such that $f(f...f(f(x))...)+y)=x+y$,where there exist $n$ notations of $f$ inside the equation.
	\flushright \href{https://artofproblemsolving.com/community/c6h565054}{(Link to AoPS)}
\end{problem}



\begin{solution}[by \href{https://artofproblemsolving.com/community/user/29428}{pco}]
	\begin{tcolorbox}Find all functions $f: \mathbb{R}\to\mathbb{R}$ such that $f(f...f(f(x))...)+y)=x+y$,where there exist $n$ notations of $f$ inside the equation.\end{tcolorbox}
Let $P(x,y)$ be the assertion $f(f^{[n-1]}(x)+y)=x+y$
Let $a=f(0)$

$P(x,0)$ $\implies$ $f^{[n]}(x)=x$
$P(f(0),x)$ $\implies$ $f(x)=x+a$

Plugging back in original equation, we get $a=0$ and so $\boxed{f(x)=x}$ $\forall x$
\end{solution}
*******************************************************************************
-------------------------------------------------------------------------------

\begin{problem}[Posted by \href{https://artofproblemsolving.com/community/user/68025}{Pirkuliyev Rovsen}]
	Does there exist a function $f: \mathbb{R}\to\mathbb{R}$ such that $f(1+f(x))=1-x$ and $f(f(x))=x$.
	\flushright \href{https://artofproblemsolving.com/community/c6h565055}{(Link to AoPS)}
\end{problem}



\begin{solution}[by \href{https://artofproblemsolving.com/community/user/29428}{pco}]
	\begin{tcolorbox}Does there exist a function $f: \mathbb{R}\to\mathbb{R}$ such that $f(1+f(x))=1-x$ and $f(f(x))=x$.\end{tcolorbox}

$f(1+f(x))=1-x$ $\implies$ $f(1+x)=1-f(x)$ $\implies$ $f(2+x)=1-f(1+x)=f(x)$ an so $f(x)$ is not injective, in contradiction with $f(f(x))=x$

So no such function.
\end{solution}



\begin{solution}[by \href{https://artofproblemsolving.com/community/user/198389}{Yue}]
	Or,
let us assume such a function exists.
Then, \[f(f(1+f(x)))=1+f(x)\Rightarrow f(1-x)=1+f(x)\]
Putting $x=0$ and $x=1$, we have \[f(1)=1+f(0)\] and \[f(0)=1+f(1)\]
Hence, $f(0)=f(1)$.
Thus, $0=1$.

So such a function cannot exist.
\end{solution}
*******************************************************************************
-------------------------------------------------------------------------------

\begin{problem}[Posted by \href{https://artofproblemsolving.com/community/user/68025}{Pirkuliyev Rovsen}]
	Let $a$ be a real number, $a{\in}(0;1)$.Determine all functions $f: \mathbb{R}\to\mathbb{R}$ that are continuous at $x=0$ and satisfy the equation $f(x)+f(ax)=x$ for all real $x$.
	\flushright \href{https://artofproblemsolving.com/community/c6h565063}{(Link to AoPS)}
\end{problem}



\begin{solution}[by \href{https://artofproblemsolving.com/community/user/29428}{pco}]
	\begin{tcolorbox}Let $a$ be a real number, $a{\in}(0;1)$.Determine all functions $f: \mathbb{R}\to\mathbb{R}$ that are continuous at $x=0$ and satisfy the equation $f(x)+f(ax)=x$ for all real $x$.\end{tcolorbox}
$f(0)=0$

$f(x)+f(ax)=x$
$f(ax)+f(a^2x)=ax$
Subtracting : $f(x)-f(a^2x)=(1-a)x$

$f(x)-f(a^2x)=(1-a)x$
$f(a^2x)-f(a^4x)=(1-a)a^2x$
...
$f(a^{2n-2}x)-f(a^{2n}x)=(1-a)a^{2n-2}x$
Summing : $f(x)-f(a^{2n}x)=(1-a)x\frac{1-a^{2n}}{1-a^2}$
Setting $n\to+\infty$ and using continuity at $0$ :

$\boxed{f(x)=\frac x{a+1}}$ $\forall x$, which indeed is a solution.
\end{solution}
*******************************************************************************
-------------------------------------------------------------------------------

\begin{problem}[Posted by \href{https://artofproblemsolving.com/community/user/125553}{lehungvietbao}]
	Find all continuous functions $f: \mathbb{R} \to \mathbb{R}$ such that
      \[\sum_{k=0}^{n}{{n\choose k} f(x^{2^{k}}})=0 \quad \forall x\in \mathbb{R} \]
	\flushright \href{https://artofproblemsolving.com/community/c6h565080}{(Link to AoPS)}
\end{problem}



\begin{solution}[by \href{https://artofproblemsolving.com/community/user/29428}{pco}]
	\begin{tcolorbox}Find all continuous functions $f: \mathbb{R} \to \mathbb{R}$ such that
      \[\sum_{k=0}^{n}{{n\choose k} f(x^{2^{k}}})=0 \quad \forall x\in \mathbb{R} \]\end{tcolorbox}
If $n=0$, equation is $f(x)=0$

If $n>0$, let $g(x)=\sum_{k=0}^{n-1}\binom{n-1}kf(x^{2^k})$ and equation is $g(x)+g(x^2)=0$
So $g(x)=g(x^4)$ and $g(x)=g(x^{4^n})$ $\forall n\in\mathbb Z$
Setting $n\to \pm\infty$ (depending on $x$) and using continuity gives $g(x)=c$ $\forall x\ge 0$
Back to $g(x)+g(x^2)=0$, we get $g(x)=0$ $\forall x$

So $\sum_{k=0}^{n}\binom{n}kf(x^{2^k})=0$ $\forall x$ $\implies$ $\sum_{k=0}^{n-1}\binom{n-1}kf(x^{2^k})=0$ $\forall x$

And so implies $\sum_{k=0}^{0}\binom{0}kf(x^{2^k})=0$

And so the unique solution $\boxed{f(x)=0}$ $\forall x$
\end{solution}
*******************************************************************************
-------------------------------------------------------------------------------

\begin{problem}[Posted by \href{https://artofproblemsolving.com/community/user/68025}{Pirkuliyev Rovsen}]
	Find all functions $f: \mathbb{R}\to\mathbb{R}$ such that $f(x+y)+y{\leq}f(f(f(x)))$ for all $x,y{\in}R$.
	\flushright \href{https://artofproblemsolving.com/community/c6h565299}{(Link to AoPS)}
\end{problem}



\begin{solution}[by \href{https://artofproblemsolving.com/community/user/29428}{pco}]
	\begin{tcolorbox}Find all functions $f: \mathbb{R}\to\mathbb{R}$ such that $f(x+y)+y{\leq}f(f(f(x)))$ for all $x,y{\in}R$.\end{tcolorbox}
Let $P(x,y)$ be the assertion $f(x+y)+y\le f(f(f(x)))$

$P(x,0)$ $\implies$ $f(x)\le f(f(f(x)))$
$P(f(x),f(f(f(x)))-f(x))$ $\implies$ $f(f(f(x)))\le f(x)$

So $f(f(f(x)))=f(x)$ $\forall x$ and $P(x,y)$ becomes new assertion $Q(x,y)$ : $f(x+y)+y\le f(x)$

$Q(x,y-x)$ $\implies$  $f(y)+y\le f(x)+x$
$Q(y,x-y)$ $\implies$  $f(x)+x\le f(y)+y$
So $f(x)+x$ is constant and $\boxed{f(x)=a-x}$ $\forall x$, which indeed is a solution, whatever is $a\in\mathbb R$
\end{solution}



\begin{solution}[by \href{https://artofproblemsolving.com/community/user/31919}{tenniskidperson3}]
	Am I wrong, or does $P(f(x), f(f(f(x)))-f(x))$ give $f(f(f(f(x))))\leq f(x)$ instead of $f(f(f(x)))\leq f(x)$?  Should there be 4 $f$'s instead of 3?
\end{solution}



\begin{solution}[by \href{https://artofproblemsolving.com/community/user/83393}{fractals}]
	Actually, it gives $f(f(f(f(x)))) + f(f(f(x))) - x \ge f(f(f(f(x))))$ so then $f(f(f(x))) \le x$ by canceling, as \begin{bolded}pco\end{bolded} said. Since there are also four $f$'s on the RHS (remembering that we plug in $f(x)$ for $x$).
\end{solution}



\begin{solution}[by \href{https://artofproblemsolving.com/community/user/157383}{jowramos}]
	First, setting $y=0$ in the original equation, we then have $f(x)\le f(f(f(x)))$. Setting $y=f(f(x))-x$, then we have that $f(f(x))\le x$. Setting $x=f(z)$, we get the reverse inequality, and, then, $f(f(f(x)))=f(x)$. Setting $x=f(f(z))$ and $y=z-f(f(z))$, then we have that $z\le f(f(z))$, and, then, as we have already gotten the reverse inequality, $f(f(x))=x$. To finish, we can see that, using what we have already, with $y=-x$, $f(0)-x\le f(x)$. 
Now, setting $x=0$, then $f(y)\le f(0)-y$, and, then, we have equality, so, the only functions that satisfy the inequality are the ones of the form $a-x$.
\end{solution}
*******************************************************************************
-------------------------------------------------------------------------------

\begin{problem}[Posted by \href{https://artofproblemsolving.com/community/user/68025}{Pirkuliyev Rovsen}]
	Find all functions $f: \mathbb{R}\to\mathbb{R}$ such that $f(x-y)=f(x^3-y^3)$ for all $x,y{\in}R$.
	\flushright \href{https://artofproblemsolving.com/community/c6h565300}{(Link to AoPS)}
\end{problem}



\begin{solution}[by \href{https://artofproblemsolving.com/community/user/86443}{roza2010}]
	[hide]$f(x)=f(x^3)$ , so $f=const$[\/hide]
\end{solution}



\begin{solution}[by \href{https://artofproblemsolving.com/community/user/29428}{pco}]
	\begin{tcolorbox}$f(x)=f(x^3)$ , so $f=const$\end{tcolorbox}
How can you conclude $f(x)=$constant, just from $f(x)=f(x^3)$ ?
There are a lot of non continuous non constant functions such that $f(x)=f(x^3)$
\end{solution}



\begin{solution}[by \href{https://artofproblemsolving.com/community/user/29428}{pco}]
	\begin{tcolorbox}Find all functions $f: \mathbb{R}\to\mathbb{R}$ such that $f(x-y)=f(x^3-y^3)$ for all $x,y{\in}R$.\end{tcolorbox}
Let $a,b\in\mathbb R\setminus\{0\}$
Consider the system $x-y=a$ and $x^3-y^3=b$.
$\iff$ $y=x-a$ and $x^3-(x-a)^3=b$ which is the quadratic $3x^2a-3xa^2+a^3-b=0$ whose discriminant is $3a(4b-a^3)$
So this quadratic has solution whenever $b$ has same sign than $a$ and $|b|\ge \frac{|a^3|}4$

So $f(x)=f(a)$ $\forall x\ge \frac{a^3}4>0$ and setting $a\to 0+$, we get $f(x)=c_1$ $\forall x>0$
And $f(x)=f(a)$ $\forall x\le \frac{a^3}4<0$ and setting $a\to 0-$, we get $f(x)=c_2$ $\forall x<0$

\begin{bolded}Hence the answer\end{underlined} \end{bolded}:
$f(x)=c_1$ $\forall x>0$
$f(0)=c_3$
$f(x)=c_2$ $\forall x<0$
Which indeed is a solution, whatever are $c_1,c_2,c_3\in\mathbb R$
\end{solution}
*******************************************************************************
-------------------------------------------------------------------------------

\begin{problem}[Posted by \href{https://artofproblemsolving.com/community/user/68025}{Pirkuliyev Rovsen}]
	Find all the continuous functions $f: \mathbb{R}\to\mathbb{R}$ such that $(f(x))^2=f(2x)$.
	\flushright \href{https://artofproblemsolving.com/community/c6h565357}{(Link to AoPS)}
\end{problem}



\begin{solution}[by \href{https://artofproblemsolving.com/community/user/29428}{pco}]
	\begin{tcolorbox}Find all the continuous functions $f: \mathbb{R}\to\mathbb{R}$ such that $(f(x))^2=f(2x)$.\end{tcolorbox}
$f(x)=0$ $\forall x$ is a solution. So let us from now look only for non allzero solutions.
$f(x)\ge 0$ $\forall x$
Let $a$ such that $f(a)>0$

$f(a2^{-n})=f(a)^{2^{-n}}$. Setting there $n\to +\infty$ and using continuity, we get $f(0)=1$

If $f(u)=0$ for some $u\ne 0$, then $f(u2^{n})=0$ $\forall n\in\mathbb Z$ and so (set $n\to-\infty$), since continuous, $f(0)=0$, impossible.

So $f(x)>0$ and we can write $f(x)=e^{g(x)}$ where $g(x)$ is some continuous function such that $g(2x)=2g(x)$

And solution of this equation is quite simple :

Let $h_1(x),h_2(x)$ any continuous functions from $[1,2)\to\mathbb R$ such that $h_1(2)=2h_1(1)$ and $h_2(2)=2h_2(1)$
Then $g(x)$ may be defined as :

$\forall x>0$ $g(x)=2^{\lfloor\log_2 |x|\rfloor}h_1(2^{\{\log_2|x|\}})$
$g(0)=0$
$\forall x<0$ $g(x)=2^{\lfloor\log_2 |x|\rfloor}h_2(2^{\{\log_2 |x|\}})$
(continuity at zero is easy to check)

And $f(x)=e^{g(x)}$
\end{solution}
*******************************************************************************
-------------------------------------------------------------------------------

\begin{problem}[Posted by \href{https://artofproblemsolving.com/community/user/200116}{xiaotabuta}]
	$f: \mathbb{R_+} \rightarrow \mathbb{R_+}$ satisfied:
$f(ab) = lcm(a,b) . gcd(f(a), f(b))$
	\flushright \href{https://artofproblemsolving.com/community/c6h565363}{(Link to AoPS)}
\end{problem}



\begin{solution}[by \href{https://artofproblemsolving.com/community/user/29428}{pco}]
	\begin{tcolorbox}$f: \mathbb{R_+} \rightarrow \mathbb{R_+}$ satisfied:
$f(ab) = lcm(a,b) . gcd(f(a), f(b))$\end{tcolorbox}
Without any precision, I suppose that domain of functional equation is the same as domain of function, so $\mathbb R_+$
Then what is the definition,  in your country, of $lcm(\sqrt 2,\sqrt3)$ where both $\sqrt2$ and $\sqrt 3$ $\in\mathbb R_+$ ?
\end{solution}



\begin{solution}[by \href{https://artofproblemsolving.com/community/user/198389}{Yue}]
	Assuming the mistake is that the domain and the range will be $\mathbb{N}$, we have

$f(k.1)=k.gcd(f(1),f(k))$
i.e., $f(k)=k.gcd(f(1),f(k))$

Let $(gcd(f(1),f(k))=d$, where $f(1)=dl, f(k)=dk; gcd(k,l)=1.$
If $l\neq 1$, we have $f(l)=dl^2$ which is wrong.
Hence $l=1$.

If $f(1)=d$, we have $f(n)=dn$ for all $n\in \mathbb{N}$.
\end{solution}



\begin{solution}[by \href{https://artofproblemsolving.com/community/user/200116}{xiaotabuta}]
	\begin{tcolorbox}Assuming the mistake is that the domain and the range will be $\mathbb{N}$, we have

$f(k.1)=k.gcd(f(1),f(k))$
i.e., $f(k)=k.gcd(f(1),f(k))$

Let $(gcd(f(1),f(k))=d$, where $f(1)=dl, f(k)=dk; gcd(k,l)=1.$
If $l\neq 1$, we have $f(l)=dl^2$ which is wrong.
Hence $l=1$.

If $f(1)=d$, we have $f(n)=dn$ for all $n\in \mathbb{N}$.\end{tcolorbox}
Can you please explain why $f(l) = dl^2$ ?
\end{solution}



\begin{solution}[by \href{https://artofproblemsolving.com/community/user/198389}{Yue}]
	\begin{tcolorbox}[quote="Yue"]Assuming the mistake is that the domain and the range will be $\mathbb{N}$, we have

$f(k.1)=k.gcd(f(1),f(k))$
i.e., $f(k)=k.gcd(f(1),f(k))$

Let $(gcd(f(1),f(k))=d$, where $f(1)=dl, f(k)=dk; gcd(k,l)=1.$
If $l\neq 1$, we have $f(l)=dl^2$ which is wrong.
Hence $l=1$.

If $f(1)=d$, we have $f(n)=dn$ for all $n\in \mathbb{N}$.\end{tcolorbox}
Can you please explain why $f(l) = dl^2$ ?\end{tcolorbox}

Because $f(l.1)=l.gcd(f(1),f(l))=l.gcd(dl,dl)=l.dl=dl^2$
\end{solution}
*******************************************************************************
-------------------------------------------------------------------------------

\begin{problem}[Posted by \href{https://artofproblemsolving.com/community/user/59765}{hungkg}]
	Find all continuous functions $f:R \to R$ such as ${\rm{f (x  +  y)  +  f (x  -  y)  =  2f (x)  +  2f (y) }}$,for all $x,y\in R$.
	\flushright \href{https://artofproblemsolving.com/community/c6h565465}{(Link to AoPS)}
\end{problem}



\begin{solution}[by \href{https://artofproblemsolving.com/community/user/29428}{pco}]
	\begin{tcolorbox}Find all functions $f:R \to R$ such as ${\rm{f (x  +  y)  +  f (x  -  y)  =  2f (x)  +  2f (y) }}$,for all $x,y\in R$.\end{tcolorbox}
Let $P(x,y)$ be the assertion $f(x+y)+f(x-y)=2f(x)+2f(y)$

$P(0,0)$ $\implies$ $f(0)=0$
$P(0,x)$ $\implies$ $f(-x)=f(x)$ and the function is even.

$P((n+1)x,x)$ $\implies$ $f((n+2)x=2f(n+1)x-f(nx)+2f(x)$ and a simple induction gives $f(nx)=n^2f(x)$ $\forall x$
So $f(x)=ax^2$ $\forall x\in\mathbb Q$

Without continuity, it is difficult to go further.

We can easily check that $f(x)=c\times a(x)^2$ is solution, whatever is $a(x)$ solution of additive Cauchy equation but I'm not sure these are the only solutions.

And i'm very interested in your own full solution. Thanks in advance.
\end{solution}



\begin{solution}[by \href{https://artofproblemsolving.com/community/user/59765}{hungkg}]
	\begin{tcolorbox}[quote="hungkg"]Find all functions $f:R \to R$ such as ${\rm{f (x  +  y)  +  f (x  -  y)  =  2f (x)  +  2f (y) }}$,for all $x,y\in R$.\end{tcolorbox}
Let $P(x,y)$ be the assertion $f(x+y)+f(x-y)=2f(x)+2f(y)$

$P(0,0)$ $\implies$ $f(0)=0$
$P(0,x)$ $\implies$ $f(-x)=f(x)$ and the function is even.

$P((n+1)x,x)$ $\implies$ $f((n+2)x=2f(n+1)x-f(nx)+2f(x)$ and a simple induction gives $f(nx)=n^2f(x)$ $\forall x$
So $f(x)=ax^2$ $\forall x\in\mathbb Q$

Without continuity, it is difficult to go further.

We can easily check that $f(x)=c\times a(x)^2$ is solution, whatever is $a(x)$ solution of additive Cauchy equation but I'm not sure these are the only solutions.

And i'm very interested in your own full solution. Thanks in advance.\end{tcolorbox}

Sorry sir PCO. I am wrong. You are right. Necessary condition Continuous function.
\end{solution}
*******************************************************************************
-------------------------------------------------------------------------------

\begin{problem}[Posted by \href{https://artofproblemsolving.com/community/user/190536}{DonaldLove}]
	Given $a\in \mathbb{R}^*$, let $f: \mathbb{R}\setminus \{1\} \to  \mathbb{R}$ be given by $f(x)=\dfrac{1}{4a^2(1-x)}$. Find $a$ such that there exists a positive integer $n$ satisfying $f(f(f(...(0)...)))=1$ ($n$ times $f$).
	\flushright \href{https://artofproblemsolving.com/community/c6h565492}{(Link to AoPS)}
\end{problem}



\begin{solution}[by \href{https://artofproblemsolving.com/community/user/29428}{pco}]
	\begin{tcolorbox}Given $a\in \mathbb{R}^*$, let $f: \mathbb{R}\setminus \{1\} \to  \mathbb{R}$ be given by $f(x)=\dfrac{1}{4a^2(1-x)}$. Find $a$ such that there exists a positive integer $n$ satisfying $f(f(f(...(0)...)))=1$ ($n$ times $f$).\end{tcolorbox}
Wlog consider $a>0$

If $a\ge 1$, it's easy to see that $f^{n}(0)\in[0,\frac 12]$ and so can never be $1$
If $a\in[0,1)$, let $t\in(0,\frac{\pi}2)$ such that $a=\cos t$

Let then $h(x)=\cot t-2\cot tf(\frac{x-\cot t}{-2\cot t})$ $=\frac{x-\tan t}{1+x\tan t}$

So $h(\tan x)=\tan(x-t)$

So $h^{n}(\tan x)=\cot t-2\cot tf^{n}(\frac{\tan x-\cot t}{-2\cot t})$ $=\tan(x-nt)$

So $h^n(\cot t)=\cot t-2\cot tf^{n}(0)$ $=\tan(\frac{\pi}2-t-nt)$

So $f^n(0)=1$ $\iff$ $-\cot t=\tan(\frac{\pi}2-t-nt)$ $\iff$ $t=\frac{k\pi}{n+2}$

Hence the answer $\boxed{a\in\left\{\pm\cos\frac pq\pi\text{    }\forall p,q\in\mathbb N\text{ such that }0<\frac pq<\frac 12\right\}}$
\end{solution}



\begin{solution}[by \href{https://artofproblemsolving.com/community/user/190536}{DonaldLove}]
	thanks Patrick. Can you explain how do you get $h(x)$ in the solution?
\end{solution}



\begin{solution}[by \href{https://artofproblemsolving.com/community/user/29428}{pco}]
	\begin{tcolorbox}thanks Patrick. Can you explain how do you get $h(x)$ in the solution?\end{tcolorbox}
I wrote $h(x)=u+vf(\frac {x-u}v)$ so that $h^n(x)=u+vf^n(\frac {x-u}v)$

And I looked for $u,v$ such that $h(x)=\frac {x+w}{1-wx}$ for some $w$  in order to have $h(\tan x)=\tan(x+s)$ and so $h^n(\tan x)=\tan (x+ns)$
\end{solution}
*******************************************************************************
-------------------------------------------------------------------------------

\begin{problem}[Posted by \href{https://artofproblemsolving.com/community/user/190536}{DonaldLove}]
	find $f:Q \to Q$ such that $f(x^2+y+f(xy))=3+(x+f(y)-2)f(x) \forall x,y \in Q$
	\flushright \href{https://artofproblemsolving.com/community/c6h565495}{(Link to AoPS)}
\end{problem}



\begin{solution}[by \href{https://artofproblemsolving.com/community/user/190536}{DonaldLove}]
	any suggestion?
\end{solution}



\begin{solution}[by \href{https://artofproblemsolving.com/community/user/29428}{pco}]
	\begin{tcolorbox}find $f:Q \to Q$ such that $f(x^2+y+f(xy))=3+(x+f(y)-2)f(x) \forall x,y \in Q$\end{tcolorbox}
Let $P(x,y)$ be the assertion $f(x^2+y+f(xy))=3+(x+f(y)-2)f(x)$
Let $a=f(0)$

If $a=0$, then $P(0,x)$ $\implies$ $f(x)=3$  $\forall x$, which is not a solution. So $a\ne 0$

$P(0,0)$ $\implies$ $f(a)=a^2-2a+3$

$P(a,0)$ $\implies$ $f(a^2+a)=3+(2a-2)f(a)$ $=2a^3-6a^2+10a-3$
$P(-a,0)$ $\implies$ $f(a^2+a)=3-2f(-a)$
Subtracting, we get $f(-a)=-a^3+3a^2-5a+3$

$P(0,-a)$ $\implies$ $f(-a)=3-\frac 3a$

So $-a^3+3a^2-5a+3=3-\frac 3a$ $\iff$ $(a-1)(a^3-2a^2+3a+3)=0$
It's easy to see that $x^3-2x^2+3x+3=0$ has no rational root. And so $a=1$

$P(0,x)$ $\implies$ $f(x+1)=f(x)+1$ and so $f(n)=n+1$ $\forall n\in\mathbb Z$ and $f(x+n)=f(x)+n$

$P(\frac pq,0)$ $\implies$ $f(\frac{p^2}{q^2})=(\frac pq-1)f(\frac pq)+2$
$P(q+\frac pq,0)$ $\implies$ $f(\frac{p^2}{q^2})=(q+\frac pq-1)f(\frac pq))-p-q+2$
Subtracting, we get $f(\frac pq)=\frac pq+1$

And so $\boxed{f(x)=x+1}$ $\forall x\in\mathbb Q$, which indeed is a solution.
\end{solution}
*******************************************************************************
-------------------------------------------------------------------------------

\begin{problem}[Posted by \href{https://artofproblemsolving.com/community/user/59765}{hungkg}]
	Find all functions $f:R \to R$ such that  $f(f(x)+f(y))=f(x)+y$, for all $x,y\in R$.
	\flushright \href{https://artofproblemsolving.com/community/c6h565648}{(Link to AoPS)}
\end{problem}



\begin{solution}[by \href{https://artofproblemsolving.com/community/user/29428}{pco}]
	\begin{tcolorbox}Find all functions $f:R \to R$ such that  $f(f(x)+f(y))=f(x)+y$, for all $x,y\in R$.\end{tcolorbox}
Let $P(x,y)$ be the assertion $f(f(x)+f(y))=f(x)+y$

Subtracting $P(x,0)$ from $P(0,x)$, we get $f(x)=x+f(0)$

Plugging this back in equation, we get $f(0)=0$ and so the unique solution $\boxed{f(x)=x}$ $\forall x$
\end{solution}
*******************************************************************************
-------------------------------------------------------------------------------

\begin{problem}[Posted by \href{https://artofproblemsolving.com/community/user/59765}{hungkg}]
	1. Find all injectives $f:\mathbb{R}\to \mathbb{R}$ such as $f(0)=0$ and $f\left( {{f}^{2}}\left( x \right)+{{f}^{2}}\left( y \right) \right)=xf\left( x \right)+yf\left( y \right),\forall x,y\in \mathbb{R}.$

2. Find all function $f:\mathbb{R}\to \mathbb{R}$ such as $f\left( {{f}^{2}}\left( x \right)+{{f}^{2}}\left( y \right) \right)={{f}^{2}}\left( x \right)+{{y}^{2}},\forall x,y\in \mathbb{R}.$
	\flushright \href{https://artofproblemsolving.com/community/c6h565665}{(Link to AoPS)}
\end{problem}



\begin{solution}[by \href{https://artofproblemsolving.com/community/user/29428}{pco}]
	\begin{tcolorbox}1. Find all function $f:\mathbb{R}\to \mathbb{R}$ such as $f\left( {{f}^{2}}\left( x \right)+{{f}^{2}}\left( y \right) \right)=xf\left( x \right)+xy,\forall x,y\in \mathbb{R}.$\end{tcolorbox}
Let $P(x,y)$ be the assertion $f(f(x)^2+f(y)^2)=xf(x)+xy$ 

Subtracting $P(x,0)$ from $P(0,x)$, we get $xf(x)=0$ and so $f(x)=0$ $\forall x\ne 0$, which is not a solution.

So no solution for this functional equation.
\end{solution}



\begin{solution}[by \href{https://artofproblemsolving.com/community/user/29428}{pco}]
	\begin{tcolorbox}2. Find all function $f:\mathbb{R}\to \mathbb{R}$ such as $f\left( {{f}^{2}}\left( x \right)+{{f}^{2}}\left( y \right) \right)={{f}^{2}}\left( x \right)+{{y}^{2}},\forall x,y\in \mathbb{R}.$\end{tcolorbox}
Let $P(x,y)$ be the assertion $f(f(x)^2+f(y)^2)=f(x)^2+y^2$ 
Let $f(0)=a$

Subtracting $P(x,0)$ from $P(0,x)$, we get $f(x)^2=x^2+a^2$

So $f(x^2+y^2+2a^2)=x^2+y^2+a^2$
So (squaring) $(x^2+y^2+2a^2)^2+a^2=(x^2+y^2+a^2)^2$ and so $a=0$ and $f(x)^2=x^2$

Equation is then $f(x^2+y^2)=x^2+y^2$ and so $f(x)=x$ $\forall x\ge 0$

\begin{bolded}Hence the solution\end{underlined} \end{bolded}:
Let $e(x)$ be any fonction from $\mathbb R\to\{-1,+1\}$ : 
$f(x)=x$ $\forall x\ge 0$
$f(x)=e(x)x$ $\forall x<0$
Which indeed is a solution.
\end{solution}



\begin{solution}[by \href{https://artofproblemsolving.com/community/user/59765}{hungkg}]
	\begin{tcolorbox}[quote="hungkg"]1. Find all function $f:\mathbb{R}\to \mathbb{R}$ such as $f\left( {{f}^{2}}\left( x \right)+{{f}^{2}}\left( y \right) \right)=xf\left( x \right)+xy,\forall x,y\in \mathbb{R}.$\end{tcolorbox}
Let $P(x,y)$ be the assertion $f(f(x)^2+f(y)^2)=xf(x)+xy$ 

Subtracting $P(x,0)$ from $P(0,x)$, we get $xf(x)=0$ and so $f(x)=0$ $\forall x\ne 0$, which is not a solution.

So no solution for this functional equation.\end{tcolorbox}

Sorry sir PCO. Problem there is mistake. I repair problem as following.

Find all functions $f:\mathbb{R}\to \mathbb{R}$ such as $f\left( {{f}^{2}}\left( x \right)+{{f}^{2}}\left( y \right) \right)=xf\left( x \right)+yf\left( y \right),\forall x,y\in \mathbb{R}.$
\end{solution}



\begin{solution}[by \href{https://artofproblemsolving.com/community/user/29428}{pco}]
	\begin{tcolorbox}Sorry sir PCO. Problem there is mistake. I repair problem as following.

Find all functions $f:\mathbb{R}\to \mathbb{R}$ such as $f\left( {{f}^{2}}\left( x \right)+{{f}^{2}}\left( y \right) \right)=xf\left( x \right)+yf\left( y \right),\forall x,y\in \mathbb{R}.$\end{tcolorbox}
We obviously have infinitely many strange solutions and I dont know how to find general solution for such a problem. 

Example of solution : $f(x)=-1_{\mathbb Z}(x)x$

Could you kindly give us your own general solution, please ?
\end{solution}



\begin{solution}[by \href{https://artofproblemsolving.com/community/user/59765}{hungkg}]
	\begin{tcolorbox}[quote="hungkg"]Sorry sir PCO. Problem there is mistake. I repair problem as following.

Find all functions $f:\mathbb{R}\to \mathbb{R}$ such as $f\left( {{f}^{2}}\left( x \right)+{{f}^{2}}\left( y \right) \right)=xf\left( x \right)+yf\left( y \right),\forall x,y\in \mathbb{R}.$\end{tcolorbox}
We obviously have infinitely many strange solutions and I dont know how to find general solution for such a problem. 

Example of solution : $f(x)=-1_{\mathbb Z}(x)x$

Could you kindly give us your own general solution, please ?\end{tcolorbox}

Sorry sir PCO. I think that there is mistake here. Necessary condition injective. I will repair problem as following.



Find all injectives $f:\mathbb{R}\to \mathbb{R}$ such as $f(0)=0$ and $f\left( {{f}^{2}}\left( x \right)+{{f}^{2}}\left( y \right) \right)=xf\left( x \right)+yf\left( y \right),\forall x,y\in \mathbb{R}.$
\end{solution}



\begin{solution}[by \href{https://artofproblemsolving.com/community/user/29428}{pco}]
	\begin{tcolorbox}

Sorry sir PCO. I think that there is mistake here. Necessary condition injective. I will repair problem as following.
...
\end{tcolorbox}
You seem quite unable to correctly copy the problem statement you got from your teacher.

I suggest you to simply post your solution and we'll create for you the precise full problem statement matching your solution.
\end{solution}



\begin{solution}[by \href{https://artofproblemsolving.com/community/user/59765}{hungkg}]
	\begin{tcolorbox}[quote="hungkg"]

Sorry sir PCO. I think that there is mistake here. Necessary condition injective. I will repair problem as following.
...
\end{tcolorbox}
You seem quite unable to correctly copy the problem statement you got from your teacher.

I suggest you to simply post your solution and we'll create for you the precise full problem statement matching your solution.\end{tcolorbox}

Sorry Sir PCO. I can't solve this problem.
\end{solution}



\begin{solution}[by \href{https://artofproblemsolving.com/community/user/29428}{pco}]
	\begin{tcolorbox}
Sorry Sir PCO. I can't solve this problem.\end{tcolorbox}
Please, read the forum definitions and respect the usage. It's just a matter of politeness :

\begin{italicized}Unsolved category\end{underlined}\end{italicized} : "Problems you couldn't solve and to which you know that there is a solution (i.e. a problem from a contest, etc.) but you don't know it."

\begin{italicized}Proposed and own category\end{underlined}\end{italicized} : "Problems \begin{bolded}that you have already solved\end{underlined}\end{bolded} and you are interested in second opinions or solutions."

\begin{italicized}Open category\end{underlined}\end{italicized} : "An open question is a question that has no known solution up to this moment, and it is not known wheter the problem has one or not."

Some users are interested in only some categories (personnaly, I generally spend more time on "proposed and own" than on "Open" since I'm sure there is a clever answer (olympiad level) and I know, when I fail findind it, that original poster will post the answer when asked.
\end{solution}



\begin{solution}[by \href{https://artofproblemsolving.com/community/user/59765}{hungkg}]
	\begin{tcolorbox}[quote="hungkg"]
Sorry Sir PCO. I can't solve this problem.\end{tcolorbox}
Please, read the forum definitions and respect the usage. It's just a matter of politeness :

\begin{italicized}Unsolved category\end{underlined}\end{italicized} : "Problems you couldn't solve and to which you know that there is a solution (i.e. a problem from a contest, etc.) but you don't know it."

\begin{italicized}Proposed and own category\end{underlined}\end{italicized} : "Problems \begin{bolded}that you have already solved\end{underlined}\end{bolded} and you are interested in second opinions or solutions."

\begin{italicized}Open category\end{underlined}\end{italicized} : "An open question is a question that has no known solution up to this moment, and it is not known wheter the problem has one or not."

Some users are interested in only some categories (personnaly, I generally spend more time on "proposed and own" than on "Open" since I'm sure there is a clever answer (olympiad level) and I know, when I fail findind it, that original poster will post the answer when asked.\end{tcolorbox}

I will learn from experience sir PCO.
\end{solution}
*******************************************************************************
-------------------------------------------------------------------------------

\begin{problem}[Posted by \href{https://artofproblemsolving.com/community/user/168537}{vutuanhien}]
	Find $f:\mathbb{R}\rightarrow \mathbb{R}$ such that:
1)$f(x)-2f(3x)+f(9x)=x^3+x^2+x$ $\forall x\in \mathbb{R}$
2)$f(3)=2014$
	\flushright \href{https://artofproblemsolving.com/community/c6h565688}{(Link to AoPS)}
\end{problem}



\begin{solution}[by \href{https://artofproblemsolving.com/community/user/29428}{pco}]
	\begin{tcolorbox}Find $f:\mathbb{R}\rightarrow \mathbb{R}$ such that:
1)$f(x)-2f(3x)+f(9x)=x^3+x^2+x$ $\forall x\in \mathbb{R}$
2)$f(3)=2014$\end{tcolorbox}
Is it a real olympiad exercise ?
There are obviously infinity many solutions without general closed form :

Let $h_1(x)$ any function from $[1,9)\to \mathbb R$ such that $h_1(3)=2014$
Let $h_2(x)$ any function from $[1,9)\to \mathbb R$
Let $a\in\mathbb R$

Let $f_k(x)$ a sequence of functions of $[3^{k},3^{k+1})\to \mathbb R$ defined as :
$f_0(x)=h_1(x)$
$f_1(x)=h_1(x)$
$f_{k+2}(x)=(\frac x9)^3+(\frac x9)^2+\frac x9+2f_{k+1}(\frac x3)-f_{k}(\frac x9)$

Let $g_k(x)$ a sequence of functions of $[3^{1-k},3^{2-k})\to \mathbb R$ defined as :
$g_0(x)=h_1(x)$
$g_1(x)=h_1(x)$
$g_{k+2}(x)=x^3+x^2+x+2g_{k+1}(3x)-g_k(9x)$

Let $u_k(x)$ a sequence of functions of $[3^{k},3^{k+1})\to \mathbb R$ defined as :
$u_0(x)=h_2(x)$
$u_1(x)=h_2(x)$
$u_{k+2}(x)=-(\frac x9)^3+(\frac x9)^2-\frac x9+2u_{k+1}(\frac x3)-u_{k}(\frac x9)$

Let $v_k(x)$ a sequence of functions of $[3^{1-k},3^{2-k})\to \mathbb R$ defined as :
$v_0(x)=h_2(x)$
$v_1(x)=h_2(x)$
$v_{k+2}(x)=-x^3+x^2-x+2v_{k+1}(3x)-v_k(9x)$

Define then $f(x)$ as :
$f(0)=a$
Let integer $n\ge 0$ : $\forall x\in[3^n,3^{n+1})$ : $f(x)=f_n(x)$
Let integer $n\ge 2$  : $\forall x\in[3^{1-n},3^{2-n})$ : $f(x)=g_n(x)$
Let integer $n\ge 0$ : $\forall x\in(-3^{n+1},-3^{n})$ : $f(x)=u_n(-x)$
Let integer $n\ge 2$  : $\forall x\in[-3^{2-n},-3^{1-n})$ : $f(x)=v_n(-x)$
\end{solution}



\begin{solution}[by \href{https://artofproblemsolving.com/community/user/168537}{vutuanhien}]
	\begin{tcolorbox}[quote="vutuanhien"]Find $f:\mathbb{R}\rightarrow \mathbb{R}$ such that:
1)$f(x)-2f(3x)+f(9x)=x^3+x^2+x$ $\forall x\in \mathbb{R}$
2)$f(3)=2014$\end{tcolorbox}
Is it a real olympiad exercise ?
There are obviously infinity many solutions without general closed form :

Let $h_1(x)$ any function from $[1,9)\to \mathbb R$ such that $h_1(3)=2014$
Let $h_2(x)$ any function from $[1,9)\to \mathbb R$
Let $a\in\mathbb R$

Let $f_k(x)$ a sequence of functions of $[3^{k},3^{k+1})\to \mathbb R$ defined as :
$f_0(x)=h_1(x)$
$f_1(x)=h_1(x)$
$f_{k+2}(x)=(\frac x9)^3+(\frac x9)^2+\frac x9+2f_{k+1}(\frac x3)-f_{k}(\frac x9)$

Let $g_k(x)$ a sequence of functions of $[3^{1-k},3^{2-k})\to \mathbb R$ defined as :
$g_0(x)=h_1(x)$
$g_1(x)=h_1(x)$
$g_{k+2}(x)=x^3+x^2+x+2g_{k+1}(3x)-g_k(9x)$

Let $u_k(x)$ a sequence of functions of $[3^{k},3^{k+1})\to \mathbb R$ defined as :
$u_0(x)=h_2(x)$
$u_1(x)=h_2(x)$
$u_{k+2}(x)=-(\frac x9)^3+(\frac x9)^2-\frac x9+2u_{k+1}(\frac x3)-u_{k}(\frac x9)$

Let $v_k(x)$ a sequence of functions of $[3^{1-k},3^{2-k})\to \mathbb R$ defined as :
$v_0(x)=h_2(x)$
$v_1(x)=h_2(x)$
$v_{k+2}(x)=-x^3+x^2-x+2v_{k+1}(3x)-v_k(9x)$

Define then $f(x)$ as :
$f(0)=a$
Let integer $n\ge 0$ : $\forall x\in[3^n,3^{n+1})$ : $f(x)=f_n(x)$
Let integer $n\ge 2$  : $\forall x\in[3^{1-n},3^{2-n})$ : $f(x)=g_n(x)$
Let integer $n\ge 0$ : $\forall x\in(-3^{n+1},-3^{n})$ : $f(x)=u_n(-x)$
Let integer $n\ge 2$  : $\forall x\in[-3^{2-n},-3^{1-n})$ : $f(x)=v_n(-x)$\end{tcolorbox}
Oh I forgot
$f$ is continuos function
Sorry
\end{solution}



\begin{solution}[by \href{https://artofproblemsolving.com/community/user/29428}{pco}]
	\begin{tcolorbox}Find $f:\mathbb{R}\rightarrow \mathbb{R}$ such that:
1)$f(x)-2f(3x)+f(9x)=x^3+x^2+x$ $\forall x\in \mathbb{R}$
2)$f(3)=2014$\end{tcolorbox}

\begin{tcolorbox}Oh I forgot
$f$ is continuos function
Sorry\end{tcolorbox}
Let $f(x)=\frac {x^3}{676}+\frac{x^2}{64}+\frac x4+\frac{21773359}{10816}+g(x)$ and problem becomes :

$g(3)=0$
$g(x)-2g(3x)+g(9x)=0$

Let $h(x)=g(x)-g(3x)$ and we get $h(x)=h(3x)$ and so $h(x)=h(3^nx)$ and so, setting $n\to -\infty$ and using continuity, $h(x)=h(0)=0$

So $g(x)=g(3x)$ and so $g(x)=g(3^nx)$ and so, setting $n\to -\infty$ and using continuity, $g(x)=g(0)$ and so $g(x)=0$ (since $g(3)=0$)

Hence the solution $\boxed{f(x)=\frac {x^3}{676}+\frac{x^2}{64}+\frac x4+\frac{21773359}{10816}}$
\end{solution}



\begin{solution}[by \href{https://artofproblemsolving.com/community/user/168537}{vutuanhien}]
	\begin{tcolorbox}
Let $f(x)=\frac {x^3}{676}+\frac{x^2}{64}+\frac x4+\frac{21773359}{10816}+g(x)$\end{tcolorbox}
Wow, nice solution, Patrick
But how can you get this?
\end{solution}



\begin{solution}[by \href{https://artofproblemsolving.com/community/user/29428}{pco}]
	\begin{tcolorbox}[quote="pco"]
Let $f(x)=\frac {x^3}{676}+\frac{x^2}{64}+\frac x4+\frac{21773359}{10816}+g(x)$\end{tcolorbox}
Wow, nice solution, Patrick
But how can you get this?\end{tcolorbox}
Just looking for a peculiar solution : looking for a cubic seemed obvious. Writing then $ax^3+bx^2+cx+d$ and identifying coefficients gives the result.
\end{solution}
*******************************************************************************
-------------------------------------------------------------------------------

\begin{problem}[Posted by \href{https://artofproblemsolving.com/community/user/59765}{hungkg}]
	Find all functions $f:R \to R$ such that $f\left( {xf\left( y \right)} \right) + f\left( {yf\left( z \right)} \right) + f\left( {zf\left( x \right)} \right) = xy + yz + zx,   \forall x,y,z \in R $.
	\flushright \href{https://artofproblemsolving.com/community/c6h565806}{(Link to AoPS)}
\end{problem}



\begin{solution}[by \href{https://artofproblemsolving.com/community/user/29428}{pco}]
	\begin{tcolorbox}Find all functions $f:R \to R$ such that $f\left( {xf\left( y \right)} \right) + f\left( {yf\left( z \right)} \right) + f\left( {zf\left( x \right)} \right) = xy + yz + zx,   \forall x,y,z \in R $.\end{tcolorbox}
Let $P(x,y,z)$ be the assertion $f(xf(y))+f(yf(z))+f(zf(x))=xy+yz+zx$

$P(0,0,0)$ $\implies$ $f(0)=0$
$P(1,1,0)$ $\implies$ $f(f(1))=1$
$P(x,f(1),0)$ $\implies$ $f(x)=f(1)x$
Plugging this back in original equation, we get $f(1)=\pm 1$ and so two solutions :

$\boxed{f(x)=x}$ $\forall x$ and $\boxed{f(x)=-x}$ $\forall x$
\end{solution}
*******************************************************************************
-------------------------------------------------------------------------------

\begin{problem}[Posted by \href{https://artofproblemsolving.com/community/user/59765}{hungkg}]
	Find all functions $f:\mathbb{R}\to \mathbb{R}$ such as $f(0)=0$ and $f\left( {{x}^{3}} \right)-f\left( {{y}^{3}} \right)=\left( x-y \right)\left( f\left( {{x}^{2}} \right)+f\left( {{y}^{2}} \right)+f\left( xy \right) \right)$, $\forall x,y\in \mathbb{R}$.
	\flushright \href{https://artofproblemsolving.com/community/c6h565811}{(Link to AoPS)}
\end{problem}



\begin{solution}[by \href{https://artofproblemsolving.com/community/user/29428}{pco}]
	\begin{tcolorbox}Find all functions $f:\mathbb{R}\to \mathbb{R}$ such as $f(0)=0$ and $f\left( {{x}^{3}} \right)-f\left( {{y}^{3}} \right)=\left( x-y \right)\left( f\left( {{x}^{2}} \right)+f\left( {{y}^{2}} \right)+f\left( xy \right) \right)$, $\forall x,y\in \mathbb{R}$.\end{tcolorbox}
Let $P(x,y)$ be the assertion $f(x^3)-f(y^3)=(x-y)(f(x^2)+f(y^2)+f(xy))$

Note that $P(1,0)$ $\implies$ $f(0)=0$ and so the condition $f(0)=0$ in the statement is useless.

Adding $P(\sqrt[3]x,0)$ with $P(-\sqrt[3]x,0)$ implies $f(-x)=-f(x)$

$P(x,1)$ $\implies$ $f(x^3)-f(1)=(x-1)(f(x^2)+f(1)+f(x))$
$P(-x,1)$ $\implies$ $-f(x^3)-f(1)=(-x-1)(f(x^2)+f(1)-f(x))$
Adding these two lines, we get $f(x^2)=xf(x)$

Then $P(x,0)$ $\implies$ $f(x^3)=xf(x^2)=x^2f(x)$

Replacing $f(x^3)$ by $x^2f(x)$ and $f(x^2)$ by $xf(x)$ in $P(x,1)$, we get $f(x)=f(1)x$

Hence the result $\boxed{f(x)=ax}$ $\forall x$, which indeed is a solution, whatever is $a\in\mathbb R$
\end{solution}



\begin{solution}[by \href{https://artofproblemsolving.com/community/user/59765}{hungkg}]
	Thanh you very much sir PCO because of your Solution beautiful.
\end{solution}
*******************************************************************************
-------------------------------------------------------------------------------

\begin{problem}[Posted by \href{https://artofproblemsolving.com/community/user/125553}{lehungvietbao}]
	Let $k$ be a real number. Determine all functions $f:\mathbb R\to\mathbb R$ such that 
\[f(x)+(f(y))^2=k\cdot f(x+y^2) \quad \forall x,y\in\mathbb R\]
	\flushright \href{https://artofproblemsolving.com/community/c6h565932}{(Link to AoPS)}
\end{problem}



\begin{solution}[by \href{https://artofproblemsolving.com/community/user/29428}{pco}]
	\begin{tcolorbox}Let $k$ be a real number. Determine all functions $f:\mathbb R\to\mathbb R$ such that 
\[f(x)+(f(y))^2=k\cdot f(x+y^2) \quad \forall x,y\in\mathbb R\]\end{tcolorbox}
Let $P(x,y)$ be the assertion $f(x)+f(y)^2=kf(x+y^2)$
Let $a=f(0)$

If $k\ne 1$, $P(x,0)$ $\implies$ $f(x)=\frac{a^2}{k-1}$ is constant. Plugging this back in original equaiton, we get two solutions :
$\boxed{f(x)=0}$ $\forall x$ and $\boxed{f(x)=k-1}$ $\forall x$

If $k=1$ :
$P(x,0)$ $\implies$ $a=0$
$P(0,y)$ $\implies$ $f(y)^2=f(y^2)$ So that $P(x,y)$ becomes $f(x+y^2)=f(x)+f(y^2)$ and since $f(y^2)=f(y)^2\ge 0$ :
$f(x)$ is a non decreasing function such that $f(x+y)=f(x)+f(y)$ $\forall x$, $\forall y\ge 0$ and so $f(x)=cx$ $\forall x\ge 0$

Plugging this back in orginal equaiton, we get $c\in\{0,1\}$
Let then $x<0$. Choosing $y>0$ such that $x+y^2>0$, $P(x,y)$ implies $f(x)=cx$

Hence the two solutions when $k=1$ :
$\boxed{f(x)=0}$ $\forall x$ and $\boxed{f(x)=x}$ $\forall x$
\end{solution}
*******************************************************************************
-------------------------------------------------------------------------------

\begin{problem}[Posted by \href{https://artofproblemsolving.com/community/user/125553}{lehungvietbao}]
	Determine all functions $f$ defined on $\mathbb R$  such that \[ f (f (x)) = k \cdot  x ^ {9} \quad \forall x\in\mathbb R\]
	\flushright \href{https://artofproblemsolving.com/community/c6h565934}{(Link to AoPS)}
\end{problem}



\begin{solution}[by \href{https://artofproblemsolving.com/community/user/29428}{pco}]
	\begin{tcolorbox}Determine all functions $f$ defined on $\mathbb R$  such that \[ f (f (x)) = k \cdot  x ^ {9} \quad \forall x\in\mathbb R\]\end{tcolorbox}
1) If $k=0$
==========
Equation is then $f(f(x))=0$ and general solution is:
Let $A\subseteq\mathbb R$ with $0\in A$
Let $h(x)$ any function from $\mathbb R\to A$
$f(x)=(1-1_{A}(x))h(x)$

2) If $k>0$
==============
Let $g(x)=|k|^{\frac 18}f(|k|^{-\frac 18}x)$ and equation becomes $g(g(x))=x^9$ whose solution is rather simple :
If $x>1$, we can write $\ln(\ln x))=\ln 9(\left\lfloor\frac{\ln(\ln x)}{\ln 9}\right\rfloor+\left\{\frac{\ln(\ln x)}{\ln 9}\right\})$ and so :

$x=(e^{\left\{\frac{\ln(\ln x)}{\ln 9}\right\}})^{9^{\left\lfloor\frac{\ln(\ln x)}{\ln 9}\right\rfloor}}$

So, any $x\notin\{-1,0,+1\}$ may be written in a unique way as $x=s_1 a^{s_29^n}$ where $s_1,s_2\in\{-1,+1\}$, $a\in[e,e^9)$ and $n\in\mathbb Z$

Let $A=\{-1,+1\}^2\times [e,e^9)\times \mathbb Z$ and $u(x)$ the above bijection from $A\to\mathbb R\setminus\{-1,0,1\}$ such that $g(s_1,s_2,a,n)=s_1 a^{s_29^n}$

It remains to split $A$ in any two equinumerous subsets $B_1,B_2$.
Let $v(s_1,s_2,a,n)=(s'_1,s'_2,a',n')$ any bijection from $B_1\to B_2$

And so $g(x)$ may be defined as :
If $x\in\{-1,0,1\}$, then $g(x)=$ any value in $\{-1,0,+1\}$
If $x\notin\{-1,0,+1\}$ :
If $u^{-1}(x)\in B_1$, then $g(x)=u(v(u^{-1}(x)))$
If $u^{-1}(x)\in B_2$ and $v^{-1}(u^{-1}(x))=(s_1,s_2,a,n)$, then $g(x)=u(s_1,s_2,a,n+1)$

3) If $k<0$
==============
Let $g(x)=|k|^{\frac 18}f(|k|^{-\frac 18}x)$ and equation becomes $g(g(x))=-x^9$ whose solution may be built in a quite similar way :
If $x\in\{-1,0,1\}$, then $g(x)=0$ 
If $x\notin\{-1,0,+1\}$ :
If $u^{-1}(x)\in B_1$, then $g(x)=u(v(u^{-1}(x)))$
If $u^{-1}(x)\in B_2$ and $v^{-1}(u^{-1}(x))=(s_1,s_2,a,n)$, then $g(x)=u(-s_1,s_2,a,n+1)$
\end{solution}
*******************************************************************************
-------------------------------------------------------------------------------

\begin{problem}[Posted by \href{https://artofproblemsolving.com/community/user/125553}{lehungvietbao}]
	Find all functions $f: \mathbb R \to\mathbb R$ such that \[f ^ {n} (x) =-x \quad\ \forall x\in\mathbb R\]
Note that $ \underbrace{f{\circ}f{\circ}\cdots{\circ}f}_{n \textrm{ times}}(x)=f^{n}(x) $.
	\flushright \href{https://artofproblemsolving.com/community/c6h565938}{(Link to AoPS)}
\end{problem}



\begin{solution}[by \href{https://artofproblemsolving.com/community/user/29428}{pco}]
	\begin{tcolorbox}Find all functions $f: \mathbb R \to\mathbb R$ such that \[f ^ {n} (x) =-x \quad\ \forall x\in\mathbb R\]
Note that $ \underbrace{f{\circ}f{\circ}\cdots{\circ}f}_{n \textrm{ times}}(x)=f^{n}(x) $.\end{tcolorbox}
General solution :

Let $h(x)$ any function from $\mathbb R\to\{-1,+1\}$
Let $A=h(\mathbb R_{>0})$
Let $d_1>d_2>... >d_k=1$ the $k$ oddd divisors of $n$
Let $a_i=\frac n{d_i}$ $\forall i\in[1,k]$
Let $B_1,B_2,...,B_k$ a split of $A$ in $k$ subsets such that either $|B_i|=+\infty$, either $|B_i|\equiv 0\pmod{a_i}$
For all $i\in[1,k]$ :
$\text{  }$If $a_i=1$, let $B_{i,1}=B_i$ and $g_{i,1}(x)=x$
$\text{  }$If $a_i>1$ :
$\text{     -- }$Let $B_{i,1},B_{i,2}, ...,B_{i,a_i}$ any split of $B_i$ in $a_i$ equipotent subsets.
$\text{     -- }$$\forall j\in[1,a_i)$ let $g_{i,j}(x)$ any bijection from $B_{i,j}\to B_{i,j+1}$
$\text{     -- }$Let $g_{i,a_i}(x)=g_{i,1}^{-1}(g_{i,2}^{-1}( ... g(_{i,a_i-1}^{-1}(x)...))$ bijection from $B_{i,a_i}\to B_{i,1}$

$f(x)$ can then be defined :

$f(0)=0$
If $x\ne 0$ and $x\in A$ : 
Let $i,j$ such that $x\in B_{i,j}$
If $j<a_i$, then $f(x)=g_{i,j}(x)$
If $j=a_i$, then $f(x)=-g_{i,a_i}(x)$
If $x\ne 0$ and $x\notin A$ :
Let $i,j$ such that $-x\in B_{i,j}$
If $j<a_i$, then $f(x)=-g_{i,j}(-x)$
If $j=a_i$, then $f(x)=g_{i,a_i}(-x)$
\end{solution}
*******************************************************************************
-------------------------------------------------------------------------------

\begin{problem}[Posted by \href{https://artofproblemsolving.com/community/user/125553}{lehungvietbao}]
	Find all functions $f:\mathbb{R} \to \mathbb{R^+}$ such that \[\begin{cases} f(x^2)=[f(x)]^2-2xf(x) \\ f(-x) = f(x-1) \\ \text{ If } 1<x<y \text{ then} f(x)<f(y) \end{cases}\]
	\flushright \href{https://artofproblemsolving.com/community/c6h565947}{(Link to AoPS)}
\end{problem}



\begin{solution}[by \href{https://artofproblemsolving.com/community/user/29428}{pco}]
	\begin{tcolorbox}Find all functions $f:\mathbb{R} \to \mathbb{R^+}$ such that \[\begin{cases} f(x^2)=[f(x)]^2-2xf(x) \\ f(-x) = f(x-1) \\ \text{ If } 1<x<y \text{ then} f(x)<f(y) \end{cases}\]\end{tcolorbox}
I suppose we must read $[f(x)]^2$ as $(f(x))^2$ and not $\lfloor f(x)\rfloor)^2$. If so :

$f(x^2)=f(x)^2-2xf(x)$
$f((-x)^2)=f(-x)^2+2xf(-x)=f(x-1)^2+2xf(x-1)$
Subtracting, we get $(f(x)-f(x-1)-2x)(f(x)+f(x-1))=0$ and so, since $f(x)>0$ $\forall x$ : $f(x)=f(x-1)+2x$

From there, simple induction gives $f(x+n)=f(x)+2nx+n^2+n$ and so $f(n)=n^2+n+f(0)$ $\forall n\in\mathbb Z$ and, plugging in first equation, $f(0)=1$

So $f(x)=x^2+x+1$ $\forall x\in\mathbb Z$
Let $A=\{x$ such that $f(x)=x^2+x+1\}$ : $\mathbb Z\subseteq A$

If $a\in A$, the equation $f(x+n)=f(x)+2nx+n^2+n$ implies $a+n\in A$
If $a\in A_{>0}$, then $a^2+a+1=f(a)=f((\sqrt a)^2)=f(\sqrt a)^2-2\sqrt af(\sqrt a)$ and so :
$(f(\sqrt a)-a-\sqrt a-1)(f(\sqrt a)+a-\sqrt a+1)$ and, since $a-\sqrt a+1>0$, we get $f(\sqrt a)=a+\sqrt a+1$
So $\sqrt a\in A$

It's then easy to conclude that $A$ is dense in $[1,+\infty)$ and last property implies $f(x)=x^2+x+1$ $\forall x\ge 1$
And since $a\in A$ implies $a+n\in A$, we get the result :

$\boxed{f(x)=x^2+x+1}$ $\forall x$, which indeed is a solution.
\end{solution}



\begin{solution}[by \href{https://artofproblemsolving.com/community/user/167483}{Math-lover123}]
	http://www.artofproblemsolving.com/Forum/viewtopic.php?p=3000070&sid=556f1efa98e8ddfd199bffff276d4bea#p3000070
\end{solution}
*******************************************************************************
-------------------------------------------------------------------------------

\begin{problem}[Posted by \href{https://artofproblemsolving.com/community/user/125553}{lehungvietbao}]
	For which integers $k$ does  there exists a function $f:\mathbb N \to \mathbb Z$ such that
(a) $f(1995)=1996$, and 
(b) $f(xy)=f(x)+f(y)+kf(\gcd(x,y)) $, for all $x,y\in\mathbb N$ ?
	\flushright \href{https://artofproblemsolving.com/community/c6h566049}{(Link to AoPS)}
\end{problem}



\begin{solution}[by \href{https://artofproblemsolving.com/community/user/29428}{pco}]
	\begin{tcolorbox}For which integers $k$ does  there exists a function $f:\mathbb N \to \mathbb Z$ such that
(a) $f(1995)=1996$, and 
(b) $f(xy)=f(x)+f(y)+kf(\gcd(x,y)) $, for all $x,y\in\mathbb N$ ?\end{tcolorbox}
As usual with your problems (but not in this forum), I suppose $0\in\mathbb N$

If so, what is the definition, in your country, of $\gcd(2,0)$ ?
\end{solution}



\begin{solution}[by \href{https://artofproblemsolving.com/community/user/89198}{chaotic_iak}]
	Usually $\gcd(k,0) = |k|$ for any integer $k$; that's the extension of $\gcd$ that I know. After all, $k$ divides $0$. Except when $k = 0$, in which $\gcd(0,0) = 0$, although I'm not sure how it makes sense.
\end{solution}



\begin{solution}[by \href{https://artofproblemsolving.com/community/user/125553}{lehungvietbao}]
	Dear Mr. Patrick

This problem is come from Czech Republic . So i think $0\notin \mathbb N$  :blush:
\end{solution}



\begin{solution}[by \href{https://artofproblemsolving.com/community/user/29428}{pco}]
	\begin{tcolorbox}For which integers $k$ does  there exists a function $f:\mathbb N \to \mathbb Z$ such that
(a) $f(1995)=1996$, and 
(b) $f(xy)=f(x)+f(y)+kf(\gcd(x,y)) $, for all $x,y\in\mathbb N$ ?\end{tcolorbox}
Let us consider $\mathbb N$ is the set of positive integers.

$f(x)=1996$ $\forall x$ fits when $k=-1$
$f(x)=1996 v_3(x)$ $\forall x$ fits when $k=0$

If $k\notin\{-1,0\}$ :

Set $y=1$ and equation is $(k+1)f(1)=0$ and so $f(1)=0$ so that equation implies $f(xy)=f(x)+f(y)$ $\forall$ coprime $x,y$

Let $p$ prime : $f(p^n)=f(p^{n-1})+(k+1)f(p)$ and so $f(p^n)=(nk-k+n)f(p)$ $\forall n>0$

So $f(\prod p_i^{n_i})=\sum (n_ik-k+n_i)f(p_i)$
If $f(p)=0$ $\forall p$ prime, then $f(1995)=0$, impossible, and so $\exists p$ such that $f(p)\ne 0$

Let then $x=p^2$ and $y=p^2$ and equation becomes : $f(p^4)=(k+2)f(p^2)$ and so $k(k+1)f(p)=0$, impossible.

Hence the answer : $\boxed{k\in\{-1,0\}}$
\end{solution}



\begin{solution}[by \href{https://artofproblemsolving.com/community/user/199494}{IMI-Mathboy}]
	Putting $P(a,a)$ we have $f(a^2)=f(a)(k+2)$, and $P(a^2,a)$  $\to$ $f(a^3)=f(a)(2k+3)$, then  $P(a^3,a)$ and $P(a^2,a^2)$  we have $f(a^4)=f(a)(k+2)^2=f(a)(3k+4)$ from this we have $(k+2)^2=(3k+4)$ so $k={0,-1}$
\end{solution}
*******************************************************************************
-------------------------------------------------------------------------------

\begin{problem}[Posted by \href{https://artofproblemsolving.com/community/user/125553}{lehungvietbao}]
	Given $n>1$, find all real-valued functions $f_{i}(x)$ on the reals  such that for all $x,y$ we have 
\[f_{1}(x)+f_{1}(y)=f_{2}(x)f_{2}(y) \\f_{2}(x^2)+f_{2}(y^2)=f_{3}(x)f_{3}(y)\\f_{3}(x^3)+f_{3}(y^3)=f_{4}(x)f_{4}(y)\\ \cdots \\ f_{n}(x^n)+f_{n}(y^n)=f_{1}(x)f_{1}(y)\]
	\flushright \href{https://artofproblemsolving.com/community/c6h566088}{(Link to AoPS)}
\end{problem}



\begin{solution}[by \href{https://artofproblemsolving.com/community/user/29428}{pco}]
	\begin{tcolorbox}Given $n>1$, find all real-valued functions $f_{i}(x)$ on the reals  such that for all $x,y$ we have 
\[f_{1}(x)+f_{1}(y)=f_{2}(x)f_{2}(y) \\f_{2}(x^2)+f_{2}(y^2)=f_{3}(x)f_{3}(y)\\f_{3}(x^3)+f_{3}(y^3)=f_{4}(x)f_{4}(y)\\ \cdots \\ f_{n}(x^n)+f_{n}(y^n)=f_{1}(x)f_{1}(y)\]\end{tcolorbox}
$(1)$ : $f_k(x^k)+f_k(x^k)=f_{k+1}(x)^2$
$(2)$ : $f_k(y^k)+f_k(y^k)=f_{k+1}(y)^2$
$(3)$ : $f_k(x^k)+f_k(y^k)=f_{k+1}(x)f_{k+1}(y)$
$(1)+(2)-2\times (3)$ : $f_{k+1}(x)=f_{k+1}(y)$ and so $f_{k+1}(x)=c_{k+1}$ is constant

So each function is constant and we easily get $c_{n-k}=2^{1-2^{k+1}}c_1^{2^{k+1}}$ and so $c_1=2^{1-2^{n}}c_1^{2^{n}}$ and so $c_1=0$ or $c_1=2$

Hence the two solutions :

$\boxed{f_k(x)=0}$ $\forall x$ and $\boxed{f_k(x)=2}$ $\forall x$ which indeed are solutions.
\end{solution}
*******************************************************************************
-------------------------------------------------------------------------------

\begin{problem}[Posted by \href{https://artofproblemsolving.com/community/user/168537}{vutuanhien}]
	Find all $f:\mathbb{R}\rightarrow \mathbb{R}$ such that $f(xf(y))+f(yf(x))=2xy$
	\flushright \href{https://artofproblemsolving.com/community/c6h566131}{(Link to AoPS)}
\end{problem}



\begin{solution}[by \href{https://artofproblemsolving.com/community/user/29428}{pco}]
	\begin{tcolorbox}Find all $f:\mathbb{R}\rightarrow \mathbb{R}$ such that $f(xf(y))+f(yf(x))=2xy$\end{tcolorbox}
Let $P(x,y)$ be the assertion $f(xf(y))+f(yf(x))=2xy$

$P(0,0)$ $\implies$ $f(0)=0$
$P(1,1)$ $\implies$ $f(f(1))=1$ and so $f(1)\ne 0$
$P(x,f(1))$ $\implies$ $f(x)+f(f(1)f(x))=2xf(1)$ and so $f(x)$ is injective.

Subtracting $P(-x,-x)$ from $P(x,x)$, we get $f(xf(x))=f(-xf(-x))$ and so, since injective and $f(0)=0$ : $f(-x)=-f(x)$ $\forall x$
As a consequence, $f(x)$ solution implies $-f(x)$ solution. So WLOG consider $f(1)>0$

$P(1,f(1))$ $\implies$ $f(f(1)^2)=f(1)$ and so $f(1)^2=1$ and so $f(1)=1$

Let $x\ne 0$ (so that, since injective, $f(x)\ne 0$)
$P(\frac 1x,\frac 1x)$ $\implies$ $f(\frac 1xf(\frac 1x))=\frac 1{x^2}$
$P(x,\frac 1xf(\frac 1x))$ $\implies$ $f(\frac 1xf(\frac 1x)f(x))=f(\frac 1x)$ and so $f(\frac 1x)=\frac 1{f(x)}$

Let $x\ne 0$
$P(x,1)$ $\implies$ $f(x)+f(f(x))=2x$
$P(\frac 1x,1)$ $\implies$ $\frac 1{f(x)}+\frac 1{f(f(x))}=\frac 2x$
Eliminating $f(f(x))$ between these two lines, we get $f(x)=x$

Hence the two solutions : $\boxed{f(x)=x}$ $\forall x$ and $\boxed{f(x)=-x}$ $\forall x$ which indeed are solutions.
\end{solution}
*******************************************************************************
-------------------------------------------------------------------------------

\begin{problem}[Posted by \href{https://artofproblemsolving.com/community/user/68025}{Pirkuliyev Rovsen}]
	Does it exist an injective function $f: \mathbb{R}\to\mathbb{R}$ such that $3f^3(x^5-x^4+x^3) - f^2(x^5-x^4+x^3) \ge f^4(x)+4$ for all $x \in \mathbb{R}$?
	\flushright \href{https://artofproblemsolving.com/community/c6h566289}{(Link to AoPS)}
\end{problem}



\begin{solution}[by \href{https://artofproblemsolving.com/community/user/64716}{mavropnevma}]
	You should make it clear if $f^k(t)$ is the $k$-th iterate of $f$ computed at $t$ (my favourite interpretation), or $f(t)^k$, the $k$-th power of $f(t)$.
\end{solution}



\begin{solution}[by \href{https://artofproblemsolving.com/community/user/29428}{pco}]
	\begin{tcolorbox}Does it exist an injective function $f: \mathbb{R}\to\mathbb{R}$ such that $3f^3(x^5-x^4+x^3) - f^2(x^5-x^4+x^3) \ge f^4(x)+4$ for all $x \in \mathbb{R}$?\end{tcolorbox}
If these are k-th powers and not k-th iterate, then (since $u^4 - 3u^3 + u^2 + 4 = (u-2)^2(u^2+u+1)$)

Setting $x=0$, we get $f(0)=2$
Setting $x=1$, we get $f(1)=2$

So no injective solution.
\end{solution}
*******************************************************************************
-------------------------------------------------------------------------------

\begin{problem}[Posted by \href{https://artofproblemsolving.com/community/user/125553}{lehungvietbao}]
	Find all function $f:\mathbb Q\to\mathbb Q$ such that
\[f(x+y)+f(x-y)=2f(x)+2f(y) \quad\forall x,y\in\mathbb Q\]
	\flushright \href{https://artofproblemsolving.com/community/c6h566422}{(Link to AoPS)}
\end{problem}



\begin{solution}[by \href{https://artofproblemsolving.com/community/user/29428}{pco}]
	\begin{tcolorbox}Find all function $f:\mathbb Q\to\mathbb Q$ such that
\[f(x+y)+f(x-y)=2f(x)+2f(y) \quad\forall x,y\in\mathbb Q\]\end{tcolorbox}
Let $P(x,y)$ be the assertion $f(x+y)+f(x-y)=2f(x)+2f(y)$

$P(0,0)$ $\implies$ $f(0)=0$

$P((n+1)x,x)$ $\implies$ $f((n+2)x)=2f((n+1)x)-f(nx)+2f(x)$ and simple induction gives $f(nx)=n^2f(x)$ and so :
$f(n)=f(1)n^2$ and $f(\frac pq)=f(1)\left(\frac pq\right)^2$

So $\boxed{f(x)=ax^2}$ $\forall x\in \mathbb Q$, which indeed is a solution, whatever is $a\in\mathbb Q$
\end{solution}
*******************************************************************************
-------------------------------------------------------------------------------

\begin{problem}[Posted by \href{https://artofproblemsolving.com/community/user/125553}{lehungvietbao}]
	Find all functions$ f:\mathbb R\to\mathbb R$ such that
\[f(x)f(yf(x)-1)=x^2f(y)-f(x) \quad \forall x,y\in\mathbb R\]
	\flushright \href{https://artofproblemsolving.com/community/c6h566423}{(Link to AoPS)}
\end{problem}



\begin{solution}[by \href{https://artofproblemsolving.com/community/user/29428}{pco}]
	\begin{tcolorbox}Find all functions$ f:\mathbb R\to\mathbb R$ such that
\[f(x)f(yf(x)-1))=x^2f(y)-f(x) \quad \forall x,y\in\mathbb R\]\end{tcolorbox}
$\boxed{f(x)=0}$ $\forall x$ is a solution. So let us from now look only for non allzero solutions.
Let $P(x,y)$ be the assertion $f(x)f(yf(x)-1)=x^2f(y)-f(x)$
Let $u$ such that $f(u)\ne 0$

$P(1,1)$ $\implies$ $f(1)f(f(1)-1)=0$ and so $\exists z$ such that $f(z)=0$
Then $P(z,u)$ $\implies$ $z=0$
And so $f(x)=0$ $\iff$ $x=0$
So $f(1)=1$

$P(u,0)$ $\implies$ $f(-1)=-1$

Let $x\ne 0$ : 
$P(x,\frac 1{f(x)})$ $\implies$ $f(\frac 1{f(x)})=\frac{f(x)}{x^2}$
$P(\frac 1{f(x)},x^2y)$ $\implies$ $f(x)f(yf(x)-1)=\frac{x^2f(x^2y)}{f(x)^2}-f(x)$

And so (comparing with $P(x,y)$) : newassertion $Q(x,y)$ : $f(x^2y)=f(x)^2f(y)$ $\forall x\ne 0$, $\forall y$, still true when $x=0$
$Q(x,1)$ $\implies$ $f(x^2)=f(x)^2$ and so $Q(x,y)$ implies $f(xy)=f(x)f(y)$ $\forall x\ge0$, $\forall y$
And since $f(-1)=-1$, we also have $f(-x)=-f(x)$ $\forall x\ge 0$.
So new assertion $R(x,y)$ :$f(xy)=f(x)f(y)$ $\forall x$, $\forall y$

$P(1,x+1)$ $\implies$ $f(x+1)=f(x)+1$ and so, multiplying by $f(y)$ : $f(xy+y)=f(xy)+f(y)$ and so $f(x+y)=f(x)+f(y)$

So we got $f(1)=1$ and $f(xy)=f(x)f(y)$ and $f(x+y)=f(x)+f(y)$ whose result is quite classic : $\boxed{f(x)=x}$ $\forall x$, which indeed is a solution.
\end{solution}



\begin{solution}[by \href{https://artofproblemsolving.com/community/user/73237}{msinghal}]
	Think there's a parentheses mismatch?

[color=#f00][mod edit: it was an extra parentheses and caused no misunderstanding issues -- fixed now.][\/color]
\end{solution}
*******************************************************************************
-------------------------------------------------------------------------------

\begin{problem}[Posted by \href{https://artofproblemsolving.com/community/user/125553}{lehungvietbao}]
	Find all functiosn $f:\mathbb{R}\to \mathbb{R}$ such that  \[f\left( x+y \right)=f\left( x \right)f\left( y \right)-c\sin x\sin y \quad \forall x,y\in\mathbb R\] Where $c>1$ is a constant.
	\flushright \href{https://artofproblemsolving.com/community/c6h566424}{(Link to AoPS)}
\end{problem}



\begin{solution}[by \href{https://artofproblemsolving.com/community/user/29428}{pco}]
	\begin{tcolorbox}Find all functiosn $f:\mathbb{R}\to \mathbb{R}$ such that  \[f\left( x+y \right)=f\left( x \right)f\left( y \right)-c\sin x\sin y \quad \forall x,y\in\mathbb R\] Where $c>1$ is a constant.\end{tcolorbox}
Let $P(x,y)$ be the assertion $f(x+y)=f(x)f(y)-c\sin x\sin y$
Let $a=f(\frac{\pi}2)$

$P(x,\frac{\pi}2)$ $\implies$ $f(x+\frac{\pi}2)=af(x)-c\sin x$
$P(x+\frac{\pi}2,\frac{\pi}2)$ $\implies$ $f(x+\pi)=af(x+\frac{\pi}2)-c\cos x$ $=a^2f(x)-ac\sin x-c\cos x$

$P(\frac{\pi}2,\frac{\pi}2)$ $\implies$ $f(\pi)=a^2-c$
$P(x,\pi)$ $\implies$ $f(x+\pi)=f(x)f(\pi)$ $=(a^2-c)f(x)$

So $a^2f(x)-ac\sin x-c\cos x=(a^2-c)f(x)$ and so  $f(x)=a\sin x+\cos x$ 

Plugging this back in original equation, we get $a^2=c-1$

Hence the result : $\boxed{f(x)=\cos x+\sqrt{c-1}\sin x}$ $\forall x$ and $\boxed{f(x)=\cos x-\sqrt{c-1}\sin x}$ $\forall x$
\end{solution}
*******************************************************************************
-------------------------------------------------------------------------------

\begin{problem}[Posted by \href{https://artofproblemsolving.com/community/user/125553}{lehungvietbao}]
	Let $f: \mathbb {R}^{+} \to  \left[\frac{-1}{ 2}, \frac{1}{2}\right ]$ such that $f (y) - f (x)\geq  2f (x)f (y) \quad \forall y> x> 0$. Show that $f (x) = 0 \quad \forall x> 0$.
	\flushright \href{https://artofproblemsolving.com/community/c6h566432}{(Link to AoPS)}
\end{problem}



\begin{solution}[by \href{https://artofproblemsolving.com/community/user/29428}{pco}]
	\begin{tcolorbox}Let $f: \mathbb {R}^{+} \to  \left[\frac{-1}{ 2}, \frac{1}{2}\right ]$ such that $f (y) - f (x)\geq  2f (x)f (y) \quad \forall y> x> 0$. Show that $f (x) = 0 \quad \forall x> 0$.\end{tcolorbox}
So $f(y)(1-2f(x))\ge f(x)$ $\forall y>x>0$
This implies $f(x)\ne \frac 12$ and so $f(x)<\frac 12$ and so assertion $P(x,y)$ : $f(y)\ge \frac {f(x)}{1-2f(x)}$ $\forall y>x>0$

Note that this implies $f(y)-f(x)\ge \frac {2f(x)^2}{1-2f(x)}\ge 0$ $\forall y>x>0$ and so $f(x)$ is non decreasing.

Let $a>0$
Let then any increasing positive sequence $\{a_n\}_{n\ge 0}$ with limit $a>0$ and the sequence $b_0=f(a_0)$ and $b_{n+1}=\frac{b_n}{1-2b_n}$
It's easy to show thru induction that $f(a)\ge f(a_n)\ge b_n$ $\forall n$

If $b_0>0$, the sequence $\{b_n\}$ is an increasing sequence whose limit is $+\infty$, impossible. 
So $b_0\le 0$ and $f(a_0)\le 0$ $\forall a_0>0$ an so $f(x)\le 0$ $\forall x>0$
Then the sequence $\{b_n\}$ is an increasing sequence whose limit is $0$ and so $f(a)\ge 0$ $\forall a>0$

So $f(x)=0$ $\forall x>0$
Q.E.D.
\end{solution}
*******************************************************************************
-------------------------------------------------------------------------------

\begin{problem}[Posted by \href{https://artofproblemsolving.com/community/user/125553}{lehungvietbao}]
	Find all functions $f: \mathbb Q\to\mathbb  Q$ satisfying the conditions:
a) $f (0, 0) = 0$
b) $f (x, y) = f (y, x) \quad \forall x, y \in\mathbb Q$
c) $f (x, f (y, z)) = f (f (x, y), z)\quad \forall x, y, z \in\mathbb Q$ 
d) $f (x + z, y + z) = f (x, y) + z\quad \forall x, y, z \in\mathbb Q$.
	\flushright \href{https://artofproblemsolving.com/community/c6h566433}{(Link to AoPS)}
\end{problem}



\begin{solution}[by \href{https://artofproblemsolving.com/community/user/29428}{pco}]
	\begin{tcolorbox}Find all functions $f: \mathbb Q\to\mathbb  Q$ satisfying the conditions:
a) $f (0, 0) = 0$
b) $f (x, y) = f (y, x) \quad \forall x, y \in\mathbb Q$
c) $f (x, f (y, z)) = f (f (x, y), z)\quad \forall x, y, z \in\mathbb Q$ 
d) $f (x + z, y + z) = f (x, y) + z\quad \forall x, y, z \in\mathbb Q$.\end{tcolorbox}
Let $g(x)=f(x,0)$

d) $\implies$ $f(x,y)=g(x-y)+y$
a) $\implies$ $g(0)=0$
b) $\implies$ assertion $B(x)$ : $g(-x)=g(x)-x$
c) $\implies$ assertion $P(x,y)$ : $g(x-g(y))+g(y)=g(g(x-y)+y)$

1) Preliminary results
=============
$g(x)=0$ $\forall x$ is not a solution (look at b) ). So let $u$ such that $g(u)=\frac pq\ne 0$
Note that $g(x)$ solution implies $-g(-x)$ solution. So WLOG $p,q$ positive integers

Let $A=\{x$ such that $g(x)=x\}$. $A\subseteq g(\mathbb Q)$
$P(x,0)$ $\implies$ $g(g(x))=g(x)$ and so $g(x)\in A$ $\forall x$ and $A=g(\mathbb Q)$

$P(x+y,y)$ shows that $g(\mathbb Q)$ is stable thru addition.

2) $g(x)\ge x$ $\forall x$
================
If $\exists m,n$ positive integers such that $-\frac mn\in A$, then :
$\frac pq\in A$ $\implies$ $pm=qm\frac pq\in A$ and so $g(pm)=pm$
$-\frac mn\in A$ $\implies$ $-pm=np(-\frac mn)\in A$ and so $g(-pm)=-pm$
But then $B(pm)$ is not true
So $x\in A$ $\implies$ $x\ge 0$

$P(x,x)$ $\implies$ $g(x-g(x))=0$
$B(x-g(x))$ $\implies$ $g(x)-x\in A$ $\forall x$ and so $g(x)\ge x$ $\forall x$
Q.E.D.

3) $g(x+ng(x))=(n+1)g(x)$ $\forall x$, $\forall$ integer $n\ge 0$
===========================================
$P(x,x)$ $\implies$ $g(x-g(x))=0)$ and so $g(x+(n-1)g(x))=ng(x)$ is true when $n=0$
Suppose $g(x+(n-1)g(x))=ng(x)$ true for some $n\ge 0$. Then :
a) : $P(2x,x)$ $\implies$ $g(2x-g(x))+g(x)=g(g(x)+x)$
b) : $P(x+g(x),x)$ $\implies$ $2g(x)=g(g(x)+x)$
c) : $P(2x+(n-1)g(x),x+(n-1)g(x))$ $\implies$ $g(x+ng(x))=g(2x-g(x))+ng(x)$
a) - b) +c) : $g(x+ng(x))=(n+1)g(x)$
Q.E.D.

4) $g(x)=x$ $\forall x>0$
================
$g(-x)=g(x)-x$ and so $g(x)-x\in A$ and so $k(g(x)-x)\in A$ $\forall x$, $\forall k\ge 0$

Let $x= \frac ab>0$ with $a,b>0$
Let $g(x)-x=\frac uv\ge 0$ with $v>0$ and $u\ge 0$
Choosing $n=bu-1$ in 3) above, we get $x+ng(x)=\frac ab+(bu-1)(\frac ab+\frac uv)$ $=(av+bu-1)\frac uv$
Since $av+bu-1\ge 0$, $(av+bu-1)\frac uv\in A$ and so $g(x+ng(x))=g((av+bu-1)\frac uv)=(av+bu-1)\frac uv$
But $g(x+ng(x))=(n+1)g(x)=bu(\frac ab+\frac uv)$

So $(av+bu-1)\frac uv=bu(\frac ab+\frac uv)$ and so $u=0$ and $g(x)=x$
Q.E.D.

5) Solutions for $g(x)$
==============
We got $g(x)=x$ $\forall x\ge 0$ 
$B(x)$ implies then $g(x)=0$ $\forall x\le 0$

And so $g(x)=\max(x,0)$ which indeed is a solution

And since $g(x)$ solution implies $-g(-x)$ solution, we also get $-\max(-x,0)$ and so $\min(x,0)$

6) Solutions for $f(x,y)$
===============
$f(x,y)=g(x-y)+y$ and so two solutions :

$\boxed{f(x,y)=\max(x,y)}$ which indeed is a solution

$\boxed{f(x,y)=\min(x,y)}$ which indeed is a solution.
\end{solution}
*******************************************************************************
-------------------------------------------------------------------------------

\begin{problem}[Posted by \href{https://artofproblemsolving.com/community/user/125553}{lehungvietbao}]
	Find all real values ​​of $a$ such that there is only one function $f$ which satisfies the following equality \[f (x ^ 2 + y + f (y)) = (f (x)) ^ 2 + ay \quad\forall x,y\in\mathbb R\]
	\flushright \href{https://artofproblemsolving.com/community/c6h566437}{(Link to AoPS)}
\end{problem}



\begin{solution}[by \href{https://artofproblemsolving.com/community/user/29428}{pco}]
	\begin{tcolorbox}Find all real values ​​of $a$ such that there is only one function $f$ which satisfies the following equality \[f (x ^ 2 + y + f (y)) = (f (x)) ^ 2 + ay \quad\forall x,y\in\mathbb R\]\end{tcolorbox}
If $a=0$, at least the two functions $f(x)=0$ $\forall x$ and $f(x)=1$ $\forall x$ are solutions. So $a\ne 0$
Let $P(x,y)$ be the assertion $f(x^2+y+f(y))=f(x)^2+ay$
$f(x)$ is bijective

Let $u$ such that $f(u)=0$. Comparaison of $P(u,0)$ with $P(-u,0)$ implies $f(-u)=f(u)=0$ and so $u=0$

Comparaison of $P(x,0)$ with $P(-x,0)$ $\implies$ $f(-x)=-f(x)$

$P(x,0)$ $\implies$ $f(x^2)=f(x)^2$
$P(0,y)$ $\implies$ $f(y+f(y))=ay$ and so $y+f(y)$ is bijective too.
So $P(x,y)$ may be written $f(x^2+y)=f(x^2)+f(y)$ and, since odd, $f(x+y)=f(x)+f(y)$

$f(x^2)=f(x)^2$ $\implies$ $f(x)>0$ $\forall x>0$ and so $f(x+y)>f(x)$ $\forall y>0$ and $f(x)$ is increasing.

So $f(x)=cx$ with $c\in\{0,1\}$ and, since bijective, $f(x)=x$

Plugging this back in original equation, we get $\boxed{a=2}$ as unique possibility
\end{solution}



\begin{solution}[by \href{https://artofproblemsolving.com/community/user/157383}{jowramos}]
	How did you get that $f$ is injective?
\end{solution}



\begin{solution}[by \href{https://artofproblemsolving.com/community/user/29428}{pco}]
	\begin{tcolorbox}How did you get that $f$ is injective?\end{tcolorbox}
:oops: a mistake ....

I'll look at it again, sorry
\end{solution}



\begin{solution}[by \href{https://artofproblemsolving.com/community/user/157383}{jowramos}]
	Somethings I got:

Set $(x,y)=(0,0)$ on the original equation to get $f(f(0))=f(0)^2$. Now, setting $x=f(0),y=0$ on the original equation, $f(f(0)^2+f(0))=f(f(0))^2=f(0)^4$. Setting now $x=0, y=f(0)$, we have $f(f(0)+f(f(0)))=f(0)^2+af(0)$. Then, we get an equality: 

$f(0)^4 = f(0)^2 + af(0)$. 

Then, either $f(0)=0$ or $f(0)^3 - f(0) - a=0$.

If $f(0)=0$, then, setting $y=0$, we get $f(x^2)=f(x)^2$. If $f(c)=f(d)$, then, setting $(x.y)=(c,d^2)$ and $(x,y)=(d,c^2)$, we have $f(c^2+d^2+f(d^2))=f(c)^2 + ad^2 = f(d^2+c^2+f(c^2))=f(d)^2 + ac^2 \iff c^2 = d^2$. Then, if $f(c)=f(d)$, either $c=d$ or $c=-d$. 

This is all I could get up to this moment...
\end{solution}



\begin{solution}[by \href{https://artofproblemsolving.com/community/user/199494}{IMI-Mathboy}]
	Assume that $f(0)\neq0$. Let$f(0)=c$ and$f(d)=0$ then putting $P(0,f(0)),P(f(0),0)$ we get $c^3-c-a=0$. Plugging $P(0,d)$ and  $P(d,0)$ we have $c^2+ad=0 and f(c+d^2)=0$. then putting $P(0,d^2+c)$ we have  $c^2+ad^2+ac=0$.   comparing  these equalities we have: $a=-1,d^3-d+1=0,c^3-c+1=0,d=c^2$ but for $d>0, d^3-d+1>0$ so we can conclude that $f(0)=0$ then by PCO's opinions we can get easily $a=2$
\end{solution}



\begin{solution}[by \href{https://artofproblemsolving.com/community/user/157383}{jowramos}]
	I couldn't see why: 

1 - $d^3-d+1=0$. In fact, I got a similar equation, but that involves $c$. 

2 - This leads to \begin{bolded}pco\end{bolded}'s opinions... We still could not prove the injectivity of $f$.
\end{solution}



\begin{solution}[by \href{https://artofproblemsolving.com/community/user/199494}{IMI-Mathboy}]
	(I'm sorry i made mistake in previous post.)EDIT:Assume that $f(0)\neq0$. Let$f(0)=c$ and$f(d)=0$.$(c,d\neq0$) then putting $P(0,f(0)),P(f(0),0)$ we get $c^3-c-a=0$. Plugging $P(0,d)$ and  $P(d,0)$ we have $c^2+ad=0$ and $f(c+d^2)=0$. then putting $P(0,d^2+c)$ we have:  $c^2+ad^2+ac=0$.   comparing  these equalities we have: $a=-1,c^3-c+1=0,$ if $c>0$ then$c^3-c+1>0$ so $c<0$ since $a=-1$, we can write original equation as: $f(x^2+y+f(y))=f(x)^2-y$. Plugging $Q(x,0)$ we have: $f(x^2+c)=f(x)^2$ \[\rightarrow\] $f(x)>0,$for all $x>0$ $(1)$.Then putting $Q(0,y)$ we get $f(y+f(y))=c^2-y$ but this is  contradiction to $(1)$ for $y>c^2$.(because LHS is positive ,but RHS is negative) so we can conclude that $f(0)=0$ if for some $u$, $f(u)=0$  Then in original equation $P(0,u)$ we get $u=0$. Then  $P(0,x)$: $f(x+f(x))=ax$ $(*)$.Hence $f(x)$ is bijective, then $x+f(x)$ is also bijective. $\rightarrow$ for $x>0$  any real $y$. $f(x+y)=f(x)+f(y)$.$(2)$  here $y\to-x$, then $f(-x)=-f(x)$ so $f$ is odd.Then $(2)$ is true for any real $x,y$. By $(1)$  and $(2)$ for $x>y$, $f(x-y+y)=f(x-y)+f(y)>f(y)$ so $f$ is strictly increasing. which means $f$ is injective.(using $(2)$ we can write $(*)$ as $f(f(x))+f(x)=ax$ hence $f$ is injective)
\end{solution}
*******************************************************************************
-------------------------------------------------------------------------------

\begin{problem}[Posted by \href{https://artofproblemsolving.com/community/user/125553}{lehungvietbao}]
	Find a rational function $f(x)$ with integer coefficients such that:
\[\cos \theta = f(\sin \theta - \cos \theta)\]
Find an identity, or show no exists an identity of this form .
	\flushright \href{https://artofproblemsolving.com/community/c6h566439}{(Link to AoPS)}
\end{problem}



\begin{solution}[by \href{https://artofproblemsolving.com/community/user/29428}{pco}]
	\begin{tcolorbox}Find a rational function $f(x)$ with integer coefficients such that:
\[\cos \theta = f(\sin \theta - \cos \theta)\]
Find an identity, or show no exists an identity of this form .\end{tcolorbox}
Let the supposed rational function be $x^n\frac {P(x)}{Q(x)}$ with $n\in\mathbb Z$, $P,Q\in\mathbb Z[X]$, $P(0)\ne 0$ and $Q(0)\ne 0$

Setting $\theta=\frac{\pi}4$, we get that $n=0$ and $\sqrt 2=\frac{Q(0)}{P(0)}\in\mathbb Q$, impossible.

So no such rational function.
\end{solution}
*******************************************************************************
-------------------------------------------------------------------------------

\begin{problem}[Posted by \href{https://artofproblemsolving.com/community/user/125553}{lehungvietbao}]
	Find all functions $f:\mathbb R\to\mathbb R$ such that
\[f(xf(y))=(1-y)f(xy)+x^{2}y^{2}f(y) \quad \forall x,y\in\mathbb R\]
	\flushright \href{https://artofproblemsolving.com/community/c6h566443}{(Link to AoPS)}
\end{problem}



\begin{solution}[by \href{https://artofproblemsolving.com/community/user/29428}{pco}]
	\begin{tcolorbox}Find all functions $f:\mathbb R\to\mathbb R$ such that
\[f(xf(y))=(1-y)f(xy)+x^{2}y^{2}f(y) \quad \forall x,y\in\mathbb R\]\end{tcolorbox}
$\boxed{f(x)=0}$ $\forall x$ is a solution. So let us from now look only for non allzero solutins.
Let $P(x,y)$ be the assertion $f(xf(y))=(1-y)f(xy)+x^2y^2f(y)$
Let $u$ such that $fu)\ne 0$

$P(0,1)$ $\implies$ $f(0)=0$ and so $u\ne 0$
If $f(1)\ne 0$, $P(\frac x{f(1)},1)$ $\implies$ $f(x)=\frac{x^2}{f(1)}$ which is not a solution. So $f(1)=0$
Let $x\notin \{0,1\}$. If $f(x)=0$, $P(\frac ux,x)$ $\implies$ $(1-x)f(u)=0$, impossible. So $f(x)\ne 0$ $\forall x\notin\{0,1\}$

If $f(a)=f(b)$, subtracting $P(1,a)$ from $P(1,b)$, we get $(a-b)(a+b-1)f(a)=0$ and so either $a=b$, either $a+b=1$, either $f(a)=0$

Let $x\notin\{0,1\}$ $P(\frac 1x,x)$ $\implies$ $f(\frac{f(x)}x)=f(x)$ ,and so $\forall x\notin\{0,1\}$ :
either $\frac{f(x)}x=x$ and so $f(x)=x^2$
either $\frac{f(x)}x+x=1$ and so $f(x)=x-x^2$

Let $x\notin\{0,\frac 12,1\}$ such that $f(x)=x^2$ : $P(1,x)$ $\implies$ $f(x^2)=x^4-x^3+x^2$ and so :
either $f(x^2)=x^4$ and so $x^3=x^2$, impossible
either $f(x^2)=x^2-x^4$ and so $2x^4=x^3$, impossible.
So $f(x)=x-x^2$ $\forall x\notin \{0,\frac 12,1\}$ and so $f(x)=x-x^2$ $\forall x\ne \frac 12$

Then $P(-\frac 14,2)$ $\implies$ $f(\frac 12)=\frac 14$

And so $\boxed{f(x)=x-x^2}$ $\forall x$, which indeed is a solution.
\end{solution}
*******************************************************************************
-------------------------------------------------------------------------------

\begin{problem}[Posted by \href{https://artofproblemsolving.com/community/user/125553}{lehungvietbao}]
	Find all functions$ f:\mathbb R\to\mathbb R$ such that
\[f((x - y)^2) = x^2 - 2yf(x) + (f(y))^2 \quad \forall x,y\in\mathbb R\]
	\flushright \href{https://artofproblemsolving.com/community/c6h566444}{(Link to AoPS)}
\end{problem}



\begin{solution}[by \href{https://artofproblemsolving.com/community/user/29428}{pco}]
	\begin{tcolorbox}Find all functions$ f:\mathbb R\to\mathbb R$ such that
\[f((x - y)^2) = x^2 - 2yf(x) + (f(y))^2 \quad \forall x,y\in\mathbb R\]\end{tcolorbox}
Let $P(x,y)$ be the assertion $f((x-y)^2)=x^2-2yf(x)+f(y)^2$

$P(0,0)$ $\implies$ $f(0)\in\{0,1\}$

If $f(0)=0$, $P(x,x)$ $\implies$ $\boxed{f(x)=x}$ $\forall x$, which indeed is a solution.

If $f(0)=1$, $P(x,x)$ $\implies$ $((f(x)-x)^2=1$ and so $\forall x$, either $f(x)x+1$, either $f(x)=x-1$

If $\exists x$ such that $f(x)=x-1$, then $P(0,x)$ $\implies$ $f(x^2)=x^2-4x+1$ and so :
Either $f(x^2)=x^2+1$ and so $x=0$, impossible
Either $f(x^2)=x^2-1$ and so $x=\frac 12$

So :

Either $\boxed{f(x)=x+1}$ $\forall x$ which indeed is a solution
Either $f(x)=x-1$ $\forall x$ which is not a solution
Either $f(x)=x+1$ $\forall x\ne \frac 12$ and $f(\frac 12)=-\frac 12$, which is not a solution
\end{solution}
*******************************************************************************
-------------------------------------------------------------------------------

\begin{problem}[Posted by \href{https://artofproblemsolving.com/community/user/125553}{lehungvietbao}]
	Find all functions $f:\mathbb{R} \to \mathbb{R}$ such that :
\[f\left( {x + \cos (2011y)} \right) = f(x) + 2011\cos f(y)\quad \forall x,y \in \mathbb{R}\]
	\flushright \href{https://artofproblemsolving.com/community/c6h566446}{(Link to AoPS)}
\end{problem}



\begin{solution}[by \href{https://artofproblemsolving.com/community/user/29428}{pco}]
	\begin{tcolorbox}Find all functions $f:\mathbb{R} \to \mathbb{R}$ such that :
\[f\left( {x + \cos (2011y)} \right) = f(x) + 2011\cos f(y)\quad \forall x,y \in \mathbb{R}\]\end{tcolorbox}
Let $a=2011$
Let $P(x,y)$ be the assertion $f(x+\cos ay)=f(x)+a\cos f(y)$

$P(0,x)$ $\implies$ $f(x)\in[f(0)-a,f(0)+a]$ $\forall x\in[-1,+1]$ and so $f(x)$ is bounded over $[-1,+1]$

$P(0,y)$ $\implies$ $f(\cos ay)=f(0)+a\cos f(y)$ and so $P(x,y)$ may be written $f(x+\cos ay)=f(x)+f(\cos ay)-f(0)$

So $f(x+y)=f(x)+f(y)-f(0)$ $\forall x$, $\forall y\in[-1,1]$
Let then $g(x)=f(x)-f(0)$ and we get $g(x+y)=g(x)+g(y)$ $\forall x$, $\forall y\in[-1,+1]$
And so, with easy induction, $g(x+y)=g(x)+g(y)$ $\forall x,y$

And since $f(x)$ is bounded over $[-1,+1]$, so is $g(x)$ and so $g(x)=cx$ and $f(x)=cx+d$

Plugging in original equation, we get $c\cos ay= a\cos (cy+d)$ and so :

Either $c=a$ and so $d=2k\pi$ and so the solution $\boxed{f(x)=2011x+2k\pi}$ $\forall x$, and whatever is $k\in\mathbb Z$

Either $c=-a$ and so $d=(2k+1)\pi$ and so the solution $\boxed{f(x)=-2011x+(2k+1)\pi}$ $\forall x$, and whatever is $k\in\mathbb Z$
\end{solution}
*******************************************************************************
-------------------------------------------------------------------------------

\begin{problem}[Posted by \href{https://artofproblemsolving.com/community/user/125553}{lehungvietbao}]
	Find all functions $f$ defined on $\mathbb R$ such that:
\[f(f(f(x))) = 2x-1\]
	\flushright \href{https://artofproblemsolving.com/community/c6h566546}{(Link to AoPS)}
\end{problem}



\begin{solution}[by \href{https://artofproblemsolving.com/community/user/29428}{pco}]
	\begin{tcolorbox}Find all functions $f$ defined on $\mathbb R$ such that:
\[f(f(f(x))) = 2x-1\]\end{tcolorbox}
Let $g(x)=f(x+1)-1$ and the equation is $g(g(g(x)))=2x$ whose general solution is rather classical :

Let $\sim$ be the equivalence relation in $\mathbb R^*$ defined as $x\sim y$ $\iff$ $\exists n\in\mathbb Z$ such that $y=x2^n$

Let $r(x)$ any function which associates to a non zero real a representant (unique per class) of its equivalence class.
Let $n(x)$ from $\mathbb R^*\to\mathbb Z$ defined as $n(x)=\log_2\frac x{r(x)}$

Let $A,B,C$ any split of $r(\mathbb R^*)$ in three equipotent subsets.

Let $u(x)$ any bijection from $A\to B$
Let $v(x)$ any bijection from $B\to C$

Then $g(x)$ may be defined as :
$g(0)=0$
$\forall x\ne 0$ :
If $r(x)\in A$, then $g(x)=u(r(x))2^{n(x)}$
If $r(x)\in B$, then $g(x)=v(r(x))2^{n(x)}$
If $r(x)\in C$, then $g(x)=u^{-1}(v^{-1}(r(x))2^{n(x)+1}$
\end{solution}
*******************************************************************************
-------------------------------------------------------------------------------

\begin{problem}[Posted by \href{https://artofproblemsolving.com/community/user/125553}{lehungvietbao}]
	Determine all  functions $f: \mathbb R\to\mathbb R$ such that:
\[f(x+f(y))=f(x)+y^{n} \quad \forall x\in\mathbb R , n\in\mathbb N\] 
Where $\mathbb N=\{0,1,2,...\}$
	\flushright \href{https://artofproblemsolving.com/community/c6h566547}{(Link to AoPS)}
\end{problem}



\begin{solution}[by \href{https://artofproblemsolving.com/community/user/29428}{pco}]
	\begin{tcolorbox}Determine all  functions $f: \mathbb R\to\mathbb R$ such that:
\[f(x+f(y))=f(x)+y^{n} \quad \forall x\in\mathbb R , n\in\mathbb N\] 
Where $\mathbb N=\{0,1,2,...\}$\end{tcolorbox}
1) If $n=0$, no solution, since $0^0$ is undefined and functional equation is not defined when $y=0$

2) If $n=1$, we immediately get that $f(x)$ is additve, bijective and involutary and so :
Let $A,B$ any two supplementary subvectorspaces of the $\mathbb Q$-vectorspace $\mathbb R$
Let $a(x)$ from $\mathbb R\to A$ and $b(x)$ from $\mathbb R\to B$ the two projections of $x$ in $A,B$ :
$f(x)=a(x)-b(x)$ (and the only two continuous solutions are $f(x)=x$ $\forall x$ and $f(x)=-x$ $\forall x$)

3) If $n>1$ odd , we get $f(x+y)=f(x)+f(y)$ $\forall x,y$ and $f(f(x))=x^n$ 
So $f(f(x+y))=f(f(x))+f(f(y))$ and so $(x+y)^n=x^n+y^n$ $\forall x,y$ which is wrong.
So no solution.

4) If $n>1$ even, we get  $f(x+y)=f(x)+f(y)-f(0)$ $\forall x$, $\forall y\ge f(0)$ and $f(f(x))=x^n+f(0)$
It's easy to get then $f(x+y)=f(x)+f(y)-f(0)$ $\forall x,y$ and $f(x)$ lowerbounded on $[0,+\infty)$ and so $f(x)=ax+b$ which is never a solution.
So no solution.
\end{solution}
*******************************************************************************
-------------------------------------------------------------------------------

\begin{problem}[Posted by \href{https://artofproblemsolving.com/community/user/125553}{lehungvietbao}]
	Find all function $f$ defined on $\mathbb{R}\setminus  \{-1, 1\}$ such that
\[f (x) + 2 f \left (\frac{x-3}{x +1}\right ) = \frac{x (x +9) }{ x +1} \quad \forall x \in \mathbb R \setminus  \{-1, +1\}\]
	\flushright \href{https://artofproblemsolving.com/community/c6h566569}{(Link to AoPS)}
\end{problem}



\begin{solution}[by \href{https://artofproblemsolving.com/community/user/29428}{pco}]
	\begin{tcolorbox}Find all function $f$ defined on $\mathbb{R}\setminus  \{-1, 1\}$ such that
\[f (x) + 2 f \left (\frac{x-3}{x +1}\right ) = \frac{x (x +9) }{ x +1} \quad \forall x \in \mathbb R \setminus  \{-1, +1\}\]\end{tcolorbox}
Let $g(x)$ from $\mathbb R\setminus\{-1,+1\}\to\mathbb R\setminus\{-1,+1\}$ defined as $g(x)=\frac{x-3}{x+1}$
Let $P(x)$ be the assertion $f(x)+2f(g(x))=\frac {x(x+9)}{x+1}$ true $\forall x\notin\{-1,+1\}$

note that $g(g(g(x)))=x$ $\forall x\notin\{-1,+1\}$

a) : $P(x)$ $\implies$ $f(x)+2f(g(x))=\frac {x(x+9)}{x+1}$

b) : $P(g(x))$ $\implies$ $f(g(x))+2f(g(g(x)))=\frac {g(x)(g(x)+9)}{g(x)+1}$ $=\frac{(x-3)(5x+3)}{x^2-1}$

c) : $P(g(g(x))$ $\implies$ $f(g(g(x)))+2f(x)=\frac{(g(x)-3)(5g(x)+3)}{g(x)^2-1}$ $=\frac{(x+3)(2x-3)}{x-1}$

a)-2b)+4c) $\implies$ $9f(x)=\frac {x(x+9)}{x+1}$ $-2\frac{(x-3)(5x+3)}{x^2-1}$ $+4\frac{(x+3)(2x-3)}{x-1}$ $=9(x+2)$

Hence the result: $\boxed{f(x)=x+2}$ $\forall x\notin\{-1,+1\}$ which indeed is a solution.
\end{solution}
*******************************************************************************
-------------------------------------------------------------------------------

\begin{problem}[Posted by \href{https://artofproblemsolving.com/community/user/125553}{lehungvietbao}]
	Determine all continuous functions $f: \mathbb R\to\mathbb R$ such that:
\[ (f (x + y))^2 f (x, y) = (f (x) f (y)) ^ 2 \quad \forall x\in\mathbb R\]
	\flushright \href{https://artofproblemsolving.com/community/c6h566570}{(Link to AoPS)}
\end{problem}



\begin{solution}[by \href{https://artofproblemsolving.com/community/user/29428}{pco}]
	\begin{tcolorbox}Determine all continuous functions $f: \mathbb R\to\mathbb R$ such that:
\[ (f (x + y))^2 f (x, y) = (f (x) f (y)) ^ 2 \quad \forall x\in\mathbb R\]\end{tcolorbox}
What is the meaning of $f(x,y)$ ?????
\end{solution}



\begin{solution}[by \href{https://artofproblemsolving.com/community/user/125553}{lehungvietbao}]
	Dear Mr. Patrick. So sorry ! 
Determine all continuous functions $f: \mathbb R\to\mathbb R$ such that:
\[ (f (x + y))^2 (f (x)+f(y)) = (f (x) f (y)) ^ 2 \quad \forall x,y\in\mathbb R\]
\end{solution}



\begin{solution}[by \href{https://artofproblemsolving.com/community/user/29428}{pco}]
	\begin{tcolorbox}Dear Mr. Patrick. So sorry ! 
Determine all continuous functions $f: \mathbb R\to\mathbb R$ such that:
\[ (f (x + y))^2 (f (x)+f(y)) = (f (x) f (y)) ^ 2 \quad \forall x,y\in\mathbb R\]\end{tcolorbox}
Let $P(x,y)$ be the assertion $f(x+y)^2(f(x)+f(y))=f(x)^2f(y)^2$

$P(0,0)$ $\implies$ $f(0)\in\{0,2\}$

If $f(0)=0$, $P(x,0)$ $\implies$ $\boxed{f(x)=0}$ $\forall x$, which indeed is a solution

If $f(0)=2$, $P(x,0)$ $\implies$ $f(x)^2(f(x)-2)=0$ and so $f(x)\in\{0,2\}$ $\forall x$ and continuity implies then $\boxed{f(x)=2}$ $\forall x$, which indeed is a solution
\end{solution}
*******************************************************************************
-------------------------------------------------------------------------------

\begin{problem}[Posted by \href{https://artofproblemsolving.com/community/user/125553}{lehungvietbao}]
	Find all differential functions $(0, +\infty) \to (0, +\infty)$  such that there is a positive real number $a$ such that:
\[f '\left (\frac{y }{ x}\right) = \frac{x }{ f (x)}\quad  \forall x> 0\]
	\flushright \href{https://artofproblemsolving.com/community/c6h566571}{(Link to AoPS)}
\end{problem}



\begin{solution}[by \href{https://artofproblemsolving.com/community/user/29428}{pco}]
	\begin{tcolorbox}Find all differential functions $(0, +\infty) \to (0, +\infty)$  such that there is a positive real number $a$ such that:
\[f '\left (\frac{y }{ x}\right) = \frac{x }{ f (x)}\quad  \forall x> 0\]\end{tcolorbox}
There is no letter "a" in the statement.
\end{solution}



\begin{solution}[by \href{https://artofproblemsolving.com/community/user/125553}{lehungvietbao}]
	Edited
 
Find all differential functions $(0, +\infty) \to (0, +\infty)$  such that there is a positive real number $a$ such that:
\[f '\left (\frac{a }{ x}\right) = \frac{x }{ f (x)}\quad  \forall x> 0\]
\end{solution}



\begin{solution}[by \href{https://artofproblemsolving.com/community/user/29428}{pco}]
	\begin{tcolorbox}Edited
 
Find all differential functions $(0, +\infty) \to (0, +\infty)$  such that there is a positive real number $a$ such that:
\[f '\left (\frac{a }{ x}\right) = \frac{x }{ f (x)}\quad  \forall x> 0\]\end{tcolorbox}
$f'(x)=\frac a{xf(\frac ax)}$ and so is differentiable too.

From there, we get $f(\frac ax)=\frac a{xf'(x)}$

Taking derivative, we get $-\frac a{x^2}f'(\frac ax)=-a\frac{f'(x)+xf''(x)}{x^2f'(x)^2}$

And so $\frac 1{x^2}\frac x{f(x)}=\frac{f'(x)+xf''(x)}{x^2f'(x)^2}$

Which becomes $\frac {f'(x)}{f(x)}=\frac{1}{x}+\frac{f''(x)}{f'(x)}$

And so $\ln f(x)=c+\ln x+\ln f'(x)$ $\implies$ $f(x)=dxf'(x)$ $\implies$ $\frac{f'(x)}{f(x)}=\frac 1{dx}$ $\implies$ $\ln f(x)=e+\frac 1d\ln x$

So $f(x)=ux^v$

Plugging this back in original equation, we get $v>0$ and, if $v=1$, the supplementary constraint $u=1$

\begin{bolded}Hence the answer\end{bolded} :
$\boxed{f(x)=ux^v}$ $\forall x$, and whatever are $u,v>0$ with $v\ne 1$

$\boxed{f(x)=x}$ $\forall x$
\end{solution}
*******************************************************************************
-------------------------------------------------------------------------------

\begin{problem}[Posted by \href{https://artofproblemsolving.com/community/user/125553}{lehungvietbao}]
	Determine all  functions $f: \mathbb {R}\to\mathbb {R}$ such that
\[f(x+y)+g(x-y)=2h(x)+2h(y) \quad x,y\in\mathbb R\]
	\flushright \href{https://artofproblemsolving.com/community/c6h566572}{(Link to AoPS)}
\end{problem}



\begin{solution}[by \href{https://artofproblemsolving.com/community/user/29428}{pco}]
	\begin{tcolorbox}Determine all  functions $f: \mathbb {R}\to\mathbb {R}$ such that
\[f(x+y)+g(x-y)=2h(x)+2h(y) \quad x,y\in\mathbb R\]\end{tcolorbox}
If $(f,g,h)$ is solution, so is $(f+c,g+d,h-\frac{c+d}4)$. So WLOG consider $f(0)=h(0)=0$

Setting $y=0$, we get $g(x)=2h(x)-f(x)$ and equation becomes $f(x+y)+2h(x-y)-f(x-y)=2h(x)+2h(y)$

Setting $y=x$, we get $f(2x)=4h(x)$ and equation becomes $2h(\frac{x+y}2)+h(x-y)-2h(\frac{x-y}2)=h(x)+h(y)$

Let then $P(x,y)$ be the assertion $2h(\frac{x+y}2)-2h(\frac{x-y}2)=h(x)+h(y)-h(x-y)$

Adding $P((n+1)x,x)$ with $P((n+1)x,-x)$, we get $h((n+2)x)=2h((n+1)x)-h(nx)+h(x)+h(-x)$ and simple induction gives :

$h(nx)=\frac{h(x)+h(-x)}2n^2+\frac{h(x)-h(-x)}2n$ 

And  then $h(x)=ax^2+bx$ $\forall x\in\mathbb Q$

This gives solutions in $\mathbb Q$ :
$f(x)=ax^2+2bx+4c$
$g(x)=ax^2+4d$
$h(x)=ax^2+bx+c+d$

Before looking for extension to $\mathbb R$, could you kindly confirm us that this is the exact statement you got in your exam \/ contest and that you did not forget some nearly useless precision (like continuous, or monotonous, ...) ?
Thanks in advance.
\end{solution}



\begin{solution}[by \href{https://artofproblemsolving.com/community/user/125553}{lehungvietbao}]
	\begin{tcolorbox}
.....
Before looking for extension to $\mathbb R$, could you kindly confirm us that this is the exact statement you got in your exam \/ contest and that you did not forget some nearly useless precision (like continuous, or monotonous, ...) ?
Thanks in advance.\end{tcolorbox}
Dear Mr.Patrick
I'm sure that it's an Olympiad problem from my contest
I didn't complete it :(
\end{solution}



\begin{solution}[by \href{https://artofproblemsolving.com/community/user/29428}{pco}]
	\begin{tcolorbox}[quote="pco"]
.....
Before looking for extension to $\mathbb R$, could you kindly confirm us that this is the exact statement you got in your exam \/ contest and that you did not forget some nearly useless precision (like continuous, or monotonous, ...) ?
Thanks in advance.\end{tcolorbox}
Dear Mr.Patrick
I'm sure that it's an Olympiad problem from my contest
I didn't complete it :(\end{tcolorbox}
And you confirm there is no more conditions (continuous, monotonous, ...) ?
\end{solution}



\begin{solution}[by \href{https://artofproblemsolving.com/community/user/125553}{lehungvietbao}]
	\begin{tcolorbox}
And you confirm there is no more conditions (continuous, monotonous, ...) ?\end{tcolorbox}
Yes, dear sir 
There is no more conditions in our problem :)
\end{solution}
*******************************************************************************
-------------------------------------------------------------------------------

\begin{problem}[Posted by \href{https://artofproblemsolving.com/community/user/125553}{lehungvietbao}]
	Find all  functions $f: \mathbb R\to\mathbb R$  such that 
\[f(f(f(f(x)))) = f(f(f(x))) + 2x  \quad \forall x\in\mathbb R\]
	\flushright \href{https://artofproblemsolving.com/community/c6h566573}{(Link to AoPS)}
\end{problem}



\begin{solution}[by \href{https://artofproblemsolving.com/community/user/29428}{pco}]
	\begin{tcolorbox}Find all  functions $f: \mathbb R\to\mathbb R$  such that 
\[f(f(f(f(x)))) = f(f(f(x))) + 2x  \quad \forall x\in\mathbb R\]\end{tcolorbox}
Besides the two trivial solutions $f(x)=-x$ and $f(x)=ax$ where $a$ is the positive real root of $x^4-x^3-2=0$, I think there are infinitely many strange solutions.


Could you kindly confirm us that this is the exact statement you got in your exam \/ contest and that you did not forget some nearly useless condition (continous, monotonous, ...) ?
Thanks in advance.
\end{solution}



\begin{solution}[by \href{https://artofproblemsolving.com/community/user/125553}{lehungvietbao}]
	\begin{tcolorbox}
Could you kindly confirm us that this is the exact statement you got in your exam \/ contest and that you did not forget some nearly useless condition (continous, monotonous, ...) ?
Thanks in advance.\end{tcolorbox}
Dear Mr.Patrick
 It's an Olympiad problem from a competion ( but i don't know about resource ) .
\end{solution}



\begin{solution}[by \href{https://artofproblemsolving.com/community/user/29428}{pco}]
	\begin{tcolorbox}[quote="pco"]
Could you kindly confirm us that this is the exact statement you got in your exam \/ contest and that you did not forget some nearly useless condition (continous, monotonous, ...) ?
Thanks in advance.\end{tcolorbox}
Dear Mr.Patrick
 It's an Olympiad problem from a competion ( but i don't know about resource ) .\end{tcolorbox}
And you confirm there is no more condition (continuous, monotonous ...) ?
\end{solution}



\begin{solution}[by \href{https://artofproblemsolving.com/community/user/125553}{lehungvietbao}]
	\begin{tcolorbox}
And you confirm there is no more condition (continuous, monotonous ...) ?\end{tcolorbox}
Dear Mr.Patrick
There is no more conditions in our problem :)
\end{solution}



\begin{solution}[by \href{https://artofproblemsolving.com/community/user/29428}{pco}]
	Example of strange non continuous solution :

Let $a$ be the positive real root of $x^4-x^3-2=0$
Let $u(x)$ any bijective non continuous additive function.

Then $f(x)=u^{-1}(a u(x))$ is a solution.

So, I'm quite sure this is not a real olympiad exercise.
I'm sure you think it is but your source (book, web site, teacher) is certainly a bad one.
It would be nice if you could stop posting problem from this source claiming you are sure there exists an olympiad-level solution :(

Posting problems from this source in "open" category would certainly be more honest, according to me.
\end{solution}
*******************************************************************************
-------------------------------------------------------------------------------

\begin{problem}[Posted by \href{https://artofproblemsolving.com/community/user/125553}{lehungvietbao}]
	Find all monotonous functions $f: \mathbb R\to\mathbb R$  such that 
\[f (f (f (x)))-3f (f (x))+6f (x) = 4x +3  \quad \forall x\in\mathbb R\]
	\flushright \href{https://artofproblemsolving.com/community/c6h566575}{(Link to AoPS)}
\end{problem}



\begin{solution}[by \href{https://artofproblemsolving.com/community/user/29428}{pco}]
	\begin{tcolorbox}Find all monotonous functions $f: \mathbb R\to\mathbb R$  such that 
\[f (f (f (x)))-3f (f (x))+6f (x) = 4x +3  \quad \forall x\in\mathbb R\]\end{tcolorbox}
$f(x)$ is injective.
Since monotonous, $f(x)$ is increasing (else $LHS$ is decreasing while $RHS$ is increasing).

Let $x\in\mathbb R$ and the sequence $a_n=f^{[n]}(x)$

We get $a_{n+3}=3a_{n+2}-6a_{n+1}+4a_n+3$ and so $a_n=n+a+2^nr\cos(\varphi +n\frac{\pi}3)$ for some $a,r,\varphi$ depending on $x$

So $\frac {f(a_{n+3})-f(a_{n})}{a_{n+3}-a_n}$ $=\frac{a_{n+4}-a_{n+1}}{a_{n+3}-a_n}$ $=\frac{1-3\times 2^{n+1}r\cos(\varphi +(n+1)\frac{\pi}3))}{1-3\times 2^{n}r\cos(\varphi +n\frac{\pi}3))}$

It's easy to see that if $r\ne 0$, and whatever is $\varphi$, this ratio is sometimes $<0$ when $n\to+\infty$, in contradiction with the fact that $f(x)$ is increasing.

So $r=0$ and $f^{[n]}(x)=n+a(x)$
Setting $n=0$, we get $a(x)=x$
Setting $n=1$, we get $\boxed{f(x)=x+1}$ $\forall x$, which indeed is a solution.
\end{solution}
*******************************************************************************
-------------------------------------------------------------------------------

\begin{problem}[Posted by \href{https://artofproblemsolving.com/community/user/198285}{ilovemath121}]
	find all function $ f:\mathbb{R}\to\mathbb{R} $ : \[ f(x+y^2)=f(x)+|yf(y)|. \]
	\flushright \href{https://artofproblemsolving.com/community/c6h566588}{(Link to AoPS)}
\end{problem}



\begin{solution}[by \href{https://artofproblemsolving.com/community/user/29428}{pco}]
	\begin{tcolorbox}find all function $ f:\mathbb{R}\to\mathbb{R} $ : \[ f(x+y^2)=f(x)+|yf(y)|. \]\end{tcolorbox}
Let $P(x,y)$ be the assertion $f(x+y^2)=f(x)+|yf(y)|$
Let $a=f(0)$

$P(0,y)$ $\implies$ $f(y^2)=a+|yf(y)|$ and so $f(x+y^2)=f(x)+f(y^2)-a$ and so $f(x+y)=f(x)+f(y)-a$ $\forall x$, $\forall y\ge 0$

Let then any $x,y$. Let $t\ge \max(0,-y)$ :
$f(x+y+t)=f((x+y)+t)=f(x+y)+f(t)-a$
$f(x+y+t)=f(x+(y+t))=f(x)+f(y+t)-a$ $=f(x)+f(y)+f(t)-2a$
And so $f(x+y)=f(x)+f(y)-a$ $\forall x,y$

And since $f(y^2)=a+|yf(y)|$ shows that $f(x)$ is lowerbounded over $\mathbb R^+$, we get $f(x)=bx+a$

Plugging this back in original equation, we get $by^2=|by^2+ay|$ and so $a=0$ and $b\ge 0$

Hence the answer : $\boxed{f(x)=cx}$ $\forall x$ and whatever is $c\ge 0$
\end{solution}
*******************************************************************************
-------------------------------------------------------------------------------

\begin{problem}[Posted by \href{https://artofproblemsolving.com/community/user/68025}{Pirkuliyev Rovsen}]
	Find all continuous  functions $f: \mathbb{R}\to\mathbb{R}$ such that $f(f(x))=\frac{f(x)}{x}$, for all $x{\in}R^{\ast}$.
	\flushright \href{https://artofproblemsolving.com/community/c6h566761}{(Link to AoPS)}
\end{problem}



\begin{solution}[by \href{https://artofproblemsolving.com/community/user/29428}{pco}]
	\begin{tcolorbox}Find all continuous  functions $f: \mathbb{R}\to\mathbb{R}$ such that $f(f(x))=\frac{f(x)}{x}$, for all $x{\in}R^{\ast}$.\end{tcolorbox}
Let $P(x)$ be the assertion $f(f(x))=\frac{f(x)}x$


$P(-1)$ $\implies$ $f(f(-1))=-f(-1)$ and so $f(x)$ can be neither always $>0$, neither always $<0$.
So $\exists u$ such that $f(u)=0$

If $u=0$, we get $f(0)=0$
If $u\ne 0$, $P(u)$ $\implies$ $f(0)=0$
So $f(0)=0$

"Reduced injectivity" property :
If $f(a)=f(b)\ne 0$, then $a,b\ne 0$ and comparaison of $P(a)$ and $P(b)$ implies $a=b$

If $f(u)=0$ for some $u>0$ and $\exists v\in(0,u)$ such that $f(v)\ne 0$, then :
Since continuous, $\exists 0<a<b<u$ such that $f(a)=f(b)\ne 0$.
Comparing $P(a)$ and $P(b)$, we get $a=b$, impossible.
If $f(u)=0$ for some $u<0$ and $\exists v\in(u,0)$ such that $f(v)\ne 0$, then :
Since continuous, $\exists u<a<b<0$ such that $f(a)=f(b)\ne 0$.
Comparing $P(a)$ and $P(b)$, we get $a=b$, impossible.

So $f^{-1}(0)$ is either $\mathbb R$, either $(-\infty,b]$, either $[a,b]$, either $[a,+\infty)$ where $a\le 0\le b$

1) $f^{-1}(0)=\mathbb R$
============
So $\boxed{f(x)=0}$ $\forall x$, which indeed is a solution

2) $f^{-1}(0)=(-\infty,b]$ for some $b\ge 0$
======================
If $b>0$, then, since $f(b)=0$ and $f(x)$ continuous, $\exists \varepsilon>0$ such that $f(b+\varepsilon)<b$ and $f(b+\varepsilon)\ne 0$ and then $P(b+\varepsilon)$ is wrong.
So $b=0$ and $f(x)\ne 0$ $\forall x>0$. So $f(x)>0$ $\forall x>0$ (else $f(f(x))=0$)
So $f(x)$ is an increasing continuous function from $[0,+\infty)\to[0,+\infty)$

$P(x)$ $\implies$ $f^{[2]}(x)=\frac {f(x)}x$
$P(f(x))$ $\implies$ $f^{[3]}(x)=\frac {f(f(x))}{f(x)}=\frac 1x$
And so $f^{[6]}(x)=x$
Then, if $f(a)>a>0$ for some $a$, then $f(f(a))>f(a)>a$ ... and $f^{[6]}(a)>a$
Same, if $a>f(a)>0$ for some $a$, then $a>f(a)>f(f(a))>0$ ... and $a>f^{[6]}(a)$
So $f(x)=x$ which, unfortunately, is not a solution.
So no solution

3) $f^{-1}(0)=[a,b]$ for some $b\ge 0\ge a$
===========================
If $b>0$, then $f(x)<0$ $\forall x>b$, else $\exists \varepsilon>0$ such that $0<f(b+\varepsilon)<b$ and then $P(b+\varepsilon)$ is wrong.
If $b>0$, then $f(x)<0$ $\forall x<a$, else $\exists \varepsilon>0$ such that $0<f(a-\varepsilon)<b$ and then $P(a-\varepsilon)$ is wrong.
But then  $\exists u<a<b<v$ such that $f(u)=f(v)<0$, in contradiction with reduced injectivity property.
So $b=0$

If $a<0$, then $f(x)>0$ $\forall x<a$, else $\exists \varepsilon>0$ such that $a<f(a-\varepsilon)<0$ and then $P(a-\varepsilon)$ is wrong.
If $a<0$, then $f(x)>0$ $\forall x>b$, else $\exists \varepsilon>0$ such that $a<f(b+\varepsilon)<0$ and then $P(b+\varepsilon)$ is wrong.
But then  $\exists u<a<b<v$ such that $f(u)=f(v)>0$, in contradiction with reduced injectivity property.
So $a=0$

So $f(x)=0$ $\iff$ $x=0$ and $f(x)$ is increasing (else $f(1)<0$ and $f(f(1))>0$ and $f(f(1))>f(1)$, in contradiction with $P(1)$
So $f(x)$ is an increasing continuous function from $(-\infty,+\infty)\to(-\infty,+\infty)$

$P(x)$ $\implies$ $f^{[2]}(x)=\frac {f(x)}x$
$P(f(x))$ $\implies$ $f^{[3]}(x)=\frac {f(f(x))}{f(x)}=\frac 1x$
And so $f^{[6]}(x)=x$
Then, if $f(a)>a$ for some $a$, then $f(f(a))>f(a)>a$ ... and $f^{[6]}(a)>a$
Same, if $a>f(a)$ for some $a$, then $a>f(a)>f(f(a))$ ... and $a>f^{[6]}(a)$
So $f(x)=x$ which, unfortunately, is not a solution.
So no solution

4) $f^{-1}(0)=[a,+\infty)$ for some $a\le 0$
======================
If $a<0$, then, since $f(a)=0$ and $f(x)$ continuous, $\exists \varepsilon>0$ such that $f(a-\varepsilon)>a$ and $f(a-\varepsilon)\ne 0$ and then $P(a-\varepsilon)$ is wrong.
So $a=0$ and $f(x)\ne 0$ $\forall x<0$. So $f(x)<0$ $\forall x<0$ (else $f(f(x))=0$)
So $f(x)$ is an increasing continuous function from $(-\infty,0]\to(-\infty,0]$

But then, $\forall x<0$, $f(f(x))<0$ while $\frac{f(x)}x>0$
So no solution.

So \begin{bolded}no other solution that the all-zero function\end{underlined}\end{bolded}.
\end{solution}
*******************************************************************************
-------------------------------------------------------------------------------

\begin{problem}[Posted by \href{https://artofproblemsolving.com/community/user/125553}{lehungvietbao}]
	1) Determine all  functions $f: \mathbb {R}^{+}\to\mathbb {R}^{+}$ such that
\[(f (x) )^ 2 \geq  f (x + y) (f (x) + y) \quad \forall x,y> 0\]

2) Determine all continuous functions $f: \mathbb R\to\mathbb R$ such that:
\[f(f (x)) = e^{x} \quad \forall x\in\mathbb R\]
	\flushright \href{https://artofproblemsolving.com/community/c6h566766}{(Link to AoPS)}
\end{problem}



\begin{solution}[by \href{https://artofproblemsolving.com/community/user/29428}{pco}]
	\begin{tcolorbox}2) Determine all continuous functions $f: \mathbb R\to\mathbb R$ such that:
\[f(f (x)) = e^{x} \quad \forall x\in\mathbb R\]\end{tcolorbox}
Your olympiad contests are far too highlevel for me. It's a pity you never get answers from your teacher for these real problems.

There are infinitely many solutions for this problem but I'm quite unable to find the general form for all of them.

Hereunder is a rather trivial classical (the piecewise construction) solution in order to build infinitely many solutions (but, unfortunately, not all, which is the real difficulty of the problem, according to me.). 

Let $a<0$
Let $h(x)$ any continuous increasing function from $[-\infty,a]\to (a,0]$ such that $\lim_{x\to -\infty}h(x)=a$ and $h(a)=0$

Let $\{a_n\}$ be the sequence $a_0=a$, $a_1=0$ and $a_{n+2}=e^{a_n}$ : $a_n$ is an increasing sequence whose limit is $+\infty$

Let $\{h_n(x)$, from $(a_n,a_{n+1}]\to (a_{n+1},a_{n+2}]\}$ be the sequence of bijections defined as :
$h_0(a)=0$ and $h_0(x)=e^{h^{-1}(x)}$ $\forall x\in(a,0)$
$h_{n+1}(x)=e^{h_n^{-1}(x)}$

It's easy to show that $f(x)$ defined as :
$\forall x\le a$ : $f(x)=h(x)$
$\forall x\in(a_n,a_{n+1}]$ : $f(x)=h_n(x)$

is increasing, continuous, and such that $f(f(x))=e^x$ $\forall x$
\end{solution}
*******************************************************************************
-------------------------------------------------------------------------------

\begin{problem}[Posted by \href{https://artofproblemsolving.com/community/user/125553}{lehungvietbao}]
	1) Find all continuous functions $f: \mathbb {R}^{+}\to\mathbb R$ such that 
\[f \left(x +\frac{1 }{x}\right) + f \left(y+\frac{1}{ y}\right) = f \left(x+\frac{1 }{ y}\right) + f \left(y+\frac{1 }{ x}\right)\quad \forall x, y> 0\]

2) \begin{bolded}Maroc 2006 \end{bolded}
Find all continuous functions $f: \mathbb R\to\mathbb R$  such that:
\[f (f (x)) = f (x) + x \quad \forall x\in\mathbb R\]
	\flushright \href{https://artofproblemsolving.com/community/c6h566767}{(Link to AoPS)}
\end{problem}



\begin{solution}[by \href{https://artofproblemsolving.com/community/user/29428}{pco}]
	\begin{tcolorbox}2) \begin{bolded}Maroc 2006 \end{bolded}
Find all continuous functions $f: \mathbb R\to\mathbb R$  such that:
\[f (f (x)) = f (x) + x \quad \forall x\in\mathbb R\]\end{tcolorbox}
See http://www.artofproblemsolving.com/Forum/viewtopic.php?f=36&t=485367 subcase 7.1.1.

Two solutions : $\boxed{f(x)=\frac{1+\sqrt 5}2x}$ $\forall x$ and $\boxed{f(x)=\frac{1-\sqrt 5}2x}$ $\forall x$
\end{solution}



\begin{solution}[by \href{https://artofproblemsolving.com/community/user/29428}{pco}]
	\begin{tcolorbox}1) Find all continuous functions $f: \mathbb {R}^{+}\to\mathbb R$ such that 
\[f \left(x +\frac{1 }{x}\right) + f \left(y+\frac{1}{ y}\right) = f \left(x+\frac{1 }{ y}\right) + f \left(y+\frac{1 }{ x}\right)\quad \forall x, y> 0\]\end{tcolorbox}
Let $P(x,y)$ be the assertion $f(x+\frac 1x)+f(y+\frac 1y)=f(x+\frac 1y)+f(y+\frac 1x)$

Let $0<a<1$ and $0<u_0<\sqrt{1-a^2}<1$ and the sequence $u_{n+1}=u_n\sqrt{1-\frac{a^2}{1-u_n^2}}$,$\forall n\ge 0$
$u_n$ is a positive decreasing sequence whose limit is $0$

$P\left(\frac{1+u_n}a\left(1+\sqrt{1-\frac{a^2}{1-u_n^2}}\right), \frac{1-u_n}a\left(1+\sqrt{1-\frac{a^2}{1-u_n^2}}\right)\right)$ $\implies$

$f(\frac 2a(1+u_{n+1}))+f(\frac 2a(1-u_{n+1}))$ $=f(\frac 2a(1+u_n))+f(\frac 2a(1-u_n))$

So $f(\frac 2a(1+u_n))+f(\frac 2a(1-u_n))=f(\frac 2a(1+u_0))+f(\frac 2a(1-u_0))$

Setting $n\to +\infty$ and using continuity, we get $f(\frac 2a(1+u_0))+f(\frac 2a(1-u_0))=2f(\frac 2a)$

So $f(x)+f(y)=2f(\frac{x+y}2)$ $\forall x>y>2$ (so that we can find $0<a<1$ and $0<u_0<\sqrt{1-a^2}<1$ from $x,y$)

And so (very classical) $f(x)=ax+b$ $\forall x>2$

Let then $0<x\le 2$. We can always choose $y\ne 1$ and $z>2$ such that $x=y+\frac 1z$

$P(z,y)$ $\implies$ $f(x)=ax+b$

Hence the answer $\boxed{f(x)=ax+b}$ $\forall x>0$, which indeed is a solution, whatever are $a,b\in\mathbb R$
\end{solution}
*******************************************************************************
-------------------------------------------------------------------------------

\begin{problem}[Posted by \href{https://artofproblemsolving.com/community/user/148231}{sqing}]
	Find all functions$ f(x)$ satisfying all conditions :
a) $f(x)$ is strictly increasing function on  $(0,\infty)$,
b) $f(x) f( f(x) + \frac{1}{x})=1$.
	\flushright \href{https://artofproblemsolving.com/community/c6h566771}{(Link to AoPS)}
\end{problem}



\begin{solution}[by \href{https://artofproblemsolving.com/community/user/29428}{pco}]
	\begin{tcolorbox}Find all functions$ f(x)$ satisfying all conditions :
a) $f(x)$ is strictly increasing function on  $(0,\infty)$,
b) $f(x) f( f(x) + \frac{1}{x})=1$.\end{tcolorbox}
I suppose domain of function and domain of functional equation both are $\mathbb R^+$
$f(x)\ne 0$ $\forall x$

Let $P(x)$ be the assertion $f(x)f(f(x)+\frac 1x)=1$

$P(x)$ $\implies$ $f(f(x)+\frac 1x)=\frac 1{f(x)}$


$P(f(x)+\frac 1x)$ $\implies$ $f(\frac 1{f(x)}+\frac 1{f(x)+\frac 1x})=f(x)$

Since strictly increasing, and so injective, this implies $\frac 1{f(x)}+\frac 1{f(x)+\frac 1x}=x$

And so $(f(x)-\frac{1+\sqrt 5}{2x})(f(x)-\frac{1-\sqrt 5}{2x})=0$

And so, since stricly increasing, $\boxed{f(x)=\frac{1-\sqrt 5}{2x}}$ $\forall x$, which indeed is a solution.
\end{solution}



\begin{solution}[by \href{https://artofproblemsolving.com/community/user/148231}{sqing}]
	Very nice.
Thanks.
\end{solution}
*******************************************************************************
-------------------------------------------------------------------------------

\begin{problem}[Posted by \href{https://artofproblemsolving.com/community/user/198285}{ilovemath121}]
	find all continuous function $ f:\mathbb{R}^{+}\to\mathbb{R}^{+} $ satisfy
1) $f(f(x))=x$
2)$ \displaystyle f(x+1)=\frac{f(x)}{f(x)+1} $
	\flushright \href{https://artofproblemsolving.com/community/c6h566821}{(Link to AoPS)}
\end{problem}



\begin{solution}[by \href{https://artofproblemsolving.com/community/user/173315}{chupungryenung}]
	$f(f(x))=x  =>  f $ is invertible $ <=>  f $ is bjiective $ =>  f $ is injective$ =>  f $ is strictly increasing or strictly decreasing

Suppose f is strictly increasing, then for any two points $x_{1}+1$ and $x_{2}+1$ we have  

$ f(x_{1}+1)<f(x_{2}+1) <=> f(x_{1})>f(x_{2})$

Suppose f is strictly decreasing, then for any two points $x_{1}+1$ and $x_{2}+1$ we have  

$ f(x_{1}+1)>f(x_{2}+1) <=> f(x_{1})<f(x_{2})$

In any of the two cases we have a contradiction, therefore there are no such functions.
(This might not be a solution, don't shout at me in case it is not...)

EDIT : wrong, don't bother reading...(Thank you, pco!)
\end{solution}



\begin{solution}[by \href{https://artofproblemsolving.com/community/user/29428}{pco}]
	\begin{tcolorbox}find all continuous function $ f:\mathbb{R}^{+}\to\mathbb{R}^{+} $ satisfy
1) $f(f(x))=x$
2)$ \displaystyle f(x+1)=\frac{f(x)}{f(x)+1} $\end{tcolorbox}
Let $p,n\in\mathbb  N$, $x\in\mathbb R^+$ :
From 2, we get $f(x+n)=\frac {f(x)}{1+nf(x)}$

From 1, we get then $f(\frac {f(x)}{1+nf(x)})=x+n$

From 2 again, we get then $f(\frac {f(x)}{1+nf(x)}+p)=\frac{x+n}{1+p(x+n)}$

Setting $n\to+\infty$ in this last equality and using continuity, we get $f(p)=\frac 1p$

Let then $A=\{x>0$ such that $f(x)=\frac 1x\}$
$x\in A$ $\implies$ $f(x)=\frac 1x$ $\implies$ $f(\frac 1x)=x$ $\implies$$\frac 1x\in A$
$x\in A$ $\implies$ $f(x+n)=\frac {f(x)}{1+nf(x)}$ $=\frac {1}{x+n}$ and so $x+n\in A$

Ans since $\mathbb N\subseteq A$, we get $f(x)=\frac 1x$ $\forall x\in\mathbb Q$ (use finite continued fraction representation)

And continuity ends the problem : $\boxed{f(x)=\frac 1x}$ $\forall x$, which indeed is a solution.
\end{solution}



\begin{solution}[by \href{https://artofproblemsolving.com/community/user/29428}{pco}]
	\begin{tcolorbox}$f(f(x))=x  =>  f $ is invertible $ <=>  f $ is bjiective $ =>  f $ is injective$ =>  f $ is strictly increasing or strictly decreasing

Suppose f is strictly increasing, then for any two points $x_{1}+1$ and $x_{2}+1$ we have  

$ f(x_{1}+1)<f(x_{2}+1) <=> f(x_{1})>f(x_{2})$

Suppose f is strictly decreasing, then for any two points $x_{1}+1$ and $x_{2}+1$ we have  

$ f(x_{1}+1)>f(x_{2}+1) <=> f(x_{1})<f(x_{2})$

In any of the two cases we have a contradiction, therefore there are no such functions.
(This might not be a solution, don't shout at me in case it is not...)\end{tcolorbox}
$ f(x_{1}+1)>f(x_{2}+1) <=> f(x_{1})<f(x_{2})$ is obviously wrong :
$ f(x_{1}+1)>f(x_{2}+1) <=> x_{1}<x_{2}$
\end{solution}
*******************************************************************************
-------------------------------------------------------------------------------

\begin{problem}[Posted by \href{https://artofproblemsolving.com/community/user/125553}{lehungvietbao}]
	1) Determine all continuous functions $f: \mathbb R\to\mathbb R$  such that 
\[f (x +2)-7f (x +1) f(10x) = 0 \quad \forall x\in\mathbb R\]

2) Determine all continuous functions $f: \mathbb R\to\mathbb R$  such that 
\[2f\left( \frac{ x + y}{ 2}\right) = f(f (x ) + f ( y) ) \quad \forall x\in\mathbb R\]
	\flushright \href{https://artofproblemsolving.com/community/c6h566942}{(Link to AoPS)}
\end{problem}



\begin{solution}[by \href{https://artofproblemsolving.com/community/user/29428}{pco}]
	\begin{tcolorbox}2) Determine all continuous functions $f: \mathbb R\to\mathbb R$  such that 
\[2f\left( \frac{ x + y}{ 2}\right) = f(f (x ) + f ( y) ) \quad \forall x\in\mathbb R\]\end{tcolorbox}
Let $P(x,y)$ be the assertion $2f(\frac{x+y}2)=f(f(x)+f(y))$

$P(x,x)$ $\implies$ $f(2f(x))=2f(x)$
So $u\in f(\mathbb R)$ $\implies$ $2u\in f(\mathbb R)$ and so four possibilities for $f(\mathbb R)$ :

1) $f(\mathbb R)=\{0\}$ and the solution $\boxed{f(x)=0}$ $\forall x$ which indeed is a solution

2) $f(\mathbb R)=\mathbb R$ and the solution ${\boxed{f(x)=x}}$ $\forall x$ which indeed is a solution

3) $f(\mathbb R)=[a,+\infty)$ or $(a,+\infty)$ for some $a\ge 0$
Then $f(2f(x))=2f(x)$ implies $f(x)=x$ $\forall x\ge 2a$
So $f(f(x)+f(y))=f(x)+f(y)$ $\forall x,y$ and $P(x,y)$ becomes $f(x)+f(y)=2f(\frac{x+y}2)$ $\forall x,y$
So $f(x)=x$ $\forall x$, which is impossible since  $f(\mathbb R)\ne\mathbb R$
So no solution.
 
4) $f(\mathbb R)=(-\infty,a]$ or $(-\infty,a)$ for some $a\le 0$
Then $f(2f(x))=2f(x)$ implies $f(x)=x$ $\forall x\le 2a$
So $f(f(x)+f(y))=f(x)+f(y)$ $\forall x,y$ and $P(x,y)$ becomes $f(x)+f(y)=2f(\frac{x+y}2)$ $\forall x,y$
So $f(x)=x$ $\forall x$, which is impossible since  $f(\mathbb R)\ne\mathbb R$
So no solution.
\end{solution}
*******************************************************************************
-------------------------------------------------------------------------------

\begin{problem}[Posted by \href{https://artofproblemsolving.com/community/user/125553}{lehungvietbao}]
	1) Find all functions $f:[0, + \infty] \to [0, + \infty)$ such that 
\[(f (x)) ^ 2\geq f (x + y) (f (x) + y) \quad \forall x\geq 0\]

2) Find all functions defined on $\mathbb R$ which satisfy the following conditions:
a) $f$ is even and periodic
b) $f(f(x)) = 1$.
	\flushright \href{https://artofproblemsolving.com/community/c6h566943}{(Link to AoPS)}
\end{problem}



\begin{solution}[by \href{https://artofproblemsolving.com/community/user/29428}{pco}]
	\begin{tcolorbox}2) Find all functions defined on $\mathbb R$ which satisfy the following conditions:
a) $f$ is even and periodic
b) $f(f(x)) = 1$.\end{tcolorbox}
I suggest you change your teacher. His \/ her so called "real" olympiad exercises are certainly not real.

This problem has infinitely many solutions and it's quite easy to build as many as we want.

For example : 

$f(x)=\frac{2+\cos\frac{\pi x}2-\left|\cos\frac{\pi x}2\right|}2$

$f(x)=\min\left(1,\frac 94\left(2\left\{\frac x6\right\}-1\right)^2\right)$

And a lot of amusing other one...

I hope your teacher will soon give you the general form (rather easy for continuous, much more complex according to me (I did not find them), for non continuous)
\end{solution}



\begin{solution}[by \href{https://artofproblemsolving.com/community/user/29428}{pco}]
	Here is a general form for all continuous solutions for problem 2 :

Let $a\le 1\le b$
Let $t\ge \max(|a|,b)$
Let $h(x)$ any continuous function from $\mathbb R\to\mathbb R$ such that $h(a)=h(b)=1$

$\forall x\in [\max(a,0),\max,(|a|,|b|)]$ : $f(x)=1$
$\forall x\in[0,t]\setminus[\max(a,0),\max,(|a|,|b|)]$ : $f(x)=\max(\min(h(x),b),a)$
$\forall x\in[-t,0]$ : $f(x)=-f(-x)$
$\forall x\notin[-t,+t]$ : $f(x)=f\left(2t\left\{\frac{x+t}{2t}\right\}-t\right)$


General solution for non continuous solutions is far more compex since $f(\mathbb R)$ may be unbounded and $\ne \mathbb R$
\end{solution}



\begin{solution}[by \href{https://artofproblemsolving.com/community/user/29428}{pco}]
	Example of funny unbounded non continuous solution :

$f(x)=1$ $\forall x\in\mathbb Q$
$f(x)=\left\lfloor e^{\frac 1{\left(2\left\{\frac{x+1}2\right\}-1\right)^2}}\right\rfloor$ $\forall x\notin\mathbb Q$

I really hope your teacher will give you the general form for these non continuous solutions 
\end{solution}
*******************************************************************************
-------------------------------------------------------------------------------

\begin{problem}[Posted by \href{https://artofproblemsolving.com/community/user/145904}{Mulpin}]
	Find all functions $f: \mathbb R\to\mathbb R$ such that

\[f(x^3)+f(y^3) = (x+y)f(x^2+y^2) - xyf(x+y) \, \, \forall x,y\in\mathbb R\]
	\flushright \href{https://artofproblemsolving.com/community/c6h567180}{(Link to AoPS)}
\end{problem}



\begin{solution}[by \href{https://artofproblemsolving.com/community/user/29428}{pco}]
	\begin{tcolorbox}Find all functions $f: \mathbb R\to\mathbb R$ such that

\[f(x^3)+f(y^3) = (x+y)f(x^2+y^2) - xyf(x+y) \, \, \forall x,y\in\mathbb R\]\end{tcolorbox}
Here is a looooong solution. I hope somebody will post a nicer one, especially for the key point $f(2x)=2f(x)$ :blush:

Let $P(x,y)$ be the assertion $f(x^3)+f(y^3)=(x+y)f(x^2+y^2)-xyf(x+y)$
Let $a=f(1)$

1) Preliminary results 
===============
$P(0,0)$ $\implies$ $f(0)=0$
$P(1,1)$ $\implies$ $f(2)=2a$
$P(x,0)$ $\implies$ $f(x^3)=xf(x^2)$
Adding $P(x,0)$ with $P(-x,0)$, we get $f(-x)=-f(x)$ and the function is odd

$a)$ : $P(x,y)$ $\implies$ $xf(x^2)+yf(y^2)=(x+y)f(x^2+y^2)-xyf(x+y)$
$b)$ : $P(x,-y)$ $\implies$ $xf(x^2)-yf(y^2)=(x-y)f(x^2+y^2)+xyf(x-y)$
$(x-y)a - (x+y)b$ $\implies$  $xy(-2f(x^2)+2f(y^2)+(x-y)f(x+y)+(x+y)f(x-y))=0$

And so new assertion $Q(x,y)$ : $2f(x^2)-2f(y^2)=(x-y)f(x+y)+(x+y)f(x-y)$ $\forall x,y\ne 0$

2) $f(x+2)=2f(x+1)-f(x)$ $\forall x$
=========================
Let $x\notin\{-2,-1,0\}$ :
$Q(x,x+1)$ $\implies$ $2f(x^2)-2f((x+1)^2)=-f(2x+1)-(2x+1)a$ 
$Q(x+1,x+2)$ $\implies$ $2f((x+1)^2)-2f((x+2)^2)=-f(2x+3)-(2x+3)a$ 
$Q(x+2,x)$ $\implies$ $2f((x+2)^2)-2f(x^2)=2f(2x+2)+(2x+2)2a$
Adding these three lines, we get $f(2x+3)=2f(2x+2)-f(2x+1)$

Setting there $x=\frac 12,1,\frac 32,2,\frac 52$, we get $f(8)=6f(3)-10a$ and $f(4)=2f(3)-2a$
$P(2,0)$ $\implies$ $f(8)=2f(4)$
Combining, we get $f(3)=3a$ and then $f(2x+3)=2f(2x+2)-f(2x+1)$ is true also when $x\in\{-2,-1,0\}$

So new assertion $R(x)$ : $f(x+2)=2f(x+1)-f(x)$ $\forall x$
Q.E.D.

3) $f(2x)=2f(x)$ and $f(x^2)=xf(x)$
==========================
$R(x)$ implies $f(n)=na$ and $f(x+n)=nf(x+1)-(n-1)f(x)$ $\forall x$, $\forall n\in\mathbb Z$

Then $P(x,n)$ $\implies$ $n^3(f(x^2+1)-f(x)-a)$ $+n^2x(f(x^2+1)-f(x^2)-f(x+1)+f(x))$ $+n(f(x^2)-xf(x))=0$
This polynomial in $n$ has infinitely many roots, and so is the zero polynomial and so :
$f(x^2+1)-f(x)-a=0$ $\forall x$
$x(f(x^2+1)-f(x^2)-f(x+1)+f(x))=0$ $\forall x$ and so $f(x+1)=f(x)+a$ and $f(x+n)=f(x)+na$
$f(x^2)-xf(x)=0$ $\forall x$ which implies that $Q(x,y)$ is true also when $x=0$ or $y=0$

Let $x\ne 0$ :
$Q((n+1)x,x)$ $\implies$ $nf((n+2)x)=2(n+1)f((n+1)x)-(n+2)f(nx)-2f(x)$
From there, we get :

$f(3x)=4f(2x)-5f(x)$ and so $f(9x^2)=3xf(3x)=12xf(2x)-15xf(x)$
$f(4x)=10f(2x)-16f(x)$ and so $f(4x^2)=2xf(2x)=10f(2x^2)-16f(x^2)$ and so $f(2x^2)=\frac x5f(2x)+8\frac x5f(x)$
$f(9x)=120f(2x)-231f(x)$ and so $f(9x^2)=120f(2x^2)-231f(x^2)$ $=120(\frac x5f(2x)+8\frac x5f(x))-231xf(x)$ $=24xf(2x)-39xf(x)$

And so $12xf(2x)-15xf(x)=24xf(2x)-39xf(x)$ and $f(2x)=2f(x)$ $\forall x$
Q.E.D.

4) $f(x+y)=f(x)+f(y)$ $\forall x,y$
========================
$a)$ : $Q(x,y)$ $\implies$ $2xf(x)-2yf(y)=(x-y)f(x+y)+(x+y)f(x-y)$
$b)$ :$Q(x+y,x-y)$ $\implies$ $(x+y)f(x+y)-(x-y)f(x-y)=2yf(x)+2xf(y)$
$(x-y)a - (x+y)b$ $\implies$  $(x^2+y^2)(f(x)+f(y))=(x^2+y^2)f(x+y)$ and so $f(x+y)=f(x)+f(y)$
Q.E.D.

5) $f(x)=ax$ $\forall x$
=============
$f((x+1)^2)=(x+1)f(x+1)$ becomes $f(x^2)+2f(x)+a=(x+1)(f(x)+a)=xf(x)+f(x)+x+a$ and so :
$\boxed{f(x)=ax}$ $\forall x$, which indeed is a solution, whatever is $a\in\mathbb R$
\end{solution}



\begin{solution}[by \href{https://artofproblemsolving.com/community/user/145904}{Mulpin}]
	Thankyou very much for your solution :D
\end{solution}
*******************************************************************************
-------------------------------------------------------------------------------

\begin{problem}[Posted by \href{https://artofproblemsolving.com/community/user/68025}{Pirkuliyev Rovsen}]
	Find all functions $f: \mathbb{R}\to\mathbb{R}$ such that $f(\frac{x+y+z+t}{4})=\frac{f(x)+f(y)+f(z)+f(t)}{4}$.
	\flushright \href{https://artofproblemsolving.com/community/c6h567268}{(Link to AoPS)}
\end{problem}



\begin{solution}[by \href{https://artofproblemsolving.com/community/user/29428}{pco}]
	\begin{tcolorbox}Find all functions $f: \mathbb{R}\to\mathbb{R}$ such that $f(\frac{x+y+z+t}{4})=\frac{f(x)+f(y)+f(z)+f(t)}{4}$.\end{tcolorbox}
Let $P(x,y,z,t)$ be the assertion $f(\frac{x+y+z+t}4)=\frac{f(x)+f(y)+f(z)+f(t)}4$

If $f(x)$ is solution, so is $f(x)+c$ and so WLOG $f(0)=0$

$P(4x,0,0,0)$ $\implies$ $f(4x)=4f(x)$

$P(4x,4y,0,0)$ $\implies$ $f(x+y)=f(x)+f(y)$

And so $\boxed{f(x)=a(x)+c}$ $\forall x$, which indeed is a solution, whatever is $c\in\mathbb R$ and $a(x)$ additive function.


If you add some constraint (monotonous, bounded, continuous, ...), you'll get $f(x)=ax+b$
\end{solution}
*******************************************************************************
-------------------------------------------------------------------------------

\begin{problem}[Posted by \href{https://artofproblemsolving.com/community/user/68025}{Pirkuliyev Rovsen}]
	Find all functions $f: \mathbb{R}\to\mathbb{R}$ such that $f(xy)=f(f(x)+f(y))$.
	\flushright \href{https://artofproblemsolving.com/community/c6h567272}{(Link to AoPS)}
\end{problem}



\begin{solution}[by \href{https://artofproblemsolving.com/community/user/29428}{pco}]
	\begin{tcolorbox}Find all functions $f: \mathbb{R}\to\mathbb{R}$ such that $f(xy)=f(f(x)+f(y))$.\end{tcolorbox}
$\boxed{f(x)=c}$ constant is a solution. So let us from now look only for non constant solutions.

Let $P(x,y)$ be the assertion $f(xy)=f(f(x)+f(y))$

Since $f(x)$ is not constant, let $u$ such that $f(u)\ne -f(0)$

$P(x,0)$ $\implies$ $f(f(x)+f(0))=f(0)$
$P(\frac x{f(u)+f(0)},f(u)+f(0))$ $\implies$ $f(x)=f(f(\frac x{f(u)+f(0)})+f(f(u)+f(0)))$ $=f(f(\frac x{f(u)+f(0)})+f(0))$ $=f(0)$

And so contradiction and no non constant solutions.
\end{solution}
*******************************************************************************
-------------------------------------------------------------------------------

\begin{problem}[Posted by \href{https://artofproblemsolving.com/community/user/68025}{Pirkuliyev Rovsen}]
	Find all functions $f: \mathbb{R}-{1}\to\mathbb{R}-{1}$ such that $f^{k}(x)+f^{k+1}(x)=\frac{3-x^2}{1-x}$, where $x{\in}R-{1}$, $k{{\in}N}$, $f^{k}(x)= \underbrace{f{\circ}f{\circ}\cdots{\circ}f}_{k\textrm{ times}}(x)$.
	\flushright \href{https://artofproblemsolving.com/community/c6h567285}{(Link to AoPS)}
\end{problem}



\begin{solution}[by \href{https://artofproblemsolving.com/community/user/29428}{pco}]
	\begin{tcolorbox}Find all functions $f: \mathbb{R}-{1}\to\mathbb{R}-{1}$ such that $f^{k}(x)+f^{k+1}(x)=\frac{3-x^2}{1-x}$, where $x{\in}R-{1}$, $k{{\in}N}$, $f^{k}(x)= \underbrace{f{\circ}f{\circ}\cdots{\circ}f}_{k\textrm{ times}}(x)$.\end{tcolorbox}
When you write "where $x{\in}R-{1}$, $k{{\in}N}$", you use the same formulation for $x$ and $k$. So, do you mean :

1) $\forall x\in\mathbb R\setminus\{1\}$ and $\forall k\in\mathbb N$ 

2) $\forall x\in\mathbb R\setminus\{1\}$ and for a given $k\in\mathbb N$ considered as a parameter of this problem

3) for a given $x\in\mathbb R\setminus\{1\}$ considered as a parameter of this problem and $\forall k\in\mathbb N$ 

4) for a given $x\in\mathbb R\setminus\{1\}$ considered as a parameter of this problem and  for a given $k\in\mathbb N$ considered as a parameter of this problem

:?:
\end{solution}



\begin{solution}[by \href{https://artofproblemsolving.com/community/user/68025}{Pirkuliyev Rovsen}]
	Sorry, in this case, the first variant. Thanks you.
\end{solution}



\begin{solution}[by \href{https://artofproblemsolving.com/community/user/29428}{pco}]
	\begin{tcolorbox}Find all functions $f: \mathbb{R}-{1}\to\mathbb{R}-{1}$ such that $f^{k}(x)+f^{k+1}(x)=\frac{3-x^2}{1-x}$, $\forall x{\in}R-{1}$, $\forall k{{\in}N}$, $f^{k}(x)= \underbrace{f{\circ}f{\circ}\cdots{\circ}f}_{k\textrm{ times}}(x)$.\end{tcolorbox}
If $f(x)=f(y)$ for some $x,y\ne 1$, we get $\frac{3-x^2}{1-x}=\frac{3-y^2}{1-y}$ and so (after some little computations) :
Either $y=x$
Either $y=\frac{x-3}{x-1}$

$f(x)+f^{[2]}(x)=f^{[2]}(x)+f^{[3]}(x)$ and so $f^{[3]}(x)=f(x)$ and so :

Either $f^{[2]}(x)=x$ and $f(x)=\frac{3-x^2}{1-x}-f^{[2]}(x)$ $=\frac{3-x^2}{1-x}-x$ $=\frac{x-3}{x-1}$
Either $f^{[2]}(x)=\frac{x-3}{x-1}$ and $f(x)=\frac{3-x^2}{1-x}-f^{[2]}(x)$ $=\frac{3-x^2}{1-x}-\frac{x-3}{x-1}$ $=x$

Second case obviously doest not fit (since $f(x)=x$ for some $x$ implies $f^{[2]}(x)=x$ and equation $x=\frac{x-3}{x-1}$ has no real root)

So $\boxed{f(x)=\frac{x-3}{x-1}}$ $\forall x\ne 1$ which indeed is a solution.
\end{solution}
*******************************************************************************
-------------------------------------------------------------------------------

\begin{problem}[Posted by \href{https://artofproblemsolving.com/community/user/125553}{lehungvietbao}]
	1) Find all continuous functions $f:\mathbb{R}\to \mathbb{R}$ such that:
\[f(xy)+f(x+y)=f(xy+x)+f(y)\quad \forall x,y\in \mathbb{R}\]

2) Find all  functions $f:\mathbb{R}\to \mathbb{R}$ such that:
\[f(f(x)+y)=f(f(x)-y)+4yf(x)\quad  \forall x,y \in \mathbb{R}.\]
	\flushright \href{https://artofproblemsolving.com/community/c6h567455}{(Link to AoPS)}
\end{problem}



\begin{solution}[by \href{https://artofproblemsolving.com/community/user/29428}{pco}]
	\begin{tcolorbox}2) Find all  functions $f:\mathbb{R}\to \mathbb{R}$ such that:
\[f(f(x)+y)=f(f(x)-y)+4yf(x)\quad  \forall x,y \in \mathbb{R}.\]\end{tcolorbox}
$\boxed{f(x)=0}$ $\forall x$ is a solution. So let us from now look only for non allzero solutions.

Let $P(x,y)$ be the assertion $f(f(x)+y)=f(f(x)-y)+4yf(x)$
Let $u$ such that $f(u)\ne 0$

$P(u,\frac x{8f(u)})$ $\implies$ $x=2f(f(u)+\frac x{4f(u)})-2f(f(u)-\frac x{4f(u)})$ and so any real may we written $x=2f(a)-2f(b)$ for some $a,b\in\mathbb R$

$P(a,2f(b)-f(a))$ $\implies$ $f(2f(b))=f(2f(a)-2f(b))+8f(a)f(b)-4f(a)^2$
$P(b,f(b))$ $\implies$ $f(2f(b))=f(0)+4f(b)^2$
Subtracting, we get $f(2f(a)-2f(b))=f(0)+(2f(a)-2f(b))^2$

And so $\boxed{f(x)=x^2+c}$ $\forall x$, which indeed is a solution, whatever is $c\in\mathbb R$
\end{solution}



\begin{solution}[by \href{https://artofproblemsolving.com/community/user/29428}{pco}]
	\begin{tcolorbox}1) Find all continuous functions $f:\mathbb{R}\to \mathbb{R}$ such that:
\[f(xy)+f(x+y)=f(xy+x)+f(y)\quad \forall x,y\in \mathbb{R}\]\end{tcolorbox}
Let $P(x,y)$ be the assertion $f(xy)+f(x+y)=f(xy+x)+f(y)$
$f(x)$ solution implies $f(x)+c$ solution. So WLOG $f(0)=0$

Subtracting $P(x,y)$ from $P(y,x)$, we get new assertion $Q(x,y)$ : $f(xy+x)+f(y)=f(xy+y)+f(x)$

1) $f(x)=xf(1)$ $\forall x\ge 0$
====================
Let $x>y>0$ :

Consider the sequence $y_0=y$ and $y_{n+1}=\frac{x^2+xy_n+y_n}{x+y_n+1}$
$y_n$ is an increasing sequence whose limit is $x$

$Q(\frac{x-y_n}{x+y_n+1},x+y_n)$ $\implies$ $f(x-y_n)+f(x+y_n)=f(x-y_{n+1})+f(x+y_{n+1})$
So $f(x-y_n)+f(x+y_n)=f(x-y)+f(x+y)$
Setting $n\to+\infty$ and using continuity, we get $f(x-y)+f(x+y)=f(2x)$ $\forall x>y>0$

And so $f(x+y)=f(x)+f(y)$ $\forall x,y>0$ $x\ne y$
Continuity implies then $f(x+y)=f(x)+f(y)$ $\forall x,y\ge 0$ and so $f(x)=xf(1)$ $\forall x\ge 0$
Q.E.D.

2) $f(-x)=-f(x)$ $\forall x$
================
Let $x>0$ : 

Consider the sequence $x_0=x$ and $x_{n+1}=\frac{x_n}{x_n+1}$
$x_n$ is a decreasing sequence whose limit is $0$

$Q(x_n,-x_{n+1})$ $\implies$ $f(x_{n+1})+f(-x_{n+1})=f(-x_n)+f(x_n)$
So $f(-x_n)+f(x_n)=f(-x)+f(x)$
Setting $n\to+\infty$ and using continuity, we get $f(-x)=-f(x)$ $\forall x>0$
Q.E.D.

3) Solution
=======
So $\boxed{f(x)=ax+b}$ $\forall x$, which indeed is a solution, whatever are $a,b\in\mathbb R$
\end{solution}
*******************************************************************************
-------------------------------------------------------------------------------

\begin{problem}[Posted by \href{https://artofproblemsolving.com/community/user/68025}{Pirkuliyev Rovsen}]
	Function $f: \mathbb{R}\to\mathbb{R}$ satisfies the conditions $f^3(x+y)+f^3(x-y)=(f(x)+f(y))^3+(f(x)-f(y))^3$.Prove that $f(x+y)=f(x)+f(y)$.
	\flushright \href{https://artofproblemsolving.com/community/c6h567464}{(Link to AoPS)}
\end{problem}



\begin{solution}[by \href{https://artofproblemsolving.com/community/user/29428}{pco}]
	\begin{tcolorbox}Function $f: \mathbb{R}\to\mathbb{R}$ satisfies the conditions $f^3(x+y)+f^3(x-y)=(f(x)+f(y))^3+(f(x)-f(y))^3$.Prove that $f(x+y)=f(x)+f(y)$.\end{tcolorbox}
Let $P(x,y)$ be the assertion $f^3(x+y)+f^3(x-y)=(f(x)+f(y))^3+(f(x)-f(y))^3$

$P(0,0)$ $\implies$ $f(0)=0$
$P(0,x)$ $\implies$ $f(-x)=-f(x)$

Let $g(x,y)=f^3(x+y)-(f(x)+f(y))^3$. Note that $g(x,y)=g(y,x)$ and that $g(-x,-y)=-g(x,y)$

Equation is $g(x,y)=-g(x,-y)$

So $g(x,y)=-g(x,-y)=-g(-y,x)=g(-y,-x)=-g(x,y)$ and so $g(x,y)=0$ $\forall x,y$

Q.E.D.
\end{solution}
*******************************************************************************
-------------------------------------------------------------------------------

\begin{problem}[Posted by \href{https://artofproblemsolving.com/community/user/68025}{Pirkuliyev Rovsen}]
	Find all functions $f: \mathbb{R}\to\mathbb{R}$ such that $f^3(x)+x^4f(x)=x^3+x^5$ for all $x{\in}R$.
	\flushright \href{https://artofproblemsolving.com/community/c6h567465}{(Link to AoPS)}
\end{problem}



\begin{solution}[by \href{https://artofproblemsolving.com/community/user/29428}{pco}]
	\begin{tcolorbox}Find all functions $f: \mathbb{R}\to\mathbb{R}$ such that $f^3(x)+x^4f(x)=x^3+x^5$ for all $x{\in}R$.\end{tcolorbox}
Write equation as $(f(x)-x)((2f(x)+x)^2+3x^2+4x^4)=0$

And so $\boxed{f(x)=x}$ $\forall x$
\end{solution}
*******************************************************************************
-------------------------------------------------------------------------------

\begin{problem}[Posted by \href{https://artofproblemsolving.com/community/user/168537}{vutuanhien}]
	Find all continuous functions such that
$C_{n}^{0}f(x)+C_{n}^{1}f(x^2)+...+C_{n}^{n}f(x^{2^n})=0$ $\forall x\in \mathbb{R}$
	\flushright \href{https://artofproblemsolving.com/community/c6h567477}{(Link to AoPS)}
\end{problem}



\begin{solution}[by \href{https://artofproblemsolving.com/community/user/29428}{pco}]
	\begin{tcolorbox}Find all continuous functions such that
$C_{n}^{0}f(x)+C_{n}^{1}f(x^2)+...+C_{n}^{n}f(x^{2^n})=0$ $\forall x\in \mathbb{R}$\end{tcolorbox}
I suppose you mean  $\forall n$, and not just for one $n$ considered as a parameter. If so :

If $n=0$, we get $f(x)=0$ $\forall x$

If $n>0$, let $g(x)=\sum_{k=0}^{n-1}\binom {n-1}kf(x^{2^k})$ and equation becomes $g(x)+g(x^2)=0$ whose unique continuous solution is easily found as $g(x)=0$ $\forall x$

And simple induction concludes $\boxed{f(x)=0}$ $\forall x$
\end{solution}



\begin{solution}[by \href{https://artofproblemsolving.com/community/user/168537}{vutuanhien}]
	\begin{tcolorbox}[quote="vutuanhien"]Find all continuous functions such that
$C_{n}^{0}f(x)+C_{n}^{1}f(x^2)+...+C_{n}^{n}f(x^{2^n})=0$ $\forall x\in \mathbb{R}$\end{tcolorbox}
I suppose you mean  $\forall n$, and not just for one $n$ considered as a parameter. If so :

If $n=0$, we get $f(x)=0$ $\forall x$

If $n>0$, let $g(x)=\sum_{k=0}^{n-1}\binom {n-1}kf(x^{2^k})$ and equation becomes $g(x)+g(x^2)=0$ whose unique continuous solution is easily found as $g(x)=0$ $\forall x$

And simple induction concludes $\boxed{f(x)=0}$ $\forall x$\end{tcolorbox}
No, $n$  is a constant, and $C_{n}^{0}f(x)+C_{n}^{1}f(x^2)+...+C_{n}^{n}f(x^{2^n})=0$ $\forall x\in \mathbb{R}$
Can you do it, Mr Patrick?
\end{solution}



\begin{solution}[by \href{https://artofproblemsolving.com/community/user/29428}{pco}]
	\begin{tcolorbox}No, $n$  is a constant, and $C_{n}^{0}f(x)+C_{n}^{1}f(x^2)+...+C_{n}^{n}f(x^{2^n})=0$ $\forall x\in \mathbb{R}$
Can you do it, Mr Patrick?\end{tcolorbox}
Yes I do.
In fact, I did : my proof above is for $n$ given : I proved that $\sum_{k=0}^{n}\binom nk f(x^{2^k})=0$ implies  $\sum_{k=0}^{n-1}\binom {n-1}k f(x^{2^k})=0$

And so ... $f(x)=0$ $\forall x$
\end{solution}
*******************************************************************************

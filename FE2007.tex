-------------------------------------------------------------------------------

\begin{problem}[Posted by \href{https://artofproblemsolving.com/community/user/461}{dreammath}]
	Find all functions $f: \mathbb R^{>0} \to  \mathbb R^{>0} $ such that
\[f(x)f(yf(x))=f(x+y)\]
holds for all reals $x$ and $y$.
	\flushright \href{https://artofproblemsolving.com/community/c6h17444}{(Link to AoPS)}
\end{problem}



\begin{mysolution}[by \href{https://artofproblemsolving.com/community/user/5820}{N.T.TUAN}]
	I know a solution, but what is your solution?  
\end{mysolution}



\begin{mysolution}[by \href{https://artofproblemsolving.com/community/user/14723}{andyciup}]
	This is from IMC, 2000 : \url{http://www.mathlinks.ro/Forum/viewtopic.php?search_id=1113833585&t=58521}
\end{mysolution}



\begin{mysolution}[by \href{https://artofproblemsolving.com/community/user/18420}{aviateurpilot}]
	it's easy,
$ f(x)=\frac{|2-x|+2-x}{(2-x)^{2}}\ if\ x\neq 2\ and\ f(2)=0$


or $ f(x)=\frac{Max(2-x,0)}{(2-x)+[10^{-|x-2|}]}$  
\end{mysolution}



\begin{mysolution}[by \href{https://artofproblemsolving.com/community/user/5820}{N.T.TUAN}]
	aviateurpilot, can you post concrete?
\end{mysolution}



\begin{mysolution}[by \href{https://artofproblemsolving.com/community/user/29428}{pco}]
$ $\newline
\begin{tcolorbox}Find all functions: R* to R* (R* means the set of positive reals.)
$ f(x)f(yf(x))=f(x+y)$.
It 's very interesting for anybody who wants to solve it. 
\end{tcolorbox}

Claim 1 : $ f(x)\leq 1 \forall  x>0$. 
Let $ x_{0}>0$ such that $ f(x_{0})>1$. Then let $ y_{0}=\frac{x_{0}}{f(x_{0})-1}>0$.
$ f(x_{0})f(y_{0}f(x_{0}))=f(x_{0}+y_{0})$ $ \implies$ $ f(x_{0}) f(\frac{x_{0}f(x_{0})}{f(x_{0})-1}) = f(\frac{x_{0}f(x_{0})}{f(x_{0})-1})$ and, since $ f(x)>0$ $ \forall x$ : $ f(x_{0})=1$, which is a contradiction. So Claim 1 is true.

Claim 2: $ f(x)$ is a non-increasing function
Obvious since $ y>x$ $ \implies$ $ f(x) f((y-x)f(x))=f(y)$ and so $ f(y)\leq f(x)$ since $ f((y-x)f(x))\leq 1$ (with Claim 1)

Claim 3 : if it exists $x_0>0$ such that $ f(x_0)=1$, then $ f(x)=1$ for all $ x>0$
Just put $ x=x_{0}$ in the original equation and you get $ f(y+x_{0})=f(y)$ $ \forall$ $ y>0$. and so $ f(x)$ is a constant (since it is a non increasing function (claim 2)) and so $ f(x)=f(x_{0})=1$ $ \forall$ $ x>$ and Claim 3 is true.

Claim 4: if $ f(x)$ is non constant, then $ f(x)$ is an injective function (and so is strictly decreasing)
Let $ f(u)=f(v)$ with $ u>v$. Then $ f(v) f((u-v)f(v))=f(u)$ and so $ f((u-v)f(v))=1$ and so $ f(x)=1$ (with claim 3). Q.E.D.

Solutions :
$ f(x)=1$ is a solution.
Consider now $ f(x)$ non is constant, that's to say $ f(x)$ injective (claim 4) :
Let $ y=\frac{z}{f(x)}$. Then the initial equation becomes $ f(x)f(z)=f(x+\frac{z}{f(x)})$.
But we also have $ f(z)f(x)=f(z+\frac{x}{f(z)})$
So  $ f(x+\frac{z}{f(x)})=f(z+\frac{x}{f(z)})$ and, since $ f(x)$ is injective : $ x+\frac{z}{f(x)}=z+\frac{x}{f(z)}$ and so (dividing by $ xz$) $ \frac{1}{z}+\frac{1}{xf(x)}=\frac{1}{x}+\frac{1}{zf(z)}$.

So $ \frac{1}{z}-\frac{1}{zf(z)}=\frac{1}{x}-\frac{1}{xf(x)}=a$ and $ f(x)=\frac{1}{1-ax}$

And it is easy to verify that this expression is a solution of initial equation as soon as $ a\leq 0$ (since we want $ f(x)>0$)

So the general solution is $ f(x)=\frac{1}{1+ax}$ for any $ a\geq 0$
\end{mysolution}



\begin{mysolution}[by \href{https://artofproblemsolving.com/community/user/29428}{pco}]
	\begin{tcolorbox}it's easy,
$ f(x)=\frac{|2-x|+2-x}{(2-x)^{2}}\ if\ x\neq 2\ and\ f(2)=0$


or $ f(x)=\frac{Max(2-x,0)}{(2-x)+[10^{-|x-2|}]}$  \end{tcolorbox}

That's wrong since $ f(x)>0$ $ \forall x>0$ (f : $ R^{*}\rightarrow R^{*}$)
\end{mysolution}



\begin{mysolution}[by \href{https://artofproblemsolving.com/community/user/18420}{aviateurpilot}]
	\begin{tcolorbox}it's easy,
$ f(x)=\frac{|2-x|+2-x}{(2-x)^{2}}\ if\ x\neq 2\ and\ f(2)=0$


or $ f(x)=\frac{Max(2-x,0)}{(2-x)+[10^{-|x-2|}]}$  \end{tcolorbox}
sorry \begin{bolded}N.T.TUAN\end{bolded}, here it's solution for another problem (easy) where $ f(2)=0$ and $ f(x+y)=f(x)f(yf(x)),\forall (x,y)\in \mathbb (R^{+})^{2}$
and $ f(x)=0\ in [0,2[$  :rotfl: .
\end{mysolution}



\begin{mysolution}[by \href{https://artofproblemsolving.com/community/user/5820}{N.T.TUAN}]
	joke :P  is it a problem from IMO? pco are right!
\end{mysolution}
*******************************************************************************
-------------------------------------------------------------------------------

\begin{problem}[Posted by \href{https://artofproblemsolving.com/community/user/5729}{ehsan2004}]
	Find all the function $f: \mathbb R^{\geq 0} \to \mathbb R^{\geq 0}$ such that for all $x, y \in \mathbb R^{\geq 0}$,
\[\sqrt[2]{f(\frac {x^2+y^2}{2})}=\frac {f(x)+f(y)}{2}.\]
	\flushright \href{https://artofproblemsolving.com/community/c6h42382}{(Link to AoPS)}
\end{problem}



\begin{mysolution}[by \href{https://artofproblemsolving.com/community/user/29428}{pco}]
	\begin{tcolorbox}Find all the function $ f: R^ + \cup {\{0}\}\longrightarrow R^ + \cup {\{0}\}$ such that:
$ \forall x,y\geq 0: \sqrt [2]{f(\frac {x^2 + y^2}{2})} = \frac {f(x) + f(y)}{2}$. \end{tcolorbox}

Let $ g(x)=\sqrt{f(x)}$

We have $ g(\frac {x^2 + y^2}{2})= \frac {g^2(x) + g^2(y)}{2}$

And so, taking $ x=y=0$ : $ g(0)=g^2(0)$ and either $ g(0)=0$, or $ g(0)=1$

1) If $ g(0)=0$, taking $ y=0$, we have $ g(\frac {x^2}{2})= \frac {g^2(x)}{2}$ and so $ g(\frac {x^2 + y^2}{2})=g(\frac {x^2}{2})+g(\frac {y^2}{2})$
And so $ g(x+y)=g(x)+g(y)$ (Cauchy equation) and since $ g(x)\geq 0$, $ g(x)=ax$ (since any non continuous solution of Cauchy's equation is unbounded on any non empty open  interval).
Putting back this expression in the original equation, we find $ a=0$ or $ a=1$

And two solutions :
$ f(x)=0$
$ f(x)=x^2$

2) If $ g(0)=1$, taking $ y=0$, we have $ g(\frac {x^2}{2})= \frac {g^2(x)}{2}+\frac{1}{2}$ and so $ g(\frac {x^2 + y^2}{2})=g(\frac {x^2}{2})+g(\frac {y^2}{2})-1$
And so $ g(x+y)-1=(g(x)-1)+(g(y)-1)$ (Cauchy equation) and, since $ g(x)\geq 0$, $ g(x)=ax+1$ (since any non continuous solution of Cauchy's equation is unbounded on any non empty open  interval).
Putting back this expression in the original equation, we find $ a=0$

And a third solution :
$ f(x)=1$
\end{mysolution}
*******************************************************************************
-------------------------------------------------------------------------------

\begin{problem}[Posted by \href{https://artofproblemsolving.com/community/user/285}{harazi}]
	Suppose $f$ is a real function such that for all $x,y$ we have $|f(x)+f(y)|=|f(x+y)|$. Then $f$ is additive.
	\flushright \href{https://artofproblemsolving.com/community/c6h45038}{(Link to AoPS)}
\end{problem}



\begin{mysolution}[by \href{https://artofproblemsolving.com/community/user/29428}{pco}]
	\begin{tcolorbox}Suppose $ f$ is a real function such that for all $ x,y$ we have $ |f(x) + f(y)| = |f(x + y)|$. Then $ f$ is additive.\end{tcolorbox}

For any pair $ (x,y)$, either $ f(x+y)=f(x)+f(y)$, or $ f(x+y)=-f(x)-f(y)$

It's rather immediate to see that $ f(0)=0$ and that $ f(-x)=-f(x)$ $ \forall x$.

Let then $ x,y$ such that $ f(x+y)=-f(x)-f(y)$
We have either $ f(x+(-2x))=f(x)+f(-2x)$, or $ f(x+(-2x))=-f(x)-f(-2x)$
1) If $ f(x+(-2x))=-f(x)-f(-2x)$, it means $ f(-x)=-f(x)-f(-2x)$ and since $ f(-x)=-f(x)$ : $ f(2x)=0$ and so $ f(x)=0$
2) If $ f(x+(-2x))=f(x)+f(-2x)$, then
We have $ f(x+y+(-2x))=\epsilon_1f(x+y)+\epsilon_1f(-2x)=-\epsilon_1f(x)-\epsilon_1f(y)+\epsilon_1f(-2x)$ where $ \epsilon_1\in\{-1,+1\}$
We also have $ f(x+y+(-2x))=\epsilon_2f(x+(-2x))+\epsilon_2f(y)=\epsilon_2f(x)+\epsilon_2f(y)+\epsilon_2f(-2x)$ where $ \epsilon_2\in\{-1,+1\}$

And so, by subtracting the two equalities, we have :
$ (\epsilon_1+\epsilon_2)(f(x)+f(y))+(\epsilon_2-\epsilon_1)f(-2x)=0$
Then :
2.1) either $ \epsilon_1=\epsilon_2$ and we have $ f(x)+f(y)=0$
2.2) or $ \epsilon_1=-\epsilon_2$ and we have $ f(-2x)=0$ and so $ f(x)=0$

So, $ f(x+y)=-f(x)-f(y)$ implies either $ f(x)=0$, or $ f(x)+f(y)=0$
With the same demonstration, using $ f(y+(-2y))$ instead of $ f(x+(-2x))$, we have :
So, $ f(x+y)=-f(x)-f(y)$ implies either $ f(y)=0$, or $ f(x)+f(y)=0$

Then $ f(x+y)=-f(x)-f(y)$ implies always $ f(x)+f(y)=0$ (since $ f(x)=0$ and $ f(y)=0$ imply too  $ f(x)+f(y)=0$).

So $ f(x+y)=-f(x)-f(y)$ implies that always $ f(x+y)=0$ and so $ f(x+y)=-f(x+y)=f(x)+f(y)$

So we always have $ f(x+y)=f(x)+f(y)$ and $ f(x)$ is additive.
\end{mysolution}
*******************************************************************************
-------------------------------------------------------------------------------

\begin{problem}[Posted by \href{https://artofproblemsolving.com/community/user/8597}{Cezar Lupu}]
	Prove that there is a function $f:\mathbb{R}\to\mathbb{R}$ such that $f(f(x))=-x$, for any $x\in\mathbb{R}$ ;)
	\flushright \href{https://artofproblemsolving.com/community/c6h49191}{(Link to AoPS)}
\end{problem}



\begin{mysolution}[by \href{https://artofproblemsolving.com/community/user/285}{harazi}]
	Very well, too bad that it's already posted.
\end{mysolution}



\begin{mysolution}[by \href{https://artofproblemsolving.com/community/user/8597}{Cezar Lupu}]
	I apologise, I didn't know that. :blush:
\end{mysolution}



\begin{mysolution}[by \href{https://artofproblemsolving.com/community/user/13}{enescu}]
	There are many such fuctions. However, there's an interesting geometric observation: rotating the graph of such $f$ with $90^{\circ}$  around the origin we obtain the same graph (I think this is an old russian problem).
Here's an example:
\end{mysolution}
*******************************************************************************
-------------------------------------------------------------------------------

\begin{problem}[Posted by \href{https://artofproblemsolving.com/community/user/7502}{Mathx}]
	find all function $f:\mathbb{N}\rightarrow \mathbb{R}$ such that
1. $f(x+22)=f(x)$ for all $x \in \mathbb N$, and
2. $f(x^2y)=f(f(x))^2 \cdot f(y)$ for all $x,y \in \mathbb N$.
	\flushright \href{https://artofproblemsolving.com/community/c6h49768}{(Link to AoPS)}
\end{problem}



\begin{mysolution}[by \href{https://artofproblemsolving.com/community/user/29428}{pco}]
	\begin{tcolorbox}find all function $ f: \mathbb{N}\rightarrow \mathbb{R}$ such that:

1.$ f(x + 22) = f(x)$
2.$ \forall x,y \in \mathbb{N}$   $ f(x^2y) = f(f(x))^2.f(y)$\end{tcolorbox}

First notice that although $ f(x)$ is stated $ f: \mathbb{N}\rightarrow \mathbb{R}$, we need $ f: \mathbb{N}\rightarrow \mathbb{N}$ in order to have $ f(f(x))$ defined in the equation in point 2.

Then, notice that $ f(x + 22) = f(x)$ implies that $ f(n)$ take at most $ 22$ different values when $ n\in\mathbb{N}$

Now, since $ f(x)\in\mathbb{N}$, $ f(f(x))^2\in\mathbb{N}$. Assume that $ \exists x_0\in\mathbb{N}$ such that $ f(f(x_0))>1$. If $ x_0=1$, take $ x_0=23$ (since $ f(23)=f(1)$). So we can consider $ x_0>1$. Then $ f(x_0^2y)>f(y)$, but also $ f(x_0^4y)>f(x_0^2y)>f(y)$ and we can easily find more than $ 22$ different values for $ f(x)$, which is impossible.

So $ f(f(x))=1$ $ \forall x\in\mathbb{N}$ and $ f(x^2y)=f(y)$ $ \forall x,y\in\mathbb{N}$

Then let $ x>y\in\mathbb{N}$ :
$ f(22^2x)=f(x)$
$ f(22^2y)=f(y)$
$ f(22^2x)=f(22^2y+22k)$ with $ k=22(x-y)$ and so $ f(22^x)=f(22^2y)$ and so $ f(x)=f(y)$ and so $ f(x)=c$ and so $ f(x)=1$ (since $ f(f(x))=1$)

And the only solution to these equations is $ f(x)=1$ $ \forall x\in\mathbb{N}$
\end{mysolution}



\begin{mysolution}[by \href{https://artofproblemsolving.com/community/user/5820}{N.T.TUAN}]
	This isn't a problem from APMO 2002, in APMO 2002 equational problem is http://www.mathlinks.ro/Forum/viewtopic.php?p=474887 .
\end{mysolution}



\begin{mysolution}[by \href{https://artofproblemsolving.com/community/user/34234}{noname69}]
	Actually it is similar to APMO 2003 - Problem 7
The conditions are the same but $ f(x^2y)=(f(x))^2f(y)$
\end{mysolution}
*******************************************************************************
-------------------------------------------------------------------------------

\begin{problem}[Posted by \href{https://artofproblemsolving.com/community/user/1598}{Arne}]
	Does there exist a function $f: \mathbb{N} \rightarrow \mathbb{N}$ such that \[f^{\left(f(n)\right)}(n) = n + 1,\;\forall n \in \mathbb{N},\] where $f^{(k)}(n) = f\left(f^{(k - 1)}(n)\right),\ \forall k \in \mathbb{N}$?
	\flushright \href{https://artofproblemsolving.com/community/c6h51524}{(Link to AoPS)}
\end{problem}



\begin{mysolution}[by \href{https://artofproblemsolving.com/community/user/29428}{pco}]
	\begin{tcolorbox}Does there exist a function $ f: \mathbb{N} \rightarrow \mathbb{N}$ such that
\[ f^{\left(f(n)\right)}(n) = n + 1,\;\forall n \in \mathbb{N},
\]
where $ f^{(k)}(n) = f\left(f^{(k - 1)}(n)\right),\ \forall k \in \mathbb{N}$?\end{tcolorbox}

I suppose that, as generally in this forum, $ 0\notin\mathbb{N}$

As a consequence, there does not exist any $ n_0\in\mathbb{N}$ such that $ f(n_0)=1$. Else we would have $ n_0+1=f^{f(n_0)}(n_0)=f(n_0)=1$ and so $ n_0=0$

$ 1)$ We know that $ f(1)\neq 1$. Assume then $ f(1)=2$. Then $ f^{f(1)}(1)=2$ and so $ f^2(1)=2$ and so $ f(f(1))=2$ and $ f(2)=2$.
But this is impossible since we would have then $ f^k(2)=2$ $ \forall k$ and $ f^{f(2)}(2)=3$ would be impossible.
So $ f(1)>2$.

$ 2)$ $ f(x)$ is injective : $ f(a)=f(b)>1$ implies $ a+1=f^{f(a)}(a)=f^{f(b)}(a)$ $ =f^{f(b)-1}(f(a))=f^{f(b)-1}(f(b))$ $ =f^{f(b)}(b)=b+1$ and so $ a=b$

$ 3)$ $ f^{f(1)}(1)=2$, then $ f^{f(1)+f(2)}(1)=f^{f(2)}(2)=3$ and so on : So $ \boxed{f^{\sum_{k=1}^{n}f(k)}(1)=n+1\: \: \forall n>0}$

$ 4)$ But, since $ f(1)>2$ and so $ f^{\sum_{k=1}^{f(1)-1}f(k)}(1)=f(1)$ and so $ f(f^{(\sum_{k=1}^{f(1)-1}f(k))-1}(1))=f(1)$ and, since $ f(x)$ is injective :

$ f(f^{(\sum_{k=1}^{f(1)-1}f(k))-2}(1))=1$ which is impossible (consider that $ f(1)>2$ implies that $ (\sum_{k=1}^{f(1)-1}f(k))-2>0$)

So, such function $ f(x)$ does not exist.
\end{mysolution}
*******************************************************************************
-------------------------------------------------------------------------------

\begin{problem}[Posted by \href{https://artofproblemsolving.com/community/user/9954}{Rushil}]
	For real numbers $a,b,c,d$ not all equal to $0$ , define a real function $f(x) = a +b\cos{2x} + c\sin{5x} +d \cos{8x}$. Suppose $f(t) = 4a$ for some real $t$. prove that there exists a real number $s$ s.t. $f(s)<0$
	\flushright \href{https://artofproblemsolving.com/community/c6h53000}{(Link to AoPS)}
\end{problem}



\begin{mysolution}[by \href{https://artofproblemsolving.com/community/user/29428}{pco}]
	Hello,

Here is a not-so-simple solution

\begin{tcolorbox}For real numbers $a,b,c,d$ not all equal to $0$ , define a real function $f(x) = a+b\cos{2x}+c\sin{5x}+d \cos{8x}$. Suppose $f(t) = 4a$ for some real $t$. prove that there exist a real number $s$ s.t. $f(s)<0$\end{tcolorbox}

I'll study $g(x)=b\cos{2x}+c\sin{5x}+d \cos{8x}$ and show that $(P1)$ :$\max_{x\in[0,2\pi]}g(x)+3\min_{x\in[0,2\pi]}g(x)<0$ $\forall (b,c,d)$

This will solve the problem since :
$f(t)=4a$ means $g(t)=3a$ $\Rightarrow $ $\max_{x\in[0,2\pi]}g(x)\geq 3a$ and, using $P1$, we then have $\min_{x\in[0,2\pi]}g(x)<-a$ $\Rightarrow $ $f(x_{min})=a+\min_{x\in[0,2\pi]}g(x)<0$

Let us now demonstrate $(P1)$. For the following of the demo, I'll say $\max_{x\in[0,2\pi]}g(x)\leq |b|+|c|+|d|$ and shall find :
Either value $g(x_{i})$ such that $S1=|b|+|c|+|d|+3g({x_{i}})<0$ and that will be enough (using  $\min_{x\in[0,2\pi]}g(x)\leq g(x_{i})$).
or values  $g(x_{i})$ and  $g(x_{j})$ such that $S2=|b|+|c|+|d|+3\frac{g({x_{i}})+g(x_{j})}{2}<0$ and that will be enough too ($\min_{x\in[0,2\pi]}g(x)\leq \frac{g({x_{i}})+g(x_{j})}{2}$).

0) First, I just want to establish some inequalities :
I1) $3\sin{\frac{pi}{4}}-1>0$ (obvious)
I2) $3\cos{\frac{pi}{4}}-1>0$ (obvious)
I3) $3\sin{\frac{pi}{8}}-1>0$ (demo : $\frac{1}{2}<\frac{49}{81}$ $\Rightarrow $ $\cos{\frac{\pi}{4}}<\frac{7}{9}$ $\Rightarrow $ $(\sin{\frac{\pi}{8}})^{2}>\frac{1}{9}$  $\Rightarrow $ $3\sin{\frac{pi}{8}}-1>0$ )
I4) $3\cos{\frac{pi}{8}}-1>0$ (obvious since $\cos{\frac{pi}{8}}>\sin{\frac{pi}{8}}$ )
I5) $\frac{3}{2}\sin{\frac{\pi}{4}}-1>0$ (easy to check)


Notice for the following that $b$, $c$ and $d$ are not all-zero since this would imply $f(x)=a\neq 0$, and it would be impossible to find $t$ such that $f(t)=4a$.

1) Case : $b\geq 0$, $c\geq 0$, $d\geq 0$ :
With $x=\frac{3\pi}{8}$, we have $g(x)=-b\sin{\frac{pi}{4}}-c\sin{\frac{\pi}{8}}-d$ $\Rightarrow $ $S1=-b(3\sin{\frac{pi}{4}}-1)-c(3\sin{\frac{\pi}{8}}-1)-2d < 0$.
We have $S1<0$ and not $S1\leq 0$ since $b$, $c$, and $d$ are not all-zero.

2) Case : $b\geq 0$, $c\geq 0$, $d\leq 0$ :
With $x=\frac{3\pi}{2}$, we have $g(x)=-b-c+d$ $\Rightarrow $ $S1=-2b-2c+2d< 0$.

3) Case : $b\geq 0$, $c\leq 0$, $d\geq 0$ :
With $x=\frac{11\pi}{8}$, we have $g(x)=-b\cos{\frac{pi}{4}}+c\sin{\frac{\pi}{8}}-d$ $\Rightarrow $ $S1=-b(3\sin{\frac{pi}{4}}-1)+c(3\sin{\frac{\pi}{8}}-1)-2d < 0$.

4) Case : $b\geq 0$, $c\leq 0$, $d\leq 0$ :
With $x=\frac{\pi}{2}$, we have $g(x)=-b+c+d$ $\Rightarrow $ $S1=-2b+2c+2d< 0$.

5) Case : $b\leq 0$, $c\geq 0$, $d\geq 0$ :
With $x=\frac{9\pi}{8}$, we have $g(x)=b\cos{\frac{pi}{4}}-c\cos{\frac{\pi}{8}}-d$ $\Rightarrow $ $S1=b(3\cos{\frac{pi}{4}}-1)-c(3\cos{\frac{\pi}{8}}-1)-2d < 0$.

6) Case : $b\leq 0$, $c\geq 0$, $d\leq 0$ :
With $x=0$ and $y=\frac{\pi}{4}$, we have $\frac{g(x)+g(y)}{2}=b\frac{1}{2}-c\frac{1}{2}\sin{\frac{\pi}{4}}+d$ $\Rightarrow $ $S2=b\frac{1}{2}-c(\frac{3}{2}\sin{\frac{\pi}{4}}-1)+2d < 0$.

7) Case : $b\leq 0$, $c\leq 0$, $d\geq 0$ :
With $x=\frac{\pi}{8}$, we have $g(x)=b\cos{\frac{pi}{4}}+c\cos{\frac{\pi}{8}}-d$ $\Rightarrow $ $S1=b(3\cos{\frac{pi}{4}}-1)+c(3\cos{\frac{\pi}{8}}-1)-2d < 0$.

8) Case : $b\leq 0$, $c\leq 0$, $d\leq 0$ :
With $x=0$ and $y=\frac{5\pi}{4}$, we have $\frac{g(x)+g(y)}{2}=b\frac{1}{2}+c\frac{1}{2}\sin{\frac{\pi}{4}}+d$ $\Rightarrow $ $S2=b\frac{1}{2}+c(\frac{3}{2}\sin{\frac{\pi}{4}}-1)+2d < 0$.


And this closes the demo (except in case of errors ... :(  )
I know this demo is rather long but since this topic have been posted more than 1 year ago, I think noone have found a simpler solution ... for the moment.

-- 
Patrick
\end{mysolution}



\begin{mysolution}[by \href{https://artofproblemsolving.com/community/user/38553}{Agr\_94\_Math}]
	Can someone post a simpler solution for this problem?
I am sure there exists some elegant and nice solution.

Great effort pco.
\end{mysolution}



\begin{mysolution}[by \href{https://artofproblemsolving.com/community/user/187896}{Ashutoshmaths}]
	thanks pco.
Any simpler solution?
\end{mysolution}



\begin{mysolution}[by \href{https://artofproblemsolving.com/community/user/172163}{joybangla}]
	Well this one is almost trivial.I don't understand its usage in IMOTC.Anyway let $g(x)=be^{2ix}-ice^{5ix}+de^{8ix}$ see that $g(x)+g\left(x+\frac{2\pi}{3}\right)+g\left(x+\frac{4\pi}{3}\right)=g(x)\left(1+e^{\frac{2\pi i}{3}}+e^{\frac{4\pi i}{3}}\right)=0$ now since $f(x)=a+\Re(g(x))$ we have $f(x)+f\left(x+\frac{2\pi}{3}\right)+f\left(x+\frac{4\pi}{3}\right)=3a....(*)$ now if $a<0$ then $s=t$ will be enough.But if $a>0$ then see that putting $x=t$ in $(*)$ we have $f\left(t+\frac{2\pi}{3}\right)+f\left(t+\frac{4\pi}{3}\right)=-a$ so one of $t+\frac{2\pi}{3},t+\frac{4\pi}{3}$ works.Finally let $a=0$ then in $(*)$ at least one term should be negative for some $x$ otherwise we get $f\equiv 0$ but that implies $a=b=c=d=0$ which is a contradiction.
\end{mysolution}
*******************************************************************************
-------------------------------------------------------------------------------

\begin{problem}[Posted by \href{https://artofproblemsolving.com/community/user/1598}{Arne}]
	Find all functions $f: \mathbb{N} \rightarrow \mathbb{Z}$ (where $\mathbb{N}$ is the set of positive integers) such that \[f(ab) + f\left(a^2 + b^2\right) = f(a) + f(b),\ \forall a, b \in \mathbb{N}\] and such that $f(a) \geq f(b)$ if $a | b$ ($\forall a, b \in \mathbb{N}$.)
	\flushright \href{https://artofproblemsolving.com/community/c6h53451}{(Link to AoPS)}
\end{problem}



\begin{mysolution}[by \href{https://artofproblemsolving.com/community/user/29428}{pco}]
	\begin{tcolorbox}Find all functions $ f: \mathbb{N} \rightarrow \mathbb{Z}$ (where $ \mathbb{N}$ is the set of positive integers) such that
\[ f(ab) + f\left(a^2 + b^2\right) = f(a) + f(b),\ \forall a, b \in \mathbb{N}
\]
and such that $ f(a) \geq f(b)$ if $ a | b$ ($ \forall a, b \in \mathbb{N}$.)\end{tcolorbox}

I have some difficulties to solve this one.

Some intermediate results :
Let $ M=f(1)$ the greatest value of $ f(x)$ and $ A_0=\{n$ such that $ f(n)=M\}$

$ 1)$ $ 1\in A_0$ (since $ 1$ divides any positive natural number)

$ 2)$ $ 2\in A_0$ (since $ f(1\times 1)+f(1+1)=f(1)+f(1)$ and so $ f(2)=f(1)$)

$ 3)$ $ p\in A_0$ for any prime $ p=1\pmod{4}$ : Consider that $ -1$ is then a quadratic residue mod $ p$ and so that there exists $ a$ such that $ a^2=-1+kp$
Then $ f(a)+f(a^2+1)=f(a)+f(1)$ and so $ f(a^2+1)=f(kp)=f(1)$ but $ f(p)\geq f(kp)$ since $ p|kp$ and so $ f(p)\geq f(1)$ and so $ f(p)=f(1)$

$ 4)$ $ a,b\in A_0$ implies $ ab\in A_0$ since $ f(ab)<f(a)=f(b)=M$ would imply $ f(a^2+b^2)>f(a)=f(b)=M$, which is impossible.

$ 5)$ $ a\in A_0$ implies $ d\in A_0$ $ \forall d|a$

$ 6)$ $ \gcd(a,b)=1$ implies $ a^2+b^2\in A_0$. It's easy to see : If prime $ p|a^2+b^2$, then $ a^2=-b^2\neq 0\pmod{p}$ and so $ -1$ is a quadratic residu mod p. So $ p=1\pmod{4}$ and $ a^2+b^2\in A_0$ (using points 1 and 4 above).

$ 7)$ Let $ P_0=\{$primes $ p=1\pmod{4}\}\cup\{2\}$. Then we have $ \{\prod_kp_k^{n_k},p_k\in P_0\}\subseteq A_0$.


Some examples of $ f(x)$ :

$ E1)$ $ f(x)=c$

$ E2)$ Let $ a<b$ and let prime $ p=3\pmod{4}$. 
$ f(x)=a$ if $ p|x$ and $ f(x)=b$ if not.

But I think these are not the complete set of solutions.
\end{mysolution}



\begin{mysolution}[by \href{https://artofproblemsolving.com/community/user/29428}{pco}]
	\begin{tcolorbox}Find all functions $ f: \mathbb{N} \rightarrow \mathbb{Z}$ (where $ \mathbb{N}$ is the set of positive integers) such that
\[ f(ab) + f\left(a^2 + b^2\right) = f(a) + f(b),\ \forall a, b \in \mathbb{N}
\]
and such that $ f(a) \geq f(b)$ if $ a | b$ ($ \forall a, b \in \mathbb{N}$.)\end{tcolorbox}

Here is a general solution :

If $ f(x)$ is a solution, $ f(x)+c$ is also a solution. So I'll study only solutions where $ f(1)=0$. Notice that since $ 1|n$, $ f(n)\leq 0$ $ \forall n$.

$ 1)$ Since $ f(1\times 1) + f(1 + 1) = f(1) + f(1)$, $ f(2) = f(1)=0$ and $ \boxed{f(2)=0}$ 

$ 2)$ Let $ n$ an integer such that $ -1$ is a quadratic residue mod $ n$. Then it exists $ a$ such that $ a^2 = - 1 + kn$.
Then $ f(a) + f(a^2 + 1) = f(a) + f(1)$ and so $ f(a^2 + 1) = f(kn) = f(1)=0$ but $ f(n)\geq f(kn)$ since $ n|kn$ and so $ f(n)\geq f(1)$ and so $ f(n) = f(1)=0$
So $ \boxed{u^2=-1\pmod{n}\implies f(n)=0}$ 

$ 3)$ A consequence of point 2 above is that $ f(p)=0$ for any prime $ p=1\pmod{4}$ since $ -1$ is always quadratic residue modulus such primes.

$ 4)$ If $ f(a)=f(b)=0$, then $ f(ab) < f(a) = f(b) = 0$ would imply $ f(a^2 + b^2) > 0$, which is impossible. So
$ \boxed{f(a)=f(b)=0\implies f(ab)=0}$

$ 5)$ Let $ a,b$ such that $ \gcd(a,b)=1$, then, let $ p$ a prime divisor of $ a^2+b^2$. We have $ a^2+b^2=0\pmod{p}$ So $ a^2=-b^2\pmod{p}$. We also have $ a^2\neq 0\pmod{p}$ else $ p$ would divide $ a$ and $ b$ but $ \gcd(a,b)=1$. So $ b$ have an inverse (mod p) and so $ (a/b)^2=-1\pmod{p}$ and so, according to point 2 above, $ f(p)=0$.
So $ a^2+b^2$ is a product of primes $ p_i$ such that $ f(p_i)=0$. And so, according to point 4 above, $ f(a^2+b^2)=0$. So, since  $ f(ab)+f(a^2+b^2)=f(a)+f(b)$, we can conclude that :
$ \boxed{\gcd(a,b)=1\implies f(ab)=f(a)+f(b)}$

$ 6)$ using $ a=bc$ in the original equation, we have $ f(b^2c)+f(b^2(c^2+1))=f(bc)+f(b)$. But $ f(b)\geq f(b^2(c^2+1))$ and $ f(bc)\geq f(b^2c)$.
Hence $ f(b^2c)=f(bc)$. So (taking $ c=1$), $ f(b^2)=f(b)$. Then (taking $ c=b$), $ f(b^3)=f(b^2)$. So, with an easy induction :
$ \boxed{f(b^k)=f(b)\: \: \forall k\geq 1}$

$ 7)$ Using points 5 and 6 above, we have $ f(\prod p_i^{n_i})=\sum f(p_i)$ where $ p_i$ are primes.


As a conclusion, we have :
$ f(1)=0$
$ f(2)=0$
$ f(p)=0$ for any prime $ p$ such that $ p=1\pmod{4}$
$ f(\prod p_i^{n_i})=\sum f(p_i)$ where $ p_i$ are primes.

We can now show that these necessary conditions are sufficient : 

Let $ f(x)$ defined as :
$ f(1)=0$
$ f(2)=0$
$ f(p)=0$ for any prime $ p$ such that $ p=1\pmod{4}$
$ f(p)=a_p\leq 0$ for any other prime p (where $ a_p$ is a nonpositive integer)
$ f(\prod p_i^{n_i})=\sum f(p_i)$ where $ p_i$ are primes.

We can show that $ f(x)$ has the required properties :

Obviously $ a|b$ implies $ f(a)\geq f(b)$
$ f(1\times 1)+f(1^2+1^2)=f(1)+f(1)=0$
$ f(a\times 1)+f(a^2+1)=f(a)+f(1)$ since all prime divisors p of $ a^2+1$ are such that $ p=1\pmod{4}$.
Let then $ a,b>1$ 
Let $ p_i$ prime divisors of $ a$ not dividing $ b$
Let $ q_i$ prime divisors of $ b$ not dividing $ a$
Let $ r_i$ prime divisors of $ a$ and $ b$
$ f(a)=\sum f(p_i)+\sum f(r_i)$
$ f(b)=\sum f(q_i)+\sum f(r_i)$
$ f(ab)=\sum f(p_i)+\sum f(q_i)+\sum f(r_i)$
$ f(a^2+b^2)=\sum f(r_i)+\sum f(s_i)$ where $ s_i$ are prime divisor of $ A=(\frac{a}{\gcd(a,b)})^2+(\frac{b}{\gcd(a,b)})^2$. But, as we demonstrate in point  5 above, all prime divisors of $ A$ are such that $ s_i=1\pmod{4}$ (since $ -1$ is quadratic residue mod $ s_i$) and so $ f(A)=0$
So $ f(a^2+b^2)=\sum f(r_i)$

And so $ f(ab)+f(a^2+b^2)=f(a)+f(b)$


And so we have the general solution :
Let $ M$ be any integer.
$ f(1)=M$
$ f(2)=M$
$ f(p)=M$ for any prime $ p$ such that $ p=1\pmod{4}$
$ f(p)=M + a_p$ for any other prime $p$, where $ a_p$ is any \begin{underlined}nonpositive integer\end{underlined}
$ f(\prod p_i^{n_i})=M+\sum (f(p_i)-M)$ where $ p_i$ are primes.
\end{mysolution}



\begin{mysolution}[by \href{https://artofproblemsolving.com/community/user/25722}{Yustas}]
	Very nice solution, long but beautiful.
\end{mysolution}
*******************************************************************************
-------------------------------------------------------------------------------

\begin{problem}[Posted by \href{https://artofproblemsolving.com/community/user/9954}{Rushil}]
	Find all functions $f : \mathbb{R} \to\mathbb{R}$ such that $f(x +y) = f(x) f(y) f(xy)$ for all $x, y \in \mathbb{R}.$
	\flushright \href{https://artofproblemsolving.com/community/c6h55433}{(Link to AoPS)}
\end{problem}



\begin{mysolution}[by \href{https://artofproblemsolving.com/community/user/12187}{t\u00b5t\u00b5}]
	First let's note that if $f(x) = 0$ for some $x$ then $f(x) = 0$ for all $x$ by translation.
So let's suppose $f(x) \neq 0 \forall x$
If $f$ is a solution  so is $-f$

$x=y=0 : f(0)^2 = 1$
If $f$ is a solution  so is $-f$ so we take for the moment $f(0) = 1$

Taking successively values for $x, y$: $1,-1;-2,1;-1,-1$ we get that $f(1) = f(-1) = f(-2) = f(2) = 2$
Then taking $x,-x$, $x-1,1$, we get $f(-x) = f(-x^2)$ or $f(x) = f(-x)$

Looking again $x,-x$ gives $f(x)^3= 1$ i.e $f(x) = 1$

Only solutions are $-1, 0, 1$

Mod edit: see pco's solution http://artofproblemsolving.com/community/c6h146995p832434
\end{mysolution}



\begin{mysolution}[by \href{https://artofproblemsolving.com/community/user/16759}{rem}]
	1. Take $x=y=2$:
$f(4)=f^2(2)f(4)$
$f^2(2)=1$
$f(2)=-1 or 1$.
2. Take $x=y=1$:
$f(2)=f^3(1)$
$f(1)=-1 or 1$, same sign as $f(2)$.
3. Take $x=0,y=1$:
$f(1)=f(1)f^2(0)$
$f^2(0)=1$, so $f(0)=-1 or 1$
4. Take $x=1,y=-1$:
$f(0)=f(1)f^2(-1)$.
$f^2(-1)=\frac{f(0)}{f(1)}$.  (*)
5. Take $x=y=-1$:
$f(-2)=f^2(-1)f(1)$, so $f(2)$ has same sign as $f(1)$.
Subst (*):
$f(-2)=f(0)$. So, $f(-2)=f(0)=-1 or 1$, have same sign as $f(1)$.
6. Take $x=-2,y=1$:
$f(-1)=f(-2)^2f(1)$
$f(-1)=f(1)=-1 or 1$.
7. Take $x=k,y=-1$ and $x=k-1,y=1$:
$f(k-1)=f^2(k)f(1)$, so $f(k-1)$ has same sign as $f(1)$
$f(k)=f^2(k-1)f(-1)$ so $f(k)$ has same sign as $f(1)$
Dividing, get: $f^3(k-1)=f^3(k)$ so $f(k)=f(k-1)$
Multiplying, get: $f(k-1)f(k)=1$. Hence $f(k)=-1 or 1$ and $f(k)$ has same sign as $f(1)$.
8. Note that if $f(0)=0$ ,then take $x=0,y=k$ to get $f(a)=0$.
So the solutions are:
$f(x)=0=const$
$f(x)=1=const$
$f(x)=-1=const$.
\end{mysolution}



\begin{mysolution}[by \href{https://artofproblemsolving.com/community/user/191127}{sayantanchakraborty}]
	\[f(x+y)=f(x)f(y)f(xy)\]         .....(1)
We note that if f(r)=0 for some real r then f(x)=0 for all reals x,
We thus work now fully on the assumption that f(r) not equals zero for any r. 
 Taking x=a+b and y=c in (1) we get
  \[f(a+b+c)
    =f(a+b)f(c)f(ac+bc)
    =f(a)f(b)f(ab)f(c)f(ac)f(bc)f(abc^2)\]
Thus we get 
\[f(a+b+c)=f(a)f(b)f(c)f(ab)f(bc)f(ca)f(abc^2)\]     .....(2)
Again taking x=a and y=b+c in (1) we get
\[f(a+b+c)
=f(a)f(b+c)f(ab+ac)
=f(a)f(b)f(c)f(bc)f(ab)f(ac)f(a^2bc)\]
Thus we get
\[f(a+b+c)=f(a)f(b)f(c)f(ab)f(bc)f(ca)f(a^2bc)\]    ......(3)
Comparing (2) and (3) we get
\[f(abc^2)=f(a^2bc)\]            ...(4)
If a,b,c are not equal to 0, then we put b=1/ac in (4) to get
\[f(c)=f(a)\]
Since a and c are arbitrary reals, it follows that f is a constant function, say 
\[f(x)=c\] for all reals x.
Then substitution in (1) leads to
\[c=c^3\]
giving values c=0,1 and -1
Thus we get three solutions:
\[ f(x)=0\]
\[f(x)=-1\]
\[f(x)=1\]
each extending over all reals x.
\end{mysolution}
*******************************************************************************
-------------------------------------------------------------------------------

\begin{problem}[Posted by \href{https://artofproblemsolving.com/community/user/10088}{silouan}]
	Prove that there is NOT existing function f:$\in R$ to $R$ such that $f(f(x))=-x$ for each $x\in R$
	\flushright \href{https://artofproblemsolving.com/community/c6h56637}{(Link to AoPS)}
\end{problem}



\begin{mysolution}[by \href{https://artofproblemsolving.com/community/user/5843}{Christian Hirsch}]
	Perhaps I'm missing something, but are you sure that we can't find such $f$  :? 
Just divide $\mathbb{R}\backslash 0$ in disjoint pairs; for each pair $(a|b)$ define:
$f(a)=b, f(b)=-a, f(-a)=-b, f(-b)=a$
\end{mysolution}



\begin{mysolution}[by \href{https://artofproblemsolving.com/community/user/10088}{silouan}]
	I think the problem is correct maybe yοu thought that any orbit was infinite but this is not true...
\end{mysolution}



\begin{mysolution}[by \href{https://artofproblemsolving.com/community/user/10846}{Tellah}]
	take :$f(x)=$
             $x+1$ if $x>0$ and $[x]$ is even
             $-x+1$ if $x>0$ and $[x]$ is odd 
             $x-1$ if $x<0$ and  $[x]$ is even
             $-x-1$ if $x<0$ and  $[x]$ is odd
i think $f(f(x))=-x$ $\forall x$ ;)
\end{mysolution}



\begin{mysolution}[by \href{https://artofproblemsolving.com/community/user/13}{enescu}]
	see [url=http://www.mathlinks.ro/Forum/viewtopic.php?t=49191]this[/url], you'll have a graph as well.
You probably missed something like "continuous" or "monotonic"...
Anyway, searching a bit before posting does not hurt....
\end{mysolution}
*******************************************************************************
-------------------------------------------------------------------------------

\begin{problem}[Posted by \href{https://artofproblemsolving.com/community/user/1991}{orl}]
	A function $ f$ defined on the positive integers (and taking positive integers values) is given by:

$ \begin{matrix} f(1) = 1, f(3) = 3 \\
f(2 \cdot n) = f(n) \\
f(4 \cdot n + 1) = 2 \cdot f(2 \cdot n + 1) - f(n) \\
f(4 \cdot n + 3) = 3 \cdot f(2 \cdot n + 1) - 2 \cdot f(n), \end{matrix}$

for all positive integers $ n.$ Determine with proof the number of positive integers $ \leq 1988$ for which $ f(n) = n.$
	\flushright \href{https://artofproblemsolving.com/community/c6h60400}{(Link to AoPS)}
\end{problem}



\begin{mysolution}[by \href{https://artofproblemsolving.com/community/user/29428}{pco}]
	\begin{tcolorbox}A function $ f$ defined on the positive integers (and taking positive integers values) is given by:

$ \begin{matrix} f(1) = 1, f(3) = 3 \\
f(2 \cdot n) = f(n) \\
f(4 \cdot n + 1) = 2 \cdot f(2 \cdot n + 1) - f(n) \\
f(4 \cdot n + 3) = 3 \cdot f(2 \cdot n + 1) - 2 \cdot f(n), \end{matrix}$

for all positive integers $ n.$ Determine with proof the number of positive integers $ \leq 1988$ for which $ f(n) = n.$\end{tcolorbox}

Considering that $ f(n)=f(2n)$, the two last equations give :
$ f(4n + 1)-f(4n) = 2(f(2n + 1) - f(2n))$
$ f(4n + 3)-f(4n+2)= 2(f(2n + 1) - f(2n))$

And so, if $ n$ is even and $ 2^{p+1}>n\geq 2^p>1$, we have $ f(n+1)-f(n)=2^p$

So if we have an even $ n=\sum_{i=1}^{k} 2^{a_i}$, where $ \{a_i\}$ is a strictly increasing sequence with $ a_1>0$ ($ n$ even) : $ f(n+1)=2^{a_k}+f(n)$
Then $ f(n)=f(\sum_{i=1}^{k} 2^{a_i})$ $ =f(\sum_{i=1}^{k} 2^{a_i-a_1})$ $ =2^{a_k-a_1}+f(\sum_{i=2}^{k} 2^{a_i-a_1})$
And so $ f((\sum_{i=1}^{k} 2^{a_i})+1)=2^{a_k}+2^{a_k-a_1}+f(\sum_{i=2}^{k} 2^{a_i-a_1})$

And it is easy to conclude that \[f\left(\sum_{i=1}^{k} 2^{a_i}\right)=\sum_{i=1}^{k} 2^{a_k-a_i}\]  And that applying $ f(n)$ means the reverse order of binary representation of $n$ (and this could be also easily shown with induction).

So $ f(n)=n$ occurs if and only if the binary representation of $n$ is symmetrical.

It remains to count these "symetric" numbers. We have exactly $ 2^{\lceil\frac{m-1}{2}\rceil}$ such numbers in $ [2^m,2^{m+1})$. So :
We have exactly 1 such numbers in $ [1,2)$
We have exactly 1 such numbers in $ [2,4)$
We have exactly 2 such numbers in $ [4,8)$
We have exactly 2 such numbers in $ [8,16)$
We have exactly 4 such numbers in $ [16,32)$
We have exactly 4 such numbers in $ [32,64)$
We have exactly 8 such numbers in $ [64,128)$
We have exactly 8 such numbers in $ [128,256)$
We have exactly 16 such numbers in $ [256,512)$
We have exactly 16 such numbers in $ [512,1024)$

Since $ 1988=B11111000100$, positions 2 to 6 may be any between $ 00000$ and $ 11101$, and so :
We have exactly 30 such numbers in $ [1024,1988]$

And so the requested number is $ 1+1+2+2+4+4+8+8+16+16+30=92$
\end{mysolution}



\begin{mysolution}[by \href{https://artofproblemsolving.com/community/user/354523}{muhammad-alhafi1}]
	my solution:
\end{mysolution}
*******************************************************************************
-------------------------------------------------------------------------------

\begin{problem}[Posted by \href{https://artofproblemsolving.com/community/user/18653}{maky}]
	Find all functions $f: (0,\infty)\rightarrow(0,\infty)$ with the following properties: $f(x+1)=f(x)+1$  and $f\left(\frac{1}{f(x)}\right)=\frac{1}{x}$.

\begin{italicized}Proposed by P. Volkmann\end{italicized}
	\flushright \href{https://artofproblemsolving.com/community/c6h102430}{(Link to AoPS)}
\end{problem}



\begin{mysolution}[by \href{https://artofproblemsolving.com/community/user/4420}{solyaris}]
	It can be shown that $f(x)=x$ is the only function from $(0,\infty)$ to  $(0,\infty)$ 
with the following properties :
(1) $f(x+1)=f(x)+1$
and
(2) $f\left(\frac{1}{f(x)}\right)=\frac{1}{x}$. 

Proof: It is clear that $f(x)=x$ works. 

Now suppose that $f$ is any function with the desired properties. 

By (1) $f$ is periodic and by (2) it follows that $f$ is injective and surjective, i.e. bijective. 
We conclude that $f( (n-1,n] ) = (n-1,n]$ for all natural $n$. Again by (2) 
we conclude $f(1)=1$, so $f(n)=n$ for natural $n$ and $f(n-1,n)=(n-1,n)$. 

Now we proceed by induction: 
Using (2) we can show that for every natural $k$ we have the following: 
$f( (n-1)/k,n/k ) = ((n-1)/k,n/k)$ and $f(n/k)=n/k$ for all natural $n$.

Now it is easy to show that $f(x) = x$ for all positive real $x$.  :)

I admit I have left out some details, but I suppose they are easy to fill in.  :D
\end{mysolution}



\begin{mysolution}[by \href{https://artofproblemsolving.com/community/user/16261}{Rust}]
	Let $g(x)=f(x)-x$. From (1), g(x) is periodic (but f is not periodic).
From (2) f(x) is bijective. It gives f((0,1))=(0,1) and bijective in (0,1).
f(1)=1 (f(1)>1 and f(1)<1 gives contradiction. From (1) $f(n)=n, f((n,n+1))=(n,n+1)$. From (2) $f((\frac{1}{n+1},\frac{1}{n}))=(\frac{1}{n+1},\frac{1}{n}),$ and $f((n+\frac{1}{k+1},n+\frac{1}{k}))=(n+\frac{1}{k+1},n+\frac{1}{k}),f(n+\frac{1}{k})=n+\frac{1}{k}$ and \[f((n_{1}+\frac{1}{n_{2}+\frac{1}{n_{3}}},n_{1}+\frac{1}{n_{2}+\frac{1}{n_{3}+1}}))=(n_{1}+\frac{1}{n_{2}+\frac{1}{n_{3}}},n_{1}+\frac{1}{n_{2}+\frac{1}{n_{3}+1}})\] e.t.c. It gives (by continiosly rations) $f(x)=x$.
\end{mysolution}



\begin{mysolution}[by \href{https://artofproblemsolving.com/community/user/7340}{AYMANE}]
	But $f$ is not contenious :!:
\end{mysolution}



\begin{mysolution}[by \href{https://artofproblemsolving.com/community/user/16261}{Rust}]
	\begin{tcolorbox}But $f$ is not contenious :!:\end{tcolorbox}
I don't say that f is continuous, this property proved that.
Let \[x(n_{0},n_{1},...,n_{k})=n_{0}+\frac{1}{n_{1}+\frac{1}{n_{2}+...+\frac{1}{n_{k}}}}.\] We have $f(x(n_{0},n_{1},...,n_{k}))=x(n_{0},n_{1},n_{2},...,n_{k}),n_{0}\ge 0,n_{i}\ge 1,i\ge 1$
Let $I(n_{0},n_{1},...,n_{k})=(x(n_{0},n_{1},...,n_{k},x(n_{0},n_{1},...,n_{k}+1))$ if k is even and $I(n_{0},n_{1},...,n_{k})=(x(n_{0},n_{1},...,n_{k}+1),x(n_{0},n_{1},...,n_{k}))$ if k is odd.
We have for any k
$if \ x\in I(n_{0},n_{1},n_{2},...,n_{k})$ then $f(x)\in I(n_{0},n_{1},n_{2},...n_{k})$. 
It give f(x)=x.
\end{mysolution}



\begin{mysolution}[by \href{https://artofproblemsolving.com/community/user/25434}{barasawala}]
	\begin{tcolorbox}
Now we proceed by induction: 
Using (2) we can show that for every natural $k$ we have the following: 
$f( (n-1)/k,n/k ) = ((n-1)/k,n/k)$ and $f(n/k)=n/k$ for all natural $n$.
\end{tcolorbox}

How to do it? I can't see.
\end{mysolution}



\begin{mysolution}[by \href{https://artofproblemsolving.com/community/user/25434}{barasawala}]
	Can anyone show me how to do the induction?
\end{mysolution}



\begin{mysolution}[by \href{https://artofproblemsolving.com/community/user/18361}{venatrix}]
	Hope this is correct, let's try

By induction on the first condition, it's easy to show that, for all $n\in\mathbb N$, $f(x+n)=f(x)+n$. The induction base is the first condition, now let's suppose that $f(x+n)=f(x)+n$ holds, consequently $f(x+n+1)=f((x+n)+1)=f(x+n)+1=f(x)+n+1$ as desired.

Now we will prove that $f(x)$ is bijective. Let $h(x)=\frac{1}{x}$. Then $\frac{1}{x}=f(h(f(x)))$, and then the conslusion follows.

Lets now define the following equality $\frac{f(x+1)}{f(x)}=\frac{1}{f(\frac{x}{x+1})}$ (*)
From the first condition we've got, defining $y=\frac{1}{f(x)}$, $f(y+1)=f(y)+1\Rightarrow f(\frac{1+f(x)}{f(x)})=f(\frac{f(x+1)}{f(x)})=f(\frac{1}{f(x)})+1=\frac{x+1}{x}$. From the second we've got $f(\frac{1}{f(\frac{x}{x+1})})=\frac{x+1}{x}$. Comparing the two equalities, and recalling that $f(x)$ is injective, we have the result.

Now, let's define $g(x)=f(x)-x$, so $g(x)=g(x+1)$, so $g(x)$ is $1-$periodic, as well as $n\in\mathbb N-$periodic

Let's suppose that $g(x)$ is non constant. Now the following holds $g(\frac{1}{x})=g(\frac{1}{x}+1)=g(\frac{x+1}{x})\Rightarrow g(f(\frac{1}{f(x)}))=g(f(\frac{1}{f(\frac{x}{x+1})}))$ (**)

Since we supposed that $g(x)$ is non constant, this implies that $f(\frac{1}{f(x)})=f(\frac{1}{f(\frac{x}{x+1})})+n$
Using (*) we get that $f(\frac{1}{f(x)})=f(\frac{f(x+1)}{f(x)})+n=f(\frac{f(x+1)+nf(x)}{f(x)})$, where the last equality of the chain is authorized by the induction on $n$ shown at the beginning.

Now, since $f(x)$ is bijective, we've got $\frac{1}{f(x)}=\frac{f(x+1)+nf(x)}{f(x)}\Rightarrow 1=f(x+1)+nf(x)$ (***)

Comparing (***) with the first condition, we've got $f(x+1)+nf(x)=f(x+1)-f(x)$, and so $f(x)\equiv 0$, absurd.

So $g(x)$ is constant, let's say $g(x)=k$, so $f(x)=x+k$. But even if this solutions fits the first condition, when plugged into the second, it leads to $f(\frac{1}{f(x)})=\frac{1}{x+k}+k=\frac{1}{x}$, so $k=0$ and finally we've got our solution, since $\boxed{f(x)=x}$
\end{mysolution}



\begin{mysolution}[by \href{https://artofproblemsolving.com/community/user/4420}{solyaris}]
	\begin{tcolorbox}Can anyone show me how to do the induction?\end{tcolorbox}

OK, I will give some more details. We will show that $f(x)=x$ is the only function from $(0,\infty)$ to  $(0,\infty)$  with 
(1) $f(x+1)=f(x)+1$  and (2) $f\left(\frac{1}{f(x)}\right)=\frac{1}{x}$. 

It is clear that $f(x)=x$ works.  On the other hand let $f$ have the above properties. 
Then by (2) it follows that $f$ is injective and surjective, i.e. bijective. 
By induction over $k$ we will show that the following holds: 
$f( (n-1)/k,n/k ) = ((n-1)/k,n/k)$ and $f(n/k)=n/k$ for all natural $n$.

Then we are done as this implies $f(x) = x$ for all rational $x$, and for irrational $x$
this implies $f(x) \in (a,b)$ for all rationals $a,b$ such that $x \in (a,b)$ and thus also $f(x) = x$. 
 
For the induction we first do the case $k=1$. 
By the bijectivity of $f$ the sets $A_{n}: = f(n,n+1] = f(0,1]+n$ (using (1)) have to be disjoint, and the union has to be $(0,\infty)$. Thus we have $A_{0}= (0,1]$. (Indeed for $x \notin A_{0}$ we have $x \in A_{n+1}$ for some $n \ge 0$, so $x-1 \in A_{n}$ by (1), so $x>1$. And for $x>1$ we have $x-1 \in A_{n}$ for some $n \ge 0$, so by (1) we have $x \in A_{n+1}$, so $x \notin A_{0}$.) Now it suffices to show $f(1)=1$. Then the rest of the case $k=1$ follows from (1). 
$f(1) \le 1$ follows from the above and $f(1)\ge 1$ follows from the following: Let $a = f(1)$, then by (2) we have $f(1/a) = 1$, and thus by the above $1/a \le 1$, so $a = f(1) \ge 1$. 

For the inductive step we assume the assertion is true for $1,\ldots,k-1$. By (1) it suffices to show that
$f( (n-1)/k,n/k ) = ((n-1)/k,n/k)$ for all $1\le n \le k$ and $f(n/k)=n/k$ for all natural $1 \le n < k$. 
The second assertion follows from 
$f(n/k) = f(1/(k/n))= f(1/f(k/n)) = 1/(k/n) = n/k,$
where the second step holds by the inductive hypothesis as $n<k$, and the third step holds by (2). Likewise the first assertion follows from 
$f((n-1)/k,n/k )= \{f(x): (n-1)/k< x< n/k\}$
$= \{ f(1/y): k/n < y < k/(n-1)\}= \{ f(1/f(y)): k/n < y < k/(n-1)\}$
$= \{ 1/y: k/n < y < k/(n-1)\}= ((n-1)/k,n/k),$
where in the first step we substituted $y = 1/x$, in the second step we have used  the inductive hypothesis as $n,n-1 < k$, so $f(k/n,(k+1)/n) = (k/n,(k+1)/n)$ and $f((k-1)/(n-1), k/(n-1)) = ((k-1)/(n-1), k/(n-1))$, which implies $f(k/n,k/(n-1)) = (k/n,k/(n-1))$, and in the third step we have used (2). 

I hope this answers you question. (And I am sorry for the delay.)
\end{mysolution}



\begin{mysolution}[by \href{https://artofproblemsolving.com/community/user/18728}{edriv}]
	\begin{tcolorbox}
Since we supposed that $g(x)$ is non constant, this implies that $f(\frac{1}{f(x)})=f(\frac{1}{f(\frac{x}{x+1})})+n$
\end{tcolorbox}

I think this passage is wrong..  :maybe:
\end{mysolution}



\begin{mysolution}[by \href{https://artofproblemsolving.com/community/user/334227}{reveryu}]
	\begin{tcolorbox}By (1) $f$ is periodic\end{tcolorbox}
\begin{tcolorbox} From (1) g(x) is periodic (but f is not periodic).\end{tcolorbox}
 :what?: 
\end{mysolution}



\begin{mysolution}[by \href{https://artofproblemsolving.com/community/user/399513}{polyethylene}]
	I still do not understand why $f((0,1))=(0,1)$.
Can anyone gives an explanation please
\end{mysolution}



\begin{mysolution}[by \href{https://artofproblemsolving.com/community/user/243405}{ThE-dArK-lOrD}]
	For each subset $S\subseteq \mathbb{R}^+$, denote $f(S)=\{ f(x)\mid x\in S\}$. We say that a set $S$ is \begin{italicized}good \end{italicized} if $f(S)=S$.
It's obvious that $f$ is bijective.
The proof is divided into three main steps, I'll show only the detailed proof of the first one:

1. $(0,1]$ is good.
Proof: From the first condition, we can easily deduce that $f(x)\geq \lfloor x\rfloor$ for all $x\in \mathbb{R}^+$.
For each $r\in (0,1]$, there exists $s=\frac{1}{f(\frac{1}{r})}$ such that $f(s)=r$. We've $\frac{1}{r}>1\implies f(\frac{1}{r})\geq 1\implies s\leq 1$.
So, for each $r\in (0,1]$, there exists $s\in (0,1]$ that $f(s)=r$.
Suppose there exists $s'\in (0,1]$ that $f(s')>1$.
We've proved that there must exist $r'\in (0,1]$ that $f(r')=f(s')-\lceil f(s')\rceil +1$.
By the first condition, we get $f(r'+\lceil f(s')\rceil -1)=f(s')$. Injectivity implies $r'+\lceil f(s')\rceil -1 =s'\in (0,1]$, clearly impossible.
So, for each $s\in (0,1]$, we get $f(s)\in (0,1]$, this completes the first part.

From now on, we'll use the (infinite) continued fraction representation of positive real numbers.
2. For any positive integers $n$, the interval set
$$\Big([a_0,a_1,a_2,...,a_{n-1},a_n],[a_0,a_1,a_2,...,a_{n-1},a_n+1]\Big]$$ is good for all non-negative integers $a_0$ and positive integers $a_1,a_2,...,a_n$.
Proof: Induction on $n$.

3. For all positive real numbers $r$ and $\epsilon$, there exists $a,b \in \mathbb{R}^+$ that $a<r<b$ and $|a-r|,|b-r|<\epsilon$ and the set $(a,b)$ is good.

After this, if there exists $r\in \mathbb{R}^+$ that $f(r)\neq r$, there exists $\epsilon \in \mathbb{R}^+$ such that for all $a,b\in \mathbb{R}^+$ that $a<r<b$ and $|a-r|,|b-r|<\epsilon$,
$$r\in (a,b],f(r)\not\in (a,b].$$
The third result gives us the contradiction.
\end{mysolution}
*******************************************************************************
-------------------------------------------------------------------------------

\begin{problem}[Posted by \href{https://artofproblemsolving.com/community/user/5820}{N.T.TUAN}]
	Find all pairs of positive real numbers $(a, b)$ such that for every $n \in\mathbb{N}$ and every real root $x_{n}$ of the equation $4n^{2}x = \log_{2}(2n^{2}x+1)$ we have $a^{x_{n}}+b^{x_{n}}\geq 2+3x_{n}.$
	\flushright \href{https://artofproblemsolving.com/community/c6h140931}{(Link to AoPS)}
\end{problem}



\begin{mysolution}[by \href{https://artofproblemsolving.com/community/user/20481}{kyoshiro\_hp}]
	In the first equation we obtain $x_{n}=0$ or $x_{n}=\frac{-1}{4n^{2}}$
so we can get $ab \leq e^{3}$
\end{mysolution}



\begin{mysolution}[by \href{https://artofproblemsolving.com/community/user/5820}{N.T.TUAN}]
	Can you post your solution? Concrete!
\end{mysolution}



\begin{mysolution}[by \href{https://artofproblemsolving.com/community/user/29428}{pco}]
	Hello N.T.TUAN
\begin{tcolorbox}Find all pairs of positive real numbers $(a, b)$ such that for every $n \in\mathbb{N}$ and every real root $x_{n}$ of the equation $4n^{2}x = \log_{2}(2n^{2}x+1)$ we have $a^{x_{n}}+b^{x_{n}}\geq 2+3x_{n}.$\end{tcolorbox}

For $n=0$, every real x is root of the equation, and so, no matter what are the $x_{i}$ for $i>0$, the requirement is

What are the pairs of positive real numbers $(a, b)$ such that for every $x \in \mathbb{R}$ $a^{x}+b^{x}\geq 2+3x.$

Let$f(x)=a^{x}+b^{x}$ and $g(x)=3x+2$
$f(x)$ is convex and  $f(0) = g(0)$
So the requirement is $f'(0)=g'(0)$ $\Rightarrow $ $ln(a)+ln(b)=3$

Hence $ab=e^{3}$

-- 
Patrick
\end{mysolution}



\begin{mysolution}[by \href{https://artofproblemsolving.com/community/user/5820}{N.T.TUAN}]
	But here $\mathbb{N}$ is set of all positive integer numbers.
\end{mysolution}



\begin{mysolution}[by \href{https://artofproblemsolving.com/community/user/29428}{pco}]
	\begin{tcolorbox}But here $\mathbb{N}$ is set of all positive integer numbers.\end{tcolorbox}

You're right. I always make the same error. Here is the correction.

We have $f(x)=a^{x}+b^{x}$ and $g(x)=3x+2$ and we want $f(x_{n})\geq g(x_{n})$
For $x_{n}=0$, the requirement is fulfilled.

So we want only $f(-\frac{1}{4n^{2}})\geq g(-\frac{1}{4n^{2}})$ $\forall n\in \mathbb{N}$

Since $f(0)=g(0)$ and $f(x)$ is convex, it's immediate to see that :
If $f(-\frac{1}{4n1^{2}})\geq g(-\frac{1}{4n1^{2}})$ and $n2 < n1$ then $f(-\frac{1}{4n2^{2}})\geq g(-\frac{1}{4n2^{2}})$

So the requirement is for the limit (when $n \rightarrow+\infty$) and so is $f'(0)\leq g'(0)$, so $ln(a)+ln(b) \leq 3$, so $ab \leq e^{3}$ as kyoshiro\_hp said (shortly  :)  ).

-- 
Patrick
\end{mysolution}



\begin{mysolution}[by \href{https://artofproblemsolving.com/community/user/5820}{N.T.TUAN}]
	Sorry, I don't understand here.
\begin{tcolorbox}

Since $f(0)=g(0)$ and $f(x)$ is convex, it's immediate to see that :
If $f(-\frac{1}{4n1^{2}})\geq g(-\frac{1}{4n1^{2}})$ and $n2 < n1$ then $f(-\frac{1}{4n2^{2}})\geq g(-\frac{1}{4n2^{2}})$
\end{tcolorbox}
But I can have $ab\leq e^{3}$, by $f'(0)\leq g'(0)$. ( Not using above text) . In fact, It is from
\[\frac{f(\frac{-1}{4n^{2}})-f(0)}{\frac{-1}{4n^{2}}-0}\leq \frac{g(\frac{-1}{4n^{2}})-g(0)}{\frac{-1}{4n^{2}}-0}, \]
now let $n\to\infty$. Maybe I know little on convex functions.
Finally, if $(a,b)$ is answer then $ab\leq e^{3}$, but if $ab\leq e^{3}$ then is it answer?  :maybe:
\end{mysolution}



\begin{mysolution}[by \href{https://artofproblemsolving.com/community/user/29428}{pco}]
	\begin{tcolorbox}Final, if $(a,b)$ is answer then $ab\leq e^{3}$, but if $ab\leq e^{3}$ then is it answer?  :maybe:\end{tcolorbox}

Ok, I'll try to be more precise :

$f(x)=a^{x}+b^{x}$  with $a>0$ and $b>0$
$g(x)=3x+2$

Let $x_{n}=\frac{-1}{4n^{2}}$

We have $f(x_{n})=f(0)+f'(0)x_{n}+f''(h_{n})\frac{x_{n}^{2}}{2}$ for some $h_{n}\in [x_{n},0]$ and $g(x_{n})=g(0)+x_{n}g'(0)$
$f(x_{n})\geq g(x_{n})$ $\Leftrightarrow$ $f(0)+f'(0)x_{n}+f''(h_{n})\frac{x_{n}^{2}}{2}\geq g(0)+x_{n}g'(0)$ and, since $f(0)=g(0)$ and $x_{n}<0$:
$f(x_{n})\geq g(x_{n})$ $\Leftrightarrow$ $f'(0)+f''(h_{n})\frac{x_{n}}{2}\leq g'(0)$ and :
$f(x_{n})\geq g(x_{n})$ $\Leftrightarrow$ $f'(0)-g'(0)\leq f''(h_{n})\frac{-x_{n}}{2}$

Then, and since $f''(h_{n})>0$ (f is convex) :
1) $f(x_{n})\geq g(x_{n})$ $\Rightarrow $ $f'(0)-g'(0)\leq f''(h_{n})\frac{-x_{n}}{2}$ $\forall n\in\mathbb{N}$ $\Rightarrow $ $f'(0)-g'(0)\leq 0$ and  $f'(0)-g'(0)\leq 0$ is a necessary condition
2)  $f'(0)-g'(0)\leq 0$ $\Rightarrow $ $f'(0)-g'(0)\leq 0\leq f''(h_{n})\frac{-x_{n}}{2}$ $\forall n\in\mathbb{N}$ $\Rightarrow $ $f(x_{n})\geq g(x_{n})$ and  $f'(0)-g'(0)\leq 0$ is a sufficient condition

So $ab\leq e^{3}$ is a necessary condition AND a sufficient condition.
I hope this would be clear enough (even with my poor english language)  :) 

-- 
Patrick
\end{mysolution}



\begin{mysolution}[by \href{https://artofproblemsolving.com/community/user/5820}{N.T.TUAN}]
	Thank you very much! But this
\begin{tcolorbox} $f'(0)-g'(0)\leq f''(h_{n})\frac{-x_{n}}{2}$ $\forall n\in\mathbb{N}$ $\Rightarrow$ $f'(0)-g'(0)\leq 0$ \end{tcolorbox}
Why?
\end{mysolution}



\begin{mysolution}[by \href{https://artofproblemsolving.com/community/user/29428}{pco}]
	\begin{tcolorbox}Thank you very much! But this
\begin{tcolorbox} $f'(0)-g'(0)\leq f''(h_{n})\frac{-x_{n}}{2}$ $\forall n\in\mathbb{N}$ $\Rightarrow$ $f'(0)-g'(0)\leq 0$ \end{tcolorbox}
Why?\end{tcolorbox}

You just have to do $n\rightarrow\infty$ in the inequality. Then $x_{n}\rightarrow 0$, $h_{n}\rightarrow 0$ and $f''(h_{n})\frac{-x_{n}}{2}\rightarrow 0$.

-- 
Patrick
\end{mysolution}



\begin{mysolution}[by \href{https://artofproblemsolving.com/community/user/5820}{N.T.TUAN}]
	Ok!
\begin{tcolorbox}Find all pairs of positive real numbers $(a, b)$ such that for every $n \in\mathbb{N}$ and every real root $x_{n}$ of the equation $4n^{2}x = \log_{2}(2n^{2}x+1)$ we have $a^{x_{n}}+b^{x_{n}}\geq 2+3x_{n}.$\end{tcolorbox}
I will post solution i know.
First, we need solve the equation $4n^{2}x = \log_{2}(2n^{2}x+1)$.
Condition $x>\frac{-1}{2n^{2}}$. This equation is equivalent to $2^{4n^{2}x}=2n^{2}x+1$. Put $t=4n^{2}x+1$ then $2^{t}=t+1$, this equation is equivalent to $t\in\{0,1\}$ because $f(0)=f(1)=0$ and $f''(t)>0\forall t$ , here $f(t)=2^{t}-t-1$. Finally  $4n^{2}x = \log_{2}(2n^{2}x+1)$ has roots $0$ and $\frac{-1}{4n^{2}}$.

If $(a,b)$ satisfy then $a^{\frac{-1}{4n^{2}}}+b^{\frac{-1}{4n^{2}}}\geq 2+3(\frac{-1}{4n^{2}})\forall n$ or $\frac{1}{2}(a^{\frac{-1}{4n^{2}}}+b^{\frac{-1}{4n^{2}}})\geq 1+\frac{3}{2}(\frac{-1}{4n^{2}})\forall n$ or $(\frac{a^{\frac{-1}{4n^{2}}}+b^{\frac{-1}{4n^{2}}}}{2})^{\frac{-1}{4n^{2}}}\geq (2+3(\frac{-1}{4n^{2}}))^{\frac{-1}{4n^{2}}}\forall n$, here let $n\to \infty$ we have $\sqrt{ab}\leq e^{\frac{3}{2}}$.

If $(a,b)$ satisfies $\sqrt{ab}\leq e^{\frac{3}{2}}$ then $\frac{1}{2}(a^{\frac{-1}{4n^{2}}}+b^{\frac{-1}{4n^{2}}})\geq (\sqrt{ab})^{\frac{-1}{4n^{2}}}\geq e^{\frac{3}{2}\cdot\frac{-1}{4n^{2}}}\geq 1+\frac{3}{2}(\frac{-1}{4n^{2}})\forall n$ and $a^{0}+b^{0}=2=2+3.0$.

Answer $\sqrt{ab}\leq e^{\frac{3}{2}}$.
\end{mysolution}



\begin{mysolution}[by \href{https://artofproblemsolving.com/community/user/29428}{pco}]
	Hello !
Nice demo.
Just a little error :


\begin{tcolorbox}
If $(a,b)$ satisfy then $a^{\frac{-1}{4n^{2}}}+b^{\frac{-1}{4n^{2}}}\geq 2+3(\frac{-1}{4n^{2}})\forall n$ or $\frac{1}{2}(a^{\frac{-1}{4n^{2}}}+b^{\frac{-1}{4n^{2}}})\geq 1+\frac{3}{2}(\frac{-1}{4n^{2}})\forall n$ or $(\frac{a^{\frac{-1}{4n^{2}}}+b^{\frac{-1}{4n^{2}}}}{2})^{\frac{-1}{4n^{2}}}\geq (2+3(\frac{-1}{4n^{2}}))^{\frac{-1}{4n^{2}}}\forall n$, here let $n\to \infty$ we have $\sqrt{ab}\leq e^{\frac{3}{2}}$\end{tcolorbox}

I think $\frac{1}{2}(a^{\frac{-1}{4n^{2}}}+b^{\frac{-1}{4n^{2}}})\geq 1+\frac{3}{2}(\frac{-1}{4n^{2}})\forall n$ implies $(\frac{a^{\frac{-1}{4n^{2}}}+b^{\frac{-1}{4n^{2}}}}{2})^{\frac{-1}{4n^{2}}}\leq (1+\frac{3}{2}(\frac{-1}{4n^{2}}))^{\frac{-1}{4n^{2}}}\forall n$ (and not "$\geq$")

Then, when $n\rightarrow+\infty$, LHS and RHS $\rightarrow 1$ and I don't see how you conclude  $\sqrt{ab}\leq e^{\frac{3}{2}}$
The mistake is in the exponent : You need to use $(-4n^{2})$ and not $\frac{-1}{4n^{2}}$. Then :


If $(a,b)$ satisfy then $a^{\frac{-1}{4n^{2}}}+b^{\frac{-1}{4n^{2}}}\geq 2+3(\frac{-1}{4n^{2}})\forall n$ or $\frac{1}{2}(a^{\frac{-1}{4n^{2}}}+b^{\frac{-1}{4n^{2}}})\geq 1+\frac{3}{2}(\frac{-1}{4n^{2}})\forall n$ or $(\frac{a^{\frac{-1}{4n^{2}}}+b^{\frac{-1}{4n^{2}}}}{2})^{-4n^{2}}\leq (1+\frac{3}{2}(\frac{-1}{4n^{2}}))^{-4n^{2}}\forall n$, here let $n\to \infty$ we have $\sqrt{ab}\leq e^{\frac{3}{2}}$

I think your demo is quicker than mine.

-- 
Patrick
\end{mysolution}



\begin{mysolution}[by \href{https://artofproblemsolving.com/community/user/5820}{N.T.TUAN}]
	Sorry  :blush: Thanks! Now this topic has got the two solutions  
\end{mysolution}
*******************************************************************************
-------------------------------------------------------------------------------

\begin{problem}[Posted by \href{https://artofproblemsolving.com/community/user/21742}{anuj kumar}]
	Find all functions $f: \mathbb N \to \mathbb N$ such that
\[f(m^{2}+f(n))=f(m)^{2}+n,\]
for all $m, n \in \mathbb N$.
	\flushright \href{https://artofproblemsolving.com/community/c6h145161}{(Link to AoPS)}
\end{problem}



\begin{mysolution}[by \href{https://artofproblemsolving.com/community/user/29428}{pco}]
	Hello anuj kumar!
\begin{tcolorbox}find all f:N->N such that
\[f(m^{2}+f(n))=f(m)^{2}+n.\ m,n belong to N \]
\end{tcolorbox}

Let the proposal P(m,n) be : $f(m^{2}+f(n))=f(m)^{2}+n$

1) $f(x)$ is injective
$f(a) = f(b)$ ==> $f(m^{2}+f(a))=f(m^{2}+f(b))$ $\Rightarrow$ $f(m)^{2}+a=f(m)^{2}+b$ $\Rightarrow$ $a=b$
Q.E.D.

2) f(0) = 0
Let $f(0)=a > 0$ Then $P(0,n) \Rightarrow f(f(n)) = n+a^{2}$ $\Rightarrow$ $f(n+a^{2}) = f(n)+a^{2}$ $\Rightarrow$ $f(n)=n+f(mod(n,a^{2}))-mod(n,a^{2})) = n+d(n)$ with $d(n)$ bounded
Then $f(m^{2}+f(n))=f(m^{2}+n+d(n))=m^{2}+n+d(n)+d(m^{2}+n+d(n))$ and $f(m)^{2}+n=(m+d(m))^{2}+n=m^{2}+n+2md(m)+d(m)^{2}$
And $d(n)+d(m^{2}+n+d(n))=2md(m)+d(m)^{2}$
But L.H.S is bounded and R.H.S is not (except if $d(m)=0$ $\forall m$, which implies $f(n)=n$ which is in contradiction with $f(0)=a > 0$ 
so $f(0)=0$
Q.E.D.

3)f(n) is bijective
$P(0,n) \Rightarrow f(f(n))=n$ $\Rightarrow$ f is surjective $\Rightarrow$ f is bijective (with point 1))
Q.E.D.

4) $f(n)=n$ is the only solution
$P(n,0) \Rightarrow f(n^{2})=(f(n))^{2}$ .
Since f is bijective, for a given n I can find an integer p > 0 such that $f(p)=2n+1$
Then $P(n,p) \Rightarrow f(n^{2}+2n+1)=(f(n))^{2}+p$ $\Rightarrow$ $f((n+1)^{2})=(f(n))^{2}+p$ $\Rightarrow$ $(f(n+1))^{2}=(f(n))^{2}+p$ $\Rightarrow$ $f(n+1) > f(n)$
So f is a strictly increasing bijection from N in N ==> f(n)=n
Q.E.D.

I'm afraid my demonstration is quite long and I think there exists some simpler one.

-- 
Patrick
\end{mysolution}



\begin{mysolution}[by \href{https://artofproblemsolving.com/community/user/8638}{me@home}]
	Sorry, I'm not sure what you mean by $d(n)$ is bounded? Do you mean linear?
\end{mysolution}



\begin{mysolution}[by \href{https://artofproblemsolving.com/community/user/29428}{pco}]
	\begin{tcolorbox}Sorry, I'm not sure what you mean by $d(n)$ is bounded? Do you mean linear?\end{tcolorbox}

No I mean it exist reals A and B such that $A < d(n) < B$ $\forall n$

And this is easy to see since $d(n)=f(mod(n,a^{2}))-mod(n,a^{2})$ : d(n) can only take at most $a^{2}$ different values ==> a finite set of values is always "bounded" (I dont know if it is the good English word : French one is "borné")

-- 
Patrick
\end{mysolution}



\begin{mysolution}[by \href{https://artofproblemsolving.com/community/user/20099}{pardesi}]
	\begin{tcolorbox}find all f:N->N such that
\[f(m^{2}+f(n))=f(m)^{2}+n.\ m,n belong to N \]
\end{tcolorbox}
\begin{tcolorbox}f(0) = 0 
\end{tcolorbox}
Since the function is only defined for natural no.s how can you prove that $f(0)=0$ 

OK pco i read your other post   so it's the problem with your country but you should seriously like to revisit the proof.
\end{mysolution}



\begin{mysolution}[by \href{https://artofproblemsolving.com/community/user/25459}{lovejrz}]
	what's QED mean?
\end{mysolution}



\begin{mysolution}[by \href{https://artofproblemsolving.com/community/user/20099}{pardesi}]
	\begin{tcolorbox}what's QED mean?\end{tcolorbox}

http://en.wikipedia.org/wiki/Q.E.D. 
\end{mysolution}



\begin{mysolution}[by \href{https://artofproblemsolving.com/community/user/29428}{pco}]
	\begin{tcolorbox}ok pco i read your other post   so it's the problem with your country but you should seriously like to revisit the proof.\end{tcolorbox}

OK, let's go :

1) f is injective :
$f(a)=f(b)$ $\Rightarrow $ $f(m+f(a))=f(m+f(b))$ $\Rightarrow $ $f(m)^{2}+a=f(m)^{2}+b$ $\Rightarrow $ $a=b$ Q.E.D.

2) f is strictly increasing :
If $f(a)>f(b)$, then $f(a)^{2}+n=f(b)^{2}+(n+f(a)^{2}-f(b)^{2})$ and then $f(a^{2}+f(n))=f(b^{2}+f(n+f(a)^{2}-f(b)^{2}))$ and, since f is injective :
$a^{2}+f(n)=b^{2}+f(n+f(a)^{2}-f(b)^{2})$ which implies $f(n+f(a)^{2}-f(b)^{2})=f(n)+(a^{2}-b^{2})$
$\Rightarrow $  $f(n+k(f(a)^{2}-f(b)^{2}))=f(n)+k(a^{2}-b^{2})$ $\Rightarrow $ $a>b$.
$f(a)>f(b)$ $\Rightarrow $ $a>b$ shows that f is monotonously increasing. Hence, since f is also injective, f is strictly increasing. QED

3) $f(n)=n$
Since f is strictly increasing, $f(a)=f(b)+1$ $\Rightarrow $ $a=b+1$
But we have $f(m^{2}+f(n+1))=f(m)^{2}+n+1 = f(m^{2}+f(n))+1$ and so $m^{2}+f(n+1)=m^{2}+f(n)+1$ and $f(n+1)=f(n)+1$
So $f(n)=n+f(1)-1$
When entering this expression in the functional equation, we conclude that $f(1)=1$ and $f(n)=n$
Q.E.D.


-- 
Patrick
\end{mysolution}



\begin{mysolution}[by \href{https://artofproblemsolving.com/community/user/20099}{pardesi}]
	:coolspeak: though i haven't gone through the proof (in no mood to do so :sleep2: )
\end{mysolution}


*******************************************************************************
-------------------------------------------------------------------------------

\begin{problem}[Posted by \href{https://artofproblemsolving.com/community/user/13427}{spix}]
	Find all functions $f: \mathbb R\to \mathbb R$ such that
\[\left | \sum_{k=1}^{n}2^{k}(f(x+ky)-f(x-ky)) \right |\leq 1 ,\]
 for all integers $n \geq 0$ and all $x,y \in \mathbb R$.
	\flushright \href{https://artofproblemsolving.com/community/c6h145321}{(Link to AoPS)}
\end{problem}



\begin{mysolution}[by \href{https://artofproblemsolving.com/community/user/29428}{pco}]
	Hello spix !
\begin{tcolorbox}Find all $f: R\to R$ such that:
 $\left | \sum_{k=1}^{n}2^{k}(f(x+ky)-f(x-ky)) \right |\leq 1 , \forall n \in N^{\star}, \forall x,y \in R$\end{tcolorbox}

Let $A= \sum_{k=1}^{n}2^{k}(f(x+ky)-f(x-ky))$

We have $\left | A \right |\leq 1$

Let $B= \sum_{k=1}^{n+1}2^{k}(f(x+ky)-f(x-ky)) = A+2^{n+1}(f(x+(n+1)y)-f(x-(n+1)y))$

We have $\left | B \right |\leq 1$, so $\left | A+2^{n+1}(f(x+(n+1)y)-f(x-(n+1)y)) \right |\leq 1$

But : $\left | A+2^{n+1}(f(x+(n+1)y)-f(x-(n+1)y)) \right |$ $\geq$ $\left | 2^{n+1}(f(x+(n+1)y)-f(x-(n+1)y)) \right |-\left | A\right |$

So : $1 \geq$ $\left | 2^{n+1}(f(x+(n+1)y)-f(x-(n+1)y)) \right |-\left | A\right |$

So : $1+\left | A\right |\geq$ $\left | 2^{n+1}(f(x+(n+1)y)-f(x-(n+1)y)) \right |$

So : $\left | 2^{n+1}(f(x+(n+1)y)-f(x-(n+1)y)) \right | \leq 2$

And : $\left | f(x+(n+1)y)-f(x-(n+1)y) \right | \leq 2^{-n}, \forall n \in N^{\star}, \forall x,y \in R$

Let a and b two reals.

Let $x=\frac{a+b}{2}$ and $y=\frac{b-a}{2(n+1)}$ 

Then $\left | f(x+(n+1)y)-f(x-(n+1)y) \right | \leq 2^{-n}$ becomes $\left | f(b)-f(a) \right | \leq 2^{-n}, \forall n \in N^{\star}, \forall a,b \in R$

And f is any constant function.

-- 
Patrick
\end{mysolution}
*******************************************************************************
-------------------------------------------------------------------------------

\begin{problem}[Posted by \href{https://artofproblemsolving.com/community/user/18812}{pohoatza}]
	Let $a$ be a real number and let $f : \mathbb{R}\rightarrow \mathbb{R}$ be a function satisfying $f(0)=\frac{1}{2}$ and 
\[f(x+y)=f(x)f(a-y)+f(y)f(a-x), \quad \forall x,y \in \mathbb{R}.\]
Prove that $f$ is constant.
	\flushright \href{https://artofproblemsolving.com/community/c6h145359}{(Link to AoPS)}
\end{problem}



\begin{mysolution}[by \href{https://artofproblemsolving.com/community/user/4229}{scorpius119}]
	
First substitute $x=y=0$ to get $f(a)=\frac{1}{2}$. Then substituting $y=0$, we get
\[f(x)=\frac{f(x)}{2}+\frac{f(a-x)}{2}\Rightarrow f(x)=f(a-x)\]
This turns the original functional equation into $f(x+y)=2f(x)f(y)$. In particular, $x=y$ gives
\[f(x)=2f(x)^{2}\Rightarrow f(x)(2f(x)-1)=0\]
We must show $f(x)=\frac{1}{2}$ for all $x$, so suppose otherwise: that there is some $b$ such that $f(b)=0$. If this were to happen, substituting $x=-b,y=b$ into $f(x+y)=2f(x)f(y)$ gives $\frac{1}{2}=0$ which is BAD! Therefore $f$ is the constant 1/2.

\end{mysolution}



\begin{mysolution}[by \href{https://artofproblemsolving.com/community/user/9797}{Jan}]
	I could be wrong, but I think you made a mistake scorpius:

\begin{tcolorbox}
This turns the original functional equation into $f(x+y)=2f(x)f(y)$. In particular, $x=y$ gives
\[f(x)=2f(x)^{2}\Rightarrow f(x)(2f(x)-1)=0 \]
\end{tcolorbox}

You only know that $f(2x)=2f(x)^{2}$
\end{mysolution}



\begin{mysolution}[by \href{https://artofproblemsolving.com/community/user/18812}{pohoatza}]
	\begin{tcolorbox}
First substitute $x=y=0$ to get $f(a)=\frac{1}{2}$. Then substituting $y=0$, we get
\[f(x)=\frac{f(x)}{2}+\frac{f(a-x)}{2}\Rightarrow f(x)=f(a-x) \]
\end{tcolorbox}

Continuing from here, just take $y = a-x$, therefore $f(0) = f^{2}(x)+f^{2}(a-x)$, so $f(x) = \frac{1}{2}$ or $-\frac{1}{2}$.
But now we have $f(\frac{x}{2}) = \frac{1}{2}$ or $-\frac{1}{2}$ and $f(\frac{a-x}{2}) = f(\frac{x}{2})$. Hence $f(x) = f(\frac{x}{2 }+\frac{x}{2}) = 2 f(x) f(a-\frac{x}{2}) =\frac{1}{2}$.
\end{mysolution}



\begin{mysolution}[by \href{https://artofproblemsolving.com/community/user/8638}{me@home}]
	Okay, first I see an easy way why $f(a)=f^{2}(x)+f^{2}(a-x)$ follows, which then shows that $f(0)=\frac{a}{2}=f(a)=...$ so why did you write $f(0)$?... I guess that doesn't really matter
Next, what does $f\left(\frac{a-x}{2}\right) = f\left(\frac{x}{2}\right)$ follows from?
Because \[\frac12=f\left(2\cdot\frac{a}{2}\right)=2f\left(\frac{a}{2}\right)^{2}\]
Also $f(x)=f(y)\implies 2f(x)^{2}=2f(y)^{2}\implies f(2x)=f(2y)$ but from the other direction, I think it can only show that $f(x)=f(y)\implies f\left(\frac{x}{2}\right)=\pm f\left(\frac{y}{2}\right)$... can you please explain your logic behind the steps because it seems like there are a couple mistakes or just sloppiness
\end{mysolution}



\begin{mysolution}[by \href{https://artofproblemsolving.com/community/user/29428}{pco}]
	Hello you at home !
\begin{tcolorbox}... can you please explain your logic behind the steps because it seems like there are a couple mistakes or just sloppiness\end{tcolorbox}

Let $P(x,y)$ be the proposal $f(x+y)=f(x)f(a-y)+f(y)f(a-x)$ \\
$P(0,a)$ gives $f(a)=\frac{1}{2}$ \\
$P(x,0)$ gives then $f(a-x)=f(x)$ \\
Replacing then $f(a-x)$ by $f(x)$ and $f(a-y)$ by $f(y)$ in $P(x,y)$, we have $Q(x,y)$ : $f(x+y)=2f(x)f(y)$
$Q(\frac{x}{2},\frac{x}{2})$ gives $f(x)\geq 0$ $\forall$ $x$ and $Q(x,-x)$ shows that $f(x)\neq 0$ and $f(-x)=\frac{1}{4f(x)}$
$Q(a,-x)$ gives $f(a-x)=f(-x)=\frac{1}{4f(x)}$ but $f(a-x)=f(x)$ and so $\frac{1}{4f(x)}=f(x)$ $\Rightarrow$ $f(x)^{2}=\frac{1}{4}$
Since we have shown that $f(x)\geq 0$, we can conclude that $f(x)=\frac{1}{2}$ $\forall$ $x\in \mathbb{R}$

Is it rather clear ?

-- 
Patrick
\end{mysolution}



\begin{mysolution}[by \href{https://artofproblemsolving.com/community/user/128045}{efang}]
	crastybow and I did this last night which was the night right after the allnighter of AMSP i.e. in 40 hours we each had a total of 2 hours of sleep so we weren't able to finish the problem but this morning after 12 hours of sleep I solved it.

Anyways 

crastybow part:

Substitute $x=y=0$ to see that $f(0) = f(a) \rightarrow f(a) = \frac{1}{2}$

Now substitute $x=y=a$ (which we thought to be the crux move in our tired state) Notice that then $f(2a) = f(a)f(0) + f(a)f(0) = f(a) = \frac{1}{2}$

Yesterday night we made the mistake of assuming that from here we could assume that $f(a) = f(2a) = f(4a) ....$ which, although is true, we didn't actually prove it and when you sub $x=y=2a$ you don't obtain $f(4a) = f(2a)$ but also that it doesn't cover when $a=0$ (we noticed that it didn't cover the $a=0$ part last night which is why crastybow didn't post the solution) 

And then I finished the rest this morning (no more random talk) 

I first noticed that 

$f(4a) = f(a+3a) = f(a)f(0) + f(3a)f(-2a) = f(a-2a) = f(a)f(0) + f(-2a)f(3a)$

Perhaps there's a generalization...

$f(a+ba) = f(a)f(0) + f(ba)f(a-ba) = f(a)f(0) + f(a-ba)f(ba) = f(2a-ba) \rightarrow f((b+1)a) = f((2-b)a)$ 

From here we conclude that 

$f(a) = f(2a) = f(0) = f(3a) = f(-a) = f(4a) = f(-2a) = f(5a) \ldots$ 

So this means that the function $f(x,y) - \frac{1}{2} = 0$ has infinite roots $\implies f(x,y)$ is a constant polynomial.

That is of course if $a \ne 0$. 

If a=0 then claiming that $f(a) = f(2a) = f(0) \ldots$ would be claiming that $f(0) = f(0) = f(0) \ldots$ 

So now we solve for when $a=0$. We notice that if $a = 0$

$f(x+y) = f(x)f(-y)+f(-x)f(y) = f(-(x+y))$ 

Now plug in $x=-y$

$f(0) = f(x)f(y) + f(x)f(y) = 2f(x)f(y) = 2f(x)f(-x) = 2f(x)f(x) = \frac{1}{4} \rightarrow f(x) = \pm \frac{1}{2}$

We already knew that $f(0) = \frac{1}{2}$ so this means $f(x) = \frac{1}{2}$ for all $x$.

Therefore the function of $f$ must always be constant.

EDIT:


you know what screw it. I don't think it's ok just to say $f(0) = \frac{1}{2}$ so $f(x) = \frac{1}{2}$ 

Let me try to prove that $f(x) > 0$ for all $x$ which would be a more satisfactory finish

$f(\frac{x}{2} + \frac{x}{2}) = 2f(\frac{x}{2})f(-\frac{x}{2}) = 2f(\frac{x}{2})^2$ which is always positive.

So $f(x) > 0 \rightarrow f(x) = \frac{1}{2}$ so $f$ is constant. 

$\square$.
\end{mysolution}
*******************************************************************************
-------------------------------------------------------------------------------

\begin{problem}[Posted by \href{https://artofproblemsolving.com/community/user/18812}{pohoatza}]
	Does there exist a function $f: \mathbb{R}\rightarrow\mathbb{R}$ such that $f(x+f(y))=f(x)+\sin y$, for all reals $x,y$ ?
	\flushright \href{https://artofproblemsolving.com/community/c6h145741}{(Link to AoPS)}
\end{problem}



\begin{mysolution}[by \href{https://artofproblemsolving.com/community/user/29428}{pco}]
	Hello pohoatza,
\begin{tcolorbox}Does there exist a function $f: \mathbb{R}\rightarrow\mathbb{R}$ such that $f(x+f(y))=f(x)+\sin y$, for all reals $x,y$ ?\end{tcolorbox}

1) $f(x)$ is surjective :\\
1.1) : $f(x+f(y)) = f(x) + sin(y)\rightarrow f(f(x)) = f(0) + sin(x) ==> f(R) contains [f(0)-1,f(0)+1]$ \\
1.2) : $f(x+f(y)) = f(x) + sin(y) \rightarrow f(x+nf(y)) = f(x) + n*sin(y) \rightarrow f(x+nf(pi/2)) = f(x)+n and f(x+nf(-pi/2)) = f(x)-n $ \\

So $f(R)$ contains one interval of width 2 (see 1.1) and any translation by $\pm$ n of such an interval (see 1.2) $\rightarrow f(R) = R$
Q.E.D.

2) $f$ is bounded \\
$f$ is surjective $\rightarrow$ it exists $x0$ such that $f(x0) = 0$
$f$ is surjective $\rightarrow$ for any $x$, it exists z such that $f(z) = x-x0$
Then : \\
f(x) = f(x0 + f(z)) = f(x0) + sin(z) = sin(z) ==> f(x) is in [-1,+1] for any x
Q.E.D.

3) $f$ doesn't exist \\
It's obvious since 1) and 2) are in contradiction.

-- 
Patrick
\end{mysolution}



\begin{mysolution}[by \href{https://artofproblemsolving.com/community/user/29721}{Erken}]
	substitute $ 0$ instead of $ x$ into the main equation:

$ f(f(x)) = f(0) + \sin{x}$

so $ f(f(x))\in [f(0) - 1,f(0) + 1]$

It means that a boundary of all possible values of $ f(f(x))$ is limited, consequently, the same claim should hold for $ f(x)$.

so if there is some $ x$ such that $ f(x) = y$, then:

$ f(x + f(z)) = y + \sin{z}$, so for every $ y_0\in[y - 1,y + 1]$ there should exist $ x_0$ such that $ f(x_0) = y_0$, it is enough to take $ x_0 = x + f(z_0)$, where $ \sin{z_0} = y_0 - y$.

So the image of $ f$ can be infinitely extended, thus the boundary of its values is unlimited. Causing a Contradiction.
\end{mysolution}



\begin{mysolution}[by \href{https://artofproblemsolving.com/community/user/201911}{sansae}]
	\begin{tcolorbox}
It means that a boundary of all possible values of $ f(f(x))$ is limited, consequently, the same claim should hold for $ f(x)$.
\end{tcolorbox}

I think this is not right... is it?
If this statement is true, why?
\end{mysolution}



\begin{mysolution}[by \href{https://artofproblemsolving.com/community/user/29428}{pco}]
	Clearly wrong :

Let $f(x)=|x|-x$ so that $f(f(x))\equiv 0$

We have an example of unbounded continuous function $f(x)$ such that $f(f(x))$ is bounded.



\end{mysolution}
*******************************************************************************
-------------------------------------------------------------------------------

\begin{problem}[Posted by \href{https://artofproblemsolving.com/community/user/1147}{stergiu}]
	Find all real functions $f$ defined on $ \mathbb R$, such that  \[f(f(x)+y) = f(f(x)-y)+4f(x)y ,\] for all real numbers $x,y$.
	\flushright \href{https://artofproblemsolving.com/community/c6h146128}{(Link to AoPS)}
\end{problem}



\begin{mysolution}[by \href{https://artofproblemsolving.com/community/user/16261}{Rust}]
	$y=f(x)$ gives $f(2y)=f(0)+(2y)^{2}.$\\
$y=-f(x)$ give $f(-2y)=f(0)+(2y)^{2}.$\\
It is easy to check, that $f(x)=x^{2}+c$ is the solution.\\ 
As proof, if that equation had no other solution sufficient enough to show, for any y exists x, such that $|y|=|f(x)|$.
\end{mysolution}



\begin{mysolution}[by \href{https://artofproblemsolving.com/community/user/18653}{maky}]
	i don`t think that the surjectivity of $f$ is very simple. in fact, i think it`s harder than the "offical" solution.
\end{mysolution}



\begin{mysolution}[by \href{https://artofproblemsolving.com/community/user/29428}{pco}]
	Hello,

$IM(f)$ is a serious problem.

For example, $f(x) = 0$ is a solution (different from $x^2 + c$)

-- 
Patrick
\end{mysolution}



\begin{mysolution}[by \href{https://artofproblemsolving.com/community/user/2501}{Anto}]
	The solution that I like is as follows.

Let $g(x)=f(x)-x^2$, then the functional relation becomes $g(g(x)+x^2+y)=g(g(x)+x^2-y)$ for all real $x,y$. Then $g(g(x)+x^2-g(y)-y^2+z)=g(g(x)+x^2+g(y)+y^2-z)=g(g(y)+y^2-g(x)-x^2+z)$, thus
$g(z)=g(2*(g(x)+x^2-g(y)-y^2) +z)$ for all real $x,y,z$. If $g(x)+x^2$ is constant, then we get the solution $f(x)=0$ for all real $x$. \\ If $g(x)+x^2$ is not constant, then $g$ is periodic, say with period $T$. Then from the last equation we put $x=y+T$ to get $g(z)=g(2*(2*y*T+T^2)+z)$ for all real $y$ and $z$. But we can choose $y$ at our will to get $g(z)=g(0)$ for all real $z$ and hence $f(x)=x^2+g(0)$ for all real $x$.
\end{mysolution}



\begin{mysolution}[by \href{https://artofproblemsolving.com/community/user/18653}{maky}]
	why is $g$ periodic ?
\end{mysolution}



\begin{mysolution}[by \href{https://artofproblemsolving.com/community/user/889}{Persu Madalina}]
	If $g(x) + x^2$ is not constant then there exists $a$ different from $b$ such that $a=g(x)+x^2$ and $b=g(z)+z^2$ for some $x$ and $z$.

 $g(a+y)=g(a-y)$ (*) and $g(b+y)=g(b-y)$ for all $y$. \\
Substitute $y$ by $a-y-b$ in the second relation and you obtain $g(b+(a-y-b))=g(b-(a-y-b)) \iff g(a-y)=g(y+2*b-a)$, but from (*), $g(a-y)=g(y+a) \implies g(y+a)=g(y+2*b-a)$.
 If we take $z = y+a$ we will get  $g(z) = g(z+2*(b-a))$ and because $z$ undergoes $\mathbb{R} \implies g$ is periodic of period $2*(b-a)$. 
\end{mysolution}



\begin{mysolution}[by \href{https://artofproblemsolving.com/community/user/29428}{pco}]
	Hello Maky!

\begin{tcolorbox}why is $g$ periodic ?\end{tcolorbox}

Because he has $g(z)=g(2(g(x)+x^{2}-g(y)-y^{2})+z)$ for all real x,y,z
So, if $g(x)+x^{2}$ is not a constant, there exist x and y such that $g(x)+x^{2}\neq g(y)+y^{2}$ \\
Then, if I call $T=2(g(x)+x^{2}-g(y)-y^{2})$, $T\neq 0$ and for any real z, $g(z)=g(2(g(x)+x^{2}-g(y)-y^{2})+z)$ $\rightarrow$ $g(z)=g(z+T)$

And $g$ is periodic.

-- 
Patrick
\end{mysolution}



\begin{mysolution}[by \href{https://artofproblemsolving.com/community/user/19490}{behemont}]
	\begin{tcolorbox}The solution that i like is as follows :

Let $g(x)=f(x)-x^2$ , then the functional relation becomes $g(g(x)+x^2+y)=g(g(x)+x^2-y)$ for all real $x,y$. Then \[g(g(x)+x^2-g(y)-y^2+z)=g(g(x)+x^2+g(y)+y^2-z)=g(g(y)+y^2-g(x)-x^2+z)\] and thus
$g(z)=g(2*(g(x)+x^2-g(y)-y^2) +z)$ for all real $x,y,z$. If $g(x)+x^2$ is constant then we get the solution $f(x)=0$ for all real $x$. If $g(x)+x^2$ is not constant then $g$ is periodic say with period $T$. Then from the last equation we put $x=y+T$ to get $g(z)=g(2*(2*y*T+T^2)+z)$ for all real $y$ and $z$. But we can choose $y$ at our will to get $g(z)=g(0)$ for all real $z$ and hence $f(x)=x^2+g(0)$ for all real $x$.\end{tcolorbox}

can someone turn this into $LaTeX$ and explain a bit?
\end{mysolution}



\begin{mysolution}[by \href{https://artofproblemsolving.com/community/user/18653}{maky}]
	all the solutions i know (including mine, in the contest) use the idea that $\mbox{Im}\, f-\mbox{Im}\, f =\mathbb{R}$. after that, it`s pure algebraic manipulations.
\end{mysolution}



\begin{mysolution}[by \href{https://artofproblemsolving.com/community/user/19490}{behemont}]
	\begin{tcolorbox}all the solutions i know (including mine, in the contest) use the idea that $\mbox{Im}\, f-\mbox{Im}\, f =\mathbb{R}$. after that, it`s pure algebraic manipulations.\end{tcolorbox}

can you show that?
\end{mysolution}



\begin{mysolution}[by \href{https://artofproblemsolving.com/community/user/939}{idioteque}]
	What is IR?  :blush:
\end{mysolution}



\begin{mysolution}[by \href{https://artofproblemsolving.com/community/user/8638}{me@home}]
	\begin{tcolorbox}The solution that i like is as follows :

Let $g(x)=f(x)-x^{2}$ , then the functional relation becomes \[g\left(g(x)+x^{2}+y\right)=g\left(g(x)+x^{2}-y\right)\qquad \forall \text{ real }x,y\]Then \[g\left(g(x)+x^{2}-g(y)-y^{2}+z\right)=g\left(g(x)+x^{2}+g(y)+y^{2}-z\right)=g\left(g(y)+y^{2}-g(x)-x^{2}+z\right)\] thus
\[g(z)=g\left(2(g(x)+x^{2}-g(y)-y^{2}\right)+z)\qquad \forall\text{ real }x,y,z\]If $g(x)+x^{2}$ is constant then we get the solution $f(x)=0$ for all real $x$. If $g(x)+x^{2}$ is not constant then $g$ is periodic say with period $T$. Then from the last equation we put $x=y+T$ to get \[g(z)=g\left(2\cdot(2yT+T^{2})+z\right)\qquad \forall \text{ real }y,z\]But we can choose $y$ at our will to get \[g(z)=g(0)\qquad\forall \text{ real }z\] and hence \[\boxed{f(x)=x^{2}+g(0) \qquad\forall \text{ real }x}\]\end{tcolorbox}

IR is pure imaginaries.
\end{mysolution}



\begin{mysolution}[by \href{https://artofproblemsolving.com/community/user/5820}{N.T.TUAN}]
	In problem IR is $\mathbb{R}$ = set of all real numbers.
\end{mysolution}



\begin{mysolution}[by \href{https://artofproblemsolving.com/community/user/18653}{maky}]
	I will post here my solution. I gave it in the contest as well.

the solutions are $f\equiv 0$ and $f(x)=x^{2}+a$, with $a$ being real.
Obviously $f\equiv 0$ satisfies the equation, so I will choose an $x_{0}$ now such that $f(x_{0})\neq 0$. I first claim that any real number can be written as $f(u)-f(v)$, with $u,v$ reals. denote $f(x_{0})$ with $t\neq 0$. then, putting in the equation, it follows that
$f(t+y)-f(t-y)=4ty$.
since $t\neq 0$ and $y$ is an arbitrary real number, it follows that any real $d$ can be written as $f(u)-f(v)$, with $u,v$ reals (take $y=d/4t$ above).
this proves my claim above.
now, let`s see that
$f\big(f(x)+f(y)+z\big)=f\big(f(x)-f(y)-z\big)+4f(x)\big(f(y)+z\big)$
and
$f\big(f(y)+f(x)+z\big)=f\big(f(y)-f(x)-z\big)+4f(y)\big(f(x)+z\big)$
(the relations are deduced from the hypothesis)
for all reals $x,y,z$. i will denote by $d$ the difference $f(x)-f(y)$.
subtracting the above relations will give me
$f(d-z)-f(-d-z)+4zd=0$.
now, let`s see to that this is true for all reals $d,z$, because $z$ was chosen arbitrary and $d=f(x)-f(y)$ as well can be chosen arbitrary because of the first claim.
now it`s easy. Just take $d=z=-x/2$ and it gets that
$f(0)-f(x)+x^{2}=0$, or
$f(x)=x^{2}+f(0)$.
This obviously satisfies the relation, and this ends the proof.
\end{mysolution}



\begin{mysolution}[by \href{https://artofproblemsolving.com/community/user/43727}{RaleD}]
	First, $ f(x)=0$ is the solution. Now suppose $f(x) \not=0$ for some $x$. We can see that any $y\in R$ can be represented by $f(a)-f(b)$ and also $2(f(a)-f(b))$ for some reals $a, b$ (1). Now placing $f(x)=y$ gives $f(2f(x))=f(0)+4f(x)^2$. Put $y=f(x)-2f(y)$ and we get 

$f(2(f(x)-f(y)))=f(2f(y))+4f(x)^2-8f(x)f(y)$
$=f(0)+4f(y)^2+4f(x)^2-8f(x)f(y)$
$=f(0)+4(f(x)-f(y))^2$.  

Because of (1) we see that all solutions other than $ f(x)=0$ are $f(x)=x^2+f(0)$
\end{mysolution}



\begin{mysolution}[by \href{https://artofproblemsolving.com/community/user/94548}{ShahinBJK}]
	From which source did you find this question. It is International Zhautykov olympiad 2011.But you wrote it at 2007 How?
\end{mysolution}



\begin{mysolution}[by \href{https://artofproblemsolving.com/community/user/44358}{crazyfehmy}]
	\begin{tcolorbox}From which source did you find this question. It is International Zhautykov olympiad 2011.But you wrote it at 2007 How?\end{tcolorbox}
It is Balkan MO 2007 second problem.
\end{mysolution}



\begin{mysolution}[by \href{https://artofproblemsolving.com/community/user/64716}{mavropnevma}]
	Not quite. Swapping $x$ and $y$, the Zhautykov writes as $f(f(x) + y) = f(y - f(x)) + 4f(x)y$ (the difference is, at the Balkan MO, the start of RHS was $f(f(x) - y)$). But, of course, the method is identical, so the problem is a spoof. See also my comment posted at the Zhautykov link \href{https://artofproblemsolving.com/community/c6h386820p2148157}{here}
\end{mysolution}



\begin{mysolution}[by \href{https://artofproblemsolving.com/community/user/94548}{ShahinBJK}]
	\begin{tcolorbox}	Not quite. Swapping $x$ and $y$, the Zhautykov writes as $f(f(x) + y) = f(y - f(x)) + 4f(x)y$ (the difference is, at the Balkan MO, the start of RHS was $f(f(x) - y)$). But, of course, the method is identical, so the problem is a spoof. See also my comment posted at the Zhautykov link \href{https://artofproblemsolving.com/community/c6h386820p2148157}{here}\end{tcolorbox}
yes i didn`t see the difference but $f:even$ can be easily found.
\end{mysolution}



\begin{mysolution}[by \href{https://artofproblemsolving.com/community/user/202190}{Blitzkrieg97}]
	IR means irrational?
\end{mysolution}



\begin{mysolution}[by \href{https://artofproblemsolving.com/community/user/208265}{john111111}]
	No,it is the set of real numbers.
\end{mysolution}



\begin{mysolution}[by \href{https://artofproblemsolving.com/community/user/223099}{MathPanda1}]
	Does this work?
$f \equiv 0$ is trivial. Assume there exists $x_0$ such that $f(x_0) \neq 0$.
The equation is equivalent to $f(x+y)=f(x-y)+4xy$ for all $x \in \mbox{Im}\,f$, $y \in \mathbb{R}$.              (1)
Since $y$ can range over all real numbers, sub $x=x_0$ gives $\mbox{Im}\,f - \mbox{Im}\,f = \mathbb{R}$.
Let $f(x)-x^2=g(x)$. Then, (1) is equivalent to $g(x+y)=g(x-y)$ or $g((x-z)+y+z)=g((x-z)-y+z)$ for all $x,z \in \mbox{Im}\,f$, $y \in \mathbb{R}$.
Since $\mbox{Im}\,f - \mbox{Im}\,f = \mathbb{R}$, we get $g(w+y+z)=g(w-y+z)$ for all $z \in \mbox{Im}\,f$, $y,w \in \mathbb{R}$. Let $y= \frac{u-v}{2}$ and $w=\frac{u+v-2z}{2}$ for arbitrary real numbers $u, v$. Then, $g(u)=g(v)$ for all $u,v \in \mathbb{R}$ i.e. $g$ is a constant i.e. $f(x)=x^2+c$ for some constant $c$, as required.

\end{mysolution}



\begin{mysolution}[by \href{https://artofproblemsolving.com/community/user/218500}{navi\_09220114}]
	Hi MathPanda1,

I am not sure why if you fix a real z in Im(f), then x-z can represent all reals? If not, then w is only the set Im(f)-z, which is not R. (since Im(f) is not necessarily R)
\end{mysolution}



\begin{mysolution}[by \href{https://artofproblemsolving.com/community/user/367931}{Vrangr}]
	Redacted...
\end{mysolution}



\begin{mysolution}[by \href{https://artofproblemsolving.com/community/user/218500}{navi\_09220114}]
	Im(f) here I mean image of f, not the imaginary part of f. I am sorry for the confusion
\end{mysolution}
*******************************************************************************
-------------------------------------------------------------------------------

\begin{problem}[Posted by \href{https://artofproblemsolving.com/community/user/9797}{Jan}]
	Determine all maps $f: \mathbb{N}\rightarrow [1,+\infty)$ that satisfy the following conditions:
[list]
[*]$f(2)=2$,
[*]$f(mn)=f(m)f(n)$, for all $m,n \in \mathbb{N}$, and
[*]$f(m) < f(n)$ if $m < n$.
[/list]
	\flushright \href{https://artofproblemsolving.com/community/c6h146348}{(Link to AoPS)}
\end{problem}



\begin{mysolution}[by \href{https://artofproblemsolving.com/community/user/16261}{Rust}]
	If $n=\prod p_{i}^{k_{i}}$, then $f(n)=\prod_{i}(f(p_{i}))^{k_{i}}$.
Let $g(p)=\frac{ln(f(p))}{ln(p)}$ function $P\to R$.
Let p and q primes and $\frac{P_{k}}{Q_{k}}$ is a good rational approach $\frac{ln(p)}{ln(q))}$.
Then $(-1)^{k}p^{Q_{k}}>(-1)^{k}q^{Q_{k}}$, therefore $(-1)^{k}f(p^{Q_{k}}))>(-1)^{k}f(q^{P_{k}})$.
It gives $(-1)^{k}\frac{ln(f(p))}{ln(f(q))}>(-1)^{k}\frac{P_{k}}{Q_{k}}$ or ${|\frac{lnf(p)}{ln(f(q)}-\frac{ln(p)}{ln(q}}|<\epsilon_{k}\to 0$.
Therefore $g(p)$ is constant and $f(n)=n^{a},a\ge 0$.
\end{mysolution}



\begin{mysolution}[by \href{https://artofproblemsolving.com/community/user/29428}{pco}]
	Hello Jan

\begin{tcolorbox}Determine all maps $f: \mathbb{N}_{0}\rightarrow [1,+\infty[$ that satisfy:
[list]
[*]$f(2)=2$
[*]$f(mn)=f(m)f(n)$, $\ \forall \ m,n \in \mathbb{N}_{0}$
[*]$f(m) < f(n)$ if $m < n$[/list]\end{tcolorbox}

First, we can obliviously ignore the requirement $f(2)=2$
, $f(m^{r})=f(m)^{r}$
Let a,b be two integers $>$ 1 and n,p two integers $>$ 0 such that $\frac{n}{p}\geq\frac{ln(a)}{ln(b)}$
Then $b^{n}\geq a^{p}$ hence $f(b^{n})\geq f(a^{p})$ then $f(b)^{n}\geq f(a)^{p}$ and $\frac{n}{p}\geq\frac{ln(f(a))}{ln(f(b))}$

So : Let a,b two integers $>$ 1 and n,p two integers $>$ 0 : $\frac{n}{p}\geq\frac{ln(a)}{ln(b)}$ ==> $\frac{n}{p}\geq\frac{ln(f(a))}{ln(f(b))}$

But it is always possible to find two integers p and q such that $\frac{n}{p}$ be as near as possible and above $\frac{ln(a)}{ln(b)}$
$rightarrow$ $\frac{ln(a)}{ln(b)}\geq\frac{ln(f(a))}{ln(f(b))}, \forall a,b$ integers > 1
Since we then also have $\frac{ln(b)}{ln(a)}\geq\frac{ln(f(b))}{ln(f(a))}$

The conclusion is $\frac{ln(a)}{ln(b)}=\frac{ln(f(a))}{ln(f(b))}$, hence $\frac{ln(f(a))}{ln(a)}=\frac{ln(f(b))}{ln(b)}$

Hence $f(x)=x^{c}$

But $f(2)$=2

Hence $f(x) = x$

-- 
Patrick
\end{mysolution}



\begin{mysolution}[by \href{https://artofproblemsolving.com/community/user/112}{Diogene}]
	$f(1)=(f(1))^{2}\Longrightarrow f(1)=1$. Let $a$ be an integer $a > 2$ . It's easy to see that $f(a^{n})=(f(a))^{n}$ . 
Let $u_{n}$ be an  integer such that $2^{u_{n}-1}\leq a^{n}\leq 2^{u_{n}}$ so $2^{\frac{u_{n}-1}n}\leq a\leq 2^{\frac{u_{n}}n}$ so  $\frac{u_{n}}n$  is convergent to $\frac{\ln (a)}{\ln (2)}$
We can write : 
$2^{u_{n}-1}\leq a^{n}\leq 2^{u_{n}}\Longrightarrow (f(2))^{u_{n}-1}\leq (f(a))^{n}\leq (f(2))2^{u_{n}}$$\Longrightarrow 2^{\frac{u_{n}-1}n}\leq f(a)\leq 2^{\frac{u_{n}}n}\Longrightarrow f(a)=2^{\frac{\ln (a)}{\ln (2)}}=a$
Of course this proof is verry known.
 :cool:
\end{mysolution}
*******************************************************************************
-------------------------------------------------------------------------------

\begin{problem}[Posted by \href{https://artofproblemsolving.com/community/user/23937}{Lee Sang Hoon}]
	Find all functions $f: \mathbb N \cup \{0\} \to \mathbb N \cup \{0\}$ such that $f(1)=1$ and
\[f(m^2+n^2)=f(m)^2+f(n)^2,\]
for all $m,n \in \mathbb N \cup \{0\}$.
	\flushright \href{https://artofproblemsolving.com/community/c6h147098}{(Link to AoPS)}
\end{problem}





\begin{mysolution}[by \href{https://artofproblemsolving.com/community/user/5820}{N.T.TUAN}]
	[hide="hint"]Choose $m=n=0$ and we have $f(0)=2f^{2}(0)$ , so $f(0)=0$ because $f(0)\in\mathbb{Z}$.
Choose $m=0$ and we have $f(m^{2})=f^{2}(m)\forall m\in S$, so $f(m^{2}+n^{2})=f(m^{2})+f(n^{2})\forall m,n\in S$.

Choose $m=n=1$ and we have $f(2)=2$.

Easy to see that $f(4)=4,f(5)=5,f(8)=8,f(26)=26$. Therefore $25=f^{2}(5)=f(25)=f(3^{2}+4^{2})=f^{2}(3)+f^{2}(4)$, so $f(3)=3$, by this method we have $f(k)=k\forall 0\leq k\leq 26$.

Assume that $f(i)=i\forall i\in\{0,1,...,k\}(k\geq 26)$, we need to prove $f(k+1)=k+1$.
Use $(12n)^{2}+n^{2}=(9n)^{2}+(8n)^{2}$ we can know that $f(12n)$, use $(12n+1)^{2}+(n-12)^{2}=(9n+8)^{2}+(8n-9)^{2}$ we can know that $f(12n+1)$, etc. 
\end{mysolution}



\begin{mysolution}[by \href{https://artofproblemsolving.com/community/user/29428}{pco}]
	Hello N.T.TUAN
\begin{tcolorbox}...by this method we have $f(k)=k\forall 0\leq k\leq 26$.\end{tcolorbox}

I'm sorry but this seems not so simple.
Finding $f(7)$, for example, we need to use $7^{2}+24^{2}=25^{2}$ and hence to know $f(24)$. this is possible (not immediate) through $24^{2}+10^{2}=26^{2}$ and $10=3^{2}+1^{2}$ and $26=5^{2}+1^{2}$ 

But, for computing $f(11)$, the only equation available is $11^{2}= 61^{2}-60^{2}$ so we need to know $f(61)$ and $f(60)$. For $f(61)$, the only equation is $61^{2}=1861^{2}-1860^{2}$ !

So, in order to compute $f(11)$, we need to compute $f(60)$, $f(1861)$ and $f(1860)$ !

So, how could you show that $f(11)=11$ ?

-- 
Patrick
\end{mysolution}






\begin{mysolution}[by \href{https://artofproblemsolving.com/community/user/5820}{N.T.TUAN}]
	Yes, But we have $11^{2}+2^{2}=125=10^{2}+5^{2}$, so $f^{2}(11)+f^{2}(2)=f^{2}(10)+f^{2}(5)$.
\end{mysolution}



\begin{mysolution}[by \href{https://artofproblemsolving.com/community/user/29428}{pco}]
	\begin{tcolorbox}Yes, But we have $11^{2}+2^{2}=125=10^{2}+5^{2}$, so $f^{2}(11)+f^{2}(2)=f^{2}(10)+f^{2}(5)$.\end{tcolorbox}
You're right. I forgot the equation $n^{2}=a^{2}+b^{2}-c^{2}$

Here is another solution

1) First we can show that $f(n)$ ($\forall n>1$) may always be computed just using other $f(i)$ with $i<n$   :   Let $n=(2k+1)2^{p}$

1.1) If $k>1$ $n^{2}=a^{2}+b^{2}-c^{2}$ with $a=(2k-1)2^{p}<n$, $b=(k+2)2^{p}<n$, and $c=(k-2)2^{p}<n$
Then $f(n)^{2}+f(c)^{2}=f(n^{2}+c^{2})=f(a^{2}+b^{2})=f(a)^{2}+f(b)^{2}$ $\Rightarrow $ $f(n)^{2}=f(a)^{2}+f(b)^{2}-f(c)^{2}$

1.2) If $k=1$ and $p$ is odd ($n=3*2^{2q+1}$) : $n^{2}=a^{2}+b^{2}-c^{2}$ with  $a=3*2^{2q}<n$,  $b=2^{2q}<n$,  and  $c=(2^{2q+1})<n$
Then $f(n)^{2}=f(a)^{2}+f(b)^{2}-f(c)^{2}$

1.3) If $k=1$ and $p$ is even ($n=3*2^{2q}$):
$f(5*2^{2q}) = f((2^{q+1})^{2}+(2^{q})^{2}) = (f(2^{q+1}))^{2}+(f(2^{q}))^{2})$  with  $2^{q+1}<n$  and  $2^{q}<n$
$f(4*2^{2q}) = (f(2^{q+1}))^{2}$
$f(n)^{2}= f((3*2^{2q})^{2})= (f(5*2^{2q}))^{2}-(f(4*2^{2q}) )^{2}$

1.4) If $k=0$ and $p$ is odd ($n=2^{2q+1}$) :
$f(n)=f((2^{q})^{2}+(2^{q})^{2}) = 2(f(2^{q}))^{2}$  with  $2^{q}<n$

1.5) If $k=0$ and $p$ is even ($n=2^{2q}$) :
$f(n) = (f(2^{q}))^{2}$  with  $2^{q}< n$

2) It's immediate then, with induction starting with $f(0)=0$ and $f(1)=1$ and using the above formulas, to show that $f(n)=n$ $\forall n\geq 0$ is a necessary condition.

3) Last easy step is to show that necessary condition $f(n)=n$ is also sufficient.

-- 
Patrick
\end{mysolution}



\begin{mysolution}[by \href{https://artofproblemsolving.com/community/user/5820}{N.T.TUAN}]
	You are wrong at 1.2 !
\end{mysolution}



\begin{mysolution}[by \href{https://artofproblemsolving.com/community/user/29428}{pco}]
	\begin{tcolorbox}You are wrong at 1.2 !\end{tcolorbox}

Yes! Sorry !  :blush:  :blush: 

1.2.new) If $k=1$ and p is odd ($n=3*2^{2q+1}$)
$f(10*2^{2q})=f(3*2^{q})^{2}+f(2^{q})^{2}$
$f(8*2^{2q})=f(2^{q+1})^{2}+f(2^{q+1})^{2}$
$f(3*2^{2q+1})^{2}=f(10*2^{2q})^{2}-f(8*2^{2q})^{2}$

-- 
Patrick
\end{mysolution}



\begin{mysolution}[by \href{https://artofproblemsolving.com/community/user/5820}{N.T.TUAN}]
	Now, you are true, i think so. Congratulate for new solution !  
\end{mysolution}
*******************************************************************************
-------------------------------------------------------------------------------

\begin{problem}[Posted by \href{https://artofproblemsolving.com/community/user/6551}{perfect\_radio}]
	Find all pairs of functions $ f : \mathbb R \to \mathbb R$, $g : \mathbb R \to \mathbb R$ such that \[f \left( x + g(y) \right) = xf(y) - y  f(x) + g(x) \quad\text{for all } x, y\in\mathbb{R}.\]
	\flushright \href{https://artofproblemsolving.com/community/c6h148108}{(Link to AoPS)}
\end{problem}



\begin{mysolution}[by \href{https://artofproblemsolving.com/community/user/9092}{TomciO}]
	I have a solution to this one (not mine) and I can post it if you want (it's not short but not very long).
If you are still interested in trying it yourself here are 2 hints:
1)Prove that $g$ is surjective (that's not very easy).
2)Therefore, it takes value $0$, using it, solve the rest of the problem (that's the easier part).
\end{mysolution}



\begin{mysolution}[by \href{https://artofproblemsolving.com/community/user/29428}{pco}]
	Hello !

My way to prove that $g(x)$ may not be surjective is very short :

$(f(x)=0, g(x)=0)$ is a solution in which $g$ is not surjective.

-- 
Patrick
\end{mysolution}



\begin{mysolution}[by \href{https://artofproblemsolving.com/community/user/6551}{perfect\_radio}]
	I think that you're totally misunderstanding e.lopes and TomciO! When $f(0) = 0$ (that's the first thing I tried) it's easy to prove that $g$ takes the value $0$. In the other case, I think it may be that $g$ is surjective.
\end{mysolution}



\begin{mysolution}[by \href{https://artofproblemsolving.com/community/user/9092}{TomciO}]
	Exactly, in fact, I wasn't clear in my previous post. We need only that $g$ takes value $0$, if $f(0)=0$ then we can do it by hand, in other case we prove that $g$ is surjective.
\end{mysolution}



\begin{mysolution}[by \href{https://artofproblemsolving.com/community/user/13603}{e.lopes}]
	I found one more nice and short way to solve our equation, my friends!

We will prove first that $g(i) = 0$, for any $i$.

If $f(0) = 0$, we find that $f(x+g(0)) = g(x)$. Take $x =-g(0). g(-g(0)) = f(0) = 0$.

If $f(0)$ isn't, it is more difficult. See:
Take $x = 0$ in the original equation. We see that f is surjective and g is injective!, because $f(g(y)) = g(0)-y.f(0)$

In original equation, put $x = g(x)$. We find $f(g(x)+g(y)) = g(x)f(y)-yf(g(x)+g(g(x))=g(x)f(y)-y(g(0)-xf(0))+g(g(x))$

This is symmetric, so, $g(x)f(y)-y(g(0)-xf(0))+g(g(x)) = g(y)f(x)-x(g(0)-yf(0))+g(g(y))$ (*).

Cause f is surjetive, we can take $y = m$, where $f(m) = 0$.

Do this in (*)!

We find that $-f(0)m+g( g(x) ) = g(m) f(x)-f(0)x+g(g(m))$. Putting $g(m) = p$ and $g(g(m))+f(0)m = q$,

this becomes $g(g(x)) = pf(x)-f(0)x+q$.

Substituting back into (*) we find  $g(x) f(y)+pf(x) = g(y)f(x)+pf(y)$ (**)

Let $y = 0$ in (**). We find $g(x) = \frac{(f(0)-p)f(x)}{g(0)}+k$. So, g is surjective! and we can take $g(i) = 0!$ for one real $i$.

Now, we will prove that f and g are linears!
So, in the original equation, let $y = i$!

We find $g(x) = (i+1)f(x)-f(i)x$. Substituting back into original equality, we find:

$f(x+g(y)) = (i+1-y)f(x)+x(f(y)-f(i))$ (***)

Let $y = i+1$ in (***).

We find that $f(x+r) = x(f(i+1)-f(i))$, where $f = g(i+1)$! So, $f$ is linear!  :) 

In the equality $g(x) = (i+1)f(x)-f(i)x$, we see the that g is also linear!


Now, is easy, let $f(x) = ax+b and g(x) = cx+d$, and good luck with the computations!

The solutions are f(x) = g(x) = 0 and $f(x) = \frac{k(x-k)}{k+1}$ , $g(x) ={k(x-k)}$! for a real k!
\end{mysolution}



\begin{mysolution}[by \href{https://artofproblemsolving.com/community/user/94615}{Pedram-Safaei}]
	suppose that $P(x,y)$ be the following assertion:$f(x+g(y))=xf(y)-yf(x)+g(x)$
$P(x,0):f(x+g(0))=xf(0)+g(x)$. (put $g(0)=c$)      $(1)$
$P(x+t+c,y):f(x+c+t+g(y))=(x+t+c)f(y)-yf(x+c+t)+g(x+c+t)$
$P(x+c,y):f(x+c+g(y))=(x+c)f(y)-yf(x+c)+g(x+c)$.now subtracting two equations and use of $(1)$ we have:$tf(0)+g(x+t+g(y))-g(x+g(y))=tf(y)-y(tf(0)+g(x+c+t)-g(x+c))+g(x+t)-g(x)$.
so if the previous assertion be $Q(x,y)$,then
$Q(x,0):g(x+t+c)-g(x+c)=g(x+t)-g(x)$ $(2)$    
then with use of $(2)$ we have $Q(x,1):g(x+t+g(1))-g(x+g(1))=t(f(1)-f(0))$ so for all $x$ we have:$g(x+t)-g(x)=ta$ (put $a=f(1)-f(0)$) put $x=0$ we have:$g(t)=at+c$ so:($put b=f(0)$)
$f(x+c)=bx+ax+c$ so $f(x)=(a+b)x+c(1-a-b)$ now by putting it into main equation we conclude that:$a(a+b)=-c(1-a-b)=-b$ and $f(x)=(a+b)x+b,g(x)=ax+c$.or for some real $a\neq -1$.$f(x)=\frac{a(x-a)}{a+1},g(x)=a(x-a)$.
\end{mysolution}



\begin{mysolution}[by \href{https://artofproblemsolving.com/community/user/184652}{CanVQ}]
	\begin{tcolorbox}Find all pairs of functions $ f, g : \mathbb R \to \mathbb R$ such that $ f \left( x + g(y) \right) = x \cdot f(y) - y \cdot f(x) + g(x)\quad (1)$ for all real $ x, y$.\end{tcolorbox}
Let $f(0)=a$ and $g(0)=b.$ Taking $x=0$ in $(1),$ we get \[f\big(g(y)\big)=-ay+b,\quad \forall y \in \mathbb R. \quad (2)\] There are two cases to consider:

\begin{bolded}Case 1: $\mathbf{a=0.}$\end{bolded} Taking $y=0$ in $(1),$ we get \[f(x+b)=g(x),\quad \forall x \in \mathbb R. \quad (3)\] From this, we deduce that $g({-b})=0.$ Now, replacing $y=-b$ in $(1),$ we have \[f(x)=x\cdot f({-b})+b\cdot f(x)+g(x),\quad \forall x \in \mathbb R. \quad (4)\] Taking $x=-b$ in $(4),$ we get $f({-b})=g({-b})=0$ and hence, the equality $(4)$ can be rewritten as \[(1-b)\cdot f(x)=g(x),\quad \forall x \in \mathbb R. \quad (5)\] Taking $x=0$ in $(5),$ we get $b=g(0)=0.$ Thus, the identities $(2)$ and $(5)$ can be written as \[f\big(g(x)\big)=0,\quad \forall x \in \mathbb R\quad (6)\] and \[f(x)=g(x),\quad \forall x \in \mathbb R. \quad (7)\] From $(6)$ and $(7),$ we get $g\big(g(x)\big)=0,\, \forall x \in \mathbb R.$ Now, replacing $y$ by $g(x)$ in $(1),$ we can easily deduce that $f(x)=g(x)=0,\, \forall x \in \mathbb R.$ These functions satisfy our condition.

\begin{bolded}Case 2: $\mathbf{a \ne 0.}$\end{bolded} From $(2),$ we can easily see that $f$ is surjective and $g$ is injective. Replacing $x$ by $g(x)$ in $(1)$ and using $(2),$ we get \[f\big(g(x)+g(y)\big)=g(x)\cdot f(y)+axy-by+g\big(g(x)\big),\quad \forall x ,\,y \in \mathbb R. \quad (8)\] Changing the position of $x$ and $y$ in $(8),$ we get \[g(x)\cdot f(y)-by+g\big(g(x)\big)=g(y)\cdot f(x)-bx+g\big(g(y)\big),\quad \forall x,\,y \in \mathbb R. \quad (9)\] Replacing $y=0$ in $(1),$ we have \[f(x+b)=ax+g(x),\quad \forall x \in \mathbb R. \quad (10)\] Replacing $x$ by $g(x)$ in $(10),$ we get \[g\big(g(x)\big)+a\cdot g(x)=f\big(b+g(x)\big)=b\cdot f(x)-x\cdot f(b)+g(b),\] or \[g\big(g(x)\big)=b\cdot f(x)-a\cdot g(x)-x\cdot f(b)+g(b),\quad \forall x \in \mathbb R. \quad (11)\] Plugging this result into $(9),$ we get \[\begin{aligned} g(x)\cdot f(y)-by+{}&b\cdot f(x)-a\cdot g(x)-x\cdot f(b)=\\ &=g(y)\cdot f(x)-bx+b\cdot f(y)-a\cdot g(y)-y\cdot f(b).\quad (12)\end{aligned}\] Since $f(b)=b$ (we can easily obtain this by setting $y=0$ in $(2)$), the above identity can be written as \[g(x)\cdot \big[ f(y)-a\big] +b\cdot f(x)=g(y)\cdot \big[ f(x)-a\big] +b\cdot f(y),\] or \[\big[ g(x)-b\big] \big[ f(y)-a\big]=\big[ g(y)-b\big] \big[ f(x)-a\big],\quad \forall x,\,y \in \mathbb R. \quad (13)\] Since $f$ is surjective, there exists $y_0$ such that $f(y_0) \ne a.$ Taking $y=y_0$ in $(13),$ we get \[g(x)-b=\frac{g(y_0)-b}{f(y_0)-a}\big[f(x)-a\big],\quad \forall x \in \mathbb R.  \quad (14)\] If $g(y_0)=b,$ then we have $g(x)=b,\,\forall x \in \mathbb R$ which is a contradiction since $g$ is injective. So we must have $g(y_0)\ne b.$   From this and the previous results, we can easily prove that $f$ and $g$ are bijectives. Thus, the exists a unique number $c$ such that $g(c)=0.$ Taking $x=y=c$ in $(1),$ we have $f(c)=g(c)=0.$ Continuously, replacing $y=c$ in $(1),$ we get \[(1+c)\cdot f(x)=g(x),\quad \forall x \in \mathbb R. \quad (15)\] Since $g$ is bijective, it is clearly that $c \ne -1,$ from which it follows that \[f(x)=\frac{1}{1+c}\cdot g(x),\quad \forall x \in \mathbb R. \quad (16)\] Now, replacing $y=1+c$ in $(1)$ and using $(15),$ we get \[f\big(x+g(1+c)\big)=x\cdot f(1+c)=\frac{x\cdot g(1+c)}{1+c},\quad \forall x \in \mathbb R. \quad (17)\] Replacing $x=c-g(1+c)$ in $(17),$ we get \[g(1+c)\cdot \big[c-g(1+c)\big]=0.\] Since $g$ is bijective, we have $g(1+c)\ne 0$ and hence, it follows that $g(1+c)=c.$ From this, we have \[f(x+c)=\frac{cx}{1+c},\quad \forall x \in \mathbb R.\quad (18)\] Replacing $x$ by $x-c$ in $(18),$ we get \[f(x)=\frac{c}{1+c}(x-c),\quad \forall x \in \mathbb R.\] It follows that \[g(x)=c(x-c),\quad \forall x \in \mathbb R.\] These functions satisfy our condition.
\end{mysolution}



\begin{mysolution}[by \href{https://artofproblemsolving.com/community/user/183149}{JuanOrtiz}]
	Notice $f(g(x))=g(0)-f(0)x$. (1)
If $f(0)=0$ we get $g(-g(0))=0$.
Otherwise $f$ is surjective and $g$ is injective. taking $x=g(x)$ and considering y as a constant, we get an equation. if we change the value of y in this equation, we find that $g(x)=0$ for some x. So $g(a)=0$ for some $a$ always.
Then in our original equation $f(x)=(f(x+g(a))=xf(a)-af(x)+g(x)$ so $f(x)=(g(x)+xf(a))/(a+1)$, unless $a=-1$ but in this case we finish easily. And we get $f(a)=af(a)/(a+1)$. so if $a=0$ we finish easily, otherwise $f(a)=0$ and we get $g=cf$ for a constant c. also f is bijective and from here we finish easily.

answer: linear $f$ and $g=cf$ for a constant c. additional equations must be satisfied by the coefficients of $f$, we obtain these relations by just plugging in.
\end{mysolution}



\begin{mysolution}[by \href{https://artofproblemsolving.com/community/user/128045}{efang}]
	I proved the $f(0) = 0$ the same way as CanVQ. I have a slightly different approach in proving $f(0) \neq 0$, specifically in proving bijectivity of $f$ and $g$ 

\begin{bolded}[size=150]f(0) is not equal to 0[/size]\end{bolded}

Then $f(g(y)) = -y*f(0) + g(0)$ (1). 

\begin{bolded}Lemma: f and g are both bijective\end{bolded}

Proof:

Clearly $f \circ g$ is a bijective function as it's linear. Thus we can immediately conclude from it's injectivity that $g$ is injective and from it's surjectivity that $f$ is surjective. 

For any $c \in \mathbb{R}$ note that if $f(c) = a$ we have $f(c) = a = \frac{a-g(0)}{f(0)} * f(0) + g(0) = f(g(\frac{g(0)-a}{f(0)}))$ and since $f(0)$ is non-zero we have that there necessarily exists some real number satisfying $g(a) = c$ for all real c, proving surjectivity.

Now, we had $a,b \in \mathbb{R}$ such that $f(a) = f(b) = c$. Then, we know there exist real m and n such that $g(m) = a$ and $g(n) = b$ 

Then we would have $f(g(m)) = f(a) = -m*f(0) + g(0)$ and $f(g(n)) = f(b) = -n*f(0)+g(0)$ Thus we have $m=n$ and $a=b$ so f is injective. This completes the lemma. 

$\square$

Now this means there exists a unique value $a$ such that $g(a) = 0$. Plug in $x=y=a$ into the original equation to get $f(a) = g(a) = 0$. (2)

Now, plug in just $y=a$ and use the result from (2) to obtain $f(x) = -a*f(x) + g(x)$ or $(a+1)f(x) = g(x)$ (3)

Plugging (3) into (1) reveals $f((a+1)f(x)) = -x*f(0)+(a+1)f(0)$. Now notice that when $x=a+1$ we have that $f((a+1)f(a+1)) = 0$. Thus $f(a+1) = \frac{a}{a+1}$ and thus also $g(a+1) = a$ (4)

Now, plug (3) back into the original equation and we get

$f(x+(a+1)f(y)) = xf(y) - yf(x)+(a+1)f(x)$. Plug in $y=a+1$ and use (4) and we get:

$f(x+a) = \frac{xa}{a+1}$ or $f(x) = \frac{(x-a)a}{a+1}$ and thus from (3) $g(x) = (x-a)a$ for some real constant $a$ which completes the problem.

$\square$

(note that since $(a+1)*g(0) = f(0)$ and we have that $g(0) \neq 0$ and $f(0) \neq 0$ this means that $a + 1 \neq 0$ and $a \neq -1$.)
\end{mysolution}
*******************************************************************************
-------------------------------------------------------------------------------

\begin{problem}[Posted by \href{https://artofproblemsolving.com/community/user/6551}{perfect\_radio}]
	Prove that there is a unique function $f : \mathbb Z_{>0}\to \mathbb Z_{>0}$ such that \[f \left( m+f(n) \right) = n+f \left( m+95 \right), \forall m, n \in \mathbb Z_{>0}.\]
Compute $\sum_{k=1}^{19}f(k)$.
	\flushright \href{https://artofproblemsolving.com/community/c6h148143}{(Link to AoPS)}
\end{problem}



\begin{mysolution}[by \href{https://artofproblemsolving.com/community/user/29428}{pco}]
	Hello perfect\_radio,

I don't know $\mathbb Z_{>0}$. I assume it is ${1,2,3, ....}$.

 \begin{tcolorbox}Prove that there is a unique function $f : \mathbb Z_{>0}\to \mathbb Z_{>0}$ such that
\[f \left( m+f(n) \right) = n+f \left( m+95 \right), \forall m, n \in \mathbb Z_{>0}. \]
\end{tcolorbox}

$f(a)=f(b)$ $\Rightarrow $ $f(m+f(a))=f(m+f(b))$ $\Rightarrow $ $a+f(m+95)=b+f(m+95)$ $\Rightarrow $ $a=b$ $\Rightarrow $ $f(x)$ is injective

$f(1+f(1))=1+f(96)$ $\Rightarrow $ $f(1+f(1)+f(n+1))=n+1+f(1+f(1))=n+2+f(96)$
$f(1+f(2))=2+f(96)$ $\Rightarrow $ $f(1+f(2)+f(n))=n+f(1+f(2))=n+2+f(96)$
So $f(1+f(1)+f(n+1))=f(1+f(2)+f(n))$ and since f is injective, $1+f(1)+f(n+1)=1+f(2)+f(n)$ and $f(n+1)=f(n)+f(2)-f(1)$.

Hence $f(n)=kn+b$ and so $f(m+f(n))=km+k^{2}n+kb+b$ and $n+f(m+95)=n+km+95k+b$
So $km+k^{2}n+kb+b=n+km+95k+b$ and $(k^{2}-1)n+k(b-95)=0$ $\Rightarrow $ $k=1$ ($f>0$) and $b=95$

So, $f(n)=n+95$

 \begin{tcolorbox}Compute $\sum_{k=1}^{19}f(k)$.\end{tcolorbox}
$\sum_{k=1}^{19}f(k)=1995$

-- 
Patrick
\end{mysolution}



\begin{mysolution}[by \href{https://artofproblemsolving.com/community/user/5820}{N.T.TUAN}]
	\begin{tcolorbox}Prove that there is a unique function $f : \mathbb Z_{>0}\to \mathbb Z_{>0}$ such that
\[f \left( m+f(n) \right) = n+f \left( m+95 \right), \forall m, n \in \mathbb Z_{>0}. \]
Compute $\sum_{k=1}^{19}f(k)$.\end{tcolorbox}
1) $f$ is injective.
2) $f(f(m)+f(n))=n+f(95+f(m))=n+m+f(2\cdot 95)$ therefore $f(f(n)+f(n+2))=f(2f(n+1))$, by injective we have $f(n+2)+f(n)=2f(n+1)$.

And that is all!
\end{mysolution}
*******************************************************************************
-------------------------------------------------------------------------------

\begin{problem}[Posted by \href{https://artofproblemsolving.com/community/user/5820}{N.T.TUAN}]
	Given $n \in\mathbb{N}$, find all continuous functions $f : \mathbb{R}\to \mathbb{R}$ such that for all $x\in\mathbb{R},$
\[\sum_{k=0}^{n}\binom{n}{k}f(x^{2^{k}})=0. \]
	\flushright \href{https://artofproblemsolving.com/community/c6h148365}{(Link to AoPS)}
\end{problem}



\begin{mysolution}[by \href{https://artofproblemsolving.com/community/user/16261}{Rust}]
	Your condition is equivalent to $f(x)+f(x^{2})=0$ (consider $x\to x^{2^{m}}, m=0,1,...,n$).
If $g(y)=f(e^{y})$, then $g(y)+g(2y)=0$, if $g(\pm 2^{t})=s_{\pm}(t)$, then $s_{\pm}(t+1)=-s_{\pm}(t)$ ($s_{\pm}(t)$ periodic with period 2).
For all $s_{\pm}(t), \ t\in [0,1)$ we get the solution $f(x)$.
\end{mysolution}



\begin{mysolution}[by \href{https://artofproblemsolving.com/community/user/29428}{pco}]
	\begin{tcolorbox}Given $n \in\mathbb{N}$, find all continuous functions $f : \mathbb{R}\to \mathbb{R}$ such that for all $x\in\mathbb{R},$
\[\sum_{k=0}^{n}\binom{n}{k}f(x^{2^{k}})=0. \]
\end{tcolorbox}

$\sum_{k=0}^{n}\binom{n}{k}f(x^{2^{k}})=0$ $\Leftrightarrow$ $\sum_{k=0}^{n-1}\binom{n-1}{k}f(x^{2^{k}})+\sum_{k=0}^{n-1}\binom{n-1}{k}f(x^{2^{k+1}})=0$.
If I call $g(x)=\sum_{k=0}^{n-1}\binom{n-1}{k}f(x^{2^{k}})$, I have $g(x^{2})=-g(x)$ and so $g(x)=g(x^{4^{-p}})$ for $x\geq 0$ and so, with continuity (when p$\rightarrow+\infty$) $g(x)=c$ for any $x\geq 0$ and, since $g(x)=-g(x^{2})$ for any x, we have $g(x)=c$ for any $x\geq 0$x, and $g(x)=-c$ for any $x\leq 0$ and so $g(x)=0$ for any x

So $\sum_{k=0}^{n-1}\binom{n-1}{k}f(x^{2^{k}})=0$

So any solution for rank n is also solution for rank $n-1$
The only solution at rank 0 is $f(x)=0$

So the only solution at rank n is $f(x)=0$.

-- 
Patrick
\end{mysolution}



\begin{mysolution}[by \href{https://artofproblemsolving.com/community/user/16261}{Rust}]
	Why $f(x)+f(x^{2})$ gives $f(x)\equiv 0$?
\end{mysolution}



\begin{mysolution}[by \href{https://artofproblemsolving.com/community/user/29428}{pco}]
	\begin{tcolorbox}Why $f(x)+f(x^{2})$ give $f(x)\equiv 0$?\end{tcolorbox}

As I said : $g(x)+g(x^{2})=0$ $\Rightarrow $ $g(x^{2})=-g(x)$ $\Rightarrow $ $g(x)=-g(x^{\frac{1}{2}})$ for $x\geq 0$ $\Rightarrow $ $g(x)=g(x^{\frac{1}{4}})$ for any $x\geq 0$.
Hence $g(x)=g(x^{4^{-p}})$ for any $x\geq 0$.
When $p\rightarrow+\infty$ and because g is continuous, $g(x)=g(1)$ for any $x>0$.
Since $g(x)=-g(x^{2})$ for any x, we then have $g(x)=-g(1)$ for any $x<0$
with the continuity at 0, the result is $g(x)=0$

-- 
Patrick
\end{mysolution}



\begin{mysolution}[by \href{https://artofproblemsolving.com/community/user/16261}{Rust}]
	Oh! I forgot about continiosly.
\end{mysolution}
*******************************************************************************
-------------------------------------------------------------------------------

\begin{problem}[Posted by \href{https://artofproblemsolving.com/community/user/5820}{N.T.TUAN}]
	Find all functions $f : \mathbb{R}\to\mathbb{R}$ such that $f(1) = 1$ and
\[f\left(f(x)y+\frac{x}{y}\right)=xyf(x^{2}+y^{2}), \]
for all real numbers $x$ and $y$ with $y\not = 0$.
	\flushright \href{https://artofproblemsolving.com/community/c6h148711}{(Link to AoPS)}
\end{problem}




\begin{mysolution}[by \href{https://artofproblemsolving.com/community/user/22804}{nayel}]
	i have the following:

let $f(0)=c$. $x=0$ gives $f(cy)=0,\forall y\neq 0$. $x=c$ gives $f(\frac{c}{y})=cyf(c^{2}+y^{2})=0, \forall y\neq 0$.
so $c=0$ or $f(c^{2}+y^{2})=0, \forall y\neq 0$. 

Assume that $c\neq 0$. then $f(cy)=0$ for $y\neq 0$ gives $f(c^{2})=0$. combining this with $f(c^{2}+y^{2})=0$ for $y\neq 0$ we get $f$ is constant and $f\equiv 0$. so this is a solution. now we turn our attention to $c=f(0)=0$.

$x=1$ implies $f(y+\frac{1}{y})=yf(y^{2}+1)$. on the other hand, $y=1$ gives $f(x+f(x))=xf(x^{2}+1)$. from these two equations we get $f(x+\frac{1}{x})=f(x+f(x))$ for $x\neq 0$. from this i inspect $f(x)=0$ for $x=0$, otherwise $f(x)=\frac{1}{x}$. just need to show that $f$ is injective...

probably that didn't help.. :blush:
\end{mysolution}



\begin{mysolution}[by \href{https://artofproblemsolving.com/community/user/4229}{scorpius119}]
	[hide="hmm..."]
We can get $f(0)=0$ because if $f(0)\neq 0$, then we can plug in $x=0,\ y=\frac{1}{f(0)}$ to get $f(1)=0$: a contradiction.

For any nonzero $x$, there exists a nonzero $y$ such that
\[f(x)y+\frac{x}{y}=x^{2}+y^{2}\]
as a cubic equation has at least one real root. Let $c$ be this common value; then we have
\[f(c)=xyf(c)\Rightarrow f(c)(xy-1)=0\]
Now to show that if $f(c)=0$, then $c=0$... :maybe: If we can show this, then we'll be done, since $xy=1$ and $f(x)y+\frac{x}{y}=x^{2}+y^{2}$ implies $f(x)=\frac{1}{x}$.

\end{mysolution}
*******************************************************************************
-------------------------------------------------------------------------------

\begin{problem}[Posted by \href{https://artofproblemsolving.com/community/user/8143}{Jutaro}]
	Given the positive real numbers $a_{1}<a_{2}<\cdots<a_{n}$, consider the function \[f(x)=\frac{a_{1}}{x+a_{1}}+\frac{a_{2}}{x+a_{2}}+\cdots+\frac{a_{n}}{x+a_{n}}\] Determine the sum of the lengths of the disjoint intervals formed by all the values of $x$ such that $f(x)>1$.
	\flushright \href{https://artofproblemsolving.com/community/c6h148816}{(Link to AoPS)}
\end{problem}



\begin{mysolution}[by \href{https://artofproblemsolving.com/community/user/29428}{pco}]
	Hello Jutaro,
\begin{tcolorbox}Given the positive real numbers $a_{1}<a_{2}<\cdots<a_{n}$, consider the function
\[f(x)=\frac{a_{1}}{x+a_{1}}+\frac{a_{2}}{x+a_{2}}+\cdots+\frac{a_{n}}{x+a_{n}}\]
Determine the sum of the lengths of the disjoint intervals formed by all the values of $x$ such that $f(x)>1$.\end{tcolorbox}

Very nice question.

Let $f(x)=\frac{P(x)}{Q(x)}$ with $P(x)=\sum_{i}a_{i}\prod_{j\neq i}(x+a_{j})$ and $Q(x)=\prod_{i}(x+a_{i})$

$f(x)=1$ $\Leftrightarrow$ $Q(x)-P(x)=0$ and this polynomial have exactly n roots $-a_{n}<r_{n}<-a_{n-1}<r_{n-1}<\cdots<-a_{1}<r_{1}$

The number requested is $\sum_{i}(r_{i}+a_{i})= \sum_{i}r_{i}+\sum_{i}a_{i}$
$\sum_{i}r_{i}$ is the opposite of the coefficient of $x^{n-1}$ in $Q(x)-P(x)$
Coefficient of $x^{n-1}$ in $Q(x)$ is $\sum_{i}a_{i}$
Coefficient of $x^{n-1}$ in $P(x)$ is $\sum_{i}a_{i}$
Hence $\sum_{i}r_{i}=0$



The the number requested is $\sum_{i}a_{i}$

-- 
Patrick
\end{mysolution}



\begin{mysolution}[by \href{https://artofproblemsolving.com/community/user/16261}{Rust}]
	$f(x)-1=0\to \frac{Q(x)}{P(x)}=0$, were $P(x)=\prod_{i}(x+a_{i}), \ Q(x)=-P(x)+\sum_{i}a_{i}\frac{P(x)}{x+a_{i}}.$
Because $f(x)\to-\infty$, when $x\to a_{i}-0$, and $f(x)\to+\infty$, when $x\to a_{i}+0$ we have n roots Q(x)$-a_{n}<y_{n}<-a_{n-1}<...<-a_{1}<y_{1}<\infty$.
We calculate ${Q(x)=-x^{n}-\sum_{i}a_{i}x^{n-1}-...+x^{n-1}(a_{1}+...+a_{n}}+...$. It gives $\sum_{i}y_{i}=0$.
Therefore $\sum_{i}L_{i}=\sum_{i}(y_{i}-(-a_{i}))=\sum_{i}a_{i}.$
\end{mysolution}



\begin{mysolution}[by \href{https://artofproblemsolving.com/community/user/43015}{modularmarc101}]
	\begin{tcolorbox}

...
The number requested is $\sum_{i}(r_{i}+a_{i})= \sum_{i}r_{i}+\sum_{i}a_{i}$
...\end{tcolorbox}

Sorry, but I'm having trouble understanding why it is this sum. I know I'm wrong, but I thought it would be something like $(r_1 - r_2) + (r_3 - r_4) + ...$ (that or $(r_2-r_3) + (r_4-r_5) + ... $).

EDIT: Thanks pco!
\end{mysolution}



\begin{mysolution}[by \href{https://artofproblemsolving.com/community/user/29428}{pco}]

No, you need to go back to the original expression of $f(x)$ :

$f((-\infty,-a_n))=(-\infty,0)$

$f((-a_n,r_n))=(1,+\infty)$ and so we get a "good" interval $(-a_n,r_n)$ whose length is $r_n+a_n$
$f((r_n,-a_{n-1}))=(-\infty,1)$

$f((-a_{n-1},r_{n-1}))=(1,+\infty)$ and so we get another "good" interval $(-a_{n-1},r_{n-1})$ whose length is $r_{n-1}+a_{n-1}$
$f((r_{n-1},-a_{n-2}))=(-\infty,1)$
...
$f((-a_{n-k},r_{n-k}))=(1,+\infty)$ and so we get a "good" interval $(-a_{n-k},r_{n-k})$ whose length is $r_{n-k}+a_{n-k}$
$f((r_{n-k},-a_{n-k-1}))=(-\infty,1)$

Hence the result $\sum_{i}(r_{i}+a_{i})$
\end{mysolution}
*******************************************************************************
-------------------------------------------------------------------------------

\begin{problem}[Posted by \href{https://artofproblemsolving.com/community/user/10088}{silouan}]
	Let $n$ be a positive integer. Find all monotone functions $f$ from $\mathbb R$ to $\mathbb R$ such that 
\[f(x+f(y))=f(x)+y^{n},\]
for all $x,y \in \mathbb R$.
	\flushright \href{https://artofproblemsolving.com/community/c6h149165}{(Link to AoPS)}
\end{problem}



\begin{mysolution}[by \href{https://artofproblemsolving.com/community/user/5820}{N.T.TUAN}]
	From hypothesis we have : $f$ is injective.
Choose $y=0$ then $f(x+f(0))=f(x)$, so $f(0)=0$.
Choose $x=0$ we have $f(f(y))=y^{n}$, so $n$ is odd. Therefore $f$ is surjective and $f$ is additive, so $f(x)=ax$, check! We will find $a$.
\end{mysolution}



\begin{mysolution}[by \href{https://artofproblemsolving.com/community/user/11776}{mathisfun1}]
	\begin{tcolorbox}From hypothesis we have $f$ is injective.
Choose $y=0$ then $f(x+f(0))=f(x)$, so $f(0)=0$.
Choose $x=0$ we have $f(f(y))=y^{n}$, so $n$ is odd. Therefore $f$ is surjective and $f$ is additive, so $f(x)=ax$, check! We will find $a$.\end{tcolorbox}

I'm sorry, could you explain how $f(f(y))=y^{n}$ implies $f$ is surjective and $f$ is additive?
\end{mysolution}



\begin{mysolution}[by \href{https://artofproblemsolving.com/community/user/5820}{N.T.TUAN}]
	1)If $n$ is odd then $\{y^{n}|y\in\mathbb{R}\}=\mathbb{R}$, therefore $f$ is surjective.
2)From above we have $f(x+f(y))=f(x)+f(f(y))$ and so $f$ is additive.
\end{mysolution}



\begin{mysolution}[by \href{https://artofproblemsolving.com/community/user/21482}{Yosh...}]
	why $n$ must be odd?

and why if we have $f(x)$ bijective and additive,so we have $f(x)=ax$ ?
thx..
\end{mysolution}



\begin{mysolution}[by \href{https://artofproblemsolving.com/community/user/5820}{N.T.TUAN}]
	\begin{tcolorbox}why $n$ must be odd?\end{tcolorbox}
Because $f(f(y))=y^{n}$ and $f_{o}f$ is increasing!
\end{mysolution}



\begin{mysolution}[by \href{https://artofproblemsolving.com/community/user/29428}{pco}]
	\begin{tcolorbox}Let $n\in N$ . Find all monotone functions f from R to R such that 
$f(x+f(y))=f(x)+y^{n}$\end{tcolorbox}

I don't agree with N.T.TUAN when he says that hypothesis implies that $f(x)$ is an injective function. It's true but not immediate (a monotonous function may be non-injective). By the way, we don't need to show that $f(x)$ is injective.

I suggest a little modification to N.T.TUAn demo (which is rather good indeed) :

$E1(x,y)$ : $f(x+f(y))=f(x)+y^{n}$
$E1(x,0)$ : $f(x+f(0))=f(x)$ and if $f(0)\neq 0$, we have a monotonous periodic function, so a constant and no constant satisfy the problem, so $f(0)=0$
Then $E1(0,x)$ gives $f(f(x))=x^{n}$ and $n$ is odd since $f(x)$ monotonous implies $f(f(x))$ increasing and $x^{n}$ is increasing implies $n$ is odd.
Then $n$ odd and $f(f(x))=x^{n}$ implies $f(x)$ is a surjective function.
$E1(0,y)$ : $f(f(y))=y^{n}$ and so $E1(x,y)$ becomes $f(x+f(y))=f(x)+f(f(y))$ and since $f(x)$ is surjective $f(x+z)=f(x)+f(z)$ and $f(x)$ is additive.

If $f(x)$ is additive, it is well known that $f(x)=f(1)x$ for any $x\in\mathbb{Q}$ and then $f(x)=f(1)x$ for any $x\in\mathbb{R}$ since $f(x)$ is monotonous.

So, if we write $f(1)=a$, $ax+a^{2}y=ax+y^{n}$ and $n=1$ and $a\in\{-1,+1\}$
\end{mysolution}



\begin{mysolution}[by \href{https://artofproblemsolving.com/community/user/5820}{N.T.TUAN}]
	I mean that ''monotone functions''= increasing or decreasing. Therefore $f$ is injective!
\end{mysolution}



\begin{mysolution}[by \href{https://artofproblemsolving.com/community/user/29428}{pco}]
	\begin{tcolorbox}I mean that ''monotone functions''= increasing or decreasing. Therefore $f$ is injective!\end{tcolorbox}

OK, but I think that "monotonic" function is a "non-decreasing" or "non increasing" one.

\end{mysolution}



\begin{mysolution}[by \href{https://artofproblemsolving.com/community/user/5820}{N.T.TUAN}]
	Thanks for your link! Wiki is GOOD!  :P
\end{mysolution}
*******************************************************************************
-------------------------------------------------------------------------------

\begin{problem}[Posted by \href{https://artofproblemsolving.com/community/user/26787}{Zamfirmihai}]
	Build a bijective function $f: [0;1]\to \mathbb R$.
	\flushright \href{https://artofproblemsolving.com/community/c6h149198}{(Link to AoPS)}
\end{problem}



\begin{mysolution}[by \href{https://artofproblemsolving.com/community/user/6551}{perfect\_radio}]
	It's really well-known.

Note that $g(x) = \frac1{x-\frac12}-2$ is a bijection from $\left( \frac12, 1 \right]$ to $\left[ 0, \infty \right)$.

Also, $h(x) = \frac1{x-\frac12}+2$ is a bijection from $\left[ 0, \frac12 \right)$ to $\left(-\infty, 0 \right]$. Define $a_{n}= h^{-1}\left(-\frac1{n}\right)$, for all integers $n \geq 1$.

We define $f$ as follows:

- $f(x) = g(x)$ if $x > \frac12$;
- $f(x) = h(x)$ if $x \in \left[ 0, \frac12 \right) \setminus \left\{ 0, a_{1}, a_{2}, a_{3}, \ldots \right\}$;
- $f(x) =-1$ if $x = \frac12$;
- $f(x) =-\frac12$ if $x = 0$;
- $f \left( a_{n}\right) =-\frac1{n+2}$ for all $n \geq 1$.
\end{mysolution}



\begin{mysolution}[by \href{https://artofproblemsolving.com/community/user/29428}{pco}]
	\begin{tcolorbox}Build a bijective function $f, f: [0;1]\to \mathbb R$\end{tcolorbox}

I have a rather similar solution.
Since it is obvious to have bijection between $(0,1)$ $\leftrightarrow$ $\mathbb{R}$, the problem is to find a bijective function between $[0,1]$ and $(0,1)$. I suggest the following one : 

Let $A=\{2^{-n}\forall n \in \mathbb{N}\cup\{0\}\}$
$f(0)=\frac{1}{2}$
$f(x)=\frac{x}{4}$  ${\forall x\in A}$
$f(x)=x$ everywhere else (in $(0,1]-A$)

-- 
Patrick
\end{mysolution}
*******************************************************************************
-------------------------------------------------------------------------------

\begin{problem}[Posted by \href{https://artofproblemsolving.com/community/user/5820}{N.T.TUAN}]
	Find all pair functions $f,g: \mathbb{N}_{0}\to\mathbb{N}_{0}$ such that \[f(m)-f(n)=(m-n)(g(m)+g(n))\quad \forall m,n\in\mathbb{N}_{0},\]
where $\mathbb{N}_{0}=\{0,1,2,...\}$.
	\flushright \href{https://artofproblemsolving.com/community/c6h149447}{(Link to AoPS)}
\end{problem}



\begin{mysolution}[by \href{https://artofproblemsolving.com/community/user/29428}{pco}]
	Hello N.T.TUAN

 \begin{tcolorbox}Find all pairs functions $f,g: \mathbb{N}_{0}\to\mathbb{N}_{0}$ such that
\[f(m)-f(n)=(m-n)(g(m)+g(n))\; \forall m,n\in\mathbb{N}_{0}, \]
where $\mathbb{N}_{0}=\{0,1,2,...\}$.\end{tcolorbox}

$f(m)-f(n)=(m-n)(g(m)+g(n))$ $\Rightarrow $ $f(m)=m(g(m)+g(0))+f(0)$ $\Rightarrow $ $f(m)-f(n)=m(g(m)+g(0))-n(g(n)+g(0))$
And so : $m(g(m)+g(0))-n(g(n)+g(0))=(m-n)(g(m)+g(n))$, which implies $n(g(m)-g(0))=m(g(n)-g(0))$ and $g(n)=a*n+b$

Then $f(n)=n(g(n)+g(0))+f(0)=an^{2}+2bn+c$

And it is easy to verify that these necessary conditions work.

So the solution is :
$g(n)=a*n+b$
$f(n)=an^{2}+2bn+c$
$a,b,c\in\mathbb{N}_{0}$

-- 
Patrick
\end{mysolution}
*******************************************************************************
-------------------------------------------------------------------------------

\begin{problem}[Posted by \href{https://artofproblemsolving.com/community/user/16261}{Rust}]
	Find all continuous functions $f:\mathbb R\to \mathbb R$ such that $f(2x)=2f^{2}(x)-a$ for all $x \in \mathbb R$ and a fixed real $a$ (example: $a=1$ and $f(x)=\cos x$).
	\flushright \href{https://artofproblemsolving.com/community/c6h149457}{(Link to AoPS)}
\end{problem}



\begin{mysolution}[by \href{https://artofproblemsolving.com/community/user/29428}{pco}]
	\begin{tcolorbox}Find continuous function $R\to R$, satisfying $f(2x)=2f^{2}(x)-a$, a fixed real parameter (for example a=1 f(x)=cosx).\end{tcolorbox}

Hello Rust!
The answer to this problem is strongly linked to the value of $a$ (in fact to the existence of fixed points for $g(x)=2x^{2}-a$).

1) First, it is obvious that we can choose $f(x)$ on $[1,2]$, for example, with the only constraints of continuity and $f(2)=2f^{2}(1)-a$ and then we can define $f(x)$ on $[1,+\infty)$ without difficulty. Same thing for an arbitrary definition on $[-2,-1]$, with continuity and $f(-2)=2f^{2}(-1)-a$, and then complete definition on $(-\infty,-1]$.

2) The problem is to be able to build $f(x)$ on $[0,1)$ and to have the continuity in $0$.

2.1) If $a<-\frac{1}{8}$, the function $g(x)=2x^{2}-a$ has no fixed point and it is rather easy to see that it is impossible to define $f(x)$ about $0$ $\Rightarrow $ no solution.

2.2) If $-\frac{1}{8}\leq a\leq 0$, it is generally possible to build $f(x)$ on $(0,1]$ by using the formula $f(x)=\sqrt{\frac{f(2x)+a}{2}}$.  We must take the positive value of the square root and not the negative one.
One condition for this to work is that the reference value ($f(x)$ in $[1,2]$ or $[-2,-1]$) must be either the lowest fixed point of $g(x)$ (and the function is constant), or $>$ to this lowest fixed point and then $f(0)=$ the highest fixed point of $g(x)$.

2.3) If $0<a<\frac{3}{8}$. it is again possible to build $f(x)$ on $(0,1]$ by using the formula $f(x)=\sqrt{\frac{f(2x)+a}{2}}$.  One condition for this to work is that the reference value ($f(x)$ in $[1,2]$ or $[-2,-1]$) must be $>-a$ and then $f(0)=$ the highest fixed point of $g(x)$.
But it is then possible to use the negative value of the square ($f(x)=-\sqrt{\frac{f(2x)+a}{2}}$ in some conditions). Due to the fact that $g'(x)>-1$ on the lowest fixed point, it is possible to have this lowest fixed point as a value for $f(0)$.

2.4) If $\frac{3}{8}\leq a$, it is impossible (except if $f(x)$ is constant) to have $f(0)=$ lowest fixed point. So $f(0)=$ highest fixed point.
The reference value ($f(x)$ in $[1,2]$ or $[-2,-1]$) must be $>-a$.
It is possible during some steps to use $f(x)=-\sqrt{\frac{f(2x)+a}{2}}$ but, at the end, we must use $f(x)=\sqrt{\frac{f(2x)+a}{2}}$

As a conclusion :
If $a<-\frac{1}{8}$ : no solution. \\
If $-\frac{1}{8}\leq a\leq 0$ : infinite many solutions (starting wih values $>$ lowest fixed point), all non constant solutions having $f(0)=$ highest fixed point.\\
If $0<a<\frac{3}{8}$ : infinite many solutions and two possible values for $f(0)$. \\
If $\frac{3}{8}\leq a$ : infinite many solutions and all non constant solutions having $f(0)=$ highest fixed point.

-- 
Patrick
\end{mysolution}



\begin{mysolution}[by \href{https://artofproblemsolving.com/community/user/16261}{Rust}]
	Ok!
Had these equation nonconstant continiosly periodic solution for $a\ge \frac{3}{8}$?
Is these periodic solution with period $2\pi$ unique (a=1, f(x)=cos x)?
Obviosly, if f(x) is solution, then f(bx) is solution. Therefore I interested about minimal period $2\pi$.
\end{mysolution}



\begin{mysolution}[by \href{https://artofproblemsolving.com/community/user/29428}{pco}]
	\begin{tcolorbox}Ok!
Had these equation nonconstant continiosly periodic solution for $a\ge \frac{3}{8}$?
Is these periodic solution with period $2\pi$ unique (a=1, f(x)=cos x)?
Obviosly, if f(x) is solution, then f(bx) is solution. Therefore I interested about minimal period $2\pi$.\end{tcolorbox}

For the moment, I can show that if $f(x)$ is continuous, non constant and periodic, then $a=1$ :

We have $f(2x)=g(f(x))$ with $g(x)=2x^{2}-a$ and $f(x)$ periodic (period $T$).

Let $p=\frac{1-\sqrt{1+8a}}{4}$ and $q=\frac{1+\sqrt{1+8a}}{4}$ be the two fixed points of $g(x)$ (which exist since $a\geq\frac{3}{8}>-\frac{1}{8}$). 

Since $f(x)$ is continuous and periodic, it exists lower and upper bounds for $f(x)$ : $A\leq f(x)\leq B$ $\forall x\in \mathbb{R}$. But, if $|f(x_{0})|>q$ for some $x_{0}$ it is easy to see that $\lim_{n\rightarrow+\infty}f(2^{n}x_{0})=+\infty$ and so we can conclude $|f(x)|\leq q$ $\forall x\in \mathbb{R}$.

I have shown in the previous post that $f(0)=q$.

If $f(\frac{kT}{2^{n}})=q$ $\forall k,n\in\mathbb{N}$, it is immediate (since f is continuous) that $f(x)=q$ $\forall x$). So, since $f(x)$ is supposed periodic non constant, it exist $k$ and $n$ such that $f(\frac{kT}{2^{n}})\neq q$. But $f(kT)=f(0)=q$. So it exists in $[1,n]$ an integer $i$ such that $f(\frac{kT}{2^{i}})= q$. and $f(\frac{kT}{2^{i+1}})\neq q$. But, if $f(x)=q$, and since $q=2q^{2}-a$, $f(\frac{x}{2})=q$ or $f(\frac{x}{2})=-q$. So $f(\frac{kT}{2^{i+1}})=-q$.

So (remember that $|f(x)|\leq q$ $\forall x\in \mathbb{R}$) lower and upper bounds of $f(x)$ are $-q$ and $+q$ and these values may be reached.
Since $f(x)$ is continuous with values in $[-q,+q]$, it exists $x_{1}$ such that $f(x_{1})=0$. Then $f(2x_{1})=-a$ and so $-a\in[-q,+q]$, so $a\leq q$.
But, since $f(x)=2f^{2}(\frac{x}{2})-a$, we have $f(x)\geq-a$ $\forall x$, and so $-q\geq-a$ and $q\leq a$

So $q=a$, and, since $q=\frac{1+\sqrt{1+8a}}{4}$, we can conclude $a=1$

So $a=1$

-- 
Patrick
\end{mysolution}
*******************************************************************************
-------------------------------------------------------------------------------

\begin{problem}[Posted by \href{https://artofproblemsolving.com/community/user/5820}{N.T.TUAN}]
	Find all functions $ f : \mathbb{R}\to\mathbb{R}$ that satisfy
\[ f (x^{3} + y^{3}) = x^{2}f (x) + yf (y^{2})
\]
for all $ x, y \in\mathbb R.$
	\flushright \href{https://artofproblemsolving.com/community/c6h150112}{(Link to AoPS)}
\end{problem}



\begin{mysolution}[by \href{https://artofproblemsolving.com/community/user/29428}{pco}]
	\begin{tcolorbox}Find all functions $f : \mathbb{R}\to\mathbb{R}$ that satisfy
\[f (x^{3}+y^{3}) = x^{2}f (x)+yf (y^{2}) \]
for all $x, y \in\mathbb R.$\end{tcolorbox}

$x=0$ and $y=0$ $\Rightarrow $ $f(0)=0$

$f (x^{3}+0^{3}) = x^{2}f (x)$
$f (0^{3}+x^{3}) = xf (x^{2})$

So $f(x^{3})=x^{2}f (x)=xf (x^{2})$ and $f (x^{3}+y^{3}) = f (x^{3})+f (y^{3})$

So $f(x+y)=f(x)+f(y)$ 
$y=-x$ $\Rightarrow $ $f(-x)=-f(x)$

Then $x^{2}f (x)=xf (x^{2})$ $\Rightarrow $ $f(x^{2})=xf(x)$ $\forall x\neq 0$
So $f((x+1)^{2})=(x+1)f(x+1)=(x+1)(f(x)+f(1))$ = $xf(x)+f(x)+xf(1)+f(1)$ $\forall x\neq-1$ \\
But $f((x+1)^{2})=f(x^{2}+x+x+1)$ = $f(x^{2})+f(x)+f(x)+f(1)=xf(x)+2f(x)+f(1)$ $\forall x\neq 0$
So $xf(x)+f(x)+xf(1)+f(1) =xf(x)+2f(x)+f(1)$  and $f(x)=xf(1)$ $\forall x\neq0$ and $x\neq-1$

The two cases $x=0$ and $x=-1$ are very easy to check ($f(0)=0$ and $f(-1)=-f(1)$)

So $f(x)=ax$, and it is easy to check that this necessary condition works.
\end{mysolution}





\begin{mysolution}[by \href{https://artofproblemsolving.com/community/user/30326}{quangpbc}]
	The same functional equation
Find all functions $ f :\mathbb{R}\to\mathbb{R}$ that satisfy

$ f (x^{3}-y^{3}) = x^{2}f (x)-yf (y^{2})$
for all $ x, y\in\mathbb R.$ 
\end{mysolution}



\begin{mysolution}[by \href{https://artofproblemsolving.com/community/user/30326}{quangpbc}]
	Nobody want to solve it  :D
\end{mysolution}



\begin{mysolution}[by \href{https://artofproblemsolving.com/community/user/30264}{lasha}]
	Here are the solution of both functional equations:    
$ x = 0$ follows $ f(y^{3}) = yf(y^{2})$ $ (1)$, If $ y = 0$, $ f(x^{3}) = x^{2}f(x)$ $ (2)$. So, $ f(x^{3}+y^{3}) = x^{2}f(x)+yf(y^{2}) = f(x^{3})+f(y^{3})$. So, for any real pair $ (a,b)$ we have: $ f(a+b) = f(a)+f(b)$ $ (3)$. $ x = y$ follows: $ x^{2}f(x) = xf(x^{2})$, $ f(x^{2}) = xf(x)$ $ (4)$. 
$ (3)$ and $ (4)$ gives: $ f(x^{2}+2x+1) = f(x^{2})+f(2x)+f(1) = xf(x)+2f(x)+f(1)$ $ (5)$.
$ (4)$ gives: $ f(x^{2}+2x+1) = f((x+1)^{2}) = (x+1)f(x+1) = (x+1)(f(x)+f(1)) = xf(x)+x+f(x)+f(1)$ $ (6)$.
$ (5)$ and $ (6)$ follows: $ f(x) = xf(1) = kx$, where $ k$ is any real number. So, $ f(x) = kx$.
Here is the second one:
$ y = 0$ follows: $ f(x^{3}) = x^{2}f(x)$ $ (1)$.
$ x = y = 0$ follows $ f(0) = 0$ $ (2)$.
$ x = 0$ follows:$ f(-y^{3}) =-yf(y^{2})$ $ (3)$.
$ x =-y$ and $ (3)$ follows:$ f(x^{3}) = xf(x^{2})$ $ (4)$.
$ (1)$ and $ (4)$ follows:$ f(x^{2}) = xf(x)$ $ (5)$.
$ (1)$ and $ (3)$ follows that for any pair of real numbers $ (z,t)$, $ f(z+t) = f(z)+f(t)$. $ (6)$.
$ f(x^{2}+2x+1) = f(x^{2})+2f(x)+f(1) = xf(x)+2f(x)+f(1)$.
$ f(x^{2}+2x+1) = f((x+1)^{2}) = (x+1)f(x+1) = (x+1)(f(x)+f(1)) = xf(x)+xf(1)+f(x)+f(1)$.
Last two equality gives:$ f(x) = xf(1) = kx$, where $ k$ is any real number. So, $ f(x) = kx$.  :blush:
\end{mysolution}



\begin{mysolution}[by \href{https://artofproblemsolving.com/community/user/30326}{quangpbc}]
	Find all functions $ f:\mathbb{R}\to\mathbb{R}$ so that 

$ f(x^{5}-y^{5})=x^{2}f(x^{3})-y^{2}f(y^{3})$

And the same problem :

1,Find all functions $ f:\mathbb{R}\to\mathbb{R}$ so that 

$ f(x^{5}-y^{5})=x^{4}f(x)-y^{4}f(y)$

2,Find all functions $ f:\mathbb{R}\to\mathbb{R}$ so that 

$ f(x^{5}-y^{5})=x^{3}f(x^{2})-y^{3}f(y^{2})$
\end{mysolution}



\begin{mysolution}[by \href{https://artofproblemsolving.com/community/user/37878}{hsbhatt}]
	Cannot these be solved using Jensen's Inequality
\end{mysolution}



\begin{mysolution}[by \href{https://artofproblemsolving.com/community/user/30326}{quangpbc}]
	\begin{tcolorbox}Cannot these be solved using Jensen's Inequality\end{tcolorbox}
[color=green]Hnm, what do you mean, \begin{bolded}hsbhatt\end{bolded}? What makes you think that we can use Jensen's Inequality to prove them?  :roll:  :o [/color]
\end{mysolution}



\begin{mysolution}[by \href{https://artofproblemsolving.com/community/user/45701}{sargeist}]
	Hi,

Can we just do the following?

We are given:

$ f(x^{3}+y^{3}) = x^{2}f(x) + yf(y^2)$

so substitute $ x=y=0$ to get $ f(0)=0$.

Then, put $ x=0$ and $ y=y$ to get: $ f(y^{3}) = y f(y^{2})$

and $ x=x$ and $ y=0$ to get: $ f(x^{3}) = x^{2}f(x)$.

Add these together and compare with the original equation, to get:

$ f(x^{3}+y^{3}) = f(x^{3}) + f(y^{3})$

which is just $ f(u+v) = f(u)+f(v)$, which is Cauchy's functional equation. This has solution $ f(x)=kx$ for some $ k$.

And if we substitute this into the original equation, we find that this works with any $ k$.

Is this ok? Am I allowed to just say: "Ah, the solution of cauchy's functional equation is known to be 'blah'?"
\end{mysolution}



\begin{mysolution}[by \href{https://artofproblemsolving.com/community/user/25714}{TaiPan~SP!}]
	Well the flaw is that we are working with $ f: \mathbb{R} \rightarrow \mathbb{R}$ and there are more than one function that satisfies $ f(x+y) = f(x) + f(y)$ for $ x,y, \in \mathbb{R}$.
\end{mysolution}



\begin{mysolution}[by \href{https://artofproblemsolving.com/community/user/43906}{Math pro}]
	\begin{tcolorbox}Find all functions $ f: \mathbb{R}\to\mathbb{R}$ so that 

$ f(x^{5} - y^{5}) = x^{2}f(x^{3}) - y^{2}f(y^{3})$

And the same problem :

1,Find all functions $ f: \mathbb{R}\to\mathbb{R}$ so that 

$ f(x^{5} - y^{5}) = x^{4}f(x) - y^{4}f(y)$

2,Find all functions $ f: \mathbb{R}\to\mathbb{R}$ so that 

$ f(x^{5} - y^{5}) = x^{3}f(x^{2}) - y^{3}f(y^{2})$\end{tcolorbox}

From NGUYEN TRONG TUAN 's book,anh QUANG! :oops:
\end{mysolution}



\begin{mysolution}[by \href{https://artofproblemsolving.com/community/user/28498}{harpeng}]
	\begin{tcolorbox}Hi,

Can we just do the following?

We are given:

$ f(x^{3} + y^{3}) = x^{2}f(x) + yf(y^2)$

so substitute $ x = y = 0$ to get $ f(0) = 0$.

Then, put $ x = 0$ and $ y = y$ to get: $ f(y^{3}) = y f(y^{2})$

and $ x = x$ and $ y = 0$ to get: $ f(x^{3}) = x^{2}f(x)$.

Add these together and compare with the original equation, to get:

$ f(x^{3} + y^{3}) = f(x^{3}) + f(y^{3})$

which is just $ f(u + v) = f(u) + f(v)$, which is Cauchy's functional equation. This has solution $ f(x) = kx$ for some $ k$.

And if we substitute this into the original equation, we find that this works with any $ k$.

Is this ok? Am I allowed to just say: "Ah, the solution of cauchy's functional equation is known to be 'blah'?"\end{tcolorbox}

As I know, it is allowed to say "the solution of cauchy's functional equation is known to be" when the function is about relative 

numbers. You need some more conditions(I can't and won't remember what was them) to say when it is about real numbers.
\end{mysolution}






\begin{mysolution}[by \href{https://artofproblemsolving.com/community/user/41491}{mlequi}]
	\begin{tcolorbox}Hi,

Can we just do the following?

We are given:

$ f(x^{3} + y^{3}) = x^{2}f(x) + yf(y^2)$

so substitute $ x = y = 0$ to get $ f(0) = 0$.

Then, put $ x = 0$ and $ y = y$ to get: $ f(y^{3}) = y f(y^{2})$

and $ x = x$ and $ y = 0$ to get: $ f(x^{3}) = x^{2}f(x)$.

Add these together and compare with the original equation, to get:

$ f(x^{3} + y^{3}) = f(x^{3}) + f(y^{3})$

which is just $ f(u + v) = f(u) + f(v)$, which is Cauchy's functional equation. This has solution $ f(x) = kx$ for some $ k$.

And if we substitute this into the original equation, we find that this works with any $ k$.

Is this ok? Am I allowed to just say: "Ah, the solution of cauchy's functional equation is known to be 'blah'?"\end{tcolorbox}

Continuing from this solution, we have $ f(x^3)=xf(x^2)=x^2f(x)$. Thus, we have $ f(x^2)=xf(x)$.
Substituting to the original functional equation, we get $ f(x^3+y^3)=xf(x)^{2}+yf(y)^{2}$....(1)
Now, suppose $ a$ and $ b$ are two real numbers such that $ a \geq b$.
Then substitute $ x= \sqrt[3]{a-b}$ and $ y=\sqrt[3]{b}$ to (1), we have $ f(a)=\sqrt[3]{a-b}f(\sqrt[3]{a-b})^2 + \sqrt[3]{b}f(\sqrt[3]{b})^2$
Since $ f(x^3)=xf(x^2)$, then we have
$ f(a)=\sqrt[3]{a-b}f(\sqrt[3]{a-b})^2 + f(b)$. We have $ f(a) \geq f(b)$.
Thus, the function is increasing.
Now it means we can use the Cauchy equation... right?
Sorry for my bad English
\end{mysolution}



\begin{mysolution}[by \href{https://artofproblemsolving.com/community/user/109774}{littletush}]
	by letting $x=0$ or $y=0$ etc. we can easily get $f(x^2)=xf(x)$ and $f(x^3+y^3)=f(x^3)+f(y^3)$
hence f is a Cauchy's function
then $f(x^2+2x+1)=(x+1)f(x+1)$
hence $(x+2)f(x)+1=(x+1)(f(x)+1)$
yielding $f(x)=x$.
QED
\end{mysolution}



\begin{mysolution}[by \href{https://artofproblemsolving.com/community/user/29428}{pco}]
	\begin{tcolorbox}
yielding $f(x)=x$.\end{tcolorbox}

And what about $f(x)=0$ $\forall x$ ?
And what about $f(x)=2x$ $\forall x$ ?

You should pay more attention when copying previous posts of the thread.
\end{mysolution}



\begin{mysolution}[by \href{https://artofproblemsolving.com/community/user/84677}{andreass}]
	You can't be serious! :(
Me, after proving that $f(x+y)=f(x)+f(y)$, which is true regardless of the number of summands, I said that it immediately follows that $f(nx)=nf(x)$ $\forall n \in N $ and $x \in R$ but since $f(-x)=-f(x)$ and $f(0)=0$ then $f(nx)=nf(x)$ $\forall n \in Z$
Then, I tried to generalise this step by step to n becoming real, by first becoming rational.
$f(nx)=nf(x) \Rightarrow f(x)=\frac{1}{n}f(nx)$ $\forall n\in Z-{0}$ $\Rightarrow f(x)=\frac{1}{n}\times mf(\frac{n}{m}x)$ where $m \in Z$. Hence if we substitute $q=\frac{n}{m}$ we get that $f(x)=\frac{1}{q} f(qx)$ $\forall q \in Q$ i.e. $f(qx)=qf(x) \Rightarrow f(q)=qf(1)$ $\forall q \in Q$.
Now consider the real number $r$. Denote by $[r]$ the floor function and $a_k$ the $k$-th decimal digit of $r$ after the point. Note the obviously every $\frac{a_k}{10^k}$ term will be rational. 
Hence $r=[r]+	\sum_{k=1}^{\infty}\frac{a_k}{10^k} \Rightarrow f(r)=f([r]+	\sum_{k=1}^{\infty}\frac{a_k}{10^k})$
$= [r]f(1)+\sum_{k=1}^{\infty}\frac{a_k}{10^k}f(1)$
$=([r]+\sum_{k=1}^{\infty}\frac{a_k}{10^k})f(1)$
$=rf(1)  \forall r \in R$ 
and finally it is generalized to reals!!!!
Can anyone comment or tell me if I am correct or not?
\end{mysolution}



\begin{mysolution}[by \href{https://artofproblemsolving.com/community/user/53448}{panos\_lo}]
	Andreass you are wrong. I 'll give you two reasons to make you realise it: 
1) Firstly, it's a well known fact that Cauchy's functional equation has also noncontinuous solutions, as we can see using Hamel basis.
2) Yes, additivity implies that the additive property holds regardless of the number of summands, but it doesn't mean that it holds for the case that they are infinite.
\end{mysolution}



\begin{mysolution}[by \href{https://artofproblemsolving.com/community/user/260346}{Takeya.O}]
	\begin{tcolorbox}
which is just $ f(u+v) = f(u)+f(v)$, which is Cauchy's functional equation. This has solution $ f(x)=kx$ for some $ k$.\end{tcolorbox}

You are wrong. :o

\begin{bolded}Cauchy Equation\end{bolded} implies that ∃$c\in \mathbb R$ s.t. $\forall x\in \mathbb Q:f(x)=cx$ not $\forall x\in \mathbb R:f(x)=cx$.

Thanks,
Takeya.O :P


\end{mysolution}



\begin{mysolution}[by \href{https://artofproblemsolving.com/community/user/260346}{Takeya.O}]
	\begin{tcolorbox}Find all functions $ f : \mathbb{R}\to\mathbb{R}$ that satisfy
\[ f (x^{3} + y^{3}) = x^{2}f (x) + yf (y^{2})
\]
for all $ x, y \in\mathbb R.$\end{tcolorbox}

Let $P(x,y)$ be $f(x^3+y^3)=x^2f(x)+yf(y^2)$.

$P(x,0)\rightarrow f(x^3)=x^2f(x),
P(0,x)\rightarrow f(x^3)=xf(x^2)$
$\rightarrow f(0)=0$ and $\forall x\neq 0,x^2f(x)=xf(x^2)\Leftrightarrow f(x^2)=xf(x)$ which also holds at $x=0$.

Thus
$f(x^3+y^3)=f(x^3)+f(y^3)\rightarrow
\forall a,\forall b\in \mathbb R:f(a+b)=f(a)+f(b)$.
Then $\forall x\in \mathbb R,\forall k\in \mathbb Q:f(kx)=kf(x)$.

On the other hand,for $\forall x\in \mathbb R,\forall k\in \mathbb Q$,
$f((x+k)^3)=(x+k)^2f(x+k)$.$LHS=f(x^3+3x^2k+3xk^2+k^3)=f(x^3)+3kf(x^2)+3k^2f(x)+k^3f(1)$.$RHS=(x^2+2xk+k^2)(f(x)+kf(1))=x^2f(x)+x^2f(1)k+2xf(x)k+2xf(1)k^2+f(x)k^2+f(1)k^3$.Since $LHS=RHS$,
$2(f(x)-f(1)x)k^2+(xf(x)-f(1)x^2)k=0$.For fixed $x\in \mathbb R$,this holds at infinitely many $k\in \mathbb Q$.Hence $f(x)-f(1)x=0\Leftrightarrow f(x)=f(1)x$.Conversely $\forall a\in \mathbb R,f(x)=ax(\forall x\in \mathbb R)$ which satisfies the condition.

Therefore the answer is
$\boxed{\forall a\in \mathbb R:f(x)=ax(\forall x\in \mathbb R)}\blacksquare$ :coool: 
\end{mysolution}
*******************************************************************************
-------------------------------------------------------------------------------

\begin{problem}[Posted by \href{https://artofproblemsolving.com/community/user/5820}{N.T.TUAN}]
	Let $n$ be a natural number divisible by $4$. Determine the number of bijections $f$ on the set $\{1,2,...,n\}$ such that $f (j )+f^{-1}(j ) = n+1$ for $j = 1,..., n.$
	\flushright \href{https://artofproblemsolving.com/community/c6h150114}{(Link to AoPS)}
\end{problem}



\begin{mysolution}[by \href{https://artofproblemsolving.com/community/user/29428}{pco}]
	$ j = f(x)$ $ \Rightarrow$ $ f(f(x)) = n + 1 - x$ (notice here that $ f(x)\neq x$ since n is even).

So, if I call $ y = f(x)$, we have, applying $ f()$ many times,  4-numbers cycles $ x,y,n + 1 - x, n + 1 - y, x, y, n + 1 - x, n + 1 - y, ...)$

In each cycle, the four numbers are different and exactly two of them are in $ \{1,2,...,\frac {n}{2}\}$.
There are $ (\frac {n}{2} - 1)(\frac {n}{2} - 3)...3*1$ ways for grouping $ (x,y)$s and each of the $ \frac {n}{4}$ couples may lead to 2 cycles $ (x,y,n + 1 - x,n + 1 - y)$ or $ (x,n + 1 - y,n + 1 - x,y)$

So the requested number is $ ((\frac {n}{2} - 1)(\frac {n}{2} - 3)...3*1)2^{\frac {n}{4}}$
And, since 
$ (\frac {n}{2} - 1)(\frac {n}{2} - 3)...3*1 = \frac {(\frac {n}{2})!}{2^{\frac {n}{4}}(\frac {n}{4})!}$, the result is $ \frac {(\frac {n}{2})!}{(\frac {n}{4})!}$
-- 
Patrick
\end{mysolution}
*******************************************************************************
-------------------------------------------------------------------------------

\begin{problem}[Posted by \href{https://artofproblemsolving.com/community/user/5820}{N.T.TUAN}]
	Let $\mathbb{R}^{*}$ denote the set of nonzero real numbers. Find all functions $f: \mathbb{R}^{*}\to\mathbb{R}^{*}$ such that \[f(x^{2}+y)=f^{2}(x)+\frac{f(xy)}{f(x)}\; \forall x,y\in\mathbb{R}^{*},x^{2}+y\in\mathbb{R}^{*}.\]
	\flushright \href{https://artofproblemsolving.com/community/c6h151360}{(Link to AoPS)}
\end{problem}



\begin{mysolution}[by \href{https://artofproblemsolving.com/community/user/29428}{pco}]
	\begin{tcolorbox}Let $\mathbb{R}^{*}$ denote the set of nonzero real numbers. Find all functions $f: \mathbb{R}^{*}\to\mathbb{R}^{*}$ such that
\[f(x^{2}+y)=f^{2}(x)+\frac{f(xy)}{f(x)}\; \forall x,y\in\mathbb{R}^{*},x^{2}+y\in\mathbb{R}^{*}. \]
\end{tcolorbox}

P1 : $f(x^{2}+y)=f^{2}(x)+\frac{f(xy)}{f(x)}\; \forall x,y\in\mathbb{R}^{*},x^{2}+y\in\mathbb{R}^{*}$

1) P2 : $f(x^{2}+1)=f^{2}(x)+1\; \forall x,\in\mathbb{R}^{*}$
Easy with P1 and $y=1$

2) $f(-x)=-f(x)$ $\forall x\neq 0$
$-x$ in P2 $\Rightarrow$ $f((-x)^{2}+1)=f^{2}(-x)+1=f(x^{2}+1)=f^{2}(x)+1$ $\Rightarrow$ $f(-x)=\pm f(x)$
Let then $\phi=\frac{1+\sqrt{5}}{2}$, we have $\phi^{2}=\phi+1$
If $f(-\phi)=f(\phi)$, then take $x=\phi$ and $y=-1$ in P1 and $f(\phi^{2}-1)=f^{2}(\phi)+1$ ,$\Rightarrow$ $f(\phi)=f^{2}(\phi)+1$ which have no real solution $\Rightarrow$ $f(-\phi)=-f(\phi)$ 
Then P1 gives : $f(\phi^{2}+\frac{x}{\phi})=f^{2}(\phi)+\frac{f(x)}{f(\phi)}$ but also $f((-\phi)^{2}+\frac{x}{\phi})=f^{2}(-\phi)+\frac{f(-x)}{f(-\phi)}$ and so $\frac{f(x)}{f(\phi)}=\frac{f(-x)}{f(-\phi)}$ and so $f(-x)=-f(x)$. Q.E.D.

3) $f(1)=1$
Let $a=f(1)$, $b=f(2)$ and $c=f(3)$
Then $x=1$ and $y=1$ in P1 gives $b=a^{2}+1$
Then $x=1$ and $y=2$ in P1 gives $c=a^{2}+\frac{b}{a}$
Then $x=2$ and $y=-1$ in P1 gives $c=b^{2}-1$
This system is easy to solve ang gives $f(1)=1$

4) $f(x+1)=f(x)+1$ $\forall$ $x\neq 0$ and $x\neq-1$
Just use P1 with $1$ and $x$

5) $f(x^{2})=f^{2}(x)$ $\forall$ $x\neq 0$ (and, as a consequence, $f(x)\geq 0$ $\forall$ $x>0$)
Immediate from points 1) and 4) above.

6) $f(a+b)=f(a)+f(b)$ for $a\neq 0$, $b\neq 0$, $a+b\neq 0$
P1 gives $f(x^{2}+y)=f^{2}(x)+\frac{f(xy)}{f(x)}$
Using $y+1$ instead of $y$, P1 gives $f(x^{2}+y+1)=f^{2}(x)+\frac{f(xy+x)}{f(x)}$
And, since (from 4) above), $f(x^{2}+y+1)=f(x^{2}+y)+1$ for $x^{2}+y\neq 0,-1$ :
$f^{2}(x)+\frac{f(xy+x)}{f(x)}=f^{2}(x)+\frac{f(xy)}{f(x)}+1$ and $f(xy+x)=f(xy)+f(x)$
Taking $x=b$ and $y=\frac{a}{b}$ gives the result.
The problem of "$x^{2}+y\neq 0,-1$", if  $b^{2}+\frac{a}{b}=0,-1$ may be solved (I've not checked) by using $y=b$ and $x=\frac{a}{b}$ or $y-1$ and not $y+1$ in P1.

7) $f(x)=x$
$f(a+b)=f(a)+f(b)$  gives easily $f(x)=x$ $\forall x\in\mathbb{Q}^{*}$
And since $f(b)\geq 0$ for $b>0$ (see 5)), then $f(x)$ is monotonous increasing and so $f(x)=x$ $\forall x\in\mathbb{R}^{*}$

And it is immediate to verify that this necessary condition works.
\end{mysolution}
*******************************************************************************
-------------------------------------------------------------------------------

\begin{problem}[Posted by \href{https://artofproblemsolving.com/community/user/22328}{sinajackson}]
	Find all real numbers $\alpha$ for which there exists a non constant function $f: \mathbb{R}\rightarrow\mathbb{R}$ such that for all $x,y\in\mathbb{R}$, we have
\[f(\alpha(x+y))=f(x)+f(y).\]
	\flushright \href{https://artofproblemsolving.com/community/c6h152295}{(Link to AoPS)}
\end{problem}



\begin{mysolution}[by \href{https://artofproblemsolving.com/community/user/14130}{Hawk Tiger}]
	\begin{tcolorbox}Find all numbers like $\alpha$ such that there exist a non constant function $f: \mathbb{R}\rightarrow\mathbb{R}$ where for $x,y\in\mathbb{R}$ we have:
$f(\alpha(x+y))=f(x)+f(y)$\end{tcolorbox}
The only possible $\alpha$ is $1$.
Let $x=y=0$ We get that $f(0)=0$
Let $x=0$ we get that $f(\alpha y)=f(y)$ i.e.$f(y)=f(\frac{1}{\alpha}y)$($\alpha\not=0$ is obvious)......................$(1)$
Then let $x=\frac{1}{\alpha}u,y=\frac{1}{\alpha}v$
We get that $f(x+y)=f(\frac{1}{\alpha}x)+f(\frac{1}{\alpha}y)=f(x)+f(y)$ (from $(1)$).................................$(2)$
We assume that $\alpha\not=1$
In sake of contradition,we consider $f(x)-f(y)$
\[f(x)-f(y)=f(\frac{(\alpha-1)x-y}{\alpha-1}+\frac{\alpha y}{\alpha-1})-f(y) =f(\frac{(\alpha-1)x-y}{\alpha-1})+f(\frac{\alpha y}{\alpha-1})-f(y) \]
(becuase of $(2)$)
\[=f(\frac{(\alpha-1)x-y}{\alpha-1})+f(\frac{y}{\alpha-1})-f(y) \]
(because of $(1)$)
\[{=f(\frac{(\alpha-1)x-y}{\alpha-1}+\frac{y}{\alpha-1}})-f(y)=f(y)-f(y)=0 \]
(becuase of $(2)$)
Then $f(x)=f(y)$, contradition!
\end{mysolution}



\begin{mysolution}[by \href{https://artofproblemsolving.com/community/user/29428}{pco}]
	\begin{tcolorbox}Find all numbers like $\alpha$ such that there exist a non constant function $f: \mathbb{R}\rightarrow\mathbb{R}$ where for $x,y\in\mathbb{R}$ we have:
$f(\alpha(x+y))=f(x)+f(y)$\end{tcolorbox}

If $\alpha\neq 1$, Let $y=\frac{-\alpha x}{\alpha-1}$ then $\alpha(x+y)=y$ and $f(y)=f(x)+f(y)$ and $f(x)=0$ $\forall x$

So the only value is $\alpha=1$ (which allows for example $f(x)=x$)
\end{mysolution}
*******************************************************************************
-------------------------------------------------------------------------------

\begin{problem}[Posted by \href{https://artofproblemsolving.com/community/user/22328}{sinajackson}]
	Find all continuous functions $f: \mathbb{R}\rightarrow\mathbb{R}$ such that for all $x\in\mathbb{R}$ we have $f(x)=f(x^{2}+c)$, where $c\in \mathbb R$ is a constant number!
	\flushright \href{https://artofproblemsolving.com/community/c6h152296}{(Link to AoPS)}
\end{problem}



\begin{mysolution}[by \href{https://artofproblemsolving.com/community/user/16261}{Rust}]
	For $x\in R$ define set $C(x)=\{x_{0}=x,x_{n+1}=x_{n}^{2}+c\}$
For any x,y and $\epsilon$ exist $z,t, |z|\le |c|+1,|t|\le |c|+1$, suth that $x\in C_{z}, y\in C_{t}, |z-t|<\epsilon$.
Therefore only constant functions work.
\end{mysolution}



\begin{mysolution}[by \href{https://artofproblemsolving.com/community/user/29428}{pco}]
	\begin{tcolorbox} Therefore only constant functions work.\end{tcolorbox}

Surely not.
If $c>\frac{1}{4}$, for example, there are infinety many nonconstant solutions :

Let g(x) be any continuous function defined on $[0,c]$ and such that $g(c)=g(0)$
Then let $a_{i}$ be :
$a_{0}=0$
$a_{n+1}=a_{n}^{2}+c$ $\forall n\geq 0$

Since $c>\frac{1}{4}$, $a_{n}$ is strictly increasing and $\lim_{n\rightarrow+\infty}a_{n}=+\infty$

Then we can define $f(x)$ :
For $x\in[a_{0},a_{1})$ $f(x)=g(x)$
For $x\in[a_{n},a_{n+1})$ $f(x)=f(\sqrt{x-c})$ $\forall n>0$
For $x<0$, $f(x)=f(-x)$
\end{mysolution}
*******************************************************************************
-------------------------------------------------------------------------------

\begin{problem}[Posted by \href{https://artofproblemsolving.com/community/user/30451}{Chessy}]
	Find all functions $f : \mathbb R \to \mathbb R$ such that
\[f(x)+f\left(1-\frac{1}{x}\right)=x,\]
holds for all real $x$.
	\flushright \href{https://artofproblemsolving.com/community/c6h152490}{(Link to AoPS)}
\end{problem}



\begin{mysolution}[by \href{https://artofproblemsolving.com/community/user/5820}{N.T.TUAN}]
	First, this is a topic has bad title : http://www.mathlinks.ro/Forum/viewtopic.php?t=144403 . But I'll not lock it because you are a new member   

2-nd, I will help you :D . If put $g(x)=1-\frac{1}{x}$ then $g(g(g(x)))=x$ and that's all!  
\end{mysolution}



\begin{mysolution}[by \href{https://artofproblemsolving.com/community/user/29428}{pco}]
	\begin{tcolorbox}What is $f(x)$ in this equation, can anyone solve it in a "step by step" way for me? Thank you !

$f(x)+f(1-\frac{1}{x})=x$\end{tcolorbox}

To go further in the remark of N.T.TUAN, you have :

E1 : $f(x)+f(g(x))=x$ with $g(x)=1-\frac{1}{x}$. So you have :
E2 : $f(g(x))+f(g(g(x)))=g(x)$ and you look at $g(g(x))=\frac{1}{1-x}$ (nothing special)
E3 : $f(g(g(x)))+f(g(g(g(x))))=g(g(x))$ and then you discover that $g(g(g(x)))=x$ So you have :
E3 : $f(g(g(x)))+f(x)=g(g(x))$ 

Then E1-E2+E3 gives : $2f(x)=x-g(x)+g(g(x))$ and then $f(x)=\frac{1}{2}(x-1+\frac{1}{x}+\frac{1}{1-x})$
And it is easy to check that this necessary condition works.
\end{mysolution}



\begin{mysolution}[by \href{https://artofproblemsolving.com/community/user/30451}{Chessy}]
	Thank you very much sir, you have been a great help.

Take care
\end{mysolution}
*******************************************************************************
-------------------------------------------------------------------------------

\begin{problem}[Posted by \href{https://artofproblemsolving.com/community/user/21688}{Zaratustra}]
	Find all functions $f: \mathbb{R}\rightarrow \mathbb{R}$ which satisfy
\[f(0)=2f^{2}(x)-f(2x) \quad \forall x \in \mathbb R.\]
	\flushright \href{https://artofproblemsolving.com/community/c6h153054}{(Link to AoPS)}
\end{problem}



\begin{mysolution}[by \href{https://artofproblemsolving.com/community/user/29428}{pco}]
	\begin{tcolorbox}Find all functions $f: \mathbb{R}\rightarrow \mathbb{R}$ who satisfy:

$f(0)=2f^{2}(x)-f(2x)$\end{tcolorbox}

This is a rather frequent problem whose general form is $f(2x)=h(f(x))$ (here $h(x)=2x^{2}-a$)

In this case, you write :
1) $f(0)=h(f(0))$, so $f(0)$ is any fixed point of h. In this special case, $f(0)=0$ or $f(0)=1$
2) for $x>0$ $f(x)=g(\frac{\ln(x)}{\ln(2)})$ with $g(x)$ such that $g(x+1)=h(g(x))$, so $g(x+p)=h^{\circ p}(g(x))$ (where $f^{\circ p}(x)$ means $f\circ f\circ f\circ \ldots \circ f(x)$ (p times)).

So you just have to choose any function $a(x)$ defined on $(0,1]$ and then $g(x)=h^{\circ\lfloor x\rfloor}(a(x-\lfloor x\rfloor)$

But CAUTION : in order $h^{\circ p}(x)$ exists for any $p<0$, the best case is $h(x)$ bijective function. When this is not the case, special attention must be made on these "negative" compositions.

3) For $x<0$ $f(x)=g(\frac{\ln(-x)}{\ln(2)})$ with $g(x)$ such that $g(x+1)=h(g(x))$ ...

So the general answer is :
Let $a(x)$ and $b(x)$ be any functions defined on $(0,1]$. Then the solutions to the equation $f(2x)=h(f(x))$ are :

1) $f(0)$ is any fixed point of $h(x)$
2) For $x>0$, $f(x)=h^{\circ\lfloor\frac{\ln(x)}{\ln(2)}\rfloor}(a(\frac{\ln(x)}{\ln(2)}-\lfloor\frac{\ln(x)}{\ln(2)}\rfloor)$
3) For $x<0$, $f(x)=h^{\circ\lfloor\frac{\ln(-x)}{\ln(2)}\rfloor}(b(\frac{\ln(-x)}{\ln(2)}-\lfloor\frac{\ln(-x)}{\ln(2)}\rfloor)$

In our special case, there are two subtle modification :
1) Let $f(0)=a$ with either $a=0$, either $a=1$ and let $h(x)= 2x^{2}-a$

We need to define $h^{\circ-1}(x)$ and I suggest $h^{\circ-1}(x)=\sqrt{\frac{x+a}{2}}$ and so we need $a(x)\geq 0$ and $b(x)\geq 0$ (for $a=1$, this choice may lead to forget some solutions, I think).

2) For $x>0$, $f(x)=h^{\circ\lfloor\frac{\ln(x)}{\ln(2)}\rfloor}(a(\frac{\ln(x)}{\ln(2)}-\lfloor\frac{\ln(x)}{\ln(2)}\rfloor)$
3) For $x<0$, $f(x)=h^{\circ\lfloor\frac{\ln(-x)}{\ln(2)}\rfloor}(b(\frac{\ln(-x)}{\ln(2)}-\lfloor\frac{\ln(-x)}{\ln(2)}\rfloor)$

If the continuity is demanded, some conditions more on $a(x)$ and $b(x)$ exist.
\end{mysolution}
*******************************************************************************
-------------------------------------------------------------------------------

\begin{problem}[Posted by \href{https://artofproblemsolving.com/community/user/28048}{A3K08}]
	Find all functions $f: \mathbb N\to \mathbb N$ such that
\[f(f(n))+f(n+1)=n+2, \quad \forall n\in \mathbb N.\]
	\flushright \href{https://artofproblemsolving.com/community/c6h153079}{(Link to AoPS)}
\end{problem}



\begin{mysolution}[by \href{https://artofproblemsolving.com/community/user/29428}{pco}]
	\begin{tcolorbox}Find all functions $f: N\to N$ such that: 
$f(f(n))+f(n+1)=n+2, \forall n\in N$\end{tcolorbox}

Nice question.
We have $P(n)$ : $f(f(n))+f(n+1)=n+2$

We have obviously $f(n+1)<n+2$ $\forall n\in N$, so $f(n)\leq n$ $\forall n>1$
Let $a=f(1)$

$P(1) \implies f(a)+f(2)=3$ and then $f(2)$ may be $1$ or $2$

1) $f(2)=1$
We have then : (i)$f(1)=a$, (ii)$f(2)=1$ and (iii)$f(a)=2$
Then $P(2)\implies f(f(2))+f(3)=4 \implies$ (iv)$f(3)=4-a$. Then a can only be $1,2$ or $3$.
$a=1 \implies$ (i) and (iii) are in contradiction.
$a=2 \implies$ (ii) and (iii) are in contradiction.
$a=3 \implies$ (iii) and (iv) are in contradiction.
So $f(2)\neq 1$

2) then $f(2)=2$
We have then : $f(1)=a$, $f(2)=2$ and $f(a)=1$
Then $P(2)\implies f(f(2))+f(3)=4 \implies f(3)=2$.
Then $P(3)\implies f(f(3))+f(4)=5 \implies f(4)=3$.
Then $P(4)\implies f(f(4))+f(5)=6 \implies f(5)=4$.

By induction, it is easy to show that $n>f(n)\geq 2$ $\forall n\geq 3$ :
a) : It's OK for $n=3$ : $3>f(3)=2\geq 2$
b) : If it's OK $\forall k\in [3,n]$, then $P(n) \implies f(n+1)=n+2-f(f(n))$
Then, if $f(n)=2$, $f(n+1)=n$ and $n+1>f(n+1)\geq 2$ 
Else, if $f(n)>2$, $f(n)\in[3,n)$ and so $f(n)>f(f(n))\geq 2$ and so $n\geq n+2-f(f(n))> n+2-f(n)>2$ and so $n+1> f(n+1)>2$
and the induction is verified.

So, $f(n)\geq 2 \forall n\geq 2$ and since $f(a)=1$, we must have $a=1$

Then, since we have $f(n)\leq n$ $\forall n>1$ and $f(1)=1\leq 1$, we have $f(n)\leq n$ $\forall n>0$

And so, we have a unique solution completely defined by :
$f(1)=1$
$f(n)=n+1-f(f(n-1))$ $\forall n>1$
Which gives values $1,2,2,3,4,4,5,5,6,7,7,...$

I don't know if there is a closed form for this unique solution but it's near of $\frac{-1+\sqrt{5}}{2}n$

I'll continue to search for such a closed form.
\end{mysolution}



\begin{mysolution}[by \href{https://artofproblemsolving.com/community/user/29428}{pco}]
	\begin{tcolorbox} And so, we have a unique solution completely defined by :
$f(1)=1$
$f(n)=n+1-f(f(n-1))$ $\forall n>1$
Which gives values $1,2,2,3,4,4,5,5,6,7,7,...$

I don't know if there is a closed form for this unique solution but it's near of $\frac{-1+\sqrt{5}}{2}n$

I'll continue to search for such a closed form.\end{tcolorbox}

Ok, the closed form for the unique solution  is $f(n)=1+\lfloor\frac{-1+\sqrt{5}}{2}n\rfloor$

Demo by induction :
Let $\alpha=\frac{-1+\sqrt{5}}{2}=0.618\ldots$. Notice that we have $\alpha^{2}+\alpha-1=0$
a) It's true for $n=1$ : $f(1)=1=1+\lfloor\alpha\rfloor$

b) assume it's true for any $0<k\leq n$
Then we have $f(n+1)=n+2-f(f(n))=n+2-1-\lfloor\alpha f(n)\rfloor=n+1-\lfloor\alpha+\alpha\lfloor\alpha*n\rfloor\rfloor$
We must show that $f(n+1)=1+\lfloor\alpha*n+\alpha\rfloor$
So we must show that $\lfloor\alpha*n+\alpha\rfloor=n-\lfloor\alpha+\alpha\lfloor\alpha*n\rfloor\rfloor$
Or : $n = \lfloor\alpha*n+\alpha\rfloor+\lfloor\alpha+\alpha\lfloor\alpha*n\rfloor\rfloor$

This equality is rather easy to verify:
Let $\alpha*n=k+r$ with $r\in(0,1)$
Then $S=\lfloor\alpha*n+\alpha\rfloor+\lfloor\alpha+\alpha\lfloor\alpha*n\rfloor\rfloor=\lfloor k+r+\alpha\rfloor+\lfloor\alpha+\alpha*k\rfloor$
So $S=\lfloor k+r+\alpha\rfloor+\lfloor\alpha+\alpha^{2}*n-\alpha*r\rfloor$
So $S=\lfloor k+r+\alpha\rfloor+\lfloor\alpha+n-\alpha*n-\alpha*r\rfloor$
So $S=\lfloor k+r+\alpha\rfloor+\lfloor\alpha+n-k-r-\alpha*r\rfloor$

Then :
1) $r=1-\alpha$ is impossible since we would have $\alpha*n=k+1-\alpha$ and $\alpha$ would be rational.
2) If $0<r<1-\alpha$, then $r(1+\alpha)<1-\alpha^{2}=\alpha$ then $0<\alpha-r-\alpha*r<\alpha<1$ and we have
$\lfloor k+r+\alpha\rfloor=k$
$\lfloor\alpha+n-k-r-\alpha*r\rfloor=n-k$
And $S=n$
3) If $2>r+\alpha>1$, then $r(1+\alpha)>1-\alpha^{2}=\alpha$ then $-1<-r<\alpha-r-\alpha*r<0$ and we have
$\lfloor k+r+\alpha\rfloor=k+1$
$\lfloor\alpha+n-k-r-\alpha*r\rfloor=n-k-1$
And $S=n$
Q.E.D.
\end{mysolution}



\begin{mysolution}[by \href{https://artofproblemsolving.com/community/user/28048}{A3K08}]
	I have tried for two hours to prove this problem but I not sure is correct.
I find that $f(n)=[\frac{1+\sqrt{5}}{2}n]-n+1$ and I used the following results:
$i)$ For each $n\in N$, $[\alpha([n\alpha]-n+1)]=n$ or $n+1$
$ii)$For each $n\in N$, $[(n+1)\alpha]=[n\alpha]+2$ if $[\alpha([n\alpha]-n+1)]=n$
where $\alpha=\frac{1+\sqrt{5}}{2}$
\end{mysolution}
*******************************************************************************
-------------------------------------------------------------------------------

\begin{problem}[Posted by \href{https://artofproblemsolving.com/community/user/28048}{A3K08}]
	Find all functions $f: \mathbb N\to \mathbb N$ such that for all positive integers $n$, we have
\[f(n)+f(n+1)=f(n+2)f(n+3)-k,\]
where $k=p-1$ for some prime $p$.
	\flushright \href{https://artofproblemsolving.com/community/c6h153287}{(Link to AoPS)}
\end{problem}



\begin{mysolution}[by \href{https://artofproblemsolving.com/community/user/29428}{pco}]
	\begin{tcolorbox}Find all $f: N\to N$ such that:
$f(n)+f(n+1)=f(n+2)f(n+3)-k$, where $k=p-1$ for some prime $p$\end{tcolorbox}
Quite nice !


$E1(n)$ : $f(n)+f(n+1)=f(n+2)f(n+3)-p+1$  with $p$ prime
$E1(n+1)$ : $f(n+1)+f(n+2)=f(n+3)f(n+4)-p+1$

$E1(n+1)-E1(n)$ : $E2(n)$ : $f(n+2)-f(n)=f(n+3)(f(n+4)-f(n+2)$

1) $f(1)=f(3)=a$ and $f(2)=f(4)=b$
Then property E2 shows that $f(2k)=b$ and $f(2k-1)=a$ for any $k>0$
Then property E1 shows that $a+b=ab+1-p$, so $(a-1)(b-1)=p$, so $(a,b)=(2,p+1)$ or $(p+1,2)$

2) $f(1)=f(3)=a$ and $f(4)=f(2)+c$ with $c\in \mathbb{Z}^{*}$
Then property E2 shows that $f(2k-1)=a$ for any $k>0$
And property E2 shows that $f(4)-f(2)=a(f(6)-f(4))=a^{2}(f(8)-f(6))=a^{3}(f(10)-f(8))=...$
And obviously $a=1$ and $f(2k)=f(2)+(k-1)c$ and $c>0$
Then $E1(2k)\implies$ $f(2)+(k-1)c+1=f(2)+kc-p+1$ and so $c=p$ and $E2(2k+1)$ is also verified

3) $f(3)=f(1)+c$ with $c\in \mathbb{Z}^{*}$
Then property E2 shows that $f(3)-f(1)=c=f(4)(f(5)-f(3))=$ $f(4)f(6)(f(7)-f(5))=f(4)f(6)f(8)(f(9)-f(7))=...$
And obviously $f(2k)=1$ for any $k>1$ and $f(2k-1)=f(1)+(k-1)c$ for any $k>0$ and $c$ must be $>0$
With $E2(2)$ we also have $f(2)=1$
Then $E1(2k)\implies$ $1+f(1)+kc= f(1)+(k+1)c-p+1$ and so $c=p$ and $E2(2k+1)$ is also verified

So, we have 4 families of solutions :

Family F1 : $f(1,2,3,...)=2,p+1,2,p+1,2,p+1, ...$
Family F2 : $f(1,2,3,...)=p+1,2,p+1,2,p+1,2, ...$
Family F3 : $f(1,2,3,...)=1,a,1,a+p,1,a+2p,1,a+3p,1,a+4p, ...$
Family F4 : $f(1,2,3,...)=a,1,a+p,1,a+2p,1,a+3p,1,a+4p,1, ...$
\end{mysolution}



\begin{mysolution}[by \href{https://artofproblemsolving.com/community/user/18420}{aviateurpilot}]
	\begin{tcolorbox}So, we have 4 families of solutions :

Family F1 : $f(1,2,3,...)=2,p+1,2,p+1,2,p+1, ...$
Family F2 : $f(1,2,3,...)=p+1,2,p+1,2,p+1,2, ...$
Family F3 : $f(1,2,3,...)=1,a,1,a+p,1,a+2p,1,a+3p,1,a+4p, ...$
Family F4 : $f(1,2,3,...)=a,1,a+p,1,a+2p,1,a+3p,1,a+4p,1, ...$\end{tcolorbox}
good   

F1: $2f(n)=(-1)^{n}(p-1)+p+3$

F2: ${2f(n)=(-1)^{n+1}(p-1)+p+3}$

F3: $4f(n)=(-1)^{n}(p(n-2)+2a-2)+2a+2+p(n-2)$

F4: $4f(n)=(-1)^{n+1}(p(n-1)+2a-2)+2a+2+p(n-1)$
\end{mysolution}
*******************************************************************************
-------------------------------------------------------------------------------

\begin{problem}[Posted by \href{https://artofproblemsolving.com/community/user/30691}{Modul}]
	Find all functions $f: \mathbb R\to \mathbb R$ such that
\[f(f(x))=5f(x)-4x, \quad \forall x \in \mathbb R.\]
	\flushright \href{https://artofproblemsolving.com/community/c6h153306}{(Link to AoPS)}
\end{problem}





\begin{mysolution}[by \href{https://artofproblemsolving.com/community/user/29428}{pco}]
	\begin{tcolorbox}Find all functions $f: R\to R$ such that
$f(f(x))=5f(x)-4x,\forall x \in R.$\end{tcolorbox}

Besides the two immediate trivial solutions $f(x)=x$ and $f(x)=4x+b$, this problem has infinitely many solutions.

From $f(f(x))=5f(x)-4x$, it is easy to see that :

1) $f(x)$ is a bijective function
2) $f^{\circ n}(x)=\frac{4^{n}-1}{3}f(x)-\frac{4^{n}-4}{3}x$ (where $f^{\circ n}(x)=f\circ f\circ\ldots\circ f(x)$ $n$ times)
3) $f(f(x))-f(x)=4(f(x)-x)$ and so $f(x)>x\Leftrightarrow f(f(x))>f(x)$
4) $g^{\circ n}(x)=\frac{4}{3}(1-\frac{1}{4^{n}})g(x)-\frac{1}{3}(1-\frac{1}{4^{n-1}})x$ where $g(x)=f^{\circ(-1)}(x)$ is the reciprocical function of $f(x)$
5) $g(g(x))-g(x)=\frac{1}{4}(g(x)-x)$ and so $g(x)>x\Leftrightarrow g(g(x))>g(x)$

With these 5 statements, it is easy to build a lot of solutions :

a) Take any couple $(a,b>a)$ and any bijective function $h(x)$ $[a,b]\rightarrow [b,5b-4a]$
b) Define $f(x)=h(x)$ for any $x\in[a,b]$
c) Using points 2) and 3) above, it's easy to build $f(x)$ on $[b,+\infty)$
d) Using points 4) and 5) above, it's easy to build $f(x)$ on $(\frac{4a-b}{3},a)$
e) Take $f(\frac{4a-b}{3})=\frac{4a-b}{3}$
f) Take any couple $(c<\frac{4a-b}{3},d<c)$ and any bijective function $k(x)$ $[d,c]\rightarrow [5d-4c,d]$
g) Define $f(x)=k(x)$ for any $x\in[d,c]$
h) Using points 2) and 3) above, it's easy to build $f(x)$ on $(-\infty,d]$ 
i) Using points 4) and 5) above, it's easy to build $f(x)$ on $[c,\frac{4a-b}{3})$


Such construction gives infinitely many continuous increasing solutions.

If you add the constraint that $f(x)$ must be differentiable, this constraint for $x=\frac{4a-b}{3}$ would imply $f(x)=x$ or $f(x)=4x+c$
\end{mysolution}
*******************************************************************************
-------------------------------------------------------------------------------

\begin{problem}[Posted by \href{https://artofproblemsolving.com/community/user/13068}{The soul of rock}]
	Let $a>0$ be a constant not equal to $1$. Find all functions $f: \mathbb R \to \mathbb R$ such that
\[f(ax)=f(x)+2f(-x),\]
for all real $x$.
	\flushright \href{https://artofproblemsolving.com/community/c6h153937}{(Link to AoPS)}
\end{problem}



\begin{mysolution}[by \href{https://artofproblemsolving.com/community/user/29428}{pco}]
	\begin{tcolorbox}Let $a=const >0$ and $a \neq 1$. Find all functions $f(x)$ such that
\[f(ax)=f(x)+2f(-x) \]
\end{tcolorbox}

We have $f(0)=0$. Consider now that $f(x)$ is defined for $x>0$, then it will be defined, for $x<0$ by $f(x)=\frac{f(-ax)-f(-x)}{2}$ and the property is verified for any $x\geq 0$.

In order the property be verified for any $x<0$, we must have, for any $x>0$, $f(-ax)=f(-x)+2f(x)$, and so $\frac{f(a^{2}x)-f(ax)}{2}=\frac{f(ax)-f(x)}{2}+2f(x)$, and so $f(a^{2}x)=2f(ax)+3f(x)$

So the problem is equivalent to find a function $f(x)$ such that $f(a^{2}x)=2f(ax)+3f(x)$ $\forall x>0$, and to expand $f(0)=0$ and $f(x)=\frac{f(-ax)-f(-x)}{2}$  $\forall x<0$

For $x>0$, let $f(x)=g(\frac{ \ln(x)}{\ln(a)})$. We have then $g(\frac{\ln(x)}{\ln(a)}+2)=2g(\frac{\ln(x)}{\ln(a)}+1)+3g(\frac{\ln(x)}{\ln(a)})$ $\forall x>0$, and hence $g(x+2)=2g(x+1)+3g(x)$ $\forall x$
This last equation is rather well known and we have $g(x+n)=\frac{3^{n}-(-1)^{n}}{4}g(x+1)+\frac{3^{n}+3(-1)^{n}}{4}g(x)$

So the solution for $g(x)$ is :
$g(x)=\frac{3^{\lfloor x\rfloor}-(-1)^{\lfloor x\rfloor}}{4}h(x-\lfloor x\rfloor+1)+\frac{3^{\lfloor x\rfloor}+3(-1)^{\lfloor x\rfloor}}{4}h(x-\lfloor x\rfloor)$, with $h(x)$ any function defined on $[0,2)$

And so the solution for $f(x)$ is :

For $x>0$, $f(x)=\frac{3^{\lfloor \frac{ \ln(x)}{\ln(a)}\rfloor}-(-1)^{\lfloor \frac{ \ln(x)}{\ln(a)}\rfloor}}{4}h(\frac{ \ln(x)}{\ln(a)}-\lfloor \frac{ \ln(x)}{\ln(a)}\rfloor+1)$ $+\frac{3^{\lfloor \frac{ \ln(x)}{\ln(a)}\rfloor}+3(-1)^{\lfloor \frac{ \ln(x)}{\ln(a)}\rfloor}}{4}h(\frac{ \ln(x)}{\ln(a)}-\lfloor \frac{ \ln(x)}{\ln(a)}\rfloor)$, with $h(x)$ any function defined on $[0,2)$

For $x=0$, $f(0)=0$;

For $x<0$, $f(x)=\frac{f(-ax)-f(-x)}{2}$


Examples :
1) $h(x)=0\implies f(x)=0$

2) $h(x)=2$ $\implies$
For $x>0$, $f(x)=3^{\lfloor \frac{ \ln(x)}{\ln(a)}\rfloor}+(-1)^{\lfloor \frac{ \ln(x)}{\ln(a)}\rfloor}$
For $x=0$, $f(0)=0$;
For $x<0$, $f(x)=3^{\lfloor \frac{ \ln(x)}{\ln(a)}\rfloor}-(-1)^{\lfloor \frac{ \ln(x)}{\ln(a)}\rfloor}$
\end{mysolution}



\begin{mysolution}[by \href{https://artofproblemsolving.com/community/user/13068}{The soul of rock}]
	I need a solution base on cyclic function! Could you help me! Thanks!
\end{mysolution}



\begin{mysolution}[by \href{https://artofproblemsolving.com/community/user/29428}{pco}]
	\begin{tcolorbox}I need a solution base on cyclic function! Could you help me! Thanks!\end{tcolorbox}

I don't understand. Give some precisions.

My solution is, imho, the general one (all solutions has this form for some h(x)).
\end{mysolution}



\begin{mysolution}[by \href{https://artofproblemsolving.com/community/user/30932}{Just}]
	pco,

Could you please give some details about the solutions to the "rather well known equation". Thanks!
\end{mysolution}



\begin{mysolution}[by \href{https://artofproblemsolving.com/community/user/29428}{pco}]
	\begin{tcolorbox}pco,

Could you please give some details about the solutions to the "rather well known equation". Thanks!\end{tcolorbox}

I'm sorry if I was not clear enough.
I say that functional equation $f(x+2)=af(x+1)+bf(x)$ ($a\neq0$ and $b\neq 0$) is rather simple and well known (according to me  :blush: ) :
If you define $f(x)$ as any function on $[0,2)$, then you can define it (thru $f(x+2)=af(x+1)+bf(x)$ with $x$ in $[0,1)$) on $[2,3)$ and then on $[3,4)$ and so on for any $x\geq 0$.
And $f(x+2)=af(x+1)+bf(x)$ ($a\neq0$ and $b\neq 0$) $\implies$ $f(x)=-\frac{a}{b}f(x+1)-\frac{1}{b}f(x+2)$ and so $f(x)$ can be defined on $(-\infty,0)$ in the same way.

In order to have a concrete expression of the solution, we can write $f(x+n)=u_{n}f(x+1)+v_{n}f(x)$ with $u_{n}$ and $v_{n}$ defined as :
$u_{n+1}=au_{n}+v_{n}$
$v_{n+1}=bu_{n}$
$u_{0}=0$, $u_{1}=1$, $v_{0}=1$, $v_{1}=0$,.

Or else :
$u_{n+1}=au_{n}+bu_{n-1}$, $u_{0}=0$, $u_{1}=1$ and $v_{n}=bu_{n-1}$

Once you have $u_{n}$ and $v_{n}$, you say :
$f(x)=h(x)$ $\forall x\in[0,2)$ with $h(x)$ being any function defined in $[0,2)$ and :
$f(x)=f( \lfloor x \rfloor+\{x\})=u_{ \lfloor x \rfloor }f(\{x\}+1)+v_{ \lfloor x \rfloor }f(\{x\})$ (with $\{x\}$ being the fractional part of $x$) and then :


$f(x)=u_{ \lfloor x \rfloor }h(\{x\}+1)+v_{ \lfloor x \rfloor }h(\{x\})$
\end{mysolution}



\begin{mysolution}[by \href{https://artofproblemsolving.com/community/user/30932}{Just}]
	Ok, got you. Thanks.

I was thinking of another method, what do you think?

$f(ax)=f(x)+2f(-x)$ and $f(-ax)=f(-x)+2f(x)$ hence $f(ax)+f(-ax)=3(f(x)+f(-x))$.

Letting $g(x)=f(x)+f(-x)$ we get $g(ax)=3g(x)$ whose solutions are (considering that $g(0)=0$) $kx^{\frac{\ln{3}}{\ln{a}}}$ (not sure they are the only ones).

Then, we try solving $f(x)+f(-x)=kx^{\frac{\ln{3}}{\ln{a}}}$.
\end{mysolution}



\begin{mysolution}[by \href{https://artofproblemsolving.com/community/user/29428}{pco}]
	\begin{tcolorbox}Ok, got you. Thanks.

I was thinking of another method, what do you think?

$f(ax)=f(x)+2f(-x)$ and $f(-ax)=f(-x)+2f(x)$ hence $f(ax)+f(-ax)=3(f(x)+f(-x))$.

Letting $g(x)=f(x)+f(-x)$ we get $g(ax)=3g(x)$ whose solutions are (considering that $g(0)=0$) $kx^{\frac{\ln{3}}{\ln{a}}}$ (not sure they are the only ones).

Then, we try solving $f(x)+f(-x)=kx^{\frac{\ln{3}}{\ln{a}}}$.\end{tcolorbox}

1) Yes, it could be a method and I think you'll obtain similar results.

2) Caution : $kx^{\frac{\ln{3}}{\ln{a}}}$ is absolutly not the only solution of $g(ax)=3g(x)$
\end{mysolution}
*******************************************************************************
-------------------------------------------------------------------------------

\begin{problem}[Posted by \href{https://artofproblemsolving.com/community/user/10745}{Kondr}]
	Find all functions $f: \mathbb{R}\rightarrow\mathbb{R}$ such that
\[f\left(xf(x)+f(y)\right)=\left(f(f(x))\right)^{2}+y \quad \forall x,y \in \mathbb R.\]
	\flushright \href{https://artofproblemsolving.com/community/c6h154349}{(Link to AoPS)}
\end{problem}



\begin{mysolution}[by \href{https://artofproblemsolving.com/community/user/29428}{pco}]
	\begin{tcolorbox}Find all functions $f: \mathbb{R}\rightarrow\mathbb{R}$:

$f\left(xf(x)+f(y)\right)=\left(f(f(x))\right)^{2}+y$\end{tcolorbox}

$E1(x,y)$ : $f\left(xf(x)+f(y)\right)=\left(f(f(x))\right)^{2}+y$
Let $f(0)=a$ and $f(a)=b$

$E1(0,x)$ : $f(f(x))=x+b^{2}$ $\implies$ $E2(x)$ : $f(x+b^{2})=f(x)+b^{2}$

$E1(0,0)$ : $b=b^{2}$ $\implies$ $b=0$ or $b=1$

1) If $b=1$, we have $f(f(x))=x+1$ and $f(x+1)=f(x)+1$ and so $E1(x,y)$ becomes $f(xf(x)+f(y))=(x+1)^{2}+y$. 
Then $f(0)=a$ implies $f(-1)=a-1$
Then $E1(-1,0)$ gives $f(a-f(-1))=0$ so $f(1)=0$ and $f(f(1))=a$ but $f(f(1))=2$ and so $a=2$ and, since $f(0)=a=2$, we have $f(n)=n+2$ for any integer n.
But then $E1(1,0)$ is wrong

2) So $b=0$, and then  $E1(0,x)$ gives $f(f(x))=x$ and then $E1(x,0)$ gives $f(xf(x))=x^{2}$ and $E1(f(x),0)$ gives $f(f(x)x)=f(x)^{2}$. So $f(x)^{2}=x^{2}$

So either $f(x)=x$, either $f(x)=-x$. The question then is to check if it is possible to have $f(x)=x$ for some $x$ and $f(x)=-x$ for some other $x$ :
Let $x$ such that $f(x)=x$ and $y$ such that $f(y)=-y$
Then $E1(x,y)$ gives $f(x^{2}-y)=x^{2}+y$ and so either $x^{2}-y=x^{2}+y$, either $-x^{2}+y=x^{2}+y$, and so either $x=0$, either $y=0$

So either $f(x)=x$ for any $x$, either $f(x)=-x$ for any $x$

It is easy to verify that these two solutions are available.

So the two solutions are $f(x)=x$ and $f(x)=-x$
\end{mysolution}
*******************************************************************************
-------------------------------------------------------------------------------

\begin{problem}[Posted by \href{https://artofproblemsolving.com/community/user/29515}{yrhc@eht}]
	Find all real functions $f: \mathbb R\to \mathbb R$ satisfying
\[f\left(x^{2}+y\right)=2f(x)+f(y^{2}),\]
for all $x$ and $y$ in the domain.
	\flushright \href{https://artofproblemsolving.com/community/c6h154364}{(Link to AoPS)}
\end{problem}



\begin{mysolution}[by \href{https://artofproblemsolving.com/community/user/20820}{arkhammedos}]
	did u mean this $f(x^{2}+y)=2f(x)+f(y^{2})$ :huh:
\end{mysolution}



\begin{mysolution}[by \href{https://artofproblemsolving.com/community/user/29428}{pco}]
	\begin{tcolorbox}did u mean this $f(x^{2}+y)=2f(x)+f(y^{2})$ :huh:\end{tcolorbox}

If it is $P(x,y)$ : $f(x^{2}+y)=2f(x)+f(y^{2})$, then :
$P(0,0)$ $\implies$  $f(0)=3f(0)$ $\implies$ $f(0)=0$
$P(0,x)$ $\implies$ $f(x)=f(x^{2})$
$P(x,0)$ $\implies$ $f(x^{2})=2f(x)$
So $2f(x)=f(x)$ and the unique solution is $f(x)=0$
\end{mysolution}



\begin{mysolution}[by \href{https://artofproblemsolving.com/community/user/29515}{yrhc@eht}]
	sorry, is =
anyway, thank for try helping.
\end{mysolution}
*******************************************************************************
-------------------------------------------------------------------------------

\begin{problem}[Posted by \href{https://artofproblemsolving.com/community/user/9928}{shyong}]
	Suppose that two given functions $f$ and $g$ are defined on the interval $[0,2k]$ for some $k>0$ . Can we always find a pair $(x,y)$ of real numbers such that $|xy+f(x)+g(y)|\geq k^{2}$?
	\flushright \href{https://artofproblemsolving.com/community/c6h155063}{(Link to AoPS)}
\end{problem}



\begin{mysolution}[by \href{https://artofproblemsolving.com/community/user/29428}{pco}]
	\begin{tcolorbox}Let the domain of two given functions $f$ and $g$ be defined on $[0,2k]$ ,$k>0$ . Can we always find a pair of $(x,y)\in\mathbb{R}$ such that $|xy+f(x)+g(y)|\geq k^{2}$ ?\end{tcolorbox}

Yes, we can.

1) \begin{underlined}Case 1:\end{underlined} $g(0)-g(2k)\leq 2k^{2} \implies k^{2}-g(2k)-2k*2k\leq-k^{2}-g(0)-2k*0$. Then:
Either $f(2k)\geq k^{2}-g(2k)-2k*2k$, either $f(2k)<-k^{2}-g(0)-2k*0$. So :
Either $2k*2k+f(2k)+g(2k)\geq k^{2}$, either $2k*0+f(2k)+g(0)<-k^{2}$. So :
Either  $|2k*2k+f(2k)+g(2k)|\geq k^{2}$, either $|2k*0+f(2k)+g(0)|> k^{2}$

2) \begin{underlined}Case 2 : $g(0)-g(2k)> 2k^{2}$\end{underlined} $\implies$ $k^{2}-g(0)-0*0<-k^{2}-g(2k)-0*2k$. Then :
Either $f(0)\geq k^{2}-g(0)-0*0$, either $f(0)<-k^{2}-g(2k)-0*2k$. So :
Either $0*0+f(0)+g(0)\geq k^{2}$, either $0*2k+f(0)+g(2k)<-k^{2}$. So :
Either  $|0*0+f(0)+g(0)|\geq k^{2}$, either $|0*2k+f(0)+g(2k)|> k^{2}$

And, as a consequence :
Either  $|2k*2k+f(2k)+g(2k)|\geq k^{2}$,
Either $|2k*0+f(2k)+g(0)|> k^{2}$,
Either  $|0*0+f(0)+g(0)|\geq k^{2}$,
Either $|0*2k+f(0)+g(2k)|> k^{2}$

And the requested property is verified.
\end{mysolution}



*******************************************************************************
-------------------------------------------------------------------------------

\begin{problem}[Posted by \href{https://artofproblemsolving.com/community/user/30851}{djuro}]
	Find all functions $f: \mathbb R^{+}\to \mathbb R^{+}$ such that for all real $x$, we have
\[f(x)^{2}=f(x^{2})+2.\]
	\flushright \href{https://artofproblemsolving.com/community/c6h155157}{(Link to AoPS)}
\end{problem}



\begin{mysolution}[by \href{https://artofproblemsolving.com/community/user/950}{goc}]
	haven't seen it, but i'll try to explain an idea for a solution.

its' easy to see that $f(1)=2$ just by pluging $1$into the equation and that $f(x)>\sqrt{2}$.
$f(x)=\sqrt{2+f(x^{2})}\geq \sqrt{2+\sqrt{2}}$ since the limit of $\sqrt{2+\sqrt{2+...}}=2$ we get that $f(x)\geq 2$ for all $x$
now you need an infinite partition of $R$ where each subset is closed under the operation of squaring and square-rooting or equivalently 
if $x$ is in $A$ then $x^{2^{n}}$ is in $A$ for all integers $n$
for any such subset $A$ pick one number $a_{0}$ and let $f(a_{0})$ be an arbitrary value greater than $2$. from that you can calculate every value for $a$ from $A$ by solving the recurence relation $a_{n+1}+2=a_{n}^{2}$ which holds for every integer $n$ and all $a_{n}$ are positive...
djuro gave me an idea for this one.
set $f(a_{0})=x+\frac{1}{x}$ for $x>1$ then $f(a_{n})=x^{2^{n}}+\frac{1}{x^{2^{n}}}$ for any integer $n$
since $x=\frac{f(a_{0})+\sqrt{f(a_{0})^{2}-4}}{2}$ we now have a direct formula for $a_{n}$
 

\end{mysolution}



\begin{mysolution}[by \href{https://artofproblemsolving.com/community/user/10647}{mornik}]
	Hmm..., goc, just a little thing, function is $f: R^{+}\to R^{+}$ and therefore $f(0)$ does not exist, as I can see it.
\end{mysolution}



\begin{mysolution}[by \href{https://artofproblemsolving.com/community/user/950}{goc}]
	thanks. doesn't change much, totaly forgot bout it...
for mornik
nadam se da vjezbas(cini mi se bar da vjezbas posto si konstantno na forumu ;) )...ak trebas kaj,javi se
\end{mysolution}



\begin{mysolution}[by \href{https://artofproblemsolving.com/community/user/30851}{djuro}]
	I see the solution $f(x) = x+\frac{1}{ x}$.

ja sam inace smislio taj zadatak al ga ne znam rijesit.

satja
\end{mysolution}



\begin{mysolution}[by \href{https://artofproblemsolving.com/community/user/10647}{mornik}]
	Well, there may be other solutions, but anyhow you should have posted it in Algebra: Proposed \& Own if you came up with the problem yourself.

@goc: A nešto malo radim. Nemam kaj doma radit pa sjedim na forumu.
\end{mysolution}



\begin{mysolution}[by \href{https://artofproblemsolving.com/community/user/1372}{Ravi B}]
	More generally, the function $f(x) = x^{c}+x^{-c}$ works, where $c$ is a real constant.

djuro, did you want to assume that the function $f$ satisfies some continuity property?  Otherwise, you could get crazy examples by choosing different $c$ above for different $x$.
\end{mysolution}



\begin{mysolution}[by \href{https://artofproblemsolving.com/community/user/950}{goc}]
	he doesn't know anything about continuity yet, he's a student of mine... :)
try to solve it with assumption of continuity...
\end{mysolution}



\begin{mysolution}[by \href{https://artofproblemsolving.com/community/user/30851}{djuro}]
	\begin{tcolorbox}he doesn't know anything about continuity yet, he's a student of mine... :)
try to solve it with assumption of continuity...\end{tcolorbox}

Let's make a good problem then :)
I want not less than 50% of Copyright.

p.s. jel continuous znaci neprekidna?
\end{mysolution}



\begin{mysolution}[by \href{https://artofproblemsolving.com/community/user/10647}{mornik}]
	Da. Pretty much all the nice functions are continuous, even though I just know goc will find one which is discontinuous and try to convince us it's really nicer than, say, $f(x)=x^{2}$.  :D 

\end{mysolution}



\begin{mysolution}[by \href{https://artofproblemsolving.com/community/user/29428}{pco}]
	\begin{tcolorbox}Find all functions $f: R^{+}\to R^{+}$ such that
\[f(x)^{2}=f(x^{2})+2. \]
I'd like to know has anyone seen this before?\end{tcolorbox}

Using the idea of Goc, it is possible to show that a general solution of this functional equation is :

Let $u(x)$ and $v(x)$ any two funtions defined in $[0,1)$ and whose values are strictly positive. Then $f(x)$ is defined as :

For $x>1$, $f(x)=u(\{\frac{\ln(\ln(x))}{\ln(2)}\})^{2^{\lfloor\frac{\ln(\ln(x))}{\ln(2)}\rfloor}}+u(\{\frac{\ln(\ln(x))}{\ln(2)}\})^{-2^{\lfloor\frac{\ln(\ln(x))}{\ln(2)}\rfloor}}$
For $x=1$, $f(x)=2$
For $0<x<1$, $f(x)=v(\{\frac{\ln(-\ln(x))}{\ln(2)}\})^{2^{\lfloor\frac{\ln(-\ln(x))}{\ln(2)}\rfloor}}+v(\{\frac{\ln(-\ln(x))}{\ln(2)}\})^{-2^{\lfloor\frac{\ln(-\ln(x))}{\ln(2)}\rfloor}}$

Where $\{y\}$ means fractional part of $y$
\end{mysolution}



\begin{mysolution}[by \href{https://artofproblemsolving.com/community/user/19490}{behemont}]
	pco, how do you come up with something like that?
\end{mysolution}



\begin{mysolution}[by \href{https://artofproblemsolving.com/community/user/19941}{delegat}]
	\begin{tcolorbox}p.s. jel continuous znaci neprekidna?\end{tcolorbox}

Jeste.
\end{mysolution}



\begin{mysolution}[by \href{https://artofproblemsolving.com/community/user/30851}{djuro}]
	Would the problem be better if one adds more conditions?
Is this problem worth something?

I ask because it's mine :) I put Copyright on this prob.
\end{mysolution}



\begin{mysolution}[by \href{https://artofproblemsolving.com/community/user/29428}{pco}]
	\begin{tcolorbox}pco, how do you come up with something like that?\end{tcolorbox}

1) $f(x)\geq 2$ $\forall x>0$
We have $f(x)>0$. Then, if it exists $a>0$such that $0<f(a)<2$, then we have $0<f(a^{2})<f(a)<2$ and the sequence $a_{n}=f(a^{2^{n}})$ is a strictly decreasing sequence whose each element is in $(0,2)$. So this sequence has a limit which verifies $L=L^{2}-2$ and so $L=-1$ or $L=2$, which is impossible if $0<f(a)<2$.
So $f(x)\geq 2$ $\forall x>0$

2) $f(x)=a(x)+\frac{1}{a(x)}$ with $a(x)\geq 1$ and $a(x)^{2}=a(x^{2})$.
For any $y\geq 2$, it exists a unique real $z\geq 1$ such that $y=z+\frac{1}{z}$. So it exists a unique function $a(x)\geq 1$ such that $f(x)=a(x)+\frac{1}{a(x)}$.
Then $f(x)^{2}=f(x^{2})+2$ becomes $a(x)^{2}+\frac{1}{a(x)^{2}}=a(x^{2})+\frac{1}{a(x^{2})}$.
Then, since there is a unique way of splitting a number $\geq 2$ in a sum of a number $\geq 1$ and its inverse, we have $a(x)^{2}=a(x^{2})$

3) Solutions of $a(x)\geq 1$ and $a(x)^{2}=a(x^{2})$ $\forall$ $x>0$
Since $x>0$, we can write $a(x)=b(\ln(x))$ and we have $b(\ln(x))^{2}=b(2\ln(x))$ $\forall$ $x>0$ and so $b(x)^{2}=b(2x)$ $\forall$ $x$
Then, we can distinguish 3 situations :

3.1) $x>0$. Let then $b(x)=c(\frac{\ln(x)}{\ln(2)})$ and we have $c(\frac{\ln(x)}{\ln(2)})^{2}=c(1+\frac{\ln(x)}{\ln(2)}$ and so $c(x+1)=c(x)^{2}$
So $c(x+n)=c(x)^{2^{n}}$ and so we can define $c(x)=u(x)$ for $x\in[0,1)$ ($u(x)$ is any function defined on $[0,1)$ and $\geq 1$) and then :
$c(x)=c(\lfloor x\rfloor+\{x\})=c(\{x\})^{2^{\lfloor x \rfloor }}=u(\{x\})^{2^{\lfloor x \rfloor }}$

3.2) $x=0$ $\implies$ $b(0)^{2}=b(0)$ and, since $a(x)\geq 1$ and $b(x)\geq 1$ : $b(0)=1$.

3.3) $x<0$. Let then $b(x)=c(\frac{\ln(-x)}{\ln(2)})$ and we have $c(\frac{\ln(-x)}{\ln(2)})^{2}=c(1+\frac{\ln(-x)}{\ln(2)}$ and so $c(x+1)=c(x)^{2}$
So $c(x+n)=c(x)^{2^{n}}$ and so we can define $c(x)=v(x)$ for $x\in[0,1)$ ($v(x)$ is any function defined on $[0,1)$ and $\geq 1$) and then :
$c(x)=c(\lfloor x\rfloor+\{x\})=c(\{x\})^{2^{\lfloor x \rfloor }}=v(\{x\})^{2^{\lfloor x \rfloor }}$

4) General solution of the requested functional equation :
Let $u(x)$ and $v(x)$ any two functions $\geq 1$ and defined in $[0,1)$ :
$x>1$, $f(x)=u(\{\frac{\ln(\ln(x))}{\ln(2)}\})^{2^{\lfloor \frac{\ln(\ln(x))}{\ln(2)}\rfloor }}+u(\{\frac{\ln(\ln(x))}{\ln(2)}\})^{-2^{\lfloor \frac{\ln(\ln(x))}{\ln(2)}\rfloor }}$
$x=1$, $f(x)=2$
$x<1$, $f(x)=v(\{\frac{\ln(-\ln(x))}{\ln(2)}\})^{2^{\lfloor \frac{\ln(-\ln(x))}{\ln(2)}\rfloor }}+v(\{\frac{\ln(-\ln(x))}{\ln(2)}\})^{-2^{\lfloor \frac{\ln(-\ln(x))}{\ln(2)}\rfloor }}$

5) We can retrieve the solution $f(x)=x^{a}+\frac{1}{x^{a}}$ : Just take $u(x)=v(x)=e^{a2^{x}}$
\end{mysolution}



\begin{mysolution}[by \href{https://artofproblemsolving.com/community/user/29428}{pco}]
	\begin{tcolorbox}Find all functions $f: R^{+}\to R^{+}$ such that
\[f(x)^{2}=f(x^{2})+2. \]
I'd like to know has anyone seen this before?\end{tcolorbox}

About your second question : this equation is in the very frequent class of equations like : $f\circ u=v\circ f$ where $u(x)$ and $v(x)$ are bijective functions.

You must first check that if $u(x)$ has fixed points, $v(x)$ has too  (since $u(x)=x$ $\implies$ $v(f(x))=f(x))$
For such an equation, you have $f\circ u^{\circ n}=v^{\circ n}\circ f$ (where $h^{\circ n}=h\circ h\circ h\cdots \circ h$ $n$ times).
Then the key point is to find (if possible) a subset $S$ of the domain of $f(x)$ such that any real $y$ may be written in a unique way as $y=u^{\circ n}(x)$ with $x\in S$ and $n\in\mathbb{Z}$

If this is possible, an if we call $n(x)$ (values in $\mathbb{Z}$) and $seed(x)$ (values in $S$) the two numbers such that $x=u^{\circ n(x)}(seed(x))$, and if we have $h(x)$ any function defined in $S$, then :

Obviously $n(u(x))=n(x)+1$ and $seed(u(x))=seed(x)$ and then :

$f(x)=v^{\circ n(x)}(h(seed(x)))$

Trivial examples :

1) $f(x+1)=f(x)+7$
We have $u(x)=x+1$ and $v(x)=x+7$
We can take $S=[0,1)$, $n(x)=\lfloor x\rfloor$ and $seed(x)=\{x\}$ and we have the general solution $f(x)=h(\{x\})+7\lfloor x \rfloor$ with $h(x)$ beeing any function defined in $[0,1)$ 
Obviously, in this case, we can write $k(x)=h(x)-7x$ and then $f(x)=k(\{x\})+7x$

2) $f(x+1)=(f(x))^{2}$
We have $u(x)=x+1$ and $v(x)=x^{2}$
We can take $S=[0,1)$, $n(x)=\lfloor x\rfloor$ and $seed(x)=\{x\}$ and we have the general solution $f(x)=h(\{x\})^{2^{\lfloor x \rfloor}}$ with $h(x)$ beeing any function defined in $[0,1)$ and with values $\geq 0$
For example, with $h(x)=a^{2^{x}}$, we have the simple solution $f(x)=a^{2^{x}}$

3) $f(x^{3})=f(x)+9$
We have $u(x)=x^{3}$ and $v(x)=x+9$
Impossible since $u(x)$ has fixed points $-1$ and $+1$ but $v(x)$ has no fixed points.

....
In your case, we have $f(x^{2})=f(x)^{2}-2$
We have $u(x)=x^{2}$ and $v(x)=x^{2}-2$ and $u(x)$ is a bijective function  $R^{+}\to R^{+}$
...
\end{mysolution}
*******************************************************************************
-------------------------------------------------------------------------------

\begin{problem}[Posted by \href{https://artofproblemsolving.com/community/user/19941}{delegat}]
	Find all funcions $ f: \mathbb{R}\to\mathbb{R}$ such that for all reals $ x$ and $ y$ the following equality holds:
\[f(x)f(yf(x)-1)=x^{2}f(y)-f(x).\]
	\flushright \href{https://artofproblemsolving.com/community/c6h156055}{(Link to AoPS)}
\end{problem}



\begin{mysolution}[by \href{https://artofproblemsolving.com/community/user/19490}{behemont}]
	Let $ *$ denotes the initial equation.
Plugging $ x=0$ into $ *$, we get $ f(0)=0$,
 then plugging $ y=0$ we get $ f(-1)=-1$.
Plugging $ x=-1$ into $ *$ we get $ f(-y-1)=-f(y)-1$. 
Using that, the inital equation becomes $ -f(x)f(-yf(x))=x^{2}f(y)$, denote it by $ **$, 
and plugging here $ y=-1$ we get $ f(x)f(f(x))=x^{2}$.  $ (1)$

Let's suppose $ f(x_{1})=f(x_{2})$. Then also $ f(yf(x_{1}))=f(yf(x_{2}))$, so $ f(x_{1})=f(x_{2})\rightarrow x_{1}^{2}=x_{2}^{2}$. $ (2)$ 

Look at $ (1)$. It's obvious that $ f(x)<0\rightarrow f(f(x))<0$ and $ f(x)>0\rightarrow f(f(x))>0$, meaning that $ t$ and $ f(t)$ have the same sign $ \forall t\in Im(f)$.
Then also $ f$ is injective $ \forall t\in Im(f)$ because of $ (1)$. $ (3)$
 
Now take $ t\in Im(f)$, $ t\not=0$. We know that $ f(t)f(f(t))=t^{2}$. Because of $ (3)$ we know that $ f$ is nonincreasing or nondecreasing for such $ t$'s. So suppose $ f(t)>t \rightarrow f(f(t))\geq f(t)>t$. By multiplying it, we get f$ (t)f(f(t))>t^{2}.$ Contradiction.. We'll get the same for $ f(t)<t$. So $ f(t)=t\longleftrightarrow f(f(x))=f(x) \forall x \in \mathbb{R}$.
Using that in $ (1)$, we get $ [f(x)]^{2}=x^{2}$.

Should be easy to prove that there isn't $ a$ such that $ f(a)=-a$, but i am in a hurry now so i'll finish it later..
\end{mysolution}



\begin{mysolution}[by \href{https://artofproblemsolving.com/community/user/950}{goc}]
	\begin{tcolorbox} Because of $ (3)$ we know that $ f$ is nonincreasing or nondecreasing for such $ t$\end{tcolorbox}
why should it be monotone? :huh:  there are plenty of nonmonotone bijections(and therefore injections) in the world...
monotone->bijective and not the other way arround
edit:
one more thing
\begin{tcolorbox}we get $ f(-1)=-1$\end{tcolorbox}
that's not quite true either. you let a solution slip by here. from that condition you can get that $ f(x)=0$ for all $ x$, or $ f(-1)=-1$...
\end{mysolution}



\begin{mysolution}[by \href{https://artofproblemsolving.com/community/user/19490}{behemont}]
ok i skipped that trivial solution, but why do you say it's not monotone? :(
\end{mysolution}



\begin{mysolution}[by \href{https://artofproblemsolving.com/community/user/29428}{pco}]
	\begin{tcolorbox} ..., but why do you say it's not monotone? :(\end{tcolorbox}

It exist non monotone function which are injective. For example $ f(x)=2\lfloor x\rfloor-x+1$. In this example, we have :
$ f(t)$ and $ t$ has the same sign
$ f(x)$ is a bijective function.
$ f(x)$ is not monotone.
\end{mysolution}



\begin{mysolution}[by \href{https://artofproblemsolving.com/community/user/29428}{pco}]
	\begin{tcolorbox}Find all funcions $ f: \mathbb{R}\to\mathbb{R}$ such that for all reals $ x$ and $ y$ the following equality holds:

$ f(x)f(yf(x)-1)=x^{2}f(y)-f(x)$\end{tcolorbox}

1) $ x=0$ implies $ f(0)=0$

2) $ y=0$ implies $ f(x)f(-1)=-f(x)$ and either $ f(x)=0$ $ \forall x$, either $ f(-1)=-1$

Assume now $ f(x)$ is not the null function.
3) $ x=-1$ implies $ f(-y-1)=-f(y)-1$ and $ f(-\frac{1}{2})=-\frac{1}{2}$

4) $ x=-\frac{1}{2}$ and $ y=-2$ implies $ f(-2)=-2$

5) $ x=-1$ and $ y=-2$ implies then $ f(1)=1$

6) $ x=1$ implies $ f(y-1)=f(y)-1$ and so $ f(x+1)=f(x)+1$ $ \forall x$

7) $ y=-1$ implies $ f(x)f(-f(x)-1)=-x^{2}-f(x)$ and, with point 3, $ f(x)f(f(x))=x^{2}$

8) Replace $ x$ by $ x+1$ in 7 above, and, using 6 : $ f(x+1)f(f(x+1))=(x+1)^{2}$ implies $ (f(x)+1)(f(f(x))+1)=x^{2}+2x+1$, so $ f(x)f(f(x))+f(x)+f(f(x))+1=x^{2}+2x+1$ and so $ f(f(x))+f(x)=2x$

9) $ f(f(x))f(x)=x^{2}$ and $ f(f(x))+f(x)=2x$ implies $ f(f(x))$ and $ f(x)$ are the two roots of $ Y^{2}-2xY+x^{2}=0$ and then $ f(x)=x$

So the only two solutions could be $ f(x)=0$ and $ f(x)=x$ and it is easy to check that these two solutions are OK.
\end{mysolution}



*******************************************************************************
-------------------------------------------------------------------------------

\begin{problem}[Posted by \href{https://artofproblemsolving.com/community/user/25101}{radio}]
	Find all functions $ f: \mathbb R^{+}\rightarrow \mathbb R$ such that
\[f(x)f(y)=f(xy)+2005\left(\frac{1}{x}+\frac{1}{y}+2004\right),\]
for all $ x,y>0$.
	\flushright \href{https://artofproblemsolving.com/community/c6h156164}{(Link to AoPS)}
\end{problem}



\begin{mysolution}[by \href{https://artofproblemsolving.com/community/user/29428}{pco}]
	\begin{tcolorbox}Find all $ f: R^{+}\rightarrow R$ such that

$ f(x)f(y)=f(xy)+2005(\frac{1}{x}+\frac{1}{y}+2004)$ for $ x,y>0$\end{tcolorbox}

1) $ x=y=1$ implies $ f(1)^{2}-f(1)-2005*2006=0$ and $ f(1)=2006$ or $ f(1)=-2005$

2) $ y=1$ gives $ f(x)(f(1)-1)=2005*(2005+\frac{1}{x})$ and, since $ f(1)\neq 1$, $ f(x)=2005\frac{2005x+1}{(f(1)-1)x}$

So $ f(x)$ could only be $ f(x)=2005+\frac{1}{x}$ or $ f(x)=2005\frac{2005x+1}{-2006x}$

And, by checking back in the initial equation, the only solution is $ f(x)=2005+\frac{1}{x}$
\end{mysolution}
*******************************************************************************
-------------------------------------------------------------------------------

\begin{problem}[Posted by \href{https://artofproblemsolving.com/community/user/5820}{N.T.TUAN}]
	Find all functions $ f: \mathbb{R}\to\mathbb{R}$ such that 
a)$ f(x)\geq e^{2004x}$ for all $x\in\mathbb{R}$, and 
b)$ f(x+y)\geq f(x)f(y)$ for all $x,y\in\mathbb{R}$.
	\flushright \href{https://artofproblemsolving.com/community/c6h158846}{(Link to AoPS)}
\end{problem}



\begin{mysolution}[by \href{https://artofproblemsolving.com/community/user/29428}{pco}]
	\begin{tcolorbox}Find all functions $ f: \mathbb{R}\to\mathbb{R}$ such that 
a)$ f(x)\geq e^{2004x}\forall x\in\mathbb{R}$ and 
b)$ f(x+y)\geq f(x)+f(y)\forall x,y\in\mathbb{R}$.\end{tcolorbox}

From a), we get $ f(0)\geq 1$
From b) we get $ f(x+0)\geq f(x)+f(0)$ and so $ f(0)\leq 0$

So we have a contradiction and no such function exists.
\end{mysolution}



\begin{mysolution}[by \href{https://artofproblemsolving.com/community/user/5820}{N.T.TUAN}]
	I am wrong! I edited it , please continuou!  
\end{mysolution}



\begin{mysolution}[by \href{https://artofproblemsolving.com/community/user/3182}{Kunihiko\_Chikaya}]
	\begin{tcolorbox}Find all functions $ f: \mathbb{R}\to\mathbb{R}$ such that 
a)$ f(x)\geq e^{2004x}\forall x\in\mathbb{R}$ and .\end{tcolorbox}

I think $ f(x)\geq 1+2004x$. :maybe:
\end{mysolution}



\begin{mysolution}[by \href{https://artofproblemsolving.com/community/user/5820}{N.T.TUAN}]
	No, problem is true now.
\end{mysolution}



\begin{mysolution}[by \href{https://artofproblemsolving.com/community/user/29386}{mszew}]
	[hide="Is it correct?"]let $ f(x)=e^{2004x}g(x)$  from $ a) g(x)\geq 1$ $ \forall x$

$ f(x+y)=e^{2004x}e^{2004y}g(x+y)$
$ f(x) f(y) = e^{2004x}e^{2004y}g(x) g(y)$
from $ b) g(x+y) \geq g(x)g(y)$ $ \forall x,y$
$ y=0$ $ g(x)\geq  g(x) g(0)$ 
then $ g(0)=1$
$ y=-x$ $ g(0)\geq  g(x) g(-x)$
$ 1\geq  g(x) g(-x)$
then $ g(x)=1$
\end{mysolution}



\begin{mysolution}[by \href{https://artofproblemsolving.com/community/user/5820}{N.T.TUAN}]
	\begin{tcolorbox}let $ f(x)=e^{2004x}g(x)$  from $ a) g(x)\geq 1$ $ \forall x$

$ f(x+y)=e^{2004x}e^{2004y}g(x+y)$
$ f(x) f(y) = e^{2004x}e^{2004y}g(x) g(y)$
from $ b) g(x+y) \geq g(x)g(y)$ $ \forall x,y$
$ y=0$ $ g(x)\geq g(x) g(0)$ 
then $ g(0)=1$
$ y=-x$ $ g(0)\geq g(x) g(-x)$
$ 1\geq g(x) g(-x)$
then $ g(x)=1$\end{tcolorbox}
You are true!  
\end{mysolution}
*******************************************************************************
-------------------------------------------------------------------------------

\begin{problem}[Posted by \href{https://artofproblemsolving.com/community/user/30354}{friendlist}]
	Find all funtions $f: \mathbb R \to \mathbb R$ such that the following equality holds for all $ x,y \in \mathbb R$:
\[f(x^{3}-y)+2y(3f^{2}(x)+y^{2})=f(y+f(x)).\]
	\flushright \href{https://artofproblemsolving.com/community/c6h159278}{(Link to AoPS)}
\end{problem}



\begin{mysolution}[by \href{https://artofproblemsolving.com/community/user/22328}{sinajackson}]
	[hide="some useful moves!"]
by $ x=y=0$ we have: $ f(0)=f(f(0))$
by putting $ -y$ instead of $ y$ we have: 
$ f(x^{3}+y)-2y\left(3f^{2}(x)+y^{2}\right)=f(f(x)-y)$ and by sum of this and the main one:
we have: $ f(x^{3}+y)+f(x^{3}-y)=f(f(x)-y)+f(f(x)+y)$
by putting $ y=f(x)$ we have:
$ f(x^{3}+f(x))-8f^{3}(x)=f(0)$
now this will help us to prve that $ f(0)=0$, by putting $ x=0$ we have: $ f(f(0))-8f^{3}(x)=f(0)\rightarrow-8f^{3}(x)=0\rightarrow f(0)=0$.
as a consequence we must have:
$ f(x^{3}+f(x))=8f^{3}(x)$ 
 :) 

\end{mysolution}



\begin{mysolution}[by \href{https://artofproblemsolving.com/community/user/29428}{pco}]
	\begin{tcolorbox}Find all funtions such that the following equality holds for all $ x;y \in R$
$ f(x^{3}-y)+2y(3f^{2}(x)+y^{2})=f(y+f(x))$\end{tcolorbox}

Choose $ y$ such that $ x^{3}-y=y+f(x)$ which implies equality of the leftmost and rightmost summands in the equation :

$ y=\frac{x^{3}-f(x)}{2}$ $ \implies$ $ f(x^{3}-y)=f(y+f(x))$ and so $ y(3f^{2}(x)+y^{2})=0$ and so $ y=0$ (since $ y\in\mathbb{R}$ and $ f(x)\in\mathbb{R}$) and so $ f(x)=x^{3}$

And it is rather easy to check that this necessary condition is sufficient.

So the only solution is $ f(x)=x^{3}$
\end{mysolution}
*******************************************************************************
-------------------------------------------------------------------------------

\begin{problem}[Posted by \href{https://artofproblemsolving.com/community/user/25872}{santosguzella}]
	Find all functions $f: \mathbb Q \to \mathbb Q$ satisfying \[f(x+y)+f(x-y)=2f(x)+2f(y)\] for all rational $ x$ and $ y$.
	\flushright \href{https://artofproblemsolving.com/community/c6h159376}{(Link to AoPS)}
\end{problem}



\begin{mysolution}[by \href{https://artofproblemsolving.com/community/user/29428}{pco}]
	\begin{tcolorbox}Find all functions $ f$  from the rational to the rational numbers satisfying $ f(x+y)+f(x-y)=2f(x)+2f(y)$ for all rational $ x$ and $ y$.\end{tcolorbox}
$ x=y=0$ $ \implies$ $ f(0)=0$
$ x=0$ $ \implies$ $ f(-y)=f(y)$ $ \forall$ $ y$ rational.

Now, we can show with induction that $ f(nx)=n^{2}f(x)$ for every nonnegative integer $ n$ :
a) It's true for $ n=0$ and $ n=1$
b) If it is true up to $ n\geq 1$, then :$ f(nx+x)+f(nx-x)=2f(nx)+2f(x)$ and so :
$ f((n+1)x)=$ $ 2f(nx)+2f(x)-f((n-1)x)=$ $ (2n^{2}+2-(n-1)^{2})f(x)=$ $ (n^{2}+2n+1)f(x)=(n+1)^{2}f(x)$ and it's true for $ n+1$

Then $ f(x)=f(q(\frac{x}{q}))=q^{2}f(\frac{x}{q})$ and so $ f(\frac{x}{q})=\frac{f(x)}{q^{2}}$

So $ f(\frac{p}{q})=f(p(\frac{1}{q}))=p^{2}f(\frac{1}{q})=p^{2}\frac{f(1)}{q^{2}}$

And, since $ f(x)$ is an even function : $ f(x)=ax^{2}$ $ \forall x\in\mathbb{Q}$
\end{mysolution}
*******************************************************************************
-------------------------------------------------------------------------------

\begin{problem}[Posted by \href{https://artofproblemsolving.com/community/user/5820}{N.T.TUAN}]
	For which integers $ n>1$ and real numbers $ r$ does the curve $ y=x^{n}+rx$ contain the vertices of a rectangle?
	\flushright \href{https://artofproblemsolving.com/community/c6h160311}{(Link to AoPS)}
\end{problem}



\begin{mysolution}[by \href{https://artofproblemsolving.com/community/user/29428}{pco}]
	\begin{tcolorbox}For which integers $ n>1$ and real numbers $ r$ does the curve $ y=x^{n}+rx$ contain the vertices of a rectangle?\end{tcolorbox}

Let $ f(x)=x^{n}+rx$ 

If the four vertices of the rectangle are $ (x_{A},y_{A}), (x_{B},y_{B}), (x_{C},y_{C}), (x_{D},y_{D})$ with $ x_{A}\leq x_{B}\leq x_{C}\leq x_{D}$, then we have obviously $ x_{A}< x_{B}< x_{C}< x_{D}$ and $ f'(x)$ have at least two zeroes in $ [x_{A},x_{D}]$. Then, since $ f'(x)=nx^{n-1}+r$, we must have $ n$ odd and $ r<0$.

Since the two values where $ f'(x)=0$ are in $ [x_{A},x_{D}]$, we have $ x_{A}<0$ and $ x_{D}>0$

Now, we can consider that none of the slopes of the sides of the rectangle are zero (either the other sides would be "vertical", which is impossible). Consider then the slope "$ a$" which is positive (the slope of $ AB$ and $ CD$).

It is rather easy to see that if $ x_{C}<0$, then the length of $ CD$ is greater than the length of $ AB$ (don't forget they are parallel).
It is easy too in the same way to see that if if $ x_{B}>0$, then the length of $ AB$ is greater than the length of $ CD$.

So  $ x_{A}< x_{B}< 0 < x_{C}< x_{D}$ 

Then if we consider the intersections $ C$ and $ D$ of $ f(x)=x^{n}+r$ ($ n$ odd and $ r<0$) with $ y=ax+b$ ($ a>0$) with $ x>0$, we have at most one couple of points $ C$ and $ D$ for a given length $ CD$.

Hence, since $ f(x)$ is an odd function, since $ AB$ is parallel to $ CD$ and since $ AB=CD$, we need to have $ A$ symetric to $ D$ and $ B$ symetric to $ C$ and the four points are $ A=(-b,-b^{n}-rb)$, $ B=(-a,-a^{n}-ra)$, $ C=(a,a^{n}+ra)$ and $ D=(b,b^{n}+rb)$ for some $ 0< a<b$.

It remains to have $ BD$ perpendiculat to $ CD$ :

Slope of BD is $ \frac{a^{n}+b^{n}+ra+rb}{a+b}$

Slope of CD is $ \frac{a^{n}-b^{n}+ra-rb}{a-b}$

$ BD$ perpendicular to $ CD$ means product of slopes equal to $ -1$, and so $ (a^{n}+ra)^{2}-(b^{n}+rb)^{2}=b^{2}-a^{2}$
And so $ (a^{n}+ra)^{2}+a^{2}=(b^{n}+rb)^{2}+b^{2}$ with $ 0<a<b$

This is possible iff $ g(x)=(x^{n}+rx)^{2}+x^{2}$ has at least one postive root (wth sign changing) for $ g'(x)=0$
$ g'(x)=2(x^{n}+rx)(nx^{n-1}+r)+2x = 2x[(Z+r)(nZ+r)+1]$ with $ Z=x^{n-1}$
The equation $ (Z+r)(nZ+r)+1=0$ has a positive root with sign changing iff $ r^{2}<\frac{4n}{(n-1)^{2}}$, that's to say $ r<-\frac{2\sqrt{n}}{n-1}$ (since we know $ r<0$)

So the curve $ f(x)=x^{n}+rx$ ($ n>1$) contains the four vertices of a rectangle if and only if $ n$is odd and  $ r<-\frac{2\sqrt{n}}{n-1}$ .
\end{mysolution}
*******************************************************************************
-------------------------------------------------------------------------------

\begin{problem}[Posted by \href{https://artofproblemsolving.com/community/user/22793}{April}]
	Suppose that $ f$ is a real-valued function for which \[ f(xy)+f(y-x)\geq f(y+x)\] for all real numbers $ x$ and $ y$. 
    
a) Give a non-constant polynomial that satisfies the condition.
b) Prove that $ f(x)\geq 0$ for all real $ x$.
	\flushright \href{https://artofproblemsolving.com/community/c6h160835}{(Link to AoPS)}
\end{problem}



\begin{mysolution}[by \href{https://artofproblemsolving.com/community/user/29428}{pco}]
	\begin{tcolorbox}Suppose that $ f$ is a real-valued function for which
\[ f(xy)+f(y-x)\geq f(y+x) \]
for all real numbers $ x$ and $ y$. 
    
$ a.$ Give a nonconstant polynomial that satisfies the condition.
$ b.$ Prove that $ f(x)\geq 0$ for all real $ x$.\end{tcolorbox}

$ a.$ $ f(x)=x^{2}+4$
Proof : $ f(xy)+f(y-x)-f(x+y)=(xy)^{2}+4+(y-x)^{2}-(x+y)^{2}$
So $ f(xy)+f(y-x)-f(x+y)=(xy)^{2}-4xy+4$
So $ f(xy)+f(y-x)-f(x+y)=(xy-2)^{2}\geq 0$
So $ f(xy)+f(y-x)\geq f(x+y)$

$ b.$ Investigate then the case where $ xy=y+x$, that's to say $ y=\frac{x}{x-1}$. Then the inequation becomes $ f(y-x)\geq 0$. So we always have $ f(\frac{x}{x-1}-x)\geq 0$

So $ f(\frac{x(2-x)}{x-1})\geq 0$ $ \forall x\in %Error. "mathBB" is a bad command.
{R}-\{1\}$

But the equation $ \frac{x(2-x)}{x-1}=a$ is equivalent to $ x^{2}+(a-2)x-a=0$ and this equation always has real roots$ \neq 1$ (discriminant is $ a^{2}+4>0$).
So, for any real $ a$, exists $ x\neq 1$ such that $ \frac{x(2-x)}{x-1}=a$, and so $ f(a)\geq 0$

So $ f(a)\geq 0$ $ \forall$ $ a\in%Error. "mathBB" is a bad command.
{R}$
\end{mysolution}



\begin{mysolution}[by \href{https://artofproblemsolving.com/community/user/61542}{AwesomeToad}]
	How would one go about finding such a polynomial for a)? (And I'm also curious what other polynomials suffice)
\end{mysolution}



\begin{mysolution}[by \href{https://artofproblemsolving.com/community/user/97340}{quantumbyte}]
	I solved it by looking at part b). I knew that the easiest nonconstant polynomial such that $f(x) \ge 0$ for all real $x$ would have to be something of the form $x^2 +c$. Now finding $c$ is easy by just completing the square.
\end{mysolution}



\begin{mysolution}[by \href{https://artofproblemsolving.com/community/user/130234}{vsathiam}]
	Not sure if this works:

From the given inequality we have:
f(0) $\geq$ 0

Suppose there exists a k such that f(k)<0.
Substituting x=x, y= $\frac{k}{x}$ into the original equation gives:

f(k)+f(0) $\geq$ f(x+$\frac{k}{x}$).

The function g(x)= $\frac{k}{x}$ for some constant k has a range of all real values (besides k) for all x (besides x=0).

So f(x) $\leq$ f(k)+f(0) < f(0).

But f(0)=f(0). Since f(k) is a finite negative value, the function has a discontinuity at f(0). This is a contradiction since all polynomials are continuous.

So f(x) $\geq$ 0 for all x.
\end{mysolution}
*******************************************************************************
-------------------------------------------------------------------------------

\begin{problem}[Posted by \href{https://artofproblemsolving.com/community/user/5820}{N.T.TUAN}]
	Find all functions $ f: (0,\infty)\to (0,\infty)$ such that $ f(x)f(y)=f(x+yf(x))$ for all $x,y>0$.
	\flushright \href{https://artofproblemsolving.com/community/c6h160911}{(Link to AoPS)}
\end{problem}



\begin{mysolution}[by \href{https://artofproblemsolving.com/community/user/25546}{Yuriy Solovyov}]
	\begin{tcolorbox}Find all functions $ f: (0,\infty)\to (0,\infty)$ such that $ f(x)f(y)=f(x+yf(x))\forall x,y>0.$\end{tcolorbox}
Is the function of monotonous? (if $ f(x)=f(y)$ then $ x=y$).
If yes, then:
$ i) x=y=1, f(1)=c; f(c+1)=c^{2};$
$ ii) x=c+1; c^{2}f(y)=f(c+1+yf(c+1));$
$ y=c+1; c^{2}f(x)=f(x+(c+1)f(x));$
So: $ f(x+(c+1)f(x))=f(c+1+xf(c+1))$ and $ c^{2}x+c+1=x+(c+1)f(x);$
$ f(x)=(c-1)x+1.$
\end{mysolution}



\begin{mysolution}[by \href{https://artofproblemsolving.com/community/user/18420}{aviateurpilot}]
	\begin{tcolorbox}Find all functions $ f: (0,\infty)\to (0,\infty)$ such that $ f(x)f(y)=f(x+yf(x))\forall x,y>0.$\end{tcolorbox}

we suppose that $ \exists x,y>0: \ f(x)<f(y)$ then $ \exists a>0: \ x+af(x)=y+af(y)$
so $ f(x)f(a)=f(x+af(x))=f(y+af(y))=f(y)f(a)$ gives $ f(x)=f(y)$ (impossible)
then $ f$ is $ increasing$.
and we have $ \forall x,y>0: \ f(x+yf(x))=f(x)f(y)=f(y+xf(y))$ then 
$ {\forall x,y>0: \exists c_{x,y}>0\ such\ that\ \forall h\ between\ x+yf(x)\ and\ y+xf(y)]\ f(x)=c_{x,y}}$ 
\end{mysolution}



\begin{mysolution}[by \href{https://artofproblemsolving.com/community/user/5820}{N.T.TUAN}]
	\begin{tcolorbox}

we suppose that $ \exists x,y>0: \ f(x)<f(y)$ then $ \exists a>0: \ x+af(x)=y+af(y)$
\end{tcolorbox}
Why you have $ a>0$?  :maybe:
\end{mysolution}



\begin{mysolution}[by \href{https://artofproblemsolving.com/community/user/29428}{pco}]


Very nice demo. We can add two parts :
1) f is constant or strictly increasing (which allows the end of your demo)
2) $ c\geq 1$

Demo :
1) f is constant or strictly increasing
1.1) $ f(x)\geq 1$ $ \forall x>0$
Assume we have $ x_{0}>0$ such that $ f(x_{0})<1$. Take then $ y=\frac{x_{0}}{1-f(x_{0}}$. We have $ y>0$ and the equation $ f(x)f(y)=f(x+yf(x))$ becomes :
$ f(x_{0})f(\frac{x_{0}}{1-f(x_{0}})=f(x_{0}+\frac{x_{0}f(x_{0})}{1-f(x_{0}})=f(\frac{x_{0}}{1-f(x_{0}})$
But this is impossible since $ f(x_{0})\neq 1$ and $ f(y)\neq 0$. So $ f(x)\geq 1$ $ \forall x>0$; Q.E.D.

1.2) $ f(x)$ is a non decreasing function.
Let $ a>b>0$. Then let $ x=b$ and $ y=\frac{a-b}{f(b)}$. The equation $ f(x)f(y)=f(x+yf(x))$ becomes : $ f(b)f(y)=f(a)$. And since $ f(y)\geq 1$, we can conclude $ f(a)\geq f(b)$
So $ a>b>0$implies $ f(a)\geq f(b)$ and $ f(x)$ is non decreasing. Q.E.D.

1.3) $ f(x)\equiv 1$ or $ f(x)$ is strictly increasing.
If it exists $ u>0$ such that $ f(u)=1$, then $ x=u$ in the original equation gives $ f(y)=f(y+u)$ $ \forall y>0$ and since $ f(x)$ is a non decreasing function,  $ f(x)$ is a constant and so $ f(x)=1$ $ \forall x>0$
If $ f(x)>1$ $ \forall x>0$, then the equation $ f(b)f(y)=f(a)$ in 1.2 above implies $ f(a)>f(b)$ and $ f(x)$ is strictly increasing.
Q.E.D.

So now, we know that either $ f(x)=1$ $ \forall x>0$, either $ f(x)$ is strictly increasing and so, injective, which allows the implication $ f(x+(c+1)f(x))=f(c+1+xf(c+1))$ $ \implies$ $ c^{2}x+c+1=x+(c+1)f(x)$ in your demonstration.


2) $ c\geq 1$
This is immediatly implied by 1.1 above.

Hope this will complement your demo.
\end{mysolution}



*******************************************************************************
-------------------------------------------------------------------------------

\begin{problem}[Posted by \href{https://artofproblemsolving.com/community/user/15306}{Euler-ls}]
	Find all function $f: \mathbb R^{+} \to \mathbb R^{+}$ satisfying
(a) $ f(f(x)f(y))=xy$ for all $ x, y \in \mathbb R^{+}$, and
(b) $ f(x)\neq x$ for all $ x>1$.
	\flushright \href{https://artofproblemsolving.com/community/c6h161306}{(Link to AoPS)}
\end{problem}



\begin{mysolution}[by \href{https://artofproblemsolving.com/community/user/29428}{pco}]
	\begin{tcolorbox}Find all function $ f: R^{+}\to R^{+}$ satisfying
(a) $ f(f(x)f(y))=xy$ for all $ x, y \in R^{+}$
(b) $ f(x)\neq x$ for all $ x>1$

I think $ f(x)= \frac{1}{x}$... but I don't know how to prove it....\end{tcolorbox}

$ P(x,y)$ : $ f(f(x)f(y))=xy$
$ f(a)=f(b)$ $ \implies$ $ a=f(f(a)f(1))=f(f(b)f(1))=b$ $ \implies$ $ f(x)$ is injective.
$ P(1,1)$ : $ f(f(1)f(1))=1$ and so $ f(a)=1$ with $ a=f(1)^{2}$
$ P(x,a)$ : $ f(f(x))=ax$ and, for example, $ f(f(a))=a^{2}$
$ P(f(a),f(a))$ : $ f(f(f(a))f(f(a)))=f(a)f(a)$ and so $ f(a^{4})=1=f(1)$ and so $ a^{4}=a$ qince $ f(x)$ is injective. So $ a=1$

$ P(f(x)f(y))$ : $ f(f(f(x))f(f(y)))=f(x)f(y)$ $ \implies$ $ f(xy)=f(x)f(y)$

So the equation $ (a)$ is equivalent to involutive solutions of Cauchy's equation $ f(xy)=f(x)f(y)$.

The general involutive solutions of Cauchy's equation $ f(xy)=f(x)f(y)$ may be written $ f(x)=e^{a(\ln(x))-b(\ln(x))}$ where $ a(x)$ and $ b(x)$ are the projections of $ x$ on $ A$ or $ B$ where $ A$ and $ B$ are supplementary $ \mathbb{Q}-vector space$ in $ %Error. "mathBB" is a bad command.
{R}$

With  ${ A=\mathbb{R}}$ and $ B=\{0\}$, we have the solution $ f(x)=x$
With $ A=\{0\}$ and $ B=\mathbb{R}$, we have the solution $ f(x)=\frac{1}{x}$
With axiom of Choice, a lot of other pairs $ (A,B)$ exist and a lot of solutions of equation $ (a)$ exist.

But, if $ A\neq\{0\}$, it exists $ a>0$, $ a\in A$ and then $ f(e^{a})=e^{a}$ with $ e^{a}>1$ and the second condition is not verified.

So the only solution is continuous and is $ f(x)=\frac{1}{x}$
\end{mysolution}
*******************************************************************************
-------------------------------------------------------------------------------

\begin{problem}[Posted by \href{https://artofproblemsolving.com/community/user/23840}{cckek}]
	Find a function $ f: \mathbb{N}\to \mathbb{N}$ such that $f(1)=2$ and for every positive integer $n$,
\[f(f(n))=f(n)+n \quad \text{and} \quad f(n)<f(n+1).\]
	\flushright \href{https://artofproblemsolving.com/community/c6h161454}{(Link to AoPS)}
\end{problem}



\begin{mysolution}[by \href{https://artofproblemsolving.com/community/user/16261}{Rust}]
	Obviosly $ f(n)=[\phi n+1]$, were $ \phi =\frac{1+\sqrt 5 }{2}$ is solution.
I think it posted before.
\end{mysolution}



\begin{mysolution}[by \href{https://artofproblemsolving.com/community/user/29428}{pco}]
	\begin{tcolorbox}Obviosly $ f(n)=[\phi n+1]$, were $ \phi =\frac{1+\sqrt 5 }{2}$ is solution.
I think it posted before.\end{tcolorbox}

Obviosly $ f(n)=[\phi n+1]$, were $ \phi =\frac{1+\sqrt 5 }{2}$ is not a solution : $ f(1)=2$, $ f(2)=4$ and $ f(f(1))\neq f(1)+1$

In fact, an answer could be $ f(n)=[\phi n+\frac{1}{2}]$, were $ \phi =\frac{1+\sqrt 5 }{2}$
\end{mysolution}



\begin{mysolution}[by \href{https://artofproblemsolving.com/community/user/16261}{Rust}]
	\begin{tcolorbox}In fact, an answer could be $ f(n)=[\phi n+\frac{1}{2}]$, were $ \phi =\frac{1+\sqrt 5 }{2}$\end{tcolorbox}
If $ f(n)=[\phi n+c]$, then $ -(\phi-1)-1+n+\phi c<f(f(n))-f(n)=[(\phi-1)[\phi n+c]+c]<n+\phi c$,
 therefore work c, suth that $ c\phi <1$ and $ (\phi-1)(c-1)+c\ge 0$. It give $ 2-\phi\le c<\phi-1$. (c=1/2 work).
But these functional equation had infinetely many solutions. For example $ f(4)$ may be 6 or 7.
\end{mysolution}



\begin{mysolution}[by \href{https://artofproblemsolving.com/community/user/29428}{pco}]
	\begin{tcolorbox}Find a function $ f: \mathbb{N}\to \mathbb{N}$, $ f(1)=2,f(f(n))=f(n)+n,\quad f(n)<f(n+1).$\end{tcolorbox}

In fact  $ f(n)=\lfloor\phi n+a\rfloor$, with $ \phi =\frac{1+\sqrt 5 }{2}$ and $ a\in[2-\phi,\phi-1)$ is a possible solution :

1) $ f(1)=2$ :
$ f(1)=\lfloor\phi+a\rfloor=2$ since $ 2-\phi\leq a<\phi-1<3-\phi$ and so $ 2\leq \phi+a<3$
Q.E.D.

2) $ f(n)<f(n+1)$
We have $ f(n)=\lfloor\phi n+a\rfloor$ $ < \lfloor\phi n+1+a\rfloor$ $ \leq \lfloor\phi n+\phi+a\rfloor = f(n+1)$
Q.E.D.

3) $ f(f(n))=f(n)+n$
Since $ f(n)=\lfloor\phi n+a\rfloor$, we have $ \phi n+a=f(n)+\{f(n)\}$ and so $ f(n)=\phi n+a-\{f(n)\}$
So $ f(f(n))=\lfloor\phi f(n)+a\rfloor$ $ =\lfloor\phi^{2}n+\phi a-\phi\{f(n)\}+a\rfloor$
So $ f(f(n))=\lfloor\phi n+n+\phi a-\phi\{f(n)\}+a\rfloor$ (remember $ \phi^{2}=\phi+1$)
So $ f(f(n))=n+\lfloor(\phi n+a)+\phi a-\phi\{f(n)\}\rfloor$
So $ f(f(n))=n+\lfloor f(n)+\{f(n)\}+\phi a-\phi\{f(n)\}\rfloor$
So $ f(f(n))=f(n)+n+\lfloor \phi a-(\phi-1)\{f(n)\}\rfloor$

It remains to show that $ \lfloor \phi a-(\phi-1)\{f(n)\}\rfloor=0$, which is easy :
$ \{f(n)\}\geq 0$ $ \implies$ $ \phi a-(\phi-1)\{f(n)\}\leq \phi a <\phi(\phi-1)=1$
$ \{f(n)\}< 1$ $ \implies$ $ \phi a-(\phi-1)\{f(n)\}>\phi a-(\phi-1)\geq \phi(2-\phi)-(\phi-1)=0$

So $ 0<\phi a-(\phi-1)\{f(n)\}<1$ and so $ \lfloor \phi a-(\phi-1)\{f(n)\}\rfloor=0$

And so $ f(f(n))=f(n)+n$
Q.E.D.
\end{mysolution}
*******************************************************************************
-------------------------------------------------------------------------------

\begin{problem}[Posted by \href{https://artofproblemsolving.com/community/user/18812}{pohoatza}]
	Find all functions $ f : \mathbb{R}\rightarrow \mathbb{R}$ such that $ f(x^{2}+y+f(y)) = 2y+f^{2}(x)$ for all real numbers $ x$ and $ y$.
	\flushright \href{https://artofproblemsolving.com/community/c6h161554}{(Link to AoPS)}
\end{problem}



\begin{mysolution}[by \href{https://artofproblemsolving.com/community/user/29428}{pco}]
	\begin{tcolorbox}Find every function $ f : \mathbb{R}\rightarrow \mathbb{R}$ such that $ f(x^{2}+y+f(y)) = 2y+f^{2}(x)$ for all real numbers $ x$ and $ y$.\end{tcolorbox}

Here is a rather complex solution (I think some simplier one must exist) :

We have the property $ P(x,y)$ : $ f(x^{2}+y+f(y)) = 2y+f^{2}(x)$

1) $ f(x)$ is surjective :
$ P(x,\frac{a-f^{2}(x)}{2})$ : $ f(\ldots)=a$. Q.E.D.

2) $ f(-x)=f(x)$ or $ f(-x)=-f(x)$
$ P(x,y)$ : $ f(x^{2}+y+f(y)) = 2y+f^{2}(x)$
$ P(-x,y)$ : $ f(x^{2}+y+f(y)) = 2y+f^{2}(-x)$
So $ f^{2}(-x)=f^{2}(x)$.
Q.E.D.

3) $ f(x)=0$ $ \Leftrightarrow$ $ x=0$
Since $ f(x)$ is surjective (see 1). above), it exist $ a$ such that $ f(a)=0$
Then $ P(0,a)$ gives $ 0 = 2a+f^{2}(0)$ and so $ a\leq 0$
But $ f(a)=0$ implies $ f(-a)=0$ (from 2). above) and so, with the same method, $ -a \leq 0$
And so $ a=0$.
Q.E.D.

4) $ f(x^{2})=f^{2}(x)$ and so $ f(x)>0$ $ \forall x>0$
Immediate with $ P(x,0)$.

5) $ f(x+f(x))=2x$ $ \forall x$
Immediate with $ P(0,x)$

6) $ f(x)\leq-x$ $ \forall x\leq 0$
Since $ f(x)\geq 0$ $ \forall x\geq 0$ (see 4.), and since $ f(x+f(x))=2x$ (see 5.), $ f(x+f(x))\leq 0$ $ \forall x\leq 0$ and so $ x+f(x)\leq 0$ $ \forall x\leq 0$
Q.E.D.

7) $ f(-x)=-f(x)$ $ \forall x$
We know (see 2.) that $ f(-x)=f(x)$ or $ f(-x)=-f(x)$ $ \forall x$
Assume then there exists $ a>0$ such that $ f(a)=f(-a)$
$ P(\sqrt{a},-a)$ : $ f(f(a)) =-2a+f(a)$
And since $ a>0$, $ f(a)>0$ and so $ f(f(a))>0$. So $ -2a+f(a)>0$ and $ f(a)>2a$
But $ f(a)=f(-a)$ and $ f(-a)\leq a$ (see 6.). So $ 0<2a<f(a)\leq a$ which is impossible and no such $ a$ exist.
So $ f(-x)=-f(x)$ $ \forall x$
Q.E.D.

8) $ f(x)+x$ is a surjective function.
Let $ x\geq 0$. $ P(\sqrt{\frac{x}{2}},-\frac{x}{2})$ : $ -f(f(\frac{x}{2})) =-x+f(\frac{x}{2})$ and so $ f(f(\frac{x}{2}))+f(\frac{x}{2})= x$
So $ f(x)+x$ may have any nonnegative value, and since $ f(x)+x$ is an odd function, $ f(x)+x$ may have any real value.
Q.E.D

9) $ f(x+y)=f(x)+f(y)$ $ \forall x,y$
From 4. and 5. above, the functional equation may be written : $ f(x^{2}+y+f(y)) = f(x^{2})+f(y+f(y))$
And so, since $ x+f(x)$ is surjective : $ f(x+y)=f(x)+f(y)$ $ \forall x\geq 0$, $ \forall y$
And since $ f(x)$ is an odd function : $ f(x+y)=f(x)+f(y)$ $ \forall x,y$
Q.E.D.

10) $ f(x)=x$ $ \forall x$
From 9., we know that $ f(x)$ is a solution of Cauchy's equation.
From 4., we know that $ f(x)>0$ $ \forall x>0$ and so $ f(x)$ is continuous (all non continuous solutions of Cauchy's equation have neither upper, nor lower bound on any non empty interval).
So $ f(x)=ax$
And, putting back this solution in the original equation, we get $ a=1$

And so $ f(x)=x$
Q.E.D.
\end{mysolution}



\begin{mysolution}[by \href{https://artofproblemsolving.com/community/user/16261}{Rust}]
	very nice.
\end{mysolution}



\begin{mysolution}[by \href{https://artofproblemsolving.com/community/user/9911}{Albanian Eagle}]
	I also posted it some time ago without knowing that it had already been posted three times before
\href{http://www.mathlinks.ro/viewtopic.php?search_id=1056481267&t=75846}{here}
 :)
\end{mysolution}



\begin{mysolution}[by \href{https://artofproblemsolving.com/community/user/18812}{pohoatza}]
	Well the good part is that we would had not seen pco's beautiful solution if not reposting it  :)
\end{mysolution}



\begin{mysolution}[by \href{https://artofproblemsolving.com/community/user/9911}{Albanian Eagle}]
	:maybe: 
Isn't it like Darij's solution at http://www.mathlinks.ro/Forum/viewtopic.php?t=68972 ??
\end{mysolution}
*******************************************************************************
-------------------------------------------------------------------------------

\begin{problem}[Posted by \href{https://artofproblemsolving.com/community/user/29721}{Erken}]
	Find all functions $f: \mathbb R \to \mathbb R$ that satisfy
\[(x+y)(f(x)-f(y))=(x-y)(f(x+y))\]
for all reals $x$ and $y$.
	\flushright \href{https://artofproblemsolving.com/community/c6h161596}{(Link to AoPS)}
\end{problem}



\begin{mysolution}[by \href{https://artofproblemsolving.com/community/user/29428}{pco}]
	\begin{tcolorbox}Find all functions $ f: R\rightarrow R$,that satisfy to the following condition:
For all $ x,y\in R$: $ (x+y)(f(x)-f(y))=(x-y)(f(x)+f(y))$.\end{tcolorbox}

Developping the products, we have $ xf(y)=yf(x)$, or also $ \frac{f(x)}{x}=\frac{f(y)}{y}=c$ $ \forall x,y\neq 0$

And so $ f(x)=cx$ (even for $ x=0$ since $ f(0)=0$
and putting back in the original equation, any value $ c$ fit.

So the general solution is $ f(x)=cx$.
\end{mysolution}



\begin{mysolution}[by \href{https://artofproblemsolving.com/community/user/29721}{Erken}]
	Sorry for my mistake;
Here is the right version:
Find all $ f: R\rightarrow R$,that satisfy the following condition:
For all $ x,y\in R$ we have that $ (x+y)(f(x)-f(y))=(x-y)(f(x+y))$
\end{mysolution}



\begin{mysolution}[by \href{https://artofproblemsolving.com/community/user/29428}{pco}]
	\begin{tcolorbox}Sorry for my mistake;
Here is the right version:
Find all $ f: R\rightarrow R$,that satisfy the following condition:
For all $ x,y\in R$ we have that $ (x+y)(f(x)-f(y))=(x-y)(f(x+y))$\end{tcolorbox}

With $ y=2$, we have $ (i)$ : $ f(x+2)=\frac{x+2}{x-2}(f(x)-f(2))$

With $ y=1$, we have $ f(x+1)=\frac{x+1}{x-1}(f(x)-f(1))$

So $ f(x+2)=\frac{x+2}{x}(f(x+1)-f(1))$

So $ f(x+2)=\frac{x+2}{x}(\frac{x+1}{x-1}(f(x)-f(1))-f(1))$

So $ (ii)$ : $ f(x+2)=\frac{x+2}{x}(\frac{x+1}{x-1}f(x)-2f(1)\frac{x}{x-1})$


Comparing $ (i)$ and $ (ii)$, we get :

$ \frac{x+2}{x-2}(f(x)-f(2))=\frac{x+2}{x}(\frac{x+1}{x-1}f(x)-2f(1)\frac{x}{x-1})$

$ \frac{1}{x-2}(f(x)-f(2))=\frac{x+1}{x(x-1)}f(x)-2f(1)\frac{1}{x-1}$

$ 2f(1)\frac{1}{x-1}-\frac{1}{x-2}f(2)=(\frac{x+1}{x(x-1)}-\frac{1}{x-2})f(x)$

$ f(x)=\frac{f(2)}{2}x(x-1)-f(1)x(x-2)$

$ f(x)=ax^{2}+bx$

Putting back this expression in original equation, we can verify that this solution works for any values of $ (a,b)$
\end{mysolution}



\begin{mysolution}[by \href{https://artofproblemsolving.com/community/user/29721}{Erken}]
	Nice solution,i've solved it the same way .
\end{mysolution}
*******************************************************************************
-------------------------------------------------------------------------------

\begin{problem}[Posted by \href{https://artofproblemsolving.com/community/user/25872}{santosguzella}]
	Find all functions $ f: \mathbb R^{+}\rightarrow \mathbb R^{+}$ such for all $ x,y \in \mathbb R^{+}$, we have
\[ f(x)f(yf(x))=f(x+y).\]
	\flushright \href{https://artofproblemsolving.com/community/c6h161649}{(Link to AoPS)}
\end{problem}



\begin{mysolution}[by \href{https://artofproblemsolving.com/community/user/29428}{pco}]
	\begin{tcolorbox}Let $ R^{+}$ be the set of positive real numbers. Find all functions $ f: R^{+}\rightarrow R^{+}$ such for all $ x,y \in R^{+}$,
\[ f(x)f(yf(x))=f(x+y) \]
\end{tcolorbox}

We have property $ P(x,y)$ : $ f(x)f(yf(x))=f(x+y)$ $ \forall x,y>0$

$ 1).$ $ f(x)\leq 1$ $ \forall x$
If it exists $ a>0$ such that $ f(a)>1$. 
Then $ P(a,\frac{a}{f(a)-1})$ gives $ f(a)f(\frac{af(a)}{f(a)-1})=f(\frac{af(a)}{f(a)-1})$ which is impossible since $ f(a)\neq 1$ and $ f(\frac{af(a)}{f(a)-1})\neq 0$. So $ f(x)\leq 1$ $ \forall x>0$
Q.E.D.

$ 2).$ Either $ f(x)=1$ $ \forall x>0$, either $ f(x)<1$ $ \forall x>0$ and $ f(x)$ is strictly decreasing, and so is an injective function.
$ f(x)f(yf(x))=f(x+y)$ and $ f(yf(x))\leq 1$ imply $ f(x+y)\leq f(x)$ and $ f(x)$ is a non increasing function.
If it exists $ a>0$ such that $ f(a)=1$, then $ P(a,x)$ gives $ f(x)=f(x+a)$ $ \forall x$ and so, since $ f(x)$ is non increasing, $ f(x)=1$ $ \forall x>0$
If it does not exist any $ a>0$ such that $ f(a)=1$, then $ f(x)<1$ $ \forall x>0$ and then $ f(x)f(yf(x))=f(x+y)$ and $ f(yf(x))< 1$ imply $ f(x+y)< f(x)$ and $ f(x)$ is a strictly decreasing function, and so $ f(x)$ is injective.
Q.E.D.

$ 3).$ If $ f(x)\neq 1$, then $ f(x)=\frac{1}{ax+1}$ for some $ a>0$
Assume $ f(x)\neq 1$ $ \forall x$ and let $ x>y>0$ and compare $ P(x,\frac{y}{f(x)})$ and $ P(y,x-y+\frac{y}{f(x)})$ :

$ P(x,\frac{y}{f(x)})$ gives $ f(x)f(y)=f(x+\frac{y}{f(x)})$

$ P(y,x-y+\frac{y}{f(x)})$ gives $ f(y)f((x-y+\frac{y}{f(x)})f(y))=f(x+\frac{y}{f(x)})$

So $ f(x)f(y)=f(y)f((x-y+\frac{y}{f(x)})f(y))$ and, since $ f(x)$ is injective : $ x=(x-y+\frac{y}{f(x)})f(y)$

So  $ \frac{x}{f(y)}-x=\frac{y}{f(x)}-y$

So $ \frac{1}{yf(y)}-\frac{1}{y}=\frac{1}{xf(x)}-\frac{1}{x}$

So $ \frac{1}{xf(x)}-\frac{1}{x}$ is a constant $ a$ and we have :

$ f(x)=\frac{1}{ax+1}$
Putting back this expression in original equation, we find that any value $ a$ fit. But, we know that $ f(x)>0$ $ \forall x>0$ and so we need $ a>0$. (inequality is strict since we assumed $ f(x)\neq 1)$
Q.E.D.


So the general solution of requested equation is $ f(x)=\frac{1}{ax+1}$ for any $ a\geq 0$ (the case $ a=0$ gives the solution $ f(x)\equiv 1$)
\end{mysolution}
*******************************************************************************
-------------------------------------------------------------------------------

\begin{problem}[Posted by \href{https://artofproblemsolving.com/community/user/22804}{nayel}]
	Let $ f: \mathbb R\to \mathbb R$ be a non-constant and continuous function such that \[ f(x+y)+f(x-y)=2f(x)f(y),\] for all $ x,y$ in $ \mathbb R$. Assume that there exists $ a\in\mathbb R$ such that $ f(a)<1$ Show that there exists infinitely many real numbers $ r$ such that $ f(r)=0$.
	\flushright \href{https://artofproblemsolving.com/community/c6h161693}{(Link to AoPS)}
\end{problem}



\begin{mysolution}[by \href{https://artofproblemsolving.com/community/user/29428}{pco}]
	\begin{tcolorbox}Let $ f: \mathbf R\to \mathbf R$ be a non-constant and continuous function such that
\[ f(x+y)+f(x-y)=2f(x)f(y) \]
for all $ x,y$ in $ \mathbf R$. Show that there exists infinitely many real numbers $ r$ such that $ f(r)=0$.\end{tcolorbox}

I'm afraid it's wrong : take $ f(x)=\cosh(x)$
\end{mysolution}



\begin{mysolution}[by \href{https://artofproblemsolving.com/community/user/22804}{nayel}]
	I think this one is true(thanks pco):

Let $ f: \mathbf R\to \mathbf R$ be continuous such that \[ f(x+y)+f(x-y)=2f(x)f(y)\qquad(1)\]for all $ x,y$ in $ \mathbf R$. Assume that there exists $ a\in\mathbf R$ such that $ f(a)<1$. Prove that there exists infinitely many real numbers $ r$ such that $ f(r)=0$.
\end{mysolution}



\begin{mysolution}[by \href{https://artofproblemsolving.com/community/user/32514}{TTsphn}]
	In this case $ f(x)=bcox$ and $ bcosa<1$
\end{mysolution}



\begin{mysolution}[by \href{https://artofproblemsolving.com/community/user/22804}{nayel}]
	I think It would be helpful if you post your solution. Actually I didn't pretty much understand what you mean. :maybe:
\end{mysolution}



\begin{mysolution}[by \href{https://artofproblemsolving.com/community/user/29126}{MellowMelon}]
	[hide="Solution with small amount of ugliness"]
With $ y = 0$, we find that either $ f(x) = 0$, solving the problem immediately, or $ f(0) = 1$. So assume the latter.

\begin{bolded}Lemma\end{bolded}: There exists a $ c\neq 0$ such that $ f(c) = 0$.
Proof: Consider the value of $ a$ such that $ f(a) < 1$. If $ f(x) < 0$ for any $ x$, then by $ f(0) = 1$ and $ f$ continuous, $ f$ must have a nonzero root, proving the lemma. So assume $ 0 < f(a) < 1$. $ x = y = a$ gives $ f(2a)+1 = 2f(a)^{2}$. If $ f(a) <\sqrt{2}/2$, we can get that $ f(2a)$ is negative, finishing the problem. So assume $ f(a) >\sqrt{2}/2$. Let $ f(a) = 1-\alpha$ for an $ \alpha$ that gives $ \sqrt{2}/2 < f(a) < 1$. Then $ f(2a)+1 = 2-4\alpha+2\alpha^{2}$. Since $ \alpha^{2}<\alpha$, $ f(2a) = 1-4\alpha+2\alpha^{2}< 1-2\alpha = f(a)-\alpha$. This shows that $ f(2a) < f(a)-\alpha$. So then if we consider $ f(2a) = 1-\beta$, we get that $ \beta < 2\alpha$. This means that $ f(2^{n}a) < f(a)-2^{n}\alpha$, and since $ \alpha$ is positive, we must eventually get an $ f(2^{n}a) <\sqrt{2}/2$. Then $ f(2^{n+1}a) < 0$, and we have proved the lemma.
[hide="Remark"]
There has got to be a better way to prove this lemma. The method here seemed cleaner than doing calculus on $ 2x^{2}-1$ or $ 2x^{2}-x-1$, but perhaps I'm not thinking along the best lines in general.


Now that we have our $ c$, plug $ x = y = c$ in. $ f(2c)+1 = 0$ so $ f(2c) =-1$. Now plug $ x = 2c, y = c$ in. $ f(3c) = 0$. So then $ f(3^{n}c) = 0$, finishing the proof.

\end{mysolution}



\begin{mysolution}[by \href{https://artofproblemsolving.com/community/user/22804}{nayel}]
	Hey that is like *exactly* my solution!!! How did you guess? :D
\end{mysolution}



\begin{mysolution}[by \href{https://artofproblemsolving.com/community/user/22804}{nayel}]
	My solution: First assume that $ \exists r: f(r) = 0$. Now we easily get $ f(2r) =-1,f(3r) = 0$, inductively implying $ f(3^{n}r) = 0$. Thus there is an infinite number of $ r$s that satisfy $ f(r) = 0$.

Now we will prove that $ \exists r: f(r) = 0$. Put $ y = 0$ in $ (1)$ to get $ f(0) = 1$ and $ x = y$ to get $ f(2x)+1 = (f(x))^{2}$. If $ f(x) < 0$ for some real $ x$, we are done by continuity as $ f(0) = 1$. So assume that $ f(x) > 0\ \forall x\in\mathbf R$.

Let $ 0 < f(a) < 1$. Then $ f(2a) = 2(f(a))^{2}-1 < 2f(a)-1$. An inductive argument shows that $ f(2^{n}a) < 2^{n}f(a)-2^{n}+1\Longrightarrow f(2^{n}a)-f(a) < (2^{n}-1)(f(a)-1)$. Now as $ f(a)-1 < 0$, we can choose $ n$ large enough so that $ f(2^{n}a)-f(a) <-1\Longrightarrow f(2^{n}a) < 0$. So we are done, because $ f(2^{n}a) < 0, f(0) = 1$ and by continuity there exists $ r$ such that $ f(r) = 0$.
\end{mysolution}
*******************************************************************************
-------------------------------------------------------------------------------

\begin{problem}[Posted by \href{https://artofproblemsolving.com/community/user/19941}{delegat}]
	Find all functions $ f: \mathbb{R^+}\to\mathbb{R^+}$   which satisfy the identity:
\[f(x)f(y)=f(x+yf(x))\] for all positive reals $ x$ and $ y$
	\flushright \href{https://artofproblemsolving.com/community/c6h161743}{(Link to AoPS)}
\end{problem}



\begin{mysolution}[by \href{https://artofproblemsolving.com/community/user/29126}{MellowMelon}]
	[hide="Incomplete solution - one large hole"]
Observe first that $ f(x) = ax+1$ for all nonnegative $ a$ works.

First, suppose that $ f(a) = f(b) = c$. Then $ cf(y) = f(a+yc) = f(b+yc)$.

Suppose $ a \neq b$, then WLOG $ a > b$. Let $ d = \frac{a-b}{c}$. Then we get that $ cf(y) = f(a+yc) = f(b+(y+d))c = cf(y+d)$. Thus $ f$ is periodic with period $ d$.

You can use the same argument above to show that $ f(x+d/f(x)) = f(x)$ for all $ x$. Assume there is an $ x$ with $ f(x) > 1$ and apply this identity infinitely many times to get a contradiction. So now $ f(x) \leq 1$. Here's the part of the solution I haven't finished: I am pretty sure there is a way to proceed from here to get $ f(x) = 1$ for all $ x$, but I haven't figured out exactly how. (though it's entirely possible I missed an answer for when $ f(x) < 1$ for some $ x$)

Now we look for solutions other than those above, so we must remove our assumption that $ a \neq b$ above. Therefore $ a = b$ and $ f$ is injective. Observe $ f(x+yf(x)) = f(x)f(y) = f(y)f(x) = f(y+xf(y))$. By $ f$ injective, $ x+yf(x) = y+xf(y)$. Then plug in $ y = 1$ to get $ f(x) = 1+xf(1)-x$. Let $ f(1)$ be any value greater than 1 and we get the remaining solutions: $ f(x) = ax+1$ for $ a$ positive.

\end{mysolution}



\begin{mysolution}[by \href{https://artofproblemsolving.com/community/user/29428}{pco}]
	\begin{tcolorbox}Find all functions $ f: \mathbb{R+}\to\mathbb{R+}$   which satisfy the following identity:
$ f(x)f(y)=f(x+yf(x))$ for all positive reals $ x$ and $ y$\end{tcolorbox}

We have the property $ P(x,y)$ : $ f(x)f(y)=f(x+yf(x))$ $ \forall x,y>0$

$ 1).$ $ f(x)\geq 1$ $ \forall x>0$
Let $ a>0$ such that $ 0<f(a)<1$. Then $ \frac{a}{1-f(a)}>0$ and $ P(a,\frac{a}{1-f(a)})$ gives $ f(a)f(\frac{a}{1-f(a)})=f(\frac{a}{1-f(a)})$ which is impossible since $ f(a)\neq 1$ and $ f(\frac{a}{1-f(a)})\neq 0$. So, such $ a$ don't exist and $ f(x)\geq 1$ $ \forall x>0$. 
Q.E.D.

$ 2).$ $ f(x)$ is a non decreasing function.
Since $ f(x)\geq 1$ $ \forall x>0$, $ P(x,\frac{h}{f(x)})$ gives $ f(x)\leq f(x)f(\frac{h}{f(x)})=f(x+h)$.
Q.E.D.

$ 3).$ Either $ f(x)=1$ $ \forall x>0$, either $ f(x)$ is a strictly increasing function, and so is injective.
If it exists $ a>0$ such that $ f(a)=1$, then $ P(a,x)$ gives $ f(x)=f(x+a)$ and, since $ f(x)$ is non decreasing, $ f(x)=1$ $ \forall x>0$
If $ f(x)>1$ $ \forall x>0$, $ P(x,\frac{h}{f(x)})$ gives $ f(x)< f(x)f(\frac{h}{f(x)})=f(x+h)$.
Q.E.D.

$ 4).$ If $ f(x)\neq 1$, then $ f(x)=ax+1$ for some $ a>0$
If $ f(x)\neq 1$, $ f(x)$ is injective. Compare then $ P(x,y)$ and $ P(y,x)$ :
$ P(x,y)$ : $ f(x)f(y)=f(x+yf(x))$ 
$ P(y,x)$ : $ f(y)f(x)=f(y+xf(y))$ 
So $ f(x+yf(x))=f(y+xf(y))$ and so ($ f(x)$ is injective) $ x+yf(x)=y+xf(y)$
So $ \frac{f(x)}{x}-\frac{1}{x}=\frac{f(y)}{y}-\frac{1}{y}$
So $ \frac{f(x)}{x}-\frac{1}{x}$ is the constant function and $ \frac{f(x)}{x}-\frac{1}{x}=a$ and so $ f(x)=ax+1$
Putting back this expression in $ P(x,y)$, we get that any value $ a$ fit but we need to have $ a>0$ in order to have $ f(x)>0$ and $ f(x)\neq 1$
Q.E.D.

The general solution of the equation is $ f(x)=ax+1$ for any value $ a\geq 0$ (the value 0 gives the constant solution $ f(x)\equiv 1$)
\end{mysolution}



\begin{mysolution}[by \href{https://artofproblemsolving.com/community/user/30326}{quangpbc}]
	Solution :
Suppose there exists $ x_{0}$ so that $ f(x_{0})<1$ .Let $ y=\frac{x_{0}}{1-f(x_{0})}$ ,we have $ f(x_{0}).f(\frac{x_{0}}{1-f(x_{0})})=f(\frac{x_{0}}{1-f(x_{0})})\to f(x_{0})=1$ contracdition.
Then $ f(x)\ge 1$ for all $ x>0$ . Now we have two cases :

Case 1 . There exists $ x_{1}$ so that $ f(x_{1})=1$ .
For any $ u( 0<u<x_{1})$ there exists $ v( 0<v<x_{1})$ so that $ u+vf(u)=x_{1}$
Then $ f(u).f(v)=f(x_{1})=1$ but $ f(u),f(v)\ge 1$ so $ f(u)=f(v)=1$ .
Hence $ f(x)=1$ for all $ 0<x\le x_{1}$.
In the equation , replace $ x$ by $ x_{0}$ we have $ f(x_{1}+y)=f(y)$ so $ f(x)$ is a circulation fucntion with cycle less than $ x_{1}$ . Then we have $ f(x)=1$ for all $ x\in \mathbb{R}^{+}$ .

Case 2 : $ f(x)>1$ for all $ x\in \mathbb{R}^{+}$ . For any pairs $ (x,y)$ with $ x>y$ there exists $ t$ so that $ x=y+tf(y)$ .Then $ f(x)=f(y).f(t)\to f(x)>f(y)$ so $ f(x)$ is increasing fucntion 
In the equation replace $ x$ by $ y$ we have :$ f(x).f(y)=f(x+yf(x))=f(y+xf(y))$.
Then $ x+yf(x)=y+xf(y)$.
Hence $ \frac{f(x)-1}{x}=constant$ .We get $ f(x)=ax+1$ for any value $ a\ge 0$  .
\end{mysolution}
*******************************************************************************
-------------------------------------------------------------------------------

\begin{problem}[Posted by \href{https://artofproblemsolving.com/community/user/25017}{Leonhard Euler}]
	Find all functions $f: \mathbb N \to \mathbb N$ such that $ f^{19}(n)+97f(n)=98n+232$. Notice in this problem that $f^{19}(n)$ means the composition of $f$ with itself $19$ times.
	\flushright \href{https://artofproblemsolving.com/community/c6h161994}{(Link to AoPS)}
\end{problem}



\begin{mysolution}[by \href{https://artofproblemsolving.com/community/user/29428}{pco}]
	\begin{tcolorbox}Find all functions f:N->N such that $ f^{19}(n)+97f(n)=98n+232$\end{tcolorbox}

I consider that $ f^{19}(n)$ is the composition of $ f(n)$ $ 19$ times (and not the $ 19^{th}$ power; in such a case, there are no solution).

First, we have $ 97f(n)<98n+232$ and so $ f(n)<\frac{98}{97}n+\frac{232}{97}$ $ \forall n$ and so $ f^{19}(n)<(\frac{98}{97})^{19}n+232((\frac{98}{97})^{19}-1)$

But $ f^{19}(n)<(\frac{98}{97})^{19}n+232((\frac{98}{97})^{19}-1)$ implies $ 97f(n)>98n+232-(\frac{98}{97})^{19}n-232((\frac{98}{97})^{19}-1)$

So $ (\frac{98}{97}-\frac{1}{97}(\frac{98}{97})^{19})n+\frac{232}{97}(2-(\frac{98}{97})^{19})<f(n)<\frac{98}{97}n+\frac{232}{97}$ $ \forall n$ which implies :

$ 0.997n+1.877<f(n)<1.011n+2.392$ $ \forall n$ or $ n+2-(0.003n+0.123)<f(n)<n+2+(0.011n+0.392)$

This implies $ f(n)=n+2$ $ \forall n<40$ and now it is easy to extend with induction :

If $ f(p)=p+2$ $ \forall p<n$ for $ n>38$, then :
$ f(n-38)=n-36$
$ f^{19}(n-38)+97f(n-38)=98(n-38)+232$ and so $ f^{19}(n-38)+97(n-36)=98(n-38)+232$ and so $ f^{19}(n-38)=n$
$ f^{19}(n-36)+97f(n-36)=98(n-36)+232$ and so $ f^{19}(n-36)+97(n-34)=98(n-36)+232$ and so $ f^{19}(n-36)=n+2$

So $ n+2=f^{19}(n-36)=f^{19}(f(n-38))=f(f^{19}(n-38))=f(n)$ and we have $ f(n)=n+2$ and so the induction is OK.


So the unique solution could be $ f(n)=n+2$ (and an immediate verification in original equation shows that this solution fit).
\end{mysolution}
*******************************************************************************
-------------------------------------------------------------------------------

\begin{problem}[Posted by \href{https://artofproblemsolving.com/community/user/25017}{Leonhard Euler}]
	Find all functions $f: \mathbb R \to \mathbb R$ such that \[f(x+f(x)+y)=f(y)+2x\] holds for all $x,y \in \mathbb R$.
	\flushright \href{https://artofproblemsolving.com/community/c6h162096}{(Link to AoPS)}
\end{problem}



\begin{mysolution}[by \href{https://artofproblemsolving.com/community/user/29428}{pco}]
	\begin{tcolorbox}Find all functions f:R->R such that $ f(x+f(x)+y)=f(y)+2x$\end{tcolorbox}

We have the property $ P(x,y)$ : $ f(x+f(x)+y)=f(y)+2x$ $ \forall x,y$

$ 1).$ $ f(x)$ is bijective
$ P(\frac{a-f(0)}{2},0)$ : $ f(\ldots)=a$ and $ f(x)$ is surjective
$ P(x,-f(x))$ : $ f(x)=f(-f(x))+2x$ and so $ f(x_{1})=f(x_{2})$ implies $ x_{1}=x_{2}$ and $ f(x)$ is injective
Q.E.D.

$ 2.)$ $ f(0)=0$
$ P(0,x)$ : $ f(x+f(0))=f(x)$ and, since $ f(x)$ is injective, $ f(0)=0$
Q.E.D.

$ 3.)$ $ x+f(x)$ is bijective
$ P(x,0)$ : $ f(x+f(x))=2x$ and so $ x+f(x)=f^{-1}(2x)$ and $ x+f(x)$ is bijective.
Q.E.D.

$ 4).$ $ f(x+y)=f(x)+f(y)$ 
Since $ f(x+f(x))=2x$, $ P(x,y)$ may be written $ f(x+f(x)+y)=f(y)+f(x+f(x))$ and since $ x+f(x)$ is surjective, for any $ z$ it exists $ x$ such that $ x+f(x)=z$ and so $ f(z+y)=f(y)+f(z)$.
Q.E.D.

$ 5).$ the only continuous solutions are $ f(x)=x$ and $ f(x)=-2x$
The only continuous solutions of this Cauchy's equation are $ f(x)=ax$ and so we have $ ax+a^{2}x+ay=ay+2x$ and so $ a^{2}+a-2=0$ and so $ a=1$ or $ a=-2$
Q.E.D.

$ 6).$ With axiom of Choice, it exists non continuous solutions.
Using (classical way) a base $ \{x_{i}\}$ for the $ \mathbb{Q}$-vector space $ \mathbb{R}$, it is immediate to see that $ f(x_{i})=x_{i}$ or $ f(x_{i})=-2x_{i}$.
So we can build two $ \mathbb{Q}$-vector space $ A$ and $ B$ such that $ A\cap B=\{0\}$ and $ A+B=\mathbb{R}$ ($ A$ using the part of the base such that $ f(x_{i})=x_{i}$ and $ B$ using the part of the base such that $ f(x_{i})=-2x_{i})$.
Then, if $ a(x)$ and $ b(x)$ are the projections of a real $ x$ on $ A$ and $ B$ (such that there is a unique decomposition $ x=a(x)+b(x)$ with $ a(x)\in A$ and $ b(x)\in B$, then $ f(x)=a(x)-2b(x)$


So the general solution of the original equation is the following one :

Let $ A$ and $ B$  be two $ \mathbb{Q}$-vector space $ A$ and $ B$ such that $ A\cap B=\{0\}$ and $ A+B=\mathbb{R}$
Let $ a(x)$ and $ b(x)$ be the projections of a real $ x$ on $ A$ and $ B$ (such that there is a unique decomposition $ x=a(x)+b(x)$ with $ a(x)\in A$ and $ b(x)\in B$

Then $ f(x)=a(x)-2b(x)$ is a solution (and all solutions are of this form).

If $ A=\mathbb{R}$ and $ B=\{0\}$, then we have the continuous solution $ f(x)=x$
If $ A=\{0\}$ and $ B=\mathbb{R}$, then we have the continuous solution $ f(x)=-2x$
With axiom of Choice, a lot of other pairs $ (A,B)$ exist and so a lot of non continuous solutions.
\end{mysolution}
*******************************************************************************
-------------------------------------------------------------------------------

\begin{problem}[Posted by \href{https://artofproblemsolving.com/community/user/32309}{mathemagics}]
	Find all functions $ f:\mathbb{R}-[0,1]\to\mathbb{R}$ such that 
\[f(x)+f\left(\frac{1}{1-x}\right) = \frac{2(1-2x)}{x(1-x)},\]
for all $x \in \mathbb R$.
	\flushright \href{https://artofproblemsolving.com/community/c6h162892}{(Link to AoPS)}
\end{problem}



\begin{mysolution}[by \href{https://artofproblemsolving.com/community/user/16261}{Rust}]
	\begin{tcolorbox}Find all functions $ f:\mathbb{R}-[0,1]\to\mathbb{R}$ such that 

$ f(x)+f\left(\frac{1}{1-x}\right) =\frac{2(1-2x)}{x(1-x)}$

India Competition\end{tcolorbox}
1. $ f(x)+f(\frac{1}{1-x}) =\frac{2}{x}+\frac{2}{x-1}$.
$ x =\frac{1}{1-y}, y\to x$ give
2. $ f(\frac{1}{1-x})+f(\frac{x-1}{x}) =\frac{2}{x}-2x$
$ x =\frac{y-1}{y}, y\to x$ give
3. $ f(\frac{x-1}{x})+f(x) =\frac{2x}{x-1}-2x$
(1.+3.-2.)/2 give
$ f(x) =\frac{x+1}{x-1}$.
\end{mysolution}



\begin{mysolution}[by \href{https://artofproblemsolving.com/community/user/32087}{asdrojas}]

Rust see that, in (*) $ x$ must be greater than 1, so by (**) $ 0<y<1$ but this implies that $ \frac{1}{2}<\frac{1}{1-x}<1$ in (***) but it's out of the domaind of $ f$
\end{mysolution}



\begin{mysolution}[by \href{https://artofproblemsolving.com/community/user/32087}{asdrojas}]
	\begin{tcolorbox}Find all functions $ f:\mathbb{R}-[0,1]\to\mathbb{R}$ such that 

$ f(x)+f\left(\frac{1}{1-x}\right) =\frac{2(1-2x)}{x(1-x)}$ (*)

India Competition\end{tcolorbox}

The domain of $ f$ is $ \mathbb{R}-[0,1]$ so in the condition (*) we must have $ x\in (-\infty,0)\cup(1,\infty)$ and $ \frac{1}{1-x}\in (-\infty,0)\cup(1,\infty)$ that implies $ x>1$ and $ \frac{1}{1-x}<0$.
(*) is equivalent to
$ f(x)+f\left(\frac{1}{1-x}\right) =\frac{2}{x-1}+\frac{2}{x}$
$ =\left(1+\frac{2}{x-1}\right)+\left(\frac{2}{x}-1\right)$
$ =\left(1+\frac{2}{x-1}\right)+\left(1+\frac{2}{\frac{1}{1-x}-1}\right)$
then
$ f(x)-\left(1+\frac{2}{x-1}\right)+f\left(\frac{1}{1-x}\right)-\left(1+\frac{2}{\frac{1}{1-x}-1}\right) =0$
Let $ g(x)=f(x)-\left(1+\frac{2}{x-1}\right)$
then
$ g(x)+g\left(\frac{1}{1-x}\right)=0$
$ g(x)=-g\left(\frac{1}{1-x}\right)$
Now if $ x$ goes from 1 to $ \infty$, $ \frac{1}{1-x}$ goes from -$ \infty$ to 0, so we can set $ g(x)$ an arbitrary function for $ x>1$, and construct $ g$ for negatives values as $ g(x)=-g\left(\frac{1}{1-x}\right)$, so $ f(x)=g(x)+1+\frac{2}{x-1}$
\end{mysolution}



\begin{mysolution}[by \href{https://artofproblemsolving.com/community/user/9882}{Virgil Nicula}]

\begin{bolded}Proof.\end{bolded} Denote the function $ g: A\to\mathbb R$, where $ g(x) =\frac{2(1-2x)}{x(1-x)}$ and the function $ \phi : A\to\mathbb R$, where $ \phi (x) =\frac{1}{1-x}$.
Observe that the function $ \phi$ is injective and $ \mathrm{Im}\phi\equiv\phi (A) = A$, i.e. the function $ \phi : A\to A$ is bijective (it is a \begin{bolded}dynamic function\end{bolded}).
Prove easily that $ \phi\circ\phi\circ\phi = 1_{A}$ (the identical function) and $ \left\{\begin{array}{c}f+f\circ\phi = g\\ \ f\circ\phi+f\circ\phi\circ\phi = g\circ\phi\\ \ f\circ\phi\circ\phi+f = g\circ\phi\circ\phi\end{array}\right\|$ $ \implies$ $ f =\frac{1}{2}\cdot (g-g\circ\phi+g\circ\phi\circ\phi )$ a.s.o.
\end{mysolution}



\begin{mysolution}[by \href{https://artofproblemsolving.com/community/user/32087}{asdrojas}]

Virgil $ \phi$ is not bijective, supose $ x =-1$ then $ x\in A$ but $ \phi(x) =\frac{1}{2}\notin A$
\end{mysolution}



\begin{mysolution}[by \href{https://artofproblemsolving.com/community/user/29428}{pco}]
	\begin{tcolorbox}Find all functions $ f:\mathbb{R}-[0,1]\to\mathbb{R}$ such that 

$ f(x)+f\left(\frac{1}{1-x}\right) =\frac{2(1-2x)}{x(1-x)}$

India Competition\end{tcolorbox}

The problem is not well defined, as the scope of values for which the equation is verified is not given.

By the way, I think there is a typo in the problem and the domain is $ f:\mathbb{R}-\{0,1\}\to\mathbb{R}$ instead of $ f:\mathbb{R}-[0,1]\to\mathbb{R}$.

Then Rust's solution would be the good one.
\end{mysolution}



\begin{mysolution}[by \href{https://artofproblemsolving.com/community/user/32309}{mathemagics}]
This is India 1992 and the problem was proposed with the domain $ \mathbb{R}-[0,1]$. Do you have any book that contains Indian Olympiads to check it ?
\end{mysolution}
*******************************************************************************
-------------------------------------------------------------------------------

\begin{problem}[Posted by \href{https://artofproblemsolving.com/community/user/25017}{Leonhard Euler}]
	Show that there exist function $ f: \mathbb N \to \mathbb N$ such that $ f(f(n)) = n^{2}$.
	\flushright \href{https://artofproblemsolving.com/community/c6h162930}{(Link to AoPS)}
\end{problem}



\begin{mysolution}[by \href{https://artofproblemsolving.com/community/user/29428}{pco}]
	\begin{tcolorbox}Show that there exist function $ f: n\to n$ such that $ f(f(n)) = n^{2}$\end{tcolorbox}

Let $ A$ be the set of positive integers which are not perfect squares.
Let $ B$ and $ C$ two disjoint infinite subsets of $ A$ such that $ A=B\cup C$ and  $ h(n)$ a bijective function from $ B$ to $ C$.
Let $ k(n)$ the function which gives, for any integer $ n>1$, the littlest integer $ m$ such that it exists an integer $ p\geq 0$ such that $ n=m^{2^{p}}$. 
Let $ e(n)$ the integer $ p\geq 0$ such that $ n=k(n)^{2^{p}}$
Obviously, $ k(n)$ is not a perfect square and either $ k(n)\in B$, either $ k(n)\in C$

Then we can define $ f(n)$ :
$ f(1)=1$
$ \forall n>1$ :
if $ k(n)\in B$, $ f(n)=(h(k(n)))^{2^{e(n)}}$
if $ k(n)\in C$, $ f(n)=(h^{-1}(k(n)))^{2^{e(n)+1}}$
\end{mysolution}



\begin{mysolution}[by \href{https://artofproblemsolving.com/community/user/13985}{BaBaK Ghalebi}]
\url{https://artofproblemsolving.com/community/c6h128748}
\url{https://artofproblemsolving.com/community/c6h24087}
\end{mysolution}
*******************************************************************************
-------------------------------------------------------------------------------

\begin{problem}[Posted by \href{https://artofproblemsolving.com/community/user/26272}{sylow\_theory}]
	Let $ f(x) =x^{n}+ \cdots + x + 1$ and let $ g(x) = f(x^{n+1})$. Find the remainder when $ g(x)$ is divided by $ f(x)$.
	\flushright \href{https://artofproblemsolving.com/community/c6h163042}{(Link to AoPS)}
\end{problem}



\begin{mysolution}[by \href{https://artofproblemsolving.com/community/user/12947}{minsoens}]
	I have a solution, but I must admit that I have seen a similar problem before..
$ f(x)=\sum_{i=0}^{n}x^{i}\implies f(x^{n+1})=\sum_{i=0}^{n}x^{(n+1)i}=n+1+\sum_{i=0}^{n}\left(x^{(n+1)i}-1\right)$. 
\begin{eqnarray*}x^{(n+1)i}-1&=&\left(x^{n+1}-1\right)\left(x^{(n+1)(i-1)}+x^{(n+1)(i-2)}+\cdots+x^{n+1}+1\right)\\ &=&\left(x-1\right)\left(x^{n}+x^{n-1}+\cdots+x+1\right)\left(x^{(n+1)(i-1)}+x^{(n+1)(i-2)}+\cdots+x^{n+1}+1\right)\\ &=&\left(x-1\right)f(x)\left(x^{(n+1)(i-1)}+x^{(n+1)(i-2)}+\cdots+x^{n+1}+1\right)\end{eqnarray*}
So all of $ x^{(n+1)i}-1$ in $ n+1+\sum_{i=0}^{n}\left(x^{(n+1)i}-1\right)$ are divisible by $ f(x)$. Therefore the remainder is $ \boxed{n+1}$.
Have I made any mistake?  :maybe:
\end{mysolution}



\begin{mysolution}[by \href{https://artofproblemsolving.com/community/user/26272}{sylow\_theory}]
	\begin{tcolorbox} Therefore the remainder is $ \boxed{n+1}$.
Have I made any mistake?  :maybe:\end{tcolorbox}
I forgot what the answer is  :P , but it looks right (to lazy to check).
But, I have a really short solution.  :ninja:
\end{mysolution}



\begin{mysolution}[by \href{https://artofproblemsolving.com/community/user/8425}{K81o7}]
	\begin{tcolorbox}
But, I have a really short solution.  :ninja:\end{tcolorbox}

Would this be it, by any chance?  :D 

$ f(x)|x^{n+1}-1|f(x^{n+1})-f(1) = g(x)-f(1)\implies g(x)\equiv f(1)\equiv n+1\mod{f(x)}$
\end{mysolution}



\begin{mysolution}[by \href{https://artofproblemsolving.com/community/user/29428}{pco}]
	\begin{tcolorbox}I was doing a problem, and I generalized it. Its not so hard, but its a nice result.

Let $ f(x) = x^{n}+...+x+1$ and let $ g(x) = f(x^{n+1})$. Find the remainder when $ g(x)$ is divided by $ f(x)$.\end{tcolorbox}

[hide="Short solution using complex numbers"]
Let $ z_{k}=e^{\frac{2ki\pi}{n+1}}$ for $ k\in\{1,2,\ldots,n\}$. We have $ z_{k}^{n+1}=1$ and $ f(z_{k})=0$ $ \forall k\in\{1,2,\ldots,n\}$

We have $ g(x)=f(x)q(x)+r(x)$ with degree of $ r(x)$ $ <n$. So $ g(z_{k})=f(z_{k}^{n+1})=f(1)=n+1=r(z_{k})$

So $ r(x)$ is a polynomial of degree $ \leq n-1$ and taking the same value for $ n$ different values. So $ r(x)$ is a constant.

And $ r(x)=n+1$
\end{mysolution}



\begin{mysolution}[by \href{https://artofproblemsolving.com/community/user/26272}{sylow\_theory}]
	Good solutions guys :)

But I did it like Patrick. Write it like $ g(x)=q(x)f(x)+r(x)$ and work with $ (n+1)$-th roots of unity. Substitute those in and claim $ r(x)$ is a constant polynomial.
\end{mysolution}
*******************************************************************************
-------------------------------------------------------------------------------

\begin{problem}[Posted by \href{https://artofproblemsolving.com/community/user/5820}{N.T.TUAN}]
	Let $ 0<a<1$ and $ I=(0,a)$. Find all functions $ f: I\to\mathbb{R}$ satisfying at least one of the conditions below:
a)$ f$ is continuous on $ I$ and $ f(xy)=xf(y)+yf(x)$ for all $x,y\in I$.
b)$ f(xy)=xf(x)+yf(y)$ for all $x,y\in I$.
	\flushright \href{https://artofproblemsolving.com/community/c6h163498}{(Link to AoPS)}
\end{problem}



\begin{mysolution}[by \href{https://artofproblemsolving.com/community/user/29428}{pco}]
	\begin{tcolorbox}Let $ 0 < a < 1$ and $ I = (0,a)$. Find all functions $ f: I\to\mathbb{R}$ satisfying at least one of the conditions below:
a)$ f$ is continuous on $ I$ and $ f(xy) = xf(y)+yf(x)\forall x,y\in I$.
b)$ f(xy) = xf(x)+yf(y)\forall x,y\in I.$\end{tcolorbox}

a) : Let $ g(x)=a^{-x}f(a^{x})$. We have $ g$ continuous,  defined on $ (1,+\infty)$ and $ g(x+y)=g(x)+g(y)$ $ \forall x\in (1,+\infty)$
It's then easy, for $ b>1$ and $ p\geq q\geq 1\in\mathbb{N}$ to write :
$ g(q\frac{pb}{q})=qg(\frac{pb}{q})$ since $ \frac{pb}{q}\in(1,+\infty)$ and $ g(pb)=pg(b)$ since $ b\in(1,+\infty)$
And so $ g(\frac{p}{q}b)=\frac{p}{q}g(b)$ $ \forall b>1\in\mathbb{R}$ and $ \forall p\geq q\geq 1\in\mathbb{N}$
And so, since $ g$ is continuous, $ g(x)=cx$ $ \forall x>1$ 
So $ f(x)=\alpha x\ln(x)$ $ \forall x\in I$ and it is easy to check that this necessary condition is sufficient.

b) : $ f(xy)=xf(x)+yf(y)$ $ \implies$ $ f(x^{2})=2xf(x)$ and then also $ 2f(xy)=f(x^{2})+f(y^{2})$. So we can write $ f(x^{2}y^{2})$ in two ways :
$ f(x^{2}y^{2})=f(XY)=Xf(X)+Yf(Y)=x^{2}f(x^{2})+y^{2}f(y^{2})$
$ f(x^{2}y^{2})=f(Z^{2})=2Zf(Z)=2xyf(xy)=xy(f(x^{2})+f(y^{2}))$
Comparing these two expressions,we get, for $ x\neq y$, $ xf(x^{2})=yf(y^{2})$ and so $ f(x)=\frac{c}{\sqrt{x}}$
Putting back this expression in $ f(x^{2})=2xf(x)$, we get $ \frac{c}{x}=2c\sqrt{x}$ and so $ c=0$
And so $ f(x)=0$

So the requested functions are $ f(x)=\alpha x\ln(x)$ for any $ \alpha\in\mathbb{R}$
\end{mysolution}
*******************************************************************************
-------------------------------------------------------------------------------

\begin{problem}[Posted by \href{https://artofproblemsolving.com/community/user/30146}{Nbach}]
	Find all continuous functions $f: \mathbb R^{+} \to \mathbb R^{+}$ such that :
\[f\left(\frac{1}{f(xy)}\right)=f(x)f(y)\] 
for all $ x,y\in \mathbb R^+$.
	\flushright \href{https://artofproblemsolving.com/community/c6h164574}{(Link to AoPS)}
\end{problem}



\begin{mysolution}[by \href{https://artofproblemsolving.com/community/user/29428}{pco}]
	\begin{tcolorbox}Find all continous function  $ f: R_{+}\mapsto R_{+}$ such that :
                             $ f(\frac{1}{f(xy)}) = f(x)f(y)$ for all $ x,y\in R_{+}$\end{tcolorbox}

Let $ g(x)=f(\frac{1}{x})$
We have $ g(g(xy))=g(x)g(y)$ and so $ g(g(x))=g(1)g(x)$ and so $ g(x)=g(1)x$ $ \forall x\in g(\mathbb{R}^{+})$
Then, since $ g(xy)\in g(\mathbb{R}^{+})$, $ g(g(xy))=g(1)g(xy)$ and so $ g(1)g(xy)=g(x)g(y)$ $ \forall x\in\mathbb{R}^{+}$

Let then $ g(x)=g(1)h(x)$ : we have $ h(xy)=h(x)h(y)$ and $ h(x)$ continuous. So $ h(x)=x^{c}$ for some $ c\in\mathbb{R}$

Putting back this expression in the original equation, we find $ f(x)=1$ or $ f(x)=\frac{a}{x}$, with $ a>0$
\end{mysolution}



\begin{mysolution}[by \href{https://artofproblemsolving.com/community/user/30326}{quangpbc}]
	Let $ f(1) = a$.

Let $ y = 1$ , $ f(\frac{1}{x}) = af(x)$.( 1 )
On ( 1 ) move $ x$ by $ xy$ , we get :

$ f(\frac{1}{xy}) = af(xy) = f(x).f(y)$.( 2 )

Let $ g(x) =\frac{f(x)}{a}$ so $ g(x)$ is a continous function .
From ( 2 ), we have $ g(xy) = g(x)g(y)$.
Then $ g(x) = x^{t}$ or $ f(x) = ax^{t}$
And easily to show that $ t = 0$ or $ t =-1$.

-$ t = 0$, we have $ a = a^{2}$, then $ a = 1\rightarrow f(x) = 1$
-$ t =-1$. we have $ f(x) =\frac{a}{x}$ with $ a\in\mathbb{R^{+}}$
\end{mysolution}
*******************************************************************************
-------------------------------------------------------------------------------

\begin{problem}[Posted by \href{https://artofproblemsolving.com/community/user/4527}{Amir.S}]
	Find all $ f:\mathbb{N}\rightarrow\mathbb{R}$ that for a given $ n\in\mathbb{N}$,
\[f(m+k)=f(mk-n)\]
 holds for all positive integers $k$ and $m$ with $mk>n$.
	\flushright \href{https://artofproblemsolving.com/community/c6h166166}{(Link to AoPS)}
\end{problem}



\begin{mysolution}[by \href{https://artofproblemsolving.com/community/user/29428}{pco}]
	\begin{tcolorbox}find all $ f:\mathbb{N}\rightarrow\mathbb{R}$ that for a given $ n\in\mathbb{N}$ we have:
$ f(m+k) = f(mk-n)\ ,\ m,k\in\mathbb{N}\ ,\ mk > n$\end{tcolorbox}

$ m=x-1$ and $ n=1$ $ \implies$ $ f(x)=f(x-(n+1))$ $ \forall x>n+1$ and so $ f(x)=f(x+n+1)$ $ \forall x>0$
$ k=n+1$ implies then $ f(m+n+1)=f((m-1)(n+1)+1)$ $ \forall m>0,n>0$
But, since $ f(x)=f(x+n+1)$ $ \forall x>0$, this last equation may be written $ f(m)=f(m+n+1)=f((m-1)(n+1)+1)=f(1)$

And so the unique solution $ f(m)=c$ $ c\in\mathbb{R}$
\end{mysolution}
*******************************************************************************
-------------------------------------------------------------------------------

\begin{problem}[Posted by \href{https://artofproblemsolving.com/community/user/30326}{quangpbc}]
	Find all increasing functions $ f:\mathbb{R}\to\mathbb{R}$ which satisfy
\[f(xf(y))=yf(2x), \quad \forall x,y \in \mathbb R.\]
	\flushright \href{https://artofproblemsolving.com/community/c6h166410}{(Link to AoPS)}
\end{problem}



\begin{mysolution}[by \href{https://artofproblemsolving.com/community/user/29428}{pco}]
	\begin{tcolorbox}Find all $ f:\mathbb{R}\to\mathbb{R}$ so that :

$ 1,f$ is an increasing function.

$ 2,f(xf(y)) = yf(2x)$

 :)\end{tcolorbox}

Let $ P(x,y)$ be the property $ f(xf(y)) = yf(2x)$

$ P(1,x)$  $ \implies$ $ f(f(x))=f(2)x$. $ f(2)\neq 0$ since $ f(x)$ is strictly increasing, so $ f(x)$ is a bijective function.
$ P(x,0)$ $ \implies$ $ f(xf(0))=0$ and so $ f(0)=0$ since $ f(x)$ is an injective function.
$ P(f(x),y)$ $ \implies$ $ f(f(x)f(y))=yf(2f(x))$
$ P(f(y),x)$ $ \implies$ $ f(f(x)f(y))=xf(2f(y))$
So $ yf(2f(x))=xf(2f(y))$ and so $ \frac{f(2f(x))}{x}=\frac{f(2f(y))}{y}=a$ $ \forall x,y\neq 0$
So $ f(2f(x))=ax$ $ \forall x$ with $ a\neq 0$
So we have a new property $ Q(x,y)$ : $ f(f(x)f(y))=axy$
Let then $ b\neq 0$ such that $ f(b)=1$ ($ b$ exists since $ f(x)$ is a surjective function)
$ Q(x,b)$ $ \implies$ $ f(f(x))=abx$ and so $ f(abx)=abf(x)$
$ Q(f(\frac{x}{ab}),f(\frac{y}{ab}))$ $ \implies$ $ f(xy)=\frac{1}{ab^{2}}f(x)f(y)$

And this is a well known Cauchy equation which gives $ f(x)=ux$ for some $ u$ and $ \forall x\in\mathbb{Q}$ and then $ f(x)=ux$ $ \forall x$ since $ f(x)$ is an increasing function.

Putting back this solution in the original equation, we find $ u=2$

And the only solution is $ f(x)=2x$
\end{mysolution}
*******************************************************************************
-------------------------------------------------------------------------------

\begin{problem}[Posted by \href{https://artofproblemsolving.com/community/user/30374}{hoangclub}]
	Determine all function $f: \mathbb N \to \mathbb N$ such that
\[f(m+n)f(m-n) = f(m^{2}),\]
for all $m,n \in  \mathbb N$.
	\flushright \href{https://artofproblemsolving.com/community/c6h166607}{(Link to AoPS)}
\end{problem}



\begin{mysolution}[by \href{https://artofproblemsolving.com/community/user/29428}{pco}]
	\begin{tcolorbox}Denote $ Z^{+}$ is set of all positive integer numbers. 
   determine all function $ f$:$ Z^{+}$ ->$ Z^{+}$ such that
                            $ f(m+n)f(m-n)$=$ f(m^{2})$ for all m,n $ \in$ $ Z^{+}$.\end{tcolorbox}

${ f(n)f(n+4)=f((n+2)^{2}})=f(n+1)f(n+3)$ and so $ \frac{f(n+4)}{f(n+3)}=\frac{f(n+1)}{f(n)}$ and so It exists $ u,a,b,c$ such that :
$ f(3p+1)=ua^{p}b^{p}c^{p}$
$ f(3p+2)=ua^{p+1}b^{p}c^{p}$
$ f(3p+3)=ua^{p+1}b^{p+1}c^{p}$

Then :
$ f(1)f(3)=f(4)$ $ \implies$ $ u uab=uabc$ and $ u=c$
$ f(1)f(5)=f(9)$ $ \implies$ $ u ua^{2}bc = ua^{3}b^{3}c^{2}$ and $ u=ab^{2}c$
$ f(1)f(7)=f(16)$ $ \implies$ $ u ua^{2}b^{2}c^{2}= ua^{5}b^{5}c^{5}$ and $ u=a^{3}b^{3}c^{3}$
$ f(1)f(9)=f(25)$ $ \implies$ $ u ua^{3}b^{3}c^{2}= ua^{8}b^{8}c^{8}$ and $ u=a^{5}b^{5}c^{6}$

And so $ a=b=c=u=1$

And $ f(n)=1$ $ \forall n$
\end{mysolution}
*******************************************************************************
-------------------------------------------------------------------------------

\begin{problem}[Posted by \href{https://artofproblemsolving.com/community/user/32804}{greatestmaths}]
	Find all function $f: \mathbb R \to \mathbb R$ such that for all real $x$, we have
\[f(f(f(x)))-3f(f(x))+6f(x)= 4x+3.\]
	\flushright \href{https://artofproblemsolving.com/community/c6h166839}{(Link to AoPS)}
\end{problem}



\begin{mysolution}[by \href{https://artofproblemsolving.com/community/user/18275}{fermat3}]
	no. i thing it's easy your problem.its classical:
f is increasing.and use the classical method of iteration .and you will find solution:$ f(x) = x+1$  :P
\end{mysolution}



\begin{mysolution}[by \href{https://artofproblemsolving.com/community/user/29428}{pco}]
	\begin{tcolorbox}find all function f: R----->R such that for all x in R. 

f(f(f(x)))-3f(f(x))+6f(x)= 4x+3\end{tcolorbox}

The difficulty is to study the non-trivial solutions (the trivial one is $ f(x)=x+1$).

I think I have shown there is no other bijective solution with an Abel form $ f(x)=g^{-1}(1+g(x))$
But it remains job to do.

I'm glad you posted this meassage in this forum, greatestmaths, since it means you have the solution.

Thanks to post it
or just a hint.
\end{mysolution}



\begin{mysolution}[by \href{https://artofproblemsolving.com/community/user/18275}{fermat3}]
	here you are a hint:let be $ x_{n}=f^{n}-n$ then its easy to see that $ x_{n+3}-3x_{n+2}+6x_{n+1}-4x_{n}=0$ then study this sequence via the fact that $ f$ is increasing
 
\end{mysolution}



\begin{mysolution}[by \href{https://artofproblemsolving.com/community/user/29428}{pco}]
	\begin{tcolorbox}here you are a hint:let be $ x_{n}= f^{n}-n$ then its easy to see that $ x_{n+3}-3x_{n+2}+6x_{n+1}-4x_{n}= 0$ then study this sequence via the fact that $ f$ is increasing
 \end{tcolorbox}

Why is $ f(x)$ increasing ?
\end{mysolution}



\begin{mysolution}[by \href{https://artofproblemsolving.com/community/user/18275}{fermat3}]
	supose that $ f is decreasing$ then $ f0f0f-3f0f+6f$ is deacreasing wich absurd because $ f^{3}-3f^{2}+6f=4x+3$ 
\end{mysolution}



\begin{mysolution}[by \href{https://artofproblemsolving.com/community/user/29428}{pco}]
	\begin{tcolorbox}supose that $ f is decreasing$ then $ f0f0f-3f0f+6f$ is deacreasing wich absurd because $ f^{3}-3f^{2}+6f = 4x+3$ \end{tcolorbox}

But f can be neither decreasing, neither increasing ! (increasing on some parts, decreasing on others), so your demo seems wrong.
\end{mysolution}



\begin{mysolution}[by \href{https://artofproblemsolving.com/community/user/31917}{daniel73}]

Yep, I agree with pco; we know that $ f$ is injective (it is enough to assume that $ x,y$ exist such that $ f(x) = f(y)$ and see that this leads to $ 4x+3 = 4y+3$), but I do not know how you can go from there to saying that $ f$ needs to be monotonous.  For all that we know (unless proved otherwise), it could even be not continuous at any $ x$!

My own efforts go more in the direction of defining $ g(x) = f(f(x))-2f(x)+4x$ so that obviously $ g(f(x))-g(x) = 3$ for all $ x$, and $ h(x) = f(x)-x$ so that $ h(f(f(x)))-2h(f(x))+4h(x) = 3$, but I have not been able to crack it yet (the idea is showing that $ h(x)$ is constant and/or $ g(x) = 3x$).
\end{mysolution}



\begin{mysolution}[by \href{https://artofproblemsolving.com/community/user/29428}{pco}]
	\begin{tcolorbox}find all function f: R----->R such that for all x in R. 

f(f(f(x)))-3f(f(x))+6f(x)= 4x+3\end{tcolorbox}

Here is a partial result : $ f(x) = x+1$ is the unique continuous solution.

Demo :
Let $ x_{0}\in\mathbb{R}$ and the sequence $ a_{1}= x_{0}$, $ a_{n+1}= f(a_{n})$.
We have $ a_{n+3}= 3a_{n+2}-6a_{n+1}+4a_{n}+3$
Let then $ b_{n}= a_{n+1}-a_{n}$. We have $ b_{n+2}= 2b_{n+1}-4b_{n}+3$ and so $ b_{n+3}-1 =-8(b_{n}-1)$

So, if $ b_{n}\neq 1$ for some $ n$, we are sure that the sequence $ b_{n}$ will have some $ b_{i}> 0$ and some $ b_{j}< 0$. Or $ a_{i+1}> a_{i}$ and $ a_{j+1}< a_{j}$ or again $ f(a_{i}) < a_{i}$ and $ f(a_{j}) > a_{j}$. 
Alternatively, $ b_{1}= 1$ implies $ a_{2}= a_{1}+1$ and so $ f(x_{0}) = x_{0}+1$

So : either $ f(x_{0}) = x_{0}+1$, either it exists reals $ u$ and $ v$ such that $ f(u) < u$ and $ f(v) > v$. Then, if $ f(x)$ is continous, it exists $ w$ such that $ f(w) = w$. But then $ f(f(f(w))) = w$ and $ 3f(f(w))-6f(w)+4w+3 = 3w-6w+4w+3 = w+3$ and so $ f(f(f(w)))\neq 3f(f(w))-6f(w)+4w+3$

So, either $ f(x) = x+1$ $ \forall x$, either $ f(x)$ is not continuous.

And, since $ x+1$ is a solution, the only continuous soluition is $ f(x) = x+1$
Q.E.D.

It remains to study non continuous solutions.

And, once again, greatestmaths, I would be happy to have some hint (or the complete solution).
\end{mysolution}


*******************************************************************************
-------------------------------------------------------------------------------

\begin{problem}[Posted by \href{https://artofproblemsolving.com/community/user/32407}{chessfreak}]
	Do there exist four real polynomials such that the sum of any three of them has a real zero while the sum of no two polynomials has a real zero?
	\flushright \href{https://artofproblemsolving.com/community/c6h166926}{(Link to AoPS)}
\end{problem}



\begin{mysolution}[by \href{https://artofproblemsolving.com/community/user/29428}{pco}]
	\begin{tcolorbox}Do there exist four real polynomials such that the sum of any three of them has a real zero while the sum of no two polynomials has a real zero?\end{tcolorbox}

No, it does not :

$ \forall i,j\in\{1,2,3,4\}$, we have either $ P_{i}+P_{j}> 0$ $ \forall x$, either $ P_{i}+P_{j}< 0$ $ \forall x$

But $ P_{1}+P_{2}$, $ P_{1}+P_{3}$ and $ P_{2}+P_{3}$ can't have same sign, else their sum would have constant sign and $ P_{1}+P_{2}+P_{3}$ would never be 0.

So, WLOG (choice or order + replacement $ P_{i}$ by $ -P_{i}$) say :
$ P_{1}+P_{2}> 0$
$ P_{2}+P_{3}> 0$
$ P_{1}+P_{3}< 0$

Since $ P_{1}+P_{3}< 0$, and since $ P_{1}+P_{3}$, $ P_{1}+P_{4}$ and $ P_{3}+P_{4}$ can't have same sign, $ P_{1}+P_{4}$ or $ P_{3}+P_{4}$ is $ > 0$. WLOG say :
$ P_{1}+P_{4}> 0$

Since $ P_{1}+P_{2}> 0$ and $ P_{1}+P_{4}> 0$, and since $ P_{1}+P_{2}$, $ P_{1}+P_{4}$ and $ P_{2}+P_{4}$ can't have same sign :
$ P_{2}+P_{4}< 0$

Then since $ P_{2}+P_{3}> 0$ and $ P_{1}+P_{4}> 0$, then $ P_{1}+P_{2}+P_{3}+P_{4}> 0$
But since $ P_{2}+P_{4}< 0$ and $ P_{1}+P_{3}< 0$, then $ P_{1}+P_{2}+P_{3}+P_{4}< 0$

And so contradiction.
(and this property is available for any set of continuous functions)
\end{mysolution}



\begin{mysolution}[by \href{https://artofproblemsolving.com/community/user/12952}{venkata}]
	i dunno how many times u need to be told that IMOTC postals are not to be posted at Mathlinks.
trust me, ppl who have easy access to the net have a gr8 advantage over those who dont.
in due recognition of the fact that u r not from IMOTC, i request u to kindly not post ANY OF THE IMOTC postal probs.
\end{mysolution}



\begin{mysolution}[by \href{https://artofproblemsolving.com/community/user/32245}{borislav\_mirchev}]
	Internet is great place for people to share their knowledge. You should post all problems that you like! I dislike monopolism in math!!!
\end{mysolution}



\begin{mysolution}[by \href{https://artofproblemsolving.com/community/user/13985}{BaBaK Ghalebi}]
	\begin{tcolorbox}Internet is great place for people to share their knowledge. You should post all problems that you like! I dislike monopolism in math!!!\end{tcolorbox}
sure but I think that venkata meant that these problems are in a contest which is still open,so the students are supposed to solve the questions by them selves untill the given time,so it is not a good idea to post it here,its kind of cheating...
\end{mysolution}



\begin{mysolution}[by \href{https://artofproblemsolving.com/community/user/32245}{borislav\_mirchev}]
	Yes, you are maybe right. But I think the problems are took from somewhere maybe some other competition or some textbook. What if the student use same textbook? I think, at this time when the world is a "globbal village" some competition without presence of the students if it is not time limited or aren't there some other limitations may be a warranty only that students have the desire and they want to solve proposed problems. They also may ask for help their friends parents, or other web sites not so good but, different from mathlinks... I think after I-st round there should be some communication with the students...maybe interview or round where they solve problems. Other idea I have is electronic competition with fixed time limit, but you again cannot be sure who is in front of the PC...
\end{mysolution}



\begin{mysolution}[by \href{https://artofproblemsolving.com/community/user/20099}{pardesi}]
	\begin{bolded}chessfreak\end{bolded} i suppose u r relly intesrested in seeing the Postal problems solved  :P 
How is it that u r not a IMOTC member(as u had claimed   ) and yet u r the first each time to post the postal problems.
well we have given u enough warnings i think it's time u better stp posting them otherwise we have to egt each of ur posts scrutinized and deleted if they are have postal problems .and if u still continue this u know what lies next.
now no more requests just
STOP POSTING THEM BECAUSE OF U EVERYONE ELSE IS DOING SO 
\end{mysolution}



\begin{mysolution}[by \href{https://artofproblemsolving.com/community/user/12952}{venkata}]
	Dear borislav\_mirchev
As Bhabak Ghalebi mentions, this is an ongoing competition , and the rules clrly state: NO EXTERNAL HELP TO BE TAKEN.
as far as "global villaging" goes, no 1 minds the posting of problems ONCE THE DEADLINE IS OVER.
also, in all these competitions, a lot hinges on TRUST and ETHICS. so things like asking frnds, parents etc usually dont happen at this level(so i bliv) :maybe:  
btw, ill surely suggest ur examination methods to our profs 
\end{mysolution}
*******************************************************************************
-------------------------------------------------------------------------------

\begin{problem}[Posted by \href{https://artofproblemsolving.com/community/user/29428}{pco}]
	For those who think that $ f(f(x))$ is always an increasing function :

Find all functions $ \mathbb{R}\rightarrow\mathbb{R}$ such that $ f(f(x)) =-x$ $ \forall x\in\mathbb{R}$
	\flushright \href{https://artofproblemsolving.com/community/c6h167038}{(Link to AoPS)}
\end{problem}



\begin{mysolution}[by \href{https://artofproblemsolving.com/community/user/30264}{lasha}]
	Suppose there exists such interval $ P$, that for any reals $ (x,y)$ from $ P$, where $ x>y$, $ f(x)>f(y)$. Than, $ f(f(x))>f(f(y))$, but as $ f(f(x))=-x$,$ f(f(y))=-y$, we have $ -x>-y$, but $ x>y$-contradiction. So, $ f$ is increasing function. Again take reals $ x$ and $ y$ with $ x>y$. $ f(x)<f(y)$. So, $ f(f(x))>f(f(y))$, or $ -x>-y$-again contradiction because $ x>y$.
So, there is no solution.
\end{mysolution}



\begin{mysolution}[by \href{https://artofproblemsolving.com/community/user/29428}{pco}]
	\begin{tcolorbox}Suppose there exists such interval $ P$, that for any reals $ (x,y)$ from $ P$, where $ x > y$, $ f(x) > f(y)$. Than, $ f(f(x)) > f(f(y))$, but as $ f(f(x)) =-x$,$ f(f(y)) =-y$, we have $ -x >-y$, but $ x > y$-contradiction. So, $ f$ is increasing function. Again take reals $ x$ and $ y$ with $ x > y$. $ f(x) < f(y)$. So, $ f(f(x)) > f(f(y))$, or $ -x >-y$-again contradiction because $ x > y$.
So, there is no solution.\end{tcolorbox}

You're wrong.
Solutions exist

When you write "Suppose there exists such interval $ P$, that for any reals $ (x,y)$ from $ P$, where $ x > y$, $ f(x) > f(y)$. Then, $ f(f(x)) > f(f(y))$", the conclusion "Then, $ f(f(x)) > f(f(y))$" is wrong because maybe $ f(x)\notin P$ when $ x\in P$
\end{mysolution}



\begin{mysolution}[by \href{https://artofproblemsolving.com/community/user/16261}{Rust}]
	$ f(0) = 0$. For all $ x\not = 0$ we can consider pairs $ (x,y), x\not = 0\not = y,\ |x|\not = |y|$ and define
$ f(x) = y,f(y) =-x,f(-x) =-y, f(-y) = x$.
\end{mysolution}



\begin{mysolution}[by \href{https://artofproblemsolving.com/community/user/29428}{pco}]
	\begin{tcolorbox}$ f(0) = 0$. For all $ x\not = 0$ we can consider pairs $ (x,y), x\not = 0\not = y,\ |x|\not = |y|$ and define
$ f(x) = y,f(y) =-x,f(-x) =-y, f(-y) = x$.\end{tcolorbox}

Perfecty right, Rust.

In other words :
Let $ \mathbb{A}$ and $ \mathbb{B}$ to subsets of $ \mathbb{R}^{+}$ such that :
$ \mathbb{A}\cup\mathbb{B}=\mathbb{R}^{+}$
$ \mathbb{A}\cap\mathbb{B}=\emptyset$
$ \exists$ $ h(x)$ bijective function $ \mathbb{A}\rightarrow\mathbb{B}$

$ f(x)$ may be defined as :
$ f(0)=0$
$ f(x)=h(x)$ $ \forall x\in\mathbb{A}$
$ f(x)=-h^{-1}(x)$ $ \forall x\in\mathbb{B}$
$ f(x)=-f(-x)$ $ \forall x<0$

Example :
$ f(0)=0$
$ f(x)=\frac{|x|}{x}-(-1)^{[-|x|]}x$ $ \forall x\neq 0$
\end{mysolution}



\begin{mysolution}[by \href{https://artofproblemsolving.com/community/user/16261}{Rust}]
	But there are not continiosly function $ f(x): R^{n}\to R^{n}$ suth that $ f(f(x))=-x$ if n is odd.
\end{mysolution}



\begin{mysolution}[by \href{https://artofproblemsolving.com/community/user/30264}{lasha}]
	You are right pco! Sorry, I made mistake. :(
\end{mysolution}



\begin{mysolution}[by \href{https://artofproblemsolving.com/community/user/6551}{perfect\_radio}]
	For pco: \url{http://www.mathlinks.ro/viewtopic.php?search_id=1499275189&t=113408}.

[In the link it is proved that every such function has infinitely many discontinuities.]
\end{mysolution}



\begin{mysolution}[by \href{https://artofproblemsolving.com/community/user/10088}{silouan}]
	PLease see this and the link of enescu \url{http://www.mathlinks.ro/viewtopic.php?search_id=831547553&t=56637}
\end{mysolution}
*******************************************************************************
-------------------------------------------------------------------------------

\begin{problem}[Posted by \href{https://artofproblemsolving.com/community/user/26965}{Shishkin}]
	Find all functions $f: \mathbb Q \to \mathbb Q$  such that $ f(x^{2}+y+f(xy)) = 3+(x+f(y)-2)f(x)$ for all $x,y \in \mathbb Q$.
	\flushright \href{https://artofproblemsolving.com/community/c6h167309}{(Link to AoPS)}
\end{problem}



\begin{mysolution}[by \href{https://artofproblemsolving.com/community/user/29428}{pco}]
	\begin{tcolorbox}Find all functions $ f$ $ Q\to Q$  such that $ f(x^{2} + y + f(xy)) = 3 + (x + f(y) - 2)f(x)$.\end{tcolorbox}

I have a rather complex solution to prove $ f(x)=x+1$ :

Let $ P(x,y)$ be the property : $ f(x^{2} + y + f(xy)) = 3 + (x + f(y) - 2)f(x)$
Let $ f(0)=a$

1) $ P(0,x)$ implies $ f(x+a)=af(x)+3-2a$ and so :
$ \boxed{f(x+na)=a^nf(x)+(3-2a)(1+a+a^2+\ldots+a^{n-1})}$

2) $ P(x,0)$ and $ P(-x,0)$ imply $ f(x^2+a)=(x+a-2)f(x)+3$ and $ f(x^2+a)=(-x+a-2)f(-x)+3$ and so :
$ \boxed{(x+a-2)f(x)=(-x+a-2)f(-x)}$

3) $ P(1,0)$ implies $ f(a+1)=(a-1)f(1)+3$ but $ 1)$ implies $ f(a+1)=af(1)+3-2a$ and so $ af(1)+3-2a=(a-1)f(1)+3$ and :
$ \boxed{f(1)=2a}$

4) If $ a=2$, then $ 1)$ gives $ f(2n)=2^n+1$ and $ 1)+3)$ gives $ f(2n+1)=3\times 2^n+1$
But then $ P(2,2)$ gives $ f(6+f(4))=12$, so $ f(11)=12$ but $ f(11)=f(2\times 5+1)=3\times 2^5+1=97$ and so $ a\neq 2$
If $ a=0$, then $ P(0,0)$ gives $ f(0)=a=3$ and so $ a\neq 0$
If $ a=3$, then $ 2)$ above with $ x=1$ gives $ 2f(1)=0$ but $ 3)$ said $ f(1)=2a=6$ and so :
$ \boxed{a\notin \{0,2,3\}}$

5) Using $ x=a-2$ in $ 2)$ above, we get $ 2(a-2)f(a-2)=0$ and so, since, from $ 4)$, $ a\neq 2$ :
$ \boxed{f(a-2)=0}$

6) Using $ x=1$ in $ 2)$ above gives $ f(-1)=2a\frac{a-1}{a-3}$ (remember with $ 4)$ above that $ a\neq 3$).
Then $ P(-1,-1)$ gives $ f(2a)=3 + (2a\frac{a-1}{a-3} - 3)2a\frac{a-1}{a-3}$
But $ 1)$ gives $ f(2a)=a^3-2a^2+a+3$ and so :
$ a^3-2a^2+a+3=3 + (2a\frac{a-1}{a-3} - 3)2a\frac{a-1}{a-3}$ which gives :
$ a(a-1)(a^3-11a^2+25a-27)= 0$

It's rather easy to see that $ a^3-11a^2+25a-27$ has no rational root and so :
$ \boxed{f(0)=a=1}$

7) Then $ 1)$ gives $ f(x+n)=f(x)+n$ and $ f(n)=n+1$
Then, Let $ p$ and $ q$ coprimes positive integers.
$ P(q,\frac{p}{q})$ gives $ f(q^2+\frac{p}{q}+p+1)=3 + (q + f(\frac{p}{q}) - 2)(q+1)$
But $ f(q^2+\frac{p}{q}+p+1)=q^2+p+1+f(\frac{p}{q})$ (since $ f(x+n)=f(x)+n$) and so :

$ q^2+p+1+f(\frac{p}{q})=3 + (q + f(\frac{p}{q}) - 2)(q+1)$

$ p+q=qf(\frac{p}{q}$ and so $ f(\frac{p}{q})=\frac{p}{q}+1$

So $ f(x)=x+1$ $ \forall x\in\mathbb{Q}^{+}$
Using $ 2)$, we have then $ f(-x)=-x+1$ $ \forall x\in\mathbb{Q}^{+}$

So $ f(x)=x+1$ $ \forall x\in\mathbb{Q}$

And putting back this expression in original equation, we find that this necessary condition is sufficient.

And the only solution is $ \boxed{f(x)=x+1}$
\end{mysolution}



\begin{mysolution}[by \href{https://artofproblemsolving.com/community/user/43536}{nguyenvuthanhha}]
	\begin{italicized}A very good solution , Pco . You really have talent in Algebra .  \end{italicized}
\end{mysolution}



\begin{mysolution}[by \href{https://artofproblemsolving.com/community/user/52090}{Dumel}]
	\begin{tcolorbox}
It's rather easy to see that $ a^3 - 11a^2 + 25a - 27$ has no rational root and so :
$ \boxed{f(0) = a = 1}$\end{tcolorbox}you're wrong
\end{mysolution}



\begin{mysolution}[by \href{https://artofproblemsolving.com/community/user/29428}{pco}]

Please, could you kindly show me where I am wrong ?

Is there a rational root to $ a^3 - 11a^2 + 25a - 27$ ? (please give us)

Or is my conclusion $ a=1$ wrong ?, and why ?
\end{mysolution}



\begin{mysolution}[by \href{https://artofproblemsolving.com/community/user/52090}{Dumel}]
	let $ g(x) = x^3 - 11x^2 + 25x - 27$
$ - 27 = g(0) < 0$
and
$ g(11) > 0$
$ g$ is a continuous funcion so there exist $ x_0$ in $ (0,11)$ such that $ g(x_0) = 0$

edit: oh oh I forgot that  $ f$ is $ Q\to Q$  and all roots of g are surd so your solution is (almost) correct :-)
\end{mysolution}



\begin{mysolution}[by \href{https://artofproblemsolving.com/community/user/29428}{pco}]
	\begin{tcolorbox}let $ g(x) = x^3 - 11x^2 + 25x - 27$
$ - 27 = g(0) < 0$
and
$ g(11) > 0$
$ g$ is a continuous funcion so there exist $ x_0$ in $ (0,11)$ such that $ g(x_0) = 0$

edit: oh oh I forgot that  $ f$ is $ Q\to Q$  and all roots of g are surd so your solution is (almost) correct :-)\end{tcolorbox}

Yes, I said that $ x^3 - 11x^2 + 25x - 27$ had no rational root.

And, why the "almost" word in your post ?
What else is not correct in my solution ?
\end{mysolution}



\begin{mysolution}[by \href{https://artofproblemsolving.com/community/user/52090}{Dumel}]
	"almost" was just due to the word "rational"  
\end{mysolution}



\begin{mysolution}[by \href{https://artofproblemsolving.com/community/user/177508}{mathuz}]
	:roll: 
Oh...ho, I  found very nice solution of the problem!
Let  $g(x)=f(x)-1$  and  $ g:Q\rightarrow Q $.  Then from original equation  we have \[ g(x^2+y+g(xy)+1)=2+(x+g(y)-1)(g(x)+1) .\]
(1) $g(0)=0; $  really it's true!
(2)  from (1),  $ g(1)=1 $  and 
\[ g(x+1)=g(x)+1 \] and \[ g(x^2)=x+g(x)(x-1) (*)\]
for any ratsional $x$.
Hence, $ g(n)=n $ any integer $n$.    Let $x=\frac{m}{n} $ some integers $m$ and $n$,  then  at $P(x,n), \rightarrow g(x)=x $ for any ratsional numbers $x$.
Therefore, $f(x)=x+1$, any ratsional $x$.
\end{mysolution}



\begin{mysolution}[by \href{https://artofproblemsolving.com/community/user/177508}{mathuz}]
	Sorry, I have latex mistake. Original version:  
 :roll: 
Oh...ho, I  found very nice solution of the problem!
Let  $g(x)=f(x)-1$  and  $ g:Q\rightarrow Q $.  Then from original equation  we have \[ g(x^2+y+g(xy)+1)=2+(x+g(y)-1)(g(x)+1) .\]
(1) $g(0)=0; $  really it's true!
(2)  from (1),  $ g(1)=1 $  and 
\[ g(x+1)=g(x)+1 \] and \[ g(x^2)=x+g(x)(x-1) (*)\]
for any ratsional $x$.
Hence, $ g(n)=n $ any integer $n$.    Let $x=\frac{m}{n} $ some integers $m$ and $n$,  then from $(*)$, at $P(x,n)$, $\rightarrow $ $g(x)=x $ for any ratsional numbers $x$.
Therefore, $f(x)=x+1$, any ratsional $x$.
\end{mysolution}



\begin{mysolution}[by \href{https://artofproblemsolving.com/community/user/288210}{tenplusten}]
	\begin{tcolorbox}Sorry, I have latex mistake. Original version:  
 :roll: 
Oh...ho, I  found very nice solution of the problem!
Let  $g(x)=f(x)-1$  and  $ g:Q\rightarrow Q $.  Then from original equation  we have \[ g(x^2+y+g(xy)+1)=2+(x+g(y)-1)(g(x)+1) .\]
(1) $g(0)=0; $  really it's true!
(2)  from (1),  $ g(1)=1 $  and 
\[ g(x+1)=g(x)+1 \] and \[ g(x^2)=x+g(x)(x-1) (*)\]
for any ratsional $x$.
Hence, $ g(n)=n $ any integer $n$.    Let $x=\frac{m}{n} $ some integers $m$ and $n$,  then from $(*)$, at $P(x,n)$, $\rightarrow $ $g(x)=x $ for any ratsional numbers $x$.
Therefore, $f(x)=x+1$, any ratsional $x$.\end{tcolorbox}

I would like to see your proof for both claims???? 
\end{mysolution}
*******************************************************************************
-------------------------------------------------------------------------------

\begin{problem}[Posted by \href{https://artofproblemsolving.com/community/user/19941}{delegat}]
	Find all functions $ \mathbb{R}\to\mathbb{R}$ that satisfy
\[f(xf(y)+f(x))=2f(x)+xy,\]
for all reals $ x$  and $y$.
	\flushright \href{https://artofproblemsolving.com/community/c6h167659}{(Link to AoPS)}
\end{problem}



\begin{mysolution}[by \href{https://artofproblemsolving.com/community/user/29428}{pco}]
	\begin{tcolorbox}Find all functions $ \mathbb{R}\to\mathbb{R}$ that satisfy:

$ f(xf(y) + f(x)) = 2f(x) + xy$ for all reals $ x$  and $ y$.\end{tcolorbox}

Here is a rather long method to show that $ f(x)=x+1$. I think someone could could find a shorter one.

Let $ P1(x,y)$ be the property $ P1(x,y)$ : $ f(xf(y) + f(x)) = 2f(x) + xy$ 

1) $ f(x)$ is a bijective function :
$ f(y_1)=f(y_2)$ implies $ 2f(1)+y_1=f(f(y_1)+f(1))=f(f(y_2)+f(1))=2f(1)+y_2$ implies $ y_1=y_2$ and $ f(x)$ is injective.
$ P1(1,z-2f(1))$ : $ f(f(z-2f(1))+f(1))=z$ and $ f(x)$ is surjective.

2) $ f(-1)=0$, $ f(0)=1$, $ f(1)=2$ and $ f(2)=3$
Let $ f(0)=a$, $ f(b)=0$ and $ f(c)=1$
$ P1(b,c)$ gives $ f(b)=bc=0$ and so either $ b=0$, either $ c=0$
If $ b=0$, $ f(0)=0$ and then $ P1(f^{-1}(x),0)$ gives $ f(x)=2x$ and it is easy to see that this solution is wrong. So $ c=0$ and $ f(0)=1$ and $ a=1$.
Then $ P1(0,y)$ gives $ f(1)=2$
Then $ P1(b,b)$ gives $ 1=b^2$ and so $ b=-1$ since $ f(1)=2\neq 0$ and $ f(-1)=0$
Finally, $ P(1,-1)$ gives $ f(2)=3$

3) Four other interesting properties :
$ P1(x,-1)$ gives $ P2(x)$ : $ f(f(x))=2f(x)-x$
$ P1(x,0)$ gives $ P3(x)$ : $ f(x+f(x))=2f(x)$
Applying $ f(x)$ on both sides of P3, we have $ f(2f(x))=f(f(x+f(x)))$. Using then P2 on RHS : $ f(2f(x))=2f(x+f(x))-x-f(x)$. Using then P3 in RHS : $ f(f(x))=3f(x)-x=(2f(x)-x)+f(x)=f(f(x))+f(x)$. And so, since $ f(x)$ is surjective :
$ P4(x)$ : $ f(2x)=f(x)+x$
Applying then $ f(x)$ on both sides of $ P1(x,y)$ : $ f(2f(x)+xy)=f(f(xf(y)+f(x)))$. Using P2 on RHS : $ f(2f(x)+xy)=2f(xf(y)+f(x))-xf(y)-f(x)$. Using P1 on RHS : 
$ P5(x,y)$ : $ f(2f(x)+xy)=3f(x)+2xy-xf(y)$

As a synthesis now, we have :
$ P1(x,y)$ : $ f(xf(y) + f(x)) = 2f(x) + xy$ 
$ P2(x)$ : $ f(f(x))=2f(x)-x$
$ P3(x)$ : $ f(x+f(x))=2f(x)$
$ P4(x)$ : $ f(2x)=f(x)+x$
$ P5(x,y)$ : $ f(2f(x)+xy)=3f(x)+2xy-xf(y)$
$ f(-1)=0$
$ f(0)=1$
$ f(1)=2$
$ f(2)=3$

4) $ f(x)=x+1$
$ P5(2,x-3)$ gives $ f(2x)=4x-3-2f(x-3)$. Using then P4 on LHS : $ f(x)=3x-3-2f(x-3)$
$ P5(1,x-3)$ gives $ f(x+1)=2x-f(x-3)$
Eliminating $ f(x-3)$ in these two equations, we obtain $ f(x)=-x-3+2f(x+1)$
From $ f(x)=-x-3+2f(x+1)$, we have $ f(x+1)=-x-4+2f(x+2)$ and so $ f(x)=-3x-11+4f(x+2)$ and so $ f(2x)=-6x-11+4f(2x+2)$
Using  P4 on both sides, we have $ f(x)=-3x-7+4f(x+1)$
Eliminating $ f(x+1)$ between $ f(x)=-x-3+2f(x+1)$ and $ f(x)=-3x-7+4f(x+1)$, we have $ f(x)=x+1$

And we just have to check that this value fit in the original equation.
\end{mysolution}



\begin{mysolution}[by \href{https://artofproblemsolving.com/community/user/145847}{kucheto}]
	Let $ P(x,y) $ be assertion of $ f(xf(y)+f(x))=2f(x)+xy. $
$ P(1,y-2f(1))\Rightarrow f(f(y-2f(1))+f(1))=y\Rightarrow f(x) $ is surjective.
Let $ f(a)=0,\ f(b)=1;\ P(a,b)\Rightarrow ab=0. $
If $ a=0, $ then $ P(x,0)\Rightarrow f(f(x))=2f(x)\Rightarrow f(z)=2z $ which is not a solution.
Therefore, $ b=0\Rightarrow f(0)=1;\ P(0,0)\Rightarrow f(1)=2. $
$ P(a,a)\Rightarrow a^2=1\Rightarrow a=-1. $
$ f(c)=-1,\ P(c,-1)\Rightarrow 0=-2-c\Rightarrow c=-2. $
$ P(x,-2)\Rightarrow f(f(x)-x)=2(f(x)-x)\Rightarrow f(u)=2u $ for $ u\in f(x)-x. $
$ P(-1,y)\Rightarrow f(-f(y))=-y.\ (1) $
$ P(y,-1)\Rightarrow f(f(y))=2f(y)-y.\ (2) $
$ (2)-(1)\Rightarrow 2f(y)=f(f(y))-f(-f(y))\Rightarrow 2z=f(z)-f(-z). $
$ z=u\Rightarrow f(-u)=0\Rightarrow -u=a=-1\Rightarrow u=1\Rightarrow f(x)=x+1 $ which indeed is a solution.

Hence, the only solution to the equation is $ f(x)=x+1. $
\end{mysolution}
*******************************************************************************
-------------------------------------------------------------------------------

\begin{problem}[Posted by \href{https://artofproblemsolving.com/community/user/6698}{ekaragoz}]
	Prove that there are no continuous functions $f: \mathbb R \to \mathbb R$ satisfying the functional equation
\[f(1+f(x))=2-3x \quad \forall x \in \mathbb R.\]
	\flushright \href{https://artofproblemsolving.com/community/c6h167749}{(Link to AoPS)}
\end{problem}



\begin{mysolution}[by \href{https://artofproblemsolving.com/community/user/16261}{Rust}]
	There are no function (without continiosly).
Because 2-3x is bijective f(x) must be bijective (for any y exist $ x = (2-y)/3, z = 1+f(x)$ suth that $ f(z) = y$, and $ f(x) = f(y)\to f(1+f(x)) = 3-2x = f(1+f(y)) = 3-2y\to x = y$).
From $ f(3-3x) = f(1+2-3x) = f(1+f(1+f(x))) = 2-3(1+f(x)) =-3f(x)-1$ we get $ f(3/4) =-1/4$ .
\end{mysolution}



\begin{mysolution}[by \href{https://artofproblemsolving.com/community/user/29428}{pco}]
	\begin{tcolorbox}Prove that there are no continuous function satisfying the functional equation below. (f:R-R)

f(1+f(x))=2-3x\end{tcolorbox}

$ f(x)$ is clearly bijective, so, if $ f(x)$ is continuous, $ f(x)$ must be stricty monotonous (since bijective and continuous), so $ f(1+f(x))$ would be stricty increasing, which is wrong since $ f(1+f(x))=2-3x$, strictly decreasing.

So $ f(x)$ can't be continuous.
\end{mysolution}



\begin{mysolution}[by \href{https://artofproblemsolving.com/community/user/29126}{MellowMelon}]
	pco, can you elaborate more on how you can conclude that $ f(x)$ is strictly increasing? I got the rest of the solution.
\end{mysolution}



\begin{mysolution}[by \href{https://artofproblemsolving.com/community/user/9105}{Hamster1800}]
	Suppose that $ f(x)$ is strictly increasing, then if $ y>x$,

$ f(1+f(y)) = 2-3y < f(1+f(x))$

But since $ f$ is strictly increasing,

$ 1+f(y) > 1+f(x)$
$ f(1+f(y)) > f(1+f(x))$, but $ f(1+f(y))<f(1+f(x))$. Contradiction.

The other way follows similarly.
\end{mysolution}



\begin{mysolution}[by \href{https://artofproblemsolving.com/community/user/29428}{pco}]
	\begin{tcolorbox}pco, can you elaborate more on how you can conclude that $ f(x)$ is strictly increasing? I got the rest of the solution.\end{tcolorbox}

$ g(x)=1+f(x)$ is strictly monotonous. So $ g(g(x))$ is stricty increasing. So $ g(g(x))-1=f(1+f(x))$ is strictly increasing.
\end{mysolution}



\begin{mysolution}[by \href{https://artofproblemsolving.com/community/user/16261}{Rust}]
	I found mistake in my solution.
Let $ g(x)=\frac{1}{4}+f(\frac{3}{4}+x)$, then equation equavalent to $ g(g(x))=-3x$ and it had infinetely many (non continiosly) solutions.
\end{mysolution}



\begin{mysolution}[by \href{https://artofproblemsolving.com/community/user/6698}{ekaragoz}]
	Thank you all
\end{mysolution}
*******************************************************************************
-------------------------------------------------------------------------------

\begin{problem}[Posted by \href{https://artofproblemsolving.com/community/user/29455}{apollo}]
	Find all functions $ f:\mathbb{N}\rightarrow\mathbb{N}$ such that $f(m+n)f(m-n) =f(m^{2})$ holds for all integers $m$ and $n$ with $m>n>0$.
	\flushright \href{https://artofproblemsolving.com/community/c6h168059}{(Link to AoPS)}
\end{problem}



\begin{mysolution}[by \href{https://artofproblemsolving.com/community/user/16261}{Rust}]
	$ m=n$ give $ f(m^{2})=0$. If $ 0\not\in N$ conradition, else $ m=1, n=2,3,...$ give $ f(n)=9\ \forall n\in N$.
\end{mysolution}



\begin{mysolution}[by \href{https://artofproblemsolving.com/community/user/29428}{pco}]
	\begin{tcolorbox}Let ℕ  denote the set of positive integers.Find all functions $ f:\mathbb{N}\rightarrow\mathbb{N}$ such that $ f(m+n)f(m-n) = f(m^{2})$\end{tcolorbox}

I think that the condition $ f(m+n)f(m-n) = f(m^{2})$ is supposed to be true $ \forall m > n > 0$

Let then $ a > 2$ : $ f(2a-1)f(1) = f(a^{2}) = f(2a-2)f(2)$ and so $ E1$ : $ \frac{f(2a-1)}{f(2a-2)}=\frac{f(2)}{f(1)}= q$.
With $ a = 3$, we have $ \frac{f(5)}{f(4)}=\frac{f(2)}{f(1)}$.

Let then $ a > 3$ : $ f(2a)f(4) = f((a+2)^{2}) = f(2a-1)f(5)$ and so $ E2$ : $ \frac{f(2a)}{f(2a-1)}=\frac{f(5)}{f(4)}=\frac{f(2)}{f(1)}= q$.

And so $ \frac{f(x+1)}{f(x)}= q$ $ \forall x > 6$ and $ f(x) =\alpha q^{x}$ $ \forall x > 6$

Then, for $ m > n+6$, $ f(m+n)f(m-n) = f(m^{2})$ becomes $ \alpha^{2}q^{2m}=\alpha q^{m^{2}}$ and $ \alpha = q = 1$ and $ f(x) = 1$ $ \forall x > 6$

Let then $ a > 0$ and $ b > 3$ : $ f(a+2b)f(a) = f((a+b)^{2})$. We have $ a+2b > 6$ and $ (a+b)^{2}> 6$ and so $ f(a+2b) = f((a+b)^{2}) = 1$ and $ f(a) = 1$ $ \forall a > 0$

And the only solution is $ f(n) = 1$ $ \forall n\in\mathbb{N}$
\end{mysolution}
*******************************************************************************
-------------------------------------------------------------------------------

\begin{problem}[Posted by \href{https://artofproblemsolving.com/community/user/14448}{hien}]
	Suppose that the function $f: \mathbb N \to \mathbb N$ satisfies the following conditions:
a) $ f(1)=1$,
b) $ 3f(n) \cdot f(2n+1) = f(2n) \cdot [1 + 3f(n)]$ for all $n \in \mathbb N$, and
c) $ f(2n)<6f(n)$ for all $n \in \mathbb N$.

Find positive integers $k$ and $m$ satisfying $f(k)+f(m)=293$.
	\flushright \href{https://artofproblemsolving.com/community/c6h168725}{(Link to AoPS)}
\end{problem}



\begin{mysolution}[by \href{https://artofproblemsolving.com/community/user/32278}{hjbrasch}]
	b) yields $ f(2n)$ is divisible by $ 3f(n)$ and then together with c) we get the recursions:

$ f(2n)=3f(n),f(2n+1)=1+3f(n)$

Claim: $ f(\sum_{i=0}^m\alpha_i2^i)=\sum_{i=0}^m\alpha_i3^i$ with $ \alpha_i\in\{0,1\}$

Proof: Apply $ f(\sum_{i=0}^m\alpha_i2^i)=\alpha_0+3\sum_{i=0}^{m-1}\alpha_{i+1}2^i$ plus induction on $ m$.

as for the solution, e.g., set $ k=47,m=5$
\end{mysolution}



\begin{mysolution}[by \href{https://artofproblemsolving.com/community/user/16261}{Rust}]
	\begin{tcolorbox}Let $ f: \mathbb{Z}^ + \rightarrow\mathbb{Z}^ +$ satify:
a) $ f(1) = 1$
b) $ 3f(n).f(2n + 1) = f(2n).[1 + 3f(n)]$
c) $ f(2n) < 6f(n)$
Find numbers $ k,m$ satifying $ f(k) + f(m) = 293$\end{tcolorbox}
Because $ (3f(n),1+3f(n))=1$ we get $ f(2n+1)=k(1+3f(n)),f(2n)=3kf(n)$, from c) we get, that k=1.
It give $ f(2n)=3f(n),f(2n+1)=3f(n)+1$. Because f(1)=1, for $ n=a_0+a_1*2+...+a_k*2^k, a_i=0,1$ 2-adic digits we get
$ f(n)=a_0+3*a_1+...+a_k*3^k$.
I think it posted before.
\end{mysolution}



\begin{mysolution}[by \href{https://artofproblemsolving.com/community/user/29428}{pco}]
	\begin{tcolorbox}b) yields $ f(2n)$ is divisible by $ 3f(n)$ and then together with c) we get the recursions:

$ f(2n) = 3f(n),f(2n + 1) = 1 + 3f(n)$

Claim: $ f(\sum_{i = 0}^m\alpha_i2^i) = \sum_{i = 0}^m\alpha_i3^i$ with $ \alpha_i\in\{0,1\}$

Proof: Apply $ f(\sum_{i = 0}^m\alpha_i2^i) = \alpha_0 + 3\sum_{i = 0}^{m - 1}\alpha_{i + 1}2^i$ plus induction on $ m$.

as for the solution, e.g., set $ k = 47,m = 5$\end{tcolorbox}

Quite OK.

And all couples $ (k,m)$ are $ (5,47),(7,45),(13,39),(15,37),(37,15),(39,13),(45,7),(47,5)$
\end{mysolution}



\begin{mysolution}[by \href{https://artofproblemsolving.com/community/user/32278}{hjbrasch}]
	actually, one can easily show that all possible solutions are given by 

$ k=32 \alpha +8\beta+2\gamma +5$
$ m=32(1-\alpha)+8(1-\beta)+2(1-\gamma)+5=52-k$

with $ \alpha,\beta,\gamma\in\{0,1\}$ (i.e. 8 different ordered solutions)
\end{mysolution}



\begin{mysolution}[by \href{https://artofproblemsolving.com/community/user/14448}{hien}]
	\begin{tcolorbox}actually, one can easily show that all possible solutions are given by 

$ k = 32 \alpha + 8\beta + 2\gamma + 5$
$ m = 32(1 - \alpha) + 8(1 - \beta) + 2(1 - \gamma) + 5 = 52 - k$

with $ \alpha,\beta,\gamma\in\{0,1\}$ (i.e. 8 different ordered solutions)\end{tcolorbox}

How can you find exactly it?
\end{mysolution}



\begin{mysolution}[by \href{https://artofproblemsolving.com/community/user/32278}{hjbrasch}]
	$ 293=1\cdot 3^5+1\cdot 3^3+2\cdot 3^2+1\cdot 3^1+2\cdot 3^0$ and now you look for decompositions

$ 293=f(k)+f(m)=\sum_{i=0}^5(a_i+b_i)3^i$ with $ 0\leq a_i,b_i\leq 1$ from which it follows that $ 0\leq a_i+b_i\leq 2$ and all sums $ a_i+b_i$ are uniquely determined.
\end{mysolution}



\begin{mysolution}[by \href{https://artofproblemsolving.com/community/user/14448}{hien}]
	Okay, i see, Thanks for your response
\end{mysolution}



\begin{mysolution}[by \href{https://artofproblemsolving.com/community/user/30326}{quangpbc}]
	\begin{tcolorbox}Okay, i see, thanks so much for reply\end{tcolorbox}


Hi teacher Hien  :D . I saw it is a problem from book " Bài toán hàm số qua các kì thi Omlympic " of Nguyễn Trọng Tuấn , it is so nice  
\end{mysolution}



\begin{mysolution}[by \href{https://artofproblemsolving.com/community/user/14448}{hien}]
	Yes, the idea of solving is that a number is presented based on base 2 and the function values in base 3
\end{mysolution}
*******************************************************************************
-------------------------------------------------------------------------------

\begin{problem}[Posted by \href{https://artofproblemsolving.com/community/user/5820}{N.T.TUAN}]
	Find a function $ f(x)$ defined for all real values of $ x$ such that
for all $ x$, $ f(x+ 2) - f(x) = x^2 + 2x + 4$,
and if $ x \in [0,2)$ , then $ f(x) = x^2.$
	\flushright \href{https://artofproblemsolving.com/community/c6h168981}{(Link to AoPS)}
\end{problem}



\begin{mysolution}[by \href{https://artofproblemsolving.com/community/user/32514}{TTsphn}]
	We consider $ f(x)$ on the $ [0,2),[2,4),..,[2n,2n+2)$ and $ [-(2n+2),-2n)$
Consider $ f(x)$ when $ x\in [2,4)$
From condition we has
$ f(x)-f(x-2)=x^2-2x+4$
Because $ x\in [2,4)$ so $ x-2\in [0,2)$ so $ f(x-2)=(x-2)^2$
So $ f(x)=x^2-2x+4+(x-2)^2=2x^2-6x+8=2(x^2-3x+4)$
We prove that  exist $ f(x)$ by induction.
\end{mysolution}



\begin{mysolution}[by \href{https://artofproblemsolving.com/community/user/29428}{pco}]
	\begin{tcolorbox}Find a function $ f(x)$ defined for all real values of $ x$ such that
for all $ x$, $ f(x + 2) - f(x) = x^2 + 2x + 4$,
and if $ x \in [0,2)$ , then $ f(x) = x^2.$\end{tcolorbox}

Although I think that TTsphn's answer is correct, here is a closed form of the unique solution :

$ f(x)=\frac{x^3+8x}{6}-\frac{4}{3}\{\frac{x}{2}\}^3+4\{\frac{x}{2}\}^2-\frac{8}{3}\{\frac{x}{2}\}$

Where $ \{a\}$ is the fractional part of $ a$
\end{mysolution}
*******************************************************************************
-------------------------------------------------------------------------------

\begin{problem}[Posted by \href{https://artofproblemsolving.com/community/user/27016}{Dtrung}]
	Let $a,b,c,$ and $d$ be given real numbers. Find all functions $f: \mathbb R \to \mathbb R$ such that
\[f(ax+b) = c\cdot f(x)+d,\]
for all $x \in \mathbb R$.
	\flushright \href{https://artofproblemsolving.com/community/c6h169113}{(Link to AoPS)}
\end{problem}



\begin{mysolution}[by \href{https://artofproblemsolving.com/community/user/29428}{pco}]
	\begin{tcolorbox}Let $ a,b,c,d$ be real numbers. Find all functions $ \large f : R \to R$ such that :
            $ f(ax + b) = c.f(x) + d$
for all $ \large x \in R$.\end{tcolorbox}

It's a rather easy problem but you must be clever about all the different cases.
Here is the answer (demo is below in the second part of this post). Let's go 

$ (a,b,c,d) = (0,b,1,0)$. Solution is $ f(x) = u$ for any real $ u$.
$ (a,b,c,d) = (0,b,1,d\neq 0)$. No solution
$ (a,b,c,d) = (0,b,c\neq 1,d)$. Solution is $ f(x) = \frac {d}{1 - c}$ constant.
$ (a,b,c,d) = (1,0,1,0)$. Solution is any function $ f(x)$.
$ (a,b,c,d) = (1,0,1,d\neq 0)$. No solution
$ (a,b,c,d) = (1,b\neq 0,1,d)$. Solution is $ f(x) = h(\{\frac {x}{b}\}) + \frac {dx}{b}$ for any function $ h(x)$ defined on $ [0,1)$.
$ (a,b,c,d) = (1,b\neq 0,c\neq 1,d)$. Solution is $ f(x) = c^{\lfloor\frac {x}{b}\rfloor }h(\{\frac {x}{b}\}) - \frac {d}{c - 1}$ for any function $ h(x)$ defined on $ [0,1)$.
$ (a,b,c,d) = ( - 1,b,c = 1,d\neq 0)$. No solution
$ (a,b,c,d) = ( - 1,b,c = 1,d = 0)$. Solution is $ f(x) = h(x - \frac {b}{2}) + h(\frac {b}{2} - x)$  for any function $ h(x)$
$ (a,b,c,d) = ( - 1,b,c = - 1,d)$. Solution is $ f(x) = h(\frac {b}{2} - x) - h(x - \frac {b}{2}) + \frac {d}{2}$  for any function $ h(x)$
$ (a,b,c,d) = ( - 1,b,c\notin\{ - 1, + 1\},d)$. Solution is $ f(x) = \frac {d}{1 - c}$ constant.
$ (a,b,c,d) = (a\notin\{ - 1,0, + 1\},b,1,d\neq 0)$. No solution
$ (a,b,c,d) = (a\notin\{ - 1,0, + 1\},b,1,0)$. Solution is $ f(x) = h(Sign(x + \frac {b}{a - 1})Sign(a)^{\lfloor\frac {\ln(|x + \frac {b}{a - 1}|)}{\ln(|a|)}\rfloor}{\{\frac {\ln(|x + \frac {b}{a - 1}|)}{\ln(|a|)}\})}$ for any function $ h(x)$ defined on $ ( - 1,1)$
$ (a,b,c,d) = (a\notin\{ - 1,0, + 1\},b,c\neq 1,d)$; Solution is $ f(x) = c^{\lfloor\frac {\ln(|x - \frac {b}{1 - a}|)}{\ln(|a|)}\rfloor}h(Sign(x - \frac {b}{1 - a})Sign(a)^{\lfloor\frac {\ln(|x - \frac {b}{1 - a}|)}{\ln(|a|)}\rfloor}{\{\frac {\ln(|x - \frac {b}{1 - a}|)}{\ln(|a|)}\}) + \frac {d}{1 - c}}$ for any function $ h(x)$ defined on $ ( - 1,1)$


Demos :

Case 1:$ a = 0$, $ c = 1$ and $ d = 0$. The equation is $ f(b) = f(x)$ and the solutions are the constant functions $ f(x) = u$.

Case 2: $ a = 0$, $ c = 1$ and $ d\neq0$. The equation is $ f(b) = f(x) + d$ and have no solution (since, for $ x = b$, we have $ 0 = d$)

Case 3: $ a = 0$ and $ c\neq 1$. The equation is $ f(b) = cf(x) + d$ and the solution is $ f(x) = \frac {d}{1 - c}$

Case 4: $ a = 1$, $ b = 0$, $ c = 1$ and $ d = 0$. The equation is $ f(x) = f(x)$ and any function $ f(x)$ is solution.

Case 5: $ a = 1$, $ b = 0$, $ c = 1$ and $ d\neq 0$. The equation is $ 0 = d$ and no solution exist.

Case 6: $ a = 1, b\neq 0, c = 1$. The equation is $ f(x+b)=f(x)+d$, so $ f(x + nb) = f(x) + nd$, so \[f(x) = f(b\{\frac {x}{b}\} + \lfloor\frac {x}{b}\rfloor b) = f(b\{\frac {x}{b}\}) + \lfloor\frac {x}{b}\rfloor d\]
So $ f(x)$ is completely defined on $ \mathbb{R}$ as soon as $ f(x)$ is defined on $ [0,b)$ (or $ (b,0]$).
The general solution is then $ f(x) = g(b\{\frac {x}{b}\}) + \lfloor\frac {x}{b}\rfloor d$ for any function $ g(x)$ defined on $ [0,b)$ (or $ (b,0]$).
Considering $ h(x) = g(bx) - dx$, we have a prettier form for the solution : $ f(x) = h(\{\frac {x}{b}\}) + \frac {dx}{b}$ for any function $ h(x)$ defined on $ [0,1)$.
And it is easy to check that this necessary condition is sufficient.

Case 7: $ a = 1$ and $ c\neq 1$. The equation is $ f(x + b) = cf(x) + d$ and so $ f(x + nb) = c^nf(x) + d\frac {c^n - 1}{c - 1}$ and so $ f(x) = f(b\{\frac {x}{b}\} + \lfloor\frac {x}{b}\rfloor b) = c^{\lfloor\frac {x}{b}\rfloor }f(b\{\frac {x}{b}\}) + d\frac {c^{\lfloor\frac {x}{b}\rfloor } - 1}{c - 1}$.
So $ f(x)$ is completely defined on $ \mathbb{R}$ as soon as $ f(x)$ is defined on $ [0,b)$ (or $ (b,0]$).
The general solution is then $ f(x) = c^{\lfloor\frac {x}{b}\rfloor }g(b\{\frac {x}{b}\}) + d\frac {c^{\lfloor\frac {x}{b}\rfloor } - 1}{c - 1}$ for any function $ g(x)$ defined on $ [0,b)$ (or $ (b,0]$).
Considering $ h(x) = g(bx) + \frac {d}{c - 1}$, we have a prettier form for the solution : $ f(x) = c^{\lfloor\frac {x}{b}\rfloor }h(\{\frac {x}{b}\}) - \frac {d}{c - 1}$ for any function $ h(x)$ defined on $ [0,1)$.
And it is easy to check that this necessary condition is sufficient.

Case 8: $ a = - 1$,$ c = 1$ and $ d\neq 0$ ; The equation is $ f(b - x) = f(x) + d$ and so, with $ x = \frac {b}{2}$ : $ 0 = d$ and no solution exist.

Case 9: $ a = - 1$,$ c = 1$ and $ d = 0$; The equation is $ f(b - x) = f(x)$. Let then $ g(x) = f(x + \frac {b/2})$. we have $ f(x) = g(x - \frac {b}{2})$ and the equation becomes $ g(\frac {b}{2} - x) = g(x - \frac {b}{2})$ and so $ g( - x) = g(x)$.
The solution is then $ f(x) = g(x - \frac {b}{2})$ for any even function $ g(x)$.
Considering that any even function can be written $ g(x) = h(x) + h( - x)$, we have another form for the solution :
$ f(x) = h(x - \frac {b}{2}) + h(\frac {b}{2} - x)$  for any function $ h(x)$
And it is easy to check that this necessary condition is sufficient.

\begin{underlined}Case 10\end{underlined}: $ a = - 1$ and $ c = - 1$; The equation is $ f(b - x) = d - f(x)$. Let then $ g(x) = f(\frac {b}{2} - x) - \frac {d}{2}$. We have $ f(x) = g(\frac {b}{2} - x) + \frac {d}{2}$ and the equation becomes $ g(x - \frac {b}{2}) = - g(\frac {b}{2} - x)$ and so $ g( - x) = - g(x)$. 
The solution is then $ f(x) = g(\frac {b}{2} - x) + \frac {d}{2}$ for any odd function $ g(x)$.
Considering that any odd function can be written $ g(x) = h(x) - h( - x)$, we have another form for the solution :
$ f(x) = h(\frac {b}{2} - x) - h(x - \frac {b}{2}) + \frac {d}{2}$  for any function $ h(x)$
And it is easy to check that this necessary condition is sufficient.


\begin{underlined}Case 11\end{underlined}: $ a = - 1$ and $ |c|\neq 1$ The equation is $ f(b - x) = cf(x) + d$ and so $ f(x) = f(b - (b - x)) = cf(b - x) + d = c^2f(x) + cd + d$. Then :
$ f(x) = \frac {d}{1 - c}$ and the original equation becomes $ \frac {d}{1 - c} = \frac {cd}{1 - c} + d$ which is true.
So the only solution is $ f(x) = \frac {d}{1 - c}$.
And it is easy to check that this necessary condition is sufficient.

\begin{underlined}Case 12\end{underlined}: $ a\neq 0$, $ |a|\neq 1$, $ c = 1$ and $ d\neq 0$. The equation is $ f(ax + b) = f(x) + d$. let then $ x = \frac {b}{1 - a}$. We have $ 0 = d$ and no solution exist.

\begin{underlined}Case 13\end{underlined}: $ a\neq 0$, $ |a|\neq 1$, $ c = 1$ and $ d = 0$. The equation is $ f(ax + b) = f(x)$. let then $ k(x) = f(x - \frac {b}{a - 1})$. We have $ f(x) = k(x + \frac {b}{a - 1})$ and the equation is $ k(a(x + \frac {b}{a - 1})) = k(x + \frac {b}{a - 1})$ and so $ k(ax) = k(x)$.
This equation is well known and a general solution is :
$ k(0) = u$ and, for $ x\neq 0$, $ k(x) = h(Sign(x)Sign(a)^{\lfloor\frac {\ln(|x|)}{\ln(|a|)}\rfloor}{\{\frac {\ln(|x|)}{\ln(|a|)\}}}$ for any function $ h(x)$ defined on $ ( - 1,1)$
And so a general solution :
$ f(\frac {b}{1 - a}) = u$ (any real constant)
For $ x\neq\frac {b}{1 - a}$ : $ f(x) = h(Sign(x + \frac {b}{a - 1})Sign(a)^{\lfloor\frac {\ln(|x + \frac {b}{a - 1}|)}{\ln(|a|)}\rfloor}{\{\frac {\ln(|x + \frac {b}{a - 1}|)}{\ln(|a|)}\})}$ for any function $ h(x)$ defined on $ ( - 1,1)$
And it is easy to check that this necessary condition is sufficient.

\begin{underlined}Case 14\end{underlined}: $ a\neq 0$, $ |a|\neq 1$ and $ c\neq 1$. The equation is $ f(ax + b) = cf(x) + d$. Let then $ k(x) = f(x + \frac {b}{1 - a}) - \frac {d}{1 - c}$. We have $ f(x) = k(x - \frac {b}{1 - a}) + \frac {d}{1 - c}$ and the equation becomes : $ k(ax) = ck(x)$
This equation is well known and a general solution is :
$ k(0) = 0$ and, for $ x\neq 0$, $ k(x) = c^{\lfloor\frac {\ln(|x|)}{\ln(|a|)}\rfloor}h(Sign(x)Sign(a)^{\lfloor\frac {\ln(|x|)}{\ln(|a|)}\rfloor}{\{\frac {\ln(|x|)}{\ln(|a|)\}}}$ for any function $ h(x)$ defined on $ ( - 1,1)$
And so a general solution :
$ f(\frac {b}{1 - a}) = 0$
For $ x\neq\frac {b}{1 - a}$ : $ f(x) = c^{\lfloor\frac {\ln(|x - \frac {b}{1 - a}|)}{\ln(|a|)}\rfloor}h(Sign(x - \frac {b}{1 - a})Sign(a)^{\lfloor\frac {\ln(|x - \frac {b}{1 - a}|)}{\ln(|a|)}\rfloor}{\{\frac {\ln(|x - \frac {b}{1 - a}|)}{\ln(|a|)}\}) + \frac {d}{1 - c}}$ for any function $ h(x)$ defined on $ ( - 1,1)$
And it is easy to check that this necessary condition is sufficient.
\end{mysolution}
*******************************************************************************
-------------------------------------------------------------------------------

\begin{problem}[Posted by \href{https://artofproblemsolving.com/community/user/25546}{Yuriy Solovyov}]
	Let $ n=111\cdots11$, where the $1$'s have been repeated $2007$ times. Does there exist a function $f: \mathbb R \to \mathbb R$ such that for all real $x$ not equal to $0$ or $1$, we have
\[f^n(x) = \left(1 - \frac{1}{\sqrt[n]{x}}\right)^{n}?\]
Note that $f^n(x)$ denotes composition of $f$ with itself $n$ times.
	\flushright \href{https://artofproblemsolving.com/community/c6h169311}{(Link to AoPS)}
\end{problem}



\begin{mysolution}[by \href{https://artofproblemsolving.com/community/user/29428}{pco}]
	\begin{tcolorbox}Let $ n = 111...11$, where we have $ 2007$ number $ 1$. Is there exist such function $ f(x)$,
that for all $ x$, except $ 1$ and $ 0$, we have:
$ f(f(f(...(f(x))...))) = (1 - \frac {1}{\sqrt [n]{x}})^{n}$?
In the left side $ f(x)$ takes $ n$ times.\end{tcolorbox}

The answer is yes.

Let $ E=\mathbb{R}-\{0,1\}$

Let $ g(x)=(1 - \frac {1}{\sqrt [n]{x}})^{n}$ from $ E\rightarrow E$ . It's easy to see that $ g(x)$ is bijective and that $ g(g(g(x)))=x$.
It's also clear that $ g(x)\neq x$ $ \forall x\in E$ and (since $ g(g(g(x)))=x$) that $ g(g(x))\neq x$ $ \forall x\in E$

Let then :
$ \mathbb{A}=\{x\in E$ such that $ x<g(x)$ and $ x<g(g(x))\}$
$ \mathbb{B}=\{g(x),x\in\mathbb{A}\}$
$ \mathbb{C}=\{g(x),x\in\mathbb{B}\}$

Clearly, $ \mathbb{A}\cap\mathbb{B}=\emptyset$, $ \mathbb{A}\cap\mathbb{C}=\emptyset$, $ \mathbb{B}\cap\mathbb{C}=\emptyset$ and $ \mathbb{A}\cup\mathbb{B}\cup\mathbb{C}=E$ and these three sets are infinite.

Let then $ A_1,A_2,...,A_n$ a partition of $ \mathbb{A}$ in $ n$ bijective subsets and $ h_i(x)$, $ i\in\{1,2,...,n-1\}$ the bijections from $ A_i$ in $ A_{i+1}$.

Let then $ f(x)$ defined as :
$ f(0)=0$
$ f(1)=1$

$ \forall x\in E$, either $ x\in\mathbb{A}$, either $ x\in\mathbb{B}$, either $ x\in\mathbb{C}$ and :

1) If $ x\in\mathbb{A}$, $ \exists k\in\{1,2,...,n\}$ such that $ x\in A_k$.
If $ k<n$, $ f(x)=h_k(x)$
If $ k=n$, $ f(x)=g(h_{n-1}^{-1}(h_{n-2}^{-1}(...(h_2^{-1}(h_1^{-1}(x)))...)))$

2) If $ x\in\mathbb{B}$, $ g^{-1}(x)=g(g(x))\in\mathbb{A}$ and let $ f(x)=g(f(g(g(x))))=g(f(g^{-1}(x)))$

3) If $ x\in\mathbb{C}$, $ g(x)=g^{-1}(g^{1}(x))\in\mathbb{A}$ and let $ f(x)=g(g(f(g(x))))=g^{-1}(f(g(x)))$

We have $ f^n(x)=g(x)$ $ \forall x\in E$ (I could give a full precise demo of this point if anybody wants).

And this is avaible for any odd $ n$.
\end{mysolution}
*******************************************************************************
-------------------------------------------------------------------------------

\begin{problem}[Posted by \href{https://artofproblemsolving.com/community/user/29382}{Gib Z}]
	Find a function $f(x)$ such that $f( f(x) ) = \exp(x)$.
	\flushright \href{https://artofproblemsolving.com/community/c6h172289}{(Link to AoPS)}
\end{problem}



\begin{mysolution}[by \href{https://artofproblemsolving.com/community/user/29428}{pco}]
	\begin{tcolorbox}Find A function f(x) such that f( f(x) ) = exp(x)\end{tcolorbox}

Rather classical problem and solution :

Let $ u < 0$
Let $ h(x)$ be any strictly increasing continuous bijection from $ ( - \infty,u]\rightarrow (u,0]$ such that : $ \lim_{x\rightarrow - \infty}h(x) = u$ and $ h(u) = 0$
Let $ \{a_n,n > 0\}$ the strictly increasing sequence defined by $ a_1 = u$, $ a_2 = 0$ and $ a_{n + 2} = e^{a_n}$ $ \forall n > 0$

Let then the sequence of functions $ \{g_k(x),k > 0\}$ defined as :
$ g_1(x)$ is the bijection from $ ( - \infty,a_1)\rightarrow (a_1,a_2)$ such that $ g_1(x) = h(x)$
$ \forall k > 1$ : 
$ g_k(x)$ is the bijection from $ [a_{k - 1},a_k)\rightarrow [a_k,a_{k + 1})$ such that $ g_k(x) = e^{g_{k - 1}^{ - 1}(x)}$

Let then $ f(x)$ the strictly increasing continuous function defined as :

$ \forall x\in( - \infty,a_1)$ $ f(x) = g_1(x)$
$ \forall x\in[a_{k - 1},a_k)$, $ f(x) = g_k(x)$ $ \forall k > 1$

Then, if $ x < a_1$, $ f(x) = h(x)\in(a_1,a_2)$, so $ f(f(x)) = e^{g_1^{ - 1}(f(x))} = e^{h^{ - 1}(h(x))} = e^x$
And, $ \forall k > 1$ :
$ \forall x\in[a_{k - 1},a_k)$, $ f(x) = g_k(x)\in[a_k,a_{k + 1})$. So $ f(f(x)) = e^{g_{k}^{ - 1}(f(x))} = e^{g_{k}^{ - 1}(g_k(x))} = e^x$

And so $ f(f(x)) = e^x$ $ \forall x\in\mathbb{R}$
\end{mysolution}
*******************************************************************************
-------------------------------------------------------------------------------

\begin{problem}[Posted by \href{https://artofproblemsolving.com/community/user/11882}{nedo477}]
	Given $a,s \in \mathbb N$, prove that there exists a function $ f: \mathbb N \rightarrow \mathbb N$ such that \[ f(f(n))=  a \cdot n^{s},\] holds for all $ n \in \mathbb N$.
	\flushright \href{https://artofproblemsolving.com/community/c6h172519}{(Link to AoPS)}
\end{problem}



\begin{mysolution}[by \href{https://artofproblemsolving.com/community/user/29428}{pco}]
	\begin{tcolorbox}Given $ a,s \in$N, prove that exist function $ f: N \rightarrow N$ such that $ f(f(n)) = a.n^{s}$ with all $ n \in N$\end{tcolorbox}

$ 1)$ If $ a=s=1$, then $ f(n)=n$ is a solution

$ 2)$ Consider now that either $ a>1$, either $ s>1$.
$ \forall x\in\mathbb{N}$, let $ g(x)$ the fewest positive integer such that it exists a non negative integer $ k$ such that $ x=a^{\sum_{i=1}^k s^{i-1}}(g(x))^{s^k}$ (with the convention that $ \sum_{i=1}^0 s^{i-1}=0$). Let then $ h(x)$ this nonnegative integer $ k$.

Let $ E=g(\mathbb{N}$. $ E$ is a countable infinite subset of $ \mathbb{N}$ (since it contains at least all the primes different from $ a$).
Let $ U$ any infinite subset of $ E$ such that $ V=E-U$ is also an infinite subset. $ U$ and $ V$ are bijective subsets (each is a countable infinite set) and let $ b(x)$ a bijection from $ U$ in $ V$.

Let then $ f(x)$ defined as :

If $ g(x)\in U$, $ f(x)=a^{\sum_{i=1}^{h(x)} s^{i-1}}(b(g(x)))^{s^{h(x)}}$

If $ g(x)\in V$, $ f(x)=a^{\sum_{i=1}^{h(x)+1} s^{i-1}}(b^{-1}(g(x)))^{s^{h(x)+1}}$

It's rather easy to check that $ f(f(x))=ax^s$ :

If $ g(x)\in U$, $ f(x)=a^{\sum_{i=1}^{h(x)} s^{i-1}}(b(g(x)))^{s^{h(x)}}$. Then $ g(f(x))=b(g(x))\in V$ and $ h(f(x))=h(x)$. And so :

$ f(f(x))=a^{\sum_{i=1}^{h(f(x))+1} s^{i-1}}(b^{-1}(g(f(x))))^{s^{h(f(x))+1}}$ $ =a^{\sum_{i=1}^{h(x)+1} s^{i-1}}(b^{-1}(b(g(x))))^{s^{h(x)+1}}$ $ =a(a^{\sum_{i=1}^{h(x)} s^{i-1}}(g(x))^{s^{h(x)}})^s$ $ =a\cdot x^s$

If $ g(x)\in V$, $ f(x)=a^{\sum_{i=1}^{h(x)+1} s^{i-1}}(b^{-1}(g(x)))^{s^{h(x)+1}}$. Then $ g(f(x))=b^{-1}(g(x))\in U$ and $ h(f(x))=h(x)+1$. And so :

$ f(f(x))=a^{\sum_{i=1}^{h(f(x))} s^{i-1}}(b(g(f(x))))^{s^{h(f(x))}}$ $ =a^{\sum_{i=1}^{h(x)+1} s^{i-1}}(b(b^{-1}(g(x))))^{s^{h(x)+1}}$ $ =a(a^{\sum_{i=1}^{h(x)} s^{i-1}}(g(x))^{s^{h(x)}})^s$ $ =a\cdot x^s$
\end{mysolution}



\begin{mysolution}[by \href{https://artofproblemsolving.com/community/user/11882}{nedo477}]
	I've read your solution. It's great! And here is my way
*Let $ g(n)=a.n^{s}; g_{m}(n)=g(..(g(n))$ ($ m$ times) and $ g_{0}(n)=n$
$ B=(b\ \in N|b \neq g(n) \forall n)$

*First we'll prove that $ \forall n$, $ \exists ! b \in B,m \in N$ such that $ n=g_{m}(b) (1)$
+$ n \in B \rightarrow m=0$, $ (1)$ right
  $ n \notin B \rightarrow \exists n' : n=g(n') \rightarrow n'<n \rightarrow n'=g_{m'}(b') \rightarrow n=g_{m'+1}(b')$, $ (1)$ right
+Assume $ \exists (m',b') \neq (m,b)$ so that $ g_{m'}(b')=n=g_{m}(b)$
If $ m'=m$, then $ b'=b$,wrong
If $ m'>m$, then $ b=g_{m'-m}(b') \notin B$, wrong
If $ m'<m$, then $ b'=g_{m-m'}(b) \notin B$, wrong
So $ (m,b)$ is unique. $ (1)$ is proved

*Now, we'll finish this problem
$ B=(b_{1},b_{2},....)$ $ (b_{1}<b_{2}<...)$
$ n=g_{m}(b_{i})$ 
If $ i \vdots 2$, $ f(n)=g_{m}(b_{i-1})$
else $ f(n)=g_{m+1}(b_{i+1})$

Then $ f(n)$ is the function we need (easy to check).
\end{mysolution}



\begin{mysolution}[by \href{https://artofproblemsolving.com/community/user/29428}{pco}]
	\begin{tcolorbox}I've read your solution. It's great! And here is my way
*Let $ g(n) = a.n^{s}; g_{m}(n) = g(..(g(n))$ ($ m$ times) and $ g_{0}(n) = n$
$ B = (b\ \in N|b \neq g(n) \forall n)$

*First we'll prove that $ \forall n$, $ \exists ! b \in B,m \in N$ such that $ n = g_{m}(b) (1)$
+$ n \in B \rightarrow m = 0$, $ (1)$ right
  $ n \notin B \rightarrow \exists n' : n = g(n') \rightarrow n' < n \rightarrow n' = g_{m'}(b') \rightarrow n = g_{m' + 1}(b')$, $ (1)$ right
+Assume $ \exists (m',b') \neq (m,b)$ so that $ g_{m'}(b') = n = g_{m}(b)$
If $ m' = m$, then $ b' = b$,wrong
If $ m' > m$, then $ b = g_{m' - m}(b') \notin B$, wrong
If $ m' < m$, then $ b' = g_{m - m'}(b) \notin B$, wrong
So $ (m,b)$ is unique. $ (1)$ is proved

*Now, we'll finish this problem
$ B = (b_{1},b_{2},....)$ $ (b_{1} < b_{2} < ...)$
$ n = g_{m}(b_{i})$ 
If $ i \vdots 2$, $ f(n) = g_{m}(b_{i - 1})$
else $ f(n) = g_{m + 1}(b_{i + 1})$

Then $ f(n)$ is the function we need (easy to check).\end{tcolorbox}

Quite OK, and we have nearly the same demos :

My $ E$ is your $ B$
My $ g(n)$ is your function mapping any $ n$ on the $ b_i$ such that $ n=g_m(b_i)$
And you choose as splitting $ U,V$ of $ E$ :
$ E=B=\{b_1,b_2,b_3,\ldots\}$ $ = U\cup V=\{b_1,b_3,b_5,\ldots\}\cup\{b_2,b_4,b_6,\ldots\}$
\end{mysolution}
*******************************************************************************
-------------------------------------------------------------------------------

\begin{problem}[Posted by \href{https://artofproblemsolving.com/community/user/29721}{Erken}]
	Find all functions $ f: \mathbb{N}\rightarrow\mathbb{R}$ such that for all $ x,y,z\in\mathbb{N}$,
\[f(xy)+f(xz)\geq f(x)f(yz)+1.\]
	\flushright \href{https://artofproblemsolving.com/community/c6h172672}{(Link to AoPS)}
\end{problem}



\begin{mysolution}[by \href{https://artofproblemsolving.com/community/user/17435}{ashwinrk\_jain}]
	this is quite interesting question
$ f(xy) + f(xz) >= f(x)f(yz)+1$
hence we get
$ f(0)+\frac{1}{f(0)}<=2$
but we know
$ f(0)+\frac{1}{f(0)}>=2$
hence we can say
$ f(0)=1$
now, put $ x=y=z=1$
we get
$ f(1)[2-f(1)]>=1$
which again gives
$ (f(1)-1)^2=0$
that is $ f(1)=1$
now,
put y=0
$ f(0)+f(xz)>=f(x)f(0)+1$
solving this we get
$ f(xz)>=f(x)$..........................(1)
put x=1
$ f(z)>=f(1)$............................(2)
also put z=1 in (1)
we $ f(0)>=f(x)$......................(3)
combining (2) and (3)
we get
$ f(1)<=f(x)<=f(0)$
that is
$ 1<=f(x)<=1$
hence we get
$ f(x)=1$
\end{mysolution}



\begin{mysolution}[by \href{https://artofproblemsolving.com/community/user/29721}{Erken}]
	In this question $ \mathbb{N}$ is a set of natural numbers,i.e $ \mathbb{N}=\{1,2,3\dots\}$.So you can't consider $ f(0)$.
\end{mysolution}



\begin{mysolution}[by \href{https://artofproblemsolving.com/community/user/29428}{pco}]
	\begin{tcolorbox}Find all functions $ f: \mathbb{N}\rightarrow\mathbb{R}$,such that for all $ x,y,z\in\mathbb{N}$:
$ f(xy) + f(xz)\geq f(x)f(yz) + 1$.\end{tcolorbox}

Let $ P(x,y,z)$ the property $ f(xy) + f(xz)\geq f(x)f(yz) + 1$

$ 1)$ $ P(1,1,1)$ gives $ 2f(1)\geq f(1)^2+1$, and so $ 0\geq f(1)^2-2f(1)+1=(f(1)-1)^2$ and so $ \boxed{f(1)=1}$

$ 2)$ $ P(x,1,1)$ gives $ 2f(x)\geq f(x)f(1)+1=f(x)+1$ and so $ \boxed{f(x)\geq 1\: \forall x\in\mathbb{N}}$

$ 3)$ $ P(x,x,x)$ gives $ 2f(x^2)\geq f(x)f(x^2)+1$ and so $ f(x^2)(f(x)-2)\leq -1$ and so $ \boxed{f(x)<2\: \forall x\in\mathbb{N}}$

$ 4)$ Let then $ a=\sup(\{f(n),n\in\mathbb{N}\})$ the least upper bound of $ f(\mathbb{N})$. $ a$ exists since $ f(\mathbb{N})\subseteq[1,2)$ and $ 2\geq a\geq 1$
Let then $ a>\epsilon>0$ and $ p$ such that $ 0\leq a-f(p)<\epsilon$
$ P(p,p,1)$ gives $ f(p^2)+f(p)\geq (f(p))^2+1$. And, since LHS is less tha $ a+a$ : $ (f(p))^2+1 \leq 2a$ and so $ (a-\epsilon)^2+1\leq 2a$
And so $ (a-1)^2<2\epsilon a -\epsilon^2<4\epsilon$ $ \forall \epsilon\in(0,1)$
And so $ a=1$
And so $ \boxed{f(x)=1\: \: \forall x\in\mathbb{N}}$
\end{mysolution}
*******************************************************************************
-------------------------------------------------------------------------------

\begin{problem}[Posted by \href{https://artofproblemsolving.com/community/user/24822}{Svejk}]
	Let $ f: (0,\infty) \rightarrow (0,\infty)$ be a non-constant function such that
\[ f(x)\cdot f(yf(x))\cdot f(zf(x + y)) = f(x + y + z)\quad \forall x,y,z \in (0 ,\infty).\]
a) Prove that $f$ is injective.
b) Find $f$.
	\flushright \href{https://artofproblemsolving.com/community/c6h172703}{(Link to AoPS)}
\end{problem}



\begin{mysolution}[by \href{https://artofproblemsolving.com/community/user/29428}{pco}]
	\begin{tcolorbox}Let $ f: (0,\infty) \rightarrow (0,\infty)$ ,  such that : 

$ f(x)\cdot f(yf(x))\cdot f(zf(x + y)) = f(x + y + z)$ $ \forall x,y,z \in (0 ,\infty)$ 

a)prove that $ f$ is injective 
b) find $ f$\end{tcolorbox}

Obviously, it is impossible to prove that $ f(x)$ is injective since $ f(x)=1$, non injective, is a solution of the equation.
\end{mysolution}



\begin{mysolution}[by \href{https://artofproblemsolving.com/community/user/24822}{Svejk}]

yes you are right..I forgot to mention that f must not be constant.I edited my initial post now.I m sorry.
\end{mysolution}



\begin{mysolution}[by \href{https://artofproblemsolving.com/community/user/29428}{pco}]
	\begin{tcolorbox}Let $ f: (0,\infty) \rightarrow (0,\infty)$ a non constant function  such that : 

$ f(x)\cdot f(yf(x))\cdot f(zf(x + y)) = f(x + y + z)$ $ \forall x,y,z \in (0 ,\infty)$ 

a)prove that $ f$ is injective 
b) find $ f$\end{tcolorbox}

Let $ P_0(x,y,z)$ the property $ f(x) f(yf(x)) f(zf(x + y)) = f(x + y + z)$

$ 1)$ Assume it exists $ x,y>0$ such that $ f(x+y)>1$. Then $ \frac{x+y}{f(x+y)-1}>0$ and :
$ P_0(x,y,\frac{x+y}{f(x+y)-1})$ gives $ f(x)f(yf(x))f(f(x+y)\frac{x+y}{f(x+y)-1})=f(x+y+\frac{x+y}{f(x+y)-1})$
And since $ f(x+y)\frac{x+y}{f(x+y)-1}=x+y+\frac{x+y}{f(x+y)-1}$, we have $ f(x)f(yf(x))=1$
Then $ P_0(x,y,\frac{x+y}{f(x+y})$ gives $ f(x)f(yf(x))f(x+y)=f(x+y+\frac{x+y}{f(x+y)})$ and, since $ f(x)f(yf(x))=1$ and $ f(x+y)>1$ :
$ f(x+y+\frac{x+y)}{f(x+y})>1$

So, $ f(x+y)>1$ implies $ \boxed{f(yf(x))=\frac{1}{f(x)}}$ and $ f(x+y+\frac{x+y}{f(x+y})>1$
Then, $ f(x+y+\frac{x+y}{f(x+y)})>1$ implies $ \boxed{f(\frac{x+y}{f(x+y)}f(x+y))=\frac{1}{f(x+y)}}$

So, $ P_0(x,y,\frac{x+y}{f(x+y)})$ gives $ f(x)f(yf(x))f(\frac{x+y}{f(x+y)}f(x+y))=f(x+y+\frac{x+y}{f(x+y)})$ and so :
$ f(x)\frac{1}{f(x)}\frac{1}{f(x+y)}=f(x+y+\frac{x+y}{f(x+y)})$ and so :
$ f(x+y+\frac{x+y}{f(x+y)})f(x+y)=1$  which is impossible since both factors are $ >1$

So $ \boxed{f(x)\leq 1\: \: \forall x>0}$

$ 2)$ Since $ f(x)\leq 1$ $ \forall x$, then $ f(x+y+z)=f(x)f(yf(x))f(zf(x+y))\leq f(x)$ and $ f(x)$ is a non increasing function.
Assume then it exists $ a>0$ such that $ f(a)=1$.
Since $ f(x)$ is a non increasing function, $ f(x)=1$ $ \forall x\leq a$.
Then $ P_0(\frac{a}{2}, \frac{a}{2},\frac{a}{2})$ gives $ f(\frac{a}{2})f(\frac{a}{2}f(\frac{a}{2}))f(\frac{a}{2}f(a))=f(\frac{3a}{2})$
But each factor in LHS is $ 1$ and so $ f(\frac{3a}{2})=1$ and so $ f((\frac{3}{2})^na)=1$ $ \forall n$ and so $ f(x)=1$ $ \forall x\leq (\frac{3}{2})^na$ $ \forall n$
And so $ f(x)=1$ $ \forall x$.
So, since $ f(x)$ is a non-constant function, such $ a>0$ respecting $ f(a)=1$ does not exist and 
$ \boxed{0<f(x)<1\: \: \forall x>0}$  

$ 3)$ Since $ f(u)<1$ $ \forall u>0$ : $ f(x+y+z)=f(x)f(yf(x))f(zf(x+y))< f(x)$ and $ f(x)$ is a stricty decreasing function.
As a consequence, $ f(x)$ is an injective function.

$ 4)$ Let then $ x,y,z>0$. Let $ u=\frac{zf(x+y)}{f(x)}$ and $ v=\frac{yf(x)}{f(x+u)}$; We have :
$ uf(x)=zf(x+y)$ and $ vf(x+u)=yf(x)$ and so $ f(x)f(yf(x))f(zf(x+y))=f(x)f(uf(x))f(vf(x+u))$ and so
$ f(x+y+z)=f(x+u+v)$ and, since $ f(x)$ is injective, $ y+z=u+v$ :

So $ y+\frac{uf(x)}{f(x+y)}=u+\frac{yf(x)}{f(x+u)}$

So $ \frac{f(x)}{yf(x+y)}-\frac{1}{y}=\frac{f(x)}{uf(x+u)}-\frac{1}{u}=c$

Sp $ f(x+y)=\frac{f(x)}{cy+1}$ $ \forall x,y>0$

So $ f(y)=\frac{f(x)}{cy-cx+1}$ $ \forall y>x>0$

And so $ f(x)=\frac{1}{ax+b}$ for $ x>d$ and some $ a$ and $ b$

Putting back this expression in the orinal equation, we find $ \boxed{f(x)=\frac{1}{cx+1}}$
\end{mysolution}
*******************************************************************************
-------------------------------------------------------------------------------

\begin{problem}[Posted by \href{https://artofproblemsolving.com/community/user/5820}{N.T.TUAN}]
	Find all functions $ f: \mathbb{R}\to\mathbb{R}$ such that \[ f(x+f(xy))=f(x)+xf(y)\quad \forall x,y\in\mathbb{R}.\]
	\flushright \href{https://artofproblemsolving.com/community/c6h172996}{(Link to AoPS)}
\end{problem}



\begin{mysolution}[by \href{https://artofproblemsolving.com/community/user/29428}{pco}]
	\begin{tcolorbox}Find all functions $ f: \mathbb{R}\to\mathbb{R}$ such that
\[ f(x + f(xy)) = f(x) + xf(y)\,\forall x,y\in\mathbb{R}.
\]
\end{tcolorbox}

Let $ P(x,y)$ be the property $ f(x + f(xy)) = f(x) + xf(y)$

$ 1)$ Assume it exists $ u\neq 0$ such that $ f(u)=0$.
Let then $ v\neq 0$. $ P(\frac{u}{v},v)$ gives $ f(\frac{u}{v} + f(u)) = f(\frac{u}{v}) + \frac{u}{v}f(v)$ and so $ 0=\frac{u}{v}f(v)$ and so $ f(v)=0$ and $ f(x)=0$ $ \forall x \neq 0$.
Then, if $ f(0)=a\neq 0$, $ P(a,0)$ gives $ f(2a)=f(a)+af(0)$ and so $ f(0)=0$ and contradiction.
So, if it exists $ u\neq 0$ such that $ f(u)=0$, then $ f(x)=0$ $ \forall x$ and this is a first solution to the problem.

Now, we'll consider $ f(x)$ not always 0. And so $ \boxed{f(u)=0\implies u=0}$

$ 2)$ $ P(-1,-1)$ gives $ f(-1+f(1))=f(-1)-f(-1)=0$ and so $ \boxed{f(1)=1}$ and $ \boxed{f(0)=0}$

$ 3)$ We have $ f(0)=0$ and $ f(1)=1$. Assume $ f(n)=n$. Then $ P(1,n)$ gives $ f(n+1)=n+1$ and so, with induction, $ \boxed{f(n)=n\: \: \forall n\in\mathbb{N}\cup\{0\}}$

$ 4)$ Let $ n\geq 1$ $ P(-1,-n)$ gives $ f(n-1)=f(-1)-f(-n)$ and so $ f(-n)=1+f(-1)-n$
Then $ P(-n,-1)$ gives $ f(0)=0=f(-n)-nf(-1)=1+f(-1)-n-nf(-1)$$ =(1+f(-1))(1-n)$ and so $ f(-1)=-1$ and so $ f(-n)=-n$
So $ \boxed{f(n)=n\: \: \forall n\in\mathbb{Z}}$

$ 5)$ $ P(-1,x)$ gives $ f(-1+f(-x))=-1-f(x)$ and so $ 1+f(-1+f(-x))=-f(x)$; Then $ P(1,-1+f(-x))$ gives $ \boxed{f(-f(x))=-f(x)}$
Then $ -f(-f(x))=f(x)$ and so $ f(-f(-f(x)))=-f(-f(x))$ and so $ \boxed{f(f(x))=f(x)}$

$ 6)$ $ P(1,x)$ gives $ f(f(x)+1)=f(x)+1$ and, replacing $ x$ with $ f(x)+1$, we obtain $ f(f(x)+2)=f(x)+2$ and, with immediate induction, $ f(f(x)+n)=f(x)+n$ $ \forall n\in\mathbb{N}\cup\{0\}$
$ P(-1,x)$ gives $ f(-1+f(-x))=-1-f(x)$ and so $ 2+f(-1+f(-x))=1-f(x)$ and so (since $ f(f(x)+n)=f(x)+n$) $ f(1-f(x))=1-f(x)$
So $ -f(1-f(x))=f(x)-1$ and, since $ f(-f(x))=-f(x)$ : $ f(f(x)-1)=f(x)-1$, and with induction $ f(f(x)-n)=f(x)-n$
So $ \boxed{f(f(x)+n)=f(x)+n\: \: \forall n\in\mathbb{Z}}$

$ 7)$ Now, replacing $ x$ by $ nx$ in point 6 above : $ f(f(nx)+n)=f(nx)+n$.
But $ P(n,x)$ gives $ f(n+f(nx))=f(n)+nf(x)=n+nf(x)$ and so, combining these two lines : $ f(nx)+n=nf(x)+n$ and so $ \boxed{f(nx)=nf(x)\: \: \forall n\in\mathbb{Z}}$

$ 8)$ Assume, for a given $ u$, $ f(u)=-u$, then $ f(-u)=u$ (since, according to point 7 above, $ f(nu)=nf(u)$; take $ n=-1$).
Then $ P(-1,-u)$ gives $ f(-1+f(u))=-1-f(-u)$ and so $ f(-1-u)=-1-u$
Then $ P(1,-1-u)$ gives $ f(1+f(-1-u))=1+f(-1-u)$ and so $ f(-u)=-u$ and, since $ f(-u)=u$, $ u=-u=0$
So, $ \boxed{f(u)=-u\implies u=0}$

$ 9)$ $ P(x,-1)$ gives $ f(x+f(-x))=f(x)-x$ and so $ f(x-f(x))=f(x)-x$, and so (according to point 8 above) $ x-f(x)=0$
And so $ \boxed{f(x)=x\: \: \forall x}$


So the two only solutions are :
$ f(x)=0$ $ \forall x$
$ f(x)=x$ $ \forall x$
\end{mysolution}
*******************************************************************************
-------------------------------------------------------------------------------

\begin{problem}[Posted by \href{https://artofproblemsolving.com/community/user/1018}{cyshine}]
	Let $ f(x) = x^2 + 2007x + 1$. Prove that for every positive integer $ n$, the equation $ \underbrace{f(f(\ldots(f}_{n\ {\rm times}}(x))\ldots)) = 0$ has at least one real solution.
	\flushright \href{https://artofproblemsolving.com/community/c6h173067}{(Link to AoPS)}
\end{problem}



\begin{mysolution}[by \href{https://artofproblemsolving.com/community/user/32514}{TTsphn}]
	Solution 1 :
\begin{underlined} Result \end{underlined}
Consider the equation :
$ f(x) = a$ where $ a\geq \frac { - 2007^2 + 4}{4}$ has solution $ x_1,x_2$ then 
${ x_1 = \max\{x_1,x_2}\geq \frac { - 2007}{2}$
Proof : From $ x_1 + x_2 = - 2007$
We prove by induction $ f_{n}x = a$ has at least real solution if $ a > \frac { - 2007^2 + 4}{4}$
Suppose it is true for $ n$ 
We prove it is true for $ n + 1$
$ f_{n + 1}(x) = f_n(x)^2 + 2007f_{n}(x) + 1$
If $ f_n(x_n) = x_1$ then $ x_n$ is a root of $ f_{n + 1}(x) = 0$
But we have $ x_1 > \frac { - 2007^2 + 4}{4}$ so the equation has solution .Call it is $ x_{11}\geq \frac{-2007}{4}$
Contine consider equation $ f_{n-1}=x_{11}$ ..
It has solution. 
Our induction claim.
Apply this result our problem was be claim.
I know that have other solution so please contine discuss it .
\end{mysolution}



\begin{mysolution}[by \href{https://artofproblemsolving.com/community/user/29428}{pco}]
	\begin{tcolorbox}Let $ f(x) = x^2 + 2007x + 1$. Prove that for every positive integer $ n$, the equation $ \underbrace{f(f(\ldots(f}_{n\ {\rm times}}(x))\ldots)) = 0$ has at least one real solution.\end{tcolorbox}

The equation $ f(x)=x$ has two solutions $ a<b<0$.

Then, if $ g(x)=\underbrace{f(f(\ldots(f}_{n\ {\rm times}}(x))\ldots))$, we have :

$ 1)$ $ g(b)=b < 0$
$ 2)$ $ g(x)$ is a polynomial whose top term is $ x^{2^n}$ and so ${ \lim_{x\rightarrow+\infty}g(x)=}\infty >0$

And so $ g(x)$ has at least one real root in $ (b,+\infty)$
\end{mysolution}



\begin{mysolution}[by \href{https://artofproblemsolving.com/community/user/32278}{hjbrasch}]
	Define $ a=-\frac{2007}{2}\,,\,b=-a^2+1$ and the intervals $ D=[a,\infty[\,,\,W=[b,\infty[$, then we have $ 0\in D\subset W$ and $ f(x)=(x-a)^2+b$, i.e. $ f(\mathbb{R})=f(D)=W$ and therefore $ f(\ldots f(\ldots (\mathbb{R})\ldots )=W$
\end{mysolution}



\begin{mysolution}[by \href{https://artofproblemsolving.com/community/user/34083}{Mithril}]
	$ f(x)=q$ has solution if $ q>1-\frac{2007^2}4$. One solution for $ f(x)=q$ is $ x_1=\frac{\sqrt s -2007}2>-2007>1-\frac{2007^2}4$. So $ f(x)=x_1$ has a root $ x_2$ such that $ f(x)=x_2$ has a root $ x_3$..., and if we assume $ q=0$, we can do this process indefinitely, which is what we had to prove.
\end{mysolution}



\begin{mysolution}[by \href{https://artofproblemsolving.com/community/user/32514}{TTsphn}]
	Here is general problem (can solve with my method )
Find then number of solution of the equation : $ f_{n}(x)=0$
\end{mysolution}



\begin{mysolution}[by \href{https://artofproblemsolving.com/community/user/53240}{SnowEverywhere}]
	I think that this works.

\begin{bolded}Solution\end{bolded}

Let $Q(x)=\underbrace{f(f(\ldots(f}_{n\ {\rm times}}(x))\ldots))$.

We have that
\[x \rightarrow \infty \quad \Rightarrow \quad f(x) \rightarrow \infty\]
Therefore we also have that
\[x \rightarrow \infty \quad \Rightarrow \quad Q(x) \rightarrow \infty\]
Hence there exists arbitrarily large value $\alpha$ such that $Q(\alpha)>0$.

Let $\beta$ denote a root of the equation $f(x)=x$. The quadratic formula yields that $\beta < 0$. This yields that $Q(\beta)=\beta<0$.

By the intermediate value theorem, there exists $x$ between $\alpha$ and $\beta$ such that $Q(x)=0$.
\end{mysolution}
*******************************************************************************
-------------------------------------------------------------------------------

\begin{problem}[Posted by \href{https://artofproblemsolving.com/community/user/31750}{primoz2}]
	Prove that there are no functions $f: \mathbb N \to \mathbb N$ satisfying $f(f(n))=n+1$ for all positive integers $n$.
	\flushright \href{https://artofproblemsolving.com/community/c6h173160}{(Link to AoPS)}
\end{problem}



\begin{mysolution}[by \href{https://artofproblemsolving.com/community/user/29428}{pco}]
	\begin{tcolorbox}Prove that there are no funkcions satisfying  f(f(n))=n+1,     where n is a natural number\end{tcolorbox}

$ 1)$ Wrong : $ f(n) = n + \frac {1}{2}$ is a solution.

$ 2)$ If you add to your problem that $ f(x)$ is a function from $ \mathbb{N}$ $ \rightarrow$ $ \mathbb{N}$, then :

\begin{underlined}First solution \end{underlined}:
If $ a = 1$, $ f(f(a) = f(f(1)) = f(a) = f(1) = a = 1\neq 1 + 1$. hence $ f(1) = a > 1$.

Now, the notation $ f^p(x)$ means $ f(f(..(x))..)$ p times and not the power.

Then $ f^{2}(1) = 2$, $ f^{4}(1) = 3$, ... and $ f^{2a - 2}(1) = a$ and $ f^{2a - 1}(1) = f(a) = f(f(1)) = 2$
But $ f^{2a - 1}(1) = f^{2a - 2}(f(1)) = f(1) + a - 1 = 2a - 1$

So $ 2a - 1 = 2$ and $ a = \frac {3}{2}$ which is impossible.
hence the result.

\begin{underlined}Second solution : another quicker method \end{underlined}: $ f(f(n))=n+1$ implies $ f(n+1)=f(n)+1$ and so $ f(n)=n+f(1)-1$
Then $ f(f(n))=n+1$ implies $ n+2(f(1)-1))=n+1$ and so $ f(1)=\frac{3}{2}$ which is impossible.
\end{mysolution}



\begin{mysolution}[by \href{https://artofproblemsolving.com/community/user/31750}{primoz2}]
	I ment it like in 2, but i didnt`t know how to write that what you aded.
\end{mysolution}
*******************************************************************************
-------------------------------------------------------------------------------

\begin{problem}[Posted by \href{https://artofproblemsolving.com/community/user/22709}{tanpham90}]
	Find all continuous functions $f: \mathbb R \to \mathbb R$ which satisfy
\[ f(f(x))=3f(x)-2x,\quad \forall x \in \mathbb R.\]
	\flushright \href{https://artofproblemsolving.com/community/c6h173247}{(Link to AoPS)}
\end{problem}



\begin{mysolution}[by \href{https://artofproblemsolving.com/community/user/29428}{pco}]
	\begin{tcolorbox}Find all continuous functions $ f : R \to R$ which is satisfy the requirements :
$ f(f(x)) = 3f(x) - 2x$ $ \forall x \in R$\end{tcolorbox}

$ 1)$ $ f(x) = x$ is a solution.
So we'll now consider functions for which it exists an $ x_0$ such that $ f(x_0)\neq x_0$

$ 2)$ $ f(x)$ is obviously injective, and so monotonous (since continuous).
If $ f(x_0) > x_0$, then $ f(f(x_0)) = 3f(x_0) - 2x_0 > f(x_0)$ and so $ f(x)$ is strictly increasing.
If $ f(x_0) < x_0$, then $ f(f(x_0)) = 3f(x_0) - 2x_0 < f(x_0)$ and so $ f(x)$ is strictly increasing.
If $ \lim_{x\rightarrow + \infty}f(x) = L$, then, in $ f(f(x)) = 3f(x) - 2x$, when $ x\rightarrow + \infty$, LHS$ \rightarrow f(f(L))$ and RHS$ \rightarrow - \infty$. So $ \lim_{x\rightarrow + \infty}f(x) = + \infty$
If $ \lim_{x\rightarrow - \infty}f(x) = l$, then, in $ f(f(x)) = 3f(x) - 2x$, when $ x\rightarrow - \infty$, LHS$ \rightarrow f(f(l))$ and RHS$ \rightarrow + \infty$. So $ \lim_{x\rightarrow - \infty}f(x) = - \infty$
So $ f(x)$ is a strictly increasing bijection and $ f(\mathbb{R}) = \mathbb{R}$

$ 3)$ Using $ f(f(x)) = 3f(x) - 2x$, it is easy to establish that $ f(u_{n}) = u_{n + 1}$ with $ u_0 = x_0$, $ u_1 = f(x_0)$ and $ u_{n + 2} = 3u_{n + 1} - 2u_n$ for any $ n\geq 0$. Then, since $ u_n = (2x_0 - f(x_0)) + (f(x_0) - x_0)2^n$, we have $ f(u_n) = 2u_n - (2x_0 - f(x_0))$
Using that $ f(x)$ is a bijection and that $ f(f(x)) = 3f(x) - 2x$ implies $ 2f^{ - 1}(x) = 3x - f(x)$, it is possible to show that the previous equalities are true even for $ n < 0$ and so :
$ f(x) = 2x - L_0$ with $ L_0 = 2x_0 - f(x_0)$ for any $ x\in\{(2x_0 - f(x_0)) + (f(x_0) - x_0)2^n\: \: \forall n\in\mathbb{Z}\}$.
With continuity, we have, when $ n\rightarrow - \infty$, $ f(L_0) = L_0$ 

$ 4)$ Now, consider $ f(x_0) > x_0$. If it exists $ x_1$( Wlog $ x_1 > x_0$) such that $ f(x_1) > x_1$ and $ f(x_1)\neq 2x_1 - L_0$, then we have $ f(L_0) = L_0$ and $ f(L_1) = L_1$.
For any $ L\in(L_0,L_1)$, with continuity, it exist a value $ a$ such that $ f(a) = 2a - L$. Applying then the demo of point 3 above starting with $ a$ and $ f(a)$, we conclude $ f(L) = L$, and so $ f(x) = x$ $ \forall x\in(L_0,L_1)$ which is in contradiction with $ f(x) = 2x - L_0$ for any $ x\in\{(2x_0 - f(x_0)) + (f(x_0) - x_0)2^n\: \: \forall n\in\mathbb{Z}\}$.
So $ f(x_0) > x_0$  implies :
$ f(x)\leq x$ $ \forall x\leq L_0$
$ f(x) = 2x - L_0$ $ \forall x\geq L_0$

$ 5)$ Now, consider $ f(x_0) < x_0$. The same demo as above gives :
$ f(x)\geq x$ $ \forall x\geq L_0$
$ f(x) = 2x - L_0$ $ \forall x\leq L_0$

And so the solutions :

$ a)$ $ f(x) = x$
$ b)$ Let $ L\in\mathbb{R}$. $ f(x) = x$ $ \forall x\leq L$ and $ f(x) = 2x - L$ $ \forall x\geq L$
$ c)$ Let $ L\in\mathbb{R}$. $ f(x) = x$ $ \forall x\geq L$ and $ f(x) = 2x - L$ $ \forall x\leq L$
$ d)$ Let $ L_0\leq L_1\in\mathbb{R}$. $ f(x) = 2x - L_0$ $ \forall x\leq L_0$, $ f(x) = x$ $ \forall x\in[L_0,L_1]$ and $ f(x) = 2x - L_1$ $ \forall x\geq L_1$
\end{mysolution}



\begin{mysolution}[by \href{https://artofproblemsolving.com/community/user/22709}{tanpham90}]
	Thank a million , Patrick  :)  ! 
\begin{tcolorbox}
$ 4)$ Now, consider $ f(x_0) > x_0$. If it exists $ x_1$( Wlog $ x_1 > x_0$) such that $ f(x_1) > x_1$ and $ f(x_1)\neq 2x_1 - L_0$, then we have $ f(L_0) = L_0$ and $ f(L_1) = L_1$.
For any $ L\in(L_0,L_1)$, with continuity, it exist a value $ a$ such that $ f(a) = 2a - L$. Applying then the demo of point 3 above starting with $ a$ and $ f(a)$, we conclude $ f(L) = L$, and so $ f(x) = x$ $ \forall x\in(L_0,L_1)$ which is in contradiction with $ f(x) = 2x - L_0$ for any $ x\in\{(2x_0 - f(x_0)) + (f(x_0) - x_0)2^n\: \: \forall n\in\mathbb{Z}\}$.
So $ f(x_0) > x_0$  implies :
$ f(x)\leq x$ $ \forall x\leq L_0$
$ f(x) = 2x - L_0$ $ \forall x\geq L_0$
\end{tcolorbox}
Can you explain more detail , please ! What is $ L_1$ and why do we have $ f(L_1)=L_1$
\end{mysolution}



\begin{mysolution}[by \href{https://artofproblemsolving.com/community/user/29428}{pco}]
	\begin{tcolorbox}Thank a million , Patrick  :)  ! 
\begin{tcolorbox}
$ 4)$ Now, consider $ f(x_0) > x_0$. If it exists $ x_1$( Wlog $ x_1 > x_0$) such that $ f(x_1) > x_1$ and $ f(x_1)\neq 2x_1 - L_0$, then we have $ f(L_0) = L_0$ and $ f(L_1) = L_1$.
For any $ L\in(L_0,L_1)$, with continuity, it exist a value $ a$ such that $ f(a) = 2a - L$. Applying then the demo of point 3 above starting with $ a$ and $ f(a)$, we conclude $ f(L) = L$, and so $ f(x) = x$ $ \forall x\in(L_0,L_1)$ which is in contradiction with $ f(x) = 2x - L_0$ for any $ x\in\{(2x_0 - f(x_0)) + (f(x_0) - x_0)2^n\: \: \forall n\in\mathbb{Z}\}$.
So $ f(x_0) > x_0$  implies :
$ f(x)\leq x$ $ \forall x\leq L_0$
$ f(x) = 2x - L_0$ $ \forall x\geq L_0$
\end{tcolorbox}
Can you explain more detail , please ! What is $ L_1$ and why do we have $ f(L_1) = L_1$\end{tcolorbox}

Sure, it's not very clear. i'm sorry.
let's try in a different manner :

Let $ x_0$ and $ x_1$ such that $ f(x_0)>x_0$ and $ f(x_1)>x_1$

Then we have shown in the point 3 above that :
$ f(x) = 2x - L_0$ with $ L_0 = 2x_0 - f(x_0)$ for any $ x\in\{(2x_0 - f(x_0)) + (f(x_0) - x_0)2^n\: \: \forall n\in\mathbb{Z}\}$.
$ f(L_0)=L_0$
And similarly :
$ f(x) = 2x - L_1$ with $ L_1 = 2x_1 - f(x_1)$ for any $ x\in\{(2x_1 - f(x_1)) + (f(x_1) - x_1)2^n\: \: \forall n\in\mathbb{Z}\}$.
$ f(L_1)=L_1$

Then, if $ L_0\neq L_1$, WLOG say $ L_0<L_1$, Let $ L\in(L_0,L_1)$
The line $ g(x)=2x-L$ is below the curve $ f(x)$ at point $ x=L_0$ and is above the curve $ f(x)$ at point $ x=L_1$.
So, using continuity, it exist a value $ a\in(L_0,L_1)$ such that $ f(x)=g(x)$ and so $ f(a) = 2a - L$. Applying then the demo of point 3 above starting with $ a$ and $ f(a)$, we conclude $ f(L) = L$, and so $ f(x) = x$ $ \forall x\in(L_0,L_1)$ which is in contradiction with $ f(x) = 2x - L_0$ for any $ x\in\{(2x_0 - f(x_0)) + (f(x_0) - x_0)2^n\: \: \forall n\in\mathbb{Z}\}$.
So $ L_0=L_1$ and :

$ f(x_0) > x_0$  implies :
$ f(x)\leq x$ $ \forall x\leq L_0$
$ f(x) = 2x - L_0$ $ \forall x\geq L_0$

And the reminder of the demo.

Is it a bit more clear ?.
\end{mysolution}



\begin{mysolution}[by \href{https://artofproblemsolving.com/community/user/22709}{tanpham90}]
	Now I understand ! Thank you Patrick   
\end{mysolution}
*******************************************************************************
-------------------------------------------------------------------------------

\begin{problem}[Posted by \href{https://artofproblemsolving.com/community/user/34189}{tdl}]
	Find all function $ f: (0,1)\rightarrow \mathbb R$ so that
\[ f(xy)=xf(x)+yf(y), \quad \forall x,y\in(0,1).\]
	\flushright \href{https://artofproblemsolving.com/community/c6h174248}{(Link to AoPS)}
\end{problem}



\begin{mysolution}[by \href{https://artofproblemsolving.com/community/user/32514}{TTsphn}]
	Good problem .
$ f(x)\equiv 0$
\end{mysolution}



\begin{mysolution}[by \href{https://artofproblemsolving.com/community/user/29428}{pco}]
	\begin{tcolorbox}Find all function $ f: (0;1)\rightarrow R$ so that:
$ f(xy) = xf(x) + yf(y)\forall x,y\in(0;1)$\end{tcolorbox}

$ f(xy^2)=f(x(y^2))=xf(x)+y^2f(y^2)=xf(x)+2y^3f(y)$
$ f(xy^2)=f((xy)y)=xyf(xy)+yf(y)$ $ =x^2yf(x)+xy^2f(y)+yf(y)=x^2yf(x)+y(xy+1)f(y)$

And so $ xf(x)+2y^3f(y)=x^2yf(x)+y(xy+1)f(y)$
And so $ y(2y^2-xy-1)f(y)=x(xy-1)f(x)$

Taking then $ y=\frac{x+\sqrt{x^2+8}}{4}$, we have $ y\in(0,1)$ and $ 2y^2-xy-1=0$ and $ xy-1\neq 0$ (since $ x\in(0,1)$ and $ y\in(0,1)$) and so $ f(x)=0$ $ \forall x$
\end{mysolution}



\begin{mysolution}[by \href{https://artofproblemsolving.com/community/user/34189}{tdl}]
	Another solution:
$ f(x^2)=2xf(x),f(x^4)=2x^2f(x^2)=4x^3f(x)$
$ f(x^4)=xf(x)+x^3f(x^3)=xf(x)+x^3(xf(x)+x^2f(x^2))=(2x^6+x^4+x)f(x)$
Then $ f(x)g(x)=0\forall x\in(0;1)$ with $ g(x)=2x^6+x^4-4x^3+x$
It mean $ f(x)=0$ with all $ x$ isn't root of $ g(x)$
If exit $ f(x_0)\neq 0$, it mean $ g(x_0)=0$ then $ f(x_0^2)=2x_0f(x_0)\neq 0$, hence $ g(x_0^2)=0$
Similarly we have $ g(x)$ have infinity number of root, contradiction!
\end{mysolution}
*******************************************************************************
-------------------------------------------------------------------------------

\begin{problem}[Posted by \href{https://artofproblemsolving.com/community/user/27967}{FOURRIER}]
	Let $f: \mathbb R \to \mathbb R$ be a function such that
\[f(x^2 + x + 3) + 2f(x^2 - 3x + 5) = 6x^2 - 10x + 17\]
holds for all real $x$. Find $ f(85)$.
	\flushright \href{https://artofproblemsolving.com/community/c6h175973}{(Link to AoPS)}
\end{problem}



\begin{mysolution}[by \href{https://artofproblemsolving.com/community/user/16261}{Rust}]
	$ f(x)=2x-3$.
\end{mysolution}



\begin{mysolution}[by \href{https://artofproblemsolving.com/community/user/32514}{TTsphn}]
	\begin{tcolorbox}$ f(x) = 2x - 3$.\end{tcolorbox}
Are you sure?Note that $ f(x)$ is not a polynomial.
Let $ g(x)=f(x)-2x+\frac{9}{2}$
Then 
$ g(x^2+x+3)+2g(x^2-3x+5)=0$
From this equation we can file some value of $ g(x)$ but can not find $ g(x)$ on R because $ h(x)=x^2+x+3$ is not inject in R.
\end{mysolution}



\begin{mysolution}[by \href{https://artofproblemsolving.com/community/user/16261}{Rust}]
	Let $ t(x) = f(x^2 - 3x + 5) - 2(x^2 - 3x + 5) + 3$, then $ f(x^2 + x + 3) - 2(x^2 + x + 3) + 3 = t(x + 2)$ and $ t(x + 2) + 2t(x) = 0,t(3 - x) = t(x)$.
Therefore $ f(n + 0.5) \equiv 0 \ \forall n\in Z$.
For prove $ t(x) \equiv 0$ we need continiosly f.
\end{mysolution}



\begin{mysolution}[by \href{https://artofproblemsolving.com/community/user/29428}{pco}]
	\begin{tcolorbox}Let $ t(x) = f(x^2 - 3x + 5) - 2(x^2 - 3x + 5) + 3$, then $ f(x^2 + x + 3) - 2(x^2 + x + 3) + 3 = t(x + 2)$ and $ t(x + 2) + 2t(x) = 0,t(3 - x) = t(x)$.
Therefore $ f(n + 0.5) \equiv 0 \ \forall n\in Z$.
For prove $ t(x) \equiv 0$ we need continiosly f.\end{tcolorbox}

The job is nearly finished :

You have  $ t(3-x)=t(x)$ and so $ t(x+2)=t(1-x)$
So $ t(x+2)+2t(x)=0$ becomes $ t(1-x)=-2t(x)$ and so $ t(x)=t(1-(1-x))=-2t(1-x)=4t(x)$ and so $ t(x)=0$ $ \forall x$

And so  $ f(x^2 - 3x + 5) = 2(x^2 - 3x + 5) - 3$
And so $ f(85)=167$
\end{mysolution}



\begin{mysolution}[by \href{https://artofproblemsolving.com/community/user/32514}{TTsphn}]
	$ f(x)$ is not equalities $ 2x-3$ for all x.
Rust have a mistake with his solution.
$ x^2-3x+5$ not take all value on R.
We can file out an solution of this equation as follow :
$ f(x)=c,\forall x\in (-\infty,0)$ and $ f(x)=0$ on other interval.
\end{mysolution}



\begin{mysolution}[by \href{https://artofproblemsolving.com/community/user/29428}{pco}]
	\begin{tcolorbox}$ f(x)$ is not equalities $ 2x - 3$ for all x.
Rust have a mistake with his solution.
$ x^2 - 3x + 5$ not take all value on R.
We can file out an solution of this equation as follow :
$ f(x) = c,\forall x\in ( - \infty,0)$ and $ f(x) = 0$ on other interval.\end{tcolorbox}

Rust gave a second more precise (and, according to me, correct) demo :

Let $ t(x) = f(x^2 - 3x + 5) - 2(x^2 - 3x + 5) + 3$
Then $ t(x + 2) = f(x^2 + x + 3) - 2(x^2 + x + 3) + 3$

So we have  $ f(x^2 - 3x + 5) = t(x) + 2(x^2 - 3x + 5) - 3$  and $ f(x^2 + x + 3) = t(x + 2) + 2(x^2 + x + 3) - 3$

And so $ 6x^2 - 10x + 17 = f(x^2 + x + 3) + 2f(x^2 - 3x + 5)$ $ = t(x + 2) + 2(x^2 + x + 3) - 3 + 2(t(x) + 2(x^2 - 3x + 5) - 3)$ $ = t(x + 2) + 2t(x) + 6x^2 - 10x + 17$

And so $ t(x + 2) = - 2t(x)$
We also have $ t(3 - x) = f((3 - x)^2 - 3(3 - x) + 5) - 2((3 - x)^2 - 3(3 - x) + 5) + 3$ $ = f(x^2 - 3x + 5) - 2(x^2 - 3x + 5) + 3$ $ = t(x)$

So $ t(3 - x) = t(x)$ and so $ t(x + 2) = t(1 - x)$
So $ t(x + 2) = - 2t(x)$ becomes $ t(1 - x) = - 2t(x)$ and so $ t(x) = t(1 - (1 - x)) = - 2t(1 - x) = 4t(x)$ and so $ t(x) = 0$ $ \forall x$

And so  $ f(x^2 - 3x + 5) = 2(x^2 - 3x + 5) - 3$
And so $ f(x) = 2x - 3$ $ \forall x\in[\frac {11}{4}, + \infty)$

And so $ f(85) = 167$
\end{mysolution}



\begin{mysolution}[by \href{https://artofproblemsolving.com/community/user/27967}{FOURRIER}]
	Why don't we do it like this :

Lets look for a simple function $ f$ wich satisfies the conditions,

Let $ f(x) = ax + b$

We can find easily $ f(x) = 2x - 3$
 
and we deduce

Is this method true?

I think this is Rust's Method
\end{mysolution}



\begin{mysolution}[by \href{https://artofproblemsolving.com/community/user/16000}{tchebytchev}]
	you can find find $ f(85$) using this method , but in general cases it isn't true , for example .
let$ f$be function defined from $ R$ to $ R$ such that $ (f(x))^2= f(2x)$. find $ f(0)$.
here there is two value $ 0$ and $ 1$ if $ f(x)=0$ for every $ x$ then $ f(0)=0$ but if $ f(x)=e^x$ then $ f(0)=1$.
\end{mysolution}



\begin{mysolution}[by \href{https://artofproblemsolving.com/community/user/29428}{pco}]
	\begin{tcolorbox}Why don't we do it like this :

Lets look for a simple function $ f$ wich satisfies the conditions,

Let $ f(x) = ax + b$

We can find easily $ f(x) = 2x - 3$
 
and we deduce

Is this method true?

I think this is Rust's Method\end{tcolorbox}

In fact, just by saying $ f(x)=2x-3$ is a solution, you can find \begin{bolded}\begin{underlined}a\end{underlined}\end{bolded} value for $ f(85)$.
The problem is that maybe different solutions exist leading to different values of $ f(85)$

So :
Either you consider that the problem says in an implicit way that there is a unique solution for $ f(85)$ and then the solution is trivial and you are right.
Either you consider that the problem does not say such a thing and you must show that there is a unique value $ 167$

And RUST used the first methoid in its first post
and nearly the second in its second post.
\end{mysolution}



\begin{mysolution}[by \href{https://artofproblemsolving.com/community/user/27967}{FOURRIER}]
	\begin{tcolorbox}Let $ t(x) = f(x^2 - 3x + 5) - 2(x^2 - 3x + 5) + 3$\end{tcolorbox}

\begin{tcolorbox}
Let $ t(x) = f(x^2 - 3x + 5) - 2(x^2 - 3x + 5) + 3$
\end{tcolorbox}

How do you think of this?

Thanks  :)
\end{mysolution}



\begin{mysolution}[by \href{https://artofproblemsolving.com/community/user/16261}{Rust}]
	Let $ g(x)=x^2-3x+5=g(3-x)$, then $ x^2+x+3=g(x+2)$, therefore $ t(x)=f(g(x))-2g(x)+3$ satisfyed
$ t(x+2)+2t(x)=0,t(3-x)=t(x).$
\end{mysolution}



\begin{mysolution}[by \href{https://artofproblemsolving.com/community/user/29428}{pco}]


What do you mean ?

The phrase before, in my post (and that you carefully omitted) was :
"Rust gave a second more precise (and, according to me, correct) demo : "

I was just explaining completely Rust's demo (and my own end of Rust's demo)
So I used Rusts's phrases.

Is there any problem with this ?

And have you read all the posts ?
\end{mysolution}



\begin{mysolution}[by \href{https://artofproblemsolving.com/community/user/27967}{FOURRIER}]
	\begin{tcolorbox}
What do you mean ?
\end{tcolorbox}

I mean where the Idea of considering $ t(x) = f(x^2 - 3x + 5) - 2(x^2 - 3x + 5) + 3$ came from , and Rust answered me  but if you have somthn to add thank you  
\end{mysolution}
*******************************************************************************
-------------------------------------------------------------------------------

\begin{problem}[Posted by \href{https://artofproblemsolving.com/community/user/1147}{stergiu}]
	Let $f: \mathbb R \to \mathbb R$ be a function with the property that
\[f(f(x)-f(y)) = f(f(x)) - y\]
holds true for all reals $x$ and $y$. Prove that $ f$ is an odd function.
	\flushright \href{https://artofproblemsolving.com/community/c6h176051}{(Link to AoPS)}
\end{problem}



\begin{mysolution}[by \href{https://artofproblemsolving.com/community/user/32514}{TTsphn}]
	Easy to check that $ f(x)$ is inject.
Suppose exist $ a,b$ satisfy $ f(a)=f(b)$
Let $ y=a,y=b$ then 
$ f(f(x)-f(a))=f(x)-a$
$ f(f(x)-f(b))=f(x)-b$ s
but $ f(a)=f(b)$ then $ a=b$
Let $ y=0$ then
$ f(f(x)-f(0))=f(f(x))$ so $ f(0)=0$
Let $ x=y$ then
$ f(f(x))=x$
Let $ x=0$ then
$ f(-f(y))=-y$
Let $ y=f(y)$ then
$ f(-f(f(y))=-f(y)$ 
But from $ f(f(y))=y$ we have 
$ f(-y)=-f(y)$
It mean that $ f(x)$ is a odd function.
\end{mysolution}



\begin{mysolution}[by \href{https://artofproblemsolving.com/community/user/29428}{pco}]
	\begin{tcolorbox}If function $ f$ has the propeprty 

  $ f(f(x) - f(y)) = f(f(x)) - y$ 

for every reals $ x , y$ , prove that $ f$ is an odd function.\end{tcolorbox}

Let $ P(x,y)$ be the property $ f(f(x) - f(y)) = f(f(x)) - y$

$ P(x,x)$ implies $ f(f(x))=x+f(0)$ and then $ f(x)$ is bijective.
$ P(0,0)$ implies $ f(f(0))=f(0)$ and, since $ f(x)$ is bijective, $ f(0)=0$

So we have $ f(f(x))=x$ and $ P(0,f(x))$ gives $ f(-x)=-f(x)$

Q.E.D.
\end{mysolution}



\begin{mysolution}[by \href{https://artofproblemsolving.com/community/user/10088}{silouan}]
	Could we find all the functions 
 $ f$ with propeprty 

  $ f(f(x) - f(y)) = f(f(x)) - y$    :)  :?:
\end{mysolution}



\begin{mysolution}[by \href{https://artofproblemsolving.com/community/user/32514}{TTsphn}]
	Let $ x\to f(x),y\to f(y)$ then
$ f(x-y)=f(x)-f(y)$ so $ f(x+y)=f(x)+f(y)$ and more : 
$ f(f(x))=x$
But with conditon we can find out it.
\end{mysolution}



\begin{mysolution}[by \href{https://artofproblemsolving.com/community/user/29428}{pco}]
	\begin{tcolorbox}Could we find all the functions 
 $ f$ with propeprty 

  $ f(f(x) - f(y)) = f(f(x)) - y$    :)  :?:\end{tcolorbox}

We have $ f(f(x))=x$ and $ f(-x)=-f(x)$ so $ f(f(x)+f(y))=x+y$ and so $ f(x+y)=f(x)+f(y)$

So the solutions are :

Let $ \mathbb{A}$ and $ \mathbb{B}$ two supplementary $ \mathbb{Q}$-vectorspaces of $ \mathbb{R}$ ($ \mathbb{A}\cap\mathbb{B}=\{0\}$ and $ \mathbb{A}+\mathbb{B}=\mathbb{R}$

Let $ a(x)$ and $ b(x)$ the projections of $ x$ on $ \mathbb{A}$ and $ \mathbb{B}$  (We have in a unique manner $ x=a(x)+b(x)$)

Then $ f(x)=a(x)-b(x)$ 

Note:
Without Axiom of Choice, the two only possibilities for $ \mathbb{A}$ and $ \mathbb{B}$ are ( $ \mathbb{A}=\mathbb{R}$ and$ \mathbb{A}=\{0\}$) or ($ \mathbb{A}=\{0\}$ and $ \mathbb{B}=\mathbb{R}$)

which give the two solutions $ f(x)=x$ and $ f(x)=-x$

With Axiom of Choice, we have infinitly many others.
\end{mysolution}
*******************************************************************************
-------------------------------------------------------------------------------

\begin{problem}[Posted by \href{https://artofproblemsolving.com/community/user/5820}{N.T.TUAN}]
	Find all functions $ f: \mathbb{R}\to\mathbb{R}$ such that $ f(2xy+y)=2f(xy)+f(y)\;\;\forall x,y\in\mathbb{R}.$
	\flushright \href{https://artofproblemsolving.com/community/c6h177135}{(Link to AoPS)}
\end{problem}



\begin{mysolution}[by \href{https://artofproblemsolving.com/community/user/32514}{TTsphn}]
	It follow that 
$ f(x+y)=f(x)+f(y)$ it is the Cauchy's  function.
\end{mysolution}



\begin{mysolution}[by \href{https://artofproblemsolving.com/community/user/29428}{pco}]
	\begin{tcolorbox}Find all functions $ f: \mathbb{R}\to\mathbb{R}$ such that $ f(2xy + y) = 2f(xy) + f(y)\;\;\forall x,y\in\mathbb{R}.$\end{tcolorbox}

Let $ P(x,y)$ be the property $ f(2xy + y) = 2f(xy) + f(y)$

$ P(0,0)$ implies $ f(0)=0$

$ P(-1,x)$ implies $ f(-x)=-f(x)$

$ P(-\frac{1}{2},x)$ implies $ 2f(\frac{x}{2})=f(x)$

$ P(\frac{u}{2v},v)$ implies $ f(u+v)=2f(\frac{u}{2})+f(v)=f(u)+f(v)$ $ \forall v\neq 0$. But $ f(u+0)=f(u)+f(0)$

So $ P(x,y)\;\;\forall x,y$ implies $ f(x+y)=f(x)+f(y)\;\;\forall x,y$
But obviously $ f(x+y)=f(x)+f(y)\;\;\forall x,y$ implies $ P(x,y)\;\;\forall x,y$ 

So the set of solution of $ P(x,y)$ is the set of solutions of Cauchy's equation $ f(x+y)=f(x)+f(y)$:
Continuous solutions : $ f(x)=ax$
+ Infinitely many non continuous solutions with AC.
\end{mysolution}
*******************************************************************************
-------------------------------------------------------------------------------

\begin{problem}[Posted by \href{https://artofproblemsolving.com/community/user/17813}{duytungct}]
	Given a prime $p$, find all functions $ f: \mathbb N \to \mathbb Z$ such that $ f(n+p)=f(n)$ and $ f(mn)=f(m)f(n)$ for all $m,n \in \mathbb N$.
	\flushright \href{https://artofproblemsolving.com/community/c6h177142}{(Link to AoPS)}
\end{problem}



\begin{mysolution}[by \href{https://artofproblemsolving.com/community/user/32514}{TTsphn}]
	Because value of $ f(n)\in\{f(1),f(2),...,f(p + 1)\}$ 
Call $ t$ is the value of $ f(n)$ such that : 
${ |f(t)| = \max\{|f(1)|,..,|f(p + 1)|}$
Let $ m = t$ then 
$ f(nt) = f(n)f(t)$ so $ |f(nt)| = |f(n)\parallel{}f(t)|$ 
But from $ |f(t)|$ is maximum of $ |f(n)|$ so 
$ f(n)\in\{0,1\}$
 \begin{underlined}Case 1\end{underlined} Exist $ n_0\in N$ such that $ f(n_0) = 0$ then 
$ f(mn_0) = 0,\forall m\in N$ 
But $ \{m{n_0}\}_{m = 1}^{p}$ is a complete reside mod p . So $ f(n) = 0,\forall n\in N$.
\begin{underlined}Case 2\end{underlined} $ f(n)\equiv 1$
So has two function satisfy condition 
$ f(n) = 1,f(n) = 0$
\end{mysolution}



\begin{mysolution}[by \href{https://artofproblemsolving.com/community/user/29428}{pco}]
	\begin{tcolorbox}So has two function satisfy condition 
$ f(n) = 1,f(n) = 0$\end{tcolorbox}

I've not checked your demo, but your conclusion seems not completely right.

Here is at least a third solution :
If $ n=0\pmod{p}$, $ f(n)=0$
Else $ f(n)=1$

For $ p=3$, here is a fourth solution :
If $ n=0\pmod{3}$, $ f(n)=0$
If $ n=1\pmod{3}$, $ f(n)=1$
If $ n=2\pmod{3}$, $ f(n)=-1$
\end{mysolution}



\begin{mysolution}[by \href{https://artofproblemsolving.com/community/user/32514}{TTsphn}]
	Oh sorry, i have a mistake : 
Must be $ |f(n)|=1$ and contine as above.
\end{mysolution}



\begin{mysolution}[by \href{https://artofproblemsolving.com/community/user/29428}{pco}]
	\begin{tcolorbox}Oh sorry, i have a mistake : 
Must be $ |f(n)| = 1$ and contine as above.\end{tcolorbox}

Continue up to what conclusion ?

The fact that $ f(n)\in\{-1,0,1\}$ is immediate :
$ f(n+p)=f(n)$ implies $ f(n)$ takes at most $ p$ values
$ f(n^k)=(f(n))^k$ implies then $ f(n)\in\{-1,0,1\}$, else $ f(n)$ would take infinitely many values.

The question is to find the general solution :

For $ p=2$, we have exactly three solutions : $ (0,0)$, $ (0,1)$ and $ (1,1)$

For $ p=3$, we have exactly four solutions : $ (0,0,0)$, $ (0,1,-1)$, $ (0,1,1)$ and $ (1,1,1)$

For $ p=5$, we have exactly four solutions : $ (0,0,0,0,0)$, $ (0,1,-1,-1,1)$, $ (0,1,1,1,1)$ and $ (1,1,1,1,1)$
\end{mysolution}



\begin{mysolution}[by \href{https://artofproblemsolving.com/community/user/29428}{pco}]
	\begin{tcolorbox}Given $ p \in P$. Find $ f: N \to Z$ satisfy $ f(n + p) = f(n)$ and $ f(mn) = f(m)f(n) \ \forall \ m,n \in N$\end{tcolorbox}

I think that we generally have four solutions :

$ 1)$ $ f(n)=0$
$ 2)$ $ f(n)=1$
$ 3)$ $ f(n)=$Legendre Symbol$ (n,p)$
$ 4)$ $ f(n)=($Legendre Symbol$ (n,p))^2=0$ for any $ n=0\pmod{p}$ and $ 1$ elsewhere.

These four solutions obviously respect the problem. It remains to prove there are no other solution.
\end{mysolution}



\begin{mysolution}[by \href{https://artofproblemsolving.com/community/user/17813}{duytungct}]
	What does Legendre Symbol(n,p) mean?
\end{mysolution}



\begin{mysolution}[by \href{https://artofproblemsolving.com/community/user/32514}{TTsphn}]
	The Lengendre symbol is the symbol of quadric reside mod p .
\end{mysolution}



\begin{mysolution}[by \href{https://artofproblemsolving.com/community/user/32514}{TTsphn}]
	Good idea pco.
Let $ n = kp$ then 
$ f(p(k + 1)) = f(pk)$
$ \Longleftrightarrow f(p)(f(k + 1) - f(k)) = 0$ 
\begin{underlined}Case 1\end{underlined} $ f(p)$ is different from 0 . 
So $ f(k + 1) = f(k),\forall k\in N$ 
It mean that 
$ f(k) = 1$
\begin{underlined}Case 2\end{underlined} $ f(p) = 0$
If $ f(1) = 0$ then $ f(n) = 0,\forall n\in N$
If $ f(1) = - 1$ then let $ n = 1$ then we have $ f(n) = 0$,tradition.
So $ f(1) = 1$
we will prove that : $ f(x^2) = 1,\forall \gcd(x,p) = 1$
from  $ f(x^2) = (f(x))^2\in \{ 1,0\}$ 
If $ f(x^2) = 0$ for some $ x\in N$ then 
$ f(x) = 0$ 
So $ f(mx) = 0$ 
But from $ \{mx\}$ is a complete reside mod p and $ f(1) = 1$ it is tradition.
It mean that $ f(x^2) = 1,\forall x\in N$ so $ f(a) = 0$ when $ (\frac {a}{p}) = 1$
We have two case. 
\begin{bolded}1\end{bolded} $ f(x)\equiv 1$
\begin{bolded}2\end{bolded} Exist $ n_0\in N$ such that $ f(n_0) = - 1$  so $ n_0$ is non quadric reside mod p
Call 
$ a_{1},,.., a_{\frac {p - 1}{2}}$
is the quadric  reside mod p. 
Then $ f(a_i.n_0) = - 1$ 
But $ \{a_i n_0\}$ take all nonquadric reside mod p . 
So $ f(b) = - 1$ when $ (\frac {b}{p}) = - 1$ 
Solution complete.
\end{mysolution}



\begin{mysolution}[by \href{https://artofproblemsolving.com/community/user/17813}{duytungct}]
	Here is my way, thought it is also similar with TTsphn's
Let $ m=n=1$ then $ f(1)=(f(1))^{2}$ , so
$ 1) f(1)=0$ then $ f(n)=0 \forall n$
$ 2) f(1)=1.$ Let $ m=n=0$ then
$ +)f(0)=1 \to f(n)=1 \forall n$
$ +)f(0)=0 \to f(n)=0 \ \forall \ p|n$
this case is solved like TTsphn's
\end{mysolution}



\begin{mysolution}[by \href{https://artofproblemsolving.com/community/user/32514}{TTsphn}]
	I think so ,but careful .This probblem take from a contest .(I don't remember year) but N is not contain 0. 
Although your solution is still true as I find $ f(p)$ 
\end{mysolution}
*******************************************************************************
-------------------------------------------------------------------------------

\begin{problem}[Posted by \href{https://artofproblemsolving.com/community/user/33274}{toanIneq}]
	Find all the functions $f: \mathbb N \to \mathbb N$ satisfying
\[f(f(f(n)))+6f(n)=3f(f(n))+4n+2007\]
for all $n \in \mathbb N$.
	\flushright \href{https://artofproblemsolving.com/community/c6h178149}{(Link to AoPS)}
\end{problem}



\begin{mysolution}[by \href{https://artofproblemsolving.com/community/user/29428}{pco}]
	\begin{tcolorbox}find all the functions $ f: N\rightarrow N$ satisfying
$ f(f(f(n)) + 6f(n) = 3f(f(n)) + 4n + 2007$\end{tcolorbox}

Parenthesis missing. 

Is it : $ f(f(f(n))) + 6f(n) = 3f(f(n)) + 4n + 2007$ ?

Or : $ f(f(f(n)) + 6f(n)) = 3f(f(n)) + 4n + 2007$ ?
\end{mysolution}



\begin{mysolution}[by \href{https://artofproblemsolving.com/community/user/33274}{toanIneq}]
	oh,I don't think so.my problem is true and try $ f(n)=n+669$
\end{mysolution}



\begin{mysolution}[by \href{https://artofproblemsolving.com/community/user/29428}{pco}]
	\begin{tcolorbox}oh,I don't think so.my problem is true and try $ f(n) = n + 669$\end{tcolorbox}

You don't think so what ?

In $ f(f(f(n))+6f(n)$, you have four left parenthesis and only three right parenthesis. So, parenthesis missing.
\end{mysolution}



\begin{mysolution}[by \href{https://artofproblemsolving.com/community/user/33274}{toanIneq}]
	yes, i see and I'm sorry that  for the parenthesis missing in my problem. I edited it
$ f(f(f(n)))+6f(n)=3f(f(n))+4n+2007$
\end{mysolution}



\begin{mysolution}[by \href{https://artofproblemsolving.com/community/user/29428}{pco}]
	\begin{tcolorbox}yes, i see and I'm sorry that  for the parenthesis missing in my problem. I edited it
$ f(f(f(n))) + 6f(n) = 3f(f(n)) + 4n + 2007$\end{tcolorbox}

Let $ f(n)=g(n)+669$ We have $ g(g(g(n))) + 6g(n) = 3g(g(n)) + 4n$

Let the the sequence $ a_k(n)$ defined as :
$ a_0(n)=n$
$ a_{k+1}(n)=g(a_k(n))$

We have $ a_{k+3}(n)=3a_{k+2}(n)-6a_{k+1}(n)+4a_k(n)$

And so $ a_k(n)=(\frac{2n}{3}+\frac{g(g(n))}{3}-2\frac{g(n)-n}{3\sqrt{3}})+$ $ (\frac{n}{3}-\frac{g(g(n))}{3}+2\frac{g(n)-n}{3\sqrt{3}})2^k\cos(k\frac{\pi}{3})+$ $ \frac{g(n)-n}{\sqrt{3}}2^k\sin(k\frac{\pi}{3})$ which may be written :

$ a_k(n)=u(n)+2^k(v(n)\cos(k\frac{\pi}{3})+w(n)\sin(k\frac{\pi}{3}))$
If $ v(n)$ and $ w(n)$ are not zero, $ |a_k(n)|$ grows up to $ +\infty$ and sign changes between $ a_k(n)$ and $ a_{k+3}(n)$

So, since $ a_k(n)$ must $ \in\mathbb{N}$, we must have $ v(n)=w(n)=0$

And so $ g(n)=n$

And so $ f(n)=n+669$, unique solution.
\end{mysolution}
*******************************************************************************
-------------------------------------------------------------------------------

\begin{problem}[Posted by \href{https://artofproblemsolving.com/community/user/15524}{phuong}]
	Finf all funtions $ f: \mathbb{R}\to\mathbb{R}$ such that
\[f(x+y)+f(x)f(y)=f(xy)+f(x)+f(y),\quad \forall x,y\in\mathbb{R}.\]
	\flushright \href{https://artofproblemsolving.com/community/c6h178957}{(Link to AoPS)}
\end{problem}



\begin{mysolution}[by \href{https://artofproblemsolving.com/community/user/27917}{shyamkumar}]
	By observation, f(x) = 0 is one such function.
Now,plug in x=0,y=0…this yields f(0)= 0 or 2.
Then plug in y=0 to give f(x)=2 (if f(0) = 2).
Hence another possible function is f(x)=2 (constant function)…..

Another possibility is f(x+y) = f(x) + f(y) and f(xy)=f(x)*f(y)
This gives (on plugging x=y=0 and later y=0,x=1,2,3…..) f(x) =x

This is all I’ve been able to do…though I feel (intuitively) that there are no more functions, I haven’t been able to prove/disprove that….:oops: 
Shyam
\end{mysolution}



\begin{mysolution}[by \href{https://artofproblemsolving.com/community/user/29428}{pco}]
	\begin{tcolorbox}Finf all funtions $ f: \mathbb{R}\to\mathbb{R}$ such that for arbitrary real numbers $ x$ and $ y$: $ f(x + y) + f(x)f(y) = f(xy) + f(x) + f(y),\forall x,y\in\mathbb{R}$\end{tcolorbox}

$ 1)$ If $ f(x)=c$ constant, we have $ c+c^2=3c$ and we find immediatly two solutions $ f(x)=0$ and $ f(x)=2$

$ 2)$ If $ f(x)$ is not constant :
Let $ P(x,y)$ the property $ f(x + y) + f(x)f(y) = f(xy) + f(x) + f(y)$
$ P(x,0)$ gives $ f(x) + f(x)f(0) = f(0) + f(x) + f(0)$ and so $ f(0)=0$ (since $ f(x)$ is not the constant $ 2$).
$ P(x,1)$ gives $ f(x+1)=(2-f(1))f(x)+f(1)$ and so 
$ P(x+1,1)$ gives $ f(x+2)=(2-f(1))f(x+1)+f(1)$ $ =(2-f(1))^2f(x)+f(1)(3-f(1))$
Since $ f(0)=0$, we can also write this last equation $ f(x+2)=(2-f(1))^2f(x)+f(2)$, with $ f(2)=f(1)(3-f(1))$
But $ P(x,2)$ gives $ f(x+2)=f(2x)+f(x)(1-f(2))+f(2)$

And so $ (2-f(1))^2f(x)+f(2)=f(2x)+f(x)(1-f(2))+f(2)$
implies $ (2-f(1))^2f(x)=f(2x)+f(x)(1-3f(1)+f(1)^2)$
implies $ f(2x)=(3-f(1))f(x)$

So we have $ f(2x)=af(x)$ and $ f(4x)=a^2f(x)$ with $ a=3-f(1)$.

Then $ P(2x,2y)$ gives $ af(x + y) + a^2f(x)f(y) = a^2f(xy) + af(x) + af(y)$.
We also have $ P(x,y)$ : $ f(x + y) + f(x)f(y) = f(xy) + f(x) + f(y)$ which implies $ a^2f(x + y) + a^2f(x)f(y) = a^2f(xy) + a^2f(x) + a^2f(y)$

Subtracting these two equations, we have :
$ a(a-1)f(x+y)=a(a-1)(f(x)+f(y))$

But, we have :
$ a\neq 0$, else $ f(2x)=af(x)$ would imply $ f(x)=0$, constant.
$ a\neq 1$, else $ f(1)=3-a=2$ and, since $ f(x+1)=(2-f(1))f(x)+f(1)$, $ f(x+1)=2$ and $ f(x)=2$, constant.

So we have $ f(x+y)=f(x)+f(y)$ and, as a consequence $ f(xy)=f(x)f(y)$

And this well known system has a unique solution $ f(x)=x$

So we just have three solutions, as Shyamkumar thought :
$ f(x)=0$
$ f(x)=2$
$ f(x)=x$
\end{mysolution}



\begin{mysolution}[by \href{https://artofproblemsolving.com/community/user/30838}{greentreeroad}]
	\begin{tcolorbox}

So we have $ f(x + y) = f(x) + f(y)$ and, as a consequence $ f(xy) = f(x)f(y)$

And this well known system has a unique solution $ f(x) = x$
\end{tcolorbox}

how do you show this, thanks :roll:
\end{mysolution}



\begin{mysolution}[by \href{https://artofproblemsolving.com/community/user/43536}{nguyenvuthanhha}]
	\begin{italicized}Pco's solution is correct and I have just found an other one  \end{italicized}
\end{mysolution}



\begin{mysolution}[by \href{https://artofproblemsolving.com/community/user/29428}{pco}]


$ f(x+y=f(x)+f(y)$ is Cauchy's equation and gives $ f(x)=f(1)x$ $ \forall x\in\mathbb Q$
$ f(xy)=f(x)f(y)$ gives $ f(x^2)=f(x)^2$ and so $ f(x)\geq 0$ $ \forall x\geq 0$

So $ f(x+y)\geq f(x)$ $ \forall x,\forall y\geq 0$ and so $ f(x)$ is monotonous (non decreasing).

And  $ f(x)=f(1)x$ $ \forall x\in\mathbb Q$ PLUS  $ f(x)$  monotonous implies $ f(x)=f(1)x$ $ \forall x\in\mathbb R$

Now $ f(xy)=f(x)f(y)$ $ \implies$ $ f(1)^2=f(1)$ and so two solutions :
$ f(x)=x$
$ f(x)=0$

And, since in paragraph 2 we supposed that $ f(x)$ was not a constant function : $ f(x)=x$
\end{mysolution}
*******************************************************************************
-------------------------------------------------------------------------------

\begin{problem}[Posted by \href{https://artofproblemsolving.com/community/user/34486}{massnet}]
	Find all pairs of functions $ f,g: \mathbb R\rightarrow \mathbb R$ such that

(a) $ f(xg(y+1))+y=xf(y)+f(x+g(y))$ for any $ x,y\in \mathbb R$, and

(b) $ f(0)+g(0)=0$.
	\flushright \href{https://artofproblemsolving.com/community/c6h179588}{(Link to AoPS)}
\end{problem}



\begin{mysolution}[by \href{https://artofproblemsolving.com/community/user/29428}{pco}]
	\begin{tcolorbox}Find all pairs of functions $ f,g: R\rightarrow R$ such that

(a) $ f(xg(y + 1)) + y = xf(y) + f(x + g(y))$ for any $ x,y\in R$

(b) $ f(0) + g(0) = 0$.\end{tcolorbox}

$ 1)$ $ x=0$  implies $ f(g(y))=y+f(0)$. And so $ f(x)$ is surjective and $ g(x)$ is injective.

$ 2)$ If $ g(1)\neq 1$, then, using $ x=\frac{g(0)}{g(1)-1}$ and $ y=0$ in $ a)$, we get: 
$ f\left(\frac{g(0)g(1)}{g(1)-1}\right)=\frac{f(0)g(0)}{g(1)-1}$ $ +f\left(\frac{g(0)g(1)}{g(1)-1}\right)$ and so $ f(0)g(0)=0$ and so $ f(0)=g(0)=0$

Using then $ y=-1$ in $ a)$, we have $ f(x+g(-1))=-f(-1)x-1$ $ \forall x$ and so $ f(x)=ax+b$ for some $ a$ and $ b$

We know that $ a\neq 0$, else $ f(x)$ would not be surjective. And so, since $ f(x)=ax+b$ and $ f(g(x))=x$ (see point $ 1$ above) : $ g(x)=\frac{x-b}{a}$.
Putting these two values in equation $ a)$, we find $ f(x)=g(x)=x$, which is a contradiction in this paragraph since we supposed $ g(1)\neq 1$
So $ \boxed{g(1)=1}$

$ 3)$ Since $ f(x)$ is surjective, exists $ u$ such that $ f(u)=0$
If $ g(u+1)\neq 1$, Putting $ x=\frac{g(u)}{g(u+1)-1}$ and $ y=u$ in $ a)$, we get : 

$ f\left(\frac{g(u)g(u+1)}{g(u+1)-1}\right)$ $ +u=f\left(\frac{g(u)g(u+1)}{g(u+1)-1}\right)$ and so $ u=0$, which is a contradiction since it implies $ 1\neq g(u+1)=g(1)=1$

So $ g(u+1)=1$, then $ g(u+1)=g(1)$ and so $ u=0$ since $ g(x)$ is injective (see point $ 1$ above).

So $ f(0)=0$ and then $ g(0)=0$ and the same demo as above (in point $ 2$) works :

Using then $ y=-1$ in $ a)$, we have $ f(x+g(-1))=-f(-1)x-1$ $ \forall x$ and so $ f(x)=ax+b$ for some $ a$ and $ b$
$ a\neq 0$, else $ f(x)$ would not be surjective. And so, since $ f(x)=ax+b$ and $ f(g(x))=x$ (see $ 1)$ above) : $ g(x)=\frac{x-b}{a}$.
Putting these two values in $ a)$, we find $ \boxed{f(x)=g(x)=x}$, unique solution.
\end{mysolution}
*******************************************************************************
-------------------------------------------------------------------------------

\begin{problem}[Posted by \href{https://artofproblemsolving.com/community/user/32247}{Jure the frEEEk}]
	Let $ f: \mathbb{R}\to\mathbb{R}$. prove that if $ f(xy+y+x)=f(xy)+f(x)+f(y)$, then $ f(x+y)=f(x)+f(y)$.
	\flushright \href{https://artofproblemsolving.com/community/c6h179772}{(Link to AoPS)}
\end{problem}



\begin{mysolution}[by \href{https://artofproblemsolving.com/community/user/29428}{pco}]
	\begin{tcolorbox}Let $ f: \mathbb{R}\to\mathbb{R}$. prove that if $ f(xy + y + x) = f(xy) + f(x) + f(y)$ than $ f(x + y) = f(x) + f(y)$\end{tcolorbox}

A not-so-simple solution :

Let $ P(x,y)$ be the property $ f(xy + y + x) = f(xy) + f(x) + f(y)$.

$ P(0,0)$ implies $ \boxed{f(0)=0}$

$ P(x,-x)$ implies $ \boxed{f(-x)=-f(x)}$ $ \forall x$

Let $ x\neq -1$. Then :
$ P(x,\frac{x}{x+1})$ gives $ f(2x) = f\left(\frac{x^2}{x+1}\right) + f(x) + f\left(\frac{x}{x+1}\right)$
$ P(x,-\frac{x}{x+1})$ gives $ f(0) = -f\left(\frac{x^2}{x+1}\right) + f(x) - f\left(\frac{x}{x+1}\right)$
Adding  these two equalities, we have $ f(2x)=2f(x)$ $ \forall x\neq -1$. So, for $ x=1$, we have $ f(2)=2f(1)$ and so $ -f(2)=-2f(1)$ and so $ f(-2)=2f(-1)$ and so 
$ \boxed{f(2x)=2f(x)}$ $ \forall x$

$ P(-x,-y)$ gives then $ f(xy - y - x) = f(xy) - f(x) - f(y)$. Adding with $ P(x,y)$, we get $ f(xy+x+y)+f(xy-x-y)=2f(xy)$ and, since $ 2f(xy)=f(2xy)$ :
$ \boxed{f(xy+x+y)=f(2xy)+f(x+y-xy)}$ $ \forall x,y$

Let then $ u\leq 0$ and $ v\in \mathbb{R}$. It is always possible to find $ x$ and $ y$ such that $ xy=\frac{u}{2}$ and $ x+y=v+\frac{u}{2}$
Hence, since ${ f(xy+x+y)=f(2xy)+f(x+y-xy)}$, we have $ f(u+v)=f(u)+f(v)$.
Now, if $ u>0$, we have $ -u<0$ and also $ f(-u-v)=f(-u)+f(-v)$ and so $ f(u+v)=f(u)+f(v)$

And so $ \boxed{f(u+v)=f(u)+f(v)}$ $ \forall u,v$
\end{mysolution}



\begin{mysolution}[by \href{https://artofproblemsolving.com/community/user/32247}{Jure the frEEEk}]
	thanks very much. i had problems with proving it for irationals
\end{mysolution}



\begin{mysolution}[by \href{https://artofproblemsolving.com/community/user/19490}{behemont}]
	I've found a nice solution..

Plugging $ y = 1$ to the equation we get
$ f(2x + 1) = 2f(x) + c$, where $ c = f(1)$. Call this property $ (*)$.

Now substituting $ y$ with $ 2y + 1$ in the initial equation we get
$ f(2xy + 2x + 2y + 1) = f(2xy + x) + f(x) + f(2y + 1)$, and now using $ (*)$ after some simplification we get 
$ 2f(xy) + f(x) = f(2xy + x)$, which obviously means $ f(2a + b) = 2f(a) + f(b), \forall a,b$. Call this property $ (**)$.

Now substituting $ x$ with $ 2x + 1$ and $ y$ with $ 2y + 1$ in the initial equation, we get 
$ f(4xy + 4x + 4y + 3) = f(4xy + 2x + 2y + 1) + f(2x + 1) + f(2y + 1)$. Now using $ (*)$ and $ (**)$ this equality leads to $ f(x + y) = f(x) + f(y)$, and that is what we wanted...
\end{mysolution}
*******************************************************************************
-------------------------------------------------------------------------------

\begin{problem}[Posted by \href{https://artofproblemsolving.com/community/user/6551}{perfect\_radio}]
	What's the minimal number $n$ of discontinuities that a real function $f$ which satisfies $f(f(x)) =-x$ can have? 
	\flushright \href{https://artofproblemsolving.com/community/c7h113408}{(Link to AoPS)}
\end{problem}



\begin{mysolution}[by \href{https://artofproblemsolving.com/community/user/18124}{PTynan89}]
	Over all of the reals, or just some interval?

And also, it's pretty obvious that it has to be a bijection.

One more thing, the function $f$ has order $4$, meaning that $f^{4}(x) = e(x) = x$, where $e$ is the identity function.

That's all I have so far, so it's probably not very helpful.
\end{mysolution}



\begin{mysolution}[by \href{https://artofproblemsolving.com/community/user/2975}{jmerry}]
	This only really makes sense as a question about all of $\mathbb{R}$.

It's not just that $f$ has order 4; every orbit except $\{0\}$ has exactly four elements.

Infinitely many discontinuities are required.
Proof:
First, we note that $f(0)=0$. Suppose $f$ has only $n$ discontinuities $x_{1},\dots,x_{n}$ (including zero), and let $S=\{x: x=f^{i}(x_{j})\text{ for some }i,j\}\cup \{0\}$. $S$ is still finite, and contains $4k+1$ elements for some integer $k\le n$. Also, $f(S)=S$ and $f^{-1}(S)=S$.
$\mathbb{R}\setminus S$ is the union of $4k+2$ open intervals, and $f$ is continuous on each of these intervals. Since $f$ maps $\mathbb{R}\setminus S$ to itself bijectively, these intervals must be mapped to each other by $f$. Let $A$ be the set of these intervals; we define $f$ on $A$ in the natural way. Since each element of $A$ is either entirely positive or entirely negative, $f^{2}(U)\neq U$ for each $U\in A$. On the other hand, $f^{4}$ is the identity on $U$, so each orbit in $U$ has exactly four elements. The number of elements in $U$ is not divisible by 4, and we have a contradiction.
\end{mysolution}
*******************************************************************************

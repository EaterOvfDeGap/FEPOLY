-------------------------------------------------------------------------------

\begin{problem}[Posted by \href{https://artofproblemsolving.com/community/user/6551}{perfect_radio}]
	Prove that any polynomial $\in \mathbb{R} [X]$ can be written as a difference of two strictly increasing polynomials.

The problem is not very hard, but it states an interesting fact.
	\flushright \href{https://artofproblemsolving.com/community/c6h66713}{(Link to AoPS)}
\end{problem}



\begin{solution}[by \href{https://artofproblemsolving.com/community/user/26}{grobber}]
	:) Cute.

Let $f$ be our polynomial, which we assume has degree $n$, and let $m$ be an even number larger than $n-1$. The polynomial $x^m-f'$has limit $\infty$ at both $\pm\infty$, so there is a positive constant $c$ such that $g(x)=x^m+c-f'(x)>0,\ \forall x\in\mathbb R$. This means that we have succeeded in writing $f'(x)$ as the difference between two polynomials $x^m+c$ and $g(x)$ which take only positive values. Now $f$ will be the difference between a primitive of $x^m+c$, which is strictly increasing, and a primitive of $g$ which is strictly increasing as well.
\end{solution}



\begin{solution}[by \href{https://artofproblemsolving.com/community/user/6551}{perfect_radio}]
	Congratulations, grobber. One of the most elegant proofs I have ever seen.

This shows me that I'm far from using analysis at its (full) potential.
\end{solution}



\begin{solution}[by \href{https://artofproblemsolving.com/community/user/285}{harazi}]
	Other method: write $f'$ as the half difference between the integrals of $(f')^2+f'+1$ and $(f')^2-f'+1$.
\end{solution}



\begin{solution}[by \href{https://artofproblemsolving.com/community/user/226093}{Taussig}]
	Is there a way to prove this without the use of calculus? My knowledge of it is extremely limited.
\end{solution}



\begin{solution}[by \href{https://artofproblemsolving.com/community/user/29428}{pco}]
	\begin{tcolorbox}Prove that any polynomial $\in \mathbb{R} [X]$ can be written as a difference of two strictly increasing polynomials.\end{tcolorbox}
$P(x)=\left(x+P(0)+\int_0^x\left(\frac{P'(t)+1}2\right)^2dt\right)$ $-\left(x+\int_0^x\left(\frac{P'(t)-1}2\right)^2dt\right)$
\end{solution}

*******************************************************************************
-------------------------------------------------------------------------------

\begin{problem}[Posted by \href{https://artofproblemsolving.com/community/user/25405}{AndrewTom}]
	Find all polynomials

$f(x) = a_{0} + a_{1}x + ... + a_{n}x^{n}$

satisfying the equation 

$f(x^{2}) = (f(x))^{2}$

for all real numbers $x$.
	\flushright \href{https://artofproblemsolving.com/community/c6h571651}{(Link to AoPS)}
\end{problem}



\begin{solution}[by \href{https://artofproblemsolving.com/community/user/31915}{Batominovski}]
	Hint: Any root of $f$, if exists, must be $0$.  

EDIT: This hint assumes that $f$ is not identically zero.
\end{solution}



\begin{solution}[by \href{https://artofproblemsolving.com/community/user/29428}{pco}]
	\begin{tcolorbox}Hint: Any root of $f$, if exists, must be $0$.\end{tcolorbox}
What about $-1, 1, i, -i,$ ... and a lot of others ....

This is not the good hint. Good hint is "identification of two highest degrees coefficients"
\end{solution}



\begin{solution}[by \href{https://artofproblemsolving.com/community/user/31915}{Batominovski}]
	If $z \neq 0$ is a root of $f$, then, for any integer $k \geq 0$, \begin{bolded}every\end{bolded} $\left(2^k\right)$-th root of $z$ will be a root of $f$ as well.  Why is my hint not good?  The only omission in my original comment is that $f$ is assumed to be nonzero.  By the way, my hint works even if $f$ is an entire function satisfying the same functional equation.  So, in a sense, it is better than equating coefficients.
\end{solution}



\begin{solution}[by \href{https://artofproblemsolving.com/community/user/29428}{pco}]
	\begin{tcolorbox}If $z \neq 0$ is a root of $f$, then, for any integer $k \geq 0$, \begin{bolded}every\end{bolded} $\left(2^k\right)$-th root of $z$ will be a root of $f$ as well.  ...\end{tcolorbox}
You're quite right. I considered only $x^{2^n}$ while, as you suggest, $x^{2^{-n}}$ kills it.

And your hint indeed is better than mine.

Sorry :blush:
\end{solution}
*******************************************************************************
-------------------------------------------------------------------------------

\begin{problem}[Posted by \href{https://artofproblemsolving.com/community/user/183939}{panther000}]
	Find all polynomials $P(x)$ such that $P(a)\in Z\Rightarrow a\in Z$
	\flushright \href{https://artofproblemsolving.com/community/c6h578390}{(Link to AoPS)}
\end{problem}



\begin{solution}[by \href{https://artofproblemsolving.com/community/user/183939}{panther000}]
	any solution?
\end{solution}



\begin{solution}[by \href{https://artofproblemsolving.com/community/user/29428}{pco}]
	\begin{tcolorbox}any solution?\end{tcolorbox}
$\boxed{P(x)=c}$ $\forall x$ is a solution, whatever is $c\in\mathbb Z$

If $P(x)$ is not constant, choosing $x$ great enough, we get that $P(x)$ is monotonous and tends toward $\infty$, WLOG $+\infty$

Choosing $n$ great enough, and $u<v$ such that $f(u)=n$ and $f(v)=n+1$, we get $\frac{f(v)-f(u)}{v-u}\le 1$ which is impossible if degree is $\ge 2$

So $f(x)=ax+b$ for some $a,b$ and we easily get $\boxed{P(x)=\frac{x+p}q}$ whatever are $p\in\mathbb Z$, $q\in\mathbb Z\setminus\{0\}$
\end{solution}
*******************************************************************************
-------------------------------------------------------------------------------

\begin{problem}[Posted by \href{https://artofproblemsolving.com/community/user/125513}{hal9v4ik}]
	Is it true that for  polynomials $P,Q,R$:
a) if $P$ and $Q$ are polynomials(with leading coefficint 1) of degree $n$ with n real roots each,then for $P+Q=2*R$, $R$ have $n$ real roots.
b)if $P$ and $Q$ don't have real roots then $R$ also don't have real root.
	\flushright \href{https://artofproblemsolving.com/community/c6h578944}{(Link to AoPS)}
\end{problem}



\begin{solution}[by \href{https://artofproblemsolving.com/community/user/29428}{pco}]
	\begin{tcolorbox}Is it true that for  polynomials $P,Q,R$:
a) if $P$ and $Q$ are polynomials(with leading coefficint 1) of degree $n$ with n real roots each,then for $P+Q=2*R$, $R$ have $n$ real roots.\end{tcolorbox}
Wrong. Choose as counter-example :
$P(x)=x^2-x$
$Q(x)=x^2-5x+6$
$R(x)=x^2-3x+3$
\end{solution}



\begin{solution}[by \href{https://artofproblemsolving.com/community/user/29428}{pco}]
	\begin{tcolorbox}Is it true that for  polynomials $P,Q,R$:
b)if $P$ and $Q$ don't have real roots then $R$ also don't have real root.\end{tcolorbox}
Trivially true if monic property is true also for question b) : $P(x)>0$ $\forall x$ and $Q(x)>0$ $\forall x$ and so $R(x)>0$ $\forall x$
Trivially false if monic property is not true for question b) : $P(x)=x^2+1$ and $Q(x)=-x^2-1$
\end{solution}
*******************************************************************************
-------------------------------------------------------------------------------

\begin{problem}[Posted by \href{https://artofproblemsolving.com/community/user/32404}{epsilon07}]
	$\overline{abc}$ is a three digit number, the roots of the polynomial $ ax^2 + bx +c $  are integers, and these integer roots divide the $3$-digit number $\overline{abc}$. How many different $\overline{abc}$ can you find?
	\flushright \href{https://artofproblemsolving.com/community/c6h581335}{(Link to AoPS)}
\end{problem}



\begin{solution}[by \href{https://artofproblemsolving.com/community/user/29428}{pco}]
	\begin{tcolorbox}$\overline{abc}$ is a three digit number, the roots of the polynomial $ ax^2 + bx +c $  are integers, and these integer roots divide the $3$-digit number $\overline{abc}$. How many different $\overline{abc}$ can you find?\end{tcolorbox}
So quadratic is $px^2+p(m+n)x+pmn$ where $-m,-n$ are the two negative integers root.

Constraints are $p(m+n),pmn\in\{1,2,3,4,5,6,7,8,9\}$ and $m|10p(n+10)$ and $n|10p(m+10)$
WLOG $m\ge n$

First constraint gives very few values $(m,n,p)$ : 
$(9,1,1),(8,1,1),(7,1,1),(6,1,1),(5,1,1)$,
$(4,2,1),(4,1,1)$,
$(3,3,1),(3,2,1),(3,1,1),(3,1,2)$,
$(2,2,1),(2,2,2),(2,1,1)(2,1,2),(2,1,3)$,
$(1,1,1),(1,1,2),(1,1,3),(1,1,4)$

Second constraint reduces this short list to :
$(5,1,1),(4,2,1),(3,2,1),(2,2,1),(1,1,1),(1,1,2),(1,1,3),(1,1,4),(2,1,1),(2,1,2),(2,2,2),(2,1,3)$

And so $\boxed{\overline{abc}\in\{165,168,156,144,121,242,363,484,132,264,288,396\}}$
\end{solution}
*******************************************************************************
-------------------------------------------------------------------------------

\begin{problem}[Posted by \href{https://artofproblemsolving.com/community/user/211457}{rod16}]
	$a,b,c$ are real numbers with $ac<0$ and $\sqrt{2}a+\sqrt{3}b+\sqrt{5}c=0$. Prove that the second degree equation $ax^2+bx+c=0$ has roots in the interval of $\left(\frac{3}{4}, 1 \right)$
	\flushright \href{https://artofproblemsolving.com/community/c6h585464}{(Link to AoPS)}
\end{problem}



\begin{solution}[by \href{https://artofproblemsolving.com/community/user/29428}{pco}]
	\begin{tcolorbox}$a,b,c$ are real numbers with $ac<0$ and $\sqrt{2}a+\sqrt{3}b+\sqrt{5}c=0$. Prove that the second degree equation $ax^2+bx+c=0$ has roots in the interval of $\left(\frac{3}{4}, 1 \right)$\end{tcolorbox}
$a\ne 0$ and, setting $u=-\frac ca>0$, equation is $f(x)=x^2+\frac{u\sqrt 5-\sqrt 2}{\sqrt 3}x-u=0$

$f(1)=u\frac{\sqrt 5-\sqrt 3}{\sqrt 3}+\frac{\sqrt 3-\sqrt 2}{\sqrt 3}>0$

$f(\frac 34)=u\frac{3\sqrt 5-4\sqrt 3}{4\sqrt 3}+3\frac{3\sqrt 3-4\sqrt 2}{16\sqrt 3} <0$

Hence the result.
\end{solution}
*******************************************************************************
-------------------------------------------------------------------------------

\begin{problem}[Posted by \href{https://artofproblemsolving.com/community/user/202190}{Blitzkrieg97}]
	find all $ P(x) $ polynoms with real coefficients,such that,for every $x$, $ (x+1)P(x-1)-(x-1)P(x) $ is const.
	\flushright \href{https://artofproblemsolving.com/community/c6h586443}{(Link to AoPS)}
\end{problem}



\begin{solution}[by \href{https://artofproblemsolving.com/community/user/29428}{pco}]
	\begin{tcolorbox}find all $ P(x) $ polynoms with real coefficients,such that,for every $x$, $ (x+1)P(x-1)-(x-1)P(x) $ is const.\end{tcolorbox}

1) If $P(x)$ has no summands, then $P(x)=0$ $\forall x$ is a solution.

2) If $P(x)$ has just one summand, then $P(x)=a_nx^n$ and we get $n=0$ and so  $P(x)=c$ $\forall x$, which indeed is a solution, whatever is $c\in\mathbb R$

3) If $P(x)$ has at least two summands, the two highest degrees being $a_nx^n+a_px^p$, identification of highest degrees summands in equation implies $n=2$ and $p=1$ and $a_n=a_p$ 

Hence the answer : $\boxed{P(x)=ax^2+ax+c\text{   }\forall x}$ which indeed is a solution, whatever are $a,c\in\mathbb R$
\end{solution}
*******************************************************************************
-------------------------------------------------------------------------------

\begin{problem}[Posted by \href{https://artofproblemsolving.com/community/user/180735}{mmaht}]
	prove that all the roots of the following polynomial can't be real:\[p(x)=x^n+2nx^{n-1}+2n^2x^{n-2}+a_{n-3}x^{n-3}+...+a_{0}\]
	\flushright \href{https://artofproblemsolving.com/community/c6h586453}{(Link to AoPS)}
\end{problem}



\begin{solution}[by \href{https://artofproblemsolving.com/community/user/29428}{pco}]
	\begin{tcolorbox}prove that all the roots of the following polynomial can't be real:\[p(x)=x^n+2nx^{n-1}+2n^2x^{n-2}+a_{n-3}x^{n-3}+...+a_{0}\]\end{tcolorbox}
I suppose we need $n\ge 3$ in order to have the definition consistant.
Take the $(n-2)^{th}$ derivative and you get $n!(\frac {x^2}2+2x+\frac {2n}{n-1})$ whose discriminant is negative.

So $(n-3)^{th}$ derivative has at most one real root
So $(n-4)^{th}$ derivative has at most two real roots
...
So first derivative has at most $n-3$ real roots
So $P(x)$ has at most $n-2$ real roots.

Hence the claim
\end{solution}
*******************************************************************************
-------------------------------------------------------------------------------

\begin{problem}[Posted by \href{https://artofproblemsolving.com/community/user/180735}{mmaht}]
	find all polynomials \[(p_{1}(x),p_{2}(x),p_{3}(x),p_{4}(x))\] s.t for every natural numbers x,y,z,t with xy-zt=1 we have  \[p_{1}(x)p_{2}(x)-p_{3}(x)p_{4}(x)=1\]
	\flushright \href{https://artofproblemsolving.com/community/c6h586461}{(Link to AoPS)}
\end{problem}



\begin{solution}[by \href{https://artofproblemsolving.com/community/user/199091}{maryam_m}]
	\begin{tcolorbox}find all polynomials \[(p_{1}(x),p_{2}(x),p_{3}(x),p_{4}(x))\] s.t for every natural numbers x,y,z,t with xy-zt=1 we have  \[p_{1}(x)p_{2}(x)-p_{3}(x)p_{4}(x)=1\]\end{tcolorbox}

youre sure the problem's true?

isnt it \[p_{1}(x)p_{2}(y)-p_{3}(z)p_{4}(t)=1\]
\end{solution}



\begin{solution}[by \href{https://artofproblemsolving.com/community/user/180735}{mmaht}]
	yes you are right
\end{solution}



\begin{solution}[by \href{https://artofproblemsolving.com/community/user/29428}{pco}]
	\begin{tcolorbox}find all polynomials \[(p_{1}(x),p_{2}(x),p_{3}(x),p_{4}(x))\] s.t for every natural numbers x,y,z,t with xy-zt=1 we have  \[p_{1}(x)p_{2}(y)-p_{3}(z)p_{4}(t)=1\]\end{tcolorbox}
Let $P(x,y,z,t)$ be the asserton $p_1(x)p_2(y)-p_3(z)p_4(t)=1$, true $\forall x,y,z,t\in\mathbb N$ such that $xy-zt=1$

If one of the four polynomials is the zero polynomial, WLOG $p_1(x)=0$ $\forall x$, then $P(xy+1,1,x,y)$ $\implies$ $p_3(x)p_4(y)=-1$ $\forall x,y\in\mathbb N$
Hence the solutions
$\boxed{\text{S1 : }(K(x),0,u,v)}$ where $K(x)$ is any polynomial$\in\mathbb R[X]$ and $u,v$ any real numbers such that $uv=-1$
$\boxed{\text{S2 : }(0,K(x),u,v)}$ where $K(x)$ is any polynomial$\in\mathbb R[X]$ and $u,v$ any real numbers such that $uv=-1$
$\boxed{\text{S3 : }(u,v,K(x),0)}$ where $K(x)$ is any polynomial$\in\mathbb R[X]$ and $u,v$ any real numbers such that $uv=1$
$\boxed{\text{S4 : }(u,v,0,K(x))}$ where $K(x)$ is any polynomial$\in\mathbb R[X]$ and $u,v$ any real numbers such that $uv=1$

If no polynomial is the zero polynomial, then let $m,n>1$ :
Comparing $P(m,n,mn-1,1)$ with $P(n,m,mn-1,1)$, we get $p_1(m)p_2(n)=p_1(n)p_2(m)$ and so $p_1,p_2$ are the same (with a multiplicative coefficient)
Comparing $P(mn+1,1,m,n)$ with $P(mn+1,1,n,m)$, we get $p_3(m)p_4(n)=p_3(n)p_4(m)$ and so $p_3,p_4$ are the same (with a multiplicative coefficient)
So $P(x,y,z,t)$ may be written $\alpha p(x)p(y)-\beta q(z)q(t)=1$ with $\alpha\beta\ne 0$

If $p(x)$ is a nonzero constant, then $P(mn+1,1,m,n)$ implies $q(m)q(n)$ is constant and so $q(x)$ is constant
If $q(x)$ is a nonzero constant, then $P(m,n,mn-1,1)$ implies $p(m)p(n)$ is constant and so $p(x)$ is constant
And so the solution $\boxed{\text{S5 : }(u,v,w,t)}$ where  $u,v,w,t$ are any real numbers such that $uv-wt=1$

If $p(x),q(x)$ are nonconstant polynomials and $\alpha\beta\ne 0$ :

$P(n+1,1,n,1)$ $\implies$ $\alpha p(n+1)p(1)-\beta q(n)q(1)=1$ and so $q(n)=a + b p(n+1)$
But, for symetry reasons, we also have $p(n)=a'+b'q(n+1)$ and so $p(n+2)=r+sp(n)$ and so $p(n),q(n)$ both have degree 1.

Then $P(m,n,mn-1,1)$ $\implies$ $p(mn)=u p(m)p(n)+v$ and so $p(x)=c x$ and, same $q(x)=d x$
And it's then easy to get the last solution :
$\boxed{\text{S6 : }(ux,vx,wx,tx)}$ where  $u,v,w,t$ are any real numbers such that $uv=wt=1$
\end{solution}



\begin{solution}[by \href{https://artofproblemsolving.com/community/user/199091}{maryam_m}]
	tnx pco ! nothin better n more complete can be written!
\end{solution}
*******************************************************************************
-------------------------------------------------------------------------------

\begin{problem}[Posted by \href{https://artofproblemsolving.com/community/user/194411}{EdsonBR}]
	let $x_1<x_2<x_3 $ be the three real ratios of the equation $x^3-3x-1=0 $ . Prove that $x_3^2 - x_2^2=x_3-x_1$
	\flushright \href{https://artofproblemsolving.com/community/c6h587812}{(Link to AoPS)}
\end{problem}



\begin{solution}[by \href{https://artofproblemsolving.com/community/user/29428}{pco}]
	\begin{tcolorbox}let $x_1<x_2<x_3 $ be the three real ratios of the equation $x^3-3x-1=0 $ . Prove that $x_3^2 - x_2^2=x_3-x_1$\end{tcolorbox}
One simple way (not the one your teacher is looking for, I think) is just to solve the equation.

We easily get $x_1=2\cos \frac{7\pi}9$ and $x_2=2\cos \frac{13\pi}9$ and $x_3=2\cos\frac{\pi}9$

And required equation becomes $4\sin\frac{8\pi}9\sin\frac{6\pi}9=$ $4\sin\frac{28\pi}9\sin\frac{24\pi}9$

which is trivially true.
\end{solution}
*******************************************************************************
-------------------------------------------------------------------------------

\begin{problem}[Posted by \href{https://artofproblemsolving.com/community/user/68025}{Pirkuliyev Rovsen}]
	Prove that for any polynomial $P$ there exists a polynomial $Q$ such that $P(x)=Q(x+1)-Q(x)$ for all $x$.
	\flushright \href{https://artofproblemsolving.com/community/c6h589250}{(Link to AoPS)}
\end{problem}



\begin{solution}[by \href{https://artofproblemsolving.com/community/user/31919}{tenniskidperson3}]
	We have that if $P$ is degree $n$, then

$P(x)=\sum_{i=0}^n\left(\binom{x}{i}\cdot\sum_{j=0}^iP(i-j)(-1)^j\binom{i}{j}\right)$,

since it's a degree $n$ polynomial agreeing with $P$ on $0, 1, 2\ldots n$.  So by Pascal's Identity we can take

$Q(x)=C+\sum_{i=0}^n\left(\binom{x}{i+1}\cdot\sum_{j=0}^iP(i-j)(-1)^j\binom{i}{j}\right)$.

The reason $P$ agrees with the sum on $0, 1, 2\ldots n$ is because of a formula known as [url=http://en.wikipedia.org\/wiki\/Binomial_transform]the binomial transform[\/url]: if $s_i=\sum_{j=0}^i\binom{i}{j}a_j$, then $a_i=\sum_{j=0}^i(-1)^{i-j}\binom{i}{j}s_j$.  Here we use $s_i=P(i)$.
\end{solution}



\begin{solution}[by \href{https://artofproblemsolving.com/community/user/29428}{pco}]
	You can also use induction and derivatives :

If $P'(x)=Q(x+1)-Q(x)$, then $P(x)=R(x+1)-R(x)$ with $R(x)=\left(P(0)-\int_{-1}^0Q(t)dt\right)x+\int_0^xQ(t)dt$
\end{solution}
*******************************************************************************
-------------------------------------------------------------------------------

\begin{problem}[Posted by \href{https://artofproblemsolving.com/community/user/167924}{utkarshgupta}]
	Show that if $a$ and $b$ are real, and the polynomial $p(x)=x^4+ax^3+2x^2+bx+1$ has a real solution,
then, $a^2+b^2 \ge 8$
	\flushright \href{https://artofproblemsolving.com/community/c6h590855}{(Link to AoPS)}
\end{problem}



\begin{solution}[by \href{https://artofproblemsolving.com/community/user/29428}{pco}]
	\begin{tcolorbox}Show that if $a$ and $b$ are real, and the polynomial $p(x)=x^4+ax^3+2x^2+bx+1$ has a real solution,
then, $a^2+b^2 \ge 8$\end{tcolorbox}
If $a^2+b^2<8$, then $a^2<8$ and we can write 

$p(x)=x^2\left(x+\frac a2\right)^2+\frac{8-a^2}4\left(x+\frac{2b}{8-a^2}\right)^2+\frac{8-a^2-b^2}{8-a^2}$ $\ge \frac{8-a^2-b^2}{8-a^2}$ $>0$

Hence the claim.
\end{solution}
*******************************************************************************
-------------------------------------------------------------------------------

\begin{problem}[Posted by \href{https://artofproblemsolving.com/community/user/145173}{Aiscrim}]
	For two quadratic trinomials $P(x)$ and $Q(x)$ there is a linear function $\ell(x)$ such that $P(x)=Q(\ell(x))$ for all real $x$. How many such linear functions $\ell(x)$ can exist?

\begin{italicized}(A. Golovanov)\end{italicized}
	\flushright \href{https://artofproblemsolving.com/community/c6h597517}{(Link to AoPS)}
\end{problem}



\begin{solution}[by \href{https://artofproblemsolving.com/community/user/29428}{pco}]
	\begin{tcolorbox}For two quadratic trinomials $P(x)$ and $Q(x)$ there is a linear function $l(x)$ such that $P(x)=Q(l(x))$ for all real $x$. How many such linear functions $l(x)$ can exist?

\begin{italicized}(A. Golovanov)\end{italicized}\end{tcolorbox}
I suppose that "quadratic trinomial" suppose that cofficient of $x^2$ is nonzero. If so :
$P(x)=Q(l_1(x))$ and $P(x)=Q(l_2(x))$ implies $Q(l_1(x))=Q(l_2(x))$ and so :
either $l_1(x)=l_2(x)$
either $l_1(x)+l_2(x)=c$ constant.

Hence at most two such functions, and trivially at least two exist. Hence the answer : $\boxed{2}$
\end{solution}



\begin{solution}[by \href{https://artofproblemsolving.com/community/user/64716}{mavropnevma}]
	A more complicated, but relevant approach.

Clearly a linear function $\ell(x) = ax+b$ (with $a\neq 0$) admit a (also linear) inverse $\ell^{-1}(x) = \dfrac {1} {a} x - \dfrac {b} {a}$. Then, if $P(x)=Q(\ell_1(x)) = Q(\ell_2(x))$, we also have $Q(x) = Q(\lambda(x))$, where $\lambda = \ell_2\circ \ell_1^{-1}$ is linear, thus $\lambda(x) = \alpha x + \beta$ (with $\alpha \neq 0$).

By identification of the dominant coefficients follows $|\alpha|=1$. (Then, for $\deg Q = n\geq 2$, and working with complex coefficients, we may have as many as $n$ such functions $\lambda$, for example $\lambda_k(x) = \textrm{e}^{2k\pi\textrm{i}\/n} x$ for $Q(x) = x^n$). In the sequel, assume real coefficients.

With $\deg Q = n\geq 2$, then for odd $n$ we have $\alpha=1$, leading to $Q(x) = Q(x+\beta)$, thus by simple iteration $Q(x) = Q(x+k\beta)$ for all natural $k$; this clearly forces $\beta = 0$, for which $\lambda = \operatorname{id}$. For even $n$ we may also have $\alpha=-1$, leading to $Q(x) = Q(-x+\beta)$, thus $\lambda \circ \lambda = \operatorname{id}$. But we cannot have $Q(x) = Q(-x+\gamma)$ for $\gamma \neq \beta$, since then $Q(-x+\beta) =  Q(-x+\gamma)$, so $Q(x) =  Q(x+(\beta -\gamma))$, with the obvious contradiction pointed at in the above. It follows in this case may exist at most two such functions $\lambda$, with an immediate example given by $Q$ made only of monomials of even degree, when $\lambda = \pm \operatorname{id}$ work.

In the case at hand, $\deg Q = 2$, which allows the simpler and more direct solution based on the symmetries of the graph of a quadratic trinomial.
\end{solution}
*******************************************************************************
-------------------------------------------------------------------------------

\begin{problem}[Posted by \href{https://artofproblemsolving.com/community/user/218339}{bnmh}]
	Find all non-constant real polynomials $f (x)$ such that for any real $x$ the following equality
holds:
\[f (\sin x + \cos x) = f (\sin x) + f (\cos x)\]
	\flushright \href{https://artofproblemsolving.com/community/c6h598639}{(Link to AoPS)}
\end{problem}



\begin{solution}[by \href{https://artofproblemsolving.com/community/user/213278}{shmm}]
	I think the answer:$f(x)=ax$ for all real $a$
\end{solution}



\begin{solution}[by \href{https://artofproblemsolving.com/community/user/29428}{pco}]
	\begin{tcolorbox}I think the answer:$f(x)=ax$ for all real $a$\end{tcolorbox}
No. $a=0$ is not a solution.
\end{solution}



\begin{solution}[by \href{https://artofproblemsolving.com/community/user/49556}{xxp2000}]
	\begin{tcolorbox}Find all non-constant real polynomials $f (x)$ such that for any real $x$ the following equality
holds:
\[f (\sin x + \cos x) = f (\sin x) + f (\cos x)\]\end{tcolorbox}

Suppose $f(x)=\sum a_nx^n$ with $n\ge2$.
We have $f(x\pm\sqrt{1-x^2})=f(x)+f(\pm\sqrt{1-x^2}),|x|\le1$.
We can write $f(x\pm\sqrt{1-x^2})-f(x)-f(\pm\sqrt{1-x^2})=h(x)\pm g(x)\sqrt{1-x^2}=0,|x|\le1$, where $h,g$ are some polynomials.
Obviously $h(x)=g(x)=0,|x|\le1$. So $h(x)=g(x)=0,\forall x$. 
In other words, $f(x\pm\sqrt{1-x^2})=f(x)+f(\pm\sqrt{1-x^2}),\forall x$.
Divide both sides with $a_nx^n$ and take limit when $x$ goes to infinity,
$(1+i)^n=1+i^n$. 
It is easy to check this equation does not hold for $n\ge2$. So $f$ is linear.

Since $f(1)=f(1)+f(0)$, $f(0)=0$. So the only solution is $f(x)=ax$ where $a\ne0$.
\end{solution}
*******************************************************************************
-------------------------------------------------------------------------------

\begin{problem}[Posted by \href{https://artofproblemsolving.com/community/user/213456}{ambiguity}]
	Determin all polynomials $f$ such that:

$f(xy)-y=f(x)f(y)+x$

I have tried to let $x=y=0$ and $x=y=1$ but I can't make any conclusions about all $f$ that satisfy the equation.
Maybe I should use that $f(xy)=f(x)f(y)$ throuhg some substituion? Any ideas?
	\flushright \href{https://artofproblemsolving.com/community/c6h608180}{(Link to AoPS)}
\end{problem}



\begin{solution}[by \href{https://artofproblemsolving.com/community/user/29428}{pco}]
	\begin{tcolorbox}Determin all polynomials $f$ such that:

$f(xy)-y=f(x)f(y)+x$
\end{tcolorbox}
Let $P(x,y)$ be the assertion $f(xy)-y=f(x)f(y)+x$
Let $a=f(0)$

$P(1,0)$ $\implies$ $a(1-f(1))=1$ and so $a\ne 0$
$P(0,0)$ $\implies$ $a=1$ (since $a\ne 0$)
$P(x,0)$ $\implies$ $f(x)=1-x$ which unfortunately is not a solution.

So no solution.
\end{solution}
*******************************************************************************
-------------------------------------------------------------------------------

\begin{problem}[Posted by \href{https://artofproblemsolving.com/community/user/226116}{icosine}]
	Determine all polynomials $Q$ with real coefficients such that:

$Q(Q(x))+Q(x)=x^9+9x^3+9$
	\flushright \href{https://artofproblemsolving.com/community/c6h608835}{(Link to AoPS)}
\end{problem}



\begin{solution}[by \href{https://artofproblemsolving.com/community/user/213456}{ambiguity}]
	Hint:
If there exists a solution, Q(x) must be of degree 3.
\end{solution}



\begin{solution}[by \href{https://artofproblemsolving.com/community/user/29428}{pco}]
	\begin{tcolorbox}Determine all polynomials $Q$ with real coefficients such that:

$Q(Q(x))+Q(x)=x^9+9x^3+9$\end{tcolorbox}

Degree of $Q(x)$ is $3$ and so $Q(x)=ax^3+bx^2+cx+d$

Identification of $x^9$ summands implies $a^4=1$ and so $a=\pm 1$
Identification of $x^8$ summands implies $3a^3b=0$ and so $b=0$
Identification of $x^7$ summands implies $3a^3c=0$ and so $c=0$
Identification of $x^6$ summands implies $3a^3d=0$ and so $d=0$

So $Q(x)=x^3$ or $Q(x)=-x^3$, none of them being a solution.

So no solution.
\end{solution}
*******************************************************************************
-------------------------------------------------------------------------------

\begin{problem}[Posted by \href{https://artofproblemsolving.com/community/user/125553}{lehungvietbao}]
	Let $f(x)=x^{3} -3x - 1$ for all $x \in \mathbb{R}$. Find the number of roots in  $f(f(x))=0$.
	\flushright \href{https://artofproblemsolving.com/community/c6h609835}{(Link to AoPS)}
\end{problem}



\begin{solution}[by \href{https://artofproblemsolving.com/community/user/3640}{Dr Sonnhard Graubner}]
	\begin{tcolorbox}Let $f(x)=x^{3} -3x - 1$ for all $x \in \mathbb{R}$. Find the number of roots in  $f(f(x))=0$.\end{tcolorbox}
hello, you will get
${x}^{9}-9\,{x}^{7}-3\,{x}^{6}+27\,{x}^{5}+18\,{x}^{4}-27\,{x}^{3}-27\,
{x}^{2}+1
=0$ and the number of roots is nine.
Sonnhard.
\end{solution}



\begin{solution}[by \href{https://artofproblemsolving.com/community/user/64716}{mavropnevma}]
	The \begin{bolded}total\end{bolded} number of roots is clearly $9$, due to the degree. The issue clearly is what is the number of \begin{bolded}real\end{bolded} roots, which incidentally is $7$.
\end{solution}



\begin{solution}[by \href{https://artofproblemsolving.com/community/user/3640}{Dr Sonnhard Graubner}]
	\begin{tcolorbox}The \begin{bolded}total\end{bolded} number of roots is clearly $9$, due to the degree. The issue clearly is what is the number of \begin{bolded}real\end{bolded} roots, which incidentally is $7$.\end{tcolorbox}
hello, the proposer asked really "find the number of roots" i can not read what is in his mind.
Sonnhard.
\end{solution}



\begin{solution}[by \href{https://artofproblemsolving.com/community/user/64716}{mavropnevma}]
	So you hasten to answer a trivial question. Say $P$ is a polynomial of degree $n$. What is the number of roots of $P(P(x)) = 0$? And what is the number of roots of $P(P(P(x))) = 0$? And so on ...
\end{solution}



\begin{solution}[by \href{https://artofproblemsolving.com/community/user/105169}{Nikpour}]
	$f(x)$ mus be a root of $f(x)={{x}^{3}}-3x-1$. First we find the roots of  $f(x)={{x}^{3}}-3x-1$.

\begin{align}
  & {{x}^{3}}-3x-1=0 \\ 
 & x=u+v\Rightarrow {{(u+v)}^{3}}-3(u+v)-1=0 \\ 
 & \Rightarrow {{u}^{3}}+{{v}^{3}}+3uv(u+v)-3(u+v)-1=0 \\ 
 & \Rightarrow {{u}^{3}}+{{v}^{3}}+(3uv-3)(u+v)-1=0 \\ 
 & 3uv-3=0\Rightarrow v=\frac{1}{u}\Rightarrow {{u}^{3}}+\frac{1}{{{u}^{3}}}-1=0 \\ 
 & \Rightarrow {{({{u}^{3}})}^{2}}-{{u}^{3}}+1=0\Rightarrow {{u}^{3}}=\frac{1\pm i\sqrt{3}}{2} \\ 
\end{align}
 
$u$ is any third roots of 
\[\frac{1\pm i\sqrt{3}}{2}\]
.
Let $\alpha $ be a root of  $f(x)$. It coffiest to solve $f(x)=\alpha $.
Any such equation has exatly $3$ roots.
but some of these roots are not real.
\end{solution}



\begin{solution}[by \href{https://artofproblemsolving.com/community/user/64716}{mavropnevma}]
	Compute the values of $f(-2),f(-1),f(0),f(1),f(2)$ to see the real roots of $f$ are localized in $(-2,-1),(-1,0),(1,2)$. On the other hand, since $f'(x)=3(x^2-1)$, $f$ has a local minimum of $-3$ at $1$ and a local maximum of $1$ at $-1$. To have $f(f(x))=0$ we need $f(x)=r$, for some root $r$ of $f$. According with the above, for the smaller two we have three possibilities each, while for the larger one we have one possibility only, for a total of $7$.

The relation (5) of above is gibberish, right? You just select some values $u$ and $v$ in order to express a root of $f$, without any knowledge of its value or position.
\end{solution}



\begin{solution}[by \href{https://artofproblemsolving.com/community/user/29428}{pco}]
	\begin{tcolorbox}Let $f(x)=x^{3} -3x - 1$ for all $x \in \mathbb{R}$. Find the number of \begin{bolded}distinct real\end{underlined}\end{bolded} roots in  $f(f(x))=0$.\end{tcolorbox}
(I modified a bit the problem statement)
$f(x)=0$ has  three real roots $2\cos\frac{\pi}9$, $2\cos\frac{7\pi}9$ and $2\cos\frac{13\pi}9$.

So we need to solve :
$x^3-3x=1+2\cos\frac{\pi}9$ which has one real root, since $RHS>2$
$x^3-3x=1+2\cos\frac{7\pi}9$ which has three real roots, since $RHS\in(-2,+2)$
$x^3-3x=1+2\cos\frac{13\pi}9$ which has three real roots, since $RHS\in(-2,+2)$

Hence the answer : $\boxed{7\text{ distinct real roots}}$
\end{solution}
*******************************************************************************
-------------------------------------------------------------------------------

\begin{problem}[Posted by \href{https://artofproblemsolving.com/community/user/29381}{james digol}]
	Let $f$ be a polynomial function with integer coefficients and $p$ be a prime number. Suppose there are at least four distinct integers satisfying $f(x) = p$. Show that $f$ does not have integer zeros.
	\flushright \href{https://artofproblemsolving.com/community/c6h609876}{(Link to AoPS)}
\end{problem}



\begin{solution}[by \href{https://artofproblemsolving.com/community/user/29428}{pco}]
	\begin{tcolorbox}Let $f$ be a polynomial function with integer coefficients and $p$ be a prime number. Suppose there are at least four distinct integers satisfying $f(x) = p$. Show that $f$ does not have integer zeros.\end{tcolorbox}
If $f(x)$ has an integer root, WLOG consider $f(0)=0$ and so $f(x)=xg(x)$

$f(t)=p$ implies then $t|p$ and so $t\in\{-p,-1,1,p\}$

$f(-p)=p$ $\implies$ $g(-p)=-1$ and $f(p)=p$ $\implies$ $g(p)=1$ and so $2p|2$, impossible

Hence the result
\end{solution}
*******************************************************************************
-------------------------------------------------------------------------------

\begin{problem}[Posted by \href{https://artofproblemsolving.com/community/user/216299}{adgj1234}]
	Show that there are infintely many pairs (a,b) of relatively prime integers (not necessarily positive) such that both the equations \begin{eqnarray*} x^2 +ax +b &=& 0 \\ x^2 + 2ax + b &=& 0 \\ \end{eqnarray*} have integer roots.
	\flushright \href{https://artofproblemsolving.com/community/c6h615950}{(Link to AoPS)}
\end{problem}



\begin{solution}[by \href{https://artofproblemsolving.com/community/user/29428}{pco}]
	\begin{tcolorbox}Show that there are infintely many pairs (a,b) of relatively prime integers (not necessarily positive) such that both the equations \begin{eqnarray*} x^2 +ax +b &=& 0 \\ x^2 + 2ax + b &=& 0 \\ \end{eqnarray*} have integer roots.\end{tcolorbox}
Let $t\in\mathbb Z\setminus\{-2,-1,0,1\}$, $t\not\equiv 1\pmod 3$ :

$x^2+(2t+1)x-(t-1)t(t+1)(t+2)=(x-(t-1)(t+1))(x+t(t+2))$
$x^2+2(2t+1)x-(t-1)t(t+1)(t+2)=(x-(t-1)t)(x+(t+1)(t+2))$
\end{solution}



\begin{solution}[by \href{https://artofproblemsolving.com/community/user/216299}{adgj1234}]
	How did you reach to this solution?
\end{solution}



\begin{solution}[by \href{https://artofproblemsolving.com/community/user/29428}{pco}]
	\begin{tcolorbox}How did you reach to this solution?\end{tcolorbox}
$a^2-4b=u^2$
$a^2-b=v^2$
So (second line) $b=a^2-v^2$ and so (first line) $u^2-4v^2=-3a^2$ and so for example $u-2v=-3$ and $u+2v=a^2$

So $u=2v-3$ and $4v-3=a^2$ and $v=\frac{a^2+3}4$ and so $a$ is odd and we can write $a=2t+1$

This gives $a=2t+1$ and $b=a^2-v^2=a^2-\left(\frac{a^2+3}4\right)^2=-(t-1)t(t+1)(t+2)$

$a,b\ne 0$ $\implies$ $t\in\mathbb Z\setminus\{-2,-1,0,1\}$
$|a|,|b|$ coprime $\implies$ $t\not\equiv 1\pmod 3$
\end{solution}



\begin{solution}[by \href{https://artofproblemsolving.com/community/user/18418}{\u0391\u03c1\u03c7\u03b9\u03bc\u03ae\u03b4\u03b7\u03c2 6}]
	\begin{tcolorbox}Show that there are infintely many pairs (a,b) of relatively prime integers (not necessarily positive) such that both the equations \begin{eqnarray*} x^2 +ax +b &=& 0 \\ x^2 + 2ax + b &=& 0 \\ \end{eqnarray*} have integer roots.\end{tcolorbox}

As i see it is true for every integer d.

$x^2+ax+b=0$

$x^2+dax+b=0$
\end{solution}



\begin{solution}[by \href{https://artofproblemsolving.com/community/user/167924}{utkarshgupta}]
	[url=http://www.artofproblemsolving.com/Forum/viewtopic.php?p=341871&sid=bedc386f5e3bd29a4c58c7485ed6b57d#p341871]India 1995[\/url]

The topic should be locked or rather merged :)
\end{solution}
*******************************************************************************
-------------------------------------------------------------------------------

\begin{problem}[Posted by \href{https://artofproblemsolving.com/community/user/152589}{rightways}]
	Zeros of a fourth-degree polynomial $f (x)$ form an arithmetic progression. Prove that the zeros of $f '(x)$ also form an arithmetic progression.
	\flushright \href{https://artofproblemsolving.com/community/c6h617386}{(Link to AoPS)}
\end{problem}



\begin{solution}[by \href{https://artofproblemsolving.com/community/user/223754}{Submathematics}]
	if the roots of the $f(x)$ are $a-3d, a-d, a+d, a+3d$ then  $f(x)=x^4-4ax^3+(6a^2-10d^2)x^2-2ax(a^2-10d^2)+c$ then its easy to see that roots of f'(x) are $a, a-sqrt(5)d, a+sqrt(5)$
looking forward to an elegant solution
\end{solution}



\begin{solution}[by \href{https://artofproblemsolving.com/community/user/29428}{pco}]
	\begin{tcolorbox}Zeros of a fourth-degree polynomial $f (x)$ form an arithmetic progression. Prove that the zeros of $f '(x)$ also form an arithmetic progression.\end{tcolorbox}
Wrong. Chose as counter-example $f(x)=x^4-3x^3+2x^2$ whose zeroes are $\{0,1,2\}$
\end{solution}



\begin{solution}[by \href{https://artofproblemsolving.com/community/user/152589}{rightways}]
	Patrick

but $\{0,0,2,1\}$ does not form a.p
\end{solution}



\begin{solution}[by \href{https://artofproblemsolving.com/community/user/89198}{chaotic_iak}]
	The \begin{italicized}distinct\end{italicized} zeroes do. The problem is unclear on this, whether the zeroes must be counted with multiplicity or not.
\end{solution}



\begin{solution}[by \href{https://artofproblemsolving.com/community/user/29428}{pco}]
	\begin{tcolorbox}The \begin{italicized}distinct\end{italicized} zeroes do. The problem is unclear on this, whether the zeroes must be counted with multiplicity or not.\end{tcolorbox}
Indeed. As too often, badly written problem.

A very simple effort could have been something like : "Let a degree-4 polynomial with four distincts real zeroes ...."
\end{solution}



\begin{solution}[by \href{https://artofproblemsolving.com/community/user/29428}{pco}]
	\begin{tcolorbox}Zeros of a fourth-degree polynomial $f (x)$ form an arithmetic progression. Prove that the zeros of $f '(x)$ also form an arithmetic progression.\end{tcolorbox}
Let us consider only the case with \begin{bolded}four real distinct\end{underlined}\end{bolded} zeroes $a,a+r;a+2r,a+3r$ with $r>0$

Then $P(x)=\alpha(x-a)(x-a-r)(x-a-2r)(x-a-3r)$ for some $\alpha,r\ne 0$

Then $P(\frac r2x+a-\frac{3r}2)=\frac{\alpha r^4}{16}(x-3)(x-1)(x+1)(x+3)$ $=\frac{\alpha r^4}{16}(x^4-10x^2+9)$

Then $\frac r2P'(\frac r2x+a-\frac{3r}2)=\frac{\alpha r^4}{16}(4x^3-20x)$ $=\frac{\alpha r^4}{4}x(x^2-5)$ $=\frac{\alpha r^4}{4}(x-\sqrt 5)x(x+\sqrt 5)$

And since $P'(\frac r2x+a-\frac{3r}2)$ has three real distinct zeroes in AP, it's immediate to get that $P'(x)$ also has.

Q.E.D.
\end{solution}
*******************************************************************************
-------------------------------------------------------------------------------

\begin{problem}[Posted by \href{https://artofproblemsolving.com/community/user/231016}{dothef1}]
	Prove that \[ \sum_{k=1}^n\frac{1}{P'(x_k)}=0 \]
where $ P(x) $ is a polynomial with n distinct real roots , namely $ x_i $ for i=1,2....,n
	\flushright \href{https://artofproblemsolving.com/community/c6h617722}{(Link to AoPS)}
\end{problem}



\begin{solution}[by \href{https://artofproblemsolving.com/community/user/29428}{pco}]
	\begin{tcolorbox}Prove that \[ \sum_{k=1}^n\frac{1}{P'(x_k)}=0 \]
where $ P(x) $ is a polynomial with n distinct real roots , namely $ x_i $ for i=1,2....,n\end{tcolorbox}
Wrong. Choose as trivial counter-example $P(x)=x$
\end{solution}



\begin{solution}[by \href{https://artofproblemsolving.com/community/user/231016}{dothef1}]
	Sorry this is not what i wanted to type , i'll fix it as soon as i get the correct one .
\end{solution}



\begin{solution}[by \href{https://artofproblemsolving.com/community/user/29428}{pco}]
	\begin{tcolorbox}Prove that \[ \sum_{k=1}^n\frac{1}{P'(x_k)}=0 \]
where $ P(x) $ is a polynomial with n distinct real roots , namely $ x_i $ for i=1,2....,n\end{tcolorbox}
You just need to add the condition $n\ge 2$ : let $n\ge 2$ distinct real numbers $x_1,x_2,...x_n$
Let $P(x)=\alpha\prod_{i=1}^{n}(x-x_i)$ with $\alpha\ne 0$ and $Q(x)=\alpha\prod_{i=1}^{n-1}(x-x_i)$ so that $P(x)=Q(x)(x-x_{n})$ 

Let $R(x)=\sum_{i=1}^{n-1}\frac{Q(x)}{Q'(x_i)(x-x_i)}$ (considered as prolongated thru continuity for $x=x_i$). Note that here we need $n\ge 2$.
$R(x)$ is a polynomial with degree at most $n-2$ and $R(x_i)=1$ for $i=1,2,...,n-1$ and so $R(x)=1$ $\forall x$

So $\sum_{i=1}^{n-1}\frac 1{Q'(x_i)(x-x_i)}=\frac 1{Q(x)}$ $\forall x\notin\{x_1,x_2,...x_{n-1}\}$ and so $\sum_{i=1}^{n-1}\frac 1{Q'(x_i)(x_i-x_{n})}+\frac 1{Q(x_{n})}=0$

Which is exactly $\sum_{i=1}^{n}\frac 1{P'(x_i)}=0$
Hence the result.
\end{solution}
*******************************************************************************
-------------------------------------------------------------------------------

\begin{problem}[Posted by \href{https://artofproblemsolving.com/community/user/226936}{junior2001}]
	Find all polynomials of degree 3, such that for each $x,y\geq 0$: \[p(x+y)\geq p(x)+p(y)\]
	\flushright \href{https://artofproblemsolving.com/community/c6h618613}{(Link to AoPS)}
\end{problem}



\begin{solution}[by \href{https://artofproblemsolving.com/community/user/29428}{pco}]
	\begin{tcolorbox}Find all polynomials of degree 3, such that for each $x,y\geq 0$: \[p(x+y)\geq p(x)+p(y)\]\end{tcolorbox}
Writing $P(x)=ax^3+bx^2+cx+d$ with $a\ne 0$, we get $xy(3a(x+y)+2b)\ge d$ and so $a>0$ and $d\le 0$

If $b\ge 0$, conditions $a>0$ and $d\le 0$ are enough
If $b<0$ and $xy=0$, conditions $a>0$ and $d\le 0$ are enough
If $b<0$ and $x,y>0$, condition is $3a(x+y)+2b\ge \frac d{xy}$ which implies $d<0$
writing then $x+y=z$, we get $xy\le \frac {z^2}4$ and, for a given $z$, LHS reaches its maximum when $xy=\frac {z^2}4$
And so we need $3az+2b\ge \frac {4d}{z^2}$ $\forall z>0$ and so $z^3+\frac{2b}{3a}z^2-\frac{4d}{3a}\ge 0$ $\forall z>0$

This cubic reaches it's minimum over $\mathbb R^+$ when $z=-\frac{4b}{9a}$ and so we need (and it's enough) $\left(-\frac{4b}{9a}\right)^3+\frac{2b}{3a}\left(-\frac{4b}{9a}\right)^2-\frac{4d}{3a}\ge 0$
And so $d\le \frac{8b^3}{243 a^2}$

Hence the result : $\boxed{P(x)=ax^3+bx^2+cx+d\text{ with }a>0\text{ and }d\le\min\left(0, \frac{8b^3}{243 a^2}\right)}$
\end{solution}
*******************************************************************************
-------------------------------------------------------------------------------

\begin{problem}[Posted by \href{https://artofproblemsolving.com/community/user/218396}{vi1lat}]
	Find value $a$ , where sum of real roots of equation $\frac{f(a)x^2 +1}{x^2+g(a)}=\sqrt{\frac{xg(a)-1}{f(a)-x}}$ was minimum value , and $f(a)=a^2-a\sqrt{20} +23$ , $g(a)=1,5a^2-a\sqrt{20}+24$
	\flushright \href{https://artofproblemsolving.com/community/c6h618688}{(Link to AoPS)}
\end{problem}



\begin{solution}[by \href{https://artofproblemsolving.com/community/user/218396}{vi1lat}]
	does anyone have solutions?
\end{solution}



\begin{solution}[by \href{https://artofproblemsolving.com/community/user/29428}{pco}]
	\begin{tcolorbox}does anyone have solutions?\end{tcolorbox}
Since you posted in this "proposed and own" category, this means that you have a solution and are just looking for another one.
Please, post your own solution, please.

Hereunder is mine

\begin{tcolorbox}Find value $a$ , where sum of real roots of equation $\frac{f(a)x^2 +1}{x^2+g(a)}=\sqrt{\frac{xg(a)-1}{f(a)-x}}$ was minimum value , and $f(a)=a^2-a\sqrt{20} +23$ , $g(a)=1,5a^2-a\sqrt{20}+24$\end{tcolorbox}
For easier writing, let $u=f(a)=a^2-a\sqrt{20}+23$ and $v=g(a)=\frac 32a^2-a\sqrt{20}+24$
1) some preliminary results
1.1 : $u=(a-\sqrt 5)^2+18\ge 18$
1.2 : $v=\frac 32(a-\frac{2\sqrt 5}3)^2+\frac{62}3\ge \frac{62}3$
1.3 : using previous results, we trivially have $uv\ge 3$ and so $\frac 1v<\frac 2v<\frac {2u}3<u$
1.4 : $4u < v^2$
$4v^2-16u=(3a^2-4a\sqrt{5}+48)^2-16(a^2-2a\sqrt{5}+23)$ $=9a^4-24\sqrt 5a^3+352a^2-352\sqrt 5a+1936$
$=(3a^2-4a\sqrt 5+44)^2+8a^2$ $>0$
Q.E.D.
1.5 : $2u^2 > 9v$
$4u^2-18v=4(a^2-2a\sqrt 5+23)^2-9(3a^2-4a\sqrt 5+48)$ $=4a^4-16\sqrt 5a^3+237a^2-332\sqrt 5a+1684$
$=(2a^2-4a\sqrt 5+37)^2+9(a-2\sqrt 5)^2+135$ $>0$
Q.E.D

2) solution
2.1 get an equivalent cubic
Equation is $\frac{ux^2+1}{x^2+v}=\sqrt{\frac{vx-1}{u-x}}$ and is only defined when $x\in[\frac 1v,u)$ (remember $uv>3>1$)

$\iff$ $x-\frac{ux^2+1}{x^2+v}=x-\sqrt{\frac{vx-1}{u-x}}$

$\iff$ $\frac{x^3-ux^2+vx-1}{x^2+v}$ $=\frac{x^2-\frac{vx-1}{u-x}}{x+\sqrt{\frac{vx-1}{u-x}}}$ $=-\frac{x^3-ux^2+vx-1}{(u-x)(x+\sqrt{\frac{vx-1}{u-x}})}$

$\iff$ $(x^3-ux^2+vx-1)\left(\frac 1{x^2+v}+\frac{1}{(u-x)(x+\sqrt{\frac{vx-1}{u-x}})}\right)=0$

$\iff$ $x\in[\frac 1v,u)$ and $x^3-ux^2+vx-1=0$ (other factor is $>0$)

2.2 Prove that equivalent cubic has no complex root and that all real roots are in the good interval
Let $P(x)=x^3-ux^2+vx-1$

$P(\frac 1v)=\frac 1{v^3}-\frac u{v^2}=\frac{1-uv}{v^3}$ $<0$ (see 1.3 above)

$P(\frac 2v)=\frac 8{v^3}-\frac{4u}{v^2}+1$ $=\frac 8{v^3}+\frac {v^2-4u}{v^2}$ $>0$ (see 1.4 above)

$P(\frac{2u}3)=\frac{8u^3}{27}-\frac{4u^3}9+\frac{2uv}3-1$ $=\frac{2u(9v-2u^2)}{27}-1$ $<0$ (see 1.5 above)

$P(u)=uv-1$ $>0$ (see 1.3 above)

So we have indeed three distinct real roots in the suitable interval and so the sum of all real roots is exactly the sum of all roots and  is $u$ whose minimum is $18$ when $\boxed{a=\sqrt 5}$ (see 1.1 above)
\end{solution}



\begin{solution}[by \href{https://artofproblemsolving.com/community/user/218396}{vi1lat}]
	Sorry, but I haven't solution.
\end{solution}



\begin{solution}[by \href{https://artofproblemsolving.com/community/user/29428}{pco}]
	\begin{tcolorbox}Sorry, but I haven't solution.\end{tcolorbox}
You are welcome.
Glad to have helped you.

And please, respect the usage of each forum. Posting in "proposed and own" means that you have a solution.
\end{solution}
*******************************************************************************
-------------------------------------------------------------------------------

\begin{problem}[Posted by \href{https://artofproblemsolving.com/community/user/233748}{Pasimbung}]
	Find all polynomial that is not a zero polynomial $P(x)$ and $Q(x)$ with real coefficient and minimal degree such that
$P(x^2)+ Q(x) = P(x) + x^5Q(x)$
for every real number $x$.
	\flushright \href{https://artofproblemsolving.com/community/c6h619099}{(Link to AoPS)}
\end{problem}



\begin{solution}[by \href{https://artofproblemsolving.com/community/user/29428}{pco}]
	\begin{tcolorbox}Find all polynomial that is not a zero polynomial $P(x)$ and $Q(x)$ with real coefficient and minimal degree such that
$P(x^2)+ Q(x) = P(x) + x^5Q(x)$
for every real number $x$.\end{tcolorbox}
Setting $x=0$, we get $Q(0)=0$ and so $Q(x)=xR(x)$ and equation is $P(x^2)-P(x)=x(x^5-1)R(x)$
LHS has an even degree and so must be degree of $R(x)$

Trying degree 0 for $R(x)$ and so $R(x)=u$ and $P(x)=ax^3+bx^2+cx+d$ leads to an impossibility (look at coefficient of $x^3$)

Trying degree 2 for $R(x)$ and so $R(x)=ux^2+vx+w$ and $P(x)=ax^4+bx^3+cx^2+dx+e$ gives the minimal degree solution :

$\boxed{P(x)=a(x^4+x^3+x^2+x+b)\text{  and  }Q(x)=a(x^3+x)}$ which indeed is a solution whatever are $a\ne 0$ and $b\in\mathbb R$
\end{solution}



\begin{solution}[by \href{https://artofproblemsolving.com/community/user/64716}{mavropnevma}]
	With $P(x^2)-P(x) = (x^5-1)Q(x)$, consider the primitive $5$-root of unity $\omega = \cos \dfrac {2\pi}{5} + \textrm{i}\sin \dfrac {2\pi}{5}$; we then have $P(\omega^2)=P(\omega)$, $P(\omega^4)=P(\omega^2)$, $P(\omega)=P(\omega^6) = P(\omega^3)$, $P(\omega^3)=P(\omega^8) = P(\omega^4)$, thus $P(\omega)=P(\omega^2) = P(\omega^3)=P(\omega^4)$, equal to some value $v$. Therefore $\dfrac {x^5-1}{x-1} = x^4+x^3+x^2+x+1 \mid P(x) - v$, so there exists a polynomial $f$ such that $P(x) = (x^4+x^3+x^2+x+1)f(x) + v$. Since obviously then $x^5-1 \mid P(x^2)-P(x)$ (with the quotient being $Q(x)$), it means these are also \begin{bolded}all\end{bolded} the solutions (and the condition that $P$ is a real polynomial also forces $v$ to be real).

For minimal degree, since $Q$ is asked not to be the nil polynomial, it follows we need take $f$ non-zero constant, thus $f(x)=a\neq 0$, and so \begin{bolded}pco\end{bolded}'s $b$ constant will be $b=1+\dfrac {v}{a}$.
\end{solution}
*******************************************************************************
